\subsection{San Ambrosio, obispo}

\subsubsection{Comentario: Nació el que es Siervo y Señor a la vez}

\src{Comentario 4-5 sobre el salmo 35: \\CCL 64, 52-53.\cite{Ambrosio_CCL064_0052}}

\begin{body}
	\ltr{C}{reo} que sobre la pobreza y sufrimientos del Señor hemos aducido testimonios muy válidos de dos santos, de los cuales uno vio y testimonió, mientras que el otro fue elegido tan sólo para testimoniar. Escuchemos todavía nuevos testimonios sobre la condición servil del Señor tomados de estos testigos fiables, o mejor, escuchemos lo que de sí mismo dice el mismo Señor por boca de ambos. Veamos lo que dice: \emph{Habla el Señor, que desde el vientre me formó siervo suyo, para que le trajese a Jacob, para que le reuniese a Israel}. Advirtamos que asumió la condición de siervo para reunir al pueblo.
	
	Estaba yo ---dice--- en las entrañas maternas, y el Señor pronunció mi nombre. Escuchemos cuál es el nombre que el Padre le da: \emph{Mirad: la Virgen concebirá y dará a luz un hijo y le pondrá por nombre Emmanuel, que significa \textquote{Dios-con-nosotros}}. ¿Cuál si no es el nombre de Cristo sino el de \textquote{Hijo de Dios}? Escucha un nuevo texto. Hablando de María a José, también Gabriel había dicho: \emph{Dará a luz un hijo, y tú le pondrás por nombre Jesús}. Escucha ahora la voz de Dios: \emph{Y tú, Belén, tierra de Judá, no eres ni mucho menos la última de las ciudades de Judá: pues de ti saldrá un jefe que será el pastor de mi pueblo}.
	
	Advierte el misterio: del seno de la Virgen nació el que es Siervo y Señor a la vez ---siervo para trabajar, señor para mandar---, a fin de implantar el reinado de Dios en el corazón del hombre. Ambos son uno: no uno del Padre y otro de la Virgen, sino que el mismo que antes de los siglos fue engendrado por el Padre se encarnará más tarde en el seno de la Virgen. Por eso se le llama Siervo y Señor: siervo por nosotros, mas, por la unidad de la naturaleza divina, Dios de Dios, príncipe de príncipe, igual de igual; pues no pudo el Padre engendrar un ser inferior a él y afirmar al mismo tiempo que en el Hijo tiene sus complacencias.
	
	\emph{Gran cosa es para ti} ---dice--- \emph{que seas mi siervo y restablezcas las tribus de Jacob}. Emplea siempre términos adecuados a su dignidad: Gran Dios y gran siervo, pues al encarnarse no perdió los atributos de su grandeza, ya que su grandeza no tiene fin. Así pues, es igual en cuanto Hijo de Dios, asumió la condición de siervo al encarnarse, sufrió la muerte aquel cuya grandeza no tiene fin, porque \emph{el fin de la ley es Cristo, y con eso se justifica a todo el que cree,} para que todos creamos en él y le adoremos con profundo afecto. Bendita servidumbre que a todos nos otorgó la libertad, bendita servidumbre que le valió el \textquote{nombre-sobre-todo-nombre}, bendita humildad que hizo que \emph{al nombre de Jesús toda rodilla se doble en el cielo, en la tierra, en el abismo, y toda lengua proclame: Jesucristo es Señor para gloria de Dios Padre}.
\end{body}