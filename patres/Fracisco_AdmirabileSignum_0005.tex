\begin{patercite}
	¡Cuánta emoción debería acompañarnos mientras colocamos en el belén las montañas, los riachuelos, las ovejas y los pastores! De esta manera recordamos, como lo habían anunciado los profetas, que toda la creación participa en la fiesta de la venida del Mesías. Los ángeles y la estrella son la señal de que también nosotros estamos llamados a ponernos en camino para llegar a la gruta y adorar al Señor.
	
	\textquote{Vayamos, pues, a Belén, y veamos lo que ha sucedido y que el Señor nos ha comunicado} (\emph{Lc} 2,15), así dicen los pastores después del anuncio hecho por los ángeles. Es una enseñanza muy hermosa que se muestra en la sencillez de la descripción. A diferencia de tanta gente que pretende hacer otras mil cosas, los pastores se convierten en los primeros testigos de lo esencial, es decir, de la salvación que se les ofrece. Son los más humildes y los más pobres quienes saben acoger el acontecimiento de la encarnación. A Dios que viene a nuestro encuentro en el Niño Jesús, los pastores responden poniéndose en camino hacia Él, para un encuentro de amor y de agradable asombro. Este encuentro entre Dios y sus hijos, gracias a Jesús, es el que da vida precisamente a nuestra religión y constituye su singular belleza, y resplandece de una manera particular en el pesebre.
	
	Tenemos la costumbre de poner en nuestros belenes muchas figuras simbólicas, sobre todo, las de mendigos y de gente que no conocen otra abundancia que la del corazón. Ellos también están cerca del Niño Jesús por derecho propio, sin que nadie pueda echarlos o alejarlos de una cuna tan improvisada que los pobres a su alrededor no desentonan en absoluto. De hecho, los pobres son los privilegiados de este misterio y, a menudo, aquellos que son más capaces de reconocer la presencia de Dios en medio de nosotros.
	
	Los pobres y los sencillos en el Nacimiento recuerdan que Dios se hace hombre para aquellos que más sienten la necesidad de su amor y piden su cercanía. Jesús, \textquote{manso y humilde de corazón} (\emph{Mt} 11,29), nació pobre, llevó una vida sencilla para enseñarnos a comprender lo esencial y a vivir de ello. Desde el belén emerge claramente el mensaje de que no podemos dejarnos engañar por la riqueza y por tantas propuestas efímeras de felicidad\ldots{}
	
	\textbf{Francisco, papa}, Carta apostólica \emph{Admirabile signum}, nn. 5-6\cite{Fracisco_AdmirabileSignum_0005}.
\end{patercite}
