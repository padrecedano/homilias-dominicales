\subsection{San Ambrosio, obispo}

\ptheme{¿Eres tú el que ha de venir o tenemos que esperar a otro?}

\src{Comentario sobre el evangelio de san Lucas, \\Lib. 5, 93-95. 99-102. 109: CCL 14, 165-166. 167-168. 171-177.\cite{Ambrosio_CCL014_0165}}

\begin{body}
	\ltr{J}{uan} \emph{envió a dos de sus discípulos a preguntar a Jesús: \textquote{¿Eres tú el que ha de venir o tenemos que esperar a otro?}}. No es sencilla la comprensión de estas sencillas palabras, o de lo contrario este texto estaría en contradicción con lo dicho anteriormente. ¿Cómo, en efecto, puede Juan afirmar aquí que desconoce a quien anteriormente había reconocido por revelación de Dios Padre? ¿Cómo es que entonces conoció al que previamente desconocía mientras que ahora parece desconocer al que ya antes conocía? \emph{Yo} ---dice--- \emph{no lo conocía, pero el que me envió a bautizar con agua me dijo: \textquote{Aquel sobre quien veas bajar el Espíritu Santo\ldots{}}}. Y Juan dio fe al oráculo, reconoció al revelado, adoró al bautizado y profetizó al enviado Y concluye: \emph{Y yo lo he visto, y he dado testimonio de que éste es el elegido de Dios}. ¿Cómo, pues, aceptar siquiera la posibilidad de que un profeta tan grande haya podido equivocarse, hasta el punto de no considerar aún como Hijo de Dios a aquel de quien había afirmado: \emph{Éste es el que quita el pecado del mundo?} 
	
	Así pues, ya que la interpretación literal es contradictoria, busquemos el sentido espiritual. Juan ---lo hemos dicho ya--- era tipo de la ley, precursora de Cristo. Y es correcto afirmar que la ley ---aherrojada materialmente como estaba en los corazones de los sin fe, como en cárceles privadas de la luz eterna, y constreñida por entrañas fecundas en sufrimientos e insensatez--- era incapaz de llevar a pleno cumplimiento el testimonio de la divina economía sin la garantía del evangelio. Por eso, envía Juan a Cristo dos de sus discípulos, para conseguir un suplemento de sabiduría, dado que Cristo es la plenitud de la ley. 
	
	Además, sabiendo el Señor que nadie puede tener una fe plena sin el evangelio ---ya que si la fe comienza en el antiguo Testamento no se consuma sino en el nuevo---, a la pregunta sobre su propia identidad, responde no con palabras, sino con hechos. \emph{Id} ---dice--- \emph{a anunciar a Juan lo que estáis viendo y oyendo: los ciegos ven y los inválidos andan; los leprosos quedan limpios y los sordos oyen; los muertos resucitan y a los pobres se les anuncia la buena noticia}. Y sin embargo, estos ejemplos aducidos por el Señor no son aún los definitivos: la plenificación de la fe es la cruz del Señor, su muerte, su sepultura. Por eso, completa sus anteriores afirmaciones añadiendo: \emph{¡Y dichoso el que no se sienta defraudado por mí!} Es verdad que la cruz se presta a ser motivo de escándalo incluso para los elegidos, pero no lo es menos que no existe mayor testimonio de una persona divina, nada hay más sobrehumano que la íntegra oblación de uno solo por la salvación del mundo; este solo hecho lo acredita plenamente como Señor. Por lo demás, así es cómo Juan lo designa: \emph{Este es el Cordero de Dios, que quita el pecado del mundo}. En realidad, esta respuesta no va únicamente dirigida a aquellos dos hombres, discípulos de Juan: va dirigida a todos nosotros, para que creamos en Cristo en base a los hechos. 
	
	\emph{Entonces, ¿a qué salisteis?, ¿a ver a un profeta? Sí, os digo, y más que profeta}. Pero, ¿cómo es que querían ver a Juan en el desierto, si estaba encerrado en la cárcel? El Señor propone a nuestra imitación a aquel que le había preparado el camino no sólo precediéndolo en el nacimiento según la carne y anunciándolo con la fe, sino también anticipándosele con su gloriosa pasión. Más que profeta, sí, ya que es él quien cierra la serie de los profetas; más que profeta, ya que muchos desearon ver a quien éste profetizó, a quien éste contempló, a quien éste bautizó.
\end{body}