\subsection{San Fulberto de Chartres, obispo}

\subsubsection{Carta: El misterio de nuestra salvación}

\src{Carta 5: PL 141, 198-199.\cite{FulbertoDeChartres_PL141_0198}}

\begin{body}
	\ltr{N}{o} nos resulta difícil sopesar la diversidad de naturalezas en Cristo. En efecto: uno es el nacimiento o la naturaleza en que, en frase de san Pablo, \emph{nació de una mujer, nació bajo la ley;} otra por la que en el principio estaba junto a Dios; una es la naturaleza por la que, engendrado de la virgen María, vivió humilde en la tierra, y otra por la que, eterno y sin principio, creó el cielo y la tierra; una es la naturaleza en la que se afirma que fue presa de la tristeza, que el cansancio le rindió, que padeció hambre, que lloró, y otra en virtud de la cual curó paralíticos, hizo caminar a los tullidos, dio la vista al ciego de nacimiento, calmó con su imperio las turgentes olas, resucitó muertos.
	
	Siendo así las cosas, es necesario que quien desee llevar el nombre de cristiano con coherencia y sin perjuicio personal, confiese que Cristo, en quien reconocemos dos naturalezas, es a la vez verdadero Dios y hombre verdadero. Así, una vez asegurada la verdad de las dos naturalezas, la fe verdadera no confunda ni divida a Cristo, verdadero en los dolores de su humanidad y verdadero en los poderes de su divinidad. Pues en él la unidad de persona no tolera división y la realidad de la doble naturaleza no admite confusión. En él no subsisten separados Dios y hombre, sino que Cristo es al mismo tiempo Dios y hombre. Efectivamente, Cristo es el mismo Dios que con su divinidad destruyó la muerte; el mismo Hijo de Dios que no podía morir en su divinidad, murió en la carne mortal que el Dios inmortal había asumido; y este mismo Cristo Hijo de Dios, muerto en la carne, resucitó, pues muriendo en la carne, no perdió la inmortalidad de su divinidad.
	
	Sabemos con plena certeza que, siendo pecadores por el primer nacimiento, el segundo nos ha purificado; siendo cautivos por el primer nacimiento, el segundo nos ha liberado; siendo terrenos por el primer nacimiento, el segundo nos hace celestes; siendo carnales por el vicio del primer nacimiento, el beneficio del segundo nacimiento nos hace espirituales; por el primer nacimiento somos hijos de ira, por el segundo nacimiento somos hijos de gracia. Por tanto, todo el que atenta contra la santidad del bautismo, sepa que está ofendiendo al mismo Dios, que dijo: \emph{El que no nazca de agua y Espíritu no puede entrar en el reino de Dios}. Constituye, por tanto, una gracia de la doctrina de la salvación, conocer la profundidad del misterio del bautismo, del que el Apóstol afirma: Si \emph{hemos muerto con Cristo, creemos que también viviremos con él}. Conmorir y ser sepultados con Cristo tiene como meta poder resucitar con él, poder vivir con él.
\end{body}