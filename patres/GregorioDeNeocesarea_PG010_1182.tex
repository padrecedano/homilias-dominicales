\subsection{San Gregorio de Neocesarea, obispo}

\ptheme{Vino a nosotros el que es el esplendor de la gloria del Padre}

\src{Homilía 4 en la santa Teofanía: \\PG 10, 1182-1183.\cite{GregorioDeNeocesarea_PG010_1182}}

\begin{body}
	\ltr{E}{stando} tú presente, me es imposible callar, pues yo soy la voz, y precisamente \emph{la voz que grita en el desierto: preparad el camino del Señor. Soy yo el que necesita que tú me bautices, ¿y tú acudes a mí?} Al nacer, yo hice fecunda la esterilidad de la madre que me engendró, y, cuando todavía era un niño, procuré medicina a la mudez de mi padre, recibiendo de ti, niño, la gracia de hacer milagros.
	
	Por tu parte, nacido de María la Virgen según quisiste y de la manera que tú solo conociste, no menoscabaste su virginidad, sino que la preservaste y se la regalaste junto con el apelativo de Madre. Ni la virginidad obstaculizó tu nacimiento ni el nacimiento lesionó la virginidad, sino que ambas realidades: nacimiento y virginidad ---realidades contradictorias si las hay---, firmaron un pacto, porque para ti, Creador de la naturaleza, esto es fácil y hacedero.
	
	Yo soy solamente hombre, partícipe de la gracia divina; tú, en cambio, eres a la vez Dios y hombre, pues eres benigno y amas con locura el género humano. \emph{Soy yo el que necesita que tú me bautices, ¿y tú acudes a mí?} Tú que eras al principio, y estabas junto a Dios y eras Dios mismo; tú que eres el esplendor de la gloria del Padre; tú que eres la imagen perfecta del padre perfecto; tú que eres \emph{la luz verdadera, que alumbra a todo hombre que viene a este mundo;} tú que para estar en el mundo viniste donde ya estabas; tú que te hiciste carne sin convertirte en carne; tú que acampaste entre nosotros y te hiciste visible a tus siervos en la condición de esclavo; tú que, con tu santo nombre como con un puente, uniste el cielo y la tierra: ¿tú acudes a mí? ¿Tú, tan grande, a un hombre como yo?, ¿el Rey al precursor?, ¿el Señor al siervo?
	
	Pues aunque tú no te hayas avergonzado de nacer en las humildes condiciones de la humanidad, yo no puedo traspasar los límites de la naturaleza. Tengo conciencia del abismo que separa la tierra del Creador. Conozco la diferencia existente entre el polvo de la tierra y el Hacedor. Soy consciente de que la claridad de tu sol de justicia me supera con mucho a mí, que soy la lámpara de tu gracia. Y aun cuando estés revestido de la blanca nube del cuerpo, reconozco no obstante tu dominación. Confieso mi condición servil y proclamo tu magnificencia. Reconozco la perfección de tu dominio, y conozco mi propia abyección y vileza. \emph{No soy digno de desatar la correa de tu sandalia;} ¿cómo, pues, voy a atreverme a tocar la inmaculada coronilla de tu cabeza? ¿Cómo voy a extender sobre ti mi mano derecha, sobre ti que extendiste los cielos como una tienda y cimentaste sobre las aguas la tierra? ¿Cómo abriré mi mano de siervo sobre tu divina cabeza? ¿Cómo lavar al inmaculado y exento de todo pecado? ¿Cómo iluminar a la misma luz? ¿Qué oración pronunciaré sobre ti, sobre ti que acoges incluso las plegarias de los que no te conocen?
\end{body}
