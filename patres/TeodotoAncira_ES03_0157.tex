\subsection{Teodoto de Ancira}

\ptheme{El Dueño de todo vino en forma de siervo}

\src{Sermón en la Natividad del Salvador \\Edic. Schwartz, ACO t. 3, parte 1, 157-159.\cite{TeodotoAncira_ES03_0157}}

\begin{body}
	\ltr{E}{l} Dueño de todo vino en forma de siervo, revestido de pobreza, para no ahuyentar la presa. Habiendo elegido para nacer la inseguridad de un campo indefenso, nace de una pobrecilla virgen, inmerso en la pobreza, para, en silencio, dar caza al hombre y así salvarlo. Pues de haber nacido en medio del boato, y si se hubiera rodeado de riqueza, los infieles habrían dicho, y con razón, que había sido la abundancia de riqueza la que había operado la transformación de la redondez de la tierra. Y si hubiera elegido Roma, entonces la ciudad más poderosa, hubieran pensado que era el poderío de sus ciudadanos el que había cambiado el mundo.
	
	De haber sido el hijo del emperador, su obra benéfica se habría inscrito en el haber de las influencias. Si hubiera nacido hijo de un legislador, su reforma social se habría atribuido al ordenamiento jurídico. Y ¿qué es lo que hizo? Escogió todo lo vil y pobre, todo lo mediocre e ignorado por la gran masa, a fin de dar a conocer que la divinidad era la única transformadora de la tierra. He aquí por qué eligió una madre pobre, una patria todavía más pobre, y él mismo falto de recursos.
	
	Aprende la lección del pesebre. No habiendo lecho en que acostar al Señor, se le coloca en un pesebre, y la indigencia de lo más imprescindible se convierte en privilegiado anuncio de la profecía. Fue colocado en un pesebre para indicar que iba a convertirse en manjar incluso de los irracionales. En efecto, viviendo en la pobreza y yaciendo en un pesebre, la Palabra e Hijo de Dios atrae a sí tanto a los ricos como a los pobres, a los elocuentes como a los de premiosa palabra.
	
	Fíjate cómo la ausencia de bienes dio cumplimiento a la profecía, y cómo la pobreza ha hecho accesible a todos a aquel que por nosotros se hizo pobre. Nadie tuvo reparo en acudir por temor a las soberbias riquezas de Cristo; nadie sintió bloqueado el acceso por la magnificencia del poder: se mostró cercano y pobre, ofreciéndose a sí mismo por la salvación de todos.
	
	Mediante la corporeidad asumida, el Verbo de Dios se hace presente en el pesebre, para hacer posible que todos, racionales e irracionales, participen del manjar de salvación. Y pienso que esto es lo que ya antes había pregonado el profeta, desvelándonos el misterio de este pesebre: \emph{Conoce el buey a su amo, y el asno, el pesebre de su dueño; Israel no me conoce, mi pueblo no recapacita}. El que es rico, por nosotros se hizo pobre, haciendo fácilmente perceptible a todos la salvación con la fuerza de la divinidad. Refiriéndose a esto decía asimismo el gran Pablo: \emph{Siendo rico, por nosotros se hizo pobre, para que nosotros, con su pobreza, nos hagamos ricos}.
	
	Pero, ¿quién era el que enriquecía?, ¿de qué enriquecía?, y, ¿cómo se hizo él pobre por nosotros? Dime, por favor: ¿quién, siendo rico, se ha hecho pobre con mi pobreza? ¿Quizá el que apareció hecho hombre? Pero éste nunca fue rico, sino que nació pobre de padres pobres. ¿Quién, pues, era rico y con qué nos enriquecía el que por nosotros se hizo pobre? Dios ---dice--- enriquece a la criatura. Es, pues, Dios quien se hizo pobre, haciendo suya la pobreza del que se hacía visible; él es efectivamente rico en su divinidad, y por nosotros se hizo pobre.
\end{body}