\subsection{San Máximo Confesor}

\ptheme{Misterio siempre nuevo}

\src{De las Cinco Centurias, Centuria 1, 8-13: \\PG 90, 1182-86.\cite{MaximoConfesor_PG090_1182}}

\begin{body}
	\ltr{L}{a} Palabra de Dios, nacida una vez en la carne (lo que nos indica la querencia de su benignidad y humanidad), vuelve a nacer siempre gustosamente en el espíritu para quienes lo desean; vuelve a hacerse niño, y se vuelve a formar en aquellas virtudes; y la amplitud de su grandeza no disminuye por malevolencia o envidia, sino que se manifiesta a sí mismo en la medida en que sabe que lo puede asimilar el que lo recibe, y así, al mismo tiempo que explora discretamente la capacidad de quienes desean verlo, sigue manteniéndose siempre fuera del alcance de su percepción, a causa de la excelencia del misterio.
	
	Por lo cual, el santo Apóstol, considerando sabiamente la fuerza del misterio, exclama: \emph{Jesucristo es el mismo ayer y hoy y siempre;} ya que entendía el misterio como algo siempre nuevo, al que nunca la comprensión de la mente puede hacer envejecer.
	
	Nace Cristo Dios, hecho hombre mediante la incorporación de una carne dotada de alma inteligente; el mismo que había otorgado a las cosas proceder de la nada. Mientras tanto, brilla en lo alto la estrella del Oriente y conduce a los Magos al lugar en que yace la Palabra encarnada; con lo que muestra que hay en la ley y los profetas una palabra místicamente superior, que dirige a las gentes a la suprema luz del conocimiento.
	
	Así pues, la palabra de la ley y de los profetas, entendida alegóricamente, conduce, como una estrella, al pleno conocimiento de Dios a aquellos que fueron llamados por la fuerza de la gracia, de acuerdo con el designio divino.
	
	Dios se hace efectivamente hombre perfecto, sin alterar nada de lo que es propio de la naturaleza, a excepción del pecado (pues ni el mismo pecado era propio de la naturaleza). Se hace efectivamente hombre perfecto a fin de provocar, con la vista del manjar de su carne, la voracidad insaciable y ávida del dragón infernal; y abatirlo por completo cuando ingiriera una carne que habría de convertírsele en veneno, porque en ella se hallaba oculto el poder de la divinidad. Esta carne sería al mismo tiempo remedio de la naturaleza humana, ya que el mismo poder divino presente en aquélla habría de restituir la naturaleza humana a la gracia primera.
	
	Y así como el dragón, deslizando su veneno en el árbol de la ciencia, había corrompido con su sabor la naturaleza, de la misma manera, al tratar de devorar la carne del Señor, se vio corrompido y destruido por la virtud de la divinidad que en ella residía.
	
	Inmenso misterio de la divina encarnación, que sigue siendo siempre misterio; pues, ¿de qué modo puede la Palabra hecha carne seguir siendo su propia persona esencialmente, siendo así que la misma persona existe al mismo tiempo con todo su ser en Dios Padre? ¿Cómo la Palabra, que es toda ella Dios por naturaleza, se hizo toda ella por naturaleza hombre, sin detrimento de ninguna de las dos naturalezas: ni de la divina, en cuya virtud es Dios, ni de la nuestra, en virtud de la cual se hizo hombre? Sólo la fe capta estos misterios, ella precisamente que es la sustancia y la base de todas aquellas realidades que exceden la percepción y razón de la mente humana en todo su alcance.
\end{body}
