\subsection{San Agustín, obispo}

\subsubsection{Sermón: Saciados con la visión de la Palabra}

\src{Sermón 194, 3-4: PL 38, 1016-1017.\cite{Agustin_PL038_1016}}

\begin{body}
	\ltr[¿]{Q}{ué} ser humano podría conocer todos los tesoros de sabiduría y de ciencia ocultos en Cristo y escondidos en la pobreza de su carne? Porque, \emph{siendo rico, se hizo pobre por vosotros, para enriqueceros con su pobreza}. Pues cuando asumió la condición mortal y experimentó la muerte, se mostró pobre: pero prometió riquezas para más adelante, y no perdió las que le habían quitado.
	
	¡Qué inmensidad la de su dulzura, que escondió para los que lo temen, y llevó a cabo para los que esperan en él!
	
	Nuestros conocimientos son ahora parciales, hasta que se cumpla lo que es perfecto. Y para que nos hagamos capaces de alcanzarlo, él, que era igual al Padre en la forma de Dios, se hizo semejante a nosotros en la forma de siervo, para reformarnos a semejanza de Dios: y, convertido en hijo del hombre --él, que era único Hijo de Dios---, convirtió a muchos hijos de los hombres en hijos de Dios; y, habiendo alimentado a aquellos siervos con su forma visible de siervo, los hizo libres para que contemplasen la forma de Dios.
	
	Pues \emph{ahora somos hijos de Dios y aún no se ha manifestado lo que seremos. Sabemos que, cuando se manifieste, seremos semejantes a él, porque lo veremos tal cual es}. Pues ¿para qué son aquellos tesoros de sabiduría y de ciencia, para qué sirven aquellas riquezas divinas sino para colmarnos? ¿Y para qué la inmensidad de aquella dulzura sino para saciarnos? \emph{Muéstranos al Padre y nos basta}.
	
	Y en algún salmo, uno de nosotros, o en nosotros, o por nosotros, le dice: \emph{Me saciaré cuando se manifieste tu gloria}. Pues él y el Padre son una misma cosa: y quien lo ve a él ve también al Padre. De modo que \emph{el Señor, Dios de los ejércitos, él es el Rey de la gloria}. Volviendo a nosotros, nos mostrará su rostro; y nos salvaremos y quedaremos saciados, y eso nos bastará.
	
	Pero mientras eso no suceda, mientras no nos muestre lo que habrá de bastarnos, mientras no le bebamos como fuente de vida y nos saciemos, mientras tengamos que andar en la fe y peregrinemos lejos de él, mientras tenemos hambre y sed de justicia y anhelamos con inefable ardor la belleza de la forma de Dios, celebremos con devota obsequiosidad el nacimiento de la forma de siervo.
	
	Si no podemos contemplar todavía al que fue engendrado por el Padre antes que el lucero de la mañana, tratemos de acercarnos al que nació de la Virgen en medio de la noche. No comprendemos aún que su \emph{nombre dura como el sol;} reconozcamos que su \emph{tienda} ha sido puesta \emph{en el sol}.
	
	Todavía no podemos contemplar al Único que permanece en su Padre; recordemos al \emph{Esposo que sale de su alcoba}. Todavía no estamos preparados para el banquete de nuestro Padre; reconozcamos al menos el pesebre de nuestro Señor Jesucristo.
\end{body}
