\begin{patercite}
	\textquote{No temáis, pues os anuncio una gran alegría. (\ldots{}) Os ha nacido hoy, en la ciudad de David, un salvador} (\emph{Lc} 2, 10-11). El mensaje de la venida de Cristo, que llegó del cielo mediante el anuncio de los ángeles, sigue resonando en esta ciudad, así como en las familias, en los hogares y en las comunidades de todo el mundo. Es una \textquote{gran alegría}, dijeron los ángeles, \textquote{para todo el pueblo}. Este mensaje proclama que el Mesías, el Hijo de Dios e hijo de David nació \textquote{por vosotros}: por ti y por mí, y por todos los hombres y mujeres de todo tiempo y lugar. En el plan de Dios, Belén, \textquote{el menor entre los clanes de Judá} (\emph{Mi} 5, 1) se convirtió en un lugar de gloria imperecedera: el lugar donde, en la plenitud de los tiempos, Dios eligió hacerse hombre, para acabar con el largo reinado del pecado y de la muerte, y para traer vida nueva y abundante a un mundo ya viejo, cansado y oprimido por la desesperación.
	
	Para los hombres y mujeres de todo lugar, Belén está asociada a este alegre mensaje de renacimiento, renovación, luz y libertad. Y, sin embargo, aquí, en medio de nosotros, ¡qué lejos de hacerse realidad parece esa magnífica promesa! ¡Qué distante parece el Reino de amplio dominio y paz, de seguridad, justicia e integridad, que el profeta Isaías anunció, como hemos escuchado en la primera lectura (cf. \emph{Is} 9, 7) y que proclamamos como definitivamente establecido con la venida de Jesucristo, Mesías y Rey!
	
	\textbf{Benedicto XVI, papa,} \emph{Homilía} en Belén, 13 de mayo del 2009\cite{BenedictoXVI_Homilia_20090513}.
\end{patercite}
