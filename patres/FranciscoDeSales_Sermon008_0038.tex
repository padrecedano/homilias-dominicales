\subsection{San Francisco de Sales, obispo}

\ptheme{Regalemos lo más grande al Niño-Dios.}

\src{Sermón VIII, 38.\cite{FranciscoDeSales_Sermon008_0038}}

\begin{body}
	\textquote{\emph{Unos magos, venidos de Oriente, llegaron a Jerusalén}} (Mt 2,2).
	
	\ltr{E}{s} una gran fiesta, en la que celebramos que la Iglesia de los Gentiles es aceptada por Cristo y recibida por Cristo. Sí, es una gran fiesta porque los gentiles llegan a Cristo y a la Casa del Pan.
	
	La Epifanía es el día de los dones. Nunca ha recibido Cristo regalos más espléndidos y ahí tenemos la manera de ofrecer nuestros presentes a Dios. Los Magos nos lo pueden enseñar, ya que el primer acto de cada clase sirve de tipo a lo demás. Veamos, pues, las circunstancias: ¿Quién? ¿Qué? ¿A quién? ¿Por qué? ¿Cómo?
	
	¿Quién? Unos Reyes sabios. Antes de haber recibido la fe, ya creían. Reyes piadosos, que observaban las estrellas siguiendo la profecía de Balaam; su devoción se demuestra al dejar sus reinos y al acudir y presentarse intrépidamente al rey Herodes y confesarle ingenuamente su fe.
	
	¿Qué? Oro, incienso y mirra. Las opiniones de los doctores están divididas cuando explican la razón de estos presentes. Strabus dice que trajeron de lo que producía su país de Arabia. Todo agrada a Dios: Abel le daba de sus rebaños y el que no tenía sino una piel de cabra, también podía ofrecérsela. Honra al Señor con tus bienes.
	
	Hay quienes ofrecen al Señor lo que no poseen. Hijo mío, ¿por qué no eres más devoto? Lo seré en mi ancianidad. Pero, ¿sabes tú que llegarás a viejo? Otro dice: Si yo fuese capuchino, ofrecería sacrificios al Señor. Honra al Señor con lo que tienes. Si yo fuese rico\ldots{} yo daría\ldots{} Honra al Señor con tu pobreza. Si yo fuera santo\ldots{} Honra al Señor con tu paciencia, si yo fuera doctor\ldots{}, honra al Señor con tu sencillez\ldots{}
	
	De lo que tienes, el valor de tu ofrenda se mide en relación con lo que posees. San Agustín dice que los Magos le trajeron oro como a rey; incienso como a Dios; mirra como a hombre. ¿A quién? ¡A Cristo nuestro Señor! ¿Por qué? ¡Hemos venido a adorar al Señor! ¿Cómo? ¡Se postraron y le adoraron!
	
	Y no digamos que no tenemos nada muy grande para regalarle. Nada hay suficientemente digno de Dios. Debéis decir: \textquote{Yo quiero, Divino Niño, darte el único bien que poseo: yo mismo, y te ruego que aceptes este don}. Y Él nos responderá: \textquote{Hijo mío, tu regalo no es pequeño sino en tu propia estima}.
\end{body}