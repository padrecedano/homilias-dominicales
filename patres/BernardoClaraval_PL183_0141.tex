\subsection{San Bernardo de Claraval, abad}

\subsubsection{Sermón: Plenitud de los tiempos y de la divinidad}

\src{Sermón 1-2 en la Epifanía del Señor, \\PL 183, 141-143.\cite{BernardoClaraval_PL183_0141}}

\begin{body}
	\ltr{H}{a} \emph{aparecido la bondad de Dios, nuestro Salvador, y su amor al hombre}. Gracias sean dadas a Dios, que ha hecho abundar en nosotros el consuelo en medio de esta peregrinación, de este destierro, de esta miseria.
	
	Antes de que apareciese la humanidad de nuestro Salvador, su bondad se hallaba también oculta, aunque ésta ya existía, pues la misericordia del Señor es eterna. ¿Pero cómo, a pesar de ser tan inmensa, iba a poder ser reconocida? Estaba prometida, pero no se la alcanzaba a ver; por lo que muchos no creían en ella. Efectivamente, \emph{en distintas ocasiones y de muchas maneras habló Dios por los profetas}. Y decía: Yo tengo \emph{designios de paz y no de aflicción}. Pero ¿qué podía responder el hombre que sólo experimentaba la aflicción e ignoraba la paz? ¿Hasta cuándo vais a estar diciendo: \emph{\textquote{Paz, paz}, y no hay paz?} A causa de lo cual \emph{los mensajeros de paz lloraban amargamente,} diciendo: \emph{Señor, ¿quién creyó nuestro anuncio?} Pero ahora los hombres tendrán que creer a sus propios ojos, ya que \emph{los testimonios de Dios se han vuelto absolutamente creíbles}. Pues para que ni una vista perturbada pueda dejar de verlo, \emph{puso su tienda al sol}.
	
	Pero de lo que se trata ahora no es de la promesa de la paz, sino de su envío; no de la dilatación de su entrega, sino de su realidad; no de su anuncio profético, sino de su presencia. Es como si Dios hubiera vaciado sobre la tierra un saco lleno de su misericordia; un saco que habría de desfondarse en la pasión, para que se derramara nuestro precio, oculto en él; un saco pequeño, pero lleno. Ya que \emph{un niño se nos ha dado}, pero \emph{en quien habita toda la plenitud de la divinidad}. Ya que, cuando llegó la plenitud del tiempo, hizo también su aparición la plenitud de la divinidad. Vino en carne mortal para que, al presentarse así ante quienes eran carnales, en la aparición de su humanidad se reconociese su bondad. Porque, cuando se pone de manifiesto la humanidad de Dios, ya no puede mantenerse oculta su bondad. ¿De qué manera podía manifestar mejor su bondad que asumiendo mi carne? La mía, no la de Adán, es decir, no la que Adán tuvo antes del pecado.
	
	¿Hay algo que pueda declarar más inequívocamente la misericordia de Dios que el hecho de haber aceptado nuestra miseria? ¿Qué hay más rebosante de piedad que la Palabra de Dios convertida en tan poca cosa por nosotros? \emph{Señor, ¿qué es el hombre, para que te acuerdes de él, el ser humano, para darle poder?} Que deduzcan de aquí los hombres lo grande que es el cuidado que Dios tiene de ellos; que se enteren de lo que Dios piensa y siente sobre ellos. No te preguntes, tú, que eres hombre, por lo que has sufrido, sino por lo que sufrió él. Deduce de todo lo que sufrió por ti, en cuánto te tasó, y así su bondad se te hará evidente por su humanidad. Cuanto más pequeño se hizo en su humanidad, tanto más grande se reveló en su bondad; y cuanto más se dejó envilecer por mí, tanto más querido me es ahora. \emph{Ha aparecido} ---dice el Apóstol--- \emph{la bondad de Dios, nuestro Salvador, y su amor al hombre}. Grandes y manifiestos son, sin duda, la bondad y el amor de Dios, y gran indicio de bondad reveló quien se preocupó de añadir a la humanidad el nombre de Dios.
\end{body}