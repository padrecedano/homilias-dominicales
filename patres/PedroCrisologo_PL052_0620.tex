\subsection{San Pedro Crisólogo, obispo}

\ptheme{El que por nosotros quiso nacer no quiso ser ignorado por nosotros}

\src{Sermón 160: PL 52, 620-622.\cite{PedroCrisologo_PL052_0620}}

\begin{body}
	\ltr{A}{unque} en el mismo misterio del nacimiento del Señor se dieron insignes testimonios de su divinidad, sin embargo, la solemnidad que celebramos manifiesta y revela de diversas formas que Dios ha asumido un cuerpo humano, para que nuestra inteligencia, ofuscada por tantas obscuridades, no pierda por su ignorancia lo que por gracia ha merecido recibir y poseer.
	
	Pues el que por nosotros quiso nacer no quiso ser ignorado por nosotros; y por esto se manifestó de tal forma que el gran misterio de su bondad no fuera ocasión de un gran error.
	
	Hoy el mago encuentra llorando en la cuna a aquel que, resplandeciente, buscaba en las estrellas. Hoy el mago contempla claramente entre pañales a aquel que, encubierto, buscaba pacientemente en los astros.
	
	Hoy el mago discierne con profundo asombro lo que allí contempla: el cielo en la tierra, la tierra en el cielo; el hombre en Dios, y Dios en el hombre; y a aquel que no puede ser encerrado en todo el universo incluido en un cuerpo de niño. Y, viendo, cree y no duda; y lo proclama con sus dones místicos: el incienso para Dios, el oro para el Rey, y la mirra para el que morirá.
	
	Hoy el gentil, que era el último, ha pasado a ser el primero, pues entonces la fe de los magos consagró la creencia de las naciones.
	
	Hoy Cristo ha entrado en el cauce del Jordán para lavar el pecado del mundo. El mismo Juan atestigua que Cristo ha venido para esto: \emph{Éste es el Cordero de Dios, que quita el pecado del mundo}. Hoy el siervo recibe al Señor, el hombre a Dios, Juan a Cristo; el que no puede dar el perdón recibe a quien se lo concederá.
	
	Hoy, como afirma el profeta, \emph{la voz del Señor sobre las aguas}. ¿Qué voz? \emph{Este es mi Hijo, el amado, mi predilecto}.
	
	Hoy el Espíritu Santo se cierne sobre las aguas en forma de paloma, para que, así como la paloma de Noé anunció el fin del diluvio, de la misma forma ésta fuera signo de que ha terminado el perpetuo naufragio del mundo. Pero a diferencia de aquélla, que sólo llevaba un ramo de olivo caduco, ésta derramará la enjundia completa del nuevo crisma en la cabeza del Autor de la nueva progenie, para que se cumpliera aquello que predijo el profeta: \emph{Por eso el Señor, tu Dios, te ha ungido con aceite de júbilo entre todos tus compañeros}.
	
	Hoy Cristo, al convertir el agua en vino, comienza los signos celestes. Pero el agua había de convertirse en el misterio de la sangre, para que Cristo ofreciese a los que tienen sed la pura bebida del vaso de su cuerpo, y se cumpliese lo que dice el profeta: \emph{Y mi copa rebosa}.
\end{body}