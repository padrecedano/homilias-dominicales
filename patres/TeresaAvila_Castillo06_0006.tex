\begin{patercite}
	\ldots{} Hermanas, tengo por mejor que nos pongamos delante del Señor y miremos su misericordia y grandeza y nuestra bajeza, y dénos El lo que quisiere, siquiera haya agua, siquiera sequedad: El sabe mejor lo que nos conviene.\\ Y con esto andaremos descansadas y el demonio no tendrá tanto lugar de hacernos trampantojos.
	
	Entre estas cosas penosas y sabrosas juntamente da nuestro Señor al alma algunas veces unos júbilos y oración extraña, que no sabe entender qué es.\\ Porque si os hiciere esta merced, le alabéis mucho y sepáis que es cosa que pasa, la pongo aquí. Es, a mi parecer, una unión grande de las potencias, sino que las deja nuestro Señor con libertad para que gocen de este gozo, y a los sentidos lo mismo, sin entender qué es lo que gozan y cómo lo gozan.
	
	Parece esto algarabía, y cierto pasa así, que es un gozo tan excesivo del alma, que no querría gozarle a solas, sino decirlo a todos para que la ayudasen a alabar a nuestro Señor, que aquí va todo su movimiento. ¡Oh, qué de fiestas haría y qué de muestras, si pudiese, para que todos entendiesen su gozo! Parece que se ha hallado a sí, y que, como el padre del hijo pródigo, querría convidar a todos y hacer grandes fiestas , por ver su alma en puesto que no puede dudar que está en seguridad, al menos por entonces. Y tengo para mí que es con razón; porque tanto gozo interior de lo muy íntimo del alma, y con tanta paz, y que todo su contento provoca a alabanzas de Dios, no es posible darle el demonio.
	
	Es harto, estando con este gran ímpetu de alegría, que calle y pueda disimular, y no poco penoso. Esto debía sentir San Francisco, cuando le toparon los ladrones, que andaba por el campo dando voces y les dijo que era pregonero del gran Rey , y otros santos que se van a los desiertos por poder pregonar lo que San Francisco estas alabanzas de su Dios.
	
	\textbf{Santa Teresa de Jesús}, \emph{El Castillo Interior,} Moradas Sextas, capítulo 6\cite{TeresaAvila_Castillo06_0006}.
\end{patercite}