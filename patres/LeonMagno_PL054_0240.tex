\subsection{San León Magno, papa}

\ptheme{Dios ha manifestado su salvación en todo el mundo}

\src{Sermón 3, 1-3. 5 en la Epifanía del Señor: \\PL 54, 240-244.\cite{LeonMagno_PL054_0240}}

\begin{body}
	\ltr{L}{a} misericordiosa providencia de Dios, que ya había decidido venir en los últimos tiempos en ayuda del mundo que perecía, determinó de antemano la salvación de todos los pueblos en Cristo.
	
	De estos pueblos se trataba en la descendencia innumerable que fue en otro tiempo prometida al santo patriarca Abrahán, descendencia que no sería engendrada por una semilla de carne, sino por fecundidad de la fe, descendencia comparada a la multitud de las estrellas, para que de este modo el padre de todas las naciones esperara una posteridad no terrestre, sino celeste.
	
	Así pues, que todos los pueblos vengan a incorporarse a la familia de los patriarcas, y que los hijos de la promesa reciban la bendición de la descendencia de Abrahán, a la cual renuncian los hijos según la carne. Que todas las naciones, en la persona de los tres Magos, adoren al Autor del universo, y que Dios sea conocido, no ya sólo en Judea, sino también en el mundo entero, para que por doquier \emph{sea grande su nombre en Israel}.
	
	Instruidos en estos misterios de la gracia divina, queridos míos, celebremos con gozo espiritual el día que es el de nuestras primicias y aquél en que comenzó la salvación de los paganos. Demos gracias al Dios misericordioso, quien, según palabras del Apóstol, \emph{nos ha hecho capaces de compartir la herencia del pueblo santo en la luz; él nos ha sacado del dominio de las tinieblas y nos ha trasladado al reino de su Hijo querido}. Porque, como profetizó Isaías, \emph{el pueblo que caminaba en tinieblas vio una luz grande; habitaban en tierra de sombras, y una luz les brilló}. También a propósito de ellos dice el propio Isaías al Señor: \emph{Naciones que no te conocían te invocarán, un pueblo que no te conocía correrá hacia ti}.
	
	Abrahán vio \emph{este día, y se llenó de alegría,} cuando supo que sus hijos según la fe serían benditos en su descendencia, a saber, en Cristo, y él se vio a sí mismo, por su fe, como futuro padre de todos los pueblos, \emph{dando gloria a Dios, al persuadirse de que Dios es capaz de hacer lo que promete}.
	
	También David anunciaba este día en los salmos cuando decía: Todos los pueblos vendrán a postrarse en tu presencia, Señor; bendecirán tu nombre; y también: El Señor da a conocer su victoria, revela a las naciones su justicia.
	
	Esto se ha realizado, lo sabemos, en el hecho de que tres magos, llamados de su lejano país, fueron conducidos por una estrella para conocer y adorar al Rey del cielo y de la tierra. La docilidad de los magos a esta estrella nos indica el modo de nuestra obediencia, para que, en la medida de nuestras posibilidades, seamos servidores de esa gracia que llama a todos los hombres a Cristo.
	
	Animados por este celo, debéis aplicaros, queridos míos, a seros útiles los unos a los otros, a fin de que brilléis como hijos de la luz en el reino de Dios, al cual se llega gracias a la fe recta y a las buenas obras; por nuestro Señor Jesucristo que, con Dios Padre y el Espíritu Santo, vive y reina por los siglos de los siglos. Amén.
\end{body}