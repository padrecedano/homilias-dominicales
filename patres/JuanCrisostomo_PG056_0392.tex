\subsection{San Juan Crisóstomo, obispo}

\ptheme{Junto al Niño Jesús están María y José}

\src{Homilía sobre el día de Navidad: PG 56, 392.\cite{JuanCrisostomo_PG056_0392}} 

\begin{body}
	\ltr{E}{ntró} Jesús en Egipto para poner fin al llanto de la antigua tristeza; suplantó las plagas por el gozo, y convirtió la noche y las tinieblas en luz de salvación. Entonces fue contaminada el agua del río con la sangre de los tiernos niños. Por eso entró en Egipto el que había convertido el agua en sangre, comunicó a las aguas vivas el poder de aflorar la salvación y las purificó de su fango e impureza con la virtud del Espíritu. Los egipcios fueron afligidos y, enfurecidos, no reconocieron a Dios. Entró, pues, Jesús en Egipto y, colmando las almas religiosas del conocimiento de Dios, dio al río el poder de fecundar una mies de mártires más copiosa que la mies de grano.
	
	¿Qué más diré o cómo seguir hablando? Veo a un artesano y un pesebre; veo a un Niño y los pañales de la cuna, veo el parto de la Virgen carente de lo más imprescindible, todo marcado por la más apremiante necesidad; todo bajo la más absoluta pobreza. ¿Has visto destellos de riqueza en la más extrema pobreza? ¿Cómo, siendo rico, se ha hecho pobre por nuestra causa? ¿Cómo es que no dispuso ni de lecho ni de mantas, sino que fue depositado en un desnudo pesebre? ¡Oh tesoro de riqueza, disimulado bajo la apariencia de pobreza! Yace en el pesebre, y hace temblar el orbe de la tierra; es envuelto en pañales, y rompe las cadenas del pecado; aún no sabe articular palabra, y adoctrina a los Magos induciéndolos a la conversión.
	
	¿Qué más diré o cómo seguir hablando? Ved a un Niño envuelto en pañales y que yace en un pesebre: está con él María, que es Virgen y Madre; le acompañaba José, que es llamado padre.
	
	José era sólo el esposo: fue el Espíritu quien la cubrió con su sombra. Por eso José estaba en un mar de dudas y no sabía cómo llamar al Niño. Esta es la razón por la que, trabajado por la duda, recibe, por medio del ángel, un oráculo del cielo: \emph{José, no tengas reparo en llevarte a tu mujer, pues la criatura que hay en ella viene del Espíritu Santo}. En efecto, el Espíritu Santo cubrió a la Virgen con su sombra. Y ¿por qué nace de la Virgen y conserva intacta su virginidad? Pues porque en otro tiempo el diablo engañó a la virgen Eva; por lo cual a María, que dio a luz siendo virgen, fue Gabriel quien le comunicó la feliz noticia. Es verdad que la seducida Eva dio a luz una palabra que introdujo la muerte; pero no lo es menos que María, acogiendo la alegre noticia, engendró al Verbo en la carne, que nos ha merecido la vida eterna.
\end{body}