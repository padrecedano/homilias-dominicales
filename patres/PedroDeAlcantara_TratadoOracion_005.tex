\begin{patercite}
	El que quisiere ver cuánto ha aprovechado en este camino de Dios, mire cuánto crece cada día en humildad interior y exterior. ¿Cómo sufre las injusticias de los otros? ¿Cómo sabe dar pasada a las flaquezas ajenas? ¿Cómo acude a las necesidades de sus prójimos? ¿Cómo se compadece y no se indigna contra los defectos ajenos? ¿Cómo sabe esperar en Dios en el tiempo de la tribulación? ¿Cómo rige su lengua? ¿Cómo guarda su corazón? ¿Cómo trae domada su carne con todos sus apetitos y sentidos? ¿Cómo se sabe valer en las prosperidades y adversidades? ¿Cómo se repara y provee en todas las cosas con gravedad y discreción?
	
	Y, sobre todo esto, mire si está muerto el amor de la honra, y del regalo, y del mundo, y según lo que en esto hubiere aprovechado o desaprovechado, así se juzgue, y no según lo que siente o no siente de Dios. Y por esto siempre ha de tener él un ojo, y el más principal en la mortificación, y el otro en la oración, porque esa misma mortificación no se puede perfectamente alcanzar sin el socorro de la oración.
	
	\textbf{San Pedro de Alcántara}, \emph{Tratado sobre la Oración,} capítulo 5\cite{PedroDeAlcantara_TratadoOracion_005}.			
\end{patercite}

\newsection