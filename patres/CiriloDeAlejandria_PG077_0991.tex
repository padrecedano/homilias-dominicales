\subsection{San Cirilo de Alejandría, obispo}

\ptheme{Alabanzas de la Madre de Dios}

\src{Homilía 4, pronunciada en el Concilio de Éfeso: PG 77, 991. 995-996. \cite{CiriloDeAlejandria_PG077_0991}}

\begin{body}
	\ltr{T}{engo} ante mis ojos la asamblea de los santos padres que, llenos de gozo y fervor, han acudido aquí, respondiendo con prontitud a la invitación de la santa Madre de Dios, la siempre Virgen María. Este espectáculo ha trocado en gozo la gran tristeza que antes me oprimía. Vemos realizadas en esta reunión aquellas hermosas palabras de David, el salmista: \emph{Ved qué dulzura, qué delicia; convivir los hermanos unidos}.
	
	Te saludamos, santa y misteriosa Trinidad, que nos has convocado a todos nosotros en esta iglesia de santa María, Madre de Dios.
	
	Te saludamos, María, Madre de Dios, tesoro digno de ser venerado por todo el orbe, lámpara inextinguible, corona de la virginidad, trono de la recta doctrina, templo indestructible, lugar propio de aquel que no puede ser contenido en lugar alguno, madre y virgen, por quien es llamado bendito, en los santos evangelios, el que viene en nombre del Señor.
	
	Te saludamos, a ti, que encerraste en tu seno virginal a aquel que es inmenso e inabarcable; a ti, por quien la santa Trinidad es adorada y glorificada; por quien la cruz preciosa es celebrada y adorada en todo el orbe; por quien exulta el cielo; por quien se alegran los ángeles y arcángeles; por quien son puestos en fuga los demonios; por quien el diablo tentador cayó del cielo; por quien la criatura, caída en el pecado, es elevada al cielo; por quien la creación, sujeta a la insensatez de la idolatría, llega al conocimiento de la verdad; por quien los creyentes obtienen la gracia del bautismo y el aceite de la alegría; por quien han sido fundamentadas las Iglesias en el orbe de la tierra; por quien todos los hombres son llamados a la conversión.
	
	Y ¿qué más diré? Por ti, el Hijo unigénito de Dios ha iluminado a \emph{los que vivían en tinieblas y en sombra de muerte}; por ti, los profetas anunciaron las cosas futuras; por ti, los apóstoles predicaron la salvación a los gentiles; por ti, los muertos resucitan; por ti, reinan los reyes, por la santísima Trinidad.
	
	¿Quién habrá que sea capaz de cantar como es debido las alabanzas de María? Ella es madre y virgen a la vez; ¡qué cosa tan admirable! Es una maravilla que me llena de estupor. ¿Quién ha oído jamás decir que le esté prohibido al constructor habitar en el mismo templo que él ha construido? ¿Quién podrá tachar de ignominia el hecho de que la sirviente sea adoptada como madre?
	
	Mirad: hoy todo el mundo se alegra; quiera Dios que todos nosotros reverenciemos y adoremos la unidad, que rindamos un culto impregnado de santo temor a la Trinidad indivisa, al celebrar, con nuestras alabanzas, a María siempre Virgen, el templo santo de Dios, y a su Hijo y esposo inmaculado: porque a él pertenece la gloria por los siglos de los siglos. Amén.
\end{body}