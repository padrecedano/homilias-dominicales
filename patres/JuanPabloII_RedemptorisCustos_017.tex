\begin{patercite}
	Durante su vida, que fue una peregrinación en la fe, José, al igual que María, permaneció fiel a la llamada de Dios hasta el final. La vida de ella fue el cumplimiento hasta sus últimas consecuencias de aquel primer \textquote{\emph{fiat}} pronunciado en el momento de la anunciación mientras que José en el momento de su \textquote{anunciación} no pronunció palabra alguna. Simplemente él \textquote{\emph{hizo} como el ángel del Señor le había mandado} (\emph{Mt} 1, 24). \emph{Y este primer \textquote{hizo} es el comienzo del \textquote{camino de José}}. A lo largo de este camino, los Evangelios no citan ninguna palabra dicha por él. Pero el \emph{silencio de José} posee una especial elocuencia: gracias a este silencio se puede leer plenamente la verdad contenida en el juicio que de él da el Evangelio: el \textquote{justo} (\emph{Mt} 1, 19).
	
	Hace falta saber leer esta verdad, porque ella contiene \emph{uno de los testimonios más importantes acerca del hombre y de su vocación}. En el transcurso de las generaciones la Iglesia lee, de modo siempre atento y consciente, dicho testimonio, casi como si sacase del tesoro de esta figura insigne \textquote{lo nuevo y lo viejo} (\emph{Mt} 13, 52).
	
	\textbf{San Juan Pablo II, papa,} \emph{Redemptoris Custos,} 17\cite{JuanPabloII_RedemptorisCustos_017}.
\end{patercite}