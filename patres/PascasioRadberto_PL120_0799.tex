\subsection{Pascasio Radberto}

\ptheme{Velad, para estar preparados}

\src{Exposición sobre el evangelio de san Mateo, \\Lib. 11, cap. 24: PL 120, 799-800.\cite{PascasioRadberto_PL120_0799}}

\begin{body}
	\ltr{V}{\emph{elad,}} \emph{porque no sabéis el día ni la hora}. Siendo una recomendación que a todos afecta, la expresa como si solamente se refiriera a los hombres de aquel entonces. Es lo que ocurre con muchos otros pasajes que leemos en las Escrituras. Y de tal modo atañe a todos lo así expresado, que a cada uno le llega el último día y para cada cual es el fin del mundo el momento mismo de su muerte. Por eso es necesario que cada uno parta de este mundo tal cual ha de ser juzgado aquel día. En consecuencia, todo hombre debe cuidar de no dejarse seducir ni abandonar la vigilancia, no sea que el día de la venida del Señor lo encuentre desprevenido. 
	
	Y aquel día encontrará desprevenido a quien hallare desprevenido el último día de su vida. Pienso que los apóstoles estaban convencidos de que el Señor no iba a presentarse en sus días para el juicio final; y sin embargo, ¿quién dudará de que ellos cuidaron de no dejarse seducir, de que no abandonaron la vigilancia y de que observaron todo lo que a todos fue recomendado, para que el Señor los hallara preparados? Por esta razón, debemos tener siempre presente una doble venida de Cristo: una, cuando aparezca de nuevo y hayamos de dar cuenta de todos nuestros actos; otra diaria, cuando a todas horas visita nuestras conciencias y viene a nosotros, para que cuando viniere, nos encuentre preparados.
	
	¿De qué me sirve, en efecto, conocer el día del juicio si soy consciente de mis muchos pecados?, ¿conocer si viene o cuándo viene el Señor, si antes no viniere a mi alma y retornare a mi espíritu?, ¿si antes no vive Cristo en mí y me habla? Sólo entonces será su venida un bien para mí, si primero Cristo vive en mí y yo vivo en Cristo. Y sólo entonces vendrá a mí, como en una segunda venida, cuando, muerto para el mundo, pueda en cierto modo hacer mía aquella expresión: \emph{El mundo está crucificado para mí, y yo para el mundo}.
	
	Considera asimismo estas palabras de Cristo: \emph{Porque muchos vendrán usando mi nombre}. Sólo el anticristo y sus secuaces se arrogan falsamente el nombre de Cristo, pero sin las obras de Cristo, sin sus palabras de verdad, sin su sabiduría. En ninguna parte de la Escritura hallarás que el Señor haya usado esta expresión y haya dicho: \emph{Yo soy el Cristo}. Le bastaba mostrar con su doctrina y sus milagros lo que era realmente, pues las obras del Padre que realizaba, la doctrina que enseñaba y su poder gritaban: \emph{Yo soy el Cristo} con más eficacia que si mil voces lo pregonaran. Cristo, que yo sepa, jamás se atribuyó verbalmente este título: lo hizo realizando las obras del Padre y enseñando la ley del amor. En cambio, los falsos cristos, careciendo de esta ley del amor, proclamaban de palabra ser lo que no eran.				
\end{body}