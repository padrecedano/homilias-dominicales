\subsection{San León Magno, papa}

\ptheme{El nacimiento del Señor es el nacimiento de la paz}

\src{Sermón 6, 2-3 en la Natividad del Señor: \\PL 54, 213-216.\cite{LeonMagno_PL054_0213}}

\begin{body}
	\ltr{A}{unque} aquella infancia, que la majestad del Hijo de Dios se dignó hacer suya, tuvo como continuación la plenitud de una edad adulta, y, después del triunfo de su pasión y resurrección, todas las acciones de su estado de humildad, que el Señor asumió por nosotros, pertenecen ya al pasado, la festividad de hoy renueva ante nosotros los sagrados comienzos de Jesús, nacido de la Virgen María; de modo que, mientras adoramos el nacimiento de nuestro Salvador, resulta que estamos celebrando nuestro propio comienzo.
	
	Efectivamente, la generación de Cristo es el comienzo del pueblo cristiano, y el nacimiento de la cabeza lo es al mismo tiempo del cuerpo.
	
	Aunque cada uno de los que llama el Señor a formar parte de su pueblo sea llamado en un tiempo determinado y aunque todos los hijos de la Iglesia hayan sido llamados cada uno en días distintos, con todo, la totalidad de los fieles, nacida en la fuente bautismal, ha nacido con Cristo en su nacimiento, del mismo modo que ha sido crucificada con Cristo en su pasión, ha sido resucitada en su resurrección y ha sido colocada a la derecha del Padre en su ascensión.
	
	Cualquier hombre que cree ---en cualquier parte del mundo---, y se regenera en Cristo, una vez interrumpido el camino de su vieja condición original, pasa a ser un nuevo hombre al renacer; y ya no pertenece a la ascendencia de su padre carnal, sino a la simiente del Salvador, que se hizo precisamente Hijo del hombre, para que nosotros pudiésemos llegar a ser hijos de Dios.
	
	Pues si él no hubiera descendido hasta nosotros revestido de esta humilde condición, nadie hubiera logrado llegar hasta él por sus propios méritos.
	
	Por eso, la misma magnitud del beneficio otorgado exige de nosotros una veneración proporcionada a la excelsitud de esta dádiva. Y, como el bienaventurado Apóstol nos enseña, \emph{no hemos recibido el espíritu de este mundo, sino el Espíritu que procede de Dios}, a fin de que conozcamos lo que Dios nos ha otorgado; y el mismo Dios sólo acepta como culto piadoso el ofrecimiento de lo que él nos ha concedido.
	
	¿Y qué podremos encontrar en el tesoro de la divina largueza tan adecuado al honor de la presente festividad como la paz, lo primero que los ángeles pregonaron en el nacimiento del Señor?
	
	La paz es la que engendra los hijos de Dios, alimenta el amor y origina la unidad, es el descanso de los bienaventurados y la mansión de la eternidad. El fin propio de la paz y su fruto específico consiste en que se unan a Dios los que el mismo Señor separa del mundo.
	
	Que los que \emph{no han nacido de sangre, ni de amor carnal, ni de amor humano, sino de Dios}, ofrezcan, por tanto, al Padre la concordia que es propia de hijos pacíficos, y que todos los miembros de la adopción converjan hacia el Primogénito de la nueva creación, que vino a cumplir la voluntad del que le enviaba y no la suya: puesto que la gracia del Padre no adoptó como herederos a quienes se hallaban en discordia e incompatibilidad, sino a quienes amaban y sentían lo mismo. Los que han sido reformados de acuerdo con una sola imagen deben ser concordes en el espíritu.
	
	El nacimiento del Señor es el nacimiento de la paz: y así dice el Apóstol: \emph{El es nuestra paz; él ha hecho de los dos pueblos una sola cosa,} ya que, tanto los judíos como los gentiles, por su medio \emph{podemos acercarnos al Padre con un mismo Espíritu}.
\end{body}
