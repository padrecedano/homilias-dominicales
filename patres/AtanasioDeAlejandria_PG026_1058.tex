\subsection{San Atanasio de Alejandría, obispo}

\ptheme{La Palabra tomó de María nuestra condición}

\src{Carta a Epicteto, 5-9: PG 26, 1058. 1062-1066\cite{AtanasioDeAlejandria_PG026_1058}}

\begin{body}
	\ltr{L}{a} Palabra \emph{tendió una mano a los hijos de Abrahán,} como afirma el Apóstol, \emph{y por eso tenía que parecerse en todo a sus hermanos} y asumir un cuerpo semejante al nuestro. Por esta razón, en verdad, María está presente en este misterio, para que de ella la Palabra tome un cuerpo, y, como propio, lo ofrezca por nosotros. La Escritura habla del parto y afirma: \emph{Lo envolvió en pañales;} y se proclaman dichosos los pechos que amamantaron al Señor, y, por el nacimiento de este primogénito, fue ofrecido el sacrificio prescrito. El ángel Gabriel había anunciado esta concepción con palabras muy precisas, cuando dijo a María no simplemente \textquote{\emph{lo que nacerá en ti}} ---para que no se creyese que se trataba de un cuerpo introducido desde el exterior---, sino \emph{de} para que creyéramos que aquel que era engendrado en María procedía realmente de ella.
	
	Las cosas sucedieron de esta forma para que la Palabra, tomando nuestra condición y ofreciéndola en sacrificio, la asumiese completamente, y revistiéndonos después a nosotros de su condición, diese ocasión al Apóstol para afirmar lo siguiente: \emph{Esto corruptible tiene que vestirse de incorrupción, y esto mortal tiene que vestirse de inmortalidad}.
	
	Estas cosas no son una ficción, como algunos juzgaron; ¡tal postura es inadmisible! Nuestro Salvador fue verdaderamente hombre, y de él ha conseguido la salvación el hombre entero. Porque de ninguna forma es ficticia nuestra salvación ni afecta sólo al cuerpo, sino que la salvación de todo el hombre, es decir, alma y cuerpo, se ha realizado en aquel que es la Palabra.
	
	Por lo tanto, el cuerpo que el Señor asumió de María era un verdadero cuerpo humano, conforme lo atestiguan las Escrituras; verdadero, digo, porque fue un cuerpo igual al nuestro. Pues María es nuestra hermana, ya que todos nosotros hemos nacido de Adán.
	
	Lo que Juan afirma: \emph{La Palabra se hizo carne,} tiene la misma significación, como se puede concluir de la idéntica forma de expresarse. En san Pablo encontramos escrito: \emph{Cristo se hizo por nosotros un maldito}. Pues al cuerpo humano, por la unión y comunión con la Palabra, se le ha concedido un inmenso beneficio: de mortal se ha hecho inmortal, de animal se ha hecho espiritual, y de terreno ha penetrado las puertas del cielo.
	
	Por otra parte, la Trinidad, también después de la encarnación de la Palabra en María, siempre sigue siendo la Trinidad, no admitiendo ni aumentos ni disminuciones; siempre es perfecta, y en la Trinidad se reconoce una única Deidad, y así la Iglesia confiesa a un único Dios, Padre de la Palabra.
\end{body}
