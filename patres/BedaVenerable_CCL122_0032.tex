\subsection{San Beda el Venerable, presbítero}

\ptheme{¡Oh grande e insondable misterio!}

\src{Homilía 5 en la vigilia de Navidad: CCL 122, 32-36.\cite{BedaVenerable_CCL122_0032}}
	
	\begin{body}
		\ltr{E}{n} breves palabras, pero llenas de verismo, describe el evangelista san Mateo el nacimiento del Señor y Salvador nuestro Jesucristo, por el que el Hijo de Dios, eterno antes del tiempo, apareció en el tiempo Hijo del hombre. Al conducir el evangelista la serie genealógica partiendo de Abrahán para acabar en José, el esposo de María, y enumerar ---según el acostumbrado orden de la humana generación--- la totalidad así de los genitores como de los engendrados, y disponiéndose a hablar del nacimiento de Cristo, subrayó la enorme diferencia existente entre éste y el resto de los nacimientos: los demás nacimientos se producen por la normal unión del hombre y de la mujer mientras que él, por ser Hijo de Dios, vino al mundo por conducto de una Virgen. Y era conveniente bajo todos los aspectos que, al decidir Dios hacerse hombre para salvar a los hombres, no naciera sino de una virgen, pues era inimaginable que una virgen engendrara a ningún otro, sino a uno que, siendo Dios, ella lo procreara como Hijo.
		
		\emph{Mirad} ---dice--- \emph{la Virgen está encinta y dará a luz un hijo, y le pondrá por nombre Emmanuel (que significa \textquote{Dios-con-nosotros})}. El nombre que el profeta da al Salvador, \textquote{\emph{Dios-con-nosotros}}, indica la doble naturaleza de su única persona. En efecto, el que es Dios nacido del Padre antes de los tiempos, es el mismo que, en la plenitud de los tiempos, se convirtió, en el seno materno, en Emmanuel, esto es, en \textquote{\emph{Dios-con-nosotros}}, ya que se dignó asumir la fragilidad de nuestra naturaleza en la unidad de su persona, cuando \emph{la Palabra se hizo carne y acampó entre nosotros}, esto es, cuando de modo admirable comenzó a ser lo que nosotros somos, sin dejar de ser lo que era asumiendo de forma tal nuestra naturaleza que no le obligase a perder lo que él era.
		
		Dio, pues, a luz María a su hijo primogénito, es decir, al hijo de su propia carne; dio a luz al que, antes de la creación, había nacido Dios de Dios, y en la humanidad en que fue creado, superaba ampliamente a toda creatura. Y él \emph{le puso} ---dice--- \emph{por nombre Jesús}.
		
		Jesús es el nombre del hijo que nació de la virgen, nombre que significa ---según la explicación del ángel--- que él iba a \emph{salvar a su pueblo de los pecados}. Y el que salva de los pecados, salvará igualmente de las corruptelas de alma y cuerpo, secuela del pecado.
		
		La palabra \textquote{\emph{Cristo}} connota la dignidad sacerdotal o regia. En la ley, tanto los sacerdotes como los reyes eran llamados \textquote{\emph{cristos}} por el crisma, es decir, por la unción con el óleo sagrado: eran un signo de quien, al manifestarse en el mundo como verdadero Rey y Pontífice, fue ungido \emph{con aceite de júbilo entre todos sus compañeros}.
		
		Debido a esta unción o crisma, se le llama \emph{Cristo}; a los que participan de esta unción, es decir, de esta gracia espiritual, se les llama \textquote{\emph{cristianos}}. Que él, por ser nuestro Salvador, nos salve de los pecados; en cuanto Pontífice, nos reconcilie con Dios Padre; en su calidad de Rey se digne darnos el reino eterno de su Padre, Jesucristo nuestro Señor, que con el Padre y el Espíritu Santo vive y reina y es Dios por todos los siglos de los siglos. Amén.
	\end{body}