\subsection{San Pedro Crisólogo, obispo}

\ptheme{Mirad: la virgen concebirá y dará a luz un hijo, y le pondrá por nombre Emmanuel}

\src{Sermón 145: PL 52, 588\cite{PedroCrisologo_PL052_0588}.}

\begin{body}
	\ltr{E}{s} mi propósito hablaros hoy, hermanos, de cómo nos relata el santo evangelista el misterio de la generación de Cristo. Dice: \emph{El nacimiento de Jesucristo fue de esta manera: La madre de Jesús estaba desposada con José y, antes de vivir juntos, resultó que ella esperaba un hijo, por obra del Espíritu Santo. José, su esposo, que era bueno y no quería denunciarla, decidió repudiarla en secreto}. Pero, ¿cómo se compagina esta bondad con la resolución de no discutir la gravidez de su esposa? Las virtudes no pueden sostenerse separadamente: la equidad desprovista de bondad se convierte en severidad, y la justicia sin piedad se torna crueldad. Con razón, pues, a José se le califica de bueno, porque era piadoso; y de piadoso, por ser bueno. Al pensar piadosamente, se libra de la crueldad; al juzgar benignamente, observó la justicia; al no querer erigirse en acusador, rehuyó la sentencia. Se requemaba el alma del justo perpleja ante la novedad del evento: tenía ante sí una esposa preñada, pero virgen; grávida del don no menos que del pudor; solícita por lo concebido, pero segura de su integridad; revestida de la función maternal, sin perder el decoro virginal.
	
	¿Cuál debía ser la conducta del esposo ante tal situación? ¿Acusarla de infidelidad? No, pues él era testigo de su inocencia. ¿Airear la culpa? Tampoco, pues era él el guardián de su pudor. ¿Inculparla de adulterio? Menos aún, pues estaba plenamente convencido de su virginidad. ¿Qué hacer, entonces? Piensa repudiarla en secreto, pues ni podía ir por ahí aireando lo sucedido ni ocultarlo en la intimidad del hogar. Decide repudiarla en secreto y confía a Dios todo el negocio, ya que nada tiene que comunicar a los hombres.
	
	\emph{José, hijo de David, no tengas reparo en llevarte a María, tu mujer, porque la criatura que hay en ella viene del Espíritu Santo. Dará a luz un hijo y tú le pondrás por nombre Jesús, porque él salvará al pueblo de los pecados}. Veis, hermanos, cómo una sola persona representa toda una raza, veis cómo un solo individuo lleva la representación de toda una estirpe, veis cómo en José se da cita la serie genealógica de David.
	
	\emph{José, hijo de David}. Nacido de la vigésima octava generación, ¿por qué se le llama hijo de David, sino porque en él se desvela el misterio de una estirpe, se cumple la fidelidad de la promesa, y en la carne virginal luce ya el sello de la sobrenatural concepción de un parto celeste? La promesa de Dios Padre hecha a David estaba expresada en estos términos: \emph{El Señor ha jurado a David una promesa que no retractará: \textquote{A uno de tu linaje pondré sobre tu trono}}. Del fruto de tu vientre: sí, fruto de tu vientre, de tu seno, sí, porque el huésped celeste, el supremo morador de tal modo descendió al receptáculo de tus entrañas que ignoró las limitaciones corporales; y de tal modo salió del claustro materno, que dejó intacto el sello de la virginidad, cumpliéndose de esta manera lo que se canta en el Cantar de los cantares: \emph{Eres jardín cerrado, hermana y novia mía; eres jardín cerrado, fuente sellada}.
	
	\emph{La criatura que hay en ella viene del Espíritu Santo}. Concibió la virgen, pero del Espíritu; parió la virgen, pero a aquel que había predicho Isaías: \emph{Mirad: la virgen concebirá y dará a luz un hijo, y le pondrá por nombre Emmanuel (que significa \textquote{Dios-con-nosotros})}.
\end{body}