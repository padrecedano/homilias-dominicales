\subsection{Beato Elredo de Rievaulx, abad}

\ptheme{Hoy nos ha nacido un Salvador}

\src{Sermón 1 de la Natividad del Señor: \\PL 195, 226-227.\cite{ElredoRievaulx_PL195_226}}

\begin{body}
	\ltr{H}{oy,} \emph{en la ciudad de David, nos ha nacido un Salvador: El Mesías, el Señor}. La ciudad de que aquí se habla es Belén, a la que debemos acudir corriendo, como lo hicieron los pastores, apenas oído este rumor. Así es como soléis cantar ---en el himno de María, la Virgen---: \textquote{Cantaron gloria a Dios, corrieron a Belén}. \emph{Y aquí tenéis la señal: encontraréis un niño envuelto en pañales y acostado en un pesebre}.
	
	Ved por qué os dije que debéis amar. Teméis al Señor de los ángeles, pero amadle chiquitín; teméis al Señor de la majestad, pero amadle envuelto en pañales; teméis al que reina en el cielo, pero amadle acostado en un pesebre. Y ¿cuál fue la señal que recibieron los pastores? \emph{Encontraréis un niño envuelto en pañales y acostado en un pesebre}. El es el Salvador, él es el Señor. Pero, ¿qué tiene de extraordinario ser envuelto en pañales y yacer en un establo? ¿No son también los demás niños envueltos en pañales? Entonces, ¿qué clase de señal es ésta? Una señal realmente grande, a condición de que sepamos comprenderla. Y la comprendemos si no nos limitamos a escuchar este mensaje de amor, sino que, además, albergamos en nuestro corazón aquella claridad que apareció junto con los ángeles. Y si el ángel se apareció envuelto en claridad, cuando por primera vez anunció este rumor, fue para enseñarnos que sólo escuchan de verdad, los que acogen en su alma la claridad espiritual.
	
	Podríamos decir muchas cosas sobre esta señal, pero como el tiempo corre, insistiré brevemente en este tema. Belén, \textquote{casa del pan}, es la santa Iglesia, en la cual se distribuye el cuerpo de Cristo, a saber, el pan verdadero. El pesebre de Belén se ha convertido en el altar de la Iglesia. En él se alimentan los animales de Cristo. De esta mesa se ha escrito: \emph{Preparas una mesa ante mí}. En este pesebre está Jesús envuelto en pañales. La envoltura de los pañales es la cobertura de los sacramentos. En este pesebre y bajo las especies de pan y vino está el verdadero cuerpo y la sangre de Cristo. En este sacramento creemos que está el mismo Cristo; pero está envuelto en pañales, es decir, invisible bajo los signos sacramentales. No tenemos señal más grande y más evidente del nacimiento de Cristo como el hecho de que cada día sumimos en el altar santo su cuerpo y su sangre; como el comprobar que a diario se inmola por nosotros, el que por nosotros nació una vez de la Virgen.
	
	Apresurémonos, hermanos, al pesebre del Señor; pero antes y en la medida de lo posible, preparémonos con su gracia para este encuentro de suerte que asociados a los ángeles, \emph{con corazón limpio, con una conciencia honrada y con una fe sentida,} cantemos al Señor con toda nuestra vida y toda nuestra conducta: \emph{Gloria a Dios en el cielo, y en la tierra, paz a los hombres que Dios ama}. Por el mismo Jesucristo, nuestro Señor, a quien sea el honor y la gloria por los siglos de los siglos. Amén.
\end{body}