\subsection{San Juan Pablo II, papa}

\subsubsection{Catequesis: La genealogía de Mateo}

\src{Catequesis en la Audiencia general del 28 de enero de 1987, nn. 5-10.\cite{JuanPabloII_Catequesis_19870128}}

\begin{body}
	\ltr{E}{l} Evangelio \emph{según Mateo} completa la narración de Lucas describiendo algunas circunstancias que precedieron al nacimiento de Jesús. Leemos: \textquote{La concepción de Jesucristo fue así: Estando desposada María, su Madre, con José, \emph{antes de que conviviesen} se halló \emph{haber concebido María del Espíritu Santo}. José, su esposo, siendo justo, no quiso denunciarla y resolvió repudiarla en secreto. Mientras reflexionaba sobre esto, he aquí que se le apareció en sueños un ángel del Señor y le dijo: José, hijo de David, no temas recibir en tu casa a María, tu esposa, pues \emph{lo concebido en ella es obra del Espíritu Santo}. Dará a luz un hijo a quien pondrás por nombre Jesús, porque salvará a su pueblo de sus pecados} (\emph{Mt} 1, 18-21 ).
	
	Como se ve, ambos textos del \textquote{Evangelio de la infancia concuerdan en la constatación fundamental}: Jesús fue concebido por obra del Espíritu Santo y nació de María Virgen; y son entre sí \emph{complementarios} en el esclarecimiento de las circunstancias de este acontecimiento extraordinario: Lucas respecto a María, Mateo respecto a José.
	
	Para identificar \emph{la fuente de la que deriva el Evangelio de la infancia}, hay que referirse a la frase de San Lucas: \textquote{\emph{María guardaba todo esto} y lo meditaba en su corazón} (\emph{Lc} 2, 19). Lucas lo dice dos veces: después de marchar los pastores de Belén y después del encuentro de Jesús en el templo (cf. 2, 51). El Evangelista mismo nos ofrece los elementos para identificar en la Madre de Jesús una de las fuentes de información utilizadas por él para escribir el \textquote{Evangelio de la infancia}. María, que \textquote{guardó todo esto en su corazón} (cf. \emph{Lc} 2, 19), pudo dar testimonio, después de la muerte y resurrección de Cristo, de lo que se refería a la propia persona y a la función de Madre precisamente en el período apostólico, en el que nacieron los textos del Nuevo Testamento y tuvo origen la primera tradición cristiana.
	
	El testimonio evangélico de \emph{la concepción virginal de Jesús} por parte de María es de gran relevancia teológica. Pues constituye un signo especial \emph{del origen divino del Hijo de María}. El que Jesús no tenga un padre terreno porque ha sido engendrado \textquote{sin intervención de varón}, pone de relieve la verdad de que Él es el Hijo de Dios, de modo que cuando asume la naturaleza humana, su Padre continúa siendo exclusivamente Dios.
	
	La revelación de la intervención del Espíritu Santo \emph{en la concepción de Jesús}, indica \emph{el comienzo} en la historia del hombre de la nueva generación espiritual que tiene un carácter estrictamente sobrenatural (cf. \emph{1 Cor} 15, 45-49). De este modo Dios Uno y Trino \textquote{se comunica} a la criatura mediante el Espíritu Santo. Es el misterio al que se pueden aplicar las palabras del Salmo: \textquote{Envía tu Espíritu, y serán creados, y renovarás la faz de la tierra} (\emph{Sal} 103 {[}104{]}, 30). En la economía de esa comunicación de Sí mismo que Dios hace a la criatura, la concepción virginal de Jesús, que sucedió por obra del Espíritu Santo, es un \emph{acontecimiento central y culminante}. Él \emph{inicia la \textquote{nueva creación}}. Dios entra así en un modo decisivo en la historia para actuar el destino sobrenatural del hombre, o sea, la predestinación de todas las cosas en Cristo. Es \emph{la expresión} definitiva del \emph{Amor salvífico} de Dios al hombre, del que hemos hablado en las catequesis sobre la Providencia.
	
	En la actuación del plan de la salvación hay siempre una participación de la criatura. Así en la concepción de Jesús por obra del Espíritu Santo \emph{María participa} de forma \emph{decisiva}. Iluminada interiormente por el mensaje del ángel sobre su vocación de Madre y sobre la conservación de su virginidad, María \emph{expresa su voluntad y consentimiento} y acepta hacerse el humilde instrumento de la \textquote{virtud del Altísimo}. La acción del Espíritu Santo hace que en María la maternidad y la virginidad estén presentes de un modo que, aunque inaccesible a la mente humana, entre de lleno en el ámbito de la predilección de la omnipotencia de Dios. En María se cumple la gran profecía de Isaías: \textquote{La virgen grávida da a luz} (7, 14; cf. \emph{Mt} 1, 22-23); su virginidad, signo en el Antiguo Testamento de la pobreza y de disponibilidad total al plan de Dios, se convierte en el terreno de la acción excepcional de Dios, que escoge a María para ser Madre del Mesías.
	
	La excepcionalidad de María se deduce también de las genealogías aducidas por Mateo y Lucas.
	
	El Evangelio \emph{según Mateo} comienza, conforme a la costumbre hebrea\emph{, con la genealogía de Jesús} (\emph{Mt} 1, 2-17) y hace un elenco partiendo de Abraham, de las generaciones masculinas. A Mateo de hecho, le importa poner de relieve, mediante la paternidad \emph{legal} de José, la descendencia de Jesús de Abraham y David y, por consiguiente, la legitimidad de su calificación de Mesías. Sin embargo, al final de la serie de los ascendientes leemos: \textquote{Y Jacob engendró a José esposo de María, \emph{de la cual nació Jesús llamado Cristo}} (\emph{Mt} 1, 16). Poniendo el acento en la maternidad de María, el Evangelista implícitamente subraya la verdad del nacimiento virginal: Jesús, como hombre, no tiene padre terreno.
	
	\emph{Según el Evangelio de Lucas}, la genealogía de Jesús (\emph{Lc} 3, 23-38) es ascendente: desde Jesús a través de sus antepasados se remonta \emph{hasta Adán}. El Evangelista ha querido mostrar la vinculación de Jesús \emph{con todo el género humano}. María, como colaboradora de Dios en dar a su Eterno Hijo la naturaleza humana, ha sido el instrumento de la unión de Jesús con toda la humanidad.
\end{body}
