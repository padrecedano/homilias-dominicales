\subsection{San Basilio Magno, obispo}

\ptheme{El Verbo se hizo carne y puso su morada entre nosotros}

\src{Homilía 2, 6: \\PG 31, 1459-1462. 1471-1474.\cite{BasilioMagno_PG031_1459}}

\begin{body}
	\ltr{D}{ios} en la tierra, Dios en medio de los hombres, no un Dios que consigna la ley entre resplandores de fuego y ruido de trompetas sobre un monte humeante, o en densa nube entre relámpagos y truenos, sembrando el terror entre quienes escuchan; sino un Dios encarnado, que habla a las creaturas de su misma naturaleza con suavidad y dulzura. Un Dios encarnado, que no obra desde lejos o por medio de profetas, sino a través de la humanidad asumida para revestir su persona, para reconducir a sí, en nuestra misma carne hecha suya, a toda la humanidad. ¿Cómo, por medio de uno solo, el resplandor alcanza a todos? ¿Cómo la divinidad reside en la carne? Como el fuego en el hierro: no por transformación, sino por participación. El fuego, efectivamente, no pasa al hierro: permaneciendo donde está, le comunica su virtud; ni por esta comunicación disminuye, sino que invade con lo suyo a quien se comunica. Así el Dios-Verbo, sin jamás separarse de sí mismo \emph{puso su morada en medio de nosotros;} sin sufrir cambio alguno \emph{se hizo carne;} el cielo que lo contenía no quedó privado de él mientras la tierra lo acogió en su seno.
	
	Busca penetrar en el misterio: Dios asume la carne justamente para destruir la muerte oculta en ella. Como los antídotos de un veneno, una vez ingeridos, anulan sus efectos, y como las tinieblas de una casa se disuelven a la luz del sol, la muerte que dominaba sobre la naturaleza humana fue destruida por la presencia de Dios. Y como el hielo permanece sólido en el agua mientras dura la noche y reinan las tinieblas, pero prontamente se diluye al calor del sol, así la muerte reinante hasta la venida de Cristo, apenas resplandeció la gracia de Dios Salvador y surgió el sol de justicia, \emph{fue engullida por la victoria} (1Co 15, 54), no pudiendo coexistir con la Vida. ¡Oh grandeza de la bondad y del amor de Dios por los hombres!
	
	Démosle gloria con los pastores, exultemos con los ángeles \emph{porque hoy ha nacido el Salvador, Cristo el Señor} (Lc 2, 11). Tampoco a nosotros se apareció el Señor en forma de Dios, porque habría asustado a nuestra fragilidad, sino en forma de siervo, para restituir a la libertad a los que estaban en la esclavitud. ¿Quién es tan tibio, tan poco reconocido que no goce, no exulte, no lleve dones? Hoy es fiesta para toda creatura. No haya nadie que no ofrezca algo, nadie se muestre ingrato. Estallemos también nosotros en un canto de exultación.
\end{body}