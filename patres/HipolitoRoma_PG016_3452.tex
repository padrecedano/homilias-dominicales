\subsection{San Hipólito de Roma, presbítero}

\subsubsection{Tratado: La Palabra hecha carne nos diviniza}

\src{Contra las herejías, Cap. 10, 33-34: \\PG 16, 3452-3453.\cite{HipolitoRoma_PG016_3452}}

\begin{body}
	\ltr{N}{o} prestamos nuestra adhesión a discursos vacíos ni nos dejamos seducir por pasajeros impulsos del corazón, como tampoco por el encanto de discursos elocuentes, sino que nuestra fe se apoya en las palabras pronunciadas por el poder divino. Dios se las ha ordenado a su Palabra, y la Palabra las ha pronunciado, tratando con ellas de apartar al hombre de la desobediencia, no dominándolo como a un esclavo por la violencia que coacciona, sino apelando a su libertad y plena decisión.
	
	Fue el Padre quien envió la Palabra, al fin de los tiempos. Quiso que no siguiera hablando por medio de un profeta, ni que se hiciera adivinar mediante anuncios velados; sino que le dijo que se manifestara a rostro descubierto, a fin de que el mundo, al verla, pudiera salvarse.
	
	Sabemos que esta Palabra tomó un cuerpo de la Virgen, y que asumió al hombre viejo, transformándolo. Sabemos que se hizo hombre de nuestra misma condición, porque, si no hubiera sido así, sería inútil que luego nos prescribiera imitarle como maestro. Porque, si este hombre hubiera sido de otra naturaleza, ¿cómo habría de ordenarme las mismas cosas que él hace, a mí, débil por nacimiento, y cómo sería entonces bueno y justo?
	
	Para que nadie pensara que era distinto de nosotros, se sometió a la fatiga, quiso tener hambre y no se negó a pasar sed, tuvo necesidad de descanso y no rechazó el sufrimiento, obedeció hasta la muerte y manifestó su resurrección, ofreciendo en todo esto su humanidad como primicia, para que tú no te descorazones en medio de tus sufrimientos, sino que, aun reconociéndote hombre, aguardes a tu vez lo mismo que Dios dispuso para él.
	
	Cuando contemples ya al verdadero Dios, poseerás un cuerpo inmortal e incorruptible, junto con el alma, y obtendrás el reino de los cielos, porque, sobre la tierra, habrás reconocido al Rey celestial; serás íntimo de Dios, coheredero de Cristo, y ya no serás más esclavo de los deseos, de los sufrimientos y de las enfermedades, porque habrás llegado a ser dios.
	
	Porque todos los sufrimientos que has soportado, por ser hombre, te los ha dado Dios precisamente porque lo eras; pero Dios ha prometido también otorgarte todos sus atributos, una vez que hayas sido divinizado y te hayas vuelto inmortal. Es decir, \emph{conócete a ti mismo} mediante el conocimiento de Dios, que te ha creado, porque conocerlo y ser conocido por él es la suerte de su elegido.
	
	No seáis vuestros propios enemigos, ni os volváis hacia atrás, porque Cristo es \emph{el Dios que está por encima de todo:} él ha ordenado purificar a los hombres del pecado, y él es quien renueva al hombre viejo, al que ha llamado desde el comienzo imagen suya, mostrando, por su impronta en ti, el amor que te tiene. Y, si tú obedeces sus órdenes y te haces buen imitador de este buen maestro, llegarás a ser semejante a él y recompensado por él; porque Dios no es pobre, y te divinizará para su gloria.
\end{body}