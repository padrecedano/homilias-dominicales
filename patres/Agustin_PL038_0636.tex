\subsection{San Agustín, obispo}

\ptheme{Convertíos, porque está cerca el Reino de los cielos}

\src{Sermón 109, 1: PL 38, 636.\cite{Agustin_PL038_0636}}

\begin{body}
	\ltr{H}{emos} escuchado el evangelio y en el evangelio al Señor descubriendo la ceguera de quienes son capaces de interpretar el aspecto del cielo, pero son incapaces de discernir el tiempo de la fe en un reino de los cielos que está ya llegando. Les decía esto a los judíos, pero sus palabras nos afectan también a nosotros. Y el mismo Jesucristo comenzó así la predicación de su evangelio: \emph{Convertíos, porque está cerca el Reino de los cielos}. Igualmente, Juan el Bautista, su Precursor, comenzó así: \emph{Convertíos, porque está cerca el Reino de los cielos}. Y ahora corrige el Señor a los que se niegan a convertirse, próximo ya el Reino de los cielos. \emph{El Reino de los cielos} ---como él mismo dice--- \emph{no vendrá espectacularmente}. Y añade: \emph{El Reino de Dios está dentro de vosotros}. 
	
	Que cada cual reciba con prudencia las admoniciones del preceptor, si no quiere perder la hora de misericordia del Salvador, misericordia que se otorga en la presente coyuntura, en que al género humano se le ofrece el perdón. Precisamente al hombre se le brinda el perdón para que se convierta y no haya a quien condenar. Eso lo ha de decidir Dios cuando llegue el fin del mundo; pero de momento nos hallamos en el tiempo de la fe. Si el fin del mundo encontrará o no aquí a alguno de nosotros, lo ignoro; posiblemente no encuentre a ninguno. Lo cierto es que el tiempo de cada uno de nosotros está cercano, pues somos mortales. Andamos en medio de peligros. Nos asustan más las caídas que si fuésemos de vidrio. ¿Y hay algo más frágil que un vaso de cristal? Y sin embargo se conserva y dura siglos. Y aunque pueda temerse la caída de un vaso de cristal, no hay miedo de que le afecte la vejez o la fiebre. 
	
	Somos, por tanto, más frágiles que el cristal porque debido indudablemente a nuestra propia fragilidad, cada día nos acecha el temor de los numerosos y continuos accidentes inherentes a la condición humana; y aunque estos temores no lleguen a materializarse, el tiempo corre: y el hombre que puede evitar un golpe, ¿podrá también evitar la muerte? Y si logra sustraerse a los peligros exteriores, ¿logrará evitar asimismo los que vienen de dentro? Unas veces son los virus que se multiplican en el interior del hombre, otras es la enfermedad que súbitamente se abate sobre nosotros; y aun cuando logre verse libre de estas taras, acabará finalmente por llegarle la vejez, sin moratoria posible.
\end{body}