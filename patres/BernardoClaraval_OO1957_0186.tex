\subsection{San Bernardo de Claraval, abad}

\subsubsection{Sermón: Habitaré y caminaré con ellos}

\src{Sermón 27, 7.9 sobre el Cantar de los cantares: \\Opera omnia. Edit. Cister. 1957, I, 186-188\cite{BernardoClaraval_OO1957_0186}.}

\begin{body}
	\ltr{L}{uego} que el divino Emmanuel implantó en la tierra el magisterio de la doctrina celeste, luego que por Cristo y en Cristo se nos manifestó la imagen visible de aquella celestial Jerusalén, que es nuestra madre, y el esplendor de su belleza, ¿qué es lo que contemplamos sino a la Esposa en el Esposo, admirando en el mismo y único Señor de la gloria al Esposo que se ciñe la corona y a la Esposa que se adorna de sus joyas? El que bajó es efectivamente el mismo que subió, pues nadie ha subido al cielo sino el que bajó del cielo: el mismo y único Señor, esposo en la cabeza y esposa en el cuerpo. Y no en vano apareció en la tierra el hombre celestial, él que convirtió en celestiales a muchos hombres terrenales haciéndolos semejantes a él, para que se cumpliera lo que dice la Escritura: \emph{Igual que el celestial son los hombres celestiales}.
	
	Desde entonces en la tierra se vive como en el cielo, pues, a imitación de aquella soberana y dichosa criatura, también ésta que viene desde los confines de la tierra a escuchar la sabiduría de Salomón, se une al esposo celeste con un vínculo de casto amor; y si bien no le está todavía como aquélla unida por la visión, está ya desposada por la fe, según la promesa de Dios, que dice por el profeta: \emph{Me casaré contigo en derecho y justicia, en misericordia y compasión, me casaré contigo en fidelidad}. Por eso se afana en conformarse más y más al modelo celestial, aprendiendo de él a ser modesta y sobria, pudorosa y santa, paciente y compasiva, aprendiendo finalmente a ser mansa y humilde de corazón. Con semejante conducta procura agradar, aunque ausente, a aquel a quien los ángeles desean contemplar, a fin de que, inflamada de angélico ardor, se comporte como ciudadana de los santos y miembro de la familia de Dios, se comporte como la amada, se comporte como la esposa.
	
	\emph{Ven, mi elegida, y pondré en ti mi trono}. ¿Por qué te acongojas ahora, alma mía, por qué te me turbas? ¿Crees que podrás disponer en tu interior un lugar para el Señor? Y ¿qué lugar, en nuestro interior, podrá parecernos idóneo para tanta gloria, capaz de tamaña majestad? ¡Ojalá pudiera merecer siquiera adorar al estrado de sus pies! ¡Quién me diera seguir al menos las huellas de cualquier alma santa, que el Señor se escogió como heredad! Mas si él se dignase derramar en mi alma la unción de su misericordia, y dilatarla como se dilata una piel engrasada, de modo que también yo pudiera decir: \emph{Correré por el camino de tus mandatos cuando me ensanches el corazón,} quizá pudiera a mi vez mostrarle en mí mismo, si no una sala grande arreglada con divanes, donde pueda sentarse a la mesa con sus discípulos, sí al menos un lugar donde pueda reclinar su cabeza. Veo en lontananza a aquellas almas realmente dichosas, de las cuales se ha dicho: \emph{Habitaré y caminaré con ellos}.
\end{body}