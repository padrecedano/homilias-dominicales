\begin{patercite}
	La estrella vino a pararse encima de donde estaba el niño. Por lo cual, los magos, al ver la estrella, se llenaron de inmensa alegría. Recibamos también nosotros esa inmensa alegría en nuestros corazones. Es la alegría que los ángeles anuncian a los pastores. Adoremos con los Magos, demos gloria con los pastores, dancemos con los ángeles. \emph{Porque hoy ha nacido un Salvador: el Mesías, el Señor. El Señor es Dios: él nos ilumina,} pero no en la condición divina, para atemorizar nuestra debilidad, sino en la condición de esclavo, para gratificar con la libertad a quienes gemían bajo la esclavitud. ¿Quién es tan insensible, quién tan ingrato, que no se alegre, que no exulte, que no se recree con tales noticias? Esta es una fiesta común a toda la creación: se le otorgan al mundo dones celestiales, el arcángel es enviado a Zacarías y a María, se forma un coro de ángeles, que cantan: \emph{Gloria a Dios en el cielo, y en la tierra, paz a los hombres que Dios ama}.
	
	Las estrellas se descuelgan del cielo, unos Magos abandonan la paganía, la tierra lo recibe en una gruta. Que todos aporten algo, que ningún hombre se muestre desagradecido. Festejemos la salvación del mundo, celebremos el día natalicio de la naturaleza humana. Hoy ha quedado cancelada la deuda de Adán. Ya no se dirá en adelante: \emph{Eres polvo y al polvo volverás,} sino: \textquote{Unido al que viene del cielo, serás admitido en el cielo}. Ya no se dirá más: \emph{Parirás hijos con dolor,} pues es dichosa la que dio a luz al Emmanuel y los pechos que le alimentaron. Precisamente por esto \emph{un niño nos ha nacido, un hijo se nos ha dado: lleva a hombros el principado}.
	
	(\textbf{San Basilio Magno}, \emph{Homilía} sobre la generación de Cristo: PG 31, 1471-1475)\cite{BasilioMagno_PG031_1471}.
\end{patercite}