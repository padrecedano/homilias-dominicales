\subsection{San Juan Pablo II, papa}

\subsubsection{Catequesis (1983)}

\src{Audiencia general. \\Miércoles 28 de diciembre de 1983.\cite{JuanPabloII_Catequesis_19831228}}

\begin{body}
	\ltr[1. ]{E}{l} misterio de Navidad hace resonar en nuestros oídos el canto con que el cielo quiere hacer participar a la tierra en el gran acontecimiento de la Encarnación: \textquote{Gloria a Dios en las alturas y paz en la tierra a los hombres de buena voluntad} (\emph{Lc} 2, 14).
	
	\emph{La paz es anunciada por toda la tierra}. No es una paz que los hombres consigan conquistar con sus fuerzas. \emph{Viene de lo alto} como don maravilloso de Dios a la humanidad. No podemos olvidar que, si todos debemos trabajar para instaurar la paz en el mundo, antes de nada debemos abrirnos al don divino de la paz poniendo toda nuestra confianza en el Señor.
	
	Según el cántico de Navidad, la paz prometida a la tierra \emph{está ligada al amor que Dios trae a los hombres}. Los hombres son llamados \textquote{hombres de buena voluntad} porque ya la buena voluntad divina les pertenece. El nacimiento de Jesús es el testimonio irrefutable y definitivo de esta buena voluntad que jamás será retirada de la humanidad.
	
	Este nacimiento pone de manifiesto \emph{la voluntad divina de reconciliación}: Dios desea reconciliar consigo al mundo pecador, perdonando y cancelando los pecados. Ya en el anuncio del nacimiento el ángel había expresado esta voluntad reconciliadora indicando el nombre que debía llevar el Niño: Jesús, o sea, \textquote{Dios salva}. \textquote{Porque salvará a su pueblo de sus pecados}, comenta el ángel (Mt 1, 21). El nombre revela el destino y la misión del Niño juntamente con su personalidad: es el Dios que salva, el que libera a la humanidad de la esclavitud del pecado y, por ello, restablece las relaciones amistosas del hombre con Dios.
	
	2. El acontecimiento que da a la humanidad un Dios Salvador supera en gran medida las expectativas del pueblo judío. Este pueblo esperaba la salvación, esperaba al Mesías, a un rey ideal del futuro que debía establecer sobre la tierra el reino de Dios. A pesar de que la esperanza judía había puesto muy en lo alto a este Mesías, para ellos no era más que un hombre.
	
	La gran novedad de la venida del Salvador consiste en el hecho de que Él es Dios y hombre a la vez. Lo que el judaísmo no había podido concebir ni esperar, es decir, un Hijo de Dios hecho hombre, se realiza en el misterio de la Encarnación. \emph{El cumplimiento es mucho más maravilloso que la promesa}.
	
	Esta es la razón por la que no podemos medir la grandeza de Jesús sólo con los oráculos proféticos del Antiguo Testamento. Cuando Él realiza estos oráculos se mueve a un nivel trascendente. Todos los tentativos de encerrar a Jesús en los límites de una personalidad humana, no tienen en cuenta lo que hay de esencial en la revelación de la Nueva Alianza: la persona divina del Hijo que se ha hecho hombre o, según la palabra de San Juan, del Verbo que se ha hecho carne y ha venido a habitar entre nosotros (cf. 1, 14). Aquí aparece la grandiosidad generosa del plan divino de salvación. El Padre ha enviado a su Hijo que es Dios como Él. No se ha limitado a enviar a siervos, a hombres que hablasen en su nombre como los Profetas. Ha querido testimoniar a la humanidad el máximo de amor y le ha hecho la sorpresa de darle un Salvador que poseía la omnipotencia divina.
	
	En este Salvador, que es Dios y hombre a la vez, podemos descubrir \emph{la intención de la obra reconciliadora}. El Padre no quiere sólo purificar a la humanidad liberándola del pecado; quiere realizar \emph{la unión más íntima de la divinidad y la humanidad}. En la única persona divina de Jesús, la divinidad y la humanidad están unidas del modo más completo. Él que es perfectamente Dios es perfectamente hombre. Ha realizado en Sí esta unión de la divinidad y la humanidad para poder hacer participar de ella a todos los hombres. Perfectamente hombre, Él, que es Dios, quiere comunicar a sus hermanos humanos una vida divina que les permita ser más perfectamente hombres, reflejando en sí mismos la perfección divina.
	
	3. Un aspecto de la reconciliación merece ser subrayado aquí. Mientras el hombre pecador podía temer para su porvenir las consecuencias de su culpa y esperarse una vida humana disminuida, en cambio recibe de Cristo Salvador \emph{la posibilidad de un completo desarrollo humano}. No sólo es liberado de la esclavitud en la que le aprisionaban sus culpas, sino que puede alcanzar \emph{una perfección humana} superior a la que poseía antes del pecado. Cristo le ofrece una vida humana más abundante y más elevada. Por el hecho de que en Cristo la divinidad no ha comprimido en modo alguno a la humanidad sino que la ha elevado a un grado supremo de desarrollo, con su vida divina comunica a los hombres una vida humana más intensa y completa.
	
	Que Jesús sea el Dios Salvador hecho hombre significa, pues, que ya \emph{en el hombre nada está perdido}. Todo lo que había sido herido, manchado por el pecado, puede revivir y florecer. Esto explica cómo la gracia cristiana favorece el pleno ejercicio de todas las facultades humanas y también la afirmación de toda personalidad, tanto la femenina como la masculina. Reconciliando al hombre con Dios, la religión cristiana tiende a promover todo lo que es humano.
	
	Por tanto, podemos unirnos al canto que resonó en la gruta de Belén y proclamar con los ángeles: \textquote{Gloria a Dios en las alturas y paz en la tierra a los hombres de buena voluntad}.					
\end{body}