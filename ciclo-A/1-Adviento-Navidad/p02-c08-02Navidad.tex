\chapter{Domingo II de Navidad}

	\section{Lecturas}

		\rtitle{PRIMERA LECTURA}
		
		\rbook{Del libro de Ben Sirá} \rred{24, 1-2. 8-12}
		
		\rtheme{La sabiduría de Dios habitó en el pueblo escogido}
		
		\begin{scripture}
			La sabiduría hace su propia alabanza
			
			encuentra su honor en Dios
			
			y se gloría en medio de su pueblo.
			
			En la asamblea del Altísimo
			
			abre su boca y se gloría ante el Poderoso.
			
			El Creador del universo me dio una orden,
			
			el que me había creado estableció mi morada
			
			y me dijo: ``Pon tu tienda en Jacob,
			
			y fija tu heredad en Israel''.
			
			Desde el principio, antes de los siglos, me creó,
			
			y nunca jamás dejaré de existir.
			
			Ejercí mi ministerio en la Tienda santa delante de él,
			
			y así me establecí en Sión.
			
			En la ciudad amada encontré descanso,
			
			y en Jerusalén reside mi poder.
			
			Arraigué en un pueblo glorioso,
			
			en la porción del Señor, en su heredad.
		\end{scripture}
	
		\rtitle{SALMO RESPONSORIAL}
		
		\rbook{Salmo} \rred{147, 12-15. 19-20}
		
		\rtheme{El Verbo se hizo carne y habitó entre nosotros}
		
		\begin{psbody}
			Glorifica al Señor, Jerusalén;
			alaba a tu Dios, Sion.
			Que ha reforzado los cerrojos de tus puertas,
			y ha bendecido a tus hijos dentro de ti.
			
			Ha puesto paz en tus fronteras,
			te sacia con flor de harina.
			Él envía su mensaje a la tierra,
			y su palabra corre veloz.
			
			Anuncia su palabra a Jacob,
			sus decretos y mandatos a Israel;
			con ninguna nación obró así,
			ni les dio a conocer sus mandatos.
		\end{psbody}
	
	
		\rtitle{SEGUNDA LECTURA}
		
		\rbook{De la carta del apóstol san Pablo a los Efesios} \rred{1, 3-6. 15-18}
		
		\rtheme{Él nos ha destinado por medio de Jesucristo a ser sus hijos}
		
		\begin{scripture}
			Bendito sea Dios, Padre de Nuestro Señor Jesucristo,
			
			que nos ha bendecido en Cristo
			
			con toda clase de bendiciones espirituales en los cielos.
			
			Él nos eligió en Cristo antes de la fundación del mundo
			
			para que fuésemos santos e intachables ante él por el amor.
			
			Él nos ha destinado por medio de Jesucristo,
			
			según el beneplácito de su voluntad,
			
			a ser sus hijos,
			
			para alabanza de la gloria de su gracia,
			
			que tan generosamente nos ha concedido en el Amado.
			
			Por eso, habiendo oído hablar de vuestra fe en Cristo y de vuestro amor a todos los santos, no ceso de dar gracias por vosotros, recordándoos en mis oraciones, a fin de que el Dios de nuestro Señor Jesucristo, el Padre de la gloria, os dé espíritu de sabiduría y revelación para conocerlo, e ilumine los ojos de vuestro corazón para que comprendáis cuál es la esperanza a la que os llama, cuál la riqueza de gloria que da en herencia a los santos.
		\end{scripture}
	
	
		\rtitle{EVANGELIO}
		
		\rbook{Del Evangelio según san Juan} \rred{1, 1-18}
		
		\rtheme{El Verbo se hizo carne y habitó entre nosotros}
		
		\begin{scripture}
			En el principio existía el Verbo, y el Verbo estaba junto a Dios, y el Verbo era Dios.
			
			Él estaba en el principio junto a Dios.
			
			Por medio de él se hizo todo, y sin él no se hizo nada de cuanto se ha hecho.
			
			En él estaba la vida, y la vida era la luz de los hombres.
			
			Y la luz brilla en la tiniebla, y la tiniebla no lo recibió.
			
			Surgió un hombre enviado por Dios, que se llamaba Juan:
			
			este venía como testigo, para dar testimonio de la luz, para que todos creyeran por medio de él.
			
			No era él la luz, sino el que daba testimonio de la luz.
			
			El Verbo era la luz verdadera, que alumbra a todo hombre, viniendo al mundo.
			
			En el mundo estaba; el mundo se hizo por medio de él, y el mundo no lo conoció.
			
			Vino a su casa, y los suyos no lo recibieron.
			
			Pero a cuantos lo recibieron, les dio poder de ser hijos de Dios, a los que creen en su nombre.
			
			Estos no han nacido de sangre, ni de deseo de carne, ni de deseo de varón, sino que han nacido de Dios.
			
			Y el Verbo se hizo carne y habitó entre nosotros, y hemos contemplado su gloria: gloria como del Unigénito del Padre, lleno de gracia y de verdad.
			
			Juan da testimonio de él y grita diciendo: «Este es de quien dije: El que viene detrás de mí se ha puesto delante de mí, porque existía antes que yo».
			
			Pues de su plenitud todos hemos recibido, gracia tras gracia.
			
			Porque la ley se dio por medio de Moisés, la gracia y la verdad nos han llegado por medio de Jesucristo.
			
			A Dios nadie lo ha visto jamás: Dios unigénito, que está en el seno del Padre, es quien lo ha dado a conocer.
		\end{scripture}


	\section{Comentario Patrístico}
	
		\subsection{San León Magno, papa}
		
			\ptheme{El nacimiento del Señor es el nacimiento de la paz}
			
			\srec{Sermón 6, 2-3 en la Natividad del Señor: PL 54, 213-216.}
			
			\begin{body}
				Aunque aquella infancia, que la majestad del Hijo de Dios se dignó hacer suya, tuvo como continuación la plenitud de una edad adulta, y, después del triunfo de su pasión y resurrección, todas las acciones de su estado de humildad, que el Señor asumió por nosotros, pertenecen ya al pasado, la festividad de hoy renueva ante nosotros los sagrados comienzos de Jesús, nacido de la Virgen María; de modo que, mientras adoramos el nacimiento de nuestro Salvador, resulta que estamos celebrando nuestro propio comienzo.
				
				Efectivamente, la generación de Cristo es el comienzo del pueblo cristiano, y el nacimiento de la cabeza lo es al mismo tiempo del cuerpo.
				
				Aunque cada uno de los que llama el Señor a formar parte de su pueblo sea llamado en un tiempo determinado y aunque todos los hijos de la Iglesia hayan sido llamados cada uno en días distintos, con todo, la totalidad de los fieles, nacida en la fuente bautismal, ha nacido con Cristo en su nacimiento, del mismo modo que ha sido crucificada con Cristo en su pasión, ha sido resucitada en su resurrección y ha sido colocada a la derecha del Padre en su ascensión.
				
				Cualquier hombre que cree --en cualquier parte del mundo--, y se regenera en Cristo, una vez interrumpido el camino de su vieja condición original, pasa a ser un nuevo hombre al renacer; y ya no pertenece a la ascendencia de su padre carnal, sino a la simiente del Salvador, que se hizo precisamente Hijo del hombre, para que nosotros pudiésemos llegar a ser hijos de Dios.
				
				Pues si él no hubiera descendido hasta nosotros revestido de esta humilde condición, nadie hubiera logrado llegar hasta él por sus propios méritos.
				
				Por eso, la misma magnitud del beneficio otorgado exige de nosotros una veneración proporcionada a la excelsitud de esta dádiva. Y, como el bienaventurado Apóstol nos enseña, \emph{no hemos recibido el espíritu de este mundo, sino el Espíritu que procede de Dios}, a fin de que conozcamos lo que Dios nos ha otorgado; y el mismo Dios sólo acepta como culto piadoso el ofrecimiento de lo que él nos ha concedido.
				
				¿Y qué podremos encontrar en el tesoro de la divina largueza tan adecuado al honor de la presente festividad como la paz, lo primero que los ángeles pregonaron en el nacimiento del Señor?
				
				La paz es la que engendra los hijos de Dios, alimenta el amor y origina la unidad, es el descanso de los bienaventurados y la mansión de la eternidad. El fin propio de la paz y su fruto específico consiste en que se unan a Dios los que el mismo Señor separa del mundo.
				
				Que los que \emph{no han nacido de sangre, ni de amor carnal, ni de amor humano, sino de Dios,} ofrezcan, por tanto, al Padre la concordia que es propia de hijos pacíficos, y que todos los miembros de la adopción converjan hacia el Primogénito de la nueva creación, que vino a cumplir la voluntad del que le enviaba y no la suya: puesto que la gracia del Padre no adoptó como herederos a quienes se hallaban en discordia e incompatibilidad, sino a quienes amaban y sentían lo mismo. Los que han sido reformados de acuerdo con una sola imagen deben ser concordes en el espíritu.
				
				El nacimiento del Señor es el nacimiento de la paz: y así dice el Apóstol: \emph{El es nuestra paz; él ha hecho de los dos pueblos una sola cosa,} ya que, tanto los judíos como los gentiles, por su medio \emph{podemos acercarnos al Padre con un mismo Espíritu}.
			\end{body}

\newsection

		\subsection{San Máximo Confesor}
		
			\ptheme{Misterio siempre nuevo}
			
			\src{De las Cinco Centurias, Centuria 1, 8-13: PG 90, 1182-86.}
			
			\begin{body}
				La Palabra de Dios, nacida una vez en la carne (lo que nos indica la querencia de su benignidad y humanidad), vuelve a nacer siempre gustosamente en el espíritu para quienes lo desean; vuelve a hacerse niño, y se vuelve a formar en aquellas virtudes; y la amplitud de su grandeza no disminuye por malevolencia o envidia, sino que se manifiesta a sí mismo en la medida en que sabe que lo puede asimilar el que lo recibe, y así, al mismo tiempo que explora discretamente la capacidad de quienes desean verlo, sigue manteniéndose siempre fuera del alcance de su percepción, a causa de la excelencia del misterio.
				
				Por lo cual, el santo Apóstol, considerando sabiamente la fuerza del misterio, exclama: \emph{Jesucristo es el mismo ayer y hoy y siempre;} ya que entendía el misterio como algo siempre nuevo, al que nunca la comprensión de la mente puede hacer envejecer.
				
				Nace Cristo Dios, hecho hombre mediante la incorporación de una carne dotada de alma inteligente; el mismo que había otorgado a las cosas proceder de la nada. Mientras tanto, brilla en lo alto la estrella del Oriente y conduce a los Magos al lugar en que yace la Palabra encarnada; con lo que muestra que hay en la ley y los profetas una palabra místicamente superior, que dirige a las gentes a la suprema luz del conocimiento.
				
				Así pues, la palabra de la ley y de los profetas, entendida alegóricamente, conduce, como una estrella, al pleno conocimiento de Dios a aquellos que fueron llamados por la fuerza de la gracia, de acuerdo con el designio divino.
				
				Dios se hace efectivamente hombre perfecto, sin alterar nada de lo que es propio de la naturaleza, a excepción del pecado (pues ni el mismo pecado era propio de la naturaleza). Se hace efectivamente hombre perfecto a fin de provocar, con la vista del manjar de su carne, la voracidad insaciable y ávida del dragón infernal; y abatirlo por completo cuando ingiriera una carne que habría de convertírsele en veneno, porque en ella se hallaba oculto el poder de la divinidad. Esta carne sería al mismo tiempo remedio de la naturaleza humana, ya que el mismo poder divino presente en aquélla habría de restituir la naturaleza humana a la gracia primera.
				
				Y así como el dragón, deslizando su veneno en el árbol de la ciencia, había corrompido con su sabor la naturaleza, de la misma manera, al tratar de devorar la carne del Señor, se vio corrompido y destruido por la virtud de la divinidad que en ella residía.
				
				Inmenso misterio de la divina encarnación, que sigue siendo siempre misterio; pues, ¿de qué modo puede la Palabra hecha carne seguir siendo su propia persona esencialmente, siendo así que la misma persona existe al mismo tiempo con todo su ser en Dios Padre? ¿Cómo la Palabra, que es toda ella Dios por naturaleza, se hizo toda ella por naturaleza hombre, sin detrimento de ninguna de las dos naturalezas: ni de la divina, en cuya virtud es Dios, ni de la nuestra, en virtud de la cual se hizo hombre? Sólo la fe capta estos misterios, ella precisamente que es la sustancia y la base de todas aquellas realidades que exceden la percepción y razón de la mente humana en todo su alcance.
			\end{body}


\newsection


	\section{Homilías}
	
		\rbr{Las lecturas para este domingo son las mismas en los tres ciclos dominicales. No obstante, las homilías han sido distribuidas en esta obra en tres grupos, tomando en cuenta el ciclo litúrgico correspondiente al año en que fueron pronunciadas. Aquí aparecen las homilías que correspondieron al año A, y las de los años B y C aparecerán en sus respectivos volúmenes.}
		
		\subsection{San Juan Pablo II, papa}
		
			\subsubsection{Homilía (1983):}
			
				\src{Celebración del Te Deum de acción de gracias por el fin de año. \\Sábado 31 de diciembre de 1983.}
				
				\rbr{Pequeñas partes de la homilía fueron adaptadas para situarla en un contexto de inicio de año.}
				
				\begin{body}
					\textquote{Hijos, esta es la última hora} (\emph{1 Jn} 2, 18).
					
					1. Estamos reunidos aquí, como siempre, en la \textquote{Iglesia de Jesús} para prepararnos al encuentro con la última hora del año del Señor {[}1983{]}, y la liturgia dirige nuestro pensamiento hacia Dios, en quien todo lo existente encuentra su comienzo y su fin.
					
					El Evangelio de San Juan nos invita a volvernos a esta Palabra que en un principio estaba con Dios.
					
					He aquí la Palabra eterna: \textquote{todo fue hecho por él, y sin él nada se hizo de todo lo que existe} (\emph{Jn} 1, 3).
					
					Por tanto, también este año, que pasa como componente del tiempo humano y del paso cósmico, \textquote{se hizo} por medio del Verbo Eterno que \textquote{estaba en el principio con Dios} (cf. \emph{Jn} 1, 2) y que era Dios (cf. \emph{Jn} 1, 1).
					
					{[}Este{]} año del Señor (\ldots{}), queremos referirlo al principio absoluto. Deseamos redescubrir su lugar en la eternidad que no pasa.
					
					2. \textquote{Y el Verbo se hizo carne y habitó entre nosotros} (\emph{Jn} 1, 14). (\ldots{}) Dios en su Hijo Eterno acogió nuestro tiempo humano y todo el pasado cósmico. Nació la noche de Belén de la Inmaculada Virgen María bajo la protección del carpintero José de Nazaret. Nació en un establo porque \textquote{no había lugar para ellos en la posada} (\emph{Lc} 2, 7), porque \textquote{su propia gente no lo aceptó} (\emph{Jn} 1, 11).
					
					Todas las decepciones, tristezas y sufrimientos de nuestro mundo humano ya están insertadas de cierta manera en este Nacimiento de Dios en la tierra. Y estarán incluidas para todos los días de la peregrinación terrestre de Jesús de Nazaret hasta llegar a Getsemaní y a la cruz. En unión con él podemos vivir cada una de nuestras acciones a lo largo del tiempo. Podemos caminar los días y las horas de este año con la memoria y el corazón, también, y, especialmente aquellas que más nos hagan sufrir, porque Cristo está presente en ellas de una manera particular. Está presente a través del misterio de la Redención.
					
					3. (\ldots{}) Jesucristo, crucificado y resucitado, que está en la gloria del Padre, existe simultáneamente en el Cuerpo de su Iglesia.
					
					El Unigénito, lleno de gracia y de verdad, obtiene la gloria del Padre (cf. \emph{Jn} 1, 14) y, al mismo tiempo, por esta gracia y verdad está con nosotros, está en su Iglesia, porque \textquote{la gracia y la verdad vino (a nosotros) por Jesucristo} (\emph{Jn} 1, 17).
					
					Esta aceptación de la gracia y la verdad cuando el Verbo se hizo carne determina que el mundo y el hombre están envueltos en el misterio de la Redención. El hombre y el mundo a través de este misterio, de una manera nueva, existen en Dios por la obra de Cristo Redentor.
					
					En {[}este año estamos invitados, junto a toda{]} la Iglesia a sumergirnos de un modo nuevo en \textquote{su plenitud} (\emph{Jn} 1, 16): en la plenitud del Redentor del mundo para recibir de esta plenitud \textquote{gracia sobre gracia} (\emph{Jn} 1, 16).
					
					{[} \ldots{}{]}
					
					6. {[}En este año{]} vayamos al encuentro del Señor, dando gloria a Dios, con espíritu de acción de gracias y pidiendo perdón.
					
					Que la gracia y la verdad que \textquote{vinieron (a nosotros) por Jesucristo} nos acompañen siempre. En el misterio de la Redención, esta gracia y esta verdad seguirán guiando al hombre y al mundo al encuentro con Aquel \textquote{que es, que era y que ha de venir} (\emph{Ap} 1, 8): al encuentro con Dios que es la eternidad y la santidad. Amén.
				\end{body}

\newsection

		\subsection{Benedicto XVI, papa}
		
			\subsubsection{Ángelus (2010): Poner en Dios nuestra esperanza}
			
				\src{3 de enero del 2010.}
				
				\begin{body}
					En este domingo ---segundo después de Navidad y primero del año nuevo--- me alegra renovar a todos mi deseo de todo bien en el Señor. No faltan los problemas, en la Iglesia y en el mundo, al igual que en la vida cotidiana de las familias. Pero, gracias a Dios, nuestra esperanza no se basa en pronósticos improbables ni en las previsiones económicas, aunque sean importantes. Nuestra esperanza está en Dios, no en el sentido de una religiosidad genérica, o de un fatalismo disfrazado de fe. Nosotros confiamos en el Dios que en Jesucristo ha revelado de modo completo y definitivo su voluntad de estar con el hombre, de compartir su historia, para guiarnos a todos a su reino de amor y de vida. Y esta gran esperanza anima y a veces corrige nuestras esperanzas humanas.
					
					De esa revelación nos hablan hoy, en la liturgia eucarística, \textbf{tres lecturas bíblicas} de una riqueza extraordinaria: el capítulo 24 del \emph{Libro del Sirácida}, el himno que abre la \emph{Carta a los Efesios} de san Pablo y el prólogo del \emph{Evangelio de san Juan}. Estos textos afirman que Dios no sólo es el creador del universo ---aspecto común también a otras religiones--- sino que es Padre, que \textquote{nos eligió antes de crear el mundo (\ldots{}) predestinándonos a ser sus hijos adoptivos} (\emph{Ef} 1, 4-5) y que por esto llegó hasta el punto inconcebible de hacerse hombre: \textquote{El Verbo se hizo carne y acampó entre nosotros} (\emph{Jn} 1, 14). El misterio de la Encarnación de la Palabra de Dios fue preparado en el Antiguo Testamento, especialmente donde la Sabiduría divina se identifica con la Ley de Moisés. En efecto, la misma Sabiduría afirma: \textquote{El creador del universo me hizo plantar mi tienda, y me dijo: \textquote{Pon tu tienda en Jacob, entra en la heredad de Israel}} (\emph{Si} 24, 8). En Jesucristo, la Ley de Dios se ha hecho testimonio vivo, escrita en el corazón de un hombre en el que, por la acción del Espíritu Santo, reside corporalmente toda la plenitud de la divinidad (cf. \emph{Col} 2, 9).
					
					Queridos amigos, esta es la verdadera razón de la esperanza de la humanidad: la historia tiene un sentido, porque en ella \textquote{habita} la Sabiduría de Dios. Sin embargo, el designio divino no se cumple automáticamente, porque es un proyecto de amor, y el amor genera libertad y pide libertad. Ciertamente, el reino de Dios viene, más aún, ya está presente en la historia y, gracias a la venida de Cristo, ya ha vencido a la fuerza negativa del maligno. Pero cada hombre y cada mujer es responsable de acogerlo en su vida, día tras día. Por eso, también 2010 será un año más o menos \textquote{bueno} en la medida en que cada uno, de acuerdo con sus responsabilidades, sepa colaborar con la gracia de Dios. Por lo tanto, dirijámonos a la Virgen María, para aprender de ella esta actitud espiritual. El Hijo de Dios tomó carne de ella, con su consentimiento. Cada vez que el Señor quiere dar un paso adelante, junto con nosotros, hacia la \textquote{tierra prometida}, llama primero a nuestro corazón; espera, por decirlo así, nuestro \textquote{sí}, tanto en las pequeñas decisiones como en las grandes. Que María nos ayude a aceptar siempre la voluntad de Dios, con humildad y valentía, a fin de que también las pruebas y los sufrimientos de la vida contribuyan a apresurar la venida de su reino de justicia y de paz.
				\end{body}
			
			\subsubsection{Ángelus (2011): Entrar en las profundidades de Dios}
			
				\src{2 de enero del 2011.}
				
				\begin{body}
				Os renuevo a todos mis mejores deseos para el año nuevo y doy las gracias a cuantos me han enviado mensajes de cercanía espiritual. La liturgia de este domingo vuelve a proponer el \textbf{Prólogo del \emph{Evangelio de san Juan}}, proclamado solemnemente en el día de Navidad. Este admirable texto expresa, en forma de himno, el misterio de la Encarnación, que predicaron los testigos oculares, los Apóstoles, especialmente san Juan, cuya fiesta, no por casualidad, se celebra el 27 de diciembre. Afirma san Cromacio de Aquileya que \textquote{Juan era el más joven de todos los discípulos del Señor; el más joven por edad, pero ya anciano por la fe} (Sermo II, 1 \emph{De Sancto Iohanne Evangelista:} CCL 9a, 101). Cuando leemos: \textquote{En el principio existía el Verbo y el Verbo estaba con Dios, y el Verbo era Dios} (\emph{Jn} 1, 1), el Evangelista ---al que tradicionalmente se compara con un águila--- se eleva por encima de la historia humana escrutando las profundidades de Dios; pero muy pronto, siguiendo a su Maestro, vuelve a la dimensión terrena diciendo: \textquote{Y el Verbo se hizo carne} (\emph{Jn} 1, 14). El Verbo es \textquote{una realidad viva: un Dios que\ldots{} se comunica haciéndose él mismo hombre} (J. Ratzinger, \emph{Teologia della liturgia}, LEV 2010, p. 618). En efecto, atestigua Juan, \textquote{puso su morada entre nosotros, y hemos contemplado su gloria} (\emph{Jn} 1, 14). \textquote{Se rebajó hasta asumir la humildad de nuestra condición ---comenta san León Magno--- sin que disminuyera su majestad} (\emph{Tractatus} XXI, 2: CCL 138, 86-87). Leemos también en el Prólogo: \textquote{De su plenitud hemos recibido todos, gracia por gracia} (\emph{Jn} 1, 16). \textquote{¿Cuál es la primera gracia que hemos recibido? ---se pregunta san Agustín, y responde--- Es la fe}. La segunda gracia, añade en seguida, es \textquote{la vida eterna} (\emph{Tractatus in Ioh}. III, 8.9: ccl 36, 24.25).
			\end{body}

\newsection

		\subsection{Francisco, papa}
		
			\subsubsection{Ángelus (2014): Profundizar el sentido de su nacimiento}
			
				\src{5 de enero del 2014.}
				
				\begin{body}
					La liturgia de este domingo nos vuelve a proponer, en el \textbf{Prólogo del Evangelio de san Juan}, el significado más profundo del Nacimiento de Jesús. Él es la Palabra de Dios que se hizo hombre y puso su \textquote{tienda}, su morada entre los hombres. Escribe el evangelista: \textquote{El Verbo se hizo carne y habitó entre nosotros} (\emph{Jn} 1, 14). En estas palabras, que no dejan de asombrarnos, está todo el cristianismo. Dios se hizo mortal, frágil como nosotros, compartió nuestra condición humana, excepto en el pecado, pero cargó sobre sí mismo los nuestros, como si fuesen propios. Entró en nuestra historia, llegó a ser plenamente Dios-con-nosotros. El nacimiento de Jesús, entonces, nos muestra que Dios quiso unirse a cada hombre y a cada mujer, a cada uno de nosotros, para comunicarnos su vida y su alegría.
					
					Así Dios es Dios con nosotros, Dios que nos ama, Dios que camina con nosotros. Éste es el mensaje de Navidad: el Verbo se hizo carne. De este modo la Navidad nos revela el amor inmenso de Dios por la humanidad. De aquí se deriva también el entusiasmo, nuestra esperanza de cristianos, que en nuestra pobreza sabemos que somos amados, visitados y acompañados por Dios; y miramos al mundo y a la historia como el lugar donde caminar juntos con Él y entre nosotros, hacia los cielos nuevos y la tierra nueva. Con el nacimiento de Jesús nació una promesa nueva, nació un mundo nuevo, pero también un mundo que puede ser siempre renovado. Dios siempre está presente para suscitar hombres nuevos, para purificar el mundo del pecado que lo envejece, del pecado que lo corrompe. En lo que la historia humana y la historia personal de cada uno de nosotros pueda estar marcada por dificultades y debilidades, la fe en la Encarnación nos dice que Dios es solidario con el hombre y con su historia. Esta proximidad de Dios al hombre, a cada hombre, a cada uno de nosotros, es un don que no se acaba jamás. ¡Él está con nosotros! ¡Él es Dios con nosotros! Y esta cercanía no termina jamás. He aquí el gozoso anuncio de la Navidad: la luz divina, que inundó el corazón de la Virgen María y de san José, y guio los pasos de los pastores y de los magos, brilla también hoy para nosotros.
					
					En el misterio de la Encarnación del Hijo de Dios hay también un aspecto vinculado con la libertad humana, con la libertad de cada uno de nosotros. En efecto, el Verbo de Dios pone su tienda entre nosotros, pecadores y necesitados de misericordia. Y todos nosotros deberíamos apresurarnos a recibir la gracia que Él nos ofrece. En cambio, continúa el Evangelio de san Juan, \textquote{los suyos no lo recibieron} (v. 11). Incluso nosotros muchas veces lo rechazamos, preferimos permanecer en la cerrazón de nuestros errores y en la angustia de nuestros pecados. Pero Jesús no desiste y no deja de ofrecerse a sí mismo y ofrecer su gracia que nos salva. Jesús es paciente, Jesús sabe esperar, nos espera siempre. Esto es un mensaje de esperanza, un mensaje de salvación, antiguo y siempre nuevo. Y nosotros estamos llamados a testimoniar con alegría este mensaje del Evangelio de la vida, del Evangelio de la luz, de la esperanza y del amor. Porque el mensaje de Jesús es éste: vida, luz, esperanza y amor.
					
					Que María, Madre de Dios y nuestra Madre de ternura, nos sostenga siempre, para que permanezcamos fieles a la vocación cristiana y podamos realizar los deseos de justicia y de paz que llevamos en nosotros al inicio de este nuevo año.
				\end{body}
			
			\subsubsection{Ángelus (2020): Revelación plena del plan de Dios}
			
				\src{5 de enero del 2020.}
				
				\begin{body}
					En este segundo domingo de la Navidad, las lecturas bíblicas nos ayudan a alargar la mirada, para tomar una conciencia plena del significado del nacimiento de Jesús.
					
					El \textbf{comienzo del Evangelio de San Juan} nos muestra una impactante novedad: el Verbo eterno, el Hijo de Dios, \textquote{se hizo carne} (v. 14). No sólo vino a vivir entre la gente, sino que se convirtió en uno del pueblo, ¡uno de nosotros! Después de este acontecimiento, para dirigir nuestras vidas, ya no tenemos sólo una ley, una institución, sino una Persona, una Persona divina, Jesús, que guía nuestras vidas, nos hace ir por el camino porque Él lo hizo antes.
					
					\textbf{San Pablo} bendice a Dios por su plan de amor realizado en Jesucristo (cf. \emph{Efesios} 1, 3-6; 15-18). En este plan, cada uno de nosotros encuentra su vocación fundamental. ¿Y cuál es? Esto es lo que dice Pablo: estamos predestinados a ser hijos de Dios por medio de Jesucristo. El Hijo de Dios se hizo hombre para hacernos a nosotros, hombres, hijos de Dios. Por eso el Hijo eterno se hizo carne: para introducirnos en su relación filial con el Padre.
					
					Así pues, hermanos y hermanas, mientras continuamos contemplando el admirable signo del belén, la liturgia de hoy nos dice que el Evangelio de Cristo no es una fábula, ni un mito, ni un cuento moralizante, no. El Evangelio de Cristo es la plena revelación del plan de Dios, el plan de Dios para el hombre y el mundo. Es un mensaje a la vez sencillo y grandioso, que nos lleva a preguntarnos: ¿qué plan concreto tiene el Señor para mí, actualizando aún hoy su nacimiento entre nosotros?
					
					Es el \textbf{apóstol Pablo} quien nos sugiere la respuesta: \textquote{{[}Dios{]} nos ha elegido [\ldots{}] para ser santos e inmaculados en su presencia, en el amor} (v. 4). Este es el significado de la Navidad. Si el Señor sigue viniendo entre nosotros, si sigue dándonos el don de su Palabra, es para que cada uno de nosotros pueda responder a esta llamada: ser santos en el amor. La santidad pertenece a Dios, es comunión con Él, transparencia de su infinita bondad. La santidad es guardar el don que Dios nos ha dado. Simplemente esto: guardar la gratuidad. En esto consiste ser santo. Por tanto, quien acepta la santidad en sí mismo como un don de gracia, no puede dejar de traducirla en acciones concretas en la vida cotidiana. Este don, esta gracia que Dios me ha dado, la traduzco en una acción concreta en la vida cotidiana, en el encuentro con los demás. Esta caridad, esta misericordia hacia el prójimo, reflejo del amor de Dios, al mismo tiempo purifica nuestro corazón y nos dispone al perdón, haciéndonos \textquote{inmaculados} día tras día. Pero inmaculados no en el sentido de que yo elimino una mancha: inmaculados en el sentido de que Dios entra en nosotros, el don, la gratuidad de Dios entra en nosotros y nosotros lo guardamos y lo damos a los demás.
					
					Que la Virgen María nos ayude a acoger con alegría y gratitud el diseño divino de amor realizado en Jesucristo.
				\end{body}

\newsection

	\section{Temas}