\chapter{Sagrada Familia (A)}

\section{Lecturas}

PRIMERA LECTURA

Del libro de Ben Sirá 3, 2-6. 12-14

Quien teme al Señor honrará a sus padres

El Señor honra más al padre que a los hijos

y afirma el derecho de la madre sobre ellos.

Quien honra a su padre expía sus pecados,

y quien respeta a su madre es como quien acumula tesoros.

Quien honra a su padre se alegrará de sus hijos

y cuando rece, será escuchado.

Quien respeta a su padre tendrá larga vida,

y quien honra a su madre obedece al Señor.

Hijo, cuida de tu padre en su vejez

y durante su vida no le causes tristeza.

Aunque pierda el juicio, sé indulgente con él

y no lo desprecies aun estando tú en pleno vigor.

Porque la compasión hacia el padre no será olvidada

y te servirá para reparar tus pecados.

SALMO RESPONSORIAL

Salmo 127, 1bc-2. 3. 4-5

Dichosos los que temen al Señor y siguen sus caminos

℣. Dichoso el que teme al Señor

y sigue sus caminos.

Comerás del fruto de tu trabajo,

serás dichoso, te irá bien. ℟.

℣. Tu mujer, como parra fecunda,

en medio de tu casa;

tus hijos, como renuevos de olivo,

alrededor de tu mesa. ℟.

℣. Esta es la bendición del hombre

que teme al Señor.

Que el Señor te bendiga desde Sión,

que veas la prosperidad de Jerusalén

todos los días de tu vida. ℟.

SEGUNDA LECTURA

De la carta del apóstol san Pablo a los Colosenses 3, 12-21

La vida de familia vivida en el Señor

Hermanos:

Como elegidos de Dios, santos y amados, revestíos de compasión
entrañable, bondad, humildad, mansedumbre, paciencia.

Sobrellevaos mutuamente y perdonaos cuando alguno tenga quejas contra
otro.

El Señor os ha perdonado: haced vosotros lo mismo.

Y por encima de todo esto, el amor, que es el vínculo de la unidad
perfecta.

Que la paz de Cristo reine en vuestro corazón: a ella habéis sido
convocados en un solo cuerpo.

Sed también agradecidos. La Palabra de Cristo habite entre vosotros en
toda su riqueza; enseñaos unos a otros con toda sabiduría; exhortaos
mutuamente.

Cantad a Dios, dando gracias de corazón, con salmos, himnos y cánticos
inspirados.

Y todo lo que de palabra o de obra realicéis, sea todo en nombre de
Jesús, dando gracias a Dios Padre por medio de él.

Mujeres, sed sumisas a vuestros maridos, como conviene en el Señor.
Maridos, amad a vuestras mujeres, y no seáis ásperos con ellas.

Hijos, obedeced a vuestros padres en todo, que eso agrada al Señor.
Padres, no exasperéis a vuestros hijos, no sea que pierdan el ánimo.

EVANGELIO

Del Santo Evangelio según san Mateo 2, 13-15. 19-23

Toma al niño y a su madre y huye a Egipto

Cuando se retiraron los magos, el ángel del Señor se apareció en sueños
a José y le dijo:

«Levántate, toma al niño y a su madre y huye a Egipto; quédate allí
hasta que yo te avise, porque Herodes va a buscar al niño para matarlo».

José se levantó, tomó al niño y a su madre, de noche, se fue a Egipto y
se quedó hasta la muerte de Herodes para que se cumpliese lo que dijo el
Señor por medio del profeta: \textquote{De Egipto llamé a mi hijo}.

Cuando murió Herodes, el ángel del Señor se apareció de nuevo en sueños
a José en Egipto y le dijo:

«Levántate, coge al niño y a su madre y vuelve a la tierra de Israel,
porque han muerto los que atentaban contra la vida del niño».

Se levantó, tomó al niño y a su madre y volvió a la tierra de Israel.

Pero al enterarse de que Arquelao reinaba en Judea como sucesor de su
padre Herodes tuvo miedo de ir allá. Y avisado en sueños se retiró a
Galilea y se estableció en una ciudad llamada Nazaret. Así se cumplió lo
dicho por medio de los profetas, que se llamaría nazareno.


\section{Comentario Patrístico}

\subsection{San Juan Crisóstomo, obispo}

Junto al Niño Jesús están María y José

Homilía sobre el día de Navidad: PG 56, 392

Entró Jesús en Egipto para poner fin al llanto de la antigua tristeza; suplantó las plagas por el gozo, y convirtió la noche y las tinieblas en luz de salvación. Entonces fue contaminada el agua del río con la sangre de los tiernos niños. Por eso entró en Egipto el que había convertido el agua en sangre, comunicó a las aguas vivas el poder de aflorar la salvación y las purificó de su fango e impureza con la virtud del Espíritu. Los egipcios fueron afligidos y, enfurecidos, no reconocieron a Dios. Entró, pues, Jesús en Egipto y, colmando las almas religiosas del conocimiento de Dios, dio al río el poder de fecundar una mies de mártires más copiosa que la mies de grano.

¿Qué más diré o cómo seguir hablando? Veo a un artesano y un pesebre; veo a un Niño y los pañales de la cuna, veo el parto de la Virgen carente de lo más imprescindible, todo marcado por la más apremiante necesidad; todo bajo la más absoluta pobreza. ¿Has visto destellos de riqueza en la más extrema pobreza? ¿Cómo, siendo rico, se ha hecho pobre por nuestra causa? ¿Cómo es que no dispuso ni de lecho ni de mantas, sino que fue depositado en un desnudo pesebre? ¡Oh tesoro de riqueza, disimulado bajo la apariencia de pobreza! Yace en el pesebre, y hace temblar el orbe de la tierra; es envuelto en pañales, y rompe las cadenas del pecado; aún no sabe articular palabra, y adoctrina a los Magos induciéndolos a la conversión.

¿Qué más diré o cómo seguir hablando? Ved a un Niño envuelto en pañales y que yace en un pesebre: está con él María, que es Virgen y Madre; le acompañaba José, que es llamado padre.

José era sólo el esposo: fue el Espíritu quien la cubrió con su sombra. Por eso José estaba en un mar de dudas y no sabía cómo llamar al Niño. Esta es la razón por la que, trabajado por la duda, recibe, por medio del ángel, un oráculo del cielo: \emph{José, no tengas reparo en llevarte a tu mujer, pues la criatura que hay en ella viene del Espíritu Santo}. En efecto, el Espíritu Santo cubrió a la Virgen con su sombra. Y ¿por qué nace de la Virgen y conserva intacta su virginidad? Pues porque en otro tiempo el diablo engañó a la virgen Eva; por lo cual a María, que dio a luz siendo virgen, fue Gabriel quien le comunicó la feliz noticia. Es verdad que la seducida Eva dio a luz una palabra que introdujo la muerte; pero no lo es menos que María, acogiendo la alegre noticia, engendró al Verbo en la carne, que nos ha merecido la vida eterna.

\section{Homilías}

Las lecturas para esta solemnidad podrían ser las mismas en los tres ciclos dominicales (hay lecturas opcionales para los años B y C). En esta obra las homilías han sido distribuidas en tres grupos, tomando en cuenta el ciclo litúrgico correspondiente al año en que fueron pronunciadas. Aquí aparecen las homilías que correspondieron al año A, y las de los años B y C aparecerán en sus respectivos volúmenes. Aquéllas homilías podrían también ser iluminadoras para esta año y viceversa.

\subsection{San Juan XXIII, papa}

\subsubsection{Discurso: Que no reine el espíritu mundano}

En la festividad de la Sagrada Familia. Domingo 10 de enero de 1960.

Hoy que la Iglesia pone a la consideración de los fieles el ejemplo de virtud de la Sagrada Familia, nos complacemos en invocar la protección de Jesús, María y José sobre las queridas familias de todos nuestros hijos.

Nos las imaginamos a todas aquí presentes, unidas con Nos en un mismo afecto, y comprendemos los deseos, angustias y temores de cada uno. Nuestro corazón sabe alegrarse con el que se alegra y sufrir con el que sufre (Rom 12,15). Conocemos también las dificultades que hay en las familias, especialmente en las numerosas, cuyos sacrificios suelen ignorarse e incluso, a veces, ni se aprecian.

Sabemos que el espíritu mundano, empleando cada vez mayores incentivos, trata de insinuarse en esta santa institución familiar, que Dios ha querido como custodia y salvaguardia de la dignidad del hombre, del primer despertar de la vida a la juventud impetuosa y de la edad madura a la vejez.

Por tanto, dirigimos, mejor, repetimos a todos la invitación de la liturgia a que miren con segura confianza el ejemplo de la Sagrada Familia que Jesús santificó con inefables virtudes.

El secreto de la verdadera paz, de la mutua y permanente concordia, de la docilidad de los hijos, del florecimiento de las buenas costumbres está en la constante y generosa imitación de la amabilidad, modestia y mansedumbre de la familia de Nazaret, en la que Jesús, Sabiduría eterna del Padre, se nos ofrece junto con María, su madre purísima, y San José, que representa al Padre celestial.

En esta luz todo se transforma en las grandes realidades de la familia cristiana como poco ha hemos puesto de manifiesto en la alocución de la misa de Nochebuena: \textquote{Esponsales iluminados por la luz de lo alto; matrimonio sagrado e inviolable dentro del respeto a sus cuatro notas características: fidelidad, castidad, amor mutuo y santo temor del Señor; espíritu de prudencia y de sacrificio en la educación cuidadosa de los hijos; y siempre, siempre y en toda circunstancia, en disposición de ayudar, de perdonar, de compartir, de otorgar a otros la confianza que nosotros quisiéramos se nos otorgara. Es así como se edifica la casa que jamás se derrumba}.

De nuestro corazón brota el deseo de esta segura esperanza que es garantía de paz inalterable y se une a cada uno de vosotros para acompañarnos en el año nuevo, y que reforzamos con una oración especial que elevamos al cielo fervorosamente con las familias de todos los que nos escuchan, especialmente de aquellas que por falta de medios, de trabajo y de salud sufren dolorosas privaciones.

Nuestro pensamiento se dirige sobre todo a la juventud, esperanza y consuelo de la Iglesia y futuro sostén de la sociedad y más que nada ---ya lo repetimos el pasado año--- a cuantos jóvenes van a formar un hogar y no pueden por dificultades económicas. A todos deseamos una vida llena de la divina gracia, que se afiance en la defensa de los valores espirituales, y llena de la prosperidad y suavidad de los bienes de este mundo.

\subsection{Juan Pablo II, papa}

\subsubsection{Homilía (1986):} 30 de noviembre de 1986. Celebración en el Hipódromo \textquote{Belmont}, Perth, Australia.

Esta homilía fue pronunciada el Domigo I de Adviento, pero la celebración estaba dedicada a la Familia, por lo que las palabras del Papa son perfectamente aplicables a la celebración de este día.

\begin{body}
	\textquote{\emph{Es hora de despertarnos del sueño, porque nuestra salvación está más cerca ahora \ldots{}}} (\emph{Rom} 13, 11).
	
	\emph{Amados hermanos y hermanas en Cristo.}
	
	1. Con estas solemnes palabras, la liturgia de este primer domingo de Adviento conduce a toda la Iglesia a un tiempo de espera y preparación. Es el momento en el que toda comunidad cristiana revive la expectativa que los profetas despertaron en el pueblo de Israel, mientras esperaban ansiosos el cumplimiento de la promesa: \textquote{La Virgen concebirá y dará a luz un hijo, al que llamará Emmanuel} (\emph{Is} 7, 14), que significa \textquote{Dios con nosotros} (cf. \emph{Mt} 1, 23). Es el tiempo de preparación para la venida del niño, el \textquote{Príncipe de la Paz}: el infante de Belén, que es al mismo tiempo el Hijo de Dios, y la segunda Persona de la Santísima Trinidad.
	
	(\ldots{}) La Navidad es un día especial para las familias (\ldots{}) en muchas otras partes del mundo. La familia en el proyecto de Dios para la humanidad y para la Iglesia (\ldots{}). El Hijo de Dios, al hacerse hombre, inicia esa familia especial que la Iglesia venera como la Sagrada Familia de Nazaret: Jesús, María y José.
	
	{[} \ldots{}{]}
	
	3. \textquote{La familia es la Iglesia doméstica}. El significado de esta idea cristiana tradicional es que la casa es la Iglesia en miniatura. La Iglesia es el sacramento del amor de Dios, es una comunión de fe y de vida. Ella es madre y maestra. Está al servicio de toda la familia humana para caminar hacia su destino final. Al mismo tiempo, la familia es una comunidad de vida y amor. Educa y orienta a sus miembros hacia la plena madurez humana y está al servicio del bien de todos en el camino de la vida. La familia es la \textquote{primera y vital célula de la sociedad} ({\emph{Apostolicam Actuositatem}}). El futuro del mundo y de la Iglesia pasa, pues, por la familia.
	
	Por tanto, no es de extrañar que la Iglesia en los últimos tiempos haya prestado mucho cuidado y atención a los problemas que afectan a la vida familiar y al matrimonio. Tampoco es de extrañar que los gobiernos y las organizaciones públicas estén constantemente involucrados en problemas que afectan directa o indirectamente el bienestar institucional del matrimonio y la familia. Y todos pudieron ver que las relaciones saludables en el matrimonio y en la familia son de gran importancia para el crecimiento y el bienestar de la persona humana.
	
	4. Las transformaciones económicas, sociales y culturales que están teniendo lugar en el mundo tienen un gran efecto en la forma en que las personas ven el matrimonio y la familia. Como resultado, muchos esposos no están seguros del significado de su relación y esto les hace sentirse incómodos y angustiados. Por otro lado, muchos otros matrimonios son más fuertes porque, habiendo superado las tensiones del mundo moderno, experimentan mucho más plenamente ese amor especial y la responsabilidad del matrimonio que les hace ver a los hijos como un don especial de Dios para ellos y para la sociedad. Dependiendo de cómo vaya la familia, también lo hará la nación y todo el mundo en el que vivimos.
	
	En cuanto a la familia, la sociedad necesita urgentemente \textquote{que todos recuperen la conciencia de la primacía de los valores morales, que son los valores de la persona humana como tal}, y también de la \textquote{re-comprensión del sentido último de la vida y sus valores fundamentales} (\emph{Familiaris Consortio}, 8). {[}Es necesario{]} saber cómo salvaguardar la familia y la estabilidad del amor conyugal si queremos tener paz y justicia en la tierra.
	
	5. La Iglesia (\ldots{}) tiene una tarea específica: explicar y promover el plan de Dios para el matrimonio y la familia y ayudar a los esposos y familias a vivir de acuerdo con este plan. La Iglesia se dirige a todas las familias: en primer lugar a aquellas familias cristianas que se esfuerzan por ser cada vez más fieles al designio de Dios, busca fortalecerlas y acompañarlas en su desarrollo. Pero también se dirige, con la compasión del corazón de Jesús, a aquellas familias que se encuentran en dificultades o en situaciones irregulares.
	
	La Iglesia no puede decir que lo malo es bueno, ni puede decir que lo que no es válido es válido. No puede dejar de proclamar la enseñanza de Cristo, incluso cuando esta enseñanza es difícil de aceptar. También sabe que fue enviada a curar, reconciliar, llamar a la conversión, encontrar lo perdido (cf. \emph{Lc} 15, 6). Por tanto, es con inmenso amor y paciencia que la Iglesia busca ayudar a quienes experimentan dificultades para responder a las exigencias del amor conyugal cristiano y de la vida familiar.
	
	La caridad de Cristo solo se puede realizar en la verdad: en la verdad sobre la vida, el amor y la responsabilidad. La Iglesia debe anunciar a Cristo: camino, verdad, vida; y al hacerlo, debe enseñar los valores y principios que corresponden a la llamada del hombre, a la \textquote{novedad} de vida en Cristo. La Iglesia a veces es incomprendida y considerada carente de compasión porque apoya el plan creador de Dios para el matrimonio y la familia: su plan para el amor humano y la transmisión de la vida. La Iglesia es siempre la verdadera y fiel amiga de la persona humana en el peregrinaje de la vida. Sabe que la defensa de la ley moral contribuye al establecimiento de una verdadera civilización humana, y desafía constantemente a las personas a no abandonar su responsabilidad personal ante los imperativos éticos y morales (cf. \emph{Humanae Vitae}, 18).
	
	6. \textquote{Venid, subamos al monte del Señor \ldots{} para que nos muestre sus caminos y recorramos sus sendas} (\emph{Is} 2, 3). Con esta invitación, el profeta Isaías nos dice cómo debemos responder a Dios y cómo esta respuesta también se puede aplicar al plan de Dios para el matrimonio y la familia. A los esposos se les ofrece la gracia y la fuerza del sacramento del matrimonio para que puedan caminar por los caminos del Señor y seguir sus sendas, observando el plan que Cristo confirmó y estableció para la familia. Este plan da testimonio de lo que era en el \textquote{principio} (cf. \emph{Mt} 19, 8), lo que Dios quiso desde el principio para el bienestar y la felicidad de la familia. En el plan de Dios, el matrimonio requiere: amor fiel y duradero entre marido y mujer; una comunión indisoluble que \textquote{tiene sus raíces en la complementariedad natural que existe entre el hombre y la mujer, y se nutre de la voluntad personal de los esposos de compartir todo el proyecto de vida, lo que tienen y lo que son} (\emph{Familiaris Consortio}, 19); una comunidad de personas en la que el amor entre marido y mujer es plenamente humano, exclusivo y abierto a una nueva vida (cf. \emph{Familiaris consortio}, 29).
	
	El amor conyugal se fortalece con el sacramento del matrimonio para que sea una imagen cada vez más real y eficaz de la unidad que existe entre Cristo y la Iglesia (cf. \emph{Ef} 5, 32).
	
	7. Vosotros sabéis cuánto valor cristiano necesitáis para vivir los mandamientos de Dios en vuestra vida y en vuestra familia. Se trata de la valentía de estar dispuestos cada día a construir el amor, ese amor del que dice san Pablo: \textquote{La caridad es paciente, la caridad es bondadosa; la caridad no es envidiosa, no se jacta, no se hincha, no falta el respeto, no busca su interés, no se enoja, no toma en cuenta el mal recibido, no disfruta de la injusticia, pero se complace con la verdad. Todo lo cree \ldots{} todo lo soporta. La caridad no se acabará nunca} (\emph{1 Cor} 13, 4-8).
	
	{[}Entonces, ¿puede el Papa venir a Australia y no pedirle a los esposos y familias australianas que reflexionen en sus corazones si están viviendo bien su amor cristiano? ¿Cuán seriamente están comprometidos con la defensa de los valores familiares? ¿Qué tan adecuadas son las políticas para la defensa de estos valores y, por tanto, para la promoción del bien común de toda la nación?{]}
	
	En un mundo cada vez más sensible a los derechos de las mujeres, ¿qué se puede decir sobre los derechos de las mujeres que quieren ser, o necesitan ser, esposas y madres a tiempo completo? ¿Deberían estar agobiadas por un sistema fiscal que las discrimine de aquellas que optan por no quedarse en casa para tener sus propios ingresos? Sin violar la libertad de cualquiera que busque satisfacción en el empleo y las actividades fuera del hogar, ¿no debería apreciarse y apoyarse adecuadamente el trabajo del ama de casa? (cf. \emph{Familiaris Consortio}, 23). Esto es posible cuando las mujeres y los hombres son tratados con pleno respeto de su dignidad personal, por lo que son, más que por lo que hacen.
	
	8. Comprendiendo la importancia esencial de la vida familiar para una sociedad justa y saludable, la Santa Sede ha presentado una Carta de los derechos de la familia basada en los derechos naturales y valores comunes de toda la humanidad. Está dirigida principalmente a gobiernos y organismos internacionales, como \textquote{modelo y punto de referencia para la elaboración de legislación y política familiar, y guía para programas de acción} (\emph{Carta de los derechos de la familia, 22 de octubre de 1983}, Introducción).
	
	Entre los diversos principios que la Iglesia apoya firmemente en todas las circunstancias se encuentran los siguientes, sobre los que quiero llamar vuestra atención: el derecho inalienable de los cónyuges \textquote{a establecer una familia y a decidir el intervalo entre los nacimientos y el número de hijos a procrear, teniendo plenamente en cuenta sus deberes para con ellos mismos, con los hijos ya nacidos, la familia y la sociedad, en una justa jerarquía de valores y conforme al orden moral objetivo \ldots{}}; todas las presiones que limitan \textquote{la libertad de los padres para decidir sobre sus hijos constituyen una grave ofensa contra la dignidad humana y la justicia}; \textquote{las familias tienen derecho a poder contar con una adecuada política familiar por parte de los poderes públicos en los ámbitos jurídico, económico, social y fiscal, sin discriminación de ningún tipo} (\emph{Carta de los derechos de la familia}, artículos 3 y 9) .
	
	9. El orden moral exige que la regla establecida para los procesos de vida por el Creador en el acto de la creación sea respetada siempre y en todas partes. La conocida oposición de la Iglesia a la anticoncepción y la esterilización no es una posición tomada arbitrariamente, ni se basa en una perspectiva parcial de la persona humana. Más bien expresa su visión integral de la persona humana, a quien se le ha dado una vocación no sólo natural y terrena, sino también sobrenatural y eterna (cf. \emph{Humanae vitae}, 7). Además, la comprensión de la Iglesia del valor intrínseco de la vida humana como un regalo irrevocable de Dios explica por qué el Concilio Vaticano II habla de una \textquote{misión muy elevada para proteger la vida} y considera el aborto como un \textquote{crimen abominable} (\emph{Gaudium et Spes}, 27. 51).
	
	10. El lugar que ocupan los niños en la cultura y la sociedad (\ldots{}) merece una consideración. Sé que amáis y respetáis a vuestros hijos. Sé que en muchos sentidos las leyes tienen como objetivo salvaguardar su bienestar y su protección. Una sociedad que ama a sus hijos es una sociedad sana y dinámica. En su nombre, os hago un llamamiento a vosotros, padres. Los niños necesitan padres que puedan brindarles un entorno familiar estable. Hacedle saber que necesitan del amor verdadero para sentirse unidos en vuestro amor por los demás y por ellos mismos. Ellos buscan en vosotros amistad y guía. De vosotros, sobre todo, deben aprender a distinguir entre lo justo y lo injusto y saber discernir el bien del mal. Os hago, pues, un llamamiento: no privéis a vuestros hijos de su herencia verdaderamente humana y espiritual. Habladle de Dios, de Jesús, de su amor y de su Evangelio. Enseñadle a amar a Dios y a respetar sus mandamientos con la certeza de que son ante todo hijos suyos. Enseñadle a orar. Ayudadle a convertirse en seres humanos maduros y responsables, ciudadanos honestos de su país. Este es un privilegio maravilloso, un deber importante y una asignación maravillosa que habéis recibido de Dios. Mediante el testimonio de vuestra vida cristiana, guiad a vuestros hijos a ocupar el lugar que les corresponde en la Iglesia de Cristo.
	
	11. ¿Y qué deciros a vosotros, niños y jóvenes, {[}presentes aquí en tan gran número{]}? Amad a vuestros padres; rezad por ellos; dad gracias a Dios todos los días por ellos. Si a veces hay malentendidos entre vosotros, si a veces os resulta difícil obedecerlos, recordad estas palabras de San Pablo: \textquote{Haced todo sin murmuraciones y sin críticas, para que seáis irreprensibles y sencillos, inmaculados, hijos de Dios \ldots{} que debes brillar como estrellas en el mundo} (\emph{Fil} 2, 14-15). Rezad también por vuestros hermanos y hermanas y por todos los niños del mundo, especialmente por los pobres y hambrientos. Orad por los que no conocen a Jesús, por los que están solos y tristes.
	
	A todos los jóvenes católicos (\ldots{}) se os confía el futuro de la Iglesia en esta tierra. La Iglesia os necesita. Hay mucho que hacer en vuestras parroquias y comunidades locales, al servicio de los pobres y los necesitados, los enfermos y los ancianos, a través de las muchas formas de servicio voluntario. En primer lugar, debéis llevar a Cristo a vuestros amigos. Vuestra generación es el campo, rico para la mies, al que Cristo os envía. Cristo es el camino, la verdad y la vida para vuestra generación y para las generaciones venideras. Vosotros sois la esperanza de la Iglesia para una nueva era de evangelización y servicio. ¡Sed generosos con los demás, sed generosos con Cristo!
	
	12. {[}Queridos padres e hijos, queridas familias \ldots{}: el Evangelio del primer domingo de Adviento nos llamaba a \textquote{velar}, porque \textquote{si el dueño de casa lo supiera \ldots{} velaría y no permitiría que entren en su casa} (\emph{Mt} 24, 43). Esta es la exhortación que os repito. ¡Velad! No permitáis que os quiten el bien precioso del fiel amor matrimonial y la vida familiar. No los rechacéis, no penséis que hay una propuesta mejor para la felicidad o la realización humana.
	
	La llamada del Evangelio a \textquote{velar} también significa construir en la familia un sentido de responsabilidad. El amor genuino es siempre amor responsable. Los esposos y las esposas se aman verdaderamente cuando son responsables ante Dios y llevan a cabo su plan para el amor y la vida humanos; cuando responden y son responsables unos de otros. La paternidad responsable implica no solo traer hijos al mundo, sino también participar personal y responsablemente en su crecimiento y educación. ¡El verdadero amor en la familia es para siempre! Finalmente, mientras nos esforzamos por ser perfectos en el amor, recordamos las palabras de San Pablo: \textquote{Por lo tanto, desechemos las obras de las tinieblas y vistámonos las armas de la luz \ldots{} en cambio, vestíos del Señor Jesucristo} (\emph{Rom} 13, 12. 14).
	
	Queridas familias (\ldots{}) esta es vuestra vocación y vuestra felicidad hoy y siempre: revestirse del Señor Jesucristo y caminar en su luz. Amén.{]}
\end{body}



\subsubsection{Ángelus (1986): Iglesia doméstica}

Domingo 28 de diciembre de 1986.

1. \textquote{Levántate, coge al niño y a su madre y huye a Egipto; quédate allí hasta que yo te avise, porque \emph{Herodes va a buscar al niño para} \emph{matarlo}} (\emph{Mt} 2, 13).

El \textbf{Evangelio} de este domingo de la octava de Navidad nos recuerda cómo la Sagrada Familia fue amenazada durante su estancia en Belén.

Es una amenaza que viene del mundo, que quiere acabar con la vida del Niño.

2. Reunidos hoy para recitar el \textquote{Ángelus}, deseamos junto con toda la Iglesia expresar veneración y amor a esta Familia que, gracias al Hijo de Dios, se hizo la \textquote{\emph{iglesia doméstica}} en la tierra, antes que Él fundase su Iglesia sobre los Apóstoles y sobre Pedro.

Al mismo tiempo, la plegaria de la Iglesia universal y apostólica abraza hoy a todas las familias de la tierra: ¡a todas las \textquote{iglesias domésticas}!

Deseamos hacer frente a todo lo que, en el mundo de hoy, amenaza a la familia \emph{desde dentro y desde fuera}:

¡A lo que amenaza el amor, la fidelidad y la honestidad conyugal, a lo que amenaza la vida!

¡La vida: la gran dignidad de la persona humana!

3. Recemos, pues, con el Apóstol:

¡Familias!: \textquote{¡La palabra de Cristo habite entre vosotros en toda su riqueza!} (\emph{Col} 3, 16).

¡Familias!: \textquote{¡Que \emph{la paz} de Cristo \emph{actúe de árbitro} en vuestro corazón!} (\emph{Col} 3, 15).

\textquote{Sea vuestro uniforme\ldots{} la comprensión. \emph{Sobrellevaos mutuamente} y perdonaos\ldots{} y por encima de todo esto, el amor, que es el ceñidor de la unidad consumada} (\emph{Col} 3, 12-14).

¡Familias! ¡Esposos e hijos! \textquote{\emph{Estad agradecidos}} por el don de la comunidad y de la unión al que Cristo os ha llamado, ofreciéndoos el modelo de la Santísima Familia de Nazaret.

4. Hoy deseo reavivar, junto con todas las familias de Roma y de la Iglesia, esta gracia que han recibido en el santo sacramento del matrimonio, para que obre en ellos de forma eficaz durante todos los días de la vida.

\subsubsection{Homilía (1989):} Misa de fin de año y \textquote{Te Deum} de acción de gracias al Señor.

\emph{Domingo 31 de diciembre de 1989}

\begin{body}
	1. \textquote{Esta es la bendición del hombre que teme al Señor} (\emph{Salmo responsorial}).
	
	Al cabo de un año más, la Iglesia, \textquote{casa} en la que el Verbo hecho hombre se complace en habitar, la familia de Dios que camina en el temor del Señor hacia el cumplimiento de los tiempos, quiere reconocer que ha sido \textquote{bendecido} por Dios, con toda bendición espiritual en Cristo Jesús (cf. \emph{Ef} 1, 2).
	
	Al mismo tiempo, siente la necesidad de bendecir y agradecer a aquel de quien proviene todo don perfecto y en quien no hay variación ni sombra de cambio (cf. \emph{St} 1, 16).
	
	Queridos hermanos y hermanas, estamos aquí, esta noche, precisamente para responder a esta necesidad íntima del alma: cantar nuestro \textquote{Te Deum} y celebrar la Eucaristía, que es precisamente acción de gracias, por los innumerables beneficios que nos concede la bondad divina en este año que está a punto de terminar \ldots{}
	
	2. \textquote{Esta es la bendición del hombre que teme al Señor}. Las palabras del Salmo adquieren hoy un significado más amplio y abren horizontes más amplios. La liturgia de este domingo después de Navidad nos invita, de hecho, a detenernos en la contemplación frente al pesebre, donde nos encontramos con María y José y con el niño Jesús; nos invita a detenernos para aprender la lección de la Sagrada Familia de Nazaret y pedirle a Dios \textquote{que las mismas virtudes y el mismo amor puedan florecer en nuestras familias} (\emph{Oratio Collecta}).
	
	Queremos hacerlo con una mirada atenta a la situación y las necesidades de las familias que viven en nuestra ciudad \ldots{}
	
	3. \textquote{Levántate, toma al niño y a su madre y huye a Egipto \ldots{}} (\emph{Mt} 2, 13).
	
	El pasaje del Evangelio, que acabamos de escuchar, nos presenta un cuadro de la Familia de Nazaret donde no todo es idilio, paz y serenidad. Esta Santa Familia pasa por la prueba de la persecución y las dificultades del exilio. Se ve obligada a huir, a refugiarse, a buscar hospitalidad en otro lugar.
	
	Estos son hechos que no deben sorprendernos. Constituyen una confirmación más de la realidad del misterio de la Encarnación, que estamos celebrando en estos días. Al hacerse hombre, el Hijo de Dios quiso vivir la experiencia concreta de la familia humana y asumir no sólo las alegrías, sino también las pruebas y dificultades: las mismas que muchas familias hoy, incluso en nuestra ciudad, conocen bien y que se intenta remediar con múltiples iniciativas de servicio y soporte.
	
	4. En nuestro tiempo, a las dificultades habituales se han añadido las trampas que la rápida y profunda transformación socio\emph{-}cultural iniciada en las últimas décadas han traído al tejido vital de la familia. Esto constituye hoy un verdadero \textquote{desafío} para {[}toda la Iglesia{]} (\ldots{})
	
	Es cierto que (\ldots{}) hay todavía un gran número de familias en las que \textquote{el amor se guarda, se revela y se comunica} (\emph{Familiaris Consortio}, 7), pero es igualmente cierto que en la actual revolución social la célula familiar está particularmente en peligro. Las normas éticas y jurídicas, que han regulado su estructura y funciones durante siglos, son a menudo cuestionadas. El laicismo progresista tiende cada vez más a oscurecer e incluso negar esos valores naturales y creaturales de la institución familiar, inscritos en el plan redentor de Dios, reconocidos y potenciados haciendo de la familia, fundada en el sacramento del Matrimonio, una imagen de la Trinidad y una \textquote{Iglesia doméstica}. Los datos, publicados {[}recientemente{]} (\ldots{}) son preocupantes: las separaciones matrimoniales aumentan, el número de \textquote{uniones libres} es cada vez mayor, las tasas de natalidad están disminuyendo, el flagelo del aborto persiste.
	
	Todo esto no puede dejar indiferente a la Iglesia, que ha recibido de Cristo, su Esposo, la misión \textquote{de iluminar y consolar a los cristianos y a todos los hombres que se esfuerzan por salvaguardar y promover la dignidad natural y el altísimo valor sagrado del estado matrimonial} \emph{Gaudium et Spes}, 47).
	
	En este sentido, se abre un campo de acción amplio y exigente no solo para la comunidad eclesial (\ldots{}), sino también para las instituciones públicas, que tienen en el corazón el bien común y la promoción integral de la persona humana.
	
	Sobre todos, en el umbral del nuevo año, invoco las bendiciones del Señor para un renovado impulso en el cumplimiento de su servicio a la Iglesia y a la ciudad y, en particular, en beneficio de la familia, que es la célula fundamental de nuestra sociedad.
	
	5. Es ampliamente reconocido que la crisis actual de la familia a menudo tiene sus raíces en la superficialidad de quienes se comprometen con ella. No pocas veces, de hecho, las parejas jóvenes muestran poca conciencia del significado y valor de la institución familiar, especialmente cuando se lo considera desde la perspectiva de la Revelación. Así sucede que incluso quienes eligen libremente casarse \textquote{en el Señor} a veces terminan distanciandose de las cuestiones morales vinculadas a este hecho, exponiéndose a un desorden fácilmente imaginable.
	
	Por tanto, como opción prioritaria, se requiere la pastoral evangelizadora de la familia y, en ella, el compromiso de una preparación más adecuada al matrimonio. Ciertamente, ya se ha hecho mucho en este ámbito en los últimos años. Sin embargo, es necesario incrementar y unificar los esfuerzos, dando vida a itinerarios educativos reales, con herramientas y ayudas adecuadas y, sobre todo, con la implicación de los matrimonios más maduros en la fe y disponibles para esta forma particular de ministerio conyugal.
	
	Un gran aporte a la pastoral familiar vendrá también de un compromiso más marcado con el establecimiento y la animación de \textquote{grupos familiares} de espiritualidad y servicio, cada vez más capaces de compartir \textquote{con generosidad \ldots{} su riqueza espiritual con otras familias} (\emph{Gaudium et Spes}, 48), para construir y expandir la comunidad eclesial, haciendo de la parroquia una \textquote{familia de familias} y, por tanto, una verdadera comunidad evangelizadora y testimonial. De hecho, \textquote{la evangelización, en el futuro, depende en gran medida de la Iglesia doméstica} (\emph{Familiaris Consortio}, 65).
	
	6. Todo esto será mucho más fácil si las familias cristianas se esfuerzan por vivir la comunión de la que el Espíritu Santo es principio y alimento, que se les da en el sacramento del matrimonio. Una comunión fundada en la escucha de la Palabra de Dios, en la oración común, en el ejercicio de las virtudes cristianas, en primer lugar la caridad, \textquote{que es el vínculo de la perfección}, según la enseñanza del apóstol Pablo que acabamos de escuchar en la segunda lectura.
	
	Dado que la familia es la primera célula fundamental de la sociedad, es de esperar que ésta sepa aprobar leyes que protejan y promuevan la institución natural de la familia basada en el matrimonio y sus características de singularidad y estabilidad.
	
	7. Hermanos y hermanas, cuando estamos a punto de concluir un año más que nos concede la bondad del Señor, escuchemos la amonestación de San Pablo: \textquote{Todo lo que de palabra o de obra realicéis, sea todo en nombre de Jesús, dando gracias a Dios Padre por medio de él}.
	
	Sí, al dar gracias a Dios Padre, a través de Cristo, en el Espíritu Santo, nos disponemos a hacer todo en su nombre y para su mayor gloria.
	
	\textquote{Y que la paz de Cristo reine en vuestros corazones, porque a ella fuisteis llamados en un solo cuerpo}. ¡Amén!
	
\end{body}

\subsubsection{Ángelus (1989): El primer seminario}

Domingo 31 de diciembre de 1989.

1. La fiesta de hoy nos invita a contemplar la Sagrada Familia de José, María y Jesús, y a admirar su armonioso entendimiento y su perfecto amor. A la luz de ese modelo podemos comprender mejor el valor de la institución familiar y la importancia de su serena convivencia.

Por la narración bíblica de la creación sabemos que la familia ha sido querida por Dios, cuando creó al hombre y la mujer y, bendiciéndolos, les dijo \textquote{Sed fecundos y multiplicaos} (\emph{Gn} 1, 28).

Además, la gracia de Cristo, transmitida mediante el sacramento del matrimonio, hace a las familias capaces de realizar la unión a la que han sido llamadas. En especial las familias cristianas están comprometidas a reproducir el ideal enunciado por Jesús en la oración sacerdotal: \textquote{Como tú, Padre, en mí y yo en ti, que ellos también sean uno en nosotros} (\emph{Jn} 17, 21). Aquel que hizo esta oración obtuvo con su sacrificio un don especial de unidad para todas las familias.

2. El Hijo de Dios se hizo sacerdote en la Encarnación, pero precisamente en virtud de ese ministerio tuvo \emph{necesidad de una educación familiar}. Jesús obedecía a \textbf{María} y a \textbf{José}: \textquote{Vivía sujeto a ellos}, dice el \textbf{Evangelio} (\emph{Lc} 2, 51). Esta sumisión contribuía a la unión del Niño con sus padres y al clima de perfecto entendimiento que reinaba en la casa de Nazaret.

La educación recibida en familia preparó de hecho a Jesús para la misión que debía realizar en la tierra, según la revelación del ángel en el momento de la Anunciación. Fue, por consiguiente, una formación para el cumplimiento de su ministerio sacerdotal, más particularmente para la ofrenda del sacrificio de sí mismo al Padre.

Así queda iluminado \emph{el papel de la familia cristiana en el desarrollo de las vocaciones sacerdotales}. {[}El próximo Sínodo no podrá dejar de considerar este papel, reconocer su importancia y reflexionar sobre los medios adecuados para favorecerlo{]}.

3. La vocación es una llamada que viene del poder soberano y gratuito de Dios. Pero dicha llamaba debe abrirse un camino en el corazón; debe entrar en las profundidades del pensamiento, del sentimiento, de la voluntad del sujeto, para llegar a influir en el comportamiento moral. El joven tiene necesidad de un ambiente familiar así, que lo ayude a tomar conciencia de la llamada y a desarrollar todas sus virtualidades.

Orando hoy por todas las familias del mundo, pediremos en particular a María, Madre de Dios y Madre nuestra, que favorezca el desarrollo de las vocaciones sacerdotales y que bendiga a aquellas familias que se han mostrado disponibles, regalando uno de sus hijos a la Iglesia.

\subsubsection{Ángelus: Redentor de la familia}

Domingo 27 de diciembre de 1992.

1. Hoy la liturgia nos invita a contemplar la Sagrada Familia de Jesús, María y José.

Familia muy singular, por la presencia en ella del Hijo de Dios hecho hombre.

Pero, precisamente por esto, \emph{familia-modelo}, en la que todas las familias del mundo pueden encontrar su ideal seguro y el secreto de su vitalidad.

No es casualidad el hecho de que la fiesta de la Sagrada Familia caiga en un día tan cercano a la Navidad, pues se trata de su desarrollo natural.

Lo es, ante todo porque el Hijo de Dios \emph{quiso tener necesidad,} como todos los niños, \emph{del calor de una familia}.

Y lo es también porque, al venir a salvar al hombre, quiso asumir todas sus dimensiones: tanto la individual como la social. Es el Redentor del hombre, y también el \emph{Redentor de la familia}. Viviendo con María y José devolvió a la familia el esplendor del designio originario de Dios.

2. La experiencia ejemplar de Nazaret nos invita, queridos hermanos y hermanas, a volver a descubrir \emph{el valor fundamental del núcleo familiar}.

La familia es una vocación al amor, una comunidad de personas llamadas a vivir una experiencia específica de comunión (cf. \emph{Familiaris consortio}, 21) dentro del vasto designio de unidad, que Dios estableció para la Iglesia y el mundo, y que tiene su modelo y su fuente en la comunión trinitaria.

Por desgracia, la unidad familiar hoy se halla a menudo amenazada por una cultura hedonista y relativista, que no favorece la indisolubilidad del matrimonio y la acogida de la vida. Quienes sufren las consecuencias son, sobre todo, los más pequeños, pero también se proyectan sus efectos negativos en todo el entramado social, pues se genera frustración, tensión, agresividad, deseos de evasión y, en ocasiones, violencia.

¿Cómo se podrá lograr una convivencia ordenada y pacífica, en una sociedad cada vez más compleja, si no se vuelve a descubrir el valor y la vocación de la familia?

3. A esa urgencia nos invita precisamente la fiesta de hoy, volviéndonos a presentar el ideal de la Sagrada Familia, donde \emph{no faltaba la cruz}, pero \emph{se hacía oración}; donde los afectos eran profundos y puros; donde la esperanza diaria de la vida se suavizaba con el acatamiento sereno de la voluntad de Dios; donde el amor no se cerraba, sino que se proyectaba lejos, en una solidaridad concreta y universal.

La Virgen santa, a quien ahora nos dirigimos con la oración del \emph{Ángelus}, obtenga a las familias cristianas del mundo entero la gracia de ser cada vez más cautivadas por este ideal evangélico, a fin de que se conviertan en fermento auténtico de nueva humanidad.

\subsubsection{Ángelus: Amor auténtico}

Domingo 31 de diciembre de 1995.

1. Hoy la Iglesia celebra la fiesta de la Sagrada Familia, que este año coincide con el último día del año. La liturgia de hoy refiere \textbf{la invitación que el ángel dirigió dos veces a José}: \textquote{Levántate, toma contigo al niño ya su madre y \emph{huye a Egipto} (\ldots{}) porque Herodes va a buscar al niño para matarle} (Mt 2, 13); y, después de la muerte de Herodes: \textquote{Levántate, toma contigo al niño y a su madre, y ponte en camino \emph{de la tierra de Israel}} (\emph{Mt} 2, 20).

En este relato se pueden distinguir \emph{dos momentos decisivos} para la Sagrada Familia: primero, en Belén, cuando el rey Herodes, quiere matar al Niño, porque ve en él un adversario para el trono; y, en Egipto, cuando pasado el peligro, la Sagrada Familia puede volver del destierro a Nazaret. Observamos, ante todo, \emph{la paternal solicitud de Dios} ---la divina solicitud del Padre por el Hijo encarnado--- y, casi como un reflejo, la solicitud humana de José. Junto a él, percibimos la presencia silenciosa y trepidante de María, que en su corazón medita en la solicitud de Dios y en la obediencia diligente de José. A esa solicitud de Dios solemos llamarla divina Providencia; mientras que la solicitud humana se podría definir \emph{providencia humana}. En virtud de esa \emph{providencia}, el padre o la madre se esmeran para evitar todo tipo de mal y garantizar todo el bien posible a los hijos y a la familia.

2. La solicitud de los padres y de las madres debería suscitar en los hijos y en las familias viva gratitud, un sentimiento que constituye también un mandamiento: \textquote{\emph{Honra}}, dice también a los padres: \textquote{\emph{Trata de merecer esa honra}}. Es preciso recordar constantemente la dimensión de la vida familiar, establecida por el cuarto mandamiento del Decálogo. La familia que, por su naturaleza y vocación, es \emph{ambiente de vida y amor}, a menudo se halla sujeta a dolorosas amenazas de todo tipo. Con la familia y en la familia, se encuentra amenazada también la vida de la persona y también de la sociedad.

3. Amadísimos hermanos y hermanas, contemplemos a la Sagrada Familia de Nazaret, ejemplo para todas las familias cristianas y humanas. Ella irradia el auténtico amor-caridad, creando no sólo un elocuente modelo para todas las familias, sino también ofreciendo una garantía de que ese amor puede realizarse en todo el núcleo familiar. En la Sagrada Familia se han de inspirar los novios al prepararse para el matrimonio; y la deben contemplar los esposos al construir su comunidad doméstica. Quiera Dios que en toda casa crezca la fe y reinen el amor, la concordia, la solidaridad, el respeto recíproco y la apertura a la vida.

Que María, \emph{Reina de la familia}, título con el que podríamos de ahora en adelante invocarla en las letanías lauretanas, ayude a las familias de los creyentes a responder cada vez más fielmente a su vocación a fin de que lleguen a ser auténticas \emph{iglesias domésticas}.

\subsubsection{Ángelus: Imagen viva de la Iglesia de Dios}

Domingo 27 de diciembre de 1998.

1. En el clima gozoso de la Navidad, la Iglesia, reviviendo con nueva admiración el misterio del Emmanuel, el Dios con nosotros, nos invita a contemplar hoy a la Sagrada Familia de Nazaret. En la contemplación de este admirable modelo la Iglesia descubre valores que vuelve a proponer a las mujeres y a los hombres de todos los tiempos y de todas las culturas.

\textquote{¡Oh, Familia de Nazaret, imagen viva de la Iglesia de Dios!}. Con estas palabras, la comunidad cristiana reconoce en la comunión familiar de Jesús, María y José, una auténtica \textquote{regla de vida}: cuanto más sepa realizar la Iglesia la \textquote{alianza de amor} que se manifiesta en la Sagrada Familia, tanto más cumplirá su misión de ser levadura, para que \textquote{los hombres constituyan en Cristo una sola familia} (cf. \emph{Ad gentes}, 1).

2. La Sagrada Familia irradia una luz de esperanza también sobre la realidad de la familia de hoy. {[}Consciente de esto, el Consejo pontificio para la familia ya ha comenzado a trabajar para preparar el \emph{III Encuentro mundial de las familias}, que se celebrará en Roma los días 14 y 15 de octubre del año 2000, en el marco del gran jubileo\ldots{} El encuentro anterior tuvo lugar en Río de Janeiro. Y el primero, hace cuatro años, en Roma. El próximo será el tercero{]}.

\ldots{} En Nazaret brotó la primavera de la vida humana del Hijo de Dios, en el instante en que fue concebido por obra del Espíritu Santo en el seno virginal de María. Entre las paredes acogedoras de la casa de Nazaret, se desarrolló en un ambiente de alegría la infancia de Jesús, que \textquote{crecía en edad, en sabiduría y en gracia ante Dios y ante los hombres} (\emph{Lc} 2, 52).

3. Así, el misterio de Nazaret enseña a toda familia a engendrar y educar a sus hijos, cooperando de modo admirable en la obra del Creador y dando al mundo, con cada niño, una nueva sonrisa. En la familia unida los hijos alcanzan la maduración de su existencia, viviendo la experiencia más significativa y rica del amor gratuito, de la fidelidad, del respeto recíproco y de la defensa de la vida.

Ojalá que las familias de hoy contemplen a la Familia de Nazaret a fin de que, imitando el ejemplo de María y José, dedicados amorosamente al cuidado del Verbo encarnado, obtengan indicaciones oportunas para sus opciones diarias de vida.

A la luz de las enseñanzas aprendidas en esa escuela insuperable, todas las familias podrán orientarse en el camino hacia la plena realización del designio de Dios.

\subsubsection{Ángelus: Eligió una familia donde nacer y crecer}

Domingo 30 de diciembre del 2001.

1. Desde la cueva de Belén, donde en la Noche santa nació el Salvador, nuestra mirada se dirige hoy \emph{hacia la humilde casa de Nazaret}, para contemplar a la Sagrada Familia de Jesús, María y José, cuya fiesta celebramos en el clima festivo y familiar de la Navidad.

El Redentor del mundo quiso elegir la familia como lugar donde nacer y crecer, santificando así esta institución fundamental de toda sociedad. El tiempo que pasó en Nazaret, el más largo de su existencia, se halla envuelto por una gran reserva: los evangelistas nos transmiten pocas noticias. Pero si deseamos comprender más profundamente la vida y la misión de Jesús, debemos acercarnos al misterio de la Sagrada Familia de Nazaret para observar y escuchar. La liturgia de hoy nos ofrece una oportunidad providencial.

2. La humilde morada de Nazaret es para todo creyente y, especialmente para las familias cristianas, \emph{una auténtica escuela del Evangelio}. En ella admiramos la realización del proyecto divino de hacer de la familia una \emph{comunidad íntima de vida y amor}; en ella aprendemos que cada hogar cristiano está llamado a ser una pequeña \emph{iglesia doméstica}, donde deben resplandecer las virtudes evangélicas. Recogimiento y oración, comprensión y respeto mutuos, disciplina personal y ascesis comunitaria, espíritu de sacrificio, trabajo y solidaridad son rasgos típicos que hacen de la familia de Nazaret un modelo para todos nuestros hogares.

Quise poner de relieve estos valores en la exhortación apostólica \emph{Familiaris consortio}, cuyo vigésimo aniversario se celebra precisamente este año. \emph{El futuro de la humanidad pasa a través de la familia} que, en nuestro tiempo, ha sido marcada, más que cualquier otra institución, por las profundas y rápidas transformaciones de la cultura y la sociedad. Pero la Iglesia jamás ha dejado de \textquote{hacer sentir su voz y ofrecer su ayuda a todo aquel que, conociendo ya el valor del matrimonio y de la familia, trata de vivirlo fielmente; a todo aquel que, en medio de la incertidumbre o de la ansiedad, busca la verdad; y a todo aquel que se ve injustamente impedido para vivir con libertad el propio proyecto familiar} (\emph{Familiaris consortio}, 1). Es consciente de esta responsabilidad suya y también hoy quiere seguir \textquote{ofreciendo su servicio a todo hombre preocupado por el destino del matrimonio y de la familia} (\emph{ib}.).

3. Para cumplir esta urgente misión, la Iglesia cuenta de modo especial con el testimonio y la aportación de las familias cristianas. Más aún, frente a los peligros y a las dificultades que afronta la institución familiar, invita a un suplemento de audacia espiritual y apostólica, convencida de que las familias están llamadas a ser \textquote{signo de unidad para el mundo} y a testimoniar \textquote{el reino y la paz de Cristo, hacia el cual el mundo entero está en camino} (\emph{ib}., 48).

Que Jesús, María y José bendigan y protejan a todas las familias del mundo, para que en ellas reinen la serenidad y la alegría, la justicia y la paz que Cristo al nacer trajo como don a la humanidad.

\subsection{Benedicto XVI, papa}

\subsubsection{Ángelus: Santificó la realidad de la familia}

Domingo 30 de diciembre del 2007.

Celebramos hoy la fiesta de la Sagrada Familia. Siguiendo los evangelios de san Mateo y san Lucas, fijamos hoy nuestra mirada en Jesús, María y José, y adoramos el misterio de un Dios que quiso nacer de una mujer, la Virgen santísima, y entrar en este mundo por el camino común a todos los hombres. Al hacerlo así, santificó la realidad de la familia, colmándola de la gracia divina y revelando plenamente su vocación y misión.

A la familia dedicó gran atención el concilio Vaticano II. Los cónyuges ---afirma--- \textquote{son testigos, el uno para el otro y ambos para sus hijos, de la fe y del amor de Cristo} (\emph{Lumen gentium}, 35). Así la familia cristiana participa de la vocación profética de la Iglesia: con su estilo de vida \textquote{proclama en voz alta tanto los valores del reino de Dios ya presentes como la esperanza en la vida eterna} (\emph{ib}.).

Como repitió incansablemente mi venerado predecesor Juan Pablo II, el bien de la persona y de la sociedad está íntimamente vinculado a la \textquote{buena salud} de la familia (cf. \emph{Gaudium et spes}, 47). Por eso, la Iglesia está comprometida en defender y promover \textquote{la dignidad natural y el eximio valor} ---son palabras del Concilio--- del matrimonio y de la familia (\emph{ib}.). Con esta finalidad se está llevando a cabo, precisamente hoy, una importante iniciativa en Madrid, a cuyos participantes me dirigiré ahora en lengua española.

[\ldots{}] Al contemplar el misterio del Hijo de Dios que vino al mundo rodeado del afecto de María y de José, invito a las familias cristianas a experimentar la presencia amorosa del Señor en sus vidas. Asimismo, les aliento a que, inspirándose en el amor de Cristo por los hombres, den testimonio ante el mundo de la belleza del amor humano, del matrimonio y la familia. Esta, fundada en la unión indisoluble entre un hombre y una mujer, constituye el ámbito privilegiado en el que la vida humana es acogida y protegida, desde su inicio hasta su fin natural. Por eso, los padres tienen el derecho y la obligación fundamental de educar a sus hijos en la fe y en los valores que dignifican la existencia humana.

Vale la pena trabajar por la familia y el matrimonio porque vale la pena trabajar por el ser humano, el ser más precioso creado por Dios. Me dirijo de modo especial a los niños, para que quieran y recen por sus padres y hermanos; a los jóvenes, para que estimulados por el amor de sus padres, sigan con generosidad su propia vocación matrimonial, sacerdotal o religiosa; a los ancianos y enfermos, para que encuentren la ayuda y comprensión necesarias. Y vosotros, queridos esposos, contad siempre con la gracia de Dios, para que vuestro amor sea cada vez más fecundo y fiel. En las manos de María, \textquote{que con su \textquote{sí} abrió la puerta de nuestro mundo a Dios} (\emph{Spe salvi}, 49), pongo los frutos de esta celebración. Muchas gracias y ¡felices fiestas!

Nos dirigimos ahora a la Virgen santísima, pidiendo por el bien de la familia y por todas las familias del mundo.

\subsubsection{Ángelus: El calor de una familia}

Domingo 26 de diciembre del 2010.

El \emph{\textbf{Evangelio según san Lucas}} narra que los pastores de Belén, después de recibir del ángel el anuncio del nacimiento del Mesías, \textquote{fueron a toda prisa, y encontraron a María y a José, y al niño acostado en el pesebre} (2, 16). Así pues, a los primeros testigos oculares del nacimiento de Jesús se les presentó la escena de una familia: madre, padre e hijo recién nacido. Por eso, el primer domingo después de Navidad, la liturgia nos hace celebrar la fiesta de la Sagrada Familia. Este año tiene lugar precisamente al día siguiente de la Navidad y, prevaleciendo sobre la de san Esteban, nos invita a contemplar este \textquote{icono} en el que el niño Jesús aparece en el centro del afecto y de la solicitud de sus padres. En la pobre cueva de Belén ---escriben los Padres de la Iglesia--- resplandece una luz vivísima, reflejo del profundo misterio que envuelve a ese Niño, y que María y José custodian en su corazón y dejan traslucir en sus miradas, en sus gestos y sobre todo en sus silencios. De hecho, conservan en lo más íntimo las palabras del anuncio del ángel a María: \textquote{El que ha de nacer será llamado Hijo de Dios} (\emph{Lc} 1, 35).

Sin embargo, el nacimiento de todo niño conlleva algo de este misterio. Lo saben muy bien los padres que lo reciben como un don y que, con frecuencia, así se refieren a él. Todos hemos escuchado decir alguna vez a un papá y a una mamá: \textquote{Este niño es un don, un milagro}. En efecto, los seres humanos no viven la procreación meramente como un acto reproductivo, sino que perciben su riqueza, intuyen que cada criatura humana que se asoma a la tierra es el \textquote{signo} por excelencia del Creador y Padre que está en el cielo. ¡Cuán importante es, por tanto, que cada niño, al venir al mundo, sea acogido por el calor de una familia! No importan las comodidades exteriores: Jesús nació en un establo y como primera cuna tuvo un pesebre, pero el amor de María y de José le hizo sentir la ternura y la belleza de ser amados. Esto es lo que necesitan los niños: el amor del padre y de la madre. Esto es lo que les da seguridad y lo que, al crecer, les permite descubrir el sentido de la vida. La Sagrada Familia de Nazaret pasó por muchas pruebas, como la de la \textquote{matanza de los inocentes} ---nos la recuerda el \emph{Evangelio según san Mateo}---, que obligó a José y María a emigrar a Egipto (cf. 2, 13-23). Ahora bien, confiando en la divina Providencia, encontraron su estabilidad y aseguraron a Jesús una infancia serena y una educación sólida.

Queridos amigos, ciertamente la Sagrada Familia es singular e irrepetible, pero al mismo tiempo es \textquote{modelo de vida} para toda familia, porque Jesús, verdadero hombre, quiso nacer en una familia humana y, al hacerlo así, la bendijo y consagró. Encomendemos, por tanto, a la Virgen y a san José a todas las familias, para que no se desalienten ante las pruebas y dificultades, sino que cultiven siempre el amor conyugal y se dediquen con confianza al servicio de la vida y de la educación.

\subsection{Francisco, papa}

\subsubsection{Ángelus: Dios se hizo como nosotros}

Domingo 29 de diciembre del 2013.

En este primer domingo después de Navidad, la Liturgia nos invita a celebrar la fiesta de la Sagrada Familia de Nazaret. En efecto, cada belén nos muestra a Jesús junto a la Virgen y a san José, en la cueva de Belén. Dios quiso nacer en una familia humana, quiso tener una madre y un padre, como nosotros.

Y hoy el Evangelio nos presenta a la Sagrada Familia por el camino doloroso del destierro, en busca de refugio en Egipto. José, María y Jesús experimentan la condición dramática de los refugiados, marcada por miedo, incertidumbre, incomodidades (cf. Mt 2, 13-15.19-23). Lamentablemente, en nuestros días, millones de familias pueden reconocerse en esta triste realidad. Casi cada día la televisión y los periódicos dan noticias de refugiados que huyen del hambre, de la guerra, de otros peligros graves, en busca de seguridad y de una vida digna para sí mismos y para sus familias.

En tierras lejanas, incluso cuando encuentran trabajo, no siempre los refugiados y los inmigrantes encuentran auténtica acogida, respeto, aprecio por los valores que llevan consigo. Sus legítimas expectativas chocan con situaciones complejas y dificultades que a veces parecen insuperables. Por ello, mientras fijamos la mirada en la Sagrada Familia de Nazaret en el momento en que se ve obligada a huir, pensemos en el drama de los inmigrantes y refugiados que son víctimas del rechazo y de la explotación, que son víctimas de la trata de personas y del trabajo esclavo. Pero pensemos también en los demás \textquote{exiliados}: yo les llamaría \textquote{exiliados ocultos}, esos exiliados que pueden encontrarse en el seno de las familias mismas: los ancianos, por ejemplo, que a veces son tratados como presencias que estorban. Muchas veces pienso que un signo para saber cómo va una familia es ver cómo se tratan en ella a los niños y a los ancianos.

Jesús quiso pertenecer a una familia que experimentó estas dificultades, para que nadie se sienta excluido de la cercanía amorosa de Dios. La huida a Egipto causada por las amenazas de Herodes nos muestra que Dios está allí donde el hombre está en peligro, allí donde el hombre sufre, allí donde huye, donde experimenta el rechazo y el abandono; pero Dios está también allí donde el hombre sueña, espera volver a su patria en libertad, proyecta y elige en favor de la vida y la dignidad suya y de sus familiares.

Hoy, nuestra mirada a la Sagrada Familia se deja atraer también por la sencillez de la vida que ella lleva en Nazaret. Es un ejemplo que hace mucho bien a nuestras familias, les ayuda a convertirse cada vez más en una comunidad de amor y de reconciliación, donde se experimenta la ternura, la ayuda mutua y el perdón recíproco. Recordemos las tres palabras clave para vivir en paz y alegría en la familia: permiso, gracias, perdón. Cuando en una familia no se es entrometido y se pide \textquote{permiso}, cuando en una familia no se es egoísta y se aprende a decir \textquote{gracias}, y cuando en una familia uno se da cuenta que hizo algo malo y sabe pedir \textquote{perdón}, en esa familia hay paz y hay alegría. Recordemos estas tres palabras. Pero las podemos repetir todos juntos: permiso, gracias, perdón. (Todos: permiso, gracias, perdón) Desearía alentar también a las familias a tomar conciencia de la importancia que tienen en la Iglesia y en la sociedad. El anuncio del Evangelio, en efecto, pasa ante todo a través de las familias, para llegar luego a los diversos ámbitos de la vida cotidiana.

Invoquemos con fervor a María santísima, la Madre de Jesús y Madre nuestra, y a san José, su esposo. Pidámosle a ellos que iluminen, conforten y guíen a cada familia del mundo, para que puedan realizar con dignidad y serenidad la misión que Dios les ha confiado.

\subsubsection{Ángelus: Obedecieron a Dios}

Domingo, 29 de diciembre de 2019

Hoy es un día hermoso\ldots{} Hoy celebramos la fiesta de la Sagrada Familia de Nazaret. El término \textquote{sagrada} coloca a esta familia en el ámbito de la santidad, que es un don de Dios pero, al mismo tiempo, es una adhesión libre y responsable al plan de Dios. Éste fue el caso de la familia de Nazaret: estaba totalmente a disposición de la voluntad de Dios.

¿Cómo no asombrarse, por ejemplo, de la \textbf{docilidad de María} a la acción del Espíritu Santo que le pide que se convierta en la madre del Mesías? Porque María, como toda joven de su tiempo, estaba a punto de realizar su proyecto de vida, es decir, casarse con José. Pero cuando se dio cuenta de que Dios la llamaba a una misión particular, no dudó en proclamarse su \textquote{esclava} (cf. \emph{Lucas} 1, 38). Jesús exaltará su grandeza no tanto por su papel de madre, sino por su obediencia a Dios. Jesús dijo: \textquote{Dichosos más bien los que oyen la Palabra de Dios y la guardan} (\emph{Lucas} 11, 28), como María. Y cuando no comprende plenamente los acontecimientos que la involucran, María medita en silencio, reflexiona y adora la iniciativa divina. Su presencia al pie de la Cruz consagra esta disponibilidad total.

Luego, en lo que respecta a \textbf{José}, el Evangelio no nos refiere ni una sola palabra: no habla, sino que actúa por obediencia. Es el hombre del silencio, el hombre de la obediencia. La página del \textbf{Evangelio de hoy} (cf. \emph{Mateo} 2, 13-15, 19-23) nos recuerda tres veces esta obediencia del justo José, refiriéndose a su huida a Egipto y a su retorno a la tierra de Israel. Bajo la guía de Dios, representada por el Ángel, José aleja a su familia de la amenaza de Herodes y los salva. De esta manera, la Sagrada Familia se solidariza con todas las familias del mundo que se ven obligadas a exiliarse, se solidariza con todos aquellos que se ven obligados a abandonar su tierra a causa de la represión, la violencia, la guerra.

Finalmente, la tercera persona de la Sagrada Familia: \textbf{Jesús}. Él es la voluntad del Padre: sobre Él, dice san Pablo, no hubo \textquote{sí} y \textquote{no}, sino sólo \textquote{sí} (cf. \emph{2 Corintios} 1, 19). Y esto se manifestó en muchos momentos de su vida terrenal. Por ejemplo, el episodio en el templo en el que, a los padres angustiados que lo buscaban, les respondió: \textquote{¿No sabíais que yo debía estar en la casa de mi Padre?} (\emph{Lucas} 2, 49); o su constante repetición: \textquote{Mi alimento es hacer la voluntad del que me ha enviado} (\emph{Juan} 4, 34); su oración en el Huerto de los Olivos: \textquote{Padre mío, si esto no puede pasar sin que yo lo beba, hágase tu voluntad} (\emph{Mateo} 26, 42). Todos estos acontecimientos son la perfecta realización de las mismas palabras de Cristo que dice: \textquote{Sacrificio y oblación no quisiste [\ldots{}]. Entonces dije: \textquote{¡He aquí que vengo [\ldots{}] a hacer, oh Dios, tu voluntad!}} (\emph{Hebreos} 10, 5-7; \emph{Salmos} 40, 7-9).

\textbf{María, José, Jesús}: la Sagrada Familia de Nazaret que representa una respuesta coral a la voluntad del Padre: los tres miembros de esta familia se ayudan mutuamente a descubrir el plan de Dios. Rezaban, trabajaban, se comunicaban. Y yo me pregunto: ¿tú, en tu familia, sabes cómo comunicarte o eres como esos chicos que en la mesa, cada uno con un teléfono móvil, están chateando? En esa mesa parece que hay un silencio como si estuvieran en misa\ldots{} Pero no se comunican entre ellos. Debemos reanudar el diálogo en la familia: padres, madres, hijos, abuelos y hermanos deben comunicarse entre sí\ldots{} Es una tarea que hay que hacer hoy, precisamente en el Día de la Sagrada Familia. Que la Sagrada Familia sea un modelo para nuestras familias, para que padres e hijos se apoyen mutuamente en la fidelidad al Evangelio, fundamento de la santidad de la familia.

Confiemos a María \textquote{Reina de la Familia} todas las familias del mundo, especialmente las que sufren o están en peligro, e invoquemos sobre ellas su protección materna.


\section{Temas}