La vigilancia humilde del corazón

CEC 2729-2733:

\textbf{Frente a las dificultades de la oración}

\textbf{2729} La dificultad habitual de la oración es la \emph{distracción}. En la oración vocal, la distracción puede referirse a las palabras y al sentido de estas. La distracción, de un modo más profundo, puede referirse a Aquél al que oramos, tanto en la oración vocal (litúrgica o personal), como en la meditación y en la oración contemplativa. Dedicarse a perseguir las distracciones es caer en sus redes; basta con volver a nuestro corazón: la distracción descubre al que ora aquello a lo que su corazón está apegado. Esta humilde toma de conciencia debe empujar al orante a ofrecerse al Señor para ser purificado. El combate se decide cuando se elige a quién se desea servir (cf. \emph{Mt} 6,21.24).

\textbf{2730} Mirado positivamente, el combate contra el ánimo posesivo y dominador es la vigilancia, la sobriedad del corazón. Cuando Jesús insiste en la vigilancia, es siempre en relación a Él, a su Venida, al último día y al \textquote{hoy}. El esposo viene en mitad de la noche; la luz que no debe apagarse es la de la fe: \textquote{Dice de ti mi corazón: busca su rostro} (\emph{Sal} 27, 8).

\textbf{2731} Otra dificultad, especialmente para los que quieren sinceramente orar, es la \emph{sequedad}. Forma parte de la oración en la que el corazón está desprendido, sin gusto por los pensamientos, recuerdos y sentimientos, incluso espirituales. Es el momento en que la fe es más pura, la fe que se mantiene firme junto a Jesús en su agonía y en el sepulcro. \textquote{El grano de trigo, si [\ldots{}] muere, da mucho fruto} (\emph{Jn} 12, 24). Si la sequedad se debe a falta de raíz, porque la Palabra ha caído sobre roca, no hay éxito en el combate sin una mayor conversión (cf. \emph{Lc} 8, 6. 13).



\textbf{Frente a las tentaciones en la oración}

\textbf{2732} La tentación más frecuente, la más oculta, es nuestra \emph{falta de fe}. Esta se expresa menos en una incredulidad declarada que en unas preferencias de hecho. Cuando se empieza a orar, se presentan como prioritarios mil trabajos y cuidados que se consideran más urgentes; una vez más, es el momento de la verdad del corazón y de su más profundo deseo. Mientras tanto, nos volvemos al Señor como nuestro único recurso; pero ¿alguien se lo cree verdaderamente? Consideramos a Dios como asociado a la alianza con nosotros, pero nuestro corazón continúa en la arrogancia. En cualquier caso, la falta de fe revela que no se ha alcanzado todavía la disposición propia de un corazón humilde: \textquote{Sin mí, no podéis hacer nada} (\emph{Jn} 15, 5).

\textbf{2733} Otra tentación a la que abre la puerta la presunción es la \emph{acedia}. Los Padres espirituales entienden por ella una forma de aspereza o de desabrimiento debidos a la pereza, al relajamiento de la ascesis, al descuido de la vigilancia, a la negligencia del corazón. \textquote{El espíritu [\ldots{}] está pronto pero la carne es débil} (\emph{Mt} 26, 41). Cuanto más alto es el punto desde el que alguien toma decisiones, tanto mayor es la dificultad. El desaliento, doloroso, es el reverso de la presunción. Quien es humilde no se extraña de su miseria; ésta le lleva a una mayor confianza, a mantenerse firme en la constancia.

El que quisiere ver cuánto ha aprovechado en este camino de Dios, mire cuánto crece cada día en humildad interior y exterior. ¿Cómo sufre las injusticias de los otros? ¿Cómo sabe dar pasada a las flaquezas ajenas? ¿Cómo acude a las necesidades de sus prójimos? ¿Cómo se compadece y no se indigna contra los defectos ajenos? ¿Cómo sabe esperar en Dios en el tiempo de la tribulación? ¿Cómo rige su lengua? ¿Cómo guarda su corazón? ¿Cómo trae domada su carne con todos sus apetitos y sentidos? ¿ Cómo se sabe valer en las prosperidades y adversidades? ¿Cómo se repara y provee en todas las cosas con gravedad y discreción?

Y, sobre todo esto, mire si está muerto el amor de la honra, y del regalo, y del mundo, y según lo que en esto hubiere aprovechado o desaprovechado, así se juzgue, y no según lo que siente o no siente de Dios. Y por esto siempre ha de tener él un ojo, y el más principal en la mortificación, y el otro en la oración, porque esa misma mortificación no se puede perfectamente alcanzar sin el socorro de la oración.

\textbf{San Pedro de Alcántara}, \emph{Tratado sobre la Oración,} capítulo 5.