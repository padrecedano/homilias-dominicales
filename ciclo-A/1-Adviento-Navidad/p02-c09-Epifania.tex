\chapter{Epifanía del Señor}

	\section{Lecturas}
	
		\rtitle{PRIMERA LECTURA}
		
		\rbook{Del libro del profeta Isaías} \rred{60, 1-6}
		
		\rtheme{La Gloria del Señor amanece sobre ti}
		
		\begin{scripture}
			¡Levántate y resplandece, Jerusalén,
			
			porque llega tu luz;
			
			la gloria del Señor amanece sobre ti!
			
			Las tinieblas cubren la tierra,
			
			la oscuridad los pueblos,
			
			pero sobre ti amanecerá el Señor
			
			y su gloria se verá sobre ti.
			
			Caminarán los pueblos a tu luz,
			
			los reyes al resplandor de tu aurora.
			
			Levanta la vista en torno,
			
			mira: todos esos se han reunido,
			
			vienen hacia ti; llegan tus hijos desde lejos,
			
			a tus hijas las traen en brazos.
			
			Entonces lo verás y estarás radiante;
			
			tu corazón se asombrará, se ensanchará,
			
			porque la opulencia del mar se vuelca sobre ti,
			
			y a ti llegan las riquezas de los pueblos.
			
			Te cubrirá una multitud de camellos,
			
			dromedarios de Madián y de Efá.
			
			Todos los de Saba llegan trayendo oro e incienso,
			
			y proclaman las alabanzas del Señor.
		\end{scripture}
	
		\rtitle{SALMO RESPONSORIAL}
		
		\rbook{Salmo} \rred{71, 1-2. 7-8. 10-13}
		
		\rtheme{Se postrarán ante ti, Señor, todos los pueblos de la tierra}
		
		\begin{psbody}
			Dios mío, confía tu juicio al rey,
			tu justicia al hijo de reyes,
			para que rija a tu pueblo con justicia,
			a tus humildes con rectitud.
			
			En sus días florezca la justicia
			y la paz hasta que falte la luna;
			domine de mar a mar,
			del Gran Río al confín de la tierra.
			
			Los reyes de Tarsis y de las islas
			le paguen tributo.
			Los reyes de Saba y de Arabia
			le ofrezcan sus dones;
			postrense ante él todos los reyes,
			y sirvanle todos los pueblos.
			
			Él librará al pobre que clamaba,
			al afligido que no tenía protector;
			él se apiadará del pobre y del indigente,
			y salvará la vida de los pobres.
		\end{psbody}

		\rtitle{SEGUNDA LECTURA}
		
		\rbook{De la carta del apóstol san Pablo a los Efesios} \rred{3, 2-3a. 5-6}
		
		\rtheme{Ahora ha sido revelado que también los gentiles son coherederos de la
promesa}

		\begin{scripture}
			Hermanos:
			
			Habéis oído hablar de la distribución de la gracia de Dios que se me ha dado en favor de vosotros, los gentiles.
			
			Ya que se me dio a conocer por revelación el misterio, que no había sido manifestado a los hombres en otros tiempos, como ha sido revelado ahora por el Espíritu a sus santos apóstoles y profetas: que también los gentiles son coherederos, miembros del mismo cuerpo, y partícipes de la misma promesa en Jesucristo, por el Evangelio.
		\end{scripture}
	
		\rtitle{EVANGELIO}
		
		\rbook{Del Santo Evangelio según san Mateo} \rred{2, 1-12}
		
		\rtheme{Venimos a adorar al Rey}
		
		\begin{scripture}
			Habiendo nacido Jesús en Belén de Judea en tiempos del rey Herodes, unos magos de Oriente se presentaron en Jerusalén preguntando:
			
			«¿Dónde está el Rey de los judíos que ha nacido? Porque hemos visto salir su estrella y venimos a adorarlo».
			
			Al enterarse el rey Herodes, se sobresaltó y toda Jerusalén con él; convocó a los sumos sacerdotes y a los escribas del país, y les preguntó dónde tenía que nacer el Mesías.
			
			Ellos le contestaron:
			
			«En Belén de Judea, porque así lo ha escrito el profeta:
			
			``Y tú, Belén, tierra de Judá,
			
			no eres ni mucho menos la última
			
			de las poblaciones de Judá,
			
			pues de ti saldrá un jefe
			
			que pastoreará a mi pueblo Israel''».
			
			Entonces Herodes llamó en secreto a los magos para que le precisaran el tiempo en que había aparecido la estrella, y los mandó a Belén, diciéndoles:
			
			«Id y averiguad cuidadosamente qué hay del niño y, cuando lo encontréis, avisadme, para ir yo también a adorarlo».
			
			Ellos, después de oír al rey, se pusieron en camino y, de pronto, la estrella que habían visto salir comenzó a guiarlos hasta que vino a pararse encima de donde estaba el niño.
			
			Al ver la estrella, se llenaron de inmensa alegría. Entraron en la casa, vieron al niño con María, su madre, y cayendo de rodillas lo adoraron; después, abriendo sus cofres, le ofrecieron regalos: oro, incienso y mirra.
			
			Y habiendo recibido en sueños un oráculo, para que no volvieran a Herodes, se retiraron a su tierra por otro camino.
		\end{scripture}


\newsection

	\section{Comentarios Patrísticos}
	
		\subsection{San León Magno, papa}
		
			\ptheme{Dios ha manifestado su salvación en todo el mundo}
			
			\src{Sermón en la Epifanía del Señor 3, 1-3. 5: PL 54, 240-244.}
			
			\begin{body}
				La misericordiosa providencia de Dios, que ya había decidido venir en los últimos tiempos en ayuda del mundo que perecía, determinó de antemano la salvación de todos los pueblos en Cristo.
				
				De estos pueblos se trataba en la descendencia innumerable que fue en otro tiempo prometida al santo patriarca Abrahán, descendencia que no sería engendrada por una semilla de carne, sino por fecundidad de la fe, descendencia comparada a la multitud de las estrellas, para que de este modo el padre de todas las naciones esperara una posteridad no terrestre, sino celeste.
				
				Así pues, que todos los pueblos vengan a incorporarse a la familia de los patriarcas, y que los hijos de la promesa reciban la bendición de la descendencia de Abrahán, a la cual renuncian los hijos según la carne. Que todas las naciones, en la persona de los tres Magos, adoren al Autor del universo, y que Dios sea conocido, no ya sólo en Judea, sino también en el mundo entero, para que por doquier \emph{sea grande su nombre en Israel}.
				
				Instruidos en estos misterios de la gracia divina, queridos míos, celebremos con gozo espiritual el día que es el de nuestras primicias y aquél en que comenzó la salvación de los paganos. Demos gracias al Dios misericordioso, quien, según palabras del Apóstol, \emph{nos ha hecho capaces de compartir la herencia del pueblo santo en la luz; él nos ha sacado del dominio de las tinieblas y nos ha trasladado al reino de su Hijo querido}. Porque, como profetizó Isaías, \emph{el pueblo que caminaba en tinieblas vio una luz grande; habitaban en tierra de sombras, y una luz les brilló}. También a propósito de ellos dice el propio Isaías al Señor: \emph{Naciones que no te conocían te invocarán, un pueblo que no te conocía correrá hacia ti}.
				
				Abrahán vio \emph{este día, y se llenó de alegría,} cuando supo que sus hijos según la fe serían benditos en su descendencia, a saber, en Cristo, y él se vio a sí mismo, por su fe, como futuro padre de todos los pueblos, \emph{dando gloria a Dios, al persuadirse de que Dios es capaz de hacer lo que promete}.
				
				También David anunciaba este día en los salmos cuando decía: Todos los pueblos vendrán a postrarse en tu presencia, Señor; bendecirán tu nombre; y también: El Señor da a conocer su victoria, revela a las naciones su justicia.
				
				Esto se ha realizado, lo sabemos, en el hecho de que tres magos, llamados de su lejano país, fueron conducidos por una estrella para conocer y adorar al Rey del cielo y de la tierra. La docilidad de los magos a esta estrella nos indica el modo de nuestra obediencia, para que, en la medida de nuestras posibilidades, seamos servidores de esa gracia que llama a todos los hombres a Cristo.
				
				Animados por este celo, debéis aplicaros, queridos míos, a seros útiles los unos a los otros, a fin de que brilléis como hijos de la luz en el reino de Dios, al cual se llega gracias a la fe recta y a las buenas obras; por nuestro Señor Jesucristo que, con Dios Padre y el Espíritu Santo, vive y reina por los siglos de los siglos. Amén.
			\end{body}
		

\newsection


		\subsection{San Francisco de Sales, obispo}
		
			\ptheme{Regalemos lo más grande al Niño-Dios.}
			
			\src{Sermón VIII, 38.}
			
			\begin{scripture}
				\textquote{Unos magos, venidos de Oriente, llegaron a Jerusalén} (Mt 2,2).
				
				Es una gran fiesta, en la que celebramos que la Iglesia de los Gentiles es aceptada por Cristo y recibida por Cristo. Sí, es una gran fiesta porque los gentiles llegan a Cristo y a la Casa del Pan.
				
				La Epifanía es el día de los dones. Nunca ha recibido Cristo regalos más espléndidos y ahí tenemos la manera de ofrecer nuestros presentes a Dios. Los Magos nos lo pueden enseñar, ya que el primer acto de cada clase sirve de tipo a lo demás. Veamos, pues, las circunstancias: ¿Quién? ¿Qué? ¿A quién? ¿Por qué? ¿Cómo?
				
				¿Quién? Unos Reyes sabios. Antes de haber recibido la fe, ya creían. Reyes piadosos, que observaban las estrellas siguiendo la profecía de Balaam; su devoción se demuestra al dejar sus reinos y al acudir y presentarse intrépidamente al rey Herodes y confesarle ingenuamente su fe.
				
				¿Qué? Oro, incienso y mirra. Las opiniones de los doctores están divididas cuando explican la razón de estos presentes. Strabus dice que trajeron de lo que producía su país de Arabia. Todo agrada a Dios: Abel le daba de sus rebaños y el que no tenía sino una piel de cabra, también podía ofrecérsela. Honra al Señor con tus bienes.
				
				Hay quienes ofrecen al Señor lo que no poseen. Hijo mío, ¿por qué no eres más devoto? Lo seré en mi ancianidad. Pero, ¿sabes tú que llegarás a viejo? Otro dice: Si yo fuese capuchino, ofrecería sacrificios al Señor. Honra al Señor con lo que tienes. Si yo fuese rico\ldots{} yo daría\ldots{} Honra al Señor con tu pobreza. Si yo fuera santo\ldots{} Honra al Señor con tu paciencia, si yo fuera doctor\ldots{}, honra al Señor con tu sencillez\ldots{}
				
				De lo que tienes, el valor de tu ofrenda se mide en relación con lo que posees. San Agustín dice que los Magos le trajeron oro como a rey; incienso como a Dios; mirra como a hombre. ¿A quién? ¡A Cristo nuestro Señor! ¿Por qué? ¡Hemos venido a adorar al Señor! ¿Cómo? ¡Se postraron y le adoraron!
				
				Y no digamos que no tenemos nada muy grande para regalarle. Nada hay suficientemente digno de Dios. Debéis decir: \textquote{Yo quiero, Divino Niño, darte el único bien que poseo: yo mismo, y te ruego que aceptes este don}. Y Él nos responderá: \textquote{Hijo mío, tu regalo no es pequeño sino en tu propia estima}.
			\end{scripture}
		
			\begin{patercite}
				La estrella vino a pararse encima de donde estaba el niño. Por lo cual, los magos, al ver la estrella, se llenaron de inmensa alegría. Recibamos también nosotros esa inmensa alegría en nuestros corazones. Es la alegría que los ángeles anuncian a los pastores. Adoremos con los Magos, demos gloria con los pastores, dancemos con los ángeles. \emph{Porque hoy ha nacido un Salvador: el Mesías, el Señor. El Señor es Dios: él nos ilumina,} pero no en la condición divina, para atemorizar nuestra debilidad, sino en la condición de esclavo, para gratificar con la libertad a quienes gemían bajo la esclavitud. ¿Quién es tan insensible, quién tan ingrato, que no se alegre, que no exulte, que no se recree con tales noticias? Esta es una fiesta común a toda la creación: se le otorgan al mundo dones celestiales, el arcángel es enviado a Zacarías y a María, se forma un coro de ángeles, que cantan: \emph{Gloria a Dios en el cielo, y en la tierra, paz a los hombres que Dios ama}.
				
				Las estrellas se descuelgan del cielo, unos Magos abandonan la paganía, la tierra lo recibe en una gruta. Que todos aporten algo, que ningún hombre se muestre desagradecido. Festejemos la salvación del mundo, celebremos el día natalicio de la naturaleza humana. Hoy ha quedado cancelada la deuda de Adán. Ya no se dirá en adelante: \emph{Eres polvo y al polvo volverás,} sino: \textquote{Unido al que viene del cielo, serás admitido en el cielo}. Ya no se dirá más: \emph{Parirás hijos con dolor,} pues es dichosa la que dio a luz al Emmanuel y los pechos que le alimentaron. Precisamente por esto \emph{un niño nos ha nacido, un hijo se nos ha dado: lleva a hombros el principado}.
				
				(\textbf{San Basilio Magno}, \emph{Homilía} sobre la generación de Cristo: PG 31, 1471-1475).
			\end{patercite}

\newsection

	\section{Homilías}
	
		\subsection{San Juan Pablo II, papa}
		
			\subsubsection{Homilía (1981): Guiados por la luz de la fe}
			
				\src{Ordenación episcopal de 11 nuevos obispos. Basílica de San Pedro. \\Martes 6 de enero de 1981.}
				
				\begin{body}
					1. \textquote{Levántate, brilla, Jerusalén, que llega tu luz; la gloria del Señor amanece sobre ti!} \emph{(Is} 60, 1). Con estas palabras del \textbf{Profeta Isaías} la liturgia de hoy anuncia la celebración de una gran fiesta: \emph{la solemnidad de la Epifanía del Señor,} que es la culminación de la \emph{fiesta de Navidad;} del nacimiento de Dios
					
					Las palabras del Profeta se dirigen a Jerusalén, a la ciudad del Pueblo de Dios, a la ciudad de la elección divina. En esta ciudad la Epifanía debía alcanzar su cénit en los días del misterio pascual del Redentor.
					
					Sin embargo, por el momento, el Redentor es todavía un \emph{niño pequeño}. Yace en una pobre gruta cerca de Belén, y la gruta sirve de refugio para los animales. Allí encontró el primer albergue para Sí mismo sobre esta tierra. Allí le rodearon el amor de la Madre y la solicitud de José de Nazaret. Y allí tuvo lugar también \emph{el comienzo de la Epifanía:} de esa gran luz que debía penetrar los corazones, guiándolos por el camino de la fe hacia Dios, con el cual solamente por esta senda puede encontrarse el hombre: el hombre viviente con el Dios viviente.
					
					Hoy en este camino de la fe vemos a los tres nuevos hombres \emph{que vienen de Oriente,} de fuera de Israel. Son hombres sabios y poderosos, que vienen a Belén conducidos por la estrella en el firmamento celeste y por la luz interna de la fe en la profundidad de sus corazones.
					
					2. En este día, tan solemne, tan elocuente, os presentáis aquí \emph{vosotros,} venerados y queridos hijos, que por el acto de la ordenación debéis venir a ser hermanos nuestros en el Episcopado, en el servicio apostólico de la Iglesia. \emph{Os saludo} cordialmente en esta basílica, la cual se trasladó la luz de la Jerusalén mesiánica juntamente con la persona del Apóstol Pedro, que vino aquí guiado por el Espíritu Santo de acuerdo con la voluntad de Cristo.
					
					Aquí, en este lugar, medito con vosotros las palabras de la liturgia de hoy, en las que se manifiestan la luz de la Epifanía \emph{y la misión nacida} en los corazones de los hombres \emph{por la fe en Jesucristo}. Que esta luz resplandezca sobre vosotros de modo particular en el día de hoy, que brille continuamente en los caminos de vuestra vida y de vuestro ministerio. Que esta luz os guíe ---como la estrella de los Magos--- y os ayude a guiar a los demás de acuerdo con la sustancia de vuestra vocación en el Episcopado.
					
					\textquote{Los obispos ---ha recordado el Concilio Vaticano II--- como sucesores de los Apóstoles, reciben del Señor, a quien ha sido dado todo poder en el cielo y en la tierra, la misión de enseñar a todas las gentes y de predicar el Evangelio a toda criatura, a fin de que todos los hombres consigan la salvación por medio de la fe, del bautismo y del cumplimiento de los mandamientos (cf. \emph{Mt} 28, 18-20; \emph{Mc} 16, 15-16; \emph{Act} 26. 17 ss.). Para cumplir esta misión, Cristo Señor prometió a los Apóstoles el Espíritu Santo, y lo envió desde el cielo el día de Pentecostés, para que, confortados con su virtud, fuesen sus testigos hasta los confines de la tierra ante las gentes, los pueblos y los reyes (cf. \emph{Act} 1, 8; 2, 1 ss.; 9, 15). Este encargo que el Señor confió a los Pastores de su pueblo es un verdadero servicio, que en la Sagrada Escritura se llama con toda propiedad `diaconía', o sea, ministerio (cf. \emph{Act} 1, 17 y 25; 21, 19; \emph{Rom} 11, 13; \emph{1 Tim} 1, 12)} \emph{(Lumen gentium,} 24).
					
					3. Debéis ser, queridos hermanos, confesores de la fe, testigos de la fe, maestros de la fe. \emph{Debéis ser los hombres de la fe}. Contemplad este maravilloso acontecimiento que la solemnidad de hoy presenta a los ojos de nuestra alma.
					
					Un día, después de la venida del Espíritu Santo, se realizó en la comunidad de la Iglesia primitiva un gran cambio. El protagonista de este cambio fue \emph{Pablo de Tarso}. Escuchemos cómo habla en la liturgia de hoy: \textquote{Se me dio a conocer por revelación el misterio\ldots{}: que también los gentiles son coherederos, miembros del mismo cuerpo y partícipes de la promesa en Jesucristo, por el Evangelio} (\emph{Ef} 3, 3. 6).
					
					\emph{Este misterio,} en virtud del cual Pablo, y luego los otros Apóstoles, llevaron la luz del Evangelio más allá de las fronteras del Pueblo de la Antigua Alianza, este misterio \emph{se anuncia ya hoy}. Ya en el momento del nacimiento del Mesías: en su pesebre de Belén, en la coparticipación de la promesa que El ha venido a realizar, son llamados con la luz de la estrella y con la luz de la fe tres hombres que provienen de fuera de Israel.
					
					Estos tres hombres hablan de todos aquellos que deben seguir la misma luz mesiánica, tanto de Oriente como de Occidente, tanto del Norte como del Sur, para encontrar juntamente \textquote{con Abraham, Isaac y Jacob} la promesa del Dios viviente.
					
					Esta \emph{promesa se realiza} hoy ante los ojos de los Magos, tal como se realizó en la noche del nacimiento de Dios ante los ojos de los pastores, cerca de Belén.
					
					¡Oh, cuánto nos dicen hoy las palabras del \textbf{Profeta}, que interpela a Jerusalén: \textquote{Levanta la vista en torno, mira\ldots{} tu corazón se asombrará, se ensanchará} (\emph{Is} 60, 4-5).
					
					4. Queridos hijos y amados hermanos:
					
					Debéis convertiros en testigos singulares de la alegría que siente hoy la Jerusalén del Señor ¡Deben \emph{palpitar y dilatarse vuestros corazones} ante el misterio que contempláis! ¡Ante la luz a la que debéis servir!
					
					¡Qué grande es la fe de los Magos! ¡Qué seguros están de la luz que el Espíritu del Señor encendió en sus corazones! Con cuánta tenacidad la siguen. Con cuánta coherencia buscan al Mesías recién nacido. Y cuando finalmente llegaron a la meta, \textquote{\ldots{}se llenaron de inmensa alegría. Entraron en la casa, vieron al Niño con María, su Madre, y, \emph{cayendo de rodillas, lo adoraron;} después, abriendo sus cofres, le ofrecieron regalos: oro. incienso y mirra} \emph{(Mt} 2, 10-11).
					
					La luz de la fe les permitió escrutar todas las incógnitas. Los caminos incógnitos. las circunstancias incógnitas. Como cuando se hallaron ante el recién Nacido, un recién nacido humano que no tenía casa. Ellos se dieron cuenta de la miseria del lugar. ¡Qué contraste con su posición de hombres instruidos y socialmente influyentes! Y, sin embargo, \textquote{cayendo de rodillas, lo adoraron} (cf. \emph{Mt} 2, 11).
					
					Si este Niño, Cristo, hubiese podido hablar entonces, tal como habló después muchas veces, les debería haber dicho: ¡Hombres, qué grande es vuestra fe! Palabras semejantes a las que una vez, más tarde, escuchó la mujer cananea: \textquote{¡Grande es tu fe!} (cf. \emph{Mt} 15, 28).
					
					5. Queridos hermanos: Dentro de poco, también vosotros os inclinaréis profundamente, y os \emph{postraréis,} y tendidos sobre el pavimento de esta basílica, prepararéis vuestros corazones para la nueva venida del Espíritu Santo, para recibir sus dones divinos. Son \emph{los mismos dones} que iluminaron y robustecieron a los Magos en el camino de Belén, en el encuentro con el recién Nacido y, luego, en el camino de retorno y en toda su vida.
					
					A estos \textbf{dones divinos} ellos respondieron con un don: el oro, el incienso y la mirra, realidades que tienen también su significado simbólico. Teniendo presente ese significado, ofreced hoy vuestros dones, a vosotros mismos en don, y estad dispuestos a ofrecer, durante toda vuestra vida el amor, la oración, el sufrimiento.
					
					Y luego, \emph{levantaos, dirigíos por el camino} por el que os conducirá el Señor, guiándoos por las sendas de vuestra misión y de vuestro ministerio.
					
					¡Levantaos, robusteceos en la fe! Como testigos del ministerio de Dios. Como siervos del Evangelio y dispensadores de la potencia de Cristo. \emph{Y caminad a la luz de la Epifanía,} guiando a los otros a la fe y fortificando en la fe a todos los que encontréis.
					
					Que os acompañe siempre la sabiduría, la humildad y la valentía de los Magos de Oriente.
				\end{body}
			
			\subsubsection{Homilía (1984):}
			
				\src{Basílica Vaticana. Jueves 6 de enero de 1984.}
				
				\begin{body}
					1. Hoy, \emph{en el horizonte de la Navidad}, aparecen tres nuevas figuras: \emph{los Magos de Oriente.}
					
					Vienen de lejos, siguiendo la luz de la estrella que se les apareció. Se dirigen a Jerusalén, llegan a la corte de Herodes. Preguntan: \textquote{¿Dónde está el \emph{rey de los judíos} que ha nacido? Vimos salir su estrella y venimos a adorarlo} (\emph{Mt} 2, 2).
					
					2. En la liturgia de la Iglesia, la solemnidad de hoy lleva el nombre de \emph{la Epifanía del Señor. Epifanía significa manifestación.}
					
					Esta expresión nos invita a pensar no solo en la estrella que apareció a los ojos de los Magos, no solo en el camino que toman estos hombres de Oriente, siguiendo el signo de la estrella. La epifanía nos invita \emph{a pensar en el camino interior}, en cuyo comienzo se encuentra el misterioso encuentro del intelecto y el corazón humano \emph{con la luz de Dios mismo.}
					
					\textquote{La luz \ldots{} que ilumina a todo hombre cuando viene al mundo} (cf. \emph{Jn} 1, 9).
					
					Los tres personajes del Oriente seguían esta luz con certeza incluso antes de que apareciera la estrella.
					
					Dios les habló \emph{con la elocuencia de toda la creación}: les dijo que Él es, que Él existe; que es el Creador y Señor del mundo.
					
					En cierto momento, más allá del velo de las criaturas, las acercó más \emph{aún a sí mismo.} Y, al mismo tiempo, comenzó a confiarles \emph{la verdad de su venida al mundo.} Ellos, de alguna manera, se han dado cuenta del plan divino de salvación.
					
					Los magos \emph{respondieron con fe} a esa epifanía interior de Dios.
					
					3. Esta fe les permitió reconocer el significado de la estrella. Esta fe también les ordenó partir. Fueron a Jerusalén, la capital de Israel, donde la verdad sobre la venida del Mesías fue transmitida de generación en generación. Los Profetas lo habían predicado y los libros sagrados habían escrito al respecto.
					
					Dios que habló con la Epifanía en los corazones de los Magos, \emph{había hablado} a través de los siglos al \emph{Pueblo Elegido} y había profetizado la misma verdad sobre su venida.
					
					4. Esta verdad se cumplió la noche del nacimiento de Dios en Belén. Esa \emph{noche ya es la Epifanía de Dios}, que vino: Dios que nació de la Virgen y fue puesto en el pesebre. Dios que \emph{ocultó su venida} en la pobreza de su nacimiento en Belén: he aquí \emph{la Epifanía del divino} ocultamiento.
					
					Solo un \emph{grupo de pastores se} había apresurado a aquel encuentro \ldots{}
					
					Pero ahora vienen \emph{los magos.} Dios, que se esconde de los ojos de los hombres que viven cerca de él, se \emph{revela} a los hombres que vienen de lejos.
					
					El profeta dice en Jerusalén: \textquote{Caminarán las naciones a tu luz, y los reyes al resplandor de tu alborada. Alza los ojos en torno y mira: todos se reúnen y vienen a ti. Tus hijos vienen de lejos, y tus hijas son llevadas en brazos}. (\emph{Is} 60, 3-4).
					
					El Concilio habla de esta fuerza de la siguiente manera:
					
					\textquote{Dispuso Dios en su sabiduría revelarse a Sí mismo y dar a conocer el misterio de su voluntad (cf. \emph{Ef} 1, 9), mediante el cual los hombres, por medio de Cristo, Verbo encarnado, tienen acceso al Padre en el Espíritu Santo y se hacen consortes de la naturaleza divina (cf. \emph{Ef} 2, 18; \emph{2 Pe} 1, 4). En consecuencia, por esta revelación, Dios invisible (cf. \emph{Col} 1, 15; \emph{1 Tim} 1, 17) habla a los hombres como amigos (cf. \emph{Ex} 33, 11; \emph{Jn} 15, 14-15), movido por su gran amor y mora con ellos, para invitarlos a la comunicación consigo y recibirlos en su compañía} (\emph{Dei Verbum}, 2).
					
					Los Magos de Oriente llevan consigo esa fuerza interior de la Epifanía que les \emph{permite reconocer al Mesías} en el Niño acostado en el pesebre. Esta fuerza les ordena postrarse ante él y ofrecer los dones: oro, incienso y mirra (cf. \emph{Mt} 2, 11).
					
					Los Magos son al mismo tiempo un presagio de que la fuerza interior de la Epifanía se extenderá ampliamente entre los pueblos de la tierra.
					
					El profeta dice: \textquote{Tú entonces al verlo te pondrás radiante, se estremecerá y se ensanchará tu corazón, porque vendrán a ti los tesoros del mar, las riquezas de las naciones vendrán a ti} (\emph{Is} 60, 5).
					
					6. Queridos hermanos (\ldots{}) \emph{la solemnidad de hoy} manifiesta al Señor al mundo entero, porque su venida es para todos.
					
					{[}Que esta celebración sea para nosotros{]} una nueva \emph{llamada} a someter toda la vida a la fuerza interior de la Epifanía, a través de la cual el Dios infinito confía a cada uno su misterio salvífico en Jesucristo, nacido en la noche de Belén de la Virgen Madre.
					
					Acepta hoy esta llamada que te dirige la Iglesia.
					
					Deja que \emph{este poder divino irradie} en tu corazón como en una Jerusalén interior, a la que la liturgia de hoy dice: \textquote{Levántate, revestido de luz, / que viene tu luz, / la gloria del Señor brilla sobre ti} (\emph{Is} 60, 1).
					
					Deja que \emph{el poder salvífico de la divina Epifanía irradie} entre los hombres y pueblos a quienes eres enviado, como testigo de verdad y misericordia.
					
					Verdaderamente: \textquote{Los bienes de los pueblos vendrán a ti} (\emph{Is} 60, 5).
					
					Y responde \emph{al don} de la solemnidad de hoy con un incesante y continuo \emph{don}: ofrece oro, incienso y mirra.
					
					Así \emph{la abundancia de la divina Epifanía} quedará en vosotros y se renovará en el camino del servicio apostólico. \emph{Amén.}					
				\end{body}
			
			
			\subsubsection{Homilía (1987):}
			
				\src{Martes 6 de enero de 1987. \\Ordenación de diez nuevos obispos.}
				
				\begin{body}
					1. \emph{Levántate, Jerusalén}, que viene tu luz. Levántate Jerusalén, vestida de luz (\emph{Is} 60, 1).
					
					Estas palabras del profeta adquieren hoy una relevancia particular. De hecho, los Magos de Oriente llegan a Jerusalén precisamente con esta noticia: \textquote{\emph{¡Viene tu luz!}}
					
					¿Dónde buscar el lugar de su nacimiento?
					
					Jerusalén es la ciudad de un gran Rey, más grande que Herodes, y este gobernante temporal, que se sienta en el trono de Israel con el consentimiento de Roma, \emph{no puede ocultar la promesa de un Rey mesiánico}.
					
					Es incapaz de oscurecer su luz.
					
					¿Dónde buscar el lugar de nacimiento del Mesías?
					
					Los escritos del Antiguo Testamento responden con certeza: \emph{en Belén}.
					
					Encuentran un recién nacido. Ningún obstáculo externo puede apagar la luz \emph{que llevan en sus corazones}.
					
					No prestan atención a la pobreza del lugar. Se postran. Le adoran. Ofrecen sus dones.
					
					2. ¡Jerusalén, ha llegado tu luz!
					
					\emph{¡Jerusalén vestida de luz!}
					
					Hombres de lejos, \textquote{los reyes de Tarsis y las islas \ldots{} los reyes de Saba y de Arabia} (\emph{Sal} 72, 10) caminan a tu luz.
					
					Y la luz brilla en la oscuridad.
					
					\emph{¡Jerusalén! El destino que Dios te ha dado es brillar}. Y ninguna oscuridad de la historia del hombre puede quitarte este destino, esta vocación.
					
					Esta es la certeza expresada en nuestro tiempo por los Padres del Concilio Vaticano II, cuando comenzaron su documento sobre la Iglesia con las palabras \emph{Lumen gentium}: \textquote{Lumen gentium cum sit Christus}.
					
					¿Quién eres tú, Iglesia? ¿Qué dices de ti misma?
					
					\textquote{Mi destino es brillar. \emph{¡Brilla con esta luz, que es Cristo!}}.
					
					Aquí está la elocuencia de la solemnidad de la Epifanía.
					
					3. Esta elocuencia, esta verdad sobre Jerusalén, esta enseñanza sobre la Iglesia (\ldots{}) manifiesta \emph{el carácter misionero de la Iglesia}, que se siente irrevocablemente enviada a todos los pueblos y a todos los hombres.
					
					Quizás no apareció la estrella que una vez indujo a los Magos de Oriente a ponerse en el horizonte de tu cielo. Pero en tu alma \emph{hay la misma luz interior que los ha guiado}.
					
					La misma luz te guía a ti también: \textquote{\emph{Lumen Gentium}}.
					
					4. En este momento (\ldots{}) me dirijo a cada uno de vosotros con las palabras del profeta:
					
					\textquote{Se estremecerá y se ensanchará tu corazón, porque vendrán a ti los tesoros del mar, las riquezas de las naciones vendrán a ti} (\emph{Is} 60, 5).
					
					{[} \ldots{}{]}
					
					Mira: los magos de Oriente ofrecen los regalos que han traído. Abren sus tesoros y presentan los regalos.
					
					Hoy estás llamado a abrir tu tesoro interior aún más profundamente. Estás llamado a entregarte aún más plenamente a Cristo, que es la Epifanía del Pastor eterno.
					
					A través de tu dedicación, que la luz que es Cristo brille sobre todos aquellos a quienes eres enviado.
				\end{body}
			
			
			\subsubsection{Homilía (1990):}
			
				\src{Sábado 6 de enero de 1990. Ordenación Episcopal de Doce nuevos Obispos.}
				
				\begin{body}
					\textquote{Entraron en la casa; vieron al niño con María su madre y, postrándose, le adoraron; abrieron luego sus cofres y le ofrecieron dones de oro, incienso y mirra} (cf. \emph{Mt} 2, 11).
					
					1. Estas palabras del Evangelio de Mateo contienen como una síntesis del misterio, que se expresa con el sustantivo griego \textquote{Epifanía}. Los Magos, procedentes de Oriente, colocan los regalos a los pies del Niño de Belén: oro, incienso y mirra.
					
					Estos dones son la respuesta al Don. El Don de arriba les fue anunciado por medio de la estrella brillante en la oscuridad. Los sabios de Oriente siguen la estrella, luego en Jerusalén le piden información a Herodes. Reciben explicaciones que se les dan con las mismas palabras que el Profeta. Van con perseverancia a Belén para recibir el Don de lo alto. Ellos llaman a este Don \textquote{el rey de los judíos que ha nacido}, mientras que el Profeta lo llama \textquote{un jefe que pastoreará al pueblo}. Él es el que el Padre ungió con el Espíritu Santo y envió al mundo: el Mesías. El Hijo Unigénito, dado por el Padre.
					
					Los Magos encuentran al Niño en los brazos de la Madre: encuentran al Hijo del hombre. Saben que él es el Don del Padre. Vienen de lejos para acogerlo: acoger el Don en el que el Eterno expresa su amor: \textquote{Porque tanto amó Dios al mundo que le dio a su Hijo Unigénito} (\emph{Jn} 3, 16). Los magos de Oriente están entre los primeros en darle la bienvenida. Son los testigos de la divina Epifanía.
					
					2. ¡Queridos hijos y hermanos! {[}Esta solemnidad nos invita a{]} acoger el mismo Don: acoger a Cristo, nacido en Belén, a quien el Padre envió al mundo. El Hijo unigénito que el Padre dio por amor al mundo que es suyo, para que todo el que crea en él no muera, mas tenga vida eterna.
					
					{[} \ldots{}{]}
					
					3. Respecto a los dones que los Magos de Oriente ofrecieron al Niño recién nacido, el profeta Isaías había dicho las siguientes palabras: \textquote{Todos ellos de Sabá vienen portadores de oro e incienso y pregonando alabanzas al Señor} (\emph{Is} 60, 6).
					
					E Isaías había dicho esto mientras tenía ante sus ojos la maravillosa procesión de las naciones, que iban a caminar hacia esa luz, que brillaría sobre Jerusalén: \textquote{Alza los ojos en torno y mira: todos se reúnen y vienen a ti. Tus hijos vienen de lejos, y tus hijas son llevadas en brazos. Tú entonces al verlo te pondrás radiante, se estremecerá y se ensanchará tu corazón, porque vendrán a ti los tesoros del mar, las riquezas de las naciones vendrán a ti} (\emph{Is} 60, 4-5).
					
					4. Cada uno de vosotros, queridos hijos, trae a este altar, {[}en la Basílica de San Pedro,{]} su \textquote{don propio}: el oro, el incienso y la mirra de su propia vida. Este don que brilla en vuestro corazón a través de la luz del Espíritu de la Verdad, que madura en el momento del ofertorio de hoy. Vuestro don debe hoy ser consagrado de nuevo y convertirse en una respuesta particular al Don de la divina Epifanía en Jesucristo.
					
					La Epifanía es la fiesta del intercambio de regalos. Entonces, cada uno de vosotros trae aquí no solo su propio regalo. A través de cada uno de vosotros se expresa \textquote{toda la riqueza de la capacidad de los pueblos}. (\ldots{}).
					
					Así, por el don que es propio de cada uno y por \textquote{toda la riqueza de la capacidad de los pueblos} que cada uno lleva en su interior, todos crecemos en esta comunión particular que es la Iglesia, el Cuerpo de Cristo, y al mismo tiempo todos formamos y enriquecemos esta comunión.
					
					5. {[} \ldots{}{]}
					
					Queridos hermanos (\ldots{}) que el Hijo de Dios nacido de la Virgen acoja vuestros dones, como acogió los dones de los Magos de Oriente, y os ayude a revelar siempre, con la luz y el poder del Espíritu Santo, a todos los hombres, a todos los pueblos y naciones de la tierra, el don del Hijo Eterno dado por el Padre, para que \textquote{todo el que crea en él no se pierda, mas tenga la vida eterna}.
					
					{[} \ldots{}{]}
				\end{body}
			
			\subsubsection{Homilía (1993):} 
			
				\src{Miércoles 6 de enero de 1993. Ordenación Episcopal de Once Nuevos Obispos.}
				
				\begin{body}
					\emph{Todos los reyes se postrarán ante él, todas las naciones le servirán} (\emph{Sal} 71, 11).
					
					1. La Constitución Dogmática \emph{Lumen gentium} del Concilio Ecuménico Vaticano II, citando un famoso pasaje de San Juan Crisóstomo, subraya esa universalidad del único pueblo de Dios que brilla de manera especial en la celebración de hoy.
					
					\textquote{Todos los hombres están llamados a formar parte del nuevo Pueblo de Dios. Por lo cual, este pueblo, sin dejar de ser uno y único, debe extenderse a todo el mundo y en todos los tiempos, para así cumplir el designio de la voluntad de Dios, quien en un principio creó una sola naturaleza humana, y a sus hijos, que estaban dispersos, determinó luego congregarlos (cf. \emph{Jn} 11,52). Para esto envió Dios a su Hijo, a quien constituyó en heredero de todo (cf. \emph{Hb} 1,2), para que sea Maestro, Rey y Sacerdote de todos, Cabeza del pueblo nuevo y universal de los hijos de Dios. Para esto, finalmente, envió Dios al Espíritu de su Hijo, Señor y Vivificador, quien es para toda la Iglesia y para todos y cada uno de los creyentes el principio de asociación y unidad en la doctrina de los Apóstoles, en la mutua unión, en la fracción del pan y en las oraciones (cf. \emph{Hch} 2,42 gr.)} (\emph{Lumen} gentium, 13).
					
					2. \textquote{Entraron en la casa; vieron al niño con María su madre y, postrándose, le adoraron} (\emph{Mt} 2, 11).
					
					En la liturgia de hoy, la Iglesia revive la verdad de estas palabras. De hecho, mientras la noche de Navidad contemplamos el correr presuroso a la gruta de Belén de unos pastores pertenecientes al pueblo de Israel, hoy --- solemnidad de la Epifanía --- recordamos la llegada de los Magos, que venían del Lejano Oriente para adorar al Rey y Salvador universal en el Niño recién nacido y ofrecerle sus dones.
					
					Los magos vienen preguntando: \textquote{¿Dónde está el rey de los judíos que nació?}. Y vienen trayendo regalos (\emph{Mt} 2, 2). Cuando llegaron al \textquote{lugar donde estaba el niño \ldots{} se postraron y lo adoraron. Entonces abrieron sus cofres y le ofrecieron regalos de oro, incienso y mirra} (\emph{Mt} 2, 9. 11).
					
					Habían preguntado al recién nacido \textquote{rey de los judíos} (\emph{Mt} 2, 2) y he aquí, ahora está ante ellos el rey no de un solo pueblo, sino de todas las naciones que le \textquote{han sido entregadas en posesión} (cf. \emph{Sal} 2, 8). De esta manera se hizo realidad la verdad ya anunciada por el salmista hace mucho tiempo. Los Magos con su gesto de adoración, por tanto, testifican que el niño Jesús no es rey de un solo pueblo, el pueblo de Israel, sino de todos los pueblos de la tierra.
					
					3. Epifanía. La solemnidad de hoy lleva este nombre significativo. Los Magos, que vinieron de Oriente para ofrecer sus dones, se convierten en testigos del don santísimo ofrecido por Dios a los hombres: en el misterio de la Encarnación, Dios Padre ofrece a la humanidad su Hijo único, consustancial con él. En el Verbo hecho carne, el amor mismo de Dios se hizo visible: \textquote{Tanto amó Dios al mundo que dio a su Hijo unigénito} (\emph{Jn 3, 16}). Los Magos, en presencia de este santísimo misterio, se postraron. Aunque sus ojos ven sólo a un niño recién nacido, la luz que los guía hacia adentro les permite reconocer lo que sus ojos no pueden percibir: les permite comprender el don santísimo ofrecido por Dios a la humanidad. Responden a este don santísimo con su don personal, presentando a Jesús oro, incienso y mirra, simples símbolos humanos, que expresan el misterio del rey a quien \textquote{todas las naciones han sido entregadas en posesión} (cf. \emph{Sal} 2, 8).
					
					Al mismo tiempo, los dones ponen de manifiesto la profunda verdad de la Encarnación. Recibiendo la naturaleza humana, el Hijo de Dios también quiso compartir la amargura de la existencia terrena, que encontró su punto culminante cuando, en la agonía de la cruz, \textquote{le ofrecieron vino mezclado con mirra} (\emph{Mc} 15, 23).
					
					{[} \ldots{}{]}
					
					{[}Que esta celebración nos revele{]} el don santísimo que, en Jesucristo, Dios concedió al mundo y la humanidad y lleguemos a conocer con qué inmenso amor Dios nos amó, sin dudar en dar por nosotros a su Hijo unigénito.
					
					\textquote{A él se postrarán todos los reyes, todas las naciones le servirán}. ¡Para siempre! Amén.
				\end{body}
			
			\subsubsection{Homilía (1996):} 
			
				\src{Ordenación en la fiesta de la Epifanía. \\Basílica Vaticana. Sábado 6 de enero de 1996.}
				
				\begin{body}
					1. La fiesta que celebramos hoy, 6 de enero, lleva el nombre de \textquote{Epifanía}. La palabra griega \emph{epifania} significa \textquote{revelación}, \textquote{manifestación}. Se dice en la Carta a Tito: \textquote{La gracia de Dios ha sido revelada, portadora de salvación para todos los hombres \ldots{}} (2, 11), y nuevamente: \textquote{La bondad de Dios, nuestro Salvador, y su amor por los hombres han sido revelados \ldots{}} (3, 4). \emph{La revelación es precisamente el descubrimiento del misterio de Dios Salvador}. Existe un estrecho vínculo de sentido entre unos y otros, entre la revelación y el misterio de la salvación.
					
					\emph{El Creador le dio al hombre la capacidad de conocer el mundo}, las cosas visibles, los hechos históricos; también le dio la capacidad de penetrar con su propia razón más allá de la superficie de lo sensible. Pero Dios también vino al encuentro del hombre \emph{hablándole directamente}. La Revelación consiste precisamente en esto: \emph{Dios ha hablado al hombre revelándole} lo que sabe y piensa de sí mismo, del hombre, del mundo. Así, gracias a la revelación, conocemos el pensamiento de Dios, lo conocemos \emph{por nuestra razón}, pero \emph{no en virtud de nuestra razón}. Lo que Dios revela lo aceptamos porque \emph{confiamos en Él}. Esta \emph{confianza en nosotros mismos a la autoridad de Dios que se revela} se llama \emph{fe}.
					
					Somos conscientes de que solo Dios mismo puede instruir al hombre sobre las realidades divinas. En la Constitución conciliar \emph{Del Verbum} sobre la Divina Revelación leemos: \textquote{Agradó a Dios, en su bondad y sabiduría, revelarse y manifestar el misterio de su voluntad (cf. \emph{Ef} 1, 9) \ldots{} Con esta revelación, en efecto, el \emph{Dios invisible} (cf. \emph{Col} 1, 15; \emph{1 Tm} 1,17) \emph{en su inmenso amor habla a los hombres como a amigos} (cf. \emph{Ex} 33, 11; \emph{Jn} 15, 14-15) \emph{y entretiene con ellos} (cf. \emph{Bar} 3, 38), \emph{para invitarlos y admitirlos a la comunión con Él}} (n. 2).
					
					El hecho de que Dios quisiera revelar al hombre la verdad sobre sí mismo, una verdad que es un misterio, atestigua que el hombre es una criatura muy querida por Dios, una criatura hecha a su semejanza, la única en el mundo visible con quien Dios puede diálogar, al que puede confiar la verdad sobre sí mismo y su vida íntima, la verdad de sus misterios divinos.
					
					2. \textquote{Hemos visto salir su estrella y hemos venido a adorarle} (\emph{Mt} 2, 2).
					
					Los Magos de Oriente pronunciaron estas palabras en Jerusalén, frente al rey Herodes, quien no solo las entendió de una manera puramente humana, sino incluso con pérfida envidia. Sin embargo, estas palabras resumen \emph{la revelación sobre el nacimiento del Señor}. Los Sabios de Oriente, junto con los pastores de Belén, son los que fueron iniciados por Dios mismo, podríamos decir, fueron iniciados en el misterio de la encarnación del Hijo de Dios. Los pastores ya estaban casi en el lugar, cerca de la \textquote{Ciudad de David}. En cambio, los magos vinieron de lejos, interpretando las señales que indicaban el momento y el lugar del nacimiento del Salvador. Y una señal particular fue la estrella, que los guió hacia la tierra de Israel: primero a Jerusalén, y luego a Belén.
					
					En el signo visible de la estrella, el Dios invisible les habló. ¿Cómo pudo haber sucedido que, entre tantas estrellas observadas por los Sabios en la bóveda celeste, precisamente aquella estrella les hablara del nacimiento del Hijo de Dios en carne humana? \emph{Esto fue posible solo a través de la fe}. Los magos, habiendo llegado a Jerusalén, buscaron la confirmación de su intuición de los escribas, expertos en la revelación de Dios a Israel. Y obtuvieron la respuesta: el profeta Miqueas había anunciado que el Mesías nacería en Belén (cf. \emph{Mi} 5, 1). Fueron, pues, a Belén y entraron en la casa donde estaba el Niño, junto con su Madre y José; cayeron de rodillas y ofrecieron sus regalos simbólicos. Todo esto testifica que la \emph{fe los había introducido por el camino correcto hasta el centro mismo del misterio del nacimiento del Señor}.
					
					{[} \ldots{}{]}
					
					4. \textquote{Levántate, vístete de luz, porque \emph{tu luz viene, la gloria del Señor brilla sobre ti} \ldots{} Pueblos caminarán a tu luz, reyes al esplendor de tu ascenso. Levanta tus ojos alrededor y mira: todos estos se han reunido, vienen a ti. Tus hijos vienen de lejos, tus hijas son llevadas en tus brazos} (\emph{Is} 60, 1,3-4).
					
					Los Magos, que llegaron a Belén desde Oriente, constituyen las primicias \emph{de la gran peregrinación de la fe}, que transcurre de generación en generación, acercando a los hombres, pueblos y naciones a Cristo, la luz del mundo. Numerosos pueblos y naciones han participado en esta peregrinación, que se lleva a cabo desde hace casi dos mil años. Y la luz, que se elevó sobre Jerusalén en la plenitud de los tiempos, no se apaga, sino que brilla con un resplandor siempre nuevo. Ilumina el camino de la humanidad en medio de la oscuridad que envuelve la tierra. Y continuamente, durante la noche de la que habla el profeta Isaías (cf. \emph{Is} 60, 2), resuena el grito de los pastores, de los magos, de todos los creyentes de todas las épocas: \textquote{Christus apparuit nobis, venite adoremus}.
				\end{body}
			
			\subsubsection{Homilía (1999): La luz de la razón y de la fe}
			
				\src{Basílica de San Pedro. 6 de enero de 1999.}
				
				\begin{body}
					1. \textquote{La luz brilla en las tinieblas, pero las tinieblas no la acogieron} (Jn 1, 5).
					
					Toda la liturgia habla hoy de la \emph{\textbf{luz de Cristo}}, de la luz que se encendió en la noche santa. La misma luz que guió a los pastores hasta el portal de Belén indicó el camino, el día de la Epifanía, a los Magos que fueron desde Oriente para adorar al Rey de los judíos, y resplandece para todos los hombres y todos los pueblos que anhelan encontrar a Dios.
					
					En su búsqueda espiritual, el ser humano ya dispone naturalmente de una luz que lo guía: es la razón, gracias a la cual puede orientarse, aunque a tientas (cf. \emph{Hch} 17, 27), hacia su Creador. Pero, dado que es fácil perder el camino, Dios mismo vino en su ayuda con la luz de la revelación, que alcanzó su plenitud en la encarnación del Verbo, Palabra eterna de verdad.
					
					La Epifanía celebra la aparición en el mundo de esta luz divina, con la que Dios salió al encuentro de la débil luz de la razón humana. Así, en la solemnidad de hoy, se propone la íntima relación que existe entre la razón y la fe, las dos alas de que dispone el espíritu humano para elevarse hacia la contemplación de la verdad, como recordé en la reciente encíclica \emph{Fides et ratio}.
					
					2. Cristo no es sólo luz que ilumina el camino del hombre. \emph{También se ha hecho camino} para sus pasos inciertos hacia Dios, fuente de vida. Un día dijo a los Apóstoles: \textquote{Yo soy el camino, la verdad y la vida. Nadie va al Padre sino por mí. Si me conocéis a mí, conoceréis también a mi Padre; desde ahora lo conocéis y lo habéis visto} (\emph{Jn} 14, 6-7). Y, ante la objeción de Felipe, añadió: \textquote{El que me ha visto a mí, ha visto al Padre. (\ldots{}) Yo estoy en el Padre y el Padre está en mí} (\emph{Jn} 14, 9. 11). \emph{La epifanía del Hijo es la epifanía del Padre}.
					
					¿No es éste, en definitiva, el objetivo de la venida de Cristo al mundo? Él mismo afirmó que había venido para \textquote{dar a conocer al Padre}, para \textquote{explicar} a los hombres quién es Dios y para revelar su rostro, su \textquote{nombre} (cf. \emph{Jn} 17, 6). La vida eterna consiste en el encuentro con el Padre (cf. \emph{Jn} 17, 3). Por eso, ¡cuán oportuna es esta reflexión, especialmente durante el año dedicado al Padre!
					
					La Iglesia prolonga en los siglos la misión de su Señor: su compromiso principal consiste en dar a conocer a todos los hombres el rostro del Padre, reflejando la luz de Cristo, \emph{lumen gentium}, luz de amor, de verdad y de paz. Para esto el divino Maestro envió al mundo a los Apóstoles, y envía continuamente, con el mismo Espíritu, a los obispos, sus sucesores\ldots{}
					
					{[}\ldots{}{]}
					
					4. El mundo, en el umbral del tercer milenio, tiene gran necesidad de \emph{experimentar la bondad divina}; de sentir el amor de Dios a toda persona.
					
					También a nuestra época se puede aplicar el oráculo del \textbf{profeta Isaías}, que acabamos de escuchar: \textquote{La oscuridad cubre la tierra, y espesa nube a los pueblos, mas sobre ti amanece el Señor y su gloria sobre ti aparece} (Is 60, 2-3). En el paso, por decirlo así, del segundo al tercer milenio, la Iglesia está llamada a revestirse de luz (cf. Is 60, 1), para resplandecer como una ciudad situada en la cima de un monte: la Iglesia no puede permanecer oculta (cf. \emph{Mt} 5, 14), porque los hombres necesitan recoger su mensaje de luz y esperanza, y glorificar al Padre que está en los cielos (cf. \emph{Mt} 5, 16).
					
					Conscientes de esta tarea apostólica y misionera, que compete a todo el pueblo cristiano, pero especialmente a cuantos el Espíritu Santo ha puesto como obispos para pastorear la Iglesia de Dios (cf. \emph{Hch} 20, 28), vamos como peregrinos a Belén, a fin de unirnos a los Magos de Oriente, mientras ofrecen dones al Rey recién nacido.
					
					Pero el verdadero don es él: Jesús, el don de Dios al mundo. Debemos acogerlo a él, para llevarlo a cuantos encontremos en nuestro camino. Él es para todos la epifanía, la manifestación de Dios, \emph{esperanza del hombre}, de Dios, \emph{liberación del hombre}, de Dios, \emph{salvación del hombre}.
					
					Cristo nació en Belén por nosotros.
					
					Venid, adorémoslo. Amén.
				\end{body}
			
			\subsubsection{Homilía (2002): Luz que guía en la noche}
			
				\src{Ordenación episcopal de diez presbíteros. \\Domingo 6 de enero de 2002.}
				
				\begin{body}
					1. \textquote{\emph{Lumen gentium (\ldots{}) Christus}, Cristo es la luz de los pueblos} (\emph{Lumen gentium}, 1).
					
					El \emph{tema de la luz} domina las solemnidades de la Navidad y de la Epifanía, que antiguamente -y aún hoy en Oriente- estaban unidas en una sola y gran \textquote{fiesta de la luz}. En el clima sugestivo de la Noche santa apareció la luz; nació Cristo, \textquote{luz de los pueblos}. Él es el \textquote{sol que nace de lo alto} (\emph{Lc} 1, 78), el sol que vino al mundo para disipar las tinieblas del mal e inundarlo con el esplendor del amor divino. El evangelista san Juan escribe: \textquote{La luz verdadera, viniendo a este mundo, ilumina a todo hombre} (\emph{Jn} 1, 9).
					
					\textquote{\emph{Deus lux est} --- Dios es luz}, recuerda también san Juan, sintetizando no una teoría gnóstica, sino \textquote{el mensaje que hemos oído de él} (\emph{1 Jn} 1, 5), es decir, de Jesús. En el evangelio recoge las palabras que oyó de los labios del Maestro: \textquote{Yo soy la luz del mundo; el que me siga no caminará en la oscuridad, sino que tendrá la luz de la vida} (\emph{Jn} 8, 12).
					
					Al encarnarse, el Hijo de Dios \emph{se manifestó como luz}. No sólo luz externa, en la historia del mundo, sino también \emph{dentro del hombre}, en su historia personal. Se hizo uno de nosotros, dando sentido y nuevo valor a nuestra existencia terrena. De este modo, respetando plenamente la libertad humana, Cristo se convirtió en \textquote{\emph{lux mundi}, la luz del mundo}. Luz que brilla en las tinieblas (cf. \emph{Jn} 1, 5).
					
					2. Hoy, solemnidad de la Epifanía, que significa \textquote{manifestación}, se propone de nuevo con vigor el tema de la luz. Hoy el Mesías, que se manifestó en Belén a humildes pastores de la región, sigue revelándose como luz de los pueblos de todos los tiempos y de todos los lugares. Para los Magos, que acudieron de Oriente a adorarlo, la luz del \textquote{rey de los judíos que ha nacido} (\emph{Mt} 2, 2) toma la forma de un astro celeste, tan brillante que atrae su mirada y los guía hasta Jerusalén. Así, les hace seguir los indicios de las antiguas profecías mesiánicas: \textquote{De Jacob avanza una estrella, un cetro surge de Israel\ldots{}} (\emph{Nm} 24, 17).
					
					¡Cuán sugestivo es el \emph{símbolo de la estrella}, que aparece en toda la iconografía de la Navidad y de la Epifanía! Aún hoy evoca profundos sentimientos, aunque como tantos otros signos de lo sagrado, a veces corre el riesgo de quedar desvirtuado por el uso consumista que se hace de él. Sin embargo, la estrella que contemplamos en el belén, situada en su contexto original, \emph{también habla a la mente y al corazón del hombre del tercer milenio}. Habla \emph{al hombre secularizado}, suscitando nuevamente en él la nostalgia de su condición de viandante que busca la verdad y \emph{anhela lo absoluto}. La etimología misma del verbo desear -en latín, \emph{desiderare}-- evoca la experiencia de los navegantes, los cuales se orientan en la noche observando los astros, que en latín se llaman \emph{sidera}.
					
					3. ¿Quién no siente la necesidad de una \textquote{estrella} que lo guíe a lo largo de su camino en la tierra? Sienten esta necesidad tanto las personas como las naciones. A fin de satisfacer este anhelo de salvación universal, el Señor se eligió un pueblo que fuera estrella orientadora para \textquote{todos los linajes de la tierra} (\emph{Gn} 12, 3). Con la encarnación de su Hijo, Dios extendió luego su elección a todos los demás pueblos, sin distinción de raza y cultura. Así nació la Iglesia, formada por hombres y mujeres que, \textquote{reunidos en Cristo, son guiados por el Espíritu Santo en su peregrinar hacia el reino del Padre y han recibido el mensaje de la salvación para proponérselo a todos} (\emph{Gaudium et spes}, 1).
					
					Por tanto, para toda la comunidad eclesial resuena el oráculo del \textbf{profeta Isaías,} que hemos escuchado en la \textbf{primera lectura}: \textquote{¡Levántate, brilla (\ldots{}), que llega tu luz; la gloria del Señor amanece sobre ti! (\ldots{}) Y caminarán los pueblos a tu luz; los reyes al resplandor de tu aurora} (\emph{Is} 60, 1. 3)\ldots{}
					
					5. {[}\ldots{}{]} Os repito las palabras del Redentor: \textquote{\emph{Duc in altum}}. \textquote{No tengáis miedo a las tinieblas del mundo, porque quien os envía es la \textquote{luz del mundo} (\emph{Jn} 8, 12), \textquote{el lucero radiante del alba}} (\emph{Ap} 22, 16).
					
					¡Y tú, Jesús, que un día dijiste a tus discípulos: \textquote{Vosotros sois la luz del mundo} (\emph{Mt} 5, 14), haz que \emph{el testimonio evangélico} de estos hermanos nuestros \emph{resplandezca ante los hombres de nuestro tiempo}. Haz eficaz su misión para que cuantos confíes a su cuidado pastoral glorifiquen siempre al Padre que está en los cielos (cf. \emph{Mt} 5, 16).
					
					Madre del Verbo encarnado, Virgen fiel, conserva a estos nuevos obispos bajo tu constante protección, para que sean misioneros valientes del Evangelio; \emph{fiel reflejo del amor de Cristo}, luz de los pueblos y esperanza del mundo.
				\end{body}
			
\newsection
			
	
		\subsection{Benedicto XVI, papa}
		
			\subsubsection{Homilía (2008): Una esperanza mayor}
			
				\src{Basílica de San Pedro. Domingo 6 de enero del 2008.}
				
				\begin{body}
					Celebramos hoy a Cristo, luz del mundo, y su manifestación a las naciones. En el día de Navidad el mensaje de la liturgia era: \textquote{\emph{Hodie descendit lux magna super terram}} --- \textquote{Hoy desciende una gran luz a la tierra} (\emph{Misal romano}). En Belén, esta \textquote{gran luz} se presentó a un pequeño grupo de personas, a un minúsculo \textquote{resto de Israel}: a la Virgen María, a su esposo José, y a algunos pastores. Una luz humilde, según el estilo del verdadero Dios. Una llamita encendida en la noche: un frágil niño recién nacido, que da vagidos en el silencio del mundo\ldots{} Pero en torno a ese nacimiento oculto y desconocido resonaba el himno de alabanza de los coros celestiales, que cantaban gloria y paz (cf. \emph{Lc} 2, 13-14).
					
					Así, aquella luz, aun siendo pequeña cuando apareció en la tierra, se proyectaba con fuerza en los cielos. El nacimiento del Rey de los judíos había sido anunciado por una estrella que se podía ver desde muy lejos. Este fue el testimonio de \textquote{algunos Magos} que llegaron desde Oriente a Jerusalén poco después del nacimiento de Jesús, en tiempos del rey Herodes (cf. \emph{Mt} 2, 1-2).
					
					Una vez más, se comunican y se responden el cielo y la tierra, el cosmos y la historia. Las antiguas profecías se cumplen con el lenguaje de los astros. \textquote{De Jacob avanza una estrella, un cetro surge de Israel} (\emph{Nm} 24, 17), había anunciado el vidente pagano Balaam, llamado a maldecir al pueblo de Israel y que, al contrario, lo bendijo porque, como Dios le reveló, \textquote{ese pueblo es bendito} (\emph{Nm} 22, 12).
					
					Cromacio de Aquileya, en su \emph{Comentario al evangelio de san Mateo}, relacionando a Balaam con los Magos, escribe: \textquote{Aquel profetizó que Cristo vendría; estos lo vieron con los ojos de la fe}. Y añade una observación importante: \textquote{Todos vieron la estrella, pero no todos comprendieron su sentido. Del mismo modo, nuestro Señor y Salvador nació para todos, pero no todos lo acogieron} (\emph{ib}., 4, 1-2). Este es, en la perspectiva histórica, el significado del símbolo de la luz aplicado al nacimiento de Cristo: expresa la bendición especial de Dios en favor de la descendencia de Abraham, destinada a extenderse a todos los pueblos de la tierra.
					
					De este modo, el acontecimiento evangélico que recordamos en la Epifanía, la visita de los Magos al Niño Jesús en Belén, nos remite a los orígenes de la historia del pueblo de Dios, es decir, a la llamada de Abraham, que encontramos en el capítulo 12 del libro del Génesis. Los primeros once capítulos son como grandes cuadros que responden a algunas preguntas fundamentales de la humanidad: ¿Cuál es el origen del universo y del género humano? ¿De dónde viene el mal? ¿Por qué hay diversas lenguas y civilizaciones?
					
					Entre los relatos iniciales de la Biblia aparece una primera \textquote{alianza}, establecida por Dios con Noé, después del diluvio. Se trata de una alianza universal, que atañe a toda la humanidad: el nuevo pacto con la familia de Noé es, a la vez, un pacto con \textquote{toda carne} (cf. \emph{Gn} 9, 15). Luego, antes de la llamada de Abraham, se encuentra otro gran cuadro, muy importante para comprender el sentido de la Epifanía: el de la torre de Babel. El texto sagrado afirma que en los orígenes \textquote{todo el mundo tenía un mismo lenguaje e idénticas palabras} (\emph{Gn} 11, 1). Después los hombres dijeron: \textquote{Ea, vamos a edificarnos una ciudad y una torre con la cúspide en los cielos, y hagámonos famosos, por si nos desperdigamos por toda la haz de la tierra} (\emph{Gn} 11, 4). La consecuencia de este pecado de orgullo, análogo al de Adán y Eva, fue la confusión de las lenguas y la dispersión de la humanidad por toda la tierra (cf. \emph{Gn} 11, 7-8). Esto es lo que significa \textquote{Babel}; fue una especie de maldición, semejante a la expulsión del paraíso terrenal.
					
					En este punto se inicia la historia de la bendición, con la llamada de Abraham: comienza el gran plan de Dios para hacer de la humanidad una familia, mediante la alianza con un pueblo nuevo, elegido por él para que sea una bendición en medio de todas las naciones (cf. \emph{Gn} 12, 1-3). Este plan divino se sigue realizando todavía y tuvo su momento culminante en el misterio de Cristo. Desde entonces se iniciaron \textquote{los últimos tiempos}, en el sentido de que el plan fue plenamente revelado y realizado en Cristo, pero debe ser acogido por la historia humana, que sigue siendo siempre historia de fidelidad por parte de Dios y, lamentablemente, también de infidelidad por parte de nosotros los hombres.
					
					La Iglesia misma, depositaria de la bendición, es santa y a la vez está compuesta de pecadores; está marcada por la tensión entre el \textquote{ya} y el \textquote{todavía no}. En la plenitud de los tiempos Jesucristo vino a establecer la alianza: él mismo, verdadero Dios y verdadero hombre, es el Sacramento de la fidelidad de Dios a su plan de salvación para la humanidad entera, para todos nosotros.
					
					La llegada de los Magos de Oriente a Belén, para adorar al Mesías recién nacido, es la señal de la manifestación del Rey universal a los pueblos y a todos los hombres que buscan la verdad. Es el inicio de un movimiento opuesto al de Babel: de la confusión a la comprensión, de la dispersión a la reconciliación. Por consiguiente, descubrimos un vínculo entre la Epifanía y Pentecostés: si el nacimiento de Cristo, la Cabeza, es también el nacimiento de la Iglesia, su cuerpo, en los Magos vemos a los pueblos que se agregan al resto de Israel, anunciando la gran señal de la \textquote{Iglesia políglota} realizada por el Espíritu Santo cincuenta días después de la Pascua.
					
					El amor fiel y tenaz de Dios, que mantiene siempre su alianza de generación en generación. Este es el \textquote{misterio} del que habla \textbf{san Pablo} en sus cartas, también en el pasaje de la \textbf{carta a los Efesios} que se acaba de proclamar. El Apóstol afirma que este misterio le \textquote{fue comunicado por una revelación} (\emph{Ef} 3, 3) y él se encargó de darlo a conocer.
					
					Este \textquote{misterio} de la fidelidad de Dios constituye la esperanza de la historia. Ciertamente, se le oponen fuerzas de división y atropello, que desgarran a la humanidad a causa del pecado y del conflicto de egoísmos. En la historia, la Iglesia está al servicio de este \textquote{misterio} de bendición para la humanidad entera. En este misterio de la fidelidad de Dios, la Iglesia sólo cumple plenamente su misión cuando refleja en sí misma la luz de Cristo Señor, y así sirve de ayuda a los pueblos del mundo por el camino de la paz y del auténtico progreso.
					
					En efecto, sigue siendo siempre válida la palabra de Dios revelada por medio del \textbf{profeta Isaías}: \textquote{La oscuridad cubre la tierra, y espesa nube a los pueblos, mas sobre ti amanece el Señor y su gloria sobre ti aparece} (\emph{Is} 60, 2). Lo que el profeta anuncia a Jerusalén se cumple en la Iglesia de Cristo: \textquote{A tu luz caminarán las naciones, y los reyes al resplandor de tu aurora} (\emph{Is} 60, 3).
					
					Con Jesucristo la bendición de Abraham se extendió a todos los pueblos, a la Iglesia universal como nuevo Israel que acoge en su seno a la humanidad entera. Con todo, también hoy sigue siendo verdad lo que decía \textbf{el profeta}: \textquote{Espesa nube cubre a los pueblos} y nuestra historia. En efecto, no se puede decir que la globalización sea sinónimo de orden mundial; todo lo contrario. Los conflictos por la supremacía económica y el acaparamiento de los recursos energéticos e hídricos, y de las materias primas, dificultan el trabajo de quienes, en todos los niveles, se esfuerzan por construir un mundo justo y solidario.
					
					Es necesaria una esperanza mayor, que permita preferir el bien común de todos al lujo de pocos y a la miseria de muchos. \textquote{Esta gran esperanza sólo puede ser Dios, (\ldots{}) pero no cualquier dios, sino el Dios que tiene un rostro humano} (\emph{Spe salvi}, 31), el Dios que se manifestó en el Niño de Belén y en el Crucificado Resucitado.
					
					Si hay una gran esperanza, se puede perseverar en la sobriedad. Si falta la verdadera esperanza, se busca la felicidad en la embriaguez, en lo superfluo, en los excesos, y los hombres se arruinan a sí mismos y al mundo. La moderación no sólo es una regla ascética, sino también un camino de salvación para la humanidad.
					
					Ya resulta evidente que sólo adoptando un estilo de vida sobrio, acompañado del serio compromiso por una distribución equitativa de las riquezas, será posible instaurar un orden de desarrollo justo y sostenible. Por esto, hacen falta hombres que alimenten una gran esperanza y posean por ello una gran valentía. La valentía de los Magos, que emprendieron un largo viaje siguiendo una estrella, y que supieron arrodillarse ante un Niño y ofrecerle sus dones preciosos. Todos necesitamos esta valentía, anclada en una firme esperanza.
					
					Que nos la obtenga María, acompañándonos en nuestra peregrinación terrena con su protección materna. Amén.
				\end{body}
			
			
			\subsubsection{Homilía (2011): Estrella que nos lleva a Dios}
			
				\src{Basílica Vaticana. Jueves 6 de enero de 2011.}
				
				\begin{body}
					En la solemnidad de la Epifanía la Iglesia sigue contemplando y celebrando el misterio del nacimiento de Jesús salvador. En particular, la fiesta de hoy subraya el destino y el significado universales de este nacimiento. Al hacerse hombre en el seno de María, el Hijo de Dios vino no sólo para el pueblo de Israel, representado por los pastores de Belén, sino también para toda la humanidad, representada por los Magos. Y la Iglesia nos invita hoy a meditar y orar precisamente sobre los Magos y sobre su camino en busca del Mesías (cf. \emph{Mt} 2, 1-12). En el \textbf{Evangelio} hemos escuchado que los Magos, habiendo llegado a Jerusalén desde el Oriente, preguntan: \textquote{¿Dónde está el Rey de los judíos que ha nacido? Hemos visto su estrella en el Oriente y hemos venido a adorarlo} (v. 2). ¿Qué clase de personas eran y qué tipo de estrella era esa? Probablemente eran sabios que escrutaban el cielo, pero no para tratar de \textquote{leer} en los astros el futuro, quizá para obtener así algún beneficio; más bien, eran hombres \textquote{en busca} de algo más, en busca de la verdadera luz, una luz capaz de indicar el camino que es preciso recorrer en la vida. Eran personas que tenían la certeza de que en la creación existe lo que podríamos definir la \textquote{firma} de Dios, una firma que el hombre puede y debe intentar descubrir y descifrar. Tal vez el modo para conocer mejor a estos Magos y entender su deseo de dejarse guiar por los signos de Dios es detenernos a considerar lo que encontraron, en su camino, en la gran ciudad de Jerusalén.
					
					Ante todo encontraron al \textbf{rey Herodes}. Ciertamente, Herodes estaba interesado en el niño del que hablaban los Magos, pero no con el fin de adorarlo, como quiere dar a entender mintiendo, sino para eliminarlo. Herodes es un hombre de poder, que en el otro sólo ve un rival contra el cual luchar. En el fondo, si reflexionamos bien, también Dios le parece un rival, más aún, un rival especialmente peligroso, que querría privar a los hombres de su espacio vital, de su autonomía, de su poder; un rival que señala el camino que hay que recorrer en la vida y así impide hacer todo lo que se quiere. Herodes escucha de sus expertos en las Sagradas Escrituras las palabras del profeta Miqueas (5, 1), pero sólo piensa en el trono. Entonces Dios mismo debe ser ofuscado y las personas deben limitarse a ser simples peones para mover en el gran tablero de ajedrez del poder. Herodes es un personaje que no nos cae simpático y que instintivamente juzgamos de modo negativo por su brutalidad. Pero deberíamos preguntarnos: ¿Hay algo de Herodes también en nosotros? ¿También nosotros, a veces, vemos a Dios como una especie de rival? ¿También nosotros somos ciegos ante sus signos, sordos a sus palabras, porque pensamos que pone límites a nuestra vida y no nos permite disponer de nuestra existencia como nos plazca? Queridos hermanos y hermanas, cuando vemos a Dios de este modo acabamos por sentirnos insatisfechos y descontentos, porque no nos dejamos guiar por Aquel que está en el fundamento de todas las cosas. Debemos alejar de nuestra mente y de nuestro corazón la idea de la rivalidad, la idea de que dar espacio a Dios es un límite para nosotros mismos; debemos abrirnos a la certeza de que Dios es el amor omnipotente que no quita nada, no amenaza; más aún, es el único capaz de ofrecernos la posibilidad de vivir en plenitud, de experimentar la verdadera alegría.
					
					Los Magos, luego, se encuentran con \textbf{los estudiosos, los teólogos, los expertos} que lo saben todo sobre las Sagradas Escrituras, que conocen las posibles interpretaciones, que son capaces de citar de memoria cualquier pasaje y que, por tanto, son una valiosa ayuda para quienes quieren recorrer el camino de Dios. Pero, afirma san Agustín, les gusta ser guías para los demás, indican el camino, pero no caminan, se quedan inmóviles. Para ellos las Escrituras son una especie de atlas que leen con curiosidad, un conjunto de palabras y conceptos que examinar y sobre los cuales discutir doctamente. Pero podemos preguntarnos de nuevo: ¿no existe también en nosotros la tentación de considerar las Sagradas Escrituras, este tesoro riquísimo y vital para la fe la Iglesia, más como un objeto de estudio y de debate de especialistas que como el Libro que nos señala el camino para llegar a la vida? Creo que, como indiqué en la exhortación apostólica \emph{Verbum Domini}, debería surgir siempre de nuevo en nosotros la disposición profunda a ver la palabra de la Biblia, leída en la Tradición viva de la Iglesia (n. 18), como la verdad que nos dice qué es el hombre y cómo puede realizarse plenamente, la verdad que es el camino a recorrer diariamente, junto a los demás, si queremos construir nuestra existencia sobre la roca y no sobre la arena.
					
					Pasemos ahora a \textbf{la estrella}. ¿Qué clase de estrella era la que los Magos vieron y siguieron? A lo largo de los siglos esta pregunta ha sido objeto de debate entre los astrónomos. Kepler, por ejemplo, creía que se trataba de una \textquote{nova} o una \textquote{supernova}, es decir, una de las estrellas que normalmente emiten una luz débil, pero que pueden tener improvisamente una violenta explosión interna que produce una luz excepcional. Ciertamente, son cosas interesantes, pero que no nos llevan a lo que es esencial para entender esa estrella. Debemos volver al hecho de que esos hombres buscaban las huellas de Dios; trataban de leer su \textquote{firma} en la creación; sabían que \textquote{el cielo proclama la gloria de Dios} (\emph{Sal} 19, 2); es decir, tenían la certeza de que es posible vislumbrar a Dios en la creación. Pero, al ser hombres sabios, sabían también que no es con un telescopio cualquiera, sino con los ojos profundos de la razón en busca del sentido último de la realidad y con el deseo de Dios, suscitado por la fe, como es posible encontrarlo, más aún, como resulta posible que Dios se acerque a nosotros. El universo no es el resultado de la casualidad, como algunos quieren hacernos creer. Al contemplarlo, se nos invita a leer en él algo profundo: la sabiduría del Creador, la inagotable fantasía de Dios, su infinito amor a nosotros. No deberíamos permitir que limiten nuestra mente teorías que siempre llegan sólo hasta cierto punto y que ---si las miramos bien--- de ningún modo están en conflicto con la fe, pero no logran explicar el sentido último de la realidad. En la belleza del mundo, en su misterio, en su grandeza y en su racionalidad no podemos menos de leer la racionalidad eterna, y no podemos menos de dejarnos guiar por ella hasta el único Dios, creador del cielo y de la tierra. Si tenemos esta mirada, veremos que el que creó el mundo y el que nació en una cueva en Belén y sigue habitando entre nosotros en la Eucaristía son el mismo Dios vivo, que nos interpela, nos ama y quiere llevarnos a la vida eterna.
					
					Herodes, los expertos en las Escrituras, la estrella. Sigamos el camino de los Magos que llegan a Jerusalén. Sobre la gran ciudad la estrella desaparece, ya no se ve. ¿Qué significa eso? También en este caso debemos leer el signo en profundidad. Para aquellos hombres era lógico buscar al nuevo rey en el palacio real, donde se encontraban los sabios consejeros de la corte. Pero, probablemente con asombro, tuvieron que constatar que aquel recién nacido no se encontraba en los lugares del poder y de la cultura, aunque en esos lugares se daban valiosas informaciones sobre él. En cambio, se dieron cuenta de que a veces el poder, incluso el del conocimiento, obstaculiza el camino hacia el encuentro con aquel Niño. Entonces la estrella los guió a Belén, una pequeña ciudad; los guió hasta los pobres, hasta los humildes, para encontrar al Rey del mundo. Los criterios de Dios son distintos de los de los hombres. Dios no se manifiesta en el poder de este mundo, sino en la humildad de su amor, un amor que pide a nuestra libertad acogerlo para transformarnos y ser capaces de llegar a Aquel que es el Amor. Pero incluso para nosotros las cosas no son tan diferentes de como lo eran para los Magos. Si se nos pidiera nuestro parecer sobre cómo Dios habría debido salvar al mundo, tal vez responderíamos que habría debido manifestar todo su poder para dar al mundo un sistema económico más justo, en el que cada uno pudiera tener todo lo que quisiera. En realidad, esto sería una especie de violencia contra el hombre, porque lo privaría de elementos fundamentales que lo caracterizan. De hecho, no se verían involucrados ni nuestra libertad ni nuestro amor. El poder de Dios se manifiesta de un modo muy distinto: en Belén, donde encontramos la aparente impotencia de su amor. Y es allí a donde debemos ir y es allí donde encontramos la estrella de Dios.
					
					Así resulta muy claro también un último elemento importante del episodio de los Magos: el lenguaje de la creación nos permite recorrer un buen tramo del camino hacia Dios, pero no nos da la luz definitiva. Al final, para los Magos fue indispensable escuchar la voz de las Sagradas Escrituras: sólo ellas podían indicarles el camino. La Palabra de Dios es la verdadera estrella que, en la incertidumbre de los discursos humanos, nos ofrece el inmenso esplendor de la verdad divina. Queridos hermanos y hermanas, dejémonos guiar por la estrella, que es la Palabra de Dios; sigámosla en nuestra vida, caminando con la Iglesia, donde la Palabra ha plantado su tienda. Nuestro camino estará siempre iluminado por una luz que ningún otro signo puede darnos. Y también nosotros podremos convertirnos en estrellas para los demás, reflejo de la luz que Cristo ha hecho brillar sobre nosotros. Amén.
				\end{body}

\newsection

		\subsection{Francisco, papa}
		
			\subsubsection{Homilía (2014): No se detuvieron en la oscuridad}
			
				\src{Basílica de San Pedro. Lunes 6 de enero del 2014.}
				
				\begin{body}
					\textquote{Lumen requirunt lumine}. Esta sugerente expresión de un himno litúrgico de la Epifanía se refiere a la experiencia de los Magos: siguiendo una luz, buscan la Luz. La estrella que aparece en el cielo enciende en su mente y en su corazón una luz que los lleva a buscar la gran Luz de Cristo. Los Magos siguen fielmente aquella luz que los ilumina interiormente y encuentran al Señor.
					
					En este recorrido que hacen los Magos de Oriente está simbolizado el destino de todo hombre: nuestra vida es un camino, iluminados por luces que nos permiten entrever el sendero, hasta encontrar la plenitud de la verdad y del amor, que nosotros cristianos reconocemos en Jesús, Luz del mundo. Y todo hombre, como los Magos, tiene a disposición dos grandes \textquote{libros} de los que sacar los signos para orientarse en su peregrinación: el libro de la creación y el libro de las Sagradas Escrituras. Lo importante es estar atentos, vigilantes, escuchar a Dios que nos habla, siempre nos habla. Como dice el Salmo, refiriéndose a la Ley del Señor: \textquote{Lámpara es tu palabra para mis pasos, / luz en mi sendero} (Sal 119,105). Sobre todo, escuchar el Evangelio, leerlo, meditarlo y convertirlo en alimento espiritual nos permite encontrar a Jesús vivo, hacer experiencia de Él y de su amor.
					
					En la \textbf{primera Lectura} resuena, por boca del \textbf{profeta Isaías}, el llamado de Dios a Jerusalén: \textquote{¡Levántate, brilla!} (60,1). Jerusalén está llamada a ser la ciudad de la luz, que refleja en el mundo la luz de Dios y ayuda a los hombres a seguir sus caminos. Ésta es la vocación y la misión del Pueblo de Dios en el mundo. Pero Jerusalén puede desatender esta llamada del Señor. Nos dice el \textbf{Evangelio} que los Magos, cuando llegaron a Jerusalén, de momento perdieron de vista la estrella. No la veían. En especial, su luz falta en el palacio del rey Herodes: aquella mansión es tenebrosa, en ella reinan la oscuridad, la desconfianza, el miedo, la envidia. De hecho, Herodes se muestra receloso e inquieto por el nacimiento de un frágil Niño, al que ve como un rival. En realidad, Jesús no ha venido a derrocarlo a él, ridículo fantoche, sino al Príncipe de este mundo. Sin embargo, el rey y sus consejeros sienten que el entramado de su poder se resquebraja, temen que cambien las reglas de juego, que las apariencias queden desenmascaradas. Todo un mundo edificado sobre el poder, el prestigio, el tener, la corrupción, entra en crisis por un Niño. Y Herodes llega incluso a matar a los niños: \textquote{Tú matas el cuerpo de los niños, porque el temor te ha matado a ti el corazón} -- escribe san Quodvultdeus (Sermón 2 sobre el Símbolo: PL 40, 655). Es así: tenía temor, y por este temor pierde el juicio.
					
					Los Magos consiguieron superar aquel momento crítico de oscuridad en el palacio de Herodes, porque creyeron en las Escrituras, en la palabra de los profetas que señalaba Belén como el lugar donde había de nacer el Mesías. Así escaparon al letargo de la noche del mundo, reemprendieron su camino y de pronto vieron nuevamente la estrella, y el \textbf{Evangelio} dice que se llenaron de \textquote{inmensa alegría} (Mt 2,10). Esa estrella que no se veía en la oscuridad de la mundanidad de aquel palacio.
					
					Un aspecto de la luz que nos guía en el camino de la fe es también la santa \textquote{astucia}. Es también una virtud, la santa \textquote{astucia}. Se trata de esa sagacidad espiritual que nos permite reconocer los peligros y evitarlos. Los Magos supieron usar esta luz de \textquote{astucia} cuando, de regreso a su tierra, decidieron no pasar por el palacio tenebroso de Herodes, sino marchar por otro camino. Estos sabios venidos de Oriente nos enseñan a no caer en las asechanzas de las tinieblas y a defendernos de la oscuridad que pretende cubrir nuestra vida. Ellos, con esta santa \textquote{astucia}, han protegido la fe. Y también nosotros debemos proteger la fe. Protegerla de esa oscuridad. Esa oscuridad que a menudo se disfraza incluso de luz. Porque el demonio, dice san Pablo, muchas veces se viste de ángel de luz. Y entonces es necesaria la santa \textquote{astucia}, para proteger la fe, protegerla de los cantos de las sirenas, que te dicen: \textquote{Mira, hoy debemos hacer esto, aquello\ldots{}} Pero la fe es una gracia, es un don. Y a nosotros nos corresponde protegerla con la santa \textquote{astucia}, con la oración, con el amor, con la caridad. Es necesario acoger en nuestro corazón la luz de Dios y, al mismo tiempo, practicar aquella astucia espiritual que sabe armonizar la sencillez con la sagacidad, como Jesús pide a sus discípulos: \textquote{Sean sagaces como serpientes y simples como palomas} (Mt 10,16).
					
					En esta fiesta de la Epifanía, que nos recuerda la manifestación de Jesús a la humanidad en el rostro de un Niño, sintamos cerca a los Magos, como sabios compañeros de camino. Su ejemplo nos anima a levantar los ojos a la estrella y a seguir los grandes deseos de nuestro corazón. Nos enseñan a no contentarnos con una vida mediocre, de \textquote{poco calado}, sino a dejarnos fascinar siempre por la bondad, la verdad, la belleza\ldots{} por Dios, que es todo eso en modo siempre mayor. Y nos enseñan a no dejarnos engañar por las apariencias, por aquello que para el mundo es grande, sabio, poderoso. No nos podemos quedar ahí. Es necesario proteger la fe. Es muy importante en este tiempo: proteger la fe. Tenemos que ir más allá, más allá de la oscuridad, más allá de la atracción de las sirenas, más allá de la mundanidad, más allá de tantas modernidades que existen hoy, ir hacia Belén, allí donde en la sencillez de una casa de la periferia, entre una mamá y un papá llenos de amor y de fe, resplandece el Sol que nace de lo alto, el Rey del universo. A ejemplo de los Magos, con nuestras pequeñas luces busquemos la Luz y protejamos la fe. Así sea.
				\end{body}
			
			
			\subsubsection{Ángelus (2014): Dios nos amó primero}
			
				\src{Plaza de San Pedro. Lunes 6 de enero de 2014.}
				
				\begin{body}
					Hoy celebramos la Epifanía, es decir la \textquote{manifestación} del Señor. Esta solemnidad está vinculada al relato bíblico de la llegada de los magos de Oriente a Belén para rendir homenaje al Rey de los judíos: un episodio que el Papa Benedicto comentó magníficamente en su libro sobre la infancia de Jesús. Esa fue precisamente la primera \textquote{manifestación} de Cristo a las gentes. Por ello la Epifanía destaca la apertura universal de la salvación traída por Jesús. La Liturgia de este día aclama: \textquote{Te adorarán, Señor, todos los pueblos de la tierra}, porque Jesús vino por todos nosotros, por todos los pueblos, por todos.
					
					En efecto, esta fiesta nos hace ver un doble movimiento: por una parte el movimiento de Dios hacia el mundo, hacia la humanidad ---toda la historia de la salvación, que culmina en Jesús---; y por otra parte el movimiento de los hombres hacia Dios ---pensemos en las religiones, en la búsqueda de la verdad, en el camino de los pueblos hacia la paz, la paz interior, la justicia, la libertad---. Y a este doble movimiento lo mueve una recíproca atracción. Por parte de Dios, ¿qué es lo que lo atrae? Es el amor por nosotros: somos sus hijos, nos ama, y quiere liberarnos del mal, de las enfermedades, de la muerte, y llevarnos a su casa, a su Reino. \textquote{Dios, por pura gracia, nos atrae para unirnos a sí} (Exhort. ap. Evangelii gaudium, 112). Y también por parte nuestra hay un amor, un deseo: el bien siempre nos atrae, la verdad nos atrae, la vida, la felicidad, la belleza nos atrae\ldots{} Jesús es es el punto de encuentro de esta atracción mutua, y de este doble movimiento. Es Dios y hombre: Jesús. Dios y hombre. ¿Pero quien toma la iniciativa? ¡Siempre Dios! El amor de Dios viene siempre antes del nuestro. Él siempre toma la iniciativa. Él nos espera, Él nos invita, la iniciativa es siempre suya. Jesús es Dios que se hizo hombre, se encarnó, nació por nosotros. La nueva estrella que apareció a los magos era el signo del nacimiento de Cristo. Si no hubiesen visto la estrella, esos hombres no se hubiesen puesto en camino. La luz nos precede, la verdad nos precede, la belleza nos precede. Dios nos precede. El profeta Isaías decía que Dios es como la flor del almendro. ¿Por qué? Porque en esa tierra el almendro es primero en florecer. Y Dios siempre precede, siempre nos busca Él primero, Él da el primer paso. Dios nos precede siempre. Su gracia nos precede; y esta gracia apareció en Jesús. Él es la epifanía. Él, Jesucristo, es la manifestación del amor de Dios. Está con nosotros.
					
					La Iglesia está toda dentro de este movimiento de Dios hacia el mundo: su alegría es el Evangelio, es reflejar la luz de Cristo. La Iglesia es el pueblo de aquellos que experimentaron esta atracción y la llevaron dentro, en el corazón y en la vida. \textquote{Me gustaría ---sinceramente---, me gustaría decir a aquellos que se sienten alejados de Dios y de la Iglesia ---decirlo respetuosamente---, decir a aquellos son temerosos e indiferentes: el Señor te llama también a ti, te llama a formar parte de su pueblo y lo hace con gran respeto y amor} (ibid., 113). El Señor te llama. El Señor te busca. El Señor te espera. El Señor no hace proselitismo, da amor, y este amor te busca, te espera, a ti que en este momento no crees o estás alejado. Esto es el amor de Dios.
					
					Pidamos a Dios, para toda la Iglesia, pidamos la alegría de evangelizar, porque ha sido \textquote{enviada por Cristo para manifestar y comunicar a todos los hombres y a todos los pueblos el amor de Dios} (Ad gentes, 10). Que la Virgen María nos ayude a ser todos discípulos-misioneros, pequeñas estrellas que reflejen su luz. Y oremos para que los corazones se abran para acoger el anuncio, y todos los hombres lleguen a ser \textquote{partícipes de la misma promesa en Jesucristo, por el Evangelio} (Ef 3, 6).
				\end{body}
			
			\subsubsection{Homilía (2017)}
			
				\src{Basílica Vaticana. \\Viernes 6 de enero de 2017.}
				
				\begin{body}
					\textquote{¿Dónde está el Rey de los judíos que acaba de nacer? Porque vimos su estrella y hemos venido a adorarlo} (\emph{Mt} 2, 2).
					
					Con estas palabras, los magos, venidos de tierras lejanas, nos dan a conocer el motivo de su larga travesía: adorar al rey recién nacido. Ver y adorar, dos acciones que se destacan en el relato evangélico: vimos una estrella y queremos adorar.
					
					Estos hombres \emph{vieron una estrella} que los puso en movimiento. El descubrimiento de algo inusual que sucedió en el cielo logró desencadenar un sinfín de acontecimientos. No era una estrella que brilló de manera exclusiva para ellos, ni tampoco tenían un ADN especial para descubrirla. Como bien supo decir un padre de la Iglesia, \textquote{los magos no se pusieron en camino porque hubieran visto la estrella, sino que vieron la estrella porque se habían puesto en camino} (cf. San Juan Crisóstomo). Tenían el corazón abierto al horizonte y lograron ver lo que el cielo les mostraba porque había en ellos una inquietud que los empujaba: estaban abiertos a una novedad.
					
					Los magos, de este modo, expresan el retrato del hombre creyente, del hombre que tiene nostalgia de Dios; del que añora su casa, la patria celeste. Reflejan la imagen de todos los hombres que en su vida no han dejado que se les anestesie el corazón.
					
					La santa nostalgia de Dios brota en el corazón creyente pues sabe que el Evangelio no es un acontecimiento del pasado sino del presente. La santa nostalgia de Dios nos permite tener los ojos abiertos frente a todos los intentos reductivos y empobrecedores de la vida. La santa nostalgia de Dios es la memoria creyente que se rebela frente a tantos profetas de desventura. Esa nostalgia es la que mantiene viva la esperanza de la comunidad creyente la cual, semana a semana, implora diciendo: \textquote{Ven, Señor Jesús}.
					
					Precisamente esta nostalgia fue la que empujó al anciano Simeón a ir todos los días al templo, con la certeza de saber que su vida no terminaría sin poder acunar al Salvador. Fue esta nostalgia la que empujó al hijo pródigo a salir de una actitud de derrota y buscar los brazos de su padre. Fue esta nostalgia la que el pastor sintió en su corazón cuando dejó a las noventa y nueve ovejas en busca de la que estaba perdida, y fue también la que experimentó María Magdalena la mañana del domingo para salir corriendo al sepulcro y encontrar a su Maestro resucitado. La nostalgia de Dios nos saca de nuestros encierros deterministas, esos que nos llevan a pensar que nada puede cambiar. La nostalgia de Dios es la actitud que rompe aburridos conformismos e impulsa a comprometerse por ese cambio que anhelamos y necesitamos. La nostalgia de Dios tiene su raíz en el pasado pero no se queda allí: va en busca del futuro. Al igual que los magos, el creyente \textquote{nostalgioso} busca a Dios, empujado por su fe, en los lugares más recónditos de la historia, porque sabe en su corazón que allí lo espera el Señor. Va a la periferia, a la frontera, a los sitios no evangelizados para poder encontrarse con su Señor; y lejos de hacerlo con una postura de superioridad lo hace como un mendicante que no puede ignorar los ojos de aquel para el cual la Buena Nueva es todavía un terreno a explorar.
					
					Como actitud contrapuesta, en el palacio de Herodes ―que distaba muy pocos kilómetros de Belén―, no se habían percatado de lo que estaba sucediendo. Mientras los magos caminaban, Jerusalén dormía. Dormía de la mano de un Herodes quien lejos de estar en búsqueda también dormía. Dormía bajo la anestesia de una conciencia cauterizada. Y quedó desconcertado. Tuvo miedo. Es el desconcierto que, frente a la novedad que revoluciona la historia, se encierra en sí mismo, en sus logros, en sus saberes, en sus éxitos. El desconcierto de quien está sentado sobre la riqueza sin lograr ver más allá. Un desconcierto que brota del corazón de quién quiere controlar todo y a todos. Es el desconcierto del que está inmerso en la cultura del ganar cueste lo que cueste; en esa cultura que sólo tiene espacio para los \textquote{vencedores} y al precio que sea. Un desconcierto que nace del miedo y del temor ante lo que nos cuestiona y pone en riesgo nuestras seguridades y verdades, nuestras formas de aferrarnos al mundo y a la vida. Y Herodes tuvo miedo, y ese miedo lo condujo a buscar seguridad en el crimen: \textquote{\emph{Necas parvulos corpore, quia te necat timor in corde}} (San Quodvultdeus, \emph{Sermo 2 sobre el símbolo}: \emph{PL}, 40, 655). Matas los niños en el cuerpo porque a ti el miedo te mata el corazón.
					
					\emph{Queremos adorar}. Los hombres de Oriente fueron a adorar, y fueron a hacerlo al lugar propio de un rey: el Palacio. Y esto es importante, allí llegaron ellos con su búsqueda, era el lugar indicado: pues es propio de un rey nacer en un palacio, y tener su corte y súbditos. Es signo de poder, de éxito, de vida lograda. Y es de esperar que el rey sea venerado, temido y adulado, sí; pero no necesariamente amado. Esos son los esquemas mundanos, los pequeños ídolos a los que le rendimos culto: el culto al poder, a la apariencia y a la superioridad. Ídolos que solo prometen tristeza, esclavitud, miedo.
					
					Y fue precisamente ahí donde comenzó el camino más largo que tuvieron que andar esos hombres venidos de lejos. Ahí comenzó la osadía más difícil y complicada. Descubrir que lo que ellos buscaban no estaba en el palacio sino que se encontraba en otro lugar, no sólo geográfico sino existencial. Allí no veían la estrella que los conducía a descubrir un Dios que quiere ser amado, y eso sólo es posible bajo el signo de la libertad y no de la tiranía; descubrir que la mirada de este Rey desconocido ―pero deseado― no humilla, no esclaviza, no encierra. Descubrir que la mirada de Dios levanta, perdona, sana. Descubrir que Dios ha querido nacer allí donde no lo esperamos, donde quizá no lo queremos. O donde tantas veces lo negamos. Descubrir que en la mirada de Dios hay espacio para los heridos, los cansados, los maltratados, abandonados: que su fuerza y su poder se llama misericordia. Qué lejos se encuentra, para algunos, Jerusalén de Belén.
					
					Herodes no puede adorar porque no quiso y no pudo cambiar su mirada. No quiso dejar de rendirse culto a sí mismo creyendo que todo comenzaba y terminaba con él. No pudo adorar porque buscaba que lo adorasen. Los sacerdotes tampoco pudieron adorar porque sabían mucho, conocían las profecías, pero no estaban dispuestos ni a caminar ni a cambiar.
					
					Los magos sintieron nostalgia, no querían más de lo mismo. Estaban acostumbrados, habituados y cansados de los Herodes de su tiempo. Pero allí, en Belén, había promesa de novedad, había promesa de gratuidad. Allí estaba sucediendo algo nuevo. Los magos pudieron adorar porque se animaron a caminar y postrándose ante el pequeño, postrándose ante el pobre, postrándose ante el indefenso, postrándose ante el extraño y desconocido Niño de Belén, allí descubrieron la Gloria de Dios.
				\end{body}
			
			\subsubsection{Ángelus (2017)} 
			
				\begin{body}
					\emph{Queridos hermanos y hermanas, ¡buenos días!}
					
					Hoy, celebramos la Epifanía del Señor, es decir, la manifestación de Jesús que brilla como luz para todas las gentes. Símbolo de esta luz que resplandece en el mundo y quiere iluminar la vida de cada uno es la estrella, que guió a los Magos a Belén. Ellos, dice el Evangelio, vieron \textquote{su estrella} (\emph{Mt} 2, 2) y decidieron seguirla: decidieron dejarse guiar por la estrella de Jesús.
					
					También en nuestra vida existen diversas estrellas, luces que brillan y orientan. Depende de nosotros elegir cuáles seguir. Por ejemplo, hay luces intermitentes, que van y vienen, como las pequeñas satisfacciones de la vida: que aunque buenas, no son suficientes, porque duran poco y no dejan la paz que buscamos. Después están las luces cegadoras del primer plano, del dinero y del éxito, que prometen todo y enseguida: son seductoras, pero con su fuerza ciegan y hacen pasar de los sueños de gloria a la oscuridad más densa. Los Magos, en cambio, invitan a seguir una luz estable, una luz amable, que no se oculta, porque no es de este mundo: viene del cielo y resplandece\ldots{} ¿Dónde? En el corazón.
					
					Esta luz verdadera es la luz del Señor, o mejor dicho, es el Señor mismo. Él es nuestra luz: una luz que no deslumbra, sino que acompaña y dona una alegría única. Esta luz es para todos y llama a cada uno: podemos escuchar así la actual invitación dirigida a nosotros por el profeta Isaías: \textquote{arriba, resplandece, que ha llegado tu luz} (60, 1). Así decía Isaías, profetizando esta alegría de hoy en Jerusalén: \textquote{arriba, resplandece, que ha llegado tu luz}. Al inicio de cada día podemos acoger esta invitación: arriba, resplandece, que ha llegado tu luz, sigue hoy, entre tantas estrellas fugaces en el mundo, la estrella luminosa de Jesús! Siguiéndola, tendremos alegría, como ocurrió a los Magos, que \textquote{al ver la estrella se llenaron de inmensa alegría} (\emph{Mt} 2, 10); porque donde esta Dios hay alegría. Quien ha encontrado a Jesús ha experimentado el milagro de la luz que rasga las tinieblas y conoce esta luz que ilumina y aclara. Querría, con mucho respeto, invitar a todos a no tener miedo de esta luz y a abrirse al Señor. Sobre todo querría decir a quien ha perdido la fuerza de buscar, está cansado, a quien, superado por las oscuridades de la vida, ha apagado el deseo: \textquote{¡Levántate, ánimo, la luz de Jesús sabe vencer las tinieblas más oscuras; levántate, ánimo!}.
					
					Y ¿cómo encontrar esta luz divina? Sigamos el ejemplo de los Magos, que el Evangelio describe siempre en movimiento. Quien quiere la luz, efectivamente, sale de sí y busca: no permanece cerrado, quieto a ver qué cosa sucede al su alrededor, sino pone en juego su propia vida; sale de sí. La vida cristiana es un camino continuo, hecho de esperanza, hecho de búsqueda; un camino que, como aquel de los Magos, prosigue incluso cuando la estrella desaparece momentáneamente de la vista. En este camino hay también insidias que hay que evitar: las charlas superficiales y mundanas, que frenan el paso; los caprichos paralizantes del egoísmo; los agujeros del pesimismo, que atrapa a la esperanza. Estos obstáculos bloquearon a los escribas, de los que habla el Evangelio de hoy. Ellos sabían dónde estaba la luz, pero no se movieron. Cuando Herodes les pregunto: \textquote{¿Dónde nacerá el Mesías?} --- \textquote{¡En Belén!}. Sabían dónde, pero no se movieron. Su conocimiento fue en vano: sabían muchas cosas, pero para nada, todo en vano. No basta saber que Dios ha nacido, si no se hace con Él Navidad en el corazón. Dios ha nacido, sí, pero ¿Ha nacido en tú corazón? ¿Ha nacido en mí corazón? ¿Ha nacido en nuestro corazón? Y así le encontraremos, como los Magos, con María, José, en el establo.
					
					Los Magos lo hicieron: encontraron al Niño, \textquote{postrándose, le adoraron} (v. 11). No le miraron solamente, dijeron solo una oración circunstancial y se fueron, no, sino que le adoraron: entraron en una comunión personal de amor con Jesús. Después le regalaron oro, incienso y mirra, es decir, sus bienes más preciados. Aprendamos de los Magos a no dedicar a Jesús sólo los ratos perdidos de tiempo y algún pensamiento de vez en cuando, de lo contrario no tendremos su luz. Como los Magos, pongámonos en camino, revistámonos de luz siguiendo la estrella de Jesús, y adoremos al Señor con todo nuestro ser.
				\end{body}
			
			\subsubsection{Homilía (2020)}
			
				\src{Basílica Vaticana.\\ Lunes, 6 de enero de 2020.}
				
				\begin{body}
					En el Evangelio (\emph{Mt} 2,1-12) hemos escuchado que los Magos comienzan manifestando sus intenciones: \textquote{Hemos visto salir su estrella y venimos a adorarlo} (v. 2). La adoración es la finalidad de su viaje, el objetivo de su camino. De hecho, cuando llegaron a Belén, \textquote{vieron al niño con María, su madre, y cayendo de rodillas lo adoraron} (v. 11). Si perdemos el sentido de la \emph{adoración}, perdemos el sentido de movimiento de la vida cristiana, que es un camino hacia el Señor, no hacia nosotros. Es el riesgo del que nos advierte el Evangelio, presentando, junto a los Reyes Magos, unos personajes que no logran adorar.
					
					En primer lugar, está el rey Herodes, que usa el verbo adorar, pero de manera engañosa. De hecho, le pide a los Reyes Magos que le informen sobre el lugar donde estaba el Niño \textquote{para ir ---dice--- yo también a adorarlo} (v. 8). En realidad, Herodes sólo se adoraba a sí mismo y, por lo tanto, quería deshacerse del Niño con mentiras. ¿Qué nos enseña esto? Que el hombre, cuando no adora a \emph{Dios}, está orientado a adorar su \emph{yo}. E incluso la vida cristiana, sin adorar al Señor, puede convertirse en una forma educada de alabarse a uno mismo y el talento que se tiene: cristianos que no saben adorar, que no saben rezar adorando. Es un riesgo grave: servirnos de Dios en lugar de servir a Dios. Cuántas veces hemos cambiado los intereses del Evangelio por los nuestros, cuántas veces hemos cubierto de religiosidad lo que era cómodo para nosotros, cuántas veces hemos confundido el poder según Dios, que es servir a los demás, con el poder según el mundo, que es servirse a sí mismo.
					
					Además de Herodes, hay otras personas en el Evangelio que no logran adorar: son los jefes de los sacerdotes y los escribas del pueblo. Ellos indican a Herodes con extrema precisión dónde nacería el Mesías: en Belén de Judea (cf. v. 5). Conocen las profecías y las citan exactamente. Saben a dónde ir ---grandes teólogos, grandes---, pero no van. También de esto podemos aprender una lección. En la vida cristiana no es suficiente saber: sin salir de uno mismo, sin encontrar, sin adorar, no se conoce a Dios. La teología y la eficiencia pastoral valen poco o nada si no se doblan las rodillas; si no se hace como los Magos, que no sólo fueron sabios organizadores de un viaje, sino que caminaron y adoraron. Cuando uno adora, se da cuenta de que la fe no se reduce a un conjunto de hermosas doctrinas, sino que es la relación con una Persona viva a quien amar. Conocemos el rostro de Jesús estando cara a cara con Él. Al adorar, descubrimos que la vida cristiana es una historia de amor con Dios, donde las buenas ideas no son suficientes, sino que se necesita ponerlo en primer lugar, como lo hace un enamorado con la persona que ama. Así debe ser la Iglesia, una adoradora enamorada de Jesús, su esposo.
					
					Al inicio del año redescubrimos la adoración como una exigencia de fe. Si sabemos arrodillarnos ante Jesús, venceremos la tentación de ir cada uno por su camino. De hecho, adorar es hacer un éxodo de la esclavitud más grande, la de uno mismo. Adorar es poner al Señor en el centro para no estar más centrados en nosotros mismos. Es poner cada cosa en su lugar, dejando el primer puesto a Dios. Adorar es poner los planes de Dios antes que mi tiempo, que mis derechos, que mis espacios. Es aceptar la enseñanza de la Escritura: \textquote{Al Señor, tu Dios, adorarás} (\emph{Mt} 4,10). Tu Dios: adorar es experimentar que, con Dios, nos pertenecemos recíprocamente. Es darle del \textquote{tú} en la intimidad, es presentarle la vida y permitirle entrar en nuestras vidas. Es hacer descender su consuelo al mundo. Adorar es descubrir que para rezar basta con decir: \textquote{¡Señor mío y Dios mío!} (\emph{Jn} 20,28), y dejarnos llenar de su ternura.
					
					Adorar es encontrarse con Jesús sin la lista de peticiones, pero con la única solicitud de estar con Él. Es descubrir que la alegría y la paz crecen con la alabanza y la acción de gracias. Cuando adoramos, permitimos que Jesús nos sane y nos cambie. Al adorar, le damos al Señor la oportunidad de transformarnos con su amor, de iluminar nuestra oscuridad, de darnos fuerza en la debilidad y valentía en las pruebas. Adorar es ir a lo esencial: es la forma de desintoxicarse de muchas cosas inútiles, de adicciones que adormecen el corazón y aturden la mente. De hecho, al adorar uno aprende a rechazar lo que no debe ser adorado: el dios del dinero, el dios del consumo, el dios del placer, el dios del éxito, nuestro yo erigido en dios. Adorar es hacerse pequeño en presencia del Altísimo, descubrir ante Él que la grandeza de la vida no consiste en tener, sino en amar. Adorar es redescubrirnos hermanos y hermanas frente al misterio del amor que supera toda distancia: es obtener el bien de la fuente, es encontrar en el Dios cercano la valentía para aproximarnos a los demás. Adorar es saber guardar silencio ante la Palabra divina, para aprender a decir palabras que no duelen, sino que consuelan.
					
					La adoración es un gesto de amor que cambia la vida. Es actuar como los Magos: es traer oro al Señor, para decirle que nada es más precioso que Él; es ofrecerle incienso, para decirle que sólo con Él puede elevarse nuestra vida; es presentarle mirra, con la que se ungían los cuerpos heridos y destrozados, para pedirle a Jesús que socorra a nuestro prójimo que está marginado y sufriendo, porque allí está Él. Por lo general, sabemos cómo orar ---le pedimos, le agradecemos al Señor---, pero la Iglesia debe ir aún más allá con la oración de adoración, debemos crecer en la adoración. Es una sabiduría que debemos aprender todos los días. Rezar adorando: la oración de adoración.
					
					Queridos hermanos y hermanas, hoy cada uno de nosotros puede preguntarse: \textquote{¿Soy un adorador cristiano?}. Muchos cristianos que oran no saben adorar. Hagámonos esta pregunta. ¿Encontramos momentos para la adoración en nuestros días y creamos espacios para la adoración en nuestras comunidades? Depende de nosotros, como Iglesia, poner en práctica las palabras que rezamos hoy en el Salmo: \textquote{Señor, que todos los pueblos te adoren}. Al adorar, nosotros también descubriremos, como los Magos, el significado de nuestro camino. Y, como los Magos, experimentaremos una \textquote{inmensa alegría} (\emph{Mt} 2,10).								
				\end{body}
			
			\subsubsection{Ángelus (2020)}
			
				\src{Plaza de San Pedro. \\Lunes, 6 de enero de 2020.}
				
				\begin{body}
					\emph{Queridos hermanos y hermanas, ¡buenos días!}
					
					Celebramos la solemnidad de la Epifanía, en memoria de los Magos que vinieron de Oriente a Belén, siguiendo la estrella, para visitar al Mesías recién nacido. Al final del relato evangélico se dice que los Magos \textquote{avisados en sueños que no volvieran donde Herodes, se retiraron a su país por otro camino} (v. 12). Por otro camino.
					
					Estos sabios, procedentes de regiones lejanas, después de haber viajado mucho, encuentran al que querían conocer, después de haberlo buscado durante mucho tiempo, seguramente también con mucho trabajo y vicisitudes. Y cuando finalmente llegan a su destino, se postran ante el Niño, lo adoran, le ofrecen sus preciosos regalos. Después de eso, se pusieron en marcha de nuevo sin demora para volver a su tierra. Pero ese encuentro con el Niño los ha cambiado.
					
					El encuentro con Jesús no detiene a los Reyes Magos, al contrario, les da un nuevo impulso para volver a su país, para contar lo que han visto y la alegría que han sentido. En esto hay una demostración del estilo de Dios, de su modo de manifestarse en la historia. La experiencia de Dios no nos bloquea, sino que nos libera; no nos aprisiona, sino que nos devuelve al camino, nos devuelve a los lugares habituales de nuestra existencia. Los lugares son y serán los mismos, pero nosotros, después del encuentro con Jesús, no somos los mismos que antes. El encuentro con Jesús nos cambia, nos transforma. El evangelista Mateo subraya que los Reyes Magos regresaron \textquote{por otro camino} (v. 12). La advertencia del ángel los lleva a cambiar sus caminos para no encontrarse con Herodes y sus tramas de poder.
					
					Cada experiencia de encuentro con Jesús nos lleva a tomar caminos diferentes, porque de Él proviene una fuerza buena que sana el corazón y nos aparta del mal.
					
					Existe una sabia dinámica entre continuidad y novedad: vuelven \textquote{a su país}, pero \textquote{por otro camino}. Esto indica que somos nosotros los que debemos cambiar, los que debemos transformar nuestra forma de vida, aunque sea en el mismo ambiente de siempre, los que debemos cambiar los criterios de juicio sobre la realidad que nos rodea. Esta es la diferencia entre el verdadero Dios y los ídolos traidores, como el dinero, el poder, el éxito\ldots{}; entre Dios y aquellos que prometen darte estos ídolos, como los magos, los adivinos, los hechiceros. La diferencia es que los ídolos nos atan a sí mismos, nos hacen dependientes de los ídolos, y nosotros tomamos posesión de ellos. El verdadero Dios no nos retiene ni se deja retener por nosotros: nos abre caminos de novedad y de libertad, porque es Padre que está siempre con nosotros para hacernos crecer.
					
					Si te encuentras con Jesús, si tienes un encuentro espiritual con Jesús, recuerda: debes volver a los mismos lugares de siempre, pero de otra manera, con otro estilo. Es así, es el Espíritu Santo, que Jesús nos da, que nos cambia el corazón.
					
					Pidamos a la Santa Virgen que podamos convertirnos en testigos de Cristo allá donde estemos, con una vida nueva, transformada por su amor.
				\end{body}
			
\newsection

\section{Temas}

%01 Ciclo | 02 Tiempo | 09 Semana

\cceth{La Epifanía del Señor}

\cceref{CEC 528, 724}


\begin{ccebody}
	\n{528} La \emph{Epifanía} es la manifestación de Jesús como Mesías de Israel, Hijo de Dios y Salvador del mundo. Con el bautismo de Jesús en el Jordán y las bodas de Caná (cf. \emph{Solemnidad de la Epifanía del Señor}, Antífona del \textquote{Magnificat} en II Vísperas, LH), la Epifanía celebra la adoración de Jesús por unos \textquote{magos} venidos de Oriente (\emph{Mt} 2, 1) En estos \textquote{magos}, representantes de religiones paganas de pueblos vecinos, el Evangelio ve las primicias de las naciones que acogen, por la Encarnación, la Buena Nueva de la salvación. La llegada de los magos a Jerusalén para \textquote{rendir homenaje al rey de los Judíos} (\emph{Mt} 2, 2) muestra que buscan en Israel, a la luz mesiánica de la estrella de David (cf. \emph{Nm} 24, 17; \emph{Ap} 22, 16) al que será el rey de las naciones (cf. \emph{Nm} 24, 17-19). Su venida significa que los gentiles no pueden descubrir a Jesús y adorarle como Hijo de Dios y Salvador del mundo sino volviéndose hacia los judíos (cf. \emph{Jn} 4, 22) y recibiendo de ellos su promesa mesiánica tal como está contenida en el Antiguo Testamento (cf. \emph{Mt} 2, 4-6). La Epifanía manifiesta que \textquote{la multitud de los gentiles entra en la familia de los patriarcas} (San León Magno, \emph{Sermones}, 23: PL 54, 224B) y adquiere la \emph{israelitica dignitas} (la dignidad israelítica) (Vigilia pascual, Oración después de la tercera lectura: \emph{Misal Romano}).
	
	\n{724} En María, el Espíritu Santo \emph{manifiesta} al Hijo del Padre hecho Hijo de la Virgen. Ella es la zarza ardiente de la teofanía definitiva: llena del Espíritu Santo, presenta al Verbo en la humildad de su carne dándolo a conocer a los pobres (cf. \emph{Lc} 2, 15-19) y a las primicias de las naciones (cf. \emph{Mt} 2, 11).									
\end{ccebody}


\cceth{Cristo, luz de las naciones}

\cceref{CEC 280, 529, 748, 1165, 2466, 2715}

\begin{ccebody}
	\n{280} La creación es el fundamento de \textquote{todos los designios salvíficos de Dios}, \textquote{el comienzo de la historia de la salvación} (DCG 51), que culmina en Cristo. Inversamente, el Misterio de Cristo es la luz decisiva sobre el Misterio de la creación; revela el fin en vista del cual, \textquote{al principio, Dios creó el cielo y la tierra} (\emph{Gn} 1,1): desde el principio Dios preveía la gloria de la nueva creación en Cristo (cf. \emph{Rm} 8,18-23).
	
	\n{529} \emph{La Presentación de Jesús en el Templo} (cf. \emph{Lc} 2, 22-39) lo muestra como el Primogénito que pertenece al Señor (cf. \emph{Ex} 13,2.12-13). Con Simeón y Ana, toda la expectación de Israel es la que viene al \emph{Encuentro} de su Salvador (la tradición bizantina llama así a este acontecimiento). Jesús es reconocido como el Mesías tan esperado, \textquote{luz de las naciones} y \textquote{gloria de Israel}, pero también \textquote{signo de contradicción}. La espada de dolor predicha a María anuncia otra oblación, perfecta y única, la de la Cruz que dará la salvación que Dios ha preparado \textquote{ante todos los pueblos}.

	\n{748} \textquote{Cristo es la luz de los pueblos. Por eso, este sacrosanto Sínodo, reunido en el Espíritu Santo, desea vehementemente iluminar a todos los hombres con la luz de Cristo, que resplandece sobre el rostro de la Iglesia (LG 1), anunciando el Evangelio a todas las criaturas}. Con estas palabras comienza la \textquote{Constitución dogmática sobre la Iglesia} del Concilio Vaticano II. Así, el Concilio muestra que el artículo de la fe sobre la Iglesia depende enteramente de los artículos que se refieren a Cristo Jesús. La Iglesia no tiene otra luz que la de Cristo; ella es, según una imagen predilecta de los Padres de la Iglesia, comparable a la luna cuya luz es reflejo del sol.
	
	\n{1165} Cuando la Iglesia celebra el Misterio de Cristo, hay una palabra que jalona su oración: \emph{¡Hoy!}, como eco de la oración que le enseñó su Señor (\emph{Mt} 6,11) y de la llamada del Espíritu Santo (\emph{Hb} 3,7-4,11; \emph{Sal} 95,7). Este \textquote{hoy} del Dios vivo al que el hombre está llamado a entrar, es la \textquote{Hora} de la Pascua de Jesús, que atraviesa y guía toda la historia humana:

\begin{quote}
	\textquote{La vida se ha extendido sobre todos los seres y todos están llenos de una amplia luz: el Oriente de los orientes invade el universo, y el que existía \textquote{antes del lucero de la mañana} y antes de todos los astros, inmortal e inmenso, el gran Cristo brilla sobre todos los seres más que el sol. Por eso, para nosotros que creemos en él, se instaura un día de luz, largo, eterno, que no se extingue: la Pascua mística} (Pseudo-Hipólito Romano, \emph{In Sanctum Pascha} 1-2).
\end{quote}
	
	\n{2466} En Jesucristo la verdad de Dios se manifestó en plenitud. \textquote{Lleno de gracia y de verdad} (\emph{Jn} 1, 14), él es la \textquote{luz del mundo} (\emph{Jn} 8, 12), \emph{la Verdad} (cf. \emph{Jn} 14, 6). El que cree en él, no permanece en las tinieblas (cf. \emph{Jn} 12, 46). El discípulo de Jesús, \textquote{permanece en su palabra}, para conocer \textquote{la verdad que hace libre} (cf. \emph{Jn} 8, 31-32) y que santifica (cf. \emph{Jn} 17, 17). Seguir a Jesús es vivir del \textquote{Espíritu de verdad} (\emph{Jn} 14, 17) que el Padre envía en su nombre (cf. \emph{Jn} 14, 26) y que conduce \textquote{a la verdad completa} (\emph{Jn} 16, 13). Jesús enseña a sus discípulos el amor incondicional de la verdad: \textquote{Sea vuestro lenguaje: \textquote{sí, sí}; \textquote{no, no}} (\emph{Mt} 5, 37).
	
	\n{2715} La oración contemplativa es mirada de fe, fijada en Jesús. \textquote{Yo le miro y él me mira}, decía a su santo cura un campesino de Ars que oraba ante el Sagrario (cf. F. Trochu, \emph{Le Curé d'Ars Saint Jean-Marie Vianney}). Esta atención a Él es renuncia a \textquote{mí}. Su mirada purifica el corazón. La luz de la mirada de Jesús ilumina los ojos de nuestro corazón; nos enseña a ver todo a la luz de su verdad y de su compasión por todos los hombres. La contemplación dirige también su mirada a los misterios de la vida de Cristo. Aprende así el \textquote{conocimiento interno del Señor} para más amarle y seguirle (cf. San Ignacio de Loyola, \emph{Exercitia spiritualia,} 104).		
\end{ccebody}


\cceth{La Iglesia, sacramento de la unidad del género humano}

\cceref{CEC 60, 442, 674, 755, 767, 774-776, 781, 831}

\begin{ccebody}
	\n{60} El pueblo nacido de Abraham será el depositario de la promesa hecha a los patriarcas, el pueblo de la elección (cf. \emph{Rm} 11,28), llamado a preparar la reunión un día de todos los hijos de Dios en la unidad de la Iglesia (cf. \emph{Jn} 11,52; 10,16); ese pueblo será la raíz en la que serán injertados los paganos hechos creyentes (cf. \emph{Rm} 11,17-18.24).
	
	\n{442} No ocurre así con Pedro cuando confiesa a Jesús como \textquote{el Cristo, el Hijo de Dios vivo} (\emph{Mt} 16, 16) porque Jesús le responde con solemnidad \textquote{\emph{no te ha revelado} esto ni la carne ni la sangre, sino \emph{mi Padre} que está en los cielos} (\emph{Mt} 16, 17). Paralelamente Pablo dirá a propósito de su conversión en el camino de Damasco: \textquote{Cuando Aquel que me separó desde el seno de mi madre y me llamó por su gracia, tuvo a bien revelar en mí a su Hijo para que le anunciase entre los gentiles\ldots{}} (\emph{Ga} 1,15-16). \textquote{Y en seguida se puso a predicar a Jesús en las sinagogas: que él era el Hijo de Dios} (\emph{Hch} 9, 20). Este será, desde el principio (cf. \emph{1 Ts} 1, 10), el centro de la fe apostólica (cf. \emph{Jn} 20, 31) profesada en primer lugar por Pedro como cimiento de la Iglesia (cf. \emph{Mt} 16, 18).
	
	%r
\n{674} La venida del Mesías glorioso, en un momento determinado de la historia (cf. \emph{Rm} 11, 31), se vincula al reconocimiento del Mesías por \textquote{todo Israel} (\emph{Rm} 11, 26; \emph{Mt} 23, 39) del que \textquote{una parte está endurecida} (\emph{Rm} 11, 25) en \textquote{la incredulidad} (\emph{Rm} 11, 20) respecto a Jesús. San Pedro dice a los judíos de Jerusalén después de Pentecostés: \textquote{Arrepentíos, pues, y convertíos para que vuestros pecados sean borrados, a fin de que del Señor venga el tiempo de la consolación y envíe al Cristo que os había sido destinado, a Jesús, a quien debe retener el cielo hasta el tiempo de la restauración universal, de que Dios habló por boca de sus profetas} (\emph{Hch} 3, 19-21). Y san Pablo le hace eco: \textquote{si su reprobación ha sido la reconciliación del mundo ¿qué será su readmisión sino una resurrección de entre los muertos?} (\emph{Rm} 11, 5). La entrada de \textquote{la plenitud de los judíos} (\emph{Rm} 11, 12) en la salvación mesiánica, a continuación de \textquote{la plenitud de los gentiles} (\emph{Rm} 11, 25; cf. \emph{Lc} 21, 24), hará al pueblo de Dios \textquote{llegar a la plenitud de Cristo} (\emph{Ef} 4, 13) en la cual \textquote{Dios será todo en nosotros} (\emph{1 Co} 15, 28).
	
	\n{755} \textquote{La Iglesia es \emph{labranza} o campo de Dios (\emph{1 Co} 3, 9). En este campo crece el antiguo olivo cuya raíz santa fueron los patriarcas y en el que tuvo y tendrá lugar la reconciliación de los judíos y de los gentiles (\emph{Rm} 11, 13-26). El labrador del cielo la plantó como viña selecta (\emph{Mt} 21, 33-43 par.; cf. \emph{Is} 5, 1-7). La verdadera vid es Cristo, que da vida y fecundidad a los sarmientos, es decir, a nosotros, que permanecemos en él por medio de la Iglesia y que sin él no podemos hacer nada (\emph{Jn} 15, 1-5)}. (LG 6).
	
	\ccesec{La Iglesia, manifestada por el Espíritu Santo}

\n{767} \textquote{Cuando el Hijo terminó la obra que el Padre le encargó realizar en la tierra, fue enviado el Espíritu Santo el día de Pentecostés para que santificara continuamente a la Iglesia} (LG 4). Es entonces cuando \textquote{la Iglesia se manifestó públicamente ante la multitud; se inició la difusión del Evangelio entre los pueblos mediante la predicación} (AG 4). Como ella es \textquote{convocatoria} de salvación para todos los hombres, la Iglesia es, por su misma naturaleza, misionera enviada por Cristo a todas las naciones para hacer de ellas discípulos suyos (cf. \emph{Mt} 28, 19-20; AG 2,5-6).
	
		\ccesec{La Iglesia, sacramento universal de la salvación}

\n{774} La palabra griega \emph{mysterion} ha sido traducida en latín por dos términos: \emph{mysterium} y \emph{sacramentum}. En la interpretación posterior, el término \emph{sacramentum} expresa mejor el signo visible de la realidad oculta de la salvación, indicada por el término \emph{mysterium}. En este sentido, Cristo es Él mismo el Misterio de la salvación: \emph{Non est enim aliud Dei mysterium, nisi Christus} (\textquote{No hay otro misterio de Dios fuera de Cristo}; san Agustín, \emph{Epistula} 187, 11, 34). La obra salvífica de su humanidad santa y santificante es el sacramento de la salvación que se manifiesta y actúa en los sacramentos de la Iglesia (que las Iglesias de Oriente llaman también \textquote{los santos Misterios}). Los siete sacramentos son los signos y los instrumentos mediante los cuales el Espíritu Santo distribuye la gracia de Cristo, que es la Cabeza, en la Iglesia que es su Cuerpo. La Iglesia contiene, por tanto, y comunica la gracia invisible que ella significa. En este sentido analógico ella es llamada \textquote{sacramento}.

	
	\n{775} \textquote{La Iglesia es en Cristo como un sacramento o signo e instrumento de la unión íntima con Dios y de la unidad de todo el género humano} (LG 1): Ser el \emph{sacramento de la unión íntima de los hombres con Dios} es el primer fin de la Iglesia. Como la comunión de los hombres radica en la unión con Dios, la Iglesia es también el sacramento de la \emph{unidad del género humano}. Esta unidad ya está comenzada en ella porque reúne hombres \textquote{de toda nación, raza, pueblo y lengua} (\emph{Ap} 7, 9); al mismo tiempo, la Iglesia es \textquote{signo e instrumento} de la plena realización de esta unidad que aún está por venir.
	
	\n{776} Como sacramento, la Iglesia es instrumento de Cristo. Ella es asumida por Cristo \textquote{como instrumento de redención universal} (LG 9), \textquote{sacramento universal de salvación} (LG 48), por medio del cual Cristo \textquote{manifiesta y realiza al mismo tiempo el misterio del amor de Dios al hombre} (GS 45, 1). Ella \textquote{es el proyecto visible del amor de Dios hacia la humanidad} (Pablo VI, \emph{Discurso a los Padres del Sacro Colegio Cardenalicio}, 22 junio 1973) que quiere \textquote{que todo el género humano forme un único Pueblo de Dios, se una en un único Cuerpo de Cristo, se coedifique en un único templo del Espíritu Santo} (AG 7; cf. LG 17).
	
	\n{781} \textquote{En todo tiempo y lugar ha sido grato a Dios el que le teme y practica la justicia. Sin embargo, quiso santificar y salvar a los hombres no individualmente y aislados, sin conexión entre sí, sino hacer de ellos un pueblo para que le conociera de verdad y le sirviera con una vida santa. Eligió, pues, a Israel para pueblo suyo, hizo una alianza con él y lo fue educando poco a poco. Le fue revelando su persona y su plan a lo largo de su historia y lo fue santificando. Todo esto, sin embargo, sucedió como preparación y figura de su alianza nueva y perfecta que iba a realizar en Cristo [\ldots{}], es decir, el Nuevo Testamento en su sangre, convocando a las gentes de entre los judíos y los gentiles para que se unieran, no según la carne, sino en el Espíritu} (LG 9).
	
	\n{831} {[}La Iglesia{]} es católica porque ha sido enviada por Cristo en misión a la totalidad del género humano (cf. \emph{Mt} 28, 19):

\begin{quote}
	\textquote{Todos los hombres están invitados al Pueblo de Dios. Por eso este pueblo, uno y único, ha de extenderse por todo el mundo a través de todos los siglos, para que así se cumpla el designio de Dios, que en el principio creó una única naturaleza humana y decidió reunir a sus hijos dispersos [\ldots{}] Este carácter de universalidad, que distingue al pueblo de Dios, es un don del mismo Señor. Gracias a este carácter, la Iglesia Católica tiende siempre y eficazmente a reunir a la humanidad entera con todos sus valores bajo Cristo como Cabeza, en la unidad de su Espíritu} (LG 13).
\end{quote}

		
\end{ccebody}


\cceth{Jesús observa la Ley y la perfecciona}

\cceref{CEC 527, 577-582}


\begin{ccebody}
	\n{527} La \emph{Circuncisión} de Jesús, al octavo día de su nacimiento (cf. \emph{Lc} 2, 21) es señal de su inserción en la descendencia de Abraham, en el pueblo de la Alianza, de su sometimiento a la Ley (cf. \emph{Ga} 4, 4) y de su consagración al culto de Israel en el que participará durante toda su vida. Este signo prefigura \textquote{la circuncisión en Cristo} que es el Bautismo (\emph{Col} 2, 11-13).
	
	\ccesec{Jesús y la Ley}

\n{577} Al comienzo del Sermón de la Montaña, Jesús hace una advertencia solemne presentando la Ley dada por Dios en el Sinaí con ocasión de la Primera Alianza, a la luz de la gracia de la Nueva Alianza:

\begin{quote}
	\textquote{No penséis que he venido a abolir la Ley y los Profetas. No he venido a abolir sino a dar cumplimiento. Sí, os lo aseguro: el cielo y la tierra pasarán antes que pase una \textquote{i} o un ápice de la Ley sin que todo se haya cumplido. Por tanto, el que quebrante uno de estos mandamientos menores, y así lo enseñe a los hombres, será el menor en el Reino de los cielos; en cambio el que los observe y los enseñe, ése será grande en el Reino de los cielos} (\emph{Mt} 5, 17-19).
\end{quote}


	

	\n{578} Jesús, el Mesías de Israel, por lo tanto el más grande en el Reino de los cielos, se debía sujetar a la Ley cumpliéndola en su totalidad hasta en sus menores preceptos, según sus propias palabras. Incluso es el único en poderlo hacer perfectamente (cf. \emph{Jn} 8, 46). Los judíos, según su propia confesión, jamás han podido cumplir la Ley en su totalidad, sin violar el menor de sus preceptos (cf. \emph{Jn} 7, 19; \emph{Hch} 13, 38-41; 15, 10). Por eso, en cada fiesta anual de la Expiación, los hijos de Israel piden perdón a Dios por sus transgresiones de la Ley. En efecto, la Ley constituye un todo y, como recuerda Santiago, \textquote{quien observa toda la Ley, pero falta en un solo precepto, se hace reo de todos} (\emph{St} 2, 10; cf. \emph{Ga} 3, 10; 5, 3).	

	\n{579} Este principio de integridad en la observancia de la Ley, no sólo en su letra sino también en su espíritu, era apreciado por los fariseos. Al subrayarlo para Israel, muchos judíos del tiempo de Jesús fueron conducidos a un celo religioso extremo (cf. \emph{Rm} 10, 2), el cual, si no quería convertirse en una casuística \textquote{hipócrita} (cf. \emph{Mt} 15, 3-7; \emph{Lc} 11, 39-54) no podía más que preparar al pueblo a esta intervención inaudita de Dios que será la ejecución perfecta de la Ley por el único Justo en lugar de todos los pecadores (cf. \emph{Is} 53, 11; \emph{Hb} 9, 15).	
	
	\n{580} El cumplimiento perfecto de la Ley no podía ser sino obra del divino Legislador que nació sometido a la Ley en la persona del Hijo (cf. \emph{Ga} 4, 4). En Jesús la Ley ya no aparece grabada en tablas de piedra sino \textquote{en el fondo del corazón} (\emph{Jr} 31, 33) del Siervo, quien, por \textquote{aportar fielmente el derecho} (\emph{Is} 42, 3), se ha convertido en \textquote{la Alianza del pueblo} (\emph{Is} 42, 6). Jesús cumplió la Ley hasta tomar sobre sí mismo \textquote{la maldición de la Ley} (\emph{Ga} 3, 13) en la que habían incurrido los que no \textquote{practican todos los preceptos de la Ley} (\emph{Ga} 3, 10) porque \textquote{ha intervenido su muerte para remisión de las transgresiones de la Primera Alianza} (\emph{Hb} 9, 15).	
	
	\n{581} Jesús fue considerado por los judíos y sus jefes espirituales como un \textquote{rabbi} (cf. \emph{Jn} 11, 28; 3, 2; \emph{Mt} 22, 23-24, 34-36). Con frecuencia argumentó en el marco de la interpretación rabínica de la Ley (cf. \emph{Mt} 12, 5; 9, 12; \emph{Mc} 2, 23-27; \emph{Lc} 6, 6-9; \emph{Jn} 7, 22-23). Pero al mismo tiempo, Jesús no podía menos que chocar con los doctores de la Ley porque no se contentaba con proponer su interpretación entre los suyos, sino que \textquote{enseñaba como quien tiene autoridad y no como los escribas} (\emph{Mt} 7, 28-29). La misma Palabra de Dios, que resonó en el Sinaí para dar a Moisés la Ley escrita, es la que en Él se hace oír de nuevo en el Monte de las Bienaventuranzas (cf. \emph{Mt} 5, 1). Esa palabra no revoca la Ley sino que la perfecciona aportando de modo divino su interpretación definitiva: \textquote{Habéis oído también que se dijo a los antepasados [\ldots{}] pero yo os digo} (\emph{Mt} 5, 33-34). Con esta misma autoridad divina, desaprueba ciertas \textquote{tradiciones humanas} (\emph{Mc} 7, 8) de los fariseos que \textquote{anulan la Palabra de Dios} (\emph{Mc} 7, 13).
	
	\n{582} Yendo más lejos, Jesús da plenitud a la Ley sobre la pureza de los alimentos, tan importante en la vida cotidiana judía, manifestando su sentido \textquote{pedagógico} (cf. \emph{Ga} 3, 24) por medio de una interpretación divina: \textquote{Todo lo que de fuera entra en el hombre no puede hacerle impuro [\ldots{}] ---así declaraba puros todos los alimentos---. Lo que sale del hombre, eso es lo que hace impuro al hombre. Porque de dentro, del corazón de los hombres, salen las intenciones malas} (\emph{Mc} 7, 18-21). Jesús, al dar con autoridad divina la interpretación definitiva de la Ley, se vio enfrentado a algunos doctores de la Ley que no aceptaban su interpretación a pesar de estar garantizada por los signos divinos con que la acompañaba (cf. \emph{Jn} 5, 36; 10, 25. 37-38; 12, 37). Esto ocurre, en particular, respecto al problema del sábado: Jesús recuerda, frecuentemente con argumentos rabínicos (cf. \emph{Mt} 2,25-27; \emph{Jn} 7, 22-24), que el descanso del sábado no se quebranta por el servicio de Dios (cf. \emph{Mt} 12, 5; \emph{Nm} 28, 9) o al prójimo (cf. \emph{Lc} 13, 15-16; 14, 3-4) que realizan sus curaciones.							
\end{ccebody}


\cceth{La Ley nueva nos libera de las restricciones de la Ley antigua}

\cceref{CEC 580, 1972}


\begin{ccebody}
	\n{580} El cumplimiento perfecto de la Ley no podía ser sino obra del divino Legislador que nació sometido a la Ley en la persona del Hijo (cf. \emph{Ga} 4, 4). En Jesús la Ley ya no aparece grabada en tablas de piedra sino \textquote{en el fondo del corazón} (\emph{Jr} 31, 33) del Siervo, quien, por \textquote{aportar fielmente el derecho} (\emph{Is} 42, 3), se ha convertido en \textquote{la Alianza del pueblo} (\emph{Is} 42, 6). Jesús cumplió la Ley hasta tomar sobre sí mismo \textquote{la maldición de la Ley} (\emph{Ga} 3, 13) en la que habían incurrido los que no \textquote{practican todos los preceptos de la Ley} (\emph{Ga} 3, 10) porque \textquote{ha intervenido su muerte para remisión de las transgresiones de la Primera Alianza} (\emph{Hb} 9, 15).
	
	\n{1972} La Ley nueva es llamada \emph{ley de amor}, porque hace obrar por el amor que infunde el Espíritu Santo más que por el temor; \emph{ley de gracia}, porque confiere la fuerza de la gracia para obrar mediante la fe y los sacramentos; \emph{ley de libertad} (cf. \emph{St} 1, 25; 2, 12), porque nos libera de las observancias rituales y jurídicas de la Ley antigua, nos inclina a obrar espontáneamente bajo el impulso de la caridad y nos hace pasar de la condición del siervo \textquote{que ignora lo que hace su señor}, a la de amigo de Cristo, \textquote{porque todo lo que he oído a mi Padre os lo he dado a conocer} (\emph{Jn} 15, 15), o también a la condición de hijo heredero (cf. \emph{Ga} 4, 1-7. 21-31; \emph{Rm} 8, 15).								
\end{ccebody}



\cceth{Por medio del Espíritu Santo podemos llamar a Dios \textquote{Abba}}

\cceref{CEC 683, 689, 1695, 2766, 2777-2778}


\begin{ccebody}
	\n{683} \textquote{Nadie puede decir: \textquote{¡Jesús es Señor!} sino por influjo del Espíritu Santo} (\emph{1 Co} 12, 3). \textquote{Dios ha enviado a nuestros corazones el Espíritu de su Hijo que clama ¡\emph{Abbá}, Padre!} (\emph{Ga} 4, 6). Este conocimiento de fe no es posible sino en el Espíritu Santo. Para entrar en contacto con Cristo, es necesario primeramente haber sido atraído por el Espíritu Santo. Él es quien nos precede y despierta en nosotros la fe. Mediante el Bautismo, primer sacramento de la fe, la vida, que tiene su fuente en el Padre y se nos ofrece por el Hijo, se nos comunica íntima y personalmente por el Espíritu Santo en la Iglesia:

\begin{quote}
	El Bautismo \textquote{nos da la gracia del nuevo nacimiento en Dios Padre por medio de su Hijo en el Espíritu Santo. Porque los que son portadores del Espíritu de Dios son conducidos al Verbo, es decir al Hijo; pero el Hijo los presenta al Padre, y el Padre les concede la incorruptibilidad. Por tanto, sin el Espíritu no es posible ver al Hijo de Dios, y, sin el Hijo, nadie puede acercarse al Padre, porque el conocimiento del Padre es el Hijo, y el conocimiento del Hijo de Dios se logra por el Espíritu Santo} (San Ireneo de Lyon, \emph{Demonstratio praedicationis apostolicae}, 7: SC 62 41-42).
\end{quote}


	
	\ccesec{La misión conjunta del Hijo y del Espíritu Santo}

\n{689} Aquel al que el Padre ha enviado a nuestros corazones, el Espíritu de su Hijo (cf. \emph{Ga} 4, 6) es realmente Dios. Consubstancial con el Padre y el Hijo, es inseparable de ellos, tanto en la vida íntima de la Trinidad como en su don de amor para el mundo. Pero al adorar a la Santísima Trinidad vivificante, consubstancial e indivisible, la fe de la Iglesia profesa también la distinción de las Personas. Cuando el Padre envía su Verbo, envía también su Aliento: misión conjunta en la que el Hijo y el Espíritu Santo son distintos pero inseparables. Sin ninguna duda, Cristo es quien se manifiesta, Imagen visible de Dios invisible, pero es el Espíritu Santo quien lo revela.	
	
	\n{1695} \textquote{Justificados [\ldots{}] en el nombre del Señor Jesucristo y en el Espíritu de nuestro Dios} (\emph{1 Co} 6,11.), \textquote{santificados y llamados a ser santos} (\emph{1 Co} 1,2.), los cristianos se convierten en \textquote{el templo [\ldots{}] del Espíritu Santo} (cf. \emph{1 Co} 6,19). Este \textquote{Espíritu del Hijo} les enseña a orar al Padre (\emph{Ga} 4, 6) y, haciéndose vida en ellos, les hace obrar (cf. \emph{Ga} 5, 25) para dar \textquote{los frutos del Espíritu} (\emph{Ga} 5, 22.) por la caridad operante. Sanando las heridas del pecado, el Espíritu Santo nos renueva interiormente mediante una transformación espiritual (cf. \emph{Ef} 4, 23.), nos ilumina y nos fortalece para vivir como \textquote{hijos de la luz} (\emph{Ef} 5, 8.), \textquote{por la bondad, la justicia y la verdad} en todo (\emph{Ef} 5,9.).	
	
	\n{2766} Pero Jesús no nos deja una fórmula para repetirla de modo mecánico (cf. \emph{Mt} 6, 7; \emph{1 R} 18, 26-29). Como en toda oración vocal, el Espíritu Santo, a través de la Palabra de Dios, enseña a los hijos de Dios a hablar con su Padre. Jesús no sólo nos enseña las palabras de la oración filial, sino que nos da también el Espíritu por el que estas se hacen en nosotros \textquote{espíritu [\ldots{}] y vida} (\emph{Jn} 6, 63). Más todavía: la prueba y la posibilidad de nuestra oración filial es que el Padre \textquote{ha enviado [\ldots{}] a nuestros corazones el Espíritu de su Hijo que clama: \textquote{¡Abbá, Padre!}} (\emph{Ga} 4, 6). Ya que nuestra oración interpreta nuestros deseos ante Dios, es también \textquote{el que escruta los corazones}, el Padre, quien \textquote{conoce cuál es la aspiración del Espíritu, y que su intercesión en favor de los santos es según Dios} (\emph{Rm} 8, 27). La oración al Padre se inserta en la misión misteriosa del Hijo y del Espíritu.	
	
	\ccesec{Acercarse a Él con toda confianza}

\n{2777} En la liturgia romana, se invita a la asamblea eucarística a rezar el Padre Nuestro con una audacia filial; las liturgias orientales usan y desarrollan expresiones análogas: \textquote{Atrevernos con toda confianza}, \textquote{Haznos dignos de}. Ante la zarza ardiendo, se le dijo a Moisés: \textquote{No te acerques aquí. Quita las sandalias de tus pies} (\emph{Ex} 3, 5). Este umbral de la santidad divina, sólo lo podía franquear Jesús, el que \textquote{después de llevar a cabo la purificación de los pecados} (\emph{Hb} 1, 3), nos introduce en presencia del Padre: \textquote{Hénos aquí, a mí y a los hijos que Dios me dio} (\emph{Hb} 2, 13):

\begin{quote}
	\textquote{La conciencia que tenemos de nuestra condición de esclavos nos haría meternos bajo tierra, nuestra condición terrena se desharía en polvo, si la autoridad de nuestro mismo Padre y el Espíritu de su Hijo, no nos empujasen a proferir este grito: \textquote{Abbá, Padre} (\emph{Rm} 8, 15) \ldots{} ¿Cuándo la debilidad de un mortal se atrevería a llamar a Dios Padre suyo, sino solamente cuando lo íntimo del hombre está animado por el Poder de lo alto?} (San Pedro Crisólogo, \emph{Sermón} 71, 3).
\end{quote}

	
	
	\n{2778} Este poder del Espíritu que nos introduce en la Oración del Señor se expresa en las liturgias de Oriente y de Occidente con la bella palabra, típicamente cristiana: \emph{parrhesia}, simplicidad sin desviación, conciencia filial, seguridad alegre, audacia humilde, certeza de ser amado (cf. \emph{Ef} 3, 12; \emph{Hb} 3, 6; 4, 16; 10, 19; \emph{1 Jn} 2,28; 3, 21; 5, 14).
\end{ccebody}



\cceth{El nombre de Jesús}

\cceref{CEC 430-435, 2666-2668, 2812}


\begin{ccebody}
	\n{430} \emph{Jesús} quiere decir en hebreo: \textquote{Dios salva}. En el momento de la anunciación, el ángel Gabriel le dio como nombre propio el nombre de Jesús que expresa a la vez su identidad y su misión (cf. \emph{Lc} 1, 31). Ya que \textquote{¿quién puede perdonar pecados, sino sólo Dios?} (\emph{Mc} 2, 7), es Él quien, en Jesús, su Hijo eterno hecho hombre \textquote{salvará a su pueblo de sus pecados} (\emph{Mt} 1, 21). En Jesús, Dios recapitula así toda la historia de la salvación en favor de los hombres.
	
	\n{431} En la historia de la salvación, Dios no se ha contentado con librar a Israel de \textquote{la casa de servidumbre} (\emph{Dt} 5, 6) haciéndole salir de Egipto. Él lo salva además de su pecado. Puesto que el pecado es siempre una ofensa hecha a Dios (cf. \emph{Sal} 51, 6), sólo Él es quien puede absolverlo (cf. \emph{Sal} 51, 12). Por eso es por lo que Israel, tomando cada vez más conciencia de la universalidad del pecado, ya no podrá buscar la salvación más que en la invocación del nombre de Dios Redentor (cf. \emph{Sal} 79, 9).	
	
	\n{432} El nombre de Jesús significa que el Nombre mismo de Dios está presente en la Persona de su Hijo (cf. \emph{Hch} 5, 41; \emph{3 Jn} 7) hecho hombre para la Redención universal y definitiva de los pecados. Él es el Nombre divino, el único que trae la salvación (cf. \emph{Jn} 3, 18; \emph{Hch} 2, 21) y de ahora en adelante puede ser invocado por todos porque se ha unido a todos los hombres por la Encarnación (cf. \emph{Rm} 10, 6-13) de tal forma que \textquote{no hay bajo el cielo otro nombre dado a los hombres por el que nosotros debamos salvarnos} (\emph{Hch} 4, 12; cf. \emph{Hch} 9, 14; \emph{St} 2, 7).	
	
	\n{433} El Nombre de Dios Salvador era invocado una sola vez al año por el sumo sacerdote para la expiación de los pecados de Israel, cuando había asperjado el propiciatorio del Santo de los Santos con la sangre del sacrificio (cf. \emph{Lv} 16, 15-16; \emph{Si} 50, 20; \emph{Hb} 9, 7). El propiciatorio era el lugar de la presencia de Dios (cf. \emph{Ex} 25, 22; \emph{Lv} 16, 2; \emph{Nm} 7, 89; \emph{Hb} 9, 5). Cuando san Pablo dice de Jesús que \textquote{Dios lo exhibió como instrumento de propiciación por su propia sangre} (\emph{Rm} 3, 25) significa que en su humanidad \textquote{estaba Dios reconciliando al mundo consigo} (\emph{2 Co} 5, 19).	
	
	\n{434} La Resurrección de Jesús glorifica el Nombre de Dios \textquote{Salvador} (cf. \emph{Jn} 12, 28) porque de ahora en adelante, el Nombre de Jesús es el que manifiesta en plenitud el poder soberano del \textquote{Nombre que está sobre todo nombre} (\emph{Flp} 2, 9). Los espíritus malignos temen su Nombre (cf. \emph{Hch} 16, 16-18; 19, 13-16) y en su nombre los discípulos de Jesús hacen milagros (cf. \emph{Mc} 16, 17) porque todo lo que piden al Padre en su Nombre, Él se lo concede (\emph{Jn} 15, 16).	
	
	\n{435} El Nombre de Jesús está en el corazón de la plegaria cristiana. Todas las oraciones litúrgicas se acaban con la fórmula \emph{Per Dominum nostrum Jesum Christum\ldots{}} (\textquote{Por nuestro Señor Jesucristo\ldots{}}). El \textquote{Avemaría} culmina en \textquote{y bendito es el fruto de tu vientre, Jesús}. La oración del corazón, en uso en Oriente, llamada \textquote{oración a Jesús} dice: \textquote{Señor Jesucristo, Hijo de Dios, ten piedad de mí pecador}. Numerosos cristianos mueren, como santa Juana de Arco, teniendo en sus labios una única palabra: \textquote{Jesús}.
	
	\n{2666} Pero el Nombre que todo lo contiene es aquel que el Hijo de Dios recibe en su encarnación: JESÚS. El nombre divino es inefable para los labios humanos (cf. \emph{Ex} 3, 14; 33, 19-23), pero el Verbo de Dios, al asumir nuestra humanidad, nos lo entrega y nosotros podemos invocarlo: \textquote{Jesús}, \textquote{YHVH salva} (cf. \emph{Mt} 1, 21). El Nombre de Jesús contiene todo: Dios y el hombre y toda la Economía de la creación y de la salvación. Decir \textquote{Jesús} es invocarlo desde nuestro propio corazón. Su Nombre es el único que contiene la presencia que significa. Jesús es el resucitado, y cualquiera que invoque su Nombre acoge al Hijo de Dios que le amó y se entregó por él (cf. \emph{Rm} 10, 13; \emph{Hch} 2, 21; 3, 15-16; \emph{Ga} 2, 20).

	\n{2667} Esta invocación de fe bien sencilla ha sido desarrollada en la tradición de la oración bajo formas diversas en Oriente y en Occidente. La formulación más habitual, transmitida por los espirituales del Sinaí, de Siria y del Monte Athos es la invocación: \textquote{Señor Jesucristo, Hijo de Dios, ten piedad de nosotros, pecadores} Conjuga el himno cristológico de \emph{Flp} 2, 6-11 con la petición del publicano y del mendigo ciego (cf. \emph{Lc} 18,13; \emph{Mc} 10, 46-52). Mediante ella, el corazón está acorde con la miseria de los hombres y con la misericordia de su Salvador.	
	
	\n{2668} La invocación del santo Nombre de Jesús es el camino más sencillo de la oración continua. Repetida con frecuencia por un corazón humildemente atento, no se dispersa en \textquote{palabrerías} (\emph{Mt} 6, 7), sino que \textquote{conserva la Palabra y fructifica con perseverancia} (cf. \emph{Lc} 8, 15). Es posible \textquote{en todo tiempo} porque no es una ocupación al lado de otra, sino la única ocupación, la de amar a Dios, que anima y transfigura toda acción en Cristo Jesús.
	
	\n{2812} Finalmente, el Nombre de Dios Santo se nos ha revelado y dado, en la carne, en Jesús, como Salvador (cf. \emph{Mt} 1, 21; \emph{Lc} 1, 31): revelado por lo que Él es, por su Palabra y por su Sacrificio (cf. \emph{Jn} 8, 28; 17, 8; 17, 17-19). Esto es el núcleo de su oración sacerdotal: \textquote{Padre santo \ldots{} por ellos me consagro a mí mismo, para que ellos también sean consagrados en la verdad} (\emph{Jn} 17, 19). Jesús nos \textquote{manifiesta} el Nombre del Padre (\emph{Jn} 17, 6) porque \textquote{santifica} Él mismo su Nombre (cf. \emph{Ez} 20, 39; 36, 20-21). Al terminar su Pascua, el Padre le da el Nombre que está sobre todo nombre: Jesús es Señor para gloria de Dios Padre (cf. \emph{Flp} 2, 9-11).																	
\end{ccebody}
	