\chapter{Santa María Madre de Dios}

\section{Lecturas}

\rtitle{PRIMERA LECTURA}

\rbook{Del libro de los Números} \rred{6, 22-27}

\rtheme{Invocarán mi nombre sobre los hijos de Israel y yo los bendeciré}

\begin{scripture}
	El Señor habló a Moisés:
	
	\textquote{Di a Aarón y a sus hijos, esta es la fórmula con la que bendeciréis a los hijos de Israel: \\\begin{readprose}
			\textquote{El Señor te bendiga y te proteja, \\ilumine su rostro sobre ti \\y te conceda su favor. \\El Señor te muestre su rostro \\y te conceda la paz}.
		\end{readprose}
		
		Así invocarán mi nombre sobre los hijos de Israel y yo los bendeciré}.
\end{scripture}


\rtitle{SALMO RESPONSORIAL}

\rbook{Salmo} \rred{66, 2-3. 5. 6 y 8}

\rtheme{Que Dios tenga piedad y nos bendiga}

\begin{psbody}
	Que Dios tenga piedad nos bendiga,
	ilumine su rostro sobre nosotros;
	conozca la tierra tus caminos,
	todos los pueblos tu salvación.
	
	Que canten de alegría las naciones,
	porque riges el mundo con justicia
	y gobiernas las naciones de la tierra.
	
	Oh Dios, que te alaben los pueblos,
	que todos los pueblos te alaben.
	Que Dios nos bendiga; que le teman
	todos los confines de la tierra.
\end{psbody}


\rtitle{SEGUNDA LECTURA}

\rbook{De la carta del apóstol san Pablo a los Gálatas} \rred{4, 4-7}

\rtheme{Envió Dios a su Hijo, nacido de mujer}

\begin{scripture}
	Hermanos:
	
	Cuando llegó la plenitud del tiempo, envió Dios a su Hijo, nacido de mujer, nacido bajo la ley, para rescatar a los que estaban bajo la ley, para que recibiéramos la adopción filial.
	
	Como sois hijos, Dios envió a nuestros corazones el Espíritu de su Hijo, que clama: \emph{\textquote{¡Abba,} Padre!}. Así que ya no eres esclavo, sino hijo; y si eres hijo, eres también heredero por voluntad de Dios.
\end{scripture}

\rtitle{EVANGELIO}

\rbook{Del Santo Evangelio según san Lucas} \rred{2, 16-21}

\rtheme{Encontraron a María y a José y al niño. Y a los ocho días, le pusieron por nombre Jesús}

\begin{scripture}
	En aquel tiempo, los pastores fueron corriendo hacia belén y encontraron a María y a José, y al niño acostado en el pesebre. Al verlo, contaron lo que se les había dicho de aquel niño.
	
	Todos los que lo oían se admiraban de lo que les habían dicho los pastores. María, por su parte, conservaba todas estas cosas, meditándolas en su corazón.
	
	Y se volvieron los pastores dando gloria y alabanza a Dios por todo lo que habían oído y visto, conforme a lo que se les había dicho.
	
	Cuando se cumplieron los ocho días para circuncidar al niño, le pusieron por nombre Jesús, como lo había llamado el ángel antes de su concepción.
\end{scripture}

\newsection

\section{Comentario Patrístico}

\subsection{San Atanasio de Alejandría, obispo}

\ptheme{La Palabra tomó de María nuestra condición}

\src{Carta a Epicteto, 5-9: PG 26, 1058. 1062-1066\cite{AtanasioDeAlejandria_PG026_1058}}

\begin{body}
	\ltr{L}{a} Palabra \emph{tendió una mano a los hijos de Abrahán,} como afirma el Apóstol, \emph{y por eso tenía que parecerse en todo a sus hermanos} y asumir un cuerpo semejante al nuestro. Por esta razón, en verdad, María está presente en este misterio, para que de ella la Palabra tome un cuerpo, y, como propio, lo ofrezca por nosotros. La Escritura habla del parto y afirma: \emph{Lo envolvió en pañales;} y se proclaman dichosos los pechos que amamantaron al Señor, y, por el nacimiento de este primogénito, fue ofrecido el sacrificio prescrito. El ángel Gabriel había anunciado esta concepción con palabras muy precisas, cuando dijo a María no simplemente \textquote{\emph{lo que nacerá en ti}} ---para que no se creyese que se trataba de un cuerpo introducido desde el exterior---, sino \emph{de} para que creyéramos que aquel que era engendrado en María procedía realmente de ella.
	
	Las cosas sucedieron de esta forma para que la Palabra, tomando nuestra condición y ofreciéndola en sacrificio, la asumiese completamente, y revistiéndonos después a nosotros de su condición, diese ocasión al Apóstol para afirmar lo siguiente: \emph{Esto corruptible tiene que vestirse de incorrupción, y esto mortal tiene que vestirse de inmortalidad}.
	
	Estas cosas no son una ficción, como algunos juzgaron; ¡tal postura es inadmisible! Nuestro Salvador fue verdaderamente hombre, y de él ha conseguido la salvación el hombre entero. Porque de ninguna forma es ficticia nuestra salvación ni afecta sólo al cuerpo, sino que la salvación de todo el hombre, es decir, alma y cuerpo, se ha realizado en aquel que es la Palabra.
	
	Por lo tanto, el cuerpo que el Señor asumió de María era un verdadero cuerpo humano, conforme lo atestiguan las Escrituras; verdadero, digo, porque fue un cuerpo igual al nuestro. Pues María es nuestra hermana, ya que todos nosotros hemos nacido de Adán.
	
	Lo que Juan afirma: \emph{La Palabra se hizo carne,} tiene la misma significación, como se puede concluir de la idéntica forma de expresarse. En san Pablo encontramos escrito: \emph{Cristo se hizo por nosotros un maldito}. Pues al cuerpo humano, por la unión y comunión con la Palabra, se le ha concedido un inmenso beneficio: de mortal se ha hecho inmortal, de animal se ha hecho espiritual, y de terreno ha penetrado las puertas del cielo.
	
	Por otra parte, la Trinidad, también después de la encarnación de la Palabra en María, siempre sigue siendo la Trinidad, no admitiendo ni aumentos ni disminuciones; siempre es perfecta, y en la Trinidad se reconoce una única Deidad, y así la Iglesia confiesa a un único Dios, Padre de la Palabra.
\end{body}




\newsection

\subsection{San Cirilo de Alejandría, obispo}

\ptheme{Alabanzas de la Madre de Dios}

\src{Homilía 4, pronunciada en el Concilio de Éfeso: PG 77, 991. 995-996. \cite{CiriloDeAlejandria_PG077_0991}}

\begin{body}
	\ltr{T}{engo} ante mis ojos la asamblea de los santos padres que, llenos de gozo y fervor, han acudido aquí, respondiendo con prontitud a la invitación de la santa Madre de Dios, la siempre Virgen María. Este espectáculo ha trocado en gozo la gran tristeza que antes me oprimía. Vemos realizadas en esta reunión aquellas hermosas palabras de David, el salmista: \emph{Ved qué dulzura, qué delicia; convivir los hermanos unidos}.
	
	Te saludamos, santa y misteriosa Trinidad, que nos has convocado a todos nosotros en esta iglesia de santa María, Madre de Dios.
	
	Te saludamos, María, Madre de Dios, tesoro digno de ser venerado por todo el orbe, lámpara inextinguible, corona de la virginidad, trono de la recta doctrina, templo indestructible, lugar propio de aquel que no puede ser contenido en lugar alguno, madre y virgen, por quien es llamado bendito, en los santos evangelios, el que viene en nombre del Señor.
	
	Te saludamos, a ti, que encerraste en tu seno virginal a aquel que es inmenso e inabarcable; a ti, por quien la santa Trinidad es adorada y glorificada; por quien la cruz preciosa es celebrada y adorada en todo el orbe; por quien exulta el cielo; por quien se alegran los ángeles y arcángeles; por quien son puestos en fuga los demonios; por quien el diablo tentador cayó del cielo; por quien la criatura, caída en el pecado, es elevada al cielo; por quien la creación, sujeta a la insensatez de la idolatría, llega al conocimiento de la verdad; por quien los creyentes obtienen la gracia del bautismo y el aceite de la alegría; por quien han sido fundamentadas las Iglesias en el orbe de la tierra; por quien todos los hombres son llamados a la conversión.
	
	Y ¿qué más diré? Por ti, el Hijo unigénito de Dios ha iluminado a \emph{los que vivían en tinieblas y en sombra de muerte}; por ti, los profetas anunciaron las cosas futuras; por ti, los apóstoles predicaron la salvación a los gentiles; por ti, los muertos resucitan; por ti, reinan los reyes, por la santísima Trinidad.
	
	¿Quién habrá que sea capaz de cantar como es debido las alabanzas de María? Ella es madre y virgen a la vez; ¡qué cosa tan admirable! Es una maravilla que me llena de estupor. ¿Quién ha oído jamás decir que le esté prohibido al constructor habitar en el mismo templo que él ha construido? ¿Quién podrá tachar de ignominia el hecho de que la sirviente sea adoptada como madre?
	
	Mirad: hoy todo el mundo se alegra; quiera Dios que todos nosotros reverenciemos y adoremos la unidad, que rindamos un culto impregnado de santo temor a la Trinidad indivisa, al celebrar, con nuestras alabanzas, a María siempre Virgen, el templo santo de Dios, y a su Hijo y esposo inmaculado: porque a él pertenece la gloria por los siglos de los siglos. Amén.
\end{body}

\begin{patercite}
	Habitaba en la tierra \\y llenaba los cielos \\la Palabra de Dios infinita. \\Su bajada amorosa hasta el hombre no cambió su morada superna. \\Era el parto divino de Virgen\\ que este canto escuchaba:
	
	Salve, mansión que contiene el Inmenso; Salve, dintel del augusto Misterio.
	
	Salve, de incrédulo equívoco anuncio; Salve, del fiel inequívoco orgullo.
	
	Salve, carroza del Santo que portan querubes;
	
	Salve, sitial del que adoran sin fin serafines.
	
	Salve, tú sola has unido dos cosas opuestas;
	
	Salve, tú sola a la vez eres Virgen y Madre.
	
	Salve, por ti fue borrada la culpa;
	
	Salve, por ti Dios abrió el Paraíso.
	
	Salve, tú llave del Reino de Cristo;
	
	Salve, esperanza de bienes eternos.
	
	Salve, ¡Virgen y Esposa!
	
	(\emph{Akathistos}, 15)\cite{Akathistos_015}
\end{patercite}


\newsection


\section{Homilías}

\homiliasABC

\subsection{San Pablo VI, papa}

\subsubsection{Homilía (1978)}
\src{Basílica de Santa María la Mayor. XI Jornada Mundial de la Paz. \\Domingo 1 de enero de 1978.}

\begin{body}
	\ltr{C}{onvocados} por la fe en esta basílica ---erigida por nuestro predecesor Sixto III pocos años después del Concilio de Efeso que había proclamado solemnemente en el año 431 a María la \emph{Theotokos}, es decir, Madre de Dios---, unamos en esta celebración la alabanza por los altísimos privilegios concedidos por Dios a la Virgen Madre, juntamente con la reflexión sobre las exigencias cristianas de la paz en el mundo.
	
	En este espléndido templo, expresión singular de la ferviente devoción mariana del pueblo romano, historia y arte se han fundido admirablemente a través de los siglos; este templo, con su belleza clásica y su atractivo misterioso, nos lleva a pensamientos de alegría serena; en los mosaicos, tan antiguos, refulgen las diversas etapas de la historia de la salvación; en lo alto del ábside resplandece la escena sublime de la \textquote{Coronación de María}, obra de Jacopo Torriti; y junto a los recuerdos de la gruta del Pesebre, los Magos adoran al Verbo encarnado, en la composición escultórica de Arnolfo di Cambio.
	
	Hemos querido celebrar la \textquote{Jornada de la Paz} precisamente en este marco estupendo, creado por la piedad de nuestros antepasados; y desde aquí nos proponemos dirigir una vez más a toda la humanidad las palabras suaves y solemnes de la paz.
	
	La Jornada de la Paz no hace referencia a la paz de un día, de un día solo. Al celebrarla en la primera jornada del año civil, aporta siempre algo al año que comienza: una celebración conjunta que es augurio y promesa al comienzo del calendario; pero presenta también un tema que hemos propuesto nosotros y que resulta ocasión y fuente de convergencia de intenciones con dimensiones universales. Convergencia en la oración, para todos los católicos y para todos los cristianos que quieran unirse a la Jornada; convergencia en el estudio y la reflexión, para los responsables de la guía colectiva de la sociedad y para todos los hombres de buena voluntad; convergencia en una acción conjunta, un testimonio presentado así al mundo a través del esfuerzo solidario para defender a todos los habitantes de nuestro planeta, tan gravemente amenazados en nuestros días por \textquote{el carácter absurdo de la guerra moderna}, según hemos subrayado en nuestro reciente Mensaje, y para construir la paz cuya necesidad perentoria la conciencia de la humanidad siente cada vez más.
	
	Cada uno de los temas de las diferentes \textquote{Jornadas de la Paz} completa a los precedentes, al igual que una piedra se añade a las otras para construir una casa: esta casa de la paz, que se funda ---como decía nuestro venerado predecesor Juan XXIII--- sobre cuatro pilares, \textquote{la verdad, la justicia, la solidaridad operante y la libertad} (cf. \emph{Pacem in terris}, 47).
	
	Pero el pensamiento dominante de esta celebración nuestra se presenta espontáneamente en el binomio \textquote{María y la paz}. ¿Acaso no hay relación entre la Maternidad divina de María y la paz que celebramos el mismo día de su fiesta, una relación que no es accidental, sino que extrae su realidad y fruto de todo el patrimonio dogmático, patrístico, teológico y místico de la Iglesia de Cristo? ¿No es verdad que existe una razón histórica que se añade a éstas y nos reúne hoy con vosotros, carísimos hijas e hijos, romanos de nacimiento o de adopción? En efecto, ¿no venís esta mañana a continuar y confirmar con vuestra presencia la práctica profundamente religiosa y filial de vuestros antepasados, diocesanos de esta Iglesia de Roma, que antes de que esa fecha señalase en Occidente el comienzo del año civil eligió ya la octava de Navidad para rendir homenaje especial a la Madre de Dios? Y en torno a vosotros, ¿no está reunida místicamente toda la Iglesia, todo el Pueblo de Dios en esta Patriarcal Basílica, para celebrar a un tiempo la Maternidad de María y la paz, esa paz que su Hijo Jesucristo vino a traer al mundo?
	
	Pero no es preciso ir muy lejos en nuestra reflexión. Si hay correlación entre la maternidad divina de María y la paz, ¿qué relación hay entre esta maternidad y la repulsa de la violencia que figura en el tema elegido para la jornada de este año de 1978? Sí, existe relación. Los estudios teológicos y exegéticos acerca de este argumento se multiplican, lo subrayan cada vez más en la perspectiva que les es propia, y añaden a sus conclusiones la opinión espontánea del pueblo.
	
	Sea que se contemple la violencia ---como lo hemos hecho en nuestro reciente Mensaje para esta Jornada--- bajo el aspecto colectivo internacional, es decir, bajo el de la guerra moderna que amenaza con su \textquote{suprema irracionalidad}, con su \textquote{carácter absurdo}, y con las tristes hipótesis de una guerra espacial; sea que se la considere bajo los aspectos múltiples de la violencia pasional de la delincuencia creciente, o de la violencia civil erigida en sistema, se plantea una pregunta fundamental: ¿cuáles son las causas de tales comportamientos o de las ideas y sentimientos que los inspiran? Repetidas veces hemos recordado estas causas en nuestros Mensajes precedentes, particularmente en los que tratan sobre el desarme y la defensa de la vida. Esta mañana sólo recordaremos una: la sacudida provocada en la sociedad por condiciones de vida deshumanizadoras (cf. \emph{Gaudium et spes}, 27).
	
	Tales condiciones de vida provocan, sobre todo en los jóvenes, frustraciones que desencadenan reacciones de violencia y agresividad contra ciertas estructuras y coyunturas de la sociedad contemporánea que quisiera reducir a los jóvenes a simples instrumentos pasivos. Pero su contestación, instintiva u organizada, se dirige no sólo a las consecuencias de estas situaciones penosas, sino también a \textquote{una sociedad rebosante de bienestar material, satisfecha y gozosa, pero privada de ideales superiores que dan sentido y valor a la vida} (Mensaje de Navidad; \emph{L'Osservatore Romano,} Edición en Lengua Española, 5 de enero de 1969, pág. 2). En una palabra, una sociedad desacralizada; una sociedad sin alma, una sociedad sin amor.
	
	De hecho, ¿quiénes son frecuentemente estos violentos cuyas acciones, provocando temor u horror, hacen necesario el deber de proteger nuestra convivencia humana? Muy a menudo, con demasiada frecuencia, los que llevan a cabo esas acciones intolerables son personas olvidadas, marginadas, despreciadas, personas que no son amadas o, al menos, no se sienten amadas. Ávidas más de tener que de ser; testigos y con frecuencia víctimas de la injusticia de los más fuertes o, en algunos casos bien conocidos, de la \textquote{violencia estructural de algunos regímenes políticos}; ¿cómo pueden no sentirse \textquote{hijos pródigos} en esta sociedad anónima que los ha engendrado y, luego, con frecuencia abandonado, sin baremo fijo de valores, y en resumen, sin brújula ni estrella, sin la estrella de Navidad?
	
	En el secreto de su corazón estos \textquote{huérfanos}, ¿acaso no aspiran desde los fondos de esta sociedad madrastra a una sociedad materna y, en fin, a la maternidad religiosa de la Madre universal, a la maternidad de María?
	
	Las palabras de Cristo en la cruz, \textquote{Mujer, he ahí a tu hijo} (\emph{Jn} 19, 26-27), ¿no es verdad que iban dirigidas a ellos a través de San Juan: \textquote{Madre, he ahí a tus hijos}? ¿Y acaso no era para ellos la frase del Señor moribundo cuando decía: \textquote{Hijos, he ahí a vuestra madre}, una madre que os ama, una madre a la que amar, una madre situada en el vértice de una sociedad del amor? Es decir, Madre de Dios y del Redentor (\emph{Lumen gentium}, 53), del nuevo Adán en el que y por el que todos los hombres son hermanos (cf. \emph{Rom} 8, 29), María, nueva Eva (cf. \emph{Lumen gentium}, 63), se transforma de este modo en la madre de todos los vivientes (cf. \emph{ib.}, 56), nuestra madre amantísima (\emph{ib}., 53). Miembro eminente y plenamente singular de la Iglesia (\emph{ib.}, 53), \emph{es tipo de la} misma (\emph{ib}., 63); es imagen y principio de la Iglesia que habrá de tener su cumplimiento en la vida futura (\emph{ib}., 68).
	
	Aquí se nos presenta una nueva visión que es el reflejo de la Virgen en la Iglesia, como dice San Agustín: María \emph{\textquote{figuram in se sasnctae Ecclesiae demonstrat:} María refleja en sí la figura misma de la Iglesia} \emph{(De Symbolo}, CI; \emph{PL} 661; H. de Lubac, \emph{Méditations sur l'Eglise,} pág. 245).
	
	Madre de Cristo Rey, Príncipe de la Paz (\emph{Is} 9, 6). María se transforma por esto mismo en Reina y Madre de la paz. El Concilio Vaticano II, al enumerar los títulos de María, jamás la separa de la Iglesia.
	
	Así la Iglesia, toda la Iglesia, a ejemplo de María debe vivir también ella cada vez con mayor intensidad la propia maternidad universa (cf. \emph{Lumen gentium}, 64), respecto de toda la familia humana actualmente deshumanizada porque está desacralizada.
	
	\textquote{Madre y Maestra}, la Iglesia de Cristo no pretende construir la paz del mundo sin El o suplantándolo; sino que proclamando el reino de Dios en todas las naciones se propone al mismo tiempo \textquote{descubrir al hombre el sentido de la propia existencia}, sabiendo que \textquote{el que sigue a Cristo, Hombre perfecto, se perfecciona cada vez más en su propia dignidad de hombre} (\emph{Gaudium et spes}, 41).
	
	Y volviendo con el pensamiento a María Reina de la Paz, nos complacemos en recordar que nuestro venerado predecesor el Papa Benedicto XV quiso exaltar este título debido a la Virgen María haciendo esculpir un monumento en su honor en esta misma basílica, al finalizar la primera guerra mundial.
	
	Nadie piense que la paz, de la que María es portadora, se pueda confundir con la debilidad o la insensibilidad de los tímidos o de los viles; recordemos el himno más bello de la liturgia mariana, el \emph{Magnificat,} en el que la voz sonora y valiente de María resuena para dar fortaleza y valor a los promotores de la paz: \textquote{Desplegó el poder de su brazo y dispersó a los que se engríen con los pensamientos de su corazón. Derribó a los potentados de sus tronos y ensalzó a los humildes} (\emph{Lc} 51-52).
	
	Nos proponemos confiar a María la causa de la paz en todo el mundo, y en particular en la querida nación del Líbano, ejemplo de país arrollado por la espiral de la violencia no tanto por causas suyas internas cuanto por el reflejo de las situaciones de esa región que no han encontrado todavía soluciones justas; es decir que, en realidad, ha sido víctima de dichas situaciones. En esta Jornada de la Paz. exhortamos, pues, a los aquí presentes y a todos los fieles a orar por el Líbano a la Virgen \emph{Notre Dame du Liban,} para que acelere la reconciliación de sus hijos \emph{} y el resurgir espiritual y moral, además del material, de la nación.
	
	Que en las esperanzas de paz que comienzan a vislumbrarse en Oriente Medio, la reconciliación de los distintos grupos libaneses y la convivencia serena de la población lleguen a ser factor de reconciliación y de repulsa de la violencia para todos los pueblos de la región.
	
	Al concluir estas reflexiones nuestras, queremos dirigir una llamada apremiante a todos nuestros hijos y a cada uno particularmente.
	
	Procure cada cual aportar su contribución práctica, generosa y auténtica a la paz del mundo, eliminando del corazón en primer lugar toda forma de violencia, todo sentimiento de avasallamiento del hermano. Actuando así os encontraréis ya en el sendero de la paz universal que se funda en la paz efectiva de cada uno.
	
	Si queréis conseguir que la paz reine en todo el mundo, hacedla reinar primero en vuestro corazón, en vuestra familia, en vuestra casa, en vuestro barrio, en vuestra ciudad, en vuestra región, en vuestra nación:
	
	De este modo los demás sentirán incluso el encanto y el gozo de poder vivir en serenidad y de esforzarse para que este inmenso bien sea aspiración, exigencia y patrimonio de todos.
	
	Esto lo queremos decir en particular a vosotros, jóvenes, y a vosotros, muchachos, presentes hoy muy numerosos en esta basílica.
	
	Hemos querido terminar nuestro reciente Mensaje para la jornada de la Paz dirigiéndonos en especial a los jóvenes y a los muchachos de todo el mundo. porque vosotros tenéis esa extraordinaria capacidad de apertura y esa gozosa disponibilidad que por desgracia a veces los adultos han olvidado o perdido.
	
	También vosotros, jóvenes y muchachos, tenéis una palabra que decir y hacer oír a los mayores, una palabra juvenil, nueva, original.
	
	Comunicad esta palabra de paz, este \textquote{no a la violencia} con energía, con fuerza, con la fuerza de vuestro corazón puro, de vuestros oros límpidos, de vuestra alegría de vivir, pero de vivir en un mundo en el que \textquote{se darán el abrazo la justicia y la paz} (\emph{Sal} 84, 11).
	
	En vuestros ideales y en vuestro comportamiento dad siempre la prioridad al amor, es decir, a la comprensión, a la benevolencia, a la solidaridad con los otros.
	
	Reforzad vuestra convicción de paz en la oración personal y comunitaria: en el dialogo y la meditación en los que os esforzáis por conocer cada vez más profundamente a Cristo y por comprender su mensaje con todas sus exigencias: en los sacramentos, y sobre todo en el sacramento de la Eucaristía, en el que el mismo Cristo os da la fe, la esperanza y, ante todo, la caridad; en fin, reforzadla en la devoción filial a la Virgen María.
	
	Si vuestra convicción es sólida y firme. en todas las manifestaciones de vuestra juventud seréis testimonios de la paz y el amor de Cristo que está: en vosotros.
	
	Jóvenes y muchachos, lleváis en vosotros el porvenir del mundo y de la historia. Este mundo será mejor, más fraterno, más justo, si ya desde ahora toda vuestra vida está abierta a la gracia de Cristo, a los ideales de amor y de paz que os enseña el Evangelio.
	
	María, Reina de la Paz, \emph{Salus Populi Romani,} interceda por estas intenciones.
\end{body}

\newsection

\subsection{San Juan Pablo II, papa}

\subsubsection{Homilía (1981)}

\src{Basílica de San Pedro. XIV Jornada Mundial de la Paz. \\Jueves 1 de enero de 1981.}

\begin{body}
	1. \textquote{\ldots{}al llegar la plenitud de los tiempos, envió Dios a su Hijo, nacido de mujer\ldots{}}. Son palabras de San Pablo, tomadas de la liturgia de hoy \emph{(Gál} 4, 4).
	
	\textquote{\emph{La plenitud de los tiempos}\ldots{}}.
	
	\ltr{E}{stas} palabras tienen hoy una particular elocuencia, puesto que nos es dado pronunciar por primera vez la nueva fecha, esto es, el nombre del nuevo año solar: 1981. Así sucede cada año el primer día de enero. Pasan los años, cambian las fechas, transcurre el tiempo. Con el tiempo pasa también toda la naturaleza, naciendo, desarrollándose, muriendo. Y pasa también \emph{el hombre}; pero él pasa conscientemente. Tiene \emph{la conciencia de su pasar, la conciencia del tiempo}. Con el metro del tiempo mide la historia del mundo y, sobre todo, la propia historia. No sólo los años, los decenios, los siglos, los milenios, sino también los días, las horas, los minutos, los segundos.
	
	La liturgia de hoy nos dice con las palabras de San Pablo que el tiempo, que es el metro del pasar de los seres humanos en el mundo, está sometido también a otra medida, es decir, \emph{a la medida de la plenitud, que proviene de Dios:} la plenitud del tiempo. Efectivamente, en el tiempo ---en el tiempo humano, terreno---, Dios lleva a cumplimiento su proyecto eterno de amor. Mediante el amor de Dios, \emph{el tiempo} está sometido a la \emph{eternidad} y al \emph{Verbo}.
	
	El Verbo se hizo carne\ldots{} en el tiempo.
	
	Los años que pasan, que terminan el 31 de diciembre y comienzan de nuevo el 1 de enero, pasan en realidad confrontándose con esa plenitud, que proviene de Dios. Pasan frente a la eternidad y al Verbo. Cada año del calendario humano lleva, juntamente con el tiempo, una pequeña parte del \textquote{Kairós} divino. Cada uno comienza, dura y transcurre en relación \emph{a esa plenitud del tiempo que viene de Dios}.
	
	Es preciso darse cuenta de esto, de modo particular hoy, que es el primer día del año nuevo.
	
	2. Qué fuerte y espléndidamente se capta esta realidad, cuando nos damos cuenta de que este primer día del año nuevo \emph{es,} al mismo tiempo, \emph{el día de la octava de Navidad}. El año nuevo nace en el esplendor del misterio en el que se ha revelado la \textquote{plenitud del tiempo}.
	
	\textquote{Dios envió a su Hijo, nacido de mujer}.
	
	Y precisamente hacia esa Mujer, hacia la Madre del Hijo de Dios, hacia la Theotokos se dirigen hoy, al comienzo del año nuevo, de modo especial, el pensamiento y el corazón de la Iglesia. María está presente durante toda la octava; sin embargo, la Iglesia desea venerarla particularmente hoy, con un día dedicado totalmente a Ella: \emph{la solemnidad de Santa María, Madre de Dios}.
	
	A ella, pues, a la Maternidad admirable de la Virgen de Nazaret, ligada a la \textquote{plenitud de los tiempos}, nos dirigimos mediante este comienzo del año, que coincide con el día de hoy.
	
	Y recordamos que es el comienzo del año del Señor 1981, durante el cual resonarán con eco lejano en los siglos las fechas conmemorativas de los dos importantes Concilios de los primeros tiempos de la Iglesia, que permaneció una y única, a pesar de las primeras grandes herejías que surgieron. Efectivamente, en el año 381 tuvo lugar el \emph{primer Concilio de Constantinopla,} que, después del Concilio de Nicea, fue el segundo Concilio Ecuménico de la Iglesia y al cual debemos el \textquote{Credo} que se recita constantemente en la liturgia. Una herencia particular de aquel Concilio es la doctrina \emph{sobre el Espíritu Santo} proclamada así en la liturgia latina: Credo in Spiritum Sanctum Dominum et vivificantem, qui ex Patre Filioque procedit ----(la formulación de la teología oriental, en cambio, dice: qui a Patre per Filium procedit)---. Qui cum Patre et Filio simul adoratur et conglorificatur qui locutus est per prophetas.
	
	Y, luego, \emph{el año 431} (hace 1550 años), se celebró el \emph{Concilio de Efeso,} que confirmó, con inmensa alegría de los participantes, la fe de la Iglesia en la Maternidad Divina de María. Aquel que \textquote{nació de María Virgen}, como hombre, es, al mismo tiempo, el verdadero Hijo de Dios, \textquote{de la misma naturaleza que el Padre}. Aquella, de la cual \textquote{fue concebido por obra del Espíritu Santo} y que lo trajo al mundo la noche de Belén, es verdadera Madre de Dios: \emph{Theotokos}.
	
	Basta recitar con atención las palabras de nuestro Credo, para darse cuenta de cuán profundamente estos dos Concilios, que recordaremos en el curso del año 1981, están orgánicamente ligados el uno al otro \emph{con la profundidad del misterio divino y humano}. Sobre este misterio se construye la fe de la Iglesia.
	
	3. El primer día del año deseamos leer de nuevo en la profundidad de ese misterio el mensaje de la paz, que, de una vez para siempre, se reveló en la noche de Belén: \emph{¡Paz a los hombres de buena voluntad! ¡Paz en la tierra!,} he aquí lo que el misterio del nacimiento de Dios quiere decirnos cada año y lo que la Iglesia pone de relieve también hoy, primer día del año nuevo.
	
	\textquote{Dios envió a su Hijo, nacido de mujer\ldots{}} para que nosotros podamos recibir la filiación adoptiva.
	
	\textquote{Como sois hijos, Dios envió a vuestros corazones al Espíritu de su Hijo, que clama: ¡Abbá! (Padre). Así que ya no eres esclavo, sino hijo\ldots{}} (\emph{Gál} 4, 6-7).
	
	Toda la humanidad desea ardientemente la paz y ve la guerra como el peligro más grande en su existencia terrena. La Iglesia se halla totalmente presente en estos deseos y, al mismo tiempo, en los miedos y en las preocupaciones que agobian a todos los hombres, manifestando estos sentimientos, de modo particular, el primer día del año nuevo.
	
	¿Qué es la paz? ¿Qué puede ser la paz en la tierra, la paz \emph{entre los hombres y los pueblos,} sino \emph{el fruto de la fraternidad,} que se manifieste más fuerte que lo que divide y contrapone recíprocamente a los hombres? De esta fraternidad habla precisamente San Pablo, cuando escribe a los Gálatas: \textquote{Vosotros sois hijos}. Y \emph{si hijos} ---los hijos de Dios en Cristo--- entonces, también \emph{hermanos}.
	
	Y a continuación escribe: \textquote{Así que ya no eres esclavo, sino hijo}. En este contexto se inserta el tema del mensaje elegido para la Jornada de la Paz del primero de enero de 1981. Dice: \emph{\textquote{Para servir a la paz, respeta la libertad}}.
	
	4. ¡Sí! Debemos apelar a la fraternidad, queridos hermanos y hermanas, si queremos superar los monstruosos \emph{mecanismos} que, en la vida y en el desarrollo de las potencias del mundo contemporáneo, \emph{trabajan en favor de la guerra}.
	
	Es necesario que nosotros consideremos a la humanidad como una única gran familia, en la cual todas las clases de personas deben ser reconocidas y acogidas como hermanos. En los umbrales de un nuevo año, dirijamos de modo especial maestro pensamiento y nuestra solicitud a aquellos, entre estos hermanos, que se hallan en particulares situaciones de necesidad y esperan que los ingentes recursos, destinados a construir instrumentos de recíproca destrucción, sean empleados, en cambio, para las urgentes obras de socorro y de mejoramiento de las condiciones de vida.
	
	5. Como es sabido, el 1981 ha sido proclamado por la ONU \textquote{Año Internacional de los minusválidos}. Se trata de millones de personas que tienen enfermedades congénitas, enfermedades crónicas, o que están afectadas por varias formas de deficiencia mental o debilidades sensoriales; estas personas en el curso del año interpelarán de manera más aguda a nuestra conciencia humana y cristiana. Según recientes estadísticas, su número asciende a más de 400 millones. También ellos son hermanos nuestros. Es necesario que su dignidad humana y sus derechos inalienables reciban pleno y efectivo reconocimiento durante todo el arco de su existencia.
	
	En el pasado noviembre, durante la reunión de un grupo de trabajo, la Pontificia Academia de las Ciencias, en su constante obra al servicio de la humanidad mediante la investigación científica, ha profundizado en el estudio de una clase especial de minusválidos, los mentales. La debilidad mental, que afecta a casi al tres por ciento de la población mundial, debe ser tenida en consideración especial, porque constituye el más grave obstáculo para la realización del hombre. La relación del mencionado grupo de trabajo ha puesto de relieve la posibilidad de cuidados preventivos de las causas de la debilidad mental, mediante oportunas terapias. La ciencia y la medicina ofrecen, pues, un mensaje de esperanza y, al mismo tiempo, de empeño para toda la humanidad. Si sólo una parte del \textquote{budget} para la carrera de armamentos fuese destinado a este objetivo, se podrían conseguir éxitos importantes y aliviar la suerte de numerosas personas que sufren.
	
	Al comienzo de este año deseo confiar todas las personas minusválidas a la materna protección de María. En la Pascua de 1971, 4.000 minusválidos mentales, divididos en pequeños grupos acompañados por familiares y educadores, fueron en peregrinación a Lourdes y vivieron días de paz y de serenidad juntamente con los otros peregrinos. Deseo de corazón que, bajo la mirada materna de María, se multipliquen las experiencias de solidaridad humana y cristiana, en una fraternidad renovada que una a los débiles y a los fuertes en el camino común de la vocación divina de la persona humana.
	
	6. Al pensar, en los umbrales de este nuevo año, sobre las más graves necesidades de la humanidad, quisiera llamar también la atención sobre esa parte de la familia humana que se encuentra en extrema necesidad a causa de la situación alimentaria. El hambre y la mal nutrición constituyen hoy, efectivamente, un problema dramático de supervivencia para millones de seres humanos, especialmente de niños en amplias zonas de nuestro globo. Mi pensamiento se dirige particularmente a algunas extensas regiones de África afectadas por la sequía, como el Sahel, y de Asia, damnificadas por calamidades naturales o que deben afrontar una considerable afluencia de refugiados.
	
	Según una relación de la FAO, al menos 26 países africanos han tenido últimamente cosechas inferiores a las del pasado. En algunas partes de ese continente persiste el hambre y se registran carestías periódicas, que causan no pocas víctimas. Según los cálculos de expertos, las reservas mundiales de cereales disminuirán por tercer año consecutivo si continua la tendencia actual. Hago votos de corazón a fin de que todos los responsables, todas las organizaciones y todos los hombres de buena voluntad den su aportación para la realización de medidas que permitan un socorro más efectivo a los hermanos que se encuentran en la indigencia y, a la vez, se cree un sistema más eficaz de seguridad alimentaria. La palabra de Cristo \textquote{Tuve hambre y me disteis de comer}, es una llamada, urgente y particularmente actual, a nuestras responsabilidades.
	
	Son penetrantes las palabras de San Pablo de la liturgia de hoy. Es necesario qué la vida de la gran familia humana en todo el mundo se transforme bajo el signo de la fraternidad universal de los hombres. Efectivamente, somos hijos de Dios: Dios ha enviado a nuestros corazones el Espíritu de su Hijo que clama: Abbá, Padre. Por lo tanto: ¡ninguno es esclavo, sino hijo!
	
	7. Durante el año que acaba de terminar se ha recordado de modo particular la \emph{figura de San Benito,} como Patrono de Europa, en relación con el 1.500 aniversario de su nacimiento.
	
	Al meditar sobre el desarrollo de los acontecimientos más antiguos y sobre los contemporáneos, ha parecido justo proclamar Copatronos de Europa, al final del año, \emph{a los Santos Cirilo y Metodio}, que representan otro gran componente en la misión cristiana y en la obra de la economía de la salvación en nuestro continente. Es la parte ligada a la heredad de la Grecia antigua y del Patriarcado de Constantinopla, desde donde estos dos hermanos fueron enviados en misión a los pueblos de Europa Meridional y Oriental, precisamente a los eslavos. En efecto, \emph{Europa se hizo cristiana} bajo la acción de estos dos componentes.
	
	Nos ha parecido, pues, que, particularmente al final del año en el que se ha entablado el \emph{diálogo} teológico definitivo \emph{entre la Iglesia católica} y \emph{toda la ortodoxia,} tiene una gran elocuencia el haber puesto de relieve la misión de los Santos Cirilo y Metodio. Es la elocuencia de la reconciliación y de la paz, que en todos los caminos de la humanidad debe demostrarse más potente que las fuerzas de la división y de la amenaza recíproca.
	
	8. Termino citando una vez más las palabras de la liturgia de hoy:
	
	\textquote{El Señor tenga piedad y nos bendiga, ilumine su rostro sobre nosotros: \emph{conozca} la tierra \emph{tus caminos,} todos los pueblos tu salvación. El Señor tenga piedad y nos bendiga} (Salmo responsorial).
\end{body}

\subsubsection{Homilía (1984):} 

\src{Basílica de San Pedro. XVII Jornada Mundial de la Paz. \\Domingo 1 de enero de 1984.}

\begin{body}
	\ltr[1. ]{H}{e} aquí, estamos en el umbral del nuevo año {[}1984{]}, y clamamos: \textquote{\emph{Dios tenga misericordia de nosotros y nos bendiga}} (\emph{Sal} 67, 2).
	
	Así clama toda la Iglesia en la liturgia del primer día del nuevo año, que es al mismo tiempo el día de la Octava de Navidad.
	
	\emph{A través del misterio del nacimiento} de Dios en el tiempo, a través de los acontecimientos de Belén, nos separamos del año \textquote{viejo} y entramos en el año \textquote{nuevo}. La Octava de Navidad une, por así decirlo, \emph{estas dos orillas del tiempo} humano y la existencia humana en la tierra. De esta manera la Iglesia quiere resaltar el hecho de que nuestra existencia en la tierra, en el mundo visible, está conectada \emph{con el Dios invisible} y que en él \textquote{vivimos, nos movemos y existimos} (\emph{Hch} 17, 28).
	
	Más aún: Dios entró en nuestro tiempo humano, porque, hijo de la misma sustancia que el Padre, se hizo hombre por obra del Espíritu Santo y nació en la noche de Belén de la Virgen María. Desde ese momento \emph{nuestro tiempo humano} se ha convertido en su tiempo; por lo tanto, se llena no sólo por la historia del hombre y de la humanidad, sino que es \emph{llenado} también \emph{por el misterio salvífico de la 	Redención}, que opera precisamente en esta historia de la humanidad.
	
	2. Hoy, en el último día dentro de la Octava de Navidad, la atención de la Iglesia, llena de la más alta veneración y amor, se concentra en María, \emph{Madre de Dios (\textquote{Theolokos}}), es decir, de aquella que ha dado al hijo de Dios naturaleza humana y vida humana.
	
	Es la solemnidad de María santísima, Madre de Dios, gracias a ella pronunciamos hoy \emph{el nombre de Jesús}, porque ese día se le dio ese nombre al hijo de María.
	
	\emph{También por ella y junto a ella clamamos en nombre de su Hijo} al comienzo del nuevo año: \textquote{¡Dios tenga misericordia de nosotros y nos bendiga!}. Deseamos con este grito, en unión con la Madre de Dios, \emph{implorar todo el bien} para la gran familia humana, y prevenir el mal, todo mal. Clamemos, por tanto, en el nombre de Jesús, que significa \textquote{Salvador}, y clamemos en unión con la Madre, a quien la Tradición de la Iglesia llama \textquote{la Omnipotencia implorante} (\textquote{\emph{Omnipotentia 	Supplex}}).
	
	{[} \ldots{}{]}
	
	3. La \emph{maternidad} siempre se explica en relación \emph{con la 	paternidad.}
	
	Los padres, padre y madre, inician una nueva vida humana en la tierra, colaborando con el poder creativo de Dios mismo.
	
	\emph{La maternidad} de María es \emph{virginal}. Por obra del Espíritu Santo concibió y dio al Hijo de Dios al mundo, \textquote{sin conocer varón}.
	
	San Pablo explica este misterio de la maternidad divina de María refiriéndose a la \emph{paternidad eterna de Dios:} \textquote{Cuando llegó la plenitud de los tiempos, Dios envió a su hijo, nacido de mujer} (\emph{Gal} 4, 4).
	
	La maternidad virginal de la Madre de Dios equivale a la paternidad eterna de Dios, se encuentra, \emph{en cierto sentido}, en el camino de la \emph{misión del Hijo}, que del Padre llega a la humanidad a través de la Madre. Así, la maternidad de María \emph{abre el camino de Dios a 	la humanidad}. Es, en cierto sentido, el punto culminante de este camino.
	
	Sabemos que el camino de esta misión, una vez abierto en la historia de la humanidad, siempre permanece. Siempre permite, a través de la historia de la humanidad, la misión salvífica del Hijo de Dios: \emph{la 	misión}, que se consuma con la cruz y la resurrección. Y junto con la misión del Hijo,permanece en la historia de la humanidad \emph{la 	maternidad salvífica} de su Madre terrena: María de Nazaret.
	
	Veneramos esta maternidad el primer día del nuevo año. En efecto, deseamos que en esta nueva etapa del tiempo humano, María abra el camino de la humanidad a Cristo, tal como lo abrió en la noche del nacimiento de Dios.
	
	4. El misterio de la solemnidad de hoy lleva consigo el siguiente \emph{llamado} a todos los hombres: Mirad, en Jesucristo \emph{todos 	hemos recibido al Padre.}
	
	Cristo en su nacimiento terrenal nos trajo la misma paternidad divina: la dirigió a todos los hombres y la dio a todos como un don indispensable.
	
	De esta paternidad de Dios hacia todos nosotros, \emph{da un testimonio} particularmente elocuente \emph{la maternidad de la Virgen Madre de 	Dios.}
	
	La paternidad de Dios nos dice a todos los hombres \emph{que somos 	hermanos,} y la maternidad de María para toda la humanidad añade a esto un rasgo particular de familiaridad.
	
	Tenemos derecho a pensar y \emph{hablar} de nosotros mismos considerándonos como \textquote{la familia humana}. Todos somos hermanos y hermanas en esta familia.
	
	¿No dice el Apóstol todo esto claramente en la liturgia de hoy?
	
	- \textquote{Dios envió a su hijo, nacido de mujer \ldots{}, para que seamos \emph{adoptados como hijos}} (\emph{Gal} 4, 4-5);
	
	- \textquote{La prueba de que sois hijos es que Dios ha enviado a nuestros corazones el Espíritu de su Hijo que clama: Abba, Padre} (\emph{Gal} 4, 6);
	
	- \textquote{Así que ya no eres esclavo, sino hijo; y si eres hijo, también eres heredero por voluntad de Dios} (\emph{Gal} 4, 7).
	
	Esta filiación adoptiva de Dios es la gran herencia que nos dejó el nacimiento de Dios. \emph{Es la realidad de la Gracia de la Redención.} Al mismo tiempo, es un punto de referencia fundamental y central para toda la humanidad, para todos los hombres, si es cierto que debemos pensar y hablar \emph{de la fraternidad universal de hombres y pueblos.}
	
	5. \emph{¿Y cuál es la realidad} que nos encontramos en nuestro gran planeta el 1 de enero del año {[}1984{]}? ¿No está acaso \emph{en 	profundo contraste} con la verdad sobre la hermandad universal de hombres y pueblos?
	
	El mundo de hoy está cada vez más marcado por contrastes, entrampado por tensiones, que se manifiestan de manera lacerante y en direcciones cruzadas en las relaciones entre Oriente y Occidente y entre Norte y Sur.
	
	Las relaciones entre Oriente y Occidente han llegado a posiciones radicalmente opuestas, con la interrupción ---que todos esperamos sea temporal y lo más breve posible--- de las negociaciones sobre las reducciones de armamentos nucleares y convencionales. Mientras tanto, la desconfianza mutua multiplica los efectos nocivos de las luchas ideológicas y exacerba los ya graves conflictos locales, de los que diariamente se ensangrentan diversas naciones, algunas de ellas muy pequeñas.
	
	En la otra dirección entre el Norte y el Sur, la brecha que separa a los países ricos de los países pobres, ya grave desde hace muchos años, se ha ensanchado aún más con la reciente crisis económica. Según los expertos, una desaceleración del uno por ciento en la expansión económica de las naciones más industrializadas correspondería a un empobrecimiento de al menos un uno y medio por ciento en los países en desarrollo. El endeudamiento de estos, que ha alcanzado dimensiones catastróficas, da la medida del agravamiento divergente de estos contrastes económicos. Pero el aspecto más preocupante lo representan los contrastes que de él se derivan en la situación del hombre. En los países ricos, la salud y la nutrición están mejorando, mientras que en los países pobres hay una gran falta de alimentos para la supervivencia y la mortalidad, especialmente la mortalidad infantil, es desenfrenada. Según datos de Unicef, cuarenta mil niños menores de un año morirían cada día en el Tercer Mundo, mientras que la FAO estima que cada día más de quince mil personas morirían de hambre o de mala alimentación.
	
	La amenaza de la catástrofe nuclear y la plaga del hambre aparecen escalofriantes en el horizonte como los jinetes fatales del Apocalipsis: fruto, ambos, de complejos fenómenos de orden económico, político, ideológico y moral, constituyen tantas fuentes de violencia que interactúan constantemente.
	
	6. ¿Cuáles son --- nos preguntamos --- las causas fundamentales de estos fenómenos?
	
	¿Y por qué el nivel de amenazas y plagas no disminuye, sino que aumenta?
	
	La humanidad se hace estas preguntas con mayor preocupación. Expertos de las distintas ramas del conocimiento intentan explicar los mecanismos específicos que les afectan directa o indirectamente. Sin embargo, \emph{en el fondo de las diversas causas} y complejos mecanismos que acompañan a los procesos de desarrollo y civilización contemporánea, ¿no hay acaso \emph{una causa fundamental y última?}
	
	¿Y acaso esta causa fundamental no está representada por el hecho de que se está perdiendo la conciencia de la hermandad radical de hombres y pueblos?
	
	Todos somos hermanos. Esta \emph{hermandad} está \emph{ligada a la 	filiación común.} Somos hermanos porque somos hijos. Y esta filiación está vinculada \emph{a la paternidad de Dios mismo.} Somos hijos porque tenemos un padre.
	
	Cuanto más perdemos, o intentamos eliminar, la conciencia de esta paternidad, más \emph{dejamos de ser hermanos} y, en consecuencia, la justicia, la paz y el amor social se alejan más de nosotros.
	
	7. \emph{El mensaje} de este año para la Jornada Mundial de la Paz lleva el título: \textquote{La paz nace de un corazón nuevo}.
	
	Con este mensaje la Sede Apostólica \emph{suma su palabra} a todos aquellos esfuerzos, a veces desesperados, que realizan hombres de buena voluntad en todo el mundo, así como diversos organismos nacionales e internacionales \emph{para asegurar la paz} en el mundo contemporáneo.
	
	Hoy queremos desarrollar plenamente, en cierto sentido, el contenido de este mensaje, a partir de esa luz que la Navidad aporta a la humanidad.
	
	Así, en el curso de este santo sacrificio de Jesucristo y de la Iglesia, \emph{clamamos a Dios} y, al mismo tiempo, \emph{a todos los hombres}, orando por:
	
	\emph{una eficacia renovada de la fraternidad universal} en el corazón de todos los hombres;
	
	\emph{una renovada eficacia de la presencia} del Padre en las diversas dimensiones de la vida y la convivencia.
	
	Solo en un corazón nuevo \emph{esta fuerza puede generar una paz} segura en la tierra.
	
	Con toda humildad y con fe, confiamos el bien de esta paz a la Madre de Cristo.
	
	Sí. ¡Unimos la esperanza de la paz, la justicia y el amor en la tierra con la maternidad de María, la Madre de Dios!
\end{body}

\subsubsection{Homilía (1987):}
\src{Basílica de San Pedro. XX Jornada Mundial de la Paz. \\Jueves 1 de enero de 1987.}

\begin{body}
	1. \textquote{Cuando llegó la plenitud de los tiempos, Dios envió a su Hijo \ldots{}} (\emph{Gal} 4, 4).
	
	\ltr{T}{e} saludamos, plenitud de los tiempos, que el Hijo eterno de Dios trajo y cumplió en la historia de la creación, haciéndose hombre.
	
	Te saludamos, plenitud de los tiempos, de la que hoy emerge el nuevo año, según la medida del paso humano.
	
	Te saludamos, \emph{Año del Señor {[}1987{]}}, en el umbral de tus días, semanas y meses.
	
	La Iglesia del Verbo Encarnado te saluda en medio de la gran familia de naciones y pueblos.
	
	La Iglesia os saluda pronunciándoos \emph{la bendición del Dios de la 	Alianza}:
	
	\textquote{Que el Señor te bendiga y te proteja. Que el Señor haga resplandecer su rostro sobre ti y tenga piedad de ti. Que el Señor vuelva sobre ti su rostro \emph{y te conceda la paz}} (\emph{Núm} 6, 24-26).
	
	2. \textquote{Cuando llegó la plenitud de los tiempos, \emph{Dios envió a su Hijo} \ldots{}}.
	
	Te saludamos, Año Nuevo, \emph{en el corazón mismo del misterio de la 	Encarnación}, en el que adoramos al Hijo de Dios hecho carne por nosotros.
	
	Te saludamos, Hijo de la misma sustancia que el Padre Eterno, que vino a nosotros en la plenitud de los tiempos, \textquote{\emph{para que recibamos la 	adopción de hijos}} (\emph{Gal} 4, 5).
	
	Te saludamos en tu humanidad, Hijo de Dios, nacido de mujer, así como cada uno de nosotros, hijos humanos, nacimos de mujer.
	
	\emph{Te saludamos en la humanidad de cada hombre} en toda la riqueza y variedad de tribus, naciones y razas, idiomas, culturas y religiones.
	
	\emph{En ti}, Hijo de María, en ti Hijo del hombre, \emph{somos hijos de 	Dios}.
	
	Deseamos celebrar este primer día del nuevo año, junto con la octava de Navidad, \emph{como la solemnidad universal del hombre} en la plenitud de su dignidad humana.
	
	Deseamos celebrar este día, gracias a tu obra, como \textquote{\emph{hijos en el 	Hijo}}. Viniste \textquote{para reunir a los hijos de Dios dispersos} (\emph{Jn} 11, 52). \emph{Eres nuestro hermano y nuestra paz}.
	
	3. \textquote{Cuando llegó la plenitud de los tiempos, Dios envió a nuestros corazones \emph{el Espíritu de su Hijo} que clama: ¡Abba, Padre!} (\emph{Gal} 4, 6).
	
	Fuiste tú quien gritó así. Tú, el Hijo. Lo dijiste en momentos de fervor y momentos de desnudez.
	
	Y tú, Hijo de la misma sustancia que el Padre, \emph{nos has enseñado a 	decirlo}; nos animaste a decir junto a ti: \textquote{Padre nuestro}.
	
	Y aunque no encontremos justificación en nuestra humanidad, nos has dado, en unidad con el Padre, tu Espíritu \textquote{que es Señor y dador vida} (Dominum et Vivificantem), \emph{para que podamos decir \textquote{Abba, Padre} 	con toda la verdad interior de nuestro corazón}. De hecho, el Espíritu del Hijo fue enviado a nuestros corazones. El Espíritu del Hijo \emph{nos ha formado de} nuevo, desde la raíz misma de nuestra humanidad, de nuestra naturaleza humana, como \textquote{hijos en el Hijo}.
	
	4. Somos, por tanto \emph{, hijos, no esclavos}. Somos \emph{herederos 	por voluntad de Dios}.
	
	Hoy, al comienzo del nuevo año, deseamos \emph{reconfirmar esta herencia 	universal de todos los hijos e hijas de esta tierra}.
	
	Todos están llamados a la libertad. En el contexto de los tiempos en que vivimos, la Iglesia ha confirmado una vez más la verdad sobre la \textquote{\emph{libertad y liberación cristianas}} como fundamento de la justicia y la paz (cf. Congr. Pro doctrina Fidei, \emph{Libertatis Conscientia}, 22 de Marzo de 1986).
	
	El Espíritu del Hijo que el Padre envía incesantemente a nuestros corazones clama constantemente: \textquote{Ya no eres esclavo, sino hijo; y si eres hijo, también eres heredero por voluntad de Dios} (\emph{Gal} 4, 7).
	
	5. \textquote{Cuando llegó la plenitud de los tiempos, Dios \emph{envió a su Hijo, 	nacido de mujer}}.
	
	Durante toda la octava de Navidad, y particularmente hoy, el corazón de la Iglesia late de manera singular por ella, por la \emph{Madre del Hijo 	de Dios}. Por la Madre de Dios.
	
	Hoy se celebra su principal solemnidad. Ella, la Mujer, da el primer testimonio \emph{materno} de la dignidad humana del Hijo de Dios, que nació de ella.
	
	\emph{Ella es su Madre}.
	
	Hoy la vemos en Belén dando la bienvenida a los pastores.
	
	Al octavo día después del nacimiento, cuando se completa el rito de la circuncisión del Antiguo Testamento, ella \emph{le da el nombre al 	Niño}. Y este es el nombre: Jesús, un nombre que habla de la salvación realizada por Dios. Esta salvación es traída por su Hijo. \emph{Jesús 	significa \textquote{Salvador}}. Así fue llamado el Hijo de María en el momento de la Anunciación, el día en que fue concebido en su seno. Y así ahora es llamado por ella antes que los hombres.
	
	La dignidad humana del Hijo de Dios se expresa en este nombre. Como hombre, es el Salvador del mundo. Su Madre es la \emph{Madre del 	Salvador}.
	
	6. \textquote{\emph{Alégrate}, llena eres de gracia, el Señor es contigo \ldots{}} (\emph{Lc} 1, 28).
	
	Bienaventurada tú que \emph{has creído} \ldots{} (cf. \emph{Lc} 1, 45). Creíste en el momento de la Anunciación. Creíste en la noche de Belén. Creíste en el Calvario. Has avanzado en la peregrinación de la fe y has conservado fielmente tu unión con el Hijo, Redentor del mundo (cf. \emph{Lumen gentium}, 58). Así, las generaciones del pueblo de Dios te han visto por toda la tierra. \emph{Así te ha mostrado}, Santísima Virgen, \emph{el Concilio de nuestro siglo}.
	
	La Iglesia fija sus ojos en ti como en su propio modelo. {[}Los fija en particular en este período en el que se prepara para celebrar el advenimiento del tercer milenio de la era cristiana. Para prepararse mejor para ese acontecimiento, la Iglesia vuelve sus ojos hacia ti, que fuiste el instrumento providencial que utilizó el Hijo de Dios para convertirse en Hijo del Hombre y comenzar los nuevos tiempos. Con esta intención quiere celebrar un año especial dedicado a ti, un año mariano, que, a partir del próximo Pentecostés, terminará el año siguiente con la gran fiesta de tu asunción al cielo. Un año que cada diócesis celebrará con iniciativas particulares, encaminadas a profundizar en tu misterio y fomentar la devoción a ti en un compromiso renovado de adhesión a la voluntad de Dios, siguiendo el ejemplo que ofreciste, tú, la sierva del Señor. Estas iniciativas se enmarcarán fructíferamente en el tejido del año litúrgico y en la \textquote{geografía} de los santuarios, que la piedad de los fieles te ha elevado, oh Virgen María, en todas las partes de la tierra.{]}
	
	{[}Deseamos, oh María, que brillas en el horizonte del advenimiento de nuestro tiempo, a medida que nos acercamos a la etapa del tercer milenio después de Cristo, profundizar \emph{la conciencia de tu presencia} en el misterio de Cristo y de la Iglesia, como nos enseñó el Concilio. Para ello, el actual Sucesor de Pedro, que te confía su ministerio, tiene la intención de dirigirse en un futuro próximo a sus hermanos en la fe con una \emph{encíclica}, dedicada a ti, Virgen María, don inestimable de Dios a la humanidad.{]}
	
	7. ¡Bienaventurada tú que has creído!
	
	El evangelista dice de ti: \textquote{María, por su parte, guardaba todas estas cosas y las meditaba en su corazón} (\emph{Lc} 2, 19).
	
	\emph{¡Eres la memoria de la Iglesia!}
	
	La Iglesia aprende de ti, María, que ser madre significa ser memoria viva, significa \textquote{guardar y meditar en el corazón} los acontecimientos de los hombres y de los pueblos; los acontecimientos alegres y dolorosos.
	
	{[}Entre tantos acontecimientos de 1987, deseamos recordar el 600 aniversario del \textquote{bautismo de Lituania} a la memoria de la Iglesia, acercándonos con oración a nuestros hermanos y hermanas, que durante muchos siglos han perseverado unidos a Cristo en la fe de la Iglesia.{]}
	
	Y cuántos otros acontecimientos, cuántas esperanzas, pero también cuántas amenazas, cuántas alegrías pero también cuántos sufrimientos \ldots{} ¡a veces cuántos grandes sufrimiento! Todos debemos, como Iglesia, guardar y meditar estos eventos en nuestro corazón. Así como la Madre. Tenemos que aprender cada vez más de ti, María, cómo ser Iglesia en este {[}tiempo{]}.
	
	8. \emph{En el umbral del nuevo año, el Obispo de Roma}, abrazando en este sacrificio eucarístico a todas las Iglesias del mundo, unidas en la comunión universal \emph{católica},
	
	- y a todos los amados hermanos \emph{cristianos} que buscan junto con nosotros los caminos de la unidad,
	
	- y a todos los seguidores \emph{de religiones no cristianas},
	
	- y, sin excepción, a todos los \emph{hombres de buena voluntad en} toda la tierra, grita desde la tumba de San Pedro con las palabras de la liturgia: \textquote{\emph{Que el Señor nos bendiga y nos proteja}. ¡Que el Señor haga resplandecer su rostro sobre nosotros y tenga misericordia de nosotros y nos dé la paz!} (\emph{Núm} 6, 24-26).
	
	Que {[}1987{]} sea un año en el que la humanidad finalmente deje a un lado las divisiones del pasado; un año en el que, en solidaridad y desarrollo, todo corazón busque la paz.
\end{body}

\subsubsection{Homilía (1990):}

\src{Basílica de San Pedro. XXIII Jornada Mundial de la Paz. \\Lunes 1 de enero de 1990.}

\begin{body}
	1. \textquote{María, por su parte, guardaba todas estas cosas, meditándolas en su corazón} (\emph{Lc} 2, 19).
	
	\ltr{E}{l} 1 de enero la Iglesia concluye la octava de Navidad, venerando la Maternidad de la Virgen María. Las palabras del Evangelio de Lucas ponen especial énfasis en la dimensión interior de esta Maternidad suya. Estas palabras son muy importantes para la Iglesia de hoy. Durante la octava, la Iglesia meditó sobre el misterio del nacimiento del Hijo de Dios en Belén. Hoy se refiere a la primera que meditó este misterio en su corazón. Ya que, como enseña el Concilio Vaticano II, \textquote{María va delante del pueblo de Dios en la peregrinación de la fe, y mantuvo fielmente su unión con el Hijo hasta la cruz} (cf. \emph{Lumen gentium}, 58), peregrinación que comienza en Belén.
	
	Comienza en el Corazón de la Madre y continúa allí sin detenerse. Cada madre vive de una manera particular del recuerdo de haber dado a luz a un hijo. Este nacimiento vive en ella, lo guarda en su corazón. ¿Y qué pensar, entonces, de este nacimiento único en el que el Hijo de Dios vino al mundo?
	
	La Iglesia de hoy se refiere a la dimensión interior de la maternidad, y así venera al mismo tiempo el misterio de la Encarnación y la extraordinaria dignidad de la Virgen-Madre.
	
	2. El misterio de la Encarnación es un principio nuevo en la historia de la salvación. Y también es un nuevo comienzo en la historia del hombre y la creación. El \textbf{apóstol Pablo} define este nuevo principio como \textquote{la plenitud de los tiempos}. \textquote{Cuando llegó la plenitud de los tiempos, Dios envió a su Hijo, nacido de una mujer \ldots{} para que seamos adoptados como hijos} (\emph{Gal} 4, 4-5).
	
	Lo que queda en la memoria viva de María --- y al mismo tiempo en la memoria viva de la Iglesia --- no es un evento único, un evento \textquote{cerrado}. El nacimiento de Dios está abierto al hombre de todos los tiempos. En él se revela y moldea la adopción como hijos de Dios, que se transmite a todos los seres humanos: \textquote{Y el Verbo se hizo carne y habitó entre nosotros \ldots{} A cuantos lo acogieron, él les dio poder para ser hijos de Dios} (\emph{Jn} 1, 14. 12). Las palabras del Prólogo de Juan, recordadas durante la octava de Navidad, atestiguan la duración continua del misterio, que comenzó la noche de Belén.
	
	¡Sí! El Hijo de Dios se hizo hombre solo una vez, nació solo una vez de la Virgen María y, sin embargo, la filiación divina es una herencia continua del hombre.
	
	3. El \textbf{apóstol Pablo} todavía habla de esta herencia. Es la obra incesante del Espíritu Santo: fruto de su acción en nosotros. \textquote{Y la prueba de que sois hijos es que Dios ha enviado el Espíritu de su Hijo a nuestros corazones que clama: ¡Abba, Padre! Así que ya no eres esclavo, sino hijo; y si eres hijo, también eres heredero por voluntad de Dios} (\emph{Gal} 4, 6-7).
	
	La Iglesia guarda esta herencia, es su custodio y administrador en la tierra. Por eso fija constantemente sus ojos en el misterio de la encarnación. Y desea mirarlo con los ojos de María, participar de su memoria. En ninguna otra criatura está la Navidad tan profundamente inscrita como en ella. De hecho, se identifica con su maternidad. La maternidad humana de esta \textquote{Mujer} es, al mismo tiempo, maternidad divina. Aquel que fue traído a la luz por ella es, en realidad, el Hombre-Dios.
	
	María como dice el Concilio Vaticano II \textquote{creyendo y obedeciendo, engendró en la tierra al mismo Hijo del Padre, y sin conocer varón, cubierta con la sombra del Espíritu Santo, como una nueva Eva, que presta su fe exenta de toda duda, no a la antigua serpiente, sino al mensajero de Dios, dio a luz al Hijo, a quien Dios constituyó primogénito entre muchos hermanos (cf. \emph{Rm} 8,29), esto es, los fieles, a cuya generación y educación coopera con amor materno} (\emph{Lumen gentium}, 63).
	
	4. Este día de la octava es, por tanto, la fiesta de la herencia divina, en la que participan todos los hombres. La filiación divina, como don del Espíritu Santo en el hombre, impregna toda la herencia de la humanidad, de la naturaleza humana; toda la herencia, de hecho, de la creación misma. De hecho, el hombre fue creado a imagen de Dios y fue colocado en el mundo visible en medio de todas las criaturas.
	
	Si la Iglesia celebra hoy el Día Internacional de la Paz, en la octava de Navidad, es porque hay una profunda lógica de fe en este hecho. De hecho, la paz exige una responsabilidad particular del hombre por toda la creación.
	
	El mensaje pontificio para el nuevo año enfatiza esta responsabilidad en particular: \textquote{Paz con Dios creador - Paz con toda la creación}. El mensaje del Evangelio de la paz nos recuerda constantemente, y hace referencia una y otra vez al mandamiento de no matar. ¡No mates a otro hombre, no mates desde el momento de su concepción en el vientre de su madre, no mates! No limites la existencia humana en la tierra con el método de lucha: de la violencia, el terrorismo, la guerra, los medios de exterminio masivo. No mates, porque toda vida humana es patrimonio común de todos los hombres.
	
	Y también: no mates, destruyendo tu entorno natural de diferentes formas. Este entorno también pertenece al patrimonio común de todos los hombres, no solo de las generaciones pasadas y contemporáneas, sino también de las futuras. ¡Se un defensor, no un destructor de la vida!
	
	El primer día del nuevo año pide una referencia particular a esta herencia. La herencia de los hijos adoptivos de Dios está íntimamente ligada al imperativo de la paz.
	
	5. {[}Hoy no es sólo el primer día del nuevo año 1990, sino también el de la nueva década. Esta es la última década del siglo XX y, al mismo tiempo, del segundo milenio desde el nacimiento de Cristo.{]}
	
	La Iglesia vuelve a Belén. Allí donde \textquote{fueron {[}los pastores{]} y encontraron a María, a José y al niño, que estaba acostado en el pesebre} (\emph{Lc} 2, 16). En el transcurso de los años que siguieron, la Iglesia no cesó de rezar a la Madre de Dios para que estuviera particularmente cerca de ella para recordar el misterio que ella acariciaba y meditaba en su corazón.
	
	{[}En el umbral de la última década de nuestro siglo y del segundo milenio,{]} deseamos participar de manera particular en este recogimiento materno de María sobre el misterio del Hijo nacido, crucificado y resucitado. En él se renueva constantemente la \textquote{adopción como hijos} de Dios de todos los hombres. Toda la creación lo espera como herencia terrena del hombre, llamado a la gloria eterna en Cristo.
\end{body}


\subsubsection{Homilía (1993):}
\src{Basílica de San Pedro. XXVI Jornada Mundial de la Paz. \\Viernes 1 de enero de 1993.}

\begin{body}
	1. \textquote{Le dieron por nombre Jesús} (\emph{Lc} 2, 21). 
	
	\ltr{H}{oy} es el octavo día desde el nacimiento del Hijo de María, en la noche de Belén. Hoy \textquote{Jesús recibió su nombre}, \textquote{como lo llamó el ángel antes de que fuera concebido en el vientre de su madre} (\emph{Lc} 2, 21). \textquote{Dios envió a su Hijo, nacido de mujer} (\emph{Gal} 4, 4). El Padre Eterno quiso que este nombre se le impusiera a su Hijo unigénito: Jesús, que significa: \textquote{Dios salva}. Es un nombre que se usa en Israel, y muchos antes de él lo habían llevado. Sin embargo, sólo el Padre Eterno le dio este nombre al Redentor, y María y José, en el día de la circuncisión, fueron humildes ejecutores de su voluntad. El Padre celestial quería que su Hijo, consustancial con él, Dios de Dios, como Hombre, como Hijo del hombre, llevara el nombre de Jesús. Este nombre significaría para siempre la misión que cumplió en la \textquote{plenitud de los tiempos} (cf. \emph{Gal} 4, 4). Dios envió a su Hijo al mundo \textquote{para que el mundo sea salvo por él} (cf. \emph{Jn} 3, 17). El nombre de Jesús tiene un carácter universal: es decir, expresa la voluntad salvífica de Dios con respecto al mundo, a la humanidad. \textquote{Dios \ldots{} quiere que todos los hombres se salven y lleguen al conocimiento de la verdad} (\emph{1 Tim} 2, 4).
	
	2. Salvar significa liberarse del mal, y Jesús nos ordenó orar al Padre por esto: \textquote{Líbranos del mal}. Unió así nuestra oración a su misión en el mundo; una misión que ya marca el momento de su nacimiento en Belén: \textquote{Natus est nobis Salvator mundi}. ¡Salvar! Liberarse del mal, vencer el mal, todo esto no significa más que dejar lugar para el bien. En el hombre, una vez eliminado el mal, no debe haber vacío: el mal retrocede y desaparece ante el bien. La venida al mundo del Hijo de Dios significa que Dios quiere erradicar definitivamente el mal que hay en la humanidad, introduciendo al hombre en la dimensión divina del bien, como nos enseña el Apóstol, de hecho, en la Carta a los Gálatas: \textquote{Dios envió a su Hijo, nacido de mujer, nacido bajo la ley \ldots{} para que recibiéramos la adopción como hijos \ldots{} así que ya no somos esclavos, sino hijos} (\emph{Gal} 4, 4-7). Eres hijo, es decir, un hombre que en el poder del Espíritu puede clamar al Padre: \textquote{¡Abba, Padre!} (\emph{Gal} 4, 6). El Espíritu Santo, de hecho, fue enviado como el Espíritu del Hijo. Los que se convierten en hijos adoptivos de Dios en su Hijo único, se convierten al mismo tiempo en herederos: tienen parte del Bien imperecedero, que es Dios mismo. Toda esta verdad está contenida en el nombre \textquote{Jesús}: el Salvador.
	
	3. Según las propias palabras de Jesús, la filiación divina está relacionada con la irradiación de la paz. \textquote{Bienaventurados los que trabajan por la paz, porque serán llamados hijos de Dios} (\emph{Mt} 5, 9). Hoy, primer día del nuevo año, queremos profesar y anunciar que Jesús es nuestra Paz. Jesús, cuyo nombre significa \textquote{Dios salva}. El bien de la paz está incluido en esta misión salvadora suya. El mismo Cristo, llamado Príncipe de Paz por el Profeta del Antiguo Testamento, logra la reconciliación entre Dios y la humanidad. Y esta reconciliación constituye la primera dimensión de la paz. En ella encuentra su comienzo y su raíz toda paz que los hijos adoptivos de Dios están llamados a realizar en el mundo y entre los hombres, haciéndose partícipes, incluso \textquote{co-arquitectos}, de la salvación mesiánica, anunciada en el nombre de Jesús y que él mismo ha traído al mundo: la paz en todas sus dimensiones es un bien que forma parte de la salvación. Constituye un aspecto integral del plan salvífico de Dios ofrecido a la humanidad en Cristo, su Hijo. Por eso quería que el nombre del Redentor fuera \textquote{Jesús}.
	
	4. \textquote{Paz en la tierra a los hombres en quien él se complace \ldots{}} (\emph{Lc} 2, 14), \textquote{a los hombres de buena voluntad}. El mensaje de la noche de Belén habla de una estrecha conexión entre la paz en la tierra y la misión del Salvador. ¿Podría ser de otra manera? Salvar significa liberarse del mal y lo que constituye la antítesis de la paz, ¿no contiene en sí mismo toda la evidencia del mal? {[}Nuestro siglo, el siglo XX, lamentablemente ha resaltado esta evidencia de una manera única a través de las terribles experiencias de las dos guerras mundiales, y también a través de muchos otros conflictos, que, aunque no se definieron como mundiales, fueron sin embargo hechos de guerra, con todo el drama que comportan tales conflictos. Durante la década de 1980, cuando la amenaza de la guerra nuclear se había vuelto extremadamente peligrosa, cristianos y representantes de las otras religiones del mundo se reunieron en Asís para gritar, en el mismo lugar, \textquote{líbranos del mal}, \textquote{dona nobis pacem}. ¿Es posible pensar que una oración tan confiada no ha sido escuchada por Aquel que es el Dios de la Paz? Hoy, el horror de la destrucción nuclear parece haberse alejado de la humanidad, pero el bien de la paz aún no se ha consolidado en todas partes. Los acontecimientos recientes ocurridos fuera de Europa y en la propia Europa así lo demuestran. Lamentablemente, incluso en nuestro continente, particularmente en las regiones de los Balcanes, la propagación del mal de la guerra y la violencia destructivas no está disminuyendo. ¿Puede Europa distanciarse de esta situación y no sentirse desafiada por ella? Los discípulos y confesores de Cristo, su Iglesia, no pueden dejar de pensar y trabajar en el espíritu de las ocho Bienaventuranzas: \textquote{Bienaventurados los que trabajan por la paz}. Precisamente por eso, todas las Conferencias Episcopales de Europa, junto con el Obispo de Roma, han proclamado este 1 de enero como Jornada de oración por la paz en Europa, en particular en los Balcanes. Por eso, como en 1986, volveremos a peregrinar a Asís.{]}
	
	5. ¡Europa! \textquote{Que el Señor vuelva sobre ti su rostro y te conceda la paz} (\emph{Núm} 6, 26). Así exclamamos en el nombre de Jesús, es decir, en el nombre de Aquel que es el Salvador del mundo --- del hombre --- de todos los hombres: de naciones, países y continentes. Él no tiene a su disposición los medios que pueden utilizar los estados y los poderosos de esta tierra. Su poder radica en la pobreza de la noche de Belén y luego en la Cruz del Gólgota. Sin embargo, es un poder que penetra más profundamente. De hecho, sólo este poder puede erradicar el odio, el principal enemigo de la paz, en las profundidades del ser humano. Sólo l es capaz de transformar a los trabajadores de la guerra y la destrucción en constructores de paz, a los que se puede dar el nombre de hijos de Dios.
	
	6. ¡María! Hoy la Iglesia medita sobre el misterio de tu Maternidad. Tú eres la \textquote{memoria} de todas las grandes obras de Dios, tú conoces los caminos por los que el Hijo, Verbo consustancial con el Padre, vino al hombre: ¡Cristo, Salvador del mundo! Él es nuestra Paz. María, Madre de la Paz, intercede por nosotros ante él, intercede por nosotros. ¡Amén!
\end{body}


\subsubsection{Homilía (1996):}

\src{Basílica de San Pedro. XXIX Jornada Mundial de la Paz. \\Lunes 1 de enero de 1996.}

\begin{body}
	1. \textquote{Se le puso el nombre de Jesús \ldots{}} (\emph{Lc} 2, 21).
	
	\ltr{E}{l} \textbf{Evangelio} que se acaba de proclamar recuerda que al hijo de María, nacido en Belén, al final de los ocho días prescritos, se le dio el nombre de Jesús, nombre con el que fue llamado por el ángel, antes de ser concebido en el vientre de la Madre (cf. \emph{Lc} 2, 21). Por tanto, fue \emph{el nombre que le dio el Padre celestial}.
	
	\emph{Jesús significa: \textquote{Dios salva}}. Con este nombre comenzamos el Año Nuevo: {[}1996{]} desde el nacimiento de Cristo. El hecho de que contamos los años de nuestra era desde el nacimiento de Cristo es muy elocuente. Indica que \emph{Jesús es el centro de la historia}. En Cristo, el Hijo de Dios asumió la naturaleza humana. Y es precisamente el misterio de la Encarnación lo que explica plenamente el significado del nombre de Jesús: \textquote{Él (Dios) viene a salvaros} (\emph{Is} 35, 4). El tiempo humano está completamente impregnado del misterio salvífico de Dios. \emph{La historia de la humanidad se ha convertido en la historia de la salvación}.
	
	El primer día del Año Nuevo, combinado con el recuerdo del nombre de Jesús, revela así este profundo significado. Es el día de la octava del nacimiento del Señor, en el que la Iglesia venera especialmente \emph{la maternidad divina de la Madre de Dios}. El primer día del Año Nuevo es su fiesta, la fiesta de la Madre del Dios-Hombre, de la \emph{Theotokos}.
	
	2. El pasaje de la Carta de \textbf{San Pablo a los Gálatas} propuesto por la liturgia de hoy es, en cierto sentido, \emph{el comentario sobre el nombre de Jesús}. El apóstol revela de manera lapidaria todo lo que encierra el significado de este nombre, mostrando \emph{cómo Dios salva}. Por tanto, leemos: \textquote{Dios envió a su Hijo, nacido de mujer \ldots{}, para que recibiéramos la adopción como hijos} (\emph{Gal} 4, 4-5). La salvación, por tanto, se realiza en la adopción como hijos: en Cristo, el unigénito Hijo de Dios, los hombres se han convertido en hijos adoptivos de Dios, por lo que entendemos cómo el nombre \textquote{Jesús} tiene en sí mismo \emph{un dinamismo particular}. Dios no sólo ordena que su Hijo sea llamado por el nombre de Jesús, sino que al mismo tiempo manifiesta en este nombre la profundidad y extensión del misterio de la salvación. \emph{El nombre de Jesús revela el misterio de la adopción como hijos de Dios}. San Pablo añade casi desde una perspectiva profética: \textquote{Y la prueba de que sois hijos es que Dios ha enviado a nuestros corazones el Espíritu de su Hijo que grita: \textquote{¡\emph{Abba, Padre}!}} (\emph{Gal} 4, 6). Jesús nos enseñó a volvernos a Dios diciendo: \textquote{¡Padre nuestro!}. Pero estas palabras humanas toman del Espíritu Santo, que es el Espíritu del Hijo, su propio poder. Cuando rezamos: \textquote{\emph{Abbà}, Padre nuestro}, estas palabras humanas nuestras son ante todo \emph{una forma de participar en la vida del Verbo eterno}, Hijo consustancial al Padre. A través de esta \textquote{participación}, la invocación \textquote{¡Abbà, Padre nuestro!} se convierte en expresión de salvación.
	
	Cristo es el Salvador del mundo, porque por él y en él todos los hombres pueden pronunciar esta palabra, palabra que le pertenece plenamente sólo a él, el Hijo eterno. \emph{En él, la filiación divina se ha convertido en nuestra herencia}. Por voluntad de Dios, como hijos adoptivos, somos coherederos con el Hijo eterno, llamados a participar de la vida de Dios, de la felicidad eterna en Él.
	
	3. \emph{El nombre de Jesús, \textquote{Dios salva}, atestigua que Él es nuestro Salvador}. Las lecturas de la liturgia de hoy nos presentan una vez más \emph{la dimensión universal de la salvación}, a la que están llamados todos los hombres y todos los pueblos, por el misterio de la Encarnación. El Salmo Responsorial lo resalta bien: \textquote{Alégrense y regocíjense las naciones, Él juzgará a los pueblos con justicia, gobernará las naciones de la tierra} (\emph{Sal} 66 {[}67{]}, 5). Lo que la \textbf{primera lectura} refiere a los hijos de Israel, el \textbf{Salmo} lo extiende a los pueblos y naciones de la tierra. \emph{La salvación está destinada a toda la humanidad}. Este secreto privilegio no está reservado a una persona o pueblo, sino que está destinado a todos los hombres.
	
	Es una participación que pasa por el \emph{santo temor de Dios}, principio de la sabiduría (cf. \emph{Sal} 110 {[}111{]}, 10). \emph{La venida del Redentor del mundo}, para quienes lo acogen con reverencia y agradecimiento, marca el comienzo de \emph{un nuevo orden}, el orden divino. Con el nacimiento del Hijo de Dios en la naturaleza humana se expresa la voluntad salvífica de Dios; la Divina Providencia se manifiesta y guía el destino del mundo; se anuncia \emph{la justicia definitiva de la historia}, justicia unida a la misericordia. Por eso el salmista proclama: \textquote{El Señor tenga misericordia de nosotros y nos bendiga, haga brillar su rostro sobre nosotros; para que sea conocido en la tierra tu camino, tu salvación entre todas las naciones} (\emph{Sal} 66 {[}67{]}, 2-3). El misterio de la Navidad y el nombre mismo de \textquote{Jesús} representan así para la humanidad el signo del orden divino, que contiene la historia de la creación y de cada pueblo y nación.
	
	Con razón, por tanto, la Iglesia celebra en este primer día del Año Nuevo \emph{la Jornada Mundial de la Paz} \ldots{} \textquote{Que el Señor vuelva sobre ti su rostro y te conceda la paz} (\emph{Núm} 6, 26), anuncia hoy la \textbf{primera lectura}. \emph{La paz, signo fundamental de la presencia divina, debe irradiar también en el orden político y en la vida de las comunidades y naciones}. La conocida expresión de Pablo VI: \textquote{Desarrollo es el nuevo nombre de la paz} (\emph{Populorum Progressio}, 87), podría ser invertida y formulada de la siguiente manera: la paz es el nuevo nombre del desarrollo y del orden social.
	
	La paz en el lenguaje bíblico indica participación en la salvación que viene de Dios. La paz ya está contenida en el nombre dado al Hijo de Dios ocho días después de su nacimiento. Ese nombre significa salvación de todo mal, especialmente del odio, la guerra y la destrucción. Por tanto, el apóstol Pablo dirá de Cristo: \textquote{\emph{Porque él es nuestra paz}} (\emph{Ef} 2, 14).
	
	4. \textquote{Muchas veces y de diversas maneras Dios habló a nuestros padres por medio de los profetas; hoy, sin embargo, nos habla por medio del Hijo} (Versículo antes del Evangelio: cf. \emph{Hb} 1, 1-2). En estas palabras encontramos el paso de la Antigua a la Nueva Alianza. Dios nos habló a través del Hijo, a través de su vida y su Evangelio. Nos habló a través de su muerte y resurrección y, \emph{de manera particular, a través de su nombre}: Jesús, que significa \textquote{Dios salva}. Todo está encerrado en Él: vida, pasión, muerte y resurrección, cruz y gloria. Toda la buena noticia.
	
	El autor de la Carta a los Hebreos nos habla de que este nombre se ha conservado para los \textquote{últimos tiempos}. Al comienzo del nuevo año somos conscientes de que, en el nombre de Jesús, \emph{los últimos tiempos, el tiempo del cumplimiento} de todas las cosas en Dios, se ha acercado a la humanidad de manera decisiva. Y en virtud de este nombre vamos al encuentro de la meta definitiva del hombre, la definitiva \textquote{\emph{plenitud de los tiempos}} (cf. \emph{Gal} 4, 4), a la que Cristo nos conduce por el Espíritu Santo.
	
	Guiados por esta fuerza repetimos: \textquote{¡\emph{Abbà}, Padre!} ya aquí, en la tierra, para prepararnos para la realización que, precisamente en el nombre de Jesús, debe manifestarse al final de los tiempos para todo hombre y para toda la familia humana.
\end{body}

\subsubsection{Homilía (1999)}

\src{Basílica de San Pedro. XXXII Jornada Mundial de la Paz. \\1 de enero de 1999.}

\begin{body}
	1. \emph{Christus heri et hodie, principium et finis, alpha et omega} \ldots{} \textquote{Cristo ayer y hoy, principio y fin, alfa y omega. Suyo es el tiempo y la eternidad. A él la gloria y el poder por los siglos de los siglos} (\emph{Misal romano}, preparación del cirio pascual).
	
	\ltr{T}{odos} los años, durante la Vigilia pascual, la Iglesia renueva esta solemne aclamación a Cristo, Señor del tiempo. También el último día del año proclamamos esta verdad, en el paso del \textquote{ayer} al \textquote{hoy}: \textquote{ayer}, al dar gracias a Dios por la conclusión del año viejo; \textquote{hoy}, al acoger el año que empieza. \emph{Heri et hodie}. Celebramos a Cristo que, como dice la Escritura, es \textquote{el mismo ayer, hoy y siempre} (\emph{Hb} 13, 8). Él es el Señor de la historia; suyos son los siglos y los milenios.
	
	Al comenzar el año 1999, el último antes del gran jubileo, parece que el misterio de la historia se revela ante nosotros con una profundidad más intensa. Precisamente por eso, la Iglesia ha querido imprimir el signo trinitario de la presencia del Dios vivo sobre el trienio de preparación inmediata para el acontecimiento jubilar.
	
	2. El primer día del nuevo año concluye la Octava de la Navidad del Señor y está dedicado a la santísima Virgen, venerada como Madre de Dios. El evangelio nos recuerda que \textquote{guardaba todas estas cosas y las meditaba en su corazón} (\emph{Lc} 2, 19). Así sucedió en Belén, en el Gólgota, al pie de la cruz, y el día de Pentecostés, cuando el Espíritu Santo descendió al cenáculo.
	
	Y lo mismo sucede también hoy. La Madre de Dios y de los hombres guarda y medita en su corazón todos los problemas de la humanidad, grandes y difíciles. La \emph{Alma Redemptoris Mater} camina con nosotros y nos guía, con ternura materna, hacia el futuro. Así, ayuda a la humanidad a cruzar todos los \textquote{umbrales} de los años, de los siglos y de los milenios, sosteniendo su esperanza en aquel que es el Señor de la historia
	
	3. \emph{Heri et hodie}. Ayer y hoy. \textquote{Ayer} invita a la retrospección. Cuando dirigimos nuestra mirada a los acontecimientos de este siglo que está a punto de terminar, se presentan ante nuestros ojos las dos guerras mundiales: cementerios, tumbas de caídos, familias destruidas, llanto y desesperación, miseria y sufrimiento. ¿Cómo olvidar los campos de muerte, a los hijos de Israel exterminados cruelmente y a los santos mártires: el padre Maximiliano Kolbe, sor Edith Stein y tantos otros?
	
	Sin embargo, nuestro siglo es también el siglo de la \emph{Declaración universal de derechos del hombre}, cuyo 50° aniversario se celebró recientemente. Teniendo presente precisamente este aniversario, en el tradicional Mensaje para la actual \emph{Jornada mundial de la paz}, quise recordar que el secreto de la paz verdadera reside en el respeto de los derechos humanos. \textquote{El reconocimiento de la dignidad innata de todos los miembros de la familia humana (\ldots{}) es el fundamento de la libertad, de la justicia y de la paz en el mundo} (n. 3: \emph{L'Osservatore Romano}, edición en lengua española, 18 de diciembre de 1998, p. 6).
	
	El concilio Vaticano II, el concilio que ha preparado a la Iglesia para entrar en el tercer milenio, reafirmó que el mundo, teatro de la historia del género humano, ha sido liberado de la esclavitud del pecado por Cristo crucificado y resucitado, \textquote{para que se transforme, según el designio de Dios, y llegue a su consumación} (\emph{Gaudium et spes}, 2). Es así como los creyentes miran al mundo de nuestros días, a la vez que avanzan gradualmente hacia el umbral del año 2000.
	
	4. El Verbo eterno, al hacerse hombre, entró en el mundo y lo acogió para redimirlo. Por tanto, el mundo no sólo está marcado por la terrible herencia del pecado; es, ante todo, un mundo salvado por Cristo, el Hijo de Dios, crucificado y resucitado. Jesús es el Redentor del mundo, el Señor de la historia. \emph{Eius sunt tempora et saecula}: suyos son los años y los siglos. Por eso creemos que, al entrar en el tercer milenio junto con Cristo, cooperaremos en la transformación del mundo redimido por él. \emph{Mundus creatus, mundus redemptus}.
	
	Desgraciadamente, la humanidad cede a la influencia del mal de muchos modos. Sin embargo, impulsada por la gracia, se levanta continuamente, y camina hacia el bien guiada por la fuerza de la redención. Camina hacia Cristo, según el proyecto de Dios Padre.
	
	\textquote{Jesucristo es el principio y el fin, el alfa y la omega. Suyo es el tiempo y la eternidad}.
	
	Empecemos este año nuevo en su nombre. Que María nos obtenga la gracia de ser fieles discípulos suyos, para que con palabras y obras lo glorifiquemos y honremos por los siglos de los siglos: \emph{Ipsi gloria et imperium per universa aeternitatis saecula}. Amén.
\end{body}

\subsubsection{Homilía (2002): Madre de la paz}

\src{Basílica de San Pedro. XXXV Jornada Mundial de la Paz. \\1 de enero del 2002.}

\begin{body}
	1. \textquote{¡Salve, Madre santa!, Virgen Madre del Rey que gobierna cielo y tierra por los siglos de los siglos} (cf. \emph{Antífona de entrada}).
	
	\ltr{C}{on} este antiguo saludo, la Iglesia se dirige hoy, octavo día después de la Navidad y primero del año 2002, a María santísima, invocándola como \emph{Madre de Dios}.
	
	El Hijo eterno del Padre tomó en ella nuestra misma carne y, a través de ella, se convirtió en \textquote{hijo de David e hijo de Abraham} (\emph{Mt} 1, 1). Por tanto, María es su verdadera Madre: ¡\emph{Theotókos}, Madre de Dios!
	
	Si Jesús es la vida, María es la Madre de la vida.
	
	Si Jesús es la esperanza, María es la Madre de la esperanza.
	
	Si Jesús es la paz, María es la Madre de la paz, Madre del Príncipe de la paz.
	
	Al entrar en el nuevo año, pidamos a esta Madre santa que nos bendiga. Pidámosle que nos dé a Jesús, \emph{nuestra bendición plena, en quien el Padre ha bendecido de una vez para siempre la historia}, transformándola en historia de salvación.
	
	2. \emph{¡Salve, Madre santa!} Bajo la mirada materna de María se sitúa esta \emph{Jornada mundial de la paz}. Reflexionamos sobre la paz en un clima de preocupación generalizada a causa de los recientes acontecimientos dramáticos que han sacudido el mundo. Pero, aunque pueda parecer humanamente difícil mirar al futuro con optimismo, no debemos ceder a la tentación del desaliento.
	
	Al contrario, debemos trabajar por la paz con valentía, conscientes de que el mal no prevalecerá.
	
	La luz y la esperanza para este compromiso nos vienen de Cristo. El \emph{Niño} nacido en Belén es la Palabra eterna del Padre hecha carne por nuestra salvación, es el \textquote{Dios con nosotros}, que trae consigo \emph{el secreto de la verdadera paz}. Es el \emph{Príncipe de la paz}.
	
	3. Con estos sentimientos, saludo con deferencia a los ilustres señores embajadores ante la Santa Sede que han querido participar en esta solemne celebración. Saludo afectuosamente al presidente del Consejo pontificio Justicia y paz, señor cardenal François Xavier Nguyên Van Thuân, y a todos sus colaboradores, y les agradezco el esfuerzo que realizan a fin de difundir mi Mensaje anual para la Jornada mundial de la paz, que este año tiene como tema: \textquote{No hay paz sin justicia, no hay justicia sin perdón}.
	
	\emph{Justicia y perdón}: estos son los dos \textquote{pilares} de la paz, que he querido poner de relieve. \emph{Entre justicia y perdón no hay contraposición, sino complementariedad}, porque ambos son esenciales para la promoción de la paz. En efecto, esta, mucho más que un cese temporal de las hostilidades, es una profunda cicatrización de las heridas abiertas que rasgan los corazones (cf. \emph{Mensaje}, 3: \emph{L'Osservatore Romano,} edición en lengua española, 14 de diciembre de 2001, p. 7). Sólo el perdón puede apagar la sed de venganza y abrir el corazón a una reconciliación auténtica y duradera entre los pueblos.
	
	4. Dirigimos hoy nuestra mirada al Niño, a quien María estrecha entre sus brazos. En él reconocemos a Aquel en quien la misericordia y la verdad se encuentran, la justicia y la paz se besan (cf. \emph{Sal} 84, 11). En él adoramos al Mesías verdadero, en quien Dios ha conjugado, para nuestra salvación, la verdad y la misericordia, la justicia y el perdón.
	
	En nombre de Dios renuevo mi llamamiento apremiante a todos, creyentes y no creyentes, para que el binomio \textquote{justicia y perdón} caracterice siempre las relaciones entre las personas, entre los grupos sociales y entre los pueblos.
	
	Este llamamiento se dirige, ante todo, a \emph{cuantos creen en Dios}, en particular a las tres grandes religiones que descienden de Abraham, \emph{judaísmo, cristianismo} e \emph{islam}, llamadas a \emph{rechazar siempre con firmeza y decisión la violencia. Nadie, por ningún motivo, puede matar en nombre de Dios, único y misericordioso}. Dios es vida y fuente de la vida. Creer en él significa testimoniar su misericordia y su perdón, evitando instrumentalizar su santo nombre.
	
	Desde diversas partes del mundo se eleva una ferviente invocación de paz; se eleva particularmente de la \emph{Tierra} que Dios bendijo con su Alianza y su Encarnación, y que por eso llamamos \emph{Santa}. \textquote{La voz de la sangre} clama a Dios desde aquella tierra (cf. \emph{Gn} 4, 10); sangre de hermanos derramada por hermanos, que se remontan al mismo patriarca Abraham; hijos, como todos los hombres, del mismo Padre celestial.
	
	5. \emph{¡Salve, Madre santa!} Virgen hija de Sión, ¡cuánto debe sufrir por esta sangre tu corazón de Madre!
	
	El Niño que estrechas contra tu pecho lleva un nombre apreciado por los pueblos de religión bíblica: \emph{Jesús}, que significa \textquote{Dios salva}. Así lo llamó el arcángel antes de que fuera concebido en tu seno (cf. \emph{Lc} 2, 21). En el rostro del Mesías recién nacido reconocemos el rostro de todos tus hijos vilipendiados y explotados. Reconocemos especialmente el rostro de los niños, cualquiera que sea su raza, nación y cultura. Por ellos, oh María, por su futuro, te pedimos que ablandes los corazones endurecidos por el odio, para que se abran al amor, y la venganza ceda finalmente el paso al perdón.
	
	Obtennos, oh Madre, que la verdad de esta afirmación -\textquote{No hay paz sin justicia, no hay justicia sin perdón}- se grabe en el corazón de todos. Así la familia humana podrá encontrar la paz verdadera, que brota del encuentro entre la justicia y la misericordia.
	
	Madre santa, Madre del Príncipe de la paz, ¡ayúdanos!
	
	Madre de la humanidad y Reina de la paz, ¡ruega por nosotros!
\end{body}

\subsubsection{Homilía(2005): Vencer al mal con el bien}

\src{Sábado 1 de enero de 2005.}

\begin{body}
	1. \textquote{¡Salve, Madre santa!, Virgen Madre del Rey, que gobierna cielo y tierra por los siglos de los siglos} (Antífona de entrada).
	
	\ltr{E}{n} el primer día del año, la Iglesia se reúne en oración ante el icono de la Madre de Dios, y honra con alegría a aquella que \emph{dio al mundo el fruto de su vientre, Jesús, el \textquote{Príncipe de la paz}} (\emph{Is} 9, 5).
	
	2. Ya es tradición consolidada celebrar en este mismo día la \emph{Jornada mundial de la paz}. En esta ocasión, me alegra expresar mi más cordial felicitación a los ilustres embajadores del Cuerpo diplomático ante la Santa Sede. Dirijo un saludo especial a los embajadores de los países particularmente afectados durante estos días por el enorme cataclismo que se abatió sobre ellos\ldots{}
	
	3. La Jornada mundial de la paz constituye una invitación a los cristianos y a todos los hombres de buena voluntad a renovar su firme compromiso de \emph{construir la paz}. Esto supone la acogida de una exigencia moral fundamental, expresada muy bien en las palabras de san Pablo: \textquote{No te dejes vencer por el mal; antes bien, vence al mal con el bien} (\emph{Rm} 12, 21).
	
	Ante las numerosas manifestaciones del mal, que por desgracia hieren a la familia humana, la exigencia prioritaria es \emph{promover la paz utilizando medios coherentes}, dando importancia al diálogo, a las obras de justicia, y educando para el perdón (cf. \emph{Mensaje para la Jornada mundial de la paz de 2005}, n. 1).
	
	4. \emph{Vencer el mal con las armas del amor} es el modo como \emph{cada uno puede contribuir a la paz de todos}. A lo largo de esta senda están llamados a caminar tanto los cristianos como los creyentes de las diversas religiones, juntamente con cuantos se reconocen en la \emph{ley moral universal}.
	
	Amadísimos hermanos y hermanas, promover la paz en la tierra es \emph{nuestra misión común}.
	
	Que la Virgen María nos ayude a realizar las palabras del Señor: \textquote{Bienaventurados los que trabajan por la paz, porque ellos serán llamados hijos de Dios} (\emph{Mt} 5, 9).
	
	¡Feliz año nuevo a todos! ¡Alabado sea Jesucristo!
\end{body}

\newsection

\subsection{Benedicto XVI, papa}

\subsubsection{Homilía (2008): Nos ayuda a conocer a su Hijo}

\src{Martes 1 de enero del 2008.}

\begin{body}
	\ltr{H}{oy} comenzamos un año nuevo y nos lleva de la mano la esperanza cristiana. Lo comenzamos invocando sobre él la bendición divina e implorando, por intercesión de María, Madre de Dios, el don de la paz para nuestras familias, para nuestras ciudades y para el mundo entero.
	
	Con este deseo os saludo a todos vosotros, aquí presentes\ldots{}
	
	La paz. En la \textbf{primera lectura}, tomada del libro de los Números, hemos escuchado la invocación: \textquote{El Señor te conceda la paz} (\emph{Nm} 6, 26). El Señor conceda la paz a cada uno de vosotros, a vuestras familias y al mundo entero. Todos aspiramos a vivir en paz, pero la paz verdadera, la que anunciaron los ángeles en la noche de Navidad, no es conquista del hombre o fruto de acuerdos políticos; es ante todo don divino, que es preciso implorar constantemente y, al mismo tiempo, compromiso que es necesario realizar con paciencia, siempre dóciles a los mandatos del Señor.
	
	Este año, en el Mensaje para esta Jornada mundial de la paz puse de relieve la íntima relación que existe entre la familia y la construcción de la paz en el mundo. La familia natural, fundada en el matrimonio entre un hombre y una mujer, es \textquote{cuna de la vida y del amor} y \textquote{la primera e insustituible educadora de la paz}. Precisamente por eso la familia es \textquote{la principal \textquote{agencia} de paz} y \textquote{la negación o restricción de los derechos de la familia, al oscurecer la verdad sobre el hombre, \emph{amenaza los fundamentos mismos de la paz}} (cf. nn. 1-5). Dado que la humanidad es una \textquote{gran familia}, si quiere vivir en paz, no puede por menos de inspirarse en esos valores, sobre los cuales se funda y se apoya la comunidad familiar.
	
	La providencial coincidencia de varias celebraciones nos impulsa este año a un esfuerzo aún mayor para realizar la paz en el mundo. Hace sesenta años, en 1948, la Asamblea general de las Naciones Unidas hizo pública la \textquote{Declaración universal de derechos humanos}. Hace cuarenta años, mi venerado predecesor Pablo VI celebró la primera Jornada mundial de la paz. Este año, además, recordaremos el 25° aniversario de la adopción por parte de la Santa Sede de la \textquote{Carta de los derechos de la familia}. \textquote{A la luz de estas significativas efemérides ---cito aquí lo que escribí precisamente al concluir el Mensaje---, invito a todos los hombres y mujeres a tomar una conciencia más clara de la pertenencia común a la única familia humana y a comprometerse para que la convivencia en la tierra refleje cada vez más esta convicción, de la cual depende la instauración de una paz verdadera y duradera} (\emph{L'Osservatore Romano}, edición en lengua española, 14 de diciembre de 2007, p. 5).
	
	Nuestro pensamiento se dirige ahora, naturalmente, a la Virgen María, a la que hoy invocamos como Madre de Dios. Fue el Papa Pablo VI quien trasladó al día 1 de enero la fiesta de la Maternidad divina de María, que antes caía el 11 de octubre. En efecto, antes de la reforma litúrgica realizada después del concilio Vaticano II, en el primer día del año se celebraba la memoria de la circuncisión de Jesús en el octavo día después de su nacimiento ---como signo de sumisión a la ley, su inserción oficial en el pueblo elegido--- y el domingo siguiente se celebraba la fiesta del nombre de Jesús.
	
	De esas celebraciones encontramos algunas huellas en la página evangélica que acabamos de proclamar, en la que \textbf{san Lucas} refiere que, ocho días después de su nacimiento, el Niño fue circuncidado y le pusieron el nombre de Jesús, \textquote{el que le dio el ángel antes de ser concebido en el seno de su madre} (\emph{Lc} 2, 21). Por tanto, esta solemnidad, además de ser una fiesta mariana muy significativa, conserva también un fuerte contenido cristológico, porque, podríamos decir, antes que a la Madre, atañe precisamente al Hijo, a Jesús, verdadero Dios y verdadero hombre.
	
	Al misterio de la maternidad divina de María, la \emph{Theotokos}, hace referencia el apóstol san Pablo en la \textbf{carta a los Gálatas}. \textquote{Al llegar la plenitud de los tiempos ---escribe--- envió Dios a su Hijo, nacido de mujer, nacido bajo la ley} (\emph{Ga} 4, 4). En pocas palabras se encuentran sintetizados el misterio de la encarnación del Verbo eterno y la maternidad divina de María: el gran privilegio de la Virgen consiste precisamente en ser Madre del Hijo, que es Dios.
	
	Así pues, \textbf{ocho días después de la Navidad}, esta fiesta mariana encuentra su lugar más lógico y adecuado. En efecto, en la noche de Belén, cuando \textquote{dio a luz a su hijo primogénito} (\emph{Lc} 2, 7), se cumplieron las profecías relativas al Mesías. \textquote{Una virgen concebirá y dará a luz un hijo}, había anunciado Isaías (\emph{Is} 7, 14). \textquote{Concebirás en tu seno y darás a luz un hijo} (\emph{Lc} 1, 31), dijo a María el ángel Gabriel. Y también un ángel del Señor ---narra el evangelista san Mateo---, apareciéndose en sueños a José, lo tranquilizó diciéndole: \textquote{No temas tomar contigo a María tu mujer, porque lo engendrado en ella es del Espíritu Santo. Dará a luz un hijo} (\emph{Mt} 1, 20-21).
	
	El título de Madre de Dios es, juntamente con el de Virgen santa, el más antiguo y constituye el fundamento de todos los demás títulos con los que María ha sido venerada y sigue siendo invocada de generación en generación, tanto en Oriente como en Occidente. Al misterio de su maternidad divina hacen referencia muchos himnos y numerosas oraciones de la tradición cristiana, como por ejemplo una antífona mariana del tiempo navideño, el \emph{Alma Redemptoris Mater}, con la que oramos así: \textquote{\emph{Tu quae genuisti, natura mirante, tuum sanctum Genitorem, Virgo prius ac posterius}} --- \textquote{Tú, ante el asombro de toda la creación, engendraste a tu Creador, Madre siempre virgen}.
	
	Queridos hermanos y hermanas, contemplemos hoy a María, Madre siempre virgen del Hijo unigénito del Padre. Aprendamos de ella a acoger al Niño que por nosotros nació en Belén. Si en el Niño nacido de ella reconocemos al Hijo eterno de Dios y lo acogemos como nuestro único Salvador, podemos ser llamados, y seremos realmente, hijos de Dios: hijos en el Hijo. El Apóstol escribe: \textquote{Envió Dios a su Hijo, nacido de mujer, nacido bajo la ley, para rescatar a los que se hallaban bajo la ley, y para que recibiéramos la filiación adoptiva} (\emph{Ga} 4, 4-5).
	
	El \textbf{evangelista san Lucas} repite varias veces que la Virgen meditaba silenciosamente esos acontecimientos extraordinarios en los que Dios la había implicado. Lo hemos escuchado también en el breve pasaje evangélico que la liturgia nos vuelve a proponer hoy. \textquote{María conservaba todas estas cosas meditándolas en su corazón} (\emph{Lc} 2, 19). El verbo griego usado, \emph{sumbállousa}, en su sentido literal significa \textquote{poner juntamente}, y hace pensar en un gran misterio que es preciso descubrir poco a poco.
	
	El Niño que emite vagidos en el pesebre, aun siendo en apariencia semejante a todos los niños del mundo, al mismo tiempo es totalmente diferente: es el Hijo de Dios, es Dios, verdadero Dios y verdadero hombre. Este misterio ---la encarnación del Verbo y la maternidad divina de María--- es grande y ciertamente no es fácil de comprender con la sola inteligencia humana.
	
	Sin embargo, en la escuela de María podemos captar con el corazón lo que los ojos y la mente por sí solos no logran percibir ni pueden contener. En efecto, se trata de un don tan grande que sólo con la fe podemos acoger, aun sin comprenderlo todo. Y es precisamente en este camino de fe donde María nos sale al encuentro, nos ayuda y nos guía. Ella es madre porque engendró en la carne a Jesús; y lo es porque se adhirió totalmente a la voluntad del Padre. San Agustín escribe: \textquote{Ningún valor hubiera tenido para ella la misma maternidad divina, si no hubiera llevado a Cristo en su corazón, con una suerte mayor que cuando lo concibió en la carne} (\emph{De sancta Virginitate} 3, 3). Y en su corazón María siguió conservando, \textquote{poniendo juntamente}, los acontecimientos sucesivos de los que fue testigo y protagonista, hasta la muerte en la cruz y la resurrección de su Hijo Jesús.
	
	Queridos hermanos y hermanas, sólo conservando en el corazón, es decir, poniendo juntamente y encontrando una unidad de todo lo que vivimos, podemos entrar, siguiendo a María, en el misterio de un Dios que por amor se hizo hombre y nos llama a seguirlo por la senda del amor, un amor que es preciso traducir cada día en un servicio generoso a los hermanos.
	
	Ojalá que el nuevo año, que hoy comenzamos con confianza, sea un tiempo en el que progresemos en ese conocimiento del corazón, que es la sabiduría de los santos. Oremos para que, como hemos escuchado en la \textbf{primera lectura}, el Señor \textquote{ilumine su rostro sobre nosotros} y nos \textquote{sea propicio} (cf. \emph{Nm} 6, 25) y nos bendiga.
	
	Podemos estar seguros de que, si buscamos sin descanso su rostro, si no cedemos a la tentación del desaliento y de la duda, si incluso en medio de las numerosas dificultades que encontramos permanecemos siempre anclados en él, experimentaremos la fuerza de su amor y de su misericordia. El frágil Niño que la Virgen muestra hoy al mundo nos haga agentes de paz, testigos de él, Príncipe de la paz. Amén.
\end{body}

\subsubsection{Homilía (2011): Recibió el don de Dios}

\src{Basílica Vaticana. Sábado 1 de enero del 2011.}

\begin{body}
	\ltr{T}{odavía} inmersos en el clima espiritual de la Navidad, en la que hemos contemplado el misterio del nacimiento de Cristo, con los mismos sentimientos celebramos hoy a la Virgen María, a quien la Iglesia venera como Madre de Dios, porque dio carne al Hijo del Padre eterno. Las lecturas bíblicas de esta solemnidad ponen el acento principalmente en el Hijo de Dios hecho hombre y en el \textquote{nombre} del Señor. La \textbf{primera lectura} nos presenta la solemne bendición que pronunciaban los sacerdotes sobre los israelitas en las grandes fiestas religiosas: está marcada precisamente por el nombre del Señor, que se repite tres veces, como para expresar la plenitud y la fuerza que deriva de esa invocación. En efecto, este texto de bendición litúrgica evoca la riqueza de gracia y de paz que Dios da al hombre, con una disposición benévola respecto a este, y que se manifiesta con el \textquote{resplandecer} del rostro divino y el \textquote{dirigirlo} hacia nosotros.
	
	La Iglesia vuelve a escuchar hoy estas palabras, mientras pide al Señor que bendiga el nuevo año que acaba de comenzar, con la conciencia de que, ante los trágicos acontecimientos que marcan la historia, ante las lógicas de guerra que lamentablemente todavía no se han superado totalmente, sólo Dios puede tocar profundamente el alma humana y asegurar esperanza y paz a la humanidad. De hecho, ya es una tradición consolidada que en el primer día del año la Iglesia, presente en todo el mundo, eleve una oración coral para invocar la paz. Es bueno iniciar un emprendiendo decididamente la senda de la paz. Hoy, queremos recoger el grito de tantos hombres, mujeres, niños y ancianos víctimas de la guerra, que es el rostro más horrendo y violento de la historia. Hoy rezamos a fin de que la paz, que los ángeles anunciaron a los pastores la noche de Navidad, llegue a todos los rincones del mundo: \textquote{\emph{Super terram pax in hominibus bonae voluntatis}} (\emph{Lc} 2, 14). Por esto, especialmente con nuestra oración, queremos ayudar a todo hombre y a todo pueblo, en particular a cuantos tienen responsabilidades de gobierno, a avanzar de modo cada vez más decidido por el camino de la paz.
	
	En la \textbf{segunda lectura}, san Pablo resume en la adopción filial la obra de salvación realizada por Cristo, en la cual está como engarzada la figura de María. Gracias a ella el Hijo de Dios, \textquote{nacido de mujer} (\emph{Ga} 4, 4), pudo venir al mundo como verdadero hombre, en la plenitud de los tiempos. Ese cumplimiento, esa plenitud, atañe al pasado y a las esperas mesiánicas, que se realizan, pero, al mismo tiempo, también se refiere a la plenitud en sentido absoluto: en el Verbo hecho carne Dios dijo su Palabra última y definitiva. En el umbral de un año nuevo, resuena así la invitación a caminar con alegría hacia la luz del \textquote{sol que nace de lo alto} (\emph{Lc}1, 78), puesto que en la perspectiva cristiana todo el tiempo está habitado por Dios, no hay futuro que no sea en la dirección de Cristo y no existe plenitud fuera de la de Cristo.
	
	El pasaje del \textbf{Evangelio} de hoy termina con la imposición del nombre de Jesús, mientras María participa en silencio, meditando en su corazón sobre el misterio de su Hijo, que de modo completamente singular es don de Dios. Pero el pasaje evangélico que hemos escuchado hace hincapié especialmente en los pastores, que se volvieron \textquote{glorificando y alabando a Dios por todo lo que habían oído y visto} (\emph{Lc} 2, 20). El ángel les había anunciado que en la ciudad de David, es decir, en Belén había nacido el Salvador y que iban a encontrar \emph{la} \emph{señal}: un niño envuelto en pañales y acostado en un pesebre (cf. \emph{Lc} 2, 11-12). Fueron a toda prisa, y encontraron a María y a José, y al Niño. Notemos que el Evangelista habla de la maternidad de María a partir del Hijo, de ese \textquote{niño envuelto en pañales}, porque es él ---el Verbo de Dios (\emph{Jn} 1, 14)--- el punto de referencia, el centro del acontecimiento que está teniendo lugar, y es él quien hace que la maternidad de María se califique como \textquote{divina}.
	
	Esta atención predominante que las lecturas de hoy dedican al \textquote{Hijo}, a Jesús, no reduce el papel de la Madre; más aún, la sitúa en la perspectiva correcta: en efecto, María es verdadera Madre de Dios precisamente en virtud de su relación total con Cristo. Por tanto, glorificando al Hijo se honra a la Madre y honrando a la Madre se glorifica al Hijo. El título de \textquote{Madre de Dios}, que hoy la liturgia pone de relieve, subraya la misión única de la Virgen santísima en la historia de la salvación: misión que está en la base del culto y de la devoción que el pueblo cristiano le profesa. En efecto, María no recibió el don de Dios sólo para ella, sino para llevarlo al mundo: en su virginidad fecunda, Dios dio a los hombres los bienes de la salvación eterna (cf. \emph{Oración Colecta}). Y María ofrece continuamente su mediación al pueblo de Dios peregrino en la historia hacia la eternidad, como en otro tiempo la ofreció a los pastores de Belén. Ella, que dio la vida terrena al Hijo de Dios, sigue dando a los hombres la vida divina, que es Jesús mismo y su Santo Espíritu. Por esto es considerada madre de todo hombre que nace a la Gracia y a la vez se la invoca como Madre de la Iglesia.
	
	En el nombre de María, Madre de Dios y de los hombres, desde el 1 de enero de 1968 se celebra en todo el mundo la Jornada mundial de la paz. La paz es don de Dios, como hemos escuchado en la \textbf{primera lectura}: \textquote{Que el Señor (\ldots{}) te conceda la paz} (\emph{Nm} 6, 26). Es el don mesiánico por excelencia, el primer fruto de la caridad que Jesús nos ha dado; es nuestra reconciliación y pacificación con Dios. La paz también es un valor humano que se ha de realizar en el ámbito social y político, pero hunde sus raíces en el misterio de Cristo (cf. \emph{Gaudium et spes}, 77-90). En esta celebración solemne, con ocasión de la 44ª Jornada mundial de la paz, me alegra dirigir mi deferente saludo a los ilustres embajadores ante la Santa Sede, con mis mejores deseos para su misión. Asimismo, dirijo un saludo cordial y fraterno a mi secretario de Estado y a los demás responsables de los dicasterios de la Curia romana, con un pensamiento particular para el presidente del Consejo pontificio \textquote{Justicia y paz} y sus colaboradores. Deseo manifestarles mi vivo reconocimiento por su compromiso diario en favor de una convivencia pacífica entre los pueblos y de la formación cada vez más sólida de una conciencia de paz en la Iglesia y en el mundo. Desde esta perspectiva, la comunidad eclesial está cada vez más comprometida a actuar, según las indicaciones del Magisterio, para ofrecer un patrimonio espiritual seguro de valores y de principios, en la búsqueda continua de la paz.
	
	En mi \emph{Mensaje} para la Jornada de hoy, que lleva por título \textquote{Libertad religiosa, camino para la paz} he querido recordar que: \textquote{El mundo tiene necesidad de Dios. Tiene necesidad de valores éticos y espirituales, universales y compartidos, y la religión puede contribuir de manera preciosa a su búsqueda, para la construcción de un orden social e internacional justo y pacífico} (n. 15). Por tanto, he subrayado que \textquote{la libertad religiosa (\ldots{}) es un elemento imprescindible de un Estado de derecho; no se puede negar sin dañar al mismo tiempo los demás derechos y libertades fundamentales, pues es su síntesis y su cumbre} (n. 5).
	
	La humanidad no puede mostrarse resignada a la fuerza negativa del egoísmo y de la violencia; no debe acostumbrarse a conflictos que provoquen víctimas y pongan en peligro el futuro de los pueblos. Frente a las amenazadoras tensiones del momento, especialmente frente a las discriminaciones, los abusos y las intolerancias religiosas, que hoy golpean de modo particular a los cristianos (cf. \emph{ib}., 1), dirijo una vez más una apremiante invitación a no ceder al desaliento y a la resignación. Os exhorto a todos a rezar a fin de que lleguen a buen fin los esfuerzos emprendidos desde diversas partes para promover y construir la paz en el mundo. Para esta difícil tarea no bastan las palabras; es preciso el compromiso concreto y constante de los responsables de las naciones, pero sobre todo es necesario que todas las personas actúen animadas por el auténtico espíritu de paz, que siempre hay que implorar de nuevo en la oración y vivir en las relaciones cotidianas, en cada ambiente.
	
	En esta celebración eucarística tenemos delante de nuestros ojos, para nuestra veneración, la imagen de la Virgen del \textquote{Sacro Monte di Viggiano}, tan querida para los habitantes de Basilicata. La Virgen María nos da a su Hijo, nos muestra el rostro de su Hijo, Príncipe de la paz: que ella nos ayude a permanecer en la luz de este rostro, que brilla sobre nosotros (cf. \emph{Nm} 6, 25), para redescubrir toda la ternura de Dios Padre; que ella nos sostenga al invocar al Espíritu Santo, para que renueve la faz de la tierra y transforme los corazones, ablandando su dureza ante la bondad desarmante del Niño, que ha nacido por nosotros. Que la Madre de Dios nos acompañe en este nuevo año; que obtenga para nosotros y para todo el mundo el deseado don de la paz. Amén.
\end{body}

\newsection

\subsection{Francisco, papa}

\subsubsection{Homilía (2014): Bendición cumplida en María}

\src{Basílica Vaticana. Miércoles 1 de enero de 2014.}

\begin{body}
	\ltr{L}{a} \textbf{primera lectura} que hemos escuchado nos propone una vez más las antiguas palabras de bendición que Dios sugirió a Moisés para que las enseñara a Aarón y a sus hijos: \textquote{Que el Señor te bendiga y te proteja. Que el Señor haga brillar su rostro sobre ti y te muestre su gracia. Que el Señor te descubra su rostro y te conceda la paz} (\emph{Nm} 6,24-25). Es muy significativo escuchar de nuevo esta bendición precisamente al comienzo del nuevo año: ella acompañará nuestro camino durante el tiempo que ahora nos espera. Son palabras de fuerza, de valor, de esperanza. No de una esperanza ilusoria, basada en frágiles promesas humanas; ni tampoco de una esperanza ingenua, que imagina un futuro mejor sólo porque es futuro. Esta esperanza tiene su razón de ser precisamente en la bendición de Dios, una bendición que contiene el mejor de los deseos, el deseo de la Iglesia para todos nosotros, impregnado de la protección amorosa del Señor, de su ayuda providente.
	
	El deseo contenido en esta bendición se ha realizado plenamente en una mujer, María, por haber sido destinada a ser la Madre de Dios, y se ha cumplido en ella antes que en ninguna otra criatura.
	
	Madre de Dios. Este es el título principal y esencial de la Virgen María. Es una cualidad, un cometido, que la fe del pueblo cristiano siempre ha experimentado, en su tierna y genuina devoción por nuestra madre celestial.
	
	Recordemos aquel gran momento de la historia de la Iglesia antigua, el Concilio de Éfeso, en el que fue definida con autoridad la divina maternidad de la Virgen. La verdad sobre la divina maternidad de María encontró eco en Roma, donde poco después se construyó la Basílica de Santa María \textquote{la Mayor}, primer santuario mariano de Roma y de todo occidente, y en el cual se venera la imagen de la Madre de Dios ---la \emph{Theotokos}--- con el título de \emph{Salus populi romani}. Se dice que, durante el Concilio, los habitantes de Éfeso se congregaban a ambos lados de la puerta de la basílica donde se reunían los Obispos, gritando: \textquote{¡Madre de Dios!}. Los fieles, al pedir que se definiera oficialmente este título mariano, demostraban reconocer ya la divina maternidad. Es la actitud espontánea y sincera de los hijos, que conocen bien a su madre, porque la aman con inmensa ternura. Pero es algo más: es el \emph{sensus fidei} del santo pueblo fiel de Dios, que nunca, en su unidad, nunca se equivoca.
	
	María está desde siempre presente en el corazón, en la devoción y, sobre todo, en el camino de fe del pueblo cristiano. \textquote{La Iglesia\ldots{} camina en el tiempo\ldots{} Pero en este camino ---deseo destacarlo enseguida--- procede recorriendo de nuevo el itinerario realizado por la Virgen María} (Juan Pablo II, Enc. \emph{Redemptoris Mater}, 2). Nuestro itinerario de fe es igual al de María, y por eso la sentimos particularmente cercana a nosotros. Por lo que respecta a la fe, que es el quicio de la vida cristiana, la Madre de Dios ha compartido nuestra condición, ha debido caminar por los mismos caminos que recorremos nosotros, a veces difíciles y oscuros, ha debido avanzar en \textquote{la peregrinación de la fe} (Conc. Ecum. Vat. II, Const. \emph{Lumen gentium}, 58).
	
	Nuestro camino de fe está unido de manera indisoluble a María desde el momento en que Jesús, muriendo en la cruz, nos la ha dado como Madre diciendo: \textquote{He ahí a tu madre} (\emph{Jn} 19,27). Estas palabras tienen un valor de testamento y dan al mundo una Madre. Desde ese momento, la Madre de Dios se ha convertido también en nuestra Madre. En aquella hora en la que la fe de los discípulos se agrietaba por tantas dificultades e incertidumbres, Jesús les confió a aquella que fue la primera en creer, y cuya fe no decaería jamás. Y la \textquote{mujer} se convierte en nuestra Madre en el momento en el que pierde al Hijo divino. Y su corazón herido se ensancha para acoger a todos los hombres, buenos y malos, a todos, y los ama como los amaba Jesús. La mujer que en las bodas de Caná de Galilea había cooperado con su fe a la manifestación de las maravillas de Dios en el mundo, en el Calvario mantiene encendida la llama de la fe en la resurrección de su Hijo, y la comunica con afecto materno a los demás. María se convierte así en fuente de esperanza y de verdadera alegría.
	
	La Madre del Redentor nos precede y continuamente nos confirma en la fe, en la vocación y en la misión. Con su ejemplo de humildad y de disponibilidad a la voluntad de Dios nos ayuda a traducir nuestra fe en un anuncio del Evangelio alegre y sin fronteras. De este modo nuestra misión será fecunda, porque está modelada sobre la maternidad de María. A ella confiamos nuestro itinerario de fe, los deseos de nuestro corazón, nuestras necesidades, las del mundo entero, especialmente el hambre y la sed de justicia y de paz y de Dios; y la invocamos todos juntos, y os invito a invocarla tres veces, imitando a aquellos hermanos de Éfeso, diciéndole: ¡Madre de Dios! ¡Madre de Dios! ¡Madre de Dios! ¡Madre de Dios! Amén.
\end{body}

\subsubsection{Ángelus (2014)} 

\src{1 de enero del 2014.}

\begin{body}
	\emph{Queridos hermanos y hermanas, ¡buenos día y feliz año!}
	
	\ltr{A}{l} inicio del nuevo año dirijo a todos vosotros los más cordiales deseos de paz y de todo bien. Mi deseo es el de la Iglesia, el deseo cristiano. No está relacionado con el sentido un poco mágico y un poco fatalista de un nuevo ciclo que inicia. Sabemos que la historia tiene un centro: Jesucristo, encarnado, muerto y resucitado, que vive entre nosotros; tiene un fin: el Reino de Dios, Reino de paz, de justicia, de libertad en el amor; y tiene una fuerza que la mueve hacia ese fin: la fuerza es el Espíritu Santo. Todos nosotros tenemos el Espíritu Santo que hemos recibido en el Bautismo, y Él nos impulsa a seguir adelante por el camino de la vida cristiana, por la senda de la historia, hacia el Reino de Dios.
	
	Este Espíritu es la potencia de amor que fecundó el seno de la Virgen María; y es el mismo que anima los proyectos y las obras de todos los constructores de paz. Donde hay un hombre o una mujer constructor de paz, es precisamente el Espíritu Santo quien le ayuda, le impulsa a construir la paz. Dos caminos que se cruzan hoy: fiesta de María santísima Madre de Dios y Jornada mundial de la paz. Hace ocho días resonaba el anuncio angelical: \textquote{Gloria a Dios y paz a los hombres}; hoy lo acogemos nuevamente de la Madre de Jesús, que \textquote{conservaba todas estas cosas, meditándolas en su corazón} (\emph{Lc} 2, 19), para hacer de ello nuestro compromiso a lo largo del año que comienza.
	
	El tema de esta Jornada mundial de la paz es \textquote{\emph{La fraternidad, fundamento y camino para la paz}}. Fraternidad: siguiendo la estela de mis Predecesores, a partir de Pablo VI, he desarrollado el tema en un Mensaje, ya difundido y hoy idealmente entrego a todos. En la base está la convicción de que todos somos hijos del único Padre celestial, formamos parte de la misma familia humana y compartimos un destino común. De aquí se deriva para cada uno la responsabilidad de obrar a fin de que el mundo llegue a ser una comunidad de hermanos que se respetan, se aceptan en su diversidad y se cuidan unos a otros. Estamos llamados también a darnos cuenta de las violencias e injusticias presentes en tantas partes del mundo y que no pueden dejarnos indiferentes e inmóviles: se necesita del compromiso de todos para construir una sociedad verdaderamente más justa y solidaria. Ayer recibí una carta de un señor, tal vez uno de vosotros, quien informándome sobre una tragedia familiar, a continuación enumeraba muchas tragedias y guerras de hoy en el mundo, y me preguntaba: ¿qué sucede en el corazón del hombre, que le lleva a hacer todo esto? Y decía, al final: \textquote{Es hora de detenerse}. También yo creo que nos hará bien detenernos en este camino de violencia, y buscar la paz. Hermanos y hermanas, hago mías las palabras de este hombre: ¿qué sucede en el corazón del hombre? ¿Qué sucede en el corazón de la humanidad? ¡Es hora de detenerse!
	
	Desde todos los rincones de la tierra, los creyentes elevan hoy la oración para pedir al Señor el don de la paz y la capacidad de llevarla a cada ambiente. En este primer día del año, que el Señor nos ayude a encaminarnos todos con más firmeza por las sendas de la justicia y de la paz. Y comencemos en casa. Justicia y paz en casa, entre nosotros. Se comienza en casa y luego se sigue adelante, a toda la humanidad. Pero debemos comenzar en casa. Que el Espíritu Santo actúe en nuestro corazón, rompa las cerrazones y las durezas y nos conceda enternecernos ante la debilidad del Niño Jesús. La paz, en efecto, requiere la fuerza de la mansedumbre, la fuerza no violenta de la verdad y del amor.
	
	En las manos de María, Madre del Redentor, ponemos con confianza filial nuestras esperanzas. A ella, que extiende su maternidad a todos los hombres, confiamos el grito de paz de las poblaciones oprimidas por la guerra y la violencia, para que la valentía del diálogo y de la reconciliación predomine sobre las tentaciones de venganza, de prepotencia y corrupción. A ella le pedimos que el Evangelio de la fraternidad, anunciado y testimoniado por la Iglesia, pueda hablar a cada conciencia y derribar los muros que impiden a los enemigos reconocerse hermanos.
\end{body}

\subsubsection{Homilía (2017): No somos huérfanos}

\src{Basílica Vaticana. 1 de enero del 2017.}

\begin{body}
	\textquote{Mientras tanto, María conservaba estas cosas y las meditaba en su corazón} (\emph{Lc} 2, 19). 
	
	\ltr{A}{sí} \textbf{Lucas} describe la actitud con la que María recibe todo lo que estaban viviendo en esos días. Lejos de querer entender o adueñarse de la situación, María es la mujer que sabe conservar, es decir proteger, \emph{custodiar} en su corazón el paso de Dios en la vida de su Pueblo. Desde sus entrañas aprendió a escuchar el latir del corazón de su Hijo y eso le enseñó, a lo largo de toda su vida, a descubrir el palpitar de Dios en la historia. Aprendió a ser madre y, en ese aprendizaje, le regaló a Jesús la hermosa experiencia de saberse Hijo. En María, el Verbo Eterno no sólo se hizo carne sino que aprendió a reconocer la ternura maternal de Dios. Con María, el Niño-Dios aprendió a escuchar los anhelos, las angustias, los gozos y las esperanzas del Pueblo de la promesa. Con ella se descubrió a sí mismo Hijo del santo Pueblo fiel de Dios.
	
	En los evangelios María aparece como mujer de pocas palabras, sin grandes discursos ni protagonismos pero con una mirada atenta que sabe custodiar la vida y la misión de su Hijo y, por tanto, de todo lo amado por Él. Ha sabido custodiar los albores de la primera comunidad cristiana, y así aprendió a ser madre de una multitud. Ella se ha acercado en las situaciones más diversas para sembrar esperanza. Acompañó las cruces cargadas en el silencio del corazón de sus hijos. Tantas devociones, tantos santuarios y capillas en los lugares más recónditos, tantas imágenes esparcidas por las casas, nos recuerdan esta gran verdad. María, nos dio el calor materno, ese que nos cobija en medio de la dificultad; el calor materno que permite que nada ni nadie apague en el seno de la Iglesia la revolución de la ternura inaugurada por su Hijo. Donde hay madre, hay ternura. Y María con su maternidad nos muestra que la humildad y la ternura no son virtudes de los débiles sino de los fuertes, nos enseña que no es necesario maltratar a otros para sentirse importantes (cf. Exhort. ap. \emph{Evangelii gaudium,} 288). Y desde siempre el santo Pueblo fiel de Dios la ha reconocido y saludado como la Santa Madre de Dios.
	
	Celebrar la maternidad de María como Madre de Dios y madre nuestra, al comenzar un nuevo año, significa recordar una certeza que acompañará nuestros días: somos un pueblo con Madre, no somos huérfanos.
	
	Las madres son el antídoto más fuerte ante nuestras tendencias individualistas y egoístas, ante nuestros encierros y apatías. Una sociedad sin madres no sería solamente una sociedad fría sino una sociedad que ha perdido el corazón, que ha perdido el \textquote{sabor a hogar}. Una sociedad sin madres sería una sociedad sin piedad que ha dejado lugar sólo al cálculo y a la especulación. Porque las madres, incluso en los peores momentos, saben dar testimonio de la ternura, de la entrega incondicional, de la fuerza de la esperanza. He aprendido mucho de esas madres que teniendo a sus hijos presos, o postrados en la cama de un hospital, o sometidos por la esclavitud de la droga, con frio o calor, lluvia o sequía, no se dan por vencidas y siguen peleando para darles a ellos lo mejor. O esas madres que en los campos de refugiados, o incluso en medio de la guerra, logran abrazar y sostener sin desfallecer el sufrimiento de sus hijos. Madres que dejan literalmente la vida para que ninguno de sus hijos se pierda. Donde está la madre hay unidad, hay pertenencia, pertenencia de hijos.
	
	Comenzar el año haciendo memoria de la bondad de Dios en el rostro maternal de María, en el rostro maternal de la Iglesia, en los rostros de nuestras madres, nos protege de la corrosiva enfermedad de \textquote{la orfandad espiritual}, esa orfandad que vive el alma cuando se siente sin madre y le falta la ternura de Dios. Esa orfandad que vivimos cuando se nos va apagando el sentido de pertenencia a una familia, a un pueblo, a una tierra, a nuestro Dios. Esa orfandad que gana espacio en el corazón narcisista que sólo sabe mirarse a sí mismo y a los propios intereses y que crece cuando nos olvidamos que la vida ha sido un regalo ---que se la debemos a otros--- y que estamos invitados a compartirla en esta casa común.
	
	Tal orfandad autorreferencial fue la que llevó a Caín a decir: \textquote{¿Acaso soy yo el guardián de mi hermano?} (\emph{Gn} 4,9), como afirmando: él no me pertenece, no lo reconozco. Tal actitud de orfandad espiritual es un cáncer que silenciosamente corroe y degrada el alma. Y así nos vamos degradando ya que, entonces, nadie nos pertenece y no pertenecemos a nadie: degrado la tierra, porque no me pertenece, degrado a los otros, porque no me pertenecen, degrado a Dios porque no le pertenezco, y finalmente termina degradándonos a nosotros mismos porque nos olvidamos quiénes somos, qué \textquote{apellido} divino tenemos. La pérdida de los lazos que nos unen, típica de nuestra cultura fragmentada y dividida, hace que crezca ese sentimiento de orfandad y, por tanto, de gran vacío y soledad. La falta de contacto físico (y no virtual) va cauterizando nuestros corazones (cf. Carta enc. \emph{Laudato si'}, 49) haciéndolos perder la capacidad de la ternura y del asombro, de la piedad y de la compasión. La orfandad espiritual nos hace perder la memoria de lo que significa ser hijos, ser nietos, ser padres, ser abuelos, ser amigos, ser creyentes. Nos hace perder la memoria del valor del juego, del canto, de la risa, del descanso, de la gratuidad.
	
	Celebrar la fiesta de la Santa Madre de Dios nos vuelve a dibujar en el rostro la sonrisa de sentirnos pueblo, de sentir que nos pertenecemos; de saber que solamente dentro de una comunidad, de una familia, las personas podemos encontrar \textquote{el clima}, \textquote{el calor} que nos permita aprender a crecer humanamente y no como meros objetos invitados a \textquote{consumir y ser consumidos}. Celebrar la fiesta de la Santa Madre de Dios nos recuerda que no somos mercancía intercambiable o terminales receptoras de información. Somos hijos, somos familia, somos Pueblo de Dios.
	
	Celebrar a la Santa Madre de Dios nos impulsa a generar y cuidar lugares comunes que nos den sentido de pertenencia, de arraigo, de hacernos sentir en casa dentro de nuestras ciudades, en comunidades que nos unan y nos ayudan (cf. Carta enc. \emph{Laudato si'}, 151 ).
	
	Jesucristo en el momento de mayor entrega de su vida, en la cruz, no quiso guardarse nada para sí y entregando su vida nos entregó también a su Madre. Le dijo a María: aquí está tu Hijo, aquí están tus hijos. Y nosotros queremos recibirla en nuestras casas, en nuestras familias, en nuestras comunidades, en nuestros pueblos. Queremos encontrarnos con su mirada maternal. Esa mirada que nos libra de la orfandad; esa mirada que nos recuerda que somos hermanos: que yo te pertenezco, que tú me perteneces, que somos de la misma carne. Esa mirada que nos enseña que tenemos que aprender a cuidar la vida de la misma manera y con la misma ternura con la que ella la ha cuidado: sembrando esperanza, sembrando pertenencia, sembrando fraternidad.
	
	Celebrar a la Santa Madre de Dios nos recuerda que tenemos Madre; no somos huérfanos, tenemos una Madre. Confesemos juntos esta verdad. Y los invito a aclamarla de pie (\emph{todos se alzan}) tres veces como lo hicieron los fieles de Éfeso: Santa Madre de Dios, Santa Madre de Dios, Santa Madre de Dios.
\end{body}

\subsubsection{Ángelus (2017)} 

\src{Plaza de San Pedro. \\Domingo 1 de enero del 2017.}

\begin{body}
	\emph{Queridos hermanos y hermanas, ¡buenos días!}
	
	\ltr{D}{urante} los días pasados hemos puesto nuestra mirada adorante sobre el Hijo de Dios, nacido en Belén; hoy, Solemnidad de María Santísima Madre de Dios, dirigimos nuestros ojos a la Madre, pero recibiendo a ambos con su estrecho vínculo. Este vínculo no se agota en el hecho de haber generado y en haber sido generado; Jesús ha \textquote{nacido de mujer} (\emph{Gal} 4, 4) para una misión de salvación y su madre no está excluida de tal misión, es más, está asociada íntimamente. María es consciente de esto, por lo tanto no se cierra a considerar sólo su relación maternal con Jesús, sino que permanece abierta y primorosa en todos los acontecimientos que suceden a su alrededor: conserva y medita, observa y profundiza, como nos recuerda el Evangelio de hoy (cf \emph{Lc} 2, 19). Ha dicho ya su \textquote{sí} y ha dado su disponibilidad para ser incluida en la aplicación del plan de salvación de Dios, que \textquote{dispersó a los que son soberbios en su propio corazón. Derribó a los potentados de sus tronos y exaltó a los humildes. A los hambrientos colmó de bienes y despidió a los ricos sin nada} (\emph{Lc} 1, 51-53). Ahora, silenciosa y atenta, intenta comprender qué quiere Dios de ella día a día. La visita de los pastores le ofrece la ocasión para percibir algún elemento de la voluntad de Dios que se manifiesta en la presencia de estas personas humildes y pobres. El evangelista Lucas nos narra la visita de los pastores a la gruta con un rápido sucederse de verbos que expresan movimiento. Dice así: ellos van sin demora, encuentran al Niño con María y José, lo ven, y cuentan lo que les ha sido dicho por Él, y al final glorifican a Dios (cf \emph{Lc} 2, 16-20). María sigue atentamente esta escena, qué dicen los pastores, qué les ha ocurrido, por qué en ello ya se discierne el movimiento de salvación que surgirá de la obra de Jesús, y se adapta, preparada ante toda petición del Señor. Dios pide a María no sólo ser la madre de su Hijo unigénito, sino también cooperar con el Hijo y por el Hijo en su plan de salvación, para que en ella, humilde sierva, se cumplan las grandes obras de la misericordia divina.
	
	Por ello, mientras, así como los pastores, contemplan el icono del Niño en brazos de su Madre, sentimos crecer en nuestro corazón un sentido de inmenso agradecimiento hacia quien ha dado al mundo al Salvador. Por ello, en el primer día de un año nuevo, le decimos:
	
	Gracias, oh Santa Madre del Hijo de Dios, Jesús, ¡Santa Madre de Dios!\\ Gracias por tu humildad que ha atraído la mirada de Dios;\\ gracias por la fe con la cual has acogido su Palabra;\\ gracias por la valentía con la cual has dicho \textquote{aquí estoy},\\ olvidada de si misma, fascinada por el Amor Santo, convertida en una única cosa junto con su esperanza.\\ Gracias, ¡oh Santa Madre de Dios!\\ Reza por nosotros, peregrinos del tiempo; ayúdanos a caminar por la vía de la paz. Amén.
\end{body}

\subsubsection{Homilía (2020): La Iglesia es Madre, como María}

\src{Basílica Vaticana. 1 de enero del 2020.}

\begin{body}
	\textquote{Cuando llegó la plenitud del tiempo, envió Dios a su Hijo, nacido de mujer} (\emph{Ga} 4,4). 
	
	\ltr{N}{acido} de mujer: así es cómo vino Jesús. No apareció en el mundo como adulto, sino como nos ha dicho el \textbf{Evangelio}, fue \textquote{concebido} en el vientre (\emph{Lc} 2,21): allí hizo suya nuestra humanidad, día tras día, mes tras mes. En el vientre de una mujer, Dios y la humanidad se unieron para no separarse nunca más. También ahora, en el cielo, Jesús vive en la carne que tomó en el vientre de su madre. En Dios está nuestra carne humana.
	
	El primer día del año celebramos estos desposorios entre Dios y el hombre, inaugurados en el vientre de una mujer. En Dios estará para siempre nuestra humanidad y María será la Madre de Dios para siempre. Ella es mujer y madre, esto es lo esencial. De ella, mujer, surgió la salvación y, por lo tanto, no hay salvación sin la mujer. Allí Dios se unió con nosotros y, si queremos unirnos con Él, debemos ir por el mismo camino: a través de María, mujer y madre. Por ello, comenzamos el año bajo el signo de Nuestra Señora, la mujer que tejió la humanidad de Dios. Si queremos tejer con humanidad las tramas de nuestro tiempo, debemos partir de nuevo de la mujer.
	
	\emph{Nacido de mujer}. El renacer de la humanidad comenzó con la mujer. Las mujeres son fuente de vida. Sin embargo, son continuamente ofendidas, golpeadas, violadas, inducidas a prostituirse y a eliminar la vida que llevan en el vientre. Toda violencia infligida a la mujer es una profanación de Dios, nacido de una mujer. La salvación para la humanidad vino del cuerpo de una mujer: de cómo tratamos el cuerpo de la mujer comprendemos nuestro nivel de humanidad. Cuántas veces el cuerpo de la mujer se sacrifica en los altares profanos de la publicidad, del lucro, de la pornografía, explotado como un terreno para utilizar. Debe ser liberado del consumismo, debe ser respetado y honrado. Es la carne más noble del mundo, pues concibió y dio a luz al Amor que nos ha salvado. Hoy, la maternidad también es humillada, porque el único crecimiento que interesa es el económico. Hay madres que se arriesgan a emprender viajes penosos para tratar desesperadamente de dar un futuro mejor al fruto de sus entrañas, y que son consideradas como números que sobrexceden el cupo por personas que tienen el estómago lleno, pero de cosas, y el corazón vacío de amor.
	
	\emph{Nacido de mujer}. Según la narración bíblica, la mujer aparece en el ápice de la creación, como resumen de todo lo creado. De hecho, ella contiene en sí el fin de la creación misma: la generación y protección de la vida, la comunión con todo, el ocuparse de todo. Es lo que hace la Virgen en el \textbf{Evangelio} hoy. \textquote{María, por su parte ―dice el texto―, conservaba todas estas cosas, meditándolas en su corazón} (v. 19). Conservaba todo: la alegría por el nacimiento de Jesús y la tristeza por la hospitalidad negada en Belén; el amor de José y el asombro de los pastores; las promesas y las incertidumbres del futuro. Todo lo tomaba en serio y todo lo ponía en su lugar en su corazón, incluso la adversidad. Porque en su corazón arreglaba cada cosa con amor y confiaba todo a Dios.
	
	En el Evangelio encontramos por segunda vez esta acción de María: al final de la vida oculta de Jesús se dice, en efecto, que \textquote{su madre conservaba todo esto en su corazón} (v. 51). Esta repetición nos hace comprender que conservar en el corazón no es un buen gesto que la Virgen hizo de vez en cuando, sino un hábito. Es propio de la mujer tomarse la vida en serio. La mujer manifiesta que el significado de la vida no es continuar a producir cosas, sino tomar en serio las que ya están. Sólo quien mira con el corazón ve bien, porque sabe \textquote{ver en profundidad} a la persona más allá de sus errores, al hermano más allá de sus fragilidades, la esperanza en medio de las dificultades; ve a Dios en todo.
	
	Al comenzar el nuevo año, preguntémonos: \textquote{¿Sé mirar con el corazón? ¿sé mirar con el corazón a las personas? ¿Me importa la gente con la que vivo, o la destruyo con la murmuración? Y, sobre todo, ¿tengo al Señor en el centro de mi corazón, o tengo otros valores, otros intereses, mi promoción, las riquezas, el poder?}. Sólo si la vida \emph{es importante} para nosotros sabremos \emph{cómo cuidarla} y superar la indiferencia que nos envuelve. Pidamos esta gracia: vivir el año con el deseo de tomar en serio a los demás, de cuidar a los demás. Y si queremos un mundo mejor, que sea una casa de paz y no un patio de batalla, que nos importe la dignidad de toda mujer. De una mujer nació el Príncipe de la paz. La mujer es donante y mediadora de paz y debe ser completamente involucrada en los procesos de toma de decisiones. Porque cuando las mujeres pueden transmitir sus dones, el mundo se encuentra más unido y más en paz. Por lo tanto, una conquista para la mujer es una conquista para toda la humanidad.
	
	\emph{Nacido de mujer}. Jesús, recién nacido, se reflejó en los ojos de una mujer, en el rostro de su madre. De ella recibió las primeras caricias, con ella intercambió las primeras sonrisas. Con ella inauguró la revolución de la ternura. La Iglesia, mirando al niño Jesús, está llamada a continuarla. De hecho, al igual que María, también ella es mujer y madre, la Iglesia es mujer y madre, y en la Virgen encuentra sus rasgos distintivos. La ve inmaculada, y se siente llamada a decir \textquote{no} al pecado y a la mundanidad. La ve fecunda y se siente llamada a anunciar al Señor, a generarlo en las vidas. La ve, madre, y se siente llamada a acoger a cada hombre como a un hijo.
	
	Acercándose a María, la Iglesia se encuentra a sí misma, encuentra su centro, encuentra su unidad. En cambio, el enemigo de la naturaleza humana, el diablo, trata de dividirla, poniendo en primer plano las diferencias, las ideologías, los pensamientos partidistas y los bandos. Pero no podemos entender a la Iglesia si la miramos a partir de sus estructuras, a partir de los programas y tendencias, de las ideologías, de las funcionalidades: percibiremos algo de ella, pero no el corazón de la Iglesia. Porque la Iglesia tiene el corazón de una madre. Y nosotros, hijos, invocamos hoy a la Madre de Dios, que nos reúne como pueblo creyente. Oh Madre, genera en nosotros la esperanza, tráenos la unidad. Mujer de la salvación, te confiamos este año, custódialo en tu corazón. Te aclamamos: ¡Santa Madre de Dios! Todos juntos, por tres veces, aclamemos a la Señora, en pie, Nuestra Señora, la Santa Madre de Dios: {[}con la asamblea{]}: ¡Santa Madre de Dios, Santa Madre de Dios!
\end{body}


\subsubsection{Ángelus (2020)} 

\src{Plaza de San Pedro. LIII Jornada Mundial de la Paz. \\Miércoles, 1 de enero de 2020.}

\begin{body}
	\emph{Queridos hermanos y hermanas, ¡buenos días! ¡Y Feliz Año Nuevo!}
	
	\ltr{A}{noche} terminamos el año 2019 agradeciendo a Dios por el don del tiempo y por todos sus beneficios. Hoy comenzamos el año 2020 con la misma actitud de \emph{gratitud} y \emph{alabanza}. No se da por sentado que nuestro planeta ha comenzado una nueva vuelta alrededor del sol y que los seres humanos seguiremos viviendo en él. No se da por sentado, al contrario, siempre es un \textquote{milagro} del que sorprenderse y estar agradecido.
	
	El primer día del año la liturgia celebra a la Santa Madre de Dios, María, la Virgen de Nazaret que dio a luz a Jesús, el Salvador. Ese Niño es la \emph{bendición de Dios} para cada hombre y mujer, para la gran familia humana y para el mundo entero. Jesús no eliminó el mal del mundo, sino que lo derrotó en su raíz. Su salvación no es mágica, sino que es una salvación \textquote{paciente}, es decir, implica la paciencia del amor, que se responsabiliza de la iniquidad y le quita su poder. La paciencia del amor: el amor nos hace pacientes. Muchas veces perdemos la paciencia; yo también, y pido disculpas por el mal ejemplo de ayer {[}se refiere a la reacción que tuvo con una persona que le tiró bruscamente del brazo en la plaza de San Pedro{]}. Por eso, contemplando el Pesebre vemos, con los ojos de la fe, el mundo renovado, liberado del dominio del mal y puesto bajo el señorío real de Cristo, el Niño acostado en el pesebre.
	
	Por eso hoy la Madre de Dios nos bendice. ¿Y cómo nos bendice la Virgen? Mostrándonos al Hijo. Lo toma en sus brazos y nos lo muestra, y así nos bendice. Bendice a toda la Iglesia, bendice al mundo entero. Jesús, como cantaban los ángeles en Belén, es la \textquote{alegría de todo el pueblo}, es la gloria de Dios y la paz para la humanidad (cf. \emph{Lucas} 2, 14). Por eso el santo Papa Pablo VI quiso dedicar el primer día del año a la paz ―es la Jornada de la Paz―, a la oración, a la conciencia y a la responsabilidad por la paz. Para este año 2020 el Mensaje es así: la paz es un \emph{camino de esperanza}, un camino en el que se avanza a través del \emph{diálogo}, la \emph{reconciliación} y la \emph{conversión ecológica}.
	
	Por lo tanto, fijemos la mirada en la Madre y en el Hijo que nos muestra. Al comienzo del año, ¡seamos bendecidos! Dejémonos bendecir por la Virgen con su Hijo.
	
	Jesús es la bendición para aquellos que están oprimidos por el yugo de la esclavitud, la esclavitud moral y la esclavitud material. Él libera con amor. A los que han perdido la autoestima por permanecer prisioneros de círculos viciosos, Jesús les dice: el Padre os ama, no os abandona, espera con una paciencia inquebrantable vuestro regreso (cf. \emph{Lucas} 15, 20). A los que son víctimas de la injusticia y la explotación y no ven la salida, Jesús les abre la puerta de la fraternidad, donde pueden encontrar rostros, corazones y manos acogedores, donde pueden compartir la amargura y la desesperación, y recuperar algo de dignidad. A los que están gravemente enfermos y se sienten abandonados y desanimados, Jesús se acerca, toca con ternura las heridas, derrama el aceite del consuelo y transforma la debilidad en fuerza del bien para desatar los nudos más enredados. A los que están encarcelados y son tentados a encerrarse en sí mismos, Jesús les vuelve a abrir un horizonte de esperanza, empezando por un pequeño rayo de luz.
	
	Queridos hermanos y hermanas, bajemos de los pedestales de nuestro orgullo ―todos tenemos la tentación del orgullo― y pidamos la bendición de la Santa Madre de Dios, la humilde Madre de Dios. Ella nos muestra a Jesús: seamos bendecidos, abramos nuestros corazones a su bondad. Así, el año que comienza será un camino de esperanza y paz, no con palabras, sino con gestos cotidianos de diálogo, reconciliación y cuidado de la creación.
\end{body}


\newsection

\section{Temas}
%01 Ciclo | 02 Tiempo | 03 Semana

\cceth{Jesucristo, verdadero Dios y verdadero hombre}

\cceref{CEC 464-469}


\begin{ccebody}
	\n{464} El acontecimiento único y totalmente singular de la Encarnación del Hijo de Dios no significa que Jesucristo sea en parte Dios y en parte hombre, ni que sea el resultado de una mezcla confusa entre lo divino y lo humano. Él se hizo verdaderamente hombre sin dejar de ser verdaderamente Dios. Jesucristo es verdadero Dios y verdadero hombre. La Iglesia debió defender y aclarar esta verdad de fe durante los primeros siglos frente a unas herejías que la falseaban.
	
	\n{465} Las primeras herejías negaron menos la divinidad de Jesucristo que su humanidad verdadera (docetismo gnóstico). Desde la época apostólica la fe cristiana insistió en la verdadera encarnación del Hijo de Dios, \textquote{venido en la carne} (cf. \emph{1 Jn} 4, 2-3; \emph{2 Jn} 7). Pero desde el siglo III, la Iglesia tuvo que afirmar frente a Pablo de Samosata, en un Concilio reunido en Antioquía, que Jesucristo es Hijo de Dios por naturaleza y no por adopción. El primer Concilio Ecuménico de Nicea, en el año 325, confesó en su Credo que el Hijo de Dios es \textquote{engendrado, no creado, \textquote{de la misma substancia} {[}en griego \emph{homousion}{]} que el Padre} y condenó a Arrio que afirmaba que \textquote{el Hijo de Dios salió de la nada} (Concilio de Nicea I: DS 130) y que sería \textquote{de una substancia distinta de la del Padre} (\emph{Ibíd}., 126).
	
	\n{466} La herejía nestoriana veía en Cristo una persona humana junto a la persona divina del Hijo de Dios. Frente a ella san Cirilo de Alejandría y el tercer Concilio Ecuménico reunido en Efeso, en el año 431, confesaron que \textquote{el Verbo, al unirse en su persona a una carne animada por un alma racional, se hizo hombre} (Concilio de Efeso: DS, 250). La humanidad de Cristo no tiene más sujeto que la persona divina del Hijo de Dios que la ha asumido y hecho suya desde su concepción. Por eso el concilio de Efeso proclamó en el año 431 que María llegó a ser con toda verdad Madre de Dios mediante la concepción humana del Hijo de Dios en su seno: \textquote{Madre de Dios, no porque el Verbo de Dios haya tomado de ella su naturaleza divina, sino porque es de ella, de quien tiene el cuerpo sagrado dotado de un alma racional [\ldots{}] unido a la persona del Verbo, de quien se dice que el Verbo nació según la carne} (DS 251).
	
	\n{467} Los monofisitas afirmaban que la naturaleza humana había dejado de existir como tal en Cristo al ser asumida por su persona divina de Hijo de Dios. Enfrentado a esta herejía, el cuarto Concilio Ecuménico, en Calcedonia, confesó en el año 451:

\begin{quote}
\textquote{Siguiendo, pues, a los Santos Padres, enseñamos unánimemente que hay que confesar a un solo y mismo Hijo y Señor nuestro Jesucristo: perfecto en la divinidad, y perfecto en la humanidad; verdaderamente Dios y verdaderamente hombre compuesto de alma racional y cuerpo; consubstancial con el Padre según la divinidad, y consubstancial con nosotros según la humanidad, \textquote{en todo semejante a nosotros, excepto en el pecado} (\emph{Hb} 4, 15); nacido del Padre antes de todos los siglos según la divinidad; y por nosotros y por nuestra salvación, nacido en los últimos tiempos de la Virgen María, la Madre de Dios, según la humanidad.
		
Se ha de reconocer a un solo y mismo Cristo Señor, Hijo único en dos naturalezas, sin confusión, sin cambio, sin división, sin separación. La diferencia de naturalezas de ningún modo queda suprimida por su unión, sino que quedan a salvo las propiedades de cada una de las naturalezas y confluyen en un solo sujeto y en una sola persona} (Concilio de Calcedonia; DS, 301-302).
\end{quote}	
	
	\n{468} Después del Concilio de Calcedonia, algunos concibieron la naturaleza humana de Cristo como una especie de sujeto personal. Contra éstos, el quinto Concilio Ecuménico, en Constantinopla, el año 553 confesó a propósito de Cristo: \textquote{No hay más que una sola hipóstasis {[}o persona{]} [\ldots{}] que es nuestro Señor Jesucristo, \emph{uno de la Trinidad}} (Concilio de Constantinopla II: DS, 424). Por tanto, todo en la humanidad de Jesucristo debe ser atribuido a su persona divina como a su propio sujeto (cf. ya Concilio de Éfeso: DS, 255), no solamente los milagros sino también los sufrimientos (cf. Concilio de Constantinopla II: DS, 424) y la misma muerte: \textquote{El que ha sido crucificado en la carne, nuestro Señor Jesucristo, es verdadero Dios, Señor de la gloria y uno de la Santísima Trinidad} (\emph{ibíd}., 432).	
	
	\n{469} La Iglesia confiesa así que Jesús es inseparablemente verdadero Dios y verdadero Hombre. Él es verdaderamente el Hijo de Dios que se ha hecho hombre, nuestro hermano, y eso sin dejar de ser Dios, nuestro Señor:

\begin{quote}
	\emph{Id quod fuit remansit et quod non fuit assumpsit} (\textquote{Sin dejar de ser lo que era ha asumido lo que no era}), canta la liturgia romana (\emph{Solemnidad de la Santísima Virgen María, Madre de Dios}, Antífona al \textquote{Benedictus}; cf. san León Magno, \emph{Sermones} 21, 2-3: PL 54, 192). Y la liturgia de san Juan Crisóstomo proclama y canta: \textquote{¡Oh Hijo unigénito y Verbo de Dios! Tú que eres inmortal, te dignaste, para salvarnos, tomar carne de la santa Madre de Dios y siempre Virgen María. Tú, Cristo Dios, sin sufrir cambio te hiciste hombre y, en la cruz, con tu muerte venciste la muerte. Tú, Uno de la Santísima Trinidad, glorificado con el Padre y el Santo Espíritu, ¡sálvanos! (\emph{Oficio Bizantino de las Horas, Himno O'Monogenés}}).
\end{quote}										
\end{ccebody}


\cceth{María es la Madre de Dios}

\cceref{CEC 495, 2677}

\begin{ccebody}
	\ccesec{La maternidad divina de María}
	
\n{495} Llamada en los Evangelios \textquote{la Madre de Jesús} (Jn 2, 1; 19, 25; cf. Mt 13, 55, etc.), María es aclamada bajo el impulso del Espíritu como \textquote{la madre de mi Señor} desde antes del nacimiento de su hijo (cf. Lc 1, 43). En efecto, aquél que ella concibió como hombre, por obra del Espíritu Santo, y que se ha hecho verdaderamente su Hijo según la carne, no es otro que el Hijo eterno del Padre, la segunda persona de la Santísima Trinidad. La Iglesia confiesa que María es verdaderamente \emph{Madre de Dios} {[}\textquote{Theotokos}{]} (cf. DS 251).

	
	\n{2677} \emph{\textquote{Santa María, Madre de Dios, ruega por nosotros\ldots{} }} Con Isabel, nos maravillamos y decimos: \textquote{¿De dónde a mí que la madre de mi Señor venga a mí?} (\emph{Lc} 1, 43). Porque nos da a Jesús su hijo, María es madre de Dios y madre nuestra; podemos confiarle todos nuestros cuidados y nuestras peticiones: ora por nosotros como oró por sí misma: \textquote{Hágase en mí según tu palabra} (\emph{Lc} 1, 38). Confiándonos a su oración, nos abandonamos con ella en la voluntad de Dios: \textquote{Hágase tu voluntad}.

\textquote{\emph{Ruega por nosotros, pecadores, ahora y en la hora de nuestra muerte}}. Pidiendo a María que ruegue por nosotros, nos reconocemos pecadores y nos dirigimos a la \textquote{Madre de la Misericordia}, a la Toda Santa. Nos ponemos en sus manos \textquote{ahora}, en el hoy de nuestras vidas. Y nuestra confianza se ensancha para entregarle desde ahora, \textquote{la hora de nuestra muerte}. Que esté presente en esa hora, como estuvo en la muerte en Cruz de su Hijo, y que en la hora de nuestro tránsito nos acoja como madre nuestra (cf. \emph{Jn} 19, 27) para conducirnos a su Hijo Jesús, al Paraíso.
	
\end{ccebody}


\cceth{Nuestra adopción como hijos de Dios}

\cceref{CEC 1, 52, 270, 294, 422, 654, 1709, 2009}

\begin{ccebody}
		\n{668} \textquote{Cristo murió y volvió a la vida para eso, para ser Señor de muertos y vivos} (\emph{Rm} 14, 9). La Ascensión de Cristo al Cielo significa su participación, en su humanidad, en el poder y en la autoridad de Dios mismo. Jesucristo es Señor: posee todo poder en los cielos y en la tierra. El está \textquote{por encima de todo principado, potestad, virtud, dominación} porque el Padre \textquote{bajo sus pies sometió todas las cosas} (\emph{Ef} 1, 20-22). Cristo es el Señor del cosmos (cf. \emph{Ef} 4, 10; \emph{1 Co} 15, 24. 27-28) y de la historia. En Él, la historia de la humanidad e incluso toda la Creación encuentran su recapitulación (\emph{Ef} 1, 10), su cumplimiento transcendente.
	
	\n{52} Dios, que \textquote{habita una luz inaccesible} (\emph{1 Tm} 6,16) quiere comunicar su propia vida divina a los hombres libremente creados por él, para hacer de ellos, en su Hijo único, hijos adoptivos (cf. \emph{Ef} 1,4-5). Al revelarse a sí mismo, Dios quiere hacer a los hombres capaces de responderle, de conocerle y de amarle más allá de lo que ellos serían capaces por sus propias fuerzas.
	
	\ccesec{\textquote{Te compadeces de todos porque lo puedes todo} (\emph{Sb} 11, 23)}

\n{270} Dios es el \emph{Padre} todopoderoso. Su paternidad y su poder se esclarecen mutuamente. Muestra, en efecto, su omnipotencia paternal por la manera como cuida de nuestras necesidades (cf. \emph{Mt} 6,32); por la adopción filial que nos da (\textquote{Yo seré para vosotros padre, y vosotros seréis para mí hijos e hijas, dice el Señor todopoderoso}: \emph{2 Co} 6,18); finalmente, por su misericordia infinita, pues muestra su poder en el más alto grado perdonando libremente los pecados.
	
	\n{294} La gloria de Dios consiste en que se realice esta manifestación y esta comunicación de su bondad para las cuales el mundo ha sido creado. Hacer de nosotros \textquote{hijos adoptivos por medio de Jesucristo, según el beneplácito de su voluntad, \emph{para alabanza de la gloria} de su gracia} (\emph{Ef} 1,5-6): \textquote{Porque la gloria de Dios es que el hombre viva, y la vida del hombre es la visión de Dios: si ya la revelación de Dios por la creación procuró la vida a todos los seres que viven en la tierra, cuánto más la manifestación del Padre por el Verbo procurará la vida a los que ven a Dios} (San Ireneo de Lyon, \emph{Adversus haereses}, 4,20,7). El fin último de la creación es que Dios, \textquote{Creador de todos los seres, sea por fin \textquote{todo en todas las cosas} (\emph{1 Co} 15,28), \emph{procurando al mismo tiempo su gloria y nuestra felicidad}} (AG 2).
	
	\ccesec{La Buena Nueva: Dios ha enviado a su Hijo}

\n{422} \textquote{Pero, al llegar la plenitud de los tiempos, envió Dios a su Hijo, nacido de mujer, nacido bajo la Ley, para rescatar a los que se hallaban bajo la Ley, y para que recibiéramos la filiación adoptiva} (\emph{Ga} 4, 4-5). He aquí \textquote{la Buena Nueva de Jesucristo, Hijo de Dios} (\emph{Mc} 1, 1): Dios ha visitado a su pueblo (cf. \emph{Lc} 1, 68), ha cumplido las promesas hechas a Abraham y a su descendencia (cf. \emph{Lc} 1, 55); lo ha hecho más allá de toda expectativa: Él ha enviado a su \textquote{Hijo amado} (\emph{Mc} 1, 11).

	
	\n{654} Hay un doble aspecto en el misterio pascual: por su muerte nos libera del pecado, por su Resurrección nos abre el acceso a una nueva vida. Esta es, en primer lugar, la \emph{justificación} que nos devuelve a la gracia de Dios (cf. \emph{Rm} 4, 25) \textquote{a fin de que, al igual que Cristo fue resucitado de entre los muertos [\ldots{}] así también nosotros vivamos una nueva vida} (\emph{Rm} 6, 4). Consiste en la victoria sobre la muerte y el pecado y en la nueva participación en la gracia (cf. \emph{Ef} 2, 4-5; \emph{1 P} 1, 3). Realiza la \emph{adopción filial} porque los hombres se convierten en hermanos de Cristo, como Jesús mismo llama a sus discípulos después de su Resurrección: \textquote{Id, avisad a mis hermanos} (\emph{Mt} 28, 10; \emph{Jn} 20, 17). Hermanos no por naturaleza, sino por don de la gracia, porque esta filiación adoptiva confiere una participación real en la vida del Hijo único, la que ha revelado plenamente en su Resurrección.
	
	\n{1709} El que cree en Cristo es hecho hijo de Dios. Esta adopción filial lo transforma dándole la posibilidad de seguir el ejemplo de Cristo. Le hace capaz de obrar rectamente y de practicar el bien. En la unión con su Salvador, el discípulo alcanza la perfección de la caridad, la santidad. La vida moral, madurada en la gracia, culmina en vida eterna, en la gloria del cielo.
	
	\n{2009} La adopción filial, haciéndonos partícipes por la gracia de la naturaleza divina, puede conferirnos, según la justicia gratuita de Dios, \emph{un verdadero mérito}. Se trata de un derecho por gracia, el pleno derecho del amor, que nos hace \textquote{coherederos} de Cristo y dignos de obtener la herencia prometida de la vida eterna (cf. Concilio de Trento: DS 1546). Los méritos de nuestras buenas obras son dones de la bondad divina (cf. Concilio de Trento: DS 1548). \textquote{La gracia ha precedido; ahora se da lo que es debido [\ldots{}] Los méritos son dones de Dios} (San Agustín, \emph{Sermo} 298, 4-5).		
\end{ccebody}


\cceth{Jesús observa la Ley y la perfecciona}

\cceref{CEC 527, 577-582}


\begin{ccebody}
	\n{527} La \emph{Circuncisión} de Jesús, al octavo día de su nacimiento (cf. \emph{Lc} 2, 21) es señal de su inserción en la descendencia de Abraham, en el pueblo de la Alianza, de su sometimiento a la Ley (cf. \emph{Ga} 4, 4) y de su consagración al culto de Israel en el que participará durante toda su vida. Este signo prefigura \textquote{la circuncisión en Cristo} que es el Bautismo (\emph{Col} 2, 11-13).
	
	\ccesec{Jesús y la Ley}

\n{577} Al comienzo del Sermón de la Montaña, Jesús hace una advertencia solemne presentando la Ley dada por Dios en el Sinaí con ocasión de la Primera Alianza, a la luz de la gracia de la Nueva Alianza:

\begin{quote}
	\textquote{No penséis que he venido a abolir la Ley y los Profetas. No he venido a abolir sino a dar cumplimiento. Sí, os lo aseguro: el cielo y la tierra pasarán antes que pase una \textquote{i} o un ápice de la Ley sin que todo se haya cumplido. Por tanto, el que quebrante uno de estos mandamientos menores, y así lo enseñe a los hombres, será el menor en el Reino de los cielos; en cambio el que los observe y los enseñe, ése será grande en el Reino de los cielos} (\emph{Mt} 5, 17-19).
\end{quote}


	

	\n{578} Jesús, el Mesías de Israel, por lo tanto el más grande en el Reino de los cielos, se debía sujetar a la Ley cumpliéndola en su totalidad hasta en sus menores preceptos, según sus propias palabras. Incluso es el único en poderlo hacer perfectamente (cf. \emph{Jn} 8, 46). Los judíos, según su propia confesión, jamás han podido cumplir la Ley en su totalidad, sin violar el menor de sus preceptos (cf. \emph{Jn} 7, 19; \emph{Hch} 13, 38-41; 15, 10). Por eso, en cada fiesta anual de la Expiación, los hijos de Israel piden perdón a Dios por sus transgresiones de la Ley. En efecto, la Ley constituye un todo y, como recuerda Santiago, \textquote{quien observa toda la Ley, pero falta en un solo precepto, se hace reo de todos} (\emph{St} 2, 10; cf. \emph{Ga} 3, 10; 5, 3).	

	\n{579} Este principio de integridad en la observancia de la Ley, no sólo en su letra sino también en su espíritu, era apreciado por los fariseos. Al subrayarlo para Israel, muchos judíos del tiempo de Jesús fueron conducidos a un celo religioso extremo (cf. \emph{Rm} 10, 2), el cual, si no quería convertirse en una casuística \textquote{hipócrita} (cf. \emph{Mt} 15, 3-7; \emph{Lc} 11, 39-54) no podía más que preparar al pueblo a esta intervención inaudita de Dios que será la ejecución perfecta de la Ley por el único Justo en lugar de todos los pecadores (cf. \emph{Is} 53, 11; \emph{Hb} 9, 15).	
	
	\n{580} El cumplimiento perfecto de la Ley no podía ser sino obra del divino Legislador que nació sometido a la Ley en la persona del Hijo (cf. \emph{Ga} 4, 4). En Jesús la Ley ya no aparece grabada en tablas de piedra sino \textquote{en el fondo del corazón} (\emph{Jr} 31, 33) del Siervo, quien, por \textquote{aportar fielmente el derecho} (\emph{Is} 42, 3), se ha convertido en \textquote{la Alianza del pueblo} (\emph{Is} 42, 6). Jesús cumplió la Ley hasta tomar sobre sí mismo \textquote{la maldición de la Ley} (\emph{Ga} 3, 13) en la que habían incurrido los que no \textquote{practican todos los preceptos de la Ley} (\emph{Ga} 3, 10) porque \textquote{ha intervenido su muerte para remisión de las transgresiones de la Primera Alianza} (\emph{Hb} 9, 15).	
	
	\n{581} Jesús fue considerado por los judíos y sus jefes espirituales como un \textquote{rabbi} (cf. \emph{Jn} 11, 28; 3, 2; \emph{Mt} 22, 23-24, 34-36). Con frecuencia argumentó en el marco de la interpretación rabínica de la Ley (cf. \emph{Mt} 12, 5; 9, 12; \emph{Mc} 2, 23-27; \emph{Lc} 6, 6-9; \emph{Jn} 7, 22-23). Pero al mismo tiempo, Jesús no podía menos que chocar con los doctores de la Ley porque no se contentaba con proponer su interpretación entre los suyos, sino que \textquote{enseñaba como quien tiene autoridad y no como los escribas} (\emph{Mt} 7, 28-29). La misma Palabra de Dios, que resonó en el Sinaí para dar a Moisés la Ley escrita, es la que en Él se hace oír de nuevo en el Monte de las Bienaventuranzas (cf. \emph{Mt} 5, 1). Esa palabra no revoca la Ley sino que la perfecciona aportando de modo divino su interpretación definitiva: \textquote{Habéis oído también que se dijo a los antepasados [\ldots{}] pero yo os digo} (\emph{Mt} 5, 33-34). Con esta misma autoridad divina, desaprueba ciertas \textquote{tradiciones humanas} (\emph{Mc} 7, 8) de los fariseos que \textquote{anulan la Palabra de Dios} (\emph{Mc} 7, 13).
	
	\n{582} Yendo más lejos, Jesús da plenitud a la Ley sobre la pureza de los alimentos, tan importante en la vida cotidiana judía, manifestando su sentido \textquote{pedagógico} (cf. \emph{Ga} 3, 24) por medio de una interpretación divina: \textquote{Todo lo que de fuera entra en el hombre no puede hacerle impuro [\ldots{}] ---así declaraba puros todos los alimentos---. Lo que sale del hombre, eso es lo que hace impuro al hombre. Porque de dentro, del corazón de los hombres, salen las intenciones malas} (\emph{Mc} 7, 18-21). Jesús, al dar con autoridad divina la interpretación definitiva de la Ley, se vio enfrentado a algunos doctores de la Ley que no aceptaban su interpretación a pesar de estar garantizada por los signos divinos con que la acompañaba (cf. \emph{Jn} 5, 36; 10, 25. 37-38; 12, 37). Esto ocurre, en particular, respecto al problema del sábado: Jesús recuerda, frecuentemente con argumentos rabínicos (cf. \emph{Mt} 2,25-27; \emph{Jn} 7, 22-24), que el descanso del sábado no se quebranta por el servicio de Dios (cf. \emph{Mt} 12, 5; \emph{Nm} 28, 9) o al prójimo (cf. \emph{Lc} 13, 15-16; 14, 3-4) que realizan sus curaciones.							
\end{ccebody}


\cceth{La Ley nueva nos libera de las restricciones de la Ley antigua}

\cceref{CEC 580, 1972}


\begin{ccebody}
	\n{580} El cumplimiento perfecto de la Ley no podía ser sino obra del divino Legislador que nació sometido a la Ley en la persona del Hijo (cf. \emph{Ga} 4, 4). En Jesús la Ley ya no aparece grabada en tablas de piedra sino \textquote{en el fondo del corazón} (\emph{Jr} 31, 33) del Siervo, quien, por \textquote{aportar fielmente el derecho} (\emph{Is} 42, 3), se ha convertido en \textquote{la Alianza del pueblo} (\emph{Is} 42, 6). Jesús cumplió la Ley hasta tomar sobre sí mismo \textquote{la maldición de la Ley} (\emph{Ga} 3, 13) en la que habían incurrido los que no \textquote{practican todos los preceptos de la Ley} (\emph{Ga} 3, 10) porque \textquote{ha intervenido su muerte para remisión de las transgresiones de la Primera Alianza} (\emph{Hb} 9, 15).
	
	\n{1972} La Ley nueva es llamada \emph{ley de amor}, porque hace obrar por el amor que infunde el Espíritu Santo más que por el temor; \emph{ley de gracia}, porque confiere la fuerza de la gracia para obrar mediante la fe y los sacramentos; \emph{ley de libertad} (cf. \emph{St} 1, 25; 2, 12), porque nos libera de las observancias rituales y jurídicas de la Ley antigua, nos inclina a obrar espontáneamente bajo el impulso de la caridad y nos hace pasar de la condición del siervo \textquote{que ignora lo que hace su señor}, a la de amigo de Cristo, \textquote{porque todo lo que he oído a mi Padre os lo he dado a conocer} (\emph{Jn} 15, 15), o también a la condición de hijo heredero (cf. \emph{Ga} 4, 1-7. 21-31; \emph{Rm} 8, 15).								
\end{ccebody}



\cceth{Por medio del Espíritu Santo podemos llamar a Dios \textquote{Abba}}

\cceref{CEC 683, 689, 1695, 2766, 2777-2778}


\begin{ccebody}
	\n{683} \textquote{Nadie puede decir: \textquote{¡Jesús es Señor!} sino por influjo del Espíritu Santo} (\emph{1 Co} 12, 3). \textquote{Dios ha enviado a nuestros corazones el Espíritu de su Hijo que clama ¡\emph{Abbá}, Padre!} (\emph{Ga} 4, 6). Este conocimiento de fe no es posible sino en el Espíritu Santo. Para entrar en contacto con Cristo, es necesario primeramente haber sido atraído por el Espíritu Santo. Él es quien nos precede y despierta en nosotros la fe. Mediante el Bautismo, primer sacramento de la fe, la vida, que tiene su fuente en el Padre y se nos ofrece por el Hijo, se nos comunica íntima y personalmente por el Espíritu Santo en la Iglesia:

\begin{quote}
	El Bautismo \textquote{nos da la gracia del nuevo nacimiento en Dios Padre por medio de su Hijo en el Espíritu Santo. Porque los que son portadores del Espíritu de Dios son conducidos al Verbo, es decir al Hijo; pero el Hijo los presenta al Padre, y el Padre les concede la incorruptibilidad. Por tanto, sin el Espíritu no es posible ver al Hijo de Dios, y, sin el Hijo, nadie puede acercarse al Padre, porque el conocimiento del Padre es el Hijo, y el conocimiento del Hijo de Dios se logra por el Espíritu Santo} (San Ireneo de Lyon, \emph{Demonstratio praedicationis apostolicae}, 7: SC 62 41-42).
\end{quote}


	
	\ccesec{La misión conjunta del Hijo y del Espíritu Santo}

\n{689} Aquel al que el Padre ha enviado a nuestros corazones, el Espíritu de su Hijo (cf. \emph{Ga} 4, 6) es realmente Dios. Consubstancial con el Padre y el Hijo, es inseparable de ellos, tanto en la vida íntima de la Trinidad como en su don de amor para el mundo. Pero al adorar a la Santísima Trinidad vivificante, consubstancial e indivisible, la fe de la Iglesia profesa también la distinción de las Personas. Cuando el Padre envía su Verbo, envía también su Aliento: misión conjunta en la que el Hijo y el Espíritu Santo son distintos pero inseparables. Sin ninguna duda, Cristo es quien se manifiesta, Imagen visible de Dios invisible, pero es el Espíritu Santo quien lo revela.	
	
	\n{1695} \textquote{Justificados [\ldots{}] en el nombre del Señor Jesucristo y en el Espíritu de nuestro Dios} (\emph{1 Co} 6,11.), \textquote{santificados y llamados a ser santos} (\emph{1 Co} 1,2.), los cristianos se convierten en \textquote{el templo [\ldots{}] del Espíritu Santo} (cf. \emph{1 Co} 6,19). Este \textquote{Espíritu del Hijo} les enseña a orar al Padre (\emph{Ga} 4, 6) y, haciéndose vida en ellos, les hace obrar (cf. \emph{Ga} 5, 25) para dar \textquote{los frutos del Espíritu} (\emph{Ga} 5, 22.) por la caridad operante. Sanando las heridas del pecado, el Espíritu Santo nos renueva interiormente mediante una transformación espiritual (cf. \emph{Ef} 4, 23.), nos ilumina y nos fortalece para vivir como \textquote{hijos de la luz} (\emph{Ef} 5, 8.), \textquote{por la bondad, la justicia y la verdad} en todo (\emph{Ef} 5,9.).	
	
	\n{2766} Pero Jesús no nos deja una fórmula para repetirla de modo mecánico (cf. \emph{Mt} 6, 7; \emph{1 R} 18, 26-29). Como en toda oración vocal, el Espíritu Santo, a través de la Palabra de Dios, enseña a los hijos de Dios a hablar con su Padre. Jesús no sólo nos enseña las palabras de la oración filial, sino que nos da también el Espíritu por el que estas se hacen en nosotros \textquote{espíritu [\ldots{}] y vida} (\emph{Jn} 6, 63). Más todavía: la prueba y la posibilidad de nuestra oración filial es que el Padre \textquote{ha enviado [\ldots{}] a nuestros corazones el Espíritu de su Hijo que clama: \textquote{¡Abbá, Padre!}} (\emph{Ga} 4, 6). Ya que nuestra oración interpreta nuestros deseos ante Dios, es también \textquote{el que escruta los corazones}, el Padre, quien \textquote{conoce cuál es la aspiración del Espíritu, y que su intercesión en favor de los santos es según Dios} (\emph{Rm} 8, 27). La oración al Padre se inserta en la misión misteriosa del Hijo y del Espíritu.	
	
	\ccesec{Acercarse a Él con toda confianza}

\n{2777} En la liturgia romana, se invita a la asamblea eucarística a rezar el Padre Nuestro con una audacia filial; las liturgias orientales usan y desarrollan expresiones análogas: \textquote{Atrevernos con toda confianza}, \textquote{Haznos dignos de}. Ante la zarza ardiendo, se le dijo a Moisés: \textquote{No te acerques aquí. Quita las sandalias de tus pies} (\emph{Ex} 3, 5). Este umbral de la santidad divina, sólo lo podía franquear Jesús, el que \textquote{después de llevar a cabo la purificación de los pecados} (\emph{Hb} 1, 3), nos introduce en presencia del Padre: \textquote{Hénos aquí, a mí y a los hijos que Dios me dio} (\emph{Hb} 2, 13):

\begin{quote}
	\textquote{La conciencia que tenemos de nuestra condición de esclavos nos haría meternos bajo tierra, nuestra condición terrena se desharía en polvo, si la autoridad de nuestro mismo Padre y el Espíritu de su Hijo, no nos empujasen a proferir este grito: \textquote{Abbá, Padre} (\emph{Rm} 8, 15) \ldots{} ¿Cuándo la debilidad de un mortal se atrevería a llamar a Dios Padre suyo, sino solamente cuando lo íntimo del hombre está animado por el Poder de lo alto?} (San Pedro Crisólogo, \emph{Sermón} 71, 3).
\end{quote}

	
	
	\n{2778} Este poder del Espíritu que nos introduce en la Oración del Señor se expresa en las liturgias de Oriente y de Occidente con la bella palabra, típicamente cristiana: \emph{parrhesia}, simplicidad sin desviación, conciencia filial, seguridad alegre, audacia humilde, certeza de ser amado (cf. \emph{Ef} 3, 12; \emph{Hb} 3, 6; 4, 16; 10, 19; \emph{1 Jn} 2,28; 3, 21; 5, 14).
\end{ccebody}



\cceth{El nombre de Jesús}

\cceref{CEC 430-435, 2666-2668, 2812}


\begin{ccebody}
	\n{430} \emph{Jesús} quiere decir en hebreo: \textquote{Dios salva}. En el momento de la anunciación, el ángel Gabriel le dio como nombre propio el nombre de Jesús que expresa a la vez su identidad y su misión (cf. \emph{Lc} 1, 31). Ya que \textquote{¿quién puede perdonar pecados, sino sólo Dios?} (\emph{Mc} 2, 7), es Él quien, en Jesús, su Hijo eterno hecho hombre \textquote{salvará a su pueblo de sus pecados} (\emph{Mt} 1, 21). En Jesús, Dios recapitula así toda la historia de la salvación en favor de los hombres.
	
	\n{431} En la historia de la salvación, Dios no se ha contentado con librar a Israel de \textquote{la casa de servidumbre} (\emph{Dt} 5, 6) haciéndole salir de Egipto. Él lo salva además de su pecado. Puesto que el pecado es siempre una ofensa hecha a Dios (cf. \emph{Sal} 51, 6), sólo Él es quien puede absolverlo (cf. \emph{Sal} 51, 12). Por eso es por lo que Israel, tomando cada vez más conciencia de la universalidad del pecado, ya no podrá buscar la salvación más que en la invocación del nombre de Dios Redentor (cf. \emph{Sal} 79, 9).	
	
	\n{432} El nombre de Jesús significa que el Nombre mismo de Dios está presente en la Persona de su Hijo (cf. \emph{Hch} 5, 41; \emph{3 Jn} 7) hecho hombre para la Redención universal y definitiva de los pecados. Él es el Nombre divino, el único que trae la salvación (cf. \emph{Jn} 3, 18; \emph{Hch} 2, 21) y de ahora en adelante puede ser invocado por todos porque se ha unido a todos los hombres por la Encarnación (cf. \emph{Rm} 10, 6-13) de tal forma que \textquote{no hay bajo el cielo otro nombre dado a los hombres por el que nosotros debamos salvarnos} (\emph{Hch} 4, 12; cf. \emph{Hch} 9, 14; \emph{St} 2, 7).	
	
	\n{433} El Nombre de Dios Salvador era invocado una sola vez al año por el sumo sacerdote para la expiación de los pecados de Israel, cuando había asperjado el propiciatorio del Santo de los Santos con la sangre del sacrificio (cf. \emph{Lv} 16, 15-16; \emph{Si} 50, 20; \emph{Hb} 9, 7). El propiciatorio era el lugar de la presencia de Dios (cf. \emph{Ex} 25, 22; \emph{Lv} 16, 2; \emph{Nm} 7, 89; \emph{Hb} 9, 5). Cuando san Pablo dice de Jesús que \textquote{Dios lo exhibió como instrumento de propiciación por su propia sangre} (\emph{Rm} 3, 25) significa que en su humanidad \textquote{estaba Dios reconciliando al mundo consigo} (\emph{2 Co} 5, 19).	
	
	\n{434} La Resurrección de Jesús glorifica el Nombre de Dios \textquote{Salvador} (cf. \emph{Jn} 12, 28) porque de ahora en adelante, el Nombre de Jesús es el que manifiesta en plenitud el poder soberano del \textquote{Nombre que está sobre todo nombre} (\emph{Flp} 2, 9). Los espíritus malignos temen su Nombre (cf. \emph{Hch} 16, 16-18; 19, 13-16) y en su nombre los discípulos de Jesús hacen milagros (cf. \emph{Mc} 16, 17) porque todo lo que piden al Padre en su Nombre, Él se lo concede (\emph{Jn} 15, 16).	
	
	\n{435} El Nombre de Jesús está en el corazón de la plegaria cristiana. Todas las oraciones litúrgicas se acaban con la fórmula \emph{Per Dominum nostrum Jesum Christum\ldots{}} (\textquote{Por nuestro Señor Jesucristo\ldots{}}). El \textquote{Avemaría} culmina en \textquote{y bendito es el fruto de tu vientre, Jesús}. La oración del corazón, en uso en Oriente, llamada \textquote{oración a Jesús} dice: \textquote{Señor Jesucristo, Hijo de Dios, ten piedad de mí pecador}. Numerosos cristianos mueren, como santa Juana de Arco, teniendo en sus labios una única palabra: \textquote{Jesús}.
	
	\n{2666} Pero el Nombre que todo lo contiene es aquel que el Hijo de Dios recibe en su encarnación: JESÚS. El nombre divino es inefable para los labios humanos (cf. \emph{Ex} 3, 14; 33, 19-23), pero el Verbo de Dios, al asumir nuestra humanidad, nos lo entrega y nosotros podemos invocarlo: \textquote{Jesús}, \textquote{YHVH salva} (cf. \emph{Mt} 1, 21). El Nombre de Jesús contiene todo: Dios y el hombre y toda la Economía de la creación y de la salvación. Decir \textquote{Jesús} es invocarlo desde nuestro propio corazón. Su Nombre es el único que contiene la presencia que significa. Jesús es el resucitado, y cualquiera que invoque su Nombre acoge al Hijo de Dios que le amó y se entregó por él (cf. \emph{Rm} 10, 13; \emph{Hch} 2, 21; 3, 15-16; \emph{Ga} 2, 20).

	\n{2667} Esta invocación de fe bien sencilla ha sido desarrollada en la tradición de la oración bajo formas diversas en Oriente y en Occidente. La formulación más habitual, transmitida por los espirituales del Sinaí, de Siria y del Monte Athos es la invocación: \textquote{Señor Jesucristo, Hijo de Dios, ten piedad de nosotros, pecadores} Conjuga el himno cristológico de \emph{Flp} 2, 6-11 con la petición del publicano y del mendigo ciego (cf. \emph{Lc} 18,13; \emph{Mc} 10, 46-52). Mediante ella, el corazón está acorde con la miseria de los hombres y con la misericordia de su Salvador.	
	
	\n{2668} La invocación del santo Nombre de Jesús es el camino más sencillo de la oración continua. Repetida con frecuencia por un corazón humildemente atento, no se dispersa en \textquote{palabrerías} (\emph{Mt} 6, 7), sino que \textquote{conserva la Palabra y fructifica con perseverancia} (cf. \emph{Lc} 8, 15). Es posible \textquote{en todo tiempo} porque no es una ocupación al lado de otra, sino la única ocupación, la de amar a Dios, que anima y transfigura toda acción en Cristo Jesús.
	
	\n{2812} Finalmente, el Nombre de Dios Santo se nos ha revelado y dado, en la carne, en Jesús, como Salvador (cf. \emph{Mt} 1, 21; \emph{Lc} 1, 31): revelado por lo que Él es, por su Palabra y por su Sacrificio (cf. \emph{Jn} 8, 28; 17, 8; 17, 17-19). Esto es el núcleo de su oración sacerdotal: \textquote{Padre santo \ldots{} por ellos me consagro a mí mismo, para que ellos también sean consagrados en la verdad} (\emph{Jn} 17, 19). Jesús nos \textquote{manifiesta} el Nombre del Padre (\emph{Jn} 17, 6) porque \textquote{santifica} Él mismo su Nombre (cf. \emph{Ez} 20, 39; 36, 20-21). Al terminar su Pascua, el Padre le da el Nombre que está sobre todo nombre: Jesús es Señor para gloria de Dios Padre (cf. \emph{Flp} 2, 9-11).																	
\end{ccebody}
