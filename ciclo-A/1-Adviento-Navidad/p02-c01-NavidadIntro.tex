\part{Tiempo de Navidad}

	\chapter{Introducción}
	
		\begin{introstyle}

			\section{Normativa litúrgica}

				En la Iglesia, la celebración más antigua después de la del Misterio Pascual es la memoria del Nacimiento del Señor y sus primeras manifestaciones, que se realiza en el tiempo de Navidad\footnote{Cf. NUALC nn. 32-38; OLM n. 95}.
				
				El tiempo de Navidad va desde las primeras vísperas de Navidad hasta el domingo después de Epifanía, o después del 6 de enero, inclusive.
				
				La Misa de la vigilia de Navidad se celebra en la tarde del 24 de diciembre, antes o después de las primeras vísperas.
				
				El día de Navidad se pueden celebrar tres Misas, según una antigua tradición de la Iglesia Romana, o sea en la noche, en la aurora y en el día.
				
				El día de Navidad tiene su octava propia dispuesta de la siguiente manera:
				
				a) Domingo dentro de la octava, o en su defecto, el día 30 de diciembre: fiesta de la Sagrada Familia.
				
				b) El 26 de diciembre: fiesta de san Esteban, el primer mártir.
				
				c) El 27 de diciembre: fiesta de san Juan, apóstol y evangelista.
				
				d) El 28 de diciembre: fiesta de los santos Inocentes.
				
				e) El 29, 30, 31 de diciembre son días \textquote{dentro de la octava}.
				
				f) El 1 de enero, octava de Navidad: solemnidad de santa María Madre de Dios, en que se conmemora también la imposición del santo Nombre de Jesús.
				
				El domingo entre el 2 y 5 de enero se llama Domingo 2° después de Navidad.
				
				La Epifanía del Señor se celebra el 6 de enero, a no ser que se transfiera -donde no es de precepto- al domingo situado entre el 2 y el 8 de enero (cf. n. 7).
				
				La fiesta del Bautismo del Señor se celebra el domingo siguiente al 6 de enero.
				
				En la vigilia y en las tres misas de Navidad, las lecturas, tanto las proféticas como las demás, se han tomado de la tradición romana.
				
				El domingo dentro de la Octava de Navidad, fiesta de la Sagrada Familia, el Evangelio es de la infancia de Jesús, las demás lecturas hablan de las virtudes de la vida doméstica.
				
				En la Octava de Navidad y solemnidad de santa María, Madre de Dios, las lecturas tratan de la Virgen, Madre de Dios, y de la imposición del santísimo nombre de Jesús.
				
				El segundo domingo después de Navidad, las lecturas tratan del misterio de la encarnación.
				
				En la Epifanía del Señor, la lectura del Antiguo Testamento y el Evangelio conservan la tradición romana; en la lectura apostólica se lee un texto relativo a la vocación de los paganos a la salvación.
				
				En la fiesta del Bautismo del Señor, los textos se refieren a este misterio.
				
			\section{Las celebraciones de la Navidad}
				
				\textquote{En la vigilia y en las tres Misas de Navidad, las lecturas, tanto las proféticas como las demás, se han tomado de la tradición Romana} (OLM 95). Un momento distintivo de la Solemnidad de la Navidad del Señor es la costumbre de celebrar tres misas diferentes: la de medianoche, la de la aurora y la del día. Con la reforma posterior al Concilio Vaticano II se ha añadido una vespertina en la vigilia. A excepción de las comunidades monásticas, no es normal que todos participen en las tres (o cuatro) celebraciones; la mayor parte de los fieles participará en una Liturgia que será su \textquote{Misa de Navidad}. Por ello se ha llevado a cabo una selección de lecturas para cada celebración. No obstante, antes de considerar algunos temas integrales y comunes a los textos litúrgicos y bíblicos, resulta ilustrativo examinar la secuencia de las cuatro misas\footnote{Cf. Congregación para el Culto Divino, \emph{Directorio Homilético} (2014), nn. 110-119}.
				
				La Navidad es la fiesta de la luz. Es opinión difundida que la celebración del Nacimiento del Señor se fijó a finales de diciembre para dar un valor cristiano a la fiesta pagana del \emph{Sol invictus}. Aunque podría también no ser así. Si ya en la primera parte del siglo III, Tertuliano escribió que en algunos calendarios Cristo fue concebido el 25 de marzo, día que se considera como el primero del año, es posible que la fiesta de la Navidad haya sido calculada a partir de esta fecha. En todo caso, ya desde el siglo IV, muchos Padres reconocen el valor simbólico del hecho de que los días se alargan después de la Fiesta de la Navidad. Las fiestas paganas que exaltan la luz en la oscuridad del invierno no eran extrañas, y las fiestas invernales de la luz aún hoy son celebradas en algunos lugares por los no creyentes. A diferencia de ello, las lecturas y las oraciones de las diversas Liturgias natalicias evidencian el tema de la verdadera Luz que viene a nosotros en Jesucristo. El primer prefacio de Navidad exclama, dirigiéndose a Dios Padre: \textquote{Porque gracias al misterio de la Palabra hecha carne, la luz de tu gloria brilló ante nuestros ojos con nuevo resplandor}. El homileta debería acentuar esta dinámica de la luz en las tinieblas, que inunda estos días gozosos. Presentamos a continuación una síntesis de las características de cada Celebración.
				
				\textbf{La Misa vespertina de la Vigilia}. Aunque la celebración de la Navidad comienza con esta Misa, las oraciones y las lecturas evocan aún un sentido de temblorosa espera; en cierto sentido, esta misa es una síntesis de todo el Tiempo de Adviento. Casi todas las oraciones están conjugadas en futuro: \textquote{Mañana contemplaréis su gloria} (antífona de entrada); \textquote{Concédenos que así como ahora acogemos, gozosos, a tu Hijo como Redentor, lo recibamos también confiados cuando venga como juez} (colecta); \textquote{Mañana quedará borrada la bondad de la tierra} (canto al Evangelio); \textquote{Concédenos, Señor, empezar estas fiestas de Navidad con una entrega digna del santo misterio del nacimiento de tu Hijo en el que has instaurado el principio de nuestra salvación} (oración sobre las ofrendas); \textquote{Se revelará la gloria del Señor} (antífona de comunión). Las lecturas de Isaías en las otras Misas de Navidad describen lo que \emph{está} sucediendo, mientras que el pasaje proclamado en esta Misa cuenta lo que \emph{sucederá}. La segunda lectura y el pasaje evangélico hablan de Jesús como el Hijo de David y de los antepasados humanos que han preparado el camino para su venida. La genealogía del Evangelio de san Mateo, describiendo a grandes rasgos el largo camino de la Historia de la Salvación que conduce al acontecimiento que vamos a celebrar, es similar a las lecturas del Antiguo Testamento de la Vigila Pascual. La letanía de nombres aumenta la sensación de espera. En la Misa de la Vigilia somos un poco como los niños que agarran con fuerza el regalo de Navidad, esperando la palabra que les permita abrirlo.
				
				\textbf{La Misa de medianoche}. En el corazón de la noche, mientras el resto del mundo duerme, los cristianos abren este regalo: el don del Verbo hecho carne. El profeta Isaías anuncia: \textquote{El pueblo que caminaba en tinieblas vio una luz grande}. Continúa refiriéndose a la gloriosa victoria del héroe conquistador que ha quebrantado la vara del opresor y ha tirado al fuego los instrumentos de guerra. Anuncia que el dominio de aquel que reinará será dilatado y con una paz sin límites y, por último, le llena de títulos: \textquote{Maravilla de Consejero, Dios guerrero, Padre perpetuo, Príncipe de la Paz}. El comienzo del Evangelio resalta la eminencia de tal dignatario, mencionando por su nombre al emperador y al gobernador que reinaban cuando Él irrumpe en escena. La narración prosigue con una revelación impresionante: este rey potente ha nacido en un modesto pueblecito de las fronteras del Imperio Romano y su madre \textquote{lo envolvió en pañales y lo acostó en un pesebre, porque no tenían sitio en la posada}. El contraste entre el héroe conquistador descrito por Isaías y el niño indefenso en el establo nos trae a la mente todas las paradojas del Evangelio. El conocimiento de estas paradojas está profundamente arraigado en el corazón de los fieles y los atrae a la Iglesia en el corazón de la noche. La respuesta apropiada es unir nuestro agradecimiento al de los ángeles, cuyo canto resuena en los cielos en esta noche.
				
				\textbf{La Misa de la Aurora}. Las lecturas propuestas para esta Celebración son particularmente concisas. Somos como aquellos que se despertaron en la gélida luz del alba, preguntándose si la aparición angélica en medio de la noche había sido un sueño. Los pastores, con ese innato buen sentido propio de los pobres, piensan entre sí: \textquote{Vamos derechos a Belén, a ver eso que ha pasado y que nos ha comunicado el Señor}. Van corriendo y encuentran exactamente lo que les había anunciado el Ángel: una pobre pareja y su Hijo apenas recién nacido, dormido en un pesebre para los animales. ¿Su reacción a esta escena de humilde pobreza? Vuelven glorificando y alabando a Dios por lo que han visto y oído, y todos los que los escuchan quedan impresionados por lo que les han referido. Los pastores vieron, y también nosotros estamos invitados a ver, algo mucho más trascendente que la escena que nos llena de emoción y que ha sido objeto de tantas representaciones artísticas. Pero esta realidad se puede ver sólo con los ojos de la fe y emerge con la luz del día, en la siguiente Celebración.
				
				\textbf{La Misa del día}. Como un sol resplandeciente ya en lo alto del cielo, el Prólogo del Evangelio de san Juan aclara la identidad del niño del pesebre. El evangelista afirma: \textquote{Y la Palabra se hizo carne, y acampó entre nosotros, y hemos contemplado su gloria: gloria propia del Hijo único del Padre, lleno de gracia y de verdad}. Con anterioridad, como recuerda la segunda lectura, Dios había hablado de muchas maneras por medio de los profetas; pero ahora \textquote{en esta etapa final, nos ha hablado por el Hijo, al que ha nombrado heredero de todo, y por medio del cual ha ido realizando las edades del mundo. Él es reflejo de su gloria \ldots{}} Esta es su grandeza, por la que lo adoran los mismos ángeles. Y aquí está la invitación para que todos se unan a ellos: \textquote{adorad al Señor, porque hoy una gran luz ha bajado a la tierra} (canto al evangelio).
				
				El Verbo se hace carne para redimirnos, gracias a su Sangre derramada, y ensalzarnos con él a la gloria de la Resurrección. Los primeros discípulos reconocieron la relación íntima entre la Encarnación y el Misterio Pascual, como testimonia el himno citado en la carta de san Pablo a los Filipenses (2,5-11). La luz de la Misa de medianoche es la misma luz de la Vigilia Pascual. Las colectas de estas dos grandes Solemnidades comienzan con términos muy similares. En Navidad, el sacerdote dice: \textquote{Oh Dios, que has iluminado esta noche santa con el nacimiento de Cristo, la luz verdadera \ldots{}}; en Pascua: \textquote{Oh Dios, que iluminas esta noche santa con la gloria de la Resurrección del Señor \ldots{}}. La segunda lectura de la Misa de la aurora propone una síntesis admirable de la revelación del Misterio de la Trinidad y de nuestra introducción al mismo a través del Bautismo: \textquote{Cuando se apareció la Bondad de Dios, nuestro Salvador, y su Amor al hombre, \ldots{} sino que según su propia misericordia nos ha salvado: con el baño del segundo nacimiento, y con la renovación por el Espíritu Santo; Dios lo derramó copiosamente sobre nosotros por medio de Jesucristo nuestro Salvador. Así, justificados por su gracia, somos, en esperanza, herederos de la vida eterna}. Las oraciones propias de la Misa del día hablan de Cristo como autor de nuestra generación divina y de cómo su nacimiento manifiesta la reconciliación que nos hace amables a los ojos de Dios. La colecta, una de las más antiguas del tesoro de las oraciones de la Iglesia, expresa sintéticamente \emph{porqué} el Verbo se hace carne: \textquote{Oh Dios, que de modo admirable has creado al hombre a tu imagen y semejanza; y de modo más admirable todavía restableciste su dignidad por Jesucristo; concédenos compartir la vida divina de aquél que hoy se ha dignado compartir con el hombre la condición humana}. Una de las finalidades fundamentales de la homilía es, como afirma el presente Directorio, la de anunciar el Misterio Pascual de Cristo. Los textos de la Navidad ofrecen explícitas oportunidades para hacerlo.
				
				Otra finalidad de la homilía es la de conducir a la comunidad hacia el Sacrificio Eucarístico, en el que el misterio Pascual se hace presente. Es un indicador claro la palabra \textquote{hoy}, a la que recurren con frecuencia los textos litúrgicos de las Misas de Navidad. El Misterio del Nacimiento de Cristo está presente en esta celebración, pero como en su primera venida, solo puede ser percibido con la mirada de la fe. Para los pastores el gran \textquote{signo} fue, simplemente, un pobre niño clocado en el pesebre, aunque en su recuerdo glorificaban y alababan a Dios por lo que habían visto. Con la mirada de la fe tenemos que percibir al mismo Cristo, nacido hoy, bajo los signos del pan y del vino. El \emph{admirabile commercium} del que nos habla la colecta del día de Navidad, según la cual Cristo comparte nuestra humanidad y nosotros su divinidad, se manifiesta de modo particular en la Eucaristía, como sugieren las oraciones de la celebración. En la media noche rezamos así en la oración sobre las ofrendas: \textquote{Acepta, Señor, nuestras ofrendas en esta noche santa, y por este intercambio de dones en el que nos muestras tu divina largueza, haznos partícipes de la divinidad de tu Hijo que, al asumir la naturaleza humana, nos ha unido a la tuya de modo admirable}. Y en la de la aurora: \textquote{Señor, que estas ofrendas sean signo del Misterio de Navidad que estamos celebrando; y así como tu Hijo, hecho hombre, se manifestó como Dios, así nuestras ofrendas de la tierra nos hagan partícipes de los dones del cielo}. Y también, en el prefacio III de Navidad: \textquote{Por él, hoy resplandece ante el mundo el maravilloso intercambio que nos salva: pues al revestirse tu Hijo de nuestra frágil condición no sólo confiere dignidad eterna a la naturaleza humana, sino que por esta unión admirable nos hace a nosotros eternos}.
				
				La referencia a la inmortalidad roza otro tema recurrente en los textos de Navidad: la celebración es sólo una parada momentánea en nuestra peregrinación. El mensaje escatológico, tan evidente en el tiempo de Adviento, también encuentra aquí su expresión. En la colecta de la Vigilia, rezamos: \textquote{\ldots{} que cada año nos alegras con la fiesta esperanzadora de nuestra redención; concédenos que así como ahora acogemos, gozosos, a tu Hijo como Redentor, lo recibamos también confiados cuando venga como juez}. En la segunda lectura de la Misa de medianoche, el Apóstol nos exhorta \textquote{a renunciar a la vida sin religión y a los deseos mundanos, y a llevar ya desde ahora una vida sobria, honrada y religiosa, aguardando la dicha que esperamos: la aparición gloriosa del gran Dios y Salvador nuestro Jesucristo}. Y por último, en la oración después de la comunión de la Misa del día, pedimos que Cristo, autor de nuestra generación divina, nacido en este día, \textquote{nos haga igualmente partícipes del don de su inmortalidad}.
				
				Las lecturas y las oraciones de Navidad ofrecen un rico alimento al pueblo de Dios peregrino en esta vida; revelando a Cristo como Luz del mundo, nos invitan a sumergirnos en el Misterio Pascual de nuestra redención a través del \textquote{hoy} de la Celebración Eucarística. El homileta puede presentar este banquete al pueblo de Dios reunido para celebrar el nacimiento del Señor, exhortándole a imitar a María, la Madre de Jesús, que \textquote{conservaba todas estas cosas, meditándolas en su corazón} (Evangelio, Misa de la aurora).
				
			\section{Una reflexión sobre la Navidad}
			
				\src{Benedicto XVI, papa, \emph{Catequesis,} Audiencia general, 17 de diciembre del 2008.}
				
				Comenzamos precisamente hoy los días del Adviento que nos preparan inmediatamente para el Nacimiento del Señor: estamos en la \emph{Novena de Navidad,} que en muchas comunidades cristianas se celebra con liturgias ricas en texto bíblicos, todos ellos orientados a alimentar la espera del nacimiento del Salvador. En efecto, toda la Iglesia concentra su mirada de fe en esta fiesta, ya cercana, disponiéndose, como cada año, a unirse al canto alegre de los ángeles, que en el corazón de la noche anunciarán a los pastores el extraordinario acontecimiento del nacimiento del Redentor, invitándolos a dirigirse a la cueva de Belén. Allí yace el Emmanuel, el Creador que se ha hecho criatura, envuelto en pañales y acostado en un pobre pesebre (cf. \emph{Lc} 2, 12-14).
				
				La Navidad, por el clima que la caracteriza, es una fiesta universal. De hecho, incluso quien se dice no creyente puede percibir en esta celebración cristiana anual algo extraordinario y trascendente, algo íntimo que habla al corazón. Es la fiesta que canta el don de la vida. El nacimiento de un niño debería ser siempre un acontecimiento que trae alegría: el abrazo de un recién nacido suscita normalmente sentimientos de atención y de solicitud, de conmoción y de ternura.
				
				La Navidad es el encuentro con un recién nacido que llora en una cueva miserable. Contemplándolo en el pesebre, ¿cómo no pensar en tantos niños que también hoy, en muchas regiones del mundo, nacen en una gran pobreza? ¿Cómo no pensar en los recién nacidos que no son acogidos sino rechazados, en los que no logran sobrevivir por falta de cuidados y atenciones? ¿Cómo no pensar también en las familias que quisieran tener la alegría de un hijo y no ven cumplida esta esperanza? Por desgracia, por el impulso de un consumismo hedonista, la Navidad corre el riesgo de perder su significado espiritual para reducirse a una mera ocasión comercial de compras e intercambio de regalos.
				
				Sin embargo, en realidad, las dificultades, las incertidumbres y la misma crisis económica que en estos meses están viviendo tantas familias, y que afecta a toda la humanidad, pueden ser un estímulo para volver a descubrir el calor de la sencillez, la amistad y la solidaridad, valores típicos de la Navidad. Así, sin las incrustaciones consumistas y materialistas, la Navidad puede convertirse en una ocasión para acoger, como regalo personal, el mensaje de esperanza que brota del misterio del nacimiento de Cristo.
				
				Todo esto, sin embargo, no basta para captar en su plenitud el valor de la fiesta a la que nos estamos preparando. Nosotros sabemos que en ella se celebra el acontecimiento central de la historia: la Encarnación del Verbo divino para la redención de la humanidad. San León Magno, en una de sus numerosas homilías navideñas, exclama: \textquote{Exultemos en el Señor, queridos hermanos, y abramos nuestro corazón a la alegría más pura. Porque ha amanecido el día que para nosotros significa la nueva redención, la antigua preparación, la felicidad eterna. Así, en el ciclo anual, se renueva para nosotros el elevado misterio de nuestra salvación, que, prometido al principio y realizado al final de los tiempos, está destinado a durar sin fin} \emph{(Homilía XXII). }
				
				San Pablo comenta muchas veces esta verdad fundamental en sus cartas. Por ejemplo, a los \emph{Gálatas} escribe: \textquote{Al llegar la plenitud de los tiempos, envió Dios a su Hijo, nacido de mujer, nacido bajo la Ley (\ldots{}) para que recibiéramos la filiación adoptiva} \emph{(Ga} 4, 4-5). En la \emph{carta a los Romanos} pone de manifiesto las lógicas y exigentes consecuencias de este acontecimiento salvador: \textquote{Si somos hijos (de Dios), también somos herederos; herederos de Dios y coherederos de Cristo, ya que sufrimos con él, para ser también con él glorificados} \emph{(Rm} 8, 17). Pero es sobre todo san Juan, en el \emph{Prólogo} del cuarto \emph{Evangelio,} quien medita profundamente en el misterio de la Encarnación. Y por eso desde los tiempos más antiguos el \emph{Prólogo} forma parte de la liturgia de la Navidad. En efecto, en él se encuentra la expresión más auténtica y la síntesis más profunda de esta fiesta y del fundamento de su alegría. San Juan escribe: \textquote{\emph{Et Verbum caro factum est et habitavit in nobis}} --- \textquote{Y el Verbo se hizo carne y habitó entre nosotros} \emph{(Jn} 1, 14).
				
				Así pues, en Navidad no nos limitamos a conmemorar el nacimiento de un gran personaje; no celebramos simplemente y en abstracto el misterio del nacimiento del hombre o en general el misterio de la vida; tampoco celebramos sólo el inicio de la nueva estación. En Navidad recordamos algo muy concreto e importante para los hombres, algo esencial para la fe cristiana, una verdad que san Juan resume en estas pocas palabras: \textquote{El Verbo se hizo carne}.
				
				Se trata de un acontecimiento histórico que el evangelista san Lucas se preocupa de situar en un contexto muy determinado: en los días en que César Augusto emanó el decreto para el primer censo, cuando Quirino era ya gobernador de Siria (cf. \emph{Lc} 2, 1-7). Por tanto, en una noche fechada históricamente se verificó el acontecimiento de salvación que Israel esperaba desde hacía siglos. En la oscuridad de la noche de Belén se encendió realmente una gran luz: el Creador del universo se encarnó uniéndose indisolublemente a la naturaleza humana, siendo realmente \textquote{Dios de Dios, luz de luz} y al mismo tiempo hombre, verdadero hombre.
				
				Aquel a quien san Juan llama en griego \textquote{ \emph{ho Logo} s} ---traducido en latín \textquote{Verbum} y en español \textquote{el Verbo} --- significa también \textquote{el Sentido}. Por tanto, la expresión de san Juan se puede entender así: el \textquote{Sentido eterno} del mundo se ha hecho perceptible a nuestros sentidos y a nuestra inteligencia: ahora podemos tocarlo y contemplarlo (cf. \emph{I Jn} 1, 1). El \textquote{Sentido} que se ha hecho carne no es simplemente una idea general inscrita en el mundo; es una \textquote{Palabra} dirigida a nosotros. El \emph{Logos} nos conoce, nos llama, nos guía. No es una ley universal, en la que nosotros desarrollamos algún papel; es una Persona que se interesa por cada persona: es el Hijo del Dios vivo, que se ha hecho hombre en Belén.
				
				A muchos hombres, y de algún modo a todos nosotros, esto parece demasiado hermoso para ser cierto. En efecto, aquí se nos reafirma: sí, existe un sentido, y el sentido no es una protesta impotente contra lo absurdo. El Sentido tiene poder: es Dios. Un Dios bueno, que no se confunde con un poder excelso y lejano, al que nunca se podría llegar, sino un Dios que se ha hecho nuestro prójimo, muy cercano a nosotros, que tiene tiempo para cada uno de nosotros y que ha venido a quedarse con nosotros.
				
				Entonces surge espontáneamente la pregunta: \textquote{¿Cómo es posible algo semejante? ¿Es digno de Dios hacerse niño?}. Para intentar abrir el corazón a esta verdad que ilumina toda la existencia humana, es necesario plegar la mente y reconocer la limitación de nuestra inteligencia. En la cueva de Belén Dios se nos muestra \textquote{niño} humilde para vencer nuestra soberbia. Tal vez nos habríamos rendido más fácilmente frente al poder, frente a la sabiduría; pero él no quiere nuestra rendición; más bien apela a nuestro corazón y a nuestra decisión libre de aceptar su amor. Se ha hecho pequeño para liberarnos de la pretensión humana de grandeza que brota de la soberbia; se ha encarnado libremente para hacernos a nosotros verdaderamente libres, libres de amarlo.
				
				{[}\ldots{}{]} La Navidad es una oportunidad privilegiada para meditar en el sentido y en el valor de nuestra existencia. La proximidad de esta solemnidad nos ayuda a reflexionar, por una parte, en el dramatismo de la historia en la que los hombres, heridos por el pecado, buscan permanentemente la felicidad y el sentido pleno de la vida y de la muerte; y, por otra, nos exhorta a meditar en la bondad misericordiosa de Dios, que ha salido al encuentro del hombre para comunicarle directamente la Verdad que salva y para hacerlo partícipe de su amistad y de su vida.
				
				Preparémonos, por tanto, para la Navidad con humildad y sencillez, disponiéndonos a recibir el don de la luz, la alegría y la paz que irradian de este misterio. Acojamos el Nacimiento de Cristo como un acontecimiento capaz de renovar hoy nuestra vida. Que el encuentro con el Niño Jesús nos haga personas que no piensen sólo en sí mismas, sino que se abran a las expectativas y necesidades de los hermanos. De esta forma nos convertiremos también nosotros en testigos de la luz que la Navidad irradia sobre la humanidad del tercer milenio.
				
				Pidamos a María santísima, tabernáculo del Verbo encarnado, y a san José, testigo silencioso de los acontecimientos de la salvación, que nos comuniquen los sentimientos que ellos tenían mientras esperaban el nacimiento de Jesús, de modo que podamos prepararnos para celebrar santamente la próxima Navidad, en el gozo de la fe y animados por el compromiso de una conversión sincera.
				
				¡Feliz Navidad a todos!
			\end{introstyle}