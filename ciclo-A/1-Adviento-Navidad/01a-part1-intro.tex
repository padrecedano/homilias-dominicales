\section{Tiempo de Adviento}

\subsection{Introducción}

El tiempo de Adviento comienza con las primeras vísperas del domingo que cae el 30 de noviembre, o más próximo a ese día, y concluye antes de las primeras vísperas de Navidad.\anote{id2}

Este tiempo tiene dos características: es a la vez un tiempo de preparación a las solemnidades de Navidad en que se conmemora la primera Venida del Hijo de Dios entre los hombres, y un tiempo en el cual, mediante esta celebración, el ánimo se dirige a esperar la segunda Venida de Cristo al fin de los tiempos. Por estos dos motivos, el Adviento se presenta como un tiempo de piadosa y alegre esperanza.

\textbf{Lecturas de los domingos de Adviento}\anote{id3}

En los domingos de Adviento las lecturas del Evangelio tienen una característica propia: se refieren a la venida del Señor al final de los tiempos (primer domingo), a Juan Bautista (segundo y tercer domingos), a
los acontecimientos que prepararon de cerca el nacimiento del Señor (cuarto domingo).

Las lecturas del Antiguo Testamento son profecías sobre el Mesías y el tiempo mesiánico, tomadas principalmente del libro de Isaías.

Las lecturas del Apóstol contienen exhortaciones y amonestaciones conformes a las diversas características de este tiempo.

\textbf{Algunos aspectos que debe tener en cuenta el homileta}\anote{id4}

El Adviento es el tiempo que prepara a los cristianos a las gracias que serán dadas, una vez más en este año, en la celebración de la gran Solemnidad de la Navidad. Ya desde el I domingo de Adviento, el homileta exhorta al pueblo para que emprenda su preparación caracterizada por distintas facetas, cada una de ellas sugerida por la rica selección de pasajes bíblicos del Leccionario de este tiempo. La primera fase del Adviento nos invita a preparar la Navidad animándonos no sólo a dirigir la mirada al tiempo de la primera Venida de nuestro Señor, cuando, como dice el prefacio I de Adviento, Él asume \textquote{la humildad de nuestra carne}, sino también, a esperar vigilantes su Venida \textquote{en la majestad de su gloria}, cuando \textquote{podamos recibir los bienes prometidos}.

En la predicación no debe perderse de vista que existe un doble significado del Adviento, un doble significado de la Venida del Señor. Este tiempo nos prepara para su Venida en la gracia de la fiesta de la Navidad y a su retorno para el juicio al final de los tiempos. Los textos bíblicos deberían ser explicados considerando este doble significado. Según el texto, se puede evidenciar una u otra Venida, aunque, con frecuencia, el mismo pasaje presenta palabras e imágenes relativas a ambas.

Existe, además, otra Venida: escuchamos la Palabra de Dios en la asamblea eucarística, donde Cristo está verdaderamente presente. Al comienzo del tiempo de Adviento la Iglesia recuerda la enseñanza de san Bernardo, es decir, que entre las dos Venidas visibles de Cristo, en la historia y al final de los tiempos, existe una venida invisible, aquí y ahora\anote{id5}, así como hace suyas las palabras de san Carlos Borromeo:

\textquote{Este tiempo (\ldots{}) nos enseña que la venida de Cristo no solo aprovechó a los que vivían en el tiempo del Salvador, sino que su eficacia continúa y aún hoy se nos comunica si queremos recibir, mediante la fe y los sacramentos, la gracia que él nos prometió, y si ordenamos nuestra conducta conforme a sus mandamientos\anote{id6}}.

\textbf{Una reflexión sobre el Adviento}\anote{id7}

{[}El término Adviento significa \textquote{venida}, \emph{parousia}, en latín \emph{adventus}, de donde viene el término Adviento (cf. \emph{1 Ts} 5, 23){]}.

Reflexionemos brevemente sobre el significado de esta palabra, que se puede traducir por \textquote{presencia}, \textquote{llegada}, \textquote{venida}. En el lenguaje del mundo antiguo era un término técnico utilizado para indicar la llegada de un funcionario, la visita del rey o del emperador a una provincia. Pero podía indicar también la venida de la divinidad, que sale de su escondimiento para manifestarse con fuerza, o que se celebra presente en el culto. Los cristianos adoptaron la palabra \textquote{Adviento} para expresar su relación con Jesucristo: Jesús es el Rey, que ha entrado en esta pobre \textquote{provincia} denominada tierra para visitar a todos; invita a participar en la fiesta de su Adviento a todos los que creen en él, a todos los que creen en su presencia en la asamblea litúrgica. Con la palabra \emph{adventus} se quería decir substancialmente: Dios está aquí, no se ha retirado del mundo, no nos ha dejado solos. Aunque no podamos verlo o tocarlo, como sucede con las realidades sensibles, él está aquí y viene a visitarnos de múltiples maneras.

Por lo tanto, el significado de la expresión \textquote{Adviento} comprende también el de \emph{visitatio}, que simplemente quiere decir \textquote{visita}; en este caso se trata de una visita de Dios: él entra en mi vida y quiere dirigirse a mí. En la vida cotidiana todos experimentamos que tenemos poco tiempo para el Señor y también poco tiempo para nosotros. Acabamos dejándonos absorber por el \textquote{hacer}. ¿No es verdad que con frecuencia es precisamente la actividad lo que nos domina, la sociedad con sus múltiples intereses lo que monopoliza nuestra atención? ¿No es verdad que se dedica mucho tiempo al ocio y a todo tipo de diversiones? A veces las cosas nos \textquote{arrollan}.

El Adviento, este tiempo litúrgico fuerte que estamos comenzando, nos invita a detenernos, en silencio, para captar una presencia. Es una invitación a comprender que los acontecimientos de cada día son gestos que Dios nos dirige, signos de su atención por cada uno de nosotros. ¡Cuán a menudo nos hace percibir Dios un poco de su amor! Escribir ---por decirlo así--- un \textquote{diario interior} de este amor sería una tarea hermosa y saludable para nuestra vida. El Adviento nos invita y nos estimula a contemplar al Señor presente. La certeza de su presencia, ¿no debería ayudarnos a ver el mundo de otra manera? ¿No debería ayudarnos a considerar toda nuestra existencia como \textquote{visita}, como un modo en que él puede venir a nosotros y estar cerca de nosotros, en cualquier situación?

Otro elemento fundamental del Adviento es la espera, una espera que es al mismo tiempo esperanza. El Adviento nos impulsa a entender el sentido del tiempo y de la historia como \textquote{\emph{kairós}}, como ocasión propicia para nuestra salvación. Jesús explicó esta realidad misteriosa en muchas parábolas: en la narración de los siervos invitados a esperar el regreso de su dueño; en la parábola de las vírgenes que esperan al esposo; o en las de la siembra y la siega. En la vida, el hombre está constantemente a la espera: cuando es niño quiere crecer; cuando es adulto busca la realización y el éxito; cuando es de edad avanzada aspira al merecido descanso. Pero llega el momento en que descubre que ha esperado demasiado poco si, fuera de la profesión o de la posición social, no le queda nada más que esperar. La esperanza marca el camino de la humanidad, pero para los cristianos está animada por una certeza: el Señor está presente a lo largo de nuestra vida, nos acompaña y un día enjugará también nuestras lágrimas. Un día, no lejano, todo encontrará su cumplimiento en el reino de Dios, reino de justicia y de paz.

Existen maneras muy distintas de esperar. Si el tiempo no está lleno de un presente cargado de sentido, la espera puede resultar insoportable; si se espera algo, pero en este momento no hay nada, es decir, si el presente está vacío, cada instante que pasa parece exageradamente largo, y la espera se transforma en un peso demasiado grande, porque el futuro es del todo incierto. En cambio, cuando el tiempo está cargado de sentido, y en cada instante percibimos algo específico y positivo, entonces la alegría de la espera hace más valioso el presente. {[}\ldots{}{]} Vivamos intensamente el presente, donde ya nos alcanzan los dones del Señor, vivámoslo proyectados hacia el futuro, un futuro lleno de esperanza. De este modo, el Adviento cristiano es una ocasión para despertar de nuevo en nosotros el sentido verdadero de la espera, volviendo al corazón de nuestra fe, que es el misterio de Cristo, el Mesías esperado durante muchos siglos y que nació en la pobreza de Belén. Al venir entre nosotros, nos trajo y sigue ofreciéndonos el don de su amor y de su salvación. Presente entre nosotros, nos habla de muchas maneras: en la Sagrada Escritura, en el año litúrgico, en los santos, en los acontecimientos de la vida cotidiana, en toda la creación, que cambia de aspecto si detrás de ella se encuentra él o si está ofuscada por la niebla de un origen y un futuro inciertos.

Nosotros podemos dirigirle la palabra, presentarle los sufrimientos que nos entristecen, la impaciencia y las preguntas que brotan de nuestro corazón. Estamos seguros de que nos escucha siempre. Y si Jesús está presente, ya no existe un tiempo sin sentido y vacío. Si él está presente, podemos seguir esperando incluso cuando los demás ya no pueden asegurarnos ningún apoyo, incluso cuando el presente está lleno de dificultades.

{[}\ldots{}{]} El Adviento es el tiempo de la presencia y de la espera de lo eterno. Precisamente por esta razón es, de modo especial, el tiempo de la alegría, de una alegría interiorizada, que ningún sufrimiento puede eliminar. La alegría por el hecho de que Dios se ha hecho niño. Esta alegría, invisiblemente presente en nosotros, nos alienta a caminar confiados. La Virgen María, por medio de la cual nos ha sido dado el Niño Jesús, es modelo y sostén de este íntimo gozo. Que ella, discípula fiel de su Hijo, nos obtenga la gracia de vivir este tiempo litúrgico vigilantes y activos en la espera. Amén.