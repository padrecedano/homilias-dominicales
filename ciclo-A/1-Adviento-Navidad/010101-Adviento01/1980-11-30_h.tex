\subsubsection{Homilía (1980): La palabra central del Adviento}

\src{Visita Pastoral a la Parroquia de San Leonardo de Porto Mauricio. \\30 de noviembre de 1980.}

\begin{body}	
	\ltr[1. ]{A}{l} escuchar las palabras del \textbf{Evangelio} de hoy según \textbf{Mateo}, ante nuestros ojos vienen espontáneamente a la memoria los acontecimientos que durante la semana pasada han sacudido a toda Italia\anote{id10}.\ldots{} Mientras nosotros todos, con espíritu de solidaridad humana, queremos ayudar a nuestros hermanos y compatriotas, arrollados por la desgracia, al mismo tiempo, estos acontecimientos traen ante nuestros ojos, con una particular fuerza comparativa, el cuadro terrible que cada año trazan los \textbf{Evangelios} de este primer domingo de Adviento: anuncios de destrucción y de muerte, en la espera escatológica de la \textquote{venida del Hijo del Hombre} (\emph{Mt} 24, 39). 
	
	2. La historia de los hombres y de las naciones, la historia de toda la humanidad suministra pruebas suficientes para afirmar que en todos los tiempos se han multiplicado desgracias y catástrofes, calamidades naturales, como terremotos, o las causadas por el hombre, como guerras, revoluciones, estragos, homicidios y genocidios. Además, cada uno de nosotros sabe que nuestra existencia terrena lleva a la muerte, llegando así un día a su término. El mundo visible, con todos los bienes y las riquezas que oculta en sí mismo, al fin no es capaz de darnos más que la muerte: el término de la vida. 
	
	Esta verdad, aunque nos la recuerda también la liturgia de hoy, primer domingo de Adviento, sin embargo, no es la verdad específica anunciada en este día festivo, y en todo el período de Adviento. No es la palabra principal del \textbf{Evangelio}. 
	
	¿Cuál es, pues, la palabra principal? La hemos leído hace poco: la venida del Hijo del Hombre. La palabra principal del Evangelio no es \textquote{la separación}, \textquote{la ausencia}, sino \textquote{la venida} y \textquote{la presencia}. Ni siquiera es la \textquote{muerte}, sino la \textquote{vida}. El Evangelio es la Buena Noticia, porque pronuncia la verdad sobre la vida en el contexto de la muerte. 
	
	La venida del Hijo del Hombre es el comienzo de esta Vida. Y de este comienzo nos habla precisamente el Adviento, que responde a la pregunta: ¿cómo debe vivir el hombre en el mundo con la perspectiva de la muerte? El hombre al que, en un abrir y cerrar de ojos, le puede ser quitada la vida, ¿cómo debe vivir en este mundo, para encontrarse con el Hijo del Hombre, cuya venida es el comienzo de la nueva vida, de la vida más potente que la muerte? 
	
	4. Nos encontramos, pues, todos en el primer domingo de Adviento. ¿Cuál es esta verdad que nos penetra y vivifica hoy? ¿Qué mensaje nos anuncia la Santa Iglesia, nuestra Madre? Como ya he dicho, no se trata de un mensaje de miedo y de muerte, sino del mensaje de la esperanza y de la llamada. 
	
	Tomemos como ejemplo la \textbf{segunda lectura}; he aquí lo que el Apóstol Pablo dice a los romanos de entonces, pero que debemos tomar en serio los [romanos] de hoy: \textquote{Daos cuenta del momento en que vivís; ya es hora de espabilarse, porque ahora nuestra salvación está más cerca que cuando empezamos a creer. La noche está avanzada, el día se echa encima} (\emph{Rm} 13, 11-12). 
	
	En realidad, al contrario de como podemos ser inducidos a pensar, la salvación está más cercana y no más lejana. Efectivamente, al vivir en una época de secularización, somos testigos de comportamientos de indiferencia religiosa y también de programas e ideologías ateas o incluso antiteístas. Se llegaría a pensar que los indicios humanos desmienten el mensaje de la liturgia de hoy. Ella, en cambio ---aun haciendo referencia también a estos \textquote{indicios humanos}--- proclama, sin embargo, la verdad divina y anuncia el designio divino que no decae jamás, que no cambia aun cuando puedan cambiar los hombres, los programas, los proyectos humanos. Ese designio divino es el designio de la salvación del hombre en Cristo, que, una vez emprendido, perdura, y consiguientemente mira a su cumplimiento. 
	
	Pero el hombre puede ser ciego y sordo a todo esto. Puede meterse cada vez más profundamente en la noche, aunque se acerque el día. Puede multiplicar las obras de las tinieblas aunque Cristo le ofrezca \textquote{las armas de la luz}. 
	
	Por lo tanto, la invitación apremiante de la liturgia de hoy es la del \textbf{Apóstol}: \textquote{Vestíos del Señor Jesucristo} (\emph{Rm} 13, 14). Esta expresión es, en cierto sentido, la definición del cristiano. Ser cristiano quiere decir \textquote{vestirse de Cristo}. El Adviento es la nueva llamada a vestirse de Jesucristo. 
	
	Dice además el Apóstol: \textquote{Conduzcámonos como en pleno día, con dignidad. Nada de comilonas ni borracheras, nada de lujuria ni desenfreno, nada de riñas ni pendencias\ldots{}, y que el cuidado de vuestro cuerpo no fomente los malos deseos} (\emph{Rm} 13, 13-14). 
	
	5. ¿Qué significa, además, el Adviento? El Adviento es el descubrimiento de una gran aspiración de los hombres y de los pueblos hacia la casa del Señor. No hacia la muerte y la destrucción, sino hacia el encuentro con Él. 
	
	Y por esto en la liturgia de hoy oímos esta invitación: \textquote{Qué alegría cuando me dijeron: vamos a la casa del Señor}. 
	
	Y el mismo \textbf{Salmo responsorial} nos traza, por decirlo así, la imagen de esa casa, de esa ciudad, de ese encuentro: \textquote{Ya están pisando nuestros pies tus umbrales, Jerusalén. Allá suben las tribus, las tribus del Señor. Según la costumbre de Israel, a celebrar el nombre del Señor. En ella están los tribunales de justicia en el palacio de David. Por mis hermanos y compañeros voy a decir: \textquote{La paz contigo}. Por la casa del Señor nuestro Dios, te deseo todo bien} (\emph{Sal} 121 {[}122{]}). 
	
	Sí. El Señor es el Dios de la paz, es el Dios de la Alianza con el hombre. Cuando en la noche de Belén los pobres pastores se pondrán en camino hacia el establo donde se realizará la primera venida del Hijo del Hombre, los conducirá el canto de los ángeles: \textquote{Gloria a Dios en las alturas y paz en la tierra a los hombres de buena voluntad} (\emph{Lc} 2, 14). 
	
	6. Esta visión de la paz divina pertenece a toda la espera mesiánica en la Antigua Alianza. Oímos hoy las palabras de \textbf{Isaías}: \textquote{Será el árbitro de las naciones, el juez de pueblos numerosos. De las espadas forjarán arados; de las lanzas, podaderas. No alzará la espada pueblo contra pueblo, no se adiestrarán para la guerra. Casa de Jacob, ven; caminemos a la luz del Señor} (\emph{Is} 2, 4-5). 
	
	El Adviento trae consigo la invitación a la paz de Dios para todos los hombres. Es necesario que nosotros construyamos esta paz y la reconstruyamos continuamente en nosotros mismos y con los otros: en las familias, en las relaciones con los cercanos, en los ambientes de trabajo, en la vida de toda la sociedad. 
	
	Trabajad con espíritu de solidaridad fraterna a fin de que vuestra parroquia crezca cada vez más como comunidad de fieles, de familias, de grupos ---me refiero particularmente a todos vuestros grupos organizados--- en comunión de verdad y de amor. La comunidad parroquial, en efecto, se edifica sobre la Palabra de Dios, transmitida y garantizada por los Pastores, se alimenta por la gracia de los sacramentos, se sostiene por la oración, se une por el vínculo de la caridad fraterna. Que cada uno de sus miembros se sienta vivo, activo, partícipe, corresponsable, implicado en tareas efectivas de evangelización cristiana y de promoción humana. De este modo, vuestra parroquia se convierte en signo e instrumento de la presencia de Cristo en el barrio, irradiación de su amor y de su paz. 
	
	Para servir a esta paz de múltiples dimensiones, es necesario escuchar también estas palabras del \textbf{Profeta}: \textquote{Venid, subamos al monte del Señor, a la casa del Dios de Jacob. El nos instruirá en sus caminos y marcharemos por sus sendas, porque de Sión saldrá la ley. de Jerusalén la palabra del Señor} (\emph{Is} 2, 3). 
	
	También para vuestra comunidad eclesial el Adviento es el tiempo en el que se deben aprender de nuevo la ley del Señor y sus palabras. Es el tiempo de una catequesis intensificada. La ley y la palabra del Señor deben penetrar de nuevo en el corazón, deben encontrar de nuevo su confirmación en la vida social. Ellas sirven al bien del hombre, ¡construyen la paz! 
	
	7. Queridos hermanos e hijos: Nos encontramos, pues, de nuevo al comienzo del camino. Ha comenzado de nuevo el Adviento: el tiempo de la gracia, el tiempo de la espera, el tiempo de la venida del Señor, que perdura siempre. Y la vida del hombre se desarrolla en el amor del Señor, a pesar de todas las dolorosas experiencias de la destrucción y de la muerte, hacia la realización final en Dios. 
	
	¡El Hijo del Hombre vendrá! Escuchemos estas palabras con la esperanza, no con el miedo, aunque estén llenas de una profunda seriedad. 
	
	Velad\ldots{} y estad preparados, porque no sabéis en qué día vendrá el Hijo del Hombre. ¡Ven, Señor Jesús! ¡Marana tha!
\end{body}