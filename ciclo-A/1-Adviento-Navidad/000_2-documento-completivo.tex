\section{Alter} JUAN PABLO II .

\emph{\textbf{AUDIENCIA GENERAL}}\\[2\baselineskip]\emph{Miércoles 30 de diciembre de 1992}

 

\emph{Queridísimos hermanos y hermanas en el Señor; queridísimos jóvenes:}

1. Hemos celebrado hace algunos días la solemnidad de la Navidad y estamos todavía penetrados por la atmósfera sugestiva de la Noche Santa. Contemplamos asombrados, junto a María santísima y a san José, el misterio del Verbo encarnado.

El nacimiento del Hijo de Dios \textquote{de una mujer} (cf. \emph{Gal} 4, 4) nos hace remontarnos de nuevo al proyecto salvífico: el Altísimo ha querido entrar directamente en la historia de la humanidad y nos ha dado a su Hijo unigénito como Salvador y Redentor.

Eso es la Navidad, \textquote{misterio} providencial de amor, en el que María, escogida como virgen Madre del Emmanuel, se encuentra \emph{asociada a la obra de la redención}. Nos detenemos en estos días a \emph{contemplar a María en Belén}. La Madre, que estrecha entre sus brazos a Jesús, nos ayuda a comprender ante todo que de la gruta, iluminada por la luz divina, viene un \emph{mensaje de verdad:} Dios se ha hecho hombre y, compartiendo nuestra naturaleza, nos habla con el poder de su misericordia salvadora. Sin embargo, \emph{es María quien nos da la Palabra que salva:} ella nos muestra a Jesús, \textquote{la luz del mundo}, que da el verdadero sentido a la vida y el pleno significado a la existencia. ¿Cómo no permanecer sorprendidos y maravillados ante tal misterio? ¿Cómo no abrir el corazón a la venida entre nosotros del Señor de la historia?

2. Queridos jóvenes, que habéis venido de diferentes partes de Europa en nombre de María: La joven de Nazaret, presente silenciosamente en el misterio de la Navidad, está presente también en el corazón de la Iglesia y en el de cada fiel. El \emph{Catecismo de la Iglesia Católica} publicado recientemente afirma, que \textquote{por su total adhesión a la voluntad del Padre, a la obra redentora de su Hijo, a toda moción del Espíritu Santo, la Virgen María es para la Iglesia el modelo de la fe y de la caridad. Por eso es \textquote{miembro muy eminente y del todo singular de la Iglesia} (\emph{Lumen gentium}, 53), incluso constituye \textquote{la figura} [\textquote{tipus}] de la Iglesia (\emph{Lumen gentium}, 63)} (n. 967).

\emph{María es madre:} madre de Cristo y madre nuestra. Su función maternal \textquote{dimana\ldots{} de la superabundancia de los méritos de Cristo; se apoya en la mediación de éste, depende totalmente de ella y de la misma saca todo su poder} (\emph{Lumen gentium}, 60). Con respecto a los creyentes, su función es ser \textquote{nuestra madre en el orden de la gracia} (\emph{ib}., \emph{} 61), y por esto \textquote{es invocada en la Iglesia con los títulos de abogada, auxiliadora, socorro, mediadora} (\emph{ib}., 62). Se trata de una misión providencial que el Señor le ha confiado y que se resume perfectamente en la expresión \emph{Per Mariam ad Iesurm}.

Ésta es, como sabéis bien, la doctrina fundamental de san Luis María Grignon de Monfort, en quien vosotros os inspiráis; y es el ideal que debe impulsar a todos los cristianos. Gracias a la ayuda de la Madre de Dios, el testimonio de los creyentes se hace cada día más coherente y fervoroso, más generoso y más abierto.

3. El concilio Vaticano II, cuyo trigésimo aniversario estamos celebrando, exhortó a los fieles a ofrecer \textquote{súplicas apremiantes a la Madre de Dios y Madre de los hombres para que ella, que ayudó con sus oraciones a la Iglesia naciente, también ahora, ensalzada en el cielo por encima de todos los ángeles y bienaventurados, interceda en la comunión de todos los santos ante su Hijo hasta que todas las familias de los pueblos, tanto los que se honran con el título de cristianos como los que todavía desconocen a su Salvador, lleguen a reunirse felizmente, en paz y concordia, en un solo pueblo de Dios, para gloria de la santísima e indivisible Trinidad} (\emph{Lumen gentium}, 69).

Queridos hermanos y hermanas, de esta profunda riqueza espiritual brota vuestra devoción a María y vuestro compromiso apostólico. Mirad siempre a María como a la estrella segura que os guía en el camino de vuestra vida cristiana.

4. Vosotros, queridos jóvenes, que representáis el futuro de la humanidad y la esperanza de la Iglesia, debéis llevar el evangelio de la bondad y de la paz a todos los rincones de los países de donde provenís. En toda Europa aumentan las dificultades y algunas regiones se hallan azotadas por la violencia. Por esto, vuestra misión es una obra de solidaridad espiritual, un servicio a la verdad, que exige un testimonio creíble del mensaje integral de Cristo.

Ante vosotros resplandece \emph{María, la Virgen fiel, la estrella de la evangelización, vuestra madre y modelo}. Acudid a ella todos los días, como lo deseáis hacer hoy.

Con la ayuda de su intercesión maternal, podréis contribuir activamente a la obra de la nueva evangelización, y podréis ser un fermento genuino de vida cristiana y de comunión fiel en vuestras comunidades eclesiales.

5. Queridísimos hermanos, os saludo cordialmente a todos vosotros y a los países de los que provenís. Vuestra presencia aquí es un signo más de la unidad entre las naciones y comunidades cristianas, que se caracteriza por la \textquote{comunicación de bienes} espirituales y materiales con la finalidad de construir un futuro común basado en la justicia y la solidaridad.

Que la Madre del Redentor acompañe vuestra misión de \emph{creyentes y apóstoles del Evangelio}. Con fe y confianza invoquemos su intercesión. Oremos a fin de que obtenga las gracias necesarias para cada uno de nosotros, para toda la humanidad, para todos los que sienten más el peso de la vida y de las adversidades. Pidámosle humildemente, llenos de la alegría de la Navidad, que suscite y mantenga en todos los bautizados una fe convencida y coherente.

De modo especial esforcémonos por escuchar sus enseñanzas y seguir el ejemplo de su vida.

Dirijámonos a ella con las palabras de la antífona del tiempo de Navidad: \textquote{Santa Madre del Redentor, sostiene al pueblo que quiere levantarse. Tú, ante el asombro de toda la creación, has engendrado a tu Creador\ldots{} Ten piedad de nosotros, pecadores}.

\subsubsection{AG 1995} JUAN PABLO II

\textbf{\emph{AUDIENCIA GENERAL\\ }}\\ \emph{Miércoles 20 de diciembre de 1995}

 

\emph{(Lectura:\\ evangelio de san Lucas, capítulo 1,\\ versículos 68-69)}

1. Ya se acerca la Navidad del Señor, para la que nos estamos preparando durante estos días de Adviento. La solemnidad de la Navidad nos trae recuerdos de ternura y bondad, suscitando cada vez nueva atención hacia los valores humanos fundamentales: la familia, la vida, la inocencia, la paz y la gratuidad.

La Navidad es la \emph{fiesta de la familia} que, reunida en torno al belén y al árbol, símbolos navideños tradicionales, se redescubre llamada a ser el santuario de la vida y del amor. La Navidad es la \emph{fiesta de los niños,} porque pone de manifiesto \textquote{el sentido profundo de todo nacimiento humano, y la alegría mesiánica constituye así el fundamento y realización de la alegría por cada niño que nace} (\emph{Evangelium vitae}, 1). La Navidad del Señor lleva a redescubrir, además, el valor de la \emph{inocencia,} invitando a los adultos a aprender de los niños a acercarse con asombro y pureza de corazón a la cuna del Salvador, recién nacido.

La Navidad es la \emph{fiesta de la paz,} porque \textquote{la paz verdadera nos viene del cielo} y \textquote{por toda la tierra los cielos destilan dulzura} (\emph{Liturgia de las Horas,} oficio de lectura de Navidad). Los ángeles cantan en Belén: \textquote{Gloria a Dios en las alturas y paz en la tierra a los hombres que él ama} (\emph{Lc} 2, 14). En este tiempo, que invita a la alegría, ¿cómo no pensar con tristeza en los que, por desgracia, en muchas partes del mundo, se hallan aún inmersos en grandes tragedias? ¿Cuándo podrán celebrar una verdadera Navidad? ¿Cuándo podrá la humanidad vivir la Navidad en un mundo completamente reconciliado? Algunos signos de esperanza, gracias a Dios, nos impulsan a proseguir incansablemente en la búsqueda de la paz.

Mi pensamiento se dirige, naturalmente, a Bosnia, donde el acuerdo logrado, aún con límites comprensibles y con notables sacrificios, constituye un gran paso adelante por el camino de la reconciliación y la paz.

La Navidad es también la \emph{fiesta de los regalos:} me imagino la alegría de los niños, y también de los adultos, que reciben un regalo navideño, al sentirse amados y comprometidos a transformarse ellos mismos en don, como el Niño que la Virgen María nos muestra en el belén.

2. Pero estas consideraciones explican sólo en parte el clima festivo y sugestivo de la Navidad. Como ya es sabido, para los creyentes el auténtico fundamento de la alegría de esta fiesta estriba en el hecho de que \emph{el Verbo eterno,} imagen perfecta del Padre \emph{se ha hecho} \textquote{\emph{carne}}, niño frágil solidario con los hombres débiles y mortales. \emph{En Jesús, Dios mismo se ha acercado y permanece con nosotros,} como don incomparable que es preciso acoger con humildad en nuestra vida.

En el nacimiento del Hijo de Dios del seno virginal de una humilde joven, María de Nazaret, los cristianos reconocen la infinita condescendencia del Altísimo hacia el hombre. Ese acontecimiento, junto con la muerte y resurrección de Cristo, constituye el culmen de la historia.

En la carta del apóstol Pablo a los Filipenses encontramos un himno a Cristo, con el que la Iglesia primitiva expresaba la gratitud y el asombro ante el sublime misterio de Dios que se hace solidario con los hombres: \textquote{(Cristo) siendo de condición divina, no retuvo ávidamente el ser igual a Dios, sino que se despojó de sí mismo tomando condición de siervo, haciéndose semejante a los hombres y apareciendo en su porte como hombre: y se humilló a sí mismo, obedeciendo hasta la muerte y muerte de cruz} (\emph{Flp} 2, 6-8).

En el decurso de los primeros siglos, la Iglesia defendió con especial tenacidad este misterio frente a varias herejías que, al negar, de vez en cuando, la verdadera humanidad de Jesús, su real filiación divina, su divinidad o la unidad de su Persona, tendían a vaciar su excepcional y sorprendente contenido y a desvirtuar el insólito y consolador mensaje que trae al hombre de todos los tiempos.

El \emph{Catecismo de la Iglesia católica} nos recuerda que \textquote{el acontecimiento único y totalmente singular de la encarnación del Hijo de Dios no significa que Jesucristo sea en parte Dios y en parte hombre, ni que sea el resultado de una mezcla confusa entre lo divino y lo humano. Él se hizo verdaderamente hombre sin dejar de ser verdaderamente Dios. Jesucristo es verdadero Dios y verdadero hombre} (n. 464).

3. ¿Qué significado tiene para nosotros el evento extraordinario del nacimiento de Jesucristo? ¿Qué \emph{buena nueva} nos trae? ¿A qué metas nos impulsa? San Lucas, el evangelista de la Navidad, en las palabras inspiradas de Zacarías nos presenta la Encarnación como la \emph{visita de Dios:} \textquote{Bendito el Señor Dios de Israel porque \emph{ha visitado y redimido a su pueblo,} y nos ha suscitado una fuerza salvadora en la casa de David, su siervo} (\emph{Lc} 1, 68-69).

Pero ¿qué efectos produce en el hombre la \emph{visita de Dios}? La sagrada Escritura testimonia que cuando el Señor interviene, trae salvación y alegría, libra de la aflicción, infunde esperanza, mira el destino del que recibe la visita y abre perspectivas nuevas de vida y salvación.

\emph{La Navidad es la visita de Dios por excelencia,} pues en este acontecimiento se hace sumamente cercano al hombre mediante su Hijo único, que manifiesta en el rostro de un niño su ternura hacia los pobres y los pecadores. En el Verbo encarnado se ofrece a los hombres la gracia de la adopción como hijos de Dios. San Lucas se preocupa de mostrar que el evento del nacimiento de Jesús cambia realmente la historia y la vida de los hombres, sobre todo de los que lo acogen con corazón sincero: Isabel, Juan Bautista, los pastores, Simeón, Ana y sobre todo María son testigos de las maravillas que Dios obra con su visita.

En María, de manera especial, el evangelista presenta no sólo un modelo que es necesario seguir para acoger a Dios que sale a nuestro encuentro, sino también las perspectivas exultantes que se abren a quien, habiéndolo acogido, está Llamado a \emph{convertirse,} a su vez, \emph{en instrumento de su visita} y heraldo de su salvación: \textquote{Apenas llegó a mis oídos la voz de tu saludo, saltó de gozo el niño en mi seno}, exclama Isabel dirigiéndose a la Virgen, que le lleva en sí misma la visita de Dios (\emph{Lc} 1, 44). La misma alegría invade a los pastores, que van a Belén por invitación del ángel y encuentran al niño con su Madre: vuelven \textquote{glorificando y alabando a Dios} (\emph{Lc} 2, 20), porque saben que el Señor los ha visitado.

A la luz del misterio que nos disponemos a celebrar, expreso a todos el deseo de que acudamos en esta Navidad, como María, a Cristo que viene a \textquote{visitarnos de lo alto} (\emph{Lc} 1, 78), con corazón abierto y disponible, para convertirnos en instrumentos de la alegre visita de Dios para cuantos encontremos en nuestro camino diario.

\subsubsection{****PARA AÑO B**** Urbi et Orbi 1999} \textbf{\emph{MENSAJE URBI ET ORBI}}

\emph{Navidad, 25 de diciembre de 1999}

 

1. \textquote{\emph{Un niño nos ha nacido.\\ un hijo se nos ha dado}} (Is 9, 5) .\\ Hoy resuena en la Iglesia y en el mundo la \textquote{buena noticia} de la Navidad.\\ Resuena con las palabras del profeta Isaías,\\ llamado por esto el \textquote{evangelista} del Antiguo Testamento, \\ el cual, hablando del misterio de la redención,\\ parece entrever los acontecimiento de siete siglos después.\\ Palabras inspiradas por Dios, palabras sorprendentes que recorren la historia, \\ y que hoy, a las puertas del Dos mil, resuenan en toda la tierra\\ anunciando el gran misterio de la Encarnación.

2. \textquote{\emph{Un Niño nos ha nacido}}.\\ Estas palabras proféticas se ven realizadas en la narración del evangelista Lucas, \\ que describe el \textquote{\emph{acontecimiento}} lleno\\ cada vez más de nueva admiración y esperanza.\\ En la noche de Belén,\\ María dio a luz un Niño, al que puso por nombre Jesús.\\ No había lugar para ellos e la pensión;\\ por esto la Madre alumbró al Hijo\\ en una gruta y lo puso en un pesebre .\\ El evangelista Juan, en el Prólogo de su evangelio,\\ penetra en el \textquote{ \emph{misterio} } de este acontecimiento.\\ Aquel que nace en la gruta es el Hijo eterno de Dios.\\ Es la Palabra, que existía en el principio, la Palabra que estaba junto a Dios,\\ la Palabra que era Dios.\\ Todo lo que ha sido hecho, por medio de la Palabra se hizo (cf. 1,1-3).\\ La Palabra eterna, el Hijo de Dios,\\ tomó la naturaleza humana.\\ Dios Padre \textquote{\emph{tanto amó al mundo\\ que le ha dado su Hijo único}} (Jn 3,16).\\ El profeta Isaías al decir: \textquote{\emph{un hijo se nos ha dado}},\\ revela en toda su plenitud \emph{el misterio de Navidad}:\\ le generación eterna de la Palabra en el Padre,\\ su nacimiento en el tiempo por obra del Espíritu Santo.

3. Se ensancha el círculo del misterio :\\ el evangelista Juan afirma: \textquote{\emph{La Palabra se hizo carne,\\ y puso su Morada entre nosotros} } (Jn 1,14)\\ y añade : \textquote{\emph{a todos tos que la recibieron\\ les dio poder de hacerse hijos de Dios,\\ a los que creen en su nombre} } (ibíd. 1,12).\\ Se ensancha el círculo del misterio:\\ el nacimiento del Hijo de Dios es el don sublime,\\ la gracia más grande en favor del hombre,\\ que la mente humana nunca hubiera podido imaginar.\\ Recordando, en este Día santo,\\ el nacimiento de Cristo,\\ vivimos, junto con este acontecimiento,\\ el \textquote{\emph{misterio de la divina adopción}},\\ por obra de Cristo que viene al mundo.\\ Por eso, la Noche y el Día de Navidad\\ son tenidos como \textquote{sagrados } por los hombres que buscan la verdad.\\ Nosotros, cristianos, los consideramos \textquote{santos } reconociendo en ellos la huella inconfundible de Aquel que es Santo, lleno de misericordia y de bondad.

4. Un motivo más se añade este año\\ para considerar más santo este día de gracia:\emph{\\ es el comienzo del Gran Jubileo}.\\ Esta Noche, antes de la Santa Misa,\\ he abierto la Puerta Santa de esta Basílica.\\ Acto simbólico con el cual se inaugura el Año Jubilar,\\ gesto que pone de relieve con elocuencia singular\\ un elemento ya contenido en el misterio de Navidad:\\ ¡\emph{Jesús}, nacido en la pobreza de Belén,\\ Cristo, \emph{el Hijo eterno} que nos ha sido dado por el Padre,\emph{\\ es}, para nosotros y para todos, \emph{la Puerta}!\emph{\\ la Puerta de nuestra salvación},\emph{\\ la Puerta de la vida},\emph{\\ la Puerta de la paz} !\\ Éste es el mensaje de Navidad y el anuncio del Gran Jubileo.

5. Dirigimos la mirada hacia ti, Cristo,\emph{\\ Puerta de nuestra salvación},\\ y te damos gracias por el bien realizado en los años, siglos y milenios pasados.\\ Debemos confesar, sin embargo, que a veces la humanidad ha buscado fuera de ti la Verdad,\\ que se ha fabricado falsas certezas, ha corrido tras ideologías falaces.\\ A veces el hombre ha excluido del propio respeto y amor\\ a hermanos de otras razas o distintos credos,\\ ha negado los derechos fundamentales a las personas y a las naciones.\\ Pero Tú sigues ofreciendo a todos el Esplendor de la Verdad que salva.\\ Te miramos a Ti, Cristo, \emph{Puerta de la Vida},\\ y te damos gracias por los prodigios\\ con que has enriquecido a cada generación.\\ A veces este mundo a veces no respeta y no ama la vida.\\ Pero Tú no te cansas de amarla,\\ más aún, en el misterio de la Navidad vienes a iluminar las mentes\\ para que los legisladores y los gobernantes, hombres y mujeres de buena voluntad se comprometan a acoger, como don precioso, la vida del hombre.\\ Tú vienes a darnos el Evangelio de la Vida.\\ Fijamos los ojos en Ti, Cristo, \emph{Puerta de la paz},\\ mientras, peregrinos en el tiempo,\\ visitamos tantos lugares del dolor y de la guerra, donde reposan las víctimas\\ de violentos conflictos y de crueles exterminios.\\ Tú, Príncipe de la paz,\\ nos invitas a abandonar el insensato uso de las armas, el recurso a la violencia y al odio que han marcado con la muerte a personas, pueblos y continentes.

6. \textquote{\emph{Un hijo se nos ha dado} }.\\ Tú, Padre, nos \emph{has dado a tu Hijo}.\\ Nos lo das también hoy, al alba del nuevo milenio .\\ Él es la Puerta para nosotros.\\ A través de El entramos en una nueva dimensión\\ y alcanzamos la plenitud del destino de la salvación\\ pensado por ti para todos.\\ Precisamente por esto, Padre, nos has dado a tu Hijo,\\ para que el hombre experimente lo que Tú quieres dar en la eternidad,\\ para que el hombre tenga la fuerza de realizar\\ tu arcano misterio de amor.\\ Cristo, Hijo de la Madre siempre Virgen,\\ luz y esperanza de quienes te buscan, aun sin conocerte y de quienes, conociéndote, te buscan cada vez más; Cristo, ¡Tú eres la Puerta!\\ A través de ti,\\ con la fuerza del Espíritu Santo, queremos entrar en el tercer milenio.\\ Tú, Cristo, eres el mismo ayer, hoy y siempre (cf. Hb 13,8)

\subsubsection{UO 1998} \textbf{\emph{MENSAJE URBI ET ORBI}}

\emph{(Navidad 1998)}

 

1. \emph{\textquote{Regem venturum Dominum, venite, adoremus}}\\ \textquote{Venid, adoremos al Rey, al Señor que ha de venir}.\\ Cuántas veces hemos repetido estas palabras\\ durante el tiempo de Adviento,\\ haciéndonos eco de la esperanza de toda la humanidad.

Proyectado hacia el futuro desde sus más remotos orígenes,\\ el hombre busca a Dios, plenitud de la vida. Desde siempre\\ invoca un Salvador que lo libre del mal y de la muerte,\\ que colme su necesidad innata de felicidad.\\ Ya en el jardín del Edén, después del pecado original,\\ Dios Padre, fiel y misericordioso,\\ había preanunciado un Salvador (cf. \emph{Gn} 3, 15),\\ que habría de restablecer la alianza destruida,\\ instaurando una nueva relación\\ de amistad, de entendimiento y de paz.

2. Este gozoso anuncio, confiado a los hijos de Abraham,\\ desde la época de la salida de Egipto (cf. \emph{Ex} 3, 6-8)\\ ha resonado a lo largo de los siglos\\ como un grito de esperanza en boca de los profetas de Israel,\\ que en diversos momentos han recordado al pueblo:\\ \emph{\textquote{Prope est Dominus: venite, adoremus}}.\\ \textquote{El Señor está cerca: ¡venid a adorarlo!}\\ Venid a adorar al Dios que no abandona\\ a quienes lo buscan con sincero corazón\\ y se esfuerzan en observar su ley.\\ Acoged su mensaje,\\ que conforta los espíritus abatidos y desorientados.\\ \emph{Prope est Dominus}: fiel a la antigua promesa,\\ Dios Padre la cumple ahora en el misterio de la Navidad.

3. Sí, su promesa, que ha alimentado\\ la espera confiada de tantos creyentes\\ se ha hecho don en Belén, en el corazón de la Noche Santa.\\ Nos lo ha recordado ayer la liturgia de la Misa:\\ \emph{\textquote{Hodie scietis quia veniet Dominus,}\\ et mane videbitis gloriam eius}.\\ \textquote{Hoy sabréis que el Señor viene:\\ con el nuevo día veréis su gloria}.\\ Esta noche hemos visto la gloria de Dios,\\ proclamada por el canto gozoso de los ángeles;\\ hemos adorado al Rey, Señor del universo,\\ junto con los pastores que vigilaban sus rebaños.\\ Con los ojos de la fe, también nosotros hemos visto,\\ recostado en un pesebre,\\ al Príncipe de la Paz,\\ y junto a Él, María y José\\ en silenciosa adoración.

4. A la multitud de los ángeles, a los pastores extasiados,\\ nos unimos también nosotros hoy, cantando con júbilo:\\ \textquote{\emph{Chistus natus est nobis: venite, adoremus}}.\\ \textquote{Cristo ha nacido por nosotros: venid, adorémosle}\\ Desde la noche de Belén hasta hoy,\\ la Navidad continúa suscitando himnos de alegría,\\ que expresan la ternura de Dios\\ sembrada en el corazón de los hombres.\\ En todas las lenguas del mundo\\ se celebra el acontecimiento más grande y más humilde:\\ el Emmanuel, Dios con nosotros para siempre.

¡Cuántos cantos sugestivos ha inspirado la Navidad\\ en los pueblos y culturas!\\ ¿Quién desconoce las emociones que evocan?\\ Sus melodías hacen a revivir\\ el misterio de la Noche Santa;\\ atestiguan el encuentro entre el Evangelio y los caminos de los hombres.\\ Sí, la Navidad ha entrado en el corazón de los pueblos,\\ que miran hacia Belén con una admiración común.\\ También la Asamblea General de las Naciones Unidas\\ ha reconocido por unanimidad la pequeña población de Judá (cf. \emph{Mt} 2, 6)\\ como la tierra en la que la celebración del nacimiento de Jesús\\ ofrecerá en el 2000 una ocasión singular\\ para proyectos de esperanza y de paz.

5. ¿Cómo no percibir el clamoroso contraste\\ entre la serenidad de los cantos navideños\\ y los muchos problemas del nuestro momento actual?\\ Conocemos los aspectos preocupantes por las crónicas\\ que aparecen cada día en la televisión y los periódicos\\ pasando de un hemisferio del globo al otro:\\ son situaciones tristísimas, a las que frecuentemente\\ no es ajena la culpa e incluso la malicia humana,\\ impregnada de odio fratricida y de violencia absurda.\\ La luz que viene de Belén\\ nos salve del peligro de resignarnos\\ a un panorama tan desconcertante y atormentado.

Que el anuncio de la Navidad aliente\\ a cuantos se esfuerzan por aliviar\\ la situación penosa del Medio Oriente\\ respetando los compromisos internacionales.\\ Que la Navidad refuerce en el mundo\\ el consenso sobre medidas urgentes y adecuadas\\ para detener la producción y el comercio de armas,\\ para defender la vida humana, para desterrar la pena de muerte,\\ para liberar a los niños y adolescentes de toda forma de explotación,\\ para frenar la mano ensangrentada\\ de los responsables de genocidios y crímenes de guerra,\\ para prestar a las cuestiones del medio ambiente,\\ sobre todo tras las recientes catástrofes naturales,\\ la atención indispensable que merecen\\ a fin de salvaguardar la creación y la dignidad del hombre.

6. La alegría de la Navidad, que canta el nacimiento del Salvador,\\ infunda a todos confianza en la fuerza de la verdad\\ y de la perseverancia paciente en hacer el bien.\\ El mensaje divino de Belén resuena para cada uno de nosotros:\\ \textquote{No temáis, pues os anuncio una gran alegría,\ldots{}\\ os ha nacido hoy, en la ciudad de David,\\ un Salvador, que es el Cristo Señor} (\emph{Lc} 2, 10-11).

Hoy resplandece \emph{Urbi et Orbi},\\ en la ciudad de Roma y en el mundo entero,\\ el rostro de Dios; Jesús nos lo revela\\ como Padre que nos ama.\\ Vosotros que buscáis el sentido de la vida;\\ vosotros que lleváis en el corazón la llama\\ de una esperanza de salvación, de libertad y de paz,\\ venid a ver al Niño que ha nacido de María.\\ Él es Dios, nuestro Salvador,\\ el único digno de tal nombre,\\ el único Señor.\\ Ha nacido por nosotros, venid, ¡adorémosle!

\subsubsection{AG } pandoc -s -r html http://www.vatican.va/content/john-paul-ii/es/messages/urbi/documents/hf_jp-ii_mes_25121997_urbi.html -o aaa.tex

\subsubsection{UO 1997 ***PARA AÑO C*** } \emph{\textbf{MENSAJE URBI ET ORBI\\ DEL SANTO PADRE JUAN PABLO II}}

(\emph{Navidad, 25 de diciembre de 1997})

 

1. \textquote{La tierra ha visto a su Salvador}\\ Hoy, Navidad del Señor, vivimos profundamente\\ la verdad de estas palabras: la tierra ha visto a su Salvador.\\ Lo han visto en primer lugar los pastores de Belén\\ que, al anuncio de los ángeles,\\ se apresuraron con alegría hacia la pobre gruta.\\ Era de noche, noche llena de misterio.\\ ¿Qué vieron sus ojos?\\ Un Niño acostado en un pesebre,\\ con María y José solícitos a su lado.\\ Vieron un niño pero, iluminados por la fe,\\ en aquella frágil criatura reconocieron a Dios hecho hombre,\\ y le ofrecieron sus pobres dones.\\ Iniciaron así, sin darse cuente,\\ aquel canto de alabanza al Emmanuel,\\ Dios venido a habitar entre nosotros,\\ que se extendería de generación en generación.\\ Cántico alegre, que es patrimonio de cuantos, hoy,\\ se dirigen espiritualmente a Belén,\\ para celebrar el nacimiento del Señor,\\ y alaban a Dios por las maravillas que ha realizado.\\ También nosotros nos unimos con fe\\ a este singular encuentro de alabanza\\ que, según la tradición, se renueva cada año en Navidad,\\ aquí, en la Plaza San Pedro, y que concluye con la bendición\\ que el Obispo de Roma imparte \emph{Urbi et Orbi}:

\emph{Urbi}, es decir, a esta Ciudad que, gracias al ministerio\\ de los santos Pedro y Pablo,\\ ha \textquote{visto} de manera singular\\ al Salvador del mundo.

\emph{Et Orbi}, es decir, al mundo entero,\\ en el que se ha difundido ampliamente\\ la Buena Nueva de la salvación,\\ que ha llegado ya hasta los confines extremos de la tierra.\\ La alegría de Navidad ha llegado a ser así\\ patrimonio de innumerables pueblos y naciones.\\ En verdad, \textquote{los confines de la tierra\\ han contemplado la victoria de nuestro Dios} (Sal 97/98,3)

2. A todos, pues, va dirigido el mensaje de la solemnidad de hoy.\\ Todos están llamados a participar\\ de la alegría de la Navidad.\\ \textquote{Aclama al Señor, tierra entera,\\ gritad, vitoread, tocad} (Sal 97/98,4).\\ Día de extraordinaria alegría es la Navidad.\\ Esta alegría ha inundado los corazones humanos\\ y ha tenido múltiples expresiones\\ en la historia y en la cultura de las naciones cristianas:\\ en el canto litúrgico y popular, en la pintura,\\ en la literatura y en cada campo del arte.\\ Para la formación cristiana de generaciones enteras,\\ tienen gran importancia las tradiciones y los cantos,\\ las representaciones sacras y, entre todas, el portal.\\ El cántico de los ángeles en Belén\\ ha encontrado así un eco universal y multiforme\\ en las costumbres, mentalidades y culturas de cada tiempo.\\ Ha encontrado un eco en el corazón de cada creyente.

3. Hoy, día de alegría para todos,\\ día lleno de tantos llamamientos a la paz y la fraternidad,\\ se hacen más intensos e incisivos el clamor y la súplica\\ de los pueblos que anhelan la libertad y la concordia,\\ en situaciones de preocupante violencia étnica y política.\\ Hoy resuena más fuerte la voz\\ de quienes están comprometidos generosamente\\ en derribar barreras de miedo y de agresividad,\\ para promover la comprensión entre hombres\\ de distinto origen, raza y credo religioso.\\ Hoy día nos resultan más dramáticos los sufrimientos\\ de gente que huye a las montañas de su propia tierra\\ o busca atracar a las costas de los Países vecinos,\\ para perseguir la esperanza incluso leve\\ de una vida menos precaria y más segura.\\ Más angustioso es hoy el silencio, lleno de tensiones,\\ de la multitud, cada vez mayor, de nuevos pobres:\\ hombres y mujeres sin trabajo y sin casa,\\ muchachos y niños ofendidos y profanados,\\ adolescentes enrolados en las guerras de los adultos,\\ víctimas jóvenes de la droga\\ o atraídos por mitos falaces.\\ Hoy es Navidad, día de confianza para pueblos por largo tiempo divididos,\\ que finalmente se han vuelto a encontrar y tratar.\\ Son perspectivas a menudo tímidas y frágiles,\\ diálogos lentos y arduos,\\ pero animados por la esperanza\\ de alcanzar finalmente acuerdos\\ respetuosos de los derechos y de los deberes de todos.

4. ¡Es Navidad! Esta humanidad nuestra descarriada,\\ en camino hacia el tercer milenio,\\ te espera, Niño de Belén,\\ que vienes a manifestar el amor del Padre.\\ Tú, Rey de la paz, nos invitas hoy a no tener miedo\\ y abrir nuestros corazones a perspectivas de esperanza.\\ Por esto \textquote{cantemos al Señor un cántico nuevo,\\ porque ha hecho maravillas} (cf. Sal 97/98,1).\\ Este es el mayor prodigio obrado por Dios:\\ El mismo se hizo hombre y nació en la noche de Belén,\\ ofreció por nosotros su vida en la Cruz,\\ resucitó al tercer día según las Escrituras\\ y a través de la Eucaristía permanece con nosotros\\ hasta el fin de los tiempos.\\ En verdad \textquote{\ldots{} la palabra se hizo carne,\\ y acampó entre nosotros} (Jn 1,14).\\ La luz de la fe nos ayuda a reconocer\\ en el Niño recién nacido\\ al Dios eterno e inmortal.\\ Somos testigos de su gloria.\\ De omnipotente como era,\\ se revistió de extrema pobreza.\\ Esta es nuestra fe, la fe de la Iglesia,\\ que nos permite confesar la gloria del Hijo unigénito de Dios,\\ aunque nuestros ojos no vean más que al hombre,\\ un Niño nacido en la gruta de Belén.

Dios hecho hombre yace hoy en el pesebre\\ y el universo lo contempla silenciosamente.\\ ¡Que la humanidad pueda reconocerlo como a su Salvador!

\subsubsection{Adviento en General}

CELEBRACIÓN DE LAS PRIMERAS VÍSPERAS DEL I DOMINGO DE ADVIENTO

\emph{\textbf{HOMILÍA DE SU SANTIDAD BENEDICTO XVI}\\[2\baselineskip]Basílica de San Pedro\\ Domingo 1 de diciembre de 2007}

\emph{Queridos hermanos y hermanas:}

El Adviento es, por excelencia, el tiempo de la esperanza. Cada año, esta actitud fundamental del espíritu se renueva en el corazón de los cristianos que, mientras se preparan para celebrar la gran fiesta del nacimiento de Cristo Salvador, reavivan la esperanza de su vuelta gloriosa al final de los tiempos. La primera parte del Adviento insiste precisamente en la \emph{parusía}, la última venida del Señor. Las antífonas de estas primeras Vísperas, con diversos matices, están orientadas hacia esa perspectiva. La lectura breve, tomada de la primera carta de san Pablo a los Tesalonicenses (\emph{1 Ts} 5, 23-24) hace referencia explícita a la venida final de Cristo, usando precisamente el término griego \emph{parusía} (v. 23). El Apóstol exhorta a los cristianos a ser irreprensibles, pero sobre todo los anima a confiar en Dios, que es \textquote{fiel} (v. 24) y no dejará de realizar la santificación en quienes correspondan a su gracia.

Toda esta liturgia vespertina invita a la esperanza, indicando en el horizonte de la historia la luz del Salvador que viene: \textquote{Aquel día brillará una gran luz} (segunda antífona); \textquote{vendrá el Señor con toda su gloria} (tercera antífona); \textquote{su resplandor ilumina toda la tierra} (antífona del Magníficat). Esta luz, que proviene del futuro de Dios, ya se ha manifestado en la plenitud de los tiempos. Por eso nuestra esperanza no carece de fundamento, sino que se apoya en un acontecimiento que se sitúa en la historia y, al mismo tiempo, supera la historia: el acontecimiento constituido por Jesús de Nazaret. El evangelista san Juan aplica a Jesús el título de \textquote{luz}: es un título que pertenece a Dios. En efecto, en el Credo profesamos que Jesucristo es \textquote{Dios de Dios, Luz de Luz}.

Al tema de la esperanza he dedicado mi segunda encíclica, publicada ayer. Me alegra entregarla idealmente a toda la Iglesia en este primer domingo de Adviento a fin de que, durante la preparación para la santa Navidad, tanto las comunidades como los fieles individualmente puedan leerla y meditarla, de modo que redescubran \emph{la belleza y la profundidad de la esperanza cristiana}. En efecto, la esperanza cristiana está inseparablemente unida al conocimiento del rostro de Dios, el rostro que Jesús, el Hijo unigénito, nos reveló con su encarnación, con su vida terrena y su predicación, y sobre todo con su muerte y resurrección.

La esperanza verdadera y segura está fundamentada en la fe en Dios Amor, Padre misericordioso, que \textquote{tanto amó al mundo que le dio a su Hijo unigénito} (\emph{Jn} 3, 16), para que los hombres, y con ellos todas las criaturas, puedan tener vida en abundancia (cf. \emph{Jn} 10, 10). Por tanto, el Adviento es tiempo favorable para redescubrir una esperanza no vaga e ilusoria, sino cierta y fiable, por estar \textquote{anclada} en Cristo, Dios hecho hombre, roca de nuestra salvación.

Como se puede apreciar en el Nuevo Testamento y en especial en las cartas de los Apóstoles, desde el inicio una nueva esperanza distinguió a los cristianos de las personas que vivían la religiosidad pagana. San Pablo, en su carta a los Efesios, les recuerda que, antes de abrazar la fe en Cristo, estaban \textquote{sin esperanza y sin Dios en este mundo} (\emph{Ef} 2, 12). Esta expresión resulta sumamente actual para el paganismo de nuestros días: podemos referirla en particular al nihilismo contemporáneo, que corroe la esperanza en el corazón del hombre, induciéndolo a pensar que dentro de él y en torno a él reina la nada: nada antes del nacimiento y nada después de la muerte.

En realidad, si falta Dios, falla la esperanza. Todo pierde sentido. Es como si faltara la dimensión de profundidad y todas las cosas se oscurecieran, privadas de su valor simbólico; como si no \textquote{destacaran} de la mera materialidad. Está en juego la relación entre la existencia aquí y ahora y lo que llamamos el \textquote{más allá}. El más allá no es un lugar donde acabaremos después de la muerte, sino la realidad de Dios, la plenitud de vida a la que todo ser humano, por decirlo así, tiende. A esta espera del hombre Dios ha respondido en Cristo con el don de la esperanza.

El hombre es la única criatura libre de decir sí o no a la eternidad, o sea, a Dios. El ser humano puede apagar en sí mismo la esperanza eliminando a Dios de su vida. ¿Cómo puede suceder esto? ¿Cómo puede acontecer que la criatura \textquote{hecha para Dios}, íntimamente orientada a él, la más cercana al Eterno, pueda privarse de esta riqueza?

Dios conoce el corazón del hombre. Sabe que quien lo rechaza no ha conocido su verdadero rostro; por eso no cesa de llamar a nuestra puerta, como humilde peregrino en busca de acogida. El Señor concede un nuevo tiempo a la humanidad precisamente para que todos puedan llegar a conocerlo. Este es también \emph{el sentido de un nuevo año litúrgico que comienza}: es un don de Dios, el cual quiere revelarse de nuevo en el misterio de Cristo, mediante la Palabra y los sacramentos.

Mediante la Iglesia quiere hablar a la humanidad y salvar a los hombres de hoy. Y lo hace saliendo a su encuentro, para \textquote{buscar y salvar lo que estaba perdido} (\emph{Lc} 19, 10). Desde esta perspectiva, la celebración del Adviento es la respuesta de la Iglesia Esposa a la iniciativa continua de Dios Esposo, \textquote{que es, que era y que viene} (\emph{Ap} 1, 8). A la humanidad, que ya no tiene tiempo para él, Dios le ofrece otro tiempo, un nuevo espacio para volver a entrar en sí misma, para ponerse de nuevo en camino, para volver a encontrar el sentido de la esperanza.

He aquí el descubrimiento sorprendente: mi esperanza, nuestra esperanza, está precedida por la espera que Dios cultiva con respecto a nosotros. Sí, Dios nos ama y precisamente por eso espera que volvamos a él, que abramos nuestro corazón a su amor, que pongamos nuestra mano en la suya y recordemos que somos sus hijos.

Esta espera de Dios precede siempre a nuestra esperanza, exactamente como su amor nos abraza siempre primero (cf. \emph{1 Jn} 4, 10). En este sentido, la esperanza cristiana se llama \textquote{teologal}: Dios es su fuente, su apoyo y su término. ¡Qué gran consuelo nos da este misterio! Mi Creador ha puesto en mi espíritu un reflejo de su deseo de vida para todos. Cada hombre está llamado a esperar correspondiendo a lo que Dios espera de él. Por lo demás, la experiencia nos demuestra que eso es precisamente así. ¿Qué es lo que impulsa al mundo sino la confianza que Dios tiene en el hombre? Es una confianza que se refleja en el corazón de los pequeños, de los humildes, cuando a través de las dificultades y las pruebas se esfuerzan cada día por obrar de la mejor forma posible, por realizar un bien que parece pequeño, pero que a los ojos de Dios es muy grande: en la familia, en el lugar de trabajo, en la escuela, en los diversos ámbitos de la sociedad. La esperanza está indeleblemente escrita en el corazón del hombre, porque Dios nuestro Padre es vida, y estamos hechos para la vida eterna y bienaventurada.

Todo niño que nace es signo de la confianza de Dios en el hombre y es una confirmación, al menos implícita, de la esperanza que el hombre alberga en un futuro abierto a la eternidad de Dios. A esta esperanza del hombre respondió Dios naciendo en el tiempo como un ser humano pequeño. San Agustín escribió: \textquote{De no haberse tu Verbo hecho carne y habitado entre nosotros, hubiéramos podido juzgarlo apartado de la naturaleza humana y desesperar de nosotros} (\emph{Confesiones} X, 43, 69, citado en \emph{Spe salvi}, 29).

Dejémonos guiar ahora por Aquella que llevó en su corazón y en su seno al Verbo encarnado. ¡Oh María, Virgen de la espera y Madre de la esperanza, reaviva en toda la Iglesia el espíritu del Adviento, para que la humanidad entera se vuelva a poner en camino hacia Belén, donde vino y de nuevo vendrá a visitarnos el Sol que nace de lo alto (cf. \emph{Lc} 1, 78), Cristo nuestro Dios! Amén.

\subsubsection{Alter Navidad} \emph{Plaza de San Pedro}\\ \emph{Miércoles 18 de diciembre de 2013}


 

\emph{Queridos hermanos y hermanas, ¡buenos días!}

Este encuentro nuestro tiene lugar en el clima espiritual del Adviento, que se hace más intenso por la Novena de la Santa Navidad, que estamos viviendo en estos días y que nos conduce a las fiestas navideñas. Por ello, hoy desearía reflexionar con vosotros sobre el Nacimiento de Jesús, fiesta de la confianza y de la esperanza, que supera la incertidumbre y el pesimismo. Y la razón de nuestra esperanza es ésta: Dios está con nosotros y Dios se fía aún de nosotros. Pero pensad bien en esto: Dios está con nosotros y Dios se fía aún de nosotros. Es generoso este Dios Padre. Él viene a habitar con los hombres, elige la tierra como morada suya para estar junto al hombre y hacerse encontrar allí donde el hombre pasa sus días en la alegría y en el dolor. Por lo tanto, la tierra ya no es sólo un \textquote{valle de lágrimas}, sino el lugar donde Dios mismo puso su tienda, es el lugar del encuentro de Dios con el hombre, de la solidaridad de Dios con los hombres.

Dios quiso compartir nuestra condición humana hasta el punto de hacerse una cosa sola con nosotros en la persona de Jesús, que es verdadero hombre y verdadero Dios. Pero hay algo aún más sorprendente. La presencia de Dios en medio de la humanidad no se realiza en un mundo ideal, idílico, sino en este mundo real, marcado por muchas cosas buenas y malas, marcado por divisiones, maldad, pobreza, prepotencias y guerras. Él eligió habitar nuestra historia así como es, con todo el peso de sus límites y de sus dramas. Actuando así demostró de modo insuperable su inclinación misericordiosa y llena de amor hacia las creaturas humanas. Él es el Dios-con-nosotros; Jesús es Dios-con-nosotros. ¿Creéis vosotros esto? Hagamos juntos esta profesión: Jesús es Dios-con-nosotros. Jesús es Dios-con-nosotros desde siempre y para siempre con nosotros en los sufrimientos y en los dolores de la historia. El nacimiento de Jesús es la manifestación de que Dios \textquote{tomó partido} de una vez para siempre de la parte del hombre, para salvarnos, para levantarnos del polvo de nuestras miserias, de nuestras dificultades, de nuestros pecados.

De aquí viene el gran \textquote{regalo} del Niño de Belén: Él nos trae una energía espiritual, una energía que nos ayuda a no hundirnos en nuestras fatigas, en nuestras desesperaciones, en nuestras tristezas, porque es una energía que caldea y transforma el corazón. El nacimiento de Jesús, en efecto, nos trae la buena noticia de que somos amados inmensamente y singularmente por Dios, y este amor no sólo nos lo da a conocer, sino que nos lo dona, nos lo comunica.

De la contemplación gozosa del misterio del Hijo de Dios nacido por nosotros, podemos sacar dos consideraciones.

La primera es que si en Navidad Dios se revela no como uno que está en lo alto y que domina el universo, sino como Aquél que se abaja, desciende sobre la tierra pequeño y pobre, significa que para ser semejantes a Él no debemos ponernos sobre los demás, sino, es más, abajarnos, ponernos al servicio, hacernos pequeños con los pequeños y pobres con los pobres. Pero es algo feo cuando se ve a un cristiano que no quiere abajarse, que no quiere servir. Un cristiano que se da de importante por todos lados, es feo: ese no es cristiano, ese es pagano. El cristiano sirve, se abaja. Obremos de manera que estos hermanos y hermanas nuestros no se sientan nunca solos.

La segunda consecuencia: si Dios, por medio de Jesús, se implicó con el hombre hasta el punto de hacerse como uno de nosotros, quiere decir que cualquier cosa que hagamos a un hermano o a una hermana la habremos hecho a Él. Nos lo recordó Jesús mismo: quien haya alimentado, acogido, visitado, amado a uno de los más pequeños y de los más pobres entre los hombres, lo habrá hecho al Hijo de Dios.

Encomendémonos a la maternal intercesión de María, Madre de Jesús y nuestra, para que nos ayude en esta Santa Navidad, ya cercana, a reconocer en el rostro de nuestro prójimo, especialmente de las personas más débiles y marginadas, la imagen del Hijo de Dios hecho hombre.

\subsubsection{Navidad, pesebre} \emph{\textbf{AUDIENCIA GENERAL}}

\emph{Aula Pablo VI\\ Miércoles, 18 de diciembre de 2019}


\begin{center}\rule{0.5\linewidth}{\linethickness}\end{center}

 

\emph{Queridos hermanos y hermanas, ¡buenos días!}

Dentro de una semana será Navidad. En estos días, mientras corremos para hacer los preparativos de la fiesta, podemos preguntarnos: \textquote{¿Cómo me preparo para el nacimiento del festejado?} Un modo sencillo pero eficaz de prepararse es \emph{hacer el belén} Este año yo también he seguido este camino: fui a Greccio, donde San Francisco hizo el primer belén, con los lugareños. Y escribí una carta para recordar el significado de esta tradición, lo que significa el belén en el tiempo de Navidad.

En efecto, el pesebre \textquote{es como un Evangelio vivo} (Carta apostólica \emph{Admirabile signum}, 1). Lleva el Evangelio a los lugares donde uno vive: a las casas, a las escuelas, a los lugares de trabajo y de reunión, a los hospitales y a las residencias de ancianos, a las cárceles y a las plazas. Y allí donde vivimos nos recuerda algo esencial: que Dios no permaneció invisible en el cielo, sino que vino a la Tierra, se hizo hombre, un niño. Hacer el pesebre es \emph{celebrar la cercanía de Dios}. Dios siempre estuvo cerca de su pueblo, pero cuando se encarnó y nació, estuvo muy cerca, muy cerca. Hacer el belén es celebrar la cercanía de Dios, es redescubrir que Dios es real, concreto, vivo y palpitante. Dios no es un señor lejano ni un juez distante, sino Amor humilde, descendido hasta nosotros. El Niño en el pesebre nos transmite su ternura. Algunas figuritas representan al Niño con los brazos abiertos, para decirnos que Dios vino a abrazar nuestra humanidad. Entonces es bonito estar delante del pesebre y allí confiar nuestras vidas al Señor, hablarle de las personas y situaciones que nos importan, hacer con Él un balance del año que está llegando a su fin, compartir nuestras expectativas y preocupaciones.

Junto a Jesús vemos a la Virgen y a san José. Podemos imaginar los pensamientos y sentimientos que tuvieron cuando el Niño nació en la pobreza: alegría, pero también consternación. Y también podemos invitar a la Sagrada Familia a nuestra casa, donde hay alegrías y preocupaciones, donde cada día nos levantamos, comemos y dormimos cerca de nuestros seres queridos. El pesebre es un \emph{Evangelio doméstico}. La palabra pesebre significa literalmente \textquote{comedero}, mientras que la ciudad del pesebre, Belén, significa \textquote{casa del pan}. Pesebre y casa del pan: el belén que hacemos en casa, donde compartimos comida y afecto, nos recuerda que Jesús es el alimento, el pan de vida (cf. \emph{Jn} 6,34). Es Él quien alimenta nuestro amor, es Él quien da a nuestras familias la fuerza para seguir adelante y perdonarnos.

El pesebre nos ofrece otra enseñanza de vida. En los ritmos de hoy, a veces frenéticos, \emph{es una invitación a la contemplación}. Nos recuerda la importancia de detenernos. Porque sólo cuando sabemos recogernos podemos acoger lo que cuenta en la vida. Sólo si dejamos el estruendo del mundo fuera de nuestras casas nos abrimos a escuchar a Dios, que habla en silencio. El pesebre es actual, es la actualidad de cada familia. Ayer me regalaron una figura de un belén especial, una pequeña, llamada: \textquote{Dejemos descansar a mamá}. Representaba a la Virgen dormida y a José que hacía que el Niño se durmiera. Cuántos de vosotros tienen que repartir la noche entre marido y mujer por el niño o la niña que llora, llora, llora, llora. \textquote{Dejemos que mamá descanse} es la ternura de una familia, de un matrimonio.

El pesebre es más actual que nunca, cuando cada día se fabrican en el mundo tantas armas y tantas imágenes violentas que entran por los ojos y el corazón. El pesebre es, en cambio, una \emph{imagen artesanal de la paz}. Por eso es un Evangelio vivo.

Queridos hermanos y hermanas, del pesebre podemos sacar también una enseñanza sobre el sentido mismo de la vida. Vemos escenas cotidianas: los pastores con las ovejas, los herreros que baten el yunque, los molineros que hacen pan; a veces se insertan paisajes y situaciones de nuestros territorios. Está bien, porque el pesebre nos recuerda que Jesús viene a nuestra vida concreta. Y esto es importante. Hacer un pequeño belén, en casa, siempre, porque es el recuerdo de Dios que vino entre nosotros, nació entre nosotros, nos acompaña en la vida, es hombre como nosotros, se hizo hombre como nosotros. En la vida diaria ya no estamos solos, Él vive con nosotros. No cambia mágicamente las cosas pero, si lo acogemos, todo puede cambiar. Os deseo, entonces, que hacer el pesebre sea la ocasión de invitar a Jesús a la vida. Cuando hacemos el belén en casa, es como si abriéramos la puerta y dijéramos: \textquote{Jesús, ¡entra!}, es hacer concreta esta cercanía, esta invitación a Jesús para que venga a nuestra vida. Porque si Él habita nuestra vida, la vida renace. Y si la vida renace es de verdad Navidad. ¡Feliz Navidad a todos!

\section{I Adviento}


\section{III Adviento} 

\subsubsection{Ángelus ()}

\section{IV Adviento} 


\section{Nochebuena} 

\begin{center}\rule{0.5\linewidth}{\linethickness}\end{center}

\textbf{\emph{FELICITACIÓN NAVIDEÑA TRAS EL MENSAJE URBI ET ORBI}}

A todos ustedes, queridos hermanos y hermanas, venidos de todas partes del mundo a esta Plaza, y a cuantos desde distintos países se unen a nosotros a través de los medios de comunicación social, les deseo Feliz Navidad.

En este día, iluminado por la esperanza evangélica que proviene de la humilde gruta de Belén, pido para todos ustedes el don navideño de la alegría y de la paz: para los niños y los ancianos, para los jóvenes y las familias, para los pobres y marginados. Que Jesús, que vino a este mundo por nosotros, consuele a los que pasan por la prueba de la enfermedad y el sufrimiento y sostenga a los que se dedican al servicio de los hermanos más necesitados. ¡Feliz Navidad a todos!

\section{Misa de la Aurora} 


\subsubsection{Homilía ()}

\subsubsection{Homilía ()}

\section{Navidad: Día} 



\subsubsection{Homilía ()}

\section{1 de enero}  SOLEMNIDAD DE LA MADRE DE DIOS\\ XI JORNADA MUNDIAL DE LA PAZ

\emph{\textbf{HOMILÍA DE SU SANTIDAD PABLO VI\\ EN LA MISA DE LA PAZ}\\[2\baselineskip]}


\section{6 de enero} 



\subsubsection{Homilía ()}

\subsubsection{Homilía ()}

\section{Bautismo del Señor} 



\subsubsection{Homilía ()}

\subsubsection{Homilía ()}
