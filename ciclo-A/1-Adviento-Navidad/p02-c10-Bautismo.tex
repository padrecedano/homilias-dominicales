\chapter{Bautismo del Señor (A)}

\section{Lecturas}

PRIMERA LECTURA

Del libro del profeta Isaías 42, 1-4. 6-7

Mirad a mi siervo, en quien me complazco

Esto dice el Señor:

«Mirad a mi Siervo, a quien sostengo;

mi elegido, en quien me complazco.

He puesto mi espíritu sobre él,

manifestará la justicia a las naciones.

No gritará, no clamará,

no voceará por las calles.

La caña cascada no la quebrará,

la mecha vacilante no la apagará.

Manifestará la justicia con verdad.

No vacilará ni se quebrará,

hasta implantar la justicia en el país.

En su ley esperan las islas.

«Yo, el Señor,

te he llamado en mi justicia,

te cogí de la mano,

te formé e hice de ti alianza de un pueblo

y luz de las naciones,

para que abras los ojos de los ciegos,

saques a los cautivos de la cárcel,

de la prisión a los que habitan en tinieblas».

SALMO RESPONSORIAL

Salmo 28, 1b y 2. 3ac-4. 3b y 9c-10

El Señor bendice a su pueblo con la paz

℣. Hijos de Dios, aclamad al Señor,

aclamad la gloria del nombre del Señor,

postraos ante el Señor en el atrio sagrado. ℟.

℣. La voz del Señor sobre las aguas,

el Señor sobre las aguas torrenciales.

La voz del Señor es potente,

la voz del Señor es magnífica. ℟.

℣. El Dios de la gloria ha tronado.

En su templo un grito unánime: \textquote{¡Gloria!}

El Señor se sienta sobre las aguas del diluvio,

el Señor se sienta como rey eterno. ℟.

SEGUNDA LECTURA

De los Hechos de los Apóstoles 10, 34-38

Ungido por Dios con la fuerza del Espíritu Santo

En aquellos días, Pedro tomó la palabra y dijo:

«Ahora comprendo con toda verdad que Dios no hace acepción de personas,
sino que acepta al que lo teme y practica la justicia, sea de la nación
que sea. Envió su palabra a los hijos de Israel, anunciando la Buena
Nueva de la paz que traería Jesucristo, el Señor de todos.

Vosotros conocéis lo que sucedió en toda Judea, comenzando por Galilea,
después del bautismo que predicó Juan. Me refiero a Jesús de Nazaret,
ungido por Dios con la fuerza del Espíritu Santo, que pasó haciendo el
bien y curando a todos los oprimidos por el diablo, porque Dios estaba
con él».

EVANGELIO

Del Santo Evangelio según san Mateo 3, 13-17

Se bautizó Jesús y vio que el Espíritu de Dios se posaba sobre él

En aquel tiempo, vino Jesús desde Galilea al Jordán y se presentó a Juan
para que lo bautizara.

Pero Juan intentaba disuadirlo diciéndole:

\textquote{Soy yo el que necesito que tú me bautices, ¿y tú acudes a mí?}.

Jesús le contestó:

\textquote{Déjalo ahora. Conviene que así cumplamos toda justicia}.

Entonces Juan se lo permitió. Apenas se bautizó Jesús, salió del agua;
se abrieron los cielos y vio que el Espíritu de Dios bajaba como una
paloma y se posaba sobre él.

Y vino una voz de los cielos que decía:

«Este es mi Hijo amado, en quien me
complazco».

\section{Comentarios Patrísticos}

\subsection{San Gregorio de Neocesarea, obispo}

Vino a nosotros el que es el esplendor de la gloria del Padre

Homilía 4 en la santa Teofanía: PG 10, 1182-1183.

Estando tú presente, me es imposible callar, pues yo soy la voz, y precisamente \emph{la voz que grita en el desierto: preparad el camino del Señor. Soy yo el que necesita que tú me bautices, ¿y tú acudes a mí?} Al nacer, yo hice fecunda la esterilidad de la madre que me engendró, y, cuando todavía era un niño, procuré medicina a la mudez de mi padre, recibiendo de ti, niño, la gracia de hacer milagros.

Por tu parte, nacido de María la Virgen según quisiste y de la manera que tú solo conociste, no menoscabaste su virginidad, sino que la preservaste y se la regalaste junto con el apelativo de Madre. Ni la virginidad obstaculizó tu nacimiento ni el nacimiento lesionó la virginidad, sino que ambas realidades: nacimiento y virginidad ---realidades contradictorias si las hay---, firmaron un pacto, porque para ti, Creador de la naturaleza, esto es fácil y hacedero.

Yo soy solamente hombre, partícipe de la gracia divina; tú, en cambio, eres a la vez Dios y hombre, pues eres benigno y amas con locura el género humano. \emph{Soy yo el que necesita que tú me bautices, ¿y tú acudes a mí?} Tú que eras al principio, y estabas junto a Dios y eras Dios mismo; tú que eres el esplendor de la gloria del Padre; tú que eres la imagen perfecta del padre perfecto; tú que eres \emph{la luz verdadera, que alumbra a todo hombre que viene a este mundo;} tú que para estar en el mundo viniste donde ya estabas; tú que te hiciste carne sin convertirte en carne; tú que acampaste entre nosotros y te hiciste visible a tus siervos en la condición de esclavo; tú que, con tu santo nombre como con un puente, uniste el cielo y la tierra: ¿tú acudes a mí? ¿Tú, tan grande, a un hombre como yo?, ¿el Rey al precursor?, ¿el Señor al siervo?

Pues aunque tú no te hayas avergonzado de nacer en las humildes condiciones de la humanidad, yo no puedo traspasar los límites de la naturaleza. Tengo conciencia del abismo que separa la tierra del Creador. Conozco la diferencia existente entre el polvo de la tierra y el Hacedor. Soy consciente de que la claridad de tu sol de justicia me supera con mucho a mí, que soy la lámpara de tu gracia. Y aun cuando estés revestido de la blanca nube del cuerpo, reconozco no obstante tu dominación. Confieso mi condición servil y proclamo tu magnificencia. Reconozco la perfección de tu dominio, y conozco mi propia abyección y vileza. \emph{No soy digno de desatar la correa de tu sandalia;} ¿cómo, pues, voy a atreverme a tocar la inmaculada coronilla de tu cabeza? ¿Cómo voy a extender sobre ti mi mano derecha, sobre ti que extendiste los cielos como una tienda y cimentaste sobre las aguas la tierra? ¿Cómo abriré mi mano de siervo sobre tu divina cabeza? ¿Cómo lavar al inmaculado y exento de todo pecado? ¿Cómo iluminar a la misma luz? ¿Qué oración pronunciaré sobre ti, sobre ti que acoges incluso las plegarias de los que no te conocen?

\subsection{San Pedro Crisólogo, obispo}

El que por nosotros quiso nacer

no quiso ser ignorado por nosotros

Sermón 160: PL 52, 620-622.

Aunque en el mismo misterio del nacimiento del Señor se dieron insignes testimonios de su divinidad, sin embargo, la solemnidad que celebramos manifiesta y revela de diversas formas que Dios ha asumido un cuerpo humano, para que nuestra inteligencia, ofuscada por tantas obscuridades, no pierda por su ignorancia lo que por gracia ha merecido recibir y poseer.

Pues el que por nosotros quiso nacer no quiso ser ignorado por nosotros; y por esto se manifestó de tal forma que el gran misterio de su bondad no fuera ocasión de un gran error.

Hoy el mago encuentra llorando en la cuna a aquel que, resplandeciente, buscaba en las estrellas. Hoy el mago contempla claramente entre pañales a aquel que, encubierto, buscaba pacientemente en los astros.

Hoy el mago discierne con profundo asombro lo que allí contempla: el cielo en la tierra, la tierra en el cielo; el hombre en Dios, y Dios en el hombre; y a aquel que no puede ser encerrado en todo el universo incluido en un cuerpo de niño. Y, viendo, cree y no duda; y lo proclama con sus dones místicos: el incienso para Dios, el oro para el Rey, y la mirra para el que morirá.

Hoy el gentil, que era el último, ha pasado a ser el primero, pues entonces la fe de los magos consagró la creencia de las naciones.

Hoy Cristo ha entrado en el cauce del Jordán para lavar el pecado del mundo. El mismo Juan atestigua que Cristo ha venido para esto: \emph{Éste es el Cordero de Dios, que quita el pecado del mundo}. Hoy el siervo recibe al Señor, el hombre a Dios, Juan a Cristo; el que no puede dar el perdón recibe a quien se lo concederá.

Hoy, como afirma el profeta, \emph{la voz del Señor sobre las aguas}. ¿Qué voz? \emph{Este es mi Hijo, el amado, mi predilecto}.

Hoy el Espíritu Santo se cierne sobre las aguas en forma de paloma, para que, así como la paloma de Noé anunció el fin del diluvio, de la misma forma ésta fuera signo de que ha terminado el perpetuo naufragio del mundo. Pero a diferencia de aquélla, que sólo llevaba un ramo de olivo caduco, ésta derramará la enjundia completa del nuevo crisma en la cabeza del Autor de la nueva progenie, para que se cumpliera aquello que predijo el profeta: \emph{Por eso el Señor, tu Dios, te ha ungido con aceite de júbilo entre todos tus compañeros}.

Hoy Cristo, al convertir el agua en vino, comienza los signos celestes. Pero el agua había de convertirse en el misterio de la sangre, para que Cristo ofreciese a los que tienen sed la pura bebida del vaso de su cuerpo, y se cumpliese lo que dice el profeta: \emph{Y mi copa rebosa}.

El Hijo llamó a su siervo Juan,

que se acercó y puso su mano derecha

sobre la cabeza de Aquél que lo había creado:

«¿Qué podré decir,

cómo podré yo bautizarte, Señor mío?

Si digo: en el nombre del Padre,

he aquí que tu estás en tu Padre.

Si digo: en el nombre del Hijo:

he aquí que tú eres este Hijo amado.

Y si digo: en el nombre del Espíritu,

este espíritu está contigo.»

\ldots{}

El Hijo que ha creado toda la creación,

ha sido bautizado y ha emergido de las aguas.

He aquí el Esposo.

Olvida tu pueblo y la casa paterna,

porque el rey se alegra de tu belleza.

(De la Liturgia Sirio-Occidental).


\section{Homilías}

\subsection{Juan Pablo II, papa}

\subsubsection{Homilía (1981): Hijos de Dios por el Bautismo}

Domingo 11 de enero de 1981.

\textquote{Este es mi Hijo muy amado, en quien tengo mis complacencias} (\emph{Mt} 3, 17).

Las palabras del \textbf{Evangelio} que acabamos de oír van a realizarse también en estos queridos niños a quienes me dispongo a administrar el bautismo. Jesús es el primogénito de muchos hermanos (cf. \emph{Rom} 8, 29); lo que se realizó en El se repite misteriosamente en cada uno de los que seguimos sus huellas y llevamos su nombre, el nombre de cristianos.

Cuando Cristo entra en el Jordán, se oye la voz del Padre que lo llama predilecto suyo y se da cumplimiento así a la profecía del Siervo de Yavé que Isaías proclama en la primera lectura; y desciende el Espíritu Santo en forma de paloma para dar comienzo visible y solemne a la misión mesiánica del Hijo de Dios. Como en El, así también ha ocurrido en nosotros; así va a suceder en estos pequeños que están aquí ante nosotros, generación nueva del Pueblo de Dios destinada a crecer continuamente en el mundo gracias a las familias cristianas. También sobre ellos el Padre va a dejar oír su voz: \textquote{Este es mi Hijo muy amado, en quien tengo mis complacencias}. El Padre se complace en estos recién nacidos porque verá impresa en su espíritu la huella inmortal de su paternidad, la semejanza íntima y verdadera con su Hijo: hijos en el Hijo. Y al mismo tiempo descenderá el Espíritu Santo invisible y a la vez presente como entonces, para colmar a estas pequeñas almas de la riqueza de sus dones, para convertirlos en morada suya, templos suyos, manifestadores suyos que deberán irradiar su presencia y testimoniarlo a lo largo de la vida, vida que nosotros no sabemos todavía cómo será, pero que El ya ve en toda su plenitud.

Vamos a poner los cimientos de nuevas vidas cristianas amadas del Padre, redimidas por Cristo, marcadas por el Espíritu Santo, objeto de una predilección eterna que se proyecta ya desde ahora hacia el futuro y a la eternidad entera en un Amor sin fin que los abraza desde ahora: \textquote{Este es mi Hijo muy amado, en quien tengo mis complacencias}.

Sobre estos hijos predilectos, que dentro de poco serán los brotes nuevos de la Iglesia, blancos con la inocencia total de la gracia simbolizada en el manto que les impondré luego, fuertes como auténticos atletas con la unción del óleo de los catecúmenos, santos con la santidad misma de Dios, invoco con vosotros la ayuda continua del Señor y hago votos para que sean fieles siempre durante toda la vida a esta nuestra común y estupenda vocación cristiana.

Con tal fin los confío a vosotros, padres cristianos, que con vuestro amor y entrega mutuos les habéis dado la vida, convirtiéndoos en colaboradores de la creación de Dios. Los habéis traído aquí siendo hijos de la naturaleza, y os los lleváis a casa hijos de la gracia. De vosotros depende gran parte de su realización plena según los planes de Dios, ¡a vosotros los confío en nombre de Dios Trinidad!

Y los confío también a vosotros, padrinos y madrinas, con la misma finalidad de que garanticéis su crecimiento cristiano completo.

Sobre todos descienda la bendición del Señor, de la que es prenda y auspicio mi bendición apostólica.

\subsubsection{Homilía (1984):} **** 01A-09-Bautismo-1984

SANTA MISA POR EL BAUTISMO DE 27 NIÑOS

HOMILIA DE JUAN PABLO II

Basílica Vaticana - Domingo 8 de enero de 1984

1. \textquote{\emph{¡Gloria y alabanza a tu nombre, oh Señor!}}.

La invitación de la liturgia de hoy a glorificar y alabar el nombre del Señor adquiere un significado particular hoy, fiesta del Bautismo de Jesús.

{[}Quería celebrar este \textquote{misterio} de la vida de Cristo, confiriendo el Sacramento del Bautismo en esta Basílica a algunos niños y niñas en el Año Jubilar de la Redención, para subrayar que a través de este Sacramento el don de la Redención llega a quienes lo reciben. ., y se les aplica la obra de \emph{salvación} y \emph{santificación}, realizada por Jesús con su ofrenda al Padre celestial.{]}

La palabra de Dios, que hemos escuchado, nos presenta a Jesús de Nazaret como el \textquote{siervo de Dios}, profetizado en el \emph{\textbf{Libro de Isaías}}; el siervo, objeto de divina elección y complacencia, cumplirá su misión con una actitud de total adhesión a la voluntad del Señor y de ejemplar humildad hacia los hombres; se establecerá como una \textquote{alianza de los pueblos}, como \textquote{luz de las Naciones}, es decir, de los pueblos paganos, para dar vista a los ciegos y libertad a los presos.

Este misterioso \textquote{Siervo de Dios} es Cristo, que trae la salvación a la humanidad. En el relato del \textbf{Evangelio} que acabamos de escuchar, tan pronto como Juan bautiza a Jesús, los cielos se abren, el Espíritu de Dios desciende sobre Cristo como una paloma y una voz --- la del Padre --- dice: \textquote{Este es mi Hijo amado, en quien me complazco}.

La realidad reemplaza ahora a la profecía: la complacencia de Dios por su Siervo es la complacencia del Padre por su Hijo eterno, que asumió la naturaleza humana y que, con un gesto de profunda humildad, pidió a Juan ese bautismo, que era sólo una figura de lo que él mismo instituiría ya no como una preparación para la gracia, sino como un otorgamiento de gracia.

Queridos hermanos y hermanas aquí presentes {[}y, en particular, ¡ustedes que son los padres, madres, padrinos y madrinas de estos niños que pronto recibirán el Bautismo!{]} La Iglesia sigue obedeciendo a lo largo de los siglos las palabras que Jesús dirigió a los Apóstoles: \textquote{Se me ha dado toda potestad en el cielo y en la tierra. Id, pues, y haced discípulos de todas las naciones, bautizándolos en el nombre del Padre y del Hijo y del Espíritu Santo} (\emph{Mt} 28, 18-19).

2. {[}Estos hijos, fruto estupendo del amor de sus padres y también del misterioso gesto creador de Dios, nacieron a \emph{la vida natural.} Pero, en unos momentos, serán protagonistas de un \emph{nuevo nacimiento}, el de la vida sobrenatural, merecido para nosotros por Cristo: \textquote{En verdad, en verdad os digo --- estas son las palabras de Jesús a Nicodemo ---, a menos que uno sea nacido de agua y del Espíritu, no puede entrar en el reino de Dios. Lo que es nacido de la carne, carne es, y lo que es nacido del Espíritu, Espíritu es} (\emph{Jn} 3, 5-6).{]}

{[}El bautismo es una regeneración espiritual, un \textquote{nacimiento de arriba} (cf. \emph{Jn} 3, 7), no menos verdadero y real que el nacimiento a la vida terrena. Estos bebés a través del Bautismo --- como nos enseña la fe cristiana --- serán liberados del pecado original, serán santificados interiormente mediante la infusión de la gracia santificante, junto con los dones de las vestiduras de las virtudes teologales de la fe, la esperanza, la caridad y con los dones del Espíritu Santo; serán incorporados y conformados a Cristo y, por tanto, serán en él hijos de Dios, insertados en la Iglesia, cuerpo místico del Verbo Encarnado.{]}

3. {[}Esta vida divina, que les será comunicada por el Sacramento, debe crecer, madurar y perfeccionarse en el camino de su vida: \textquote{Los seguidores de Cristo, llamados por Dios no en razón de sus obras, sino en virtud del designio y gracia divinos y justificados en el Señor Jesús, han sido hechos por el bautismo, sacramento de la fe, verdaderos hijos de Dios y partícipes de la divina naturaleza, y, por lo mismo, realmente santos. En consecuencia, es necesario que con la ayuda de Dios conserven y perfeccionen en su vida la santificación que recibieron} \emph{Lumen gentium}, 40).{]}

{[}Es tarea y deber de toda la Iglesia, pero especialmente de los padres, padrinos y madrinas asumir la misión de ayudar a estos nuevos hijos de Dios en su crecimiento sobrenatural, ofreciéndoles cada día el ejemplo constante, generoso y constructivo de una existencia vivida enteramente de acuerdo con las exigencias del Evangelio.{]}

{[}¡Este es el deseo que hoy dirijo a estos pequeños, nuevos cristianos, nuevos miembros de la Iglesia exultantes, \emph{nuevos frutos de la Redención!}{]}

{[}¡Este es el compromiso de responsabilidad cristiana, que confío y encomiendo a todos los que me escuchan, en esta celebración comunitaria del Santo Bautismo!{]}

¡Que así sea!

\subsubsection{Homilía (1987):} 01A-09-Bautismo-1987ES

HOMILIA DE JUAN PABLO II

SANTA MISA POR EL BAUTISMO DE CUARENTA Y NUEVE BEBÉS

Domingo 11 de enero de 1987

\emph{!Queridísimos hermanos!}

Estamos aquí reunidos para celebrar hoy la fiesta litúrgica del Bautismo de Jesús: acontecimiento que, al revelar el misterio de la plena manifestación del Señor como Hijo de Dios ante todo el pueblo, inauguró la vida pública del Redentor. \textquote{Este es mi Hijo amado, en quien tengo mi complacencia} (\emph{Mt} 3, 17).

{[}Deseo hacerles llegar a todos mi cordial saludo y expresarles mi profunda alegría por poder conferir el sacramento del bautismo a estos niños. Los saludo, padres, y me alegra que hayan querido compartir con sus hijos, desde los primeros días de su vida, el don de la gracia, la vida divina y su fe. Los saludo a ustedes, padrinos y madrinas, que han aceptado con espíritu cristiano hacerse colaboradores en la formación cristiana y camino espiritual de estos niños, comprometiéndose a ser para ellos testigos de la verdad revelada y de la moral evangélica. Os saludo, como nuevos hermanos e hijos, a todos estos pequeños, a quienes deseo sinceramente una vida y un futuro lleno de felicidad, con el profundo deseo de que la gracia bautismal no falte nunca en su vida y que esta primera llamada a la fe algún día encuentre su feliz desarrollo en el pleno seguimiento de la voz del Señor.{]}

El bautismo ofrecido por Juan fue un signo de penitencia, un testimonio de conversión combinado con el deseo de volver a Dios, purificado en el espíritu, para acoger el cumplimiento de las promesas mesiánicas. A este bautismo, con un gesto que revela una gran humildad, el Hijo de Dios hecho carne se somete en nombre de todo el género humano. Así entra en un bautismo de penitencia que es la remisión de los pecados de todos los hombres, porque, como verdadero hombre, se ha hecho solidario con nosotros. Pero Jesús predice su futuro con este gesto: como \textquote{siervo elegido} de Dios (cf. \emph{Is} 42, 1) llevará sobre sí el peso de los pecados de todos y se ofrecerá un día en sacrificio para obtener, lavándonos con su propia sangre, el perdón de toda culpa.

{[}El bautismo que hoy se concederá a estos niños evoca el gesto de Cristo en el Jordán, porque el sacramento que la Iglesia confiere a sus nuevos hijos encuentra su fundamento precisamente en la ofrenda que Jesús entonces hizo de sí mismo. A través del bautismo recibido de Juan, nuestro Salvador quiso prefigurar el sacramento que un día, en la cruz, brotaría de su sacrificio como fuente de regeneración y renovación de vida: el bautismo, por el cual somos sepultados junto con él en la muerte, porque, como Cristo resucitó, también nosotros podemos caminar en una nueva vida (cf. \emph{Rm} 6, 4).{]}

{[}El agua del bautismo liberará a estos pequeños hoy del pecado original, los introducirá en la vida misteriosa escondida en Dios desde la eternidad y revelada en Cristo y en la Iglesia, los hará nuestros hermanos, miembros vivos del mismo cuerpo del Señor. Al sumergirlos en su misterio de muerte y resurrección, Cristo les dará un nuevo nacimiento y les dará su vida divina. Por eso, incluso a vuestros hijos hoy Dios Padre podrá decirles \textquote{tú eres mi Hijo}, ya que es la vida de Cristo el Hijo de Dios la que les será conferida como don de gracia, una vida que no se originó en ellos por voluntad de la carne, ni por voluntad del hombre, sino sólo por Dios (cf. \emph{Jn} 1, 13).{]}

{[}Preparémonos todos con fe para celebrar este misterio con corazón agradecido hacia el Señor, y pidamos que renueve la gracia bautismal en nosotros, orando por la nueva vida cristiana de estos pequeños.{]}

\subsubsection{Homilía (1990):} 01A-09-Bautismo-1990ES

ADMINISTRACIÓN DEL SACRAMENTO DEL BAUTISMO A ALGUNOS RECIÉN NACIDOS

HOMILIA DE JUAN PABLO II

Fiesta del Bautismo de Jesús

Salón de Bendición - Domingo 7 de enero de 1990

\textquote{Este es mi Hijo amado, en quien me complazco} (\emph{Mt} 3, 17).

1. {[}Después de haber celebrado ayer, queridos hermanos y hermanas, la solemnidad de la Epifanía,{]} nos volvemos a encontrar hoy, {[}en este primer domingo del año nuevo,{]} para la fiesta del Bautismo de Jesús en el Jordán. Con este evento se inaugura la misión pública del Mesías, que se manifiesta a todos como \textquote{el Hijo amado del Padre} para ser escuchado, acogido, seguido. Jesús permanece con nosotros, en cada uno de nosotros, como nuestro Salvador. Su salvación nos llega a través de la fe y la gracia del Bautismo, el sacramento principal de la Iglesia.{]}

{[}En este contexto litúrgico, hoy es para mí, como cada año, una ocasión renovada de alegría poder acogerlos a ustedes, padres, padrinos y madrinas, por el Bautismo de estos niños (\ldots{}). Y es también una circunstancia providencial que nos hace reflexionar sobre el sacramento del Bautismo, puerta por la que entramos de lleno en la comunidad eclesial.{]}

De hecho, la Iglesia es consciente de que su misión profética, sacerdotal y real tiene su origen en el Bautismo: de él toma fuerza la tarea de testificar y difundir a todos los hombres, alcanzada por su anuncio misionero, la salvación realizada por Cristo, \textquote{el amado Hijo del Padre}. Con el Bautismo el cristiano acoge al Salvador y, en virtud del agua y del Espíritu, recorre el camino del amor de Dios, habiéndose renovado profundamente a imagen de Aquel que se manifestó en nuestra carne mortal.

2. Queridos hermanos y hermanas, no olvidéis nunca el don recibido y la alta misión que se os confió el día de vuestro Bautismo. No lo olvidéis especialmente vosotros, padres, padrinos y madrinas de estos niños, llamados por la bondad del Padre celestial a participar de la herencia inmortal de los redimidos.

En ti, estas tiernas criaturas, renovadas por la fuerza del Espíritu, pueden encontrar siempre testigos valientes y verdaderos \textquote{padres} en el itinerario de la vida cristiana. El agua del Bautismo los libera hoy de la esclavitud del pecado original y los introduce en la plenitud de la vida de Dios, manifestada en Jesucristo y comunicada por él a su Iglesia a través del misterio de su muerte y resurrección. Precisamente porque están insertados en la Iglesia, estos niños se convierten en miembros vivos del mismo cuerpo de Cristo, nuestros hermanos en la fe, coherederos con nosotros en la salvación y compartiendo desde este momento nuestra misión común en el mundo.

3. La liturgia de hoy, que presenta la teofonía del Jordán, nos muestra al Mesías como el que devuelve la vista a los ciegos y la libertad a los presos. En esta ocasión el Padre proclama solemnemente a Jesús como su \textquote{Hijo amado} y sanciona así el paso definitivo del Antiguo al Nuevo Testamento: del bautismo de Juan --- signo de penitencia y conversión a la espera del cumplimiento de las promesas mesiánicas --- al Bautismo de Jesús, en \textquote{Espíritu y fuego} (\emph{Lc} 3, 16). Por tanto, el Espíritu Santo es el autor del Bautismo de los cristianos: de nosotros mismos y también de estos niños. Él es quien hace resonar en nuestro espíritu la Palabra reveladora del Padre: \textquote{Este es mi Hijo amado, en quien me complazco}.

Es él, el Espíritu Santo, quien abre los ojos del corazón a la Verdad, a toda la Verdad. Es él, el Espíritu Santo, quien impulsa nuestra vida por el camino renovado de la caridad. Es él, el Espíritu Santo, el don extraordinario e inconmensurable que el Padre da a cada uno de estos recién bautizados. Es él, el Espíritu Santo, quien nos reconcilia con la ternura del perdón divino y nos invade totalmente con el poder de la verdad y del amor.

4. Queridos hermanos y hermanas, la celebración de hoy, que es un derramamiento sobreabundante del Espíritu Santo, debe, por tanto, llenarnos de alegría espiritual y empujarnos a renovar nuestra vocación cristiana. En efecto, el Bautismo es vida para transmitir, luz para comunicar. Reafirmamos, por tanto, nuestra fiel adhesión a Jesús, único Salvador que hoy se manifiesta en la plenitud de su misión mesiánica. Acabamos de orar por esa intención en la Oración post-comunión: \textquote{Señor (\ldots{}) imploramos de tu bondad, que, escuchando fielmente a tu Unigénito, de verdad nos llamemos y seamos hijos tuyos. Por Jesucristo, nuestro Señor}.

Que María, Madre del Salvador, acompañe los pasos de nuestra vida cristiana en el camino de la Verdad y el Amor y obtenga el don de la perseverancia y la fidelidad para estos hijos, padres, padrinos y madrinas.

\subsubsection{Homilía (1993):} 01A-09-Bautismo-1993ES

VISITA PASTORAL Asís

SANTA MISA DE ASIS EN LA BASÍLICA SUPERIOR

HOMILIA DE JUAN PABLO II

Asís (Perugia) - Domingo 10 de enero de 1993

1. \textquote{Domine, murum odii everte, nationes dividentem, et vias concordiae fac hominibus planas} --- \textquote{Oh Señor, rompe las barreras del odio que dividen a las naciones, abre el camino a la armonía y la paz} (Martes de la tercera semana de Adviento, Invocaciones de Laudes). Queridos hermanos y hermanas, el clamor que hoy elevamos a Dios proviene de la liturgia del Adviento. La oración por la paz en Europa, y en particular en los Balcanes, se plantea en este período en las lenguas de los diferentes pueblos del continente europeo. Junto con los presidentes de los episcopados de toda Europa hemos implorado al Señor la paz. También les hemos pedido a nuestros hermanos cristianos, así como a los hijos de Israel y musulmanes, que oren por esto. Nos encontramos aquí en Asís siguiendo los pasos de San Francisco, que amaba eminentemente a Cristo, a los hombres ya toda la creación. Junto a él revivimos el misterio del Bautismo de Cristo en el Jordán, hecho clave en la misión mesiánica de Jesús de Nazaret.

2. \textquote{En ese momento Jesús de Galilea fue al Jordán donde Juan para ser bautizado por él. Pero Juan quiso impedírselo, diciendo: \textquote{Necesito ser bautizado por ti y ¿vienes a mí?}. Pero Jesús le dijo: \textquote{Déjalo por ahora, porque conviene que así cumplamos toda justicia}} (\emph{Mt} 3, 13-15). El bautismo de penitencia, conferido por Juan en el Jordán, es un signo de la justicia que el hombre espera de Dios, buscándolo con todo su corazón. También es un signo de paz, deseado por todo espíritu humano en todos los pueblos y naciones de la tierra. Y he aquí, encontramos a Jesús de Nazaret en la procesión de hombres que, animados por tal deseo, vienen a recibir el Bautismo de la penitencia, confesando sus pecados. Jesús no tiene pecado, pero sin embargo se inserta entre los pecadores. Este es un hecho muy revelador. \textquote{Este es mi Hijo amado, en quien tengo complacencia} (\emph{Mt} 3, 17). Precisamente el Hijo, infinita complacencia del Padre, se inserta entre los pecadores y junto a ellos recibe el Bautismo de la penitencia. \textquote{No he venido a llamar a justos, sino a pecadores} (\emph{Mt} 9, 13). Eventualmente, esta tarea lo llevará a la Cruz. Esto es lo que el mismo Juan expresó a orillas del Jordán, cuando dijo: \textquote{¡He aquí el Cordero de Dios, he aquí el que quita el pecado del mundo!} (\emph{Jn} 1, 29).

3. Hemos venido aquí hoy, asumiendo los grandes pecados de nuestro tiempo, de nuestro continente. {[}La guerra en curso en los Balcanes constituye una acumulación particular de pecados. Los humanos usan herramientas de destrucción para matar y exterminar a otros de su especie. ¡Qué terribles experiencias de guerras, especialmente en Europa, vivió el siglo XX! Fue un siglo marcado por el odio y el profundo desprecio por la humanidad, el odio y el desprecio que no abandonó ningún medio y método para aniquilar y exterminar al otro. El divino precepto del amor ha sido violado muchas veces y de diversas formas, hasta el punto de llegar incluso a preguntarse con temor si el europeo sería capaz de levantarse de ese abismo al que lo había empujado un loco deseo de poder, de dominación, a expensas de otros: de otros hombres, de otras naciones. Desafortunadamente, una experiencia tan trágica parece haber renacido de alguna manera en los últimos años; continúa extendiéndose directamente en la península de los Balcanes. Ésta es la razón por la que toda Europa se reúne en oración; por eso vinimos en peregrinación a Asís, para invocar a Dios por medio de Cristo: \textquote{Romper las barreras del odio \ldots{} abrir el camino a la armonía y la paz}.{]}

4. Cristo ora junto con nosotros. Entró en la procesión de los pecadores no solo una vez, a orillas del Jordán, donde recibió el bautismo de penitencia de Juan. En cada siglo y en cada generación vuelve para mezclarse en esta procesión en los diversos lugares de la tierra. Cristo es, de hecho, el Redentor del mundo, a quien Dios \textquote{hizo pecado en nuestro favor, para que por él seamos justicia de Dios} (\emph{2 Co} 5, 21). De ahí nuestra firme convicción, iluminada por la fe, de que en la tierra atormentada de los hombres y naciones {[}de los pueblos balcánicos{]}, Cristo está presente entre todos los que sufren y sufren una absurda violación de los derechos humanos. Él, Cristo, es siempre testigo y defensor de los derechos humanos: tenía hambre, tenía sed, era un extraño, estaba desnudo, fue torturado, destrozado, violado, ultrajado en la dignidad humana \ldots{} (cf. \emph{Mt} 25, 31-46). En él los derechos de la persona no son solo palabras, sino vida: la vida que prevalece sobre la muerte y por la Cruz se afirma en la victoria de la Resurrección. Hoy rezamos junto con él y por él, porque estamos firmemente convencidos de que Él reza incesantemente con nosotros.

5. El Padre se complace en Él. Creemos, por tanto, que en Él y por Él el hombre, incluso el más ultrajado y también el más culpable, es abrazado por el único Amor, más fuerte que todo odio, pecado y maldad inhumana. Él \ldots{}, siervo de nuestra justificación, \textquote{no quebrará la caña cascada, no apagará la mecha humeante \ldots{} no fallará ni caerá, hasta que haya establecido el derecho en la tierra} (\emph{Is} 42, 3-4). El Padre le dice: \textquote{Yo te formé y te establecí como alianza de los pueblos y luz de las naciones \ldots{}} (\emph{Is} 42, 6). {[}Aquí: los pueblos, las naciones de esa tierra, envueltos en el horrendo conflicto que tiene lugar en los Balcanes, constituyen comunidades unidas por muchos lazos, inscritos no solo en la memoria del pasado, sino también en la esperanza común de un futuro mejor basado sobre los valores de la justicia y la paz. Cada una de estas naciones representa un bien particular, una confirmación de la riqueza multiforme dada por el Creador al hombre y a toda la humanidad. Además, cada nación tiene derecho a la autodeterminación como comunidad. Es un derecho que puede realizarse mediante la propia soberanía política o mediante una federación o confederación con otras naciones. ¿Se podría salvar de una manera u otra entre las naciones de la ex Yugoslavia? Es difícil descartarlo. Sin embargo, la guerra que ha estallado parece haber rechazado esa posibilidad. Y la guerra aún continúa. Hablando humanamente, puede parecer difícil ver el final. Y, sin embargo: \textquote{Sanabiles fecit Deus nationes \ldots{}} \emph{(Sb} 1, 14).{]}

6. Nos volvemos, por tanto, a Ti, Cristo, Hijo del Dios vivo, Verbo en quien el Padre se complace, y que quiso cumplir la misión de servidor de nuestra Redención. Tú eres la justificación del pecador, de todos los pecadores y malhechores de la historia humana. Tú eres la Alianza con los hombres, la luz de las naciones. Quédate con nosotros. Intercede por nosotros. Ora con nosotros pecadores, para que no prevalezcan las tinieblas. Perdona nuestros pecados, los terribles pecados de los hombres dominados por el odio, así como nosotros perdonamos \ldots{} Tratando de romper la espiral del mal \ldots{} Destruye tú mismo el odio que divide a las naciones. Allí, donde ahora abunda el pecado, haz que abunden la justicia y el amor, a lo que todo hombre, todo pueblo y nación es llamado en Ti, Príncipe de Paz. En esta hora difícil, nos dirigimos también a tu Santísima Madre, que es también Madre de todos los pueblos, Madre en particular de los pueblos de Europa, que a lo largo de los siglos le han levantado famosos santuarios, que son también hoy el destino de multitudes de peregrinos. En este momento pienso ante todo en el templo mariano más antiguo de Santa Maria Maggiore en Roma, en el \textquote{Muro Indestructible} en Ucrania y en esos lugares de devoción en Rusia, donde se venera la imagen de la Madre de Dios bajo la título de Nuestra Señora de Wladimir, de Kazán, de Smolensk. Mi pensamiento también va a los santuarios de Mariapocs en Hungría, de Marija Bistrica en Croacia, de Studenica en Serbia, al santuario nacional de la \textquote{Addolorata} en Eslovaquia, a la \textquote{Puerta de la Aurora} en Lituania, a los santuarios de Aglona en Letonia., de Marija Pomagaj en Eslovenia, de Czestochowa en Polonia, de Montserrat en España, de Lourdes en Francia, de Fátima en Portugal \ldots{} y muchos otros. A María Santísima, tu Madre y Madre nuestra, oh Cristo, toda Europa confía esta oración suya por la paz, utilizando todos los idiomas que se hablan en el Continente en la celebración de hoy.

7. ¡Que se rompan las barreras del odio! ¡Oh Dios de la paz! Endereza los caminos de los hombres, para que sepan volver a vivir juntos como prójimos, hermanos y hermanas, \textquote{Hijos del Padre en el Hijo Unigénito} (cf. \emph{Ef} 1, 4-5): en Cristo Jesús nuestra paz auténtica.

\subsubsection{Homilía (1996):} FIESTA DEL BAUTISMO DEL SEÑOR

HOMILIA DI JUAN PABLO II

Capilla Sixtina - Domingo 7 de enero de 1996

\emph{¡Queridos hermanos y hermanas!}

1. La fiesta de hoy del Bautismo de Jesús concluye el tiempo de Navidad, el tiempo litúrgico de las progresivas \textquote{manifestaciones} de Jesús: en su nacimiento en Belén, cuando aparece con rostro de Niño, \textquote{el primogénito de toda criatura}, \textquote{imagen visible del Dios invisible} (cf. \emph{Col} 1-15); en la fiesta de la Epifanía, en la que se ofrece como don esperado y buscado por todos los pueblos de la tierra y como luz hacia la que confluye el camino interior de la historia; y finalmente en la celebración de hoy, en la que, en las aguas del Jordán, se solidariza con el hombre, \textquote{inclina su cabeza inmaculada ante el Precursor; y, bautizado, libera a la humanidad de la esclavitud, amante de los hombres} (Liturgia bizantina: \emph{EE}, 3038). Así es consagrado Siervo \textquote{con unción sacerdotal, profética y real, para que los hombres reconozcan en él al Mesías, enviado para llevar la buena nueva a los pobres} (\textquote{Praefatio} in festo Baptismatis Domini).

Estas son las etapas de una manifestación de Cristo que se hace cada vez más interior y profunda.

2. {[}Hoy, de manera muy singular, se traslada a la celebración del Bautismo de estos 20 niños (\ldots{}). Este sacramento renueva en ellos el don misterioso de la gracia divina, que imprime un \emph{sello indeleble en sus almas}, dando lugar a un nuevo nacimiento: \textquote{ \ldots{} de Dios fueron engendrados \ldots{} A quienes lo acogieron, les dio poder de convertirse en hijos de Dios \ldots{}} (\emph{Jn} 1, 12-13).

La gracia santificante, que elimina el pecado original, infunde en ellos con el Bautismo las virtudes teologales y los dones del Espíritu Santo; \emph{también los inserta en el Cuerpo místico de Cristo}, que es la Iglesia.

¡Qué grande es el Bautismo, el primero de los sacramentos y el más necesario para la salvación! Es \emph{la piedra angular de la vida cristiana}, el umbral de todos los demás sacramentos y de la regeneración a esa vida inmortal de la que también nos hablan los maravillosos frescos de Miguel Ángel, que podemos admirar en esta sugerente Capilla Sixtina.

Es a partir de esta conciencia que la práctica de bautizar a los niños se desarrolló desde el comienzo mismo de su existencia terrenal. Evidentemente, esta práctica presupone que los años siguientes, especialmente los de la infancia, la niñez y la juventud, se configuran entonces como un auténtico \emph{catecumenado}, un itinerario de iniciación a la vida cristiana y de progresiva inserción en la comunidad de los creyentes.{]}

3. {[}Queridos padres, queridos padrinos y madrinas, estos pequeños, a los que hoy se administra el Bautismo, deberán comprender, repensar y apreciar \emph{el don inestimable} del sacramento recibido. Depende de ustedes unirse a ellos, que ahora no saben y no entienden, y ser \emph{sus primeros maestros} en la enseñanza de las verdades cristianas.

\emph{¡Escuchen a estos niños!} La fe viene de la escucha de la Palabra de Dios, y la escucha es una actitud que, como cualquier otra, se aprende ante todo en la familia. Quien ha sido escuchado, sabe escuchar, así como quien ha sido amado puede amar más fácilmente.

Ayuden a estos niños \emph{a crecer fieles al Evangelio}, dispuestos a amar a Dios y a sus hermanos. Guíenlos, con el ejemplo y la palabra, por el camino de la santidad cristiana.

Su misión como padres no se limita únicamente al don de la vida física. Estás llamado a generar a tus hijos también en la fe y en la dimensión del espíritu.

Imitad a la Sagrada Familia de Nazaret e invocad la protección constante de la Santísima Virgen y de San José en vuestros hogares.

En esta solemne ocasión, frente a un número tan significativo de niños que por el bautismo están a punto de convertirse en hijos adoptivos de Dios, parece que volvemos a escuchar las palabras del Padre celestial recién escuchadas en el Evangelio: Cada uno de estos niños y niñas \textquote{es mi hijo amado, en quien me complazco}.

Y para ustedes, padres, padrinos, madrinas, adultos y cristianos conscientes de nuestra vocación, resuena la invitación: \textquote{¡Escuchadlo!} (\emph{Mc} 9, 7).

Que María, Madre de Dios y de la Iglesia y todos los santos a los que invocaremos en breve, os ayuden en tan exigente misión.{]}

\subsubsection{Homilía (1999): Jesús entre la multitud penitente}

Domingo 10 de enero de 1999.

1. \textquote{Éste es mi Hijo amado, en quien tengo mis complacencias} (Mt 3, 17).

En la fiesta del Bautismo del Señor, que estamos celebrando, resuenan estas palabras solemnes. Nos invitan a revivir el momento en que Jesús, bautizado por Juan, sale de las aguas del río Jordán y Dios Padre lo presenta como su Hijo unigénito, el Cordero que toma sobre sí el pecado del mundo. Se oye una voz del cielo, mientras el Espíritu Santo, en forma de paloma, se posa sobre Jesús, que comienza públicamente su misión salvífica; misión que se caracteriza por el estilo del \emph{siervo humilde y manso}, dispuesto a compartir y entregarse totalmente: \textquote{No gritará, no clamará. (\ldots{}) No quebrará la caña cascada, no apagará el pabilo vacilante. Promoverá fielmente el derecho} (\emph{Is} 42, 2-3).

La liturgia nos hace revivir la sugestiva \textbf{escena evangélica}: entre la multitud penitente que avanza hacia Juan el Bautista para recibir el bautismo está también Jesús. La promesa está a punto de cumplirse y se abre una nueva era para toda la humanidad. Este hombre, que aparentemente no es diferente de todos los demás, en realidad es Dios, que viene a nosotros para dar a cuantos lo reciban el poder de \textquote{convertirse en hijos de Dios, a los que creen en su nombre; los cuales no nacieron de sangre, ni de deseo de hombre, sino que nacieron de Dios} (\emph{Jn} 1, 12-13).

2. \textquote{Éste es mi Hijo amado; escuchadle} (\emph{Aleluya}).

Hoy, este anuncio y esta invitación, llenos de esperanza para la humanidad, resuenan particularmente para los niños que, dentro de poco, mediante el sacramento del bautismo, se convertirán en nuevas criaturas. Al participar en el misterio de la muerte y resurrección de Cristo, se enriquecerán con el don de la fe y se incorporarán al pueblo de la nueva y definitiva alianza, que es la Iglesia. El Padre los hará en Cristo hijos adoptivos suyos, revelándoles un singular proyecto de vida: escuchar como discípulos a su Hijo, para ser llamados y ser realmente sus hijos.

Sobre cada uno de ellos bajará el Espíritu Santo y, como sucedió con nosotros el día de nuestro bautismo, también ellos gozarán de la vida que el Padre da a los creyentes por medio de Jesús, el Redentor del hombre. Esta riqueza tan grande de dones les exigirá, como a todo bautizado, una única tarea, que el apóstol Pablo no se cansa de indicar a los primeros cristianos con las palabras: \textquote{Caminad según el Espíritu} (\emph{Ga} 5, 16), es decir, vivid y obrad constantemente en el amor a Dios.

Expreso mis mejores deseos de que el bautismo, que hoy reciben estos niños, los convierta a lo largo de toda su vida en valientes testigos del Evangelio. Esto será posible gracias a su empeño constante. Pero también será necesaria vuestra labor educativa, queridos padres, que hoy dais gracias a Dios por los dones extraordinarios que concede a estos hijos vuestros, del mismo modo que será necesario el apoyo de sus padrinos y sus madrinas.

3. Amadísimos hermanos y hermanas, aceptad la invitación que la Iglesia os hace: sed sus \textquote{educadores en la fe}, para que se desarrolle en ellos el germen de la vida nueva y llegue a su plena madurez. Ayudadles con vuestras palabras y, sobre todo, con vuestro ejemplo.

Que aprendan pronto de vosotros a amar a Cristo, a invocarlo sin cesar, y a imitarlo con constante adhesión a su llamada. En su nombre habéis recibido, con el símbolo del cirio, la llama de la fe: cuidad de que esté continuamente alimentada, para que cada uno de ellos, conociendo y amando a Jesús, obre siempre según la sabiduría evangélica. De este modo, llegarán a ser verdaderos discípulos del Señor y apóstoles alegres de su Evangelio.

Encomiendo a la Virgen María a cada uno de estos niños y a sus respectivas familias. Que la Virgen ayude a todos a recorrer con fidelidad el camino inaugurado con el sacramento del bautismo.

\subsubsection{Homilía (2002): Cooperadores de la paternidad divina}

Domingo 13 de enero del 2002.

1. \textquote{Este es mi Hijo, el amado, mi predilecto} (Mt 3, 17).

Acabamos de escuchar de nuevo en el \textbf{evangelio} las palabras que resonaron en el cielo cuando Jesús fue bautizado por Juan en el río Jordán. Las pronunció una voz desde lo alto: la voz de Dios Padre. Revelan el misterio que celebramos hoy, el bautismo de Cristo. El Hombre sobre el que desciende, en forma de paloma, el Espíritu Santo es el Hijo de Dios, que tomó de la Virgen María nuestra carne para redimirla del pecado y de la muerte.

¡Grande es este \emph{misterio de salvación}! Misterio en el que se insertan hoy los niños que presentáis, queridos padres, padrinos y madrinas. Al recibir en la Iglesia el sacramento del bautismo, se convierten en hijos de Dios, \emph{hijos en el Hijo}. Es el misterio del \textquote{segundo nacimiento}.

2. Queridos padres, me dirijo con afecto especialmente a vosotros, que habéis dado la vida a estas criaturas, colaborando en la obra de Dios, autor de la vida y, de modo singular, de toda vida humana. Los habéis engendrado y hoy los presentáis a la fuente bautismal, \emph{para que vuelvan a nacer por el agua y por el Espíritu Santo}. La gracia de Cristo transformará su existencia de mortal en inmortal, liberándola del pecado original. Dad gracias al Señor por el don de su nacimiento y del nuevo nacimiento espiritual de hoy.

Pero ¿cuál fuerza permite a estos inocentes e inconscientes niños realizar un \textquote{paso} espiritual tan profundo? Es la \emph{fe}, la fe de la Iglesia, profesada en particular por vosotros, queridos padres, padrinos y madrinas. Precisamente en esta fe son bautizados vuestros hijos. Cristo no realiza el milagro de regenerar al hombre sin la colaboración del hombre mismo, y la primera cooperación de la criatura humana es la fe, con la que, atraída interiormente por Dios, se abandona libremente en sus manos.

Estos niños reciben hoy el bautismo sobre la base de vuestra fe, que dentro de poco os pediré profesar. ¡Cuánto amor, amadísimos hermanos, cuánta responsabilidad implica el gesto que realizaréis en nombre de vuestros hijos!

3. En el futuro, cuando sean capaces de comprender, ellos mismos deberán recorrer, personal y libremente, un camino espiritual que, con la gracia de Dios, los llevará a \emph{confirmar}, en el sacramento de la confirmación, el don que reciben hoy.

Pero ¿podrán abrirse a la fe si los adultos que los rodean no les dan un buen testimonio? Estos niños os necesitan, ante todo, a vosotros, queridos padres; os necesitan también a vosotros, queridos padrinos y madrinas, para aprender a conocer al verdadero Dios, que es amor misericordioso. A vosotros os corresponde introducirlos en este conocimiento, en primer lugar a través del testimonio de vuestro comportamiento en las relaciones con ellos y con los demás, relaciones que se han de caracterizar por la atención, la acogida y el perdón. Comprenderán que Dios es fidelidad si pueden reconocer su reflejo, aunque sea limitado y débil, ante todo en vuestra presencia amorosa.

Es grande la responsabilidad de la cooperación de los padres en el crecimiento espiritual de sus hijos. Eran muy conscientes de esa responsabilidad los beatos esposos Luis y María Beltrame Quattrocchi, a los que recientemente tuve la alegría de elevar al honor de los altares y que os exhorto a conocer mejor y a imitar. Si ya es grande vuestra misión de ser padres \textquote{según la carne}, ¡cuánto más lo es la de \emph{colaborar en la paternidad divina}, dando vuestra contribución para modelar en estas criaturas la imagen misma de Jesús, Hombre perfecto!

4. \emph{Nunca os sintáis solos} en esta misión tan comprometedora. Os conforte, ante todo, la confianza en el ángel de la guarda, al que Dios ha encomendado su singular mensaje de amor para cada uno de vuestros hijos. Además, toda la Iglesia, a la que tenéis la gracia de pertenecer, está comprometida a asistiros: en el cielo velan los santos, en particular aquellos cuyos nombres tienen estos niños y que serán sus \textquote{patronos}. En la tierra está la comunidad eclesial, en la que es posible fortalecer la propia fe y la propia vida cristiana, alimentándola con la oración y los sacramentos. No podréis dar a vuestros hijos lo que vosotros no habéis recibido y asimilado antes.

Además, todos tenemos una Madre según el Espíritu: María santísima. A ella le encomiendo a vuestros hijos, para que lleguen a ser cristianos auténticos; a María os encomiendo también a vosotros, queridos padres, queridos padrinos y madrinas, para que transmitáis siempre a estos niños el amor que necesitan para \emph{crecer} y para \emph{creer}. En efecto, \emph{la vida y la fe caminan juntas}.

Que así sea en la existencia de cada bautizado con la ayuda de Dios.

\subsection{Benedicto XVI, papa}

\subsubsection{Homilía: Se sumergió en nuestra muerte}

Domingo 13 de enero del 2008.

La celebración de hoy es siempre para mí motivo de especial alegría. En efecto, administrar el sacramento del bautismo en el día de la fiesta del Bautismo del Señor es, en realidad, uno de los momentos más expresivos de nuestra fe, en la que podemos ver de algún modo, a través de los signos de la liturgia, el misterio de la vida. En primer lugar, la vida humana, representada aquí en particular por estos trece niños que son el fruto de vuestro amor, queridos padres, a los cuales dirijo mi saludo cordial, extendiéndolo a los padrinos, a las madrinas y a los demás parientes y amigos presentes. Está, luego, el misterio de la vida divina, que hoy Dios dona a estos pequeños mediante el renacimiento por el agua y el Espíritu Santo. Dios es vida, como está representado estupendamente también en algunas pinturas que embellecen esta Capilla Sixtina.

Sin embargo, no debe parecernos fuera de lugar comparar inmediatamente la experiencia de la vida con la experiencia opuesta, es decir, con la realidad de la muerte. Todo lo que comienza en la tierra, antes o después termina, como la hierba del campo, que brota por la mañana y se marchita al atardecer. Pero en el bautismo el pequeño ser humano recibe una vida nueva, la vida de la gracia, que lo capacita para entrar en relación personal con el Creador, y esto para siempre, para toda la eternidad.

Por desgracia, el hombre es capaz de apagar esta nueva vida con su pecado, reduciéndose a una situación que la sagrada Escritura llama \textquote{segunda muerte}. Mientras que en las demás criaturas, que no están llamadas a la eternidad, la muerte significa solamente el fin de la existencia en la tierra, en nosotros el pecado crea una vorágine que amenaza con tragarnos para siempre, si el Padre que está en los cielos no nos tiende su mano.

Este es, queridos hermanos, el misterio del bautismo: Dios ha querido salvarnos yendo él mismo hasta el fondo del abismo de la muerte, con el fin de que todo hombre, incluso el que ha caído tan bajo que ya no ve el cielo, pueda encontrar la mano de Dios a la cual asirse a fin de subir desde las tinieblas y volver a ver la luz para la que ha sido creado. Todos sentimos, todos percibimos interiormente que nuestra existencia es un deseo de vida que invoca una plenitud, una salvación. Esta plenitud de vida se nos da en el bautismo.

Acabamos de oír el \textbf{relato del bautismo de Jesús en el Jordán}. Fue un bautismo diverso del que estos niños van a recibir, pero tiene una profunda relación con él. En el fondo, todo el misterio de Cristo en el mundo se puede resumir con esta palabra: \textquote{bautismo}, que en griego significa \textquote{inmersión}. El Hijo de Dios, que desde la eternidad comparte con el Padre y con el Espíritu Santo la plenitud de la vida, se \textquote{sumergió} en nuestra realidad de pecadores para hacernos participar en su misma vida: se encarnó, nació como nosotros, creció como nosotros y, al llegar a la edad adulta, manifestó su misión iniciándola precisamente con el \textquote{bautismo de conversión}, que recibió de Juan el Bautista. Su primer acto público, como acabamos de escuchar, fue bajar al Jordán, entre los pecadores penitentes, para recibir aquel bautismo. Naturalmente, Juan no quería, pero Jesús insistió, porque esa era la voluntad del Padre (cf. \emph{Mt} 3, 13-15).

¿Por qué el Padre quiso eso? ¿Por qué mandó a su Hijo unigénito al mundo como Cordero para que tomara sobre sí el pecado del mundo? (cf. \emph{Jn} 1, 29). El evangelista narra que, cuando Jesús salió del agua, se posó sobre él el Espíritu Santo en forma de paloma, mientras la voz del Padre desde el cielo lo proclamaba \textquote{Hijo predilecto} (\emph{Mt} 3, 17). Por tanto, desde aquel momento Jesús fue revelado como aquel que venía para bautizar a la humanidad en el Espíritu Santo: venía a traer a los hombres la vida en abundancia (cf. \emph{Jn} 10, 10), la vida eterna, que resucita al ser humano y lo sana en su totalidad, cuerpo y espíritu, restituyéndolo al proyecto originario para el cual fue creado.

El fin de la existencia de Cristo fue precisamente dar a la humanidad la vida de Dios, su Espíritu de amor, para que todo hombre pueda acudir a este manantial inagotable de salvación. Por eso san Pablo escribe a los Romanos que hemos sido bautizados en la muerte de Cristo para tener su misma vida de resucitado (cf. \emph{Rm} 6, 3-4). Y por eso mismo los padres cristianos, como hoy vosotros, tan pronto como les es posible, llevan a sus hijos a la pila bautismal, sabiendo que la vida que les han transmitido invoca una plenitud, una salvación que sólo Dios puede dar. De este modo los padres se convierten en colaboradores de Dios no sólo en la transmisión de la vida física sino también de la vida espiritual a sus hijos.

Queridos padres, juntamente con vosotros doy gracias al Señor por el don de estos niños e invoco su asistencia para que os ayude a educarlos y a insertarlos en el Cuerpo espiritual de la Iglesia. A la vez que les ofrecéis lo que es necesario para el crecimiento y para la salud, vosotros, con la ayuda de los padrinos, os habéis comprometido a desarrollar en ellos la fe, la esperanza y la caridad, las virtudes teologales que son propias de la vida nueva que han recibido con el sacramento del bautismo.

Aseguraréis esto con vuestra presencia, con vuestro afecto; y lo aseguraréis, ante todo y sobre todo, con la oración, presentándolos diariamente a Dios, encomendándolos a él en cada etapa de su existencia. Ciertamente, para crecer sanos y fuertes, estos niños y niñas necesitarán cuidados materiales y muchas atenciones; pero lo que les será más necesario, más aún indispensable, es conocer, amar y servir fielmente a Dios, para tener la vida eterna. Queridos padres, sed para ellos los primeros testigos de una fe auténtica en Dios.

En el rito del bautismo hay un signo elocuente, que expresa precisamente la transmisión de la fe: es la entrega, a cada uno de los bautizandos, de una vela encendida en la llama del cirio pascual: es la luz de Cristo resucitado que os comprometéis a transmitir a vuestros hijos. Así, de generación en generación, los cristianos nos transmitimos la luz de Cristo, de modo que, cuando vuelva, nos encuentre con esta llama ardiendo entre las manos.

Durante el rito, os diré: \textquote{A vosotros, padres y padrinos, se os confía este signo pascual, una llama que debéis alimentar siempre}. Alimentad siempre, queridos hermanos y hermanas, la llama de la fe con la escucha y la meditación de la palabra de Dios y con la Comunión asidua de Jesús Eucaristía.

Que en esta misión estupenda, aunque difícil, os ayuden los santos protectores cuyos nombres recibirán estos trece niños. Que estos santos les ayuden sobre todo a ellos, los bautizandos, a corresponder a vuestra solicitud de padres cristianos. En particular, que la Virgen María los acompañe a ellos y a vosotros, queridos padres, ahora y siempre. Amén.

\subsubsection{Ángelus: Ungido con el Espíritu Santo}

Domingo 13 de enero del 2008.

Con la fiesta del Bautismo del Señor, que celebramos hoy, se concluye el tiempo litúrgico de Navidad. El Niño, a quien los Magos de Oriente vinieron a adorar en Belén, ofreciéndole sus dones simbólicos, lo encontramos ahora adulto, en el momento en que se hace bautizar en el río Jordán por el gran profeta Juan (cf. \emph{Mt} 3, 13). El \textbf{Evangelio} narra que cuando Jesús, recibido el bautismo, salió del agua, se abrieron los cielos y bajó sobre él el Espíritu Santo en forma de paloma (cf. \emph{Mt} 3, 16). Se oyó entonces una voz del cielo que decía: \textquote{Este es mi Hijo amado, en quien me complazco} (\emph{Mt} 3, 17). Esa fue su primera manifestación pública, después de casi treinta años de vida oculta en Nazaret.

Testigos oculares de ese singular acontecimiento fueron, además del Bautista, sus discípulos, algunos de los cuales se convirtieron desde entonces en seguidores de Cristo (cf. \emph{Jn} 1, 35-40). Se trató simultáneamente de cristofanía y teofanía: ante todo, Jesús se manifestó como el \emph{Cristo}, término griego para traducir el hebreo \emph{Mesías}, que significa \textquote{ungido}. Jesús no fue ungido con óleo a la manera de los reyes y de los sumos sacerdotes de Israel, sino con el Espíritu Santo. Al mismo tiempo, junto con el Hijo de Dios aparecieron los signos del Espíritu Santo y del Padre celestial.

¿Cuál es el significado de este acto, que Jesús quiso realizar ---venciendo la resistencia del Bautista--- para obedecer a la voluntad del Padre? (cf. \emph{Mt} 3, 14-15). Su sentido profundo se manifestará sólo al final de la vida terrena de Cristo, es decir, en su muerte y resurrección. Haciéndose bautizar por Juan juntamente con los pecadores, Jesús comenzó a tomar sobre sí el peso de la culpa de toda la humanidad, como Cordero de Dios que \textquote{quita} el pecado del mundo (cf. \emph{Jn} 1, 29). Obra que consumó en la cruz, cuando recibió también su \textquote{bautismo} (cf. \emph{Lc} 12, 50). En efecto, al morir se \textquote{sumergió} en el amor del Padre y derramó el Espíritu Santo, para que los creyentes en él pudieran renacer de aquel manantial inagotable de vida nueva y eterna.

Toda la misión de Cristo se resume en esto: bautizarnos en el Espíritu Santo, para librarnos de la esclavitud de la muerte y \textquote{abrirnos el cielo}, es decir, el acceso a la vida verdadera y plena, que será \textquote{sumergirse siempre de nuevo en la inmensidad del ser, a la vez que estamos desbordados simplemente por la alegría} (\emph{Spe salvi}, 12).

Es lo que sucedió también a los trece niños a los cuales administré el sacramento del bautismo esta mañana en la capilla Sixtina. Invoquemos sobre ellos y sobre sus familiares la protección materna de María santísima. Y oremos por todos los cristianos, para que comprendan cada vez más el don del bautismo y se comprometan a vivirlo con coherencia, testimoniando el amor del Padre y del Hijo y del Espíritu Santo.

\subsubsection{Homilía: Comunión plena con la humanidad}

Domingo 9 de enero del 2011.

Me alegra daros una cordial bienvenida, en particular a vosotros, padres, padrinos y madrinas de los 21 recién nacidos a los que, dentro de poco, tendré la alegría de administrar el sacramento del Bautismo. Como ya es tradición, también este año este rito tiene lugar en la santa Eucaristía con la que celebramos el Bautismo del Señor. Se trata de la fiesta que, en el primer domingo después de la solemnidad de la Epifanía, cierra el tiempo de Navidad con la manifestación del Señor en el Jordán.

Según el relato del \textbf{evangelista san Mateo} (3, 13-17), Jesús fue de Galilea al río Jordán para que lo bautizara Juan; de hecho, acudían de toda Palestina para escuchar la predicación de este gran profeta, el anuncio de la venida del reino de Dios, y para recibir el bautismo, es decir, para someterse a ese signo de penitencia que invitaba a convertirse del pecado. Aunque se llamara bautismo, no tenía el valor sacramental del rito que celebramos hoy; como bien sabéis, con su muerte y resurrección Jesús instituye los sacramentos y hace nacer la Iglesia. El que administraba Juan era un acto penitencial, un gesto que invitaba a la humildad frente a Dios, invitaba a un nuevo inicio: al sumergirse en el agua, el penitente reconocía que había pecado, imploraba de Dios la purificación de sus culpas y se le enviaba a cambiar los comportamientos equivocados, casi como si muriera en el agua y resucitara a una nueva vida.

Por esto, cuando Juan Bautista ve a Jesús que, en fila con los pecadores, va para que lo bautice, se sorprende; al reconocer en él al Mesías, al Santo de Dios, a aquel que no tenía pecado, Juan manifiesta su desconcierto: él mismo, el que bautizaba, habría querido hacerse bautizar por Jesús. Pero Jesús lo exhorta a no oponer resistencia, a aceptar realizar este acto, para hacer lo que es conveniente para \textquote{cumplir toda justicia}. Con esta expresión Jesús manifiesta que vino al mundo para hacer la voluntad de Aquel que lo mandó, para realizar todo lo que el Padre le pide; aceptó hacerse hombre para obedecer al Padre. Este gesto revela ante todo quién es Jesús: es el Hijo de Dios, verdadero Dios como el Padre; es aquel que \textquote{se rebajó} para hacerse uno de nosotros, aquel que se hizo hombre y aceptó humillarse hasta la muerte de cruz (cf. \emph{Flp} 2, 7). El bautismo de Jesús, que hoy recordamos, se sitúa en esta lógica de la humildad y de la solidaridad: es el gesto de quien quiere hacerse en todo uno de nosotros y se pone realmente en la fila con los pecadores; él, que no tiene pecado, deja que lo traten como pecador (cf. \emph{2 Co} 5, 21), para cargar sobre sus hombros el peso de la culpa de toda la humanidad, también de nuestra culpa. Es el \textquote{siervo de Dios} del que nos habló el \textbf{profeta Isaías en la primera lectura} (cf. 42, 1). Lo que dicta su humildad es el deseo de establecer una comunión plena con la humanidad, el deseo de realizar una verdadera solidaridad con el hombre y con su condición. El gesto de Jesús anticipa la cruz, la aceptación de la muerte por los pecados del hombre. Este acto de anonadamiento, con el que Jesús quiere uniformarse totalmente al designio de amor del Padre y asemejarse a nosotros, manifiesta la plena sintonía de voluntad y de fines que existe entre las personas de la santísima Trinidad. Para ese acto de amor, el Espíritu de Dios se manifiesta como paloma y baja sobre él, y en aquel momento el amor que une a Jesús al Padre se testimonia a cuantos asisten al bautismo, mediante una voz desde lo alto que todos oyen. El Padre manifiesta abiertamente a los hombres ---a nosotros--- la comunión profunda que lo une al Hijo: la voz que resuena desde lo alto atestigua que Jesús es obediente en todo al Padre y que esta obediencia es expresión del amor que los une entre sí. Por eso, el Padre se complace en Jesús, porque reconoce en las acciones del Hijo el deseo de seguir en todo su voluntad: \textquote{Este es mi Hijo amado, en quien me complazco} (\emph{Mt} 3, 17). Y esta palabra del Padre alude también, anticipadamente, a la victoria de la resurrección y nos dice cómo debemos vivir para complacer al Padre, comportándonos como Jesús.

Queridos padres, el Bautismo que hoy pedís para vuestros hijos los inserta en este intercambio de amor recíproco que existe en Dios entre el Padre, el Hijo y el Espíritu Santo; por este gesto que voy a realizar, se derrama sobre ellos el amor de Dios, y los inunda con sus dones. Mediante el lavatorio del agua, vuestros hijos son insertados en la vida misma de Jesús, que murió en la cruz para librarnos del pecado y resucitando venció a la muerte. Por eso, inmersos espiritualmente en su muerte y resurrección, son liberados del pecado original e inicia en ellos la vida de la gracia, que es la vida misma de Jesús resucitado. \textquote{Él se entregó por nosotros ---afirma san Pablo--- a fin de rescatarnos de toda iniquidad y formar para sí un pueblo puro que fuese suyo, fervoroso en buenas obras} (\emph{Tt} 2, 14).

Queridos amigos, al darnos la fe, el Señor nos ha dado lo más precioso que existe en la vida, es decir, el motivo más verdadero y más bello por el cual vivir: por gracia hemos creído en Dios, hemos conocido su amor, con el cual quiere salvarnos y librarnos del mal. La fe es el gran don con el que nos da también la vida eterna, la verdadera vida. Ahora vosotros, queridos padres, padrinos y madrinas, pedís a la Iglesia que acoja en su seno a estos niños, que les dé el Bautismo; y esta petición la hacéis en razón del don de la fe que vosotros mismos, a vuestra vez, habéis recibido. Todo cristiano puede repetir con el \textbf{profeta Isaías}: \textquote{El Señor me plasmó desde el seno materno para siervo suyo} (cf. 49, 5); así, queridos padres, vuestros hijos son un don precioso del Señor, el cual se ha reservado para sí su corazón, para poderlo colmar de su amor. Por el sacramento del Bautismo hoy los consagra y los llama a seguir a Jesús, mediante la realización de su vocación personal según el particular designio de amor que el Padre tiene pensado para cada uno de ellos; meta de esta peregrinación terrena será la plena comunión con él en la felicidad eterna.

Al recibir el Bautismo, estos niños obtienen como don un sello espiritual indeleble, el \textquote{carácter}, que marca interiormente para siempre su pertenencia al Señor y los convierte en miembros vivos de su Cuerpo místico, que es la Iglesia. Mientras entran a formar parte del pueblo de Dios, para estos niños comienza hoy un camino que debería ser un camino de santidad y de configuración con Jesús, una realidad que se deposita en ellos como la semilla de un árbol espléndido, que es preciso ayudar a crecer. Por esto, al comprender la grandeza de este don, desde los primeros siglos se ha tenido la solicitud de dar el Bautismo a los niños recién nacidos. Ciertamente, luego será necesaria una adhesión libre y consciente a esta vida de fe y de amor, y por esto es preciso que, tras el Bautismo, sean educados en la fe, instruidos según la sabiduría de la Sagrada Escritura y las enseñanzas de la Iglesia, a fin de que crezca en ellos este germen de la fe que hoy reciben y puedan alcanzar la plena madurez cristiana. La Iglesia, que los acoge entre sus hijos, debe hacerse cargo, juntamente con los padres y los padrinos, de acompañarlos en este camino de crecimiento. La colaboración entre la comunidad cristiana y la familia es más necesaria que nunca en el contexto social actual, en el que la institución familiar se ve amenazada desde varias partes y debe afrontar no pocas dificultades en su misión de educar en la fe. La pérdida de referencias culturales estables y la rápida transformación a la cual está continuamente sometida la sociedad, hacen que el compromiso educativo sea realmente arduo. Por eso, es necesario que las parroquias se esfuercen cada vez más por sostener a las familias, pequeñas iglesias domésticas, en su tarea de transmisión de la fe.

Queridos padres, junto con vosotros doy gracias al Señor por el don del Bautismo de estos hijos vuestros; al elevar nuestra oración por ellos, invocamos el don abundante del Espíritu Santo, que hoy los consagra a imagen de Cristo sacerdote, rey y profeta. Encomendándolos a la intercesión materna de María santísima, pedimos para ellos vida y salud, para que puedan crecer y madurar en la fe, y dar, con su vida, frutos de santidad y de amor. Amén.

\subsubsection{Ángelus: Obra de la Trinidad}

Domingo 9 de enero del 2011.

Hoy la Iglesia celebra el Bautismo del Señor, fiesta que concluye el tiempo litúrgico de la Navidad. Este misterio de la vida de Cristo muestra visiblemente que su venida en la carne es el acto sublime de amor de las tres personas divinas. Podemos decir que desde este solemne acontecimiento la acción creadora, redentora y santificadora de la santísima Trinidad será cada vez más manifiesta en la misión pública de Jesús, en su enseñanza, en sus milagros, en su pasión, muerte y resurrección. En efecto, leemos en el \emph{\textbf{Evangelio según san Mateo}} que \textquote{bautizado Jesús, salió luego del agua; y en esto se abrieron los cielos y vio al Espíritu de Dios que bajaba en forma de paloma y venía sobre él. Y una voz que salía de los cielos decía: \textquote{Este es mi Hijo amado, en quien me complazco}} (3, 16-17). El Espíritu Santo \textquote{mora} en el Hijo y da testimonio de su divinidad, mientras la voz del Padre, proveniente de los cielos, expresa la comunión de amor. \textquote{La conclusión de la escena del bautismo nos dice que Jesús ha recibido esta \textquote{unción} verdadera, que él es el Ungido {[}el Cristo{]} esperado} (\emph{Jesús de Nazaret}, Madrid 2007, p. 49), como confirmación de la profecía de Isaías: \textquote{He aquí mi siervo que yo sostengo, mi elegido en quien se complace mi alma} (\emph{Is} 42, 1). Verdaderamente es el Mesías, el Hijo del Altísimo que, al salir de las aguas del Jordán, establece la regeneración en el Espíritu y da, a quienes lo deseen, la posibilidad de convertirse en hijos de Dios. De hecho, no es casualidad que todo bautizado adquiera el carácter de hijo a partir del \emph{nombre cristiano}, signo inconfundible de que el Espíritu Santo hace nacer \textquote{de nuevo} al hombre del seno de la Iglesia. El beato Antonio Rosmini afirma que \textquote{el bautizado sufre una operación secreta pero potentísima, por la cual es elevado al orden sobrenatural, es puesto en comunicación con Dios} (\emph{Del principio supremo della metodica}\ldots{}, Turín 1857, n. 331). Todo esto se ha verificado de nuevo esta mañana, durante la celebración eucarística en la Capilla Sixtina, donde he conferido el sacramento del Bautismo a veintiún recién nacidos.

Queridos amigos, el Bautismo es el inicio de la vida espiritual, que encuentra su plenitud por medio de la Iglesia. En la hora propicia del sacramento, mientras la comunidad eclesial reza y encomienda a Dios un nuevo hijo, los padres y los padrinos se comprometen a acoger al recién bautizado sosteniéndolo en la formación y en la educación cristiana. Es una gran responsabilidad, que deriva de un gran don. Por esto, deseo alentar a todos los fieles a redescubrir la belleza de ser bautizados y pertenecer así a la gran familia de Dios, y a dar testimonio gozoso de la propia fe, a fin de que esta fe produzca frutos de bien y de concordia.

Lo pedimos por intercesión de la santísima Virgen María, Auxilio de los cristianos, a quien encomendamos a los padres que se están preparando al Bautismo de sus hijos, al igual que a los catequistas. Que toda la comunidad participe de la alegría del renacimiento del agua y del Espíritu Santo.

\subsection{Francisco, papa}

\subsubsection{Homilía (2014): La mejor herencia}

Domingo 12 de enero del 2014.

Jesús no tenía necesidad de ser bautizado, pero los primeros teólogos dicen que, con su cuerpo, con su divinidad, en su bautismo bendijo todas las aguas, para que las aguas tuvieran el poder de dar el Bautismo. Y luego, antes de subir al Cielo, Jesús nos pidió ir por todo el mundo a bautizar. Y desde aquel día hasta el día de hoy, esto ha sido una cadena ininterrumpida: se bautizan a los hijos, y los hijos después a los hijos, y los hijos\ldots{} Y hoy también esta cadena prosigue.

Estos niños son el eslabón de una cadena. Vosotros padres traéis a bautizar al niño o la niña, pero en algunos años serán ellos los que traerán a bautizar a un niño, o un nietecito\ldots{} Así es la cadena de la fe. ¿Qué quiere decir esto? Desearía solamente deciros esto: vosotros sois los que transmitís la fe, los transmisores; vosotros tenéis el deber de transmitir la fe a estos niños. Es la más hermosa herencia que vosotros les dejaréis: la fe. Sólo esto. Llevad hoy a casa este pensamiento. Debemos ser transmisores de la fe. Pensad en esto, pensad siempre cómo transmitir la fe a los niños.

Hoy canta el coro, pero el coro más bello es este de los niños, que hacen ruido\ldots{} Algunos llorarán, porque no están cómodos o porque tienen hambre: si tienen hambre, mamás, dadles de comer, tranquilas, porque ellos son aquí los protagonistas. Y ahora, con esta conciencia de ser transmisores de la fe, continuemos la ceremonia del Bautismo.

\subsubsection{Ángelus (2014): Los cielos se abrieron}

Domingo 12 de enero del 2014.

Hoy es la fiesta del Bautismo del Señor. Esta mañana he bautizado a treinta y dos recién nacidos. Doy gracias con vosotros al Señor por estas criaturas y por cada nueva vida. A mí me gusta bautizar a los niños. ¡Me gusta mucho! Cada niño que nace es un don de alegría y de esperanza, y cada niño que es bautizado es un prodigio de la fe y una fiesta para la familia de Dios.

La página del \textbf{Evangelio de hoy} subraya que, cuando Jesús recibió el bautismo de Juan en el río Jordán, \textquote{se abrieron los cielos} (\emph{Mt} 3, 16). Esto realiza las profecías. En efecto, hay una invocación que la liturgia nos hace repetir en el tiempo de Adviento: \textquote{Ojalá rasgases el cielo y descendieses!} (\emph{Is} 63, 19). Si el cielo permanece cerrado, nuestro horizonte en esta vida terrena es sombrío, sin esperanza. En cambio, celebrando la Navidad, la fe una vez más nos ha dado la certeza de que el cielo se rasgó con la venida de Jesús. Y en el día del bautismo de Cristo contemplamos aún el cielo abierto. La manifestación del Hijo de Dios en la tierra marca el inicio del gran tiempo de la misericordia, después de que el pecado había cerrado el cielo, elevando como una barrera entre el ser humano y su Creador. Con el nacimiento de Jesús, el cielo se abre. Dios nos da en Cristo la garantía de un amor indestructible. Desde que el Verbo se hizo carne es, por lo tanto, posible ver el cielo abierto. Fue posible para los pastores de Belén, para los Magos de Oriente, para el Bautista, para los Apóstoles de Jesús, para san Esteban, el primer mártir, que exclamó: \textquote{Veo los cielos abiertos} (\emph{Hch} 7, 56). Y es posible también para cada uno de nosotros, si nos dejamos invadir por el amor de Dios, que nos es donado por primera vez en el Bautismo. ¡Dejémonos invadir por el amor de Dios! ¡Éste es el gran tiempo de la misericordia! No lo olvidéis: ¡éste es el gran tiempo de la misericordia!

Cuando Jesús recibió el Bautismo de penitencia de Juan el Bautista, solidarizándose con el pueblo penitente ---Él sin pecado y sin necesidad de conversión---, Dios Padre hizo oír su voz desde el cielo: \textquote{Éste es mi Hijo amado, en quien me complazco} (v. 17). Jesús recibió la aprobación del Padre celestial, que lo envió precisamente para que aceptara compartir nuestra condición, nuestra pobreza. Compartir es el auténtico modo de amar. Jesús no se disocia de nosotros, nos considera hermanos y comparte con nosotros. Así, nos hace hijos, juntamente con Él, de Dios Padre. Ésta es la revelación y la fuente del amor auténtico. Y, ¡este es el gran tiempo de la misericordia!

¿No os parece que en nuestro tiempo se necesita un suplemento de fraternidad y de amor? ¿No os parece que todos necesitamos un suplemento de caridad? No esa caridad que se conforma con la ayuda improvisada que no nos involucra, no nos pone en juego, sino la caridad que comparte, que se hace cargo del malestar y del sufrimiento del hermano. ¡Qué buen sabor adquiere la vida cuando dejamos que la inunde el amor de Dios!

Pidamos a la Virgen Santa que nos sostenga con su intercesión en nuestro compromiso de seguir a Cristo por el camino de la fe y de la caridad, la senda trazada por nuestro Bautismo.

Hoy la creación es iluminada, hoy todas las cosas están llenas de júbilo,

los seres celestes y terrestres. Ángeles y hombres se unen,

porque donde está presente el Rey, allí también está su séquito.

En las aguas del Jordán el Rey de los siglos, el Señor,

moldea de nuevo a Adán que se había corrompido,

destruye las cabezas de los dragones allí anidados \ldots{}

Jesús, autor de la vida,

ha venido a destruir la condena de Adán, el primer creado:

él, que no tiene necesidad de purificación, en cuanto Dios,

en el Jordán se purifica en favor del hombre caído,

y matando allí la enemistad, otorga la paz que sobrepasa toda inteligencia.

Un tiempo estéril, amargamente privada de prole,

alégrate hoy, oh Iglesia de Cristo:

porque del agua y del Espíritu

han sido engendrados hijos que con fe aclaman:

No hay santo como nuestro Dios,

y no hay otro justo fuera de ti, Señor.

(De la Liturgia Bizantina).

\subsubsection{Homilía (2017)} FIESTA DEL BAUTISMO DEL SEÑOR\\ CELEBRACIÓN DE LA SANTA MISA Y BAUTISMO DE ALGUNOS NIÑOS

\textbf{\emph{HOMILÍA DEL SANTO PADRE FRANCISCO}}

\emph{Capilla Sixtina\\ Domingo 8 de enero de 2017}


\emph{Queridos padres:}

Vosotros habéis pedido para vuestros niños la fe, la fe que será dada en el Bautismo. La fe: eso significa vida de fe, porque la fe es vivida; caminar por el camino de la fe y dar testimonio de la fe. La fe no es recitar el \textquote{Credo} el domingo, cuando vamos a misa: no es solo esto. La fe es creer lo que es la Verdad: Dios Padre que ha enviado a su Hijo y al Espíritu que nos vivifica. Pero la fe es también encomendarse a Dios, y esto vosotros se lo tenéis que enseñar a ellos, con vuestro ejemplo, con vuestra vida.

Y la fe es luz: en la ceremonia del Bautismo se os dará una vela encendida, como en los primeros tiempos de la Iglesia. Y por esto el Bautismo, en esos tiempos, se llamaba \textquote{iluminación}, porque la fe ilumina el corazón, hace ver las cosas con otra luz. Vosotros habéis pedido la fe: la Iglesia da la fe a vuestros hijos con el Bautismo, y vosotros tenéis el deber de hacerla crecer, cuidarla, y que se convierta en testimonio para todos los demás. Este es el sentido de esta ceremonia. Y solamente quería deciros esto: cuidar la fe, hacerla crecer, que sea testimonio para los demás.

Y después\ldots{} ¡ha comenzado el concierto! {[}los niños lloran{]}: es porque los niños se encuentran en un lugar que no conocen, se han despertado antes de lo normal. Comienza uno, da la nota y después otros \textquote{imitan}\ldots{} Algunos lloran solamente porque ha llorado el otro\ldots{} Jesús hizo lo mismo, ¿sabéis? A mí me gusta pensar que la primera predicación de Jesús en el establo fue un llanto, la primera\ldots{} Y después, como la ceremonia es un poco larga, alguno llora por hambre. Si es así, vosotras madres amamantadles también, sin miedo, con toda normalidad. Como la Virgen amamantaba a Jesús\ldots{}

No olvidéis: habéis pedido la fe, a vosotros la tarea de cuidar la fe, hacerla crecer, que sea testimonio para todos nosotros, para todos nosotros: también para nosotros sacerdotes, obispos, todos. Gracias.

\subsubsection{Ángelus (2017)}

\emph{Queridos hermanos y hermanas, ¡buenos días!}

Hoy, fiesta del Bautismo de Jesús, el Evangelio (\emph{Mt} 3, 13-17) nos presenta el episodio ocurrido a orillas del río Jordán: en medio de la muchedumbre penitente que avanza hacia Juan Bautista para recibir el Bautismo también se encuentra Jesús ---hacía fila---. Juan querría impedírselo diciendo: \textquote{Soy yo el que necesita ser bautizado por ti} (\emph{Mt} 3, 14). En efecto, el Bautista es consciente de la gran distancia que hay entre él y Jesús. Pero Jesús vino precisamente para colmar la distancia entre el hombre y Dios: si Él está completamente de parte de Dios también está completamente de parte del hombre, y reúne aquello que estaba dividido. Por eso pide a Juan que le bautice, para que se cumpla toda justicia (cf. v. 15), es decir, se realice el proyecto del Padre, que pasa a través de la vía de la obediencia y de la solidaridad con el hombre frágil y pecador, la vía de la humildad y de la plena cercanía de Dios a sus hijos. ¡Porque Dios está muy cerca de nosotros, mucho!

En el momento en el que Jesús, bautizado por Juan, sale de las aguas del río Jordán, la voz de Dios Padre se hace oír desde lo alto: \textquote{Este es mi Hijo amado, en quien me complazco} (v. 17). Y al mismo tiempo el Espíritu Santo, en forma de paloma, se posa sobre Jesús, que da públicamente inicio a su misión de salvación; misión caracterizada por un estilo, el estilo del siervo humilde y dócil, dotado sólo de la fuerza de la verdad, como había profetizado Isaías: \textquote{no vociferará ni alzará el tono, {[}\ldots{}{]} la caña quebrada no partirá, y la mecha mortecina no apagará. Lealmente hará justicia} (42, 2-3). Siervo humilde y manso, he aquí el estilo de Jesús, y también el estilo misionero de los discípulos de Cristo: anunciar el Evangelio con docilidad y firmeza, sin gritar, sin regañar a alguien, sino con docilidad y firmeza, sin arrogancia o imposición. La verdadera misión nunca es proselitismo sino atracción a Cristo. ¿Pero cómo? ¿Cómo se hace esta atracción a Cristo? Con el propio testimonio, a partir de la fuerte unión con Él en la oración, en la adoración y en la caridad concreta, que es servicio a Jesús presente en el más pequeño de los hermanos. Imitando a Jesús, pastor bueno y misericordioso, y animados por su gracia, estamos llamados a hacer de nuestra vida un testimonio alegre que ilumina el camino, que lleva esperanza y amor.

Esta fiesta nos hace redescubrir el don y la belleza de ser un pueblo de bautizados, es decir, de pecadores ---todos lo somos--- de pecadores salvados por la gracia de Cristo, inseridos realmente, por obra del Espíritu Santo, en la relación filial de Jesús con el Padre, acogidos en el seno de la madre Iglesia, hechos capaces de una fraternidad que no conoce confines ni barreras.

Que la Virgen María nos ayude a todos nosotros cristianos a conservar una conciencia siempre viva y agradecida de nuestro Bautismo y a recorrer con fidelidad el camino inaugurado por este Sacramento de nuestro renacimiento. Y siempre humildad, docilidad y firmeza.

\subsubsection{Homilía (2020)} \emph{Capilla Sixtina\\ Domingo, 12 de enero de 2020}

\begin{center}\rule{0.5\linewidth}{\linethickness}\end{center}



Como Jesús, que fue a hacerse bautizar, así hacéis vosotros con vuestros hijos.

Jesús responde a Juan: \textquote{Hágase toda justicia} (cf. \emph{Mt} 3,15). Bautizar a un hijo es un acto de justicia para él. ¿Y por qué? Porque nosotros con el Bautismo le damos un tesoro, nosotros con el Bautismo le damos en prenda el \emph{Espíritu Santo}. El niño sale {[}del Bautismo{]} con la fuerza del Espíritu en su interior: el Espíritu que lo defenderá, que lo ayudará, durante toda su vida. Por eso es tan importante bautizarlos cuando son pequeños, para que crezcan con la fuerza del Espíritu Santo.

Este es el mensaje que quisiera daros hoy. Vosotros traéis hoy a vuestros hijos, {[}para que tengan{]} el Espíritu Santo dentro de ellos. Y cuidad de que crezcan con la luz, con la fuerza del Espíritu Santo, a través de la catequesis, la ayuda, la enseñanza, los ejemplos que les daréis en casa\ldots{} Este es el mensaje.

No quisiera deciros nada más importante. Sólo una advertencia. Los niños no están acostumbrados a venir a la Sixtina, ¡es la primera vez! Tampoco están acostumbrados a estar en un ambiente algo caluroso. Y no están acostumbrados a vestirse así para una fiesta tan hermosa como la de hoy. Se sentirán un poco incómodos en algún momento. Y uno empezará a llorar\ldots{} ―¡El concierto no ha empezado todavía!― pero empezará uno, luego otro\ldots{} No os asustéis, dejad que los niños lloren y griten. A lo mejor si tu niño llora y se queja, quizás sea porque tiene demasiado calor: quitadle algo; o porque tiene hambre: dale de mamar, aquí, sí, siempre en paz. Es algo que dije también el año pasado: tienen una dimensión \textquote{coral}: es suficiente que uno dé la primera nota y empiezan todos y habrá un concierto. No os asustéis. Es un sermón muy bonito el de un niño que llora en una iglesia. Haced que esté cómodo y sigamos adelante.

No lo olvidéis: vosotros lleváis el Espíritu Santo a los niños.



\subsubsection{Ángelus (2020)} \emph{Plaza de San Pedro\\ Domingo, 12 de enero de 2020}

\begin{center}\rule{0.5\linewidth}{\linethickness}\end{center}



\emph{Queridos hermanos y hermanas}, ¡buenos días!

Una vez más he tenido la alegría de bautizar a algunos niños en la fiesta de hoy del Bautismo del Señor. Hoy eran treinta y dos. Recemos por ellos y sus familias.

La liturgia de este año nos propone el acontecimiento del bautismo de Jesús según el relato evangélico de Mateo (cf. 3, 13-17). El evangelista describe el diálogo entre Jesús, que pide el bautismo, y Juan el Bautista, que se niega y observa: \textquote{Soy yo el que necesita ser bautizado por ti, ¿y tú vienes a mí?} (v. 14). Esta decisión de Jesús sorprende al Bautista: de hecho, el Mesías no necesita ser purificado, sino que es Él quien purifica. Pero Dios es Santo, sus caminos no son los nuestros, y Jesús es el Camino de Dios, un camino impredecible. Recordemos que Dios es el Dios de las sorpresas.

Juan había declarado que existía una distancia abismal e insalvable entre él y Jesús. \textquote{No soy digno de llevarle las sandalias} (\emph{Mateo} 3, 11), dijo. Pero el Hijo de Dios vino precisamente para salvar esta distancia entre el hombre y Dios. Si Jesús está del lado de Dios, también está del lado del hombre, y reúne lo que estaba dividido. Por eso le respondió a Juan: \textquote{Déjame ahora, pues conviene que así cumplamos toda justicia} (v. 15). El Mesías pide ser bautizado para que se cumpla toda justicia, para que se realice el proyecto del Padre, que pasa por el camino de la obediencia filial y de la solidaridad con el hombre frágil y pecador. Es el camino de la humildad y de la plena cercanía de Dios a sus hijos.

El profeta Isaías proclama también la justicia del Siervo de Dios, que lleva a cabo su misión en el mundo con un estilo contrario al espíritu mundano: \textquote{No vociferará ni alzará el tono, y no hará oír en la calle su voz. Caña quebrada no partirá, y mecha mortecina no apagará} (42, 2-3). Es la actitud de mansedumbre ―es lo que Jesús nos enseña con su humildad, la mansedumbre―, la actitud de sencillez, respeto, moderación y ocultamiento, que se requiere aún hoy de los discípulos del Señor. Cuántos ―es triste decirlo―, cuántos discípulos del Señor alardean como discípulos del Señor. No es un buen discípulo el que alardea de ello. El buen discípulo es el humilde, el manso que hace el bien sin ser visto. En la acción misionera, la comunidad cristiana está llamada a salir al encuentro de los demás siempre proponiendo y no imponiendo, dando testimonio, compartiendo la vida concreta de la gente.

Tan pronto como Jesús fue bautizado en el río Jordán, los cielos se abrieron y el Espíritu Santo descendió sobre él como una paloma, mientras que desde lo alto resonaba una voz que decía: \textquote{Este es mi Hijo amado; en el que me complazco} (\emph{Mateo} 3, 17). En la fiesta del Bautismo de Jesús redescubrimos nuestro bautismo. Así como Jesús es el Hijo amado del Padre, también nosotros, renacidos del agua y del Espíritu Santo, sabemos que somos hijos amados ―¡el Padre nos ama a todos!―, que somos objeto de la satisfacción de Dios, hermanos y hermanas de muchos otros, con una gran misión de testimoniar y anunciar a todos los hombres y mujeres el amor ilimitado del Padre.

Esta fiesta del Bautismo de Jesús nos recuerda nuestro bautismo. Nosotros también renacemos en el bautismo. En el bautismo el Espíritu Santo vino a permanecer en nosotros. Por eso es importante saber la fecha del bautismo. Sabemos la fecha de nuestro nacimiento, pero no siempre sabemos la fecha de nuestro bautismo. Seguramente algunos de vosotros no la saben\ldots{} Una tarea. Cuando regreses a casa pregunta: ¿Cuándo fui bautizada? ¿Cuándo fui bautizado? Y celebra la fecha de tu bautismo en tu corazón cada año. Hazlo. Es también un deber de justicia hacia el Señor que ha sido tan bueno con nosotros.

Que María Santísima nos ayude a comprender cada vez más el don del bautismo y a vivirlo coherentemente en las situaciones cotidianas.

\section{Temas}

\rbr{El Directorio Homilético no indica temas del Catecismo para esta fiesta. Pero podemos considerar los siguientes:}
