\chapter{Misa de la Aurora}

	\section{Lecturas}

		\rtitle{PRIMERA LECTURA}
		
		\rbook{Del libro del profeta Isaías} \rred{62, 11-12}
		
		\rtheme{Mira a tu salvador, que llega}
		
		\begin{readprose}
			El Señor hace oír esto
			
			hasta el confín de la tierra:
			
			«Decid a la hija de Sión:
			
			Mira a tu salvador, que llega,
			
			el premio de su victoria lo acompaña,
			
			la recompensa lo precede».
			
			Los llamarán \textquote{Pueblo santo},
			
			\textquote{Redimidos del Señor},
			
			y a ti te llamarán \textquote{Buscada},
			
			\textquote{Ciudad no abandonada}.
		\end{readprose}
	
		\rtitle{SALMO RESPONSORIAL}
		
		\rbook{Salmo} \rred{96, 1 y 6. 11-12}
		
		\rtheme{Hoy brillará una luz sobre nosotros, porque nos ha nacido el Señor}
		
		\begin{psbody}
			El Señor reina, la tierra goza,
			se alegran las islas innumerables.
			Los cielos pregonan su justicia,
			y todos los pueblos contemplan su gloria.
			
			Amanece la luz para el justo,
			y la alegría para los rectos de corazón.
			Alegraos, justos, con el Señor,
			celebrad su santo nombre.
		\end{psbody}
	
		\rtitle{SEGUNDA LECTURA}
		
		\rbook{De la carta del apóstol san Pablo a Tito} \rred{3, 4-7}
		
		\rtheme{Según su propia misericordia, nos salvó}
		
		\begin{scripture}
			Querido hermano:
			
			Cuando se manifestó la bondad de Dios nuestro Salvador y su amor al hombre, no por las obras de justicia que hubiéramos hecho nosotros, sino, según su propia misericordia, nos salvó por el baño del nuevo nacimiento y de la renovación del Espíritu Santo, que derramó copiosamente sobre nosotros por medio de Jesucristo nuestro Salvador, para que, justificados por su gracia, seamos, en esperanza, herederos de la vida eterna.
		\end{scripture}

		\rtitle{EVANGELIO}
		
		\rbook{Del Santo Evangelio según san Lucas} \rred{2, 15-20}
		
		\rtheme{Los pastores encontraron a María y a José y al niño}
		
		\begin{scripture}
			Sucedió que, cuando los ángeles se marcharon al cielo, los pastores se decían unos a otros:
			
			«Vayamos, pues, a Belén, y veamos lo que ha sucedido y que el Señor nos ha comunicado».
			
			Fueron corriendo y encontraron a María y a José, y al niño acostado en el pesebre. Al verlo, contaron lo que se les había dicho de aquel niño.
			
			Todos los que lo oían se admiraban de lo que les habían dicho los pastores. María, por su parte, conservaba todas estas cosas, meditándolas en su corazón. Y se volvieron los pastores dando gloria y alabanza a Dios por todo lo que habían oído y visto, conforme a lo que se les había dicho.
		\end{scripture}
	
	
\newsection

	
	\section{Comentario Patrístico}
	
		\subsection{Teodoto de Ancira}
		
			\ptheme{El Dueño de todo vino en forma de siervo}
			
			\src{Sermón en la Natividad del Salvador \\Edic. Schwartz, ACO t. 3, parte 1, 157-159.}
			
			\begin{body}
				El Dueño de todo vino en forma de siervo, revestido de pobreza, para no ahuyentar la presa. Habiendo elegido para nacer la inseguridad de un campo indefenso, nace de una pobrecilla virgen, inmerso en la pobreza, para, en silencio, dar caza al hombre y así salvarlo. Pues de haber nacido en medio del boato, y si se hubiera rodeado de riqueza, los infieles habrían dicho, y con razón, que había sido la abundancia de riqueza la que había operado la transformación de la redondez de la tierra. Y si hubiera elegido Roma, entonces la ciudad más poderosa, hubieran pensado que era el poderío de sus ciudadanos el que había cambiado el mundo.
				
				De haber sido el hijo del emperador, su obra benéfica se habría inscrito en el haber de las influencias. Si hubiera nacido hijo de un legislador, su reforma social se habría atribuido al ordenamiento jurídico. Y ¿qué es lo que hizo? Escogió todo lo vil y pobre, todo lo mediocre e ignorado por la gran masa, a fin de dar a conocer que la divinidad era la única transformadora de la tierra. He aquí por qué eligió una madre pobre, una patria todavía más pobre, y él mismo falto de recursos.
				
				Aprende la lección del pesebre. No habiendo lecho en que acostar al Señor, se le coloca en un pesebre, y la indigencia de lo más imprescindible se convierte en privilegiado anuncio de la profecía. Fue colocado en un pesebre para indicar que iba a convertirse en manjar incluso de los irracionales. En efecto, viviendo en la pobreza y yaciendo en un pesebre, la Palabra e Hijo de Dios atrae a sí tanto a los ricos como a los pobres, a los elocuentes como a los de premiosa palabra.
				
				Fíjate cómo la ausencia de bienes dio cumplimiento a la profecía, y cómo la pobreza ha hecho accesible a todos a aquel que por nosotros se hizo pobre. Nadie tuvo reparo en acudir por temor a las soberbias riquezas de Cristo; nadie sintió bloqueado el acceso por la magnificencia del poder: se mostró cercano y pobre, ofreciéndose a sí mismo por la salvación de todos.
				
				Mediante la corporeidad asumida, el Verbo de Dios se hace presente en el pesebre, para hacer posible que todos, racionales e irracionales, participen del manjar de salvación. Y pienso que esto es lo que ya antes había pregonado el profeta, desvelándonos el misterio de este pesebre: \emph{Conoce el buey a su amo, y el asno, el pesebre de su dueño; Israel no me conoce, mi pueblo no recapacita}. El que es rico, por nosotros se hizo pobre, haciendo fácilmente perceptible a todos la salvación con la fuerza de la divinidad. Refiriéndose a esto decía asimismo el gran Pablo: \emph{Siendo rico, por nosotros se hizo pobre, para que nosotros, con su pobreza, nos hagamos ricos}.
				
				Pero, ¿quién era el que enriquecía?, ¿de qué enriquecía?, y, ¿cómo se hizo él pobre por nosotros? Dime, por favor: ¿quién, siendo rico, se ha hecho pobre con mi pobreza? ¿Quizá el que apareció hecho hombre? Pero éste nunca fue rico, sino que nació pobre de padres pobres. ¿Quién, pues, era rico y con qué nos enriquecía el que por nosotros se hizo pobre? Dios ---dice--- enriquece a la criatura. Es, pues, Dios quien se hizo pobre, haciendo suya la pobreza del que se hacía visible; él es efectivamente rico en su divinidad, y por nosotros se hizo pobre.
			\end{body}
		
			\begin{patercite}
				¡Cuánta emoción debería acompañarnos mientras colocamos en el belén las montañas, los riachuelos, las ovejas y los pastores! De esta manera recordamos, como lo habían anunciado los profetas, que toda la creación participa en la fiesta de la venida del Mesías. Los ángeles y la estrella son la señal de que también nosotros estamos llamados a ponernos en camino para llegar a la gruta y adorar al Señor.
				
				\textquote{Vayamos, pues, a Belén, y veamos lo que ha sucedido y que el Señor nos ha comunicado} (\emph{Lc} 2,15), así dicen los pastores después del anuncio hecho por los ángeles. Es una enseñanza muy hermosa que se muestra en la sencillez de la descripción. A diferencia de tanta gente que pretende hacer otras mil cosas, los pastores se convierten en los primeros testigos de lo esencial, es decir, de la salvación que se les ofrece. Son los más humildes y los más pobres quienes saben acoger el acontecimiento de la encarnación. A Dios que viene a nuestro encuentro en el Niño Jesús, los pastores responden poniéndose en camino hacia Él, para un encuentro de amor y de agradable asombro. Este encuentro entre Dios y sus hijos, gracias a Jesús, es el que da vida precisamente a nuestra religión y constituye su singular belleza, y resplandece de una manera particular en el pesebre.
				
				Tenemos la costumbre de poner en nuestros belenes muchas figuras simbólicas, sobre todo, las de mendigos y de gente que no conocen otra abundancia que la del corazón. Ellos también están cerca del Niño Jesús por derecho propio, sin que nadie pueda echarlos o alejarlos de una cuna tan improvisada que los pobres a su alrededor no desentonan en absoluto. De hecho, los pobres son los privilegiados de este misterio y, a menudo, aquellos que son más capaces de reconocer la presencia de Dios en medio de nosotros.
				
				Los pobres y los sencillos en el Nacimiento recuerdan que Dios se hace hombre para aquellos que más sienten la necesidad de su amor y piden su cercanía. Jesús, \textquote{manso y humilde de corazón} (\emph{Mt} 11,29), nació pobre, llevó una vida sencilla para enseñarnos a comprender lo esencial y a vivir de ello. Desde el belén emerge claramente el mensaje de que no podemos dejarnos engañar por la riqueza y por tantas propuestas efímeras de felicidad\ldots{}
				
				\textbf{Francisco, papa,} Carta apostólica \emph{Admirabile signum,} nn. 5-6.
			\end{patercite}
		
\newsection		

	\section{Homilías}
		\rbr{Las homilías para esta celebración están tomadas de textos de las Padres de la Iglesia que tocan algunos aspectos de la Navidad en particular o relacionados con alguno de los textos bíblicos que se leen en la misma. \\Conviene señalar que estas homilías pueden iluminar aspectos de cualquiera de las otras celebraciones durante el tiempo de Navidad.}

		
		\subsection{San Bernardo de Claraval, abad}
		
			\subsubsection{Sermón: Plenitud de los tiempos y de la divinidad}
			
				\src{Sermón 1-2 en la Epifanía del Señor, \\ PL 183, 141-143.}
				
				\begin{body}
					\emph{Ha aparecido la bondad de Dios, nuestro Salvador, y su amor al hombre}. Gracias sean dadas a Dios, que ha hecho abundar en nosotros el consuelo en medio de esta peregrinación, de este destierro, de esta miseria.
					
					Antes de que apareciese la humanidad de nuestro Salvador, su bondad se hallaba también oculta, aunque ésta ya existía, pues la misericordia del Señor es eterna. ¿Pero cómo, a pesar de ser tan inmensa, iba a poder ser reconocida? Estaba prometida, pero no se la alcanzaba a ver; por lo que muchos no creían en ella. Efectivamente, \emph{en distintas ocasiones y de muchas maneras habló Dios por los profetas}. Y decía: Yo tengo \emph{designios de paz y no de aflicción}. Pero ¿qué podía responder el hombre que sólo experimentaba la aflicción e ignoraba la paz? ¿Hasta cuándo vais a estar diciendo: \emph{\textquote{Paz, paz}, y no hay paz?} A causa de lo cual \emph{los mensajeros de paz lloraban amargamente,} diciendo: \emph{Señor, ¿quién creyó nuestro anuncio?} Pero ahora los hombres tendrán que creer a sus propios ojos, ya que \emph{los testimonios de Dios se han vuelto absolutamente creíbles}. Pues para que ni una vista perturbada pueda dejar de verlo, \emph{puso su tienda al sol}.
					
					Pero de lo que se trata ahora no es de la promesa de la paz, sino de su envío; no de la dilatación de su entrega, sino de su realidad; no de su anuncio profético, sino de su presencia. Es como si Dios hubiera vaciado sobre la tierra un saco lleno de su misericordia; un saco que habría de desfondarse en la pasión, para que se derramara nuestro precio, oculto en él; un saco pequeño, pero lleno. Ya que \emph{un niño se nos ha dado,} pero \emph{en quien habita toda la plenitud de la divinidad}. Ya que, cuando llegó la plenitud del tiempo, hizo también su aparición la plenitud de la divinidad. Vino en carne mortal para que, al presentarse así ante quienes eran carnales, en la aparición de su humanidad se reconociese su bondad. Porque, cuando se pone de manifiesto la humanidad de Dios, ya no puede mantenerse oculta su bondad. ¿De qué manera podía manifestar mejor su bondad que asumiendo mi carne? La mía, no la de Adán, es decir, no la que Adán tuvo antes del pecado.
					
					¿Hay algo que pueda declarar más inequívocamente la misericordia de Dios que el hecho de haber aceptado nuestra miseria? ¿Qué hay más rebosante de piedad que la Palabra de Dios convertida en tan poca cosa por nosotros? \emph{Señor, ¿qué es el hombre, para que te acuerdes de él, el ser humano, para darle poder?} Que deduzcan de aquí los hombres lo grande que es el cuidado que Dios tiene de ellos; que se enteren de lo que Dios piensa y siente sobre ellos. No te preguntes, tú, que eres hombre, por lo que has sufrido, sino por lo que sufrió él. Deduce de todo lo que sufrió por ti, en cuánto te tasó, y así su bondad se te hará evidente por su humanidad. Cuanto más pequeño se hizo en su humanidad, tanto más grande se reveló en su bondad; y cuanto más se dejó envilecer por mí, tanto más querido me es ahora. \emph{Ha aparecido} ---dice el Apóstol--- \emph{la bondad de Dios, nuestro Salvador, y su amor al hombre}. Grandes y manifiestos son, sin duda, la bondad y el amor de Dios, y gran indicio de bondad reveló quien se preocupó de añadir a la humanidad el nombre de Dios.
				\end{body}

\newsection

		\subsection{San Hipólito de Roma, presbítero}
		
			\subsubsection{Tratado: La Palabra hecha carne nos diviniza}
			
				\src{Contra las herejías, Cap. 10, 33-34: \\PG 16, 3452-3453.}
				
				\begin{body}
					No prestamos nuestra adhesión a discursos vacíos ni nos dejamos seducir por pasajeros impulsos del corazón, como tampoco por el encanto de discursos elocuentes, sino que nuestra fe se apoya en las palabras pronunciadas por el poder divino. Dios se las ha ordenado a su Palabra, y la Palabra las ha pronunciado, tratando con ellas de apartar al hombre de la desobediencia, no dominándolo como a un esclavo por la violencia que coacciona, sino apelando a su libertad y plena decisión.
					
					Fue el Padre quien envió la Palabra, al fin de los tiempos. Quiso que no siguiera hablando por medio de un profeta, ni que se hiciera adivinar mediante anuncios velados; sino que le dijo que se manifestara a rostro descubierto, a fin de que el mundo, al verla, pudiera salvarse.
					
					Sabemos que esta Palabra tomó un cuerpo de la Virgen, y que asumió al hombre viejo, transformándolo. Sabemos que se hizo hombre de nuestra misma condición, porque, si no hubiera sido así, sería inútil que luego nos prescribiera imitarle como maestro. Porque, si este hombre hubiera sido de otra naturaleza, ¿cómo habría de ordenarme las mismas cosas que él hace, a mí, débil por nacimiento, y cómo sería entonces bueno y justo?
					
					Para que nadie pensara que era distinto de nosotros, se sometió a la fatiga, quiso tener hambre y no se negó a pasar sed, tuvo necesidad de descanso y no rechazó el sufrimiento, obedeció hasta la muerte y manifestó su resurrección, ofreciendo en todo esto su humanidad como primicia, para que tú no te descorazones en medio de tus sufrimientos, sino que, aun reconociéndote hombre, aguardes a tu vez lo mismo que Dios dispuso para él.
					
					Cuando contemples ya al verdadero Dios, poseerás un cuerpo inmortal e incorruptible, junto con el alma, y obtendrás el reino de los cielos, porque, sobre la tierra, habrás reconocido al Rey celestial; serás íntimo de Dios, coheredero de Cristo, y ya no serás más esclavo de los deseos, de los sufrimientos y de las enfermedades, porque habrás llegado a ser dios.
					
					Porque todos los sufrimientos que has soportado, por ser hombre, te los ha dado Dios precisamente porque lo eras; pero Dios ha prometido también otorgarte todos sus atributos, una vez que hayas sido divinizado y te hayas vuelto inmortal. Es decir, \emph{conócete a ti mismo} mediante el conocimiento de Dios, que te ha creado, porque conocerlo y ser conocido por él es la suerte de su elegido.
					
					No seáis vuestros propios enemigos, ni os volváis hacia atrás, porque Cristo es \emph{el Dios que está por encima de todo:} él ha ordenado purificar a los hombres del pecado, y él es quien renueva al hombre viejo, al que ha llamado desde el comienzo imagen suya, mostrando, por su impronta en ti, el amor que te tiene. Y, si tú obedeces sus órdenes y te haces buen imitador de este buen maestro, llegarás a ser semejante a él y recompensado por él; porque Dios no es pobre, y te divinizará para su gloria.
				\end{body}

\newsection


		\subsection{San Juan Pablo II, papa}
		
			\subsubsection{Catequesis (1983)}
			
				\src{Audiencia general. \\Miércoles 28 de diciembre de 1983.}
				
				\begin{body}
					1. El misterio de Navidad hace resonar en nuestros oídos el canto con que el cielo quiere hacer participar a la tierra en el gran acontecimiento de la Encarnación: \textquote{Gloria a Dios en las alturas y paz en la tierra a los hombres de buena voluntad } (\emph{Lc} 2, 14).
					
					\emph{La paz es anunciada por toda la tierra}. No es una paz que los hombres consigan conquistar con sus fuerzas. \emph{Viene de lo alto} como don maravilloso de Dios a la humanidad. No podemos olvidar que, si todos debemos trabajar para instaurar la paz en el mundo, antes de nada debemos abrirnos al don divino de la paz poniendo toda nuestra confianza en el Señor.
					
					Según el cántico de Navidad, la paz prometida a la tierra \emph{está ligada al amor que Dios trae a los hombres}. Los hombres son llamados \textquote{hombres de buena voluntad} porque ya la buena voluntad divina les pertenece. El nacimiento de Jesús es el testimonio irrefutable y definitivo de esta buena voluntad que jamás será retirada de la humanidad.
					
					Este nacimiento pone de manifiesto \emph{la voluntad divina de reconciliación}: Dios desea reconciliar consigo al mundo pecador, perdonando y cancelando los pecados. Ya en el anuncio del nacimiento el ángel había expresado esta voluntad reconciliadora indicando el nombre que debía llevar el Niño: Jesús, o sea, \textquote{Dios salva}. \textquote{Porque salvará a su pueblo de sus pecados}, comenta el ángel (Mt 1, 21). El nombre revela el destino y la misión del Niño juntamente con su personalidad: es el Dios que salva, el que libera a la humanidad de la esclavitud del pecado y, por ello, restablece las relaciones amistosas del hombre con Dios.
					
					2. El acontecimiento que da a la humanidad un Dios Salvador supera en gran medida las expectativas del pueblo judío. Este pueblo esperaba la salvación, esperaba al Mesías, a un rey ideal del futuro que debía establecer sobre la tierra el reino de Dios. A pesar de que la esperanza judía había puesto muy en lo alto a este Mesías, para ellos no era más que un hombre.
					
					La gran novedad de la venida del Salvador consiste en el hecho de que Él es Dios y hombre a la vez. Lo que el judaísmo no había podido concebir ni esperar, es decir, un Hijo de Dios hecho hombre, se realiza en el misterio de la Encarnación. \emph{El cumplimiento es mucho más maravilloso que la promesa}.
					
					Esta es la razón por la que no podemos medir la grandeza de Jesús sólo con los oráculos proféticos del Antiguo Testamento. Cuando Él realiza estos oráculos se mueve a un nivel trascendente. Todos los tentativos de encerrar a Jesús en los límites de una personalidad humana, no tienen en cuenta lo que hay de esencial en la revelación de la Nueva Alianza: la persona divina del Hijo que se ha hecho hombre o, según la palabra de San Juan, del Verbo que se ha hecho carne y ha venido a habitar entre nosotros (cf. 1, 14). Aquí aparece la grandiosidad generosa del plan divino de salvación. El Padre ha enviado a su Hijo que es Dios como Él. No se ha limitado a enviar a siervos, a hombres que hablasen en su nombre como los Profetas. Ha querido testimoniar a la humanidad el máximo de amor y le ha hecho la sorpresa de darle un Salvador que poseía la omnipotencia divina.
					
					En este Salvador, que es Dios y hombre a la vez, podemos descubrir \emph{la intención de la obra reconciliadora}. El Padre no quiere sólo purificar a la humanidad liberándola del pecado; quiere realizar \emph{la unión más íntima de la divinidad y la humanidad}. En la única persona divina de Jesús, la divinidad y la humanidad están unidas del modo más completo. Él que es perfectamente Dios es perfectamente hombre. Ha realizado en Sí esta unión de la divinidad y la humanidad para poder hacer participar de ella a todos los hombres. Perfectamente hombre, Él, que es Dios, quiere comunicar a sus hermanos humanos una vida divina que les permita ser más perfectamente hombres, reflejando en sí mismos la perfección divina.
					
					3. Un aspecto de la reconciliación merece ser subrayado aquí. Mientras el hombre pecador podía temer para su porvenir las consecuencias de su culpa y esperarse una vida humana disminuida, en cambio recibe de Cristo Salvador \emph{la posibilidad de un completo desarrollo humano}. No sólo es liberado de la esclavitud en la que le aprisionaban sus culpas, sino que puede alcanzar \emph{una perfección humana} superior a la que poseía antes del pecado. Cristo le ofrece una vida humana más abundante y más elevada. Por el hecho de que en Cristo la divinidad no ha comprimido en modo alguno a la humanidad sino que la ha elevado a un grado supremo de desarrollo, con su vida divina comunica a los hombres una vida humana más intensa y completa.
					
					Que Jesús sea el Dios Salvador hecho hombre significa, pues, que ya \emph{en el hombre nada está perdido}. Todo lo que había sido herido, manchado por el pecado, puede revivir y florecer. Esto explica cómo la gracia cristiana favorece el pleno ejercicio de todas las facultades humanas y también la afirmación de toda personalidad, tanto la femenina como la masculina. Reconciliando al hombre con Dios, la religión cristiana tiende a promover todo lo que es humano.
					
					Por tanto, podemos unirnos al canto que resonó en la gruta de Belén y proclamar con los ángeles: \textquote{Gloria a Dios en las alturas y paz en la tierra a los hombres de buena voluntad}.					
				\end{body}
				
\newsection


\section{Temas}

\rbr{El Directorio Homilético recoge los temas de la Navidad en un solo grupo, ver página \pageref{navidad_temas}}.