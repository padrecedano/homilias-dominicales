\chapter{Domingo IV de Adviento (A)}

	\section{Lecturas}

		\rtitle{PRIMERA LECTURA}

		\rbook{Del libro del profeta Isaías} \rred{7, 10-14}

		\rtheme{Mirad: la virgen está encinta}
		
		\begin{scripture}
			
			El Señor volvió a hablar a Ajaz y le dijo:
			
			\>{Pide un signo al Señor, tu Dios: en lo hondo del abismo o en lo alto del cielo}.
			
			Respondió Ajaz:
			
			\>{No lo pido, no quiero tentar al Señor}.
			
			Entonces dijo Isaías:
			
			\>{Escucha, casa de David: ¿no os basta cansar a los hombres, que cansáis incluso a mi Dios? Pues el Señor, por su cuenta, os dará un signo. Mirad: la virgen está encinta y da a luz un hijo, y le pondrá por nombre Enmanuel}.
		\end{scripture}

			\rtitle{SALMO RESPONSORIAL}

			\rbook{Salmo} \rred{23, 1-2. 3-4ab. 5-6}

			\rtheme{Va a entrar el Señor; Él es el Rey de la gloria}
			
			\begin{psbody}
				Del Señor es la tierra y cuanto la llena, 
				el orbe y todos sus habitantes: 
				él la fundó sobre los mares, 
				él la afianzó sobre los ríos. 
				
				¿Quién puede subir al monte del Señor? 
				¿Quién puede estar en el recinto sacro? 
				El hombre de manos inocentes y puro corazón, 
				que no confía en los ídolos. 
				
				Ese recibirá la bendición del Señor, 
				le hará justicia el Dios de salvación. 
				Esta es la generación que busca al Señor, 
				que busca tu rostro, Dios de Jacob.
			\end{psbody}
		

			\rtitle{SEGUNDA LECTURA}

			\rbook{De la carta del apóstol san Pablo a los Romanos} \rred{1, 1-7}
			
			\rtheme{Jesucristo, de la estirpe de David, Hijo de Dios	}		
			
			\begin{scripture}
				Pablo, siervo de Cristo Jesús, llamado a ser apóstol, escogido para el Evangelio de Dios, que fue prometido por sus profetas en las Escrituras Santas y se refiere a su Hijo, nacido de la estirpe de David según la carne, constituido Hijo de Dios en poder según el Espíritu de santidad por la resurrección de entre los muertos: Jesucristo nuestro Señor.
				
				Por él hemos recibido la gracia del apostolado, para suscitar la obediencia de la fe entre todos los gentiles, para gloria de su nombre. Entre ellos os encontráis también vosotros, llamados de Jesucristo.
				
				A todos los que están en Roma, amados de Dios, llamados santos, gracia y paz de Dios nuestro Padre y del Señor Jesucristo.
			\end{scripture}

			\rtitle{EVANGELIO}

			\rbook{Del Santo Evangelio según san Mateo} \rred{1, 18-24}

			\rtheme{Jesús nacerá de María, desposada con José, hijo de David}

			\begin{scripture}
				La generación de Jesucristo fue de esta manera: 
				
				María, su madre, estaba desposada con José y, antes de vivir juntos, resultó que ella esperaba un hijo por obra del Espíritu Santo.
				
				José, su esposo, como era justo y no quería difamarla, decidió repudiarla en privado. Pero, apenas había tomado esta resolución, se le apareció en sueños un ángel del Señor que le dijo: \>{José, hijo de David, no temas acoger a María, tu mujer, porque la criatura que hay en ella viene del Espíritu Santo. Dará a luz un hijo y tú le pondrás por nombre Jesús, porque él salvará a su pueblo de sus pecados}.
				
				Todo esto sucedió para que se cumpliese lo que había dicho el Señor por medio del profeta:
				
				\>{Mirad: la Virgen concebirá y dará a luz un hijo y le pondrán por nombre Enmanuel, que significa \textquote{Dios-con-nosotros}}.
				
				Cuando José se despertó, hizo lo que le había mandado el ángel del Señor y acogió a su mujer.
			\end{scripture}

	
	\newsection

	\section{Comentario Patrístico}
	
		\subsection{San Beda el Venerable, presbítero}
		
			\ptheme{¡Oh grande e insondable misterio!}

			\src{Homilía 5 en la vigilia de Navidad: CCL 122, 32-36}
			
			\begin{body}
				En breves palabras, pero llenas de verismo, describe el evangelista san Mateo el nacimiento del Señor y Salvador nuestro Jesucristo, por el que el Hijo de Dios, eterno antes del tiempo, apareció en el tiempo Hijo del hombre. Al conducir el evangelista la serie genealógica partiendo de Abrahán para acabar en José, el esposo de María, y enumerar ---según el acostumbrado orden de la humana generación--- la totalidad así de los genitores como de los engendrados, y disponiéndose a hablar del nacimiento de Cristo, subrayó la enorme diferencia existente entre éste y el resto de los nacimientos: los demás nacimientos se producen por la normal unión del hombre y de la mujer mientras que él, por ser Hijo de Dios, vino al mundo por conducto de una Virgen. Y era conveniente bajo todos los aspectos que, al decidir Dios hacerse hombre para salvar a los hombres, no naciera sino de una virgen, pues era inimaginable que una virgen engendrara a ningún otro, sino a uno que, siendo Dios, ella lo procreara como Hijo.
				
				\emph{Mirad} ---dice--- \emph{la Virgen está encinta y dará a luz un hijo, y le pondrá por nombre Emmanuel (que significa \textquote{Dios-con-nosotros})}. El nombre que el profeta da al Salvador, \textquote{\emph{Dios-con-nosotros}}, indica la doble naturaleza de su única persona. En efecto, el que es Dios nacido del Padre antes de los tiempos, es el mismo que, en la plenitud de los tiempos, se convirtió, en el seno materno, en Emmanuel, esto es, en \textquote{\emph{Dios-con-nosotros}}, ya que se dignó asumir la fragilidad de nuestra naturaleza en la unidad de su persona, cuando \emph{la Palabra se hizo carne y acampó entre nosotros}, esto es, cuando de modo admirable comenzó a ser lo que nosotros somos, sin dejar de ser lo que era asumiendo de forma tal nuestra naturaleza que no le obligase a perder lo que él era.
				
				Dio, pues, a luz María a su hijo primogénito, es decir, al hijo de su propia carne; dio a luz al que, antes de la creación, había nacido Dios de Dios, y en la humanidad en que fue creado, superaba ampliamente a toda creatura. Y él \emph{le puso} ---dice--- \emph{por nombre Jesús}.
				
				Jesús es el nombre del hijo que nació de la virgen, nombre que significa ---según la explicación del ángel--- que él iba a \emph{salvar a su pueblo de los pecados}. Y el que salva de los pecados, salvará igualmente de las corruptelas de alma y cuerpo, secuela del pecado.
				
				La palabra \textquote{\emph{Cristo}} connota la dignidad sacerdotal o regia. En la ley, tanto los sacerdotes como los reyes eran llamados \textquote{\emph{cristos}} por el crisma, es decir, por la unción con el óleo sagrado: eran un signo de quien, al manifestarse en el mundo como verdadero Rey y Pontífice, fue ungido \emph{con aceite de júbilo entre todos sus compañeros}.
				
				Debido a esta unción o crisma, se le llama \emph{Cristo}; a los que participan de esta unción, es decir, de esta gracia espiritual, se les llama \textquote{\emph{cristianos}}. Que él, por ser nuestro Salvador, nos salve de los pecados; en cuanto Pontífice, nos reconcilie con Dios Padre; en su calidad de Rey se digne darnos el reino eterno de su Padre, Jesucristo nuestro Señor, que con el Padre y el Espíritu Santo vive y reina y es Dios por todos los siglos de los siglos. Amén.
			\end{body}
				
				
				
				Durante su vida, que fue una peregrinación en la fe, José, al igual que María, permaneció fiel a la llamada de Dios hasta el final. La vida de ella fue el cumplimiento hasta sus últimas consecuencias de aquel primer \textquote{\emph{fiat}} pronunciado en el momento de la anunciación mientras que José en el momento de su \textquote{anunciación} no pronunció palabra alguna. Simplemente él \textquote{\emph{hizo} como el ángel del Señor le había mandado} (\emph{Mt} 1, 24). \emph{Y este primer \textquote{hizo} es el comienzo del \textquote{camino de José}}. A lo largo de este camino, los Evangelios no citan ninguna palabra dicha por él. Pero el \emph{silencio de José} posee una especial elocuencia: gracias a este silencio se puede leer plenamente la verdad contenida en el juicio que de él da el Evangelio: el \textquote{justo} (\emph{Mt} 1, 19).
				
				Hace falta saber leer esta verdad, porque ella contiene \emph{uno de los testimonios más importantes acerca del hombre y de su vocación}. En el transcurso de las generaciones la Iglesia lee, de modo siempre atento y consciente, dicho testimonio, casi como si sacase del tesoro de esta figura insigne \textquote{lo nuevo y lo viejo} (\emph{Mt} 13, 52).

\textbf{San Juan Pablo II, papa,} \emph{Redemptoris Custos,} 17.


	\newsection
	
	\section{Homilías}

		\subsection{San Juan Pablo II, papa}

			\subsubsection{Homilía (1983): Nos enseñan a acoger a Cristo}

				\src{Visita pastoral a la Parroquia romana de San Jorge Mártir. \\18 de diciembre de 1983.}
				
				\begin{body}
					1. \textquote{¿Quién puede subir al monte del Señor? ¿Quién puede estar en el recinto sacro? El hombre de manos inocentes y puro corazón} (Sal 24, 3-4). 
					
					Queridísimos hermanos y hermanas: La liturgia de este domingo cuarto y último de Adviento, insiste sobre el tema de la cercanía, recordando la llegada inminente del que debe venir, y trazando al mismo tiempo las características de quien, con motivo de esta venida, se acerca, a su vez, a Dios.
					
					Desde los primeros versículos, el \textbf{Salmo responsorial} nos lleva a lo alto, a Aquel que es Señor de la tierra, de cuanto la llena, del universo y de sus habitantes. Dios creó todo para regalárselo al hombre, a fin de que este, por la contemplación de lo creado, pueda reconocerlo y acercarse a él.
					
					Según la expresión del Salmista, Dios, precisamente porque trasciende todo el universo material, está \textquote{por encima} del mundo; y así, el acercamiento a él se presenta como un \textquote{subir}. Pero no se trata de un desplazamiento material en el espacio, sino de una apertura, una orientación del espíritu; una actividad \textquote{santa}, propia de los que buscan a Dios, \textquote{el grupo que busca el rostro del Dios de Jacob}.
					
					La liturgia de hoy nos hace ver concretamente las dos figuras a las que les fue dado acercarse más a Aquel que tenía que venir: \textbf{María y José}. Son las dos personas culminantes del tiempo del Adviento, situadas en la etapa de la cercanía más grande de Dios mismo.
					
					2. La figura de \textbf{María}, en la presente liturgia, queda delineada en dos pasajes de la Escritura: en el Antiguo Testamento, como prefiguración, con el texto de Isaías (Is 7, 10-14); y en el Nuevo, como realización, con el texto de Mateo (Mt 1, 18-24).
					
					Los libros del Antiguo Testamento, al describirnos la historia de la salvación, ponen de relieve, paso a paso ---como observa el Concilio (Lumen gentium, 55)---, cada vez con más claridad a la Madre del Redentor. Bajo este haz de luz ella queda proféticamente bosquejada en la imagen de la Virgen que concebirá y dará a luz un Hijo, cuyo nombre será Emmanuel, que quiere decir \textquote{Dios con nosotros}. Es apenas una anticipación eficaz para prefigurar un ser sin igual predestinado por Dios, el cual, ya muchos siglos antes, comienza a proyectar hacia nosotros algunos rasgos de su grandeza.
					
					Este texto de \textbf{Isaías}, durante el curso de los siglos, se lee y entiende en la Iglesia a la luz de la revelación posterior. Lo que en el Antiguo Testamento, con sus aperturas mesiánicas, era un comienzo, se convierte en claridad en el Nuevo Testamento. \textbf{San Mateo} reconoce en las palabras de Isaías a la mujer que, por obra del Espíritu Santo, concibió virginalmente, con exclusión de intervención de varón.
					
					Jesús es Aquel que salvará al pueblo de sus pecados. Y ella, María, es la Madre de Jesús. El Hijo de Dios \textquote{viene} a su seno para hacerse hombre. Ella lo acoge. Jamás Dios se acercó tanto al ser humano como en este caso en el que el Hijo y su Madre entran en una relación estrecha.
					
					3. Al mismo tiempo, Mateo procura cuidadosamente hacernos partícipes de la acogida consciente y amorosa de parte de \textbf{José}. Él, el esposo, que por sí solo no puede explicarse el acontecimiento nuevo que se realizaba ante sus ojos, es iluminado por la intervención del Ángel del Señor sobre la naturaleza de la maternidad de María. \textquote{Lo concebido en ella es obra del Espíritu Santo} (Mt 1, 20).
					
					De esta manera, José es puesto al corriente de los hechos y es llamado a insertarse en el designio salvífico de Dios. Ahora él sabe quién es el Niño que ha de nacer y quién es la Madre. De acuerdo con la invitación del ángel, llevó consigo a su esposa, no la repudió. \textquote{Acogiendo} a María, José acoge también al que en ella ha sido concebido por obra admirable de Dios, para quien nada es imposible.
					
					Concentrándose en estos dos personajes del Adviento, la liturgia, nos conduce directamente al terreno de la Navidad.
					
					4. Llegados a este punto, abramos el oído para escuchar la \textbf{segunda lectura}, tomada de la Carta dirigida por el apóstol san Pablo a los \textbf{Romanos}. Ella nos habla a nosotros, habitantes de la {[}Roma{]} moderna. El apóstol Pablo proclama la venida de Cristo a {[}Roma{]}, a nuestra propia ciudad: es la venida mediante el Evangelio, \textquote{el Evangelio de Dios\ldots{} acerca de su Hijo, nacido del linaje de David según la carne, constituido Hijo de Dios con poder según el Espíritu de santidad por su resurrección de entre los muertos, Jesucristo Señor nuestro, por quien recibimos la gracia y el apostolado} (Rm 1, 1-5).
					
					5. Han pasado casi dos mil años desde que el apóstol escribió estas palabras y hoy siguen siendo actuales. Estas palabras se dirigen a nosotros hoy y no nos queda más que ponernos en actitud de disponibilidad para acoger a Jesucristo por medio del Evangelio que anuncia la Iglesia, del mismo modo que lo acogieron los primeros cristianos \ldots{} Queremos acogerlo, por utilizar la expresión del Apóstol, en toda la verdad de su Divinidad y de su Humanidad.
					
					Recibámoslo la noche de Belén en el conjunto de su misterio pascual. \textquote{Por su resurrección de entre los muertos} Cristo ha sido \textquote{constituido Hijo de Dios con poder según el Espíritu de santidad}. Mediante el misterio pascual se ha revelado plenamente la filiación divina del que nació la noche de Belén.
					
					Acojamos a Cristo Hijo de Dios, Aquel que debe venir; y, al acogerlo, esforcémonos por asemejarnos a María y a José, que fueron los primeros en acogerlo mediante la fe con la fuerza del Espíritu Santo. Efectivamente, en ellos se manifiesta la plena madurez del Adviento.
					
					6. Hoy quiero desear a todos vosotros esa madurez de vida cristiana y esa disponibilidad abierta y generosa para acoger en la riqueza de su verdad al Hijo de Dios nacido según la carne, os la quiero desear hoy a todos \ldots{} Os animo a cada uno de vosotros a seguir adelante.
					
					7. Queridísimos hermanos y hermanas: \textquote{Por él hemos recibido la gracia y el apostolado} \ldots{} Deseo que esta celebración traiga para nosotros la Gracia del Adviento divino.
					
					\textquote{Mirad: la Virgen concebirá y dará a luz un Hijo, y le pondrán por nombre Emmanuel, que significa: Dios con nosotros} (Mt 1, 23). Que la Navidad traiga el cumplimiento del Adviento en cada uno de nosotros \ldots{} Que el Emmanuel, el Dios-con-nosotros, se convierta en la alegría y en la esperanza de todos los corazones humanos.
				\end{body}

			\subsubsection{Homilía (1986):}

				\src{Concelebración Eucarística en el Colegio Belga de Roma. \\21 de diciembre de 1986.}

				\begin{body}
					1. \textquote{\emph{El hijo que en ella se engendra proviene del Espíritu Santo}}.
					
					En este cuarto domingo de Adviento, que nos prepara para la Navidad, contemplamos a María que lleva en ella al Salvador del mundo. La miramos con la mirada silenciosa y admirativa de José.
					
					Desde la Anunciación, la Virgen María guardó dentro de sí un secreto \ldots{} que la llenó al mismo tiempo de alegría y temor de Dios, de gratitud y responsabilidad hacia el Hijo del Altísimo que había concebido, de disponibilidad hacia el Espíritu Santo que la había tomado bajo su sombra en una alianza íntima.
					
					Ella no pudo tomar la iniciativa de compartir este secreto divino. Es el ángel del Señor quien se lo revela a José, el justo prometido en matrimonio a la Virgen de Nazaret.
					
					Con José, veneramos a María, la bendita entre todas las mujeres. Adoramos el fruto bendito de su vientre, que ella se prepara para traer al mundo y darlo para la salvación de los hombres, cooperando en la obra de Dios que entregó a su Hijo único al mundo. Releamos a la luz de la fe, como María, como los evangelistas, como la primera comunidad cristiana, la profecía de Isaías: \textquote{He aquí que la Virgen concebirá \ldots{} un hijo al que se le dará el nombre de Emmanuel, Dios con nosotros}.
					
					Este tiempo feliz que precede a la natividad fue para María y José el de espera y esperanza, el de maduración y florecimiento de la vida del Niño Jesús, el de la meditación silenciosa de su Madre ante el designio misterioso de Dios, ante la maravilla de su amor.
					
					Queridos hermanos y hermanas, también nosotros damos la bienvenida al Salvador que viene, saltamos con alegría y confianza, ¡presentámosle todas las intenciones de este mundo con el que somos solidarios y que tanto necesita la luz y la salvación de Dios!
					
					{[} \ldots{}{]}
					
					Que la Santísima Virgen María nos proteja, en la alegría de recibir incesantemente la gracia del Redentor y sobre todo el don de Dios mismo, Emmanuel, que permanece en nosotros por su Palabra, su Espíritu y su Eucaristía.
				\end{body}

			\subsubsection{Homilía (1989):}

				\src{Visita pastoral a la Parroquia Romana de Santa Ana. \\24 de diciembre de 1989.}

				\begin{body}
					1. \textquote{Lo llamarás Jesús, porque él salvará a su pueblo de sus pecados} (\emph{Mt} 1, 21).
					
					En este último domingo de Adviento y Nochebuena, nos reunimos para la Santa Misa, queridos hermanos y hermanas de la parroquia de Santa Ana en el Vaticano. Estamos invitados a reflexionar sobre el misterio del Emmanuel---Dios con nosotros, mientras ya nos preparamos para recibir su venida en la ya inminente liturgia de la noche santa.
					
					Escuchamos con esta expectativa el testimonio de dos personas particularmente cercanas al acontecimiento del nacimiento de Jesús: María y José. El \textbf{evangelio} de hoy nos habla de ellos.
					
					2. La visión de José, narrada por Mateo, debe acercarse a la página lucana del anuncio que el ángel le dio a María. Ambas visiones se despliegan según un modo de proceder, que en la tradición bíblica pretende hacer comprender y aceptar la misión a la que Dios llama: \textquote{Darás a luz un hijo \ldots{} el Espíritu Santo vendrá sobre ti \ldots{} He aquí la esclava del Señor} (cf. \emph{Lc} 1, 35-38); \textquote{No tengas miedo de tomar contigo a María, tu esposa \ldots{} ella dará a luz un hijo y lo llamarás Jesús \ldots{} José hizo lo que le dijo el ángel} (cf. \emph{Mt} 1, 20-24).
					
					La visión y el anuncio trazan también al mismo tiempo la línea de fe necesaria para acoger adecuadamente la revelación del misterio de Dios y reconocer a Jesús de Nazaret en su verdad. Por eso el \textbf{pasaje evangélico} que hemos leído, si bien insistimos en la actitud interior de humilde disponibilidad hacia la iniciativa de Dios, vislumbra también con fuerza y ​​claridad las cualidades del Mesías: será el Hijo de David, el anunciado por los profetas, el salvador de su pueblo. Con esta clara perspectiva podremos escuchar mañana el anuncio del Evangelio de Juan, que nos hablará del Verbo hecho carne, \textquote{lleno de gracia y de verdad} (\emph{Jn} 1, 14).
					
					{[} \ldots{}{]}
					
					5. \textquote{Va a entrar el Señor; Él es el Rey de la gloria}. Así cantamos en el salmo responsorial.
					
					El Señor viene, lo esperamos con gloria y esperanza viva. El mensaje de la liturgia nos ha dicho quién es.
					
					Viene el Rey de la gloria, el que también prometió volver a nosotros al final de los tiempos, \textquote{juzgará entre las naciones \ldots{} será árbitro de pueblos numerosos} (\emph{Is} 2, 4).
					
					Mañana, y ya esta noche, lo contemplaremos manso y humilde, un niño como cualquier otro \textquote{nacido de mujer}; seremos invitados a amarlo y reconocerlo, como un día los pastores de Belén. Dará \textquote{gracia y paz} a todos los que son \textquote{amados por Dios y santos por vocación} (\emph{Rom} 1, 7).
					
					Volveremos a escuchar a los ángeles anunciar \textquote{una gran alegría que será para todo el pueblo: hoy nos ha nacido \ldots{} un Salvador, que es Cristo el Señor} (\emph{Lc} 2, 10-11).
					
					Ésta es nuestra fe; nuestra verdadera alegría se basa en estas palabras y en este anuncio. Esperamos a Dios con nosotros para siempre, y hacia él caminamos en la fe para encontrarlo con el alma pura de los \textquote{hombres que él ama} (\emph{Lc} 2, 14).
				\end{body}

			\subsubsection{Ángelus (1992)}
				\src{20 de diciembre de 1992.}
				
				\begin{body}
					\emph{Amadísimos hermanos y hermanas:}
					
					1. Faltan ya pocos días para la celebración de la Navidad del Señor y queremos vivirlos siguiendo las huellas de María y haciendo nuestros en la medida de lo posible, los sentimientos que ella experimentó en la trémula espera del nacimiento de Jesús.
					
					El evangelista Lucas narra que la Virgen santa y su esposo José se dirigieron de Galilea a Judea para ir a Belén, la ciudad de David, obedeciendo un decreto del emperador romano que ordenaba un censo general del Imperio.
					
					Pero, ¿quién podía reparar en ellos? Pertenecían a la innumerable legión de pobres, a quienes la vida a duras penas regala un rincón para vivir, y que no dejan rastro en las crónicas. De hecho no encontraron acomodo en ningún sitio, a pesar de que llevaban el \textquote{secreto} del mundo.
					
					Podemos intuir cuáles eran los sentimientos de María, totalmente abandonada en las manos del Señor. Ella es la mujer creyente: en la profundidad de su obediencia interior madura la plenitud de los tiempos (cf. \emph{Ga} 4, 4).
					
					2. Por estar enraizada en la fe, la Madre del Verbo hecho hombre \emph{encarna la gran esperanza del mundo}. En ella confluye tanto la espera mesiánica de Israel como el anhelo de salvación de la humanidad entera. En su espíritu resuena el grito de dolor de los que, en toda época de la historia, se sienten abrumados por las dificultades de la vida: los hambrientos y los necesitados, los enfermos y las víctimas del odio y la guerra, los que no tienen hogar ni trabajo y los que viven solos y marginados, los que se sienten aplastados por la violencia y la injusticia o rechazados por la desconfianza y la indiferencia, los desanimados y los defraudados.
					
					Para los hombres de toda raza y cultura, sedientos de amor, de fraternidad y de paz, María se prepara a dar a luz el fruto divino de su vientre. Por más oscuro que pueda parecer el horizonte, hay un alba que despunta. La humanidad, como recuerda san Pablo, gime y \textquote{sufre dolores de parto} (\emph{Rm} 8, 22): en el nacimiento del Hijo de Dios todo renace, todo está llamado a vida nueva.
					
					3. Queridos hermanos y hermanas preparémonos para la Navidad con la fe y la esperanza de María. Dejemos que el mismo amor que vibra en su adhesión al plan divino toque nuestro corazón. La Navidad es tiempo de renovación y fraternidad: miremos a nuestro alrededor, miremos a lo lejos. El hombre que sufre, dondequiera que se encuentre, nos atañe. Allí se encuentra el belén al que debemos dirigirnos, con solidaridad activa, para encontrar de verdad al Redentor que nace en el mundo. Caminemos, por consiguiente, hacia la Noche Santa con María, la Madre del Amor. Con ella esperemos el cumplimiento del misterio de la salvación.
				\end{body}

\newsection

		\subsection{Benedicto XVI, papa}

			\subsubsection{Ángelus (2010): San José miró el futuro con fe}

				\src{19 de diciembre del 2010.}
				
				\begin{body}
					En este cuarto domingo de Adviento el \textbf{evangelio de san Mateo} narra cómo sucedió el nacimiento de Jesús situándose desde el punto de vista de san José. Él era el prometido de María, la cual \textquote{antes de empezar a estar juntos ellos, se encontró encinta por obra del Espíritu Santo} (\emph{Mt} 1, 18). El Hijo de Dios, realizando una antigua profecía (cf. \emph{Is} 7, 14), se hace hombre en el seno de una virgen, y ese misterio manifiesta a la vez el amor, la sabiduría y el poder de Dios a favor de la humanidad herida por el pecado. \textbf{San José} se presenta como hombre \textquote{justo} (\emph{Mt} 1, 19), fiel a la ley de Dios, disponible a cumplir su voluntad. Por eso entra en el misterio de la Encarnación después de que un ángel del Señor, apareciéndosele en sueños, le anuncia: \textquote{José, hijo de David, no temas tomar contigo a María, tu mujer, porque lo engendrado en ella es del Espíritu Santo. Dará a luz un hijo y tú le pondrás por nombre Jesús, porque él salvará a su pueblo de sus pecados} (\emph{Mt} 1, 20-21). Abandonando el pensamiento de repudiar en secreto a María, la toma consigo, porque ahora sus ojos ven en ella la obra de Dios.
					
					San Ambrosio comenta que \textquote{en José se dio la amabilidad y la figura del justo, para hacer más digna su calidad de testigo} (\emph{Exp. Ev. sec. Lucam} II, 5: ccl 14, 32-33). Él ---prosigue san Ambrosio--- \textquote{no habría podido contaminar el templo del Espíritu Santo, la Madre del Señor, el seno fecundado por el misterio} (\emph{ib}., II, 6: CCL 14, 33). A pesar de haber experimentado turbación, José actúa \textquote{como le había ordenado el ángel del Señor}, seguro de hacer lo que debía. También poniendo el nombre de \textquote{Jesús} a ese Niño que rige todo el universo, él se inserta en el grupo de los servidores humildes y fieles, parecido a los ángeles y a los profetas, parecido a los mártires y a los apóstoles, como cantan antiguos himnos orientales. San José anuncia los prodigios del Señor, dando testimonio de la virginidad de María, de la acción gratuita de Dios, y custodiando la vida terrena del Mesías. Veneremos, por tanto, al padre legal de Jesús (cf. \emph{Catecismo de la Iglesia católica}, n. 532), porque en él se perfila el hombre nuevo, que mira con fe y valentía al futuro, no sigue su propio proyecto, sino que se confía totalmente a la infinita misericordia de Aquel que realiza las profecías y abre el tiempo de la salvación.
					
					Queridos amigos, a san José, patrono universal de la Iglesia, deseo confiar a todos los pastores, exhortándolos a ofrecer \textquote{a los fieles cristianos y al mundo entero la humilde y cotidiana propuesta de las palabras y de los gestos de Cristo} (Carta de convocatoria del Año sacerdotal). Que nuestra vida se adhiera cada vez más a la Persona de Jesús, precisamente porque \textquote{el que es la Palabra asume él mismo un cuerpo; viene de Dios como hombre y atrae a sí toda la existencia humana, la lleva al interior de la palabra de Dios} (\emph{Jesús de Nazaret,} Madrid 2007, p. 387). Invoquemos con confianza a la Virgen María, la llena de gracia \textquote{adornada de Dios}, para que, en la Navidad ya inminente, nuestros ojos se abran y vean a Jesús, y el corazón se alegre en este admirable encuentro de amor.
				\end{body}
			
	\newsection

		\subsection{Francisco, papa}

			\subsubsection{Ángelus (2013): La prueba de José}
			
				\src{Plaza de San Pedro. \\22 de diciembre del 2013.}
				
				\begin{body}
					En este cuarto domingo de Adviento, el Evangelio nos relata los hechos que precedieron el nacimiento de Jesús, y el evangelista Mateo los presenta desde el punto de vista de san José, el prometido esposo de la Virgen María.
					
					\textbf{José} y \textbf{María} vivían en Nazaret; aún no vivían juntos, porque el matrimonio no se había realizado todavía. Mientras tanto, María, después de acoger el anuncio del Ángel, quedó embarazada por obra del Espíritu Santo. Cuando José se dio cuenta del hecho, quedó desconcertado. El Evangelio no explica cuáles fueron sus pensamientos, pero nos dice lo esencial: él busca cumplir la voluntad de Dios y está preparado para la renuncia más radical. En lugar de defenderse y hacer valer sus derechos, José elige una solución que para él representa un enorme sacrificio. Y el Evangelio dice: \textquote{Como era justo y no quería difamarla, decidió repudiarla en privado} (1, 19).
					
					Esta breve frase resume un verdadero drama interior, si pensamos en el amor que José tenía por María. Pero también en esa circunstancia José quiere hacer la voluntad de Dios y decide, seguramente con gran dolor, repudiar a María en privado. Hay que meditar estas palabras para comprender cuál fue la prueba que José tuvo que afrontar los días anteriores al nacimiento de Jesús. Una prueba semejante a la del sacrificio de Abrahán, cuando Dios le pidió el hijo Isaac (cf. \emph{Gn} 22): renunciar a lo más precioso, a la persona más amada.
					
					Pero, como en el caso de Abrahán, el Señor interviene: encontró la fe que buscaba y abre una vía diversa, una vía de amor y de felicidad: \textquote{José ---le dice--- no temas acoger a María, tu mujer, porque la criatura que hay en ella viene del Espíritu Santo} (\emph{Mt} 1, 20).
					
					Este \textbf{Evangelio} nos muestra toda la grandeza del alma de san José. Él estaba siguiendo un buen proyecto de vida, pero Dios reservaba para él otro designio, una misión más grande. José era un hombre que siempre dejaba espacio para escuchar la voz de Dios, profundamente sensible a su secreto querer, un hombre atento a los mensajes que le llegaban desde lo profundo del corazón y desde lo alto. No se obstinó en seguir su proyecto de vida, no permitió que el rencor le envenenase el alma, sino que estuvo disponible para ponerse a disposición de la novedad que se le presentaba de modo desconcertante. Y así, era un hombre bueno. No odiaba, y no permitió que el rencor le envenenase el alma. ¡Cuántas veces a nosotros el odio, la antipatía, el rencor nos envenenan el alma! Y esto hace mal. No permitirlo jamás: él es un ejemplo de esto. Y así, José llegó a ser aún más libre y grande.
					
					Aceptándose según el designio del Señor, José se encuentra plenamente a sí mismo, más allá de sí mismo. Esta libertad de renunciar a lo que es suyo, a la posesión de la propia existencia, y esta plena disponibilidad interior a la voluntad de Dios, nos interpelan y nos muestran el camino.
					
					Nos disponemos entonces a celebrar la Navidad contemplando a María y a José: María, la mujer llena de gracia que tuvo la valentía de fiarse totalmente de la Palabra de Dios; José, el hombre fiel y justo que prefirió creer al Señor en lugar de escuchar las voces de la duda y del orgullo humano. Con ellos, caminamos juntos hacia Belén.
				\end{body}
			

			\subsubsection{Ángelus (2016): Respuesta a la cercanía de Dios}
			
				\src{Plaza de San Pedro. \\18 de diciembre del 2016.}
				
				\begin{body}
					La liturgia de hoy, cuarto y último domingo de Adviento, está caracterizada por el tema de la cercanía, la cercanía de Dios a la humanidad. El pasaje del \textbf{Evangelio} (cfr. Mt 1,18-24) nos muestra a las dos personas que más que cualquier otra están envueltas en este misterio de amor: la Virgen \textbf{María} y su esposo \textbf{José}. Misterio de amor, misterio de cercanía de Dios con la humanidad.
					
					\textbf{María} es presentada a la luz de la profecía que dice: \textquote{La Virgen concebirá y dará a luz un hijo} (Mt 1, 23). El evangelista Mateo reconoce que aquello ha acontecido en María, quien ha concebido a Jesús por obra del Espíritu Santo (cfr. v. 18). El hijo de Dios \textquote{viene} en su vientre para convertirse en hombre y Ella lo acoge. Así, de manera única, Dios se ha acercado al ser humano tomando la carne de una mujer: Dios se acercó a nosotros y tomó la carne de una mujer. También a nosotros, de manera diferente, Dios se acerca con su gracia para entrar en nuestra vida y ofrecernos en don a su Hijo. Y nosotros ¿qué hacemos? ¿Lo acogemos, lo dejamos acercarse o lo rechazamos, lo echamos? Como a María, que ofreciéndose libremente al Señor de la historia, se le ha permitido cambiar el destino de la humanidad, así también nosotros, acogiendo a Jesús y tratando de seguirlo cada día, podemos cooperar con su diseño de salvación sobre nosotros mismos y sobre el mundo. Por lo tanto María se nos presenta como el modelo al cual mirar y apoyo sobre el cual contar en nuestra búsqueda de Dios, en nuestra cercanía a Dios, con este dejar que Dios se acerque a nosotros, y en nuestro empeño por construir la civilización del amor.
					
					El otro protagonista del Evangelio de hoy es San \textbf{José}. El evangelista pone en evidencia cómo José por sí solo no pueda darse una explicación del acontecimiento que ve verificarse ante sus ojos, o sea el embarazo de María. Precisamente entonces, en aquel momento de la duda, también del miedo, Dios se le acerca a través de un mensajero suyo y él es iluminado sobre la naturaleza de aquella maternidad: \textquote{porque lo que ha sido engendrado en ella proviene del Espíritu Santo} (v. 20). Así, frente al evento extraordinario, que ciertamente suscita en su corazón tantas interrogantes, José confía totalmente en Dios que se le acerca y, siguiendo su invitación, no repudia a su comprometida sino que la toma consigo y la desposa. Acogiendo a María, José acoge conscientemente y con amor a Aquel que ha sido concebido en ella por obra admirable de Dios, para quien nada es imposible. José, hombre humilde y justo (cfr v. 19), nos enseña a confiarnos siempre en Dios, que se nos acerca: cuando Dios se nos acerca debemos confiarnos. José nos enseña a dejarnos guiar por Él con voluntaria obediencia.
					
					Estas dos figuras, \textbf{María y José}, que han sido los primeros en acoger a Jesús mediante la fe, nos introducen en el misterio de la Navidad. María nos ayuda a colocarnos en actitud de disponibilidad para acoger al Hijo de Dios en nuestra vida concreta, en nuestra carne. José nos insta a buscar siempre la voluntad de Dios y a seguirla con total confianza. \textquote{La Virgen concebirá y dará a luz un hijo a quien pondrás el nombre de Emanuel, que traducido significa: Dios-con-nosotros} (Mt 1,23). ). Así dice el ángel: \textquote{Emanuel se llamará el niño, que significa Dios-con-nosotros} o sea Dios cerca a nosotros. Y a Dios que se acerca yo le abro la puerta ---al Señor--- cuando siento una inspiración interior, cuando siento que me pide hacer algo más por los demás, cuando me llama a la oración. Dios-con-nosotros, Dios que se acerca. Que este anuncio de esperanza, que se cumple en Navidad, lleve a cumplimiento la espera de Dios también en cada uno de nosotros, en toda la Iglesia, y en tantos pequeños que el mundo desprecia, pero que Dios ama y a los cuales se acerca.
				\end{body}

			\subsubsection{Ángelus (2019): La fe inquebrantable de José}
			
				\src{Plaza de San Pedro. \\22 de diciembre del 2019.}
				
				\begin{body}
					En este cuarto y último domingo de Adviento, el \textbf{Evangelio} (cf. \emph{Mateo} 1, 18-24) nos guía hacia la Navidad, a través de la experiencia de san \textbf{José}, una figura aparentemente de segundo plano, pero en cuya actitud está contenida toda la sabiduría cristiana. Él, junto con Juan Bautista y \textbf{María}, es uno de los personajes que la liturgia nos propone para el tiempo de Adviento; y de los tres es el más modesto. El que no predica, no habla, sino que trata de hacer la voluntad de Dios; y lo hace al estilo del Evangelio y de las Bienaventuranzas. Pensemos: \textquote{Bienaventurados los pobres de espíritu, porque de ellos es el Reino de los Cielos} (\emph{Mateo} 5, 3). Y José es pobre porque vive de lo esencial, trabaja, vive del trabajo; es la pobreza típica de quien es consciente de que depende en todo de Dios y pone en Él toda su confianza.
					
					La narración del \textbf{Evangelio de hoy} presenta una situación humanamente incómoda y conflictiva. \textbf{José} y \textbf{María} están comprometidos; todavía no viven juntos, pero ella está esperando un hijo por obra de Dios. José, ante esta sorpresa, naturalmente permanece perturbado pero, en lugar de reaccionar de manera impulsiva y punitiva ―como era costumbre, la ley lo protegía― busca una solución que respete la dignidad y la integridad de su amada María. El Evangelio lo dice así: \textquote{Su marido José, como era justo y no quería ponerla en evidencia, resolvió repudiarla en secreto} (v. 19). José sabía que si denunciaba a su prometida, la expondría a graves consecuencias, incluso a la muerte. Tenía plena confianza en María, a quien eligió como su esposa. No entiende, pero busca otra solución.
					
					Esta circunstancia inexplicable le llevó a cuestionar su compromiso; por eso, con gran sufrimiento, decidió separarse de María sin crear escándalo. Pero \textbf{el Ángel del Señor} interviene para decirle que la solución que él propone no es la deseada por Dios. Por el contrario, el Señor le abrió un nuevo camino, un camino de unión, de amor y de felicidad, y le dijo: \textquote{José, hijo de David, no temas tomar contigo a María tu mujer porque lo engendrado en ella es del Espíritu Santo} (v. 20).
					
					En este punto, José confía totalmente en Dios, obedece las palabras del Ángel y se lleva a María con él. Fue precisamente esta confianza inquebrantable en Dios la que le permitió aceptar una situación humanamente difícil y, en cierto sentido, incomprensible. José entiende, en la fe, que el niño nacido en el seno de María no es su hijo, sino el Hijo de Dios, y él, José, será su guardián, asumiendo plenamente su paternidad terrenal. El ejemplo de este hombre gentil y sabio nos exhorta a levantar la vista, a mirar más allá. Se trata de recuperar la sorprendente lógica de Dios que, lejos de pequeños o grandes cálculos, está hecha de apertura hacia nuevos horizontes, hacia Cristo y Su Palabra.
					
					Que la Virgen María y su casto esposo José nos ayuden a escuchar a Jesús que viene, y que pide ser acogido en nuestros planes y elecciones.
				\end{body}
			
\newsection

	\section{Temas}
	
		Maternidad virginal de María

CEC 496-507, 495:

\textbf{La virginidad de María}

\textbf{496} Desde las primeras formulaciones de la fe (cf. DS 10-64), la Iglesia ha confesado que Jesús fue concebido en el seno de la Virgen María únicamente por el poder del Espíritu Santo, afirmando también el aspecto corporal de este suceso: Jesús fue concebido \textquote{absque semine ex Spiritu Sancto} (Cc Letrán, año 649; DS 503), esto es, sin elemento humano, por obra del Espíritu Santo. Los Padres ven en la concepción virginal el signo de que es verdaderamente el Hijo de Dios el que ha venido en una humanidad como la nuestra:

Así, S. Ignacio de Antioquía (comienzos del siglo II):

\textquote{Estáis firmemente convencidos acerca de que nuestro Señor es verdaderamente de la raza de David según la carne (cf. Rm 1, 3), Hijo de Dios según la voluntad y el poder de Dios (cf. Jn 1, 13), nacido verdaderamente de una virgen, \ldots{} Fue verdaderamente clavado por nosotros en su carne bajo Poncio Pilato \ldots{} padeció verdaderamente, como también resucitó verdaderamente} (Smyrn. 1-2).

\textbf{497} Los relatos evangélicos (cf. Mt 1, 18-25; Lc 1, 26-38) presentan la concepción virginal como una obra divina que sobrepasa toda comprensión y toda posibilidad humanas (cf. Lc 1, 34): \textquote{Lo concebido en ella viene del Espíritu Santo}, dice el ángel a José a propósito de María, su desposada (Mt 1, 20). La Iglesia ve en ello el cumplimiento de la promesa divina hecha por el profeta Isaías: \textquote{He aquí que la virgen concebirá y dará a luz un Hijo} (Is 7, 14 según la traducción griega de Mt 1, 23).

\textbf{498} A veces ha desconcertado el silencio del Evangelio de S. Marcos y de las cartas del Nuevo Testamento sobre la concepción virginal de María. También se ha podido plantear si no se trataría en este caso de leyendas o de construcciones teológicas sin pretensiones históricas. A lo cual hay que responder: La fe en la concepción virginal de Jesús ha encontrado viva oposición, burlas o incomprensión por parte de los no creyentes, judíos y paganos (cf. S. Justino, Dial 99, 7; Orígenes, Cels. 1, 32, 69; entre otros); no ha tenido su origen en la mitología pagana ni en una adaptación de las ideas de su tiempo. El sentido de este misterio no es accesible más que a la fe que lo ve en ese \textquote{nexo que reúne entre sí los misterios} (DS 3016), dentro del conjunto de los Misterios de Cristo, desde su Encarnación hasta su Pascua. S. Ignacio de Antioquía da ya testimonio de este vínculo: \textquote{El príncipe de este mundo ignoró la virginidad de María y su parto, así como la muerte del Señor: tres misterios resonantes que se realizaron en el silencio de Dios} (Eph. 19, 1;cf. 1 Co 2, 8).\textbf{\\ }

\textbf{María, la \textquote{siempre Virgen}}

\textbf{499} La profundización de la fe en la maternidad virginal ha llevado a la Iglesia a confesar la virginidad real y perpetua de María (cf. DS 427) incluso en el parto del Hijo de Dios hecho hombre (cf. DS 291; 294; 442; 503; 571; 1880). En efecto, el nacimiento de Cristo \textquote{lejos de disminuir consagró la integridad virginal} de su madre (LG 57). La liturgia de la Iglesia celebra a María como la \textquote{Aeiparthenos}, la \textquote{siempre-virgen} (cf. LG 52).

\textbf{500} A esto se objeta a veces que la Escritura menciona unos hermanos y hermanas de Jesús (cf. Mc 3, 31-55; 6, 3; 1 Co 9, 5; Ga 1, 19). La Iglesia siempre ha entendido estos pasajes como no referidos a otros hijos de la Virgen María; en efecto, Santiago y José \textquote{hermanos de Jesús} (Mt 13, 55) son los hijos de una María discípula de Cristo (cf. Mt 27, 56) que se designa de manera significativa como \textquote{la otra María} (Mt 28, 1). Se trata de parientes próximos de Jesús, según una expresión conocida del Antiguo Testamento (cf. Gn 13, 8; 14, 16;29, 15; etc.).

\textbf{501} Jesús es el Hijo único de María. Pero la maternidad espiritual de María se extiende (cf. Jn 19, 26-27; Ap 12, 17) a todos los hombres a los cuales, El vino a salvar: \textquote{Dio a luz al Hijo, al que Dios constituyó el mayor de muchos hermanos (Rom 8,29), es decir, de los creyentes, a cuyo nacimiento y educación colabora con amor de madre} (LG 63).

\textbf{La maternidad virginal de María en el designio de Dios}

\textbf{502} La mirada de la fe, unida al conjunto de la Revelación, puede descubrir las razones misteriosas por las que Dios, en su designio salvífico, quiso que su Hijo naciera de una virgen. Estas razones se refieren tanto a la persona y a la misión redentora de Cristo como a la aceptación por María de esta misión para con los hombres.

\textbf{503} La virginidad de María manifiesta la iniciativa absoluta de Dios en la Encarnación. Jesús no tiene como Padre más que a Dios (cf. Lc 2, 48-49). \textquote{La naturaleza humana que ha tomado no le ha alejado jamás de su Padre \ldots{}; consubstancial con su Padre en la divinidad, consubstancial con su Madre en nuestras humanidad, pero propiamente Hijo de Dios en sus dos naturalezas} (Cc. Friul en el año 796: DS 619).

\textbf{504} Jesús fue concebido por obra del Espíritu Santo en el seno de la Virgen María porque El es el \emph{Nuevo Adán} (cf. 1 Co 15, 45) que inaugura la nueva creación: \textquote{El primer hombre, salido de la tierra, es terreno; el segundo viene del cielo} (1 Co 15, 47). La humanidad de Cristo, desde su concepción, está llena del Espíritu Santo porque Dios \textquote{le da el Espíritu sin medida} (Jn 3, 34). De \textquote{su plenitud}, cabeza de la humanidad redimida (cf. Col 1, 18), \textquote{hemos recibido todos gracia por gracia} (Jn 1, 16).

\textbf{505} Jesús, el nuevo Adán, inaugura por su concepción virginal el \emph{nuevo nacimiento} de los hijos de adopción en el Espíritu Santo por la fe \textquote{¿Cómo será eso?} (Lc 1, 34;cf. Jn 3, 9). La participación en la vida divina no nace \textquote{de la sangre, ni de deseo de carne, ni de deseo de hombre, sino de Dios} (Jn 1, 13). La acogida de esta vida es virginal porque toda ella es dada al hombre por el Espíritu. El sentido esponsal de la vocación humana con relación a Dios (cf. 2 Co 11, 2) se lleva a cabo perfectamente en la maternidad virginal de María.

\textbf{506} María es virgen porque su virginidad es \emph{el signo de su fe} \textquote{no adulterada por duda alguna} (LG 63) y de su entrega total a la voluntad de Dios (cf. 1 Co 7, 34-35). Su fe es la que le hace llegar a ser la madre del Salvador: \textquote{Beatior est Maria percipiendo fidem Christi quam concipiendo carnem Christi} (\textquote{Más bienaventurada es María al recibir a Cristo por la fe que al concebir en su seno la carne de Cristo} (S. Agustín, virg. 3).

\textbf{507} María es a la vez virgen y madre porque ella es la figura y la más perfecta realización de la Iglesia (cf. LG 63): \textquote{La Iglesia se convierte en Madre por la palabra de Dios acogida con fe, ya que, por la predicación y el bautismo, engendra para una vida nueva e inmortal a los hijos concebidos por el Espíritu Santo y nacidos de Dios. También ella es virgen que guarda íntegra y pura la fidelidad prometida al Esposo} (LG 64).

\textbf{La maternidad divina de María}

\textbf{495} Llamada en los Evangelios \textquote{la Madre de Jesús} (Jn 2, 1; 19, 25; cf. Mt 13, 55, etc.), María es aclamada bajo el impulso del Espíritu como \textquote{la madre de mi Señor} desde antes del nacimiento de su hijo (cf. Lc 1, 43). En efecto, aquél que ella concibió como hombre, por obra del Espíritu Santo, y que se ha hecho verdaderamente su Hijo según la carne, no es otro que el Hijo eterno del Padre, la segunda persona de la Santísima Trinidad. La Iglesia confiesa que María es verdaderamente \emph{Madre de Dios} {[}\textquote{Theotokos}{]} (cf. DS 251).

		María, madre de Dios por obra del Espíritu Santo

CEC 437, 456, 484-486, 721-726:

\textbf{437} El ángel anunció a los pastores el nacimiento de Jesús como el del Mesías prometido a Israel: \textquote{Os ha nacido hoy, en la ciudad de David, un salvador, que es el Cristo Señor} (\emph{Lc} 2, 11). Desde el principio él es \textquote{a quien el Padre ha santificado y enviado al mundo} (\emph{Jn} 10, 36), concebido como \textquote{santo} (\emph{Lc} 1, 35) en el seno virginal de María. José fue llamado por Dios para \textquote{tomar consigo a María su esposa} encinta \textquote{del que fue engendrado en ella por el Espíritu Santo} (\emph{Mt} 1, 20) para que Jesús \textquote{llamado Cristo} nazca de la esposa de José en la descendencia mesiánica de David (\emph{Mt} 1, 16; cf. \emph{Rm} 1, 3; \emph{2 Tm} 2, 8; \emph{Ap} 22, 16).

\textbf{\\ }

\textbf{Por qué el Verbo se hizo carne}

\textbf{456} Con el Credo Niceno-Constantinopolitano respondemos confesando: \textquote{\emph{Por nosotros los hombres y por nuestra salvación} bajó del cielo, y por obra del Espíritu Santo se encarnó de María la Virgen y se hizo hombre} (DS 150).

\textbf{457} El Verbo se encarnó \emph{para salvarnos reconciliándonos con Dios}: \textquote{Dios nos amó y nos envió a su Hijo como propiciación por nuestros pecados} (\emph{1 Jn} 4, 10). \textquote{El Padre envió a su Hijo para ser salvador del mundo} (\emph{1 Jn} 4, 14). \textquote{Él se manifestó para quitar los pecados} (\emph{1 Jn} 3, 5):

\begin{quote} \textquote{Nuestra naturaleza enferma exigía ser sanada; desgarrada, ser restablecida; muerta, ser resucitada. Habíamos perdido la posesión del bien, era necesario que se nos devolviera. Encerrados en las tinieblas, hacía falta que nos llegara la luz; estando cautivos, esperábamos un salvador; prisioneros, un socorro; esclavos, un libertador. ¿No tenían importancia estos razonamientos? ¿No merecían conmover a Dios hasta el punto de hacerle bajar hasta nuestra naturaleza humana para visitarla, ya que la humanidad se encontraba en un estado tan miserable y tan desgraciado?} (San Gregorio de Nisa, \emph{Oratio catechetica}, 15: PG 45, 48B). \end{quote}

\textbf{Concebido por obra y gracia del Espíritu Santo \ldots{}}

\textbf{484} La Anunciación a María inaugura \textquote{la plenitud de los tiempos} (\emph{Ga} 4, 4), es decir, el cumplimiento de las promesas y de los preparativos. María es invitada a concebir a aquel en quien habitará \textquote{corporalmente la plenitud de la divinidad} (\emph{Col} 2, 9). La respuesta divina a su \textquote{¿cómo será esto, puesto que no conozco varón?} (\emph{Lc} 1, 34) se dio mediante el poder del Espíritu: \textquote{El Espíritu Santo vendrá sobre ti} (\emph{Lc} 1, 35).

\textbf{485} La misión del Espíritu Santo está siempre unida y ordenada a la del Hijo (cf. \emph{Jn} 16, 14-15). El Espíritu Santo fue enviado para santificar el seno de la Virgen María y fecundarla por obra divina, él que es \textquote{el Señor que da la vida}, haciendo que ella conciba al Hijo eterno del Padre en una humanidad tomada de la suya.

\textbf{486} El Hijo único del Padre, al ser concebido como hombre en el seno de la Virgen María es \textquote{Cristo}, es decir, el ungido por el Espíritu Santo (cf. \emph{Mt} 1, 20; \emph{Lc} 1, 35), desde el principio de su existencia humana, aunque su manifestación no tuviera lugar sino progresivamente: a los pastores (cf. \emph{Lc} 2,8-20), a los magos (cf. \emph{Mt} 2, 1-12), a Juan Bautista (cf. \emph{Jn} 1, 31-34), a los discípulos (cf. \emph{Jn} 2, 11). Por tanto, toda la vida de Jesucristo manifestará \textquote{cómo Dios le ungió con el Espíritu Santo y con poder} (\emph{Hch} 10, 38).

\textbf{\textquote{Alégrate, llena de gracia}}

\textbf{721} María, la Santísima Madre de Dios, la siempre Virgen, es la obra maestra de la Misión del Hijo y del Espíritu Santo en la Plenitud de los tiempos. Por primera vez en el designio de Salvación y porque su Espíritu la ha preparado, el Padre encuentra la Morada en donde su Hijo y su Espíritu pueden habitar entre los hombres. Por ello, los más bellos textos sobre la Sabiduría, la Tradición de la Iglesia los ha entendido frecuentemente con relación a María (cf. \emph{Pr} 8, 1-9, 6; \emph{Si} 24): María es cantada y representada en la Liturgia como el \textquote{Trono de la Sabiduría}.

En ella comienzan a manifestarse las \textquote{maravillas de Dios}, que el Espíritu va a realizar en Cristo y en la Iglesia:

\textbf{722} El Espíritu Santo \emph{preparó} a María con su gracia. Convenía que fuese \textquote{llena de gracia} la Madre de Aquel en quien \textquote{reside toda la plenitud de la divinidad corporalmente} (\emph{Col} 2, 9). Ella fue concebida sin pecado, por pura gracia, como la más humilde de todas las criaturas, la más capaz de acoger el don inefable del Omnipotente. Con justa razón, el ángel Gabriel la saluda como la \textquote{Hija de Sión}: \textquote{Alégrate} (cf. \emph{So} 3, 14; \emph{Za} 2, 14). Cuando ella lleva en sí al Hijo eterno, hace subir hasta el cielo con su cántico al Padre, en el Espíritu Santo, la acción de gracias de todo el pueblo de Dios y, por tanto, de la Iglesia (cf. \emph{Lc} 1, 46-55).

\textbf{723} En María el Espíritu Santo \emph{realiza} el designio benevolente del Padre. La Virgen concibe y da a luz al Hijo de Dios por obra del Espíritu Santo. Su virginidad se convierte en fecundidad única por medio del poder del Espíritu y de la fe (cf. \emph{Lc} 1, 26-38; \emph{Rm} 4, 18-21; \emph{Ga} 4, 26-28).

\textbf{724} En María, el Espíritu Santo \emph{manifiesta} al Hijo del Padre hecho Hijo de la Virgen. Ella es la zarza ardiente de la teofanía definitiva: llena del Espíritu Santo, presenta al Verbo en la humildad de su carne dándolo a conocer a los pobres (cf. \emph{Lc} 2, 15-19) y a las primicias de las naciones (cf. \emph{Mt} 2, 11).

\textbf{725} En fin, por medio de María, el Espíritu Santo comienza a \emph{poner en comunión} con Cristo a los hombres \textquote{objeto del amor benevolente de Dios} (cf. \emph{Lc} 2, 14), y los humildes son siempre los primeros en recibirle: los pastores, los magos, Simeón y Ana, los esposos de Caná y los primeros discípulos.

\textbf{726} Al término de esta misión del Espíritu, María se convierte en la \textquote{Mujer}, nueva Eva \textquote{madre de los vivientes}, Madre del \textquote{Cristo total} (cf. \emph{Jn} 19, 25-27). Así es como ella está presente con los Doce, que \textquote{perseveraban en la oración, con un mismo espíritu} (\emph{Hch} 1, 14), en el amanecer de los \textquote{últimos tiempos} que el Espíritu va a inaugurar en la mañana de Pentecostés con la manifestación de la Iglesia.	
		Jesús viene revelado como Salvador a José

CEC 1846:

\textbf{1846} El Evangelio es la revelación, en Jesucristo, de la misericordia de Dios con los pecadores (cf. \emph{Lc} 15). El ángel anuncia a José: \textquote{Tú le pondrás por nombre Jesús, porque él salvará a su pueblo de sus pecados} (\emph{Mt} 1, 21). Y en la institución de la Eucaristía, sacramento de la redención, Jesús dice: \textquote{Esta es mi sangre de la alianza, que va a ser derramada por muchos para remisión de los pecados} (\emph{Mt} 26, 28).

		Cristo, el Hijo de Dios en su Resurrección

CEC 445, 648, 695:

\n{445} Después de su Resurrección, su filiación divina aparece en el poder de su humanidad glorificada: \textquote{Constituido Hijo de Dios con poder, según el Espíritu de santidad, por su Resurrección de entre los muertos} (\emph{Rm} 1, 4; cf. \emph{Hch} 13, 33). Los apóstoles podrán confesar \textquote{Hemos visto su gloria, gloria que recibe del Padre como Hijo único, lleno de gracia y de verdad } (\emph{Jn} 1, 14).

\ccesec{La Resurrección obra de la Santísima Trinidad}

\n{648} La Resurrección de Cristo es objeto de fe en cuanto es una intervención transcendente de Dios mismo en la creación y en la historia. En ella, las tres Personas divinas actúan juntas a la vez y manifiestan su propia originalidad. Se realiza por el poder del Padre que \textquote{ha resucitado} (\emph{Hch} 2, 24) a Cristo, su Hijo, y de este modo ha introducido de manera perfecta su humanidad ---con su cuerpo--- en la Trinidad. Jesús se revela definitivamente \textquote{Hijo de Dios con poder, según el Espíritu de santidad, por su resurrección de entre los muertos} (\emph{Rm} 1, 3-4). San Pablo insiste en la manifestación del poder de Dios (cf. \emph{Rm} 6, 4; 2 Co 13, 4; \emph{Flp} 3, 10; \emph{Ef} 1, 19-22; \emph{Hb} 7, 16) por la acción del Espíritu que ha vivificado la humanidad muerta de Jesús y la ha llamado al estado glorioso de Señor.

\ccesec{Los símbolos del Espíritu Santo}

\n{695} \emph{La unción}. El simbolismo de la unción con el óleo es también significativo del Espíritu Santo, hasta el punto de que se ha convertido en sinónimo suyo (cf. \emph{1 Jn} 2, 20. 27; \emph{2 Co} 1, 21). En la iniciación cristiana es el signo sacramental de la Confirmación, llamada justamente en las Iglesias de Oriente \textquote{Crismación}. Pero para captar toda la fuerza que tiene, es necesario volver a la Unción primera realizada por el Espíritu Santo: la de Jesús. Cristo {[}\textquote{Mesías} en hebreo{]} significa \textquote{Ungido} del Espíritu de Dios. En la Antigua Alianza hubo \textquote{ungidos} del Señor (cf. \emph{Ex} 30, 22-32), de forma eminente el rey David (cf. \emph{1 S} 16, 13). Pero Jesús es el Ungido de Dios de una manera única: la humanidad que el Hijo asume está totalmente \textquote{ungida por el Espíritu Santo}. Jesús es constituido \textquote{Cristo} por el Espíritu Santo (cf. \emph{Lc} 4, 18-19; \emph{Is} 61, 1).

La Virgen María concibe a Cristo del Espíritu Santo, quien por medio del ángel lo anuncia como Cristo en su nacimiento (cf. \emph{Lc} 2,11) e impulsa a Simeón a ir al Templo a ver al Cristo del Señor (cf. \emph{Lc} 2, 26-27); es de quien Cristo está lleno (cf. \emph{Lc} 4, 1) y cuyo poder emana de Cristo en sus curaciones y en sus acciones salvíficas (cf. \emph{Lc} 6, 19; 8, 46). Es él en fin quien resucita a Jesús de entre los muertos (cf. \emph{Rm} 1, 4; 8, 11). Por tanto, constituido plenamente \textquote{Cristo} en su humanidad victoriosa de la muerte (cf. \emph{Hch} 2, 36), Jesús distribuye profusamente el Espíritu Santo hasta que \textquote{los santos} constituyan, en su unión con la humanidad del Hijo de Dios, \textquote{ese Hombre perfecto [\ldots{}] que realiza la plenitud de Cristo} (\emph{Ef} 4, 13): \textquote{el Cristo total} según la expresión de San Agustín (\emph{Sermo} 341, 1, 1: PL 39, 1493; Ibíd., 9, 11: PL 39, 1499).

		\textquote{La obediencia de la fe}

CEC 143-149, 494, 2087:

\n{143} \emph{Por la fe}, el hombre somete completamente su inteligencia y su voluntad a Dios. Con todo su ser, el hombre da su asentimiento a Dios que revela (cf. DV 5). La sagrada Escritura llama \textquote{obediencia de la fe} a esta respuesta del hombre a Dios que revela (cf. \emph{Rm} 1,5; 16,26).

\ccesec{La obediencia de la fe. Abraham, \textquote{padre de todos los creyentes}}

\n{144} Obedecer (\emph{ob-audire}) en la fe es someterse libremente a la palabra escuchada, porque su verdad está garantizada por Dios, la Verdad misma. De esta obediencia, Abraham es el modelo que nos propone la Sagrada Escritura. La Virgen María es la realización más perfecta de la misma.

\n{145} La carta a los Hebreos, en el gran elogio de la fe de los antepasados, insiste particularmente en la fe de Abraham: \textquote{Por la fe, Abraham obedeció y salió para el lugar que había de recibir en herencia, y salió sin saber a dónde iba} (\emph{Hb} 11,8; cf. \emph{Gn} 12,1-4). Por la fe, vivió como extranjero y peregrino en la Tierra prometida (cf. \emph{Gn} 23,4). Por la fe, a Sara se le otorgó el concebir al hijo de la promesa. Por la fe, finalmente, Abraham ofreció a su hijo único en sacrificio (cf. \emph{Hb} 11,17).

\n{146} Abraham realiza así la definición de la fe dada por la carta a los Hebreos: \textquote{La fe es garantía de lo que se espera; la prueba de las realidades que no se ven} (\emph{Hb} 11,1). \textquote{Creyó Abraham en Dios y le fue reputado como justicia} (\emph{Rm} 4,3; cf. \emph{Gn} 15,6). Y por eso, fortalecido por su fe, Abraham fue hecho \textquote{padre de todos los creyentes} (\emph{Rm} 4,11.18; cf. \emph{Gn} 15, 5).

\n{147} El Antiguo Testamento es rico en testimonios acerca de esta fe. La carta a los Hebreos proclama el elogio de la fe ejemplar por la que los antiguos \textquote{fueron alabados} (\emph{Hb} 11, 2.39). Sin embargo, \textquote{Dios tenía ya dispuesto algo mejor}: la gracia de creer en su Hijo Jesús, \textquote{el que inicia y consuma la fe} (\emph{Hb} 11,40; 12,2).

\ccesec{María: \textquote{Dichosa la que ha creído}}

\n{148} La Virgen María realiza de la manera más perfecta la obediencia de la fe. En la fe, María acogió el anuncio y la promesa que le traía el ángel Gabriel, creyendo que \textquote{nada es imposible para Dios} (\emph{Lc} 1,37; cf. \emph{Gn} 18,14) y dando su asentimiento: \textquote{He aquí la esclava del Señor; hágase en mí según tu palabra} (\emph{Lc} 1,38). Isabel la saludó: \textquote{¡Dichosa la que ha creído que se cumplirían las cosas que le fueron dichas de parte del Señor!} (\emph{Lc} 1,45). Por esta fe todas las generaciones la proclamarán bienaventurada (cf. \emph{Lc} 1,48).

\n{149} Durante toda su vida, y hasta su última prueba (cf. \emph{Lc} 2,35), cuando Jesús, su hijo, murió en la cruz, su fe no vaciló. María no cesó de creer en el \textquote{cumplimiento} de la palabra de Dios. Por todo ello, la Iglesia venera en María la realización más pura de la fe.

\ccesec{\textquote{Hágase en mí según tu palabra \ldots{}}}

\n{494} Al anuncio de que ella dará a luz al \textquote{Hijo del Altísimo} sin conocer varón, por la virtud del Espíritu Santo (cf. Lc 1, 28-37), María respondió por \textquote{la obediencia de la fe} (Rm 1, 5), segura de que \textquote{nada hay imposible para Dios}: \textquote{He aquí la esclava del Señor: hágase en mí según tu palabra} (Lc 1, 37-38). Así dando su consentimiento a la palabra de Dios, María llegó a ser Madre de Jesús y, aceptando de todo corazón la voluntad divina de salvación, sin que ningún pecado se lo impidiera, se entregó a sí misma por entero a la persona y a la obra de su Hijo, para servir, en su dependencia y con él, por la gracia de Dios, al Misterio de la Redención (cf. LG 56):

\begin{quote} Ella, en efecto, como dice S. Ireneo, \textquote{por su obediencia fue causa de la salvación propia y de la de todo el género humano}. Por eso, no pocos Padres antiguos, en su predicación, coincidieron con él en afirmar \textquote{el nudo de la desobediencia de Eva lo desató la obediencia de María. Lo que ató la virgen Eva por su falta de fe lo desató la Virgen María por su fe}. Comparándola con Eva, llaman a María `Madre de los vivientes' y afirman con mayor frecuencia: \textquote{la muerte vino por Eva, la vida por María}. (LG. 56). \end{quote}

\ccesec{La fe}

\n{2087} Nuestra vida moral tiene su fuente en la fe en Dios que nos revela su amor. San Pablo habla de la \textquote{obediencia de la fe} (\emph{Rm} 1, 5; 16, 26) como de la primera obligación. Hace ver en el \textquote{desconocimiento de Dios} el principio y la explicación de todas las desviaciones morales (cf. \emph{Rm} 1, 18-32). Nuestro deber para con Dios es creer en Él y dar testimonio de Él.

