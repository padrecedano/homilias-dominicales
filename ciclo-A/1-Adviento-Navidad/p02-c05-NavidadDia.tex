\chapter{Misa del Día}

	\section{Lecturas}

		\rtitle{PRIMERA LECTURA}

		\rbook{Del libro del profeta Isaías} \rred{52, 7-10}
		
		\rtheme{Verán los confines de la tierra la salvación de nuestro Dios}
		
			\begin{scripture}
				\begin{readprose}
					¡Qué hermosos son sobre los montes
					
					los pies del mensajero que proclama la paz,
					
					que anuncia la buena noticia,
					
					que pregona la justicia,
					
					que dice a Sión: \textquote{¡Tu Dios reina!}.
					
					Escucha: tus vigías gritan, cantan a coro,
					
					porque ven cara a cara al Señor,
					
					que vuelve a Sión.
					
					Romped a cantar a coro,
					
					ruinas de Jerusalén,
					
					porque el Señor ha consolado a su pueblo,
					
					ha rescatado a Jerusalén.
					
					Ha descubierto el Señor su santo brazo
					
					a los ojos de todas las naciones,
					
					y verán los confines de la tierra
					
					la salvación de nuestro Dios.
				\end{readprose}
			\end{scripture}
	
		\rtitle{SALMO RESPONSORIAL}
		
		\rbook{Salmo} \rred{97, 1bcde. 2-3ab. 3cd-4. 5-6}
		
		\rtheme{Los confines de la tierra han contemplado la salvación de nuestro Dios}
		
			\begin{psbody}
				Cantad al Señor un cántico nuevo,
				porque ha hecho maravillas:
				su diestra le ha dado la victoria,
				su santo brazo.
				
				El Señor da a conocer su salvación,
				revela a las naciones su justicia.
				Se acordó de su misericordia y su fidelidad
				en favor de la casa de Israel.
				
				Los confines de la tierra han contemplado
				la salvación de nuestro Dios.
				Aclamad al Señor, tierra entera;
				gritad, vitoread, tocad.
				
				Tañed la cítara para el Señor,
				suenen los instrumentos:
				con clarines y al son de trompetas,
				aclamad al Rey y Señor.
			\end{psbody}
	
		\rtitle{SEGUNDA LECTURA}
		
		\rbook{De la carta a los Hebreos} \rred{1, 1-6}
		
		\rtheme{Dios nos ha hablado por el Hijo}
		
			\begin{scripture}
				En muchas ocasiones y de muchas maneras habló Dios antiguamente a los padres por los profetas.
				
				En esta etapa final, nos ha hablado por el Hijo, al que ha nombrado heredero de todo, y por medio del cual ha realizado los siglos.
				
				Él es reflejo de su gloria, impronta de su ser. Él sostiene el universo con su palabra poderosa. Y, habiendo realizado la purificación de los pecados, está sentado a la derecha de la Majestad en las alturas; tanto más encumbrado sobre los ángeles cuanto más sublime es el nombre que ha heredado.
				
				Pues ¿a qué ángel dijo jamás: 
				
				Hijo mío eres tú, yo te he engendrado hoy;
				
				y en otro lugar:
				
				\begin{readprose}
					Yo seré para él un padre,
					
					y él será para mí un hijo?
				\end{readprose}
			
			Asimismo, cuando introduce en el mundo al primogénito, dice:
			
			Adórenlo todos los ángeles de Dios.
		\end{scripture}
	
		\rtitle{EVANGELIO (forma larga)}
		
		\rbook{Del Evangelio según san Juan} \rred{1, 1-18}
		
		\rtheme{El Verbo se hizo carne y habitó entre nosotros}
		
		\begin{scripture}
			En el principio existía el Verbo, y el Verbo estaba junto a Dios, y el Verbo era Dios.
			
			Él estaba en el principio junto a Dios.
			
			Por medio de él se hizo todo, y sin él no se hizo nada de cuanto se ha hecho.
			
			En él estaba la vida, y la vida era la luz de los hombres.
			
			Y la luz brilla en la tiniebla, y la tiniebla no lo recibió.
			
			Surgió un hombre enviado por Dios, que se llamaba Juan: este venía como testigo, para dar testimonio de la luz, para que todos creyeran por medio de él.
			
			No era él la luz, sino el que daba testimonio de la luz.
			
			El Verbo era la luz verdadera, que alumbra a todo hombre, viniendo al mundo.
			
			En el mundo estaba; el mundo se hizo por medio de él, y el mundo no lo conoció.
			
			Vino a su casa, y los suyos no lo recibieron.
			
			Pero a cuantos lo recibieron, les dio poder de ser hijos de Dios, a los que creen en su nombre.
			
			Estos no han nacido de sangre, ni de deseo de carne, ni de deseo de varón, sino que han nacido de Dios.
			
			Y el Verbo se hizo carne y habitó entre nosotros, y hemos contemplado su gloria: gloria como del Unigénito del Padre, lleno de gracia y de verdad.
			
			Juan da testimonio de él y grita diciendo: «Este es de quien dije: El que viene detrás de mí se ha puesto delante de mí, porque existía antes que yo».
			
			Pues de su plenitud todos hemos recibido, gracia tras gracia.
			
			Porque la ley se dio por medio de Moisés, la gracia y la verdad nos han llegado por medio de Jesucristo.
			
			A Dios nadie lo ha visto jamás: Dios unigénito, que está en el seno del Padre, es quien lo ha dado a conocer.
		\end{scripture}
	
		
		\rtitle{EVANGELIO (forma breve)}
		
		\rbook{Del Evangelio según san Juan} \rred{1, 1-5. 9-14}
		
		\rtheme{El Verbo se hizo carne y habitó entre nosotros}
		
		\begin{scripture}
			En el principio existía el Verbo, y el Verbo estaba junto a Dios, y el Verbo era Dios.
			
			Él estaba en el principio junto a Dios.
			
			Por medio de él se hizo todo, y sin él no se hizo nada de cuanto se ha hecho.
			
			En él estaba la vida, y la vida era la luz de los hombres.
			
			Y la luz brilla en la tiniebla, y la tiniebla no lo recibió.
			
			El Verbo era la luz verdadera, que alumbra a todo hombre, viniendo al mundo.
			
			En el mundo estaba; el mundo se hizo por medio de él, y el mundo no lo conoció.
			
			Vino a su casa, y los suyos no lo recibieron.
			
			Pero a cuantos lo recibieron, les dio poder de ser hijos de Dios, a los que creen en su nombre.
			
			Estos no han nacido de sangre, ni de deseo de carne, ni de deseo de varón, sino que han nacido de Dios.
			
			Y el Verbo se hizo carne y habitó entre nosotros, y hemos contemplado su gloria: gloria como del Unigénito del Padre, lleno de gracia y de verdad.
		\end{scripture}

	\section{Comentario Patrístico}

		\subsection{San Basilio Magno, obispo}

			\ptheme{El Verbo se hizo carne y puso su morada entre nosotros}

			\src{Homilía 2, 6: \\PG 31, 1459-1462. 1471-1474.}
			
			\begin{body}
				Dios en la tierra, Dios en medio de los hombres, no un Dios que consigna la ley entre resplandores de fuego y ruido de trompetas sobre un monte humeante, o en densa nube entre relámpagos y truenos, sembrando el terror entre quienes escuchan; sino un Dios encarnado, que habla a las creaturas de su misma naturaleza con suavidad y dulzura. Un Dios encarnado, que no obra desde lejos o por medio de profetas, sino a través de la humanidad asumida para revestir su persona, para reconducir a sí, en nuestra misma carne hecha suya, a toda la humanidad. ¿Cómo, por medio de uno solo, el resplandor alcanza a todos? ¿Cómo la divinidad reside en la carne? Como el fuego en el hierro: no por transformación, sino por participación. El fuego, efectivamente, no pasa al hierro: permaneciendo donde está, le comunica su virtud; ni por esta comunicación disminuye, sino que invade con lo suyo a quien se comunica. Así el Dios-Verbo, sin jamás separarse de sí mismo \emph{puso su morada en medio de nosotros;} sin sufrir cambio alguno \emph{se hizo carne;} el cielo que lo contenía no quedó privado de él mientras la tierra lo acogió en su seno.
				
				Busca penetrar en el misterio: Dios asume la carne justamente para destruir la muerte oculta en ella. Como los antídotos de un veneno, una vez ingeridos, anulan sus efectos, y como las tinieblas de una casa se disuelven a la luz del sol, la muerte que dominaba sobre la naturaleza humana fue destruida por la presencia de Dios. Y como el hielo permanece sólido en el agua mientras dura la noche y reinan las tinieblas, pero prontamente se diluye al calor del sol, así la muerte reinante hasta la venida de Cristo, apenas resplandeció la gracia de Dios Salvador y surgió el sol de justicia, \emph{fue engullida por la victoria} (1Co 15, 54), no pudiendo coexistir con la Vida. ¡Oh grandeza de la bondad y del amor de Dios por los hombres!
				
				Démosle gloria con los pastores, exultemos con los ángeles \emph{porque hoy ha nacido el Salvador, Cristo el Señor} (Lc 2, 11). Tampoco a nosotros se apareció el Señor en forma de Dios, porque habría asustado a nuestra fragilidad, sino en forma de siervo, para restituir a la libertad a los que estaban en la esclavitud. ¿Quién es tan tibio, tan poco reconocido que no goce, no exulte, no lleve dones? Hoy es fiesta para toda creatura. No haya nadie que no ofrezca algo, nadie se muestre ingrato. Estallemos también nosotros en un canto de exultación.
			\end{body}


	\section{Homilías}
		\homiliasNavidad

		\subsection{San León Magno, papa}
		
			\subsubsection{Sermón: Nace el Señor, nace la paz}
			
				\src{Sermón 6, 2-3 en la Natividad del Señor: \\PL 54, 213-216.}
				
				\begin{body}
					Aunque aquella infancia, que la majestad del Hijo de Dios se dignó hacer suya, tuvo como continuación la plenitud de una edad adulta, y, después del triunfo de su pasión y resurrección, todas las acciones de su estado de humildad, que el Señor asumió por nosotros, pertenecen ya al pasado, la festividad de hoy renueva ante nosotros los sagrados comienzos de Jesús, nacido de la Virgen María; de modo que, mientras adoramos el nacimiento de nuestro Salvador, resulta que estamos celebrando nuestro propio comienzo. Efectivamente, la generación de Cristo es el comienzo del pueblo cristiano, y el nacimiento de la cabeza lo es al mismo tiempo del cuerpo.
					
					Aunque cada uno de los que llama el Señor a formar parte de su pueblo sea llamado en un tiempo determinado y aunque todos los hijos de la Iglesia hayan sido llamados cada uno en días distintos, con todo, la totalidad de los fieles, nacida en la fuente bautismal, ha nacido con Cristo en su nacimiento, del mismo modo que ha sido crucificada con Cristo en su pasión, ha sido resucitada en su resurrección y ha sido colocada a la derecha del Padre en su ascensión.
					
					Cualquier hombre que cree --en cualquier parte del mundo--, y se regenera en Cristo, una vez interrumpido el camino de su vieja condición original, pasa a ser un nuevo hombre al renacer; y ya no pertenece a la ascendencia de su padre carnal, sino a la simiente del Salvador, que se hizo precisamente Hijo del hombre, para que nosotros pudiésemos llegar a ser hijos de Dios.
					
					Pues si él no hubiera descendido hasta nosotros revestido de esta humilde condición, nadie hubiera logrado llegar hasta él por sus propios méritos. Por eso, la misma magnitud del beneficio otorgado exige de nosotros una veneración proporcionada a la excelsitud de esta dádiva. Y, como el bienaventurado Apóstol nos enseña, \emph{no hemos recibido el espíritu de este mundo, sino el Espíritu que procede de Dios}, a fin de que conozcamos lo que Dios nos ha otorgado; y el mismo Dios sólo acepta como culto piadoso el ofrecimiento de lo que él nos ha concedido.
					
					¿Y qué podremos encontrar en el tesoro de la divina largueza tan adecuado al honor de la presente festividad como la paz, lo primero que los ángeles pregonaron en el nacimiento del Señor?
					
					La paz es la que engendra los hijos de Dios, alimenta el amor y origina la unidad, es el descanso de los bienaventurados y la mansión de la eternidad. El fin propio de la paz y su fruto específico consiste en que se unan a Dios los que el mismo Señor separa del mundo.
					
					Que los que \emph{no han nacido de sangre, ni de amor carnal, ni de amor humano, sino de Dios}, ofrezcan, por tanto, al Padre la concordia que es propia de hijos pacíficos, y que todos los miembros de la adopción converjan hacia el Primogénito de la nueva creación, que vino a cumplir la voluntad del que le enviaba y no la suya: puesto que la gracia del Padre no adoptó como herederos a quienes se hallaban en discordia e incompatibilidad, sino a quienes amaban y sentían lo mismo. Los que han sido reformados de acuerdo con una sola imagen deben ser concordes en el espíritu.
					
					El nacimiento del Señor es el nacimiento de la paz: y así dice el Apóstol: \emph{El es nuestra paz; él ha hecho de los dos pueblos una sola cosa,} ya que, tanto los judíos como los gentiles, por su medio \emph{podemos acercarnos al Padre con un mismo Espíritu}.
				\end{body}
			
\newsection
			
		\subsection{San Ambrosio, obispo}
			
			\subsubsection{Comentario: Nació el que es Siervo y Señor a la vez}
				
				\src{Comentario 4-5 sobre el salmo 35: \\CCL 64, 52-53.}
					
				\begin{body}
					Creo que sobre la pobreza y sufrimientos del Señor hemos aducido testimonios muy válidos de dos santos, de los cuales uno vio y testimonió, mientras que el otro fue elegido tan sólo para testimoniar. Escuchemos todavía nuevos testimonios sobre la condición servil del Señor tomados de estos testigos fiables, o mejor, escuchemos lo que de sí mismo dice el mismo Señor por boca de ambos. Veamos lo que dice: \emph{Habla el Señor, que desde el vientre me formó siervo suyo, para que le trajese a Jacob, para que le reuniese a Israel}. Advirtamos que asumió la condición de siervo para reunir al pueblo.
					
					Estaba yo ---dice--- en las entrañas maternas, y el Señor pronunció mi nombre. Escuchemos cuál es el nombre que el Padre le da: \emph{Mirad: la Virgen concebirá y dará a luz un hijo y le pondrá por nombre Emmanuel, que significa \textquote{Dios-con-nosotros}}. ¿Cuál si no es el nombre de Cristo sino el de \textquote{Hijo de Dios}? Escucha un nuevo texto. Hablando de María a José, también Gabriel había dicho: \emph{Dará a luz un hijo, y tú le pondrás por nombre Jesús}. Escucha ahora la voz de Dios: \emph{Y tú, Belén, tierra de Judá, no eres ni mucho menos la última de las ciudades de Judá: pues de ti saldrá un jefe que será el pastor de mi pueblo}.
					
					Advierte el misterio: del seno de la Virgen nació el que es Siervo y Señor a la vez ---siervo para trabajar, señor para mandar---, a fin de implantar el reinado de Dios en el corazón del hombre. Ambos son uno: no uno del Padre y otro de la Virgen, sino que el mismo que antes de los siglos fue engendrado por el Padre se encarnará más tarde en el seno de la Virgen. Por eso se le llama Siervo y Señor: siervo por nosotros, mas, por la unidad de la naturaleza divina, Dios de Dios, príncipe de príncipe, igual de igual; pues no pudo el Padre engendrar un ser inferior a él y afirmar al mismo tiempo que en el Hijo tiene sus complacencias.
					
					\emph{Gran cosa es para ti} ---dice--- \emph{que seas mi siervo y restablezcas las tribus de Jacob}. Emplea siempre términos adecuados a su dignidad: Gran Dios y gran siervo, pues al encarnarse no perdió los atributos de su grandeza, ya que su grandeza no tiene fin. Así pues, es igual en cuanto Hijo de Dios, asumió la condición de siervo al encarnarse, sufrió la muerte aquel cuya grandeza no tiene fin, porque \emph{el fin de la ley es Cristo, y con eso se justifica a todo el que cree,} para que todos creamos en él y le adoremos con profundo afecto. Bendita servidumbre que a todos nos otorgó la libertad, bendita servidumbre que le valió el \textquote{nombre-sobre-todo-nombre}, bendita humildad que hizo que \emph{al nombre de Jesús toda rodilla se doble en el cielo, en la tierra, en el abismo, y toda lengua proclame: Jesucristo es Señor para gloria de Dios Padre}.
				\end{body}
				
		\subsection{San Agustín, obispo}
		
			\subsubsection{Sermón: Saciados con la visión de la Palabra}
					
				\src{Sermón 194, 3-4: PL 38, 1016-1017.}
						
				\begin{body}
					¿Qué ser humano podría conocer todos los tesoros de sabiduría y de ciencia ocultos en Cristo y escondidos en la pobreza de su carne? Porque, \emph{siendo rico, se hizo pobre por vosotros, para enriqueceros con su pobreza}. Pues cuando asumió la condición mortal y experimentó la muerte, se mostró pobre: pero prometió riquezas para más adelante, y no perdió las que le habían quitado.
					
					¡Qué inmensidad la de su dulzura, que escondió para los que lo temen, y llevó a cabo para los que esperan en él!
					
					Nuestros conocimientos son ahora parciales, hasta que se cumpla lo que es perfecto. Y para que nos hagamos capaces de alcanzarlo, él, que era igual al Padre en la forma de Dios, se hizo semejante a nosotros en la forma de siervo, para reformarnos a semejanza de Dios: y, convertido en hijo del hombre --él, que era único Hijo de Dios---, convirtió a muchos hijos de los hombres en hijos de Dios; y, habiendo alimentado a aquellos siervos con su forma visible de siervo, los hizo libres para que contemplasen la forma de Dios.
					
					Pues \emph{ahora somos hijos de Dios y aún no se ha manifestado lo que seremos. Sabemos que, cuando se manifieste, seremos semejantes a él, porque lo veremos tal cual es}. Pues ¿para qué son aquellos tesoros de sabiduría y de ciencia, para qué sirven aquellas riquezas divinas sino para colmarnos? ¿Y para qué la inmensidad de aquella dulzura sino para saciarnos? \emph{Muéstranos al Padre y nos basta}.
					
					Y en algún salmo, uno de nosotros, o en nosotros, o por nosotros, le dice: \emph{Me saciaré cuando se manifieste tu gloria}. Pues él y el Padre son una misma cosa: y quien lo ve a él ve también al Padre. De modo que \emph{el Señor, Dios de los ejércitos, él es el Rey de la gloria}. Volviendo a nosotros, nos mostrará su rostro; y nos salvaremos y quedaremos saciados, y eso nos bastará.
					
					Pero mientras eso no suceda, mientras no nos muestre lo que habrá de bastarnos, mientras no le bebamos como fuente de vida y nos saciemos, mientras tengamos que andar en la fe y peregrinemos lejos de él, mientras tenemos hambre y sed de justicia y anhelamos con inefable ardor la belleza de la forma de Dios, celebremos con devota obsequiosidad el nacimiento de la forma de siervo.
					
					Si no podemos contemplar todavía al que fue engendrado por el Padre antes que el lucero de la mañana, tratemos de acercarnos al que nació de la Virgen en medio de la noche. No comprendemos aún que su \emph{nombre dura como el sol;} reconozcamos que su \emph{tienda} ha sido puesta \emph{en el sol}.
					
					Todavía no podemos contemplar al Único que permanece en su Padre; recordemos al \emph{Esposo que sale de su alcoba}. Todavía no estamos preparados para el banquete de nuestro Padre; reconozcamos al menos el pesebre de nuestro Señor Jesucristo.
				\end{body}
				
\newsection 
					
		\subsection{San Fulberto de Chartres, obispo}
				
			\subsubsection{Carta: El misterio de nuestra salvación}
					
				\src{Carta 5: PL 141, 198-199.}
						
				\begin{body}
					No nos resulta difícil sopesar la diversidad de naturalezas en Cristo. En efecto: uno es el nacimiento o la naturaleza en que, en frase de san Pablo, \emph{nació de una mujer, nació bajo la ley;} otra por la que en el principio estaba junto a Dios; una es la naturaleza por la que, engendrado de la virgen María, vivió humilde en la tierra, y otra por la que, eterno y sin principio, creó el cielo y la tierra; una es la naturaleza en la que se afirma que fue presa de la tristeza, que el cansancio le rindió, que padeció hambre, que lloró, y otra en virtud de la cual curó paralíticos, hizo caminar a los tullidos, dio la vista al ciego de nacimiento, calmó con su imperio las turgentes olas, resucitó muertos.
							
					Siendo así las cosas, es necesario que quien desee llevar el nombre de cristiano con coherencia y sin perjuicio personal, confiese que Cristo, en quien reconocemos dos naturalezas, es a la vez verdadero Dios y hombre verdadero. Así, una vez asegurada la verdad de las dos naturalezas, la fe verdadera no confunda ni divida a Cristo, verdadero en los dolores de su humanidad y verdadero en los poderes de su divinidad. Pues en él la unidad de persona no tolera división y la realidad de la doble naturaleza no admite confusión. En él no subsisten separados Dios y hombre, sino que Cristo es al mismo tiempo Dios y hombre. Efectivamente, Cristo es el mismo Dios que con su divinidad destruyó la muerte; el mismo Hijo de Dios que no podía morir en su divinidad, murió en la carne mortal que el Dios inmortal había asumido; y este mismo Cristo Hijo de Dios, muerto en la carne, resucitó, pues muriendo en la carne, no perdió la inmortalidad de su divinidad.
							
					Sabemos con plena certeza que, siendo pecadores por el primer nacimiento, el segundo nos ha purificado; siendo cautivos por el primer nacimiento, el segundo nos ha liberado; siendo terrenos por el primer nacimiento, el segundo nos hace celestes; siendo carnales por el vicio del primer nacimiento, el beneficio del segundo nacimiento nos hace espirituales; por el primer nacimiento somos hijos de ira, por el segundo nacimiento somos hijos de gracia. Por tanto, todo el que atenta contra la santidad del bautismo, sepa que está ofendiendo al mismo Dios, que dijo: \emph{El que no nazca de agua y Espíritu no puede entrar en el reino de Dios}. Constituye, por tanto, una gracia de la doctrina de la salvación, conocer la profundidad del misterio del bautismo, del que el Apóstol afirma: Si \emph{hemos muerto con Cristo, creemos que también viviremos con él}. Conmorir y ser sepultados con Cristo tiene como meta poder resucitar con él, poder vivir con él.
				\end{body}
			
			
\newsection

					
		\subsection{San Juan Pablo II, papa}
				
			\subsubsection{Urbi et Orbi (1986):}
					
				\src{Desde Asís.}
						
				\begin{body}
					1. \textquote{\emph{Cuán hermosos son sobre los montes los pies del mensajero de la buena nueva que anuncia la paz \ldots{} que dice a Sión: Tu Dios reina}} (Is 52, 7).
							
					Sí. Sión, tu Dios reina. Tu Dios admirable aquí está acostado en el pesebre de los animales. ¡Y así comienza a reinar tu Dios, oh Sion!
							
					Tu Dios incomprensible: \textquote{Sus pensamientos no son nuestros pensamientos, y nuestros caminos no son sus caminos} (\emph{Is} 55, 8).
							
					Entonces comenzó a reinar bajo el signo de la extrema pobreza: \textquote{Se hizo pobre por nosotros, para que nosotros nos hiciéramos ricos con su pobreza} (cf.\emph{2 Co} 8, 9).
					
					2. ¡Oh, qué hermosos son los pies de ese mensajero de la buena nueva que se llama Francisco il Poverello de Asís, de Greccio y de La Verna, Francisco, amante de todas las criaturas; Francisco conquistado por el amor del divino Niño, nacido en la noche de Belén; Francisco en cuyo corazón Cristo comenzó a reinar, para que incluso a través de la pobreza del discípulo comprendamos mejor la pobreza del Maestro y seamos llevados a pensamientos de amor y paz.
					
					Gloria a Dios en las alturas y paz en la tierra a los hombres de buena voluntad; paz a los hombres que Él ama (cf. \emph{Lc} 2, 14). Gloria a Dios \ldots{}
					
					3. {[}Escuchemos{]} el mensaje de Francisco, amante del Creador y de toda criatura; de Francisco, heraldo de la Gloria de ese Dios, que \textquote{en las alturas de los cielos} es el Amor; de Francisco promotor de la paz en la tierra. {[}Reflexionemos{]} ante el \textquote{último e inefable misterio que envuelve nuestra existencia, del que tomamos nuestro origen y hacia el que nos dirigimos} ({\emph{Nostra Aetate}}, 1), {[}oremos{]} por la paz en la tierra.
					
					4. {[} \ldots{}{]} Hemos decidido ser pobres --- frente a todos los poderes de esta tierra que devoran incalculables riquezas en armamento, disipan recursos preciosos en bienes superfluos --- hemos decidido ser pobres como Cristo, Hijo de Dios y Salvador del mundo; pobres como Francisco, imagen elocuente de Cristo; pobres como tantas grandes almas que han iluminado el camino de la humanidad.
					
					Lo hemos decidido teniendo a nuestra disposición sólo este medio, el medio de la pobreza, y sólo este poder, el poder de la debilidad: sólo la oración y sólo el ayuno.
					
					5. ¿No es necesario que el mundo escuche esta voz? ¿No es necesario que escuche el testimonio de la noche de Belén? ¿Que escuche a Dios nacido en la pobreza? ¿Que escuche a Francisco, heraldo de las ocho bienaventuranzas? ¿No es necesario que el estruendo del odio y el estruendo de las detonaciones mortíferas en muchos lugares de la tierra se callen? ¿No es necesario que Dios finalmente pueda escuchar la voz de nuestro silencio? ¿A través del silencio le llegará la oración y el grito de todos los hombres de bien? ¿El grito de tantos corazones atormentados, la voz de tantos millones de hombres que no tienen voz?
					
					6. Escuchad y comprended todos: Dios que todo lo abraza, Dios en quien \textquote{vivimos}, nos movemos y existimos (\emph{Hch} 17, 28), ha elegido la tierra como su morada; nació en Belén; ¡ha establecido su Reino en los corazones humanos!
					
					¿Podemos ignorar o distorsionar todo esto? ¿Es lícito destruir la morada de Dios entre los hombres? ¿No es necesario, en cambio, cambiar de raíz los planes del dominio humano sobre la tierra?
					
					7. ¡Hermanos y hermanas de toda la tierra! Si Dios nos amó tanto que se hizo hombre con nosotros, ¿cómo no amarnos unos a otros, hasta el punto de compartir con los demás lo que cada uno posea, para el gozo de todos? Solo el amor que se hace don puede transformar la faz de nuestro planeta, haciendo que las mentes y los corazones se vuelvan a pensamientos de hermandad y de paz.
					
					Hombres y mujeres del mundo, Cristo nos pide que nos amemos unos a otros. Este es el mensaje de Navidad, este es el deseo que dirijo a todos desde el fondo de mi corazón.
				\end{body}


			\subsubsection{Urbi et Orbi (1989):}

				\begin{body}
					1. \textquote{ \ldots{} les dio poder de hacerse hijos de Dios} (\emph{Jn} 1, 12)
					
					Es la solemnidad de la Navidad. Los ojos de nuestra alma ven al Niño, colocado en el pesebre. La mirada de nuestra fe se detiene en las palabras del prólogo de Juan. \textquote{A los que le recibieron les dio poder de hacerse hijos de Dios}.
					
					2. Te bendecimos, Hijo del hombre, que eres el Verbo eterno.
					
					Gloria al Padre que nos dio a ti, el Unigénito.
					
					Gloria al Espíritu, que procede del Padre y de ti, Hijo de Dios.
					
					Gloria al misterio eterno, que todo lo abarca.
					
					En esta noche se acercó al hombre, entró en su vida y en su historia.
					
					Ha cruzado el umbral de nuestra existencia humana.
					
					3. El Niño envuelto en pañales y colocado en un pesebre. El Niño humano indefenso - y al mismo tiempo el Poder, que sobrepasa todo lo que el hombre es, todo lo que puede.
					
					Porque el hombre no puede llegar a ser como Dios con sus propias fuerzas ---como ha sido confirmado por la historia, desde el principio---.
					
					Y, al mismo tiempo, el hombre puede llegar a ser como Dios por el poder de Dios.
					
					Este poder está en el Hijo, Verbo eterno, que \textquote{se hizo carne y habitó entre nosotros} (\emph{Jn} 1, 14).
					
					Este es el primer día de su morada entre nosotros.
					
					\textquote{Él estaba en el mundo, y el mundo fue hecho por él, pero el mundo no lo reconoció. Vino entre su propia gente, pero los suyos no lo recibieron. Pero a los que lo recibieron, les dio poder para convertirse en hijos de Dios: quienes \ldots{} fueron engendrados por Dios} (\emph{Jn} 1, 10-12).
					
					4. La historia sigue su camino \ldots{} muchos, incontables hombres, naciones, pueblos, lenguas, razas, culturas \ldots{} millones y miles de millones \ldots{} y él es el único: ahora colocado como Niño en el pesebre (\textquote{no había sitio para él \ldots{} en la posada}), y luego en la Cruz.
					
					Él, el único. Y luego, resucitó, él, el único.
					
					¿Cuántos no lo han aceptado? ¿Cuántos no le han acogido?
					
					¿Cuántos saben de él? ¿Cuántos no lo saben?
					
					Quisiéramos calcular con estadísticas humanas hasta dónde llega este poder que está en él: nacido - crucificado - resucitado.
					
					Humanamente, nos gustaría saber cuántos se han convertido, en él y para él, en hijos de Dios, hijos en el Hijo.
					
					Pero los medidores humanos no pueden medir el misterio de Dios.
					
					No pueden medir el don del nacimiento de Dios, que está presente en la historia del hombre y en la historia del mundo, que obra en las almas humanas por la fuerza del Espíritu que da la vida.
					
					5. \textquote{Todos los confines de la tierra vieron la salvación de nuestro Dios} (\emph{Sal} 98, 3).
					
					Sí. Vinieron los pastores de Belén y vieron.
					
					Sí. También vinieron los sabios de Oriente, y vieron.
					
					Y vieron el viejo Simeón y la profetisa Ana en el templo de Jerusalén.
					
					¿Con qué mirada te ven, Verbo Encarnado, todos los confines de la tierra?
					
					De hecho, tú has venido para todos. Tú eres la salvación de nuestro Dios, que es para todos y viene a través de ti.
					
					Dios \textquote{quiere que todos los hombres se salven y lleguen al conocimiento de la verdad} (\emph{1 Tim} 2, 4).
					
					La verdad y la gracia han venido a través de ti.
					
					Tú eres la verdad.
					
					Tú eres el camino y la vida (cf. \emph{Jn} 14, 6).
					
					Y aunque los tuyos no te han acogido \ldots{} aunque no había lugar para ti en la posada \ldots{}
					
					en ti Dios nos ha acogido \ldots{} nos ha acogido a todos.
					
					6. En ti, Dios también nos ha acogido a nosotros, {[}hombres y mujeres del segundo milenio, que está por terminar{]}.
					
					No miró nuestras contradicciones, nuestras infidelidades, nuestros desequilibrios.
					
					De hecho, te envió a ti, su Palabra, para sanarnos.
					
					Para decirnos que, por este camino, corremos hacia la autodestrucción.
					
					El mundo aspira a la paz: sin embargo, cada día nuestros hermanos y hermanas mueren en los conflictos en curso, {[}en el Líbano, en Tierra Santa, en América Central;{]} mueren en luchas fratricidas por las supremacías racistas, ideológicas, económicas; mueren de absurda imprudencia.
					
					El mundo aspira a la reconciliación: sin embargo, cada día miles de refugiados son abandonados y rechazados; las minorías étnicas y religiosas son ignoradas en sus necesidades fundamentales; sectores enteros de la población se mantienen al margen de la sociedad en un aislamiento cada vez mayor.
					
					El mundo aspira al equilibrio, interno y externo: sin embargo, el medio ambiente se degrada cada día más por motivos de interés o inconsciencia.
					
					7. El anuncio de la verdad y de la gracia que nos llega en Navidad a través de ti debe tocarnos a todos. Ese anuncio es para nosotros, porque viniste por nosotros, te convertiste en uno de nosotros.
					
					¡Haz que te acojamos, Palabra eterna del Padre!
					
					Que el mundo te acoja.
					
					Suscita en los corazones el rechazo de toda barrera de raza, de ideología, de intolerancia.
					
					Favorece el progreso de las negociaciones en curso sobre control y reducción de armas.
					
					Sostén a quienes se comprometen a superar los conflictos que se han prolongado durante demasiado tiempo {[}en África y Asia{]}, para que los pueblos involucrados en ellos recuperen su libertad y sus derechos, a través de un diálogo leal y confiado.
					
					8. ¡Que te acoja, Verbo Encarnado, también nuestra vieja Europa!
					
					Ella lleva profundamente grabado el estigma de tu Evangelio, del que nació su civilización, su arte, su concepción de la inviolable dignidad del hombre.
					
					Que esta Europa abra sus puertas y su corazón para comprender y acoger las ansiedades, preocupaciones, problemas de las naciones que piden su ayuda.
					
					Que sepa responder con el vigor y la generosidad de sus raíces cristianas a este momento histórico tan particular, verdadero Kairòs providencial, que el mundo vive ahora como liberado de una pesadilla y abierto a una mejor esperanza.
					
					{[}En particular, bendice en esta hora, oh Señor, a la noble tierra de Rumanía, que celebra esta Navidad con temor, en el dolor de tantas vidas humanas trágicamente perdidas y en la alegría de haber retomado el camino de la libertad.{]}
					
					9. Hermanos y hermanas, aquí presentes.
					
					{[}Hermanos y hermanas, que me escuchan en la radio y la televisión, en todos los Continentes,{]} venid a la cuna del Niño indefenso, que es el Poder de Dios, Él nació por nosotros.
					
					Venid \ldots{} y veréis \ldots{} y seréis acogidos, porque hoy se ha manifestado la bondad de Dios y su amor por los hombres.
				\end{body}


			\subsubsection{Urbi et Orbi (1992):}

				\begin{body}
					1. \textquote{Gloria a Dios en las alturas y paz en la tierra a los hombres que Él ama} (\emph{Lc} 2, 14).
					
					Este es el mensaje que escuchamos nuevamente a la medianoche, cuando los pastores llegaron a la gruta de Belén.
					
					Y ahora, habiendo llegado al corazón de este día bendito, la Iglesia nos anuncia el Misterio: \textquote{El Verbo se hizo carne y habitó entre nosotros} (\emph{Jn} 1, 14).
					
					El Hijo eterno del Padre está en el mundo; el Verbo, por quien todo fue hecho.
					
					Él estaba al principio con Dios, era Dios (cf. \emph{Jn} 1, 1-2). El Padre le ha dicho desde el principio de los siglos: Tú eres mi Hijo, yo te he engendrado en el \textquote{hoy} eterno y divino (cf. \emph{Hb} 1, 5).
					
					El Verbo --- el Hijo: Dios de Dios, Luz de Luz. El Verbo se hizo carne y habitó entre nosotros.
					
					La noche de Belén es el comienzo de su morada entre los hombres. A plena luz del día, la Iglesia proclama el Misterio del Verbo hecho carne.
					
					2. Cur Deus homo? ¿Por qué Dios se hizo hombre?
					
					El hombre pregunta: ¿Por qué? Muéstrame el camino a las profundidades de tu Misterio. El hombre le ha estado haciendo a Dios esta pregunta durante {[}dos mil años{]}.
					
					Pero muchas veces se responde a sí mismo, sin esperar la respuesta de Dios.
					
					Tú, oh Dios, estás por encima de todas las cosas, dice. Tú sólo puedes estar por encima del mundo: Uno y solo en tu infinita Majestad. ¡Dios, quédate solo! ¡No te rebajes a la criatura, no te rebajes al hombre! Así responde el hombre. Y a veces incluso llega a decir: ¡Oh Dios, mantente fuera del mundo! ¡Deja el mundo en manos del hombre solo! Aquí limitas al hombre; aquí no podemos vivir juntos. Y cree que tal respuesta es un signo de progreso y autonomía para la humanidad.
					
					Cur Deus homo? ¿Por qué Dios se hizo hombre? El hombre le hace la pregunta a Dios, pero luego se la responde a sí mismo. Sin embargo, sólo Dios puede señalar el camino a las profundidades de su Misterio.
					
					3. La respuesta de Dios se llama Evangelio.
					
					La respuesta de Dios tiene su inicio en la noche de Belén, para luego convertirse en testimonio de Aquel que nació esa misma noche.
					
					De hecho, tanto amó Dios al mundo que dio a su Hijo para que el hombre no muera, sino que tenga vida eterna en él (cf. \emph{Jn} 3, 16).
					
					4. Hermanos y hermanas, no nos cerremos ante Dios, no impidamos que viva en nosotros Aquel que nació hoy, consustancial con el Padre, el Primogénito de toda criatura.
					
					Él viene a su propiedad, no se lo impidamos.
					
					No pensamos que Dios deba permanecer solo, revestido de inefable Majestad, pero solo, por encima del mundo y fuera de él.
					
					El mundo le pertenece; y, en el mundo, el hombre es el ser más suyo, creado a su imagen y semejanza, imagen de lo Invisible en el mundo visible.
					
					Amor es el nombre que mejor se adapta a la divina Majestad. Y el amor sólo es tal cuando se entrega, cuando se hace don para los demás.
					
					¿Puede el hombre realizarse plenamente a sí mismo sin amor?
					
					¿Qué más puede salvarlo aparte del Amor todopoderoso, revelado en ese Niño indefenso?
					
					¿Quién más puede revelar plenamente al hombre a sí mismo, sino Él?
					
					Su nombre es Jesús, que significa \textquote{Dios salva}.
					
					5. Queridos hermanos y hermanas, hombres y mujeres de toda la humanidad, Cristo ---Dios que salva---, desea encontrarnos.
					
					Él está entre nosotros: acojámosle, abramos a Él nuestro corazón.
					
					Escuchen su voz, ustedes, líderes de las naciones, llamados a gestionar el destino de los pueblos: la solidaridad --- ha proclamado Él en silencio en la noche de la esperanza --- es el camino de la justicia y la paz.
					
					Tú que sufres en los caminos de la existencia, oprimido por la injusticia y el mal, decepcionado e insatisfecho con todo bienestar transitorio: la vida --- anuncia el Verbo hecho carne --- se ha manifestado hoy en todo su esplendor.
					
					Es un canto de alegría que acalla el grito amenazador de muerte.
					
					Escucha la voz del amor, dulce y poderosa al mismo tiempo, especialmente tú, que empuñas armas violentas y asesinas.
					
					6. Ante el pesebre, donde el Hombre-Dios llora bajo la mirada ansiosa de María y José, nuestro pensamiento se dirige espontáneamente a nuestros muchos hermanos para quienes la Navidad también está marcada este año por el miedo, la tristeza y el dolor.
					
					{[}Pienso en los niños de Sarajevo, de Banja Luka, de las poblaciones de Bosnia y Herzegovina, rehenes de una violencia planificada e inhumana; en Liberia, devastada y destrozada por luchas locas y fratricidas durante más de tres años; en Somalia, donde afortunadamente, gracias a la ayuda, se enciende la confianza de un futuro mejor.
					
					¿Cómo olvidar en este momento la expectativa de una paz segura y duradera en Angola, en Mozambique?
					
					¿Cómo no preocuparnos por el clima de odio y lucha que en Tierra Santa, suelo santificado por el nacimiento del divino Hacedor de la paz, persiste y aleja aún más las esperanzas suscitadas por el proceso de pacificación iniciado en Madrid?
					
					7. Cur Deus homo?
					
					Aunque oscurecido por las brumas y tormentas de la historia, el camino de la humanidad está iluminado por la respuesta de Dios, que aumenta nuestra esperanza.
					
					Tu amor, oh Verbo encarnado, es más fuerte que el odio, es más fuerte que la muerte misma (cf. \emph{Ct} 8, 6).
					
					¡Sí! Nada puede evitar que vengas a nosotros, incluso en los lugares maltratados del mundo donde la gente todavía mata y el mal parece reinar sin oposición.
					
					Filius datus est nobis!
					
					Tú vienes, oh Señor, a curar las heridas abiertas en el costado de la humanidad.
					
					Ven allí donde el rugido de las armas te impida escuchar siquiera el llanto desconsolado de mujeres y niños, los lamentos de los heridos, las débiles invocaciones de los moribundos.
					
					A veces la tierra parece sorda e impenetrable al Misterio de tu presencia.
					
					Ven, te lo rogamos, para que triunfe tu Amor, don de la paz.
					
					{[} \ldots{}{]}
					
					8. En el esplendor de este día santo resuena el cántico de alegría celestial: \textquote{Gloria a Dios en lo alto del cielo y paz en la tierra a los hombres que Él ama}.
					
					Resplandece la victoria del Amor todopoderoso, que llena plenamente todas nuestras expectativas humanas.
					
					Cur Deus homo?
					
					Puer natus est nobis! Filius datus est nobis!
					
					Es la respuesta de Dios, así responde el Verbo Encarnado.
					
					Y su voz llega al hombre cuando él, ante el divino Nacimiento de Belén, deja que Dios hable.
					
					Muéstrame, Señor, el camino a las profundidades de tu Misterio.
										
					¡Muéstrame el camino! Amén.
				\end{body}

			\subsubsection{Urbi et Orbi (1995):}

				\begin{body}
					1. \textquote{Tú eres mi hijo; yo te he engendrado hoy} (\emph{Hb} 1, 5).
					
					Las palabras de la liturgia de hoy nos introducen en el misterio del nacimiento eterno, más allá del tiempo, del Hijo de Dios, Hijo consustancial al Padre.
					
					El Evangelio de Juan dice: \textquote{En el principio era el Verbo, y el Verbo era con Dios y el Verbo era Dios. Él era en el principio con Dios} (\emph{Jn} 1, 1-2).
					
					Profesamos la misma verdad en el Credo: \textquote{Dios de Dios, Luz de Luz, Dios verdadero de Dios verdadero, engendrado, no creado, de la misma sustancia que el Padre; por él fueron creadas todas las cosas. Por nosotros los hombres y por nuestra salvación descendió del cielo, y por obra del Espíritu Santo se encarnó en el seno de la Virgen María y se hizo hombre}.
					
					He aquí la alegre noticia del nacimiento del Señor, tal como la transmitieron los evangelistas y la tradición apostólica de la Iglesia.
					
					Hoy queremos anunciarlo \textquote{a la Ciudad y al Mundo}, Urbi et Orbi.
					
					2. \textquote{En el mundo estaba y el mundo por él fue hecho} (\emph{Jn} 1, 10).
					
					Viene a los suyos El que sale a la luz en la noche de Navidad.
					
					¿Por qué viene? Viene a comunicar una \textquote{nueva fuerza}, un \textquote{poder} diferente al del mundo.
					
					Viene pobre en un establo de Belén, con el mayor don: da a los hombres la filiación divina.
					
					A todos los que le acogen les da la \textquote{potestad de llegar a ser hijos de Dios} (\emph{Jn} 1, 12), para que en él, Hijo eterno del Padre eterno, \textquote{sean engendrados de Dios} (cf. \emph{Jn} 1 , 13).
					
					De hecho, en él, en el Niño de la Noche Santa, habita la vida (cf. \emph{Jn} 1, 4): vida que no conoce la muerte; vida de Dios mismo; vida que, como dice San Juan, es la luz de los hombres.
					
					La luz brilla en las tinieblas, y las tinieblas no la vencieron (cf. \emph{Jn} 1, 4-5).
					
					En la noche de Navidad surge la luz que es Cristo. Brilla y penetra en los corazones de los hombres, injertando en ellos una nueva vida. Enciende en ellos la luz eterna, que siempre ilumina al ser humano incluso cuando la oscuridad de la muerte envuelve su cuerpo.
					
					Por eso \textquote{el Verbo se hizo carne y habitó entre nosotros} (\emph{Jn} 1, 14).
					
					3. \textquote{Vino a los suyos, pero los suyos no la recibieron} (\emph{Jn} 1,11), recuerda el Prólogo del Evangelio de Juan.
					
					Lucas el evangelista confirma esta verdad y recuerda que \textquote{no había lugar para ellos en la posada} (\emph{Lc} 2, 7).
					
					\textquote{Para ellos}, es decir, para María y José y para el Niño que estaba por nacer.
					
					He aquí un hecho que a menudo se menciona en los villancicos: \textquote{Su pueblo no lo aceptó \ldots{}}.
					
					En la gran posada de la comunidad humana, como la pequeña posada de nuestro corazón, ¡cuántos pobres aún hoy, {[}en el umbral del año 2000{]}, llaman a la puerta!
					
					4. Es Navidad: ¡una fiesta de acogida y amor!
					
					{[}¿Encontrarán un lugar en este día las familias desplazadas de Bosnia y Herzegovina, que todavía esperan ansiosamente los frutos de la paz, de esa paz recientemente proclamada? ¿Podrán los refugiados de Ruanda regresar a un país verdaderamente reconciliado? ¿Podrá el pueblo de Burundi redescubrir el camino de la paz fraterna? ¿Tendrá el pueblo de Sri Lanka la oportunidad de mirar juntos, de la mano, hacia un futuro de fraternidad y solidaridad? Por último, ¿se le dará al pueblo iraquí la alegría de recuperar una existencia normal, después de los largos años de embargo?{]}
					
					{[}¿Serán acogidas las poblaciones de Kurdistán, entre las que muchas personas se ven obligadas a afrontar el invierno, una vez más, en la más dura precariedad? ¿Y cómo no pensar en los hermanos y hermanas del sur de Sudán, que todavía sufren violencia armada, alimentada sin descanso?{]}
					
					{[}Por último, no podemos olvidar al pueblo de Argelia, que sigue sufriendo, víctima de atroces juicios.{]}
					
					¡Es en este mundo herido donde irrumpe el Niño Jesús, amoroso y frágil!
					
					Viene a liberar al hombre atrapado en el odio y esclavo de particularismos y divisiones.
					
					Viene a abrir nuevos horizontes.
					
					El Hijo de Dios hace nacer la esperanza de que, a pesar de tantas dificultades graves, la paz finalmente emerja en el horizonte.
					
					También hay signos prometedores de esto en países con problemas como Irlanda del Norte y Oriente Medio.
					
					Es nuestro deseo que los hombres abran su corazón a la Palabra de Dios encarnada en la pobreza de Belén.
					
					5. Este es el Misterio que celebramos hoy: Dios \textquote{nos ha hablado por el Hijo} (\emph{Hb} 1, 2).
					
					Muchas veces y de diversas maneras Dios había hablado a través de los Profetas, pero cuando \textquote{llegó la plenitud de los tiempos} (\emph{Gál} 4, 4), habló por medio del Hijo.
					
					El Hijo es el reflejo de la gloria del Padre; la irradiación de su sustancia, que todo lo sostiene con el poder de su palabra. Esto es lo que dice el autor de la carta a los Hebreos del Hijo recién nacido de María (cf. \emph{Hb} 1,3). Si por medio de él Dios Padre creó el cosmos, también es el Primogénito y Heredero de toda la creación (cf. \emph{Hb} 1,1-2).
					
					Este pobre Niño, para quien \textquote{no había lugar en la posada}, a pesar de las apariencias, es el único Heredero de toda la creación.
					
					Vino a compartir su herencia con nosotros, para que nosotros, habiendo llegado a ser hijos por la adopción divina, participemos de la herencia que ha traído consigo al mundo.
					
					Palabra eterna, hoy contemplamos tu gloria, \textquote{gloria como del unigénito del Padre, lleno de gracia y de verdad} (\emph{Jn} 1, 14).
					
					Que las buenas nuevas de tu nacimiento, antiguas y siempre nuevas, lleguen a los pueblos y naciones de todos los continentes sobre las olas del éter y traigan la paz al mundo.
				\end{body}


			\subsubsection{Urbi et Orbi (2001)} 
					
				\begin{body}
					1. \textquote{\emph{Christus est pax nostra}},\\ \textquote{\emph{Cristo es nuestra paz.\\ Él ha hecho de los dos pueblos una sola cosa}} (\emph{Ef} 2, 14).\\ En el alba del nuevo milenio,\\ comenzado con tantas esperanzas,\\ pero ahora amenazado por nubes tenebrosas\\ de violencia y de guerra,\\ las palabras del apóstol Pablo\\ que escuchamos esta Navidad\\ es un rayo de luz penetrante,\\ un clamor de confianza y optimismo.\\ El divino Niño nacido en Belén\\ lleva en sus pequeñas manos, como un don,\\ el secreto de la paz para la humanidad.\\ ¡Él es el Príncipe de la paz!\\ He aquí el gozoso anuncio que se oyó aquella noche en Belén,\\ y que quiero repetir al mundo\\ en este día bendito.\\ Escuchemos una vez más las palabras del ángel:\\ \textquote{\emph{os traigo la buena noticia,\\ la gran alegría para todo el pueblo:\\ hoy, en la ciudad de David,\\ os ha nacido un salvador: el Mesías, el Señor}} (\emph{Lc} 2, 10-11).\\ En el día de hoy, la Iglesia se hace eco de los ángeles,\\ y reitera su extraordinario mensaje,\\ que sorprendió en primer lugar a los pastores\\ en las alturas de Belén.
					
					2. \textquote{\emph{Christus est pax nostra}!}\\ Cristo, el \textquote{\emph{niño envuelto en pañales\\ y acostado en un pesebre}} (\emph{Lc} 2, 10-12),\\ Él es precisamente nuestra paz.\\ Un Niño indefenso, recién nacido en la humildad de una cueva,\\ devuelve la dignidad a cada vida que nace,\\ da esperanza a quien yace en la duda y en el desaliento.\\ Él ha venido para curar a los heridos de la vida\\ y para dar nuevo sentido incluso a la muerte.\\ En aquel Niño, dócil y desvalido,\\ que llora en una gruta fría y destartalada,\\ Dios ha destruido el pecado\\ y ha puesto el germen de una humanidad nueva,\\ llamada a llevar a término\\ el proyecto original de la creación\\ y a transcenderlo con la gracia de la redención.
					
					3. \textquote{\emph{Christus est pax nostra}!}\\ Hombres y mujeres del tercer milenio,\\ vosotros que tenéis hambre de justicia y de paz,\\ ¡acoged el mensaje de Navidad\\ que se propaga hoy por todo el mundo!\\ Jesús ha nacido para consolidar las relaciones\\ entre los hombres y los pueblos,\\ y hacer de todos ellos hermanos en Él.\\ Ha venido para derribar \textquote{el muro que los separaba:\\ el odio} (\emph{Ef} 2, 14),\\ y para hacer de la humanidad una sola familia.\\ Sí, podemos repetir con certeza:\\ ¡Hoy, con el Verbo encarnado, ha nacido la paz!\\ Paz que se ha de implorar,\\ porque sólo Dios es su autor y garante.\\ Paz que se ha de construir\\ en un mundo en el que pueblos y naciones,\\ afectados por tantas y tan diversas dificultades,\\ esperan en una humanidad\\ no sólo globalizada por intereses económicos,\\ sino por el esfuerzo constante\\ en favor de una convivencia más justa y solidaria.
					
					4. Como los pastores, acudamos a Belén,\\ quedémonos en adoración ante la gruta,\\ fijando la mirada en el Redentor recién nacido.\\ En Él podemos reconocer los rasgos\\ de cada pequeño ser humano que viene a la luz,\\ sea cual fuere su raza o nación:\\ es el pequeño palestino y el pequeño israelí;\\ es el bebé estadounidense y el afgano;\\ es el hijo del hutu y el hijo del tutsi\ldots{}\\ es el niño cualquiera, que es alguien para Cristo.\\ Hoy pienso en todos los pequeños del mundo:\\ muchos, demasiados, son los niños\\ que nacen ya condenados a sufrir, sin culpa,\\ las consecuencias de conflictos inhumanos.\\ ¡Salvemos a los niños,\\ para salvar la esperanza de la humanidad!\\ Nos lo pide hoy con fuerza\\ aquel Niño nacido en Belén,\\ el Dios que se hizo hombre,\\ para devolvernos el derecho de esperar.
					
					5. Supliquemos a Cristo el don de la paz\\ para cuantos sufren a causa de conflictos, antiguos y nuevos.\\ Todos los días siento en mi corazón\\ los dramáticos problemas de Tierra Santa;\\ cada día pienso con preocupación\\ en cuantos mueren de hambre y de frío;\\ día tras día me llega, angustiado,\\ el grito de quien, en tantas partes del mundo,\\ invoca una distribución más ecuánime de los recursos\\ y un trabajo dignamente retribuido para todos.\\ ¡Que nadie deje de esperar\\ en el poder del amor de Dios!\\ Que Cristo sea luz y sustento\\ de quien, a veces contracorriente, cree y actúa\\ en favor del encuentro, del diálogo, de la cooperación\\ entre las culturas y las religiones.\\ Que Cristo guíe en la paz los pasos\\ de quien se afana incansablemente\\ por el progreso de la ciencia y la técnica.\\ Que nunca se usen estos grandes dones de Dios\\ contra el respeto y la promoción de la dignidad humana.\\ ¡Que jamás se utilice el nombre santo de Dios\\ para corroborar el odio!\\ ¡Que jamás se haga de Él motivo de intolerancia y violencia!\\ Que el dulce rostro del Niño de Belén\\ recuerde a todos que tenemos un único Padre.
					
					6. \textquote{\emph{Christus est pax nostra}!}\\ Hermanos y hermanas que me escucháis,\\ abrid el corazón a este mensaje de paz,\\ abridlo a Cristo, Hijo de la Virgen María,\\ a Aquel que se ha hecho \textquote{nuestra paz}.\\ Abridlo a Él, que nada nos quita\\ si no es el pecado,\\ y nos da en cambio\\ plenitud de humanidad y de alegría.\\ Y Tú, adorado Niño de Belén,\\ lleva la paz a cada familia y ciudad,\\ a cada nación y continente.\\ ¡Ven, Dios hecho hombre!\\ ¡Ven a ser el corazón del mundo renovado por el amor!\\ ¡Ven especialmente allí donde más peligra\\ la suerte de la humanidad!\\ ¡Ven, y no tardes!\\ ¡Tú eres \textquote{\emph{nuestra paz}}! (\emph{Ef} 2,14).
				\end{body}


\newsection

	\section{Temas}\label{navidad_temas}
		%\input{../../CEC/CEC_}
		%\input{../../CEC/CEC_}
		%\input{../../CEC/CEC_}
		%\input{../../CEC/CEC_}
		%\input{../../CEC/CEC_}
		%\input{../../CEC/CEC_}
		%\input{../../CEC/CEC_}
		%\input{../../CEC/CEC_}
		