El gozo

CEC 30, 163, 301, 736, 1829, 1832, 2015, 2362:

\textbf{30} \textquote{Alégrese el corazón de los que buscan a Dios} (\emph{Sal} 105,3). Si el hombre puede olvidar o rechazar a Dios, Dios no cesa de llamar a todo hombre a buscarle para que viva y encuentre la dicha. Pero esta búsqueda exige del hombre todo el esfuerzo de su inteligencia, la rectitud de su voluntad, \textquote{un corazón recto}, y también el testimonio de otros que le enseñen a buscar a Dios.

\textquote{Tú eres grande, Señor, y muy digno de alabanza: grande es tu poder, y tu sabiduría no tiene medida. Y el hombre, pequeña parte de tu creación, pretende alabarte, precisamente el hombre que, revestido de su condición mortal, lleva en sí el testimonio de su pecado y el testimonio de que tú resistes a los soberbios. A pesar de todo, el hombre, pequeña parte de tu creación, quiere alabarte. Tú mismo le incitas a ello, haciendo que encuentre sus delicias en tu alabanza, porque nos has hecho para ti y nuestro corazón está inquieto mientras no descansa en ti} (San Agustín, \emph{Confessiones,} 1,1,1).

\textbf{La fe, comienzo de la vida eterna}

\textbf{163} La fe nos hace gustar de antemano el gozo y la luz de la visión beatífica, fin de nuestro caminar aquí abajo. Entonces veremos a Dios \textquote{cara a cara} (\emph{1 Co} 13,12), \textquote{tal cual es} (\emph{1 Jn} 3,2). La fe es, pues, ya el comienzo de la vida eterna:

\textquote{Mientras que ahora contemplamos las bendiciones de la fe como reflejadas en un espejo, es como si poseyésemos ya las cosas maravillosas de que nuestra fe nos asegura que gozaremos un día} (San Basilio Magno, \emph{Liber de Spiritu Sancto} 15,36: PG 32, 132; cf. Santo Tomás de Aquino, \emph{S.Th}., 2-2, q.4, a.1, c).

\textbf{Dios mantiene y conduce la creación}

\textbf{301} Realizada la creación, Dios no abandona su criatura a ella misma. No sólo le da el ser y el existir, sino que la mantiene a cada instante en el ser, le da el obrar y la lleva a su término. Reconocer esta dependencia completa con respecto al Creador es fuente de sabiduría y de libertad, de gozo y de confianza:

\textquote{Amas a todos los seres y nada de lo que hiciste aborreces pues, si algo odiases, no lo hubieras creado. Y ¿cómo podría subsistir cosa que no hubieses querido? ¿Cómo se conservaría si no la hubieses llamado? Mas tú todo lo perdonas porque todo es tuyo, Señor que amas la vida} (\emph{Sb} 11, 24-26).

\textbf{736} Gracias a este poder del Espíritu Santo los hijos de Dios pueden dar fruto. El que nos ha injertado en la Vid verdadera hará que demos \textquote{el fruto del Espíritu, que es caridad, alegría, paz, paciencia, afabilidad, bondad, fidelidad, mansedumbre, templanza} (\emph{Ga} 5, 22-23). \textquote{El Espíritu es nuestra Vida}: cuanto más renunciamos a nosotros mismos (cf. \emph{Mt} 16, 24-26), más \textquote{obramos también según el Espíritu} (\emph{Ga} 5, 25):

«Por el Espíritu Santo se nos concede de nuevo la entrada en el paraíso, la posesión del reino de los cielos, la recuperación de la adopción de hijos: se nos da la confianza de invocar a Dios como Padre, la participación de la gracia de Cristo, el podernos llamar hijos de la luz, el compartir la gloria eterna (San Basilio Magno, \emph{Liber de Spiritu Sancto}, 15, 36: PG 32, 132).

\textbf{1829} La caridad tiene por \emph{frutos} el gozo, la paz y la misericordia. Exige la práctica del bien y la corrección fraterna; es benevolencia; suscita la reciprocidad; es siempre desinteresada y generosa; es amistad y comunión:

\textquote{La culminación de todas nuestras obras es el amor. Ese es el fin; para conseguirlo, corremos; hacia él corremos; una vez llegados, en él reposamos} (San Agustín, \emph{In epistulam Ioannis tractatus,} 10, 4).

\textbf{1832} Los \emph{frutos} del Espíritu son perfecciones que forma en nosotros el Espíritu Santo como primicias de la gloria eterna. La tradición de la Iglesia enumera doce: \textquote{caridad, gozo, paz, paciencia, longanimidad, bondad, benignidad, mansedumbre, fidelidad, modestia, continencia, castidad} (\emph{Ga} 5,22-23, vulg.).

\textbf{2015} El camino de la perfección pasa por la cruz. No hay santidad sin renuncia y sin combate espiritual (cf. \emph{2 Tm} 4). El progreso espiritual implica la ascesis y la mortificación que conducen gradualmente a vivir en la paz y el gozo de las bienaventuranzas:

\textquote{El que asciende no termina nunca de subir; y va paso a paso; no se alcanza nunca el final de lo que es siempre susceptible de perfección. El deseo de quien asciende no se detiene nunca en lo que ya le es conocido} (San Gregorio de Nisa, \emph{In Canticum} homilia 8).

\textbf{2362} \textquote{Los actos [\ldots{}] con los que los esposos se unen íntima y castamente entre sí son honestos y dignos, y, realizados de modo verdaderamente humano, significan y fomentan la recíproca donación, con la que se enriquecen mutuamente con alegría y gratitud} (\href{http://www.vatican.va/archive/hist_councils/ii_vatican_council/documents/vat-ii_const_19651207_gaudium-et-spes_sp.html}{\emph{GS}} 49). La sexualidad es fuente de alegría y de agrado:

\textquote{El Creador [\ldots{}] estableció que en esta función {[}de generación{]} los esposos experimentasen un placer y una satisfacción del cuerpo y del espíritu. Por tanto, los esposos no hacen nada malo procurando este placer y gozando de él. Aceptan lo que el Creador les ha destinado. Sin embargo, los esposos deben saber mantenerse en los límites de una justa moderación} (Pío XII, \emph{Discurso a los participantes en el Congreso de la Unión Católica Italiana de especialistas en Obstetricia}, 29 octubre 1951).