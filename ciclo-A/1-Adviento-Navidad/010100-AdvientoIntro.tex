\part{Tiempo de Adviento}

\chapter{Introducción}

\begin{introstyle}
	
	El Adviento es tiempo de espera, de conversión, de esperanza\anote{id2}\label{fn2}:
	
	-- espera-memoria de la primera y humilde venida del Salvador en nuestra carne mortal; espera-súplica de la última y gloriosa venida de Cristo, Señor de la historia y Juez universal;
	
	-- conversión, a la cual invita con frecuencia la Liturgia de este tiempo, mediante la voz de los profetas y sobre todo de Juan Bautista: \textquote{Convertíos, porque está cerca el reino de los cielos} (Mt 3,2);
	
	-- esperanza gozosa de que la salvación ya realizada por Cristo (cfr. Rom 8,24-25) y las realidades de la gracia ya presentes en el mundo lleguen a su madurez y plenitud, por lo que la promesa se convertirá en posesión, la fe en visión y \textquote{nosotros seremos semejantes a Él porque le veremos tal cual es} (1 Jn 3,2).
	
	\section{Normativa litúrgica}
	El tiempo de Adviento comienza con las primeras vísperas del domingo que cae el 30 de noviembre, o más próximo a ese día, y concluye antes de las primeras vísperas de Navidad\anote{id3}\label{fn3}.
	
	Este tiempo tiene dos características: es a la vez un tiempo de preparación a las solemnidades de Navidad en que se conmemora la primera Venida del Hijo de Dios entre los hombres, y un tiempo en el cual, mediante esta celebración, el ánimo se dirige a esperar la segunda Venida de Cristo al fin de los tiempos. Por estos dos motivos, el Adviento se presenta como un tiempo de piadosa y alegre esperanza.
	
	\newpage
	\section{Lecturas de los domingos de Adviento}
	
	En los domingos de Adviento las lecturas\anote{id4}\label{fn4} del Evangelio tienen una característica propia: se refieren a la venida del Señor al final de los tiempos (primer domingo), a Juan Bautista (segundo y tercer domingos), a los acontecimientos que prepararon de cerca el nacimiento del Señor (cuarto domingo).
	
	Las lecturas del Antiguo Testamento son profecías sobre el Mesías y el tiempo mesiánico, tomadas principalmente del libro de Isaías.
	
	Las lecturas del Apóstol contienen exhortaciones y amonestaciones conformes a las diversas características de este tiempo.
	
	\subsection{Algunos aspectos que debe tener en cuenta el homileta}
	
	El Adviento es el tiempo que prepara a los cristianos a las gracias que serán dadas, una vez más en este año, en la celebración de la gran Solemnidad de la Navidad. Ya desde el I domingo de Adviento, el homileta exhorta al pueblo para que emprenda su preparación caracterizada por distintas facetas, cada una de ellas sugerida por la rica selección de pasajes bíblicos del Leccionario de este tiempo. La primera fase del Adviento nos invita a preparar la Navidad animándonos no sólo a dirigir la mirada al tiempo de la primera Venida de nuestro Señor, cuando, como dice el prefacio I de Adviento, Él asume \textquote{la humildad de nuestra carne}, sino también, a esperar vigilantes su Venida \textquote{en la majestad de su gloria}, cuando \textquote{podamos recibir los bienes prometidos}\anote{id5}\label{fn5}.
	
	En la predicación no debe perderse de vista que existe un doble significado del Adviento, un doble significado de la Venida del Señor. Este tiempo nos prepara para su Venida en la gracia de la fiesta de la Navidad y a su retorno para el juicio al final de los tiempos. Los textos bíblicos deberían ser explicados considerando este doble significado. Según el texto, se puede evidenciar una u otra Venida, aunque, con frecuencia, el mismo pasaje presenta palabras e imágenes relativas a ambas.
	
	Existe, además, otra Venida: escuchamos la Palabra de Dios en la asamblea eucarística, donde Cristo está verdaderamente presente. Al comienzo del tiempo de Adviento la Iglesia recuerda la enseñanza de san Bernardo, es decir, que entre las dos Venidas visibles de Cristo, en la historia y al final de los tiempos, existe una venida invisible, aquí y ahora\anote{id6}\label{fn6}, así como hace suyas las palabras de san Carlos Borromeo:
	
	\textquote{Este tiempo (\ldots{}) nos enseña que la venida de Cristo no solo aprovechó a los que vivían en el tiempo del Salvador, sino que su eficacia continúa y aún hoy se nos comunica si queremos recibir, mediante la fe y los sacramentos, la gracia que él nos prometió, y si ordenamos nuestra conducta conforme a sus mandamientos\anote{id7}\label{fn7}}.
	
	\section{Algunos aspectos sobre el Adviento}
	
	El \emph{Directorio sobre la Piedad Popular y la Liturgia} nos indica algunos aspectos sobre el Adviento que pueden ayudarnos a preparar, vivir y celebrar mejor este tiempo\anote{id8}\label{fn8}.
	
	La piedad popular es sensible al tiempo de Adviento, sobre todo en cuanto memoria de la preparación a la venida del Mesías. Está sólidamente enraizada en el pueblo cristiano la conciencia de la larga espera que precedió a la venida del Salvador. Los fieles saben que Dios mantenía, mediante las profecías, la esperanza de Israel en la venida del Mesías.
	
	A la piedad popular no se le escapa, es más, subraya llena de estupor, el acontecimiento extraordinario por el que el Dios de la gloria se ha hecho niño en el seno de una mujer virgen, pobre y humilde. Los fieles son especialmente sensibles a las dificultades que la Virgen María tuvo que afrontar durante su embarazo y se conmueven al pensar que en la posada no hubo un lugar para José ni para María, que estaba a punto de dar a luz al Niño (cfr. Lc 2,7).
	
	Con referencia al Adviento han surgido diversas expresiones de piedad popular, que alientan la fe del pueblo cristiano y transmiten, de una generación a otra, la conciencia de algunos valores de este tiempo litúrgico.
	
	\subsection{La Corona de Adviento}
	
	La colocación de cuatro cirios sobre una corona de ramos verdes, que es costumbre sobre todo en los países germánicos y en América del Norte, se ha convertido en un símbolo del Adviento en los hogares cristianos.
	
	La Corona de Adviento, cuyas cuatro luces se encienden progresivamente, domingo tras domingo hasta la solemnidad de Navidad, es memoria de las diversas etapas de la historia de la salvación antes de Cristo y símbolo de la luz profética que iba iluminando la noche de la espera, hasta el amanecer del Sol de justicia (cfr. Mal 3,20; Lc 1,78).
	
	\subsection{Las Procesiones de Adviento}
	
	En el tiempo de Adviento se celebran, en algunas regiones, diversas procesiones, que son un anuncio por las calles de la ciudad del próximo nacimiento del Salvador (la \textquote{clara estrella} en algunos lugares de Italia), o bien representaciones del camino de José y María hacia Belén, y su búsqueda de un lugar acogedor para el nacimiento de Jesús (las \textquote{posadas} de la tradición española y latinoamericana).

	\subsection{Las \textquote{Témporas de invierno}}
	
	En el hemisferio norte, en el tiempo de Adviento se celebran las \textquote{témporas de invierno}. Indican el paso de una estación a otra y son un momento de descanso en algunos campos de la actividad humana. La piedad popular está muy atenta al desarrollo del ciclo vital de la naturaleza: mientras se celebran las \textquote{témporas de invierno}, las semillas se encuentran enterradas, en espera de que la luz y el calor del sol, que precisamente en el solsticio de invierno vuelve a comenzar su ciclo, las haga germinar.
	
	Donde la piedad popular haya establecido expresiones celebrativas del cambio de estación, consérvense y valórense como tiempo de súplica al Señor y de meditación sobre el significado del trabajo humano, que es colaboración con la obra creadora de Dios, realización de la persona, servicio al bien común, actualización del plan de la Redención.
	
	\subsection{La Virgen María en el Adviento}
	
	Durante el tiempo de Adviento, la Liturgia celebra con frecuencia y de modo ejemplar a la Virgen María: recuerda algunas mujeres de la Antigua Alianza, que eran figura y profecía de su misión; exalta la actitud de fe y de humildad con que María de Nazaret se adhirió, total e inmediatamente, al proyecto salvífico de Dios; subraya su presencia en los acontecimientos de gracia que precedieron el nacimiento del Salvador. También la piedad popular dedica, en el tiempo de Adviento, una atención particular a Santa María; lo atestiguan de manera inequívoca diversos ejercicios de piedad, y sobre todo las novenas de la Inmaculada y de la Navidad.
	
	Sin embargo, la valoración del Adviento \textquote{como tiempo particularmente apto para el culto de la Madre del Señor} no quiere decir que este tiempo se deba presentar como un \textquote{mes de María}.
	
	En los calendarios litúrgicos del Oriente cristiano, el periodo de preparación al misterio de la manifestación (Adviento) de la salvación divina (Teofanía) en los misterios de la Navidad-Epifanía del Hijo Unigénito de Dios Padre, tiene un carácter marcadamente mariano. Se centra la atención sobre la preparación a la venida del Señor en el misterio de la \emph{Deípara}. Para el Oriente, todos los misterios marianos son misterios cristológicos, esto es, referidos al misterio de nuestra salvación en Cristo. Así, en el rito copto durante este periodo se cantan las Laudes de María en los \emph{Theotokia}; en el Oriente sirio este tiempo es denominado \emph{Subbara}, esto es, Anunciación, para subrayar de esta manera su fisonomía mariana. En el rito bizantino se nos prepara a la Navidad mediante una serie creciente de fiestas y cantos marianos.
	
	La solemnidad de la Inmaculada (8 de Diciembre), profundamente sentida por los fieles, da lugar a muchas manifestaciones de piedad popular, cuya expresión principal es la novena de la Inmaculada. No hay duda de que el contenido de la fiesta de la Concepción purísima y sin mancha de María, en cuanto preparación fontal al nacimiento de Jesús, se armoniza bien con algunos temas principales del Adviento: nos remite a la larga espera mesiánica y recuerda profecías y símbolos del Antiguo Testamento, empleados también en la Liturgia del Adviento.
	
	Donde se celebre la Novena de la Inmaculada se deberían destacar los textos proféticos que partiendo del vaticinio de Génesis 3,15, desembocan en el saludo de Gabriel a la \textquote{llena de gracia} (Lc 1,28) y en el anuncio del nacimiento del Salvador (cfr. Lc 1,31-33).
	
	Acompañada por múltiples manifestaciones populares, en el Continente Americano se celebra, al acercarse la Navidad, la fiesta de Nuestra Señora de Guadalupe (12 de Diciembre), que acrecienta en buena medida la disposición para recibir al Salvador: María \textquote{unida íntimamente al nacimiento de la Iglesia en América, fue la Estrella radiante que iluminó el anunció de Cristo Salvador a los hijos de estos pueblos}.
	
	\subsection{La Novena de Navidad}
	
	La Novena de Navidad nació para comunicar a los fieles las riquezas de una Liturgia a la cual no tenían fácil acceso. La novena navideña ha desempeñado una función valiosa y la puede continuar desempeñando. Sin embargo en nuestros días, en los que se ha facilitado la participación del pueblo en las celebraciones litúrgicas, sería deseable que en los días 17 al 23 de Diciembre se solemnizara la celebración de las Vísperas con las \textquote{antífonas mayores} y se invitara a participar a los fieles. Esta celebración, antes o después de la cual podrían tener lugar algunos de los elementos especialmente queridos por la piedad popular, sería una excelente \textquote{novena de Navidad} plenamente litúrgica y atenta a las exigencias de la piedad popular. En la celebración de las Vísperas se pueden desarrollar algunos elementos, tal como está previsto (p. ej. homilía, uso del incienso, adaptación de las preces).
	
	\subsection{El Nacimiento}
	
	Como es bien sabido, además de las representaciones del pesebre de Belén, que existían desde la antigüedad en las iglesias, a partir del siglo XIII se difundió la costumbre de preparar pequeños nacimientos en las habitaciones de la casa, sin duda por influencia del \textquote{nacimiento} construido en Greccio por San Francisco de Asís, en el año 1223. La preparación de los mismos (en la cual participan especialmente los niños) se convierte en una ocasión para que los miembros de la familia entren en contacto con el misterio de la Navidad, y para que se recojan en un momento de oración o de lectura de las páginas bíblicas referidas al episodio del nacimiento de Jesús.
	
	\subsection{La piedad popular y el espíritu del Adviento}
	
	La piedad popular, a causa de su comprensión intuitiva del misterio cristiano, puede contribuir eficazmente a salvaguardar algunos de los valores del Adviento, amenazados por la costumbre de convertir la preparación a la Navidad en una \textquote{operación comercial}, llena de propuestas vacías, procedentes de una sociedad consumista.
	
	La piedad popular percibe que no se puede celebrar el Nacimiento del Señor si no es en un clima de sobriedad y de sencillez alegre, y con una actitud de solidaridad para con los pobres y marginados; la espera del nacimiento del Salvador la hace sensible al valor de la vida y al deber de respetarla y protegerla desde su concepción; intuye también que no se puede celebrar con coherencia el nacimiento del que \textquote{salvará a su pueblo de sus pecados} (Mt 1,21) sin un esfuerzo para eliminar de sí el mal del pecado, viviendo en la vigilante espera del que volverá al final de los tiempos.
	
	\newsection
	\section{Una reflexión sobre el Adviento}
	
	\src{cf. Benedicto XVI, papa, \emph{Homilía}, 28 de noviembre de 2009.\cite{BenedictoXVI_Homilia_20091128}}
	
	\ltr{E}{l} término Adviento significa \textquote{venida}, \emph{parousia}, en latín \emph{adventus}, de donde viene el término Adviento (cf. \emph{1 Ts} 5, 23). Reflexionemos brevemente sobre el significado de esta palabra, que se puede traducir por \textquote{presencia}, \textquote{llegada}, \textquote{venida}. En el lenguaje del mundo antiguo era un término técnico utilizado para indicar la llegada de un funcionario, la visita del rey o del emperador a una provincia. Pero podía indicar también la venida de la divinidad, que sale de su escondimiento para manifestarse con fuerza, o que se celebra presente en el culto. 
	
	Los cristianos adoptaron la palabra \textquote{Adviento} para expresar su relación con Jesucristo: Jesús es el Rey, que ha entrado en esta pobre \textquote{provincia} denominada tierra para visitar a todos; invita a participar en la fiesta de su Adviento a todos los que creen en él, a todos los que creen en su presencia en la asamblea litúrgica. Con la palabra \emph{adventus} se quería decir substancialmente: Dios está aquí, no se ha retirado del mundo, no nos ha dejado solos. Aunque no podamos verlo o tocarlo, como sucede con las realidades sensibles, él está aquí y viene a visitarnos de múltiples maneras.
	
	Por lo tanto, el significado de la expresión \textquote{Adviento} comprende también el de \emph{visitatio}, que simplemente quiere decir \textquote{visita}; en este caso se trata de una visita de Dios: él entra en mi vida y quiere dirigirse a mí. En la vida cotidiana todos experimentamos que tenemos poco tiempo para el Señor y también poco tiempo para nosotros. Acabamos dejándonos absorber por el \textquote{hacer}. ¿No es verdad que con frecuencia es precisamente la actividad lo que nos domina, la sociedad con sus múltiples intereses lo que monopoliza nuestra atención? ¿No es verdad que se dedica mucho tiempo al ocio y a todo tipo de diversiones? A veces las cosas nos \textquote{arrollan}.
	
	El Adviento, este tiempo litúrgico fuerte (\ldots{}) nos invita a detenernos, en silencio, para captar una presencia. Es una invitación a comprender que los acontecimientos de cada día son gestos que Dios nos dirige, signos de su atención por cada uno de nosotros. ¡Cuán a menudo nos hace percibir Dios un poco de su amor! Escribir ---por decirlo así--- un \textquote{diario interior} de este amor sería una tarea hermosa y saludable para nuestra vida. El Adviento nos invita y nos estimula a contemplar al Señor presente. La certeza de su presencia, ¿no debería ayudarnos a ver el mundo de otra manera? ¿No debería ayudarnos a considerar toda nuestra existencia como \textquote{visita}, como un modo en que él puede venir a nosotros y estar cerca de nosotros, en cualquier situación?
	
	Otro elemento fundamental del Adviento es la espera, una espera que es al mismo tiempo esperanza. El Adviento nos impulsa a entender el sentido del tiempo y de la historia como \textquote{\emph{kairós}}, como ocasión propicia para nuestra salvación. Jesús explicó esta realidad misteriosa en muchas parábolas: en la narración de los siervos invitados a esperar el regreso de su dueño; en la parábola de las vírgenes que esperan al esposo; o en las de la siembra y la siega. En la vida, el hombre está constantemente a la espera: cuando es niño quiere crecer; cuando es adulto busca la realización y el éxito; cuando es de edad avanzada aspira al merecido descanso. Pero llega el momento en que descubre que ha esperado demasiado poco si, fuera de la profesión o de la posición social, no le queda nada más que esperar. La esperanza marca el camino de la humanidad, pero para los cristianos está animada por una certeza: el Señor está presente a lo largo de nuestra vida, nos acompaña y un día enjugará también nuestras lágrimas. Un día, no lejano, todo encontrará su cumplimiento en el reino de Dios, reino de justicia y de paz.
	
	Existen maneras muy distintas de esperar. Si el tiempo no está lleno de un presente cargado de sentido, la espera puede resultar insoportable; si se espera algo, pero en este momento no hay nada, es decir, si el presente está vacío, cada instante que pasa parece exageradamente largo, y la espera se transforma en un peso demasiado grande, porque el futuro es del todo incierto. En cambio, cuando el tiempo está cargado de sentido, y en cada instante percibimos algo específico y positivo, entonces la alegría de la espera hace más valioso el presente. {[}\ldots{}{]} Vivamos intensamente el presente, donde ya nos alcanzan los dones del Señor, vivámoslo proyectados hacia el futuro, un futuro lleno de esperanza. De este modo, el Adviento cristiano es una ocasión para despertar de nuevo en nosotros el sentido verdadero de la espera, volviendo al corazón de nuestra fe, que es el misterio de Cristo, el Mesías esperado durante muchos siglos y que nació en la pobreza de Belén. Al venir entre nosotros, nos trajo y sigue ofreciéndonos el don de su amor y de su salvación. Presente entre nosotros, nos habla de muchas maneras: en la Sagrada Escritura, en el año litúrgico, en los santos, en los acontecimientos de la vida cotidiana, en toda la creación, que cambia de aspecto si detrás de ella se encuentra él o si está ofuscada por la niebla de un origen y un futuro inciertos.
	
	Nosotros podemos dirigirle la palabra, presentarle los sufrimientos que nos entristecen, la impaciencia y las preguntas que brotan de nuestro corazón. Estamos seguros de que nos escucha siempre. Y si Jesús está presente, ya no existe un tiempo sin sentido y vacío. Si él está presente, podemos seguir esperando incluso cuando los demás ya no pueden asegurarnos ningún apoyo, incluso cuando el presente está lleno de dificultades.
	
	{[}\ldots{}{]} El Adviento es el tiempo de la presencia y de la espera de lo eterno. Precisamente por esta razón es, de modo especial, el tiempo de la alegría, de una alegría interiorizada, que ningún sufrimiento puede eliminar. La alegría por el hecho de que Dios se ha hecho niño. Esta alegría, invisiblemente presente en nosotros, nos alienta a caminar confiados. La Virgen María, por medio de la cual nos ha sido dado el Niño Jesús, es modelo y sostén de este íntimo gozo. Que ella, discípula fiel de su Hijo, nos obtenga la gracia de vivir este tiempo litúrgico vigilantes y activos en la espera. Amén.
\end{introstyle}