\chapter{Domingo II de Adviento (A)}

\section{Lecturas}

\rtitle{PRIMERA LECTURA}

\rbook{Del libro del profeta Isaías} \rred{11, 1-10}

\rtheme{Juzgará a los pobres con justicia}

\begin{readprose}
	En aquel día brotará un renuevo del tronco de Jesé, 
	y de su raíz florecerá un vástago.
	
	Sobre él se posará el espíritu del Señor: 
	espíritu de sabiduría y entendimiento, 
	espíritu de consejo y fortaleza, 
	espíritu de ciencia y temor del Señor.
	
	Lo inspirará el temor del Señor. 
	
	No juzgará por apariencias 
	ni sentenciará de oídas; 
	juzgará a los pobres con justicia, 
	sentenciará con rectitud a los sencillos de la tierra; 
	pero golpeará al violento con la vara de su boca, 
	y con el soplo de sus labios hará morir al malvado.
	
	La justicia será ceñidor de su cintura, 
	y la lealtad, cinturón de sus caderas.
	
	Habitará el lobo con el cordero, 
	el leopardo se tumbará con el cabrito, 
	el ternero y el león pacerán juntos: 
	un muchacho será su pastor.
	
	La vaca pastará con el oso, 
	sus crías se tumbarán juntas; 
	el león como el buey, comerá paja.
	
	El niño de pecho retoza junto al escondrijo de la serpiente, 
	y el recién destetado extiende la mano 
	hacia la madriguera del áspid.
	
	Nadie causará daño ni estrago 
	por todo mi monte santo: 
	porque está lleno el país del conocimiento del Señor, 
	como las aguas colman el mar.
	
	Aquel día, la raíz de Jesé será elevada 
	como enseña de los pueblos: 
	se volverán hacia ella las naciones 
	y será gloriosa su morada.
\end{readprose}

\rtitle{SALMO RESPONSORIAL}

\rbook{Salmo} \rred{71, 1bc-2. 7-8. 12-13. 17}

\rtheme{Que en sus días florezca la justicia, y la paz abunde eternamente}

\begin{psbody}
	Dios mío, confía tu juicio al rey, 
	tu justicia al hijo de reyes, 
	para que rija a tu pueblo con justicia, 
	a tus humildes con rectitud.
	
	En sus días florezca la justicia 
	y la paz hasta que falte la luna; 
	domine de mar a mar, 
	del Gran Río al confín de la tierra. 
	
	Él librará al pobre que clamaba, 
	al afligido que no tenía protector; 
	él se apiadará del pobre y del indigente, 
	y salvará la vida de los pobres. 
	
	Que su nombre sea eterno, 
	y su fama dure como el sol; 
	él sea la bendición de todos los pueblos, 
	y lo proclamen dichoso todas las razas de la tierra.
\end{psbody}

\rtitle{SEGUNDA LECTURA} 

\rbook{De la carta del apóstol san Pablo a los Romanos} \rred{15, 4-9}

\rtheme{Cristo salva a todos los hombres}

\begin{scripture}
	Hermanos: 
	
	Todo lo que se escribió en el pasado, se escribió para enseñanza nuestra, a fin de que a través de nuestra paciencia y del consuelo que dan las Escrituras mantengamos la esperanza. 
	
	Que el Dios de la paciencia y del consuelo os conceda tener entre vosotros los mismos sentimientos, según Cristo Jesús; de este modo, unánimes, a una voz, glorificaréis al Dios y Padre de nuestro Señor Jesucristo. 
	
	Por eso, acogeos mutuamente, como Cristo os acogió para gloria de Dios. Es decir, Cristo se hizo servidor de la circuncisión en atención a lafidelidad de Dios, para llevar a cumplimiento las promesas hechas a lospatriarcas y, en cuanto a los gentiles, para que glorifiquen a Dios por su misericordia; como está escrito: \\ \textquote{Por esto te alabaré entre los gentiles \\y cantaré para tu nombre}.
\end{scripture}

\rtitle{EVANGELIO}

\rbook{Del Santo Evangelio según san Mateo} \rred{3, 1-12}

\rtheme{Convertíos, porque está acerca el reino de los cielos}

\begin{scripture}
	
	Por aquellos días, Juan el Bautista se presentó en el desierto de Judea,
	predicando: 
	
	\>{Convertíos, porque está cerca el reino de los cielos}.
	
	Este es el que anunció el profeta Isaías diciendo:
	
	\>{Voz del que grita en el desierto: \\\textquote{Preparad el camino del Señor, \\allanad sus senderos}}. 
	
	Juan llevaba un vestido de piel de camello, con una correa de cuero a la cintura, y se alimentaba de saltamontes y miel silvestre. 
	
	Y acudía a él toda la gente de Jerusalén, de Judea y de la comarca del Jordán; confesaban sus pecados y él los bautizaba en el Jordán. 
	
	Al ver que muchos fariseos y saduceos venían a que los bautizara, les dijo: 
	
	\>{¡Raza de víboras!, ¿quién os ha enseñado a escapar del castigo inminente? \\Dad el fruto que pide la conversión. \\Y no os hagáis ilusiones, pensando: \textquote{Tenemos por padre a Abrahán}, pues os digo que Dios es capaz de sacar hijos de Abrahán de estas piedras. \\Ya toca el hacha la raíz de los árboles, y todo árbol que no dé buen fruto será talado y echado al fuego. \\Yo os bautizo con agua para que os convirtáis; pero el que viene detrás de mí es más fuerte que yo y no merezco ni llevarle las sandalias. \\Él os bautizará con Espíritu Santo y fuego. \\Él tiene el bieldo en la mano: aventará su parva, reunirá su trigo en el granero y quemará la paja en una hoguera que no se apaga}.
\end{scripture}

\newsection

\section{Comentario Patrístico} 

\subsection{San Agustín, obispo}

\ptheme{Convertíos, porque está cerca el Reino de los cielos}

\src{Sermón 109, 1: PL 38, 636.\cite{Agustin_PL038_0636}}

\begin{body}
	\ltr{H}{emos} escuchado el evangelio y en el evangelio al Señor descubriendo la ceguera de quienes son capaces de interpretar el aspecto del cielo, pero son incapaces de discernir el tiempo de la fe en un reino de los cielos que está ya llegando. Les decía esto a los judíos, pero sus palabras nos afectan también a nosotros. Y el mismo Jesucristo comenzó así la predicación de su evangelio: \emph{Convertíos, porque está cerca el Reino de los cielos}. Igualmente, Juan el Bautista, su Precursor, comenzó así: \emph{Convertíos, porque está cerca el Reino de los cielos}. Y ahora corrige el Señor a los que se niegan a convertirse, próximo ya el Reino de los cielos. \emph{El Reino de los cielos} ---como él mismo dice--- \emph{no vendrá espectacularmente}. Y añade: \emph{El Reino de Dios está dentro de vosotros}. 
	
	Que cada cual reciba con prudencia las admoniciones del preceptor, si no quiere perder la hora de misericordia del Salvador, misericordia que se otorga en la presente coyuntura, en que al género humano se le ofrece el perdón. Precisamente al hombre se le brinda el perdón para que se convierta y no haya a quien condenar. Eso lo ha de decidir Dios cuando llegue el fin del mundo; pero de momento nos hallamos en el tiempo de la fe. Si el fin del mundo encontrará o no aquí a alguno de nosotros, lo ignoro; posiblemente no encuentre a ninguno. Lo cierto es que el tiempo de cada uno de nosotros está cercano, pues somos mortales. Andamos en medio de peligros. Nos asustan más las caídas que si fuésemos de vidrio. ¿Y hay algo más frágil que un vaso de cristal? Y sin embargo se conserva y dura siglos. Y aunque pueda temerse la caída de un vaso de cristal, no hay miedo de que le afecte la vejez o la fiebre. 
	
	Somos, por tanto, más frágiles que el cristal porque debido indudablemente a nuestra propia fragilidad, cada día nos acecha el temor de los numerosos y continuos accidentes inherentes a la condición humana; y aunque estos temores no lleguen a materializarse, el tiempo corre: y el hombre que puede evitar un golpe, ¿podrá también evitar la muerte? Y si logra sustraerse a los peligros exteriores, ¿logrará evitar asimismo los que vienen de dentro? Unas veces son los virus que se multiplican en el interior del hombre, otras es la enfermedad que súbitamente se abate sobre nosotros; y aun cuando logre verse libre de estas taras, acabará finalmente por llegarle la vejez, sin moratoria posible.
\end{body}

\newsection

\section{Homilías}

\subsection{San Juan Pablo II, papa}

\subsubsection{Homilía (1983):} 

\src{Visita Pastoral a la Parroquia Romana de Santa Francesca Saverio Cabrini, \\4 de diciembre de 1983.}

\begin{body}
	\emph{Queridos fieles:} 
	
	\ltr[1. ]{E}{ste} segundo domingo de Adviento se desarrolla íntegramente sobre los temas de la venida de Cristo y la preparación necesaria para este maravilloso acontecimiento. 
	
	En el centro de la liturgia está la persona de \textbf{Juan el Bautista}. Mateo el evangelista lo describe como un hombre de intensa oración, de austera penitencia, de profunda fe: de hecho, es el último de los profetas del Antiguo Testamento, que actúa como un pasaje al Nuevo, indicando que es Jesús el Mesías esperado por los judíos. En las orillas del río Jordán, Juan el Bautista confiere el bautismo de penitencia: mucha gente \textquote{acudía a él desde Jerusalén, de toda Judea y de la zona adyacente al Jordán; y, confesando sus pecados, fueron bautizados por él} (\emph{Mt} 3, 5-6). Este bautismo no es un simple rito de adhesión, sino que indica y exige el arrepentimiento de los pecados y un sentido sincero de expectativa del Mesías. 
	
	Y Juan enseña. Predica la conversión: \textquote{convertíos, porque el Reino de los cielos está cerca}. 
	
	Juan enseña. Y de acuerdo con la profecía de Isaías, \textquote{endereza los caminos} para el Señor (cf. \emph{Mt} 3, 1-3). 
	
	2. Incluso hoy nos resuena esta palabra. 
	
	¿Quién es ese Señor que ha de venir? De sus propias palabras podemos calificar la persona, misión y autoridad del Mesías. 
	
	Juan el Bautista ante todo enmarca claramente su \textquote{persona}: \textquote{él --- dice el Bautista --- es más poderoso que yo y ni siquiera soy digno de desatar la correa de sus sandalias} (\emph{Mt} 3, 11). Con estas palabras típicas orientales reconoce la distancia infinita que se abre entre él y el que está por venir, y subraya también su tarea de preparación inmediata para el gran acontecimiento. 
	
	Luego indica la misión del Mesías: \textquote{Él os bautizará en Espíritu Santo y fuego} (\emph{Mt} 3, 12). Es la primera vez que, después del anuncio del ángel a María, aparece la impresionante dicción \textquote{Espíritu Santo}, que luego será parte de la enseñanza trinitaria fundamental de Jesús. Juan el Bautista, divinamente iluminado, anuncia que Jesús, el Mesías, continuará impartiendo el Bautismo, pero este rito dará la \textquote{gracia} de Dios, el Espíritu Santo, entendido bíblicamente como un \textquote{fuego} místico, que anula (quema) el pecado insertándonos en la vida divina misma (enciende la llama de su amor). 
	
	Finalmente, el Bautista aclara la autoridad del Mesías: \textquote{Tiene el aventador en la mano, limpiará su era y recogerá su grano en el granero, pero quemará la paja con un fuego inextinguible} (\emph{Mt} 3, 12). Según la palabra de la enseñanza de Juan, el que vendrá es el \textquote{juez de conciencias}; es decir, es él quien establece lo bueno y lo malo (el grano y la paja), la verdad y el error; es él quien determina qué árboles dan buenos frutos y cuáles dan frutos malos y cuáles deben ser cortados y quemados. Con estas afirmaciones Juan Bautista anuncia la \textquote{divinidad} del Mesías, porque sólo Dios puede ser el árbitro supremo del bien, indicar con absoluta certeza el camino positivo de la conducta moral, juzgar las conciencias, premiar o condenar. 
	
	De ahí la necesidad de prepararse para la venida del Mesías. Sin duda, la Navidad es un día de gran y serena alegría, incluso desde fuera; pero es sobre todo un acontecimiento sobrenatural y decisivo, para el que es necesaria una seria preparación moral: \textquote{¡Preparad el camino del Señor! ¡Enderezad sus senderos!}. En palabras de Juan, es toda la herencia profunda de la Antigua Alianza. 
	
	3. Pero, al mismo tiempo, se abre ante ellos la Nueva Alianza: en el que ha de venir \textquote{todo hombre verá la salvación de Dios} (\emph{Lc} 3, 6). El que viene, Cristo, es enviado \textquote{para reuniros para la gloria de Dios} (\emph{Rom} 15, 7). Viene para demostrar la \textquote{veracidad de Dios, para cumplir las promesas de los Padres \ldots{}} (\emph{Rom} 15, 8); viene a revelar que el Señor es \textquote{el Dios de perseverancia y consolación} (\emph{Rom} 15, 5); viene a \textquote{acogeros para la gloria de Dios} (\emph{Rom} 15, 7). 
	
	Y, por tanto, Aquel que viene debe ocuparse de que \textquote{os acojáis unos a otros} (\emph{Rom} 15, 7). De hecho, indica la conducta moral verdadera y auténtica, que consiste en dar gloria a Dios Padre, siguiendo su ejemplo y con los mismos sentimientos, y en amar al prójimo. San Pablo, escribiendo a los Romanos, tenía en mente tanto a los conversos del judaísmo como a los del paganismo; pero para todos habla del compromiso de \textquote{acoger}: la Palabra de Dios, que viene, debe hacer que \textquote{tengáis los mismos sentimientos los unos hacia los otros, siguiendo a Cristo Jesús \ldots{}} (cf. \emph{Rm} 15, 5) para que \textquote{con un solo corazón y una sola voz den gloria a Dios Padre} (cf. \emph{Rm} 15, 6). 
	
	4. Así, pues, el \textquote{enderezar los caminos}, predicado por \textbf{Juan Bautista}, se hace a la luz de la enseñanza de san Pablo en la \textbf{Carta a los Romanos}, acogiendo todo el programa mesiánico del Evangelio: el programa de adoración a Dios --- ¡la gloria! --- por el amor del hombre, el amor mutuo. Con este espíritu, la Iglesia anuncia el Adviento como dimensión continua de la existencia del hombre hacia Dios: hacia ese Dios \textquote{que es, que era y que ha de venir} (\emph{Ap} 1, 4). 
	
	Esta dimensión esencial de la existencia cristiana del hombre corresponde a la \textquote{preparación} enseñada por la liturgia de hoy. El hombre debe volver siempre al corazón, a la conciencia, para existir en la perspectiva de la \textquote{Venida}. 
	
	Para cumplir con este requisito, el cristiano también debe ser sensible a la acción del Espíritu Santo: el que viene, viene en el Espíritu Santo, como anunció \textbf{Isaías}: \textquote{Sobre él reposará el espíritu del Señor, espíritu de sabiduría e inteligencia espíritu de consejo y fortaleza, espíritu de conocimiento y temor del Señor} (\emph{Is} 11, 2). Con el Mesías y con la presencia del Espíritu Santo, la justicia y la paz entran en la historia humana como dones del reino de Dios: abre así la perspectiva de la reconciliación \textquote{cósmica} en toda la creación, en el hombre y en el mundo, que había sido perdida debido al pecado. \textquote{Ven, Señor, rey de justicia y de paz}: rezamos juntos en el \textbf{salmo responsorial}. 
	
	5. {[}¡Queridos fieles de la parroquia de Santa Francesca Saverio Cabrini! Estoy muy feliz de estar hoy con ustedes, de encontrarme con ustedes y de haber podido realizar con ustedes esta meditación sobre las lecturas litúrgicas, tan llenas de luz sobrenatural, afirmaciones consoladoras y pautas concretas para la vida diaria.{]} 
	
	{[} \ldots{}{]} 
	
	6. (\ldots{}) \textquote{Preparad el camino del Señor}. Este mensaje es actual, siempre y para todos. De hecho, todos vivimos en la dimensión del advenimiento de Dios. Nuestra vida es una \textquote{preparación} continua. 
	
	Finalmente, ruego a la Madre de Aquel que debe venir, para que los bienes mesiánicos del reino de Dios, la justicia y la paz, sean compartidos por vosotros en Jesucristo, su divino Hijo. 
	
	Sí, ven, Señor, rey de justicia y paz. ¡Por María! Amén.
\end{body}

\subsubsection{Homilía (1986):}

\src{Visita Pastoral a la Parroquia Romana de Santa María, Reina de los Mártires, \\7 de diciembre de 1986.} 

\begin{body} 
	1. \textquote{\emph{¡Regem venturum Dominum come adoremus!}}. 
	
	\ltr{L}{a} liturgia del Segundo Domingo de Adviento nos permite mirar al Mesías, cuya venida Israel --- el pueblo de Dios de la Antigua Alianza --- ha esperado. Veamos primero con la mirada profética de \textbf{Isaías} (en el siglo VIII a. C.): ¿quién será el Mesías esperado? Será un gran Maestro, el que vendrá en el poder del Espíritu Santo, lleno de sus dones: \textquote{El Espíritu del Señor reposará sobre él \ldots{}} (\emph{Is} 11, 2). 
	
	Al mismo tiempo, debe \textquote{brotar de la raíz de Jesé}, es decir, del linaje de David, como \textquote{un brote de su tronco}. El Mesías, por tanto, en la profecía de Isaías, aparece como un hombre, descendiente de David, tan animado por la fuerza del Espíritu de Dios, que su misión nos permite profundizar decisivamente en el misterio de este mismo Espíritu. De hecho, como dije en mi encíclica \emph{Dominum et Vivificantem} (n. 15), este texto bíblico \textquote{es importante para toda la pneumatología del Antiguo Testamento, porque casi constituye un puente entre el antiguo concepto bíblico de \textquote{espíritu}, entendido en primer lugar como \textquote{aliento carismático}, y el \textquote{Espíritu} como persona y como don, don para la persona}. 
	
	Como se desprende de otros pasajes del profeta, el Mesías, el \textquote{consagrado y ungido del Espíritu}, estará tan lleno de este Espíritu que él mismo, junto con el Padre, tendrá el poder de \textquote{enviar el Espíritu} (cf. \emph{Jn} 15, 26; 16, 7), de \textquote{conceder este Espíritu a todo el pueblo} \emph{(Dominum et V}ivificantem, 15). 
	
	2. A la luz de las palabras del \textbf{salmista} (alrededor del Siglo VII, Salmo 71), el Mesías, que ha de venir, será rey de paz basado en la justicia, llevando la liberación a los \textquote{pobres} y a los que padecen la opresión en múltiples formas. En este punto el Salmo se encuentra con el pasaje de Isaías, incluso si el profeta expresa esta verdad sobre el Mesías de otra manera: mientras que Isaías ve aquí al Mesías como un hombre lleno del Espíritu Santo, un \textquote{hombre sabio}, fuerte y justo, el salmista subraya el \textquote{reinado} del Mesías, acentuando así la justicia y eficacia de su gobierno universal y eterno. 
	
	Sea como fuere, vemos en estos pasajes de la Escritura la idea de un futuro Mesías liberando a los pobres y oprimidos. Y este es efectivamente uno de los aspectos esenciales de la misión de Cristo. Como se dice, de hecho, en la reciente \textquote{Instrucción sobre la libertad y liberación cristianas} (n. 51): \textquote{Por la fuerza de su misterio pascual, Cristo nos ha hecho libres. Con su perfecta obediencia en la cruz y con la gloria de la resurrección, el Cordero de Dios quitó el pecado del mundo y nos abrió el camino de la liberación definitiva}. 
	
	3. Para \textbf{Juan el Bautista} \textquote{en el desierto de Judea}, el Mesías, cuya venida es inmediatamente precedida y preparada en Israel por el profeta, es el que \textquote{bautizará en Espíritu Santo y fuego} (\emph{Mt} 3, 11). En el corazón de la visión del Bautista está la necesidad de la conversión moral debida a la proximidad del reino celestial: \textquote{Todo árbol que no da buen fruto es cortado y echado al fuego} (\emph{Mt} 3, 10). Juan advierte a los hombres de su tiempo y de todos los tiempos que es imposible obtener la salvación sin \textquote{dar frutos dignos de conversión} (\emph{Mt} 3, 8). Estimula vigorosamente las conciencias a la renovación de las costumbres recordándoles su responsabilidad ante Dios e inspirándoles un sano temor. Aunque tengamos por padre a Abraham --- nos recuerda el profeta --- esto no es motivo suficiente para estar seguros: para que la misericordia de Dios tenga su eficacia en nosotros, debemos corresponderle con las obras de arrepentimiento, justicia y caridad. Sólo bajo estas condiciones los hombres pueden recibir verdaderamente el \textquote{fuego del Espíritu Santo} contenido en el bautismo cristiano. 
	
	4. ¿Quién es el Mesías? Para san Pablo, que se refiere a lo que \textquote{estaba escrito antes} (\emph{Rm} 15, 4), el Mesías es Cristo, es Él, Cristo el que \textquote{os acogió para gloria de Dios \ldots{} El que se hizo siervo de los circuncisos en favor de la veracidad de Dios, para cumplir las promesas hechas a los patriarcas \ldots{}} (\emph{Rm} 15, 7-8), mientras que a las naciones paganas, que no conocían las profecías del Antiguo Testamento, se manifestó en la gratuidad e imprevisibilidad de su misericordia. 
	
	Cristo se ha mostrado de diferentes maneras a ambos, aunque es el mismo \textquote{servidor} de todos. A Israel, que estaba esperando al Mesías, se mostró como la plenitud de esa verdad prefigurada por las profecías: por eso mostró fidelidad a Dios. Y a los pueblos paganos, que no lo esperaban, a esos paganos que --- como dice Pablo --- \textquote{no tienen la ley} (\emph{Rm} 2, 14-15), el Mesías se ha mostrado como el cumplimiento superior e inesperado de esa ley que \textquote{está escrita en sus corazones}: la ley moral de la conciencia natural. 
	
	5. Hoy miramos a Cristo Mesías, cuya venida renueva la Iglesia cada año en la liturgia del Adviento; miremos con los ojos del profeta, el salmista, el Bautista y finalmente el Apóstol de los gentiles, que también {[}aquí{]} en Roma proclamó la buena nueva de Cristo. Miremos juntos con los ojos de los hombres {[}del siglo XX, que se acercan al final del segundo milenio después de Cristo{]}. Miremos con los ojos de la fe la acogida de esta invocación de Adviento: \textquote{Preparad el camino del Señor, enderezad sus sendas} (\emph{Mt} 3, 3). Esta invocación nunca pierde su relevancia. 
	
	Miremos junto con la comunidad, {[}que constituye esta parroquia romana de Santa Maria Regina dei Martiri. Los mártires: aquellos que en diferentes épocas y lugares dieron un testimonio eminente de la venida de Cristo. Vuestra parroquia lleva su nombre. Y lleva el nombre de la Madre de Cristo como Reina de los mártires.{]} 
	
	6. (\ldots{}) Os exhorto a aceptar vuestro compromiso de que la parroquia sea verdadera y cada vez más la \textquote{familia de Dios}, en la conciencia y el cuidado de ser \textquote{comunión}, de ser Iglesia. Para ello es necesario fomentar la corresponsabilidad de todos, para que cada uno se sienta desafiado e involucrado según los dones recibidos de Dios y sus propias posibilidades. 
	
	7. \textquote{¡Regem venturum Dominum come adoremus!} Quiero que {[}esta celebración nos ayude a{]} preparar la venida del Señor, a fin de que vivamos con mayor plenitud en el encuentro con Aquel que es el Rey de la paz y la justicia y que, ungido con el Espíritu Santo en la plenitud de sus dones, viene a \textquote{bautizarnos en Espíritu Santo y fuego}: para renovar en cada uno la gracia recibida en ese primer sacramento por el cual nos hacemos cristianos . . Para que --- todos y cada uno --- salgamos al encuentro de Aquel que \textquote{os acogió para gloria de Dios}. Y apoyándonos en él, \textquote{nos acojamos unos a otros} (\emph{Rom} 15, 7) en el espíritu del mandamiento evangélico del amor. 
	
	\emph{¡Que la Madre de Cristo, Reina de los mártires, una vez más os acerque a Cristo en el misterio del Nacimiento del Señor!}
\end{body}

\subsubsection{Homilía (1989):}

\src{Canonización del Hermano Mutien-Marie Wiaux, \\10 de diciembre de 1989.} 

\begin{body}
	1. \textquote{¡Convertíos porque el Reino de los Cielos está cerca!} (\emph{Mt} 3, 2). 
	
	\ltr{E}{n} este tiempo de Adviento, la Iglesia vuelve a dirigirnos la invitación de Juan Bautista. ¡Se lo dirige a cada uno de nosotros! Juan, el precursor, se retiró en el desierto, con austeridad. No tiene otra tarea que preparar el camino del Señor. Se escucha su voz y llega la multitud. Por invitación de Juan, se bautizan reconociendo sus pecados. Su camino es el de la conversión. 
	
	Con Juan, estas multitudes preparan el camino del Señor. De hecho, ¡el Reino de los Cielos ya está cerca! Juan anuncia: \textquote{El que viene después de mí es más poderoso que yo} (cf. \emph{Mt} 3, 11). El último y más grande de los profetas anuncia al Mesías, el retoño brotado de las raíces de Jesé, que el \textbf{profeta Isaías} estaba esperando. 
	
	2. Cuando hoy escuchamos a los profetas, cuando en este día escuchamos a Juan el Bautista, que nos conduce por el camino del Mesías, no percibimos sólo el eco de una palabra antigua. Es la Iglesia de Cristo la que vuelve a cada generación, a esta generación {[}al final del segundo milenio{]}, para decir que el Reino de los Cielos ya está cerca en el Mesías anunciado por los profetas. 
	
	Viene entre nosotros Aquel sobre quien \textquote{reposa el Espíritu del Señor, espíritu de sabiduría e inteligencia, espíritu de consejo y fortaleza, espíritu de conocimiento y temor del Señor \ldots{}} (\emph{Is} 11, 2). 
	
	Si se nos pide conversión, es para acoger al que viene, para recibir los dones de la justicia y la paz, para convertirnos en constructores de paz, ya que, como dice el \textbf{profeta Isaías}: \textquote{Juzgará a los pobres con justicia y tomará decisiones justas para los oprimidos de la tierra} (\emph{Is} 11, 4). Para prepararse para su venida, para convertirse en sus discípulos, es necesario dejarse transformar por su justicia y guiarse por su sabiduría. Entendemos que \textquote{mediante el espíritu de sabiduría y entendimiento} (\emph{Is} 11, 2), el Mesías descubre nuestro pecado y nos invita a conformarnos a la ley del amor y la verdad de su Reino. 
	
	3. El profeta también dice del Mesías que \textquote{la justicia será el ceñidor de su cintura} (\emph{Is} 11, 5). Inaugurará un reino de paz: \textquote{El lobo morará junto con el cordero \ldots{}, el buey y el cachorro de león pastarán juntos y un niño los guiará. (\ldots{}) No volverán a obrar inicuamente ni saquearán todo mi santo monte} (\emph{Is} 11, 6. 9). 
	
	¿Es esto una ilusión, ya que los conflictos interminables en tantas partes del mundo parecen hacer vana esta profecía? ¡Sería así si esta palabra no viniera de Dios, si no se dirigiera a las conciencias, si no fuera ella misma fuente de justicia, justicia según Dios, justicia de Dios! El mensaje de Isaías, el ardiente llamamiento del precursor y luego la misma venida de Jesús dan frutos de paz en el corazón y en las acciones de quienes se dejan convertir. 
	
	\textbf{Juan el Bautista} proclama la urgencia de la conversión: \textquote{El hacha ya está puesta a la raíz de los árboles} (\emph{Mt} 3, 10). Escuchemos con atención esta advertencia: toda persona está llamada a producir buenos frutos de justicia y paz; cada momento, cada acción tiene su importancia según el Reino que está por venir y cuyo espíritu no debe traicionar. Todo hombre, por modesto que sea, es insustituible en la familia humana; el gesto más humilde inspirado por el amor da gloria a Dios.
	
	{[} \ldots{}{]}
	
	6. (\ldots{}) Las palabras de \textbf{San Pablo} que la Iglesia nos hace escuchar en este segundo domingo de Adviento nos recuerdan que \textquote{todo lo que se escribió en el pasado, se escribió para enseñanza nuestra, a fin de que a través de nuestra paciencia y del consuelo que dan las Escrituras mantengamos la esperanza} (\emph{Rom} 15, 4). {[}Que este tiempo nos ayude a mostrar{]} perseverancia y valentía y a apoyarnos en el poder de la Palabra de Dios para enfrentar los desafíos de cada una de nuestras vidas, los desafíos de cada una de nuestras familias y comunidades.
	
	San Pablo añadía: \textquote{Que el Dios de la paciencia y del consuelo os conceda tener entre vosotros los mismos sentimientos, según Cristo Jesús; de este modo, unánimes, a una voz, glorificaréis al Dios y Padre de nuestro Señor Jesucristo} (\emph{Rm} 15, 5). (\ldots{}) plenamente imbuidos del espíritu de Cristo, recibiremos mejor al Salvador que viene entre nosotros si, unidos, recibimos la Palabra y actuamos en comunión en el mismo Espíritu y compartimos los mismos dones. \textquote{Por eso, acogeos mutuamente, como Cristo os acogió para gloria de Dios} (\emph{Rm} 15, 7-9). Vivir el Adviento es preparar los días en los que \textquote{florecerá la justicia}, los días en los que conoceremos \textquote{la abundancia de la paz} (cf. \emph{Sal} 72, 7), los días en los que \textquote{todas las razas de la tierra serán bendecidas} (\emph{Sal} 72, 17).
	
	7. Hermanos y hermanas, {[}que este tiempo nos ayude a{]} fortalecer el valor de la esperanza en nosotros. Que se nos conceda allanar el camino del Señor y el camino del hombre, que es el camino de la Iglesia. Que se nos dé la capacidad de producir frutos que expresen una conversión real, de anunciar sin cesar el Reino de los cielos ahora cercano en Aquel que bautiza en el Espíritu Santo (cf. \emph{Mt} 3, 2-11). 
	
	Y así podremos \textquote{con una sola alma y una sola voz dar gloria a Dios, Padre de nuestro Señor Jesucristo} (cf. \emph{Rm} 15, 6).
\end{body}

\subsubsection{Homilía (1992):}

\src{Visita Pastoral a la Parroquia Romana de Santa María del Buen Consejo, \\6 de diciembre de 1992.}

\begin{body}
	1. \textquote{Preparad el camino del Señor, enderezad sus sendas} (\emph{Mt} 3, 3).
	
	\ltr{Q}{ueridos} hermanos y hermanas (\ldots{}) El tiempo litúrgico de Adviento llega hoy a su segunda semana, guiándonos hacia el encuentro con el recién nacido de Belén, Jesús Salvador. Acabamos de escuchar un pasaje del \textbf{Evangelio de Mateo}, que nos habla de la predicación de Juan el Bautista, enviado para preparar los caminos del Señor en el corazón de los hombres de su tiempo. La liturgia nos lo presenta en este camino espiritual de espera y oración que es el Adviento, para que también nosotros escuchemos sus llamadas y hagamos nuestra su invitación urgente a la conversión. El austero precursor de Cristo repite hoy en nuestra asamblea litúrgica estas mismas palabras: \textquote{¡Arrepentíos, porque el reino de los cielos está cerca!} (\emph{Mt} 3, 2). Aquí, en esta parroquia, en todas las demás parroquias de Roma, de Europa, resuena la misma palabra divina. Convertirnos, abrir el corazón a la fuerza renovadora del Evangelio: este debe ser el programa diario de todo creyente, un programa particularmente exigente y elocuente en el período litúrgico que vivimos. De hecho, este es el significado del tiempo de Adviento: recordar que Jesús vino entre nosotros y por nosotros en la humildad del pesebre. Por tanto, a cada uno de nosotros le corresponde estar dispuesto a acogerle, con espíritu de penitencia, en una vida renovada por una fe auténtica y activa.
	
	2. La Iglesia nos indica hoy como modelo y guía de tal camino espiritual a Juan Bautista (\ldots{}) Vivía en el desierto en gran penitencia cuando el Espíritu de Dios lo envió a predicar la inminente venida del Mesías al mundo. Luego fue al Jordán, antes de que Jesús bajara allí, y comenzó a predicar la conversión de los corazones y a administrar el bautismo de penitencia. El suyo era un lenguaje franco y directo, casi grosero, que sin embargo la multitud escuchaba con interés. Proclamó que \textquote{el hacha ya está puesta a la raíz de los árboles: todo árbol que no da buen fruto es cortado y echado al fuego} (\emph{Mt} 3, 10). Y para ello exhortaba insistentemente a los que acudían a él: \textquote{Dad frutos dignos de conversión}. Convertirse significa \textquote{cambiar de vida}, adaptando la propia existencia al don gratuito de la salvación divina. Qué gran lección también para nuestra comunidad eclesial, que hoy se reúne con alegría alrededor del altar del Señor. La figura del Precursor es motivo de una revisión concreta de nuestra existencia cristiana y un estímulo para una adhesión más profunda al Evangelio. Su predicación fue confirmada por el testimonio del martirio. Juan el Bautista fue asesinado por el rey Herodes Antipas por su fidelidad a la palabra de Dios y, después de tantos siglos, la humanidad sigue honrando su fe y su fuerza moral, sellada por el sacrificio de la vida.
	
	3. \textquote{¡Convertíos porque el reino de los cielos está cerca!}. Esta exhortación de Juan el Bautista resuena una vez más en nuestro espíritu. Nos impulsa a tomar conciencia de la urgencia de adherirse firme y totalmente a Cristo, preparando sus caminos, enderezando sus sendas. La misión de la Iglesia que vive en nuestro tiempo de profundo y rápido cambio social es muy similar a la misión del Precursor. Todos estamos llamados a participar activamente en ella: es la nueva evangelización. \textquote{Evangelizar significa actuar como testigos y el imperativo de la evangelización es, por tanto, siempre actual}. Cuando hablamos de \textquote{nueva evangelización} lo hacemos porque es siempre y en todas partes \textquote{nueva}. \textquote{Jesucristo es el mismo ayer, hoy y por los siglos} (\emph{Hb} 13, 8). Esta \textquote{novedad} pertenece a la identidad del Evangelio. Orienta y anima el compromiso evangelizador, que constituye un imperativo continuo y permanente para los testigos de Cristo. Vivimos una época atravesada por fuertes corrientes de \textquote{contraevangelización} que, aunque en su expresión más radical se han debilitado considerablemente, no dejan de influir negativamente en el ámbito de los principios y en el de las opciones de vida. La palabra de Dios parece tan abrumada por las llamadas del utilitarismo y el consumismo, del secularismo y del materialismo práctico. Por eso es indispensable que por parte de todos los creyentes en Cristo se renueve y fortalezca la voluntad de ofrecer un testimonio renovado y coherente a favor de Cristo.
	
	4. ¡Queridos hermanos y hermanas! Esta invitación espiritual a trabajar activamente por el reino de los cielos está dirigida especialmente a vosotros (\ldots{})
	
	{[} \ldots{}{]}
	
	6. El Redentor que esperamos en este tiempo de Adviento vendrá trayendo consigo los dones del Espíritu Santo y \textquote{hará oír su poderosa voz para el gozo de nuestro corazón} (Antífona de entrada).
	
	Como nos recordó la \textbf{primera lectura} del Libro del \textbf{profeta Isaías}, su venida será una fuente de justicia y paz.
	
	\textquote{En aquel día se levantará la raíz de Jesé como estandarte para los pueblos}, \textquote{será gloriosa su morada} (\emph{Is} 11, 10). 
	
	Queridos hermanos y hermanas, que este anuncio de alegría y victoria definitiva sobre el mal y el pecado sostenga vuestro camino espiritual. La alegría del Señor, proclamada en nuestra atenta asamblea litúrgica, se nos ofrece como un don para que, en virtud de la perseverancia y el consuelo que nos llegan de las Escrituras, \textquote{mantengamos viva nuestra esperanza} (\emph{Rm} 15, 4). 
	
	Miremos y sigamos el ejemplo de María, {[}cuya solemnidad de la Inmaculada Concepción nos estamos preparando para celebrar{]}. ¡Sigamos despertando nuestra confianza a través de la oración incesante y la escucha de Dios! Por tanto, nuestra esperanza nunca fallará. 
	
	Caminemos en una vida nueva, teniendo \textquote{entre vosotros los mismos sentimientos, según Cristo Jesús} (\emph{Rom} 15, 5). Como san Pablo nos amonesta hoy en la Carta a los Romanos (\ldots{}). Con esta breve palabra, con esta breve invocación del Adviento quiero terminar. 
	
	\textquote{Preparemos el camino del Señor}. ¡Amén!
\end{body}

\subsubsection{Homilía (1998): Renovar la espera}

\src{Visita Pastoral a la Parroquia Romana Santa Rosa de Viterbo. \\6 de diciembre de 1998.}

\begin{body}
	1. \textquote{Preparad el camino del Señor} (\emph{Mt 3,3}). 
	
	\ltr{E}{stas} palabras, tomadas del libro del profeta Isaías (cf. \emph{Is 40,3}), las pronunció san Juan Bautista, a quien Jesús mismo definió en una ocasión el más grande entre los nacidos de mujer (cf. \emph{Mt 11,11}). El \textbf{evangelista san Mateo} lo presenta como el Precursor, es decir, el que recibió la misión de \textquote{preparar el camino} al Mesías. 
	
	Su apremiante exhortación a la penitencia y a la conversión sigue resonando en el mundo e impulsa a los creyentes (\ldots{}) a acoger dignamente al Señor que viene\ldots{} 
	
	Amadísimos hermanos y hermanas, preparémonos para el encuentro con Cristo. Preparémosle el camino en nuestro corazón y en nuestras comunidades. La figura del \textbf{Bautista}, que viste con pobreza y se alimenta con langostas y miel silvestre, constituye un fuerte llamamiento a la vigilancia y a la espera del Salvador. 
	
	2. \textquote{Aquel día, brotará un renuevo del tronco de Jesé} (\emph{Is 11,1}). En el tiempo del Adviento, la liturgia pone de relieve otra gran figura: el profeta \textbf{Isaías}, que, en el seno del pueblo elegido, mantuvo viva la expectativa, llena de esperanza, en la venida del Salvador prometido. Como hemos escuchado en la primera lectura, Isaías describe al Mesías como un vástago que sale del antiguo tronco de Jesé. El Espíritu de Dios se posará plenamente sobre él y su reino se caracterizará por el restablecimiento de la justicia y la consolidación de la paz universal. 
	
	También nosotros necesitamos renovar esta espera confiada en el Señor. Escuchemos las palabras del \textbf{profeta}. Nos invitan a aguardar con esperanza la instauración definitiva del reino de Dios, que él describe con imágenes muy poéticas, capaces de poner de relieve el triunfo de la justicia y la paz por obra del Mesías. \textquote{Habitarán el lobo y el cordero, (\ldots{}) el novillo y el león pacerán juntos, y un niño pequeño los pastoreará} (\emph{Is 11,6}). Se trata de expresiones simbólicas, que anticipan la realidad de una reconciliación universal. En esta obra de renovación cósmica todos estamos llamados a colaborar, sostenidos por la certeza de que un día toda la creación se someterá completamente al señorío universal de Cristo. 
	
	5. \textquote{Acogeos mutuamente como os acogió Cristo} (\emph{Rm 15,7}). San \textbf{Pablo}, indicándonos el sentido profundo del Adviento, manifiesta la necesidad de la acogida y la fraternidad en cada familia y en cada comunidad. Acoger a Cristo y abrir el corazón a los hermanos es nuestro compromiso diario, al que nos impulsa el clima espiritual de este tiempo litúrgico. 
	
	El Apóstol prosigue: \textquote{El Dios de la paciencia y del consuelo os conceda tener los unos para con los otros los mismos sentimientos, según Cristo Jesús, para que unánimes, a una voz, glorifiquéis al Dios y Padre de nuestro Señor Jesucristo} (\emph{Rm 15,5-6}). Que el Adviento y la próxima celebración del nacimiento de Jesús refuercen en cada creyente este sentido de unidad y comunión. 
	
	Que María, la Virgen de la escucha y la acogida, nos acompañe en el itinerario del Adviento, y nos guíe para ser testigos creíbles y generosos del amor salvífico de Dios. Amén.
\end{body}

\newsection 

\subsection{Benedicto XVI, Papa}

\subsubsection{Ángelus (2007): Palabras saludables}

\src{Plaza de San Pedro. \\Domingo 9 de diciembre del 2007.} 

\begin{body}
	\ltr[\ldots{} ]{H}{oy,} segundo domingo de Adviento, se nos presenta la figura austera del \textbf{Precursor}, que el \textbf{evangelista san Mateo} introduce así: \textquote{Por aquel tiempo, Juan Bautista se presentó en el desierto de Judea predicando: \textquote{Convertíos, porque está cerca el reino de los cielos}} (\emph{Mt} 3, 1-2). Tenía la misión de preparar y allanar el sendero al Mesías, exhortando al pueblo de Israel a arrepentirse de sus pecados y corregir toda injusticia. Con palabras exigentes, Juan Bautista anunciaba el juicio inminente: \textquote{El árbol que no da fruto será talado y echado al fuego} (\emph{Mt} 3, 10). Sobre todo ponía en guardia contra la hipocresía de quien se sentía seguro por el mero hecho de pertenecer al pueblo elegido: ante Dios ---decía--- nadie tiene títulos para enorgullecerse, sino que debe dar \textquote{frutos dignos de conversión} (\emph{Mt} 3, 8).
	
	Mientras prosigue el camino del Adviento, mientras nos preparamos para celebrar el Nacimiento de Cristo, resuena en nuestras comunidades esta exhortación de \textbf{Juan Bautista} a la conversión. Es una invitación apremiante a abrir el corazón y acoger al Hijo de Dios que viene a nosotros para manifestar el juicio divino. El Padre ---escribe el evangelista san Juan--- no juzga a nadie, sino que ha dado al Hijo el poder de juzgar, porque es Hijo del hombre (cf. \emph{Jn} 5, 22. 27). Hoy, en el presente, es cuando se juega nuestro destino futuro; con el comportamiento concreto que tenemos en esta vida decidimos nuestro destino eterno. En el ocaso de nuestros días en la tierra, en el momento de la muerte, seremos juzgados según nuestra semejanza o desemejanza con el Niño que está a punto de nacer en la pobre cueva de Belén, puesto que él es el criterio de medida que Dios ha dado a la humanidad.
	
	El Padre celestial, que en el nacimiento de su Hijo unigénito nos manifestó su amor misericordioso, nos llama a seguir sus pasos convirtiendo, como él, nuestra existencia en un don de amor. Y los frutos del amor son los \textquote{frutos dignos de conversión} a los que hacía referencia san \textbf{Juan Bautista} cuando, con palabras tajantes, se dirigía a los fariseos y a los saduceos que acudían entre la multitud a su bautismo.
	
	Mediante el \textbf{Evangelio}, Juan Bautista sigue hablando a lo largo de los siglos a todas las generaciones. Sus palabras claras y duras resultan muy saludables para nosotros, hombres y mujeres de nuestro tiempo, en el que, por desgracia, también el modo de vivir y percibir la Navidad muy a menudo sufre las consecuencias de una mentalidad materialista. La \textquote{voz} del gran profeta nos pide que preparemos el camino del Señor que viene, en los desiertos de hoy, desiertos exteriores e interiores, sedientos del agua viva que es Cristo.
	
	Que la Virgen María nos guíe a una auténtica conversión del corazón, a fin de que podamos realizar las opciones necesarias para sintonizar nuestra mentalidad con el Evangelio.
\end{body}

\subsubsection{Ángelus (2010): Voz de Dios en el desierto del mundo}

\src{Plaza de San Pedro. \\5 de diciembre del 2010.}

\begin{body}
	\ltr{E}{l} \textbf{Evangelio} de este segundo domingo de Adviento (\emph{Mt} 3, 1-12) nos presenta la figura de san Juan Bautista, el cual, según una célebre profecía de Isaías (cf. 40, 3), se retiró al desierto de Judea y, con su predicación, llamó al pueblo a convertirse para estar preparado para la inminente venida del Mesías. San Gregorio Magno comenta que el Bautista \textquote{predica la recta fe y las obras buenas\ldots{} para que la fuerza de la gracia penetre, la luz de la verdad resplandezca, los caminos hacia Dios se enderecen y nazcan en el corazón pensamientos honestos tras la escucha de la Palabra que guía hacia el bien} (\emph{Hom. in Evangelia,} XX, 3: CCL 141, 155). El precursor de Jesús, situado entre la Antigua y la Nueva Alianza, es como una estrella que precede la salida del Sol, de Cristo, es decir, de Aquel sobre el cual ---según otra profecía de Isaías--- \textquote{reposará el espíritu del Señor: espíritu de sabiduría e inteligencia, espíritu de consejo y fortaleza, espíritu de ciencia y temor del Señor} (\emph{Is} 11, 2).
	
	En el tiempo de Adviento, también nosotros estamos llamados a escuchar la voz de Dios, que resuena en el desierto del mundo a través de las Sagradas Escrituras, especialmente cuando se predican con la fuerza del Espíritu Santo. De hecho, la fe se fortalece cuanto más se deja iluminar por la Palabra divina, por \textquote{todo cuanto ---como nos recuerda el \textbf{apóstol san Pablo}--- fue escrito en el pasado\ldots{} para enseñanza nuestra, para que con la paciencia y el consuelo que dan las Escrituras mantengamos la esperanza} (\emph{Rm} 15, 4). El modelo de la escucha es la Virgen María: \textquote{Contemplando en la Madre de Dios una existencia totalmente modelada por la Palabra, también nosotros nos sentimos llamados a entrar en el misterio de la fe, con la que Cristo viene a habitar en nuestra vida}. San Ambrosio nos recuerda que \textquote{todo cristiano que cree, concibe en cierto sentido y engendra al Verbo de Dios en sí mismo} (\emph{Verbum Domini,} 28).
	
	Queridos amigos, \textquote{nuestra salvación se basa en una venida}, escribió Romano Guardini (\emph{La santa notte. Dall'Avvento all'Epifania}, Brescia 1994, p. 13). \textquote{El Salvador vino por la libertad de Dios\ldots{} Así la decisión de la fe consiste\ldots{} en acoger a Aquel que se acerca} (\emph{ib}., p. 14). \textquote{El Redentor ---añade--- viene a cada hombre: en sus alegrías y penas, en sus conocimientos claros, en sus dudas y tentaciones, en todo lo que constituye su naturaleza y su vida} (\emph{ib}., p. 15).
	
	A la Virgen María, en cuyo seno habitó el Hijo del Altísimo, {[}y que el miércoles próximo, 8 de diciembre, celebraremos en la solemnidad de la Inmaculada Concepción,{]} pedimos que nos sostenga en este camino espiritual, para acoger con fe y con amor la venida del Salvador.
\end{body}

\newsection

\subsection{Francisco, papa}

\subsubsection{Ángelus (2016): El Reino ha llegado ya}

\src{Plaza de San Pedro. \\4 de diciembre del 2016.}

\begin{body}
	\ltr{E}{n} el \textbf{Evangelio} de este segundo domingo de Adviento resuena la invitación de Juan Bautista: \textquote{¡Convertíos porque el reino de los cielos está cerca!} (Mt 3,2). Con estas palabras Jesús dará inicio a su misión en Galilea (cfr Mt 4,17); y tal será también el anuncio que deberán llevar los discípulos en su primera experiencia misionera (cfr Mt 10,7).
	
	El evangelista Mateo quiere así presentar a \textbf{Juan} como el que prepara el camino al Cristo que viene, y los discípulos como los continuadores de la predicación de Jesús. Se trata del mismo anuncio alegre: ¡viene el reino de Dios, es más, está cerca, está en medio de nosotros! Esta palabra es muy importante: \textquote{el reino de Dios está en medio de vosotros}, dice Jesús. Y Juan anuncia esto que Jesús luego dirá: \textquote{El reino de Dios ha venido, ha llegado, está en medio de vosotros}. Este es el mensaje central de toda misión cristiana. Cuando un misionero va, un cristiano va a anunciar a Jesús, no va a hacer proselitismo como si fuera un hincha que busca más seguidores para su equipo. No, va simplemente a anunciar: \textquote{¡El reino de Dios está en medio de vosotros!}. Y así el misionero prepara el camino a Jesús, que encuentra a su pueblo.
	
	¿Pero qué es este reino de Dios, reino de los cielos? Son sinónimos. Nosotros pensamos enseguida en algo que se refiere al más allá: la vida eterna. Cierto, esto es verdad, el reino de Dios se extenderá sin fin más allá de la vida terrena, pero la buena noticia que Jesús nos trae ---y que Juan anticipa--- es que el reino de Dios no tenemos que esperarlo en el futuro: se ha acercado, de alguna manera está ya presente y podemos experimentar desde ahora el poder espiritual. Dios viene a establecer su señorío en la historia, en nuestra vida de cada día; y allí donde esta viene acogida con fe y humildad brotan el amor, la alegría y la paz.
	
	La condición para entrar a formar parte de este reino es cumplir un cambio en nuestra vida, es decir, convertirnos. Convertirnos cada día, un paso adelante cada día. Se trata de dejar los caminos, cómodos pero engañosos, de los ídolos de este mundo: el éxito a toda costa, el poder a costa de los más débiles, la sed de riquezas, el placer a cualquier precio. Y de abrir sin embargo el camino al Señor que viene: Él no nos quita nuestra libertad, sino que nos da la verdadera felicidad. Con el nacimiento de Jesús en Belén, es Dios mismo que viene a habitar en medio de nosotros para librarnos del egoísmo, del pecado y de la corrupción, de estas actitudes que son del diablo: buscar éxito a toda costa, el poder a costa de los más débiles, tener sed de riquezas y buscar el placer a cualquier precio.
	
	La Navidad es un día de gran alegría también exterior, pero es sobre todo un evento religioso por lo que es necesaria una preparación espiritual. En este tiempo de Adviento, dejémonos guiar por la \textbf{exhortación del Bautista}: \textquote{Preparad el camino al Señor, allanad sus senderos} (v. 3).
	
	Nosotros preparamos el camino del Señor y allanamos sus senderos cuando examinamos nuestra conciencia, cuando escrutamos nuestras actitudes, cuando con sinceridad y confianza confesamos nuestros pecados en el sacramento de la penitencia. En este sacramento experimentamos en nuestro corazón la cercanía del reino de Dios y su salvación.
	
	La salvación de Dios es trabajo de un amor más grande que nuestro pecado; solamente el amor de Dios puede cancelar el pecado y liberar del mal, y solamente el amor de Dios puede orientarnos sobre el camino del bien. Que la Virgen María nos ayude a prepararnos al encuentro con este Amor cada vez más grande que en la noche de Navidad se ha hecho pequeño pequeño, como una semilla caída en la tierra, la semilla del reino de Dios.
\end{body}

\newsection

\section{Temas}
%\cceth{Los profetas y la espera del Mesías}

\cceref{CEC 522, 711-716, 722}

\begin{ccebody}
	
	\ccesec{Los misterios de la infancia y de la vida oculta de Jesús: Los preparativos}
	
	\n{522} La venida del Hijo de Dios a la tierra es un acontecimiento tan inmenso que Dios quiso prepararlo durante siglos. Ritos y sacrificios, figuras y símbolos de la \textquote{Primera Alianza} (\emph{Hb} 9,15), todo lo hace converger hacia Cristo; anuncia esta venida por boca de los profetas que se suceden en Israel. Además, despierta en el corazón de los paganos una espera, aún confusa, de esta venida.
	
	\ccesec{La espera del Mesías y de su Espíritu}
	
	\n{711} \textquote{He aquí que yo lo renuevo} (\emph{Is} 43, 19): dos líneas proféticas se van a perfilar, una se refiere a la espera del Mesías, la otra al anuncio de un Espíritu nuevo, y las dos convergen en el pequeño Resto, el pueblo de los Pobres (cf. \emph{So} 2, 3), que aguardan en la esperanza la \textquote{consolación de Israel} y \textquote{la redención de Jerusalén} (cf. \emph{Lc} 2, 25. 38).
	
	Ya se ha dicho cómo Jesús cumple las profecías que a Él se refieren. A continuación se describen aquéllas en que aparece sobre todo la relación del Mesías y de su Espíritu.
	
	\n{712} Los rasgos del rostro del \emph{Mesías} esperado comienzan a aparecer en el Libro del Emmanuel (cf. \emph{Is} 6, 12) (cuando \textquote{Isaías vio [\ldots{}] la gloria} de Cristo \emph{Jn} 12, 41), especialmente en \emph{Is} 11, 1-2:
	
	\begin{cceprose}
			«Saldrá un vástago del tronco de Jesé, 
			y un retoño de sus raíces brotará. 			
			Reposará sobre él el Espíritu del Señor: 			
			espíritu de sabiduría e inteligencia, 			
			espíritu de consejo y de fortaleza, 
			espíritu de ciencia y temor del Señor».
	\end{cceprose}
	

	
	\n{713} Los rasgos del Mesías se revelan sobre todo en los Cantos del Siervo (cf. \emph{Is} 42, 1-9; cf. \emph{Mt} 12, 18-21; \emph{Jn} 1, 32-34; y también \emph{Is} 49, 1-6; cf. \emph{Mt} 3, 17; \emph{Lc} 2, 32, y por último \emph{Is} 50, 4-10 y 52, 13-53, 12). Estos cantos anuncian el sentido de la Pasión de Jesús, e indican así cómo enviará el Espíritu Santo para vivificar a la multitud: no desde fuera, sino desposándose con nuestra \textquote{condición de esclavos} (\emph{Flp} 2, 7). Tomando sobre sí nuestra muerte, puede comunicarnos su propio Espíritu de vida.
	
	\n{714} Por eso Cristo inaugura el anuncio de la Buena Nueva haciendo suyo este pasaje de Isaías (\emph{Lc} 4, 18-19; cf. \emph{Is} 61, 1-2):
	
	«El Espíritu del Señor está sobre mí, porque me ha ungido. Me ha enviado a anunciar a los pobres la Buena Nueva, a proclamar la liberación a los cautivos y la vista a los ciegos, para dar la libertad a los oprimidos y proclamar un año de gracia del Señor».
	
	\n{715} Los textos proféticos que se refieren directamente al envío del Espíritu Santo son oráculos en los que Dios habla al corazón de su Pueblo en el lenguaje de la Promesa, con los acentos del \textquote{amor y de la fidelidad} (cf. \emph{Ez} 11, 19; 36, 25-28; 37, 1-14; \emph{Jr} 31, 31-34; y \emph{Jl} 3, 1-5), cuyo cumplimiento proclamará San Pedro la mañana de Pentecostés (cf. \emph{Hch} 2, 17-21). Según estas promesas, en los \textquote{últimos tiempos}, el Espíritu del Señor renovará el corazón de los hombres grabando en ellos una Ley nueva; reunirá y reconciliará a los pueblos dispersos y divididos; transformará la primera creación y Dios habitará en ella con los hombres en la paz.
	
	\n{716} El Pueblo de los \textquote{pobres} (cf. \emph{So} 2, 3; \emph{Sal} 22, 27; 34, 3; \emph{Is} 49, 13; 61, 1; etc.), los humildes y los mansos, totalmente entregados a los designios misteriosos de Dios, los que esperan la justicia, no de los hombres sino del Mesías, todo esto es, finalmente, la gran obra de la Misión escondida del Espíritu Santo durante el tiempo de las Promesas para preparar la venida de Cristo. Esta es la calidad de corazón del Pueblo, purificado e iluminado por el Espíritu, que se expresa en los Salmos. En estos pobres, el Espíritu prepara para el Señor \textquote{un pueblo bien dispuesto} (cf. \emph{Lc} 1, 17).
	
	\n{722} El Espíritu Santo \emph{preparó} a María con su gracia. Convenía que fuese \textquote{llena de gracia} la Madre de Aquel en quien \textquote{reside toda la plenitud de la divinidad corporalmente} (\emph{Col} 2, 9). Ella fue concebida sin pecado, por pura gracia, como la más humilde de todas las criaturas, la más capaz de acoger el don inefable del Omnipotente. Con justa razón, el ángel Gabriel la saluda como la \textquote{Hija de Sión}: \textquote{Alégrate} (cf. \emph{So} 3, 14; \emph{Za} 2, 14). Cuando ella lleva en sí al Hijo eterno, hace subir hasta el cielo con su cántico al Padre, en el Espíritu Santo, la acción de gracias de todo el pueblo de Dios y, por tanto, de la Iglesia (cf. \emph{Lc} 1, 46-55).

\end{ccebody}
%\cceth{La misión de Juan Bautista}

\cceref{CEC 523, 717-720}

\begin{ccebody}
	
	\n{523} \emph{San Juan Bautista} es el precursor (cf. \emph{Hch} 13, 24) inmediato del Señor, enviado para prepararle el camino (cf. \emph{Mt} 3, 3). \textquote{Profeta del Altísimo} (\emph{Lc} 1, 76), sobrepasa a todos los profetas (cf. \emph{Lc} 7, 26), de los que es el último (cf. \emph{Mt} 11, 13), e inaugura el Evangelio (cf. \emph{Hch} 1, 22; \emph{Lc} 16,16); desde el seno de su madre (cf. \emph{Lc} 1,41) saluda la venida de Cristo y encuentra su alegría en ser \textquote{el amigo del esposo} (\emph{Jn} 3, 29) a quien señala como \textquote{el Cordero de Dios que quita el pecado del mundo} (\emph{Jn} 1, 29). Precediendo a Jesús \textquote{con el espíritu y el poder de Elías} (\emph{Lc} 1, 17), da testimonio de él mediante su predicación, su bautismo de conversión y finalmente con su martirio (cf. \emph{Mc} 6, 17-29).
	
	\ccesec{El Espíritu de Cristo en la plenitud de los tiempos. Juan, Precursor, Profeta y Bautista}
	
	\n{717} \textquote{Hubo un hombre, enviado por Dios, que se llamaba Juan} (\emph{Jn} 1, 6). Juan fue \textquote{lleno del Espíritu Santo ya desde el seno de su madre} (\emph{Lc} 1, 15. 41) por obra del mismo Cristo que la Virgen María acababa de concebir del Espíritu Santo. La \textquote{Visitación} de María a Isabel se convirtió así en \textquote{visita de Dios a su pueblo} (\emph{Lc} 1, 68).
	
	\n{718} Juan es \textquote{Elías que debe venir} (\emph{Mt} 17, 10-13): El fuego del Espíritu lo habita y le hace correr delante {[}como \textquote{precursor}{]} del Señor que viene. En Juan el Precursor, el Espíritu Santo culmina la obra de \textquote{preparar al Señor un pueblo bien dispuesto} (\emph{Lc} 1, 17).
	
	\n{719} Juan es \textquote{más que un profeta} (\emph{Lc} 7, 26). En él, el Espíritu Santo consuma el \textquote{hablar por los profetas}. Juan termina el ciclo de los profetas inaugurado por Elías (cf. \emph{Mt} 11, 13-14). Anuncia la inminencia de la consolación de Israel, es la \textquote{voz} del Consolador que llega (\emph{Jn} 1, 23; cf. \emph{Is} 40, 1-3). Como lo hará el Espíritu de Verdad, \textquote{vino como testigo para dar testimonio de la luz} (\emph{Jn} 1, 7; cf. \emph{Jn} 15, 26; 5, 33). Con respecto a Juan, el Espíritu colma así las \textquote{indagaciones de los profetas} y la ansiedad de los ángeles (\emph{1 P} 1, 10-12): \textquote{Aquél sobre quien veas que baja el Espíritu y se queda sobre él, ése es el que bautiza con el Espíritu Santo. Y yo lo he visto y doy testimonio de que éste es el Hijo de Dios [\ldots{}] He ahí el Cordero de Dios} (\emph{Jn} 1, 33-36).
	
	\n{720} En fin, con Juan Bautista, el Espíritu Santo, inaugura, prefigurándolo, lo que realizará con y en Cristo: volver a dar al hombre la \textquote{semejanza} divina. El bautismo de Juan era para el arrepentimiento, el del agua y del Espíritu será un nuevo nacimiento (cf. \emph{Jn} 3, 5).
	
\end{ccebody}
%\cceth{La conversión de los bautizados}

\cceref{CEC 1427-1429}

\begin{ccebody}
	\n{1427} Jesús llama a la conversión. Esta llamada es una parte esencial del anuncio del Reino: \textquote{El tiempo se ha cumplido y el Reino de Dios está cerca; convertíos y creed en la Buena Nueva} (\emph{Mc} 1,15). En la predicación de la Iglesia, esta llamada se dirige primeramente a los que no conocen todavía a Cristo y su Evangelio. Así, el Bautismo es el lugar principal de la conversión primera y fundamental. Por la fe en la Buena Nueva y por el Bautismo (cf. \emph{Hch} 2,38) se renuncia al mal y se alcanza la salvación, es decir, la remisión de todos los pecados y el don de la vida nueva.
	
	\n{1428} Ahora bien, la llamada de Cristo a la conversión sigue resonando en la vida de los cristianos. Esta \emph{segunda conversión} es una tarea ininterrumpida para toda la Iglesia que \textquote{recibe en su propio seno a los pecadores} y que siendo \textquote{santa al mismo tiempo que necesitada de purificación constante, busca sin cesar la penitencia y la renovación} (LG 8). Este esfuerzo de conversión no es sólo una obra humana. Es el movimiento del \textquote{corazón contrito} (\emph{Sal} 51,19), atraído y movido por la gracia (cf. \emph{Jn} 6,44; 12,32) a responder al amor misericordioso de Dios que nos ha amado primero (cf. \emph{1 Jn} 4,10).
	
	\n{1429} De ello da testimonio la conversión de san Pedro tras la triple negación de su Maestro. La mirada de infinita misericordia de Jesús provoca las lágrimas del arrepentimiento (\emph{Lc} 22,61) y, tras la resurrección del Señor, la triple afirmación de su amor hacia él (cf. \emph{Jn} 21,15-17). La segunda conversión tiene también una dimensión \emph{comunitaria}. Esto aparece en la llamada del Señor a toda la Iglesia: \textquote{¡Arrepiéntete!} (\emph{Ap} 2,5.16).
	
	San Ambrosio dice acerca de las dos conversiones que, \textquote{en la Iglesia, existen el agua y las lágrimas: el agua del Bautismo y las lágrimas de la Penitencia} (\emph{Epistula extra collectionem} 1 {[}41{]}, 12).
	
	Mas yo, tirándome debajo de una higuera, no sé cómo, solté la rienda a las lágrimas, brotando dos ríos de mis ojos, sacrificio tuyo aceptable. Y aunque no con estas palabras, pero sí con el mismo sentido, te dije muchas cosas como éstas: \emph{¡Y tú, Señor, hasta cuándo! ¡Hasta cuándo, Señor, has de estar irritado!} No te acuerdes más de nuestras maldades pasadas. Me sentía aún cautivo de ellas y lanzaba voces lastimeras: \textquote{¿Hasta cuándo, hasta cuándo, ¡mañana!, ¡mañana!? ¿Por qué no hoy? ¿Por qué no poner fin a mis torpezas ahora mismo?}.
	
	Decía estas cosas y lloraba con muy dolorosa contrición de mi corazón. Pero he aquí que oigo de la casa vecina una voz, como de niño o niña, que decía cantando y repetía muchas veces: \textquote{\emph{Toma y lee, toma y lee}} (tolle lege, tolle lege).
	
	\ldots{} Así que, apresurado, volví al lugar donde estaba sentado Alipio y yo había dejado el códice del Apóstol al levantarme de allí. Lo tomé, lo abrí y leí en silencio el primer capítulo que se me vino a los ojos, que decía: \emph{No en comilonas y embriagueces, no en lechos y en liviandades, no en contiendas y emulaciones sino revestíos de nuestro Señor Jesucristo y no cuidéis de la carne con demasiados deseos}.
	
	No quise leer más, ni era necesario tampoco, pues al punto que di fin a la sentencia, como si se hubiera infiltrado en mi corazón una luz de seguridad, se disiparon todas las tinieblas de mis dudas.
	
	Después entramos a ver a mi madre, indicándoselo, y se llenó de gozo; le contamos el modo como había sucedido, y saltaba de alegría y cantaba victoria, por lo cual te bendecía a ti, que eres poderoso para darnos más de lo que pedimos o entendemos, porque veía que le habías concedido, respecto de mí, mucho más de lo que constantemente te pedía con sollozos y lágrimas piadosas.
	
	\textbf{San Agustín}, \emph{Confesiones ,} Libro VIII, capítulo 12.
	
\end{ccebody}
%%01 Ciclo | 01 Tiempo | 02 Semana

\cceth{Los profetas y la espera del Mesías}

\cceref{CEC 522, 711-716, 722}

\begin{ccebody}
		\ccesec{Los misterios de la infancia y de la vida oculta de Jesús: Los preparativos}
	
	\n{522} La venida del Hijo de Dios a la tierra es un acontecimiento tan inmenso que Dios quiso prepararlo durante siglos. Ritos y sacrificios, figuras y símbolos de la \textquote{Primera Alianza} (\emph{Hb} 9,15), todo lo hace converger hacia Cristo; anuncia esta venida por boca de los profetas que se suceden en Israel. Además, despierta en el corazón de los paganos una espera, aún confusa, de esta venida.

	
		\ccesec{La espera del Mesías y de su Espíritu}
	
	\n{711} \textquote{He aquí que yo lo renuevo} (\emph{Is} 43, 19): dos líneas proféticas se van a perfilar, una se refiere a la espera del Mesías, la otra al anuncio de un Espíritu nuevo, y las dos convergen en el pequeño Resto, el pueblo de los Pobres (cf. \emph{So} 2, 3), que aguardan en la esperanza la \textquote{consolación de Israel} y \textquote{la redención de Jerusalén} (cf. \emph{Lc} 2, 25. 38).
	
	Ya se ha dicho cómo Jesús cumple las profecías que a Él se refieren. A continuación se describen aquéllas en que aparece sobre todo la relación del Mesías y de su Espíritu.
	
	\n{712} Los rasgos del rostro del \emph{Mesías} esperado comienzan a aparecer en el Libro del Emmanuel (cf. \emph{Is} 6, 12) (cuando \textquote{Isaías vio [\ldots{}] la gloria} de Cristo \emph{Jn} 12, 41), especialmente en \emph{Is} 11, 1-2:
	
\begin{quote}
	«Saldrá un vástago del tronco de Jesé, \\y un retoño de sus raíces brotará. \\Reposará sobre él el Espíritu del Señor: \\espíritu de sabiduría e inteligencia, \\espíritu de consejo y de fortaleza, \\espíritu de ciencia y temor del Señor».
\end{quote}
	
		\n{713} Los rasgos del Mesías se revelan sobre todo en los Cantos del Siervo (cf. \emph{Is} 42, 1-9; cf. \emph{Mt} 12, 18-21; \emph{Jn} 1, 32-34; y también \emph{Is} 49, 1-6; cf. \emph{Mt} 3, 17; \emph{Lc} 2, 32, y por último \emph{Is} 50, 4-10 y 52, 13-53, 12). Estos cantos anuncian el sentido de la Pasión de Jesús, e indican así cómo enviará el Espíritu Santo para vivificar a la multitud: no desde fuera, sino desposándose con nuestra \textquote{condición de esclavos} (\emph{Flp} 2, 7). Tomando sobre sí nuestra muerte, puede comunicarnos su propio Espíritu de vida.

	
		\n{714} Por eso Cristo inaugura el anuncio de la Buena Nueva haciendo suyo este pasaje de Isaías (\emph{Lc} 4, 18-19; cf. \emph{Is} 61, 1-2):
	
	«El Espíritu del Señor está sobre mí, porque me ha ungido. Me ha enviado a anunciar a los pobres la Buena Nueva, a proclamar la liberación a los cautivos y la vista a los ciegos, para dar la libertad a los oprimidos y proclamar un año de gracia del Señor».

	
		\n{715} Los textos proféticos que se refieren directamente al envío del Espíritu Santo son oráculos en los que Dios habla al corazón de su Pueblo en el lenguaje de la Promesa, con los acentos del \textquote{amor y de la fidelidad} (cf. \emph{Ez} 11, 19; 36, 25-28; 37, 1-14; \emph{Jr} 31, 31-34; y \emph{Jl} 3, 1-5), cuyo cumplimiento proclamará San Pedro la mañana de Pentecostés (cf. \emph{Hch} 2, 17-21). Según estas promesas, en los \textquote{últimos tiempos}, el Espíritu del Señor renovará el corazón de los hombres grabando en ellos una Ley nueva; reunirá y reconciliará a los pueblos dispersos y divididos; transformará la primera creación y Dios habitará en ella con los hombres en la paz.
	
		\n{716} El Pueblo de los \textquote{pobres} (cf. \emph{So} 2, 3; \emph{Sal} 22, 27; 34, 3; \emph{Is} 49, 13; 61, 1; etc.), los humildes y los mansos, totalmente entregados a los designios misteriosos de Dios, los que esperan la justicia, no de los hombres sino del Mesías, todo esto es, finalmente, la gran obra de la Misión escondida del Espíritu Santo durante el tiempo de las Promesas para preparar la venida de Cristo. Esta es la calidad de corazón del Pueblo, purificado e iluminado por el Espíritu, que se expresa en los Salmos. En estos pobres, el Espíritu prepara para el Señor \textquote{un pueblo bien dispuesto} (cf. \emph{Lc} 1, 17).

	
		\n{722} El Espíritu Santo \emph{preparó} a María con su gracia. Convenía que fuese \textquote{llena de gracia} la Madre de Aquel en quien \textquote{reside toda la plenitud de la divinidad corporalmente} (\emph{Col} 2, 9). Ella fue concebida sin pecado, por pura gracia, como la más humilde de todas las criaturas, la más capaz de acoger el don inefable del Omnipotente. Con justa razón, el ángel Gabriel la saluda como la \textquote{Hija de Sión}: \textquote{Alégrate} (cf. \emph{So} 3, 14; \emph{Za} 2, 14). Cuando ella lleva en sí al Hijo eterno, hace subir hasta el cielo con su cántico al Padre, en el Espíritu Santo, la acción de gracias de todo el pueblo de Dios y, por tanto, de la Iglesia (cf. \emph{Lc} 1, 46-55).
		
\end{ccebody}


\cceth{La misión de Juan Bautista}

\cceref{CEC 523, 717-720}


\begin{ccebody}
		\n{523} \emph{San Juan Bautista} es el precursor (cf. \emph{Hch} 13, 24) inmediato del Señor, enviado para prepararle el camino (cf. \emph{Mt} 3, 3). \textquote{Profeta del Altísimo} (\emph{Lc} 1, 76), sobrepasa a todos los profetas (cf. \emph{Lc} 7, 26), de los que es el último (cf. \emph{Mt} 11, 13), e inaugura el Evangelio (cf. \emph{Hch} 1, 22; \emph{Lc} 16,16); desde el seno de su madre (cf. \emph{Lc} 1,41) saluda la venida de Cristo y encuentra su alegría en ser \textquote{el amigo del esposo} (\emph{Jn} 3, 29) a quien señala como \textquote{el Cordero de Dios que quita el pecado del mundo} (\emph{Jn} 1, 29). Precediendo a Jesús \textquote{con el espíritu y el poder de Elías} (\emph{Lc} 1, 17), da testimonio de él mediante su predicación, su bautismo de conversión y finalmente con su martirio (cf. \emph{Mc} 6, 17-29).
	
		\ccesec{El Espíritu de Cristo en la plenitud de los tiempos. Juan, Precursor, Profeta y Bautista}
	
	\n{717} \textquote{Hubo un hombre, enviado por Dios, que se llamaba Juan} (\emph{Jn} 1, 6). Juan fue \textquote{lleno del Espíritu Santo ya desde el seno de su madre} (\emph{Lc} 1, 15. 41) por obra del mismo Cristo que la Virgen María acababa de concebir del Espíritu Santo. La \textquote{Visitación} de María a Isabel se convirtió así en \textquote{visita de Dios a su pueblo} (\emph{Lc} 1, 68).

	
		\n{718} Juan es \textquote{Elías que debe venir} (\emph{Mt} 17, 10-13): El fuego del Espíritu lo habita y le hace correr delante {[}como \textquote{precursor}{]} del Señor que viene. En Juan el Precursor, el Espíritu Santo culmina la obra de \textquote{preparar al Señor un pueblo bien dispuesto} (\emph{Lc} 1, 17).
	
		\n{719} Juan es \textquote{más que un profeta} (\emph{Lc} 7, 26). En él, el Espíritu Santo consuma el \textquote{hablar por los profetas}. Juan termina el ciclo de los profetas inaugurado por Elías (cf. \emph{Mt} 11, 13-14). Anuncia la inminencia de la consolación de Israel, es la \textquote{voz} del Consolador que llega (\emph{Jn} 1, 23; cf. \emph{Is} 40, 1-3). Como lo hará el Espíritu de Verdad, \textquote{vino como testigo para dar testimonio de la luz} (\emph{Jn} 1, 7; cf. \emph{Jn} 15, 26; 5, 33). Con respecto a Juan, el Espíritu colma así las \textquote{indagaciones de los profetas} y la ansiedad de los ángeles (\emph{1 P} 1, 10-12): \textquote{Aquél sobre quien veas que baja el Espíritu y se queda sobre él, ése es el que bautiza con el Espíritu Santo. Y yo lo he visto y doy testimonio de que éste es el Hijo de Dios [\ldots{}] He ahí el Cordero de Dios} (\emph{Jn} 1, 33-36).

	
		\n{720} En fin, con Juan Bautista, el Espíritu Santo, inaugura, prefigurándolo, lo que realizará con y en Cristo: volver a dar al hombre la \textquote{semejanza} divina. El bautismo de Juan era para el arrepentimiento, el del agua y del Espíritu será un nuevo nacimiento (cf. \emph{Jn} 3, 5).	
\end{ccebody}


\cceth{La conversión de los bautizados}

\cceref{CEC 1427-1429}

\begin{ccebody}
		\n{1427} Jesús llama a la conversión. Esta llamada es una parte esencial del anuncio del Reino: \textquote{El tiempo se ha cumplido y el Reino de Dios está cerca; convertíos y creed en la Buena Nueva} (\emph{Mc} 1,15). En la predicación de la Iglesia, esta llamada se dirige primeramente a los que no conocen todavía a Cristo y su Evangelio. Así, el Bautismo es el lugar principal de la conversión primera y fundamental. Por la fe en la Buena Nueva y por el Bautismo (cf. \emph{Hch} 2,38) se renuncia al mal y se alcanza la salvación, es decir, la remisión de todos los pecados y el don de la vida nueva.
	
		\n{1428} Ahora bien, la llamada de Cristo a la conversión sigue resonando en la vida de los cristianos. Esta \emph{segunda conversión} es una tarea ininterrumpida para toda la Iglesia que \textquote{recibe en su propio seno a los pecadores} y que siendo \textquote{santa al mismo tiempo que necesitada de purificación constante, busca sin cesar la penitencia y la renovación} (LG 8). Este esfuerzo de conversión no es sólo una obra humana. Es el movimiento del \textquote{corazón contrito} (\emph{Sal} 51,19), atraído y movido por la gracia (cf. \emph{Jn} 6,44; 12,32) a responder al amor misericordioso de Dios que nos ha amado primero (cf. \emph{1 Jn} 4,10).
	
		\n{1429} De ello da testimonio la conversión de san Pedro tras la triple negación de su Maestro. La mirada de infinita misericordia de Jesús provoca las lágrimas del arrepentimiento (\emph{Lc} 22,61) y, tras la resurrección del Señor, la triple afirmación de su amor hacia él (cf. \emph{Jn} 21,15-17). La segunda conversión tiene también una dimensión \emph{comunitaria}. Esto aparece en la llamada del Señor a toda la Iglesia: \textquote{¡Arrepiéntete!} (\emph{Ap} 2,5.16).
	
	San Ambrosio dice acerca de las dos conversiones que, \textquote{en la Iglesia, existen el agua y las lágrimas: el agua del Bautismo y las lágrimas de la Penitencia} (\emph{Epistula extra collectionem} 1 {[}41{]}, 12).	
\end{ccebody}

%01 Ciclo | 01 Tiempo | 02 Semana

\cceth{Los profetas y la espera del Mesías}

\cceref{CEC 522, 711-716, 722}

\begin{ccebody}
		\ccesec{Los misterios de la infancia y de la vida oculta de Jesús: Los preparativos}
	
	\n{522} La venida del Hijo de Dios a la tierra es un acontecimiento tan inmenso que Dios quiso prepararlo durante siglos. Ritos y sacrificios, figuras y símbolos de la \textquote{Primera Alianza} (\emph{Hb} 9,15), todo lo hace converger hacia Cristo; anuncia esta venida por boca de los profetas que se suceden en Israel. Además, despierta en el corazón de los paganos una espera, aún confusa, de esta venida.

	
		\ccesec{La espera del Mesías y de su Espíritu}
	
	\n{711} \textquote{He aquí que yo lo renuevo} (\emph{Is} 43, 19): dos líneas proféticas se van a perfilar, una se refiere a la espera del Mesías, la otra al anuncio de un Espíritu nuevo, y las dos convergen en el pequeño Resto, el pueblo de los Pobres (cf. \emph{So} 2, 3), que aguardan en la esperanza la \textquote{consolación de Israel} y \textquote{la redención de Jerusalén} (cf. \emph{Lc} 2, 25. 38).
	
	Ya se ha dicho cómo Jesús cumple las profecías que a Él se refieren. A continuación se describen aquéllas en que aparece sobre todo la relación del Mesías y de su Espíritu.
	
	\n{712} Los rasgos del rostro del \emph{Mesías} esperado comienzan a aparecer en el Libro del Emmanuel (cf. \emph{Is} 6, 12) (cuando \textquote{Isaías vio [\ldots{}] la gloria} de Cristo \emph{Jn} 12, 41), especialmente en \emph{Is} 11, 1-2:
	
\begin{quote}
	«Saldrá un vástago del tronco de Jesé, \\y un retoño de sus raíces brotará. \\Reposará sobre él el Espíritu del Señor: \\espíritu de sabiduría e inteligencia, \\espíritu de consejo y de fortaleza, \\espíritu de ciencia y temor del Señor».
\end{quote}
	
		\n{713} Los rasgos del Mesías se revelan sobre todo en los Cantos del Siervo (cf. \emph{Is} 42, 1-9; cf. \emph{Mt} 12, 18-21; \emph{Jn} 1, 32-34; y también \emph{Is} 49, 1-6; cf. \emph{Mt} 3, 17; \emph{Lc} 2, 32, y por último \emph{Is} 50, 4-10 y 52, 13-53, 12). Estos cantos anuncian el sentido de la Pasión de Jesús, e indican así cómo enviará el Espíritu Santo para vivificar a la multitud: no desde fuera, sino desposándose con nuestra \textquote{condición de esclavos} (\emph{Flp} 2, 7). Tomando sobre sí nuestra muerte, puede comunicarnos su propio Espíritu de vida.

	
		\n{714} Por eso Cristo inaugura el anuncio de la Buena Nueva haciendo suyo este pasaje de Isaías (\emph{Lc} 4, 18-19; cf. \emph{Is} 61, 1-2):
	
	«El Espíritu del Señor está sobre mí, porque me ha ungido. Me ha enviado a anunciar a los pobres la Buena Nueva, a proclamar la liberación a los cautivos y la vista a los ciegos, para dar la libertad a los oprimidos y proclamar un año de gracia del Señor».

	
		\n{715} Los textos proféticos que se refieren directamente al envío del Espíritu Santo son oráculos en los que Dios habla al corazón de su Pueblo en el lenguaje de la Promesa, con los acentos del \textquote{amor y de la fidelidad} (cf. \emph{Ez} 11, 19; 36, 25-28; 37, 1-14; \emph{Jr} 31, 31-34; y \emph{Jl} 3, 1-5), cuyo cumplimiento proclamará San Pedro la mañana de Pentecostés (cf. \emph{Hch} 2, 17-21). Según estas promesas, en los \textquote{últimos tiempos}, el Espíritu del Señor renovará el corazón de los hombres grabando en ellos una Ley nueva; reunirá y reconciliará a los pueblos dispersos y divididos; transformará la primera creación y Dios habitará en ella con los hombres en la paz.
	
		\n{716} El Pueblo de los \textquote{pobres} (cf. \emph{So} 2, 3; \emph{Sal} 22, 27; 34, 3; \emph{Is} 49, 13; 61, 1; etc.), los humildes y los mansos, totalmente entregados a los designios misteriosos de Dios, los que esperan la justicia, no de los hombres sino del Mesías, todo esto es, finalmente, la gran obra de la Misión escondida del Espíritu Santo durante el tiempo de las Promesas para preparar la venida de Cristo. Esta es la calidad de corazón del Pueblo, purificado e iluminado por el Espíritu, que se expresa en los Salmos. En estos pobres, el Espíritu prepara para el Señor \textquote{un pueblo bien dispuesto} (cf. \emph{Lc} 1, 17).

	
		\n{722} El Espíritu Santo \emph{preparó} a María con su gracia. Convenía que fuese \textquote{llena de gracia} la Madre de Aquel en quien \textquote{reside toda la plenitud de la divinidad corporalmente} (\emph{Col} 2, 9). Ella fue concebida sin pecado, por pura gracia, como la más humilde de todas las criaturas, la más capaz de acoger el don inefable del Omnipotente. Con justa razón, el ángel Gabriel la saluda como la \textquote{Hija de Sión}: \textquote{Alégrate} (cf. \emph{So} 3, 14; \emph{Za} 2, 14). Cuando ella lleva en sí al Hijo eterno, hace subir hasta el cielo con su cántico al Padre, en el Espíritu Santo, la acción de gracias de todo el pueblo de Dios y, por tanto, de la Iglesia (cf. \emph{Lc} 1, 46-55).
		
\end{ccebody}


\cceth{La misión de Juan Bautista}

\cceref{CEC 523, 717-720}


\begin{ccebody}
		\n{523} \emph{San Juan Bautista} es el precursor (cf. \emph{Hch} 13, 24) inmediato del Señor, enviado para prepararle el camino (cf. \emph{Mt} 3, 3). \textquote{Profeta del Altísimo} (\emph{Lc} 1, 76), sobrepasa a todos los profetas (cf. \emph{Lc} 7, 26), de los que es el último (cf. \emph{Mt} 11, 13), e inaugura el Evangelio (cf. \emph{Hch} 1, 22; \emph{Lc} 16,16); desde el seno de su madre (cf. \emph{Lc} 1,41) saluda la venida de Cristo y encuentra su alegría en ser \textquote{el amigo del esposo} (\emph{Jn} 3, 29) a quien señala como \textquote{el Cordero de Dios que quita el pecado del mundo} (\emph{Jn} 1, 29). Precediendo a Jesús \textquote{con el espíritu y el poder de Elías} (\emph{Lc} 1, 17), da testimonio de él mediante su predicación, su bautismo de conversión y finalmente con su martirio (cf. \emph{Mc} 6, 17-29).
	
		\ccesec{El Espíritu de Cristo en la plenitud de los tiempos. Juan, Precursor, Profeta y Bautista}
	
	\n{717} \textquote{Hubo un hombre, enviado por Dios, que se llamaba Juan} (\emph{Jn} 1, 6). Juan fue \textquote{lleno del Espíritu Santo ya desde el seno de su madre} (\emph{Lc} 1, 15. 41) por obra del mismo Cristo que la Virgen María acababa de concebir del Espíritu Santo. La \textquote{Visitación} de María a Isabel se convirtió así en \textquote{visita de Dios a su pueblo} (\emph{Lc} 1, 68).

	
		\n{718} Juan es \textquote{Elías que debe venir} (\emph{Mt} 17, 10-13): El fuego del Espíritu lo habita y le hace correr delante {[}como \textquote{precursor}{]} del Señor que viene. En Juan el Precursor, el Espíritu Santo culmina la obra de \textquote{preparar al Señor un pueblo bien dispuesto} (\emph{Lc} 1, 17).
	
		\n{719} Juan es \textquote{más que un profeta} (\emph{Lc} 7, 26). En él, el Espíritu Santo consuma el \textquote{hablar por los profetas}. Juan termina el ciclo de los profetas inaugurado por Elías (cf. \emph{Mt} 11, 13-14). Anuncia la inminencia de la consolación de Israel, es la \textquote{voz} del Consolador que llega (\emph{Jn} 1, 23; cf. \emph{Is} 40, 1-3). Como lo hará el Espíritu de Verdad, \textquote{vino como testigo para dar testimonio de la luz} (\emph{Jn} 1, 7; cf. \emph{Jn} 15, 26; 5, 33). Con respecto a Juan, el Espíritu colma así las \textquote{indagaciones de los profetas} y la ansiedad de los ángeles (\emph{1 P} 1, 10-12): \textquote{Aquél sobre quien veas que baja el Espíritu y se queda sobre él, ése es el que bautiza con el Espíritu Santo. Y yo lo he visto y doy testimonio de que éste es el Hijo de Dios [\ldots{}] He ahí el Cordero de Dios} (\emph{Jn} 1, 33-36).

	
		\n{720} En fin, con Juan Bautista, el Espíritu Santo, inaugura, prefigurándolo, lo que realizará con y en Cristo: volver a dar al hombre la \textquote{semejanza} divina. El bautismo de Juan era para el arrepentimiento, el del agua y del Espíritu será un nuevo nacimiento (cf. \emph{Jn} 3, 5).	
\end{ccebody}


\cceth{La conversión de los bautizados}

\cceref{CEC 1427-1429}

\begin{ccebody}
		\n{1427} Jesús llama a la conversión. Esta llamada es una parte esencial del anuncio del Reino: \textquote{El tiempo se ha cumplido y el Reino de Dios está cerca; convertíos y creed en la Buena Nueva} (\emph{Mc} 1,15). En la predicación de la Iglesia, esta llamada se dirige primeramente a los que no conocen todavía a Cristo y su Evangelio. Así, el Bautismo es el lugar principal de la conversión primera y fundamental. Por la fe en la Buena Nueva y por el Bautismo (cf. \emph{Hch} 2,38) se renuncia al mal y se alcanza la salvación, es decir, la remisión de todos los pecados y el don de la vida nueva.
	
		\n{1428} Ahora bien, la llamada de Cristo a la conversión sigue resonando en la vida de los cristianos. Esta \emph{segunda conversión} es una tarea ininterrumpida para toda la Iglesia que \textquote{recibe en su propio seno a los pecadores} y que siendo \textquote{santa al mismo tiempo que necesitada de purificación constante, busca sin cesar la penitencia y la renovación} (LG 8). Este esfuerzo de conversión no es sólo una obra humana. Es el movimiento del \textquote{corazón contrito} (\emph{Sal} 51,19), atraído y movido por la gracia (cf. \emph{Jn} 6,44; 12,32) a responder al amor misericordioso de Dios que nos ha amado primero (cf. \emph{1 Jn} 4,10).
	
		\n{1429} De ello da testimonio la conversión de san Pedro tras la triple negación de su Maestro. La mirada de infinita misericordia de Jesús provoca las lágrimas del arrepentimiento (\emph{Lc} 22,61) y, tras la resurrección del Señor, la triple afirmación de su amor hacia él (cf. \emph{Jn} 21,15-17). La segunda conversión tiene también una dimensión \emph{comunitaria}. Esto aparece en la llamada del Señor a toda la Iglesia: \textquote{¡Arrepiéntete!} (\emph{Ap} 2,5.16).
	
	San Ambrosio dice acerca de las dos conversiones que, \textquote{en la Iglesia, existen el agua y las lágrimas: el agua del Bautismo y las lágrimas de la Penitencia} (\emph{Epistula extra collectionem} 1 {[}41{]}, 12).	
\end{ccebody}


\begin{patercite}
	Mas yo, tirándome debajo de una higuera, no sé cómo, solté la rienda a las lágrimas, brotando dos ríos de mis ojos, sacrificio tuyo aceptable. Y aunque no con estas palabras, pero sí con el mismo sentido, te dije muchas cosas como éstas: \emph{¡Y tú, Señor, hasta cuándo! ¡Hasta cuándo, Señor, has de estar irritado!} No te acuerdes más de nuestras maldades pasadas. Me sentía aún cautivo de ellas y lanzaba voces lastimeras: \textquote{¿Hasta cuándo, hasta cuándo, ¡mañana!, ¡mañana!? ¿Por qué no hoy? ¿Por qué no poner fin a mis torpezas ahora mismo?}.
	
	Decía estas cosas y lloraba con muy dolorosa contrición de mi corazón. Pero he aquí que oigo de la casa vecina una voz, como de niño o niña, que decía cantando y repetía muchas veces: \textquote{\emph{Toma y lee, toma y lee}} (tolle lege, tolle lege).
	
	\ldots{} Así que, apresurado, volví al lugar donde estaba sentado Alipio y yo había dejado el códice del Apóstol al levantarme de allí. Lo tomé, lo abrí y leí en silencio el primer capítulo que se me vino a los ojos, que decía: \emph{No en comilonas y embriagueces, no en lechos y en liviandades, no en contiendas y emulaciones sino revestíos de nuestro Señor Jesucristo y no cuidéis de la carne con demasiados deseos}.
	
	No quise leer más, ni era necesario tampoco, pues al punto que di fin a la sentencia, como si se hubiera infiltrado en mi corazón una luz de seguridad, se disiparon todas las tinieblas de mis dudas.
	
	Después entramos a ver a mi madre, indicándoselo, y se llenó de gozo; le contamos el modo como había sucedido, y saltaba de alegría y cantaba victoria, por lo cual te bendecía a ti, que eres poderoso para darnos más de lo que pedimos o entendemos, porque veía que le habías concedido, respecto de mí, mucho más de lo que constantemente te pedía con sollozos y lágrimas piadosas.
	
	\textbf{San Agustín}, \emph{Confesiones ,} Libro VIII, capítulo 12.
\end{patercite}