% !TeX program = lualatex
%\documentclass[10pt,twoside,openany,showtrims,titlepage]{memoir}
\newcommand\theAuthor{Alfertson Cedano Guerrero }
\newcommand\titleA{Homilías Dominicales (B) *Documento Provisional*}
\newcommand\titleB{Cuaresma - Triduo Pascual}
\newcommand\theKeywords{Homilías\sep Liturgia\sep Cristianismo\sep Iglesia Católica}
\newcommand\theSubject{Esta obra recoge las homilías dominicales de los últimos papas. Contiene además las lecturas bíblicas según el nuevo Leccionario de la Misa, Comentarios Patrísticos a los Evangelios y los Temas del Catecismo de la Iglesia Católica sugeridos por el Directorio Homilético}
\newcommand\theOrg{www.deiverbum.org}
\newcommand\isCopyrighted{FALSE}
\newcommand\thePublicationType{Book}
\newcommand\theVolume{2}
\newcommand\thePublisher{deiverbum.org}
\newcommand\theYear{2021}
\newcommand\volume{Volumen 2}
\newcommand\subTitle{Textos bíblicos del nuevo Leccionario de la Misa, \\Comentarios Patrísticos a los Evangelios \\y Temas del Directorio Homilético}

\RequirePackage{filecontents} %if filecontents should overwrite old files

\begin{filecontents}{\jobname.xmpdata}
	\Title{\titleA \titleB}
	\Author{\theAuthor}
	\Volume{\theVolume}
	\Subject{\theSubject}
	\Keywords{\theKeywords}	
	\Org{\theOrg}
	\Copyrighted{\isCopyrighted}
	\PublicationType{\thePublicationType}
\end{filecontents}

\documentclass[ebook,11pt,twoside,openright]{memoir}
%\usepackage[paperwidth=6in, paperheight=9in,bindingoffset=.75in]{geometry}
\usepackage[x-1a1]{pdfx}
\immediate\pdfobj stream attr{/N 4} file{ISOcoated_v2_300_eci.icc}
\pdfcatalog{%
	/OutputIntents [ <<
	/Type /OutputIntent
	/S/GTS_PDFX
	/DestOutputProfile \the\pdflastobj\space 0 R
	/OutputConditionIdentifier (ISO Coated v2 300 (ECI))
	/Info(ISO Coated v2 300 (ECI))
	/RegistryName (http://www.color.org/)
	>> ]
}
\pdfinfo{% not needed with newer PDF/X versions
	/GTS_PDFXVersion (PDF/X-1:2001)%
	/GTS_PDFXConformance (PDF/X-1a:2001)%
}
% -------------------------------------------------------Packages
\usepackage[no-math]{fontspec}
\usepackage{polyglossia}

%\usepackage{ucs}
\usepackage{lettrine}
\usepackage{csquotes} 
\usepackage{xcolor}  
%\usepackage[showframe, pass]{geometry}
%\usepackage{layout}
%\usepackage{gentium}
%\usepackage{libertine}
%\usepackage{gregoriotex}   
%\usepackage{needspace}  
\usepackage{lipsum} 
\usepackage{alltt} 
%\usepackage[showframe, pass]{geometry}
%\usepackage{showframe}
%\usepackage{atbegshi}
%\usepackage{graphicx}
%
%\usepackage[nobottomtitles*]{titlesec}
\usepackage{mdframed}
\usepackage{graphicx}
\graphicspath{ {../figures/} }
% -------------------------------------------------------Book layout
\setstocksize{9in}{6in}
\settrimmedsize{9in}{6in}{*}
\setbinding{0.4in}
%\setlrmarginsandblock{0.5in}{0.4in}{*}
%\setulmarginsandblock{0.5in}{0.5in}{*}
\setheadfoot{12.07pt}{\footskip}

\setlrmarginsandblock{0.5in}{0.4in}{*}
\setulmarginsandblock{0.5in}{*}{1}
%\setstocksize{9in}{6in}
%\settrimmedsize{\stockheight}{\dimexpr 6in-15mm}{*}
%\settypeblocksize{516.9312pt}{243.6962pt}{*}
\checkandfixthelayout

%\fixpdflayout


%\renewcommand{\bottomtitlespace}{0.25\textheight}

% -------------------------------------------------------Global settings
\setdefaultlanguage{spanish}
\setcounter{tocdepth}{4}
%\settrimmedsize{6in}{9in}{*}

%\setromanfont[BoldFont={Gentium Basic Bold},ItalicFont={Gentium Italic}]{Gentium}
%\setmainfont{Gentium}
%\setmainfont{Noto Serif Georgian}
%\setmainfont{Gentium Plus-R}
%\setmainfont{FreeSerif}
% -------------------------------------------------------Fonts definition
\setmainfont{LinuxLibertineO}
%\setmainfont{TeX Gyre Pagella}
\setsansfont{Ubuntu}
%\setlength{\headheight}{12.4pt}
%\checkandfixthelayout

%\setsansfont{Linux Libertine}

%\newfontfamily\gfont{Gentium}

%\newfontfamily\gbbfont{Gentium Book Basic}
%\newfontfamily\opensansfont{Free Serif}
%\newfontfamily\fsfont{FreeSerif}
%\newfontfamily\ssfont{FreeSans}
%\newfontfamily\alterfont{EB Garamond}[Numbers=Lining]
%\newfontfamily\headerfont{Libertinus Sans}
%\newfontfamily\notoserif{Latin Modern Roman}
\newfontfamily\notoserif{Noto Serif}		
%\newfontfamily\notoserif{Tinos}		
% -------------------------------------------------------Heading formats and styles
\renewcommand{\chapnamefont}{\centering\normalfont} 
%\renewcommand{\chapnumfont}{\centering\normalfont} 
\renewcommand\chaptitlefont{\centering\huge\bfseries\sffamily}

\renewcommand\secheadstyle{\scshape\centering\huge\sffamily\noindent}
\renewcommand\subsecheadstyle{\centering\Large\scshape\sffamily\bfseries\noindent}
\renewcommand\subsubsecheadstyle{\centering\large\sffamily\noindent\ruleline}
%\renewcommand\cftchapterfont{\scshape\MakeTextLowercase}
%\setsecnumdepth{chapter}
\setsecnumdepth{none}
%\sethangfrom{\noindent #1}
\nouppercaseheads

\makeevenhead{headings}{\sffamily \scriptsize \thepage {\enspace \textemdash \enspace}\rightmark}{}{}
\makeoddhead{headings}{}{}{\sffamily \scriptsize \leftmark {\enspace \textemdash \enspace} \thepage}
\makeevenfoot{headings}{}{}{}
\makeoddfoot{headings}{}{}{}


%\setaftersecskip{3\onelineskip plus .2ex}

%\setbeforesubsecskip{2\onelineskip plus .2ex}
%\setbeforesecskip{1.5cm}
%\setaftersecskip{0.5cm}
\setbeforesubsecskip{1cm}
\setaftersubsecskip{1.5cm}
%\setafterparaskip(1ex)
%\setbeforesubsubsecskip{1.5cm}
%\setbeforesecskip{-1\onelineskip plus -1ex minus -.2ex}
%\setaftersecskip{1\onelineskip plus .2ex}
%\setaftersecskip{1.5cm}
%\makeevenhead{headings}{\scriptsize\thepage}{}{\slshape\leftmark}
%\makeoddhead{headings}{\slshape\rightmark}{}{\scriptsize\thepage}
%\pagestyle{headings}
%break after 2nd: 
%\newif\iffirstsection
%\renewcommand{\memendofchapterhook}{\global\firstsectiontrue}
%\setsechook{\iffirstsection\global\firstsectionfalse\else\clearpage\fi}
%\OnehalfSpacing
%\setbeforesubsecskip{11sp}
%\setbeforesecskip{4.5cm}

%\setsechook{\iffirstsubsection\global\firstsubsectionfalse\else\clearpage\fi}
%\setulmargins{4.5iccethn}{*}{*}
%\newenvironment{gbbfontenv}{\gbbfont}{\par}

%\newfontfamily\opensansfont{Open Sans Extra Bold}
%\newfontfamily\opensansfont{Gentium Basic}

%\newfontfamily\opensansfont{Linux Biolinum O}
%\newfontfamily\opensansfont{Free Sans}
%\newenvironment{opensans}{\opensansfont}{\par}


%\abnormalparskip{1\baselineskip}

%\needspace{15\baselineskip}
%--------------------------------------------------------------------- Temporal
\renewcommand{\afterpartskip}{\relax}
\nopartblankpage
% Defined commands
%--------------------------------------------------------------------- Commands
\newcommand{\redtext}[1]{\renewcommand{\redtext}{#1}}
\newcommand{\smalltext}[1]{\renewcommand{\smalltext}{#1}}


%Define color

%Command to use in the context
\newcommand{\txtred}{\textcolor{red}}
\newcommand{\bfred}{\textcolor{red}}
\newcommand{\newsection}{\newpage \vspace*{1cm}}
\newcommand\redbf[1]{\txtred{\textbf{#1}}}


\label{introstyles}
%---------------------------------------------------------------------Intro styles
\newenvironment{bodyintro}{%
	%\fontfamily{gentium}\selectfont
}%

\newenvironment{introfirst}{%
	\notoserif
}%

\newenvironment{introcite}{%
	\notoserif
}%

\newenvironment{introsection}{%
	\addvspace{\topsep}% Space above
}{%
	\unskip\par
	\addvspace{\topsep}% Space below
}

\newenvironment{introstyle}{%
	\setsecheadstyle{\bfseries\large\centering}
	\setsubsecheadstyle{\bfseries\scshape\centering}		
	\setsubsubsecheadstyle{\itshape\raggedright}
	%\setparaheadstyle{\par\itshape\raggedright\par}	
	%\addvspace{\topsep}% Space above
	%\fontt
}{%
	%\unskip\par
	%\addvspace{\topsep}% Space below
}
\label{bodystyles}
%---------------------------------------------------------------------Body styles
\newenvironment{body}{%
	\addvspace{\topsep}% Space above
}{%
	\unskip\par
	\addvspace{\topsep}% Space below
	%\needspace{15\baselineskip}
}

\newenvironment{bodyprose}
{%
	\begin{alltt}\normalfont}
	{\end{alltt}
}



\label{readstyles}
%---------------------------------------------------------------------read styles
\newcommand{\rtitle}[1]{
	{	
		\sffamily
		\footnotesize
		\bfseries
		\abnormalparskip{25pt}
		\noindent 
		\color{red} #1	
	}
}

\newcommand{\rbook}[1]{
	{	
		\notoserif
		\footnotesize
		\bfseries
		\abnormalparskip{5pt}
	    \raggedright
		%\noindent 
		#1	
		%\abnormalparskip{15pt}
	}
}

\newcommand{\rred}{\mbox{}\hfill\txtred}

\newcommand{\rtheme}[1]{
	{	
		\notoserif
		\footnotesize
	    \itshape
	    \color{red} 
	    \abnormalparskip{15pt}
		\noindent 
		#1\bigskip
	}
}

\newenvironment{scripture}
{%
	\par
	\notoserif\small
	\parindent=0pt
	\everypar{\hangindent0.8cm\hangafter=1}%
	\let\>\talk
}
{\par}
\newcommand{\talk}[2]{%
	\par
	{% start a group
		\rightskip=0.8cm
		\hspace*{0.8cm}%
		\textquote{#1}%
		\ifx.#2.\par\else#2\fi
	}% end the group
}

\newenvironment{readprose}
{%
	\begin{alltt}\small\linespread{.5}\notoserif}
	{\end{alltt}
}

\makeatletter
\catcode13\active %
\newenvironment{psbody}{%
	\notoserif
	\small		
	\setlength\parskip{\bigskipamount}%
	\setlength\parindent{0pt}%
	\setlength\leftskip{10mm}%
	\catcode13\active %
	\def^^M{\@ifnextchar^^M{\par}{\@ifnextchar\end{\ifhmode\par\fi}{\ifhmode\\\fi}}}%
\def\par{~\txtred ℟.\endgraf}%
\everypar{\makebox[0pt]{\txtred ℣.\qquad}}%
}{}%
\catcode13=5 %
\makeatother

\newenvironment{readbodyold}{%
	\notoserif
	\small

}%{

\newenvironment{readbodydelete}{\everypar{\hangafter=1 \notoserif \small     	\leftskip1cm\relax \setlength{\hangindent}{3em}}}{\par}


\newenvironment{rprosedelete}{%
	% \begin{cr}{#1}
	%\par% Start a new paragraph
	%\addvspace{\baselineskip}% Space above
	
	\begin{alltt}
		\notoserif\small
	\end{alltt}				
	
	
}
{% \end{cr}
	\par% Start a new paragraph
	\addvspace{\baselineskip}% Space below
}

\newenvironment{firstcite}
{%
	%\label{#1}%
	%\tiny
	%$\langle\textit{#1}\ \rangle\equiv$%
	%\hspace{2in}
	\begin{alltt}\tiny\notoserif\hspace*{2in}}
	{\end{alltt}}


\label{paterstyles}
%---------------------------------------------------------------------Pater styles

\newcommand{\ptheme}[1]{
	{	
		\itshape
\noindent
		\color{red} #1
		\centering\par		
		%\smallskip
	}
}

\mdfsetup{  %% used this to change the setup of the frame
	%outermargin = 2cm, % 2cm left margin 
	innerleftmargin = 1.5em,
	innertopmargin = 4pt,
	innerbottommargin = 4pt,
	innerrightmargin = 1.5em,
	innerleftmargin = 1.5em,	
	innermargin = 40pt,
	leftmargin=40pt,
	rightmargin=40pt,
	linecolor=red,
	linewidth = 0.7pt,  
	topline = false,
	rightline = true,
	bottomline =false,
	skipabove=10pt,
	skipbelow=20pt,
	font={\sffamily\footnotesize},
}

\newenvironment{patercite}{%
	\vspace*{\fill}% 
	\begin{mdframed} 
		\hspace{\parindent}
	}
	{\end{mdframed}}{% 
	\vspace*{\fill}%
}

\newenvironment{paterciteold}{%
	% \begin{cr}{#1}

	\par% Start a new paragraph
	%\addvspace{40pt}% Space above
	  \vspace*{\fill}%
	\sffamily
	\scriptsize	
	\noindent
	\ignorespaces
	\leftskip2cm\relax
	%\rightskip1cm\relax

}
{% \end{cr}
	  \vspace*{\fill}%
	\par% Start a new paragraph
	%\addvspace{\baselineskip}% Space below
}

\label{homstyles}
%--------------------------------------------------------------------- hom styles
\newcommand{\homsec}[1]{%
	{
		\par\raggedright% Start a new paragraph
		\addvspace{\baselineskip}% Space above
		\textbf{\ignorespaces#1}%
		\par
	}%
}

\newcommand*\ruleline[1]{\par\noindent\raisebox{.8ex}{\makebox[\linewidth]{\textcolor{red}{\hrulefill}\hspace{1ex}\raisebox{-.8ex}{#1}\hspace{1ex}\textcolor{red}{\hrulefill}}}}


%--------------------------------------------------------------------- CCE styles

\label{ccestyles}
\newcommand{\cceth}[1]{
	{	
		%\par \addvspace{\onelineskip}% Space above		
		\sffamily
		%\tiny
		\addvspace{25pt}% Space above
		\centering
		\color{red} #1
		\par
	}
}

\newcommand{\cceref}[1]{
	{	
		%\par 
		\sffamily
		%\tiny
		\centering
		\color{red} #1
		\par 
		%\bigskip
	}
}

\newcommand\n[1]{\txtred{\textbf{#1}}}

\newcommand{\ccesec}[1]{%
	{\par\raggedright% Start a new paragraph
		\addvspace{\baselineskip}% Space above
		\textbf{\ignorespaces#1}%
	}%
}

\newenvironment{ccebody}{%
	\noindent\ignorespaces
	\setlength{\parskip}{1ex plus 0.5ex minus 0.2ex}
	\setlength\parindent{0pt}
	%\textit{redtext}\
	%\fontfamily{gentium}\selectfont
	\footnotesize
	\notoserif	
	
}%
{% \end{cr}
	%\nopagebreak
	\par% Start a new paragraph
	\addvspace{\baselineskip}% Space below
}


\newenvironment{cceprose}
{%
	\begin{alltt}\notoserif}
	{\end{alltt}%
}

\newcommand{\ccecite}[1]{
	{	
		\par 
		\notoserif
		\scriptsize #1
		\ignorespaces
		\leftskip1cm\relax
		\rightskip1cm\relax		
		\par%
	}
}


%--------------------------------------------------------------------- Test




\usepackage[colorinlistoftodos,prependcaption,textsize=tiny]{todonotes}

\newcommand{\img}[1]{
	\begin{centering}
	\vfill
	\includegraphics{#1}
	\vfill
	\end{centering}
}

\usepackage{sepfootnotes}
% this defines new footnotes and the commands
%   \Anotecontent{<id>}{<text>}
%   \Anote{<id>}
% and a few more:
\newfootnotes{a}

% input the footnote definitions:


\raggedbottom

\newcommand{\txtbig}[1]{{\Huge \txtred#1}} % version 3

\newcommand{\txtsmall}[1]{{\small\addfontfeatures{LetterSpace=1} #1}} % version 3


\label{otherstyles}

\newcommand{\src}[1]{
	{	
		\sffamily
		\scriptsize
		\centering\par
		\color{red}#1%

	}
	\par
}

\newcommand{\rbr}[1]{
	\par
	{	
		\par 
		\addvspace{\baselineskip}% Space above		
		\noindent\ignorespaces		
		\sffamily
		\scriptsize
		\color{red} #1		
	}
	\par
}

\renewcommand{\LettrineFontHook}{\color{red}}

\makeatletter
%\ltr[«]» 
%“  ” * “ ”
\def\ltr{\@ifnextchar[{\@with}{\@without}}
\def\@with[#1]#2{\lettrine[findent=.4em, ante={#1}]{#2}}
\def\@without#1{\lettrine[findent=.4em]{#1}}
\makeatother


\newenvironment{smallinfo}{%
	\linespread{0.3}
	\vspace*{\fill}\noindent
	\tiny	
}
{%
	
}

%\layout

%\renewcommand\cftchapteraftersnumb{\normalfont}
%\renewcommand\cftbeforechapterskip{5pt plus 1pt}	
%\frontmatter
%\tableofcontents*
%
\renewcommand{\cftchapterpagefont}{\normalfont\sffamily}   
\renewcommand{\cftsectionpagefont}{\normalfont\sffamily} 
%\renewcommand{\cftsectionfont}{\alterfont} 
%\renewcommand{\cftKfont}{\alterfont}
\renewcommand{\cftchapterfont}[1]{\sffamily\scshape#1}
\renewcommand{\cftsectionfont}[1]{\sffamily{#1}}
\renewcommand{\cftsubsectionfont}[1]{\sffamily\bfseries{#1}}
\renewcommand{\cftsubsubsectionfont}[1]{\sffamily{#1}}

\newlength\drop


\iffalse
\newcommand*{\titleTH}{%
	\thispagestyle{empty}

	\begingroup% TH Typography
	\raggedleft
	\vspace*{\baselineskip}
	{\Large The Author}\\[0.167\textheight]
	%{\bfseries The Big Book of}\\[\baselineskip]
	{\textcolor{red}{\Huge \scshape \titleA}}\\[\baselineskip]
	{\textcolor{red}{\huge \titleB}}\\[\baselineskip]	
	{\bfseries \subTitle}\par
	\vfill
	{\Large The Publisher}\par
	\vspace*{3\baselineskip}
	\endgroup}
\fi

\newcommand*{\titleCC}
{\begingroup% City of Cambridge
	\drop=0.1
	\textheight
	\vspace*{\drop}
	\centering
	%{\Large\itshape THE BIG BOOK OF}\\[0.5\drop]
	{\textcolor{red}{\HUGE\bfseries \scshape \titleA}}\\[0.5\drop]
	%\vspace{\drop}
	{\large \bfseries \volume}\\[1\baselineskip]
		%\vspace{\drop}
	%\vspace*{3\baselineskip} 
	{\LARGE \bfseries \titleB}\\[0.5\drop]
	{\large \bfseries \subTitle}
	\vfill{Compilado por \\ \large \scshape Alfertson Cedano Guerrero}\\[0.5\drop]
	{Primera Edición}\\[0.5\drop]	
	%\vfill{\plogo}\\[0.5\baselineskip] 
	{\itshape \thePublisher}\par
	{\scshape \theYear}\par
	%\vfill
	\vspace*{\drop}
	\endgroup}



%\title{\Huge\textbf{ORTHODOX RULE OF PRAYER}}
%\author{\Large\textbf{Father Thomas Moore, Michael Dykes}}
%\date{\Large{\today}}
%\makeatletter
%\title{\xmp@Title}
%\author{\xmp@Author}
%\makeatother
%\usepackage{lineno}
%\linenumbers



	

\begin{document}	
	
	%\frontmatter
	%\pagestyle{empty}
	%\aliaspagestyle{chapter}{empty}
	\titleCC

\centerline {Actualizado: \today}

\newpage

%Text that will now be at the bottom of the page

\begin{smallinfo}
	
	1ª Edición: \\
	
	Sevilla (España), Noviembre del 2020. \\
	
	Alfertson Cedano Guerrero, presbítero \\
	
	a.cedano@deiverbum.org \\
	
	www.deiverbum.org/publicaciones \\
	
	Imagen de portada: \textit{La Natividad} (detalle) \\
	
	Manuscrito iluminado, artista desconocido, Ms. 14 (85.MK.239) \\
	
	J. Paul Getty Museum, Los Ángeles, CA.
\end{smallinfo}

\newsection

\begin{firstcite}
	\begin{vplace}
		\hspace{2.5in}{Yo repuse: }
		\hspace{2.5in}    «¡Ay, Señor, Dios mío! 
		\hspace{2.5in}     Mira que no sé hablar, que solo soy un niño». 
		\hspace{2.5in}Y el Señor me contestó: 
		\hspace{2.5in}   «No digas que eres un niño, 
		\hspace{2.5in}    pues irás adonde yo te envíe 
		\hspace{2.5in}    y dirás lo que yo te ordene». (\ldots) 
		\hspace{2.5in}El Señor extendió la mano, 
		\hspace{2.5in}tocó mi boca y me dijo: 
		\hspace{2.5in}   «Voy a poner mis palabras en tu boca». 
		
		\hspace{2.5in}(Jeremías 1, 6-7.9)	
		
	\end{vplace}
\end{firstcite}

\newsection

\tableofcontents	
	\mainmatter
	
	\label{files}
	\anotecontent{id1}{\label{id1} A partir de esa nueva traducción de la Biblia se publicaron los nuevos Leccionarios de la Misa, que son los leccionarios oficiales desde el mes de septiembre del año 2016.}

\anotecontent{id2}{\label{id2} NUALC n. 40.}

\anotecontent{id3}{\label{id3} Cf. OLM n. 93.}

\anotecontent{id4}{\label{id4} Cf. Congregación para el Culto Divino, \emph{Directorio Homilético} (2014)\emph{,} nn. 78-109.}

\anotecontent{id5}{\label{id5} Cf. Oficio de lecturas, Lunes, I semana de Adviento.}

\anotecontent{id6}{\label{id6} Benedicto XVI, papa, \emph{Homilía}, 28 de noviembre de 2009.}

	%~
	%\titleCC

\centerline {Actualizado: \today}

\newpage

%Text that will now be at the bottom of the page

\begin{smallinfo}
	
	1ª Edición: \\
	
	Sevilla (España), Noviembre del 2020. \\
	
	Alfertson Cedano Guerrero, presbítero \\
	
	a.cedano@deiverbum.org \\
	
	www.deiverbum.org/publicaciones \\
	
	Imagen de portada: \textit{La Natividad} (detalle) \\
	
	Manuscrito iluminado, artista desconocido, Ms. 14 (85.MK.239) \\
	
	J. Paul Getty Museum, Los Ángeles, CA.
\end{smallinfo}

\newsection

\begin{firstcite}
	\begin{vplace}
		\hspace{2.5in}{Yo repuse: }
		\hspace{2.5in}    «¡Ay, Señor, Dios mío! 
		\hspace{2.5in}     Mira que no sé hablar, que solo soy un niño». 
		\hspace{2.5in}Y el Señor me contestó: 
		\hspace{2.5in}   «No digas que eres un niño, 
		\hspace{2.5in}    pues irás adonde yo te envíe 
		\hspace{2.5in}    y dirás lo que yo te ordene». (\ldots) 
		\hspace{2.5in}El Señor extendió la mano, 
		\hspace{2.5in}tocó mi boca y me dijo: 
		\hspace{2.5in}   «Voy a poner mis palabras en tu boca». 
		
		\hspace{2.5in}(Jeremías 1, 6-7.9)	
		
	\end{vplace}
\end{firstcite}

\newsection

\tableofcontents

\section{Comentario Patrístico}

\subsection{San Ambrosio, obispo}

¿Eres tú el que ha de venir o tenemos que esperar a otro?

Comentario sobre el evangelio de san Lucas,

Lib. 5, 93-95. 99-102. 109: CCL 14, 165-166. 167-168. 171-177.

\emph{Juan envió a dos de sus discípulos a preguntar a Jesús: \textquote{¿Eres tú el que ha de venir o tenemos que esperar a otro?}}. No es sencilla la comprensión de estas sencillas palabras, o de lo contrario este texto estaría en contradicción con lo dicho anteriormente. ¿Cómo, en efecto, puede Juan afirmar aquí que desconoce a quien anteriormente había reconocido por revelación de Dios Padre? ¿Cómo es que entonces conoció al que previamente desconocía mientras que ahora parece desconocer al que ya antes conocía? Yo ---dice--- \emph{no lo conocía, pero el que me envió a bautizar con agua me dijo: \textquote{Aquel sobre quien veas bajar el Espíritu Santo\ldots{}}. Y Juan dio fe al oráculo, reconoció al revelado, adoró al bautizado y profetizó al enviado Y concluye: \emph{Y yo lo he visto, y he dado testimonio de que éste es el elegido de Dios}. ¿Cómo, pues, aceptar siquiera la posibilidad de que un profeta tan grande haya podido equivocarse, hasta el punto de no considerar aún como Hijo de Dios a aquel de quien había afirmado: \emph{Éste es el que quita el pecado del mundo?}
	

	

	
	

	
	%\chapter{Introducción General}

\begin{bodyintro}
	Durante más de veinte años me he dedicado a recopilar las homilías litúrgicas y comentarios a los Evangelios de los Padres de la Iglesia y de los últimos sucesores de Pedro. Ha sido un trabajo paciente que ha supuesto no solamente recopilar, sino también traducir, ordenar, clasificar, corregir \ldots porque muchas de las fuentes de las que provenía el texto eran antiguas y los errores (tipográficos y ortográficos) varios.
	
	Esta obra recoge todo ese trabajo y también alguna novedad. El contenido se ha organizado de acuerdo a la siguiente estructura:
	
	1. Las \textbf{lecturas bíblicas} a partir de los textos de la \textit{Sagrada Biblia, versión oficial de la Conferencia Episcopal Española}\anote{id1}\label{fn1};
	
	2. Uno o varios \textbf{comentarios patrísticos} relativos al Evangelio o a la fiesta del día;
	
	3. Varias \textbf{homilías}(u otros escritos en su defecto), generalmente de los últimos papas o de los padres de la Iglesia, para la celebración correspondiente;
	
	4. Los \textbf{temas doctrinales} que sugiere el \textit{Directorio Homilético} para esa celebración, acompañados de los textos del \textit{Catecismo de la Iglesia Católica} indicados por ese mismo \textit{Directorio}.
	
	Son cuatro pilares sobre los que podremos apoyarnos en la preparación de las homilías, sabiendo que cada contexto es distinto y que cada predicación, también la nuestra, es una obra de arte en la que se mezcla nuestra criatura de barro con la asistencia consoladora del Espíritu Santo.
\end{bodyintro}

\newpage

\section {Plan de la Colección}

\begin{bodyintro}
	Las \textbf{homilías dominicales} estarán agrupadas por ciclo litúrgico (A, B, y C) y dentro de cada ciclo cinco volúmenes (cada uno con unas 400 páginas). La intención es recoger en una obra todas las homilías de los últimos pontífices de una forma exhaustiva (al menos a partir de Juan Pablo II).
	
	La distribución es la siguiente:

\begin{itemize}
	\item \textbf {Homilías Dominicales (A)}
	\begin{enumerate}
		\renewcommand{\labelenumii}{\arabic{enumii}.}
		\item Adviento-Navidad
		\item Cuaresma-Triduo Pascual
		\item Pascua
		\item Tiempo Ordinario (Semanas II-XVII) *
		\item Tiempo Ordinario (Semanas XVIII-XXXIV)
	\end{enumerate}
\end{itemize}

\begin{itemize}
	\item \textbf {Homilías Dominicales (B)}
	\begin{enumerate}
		\renewcommand{\labelenumii}{\arabic{enumii}.}
		\item Adviento-Navidad
		\item Cuaresma-Triduo Pascual
		\item Pascua
		\item Tiempo Ordinario (Semanas II-XVII) *
		\item Tiempo Ordinario (Semanas XVIII-XXXIV)
	\end{enumerate}
\end{itemize}

\begin{itemize}
	\item \textbf {Homilías Dominicales (C)}
	\begin{enumerate}
		\renewcommand{\labelenumii}{\arabic{enumii}.}
		\item Adviento-Navidad
		\item Cuaresma-Triduo Pascual
		\item Pascua
		\item Tiempo Ordinario (Semanas II-XVII) *
		\item Tiempo Ordinario (Semanas XVIII-XXXIV)
	\end{enumerate}
\end{itemize}


	\small *Incluye también las tres celebraciones del Señor durante el Tiempo Ordinario (Santísima Trinidad, Corpus Christi y Sagrado Corazón de Jesús).
\end{bodyintro}
	
	%\subsection{En este volumen}

Se recogen *** homilías u otras intervenciones de los Papas, de las cuales *** han sido traducidas de su original italiano. *** comentarios patrísticos y unos *** incisos aprovechando los espacios en blanco para seguir nutriendo esta obra con verdaderos tesoros sacados de la fuente inagotable de la Tradición.
	
	%\section{Tiempo de Adviento}

\subsection{Introducción}

El tiempo de Adviento comienza con las primeras vísperas del domingo que cae el 30 de noviembre, o más próximo a ese día, y concluye antes de las primeras vísperas de Navidad.\anote{id2}

Este tiempo tiene dos características: es a la vez un tiempo de preparación a las solemnidades de Navidad en que se conmemora la primera Venida del Hijo de Dios entre los hombres, y un tiempo en el cual, mediante esta celebración, el ánimo se dirige a esperar la segunda Venida de Cristo al fin de los tiempos. Por estos dos motivos, el Adviento se presenta como un tiempo de piadosa y alegre esperanza.

\textbf{Lecturas de los domingos de Adviento}\anote{id3}

En los domingos de Adviento las lecturas del Evangelio tienen una característica propia: se refieren a la venida del Señor al final de los tiempos (primer domingo), a Juan Bautista (segundo y tercer domingos), a
los acontecimientos que prepararon de cerca el nacimiento del Señor (cuarto domingo).

Las lecturas del Antiguo Testamento son profecías sobre el Mesías y el tiempo mesiánico, tomadas principalmente del libro de Isaías.

Las lecturas del Apóstol contienen exhortaciones y amonestaciones conformes a las diversas características de este tiempo.

\textbf{Algunos aspectos que debe tener en cuenta el homileta}\anote{id4}

El Adviento es el tiempo que prepara a los cristianos a las gracias que serán dadas, una vez más en este año, en la celebración de la gran Solemnidad de la Navidad. Ya desde el I domingo de Adviento, el homileta exhorta al pueblo para que emprenda su preparación caracterizada por distintas facetas, cada una de ellas sugerida por la rica selección de pasajes bíblicos del Leccionario de este tiempo. La primera fase del Adviento nos invita a preparar la Navidad animándonos no sólo a dirigir la mirada al tiempo de la primera Venida de nuestro Señor, cuando, como dice el prefacio I de Adviento, Él asume \textquote{la humildad de nuestra carne}, sino también, a esperar vigilantes su Venida \textquote{en la majestad de su gloria}, cuando \textquote{podamos recibir los bienes prometidos}.

En la predicación no debe perderse de vista que existe un doble significado del Adviento, un doble significado de la Venida del Señor. Este tiempo nos prepara para su Venida en la gracia de la fiesta de la Navidad y a su retorno para el juicio al final de los tiempos. Los textos bíblicos deberían ser explicados considerando este doble significado. Según el texto, se puede evidenciar una u otra Venida, aunque, con frecuencia, el mismo pasaje presenta palabras e imágenes relativas a ambas.

Existe, además, otra Venida: escuchamos la Palabra de Dios en la asamblea eucarística, donde Cristo está verdaderamente presente. Al comienzo del tiempo de Adviento la Iglesia recuerda la enseñanza de san Bernardo, es decir, que entre las dos Venidas visibles de Cristo, en la historia y al final de los tiempos, existe una venida invisible, aquí y ahora\anote{id5}, así como hace suyas las palabras de san Carlos Borromeo:

\textquote{Este tiempo (\ldots{}) nos enseña que la venida de Cristo no solo aprovechó a los que vivían en el tiempo del Salvador, sino que su eficacia continúa y aún hoy se nos comunica si queremos recibir, mediante la fe y los sacramentos, la gracia que él nos prometió, y si ordenamos nuestra conducta conforme a sus mandamientos\anote{id6}}.

\textbf{Una reflexión sobre el Adviento}\anote{id7}

{[}El término Adviento significa \textquote{venida}, \emph{parousia}, en latín \emph{adventus}, de donde viene el término Adviento (cf. \emph{1 Ts} 5, 23){]}.

Reflexionemos brevemente sobre el significado de esta palabra, que se puede traducir por \textquote{presencia}, \textquote{llegada}, \textquote{venida}. En el lenguaje del mundo antiguo era un término técnico utilizado para indicar la llegada de un funcionario, la visita del rey o del emperador a una provincia. Pero podía indicar también la venida de la divinidad, que sale de su escondimiento para manifestarse con fuerza, o que se celebra presente en el culto. Los cristianos adoptaron la palabra \textquote{Adviento} para expresar su relación con Jesucristo: Jesús es el Rey, que ha entrado en esta pobre \textquote{provincia} denominada tierra para visitar a todos; invita a participar en la fiesta de su Adviento a todos los que creen en él, a todos los que creen en su presencia en la asamblea litúrgica. Con la palabra \emph{adventus} se quería decir substancialmente: Dios está aquí, no se ha retirado del mundo, no nos ha dejado solos. Aunque no podamos verlo o tocarlo, como sucede con las realidades sensibles, él está aquí y viene a visitarnos de múltiples maneras.

Por lo tanto, el significado de la expresión \textquote{Adviento} comprende también el de \emph{visitatio}, que simplemente quiere decir \textquote{visita}; en este caso se trata de una visita de Dios: él entra en mi vida y quiere dirigirse a mí. En la vida cotidiana todos experimentamos que tenemos poco tiempo para el Señor y también poco tiempo para nosotros. Acabamos dejándonos absorber por el \textquote{hacer}. ¿No es verdad que con frecuencia es precisamente la actividad lo que nos domina, la sociedad con sus múltiples intereses lo que monopoliza nuestra atención? ¿No es verdad que se dedica mucho tiempo al ocio y a todo tipo de diversiones? A veces las cosas nos \textquote{arrollan}.

El Adviento, este tiempo litúrgico fuerte que estamos comenzando, nos invita a detenernos, en silencio, para captar una presencia. Es una invitación a comprender que los acontecimientos de cada día son gestos que Dios nos dirige, signos de su atención por cada uno de nosotros. ¡Cuán a menudo nos hace percibir Dios un poco de su amor! Escribir ---por decirlo así--- un \textquote{diario interior} de este amor sería una tarea hermosa y saludable para nuestra vida. El Adviento nos invita y nos estimula a contemplar al Señor presente. La certeza de su presencia, ¿no debería ayudarnos a ver el mundo de otra manera? ¿No debería ayudarnos a considerar toda nuestra existencia como \textquote{visita}, como un modo en que él puede venir a nosotros y estar cerca de nosotros, en cualquier situación?

Otro elemento fundamental del Adviento es la espera, una espera que es al mismo tiempo esperanza. El Adviento nos impulsa a entender el sentido del tiempo y de la historia como \textquote{\emph{kairós}}, como ocasión propicia para nuestra salvación. Jesús explicó esta realidad misteriosa en muchas parábolas: en la narración de los siervos invitados a esperar el regreso de su dueño; en la parábola de las vírgenes que esperan al esposo; o en las de la siembra y la siega. En la vida, el hombre está constantemente a la espera: cuando es niño quiere crecer; cuando es adulto busca la realización y el éxito; cuando es de edad avanzada aspira al merecido descanso. Pero llega el momento en que descubre que ha esperado demasiado poco si, fuera de la profesión o de la posición social, no le queda nada más que esperar. La esperanza marca el camino de la humanidad, pero para los cristianos está animada por una certeza: el Señor está presente a lo largo de nuestra vida, nos acompaña y un día enjugará también nuestras lágrimas. Un día, no lejano, todo encontrará su cumplimiento en el reino de Dios, reino de justicia y de paz.

Existen maneras muy distintas de esperar. Si el tiempo no está lleno de un presente cargado de sentido, la espera puede resultar insoportable; si se espera algo, pero en este momento no hay nada, es decir, si el presente está vacío, cada instante que pasa parece exageradamente largo, y la espera se transforma en un peso demasiado grande, porque el futuro es del todo incierto. En cambio, cuando el tiempo está cargado de sentido, y en cada instante percibimos algo específico y positivo, entonces la alegría de la espera hace más valioso el presente. {[}\ldots{}{]} Vivamos intensamente el presente, donde ya nos alcanzan los dones del Señor, vivámoslo proyectados hacia el futuro, un futuro lleno de esperanza. De este modo, el Adviento cristiano es una ocasión para despertar de nuevo en nosotros el sentido verdadero de la espera, volviendo al corazón de nuestra fe, que es el misterio de Cristo, el Mesías esperado durante muchos siglos y que nació en la pobreza de Belén. Al venir entre nosotros, nos trajo y sigue ofreciéndonos el don de su amor y de su salvación. Presente entre nosotros, nos habla de muchas maneras: en la Sagrada Escritura, en el año litúrgico, en los santos, en los acontecimientos de la vida cotidiana, en toda la creación, que cambia de aspecto si detrás de ella se encuentra él o si está ofuscada por la niebla de un origen y un futuro inciertos.

Nosotros podemos dirigirle la palabra, presentarle los sufrimientos que nos entristecen, la impaciencia y las preguntas que brotan de nuestro corazón. Estamos seguros de que nos escucha siempre. Y si Jesús está presente, ya no existe un tiempo sin sentido y vacío. Si él está presente, podemos seguir esperando incluso cuando los demás ya no pueden asegurarnos ningún apoyo, incluso cuando el presente está lleno de dificultades.

{[}\ldots{}{]} El Adviento es el tiempo de la presencia y de la espera de lo eterno. Precisamente por esta razón es, de modo especial, el tiempo de la alegría, de una alegría interiorizada, que ningún sufrimiento puede eliminar. La alegría por el hecho de que Dios se ha hecho niño. Esta alegría, invisiblemente presente en nosotros, nos alienta a caminar confiados. La Virgen María, por medio de la cual nos ha sido dado el Niño Jesús, es modelo y sostén de este íntimo gozo. Que ella, discípula fiel de su Hijo, nos obtenga la gracia de vivir este tiempo litúrgico vigilantes y activos en la espera. Amén.	
	%\input{01b-part1-chapter01}

	%			\chapter{Nota final}

Esta obra nació en un momento de oscuridad: la epidemia del Coronavirus vino de improviso y con ella un confinamiento de muchos días. ¿Qué hacer con tantas horas que antes ocupaba en la misión de manera presencial? Me decidí por preparar y publicar esta colección, esperando que pueda servir también a otros hermanos sacerdotes y diáconos.

El precio de los libros se ha establecido con el menor margen de ganancia posible, porque los mismos no se publican con fines de lucro, sino sobre todo con una finalidad pastoral (quienes no puedan comprarlos podrán descargarlos en formato electrónico sin coste alguno). Las ganancias que me revengan serán destinadas a la ayuda pastoral de mi Parroquia San Pedro Claver en la República Dominicana. Es el lugar donde he recibido la fe y en el que la he vivido durante todos estos años unido a una comunidad de hermanos con rostros y vidas concretos. La tierra buena, donde Dios quiso plantarme.

Mi deseo sincero es que esta obra, nacida en un momento difícil para millones de personas, sea una luz a través del ministerio de la predicación.

Sevilla (España)

31 de mayo del 2020

Solemnidad de Pentecostés \\y Visitación de la Bienaventurada Virgen María

\begin{patercite}\textbf{El Espíritu Santo en la experiencia del desierto}\end{patercite}

\begin{patercite}Al \textquote{comienzo} de la misión mesiánica de Jesús vemos otro hecho interesante y sugestivo, narrado por los evangelistas, que lo hacen depender de la acción del Espíritu Santo: se trata de \textit{la experiencia del desierto}. Leemos en el evangelio según san Marcos: \textquote{A continuación (del bautismo), \textit{el Espíritu le empuja al desierto}} (\textit{Mc} 1, 12). Además, Mateo (4, 1) y Lucas (4, 1) afirman que Jesús \textquote{fue conducido por el Espíritu al desierto}. Estos textos ofrecen puntos de reflexión que nos llevan a una ulterior investigación sobre el misterio de la íntima unión de Jesús-Mesías con el Espíritu Santo, ya desde el inicio de la obra de la redención.\end{patercite}

\begin{patercite}En primer lugar, una observación de carácter lingüístico: los verbos usados por los evangelistas (\textquote{fue conducido} por Mateo y Lucas; \textquote{le empuja}, por Marcos) expresan \textit{una iniciativa especialmente enérgica} por parte del Espíritu Santo, iniciativa que se inserta en la lógica de la vida espiritual y en la misma psicología de Jesús: acaba de recibir de Juan un \textquote{bautismo de penitencia}, y por ello siente la necesidad de un período de reflexión y de austeridad, aunque personalmente no tenía necesidad de penitencia, dado que estaba \textquote{lleno de gracia} y era \textquote{santo} desde el momento de su concepción (cf. \textit{Jn} 1, 14; \textit{Lc} 1, 35): como preparación para su ministerio mesiánico. Su misión le exige también vivir en medio de los hombres-pecadores, a quienes ha sido enviado a evangelizar y salvar (cf. santo Tomás, \textit{Summa Theol.,} III, q. 40, a. 1), en lucha contra el poder del demonio. De aquí la conveniencia de esta pausa en el desierto \textquote{\textit{para ser tentado por el diablo}}. Por lo tanto, Jesús sigue el impulso interior y se dirige adonde le sugiere el Espíritu Santo.\end{patercite}

\begin{patercite}\textit{El desierto}, además de ser lugar de encuentro con Dios, es también lugar de tentación y de lucha espiritual. Durante la peregrinación a través del desierto, que se prolongó durante cuarenta años, el pueblo de Israel había sufrido muchas tentaciones y había cedido (cf. \textit{Ex} 32, 1-6; \textit{Nm} 14, 1-4; 21, 4-5; 25, 1-3; \textit{Sal} 78, 17; \textit{1 Co} 10, 7-10). Jesús va al desierto, casi remitiéndose a la experiencia histórica de su pueblo. Pero, a diferencia del comportamiento de Israel, en el momento de inaugurar su actividad mesiánica, es sobre todo \textit{dócil a la acción del Espíritu Santo, que le pide desde el interior aquella definitiva preparación para el cumplimiento de su misión}. Es un período de soledad y de prueba espiritual, que supera con la ayuda de la palabra de Dios y con la oración. \end{patercite}

\begin{patercite}En el espíritu de la tradición bíblica, y en la línea con la psicología israelita, aquel número de \textquote{cuarenta días} podía relacionarse fácilmente con otros acontecimientos históricos, llenos de significado para la historia de la salvación: los cuarenta días del diluvio (cf. \textit{Gn} 7, 4. 17); los cuarenta días de permanencia de Moisés en el monte (cf. \textit{Ex} 24, 18); los cuarenta días de camino de Elías, alimentado con el pan prodigioso que le había dado nueva fuerza (cf. \textit{1 R} 19, 8). Según los evangelistas, Jesús, bajo la moción del Espíritu Santo, se acomoda, en lo que se refiere a la permanencia en el desierto, a este número tradicional y casi sagrado (cf. \textit{Mt} 4, 1; Lc 4, 1). Lo mismo hará también en el período de las apariciones a los Apóstoles tras la resurrección y la ascensión al cielo (cf. \textit{Hch} 1, 3). [+]\end{patercite}

\begin{patercite}\textbf{San Juan Pablo II, papa}. \textit{Catequesis,} audiencia general, 21 de julio 1990, nn. 1-2.\end{patercite}

\begin{patercite}[+] Jesús, por tanto, es conducido al desierto con el fin de afrontar \textit{las tentaciones de Satanás} y para que pueda tener, a la vez, un contacto más libre e íntimo con el Padre. Aquí conviene tener presente que los evangelistas suelen presentarnos \textit{el desierto} como \textit{el lugar donde reside Satanás}: baste recordar el pasaje de Lucas sobre el \textquote{espíritu inmundo} que \textquote{cuando sale del hombre, anda vagando por lugares áridos, en busca de reposo\ldots} (\textit{Lc} 11, 24); y en el pasaje que nos narra el episodio del endemoniado de Gerasa que \textquote{era empujado por el demonio al desierto} (\textit{Lc} 8, 29).\end{patercite}

\begin{patercite}En el caso de las tentaciones de Jesús, el ir al desierto es obra del Espíritu Santo, y ante todo significa el inicio de una demostración –se podría decir, incluso, de una nueva toma de conciencia– de la lucha que deberá mantener hasta el final de su vida contra Satanás, artífice del pecado. Venciendo sus tentaciones, manifiesta su propio poder salvífico sobre el pecado y la llegada del reino de Dios, como dirá un día: \textquote{Si por el Espíritu de Dios expulso yo los demonios, es que ha llegado a vosotros el reino de Dios} (\textit{Mt} 12, 28).\end{patercite}

\begin{patercite}\textbf{San Juan Pablo II, papa}. \textit{Catequesis,} audiencia general, 21 de julio 1990, n. 3.\end{patercite}

\begin{patercite}[+] Si observamos bien, en las tentaciones sufridas y vencidas por Jesús durante la \textquote{experiencia del desierto} se nota la oposición de Satanás contra la llegada del reino de Dios al mundo humano, directa o indirectamente expresada en los textos de los evangelistas. Las respuestas que da Jesús al tentador desenmascaran las intenciones esenciales del \textquote{padre de la mentira} (\textit{Jn} 8, 44), que trata de servirse, de modo perverso, de las palabras de la Escritura para alcanzar sus objetivos. Pero Jesús lo refuta apoyándose en la misma palabra de Dios, aplicada correctamente. La narración de los evangelistas incluye, tal vez, alguna reminiscencia y establece un paralelismo tanto con las análogas tentaciones del pueblo de Israel en los cuarenta años de peregrinación por el desierto (la búsqueda de alimento: cf. \textit{Dt} 8, 3; \textit{Ex} 16; la pretensión de la protección divina para satisfacerse a sí mismos: cf. \textit{Dt} 6, 16; \textit{Ex} 17, 1-7; la idolatría: cf. \textit{Dt} 6, 13; \textit{Ex} 32, 1-6), como con diversos momentos de la vida de Moisés. Pero se podría decir que el episodio entra específicamente en la historia de Jesús por su lógica biográfica y teológica. Aún estando libre de pecado, Jesús pudo conocer las seducciones externas del mal (cf. \textit{Mt} 16, 23): y era conveniente que fuese tentado para llegar a ser el Nuevo Adán, nuestro guía, nuestro redentor clemente (cf. \textit{Mt} 26, 36-46; \textit{Hb} 2, 10. 17-18; 4, 15; 5, 2. 7-9).\end{patercite}

\begin{patercite}En el fondo de todas las tentaciones estaba la perspectiva de \textit{un mesianismo político y} \textit{glorioso}, como se había difundido y había penetrado en el alma del pueblo de Israel. El diablo trata de inducir a Jesús a acoger esta falsa perspectiva, \end{patercite}

\begin{patercite}porque es el enemigo del plan de Dios, de su ley, de su economía de salvación, y por tanto de Cristo, como aparece claro por el evangelio y los demás escritos del Nuevo Testamento (cf. \textit{Mt} 13, 39; \textit{Jn} 8, 44; 13, 2; \textit{Hch} 10, 38; \textit{Ef} 6, 11; \textit{1 Jn} 3, 8, etc.). Si también Cristo cayese, el imperio de Satanás, que se gloría de ser el amo del mundo (\textit{Lc} 4, 5-6), obtendría la victoria definitiva en la historia. Aquel momento de la lucha en el desierto es, por consiguiente, decisivo.\end{patercite}

\begin{patercite}Jesús es consciente de ser enviado por el Padre para hacer presente el reino de Dios entre los hombres. Con ese fin acepta la tentación, tomando su lugar entre los pecadores, como había hecho ya en el Jordán, para servirles a todos de ejemplo (cf. San Agustín, \textit{De Trinitate}, 4, 13). Pero, por otra parte, en virtud de la \textquote{unción} del Espíritu Santo, llega a las mismas raíces del pecado y derrota al \textquote{padre de la mentira} (\textit{Jn} 8, 44). Por eso, va voluntariamente al encuentro de la tentación desde el comienzo de su ministerio, siguiendo el impulso del Espíritu Santo (cf. San Agustín, \textit{De Trinitate}, 13, 13).\end{patercite}

\begin{patercite}Un día, dando cumplimiento a su obra, podrá proclamar: \textquote{Ahora es el juicio de este mundo; ahora el príncipe de este mundo será echado fuera} (\textit{Jn} 12, 31). Y la víspera de su pasión repetirá una vez más: \textquote{Llega el príncipe de este mundo. En mí no tiene ningún poder} (\textit{Jn} 14, 30); es más, \textquote{el príncipe de este mundo está (ya) juzgado} (\textit{Jn} 16, 11); \textquote{¡Ánimo!, yo he vencido al mundo} (\textit{Jn} 16, 33). La lucha contra el \textquote{padre de la mentira}, que es el \textquote{principe de este mundo}, iniciada en el desierto, alcanzará su culmen en el Gólgota: la victoria se alcanzará por medio de la cruz del Redentor.\end{patercite}

\begin{patercite}Estamos, por tanto, llamados a reconocer el valor integral del desierto como lugar de una particular experiencia de Dios, como sucedió con Moisés (cf. \textit{Ex} 24, 18), con Elías (\textit{1 R} 19, 8), y sobre todo con Jesús que, \textquote{conducido} por el Espíritu Santo, acepta realizar la misma experiencia: \textit{el contacto con Dios Padre} (cf. \textit{Os} 2, 16) \textit{en lucha contra las potencias} \textit{opuestas a Dios}. Su experiencia es ejemplar, y nos puede servir también como lección sobre la necesidad de la penitencia, no para Jesús que estaba libre de pecado, sino para todos nosotros. Jesús mismo un día alertará a sus discípulos sobre la necesidad \textit{de la oración y del ayuno} para echar a los \textquote{espíritus inmundos} (cf. \textit{Mc} 9, 29) y, en la tensión de la solitaria oración de Getsemaní, recomendará a los Apóstoles presentes: \textquote{\textit{Velad y orad,} \textit{para que no caigáis en tentación}; que el espíritu está pronto, pero la carne es débil} (\textit{Mc} 14, 38). Seamos conscientes de que, amoldándonos a Cristo victorioso en la experiencia del desierto, también nosotros tendremos un divino confortador: el Espíritu Santo Paráclito, pues el mismo Cristo ha prometido que \textquote{recibirá de lo suyo} y nos lo dará (cf. \textit{Jn} 16, 14): Él, que condujo al Mesías al desierto no sólo \textquote{para ser tentado} sino también para que diera la primera demostración de su poderosa victoria sobre el diablo y sobre su reino, tomará de la victoria de Cristo sobre el pecado y sobre Satanás, su primer artífice, para hacer partícipe de ella a todo el que sea tentado.\end{patercite}

\begin{patercite}\textbf{San Juan Pablo II, papa}. \textit{Catequesis,} audiencia general, 21 de julio 1990, nn. 4-6.\end{patercite}

		
	%\chapter{Introducción General}

\begin{bodyintro}Durante más de veinte años me he dedicado a recopilar las homilías litúrgicas y comentarios a los Evangelios de los Padres de la Iglesia y de los últimos sucesores de Pedro. Ha sido un trabajo paciente que ha supuesto no solamente recopilar, sino también traducir, ordenar, clasificar, corregir… porque muchas de las fuentes de las que provenía el texto eran antiguas y los errores (tipográficos y ortográficos) varios.\end{bodyintro}

\begin{bodyintro}Esta obra recoge todo ese trabajo y también alguna novedad. El contenido se ha organizado de acuerdo a la siguiente estructura:\end{bodyintro}

\begin{bodyintro}1. Las \textbf{lecturas bíblicas} a partir de los textos de la \textit{Sagrada Biblia, versión oficial de la Conferencia Episcopal Española}\anote{id1};\end{bodyintro}

\begin{bodyintro}2. Uno o varios \textbf{comentarios patrísticos} relativos al Evangelio o a la fiesta del día;\end{bodyintro}

\begin{bodyintro}3. Varias \textbf{homilías }(u otros escritos en su defecto), generalmente de los últimos papas o de los padres de la Iglesia, para la celebración correspondiente;\end{bodyintro}

\begin{bodyintro}4. Los \textbf{temas doctrinales} que sugiere el \textit{Directorio Homilético} para esa celebración, acompañados de los textos del \textit{Catecismo de la Iglesia Católica} indicados por ese mismo \textit{Directorio}.\end{bodyintro}

\begin{bodyintro}Son cuatro pilares sobre los que podremos apoyarnos en la preparación de las homilías, sabiendo que cada contexto es distinto y que cada predicación, también la nuestra, es una obra de arte en la que se mezcla nuestra criatura de barro con la asistencia consoladora del Espíritu Santo.\end{bodyintro}
\newpage		
\section {Plan de la Colección}


\begin{bodyintro}Las \textbf{homilías dominicales} estarán agrupadas por ciclo litúrgico (A, B, y C) y dentro de cada ciclo cinco volúmenes (cada uno con unas 400 páginas). La intención es recoger en una obra todas las homilías de los útimos pontífices de una forma exhaustiva (al menos a partir de Juan Pablo II).\end{bodyintro}

\begin{bodyintro}La distribución es la siguiente:\end{bodyintro}

\begin{itemize}
	\item \textbf {Homilías Dominicales (A)}
	\begin{enumerate}
		\renewcommand{\labelenumii}{\arabic{enumii}.}
		\item Adviento-Navidad
		\item Cuaresma-Triduo Pascual
		\item Pascua
		\item Tiempo Ordinario (Semanas II-XVII) *
		\item Tiempo Ordinario (Semanas XVIII-XXXIV)
	\end{enumerate}
\end{itemize}

\begin{itemize}
	\item \textbf {Homilías Dominicales (B)}
	\begin{enumerate}
		\renewcommand{\labelenumii}{\arabic{enumii}.}
		\item Adviento-Navidad
		\item Cuaresma-Triduo Pascual
		\item Pascua
		\item Tiempo Ordinario (Semanas II-XVII) *
		\item Tiempo Ordinario (Semanas XVIII-XXXIV)
	\end{enumerate}
\end{itemize}

\begin{itemize}
	\item \textbf {Homilías Dominicales (C)}
	\begin{enumerate}
		\renewcommand{\labelenumii}{\arabic{enumii}.}
		\item Adviento-Navidad
		\item Cuaresma-Triduo Pascual
		\item Pascua
		\item Tiempo Ordinario (Semanas II-XVII) *
		\item Tiempo Ordinario (Semanas XVIII-XXXIV)
	\end{enumerate}
\end{itemize}


\begin{bodyintro}\small *Incluye también las tres celebraciones del Señor durante el Tiempo Ordinario (Santísima Trinidad, Corpus Christi y Sagrado Corazón de Jesús).\end{bodyintro}

\subsection{En este volumen}

Se recogen *** homilías u otras intervenciones de los Papas, de las cuales *** han sido traducidas de su original italiano. *** comentarios patrísticos y unos *** incisos aprovechando los espacios en blanco para seguir nutriendo esta obra con verdaderos tesoros sacados de la fuente inagotable de la Tradición.

  
	%\part{Tiempo de Cuaresma}
\chapter{Introducción al Tiempo~de~Cuaresma}
\begin{introstyle}
\section{Normativa litúrgica}
El tiempo de Cuaresma\anote{id2} está ordenado a la preparación de la celebración de Pascua. En efecto, la liturgia cuaresmal dispone a la celebración del Misterio Pascual, tanto a los catecúmenos, haciéndolos pasar por los diversos grados de la iniciación cristiana, como a los fieles, que recuerdan el bautismo y hacen penitencia\anote{id3}. Este tiempo va desde el Miércoles de Ceniza hasta la Misa de la Cena del Señor, exclusive. Desde el comienzo de Cuaresma hasta la Vigilia Pascual no se dice \textit{Aleluya}.

El miércoles que comienza la Cuaresma, que es en todas partes día de ayuno\anote{id4}, se imponen las cenizas.

Los domingos de este tiempo se llaman: primer, segundo, tercer, cuarto, y quinto, domingo de Cuaresma. El sexto domingo, con el que comienza la Semana Santa, se llama \textquote{Domingo de Ramos de la Pasión del Señor}.

La Semana Santa está destinada a conmemorar la Pasión de Cristo desde su entrada mesiánica en Jerusalén. 

Durante la mañana del Jueves Santo\anote{id5}, el Obispo, que concelebra la Misa con su presbiterio, bendice los óleos sagrados y consagra el santo crisma.

\newpage
\section{Lecturas del Leccionario}

Las lecturas\anote{id6} del Evangelio están distribuidas de la siguiente manera: en los domingos primero y segundo se conservan las narraciones de las tentaciones y de la transfiguración del Señor, según el Evangelio de Marcos. 

En los tres domingos siguientes se leen unos textos de san Juan sobre la futura glorificación de Cristo por su cruz y resurrección. Los Evangelios de la samaritana, del ciego de nacimiento y de la resurrección de Lázaro, que se leen en el año A, pueden leerse también en los años B y C, sobre todo cuando hay catecúmenos, debido a su gran importancia en relación con la iniciación cristiana.

El domingo de Ramos en la Pasión del Señor: para la procesión, se han escogido los textos que se refieren a la solemne entrada del Señor en Jerusalén. En el año B se puede optar por el texto de Marcos o de Juan, sobre el mismo tema; en la misa, se lee el relato de la pasión del Señor según san Marcos.

Las lecturas del Antiguo Testamento se refieren a la historia de la salvación, que es uno de los temas propios de la catequesis cuaresmal. Cada año hay una serie de textos que presentan los principales elementos de esta historia, desde el principio hasta la promesa de la nueva alianza. 

Las lecturas del Apóstol se han escogido de manera que tengan relación con las lecturas del Evangelio y del Antiguo Testamento y haya, en lo posible, una adecuada conexión entre las mismas.

Para el año B concretamente en el primer domingo leemos como primera lectura un pasaje sobre la alianza de Dios con Noé después del diluvio, el cual el apóstol Pedro en su primera carta relaciona con el Bautismo cristiano. 

El segundo domingo la primera lectura narra el sacrificio de Isaac, en quien el apóstol Pablo en la carta a los Romanos ve una imagen del sacrificio de Cristo.

El tercer domingo la primera lectura trae el tema del Decálogo, cuyo cumplimiento se realiza en la cruz de Cristo como relata san Pablo en la primera carta a los Corintios.

El cuatro domingo la primera lectura habla sobre la destrucción del tempo y el exilio del pueblo elegido a causa de su infidelidad. La respuesta de Dios es mostrada por san Pablo en la carta a los Efesios: Dios envía de nuevo su salvación por medio de Jesucristo.


\newpage 
\section{Indicaciones pastorales y de piedad sobre la Cuaresma}

La Cuaresma\anote{id7} es un tiempo de escucha de la Palabra de Dios y de conversión, de preparación y de memoria del Bautismo, de reconciliación con Dios y con los hermanos, de recurso más frecuente a las \textquote{armas de la penitencia cristiana}: la oración, el ayuno y la limosna (cfr. Mt 6,1-6.16-18).

En el ámbito de la piedad popular no se percibe fácilmente el sentido mistérico de la Cuaresma y no se han asimilado algunos de los grandes valores y temas, como la relación entre el \textquote{sacramento de los cuarenta días} y los sacramentos de la iniciación cristiana, o el misterio del \textquote{éxodo}, presente a lo largo de todo el itinerario cuaresmal. Según una constante de la piedad popular, que tiende a centrarse en los misterios de la humanidad de Cristo, en la Cuaresma los fieles concentran su atención en la Pasión y Muerte del Señor.

El comienzo de los cuarenta días de penitencia, en el Rito romano, se caracteriza por el austero símbolo de las Cenizas, que distingue la Liturgia del Miércoles de Ceniza. Propio de los antiguos ritos con los que los pecadores convertidos se sometían a la penitencia canónica, el gesto de cubrirse con ceniza tiene el sentido de reconocer la propia fragilidad y mortalidad, que necesita ser redimida por la misericordia de Dios. Lejos de ser un gesto puramente exterior, la Iglesia lo ha conservado como signo de la actitud del corazón penitente que cada bautizado está llamado a asumir en el itinerario cuaresmal. Se debe ayudar a los fieles, que acuden en gran número a recibir la Ceniza, a que capten el significado interior que tiene este gesto, que abre a la conversión y al esfuerzo de la renovación pascual.

A pesar de la secularización de la sociedad contemporánea, el pueblo cristiano advierte claramente que durante la Cuaresma hay que dirigir el espíritu hacia las realidades que son verdaderamente importantes; que hace falta un esfuerzo evangélico y una coherencia de vida, traducida en buenas obras, en formas de renuncia a lo superfluo y suntuoso, en expresiones de solidaridad con los que sufren y con los necesitados.

También los fieles que frecuentan poco los sacramentos de la Penitencia y de la Eucaristía saben, por una larga tradición eclesial, que el tiempo de Cuaresma-Pascua está en relación con el precepto de la Iglesia de confesar los propios pecados graves, al menos una vez al año, preferentemente en el tiempo pascual.

La divergencia existente entre la concepción litúrgica y la visión popular de la Cuaresma, no impide que el tiempo de los \textquote{Cuarenta días} sea un espacio propicio para una interacción fecunda entre Liturgia y piedad popular.

Un ejemplo de esta interacción lo tenemos en el hecho de que la piedad popular favorece algunos días, algunos ejercicios de piedad y algunas actividades apostólicas y caritativas, que la misma Liturgia cuaresmal prevé y recomienda. La práctica del ayuno, tan característica desde la antigüedad en este tiempo litúrgico, es un \textquote{ejercicio} que libera voluntariamente de las necesidades de la vida terrena para redescubrir la necesidad de la vida que viene del cielo: \textquote{No sólo de pan vive el hombre, sino de toda palabra que sale de la boca de Dios} (Mt 4,4; cfr. Dt 8,3; Lc 4,4; antífona de comunión del I Domingo de Cuaresma)

\subsubsection{La veneración de Cristo crucificado}

El camino cuaresmal termina con el comienzo del Triduo pascual, es decir, con la celebración de la Misa In Cena Domini. En el Triduo pascual, el Viernes Santo, dedicado a celebrar la Pasión del Señor, es el día por excelencia para la \textquote{Adoración de la santa Cruz}.

Sin embargo, la piedad popular desea anticipar la veneración cultual de la Cruz. De hecho, a lo largo de todo el tiempo cuaresmal, el viernes, que por una antiquísima tradición cristiana es el día conmemorativo de la Pasión de Cristo, los fieles dirigen con gusto su piedad hacia el misterio de la Cruz.

Contemplando al Salvador crucificado captan más fácilmente el significado del dolor inmenso e injusto que Jesús, el Santo, el Inocente, padeció por la salvación del hombre, y comprenden también el valor de su amor solidario y la eficacia de su sacrificio redentor.

Las expresiones de devoción a Cristo crucificado, numerosas y variadas, adquieren un particular relieve en las iglesias dedicadas al misterio de la Cruz o en las que se veneran reliquias, consideradas auténticas, del lignum Crucis. La \textquote{invención de la Cruz}, acaecida según la tradición durante la primera mitad del siglo IV, con la consiguiente difusión por todo el mundo de fragmentos de la misma, objeto de grandísima veneración, determinó un aumento notable del culto a la Cruz.

En las manifestaciones de devoción a Cristo crucificado, los elementos acostumbrados de la piedad popular como cantos y oraciones, gestos como la ostensión y el beso de la cruz, la procesión y la bendición con la cruz, se combinan de diversas maneras, dando lugar a ejercicios de piedad que a veces resultan preciosos por su contenido y por su forma.

No obstante, la piedad respecto a la Cruz, con frecuencia, tiene necesidad de ser iluminada. Se debe mostrar a los fieles la referencia esencial de la Cruz al acontecimiento de la Resurrección: la Cruz y el sepulcro vacío, la Muerte y la Resurrección de Cristo, son inseparables en la narración evangélica y en el designio salvífico de Dios. En la fe cristiana, la Cruz es expresión del triunfo sobre el poder de las tinieblas, y por esto se la presenta adornada con gemas y convertida en signo de bendición, tanto cuando se traza sobre uno mismo, como cuando se traza sobre otras personas y objetos.

El texto evangélico, particularmente detallado en la narración de los diversos episodios de la Pasión, y la tendencia a especificar y a diferenciar, propia de la piedad popular, ha hecho que los fieles dirijan su atención, también, a aspectos particulares de la Pasión de Cristo y hayan hecho de ellos objeto de diferentes devociones: el \textquote{Ecce homo}, el Cristo vilipendiado, \textquote{con la corona de espinas y el manto de púrpura} (Jn 19,5), que Pilato muestra al pueblo; las llagas del Señor, sobre todo la herida del costado y la sangre vivificadora que brota de allí (cfr. Jn 19,34); los instrumentos de la Pasión, como la columna de la flagelación, la escalera del pretorio, la corona de espinas, los clavos, la lanza de la transfixión; la sábana santa o lienzo de la deposición.

Estas expresiones de piedad, promovidas en ocasiones por personas de santidad eminente, son legítimas. Sin embargo, para evitar una división excesiva en la contemplación del misterio de la Cruz, será conveniente subrayar la consideración de conjunto de todo el acontecimiento de la Pasión, conforme a la tradición bíblica y patrística.

\subsubsection{La lectura de la Pasión del Señor}

La Iglesia exhorta a los fieles a la lectura frecuente, de manera individual o comunitaria, de la Palabra de Dios. Ahora bien, no hay duda de que entre las páginas de la Biblia, la narración de la Pasión del Señor tiene un valor pastoral especial, por lo que, por ejemplo, el Ordo unctionis infirmorum eorumque pastoralis curae sugiere la lectura, en el momento de la agonía del cristiano, de la narración de la Pasión del Señor o de algún pasaje de la misma.

Durante el tiempo de Cuaresma, el amor a Cristo crucificado deberá llevar a la comunidad cristiana a preferir el miércoles y el viernes, sobre todo, para la lectura de la Pasión del Señor.

Esta lectura, de gran sentido doctrinal, atrae la atención de los fieles tanto por el contenido como por la estructura narrativa, y suscita en ellos sentimientos de auténtica piedad: arrepentimiento de las culpas cometidas, porque los fieles perciben que la Muerte de Cristo ha sucedido para remisión de los pecados de todo el género humano y también de los propios; compasión y solidaridad con el Inocente injustamente perseguido; gratitud por el amor infinito que Jesús, el Hermano primogénito, ha demostrado en su Pasión para con todos los hombres, sus hermanos; decisión de seguir los ejemplos de mansedumbre, paciencia, misericordia, perdón de las ofensas y abandono confiado en las manos del Padre, que Jesús dio de modo abundante y eficaz durante su Pasión.

Fuera de la celebración litúrgica, la lectura de la Pasión se puede \textquote{dramatizar} si es oportuno, confiando a lectores distintos los textos correspondientes a los diversos personajes; asimismo, se pueden intercalar cantos o momentos de silencio meditativo.

\subsubsection{El \textquote{Vía Crucis}}

Entre los ejercicios de piedad con los que los fieles veneran la Pasión del Señor, hay pocos que sean tan estimados como el Vía Crucis. A través de este ejercicio de piedad los fieles recorren, participando con su afecto, el último tramo del camino recorrido por Jesús durante su vida terrena: del Monte de los Olivos, donde en el \textquote{huerto llamado Getsemaní} (Mc 14,32) el Señor fue \textquote{presa de la angustia} (Lc 22,44), hasta el Monte Calvario, donde fue crucificado entre dos malhechores (cfr. Lc 23,33), al jardín donde fue sepultado en un sepulcro nuevo, excavado en la roca (cfr. Jn 19,40-42).

Un testimonio del amor del pueblo cristiano por este ejercicio de piedad son los innumerables Vía Crucis erigidos en las iglesias, en los santuarios, en los claustros e incluso al aire libre, en el campo, o en la subida a una colina, a la cual las diversas estaciones le confieren una fisonomía sugestiva.

El Vía Crucis es la síntesis de varias devociones surgidas desde la alta Edad Media: la peregrinación a Tierra Santa, durante la cual los fieles visitan devotamente los lugares de la Pasión del Señor; la devoción a las \textquote{caídas de Cristo} bajo el peso de la Cruz; la devoción a los \textquote{caminos dolorosos de Cristo}, que consiste en ir en procesión de una iglesia a otra en memoria de los recorridos de Cristo durante su Pasión; la devoción a las \textquote{estaciones de Cristo}, esto es, a los momentos en los que Jesús se detiene durante su camino al Calvario, o porque le obligan sus verdugos o porque está agotado por la fatiga, o porque, movido por el amor, trata de entablar un diálogo con los hombres y mujeres que asisten a su Pasión.

En su forma actual, que está ya atestiguada en la primera mitad del siglo XVII, el Vía Crucis, difundido sobre todo por San Leonardo de Porto Mauricio (+1751), ha sido aprobado por la Sede Apostólica, dotado de indulgencias y consta de catorce estaciones.

El Vía Crucis es un camino trazado por el Espíritu Santo, fuego divino que ardía en el pecho de Cristo (cfr. Lc 12,49-50) y lo impulsó hasta el Calvario; es un camino amado por la Iglesia, que ha conservado la memoria viva de las palabras y de los acontecimientos de los último días de su Esposo y Señor.

En el ejercicio de piedad del Vía Crucis confluyen también diversas expresiones características de la espiritualidad cristiana: la comprensión de la vida como camino o peregrinación; como paso, a través del misterio de la Cruz, del exilio terreno a la patria celeste; el deseo de conformarse profundamente con la Pasión de Cristo; las exigencias de la sequela Christi, según la cual el discípulo debe caminar detrás del Maestro, llevando cada día su propia cruz (cfr. Lc 9,23)

Por todo esto el Vía Crucis es un ejercicio de piedad especialmente adecuado al tiempo de Cuaresma.

Para realizar con fruto el Vía Crucis pueden ser útiles las siguientes indicaciones:

– la forma tradicional, con sus catorce estaciones, se debe considerar como la forma típica de este ejercicio de piedad; sin embargo, en algunas ocasiones, no se debe excluir la sustitución de una u otra \textquote{estación} por otras que reflejen episodios evangélicos del camino doloroso de Cristo, y que no se consideran en la forma tradicional;

– en todo caso, existen formas alternativas del Vía Crucis aprobadas por la Sede Apostólica o usadas públicamente por el Romano Pontífice: estas se deben considerar formas auténticas del mismo, que se pueden emplear según sea oportuno;

– el Vía Crucis es un ejercicio de piedad que se refiere a la Pasión de Cristo; sin embargo es oportuno que concluya de manera que los fieles se abran a la expectativa, llena de fe y de esperanza, de la Resurrección; tomando como modelo la estación de la Anastasis al final del Vía Crucis de Jerusalén, se puede concluir el ejercicio de piedad con la memoria de la Resurrección del Señor.

Los textos para el Vía Crucis son innumerables. Han sido compuestos por pastores movidos por una sincera estima a este ejercicio de piedad y convencidos de su eficacia espiritual; otras veces tienen por autores a fieles laicos, eminentes por la santidad de vida, doctrina o talento literario.

La selección del texto, teniendo presente las eventuales indicaciones del Obispo, se deberá hacer considerando sobre todo las características de los que participan en el ejercicio de piedad y el principio pastoral de combinar sabiamente la continuidad y la innovación. En todo caso, serán preferibles los textos en los que resuenen, correctamente aplicadas, las palabras de la Biblia, y que estén escritos con un estilo digno y sencillo.

Un desarrollo inteligente del Vía Crucis, en el que se alternan de manera equilibrada: palabra, silencio, canto, movimiento procesional y parada meditativa, contribuye a que se obtengan los frutos espirituales de este ejercicio de piedad.

\newpage
\subsubsection{El \textquote{Vía Matris}}

Así como en el plan salvífico de Dios (cfr. Lc 2,34-35) están asociados Cristo crucificado y la Virgen dolorosa, también los están en la Liturgia y en la piedad popular.

Como Cristo es el \textquote{hombre de dolores} (Is 53,3), por medio del cual se ha complacido Dios en \textquote{reconciliar consigo todos los seres: los del cielo y los de la tierra, haciendo la paz por la sangre de su cruz} (Col 1,20), así María es la \textquote{mujer del dolor}, que Dios ha querido asociar a su Hijo, como madre y partícipe de su Pasión (socia Passionis).

Desde los días de la infancia de Cristo, toda la vida de la Virgen, participando del rechazo de que era objeto su Hijo, transcurrió bajo el signo de la espada (cfr. Lc 2,35). Sin embargo, la piedad del pueblo cristiano ha señalado siete episodios principales en la vida dolorosa de la Madre y los ha considerado como los \textquote{siete dolores} de Santa María Virgen.

Así, según el modelo del Vía Crucis, ha nacido el ejercicio de piedad del Vía Matris dolorosae, o simplemente Vía Matris, aprobado también por la Sede Apostólica. Desde el siglo XVI hay ya formas incipientes del Vía Matris, pero en su forma actual no es anterior al siglo XIX. La intuición fundamental es considerar toda la vida de la Virgen, desde el anuncio profético de Simeón (cfr. Lc 2,34-35) hasta la muerte y sepultura del Hijo, como un camino de fe y de dolor: camino articulado en siete \textquote{estaciones}, que corresponden a los \textquote{siete dolores} de la Madre del Señor.

El ejercicio de piedad del Vía Matris se armoniza bien con algunos temas propios del itinerario cuaresmal. Como el dolor de la Virgen tiene su causa en el rechazo que Cristo ha sufrido por parte de los hombres, el Vía Matris remite constante y necesariamente al misterio de Cristo, siervo sufriente del Señor (cfr. Is 52,13–53,12), rechazado por su propio pueblo (cfr. Jn 1,11; Lc 2,1-7; 2,34-35; 4,28-29; Mt 26,47-56; Hech 12,1-5). Y remite también al misterio de la Iglesia: las estaciones del Vía Matris son etapas del camino de fe y dolor en el que la Virgen ha precedido a la Iglesia y que esta deberá recorrer hasta el final de los tiempos.

El Vía Matris tiene como máxima expresión la \textquote{Piedad}, tema inagotable del arte cristiano desde la Edad Media.

\newpage
\section{La Semana Santa}

\textquote{Durante la Semana Santa la Iglesia celebra los misterios de la salvación actuados por Cristo en los últimos días de su vida, comenzando por su entrada mesiánica en Jerusalén}.

Es muy intensa la participación del pueblo en los ritos de la Semana Santa. Algunos muestran todavía señales de su origen en el ámbito de la piedad popular. Sin embargo ha sucedido que, a lo largo de los siglos, se ha producido en los ritos de la Semana Santa una especie de paralelismo celebrativo, por lo cual se dan prácticamente dos ciclos con planteamiento diverso: uno rigurosamente litúrgico, otro caracterizado por ejercicios de piedad específicos, sobre todo las procesiones.

Esta diferencia se debería reconducir a una correcta armonización entre las celebraciones litúrgicas y los ejercicios de piedad. En relación con la Semana Santa, el amor y el cuidado de las manifestaciones de piedad tradicionalmente estimadas por el pueblo debe llevar necesariamente a valorar las acciones litúrgicas, sostenidas ciertamente por los actos de piedad popular.

\section{Domingo de Ramos}

\textit{Las palmas y los ramos de olivo o de otros árboles}

La Semana Santa comienza con el Domingo de Ramos \textquote{de la Pasión del Señor}, que comprende a la vez el triunfo real de Cristo y el anuncio de la Pasión.

La procesión que conmemora la entrada mesiánica de Jesús en Jerusalén tiene un carácter festivo y popular. A los fieles les gusta conservar en sus hogares, y a veces en el lugar de trabajo, los ramos de olivo o de otros árboles, que han sido bendecidos y llevados en la procesión.

Sin embargo es preciso instruir a los fieles sobre el significado de la celebración, para que entiendan su sentido. Será oportuno, por ejemplo, insistir en que lo verdaderamente importante es participar en la procesión y no simplemente procurarse una palma o ramo de olivo; que estos no se conserven como si fueran amuletos, con un fin curativo o para mantener alejados a los malos espíritus y evitar así, en las casas y los campos, los daños que causan, lo cual podría ser una forma de superstición.

La palma y el ramo de olivo se conservan, ante todo, como un testimonio de la fe en Cristo, rey mesiánico, y en su victoria pascual.
\end{introstyle}  %Cuaresma		
	%\chapter{Domingo I de Cuaresma (B)}

\section{Lecturas}

\rtitle{PRIMERA LECTURA}

\rbook{Del libro del Génesis} \rred{9, 8-15}

\rtheme{Pacto de Dios con Noé liberado del diluvio de las aguas}

\begin{scripture}
Dios dijo a Noé y a sus hijos:

\>{Yo establezco mi alianza con vosotros y con vuestros descendientes, con todos los animales que os acompañan, aves, ganados y fieras, con todos los que salieron del arca y ahora viven en la tierra. Establezco, pues, mi alianza con vosotros: el diluvio no volverá a destruir criatura alguna ni habrá otro diluvio que devaste la tierra}.

Y Dios añadió:

\>{Esta es la señal de la alianza que establezco con vosotros y con todo lo que vive con vosotros, para todas las generaciones: pondré mi arco en el cielo, como señal de mi alianza con la tierra. Cuando traiga nubes sobre la tierra, aparecerá en las nubes el arco y recordaré mi alianza con vosotros y con todos los animales, y el diluvio no volverá a destruir a los vivientes}.
\end{scripture}

\newpage
\rtitle{SALMO RESPONSORIAL}


\rbook{Salmo} \rred{24, 4-5a. 6-7cd. 8-9}

\rtheme{Tus sendas, Señor, son misericordia y lealtad para los que guardan tu alianza}

\begin{psbody}
Señor, enséñame tus caminos,
instrúyeme en tus sendas:
haz que camine con lealtad;
enséñame, porque tú eres mi Dios y Salvador. 

Recuerda, Señor, que tu ternura
y tu misericordia son eternas;
acuérdate de mí con misericordia,
por tu bondad, Señor. 

El Señor es bueno y es recto,
enseña el camino a los pecadores;
hace caminar a los humilles con rectitud,
enseña su camino a los humildes. 
\end{psbody}

\rtitle{SEGUNDA LECTURA}

\rbook{De la primera carta del apóstol san Pedro} \rred{3, 18-22}

\rtheme{El bautismo que actualmente os está salvando}

\begin{scripture}
Queridos hermanos:

Cristo sufrió su pasión, de una vez para siempre, por los pecados, el justo por los injustos, para conduciros a Dios.

Muerto en la carne pero vivificado en el Espíritu; en el espíritu fue a predicar incluso a los espíritus en prisión, a los desobedientes en otro tiempo, cuando la paciencia de Dios aguardaba, en los días de Noé, a que se construyera el arca, para que unos pocos, es decir, ocho personas, se salvaran por medio del agua.

Aquello era también un símbolo del bautismo que actualmente os está salvando, que no es purificación de una mancha física, sino petición a Dios de una buena conciencia, por la resurrección de Jesucristo, el cual fue al cielo, está sentado a la derecha de Dios y tiene a su disposición ángeles, potestades y poderes.
\end{scripture}

\newpage
\rtitle{EVANGELIO}

\rbook{Del Santo Evangelio según san Marcos} \rred{1, 12-15}

\rtheme{Era tentado por Satanás, y los ángeles lo servían}

\begin{scripture}
En aquel tiempo, el Espíritu empujó a Jesús al desierto.

Se quedó en el desierto cuarenta días, siendo tentado por Satanás; vivía con las fieras y los ángeles lo servían.

Después de que Juan fue entregado, Jesús se marchó a Galilea a proclamar el Evangelio de Dios; decía:

\>{Se ha cumplido el tiempo y está cerca el reino de Dios. Convertíos y creed en el Evangelio}.
\end{scripture}

\begin{patercite}
	El primer domingo del itinerario cuaresmal subraya nuestra condición de hombre en esta tierra. La batalla victoriosa contra las tentaciones, que da inicio a la misión de Jesús, es una invitación a tomar conciencia de la propia fragilidad para acoger la Gracia que libera del pecado e infunde nueva fuerza en Cristo, camino, verdad y vida (cf. \textit{Ordo Initiationis Christianae Adultorum}, n. 25). Es una llamada decidida a	recordar que la fe cristiana implica, siguiendo el ejemplo de Jesús y en unión con él, una lucha \textquote{contra los Dominadores de este mundo tenebroso}	(\textit{Ef} 6, 12), en el cual el diablo actúa y no se cansa, tampoco hoy, de tentar al hombre que quiere acercarse al Señor: Cristo sale victorioso, para abrir también nuestro corazón a la esperanza y guiarnos a vencer las seducciones del mal.
	
	En síntesis, el itinerario cuaresmal, en el cual se nos invita a contemplar el Misterio de la cruz, es \textquote{hacerme semejante a él en su muerte} (\textit{Flp} 3, 10), para llevar a cabo una \textit{conversión} profunda de nuestra vida: dejarnos transformar por la acción del Espíritu Santo, como san Pablo en el camino de Damasco; orientar con decisión nuestra existencia según la voluntad de Dios; liberarnos de nuestro egoísmo, superando el instinto de dominio sobre los demás y abriéndonos a la caridad de Cristo. El período cuaresmal es el momento favorable para reconocer nuestra debilidad, acoger, con una sincera revisión de vida, la Gracia renovadora del Sacramento de la Penitencia y caminar con decisión hacia Cristo.
		
	\textbf{Benedicto XVI, papa}, \textit{Mensaje} para la Cuaresma del 2011, n. 2, parr. 2 y n. 3, parr. 4.
\end{patercite}

\newsection
\section{Comentario Patrístico}

\subsection{San Agustín, obispo}

\ptheme{En Cristo fuimos tentados, en él vencimos al diablo}

\src{Comentario sobre el salmo 60, 2-3: \\CCL 39, 766.}

\begin{body}
\ltr{D}{\textit{ios}} \textit{mío, escucha mi clamor, atiende a mi súplica}. ¿Quién es el que habla? Parece que sea uno solo. Pero veamos si es uno solo: \textit{Te invoco desde los confines de la tierra con el corazón abatido}. Por lo tanto, si invoca desde los confines de la tierra, no es uno solo; y, sin embargo, es uno solo, porque Cristo es uno solo, y todos nosotros somos sus miembros. ¿Y quién es ese único hombre que clama desde los confines de la tierra? Los que invocan desde los confines de la tierra son los llamados a aquella herencia, a propósito de la cual se dijo al mismo Hijo: \textit{Pídemelo: te daré en herencia las naciones, en posesión, los confines de la tierra}. De manera que quien clama desde los confines de la tierra es el cuerpo de Cristo, la heredad de Cristo, la única Iglesia de Cristo, esta unidad que formamos todos nosotros.

Y ¿qué es lo que pide? Lo que he dicho antes: \textit{Dios mío, escucha mi clamor, atiende a mi súplica; te invoco desde los confines de la tierra}. O sea: \textquote{Esto que pido, lo pido desde los confines de la tierra}, es decir, desde todas partes.

Pero, ¿por qué ha invocado así? Porque tenía \textit{el corazón abatido}. Con ello da a entender que el Señor se halla presente en todos los pueblos y en los hombres del orbe entero no con gran gloria, sino con graves tentaciones.

Pues nuestra vida en medio de esta peregrinación no puede estar sin tentaciones, ya que nuestro progreso se realiza precisamente a través de la tentación, y nadie se conoce a sí mismo si no es tentado, ni puede ser coronado si no ha vencido, ni vencer si no ha combatido, ni combatir si carece de enemigo y de tentaciones.

Este que invoca desde los confines de la tierra está angustiado, pero no se encuentra abandonado. Porque a nosotros mismos, esto es, a su cuerpo, quiso prefigurarnos también en aquel cuerpo suyo en el que ya murió, resucitó y ascendió al cielo, a fin de que sus miembros no desesperen de llegar adonde su cabeza los precedió.

De forma que nos incluyó en sí mismo cuando quiso verse tentado por Satanás. Nos acaban de leer que Jesucristo, nuestro Señor, \textit{se dejó tentar por el diablo}. ¡Nada menos que Cristo tentado por el diablo! Pero en Cristo estabas siendo tentado tú, porque Cristo tenía de ti la carne, y de él procedía para ti la salvación; de ti procedía la muerte para él, y de él para ti la vida; de ti para él los ultrajes, y de él para ti los honores; en definitiva, de ti para él la tentación, y de él para ti la victoria.

Si hemos sido tentados en él, también en él vencemos al diablo. ¿Te fijas en que Cristo fue tentado, y no te fijas en que venció? Reconócete a ti mismo tentado en él, y reconócete también vencedor en él. Podía haber evitado al diablo; pero, si no hubiese sido tentado, no te habría aleccionado para la victoria cuando tú fueras tentado.
\end{body}

\begin{patercite}
[\ldots] Tratemos de nuevo de pensar en el desierto. El desierto es \textit{el lugar de lo esencial}. Miremos nuestras vidas: ¡cuántas cosas inútiles nos rodean! Perseguimos mil cosas que parecen necesarias y en realidad no lo son. ¡Qué bien nos haría liberarnos de tantas realidades superfluas, para redescubrir lo que de verdad importa, para encontrar los rostros de quienes están a nuestro lado! También en esto Jesús nos da su ejemplo, ayunando. \textit{Ayunar} es saber renunciar a las cosas vanas, a lo superfluo, para ir a lo esencial. Ayunar no es solamente adelgazar, ayunar es ir precisamente a lo esencial, es buscar la belleza de una vida más sencilla.  

El desierto, finalmente, es el lugar de la soledad. También hoy, cerca de nosotros, hay tantos desiertos. Son las personas solas y abandonadas. Cuantos pobres y ancianos están cerca de nosotros y viven en silencio, sin clamor, marginados y descartados. Hablar de ellos no aumenta las audiencias. Pero el desierto nos lleva a ellos, a cuantos, forzados a callar, piden en silencio nuestra ayuda. Tantas miradas silenciosas que piden nuestra ayuda. El camino en el desierto cuaresmal es un camino de \textit{caridad} hacia quien es más débil.

Oración, ayuno, obras de misericordia: he aquí el camino en el desierto cuaresmal.

(\ldots) con la voz del profeta Isaías, Dios hizo esta promesa: \textquote{Pues bien, he aquí que yo lo renuevo: pongo en el desierto un camino} (\textit{Isaías} 43, 19). En el desierto se abre el camino que nos lleva de la muerte a la vida. Entremos en el desierto con Jesús, saldremos saboreando la Pascua, la potencia del amor de Dios que nos renueva la vida. Sucederá a nosotros como a esos desiertos que en primavera florecen, haciendo germinar de repente \textquote{de la nada} gemas y plantas. Ánimo, entremos en este desierto de la Cuaresma. Sigamos a Jesús en el desierto: con Él nuestros desiertos florecerán.

\textbf{Francisco, papa}, \textit{Catequesis}, Audiencia general, 26 de febrero de 2020, parr. 4-7.
\end{patercite}

\newsection 
\section{Homilías}

\subsection{San Juan Pablo II, papa}

\subsubsection{Homilía (1982): Despertar las conciencias}

\src{Visita pastoral a la parroquia Romana de Sant’Andrea delle Fratte.\par28 de febrero de 1982.}

\begin{body}
\ltr{C}{on} palabras muy concisas, el \textbf{evangelista Marcos} alude a ese ayuno de Jesús de Nazaret, que duró cuarenta días, y que cada año encuentra su reflejo en la liturgia de Cuaresma: \textquote{El Espíritu lo empujó al desierto y permaneció allí durante cuarenta días, siendo tentado por satanás; estaba entre los animales del campo y los ángeles le servían} (\textit{Mc} 1, 12). Luego, después del encarcelamiento de Juan el Bautista, Jesús fue a Galilea y comenzó a enseñar. Decía: \textquote{El tiempo se ha cumplido y el reino de Dios está cerca; convertíos y creed en el Evangelio} (\textit{Mc} 1, 15). El ayuno de cuarenta días de Jesús de Nazaret fue una introducción al anuncio del Evangelio del Reino de Dios. Trazó en las almas de los hombres el camino de la fe, sin el cual el Evangelio del Reino permanece como grano arrojado en tierra estéril.

2. La liturgia de hoy compara este comienzo del Evangelio del Reino, que llega a la Iglesia a través del ayuno de cuarenta días, con el arco iris, que fue un signo de la \textbf{alianza de Dios con los descendientes de Noé} después del diluvio. La Iglesia también se compara con el Arca de Noé en la \textbf{primera carta del Apóstol San Pedro}, en la que Cristo, después de haber ganado la victoria sobre la muerte y el pecado, realiza continuamente la obra de la redención. Sin embargo, el Arca de Noé era un espacio cerrado. La obra de Cristo es ilimitada en tiempo y espacio. La Iglesia sirve a esta obra como signo e instrumento.

He aquí Cristo: Aquel que murió de una vez por todas por los pecados; el Justo por el injusto, para llevarnos de regreso a Dios.

He aquí Cristo: muerto, es cierto, en el cuerpo, pero llamado a la vida en el Espíritu.

He aquí Cristo: sentado a la diestra de Dios, porque ascendió al cielo, donde los ángeles, los poderes y las dominaciones le fueron sometidos.

El mismo Cristo que, en el Espíritu Santo, \textquote{fue a anunciar la salvación también a los espíritus encarcelados; que en otro tiempo se negaron a creer} (\textit{1 Pe} 3, 19), como en los días de Noé. Cristo mismo en el bautismo nos salva, es decir, nos redime \textquote{no quitando la suciedad del cuerpo, sino mediante la invocación de la salvación dirigida a Dios por la buena conciencia} (cf. \textit{1 Pe} 3, 21): nos salva y nos redime gracias a su resurrección .

3. Así pues, la liturgia dominical de hoy abre el ayuno de Cuaresma, refiriéndose primero al ejemplo de Cristo, y luego al poder redentor de Cristo, que opera en la Iglesia y en toda la creación: a su poder redentor y santificador. La Cuaresma es el camino que se abre ante nosotros. Toda la Iglesia desea caminar así durante estos cuarenta días.

4. Y por eso ora hoy: \textquote{Señor, hazme conocer tus caminos, enséñame tus sendas. Guíame en tu verdad y enséñame, porque tú eres el Dios de mi salvación, en ti siempre he esperado} (\textit{Sal} 25 [24], 4-5). La Cuaresma es el camino de la verdad. El hombre debe encontrarse en toda su verdad ante Dios. También debe releer la verdad de las enseñanzas divinas, de los mandamientos divinos, de la voluntad divina, y debe confrontar su conciencia con ellos. Por aquí pasa el camino de la salvación. Es el camino de la esperanza.

5. Así pues, la Iglesia todavía reza: \textquote{Acuérdate, Señor, de tu amor, y de tu fidelidad que son eternos. Acuérdate de mí en tu misericordia, por tu bondad, Señor} (\textit{Sal} 25 [24], 6-7). La Cuaresma es el camino de la verdad, es el momento del despertar de las conciencias. Pero sobre todo, es el camino del Amor y la Misericordia. Sólo a través del Amor la verdad despierta al hombre a la vida. Sólo el Amor, que es Misericordia, enciende la esperanza. El ayuno durante la Cuaresma es un gran grito de amor. Un grito desgarrador. Un llanto definitivo. Es el gran momento de la misericordia. ¡Que todos reconozcan este camino!

6. Por tanto, la Iglesia sigue rezando en la liturgia de hoy: \textquote{Bueno y recto es el Señor, muestra el camino recto a los pecadores; conduce a los humildes según la justicia, enseña a los pobres su sendero} (\textit{Sal} 25 [24], 8-9). La Iglesia reza por la humildad del corazón humano. Ora para que el hombre, a través de la humildad, se encuentre a sí mismo en la verdad, para que pueda encontrarse en la verdad interior, para que así pueda encontrarse con el Amor, que es más fuerte que el pecado y la muerte, más fuerte que todo mal, y porque se deje guiar por el Verbo Divino: \textquote{No sólo de pan vive el hombre, sino de toda palabra que sale de la boca de Dios} (\textit{Mt} 4, 4).

7. He aquí el programa del Primer Domingo de Cuaresma. [\ldots] [En este tiempo] descubramos de nuevo la belleza de ser cristianos y ofrezcamos un testimonio consecuente y luminoso de ello.

8. En este primer domingo de Cuaresma deseo repetir las palabras de \textbf{San Pedro Apóstol}, primer obispo de la Iglesia de Roma: Queridos amigos, \textquote{Cristo murió una sola vez por los pecados, el justo por los injustos, para llevarnos de regreso a Dios} (\textit{1 Pe} 3, 18).

Amén.
\end{body}

\subsubsection{Homilía (1985): Guíame en tu verdad}

\src{Celebración eucarística en la Parroquia Romana de San Lorenzo in Damaso. \\24 de febrero de 1985.}

\begin{body}
\ltr[1. «]{E}{l} Espíritu llevó a Jesús al desierto, y permaneció allí cuarenta días, siendo tentado por Satanás» (cf. \textit{Mc} 1, 12-13). Cada año, en este primer domingo de Cuaresma, recordamos el ayuno de cuarenta días de Jesús y las tentaciones de satanás. El texto del \textbf{Evangelio según Marcos}, que leemos este año, es muy conciso.

Jesús de Nazaret comienza su misión mesiánica con su bautismo en el Jordán. Recibió de manos de Juan el Bautista el bautismo de penitencia, asemejándose a todos aquellos a quienes Juan lo administraba. Y, después, Jesús va al desierto, donde ayuna durante cuarenta días. Este ayuno se refiere a los cuarenta años de peregrinaje de Israel desde la esclavitud de Egipto a la Tierra Prometida.

El ayuno de cuarenta días de Jesús en el desierto es un modelo para la Cuaresma de la Iglesia. Debe llevarnos de la esclavitud del pecado a la victoria y la libertad en la resurrección de Cristo. Con un ayuno de cuarenta días nos preparamos para la Pascua.

2. Durante este período, Jesús predica el Evangelio de Dios de una manera particularmente intensa. Dice: \textquote{El tiempo se ha cumplido y el reino de Dios está cerca; convertíos y creed en el Evangelio} (\textit{Mc} 1, 15).

La Iglesia desea imitar a su Maestro. Con especial intensidad predica el Evangelio. La Cuaresma se define en la liturgia como un \textquote{tiempo fuerte}.

El Evangelio es un mensaje de conversión: \textquote{conviértete}. Al anunciar la conversión, Jesús da a conocer al hombre el estado de amenaza de múltiples males. Por lo tanto, incluso con respecto a sí mismo, admite la tentación de Satanás y la vence. Este es un doble momento en el que el camino mesiánico de Jesús pasa por los caminos del hombre, prisionero del pecado. El primer aspecto es el bautismo del Jordán, bautismo de penitencia; el segundo es la tentación.

La Iglesia recuerda esta tentación de Jesús el primer domingo de Cuaresma. De esta manera, quiere llegar a todos los caminos del hombre amenazado por múltiples tentaciones; quiere llegar a los caminos en que están los hombres envueltos en pecado. En estos caminos Cristo está verdaderamente presente con su poder salvador.

3. El estado de tentación –de amenaza de múltiples males, con pecado leve o grave– es un estado ordinario del hombre. Por eso Cristo nos recomendó orar al Padre: \textquote{Y no nos dejes caer en tentación, mas líbranos del mal} (\textit{Mt} 6, 13).

Por eso, en la liturgia de hoy, el \textbf{salmista} implora con fervor: \textquote{Señor, hazme conocer tus caminos, enséñame tus caminos, guíame en tu verdad e instrúyeme \ldots} (\textit{Sal} 25, 4-5). \textquote{Guíame en tu verdad} significa exactamente: ¡no permitas que caiga en la tentación! De hecho, la tentación siempre está relacionada con la pérdida de la verdad en el comportamiento humano. El corazón y la voluntad son \textquote{seducidos} de tal manera que, al obrar, se desprenden del verdadero bien y siguen un aparente bien. La tentación es siempre mentira y tiene su origen definitivo en aquel a quien la Escritura llama \textquote{el padre de la mentira} (\textit{Jn} 8, 44).

Si el hombre es tentado por el \textquote{mundo}, si la fuente de las tentaciones se encuentra en la concupiscencia de los ojos, de la carne y en la soberbia de la vida (es decir, \textquote{dentro del hombre}), entonces el inicio de toda tentación se origina de aquel por quien Cristo mismo permitió ser tentado durante el ayuno de cuarenta días en el desierto, el que es \textquote{el padre de la mentira}.

4. La Iglesia, por tanto, entiende su Cuaresma como un desafío particular en la lucha contra el mal, incluso hasta sus raíces. La tentación no es solo una ocasión para pecar, sino que también es la raíz del pecado. El hombre no solo se siente atraído por el mal, sino que a veces también está rodeado por él.

Todo esto Cristo lo hace presente al hombre desde el comienzo mismo de ese camino que es la Cuaresma. Al mismo tiempo, hace presente a cada uno de nosotros la fuerza salvífica del Evangelio. El Evangelio no es solo la palabra de Dios, es \textquote{poder de salvación} (\textit{Rm} 1, 16). Y en este sentido es una buena noticia. Tiene sus raíces en el pacto de Dios con la creación. La liturgia de hoy recuerda la antigua \textbf{alianza} que se estableció con \textbf{Noé} y sus hijos después del diluvio. En primer lugar, el Evangelio se expresa con la alianza nueva y eterna. Es la alianza estipulada en la cruz de Cristo, –por medio de su cuerpo y de su sangre–, y reconfirmada con la resurrección.

Cada año, ayunando durante cuarenta días, la Iglesia se prepara para la renovación singular de esta alianza, la cual contiene también la fuerza definitiva del poder salvador, que es capaz de conducir al hombre a través del estado de amenaza de múltiples males. Solo es necesario que el hombre se arraigue en esta alianza, resistiendo todo lo que viene del \textquote{padre de la mentira}.

5. Meditamos este importante contenido litúrgico del primer domingo de Cuaresma [en esta parroquia de San Lorenzo en Dámaso.]

[\ldots]

6. [¡Aceptad, queridos hermanos y hermanas, la visita de hoy del Obispo de Roma!] ¡Aceptad el mensaje del primer domingo de Cuaresma! Entre los caminos por los que os lleva la vida diaria, no dejéis de orar con las palabras del \textbf{salmista}: \textquote{Guíame, Señor, en tu verdad e instrúyeme} \ldots, y con las palabras del Padrenuestro: \textquote{No nos dejes caer en tentación, mas líbranos del mal}.

Que estas palabras marquen el tiempo de la Cuaresma en vuestras almas. En vuestra parroquia, durante este tiempo fuerte.

Amén.
\end{body}


\subsubsection{Homilía (1988): Alianza nueva y eterna}

\src{Visita pastoral a la Parroquia Romana de Santa Prisca.\\21 de febrero de 1988.}

\begin{body}
1. \textquote{He aquí que yo establezco mi alianza contigo} (\textit{Gn} 9, 9).

\ltr{H}{oy}, primer domingo de Cuaresma, la Iglesia nos recuerda en la liturgia la alianza concertada por Dios con el patriarca \textbf{Noé} después del diluvio. Y esta es una de las alianzas que forman la historia de la salvación en el Antiguo Testamento: \textquote{Muchas veces y de diferentes maneras Dios habló a los padres por medio de los profetas}, leemos en la carta a los Hebreos (cf. \textit{Hb} 1, 1); \textquote{Muchas veces haz ofrecido tu alianza a los hombres} proclama la \textit{Plegaria Eucarística IV}.

\textquote{He aquí que establezco mi alianza contigo y con tu descendencia después de ti; con cada ser vivo\ldots aves, vacas y fieras, con todos los animales que han salido del arca} (\textit{Gn} 9, 9-10). En estas palabras del Libro del Génesis escuchamos un claro eco del primer capítulo del mismo libro, en el que Dios somete toda la creación al dominio del hombre. En la historia bíblica, la obra de la creación y la alianza van de la mano.

2. La alianza con el patriarca Noé se caracteriza por el hecho de que se estableció después del diluvio. Esto fue causado por los pecados cometidos por los hombres de la época. La alianza fue, por tanto, un signo de perdón y gracia de parte de Dios.

\textquote{Yo establezco mi alianza contigo: ningún ser vivo será destruido por las aguas del diluvio, ni el diluvio devastará la tierra} (\textit{Gn} 9, 11).

Del \textbf{libro del Génesis} se puede deducir que el diluvio bíblico, que devastó la tierra y todo lo que en ella existía, excepto los seres salvados en el arca de Noé, fue el castigo por otro diluvio, el del pecado (cf. \textit{Gn} 6), en el que pronto se hicieron evidentes los efectos de la corrupción causada por el pecado original en el corazón y la conciencia de la humanidad. A causa de la primera transgresión, el hombre se encuentra bajo la influencia del \textquote{padre de la mentira} (cf. \textit{Jn} 8, 44), a quien en la Sagrada Escritura también se le llama \textquote{el príncipe de este mundo} (\textit{Jn} 12, 31; \textit{Jn} 14, 30; 16, 11).

3. Si la Iglesia nos recuerda todo esto el primer domingo de Cuaresma, lo hace para introducirnos en el misterio y al mismo tiempo en la realidad de la Nueva y Eterna Alianza, que el Padre Eterno concluyó con los hombres en Cristo: en su cruz y en su sangre. En su muerte y resurrección.

He aquí que Cristo ya está presente en el mundo. El pasaje del \textbf{Evangelio de Marcos} nos dice que vino a Galilea para proclamar el Evangelio de Dios, y que, incluso antes, sufrió una tentación en el desierto por obra del mismo \textquote{padre de la mentira} y \textquote{príncipe de este mundo}.

\textquote{El Espíritu lo empujó al desierto y permaneció allí cuarenta días, siendo tentado por Satanás} (\textit{Mc} 1, 12).

La historia es concisa. Los otros evangelistas sinópticos dan detalles más extensos. El hecho de la tentación de Jesús en el desierto debe leerse en el contexto de la lógica de la encarnación. Desde que el Hijo de Dios se hizo hombre, desde que vino \textquote{al mundo} y quiso mostrar que acoge a este mundo y al hombre como realmente son, quiso también someter su verdadera humanidad a la influencia tentadora del \textquote{príncipe de las tinieblas}. Sólo en ese contexto es posible comprender plenamente las palabras: \textquote{Yo soy la luz del mundo} (\textit{Jn} 8, 12), o la expresión de Simeón: \textquote{Luz para iluminar a las naciones} (\textit{Lc} 2, 32).

4. La Iglesia recuerda todo esto al comienzo del período que, como el ayuno de Cristo en el desierto, debe durar cuarenta días. De ahí esta referencia en la liturgia de hoy.

Sin embargo, el pensamiento de esta celebración no se detiene solo en este evento. Va hacia la alianza de Dios con el hombre, que debe cumplirse definitivamente en la muerte de Cristo.

Aquí están las palabras del \textbf{apóstol Pedro}: \textquote{Cristo murió una vez por todas por los pecados, el justo por los injustos, para llevarlos de regreso a Dios; muerto en la carne, pero vivificado en el espíritu} (\textit{1 Pe} 3, 18). Aquí San Pedro se refiere a Noé y su arca, para decir que la muerte de Cristo anuncia la salvación a los que murieron entonces (cf. \textit{1 Pe} 3, 19-20). Pero luego el apóstol explica el significado del bautismo, en el que ve una analogía con la experiencia bíblica del diluvio y el arca, cuando los hombres se salvaron por medio del agua (cf. \textit{1 Pe} 3, 20). En el bautismo, el poder salvador del sacramento no deriva del agua misma, que es sólo un símbolo expresivo, sino del poder de la resurrección de Cristo.

Es la misma verdad que proclama san Pablo en la carta a los Romanos, escribiendo sobre el bautismo que recibimos en la muerte de Cristo, para luego participar de su vida, revelada por la resurrección (cf. \textit{Rm} 6, 1 ss).

5. La liturgia de Cuaresma –como vemos– nos prepara desde el principio para los acontecimientos pascuales. Este es su significado y propósito fundamental. Con tal espíritu, cada uno de nosotros debe meditar en las palabras del \textbf{salmista} de la liturgia de hoy y orar con él: \textquote{Hazme conocer, Señor, tus caminos, enséñame tus senderos. Guíame en tu verdad e instrúyeme} (\textit{Sal} 25 [24], 4-5).

Se trata aquí de las enseñanzas más importantes para la vida de la Iglesia, las verdades decisivas y definitivas, que en el tiempo pascual, e incluso antes de la Cuaresma, están particularmente condensadas.

A la luz de estas verdades y enseñanzas, podemos reconocer plenamente que: \textquote{Bueno y recto es el Señor, indica a los pecadores el justo camino; juzga a los humildes según la justicia, enseña sus caminos a los pobres} (\textit{Sal} 25 [24], 8-9).

Precisamente este es el sentido de la alianza, que Dios ha ofrecido muchas veces a los hombres, en la historia de la salvación, para preparar la alianza última y definitiva en la sangre de Cristo, en su cruz y en su resurrección.

Si el \textbf{salmista} reza: \textquote{Señor, recuerda tu amor y tu fidelidad que son eternas} (\textit{Sal} 25 [24], 6), el misterio de la redención de Cristo constituye la realización de estas palabras.

Una vez, después del diluvio, la señal de la alianza fue un arco iris en el horizonte. Ahora, este arco iris de paz es, en última instancia, la cruz del Gólgota: la cruz que se extiende sobre el mundo entero.

[\ldots]

7. [\ldots] Quisiera deciros a todos: tened los ojos abiertos y vigilantes, tened un corazón generoso, tanto para notar las situaciones que os desafían, considerándolas con el alma iluminada por la fe, como para responder con generosidad a los necesitados, aunque sean personas que vivan lejos de vosotros. 

\txtsmall{[Por eso quiero animar fuertemente todas las iniciativas del grupo \textquote{caritas} y de los grupos de voluntariado organizados por la parroquia. \ldots Un pensamiento a los grupos de oración y animación litúrgica, con especial atención a los ancianos que, como \textquote{lámparas vivientes}, haciendo sus adoraciones y súplicas ascienden permanentemente al Señor en esta iglesia parroquial.]}

8. \textquote{No sólo de pan vivirá el hombre, sino de toda palabra que sale de la boca de Dios} (\textit{Mt} 4, 4). La liturgia nos recuerda estas palabras de Cristo dirigidas al tentador. ¡Cuán importantes son al comienzo de la Cuaresma!

\textquote{No solo de pan}\ldots este es el sentido del ayuno, se trata de practicar la templanza al comer y al usar los bienes materiales. Sino: \textquote{de toda palabra que sale de la boca de Dios}. Así que dediquemos más tiempo y espacio a este alimento que nutre la mente y el corazón, que nutre el alma.

En el espíritu de estas palabras de Cristo, repetimos a menudo en el período actual: \textquote{Gloria a ti, oh Cristo, Palabra de Dios} (\textit{Cantus ad Evangelium}).

Sí. Gloria a ti, Verbo, que te hiciste carne. Gloria a ti, Cristo, nuestro Redentor. \textquote{Tú tienes palabras de vida eterna} (\textit{Jn} 6, 68).
\end{body}


\label{b-03-01-1988H}
\newpage

\subsubsection{Homilía (1991): Dejar que el Espíritu nos empuje}

\src{Visita pastoral a la Parroquia Romana de Santa Dorotea. \\17 de febrero de 1991.}

\begin{body}
\ltr[1. «]{E}{l} Espíritu empujó a Jesús al desierto y allí permaneció cuarenta días siendo tentado por Satanás» (\textit{Mc} 1, 12). Con estas pocas palabras el \textbf{evangelista Marcos} describe la prueba que sufrió Jesús antes de comenzar su misión. Es una prueba de la que el Señor sale victorioso y que lo capacita para anunciar el Evangelio del Reino, llamando a todos a acogerlo en la fe, en actitud de conversión, para hacerse sus discípulos. (\ldots) En los cuarenta días de este tiempo de Cuaresma, que acaba de comenzar, también vuestra comunidad cristiana es impulsada por el Espíritu a adentrarse en el desierto\ldots Es decir, en ese clima espiritual donde, a través de la escucha asidua de la Palabra de Dios, la oración y la Caridad activa es concedido a los bautizados entrar en un diálogo más intenso con el Padre, que está en los Cielos, a abandonar los \textquote{ídolos} de este mundo, a madurar opciones valientes orientadas hacia una auténtica fidelidad a las necesidades evangélicas y a redescubrir una fuerte solidaridad con los hermanos. Entremos, por tanto, en este \textquote{tiempo propicio} de purificación e iluminación. Entremos y salgamos todos renovados, siguiendo a Cristo nuestro guía, nuestro maestro y modelo. Caminemos este \textquote{camino espiritual}, dejándonos llevar por el Espíritu, que quiere hacer de cada uno de nosotros una \textquote{nueva criatura}, capaz de anunciar y dar testimonio del Evangelio a todos los hombres.

2. El camino en el \textquote{desierto}, que la Iglesia está urgida a realizar en los cuarenta días de Cuaresma, adquiere un significado rico, una profundidad salvífica e implica elecciones de renovación exigentes, que conviene destacar. El itinerario de Cuaresma parte de una fuerte convicción de fe. Implica ser conscientes de que en el camino uno no está solo y abandonado a sí mismo, sino que es guiado por el Espíritu. Ese mismo Espíritu dado a Cristo y que ha sido comunicado también a los creyentes en el bautismo, el primero de los sacramentos de la nueva Alianza.

La \textbf{primera lectura} de esta celebración eucarística en clave profética de figura y anuncio, y la \textbf{segunda lectura}, que declara su cumplimiento en Cristo y en la Iglesia, ofrecen la correcta comprensión del acontecimiento que tuvo lugar con el \textquote{paso} de la muerte a la vida, realizado en el bautismo. A través de las aguas del diluvio, Dios destruyó el pecado de rebelión de los hombres y dio lugar a una nueva humanidad reconciliada con Él. Esto lo confirma la Alianza hecha con Noé, simbolizada en la señal del arco iris, casi un arco de caza que estaba colgado en las nubes, para indicar una pacificación universal y cósmica que ya no debe ser perturbada. Estamos ante un anuncio de la victoria de Dios y de su misericordia sobre un mundo que siempre está tentado a rebelarse contra Dios y prescindir de él.

La Cuaresma, por tanto, debe llevarnos a ser cada vez más conscientes de lo que el Espíritu, invocado en el bautismo, ha obrado en nosotros, para que podamos renovar con mayor conciencia, en la vigilia pascual, la alianza bautismal y los compromisos que de ella brotan. Como en el antiguo Israel vagando por el desierto, Dios ofrece al pueblo de la nueva Alianza, que camina hacia la Pascua eterna, los signos de la benevolencia y la gracia que libera y salva, con la condición, sin embargo, de que el hombre diga el \textquote{sí} de su fidelidad y obediencia a la divina propuesta de salvación.

3. Queridos hermanos y hermanas\ldots la liturgia de hoy constituye para vuestra comunidad\ldots un \textquote{programa} de vida estimulante y exigente. [Este tiempo puede ser para nosotros] un verdadero itinerario de purificación e iluminación de cara a una comunión más intensa con Dios y con los hermanos que se realiza a través de la oración, la penitencia y un servicio evangélico más fiel y generoso [a la ciudad]. Esta realmente puede considerarse como una especie de \textquote{desierto}, es decir, un lugar en el que se destacan por un lado los signos de la presencia del Tentador, que induce a los hombres a apartarse de Dios, a ceder a muchas formas de idolatría y pecado, provocando desintegración y división; y, por otro, el espacio en el que el Señor sigue dando el Espíritu a través de múltiples signos de misericordia, gracia y caridad.

\txtsmall{[Vuestro barrio también está marcado por esta \textquote{ambivalencia}. Lo caracterizan la indiferencia religiosa, el pluralismo ideológico (\ldots) múltiples formas de marginación social y pobreza. Todo esto exige una renovación en la fe, un compromiso misionero más fuerte, una pastoral más orgánica y armoniosa para servir al hombre inspirándose en el Evangelio. Hay mucha gente pobre que ve a esta comunidad parroquial como un punto de referencia y un centro de acogida, ayuda y apoyo, tanto material como espiritual. Siguen vivas en vuestra comunidad las huellas de los muchos santos que han vivido aquí, dejando vestigios de su testimonio y su amor por los pobres, los marginados, los jóvenes: San José de Calasanz, San Gaetano da Thiene, Santa Paola Frassinetti, y San Maximiliano Kolbe, el padre conventual polaco, que se ha detenido aquí varias veces en oración y que honra a la Iglesia y a vuestra Orden con su fidelidad a Cristo y al hombre, hasta el holocausto de su vida. Seguid sus pasos para dar respuesta a las viejas y nuevas miserias que hoy es posible encontrar en las calles de vuestro barrio. Hacedlo sobre todo \textquote{juntos}, sintiéndoos y trabajando en comunidad, evitando la tentación del absentismo o la delegación a pocas personas. Cread cada vez más un clima de fraternidad y de corresponsabilidad entre vosotros (\ldots)] [\ldots]}

5. \textquote{Los caminos del Señor son verdad y gracia} (Salmo Responsorial). Queridos amigos, el tiempo de Cuaresma os abre un camino y traza las vías para recorrerlo. Ante todo, la vía de la verdad, de una adhesión más consciente a la verdad evangélica, para haceros testigos y heraldos de ella para una \textquote{nueva evangelización}. Y luego la vía de la gracia, es decir, de una participación más activa y fecunda en los sacramentos, para vivir como hombres nuevos, capaces de renovar el mundo. Dios os guíe y el Espíritu os sostenga. ¡En esta Cuaresma y siempre! ¡Amén!
\end{body}

\subsubsection{Homilía (1997): Hacer alianza con Dios}

\src{Visita pastoral a la Parroquia Romana de San Andrés Avellino. \\16 de febrero de 1997.}

\begin{body}
\ltr[1. «]{Y}{o} establezco mi alianza con vosotros» (\textit{Gn} 9, 8). La liturgia de la Palabra de este primer domingo de Cuaresma nos presenta la alianza que Dios establece con los hombres y con la creación, después del diluvio, a través de \textbf{Noé}. Hemos vuelto a escuchar las solemnes palabras que pronunció Dios: \textquote{Yo hago una alianza con vosotros y con vuestros descendientes, con todos los animales que os acompañaron (\ldots). Hago una alianza con vosotros: el diluvio no volverá a destruir la vida, ni habrá otro diluvio que devaste la tierra} (\textit{Gn} 9, 9-11). Esta alianza tiene su valor típico en el Antiguo Testamento. Dios, creador del hombre y de todos los seres vivos, en cierto sentido había aniquilado con el diluvio cuanto él mismo había creado. Ese castigo tuvo como causa el pecado, difundido en el mundo después de la caída de nuestros primeros padres. Sin embargo, las aguas no exterminaron a Noé y a su familia, y tampoco a los animales que había recogido en el arca. De ese modo, se salvaron el hombre y los demás seres vivos que, habiendo sobrevivido al castigo del Creador, constituyeron después del diluvio el comienzo de una nueva alianza entre Dios y la creación. Esa alianza tuvo su signo tangible en el arco iris: \textquote{Pondré mi arco en el cielo –dice Dios–, como señal de mi alianza con la tierra. Cuando traiga nubes sobre la tierra, aparecerá en las nubes el arco, y recordaré mi alianza con vosotros} (\textit{Gn} 9, 13-15).

2. Las lecturas de hoy nos permiten, por tanto, mirar de un modo nuevo al hombre y al mundo en el que vivimos. En efecto, el mundo y el hombre no sólo representan la realidad de la existencia en cuanto expresión de la obra creadora de Dios; también son la imagen de la alianza. Toda la creación habla de esta alianza. A lo largo de las diversas épocas de la historia los hombres han seguido cometiendo pecados, tal vez incluso mayores que los descritos antes del diluvio. Sin embargo, las palabras de la alianza que Dios estableció con Noé nos permiten comprender que ya ningún pecado podrá llevar a Dios a aniquilar el mundo que él mismo creó. La liturgia de hoy abre ante nuestros ojos una visión nueva del mundo. Nos ayuda a tomar conciencia del valor que el mundo tiene a los ojos de Dios, quien incluyó toda la obra de la creación en la alianza que selló con Noé, y se comprometió a salvarla de la destrucción.

3. El miércoles pasado, con la imposición de la ceniza, comenzó la Cuaresma, y hoy es el primer domingo de este tiempo fuerte, que hace referencia al ayuno de cuarenta días que Jesús empezó después de su bautismo en el Jordán. A este propósito, \textbf{san Marcos}, que nos acompaña este año en la liturgia dominical, escribe: \textquote{El Espíritu impulsó a Jesús al desierto. Se quedó en el desierto cuarenta días, dejándose tentar por Satanás; vivía entre alimañas, y los ángeles le servían} (\textit{Mc} 1, 12-13). San Mateo, en el pasaje paralelo, anota sólo la respuesta que el Señor dio al tentador que lo provocaba para que transformara las piedras en panes: \textquote{Si eres Hijo de Dios, di que estas piedras se conviertan en panes} (\textit{Mt} 4, 3). Jesús respondió: \textquote{No sólo de pan vive el hombre, sino de toda palabra que sale de la boca de Dios} (\textit{Mt} 4, 4; cf. \textit{Aleluya}). Esta es una de las tres respuestas de Cristo a Satanás, que trataba de engañarlo y vencerlo, haciendo referencia a las tres concupiscencias de la naturaleza humana caída. En el umbral de la Cuaresma, la victoria de Cristo contra el diablo constituye, en cierta manera, una invitación a vencer el mal con el esfuerzo ascético, una de cuyas manifestaciones es el ayuno, a fin de vivir este período con autenticidad. \txtsmall{[\ldots]}

5. \textquote{Se ha cumplido el plazo, está cerca el reino de Dios: convertíos y creed en el Evangelio} (\textit{Mc} 1, 15). Estas palabras del \textbf{evangelista Marcos} resuenan en nuestro corazón. El evangelio comienza con la misión de Jesús, misión que se cumplirá con los acontecimientos pascuales. La Iglesia prosigue en el tiempo esta misión, a la que cada uno de nosotros está llamado a dar su propia aportación personal, anunciando y testimoniando a Cristo, muerto y resucitado por la salvación del mundo. \txtsmall{[\ldots]}

6. Escribe \textbf{san Pedro} en su \textbf{primera carta}: \textquote{Cristo murió por los pecados una vez para siempre: el inocente por los culpables (\ldots). Con este espíritu, fue a proclamar su mensaje a los espíritus encarcelados que en un tiempo habían sido rebeldes, cuando la paciencia de Dios aguardaba en tiempos de Noé, mientras se construía el arca, en la que unos pocos –ocho personas– se salvaron cruzando las aguas} (\textit{1 Pe} 3, 18-20). Estas palabras de Pedro hacen referencia a la alianza de Noé, de la que nos ha hablado la \textbf{primera lectura}. Esa alianza representa un modelo, un símbolo, una figura de la nueva alianza que Dios concluyó con toda la humanidad en Jesucristo, por medio de su muerte en la cruz y de su resurrección. Si la antigua alianza tenía que ver, ante todo, con la creación, la nueva, fundada en el misterio pascual de Cristo, es la alianza de la Redención.

En el texto que hemos escuchado, el \textbf{apóstol Pedro} alude al sacramento del bautismo. Las aguas destructoras del diluvio son sustituidas por las aguas bautismales, que santifican. El bautismo es el sacramento fundamental en el que se hace realidad la alianza de la redención del hombre. Ya desde el origen de la tradición cristiana, la Cuaresma era prácticamente una preparación para el bautismo, que se administraba a los catecúmenos en la solemne Vigilia de Pascua.

Amadísimos hermanos y hermanas, renovemos en nosotros mismos, especialmente durante este período cuaresmal, la conciencia de nuestra alianza con Dios. Dios estableció una alianza con Noé y la inscribió en la obra de la creación. Cristo, Redentor del hombre y de todo el hombre, llevó a plenitud la obra del Creador con su muerte y su resurrección.

Hemos sido redimidos por la sangre de Cristo. Cristo murió por los pecados una vez para siempre: el inocente por los culpables. Amén.
\end{body}

\subsubsection{Homilía (2000): Pedir perdón y perdonar}

\src{Santa Misa de la Jornada del Perdón del Año Santo 2000. \\12 de marzo del 2000.}

\rbr{Esta es una celebración especial, en el contexto del Gran Jubileo del año 2000, por ese motivo, algunas de las lecturas referidas en la homilía no son las habituales de este domingo. No obstante, la reflexiones del Papa pueden aplicar tanto para este domingo, como para una celebración penitencial durante el tiempo de Cuaresma.}


\begin{body}
\ltr[1. «]{E}{n} nombre de Cristo os suplicamos: ¡reconciliaos con Dios! A quien no conoció pecado, le hizo pecado por nosotros, para que viniésemos a ser justicia de Dios en él» (\textit{2 Co} 5, 20-21). La Iglesia relee estas palabras de san Pablo cada año, el miércoles de Ceniza, al comienzo de la Cuaresma. Durante el tiempo cuaresmal, la Iglesia desea unirse de modo particular a Cristo, que, impulsado interiormente por el Espíritu Santo, inició su misión mesiánica dirigiéndose al desierto, donde ayunó durante cuarenta días y cuarenta noches (cf. \textit{Mc} 1, 12-13). Al término de ese ayuno fue tentado por Satanás, como narra sintéticamente, en la liturgia de hoy, el \textbf{evangelista san Marcos} (cf. \textit{Mc} 1, 13). San Mateo y san Lucas, en cambio, tratan con mayor amplitud ese combate de Cristo en el desierto y su victoria definitiva sobre el tentador: \textquote{Apártate, Satanás, porque está escrito: \textquote{Al Señor tu Dios adorarás, y sólo a él darás culto}} (\textit{Mt} 4, 10). Quien habla así es aquel \textquote{que no conoció pecado} (\textit{2 Co} 5, 21), Jesús, \textquote{el Santo de Dios} (\textit{Mc} 1, 24).

2. \textquote{A quien no conoció pecado, le hizo pecado por nosotros} (\textit{2 Co} 5, 21). Acabamos de escuchar en la segunda lectura esta afirmación sorprendente del Apóstol. ¿Qué significan estas palabras? Parecen una paradoja y, efectivamente, lo son. ¿Cómo pudo Dios, que es la santidad misma, \textquote{hacer pecado} a su Hijo unigénito, enviado al mundo? Sin embargo, esto es precisamente lo que leemos en el pasaje de la segunda carta de san Pablo a los Corintios. Nos encontramos ante un misterio: misterio que, a primera vista, resulta desconcertante, pero que se inscribe claramente en la Revelación divina. Ya en el Antiguo Testamento, el libro de Isaías habla de ello con inspiración profética en el cuarto canto del Siervo de Yahveh: \textquote{Todos nosotros como ovejas erramos, cada uno marchó por su camino, y el Señor descargó sobre él la culpa de todos nosotros} (\textit{Is} 53, 6). Cristo, el Santo, a pesar de estar absolutamente sin pecado, acepta tomar sobre sí nuestros pecados. Acepta para redimirnos cargar con nuestros pecados para cumplir la misión recibida del Padre, que, como escribe el evangelista san Juan, \textquote{tanto amó al mundo que dio a su Hijo único, para que todo el que crea en él (\ldots) tenga vida eterna} (\textit{Jn} 3, 16).

3. Ante Cristo que, por amor, cargó con nuestras iniquidades, todos estamos invitados a un profundo examen de conciencia. Uno de los elementos característicos del gran jubileo es el que he calificado como \textquote{purificación de la memoria} (\textit{Incarnationis mysterium}, 11). Como Sucesor de Pedro, he pedido que \textquote{en este año de misericordia la Iglesia, persuadida de la santidad que recibe de su Señor, se postre ante Dios e implore perdón por los pecados pasados y presentes de sus hijos} (\textit{ib}.). Este primer domingo de Cuaresma me ha parecido la ocasión propicia para que la Iglesia, reunida espiritualmente en torno al Sucesor de Pedro, implore el perdón divino por las culpas de todos los creyentes. ¡Perdonemos y pidamos perdón!

Esta exhortación ha suscitado en la comunidad eclesial una profunda y provechosa reflexión, que ha llevado a la publicación, en días pasados, de un documento de la Comisión teológica internacional, titulado: \textquote{\textit{Memoria y reconciliación: la Iglesia y las culpas del pasado}}. Doy las gracias a todos los que han contribuido a la elaboración de este texto. Es muy útil para una comprensión y aplicación correctas de la auténtica petición de perdón, fundada en la responsabilidad objetiva que une a los cristianos, en cuanto miembros del Cuerpo místico, y que impulsa a los fieles de hoy a reconocer, además de sus culpas propias, las de los cristianos de ayer, a la luz de un cuidadoso discernimiento histórico y teológico. En efecto, \textquote{por el vínculo que une a unos y otros en el Cuerpo místico, y aun sin tener responsabilidad personal ni eludir el juicio de Dios, el único que conoce los corazones, somos portadores del peso de los errores y de las culpas de quienes nos han precedido} (\textit{Incarnationis mysterium}, 11). Reconocer las desviaciones del pasado sirve para despertar nuestra conciencia ante los compromisos del presente, abriendo a cada uno el camino de la conversión.

4. ¡Perdonemos y pidamos perdón! A la vez que alabamos a Dios, que, en su amor misericordioso, ha suscitado en la Iglesia una cosecha maravillosa de santidad, de celo misionero y de entrega total a Cristo y al prójimo, no podemos menos de reconocer las infidelidades al Evangelio que han cometido algunos de nuestros hermanos, especialmente durante el segundo milenio. Pidamos perdón por las divisiones que han surgido entre los cristianos, por el uso de la violencia que algunos de ellos hicieron al servicio de la verdad, y por las actitudes de desconfianza y hostilidad adoptadas a veces con respecto a los seguidores de otras religiones.

Confesemos, con mayor razón, nuestras responsabilidades de cristianos por los males actuales. Frente al ateísmo, a la indiferencia religiosa, al secularismo, al relativismo ético, a las violaciones del derecho a la vida, al desinterés por la pobreza de numerosos países, no podemos menos de preguntarnos cuáles son nuestras responsabilidades. Por la parte que cada uno de nosotros, con sus comportamientos, ha tenido en estos males, contribuyendo a desfigurar el rostro de la Iglesia, pidamos humildemente perdón. Al mismo tiempo que confesamos nuestras culpas, perdonemos las culpas cometidas por los demás contra nosotros. En el curso de la historia los cristianos han sufrido muchas veces atropellos, prepotencias y persecuciones a causa de su fe. Al igual que perdonaron las víctimas de dichos abusos, así también perdonemos nosotros. La Iglesia de hoy y de siempre se siente comprometida a purificar la memoria de esos tristes hechos de todo sentimiento de rencor o venganza. De este modo, el jubileo se transforma para todos en ocasión propicia de profunda conversión al Evangelio. De la acogida del perdón divino brota el compromiso de perdonar a los hermanos y de reconciliación recíproca.

5. Pero ¿qué significa para nosotros el término \textquote{reconciliación}? Para captar su sentido y su valor exactos, es necesario ante todo darse cuenta de la posibilidad de la división, de la separación. Sí, el hombre es la única criatura en la tierra que puede establecer una relación de comunión con su Creador, pero también es la única que puede separarse de él. De hecho, por desgracia, con frecuencia se aleja de Dios.

Afortunadamente, muchos, como el hijo pródigo, del que habla el evangelio de san Lucas (cf. \textit{Lc} 15, 13), después de abandonar la casa paterna y disipar la herencia recibida, al tocar fondo, se dan cuenta de todo lo que han perdido (cf. \textit{Lc} 15, 13-17). Entonces, emprenden el camino de vuelta: \textquote{Me levantaré, iré a mi padre y le diré: \textquote{Padre, pequé\ldots}} (\textit{Lc} 15, 18). Dios, bien representado por el padre de la parábola, acoge a todo hijo pródigo que vuelve a él. Lo acoge por medio de Cristo, en quien el pecador puede volver a ser \textquote{justo} con la justicia de Dios. Lo acoge, porque hizo pecado por nosotros a su Hijo eterno. Sí, sólo por medio de Cristo podemos llegar a ser justicia de Dios (cf. \textit{2 Co} 5, 21).

6. \textquote{Dios tanto amó al mundo que dio a su Hijo único}. ¡Éste es en síntesis, el significado, del misterio de la redención del mundo! Hay que darse cuenta plenamente del valor del gran don que el Padre nos ha hecho en Jesús. Es necesario que ante la mirada de nuestra alma se presente Cristo, el Cristo de Getsemaní, el Cristo flagelado, coronado de espinas, con la cruz a cuestas y, por último, crucificado. Cristo tomó sobre sí el peso de los pecados de todos los hombres, el peso de nuestros pecados, para que, en virtud de su sacrificio salvífico, pudiéramos reconciliarnos con Dios.

Saulo de Tarso, convertido en san Pablo, se presenta hoy ante nosotros como testigo: él experimentó, de modo singular, la fuerza de la cruz en el camino de Damasco. El Resucitado se le manifestó con todo el esplendor de su poder: \textquote{Saulo, Saulo, ¿por qué me persigues? (\ldots) ¿Quién eres, Señor? (\ldots) Yo soy Jesús, a quien tú persigues} (\textit{Hch} 9, 4-5). San Pablo, que experimentó con tanta fuerza el poder de la cruz de Cristo, se dirige hoy a nosotros con una ardiente súplica: \textquote{Os exhortamos a que no recibáis en vano la gracia de Dios}. San Pablo insiste en que esta gracia nos la ofrece Dios mismo, que nos dice hoy a nosotros: \textquote{En el tiempo favorable te escuché y en el día de salvación te ayudé} (\textit{2 Co} 6, 2).

María, Madre del perdón, ayúdanos a acoger la gracia del perdón que el jubileo nos ofrece abundantemente. Haz que la Cuaresma de este extraordinario Año santo sea para todos los creyentes, y para cada hombre que busca a Dios, el momento favorable, el tiempo de la reconciliación, el tiempo de la salvación.
\end{body}


\subsubsection{Ángelus (2003): Purificar la conciencia}

\src{9 de marzo de 2003.}

\begin{body}
\ltr[1. ]{E}{l} miércoles pasado, con el rito de la ceniza, entramos en la Cuaresma, itinerario penitencial de preparación para la Pascua, ocasión para que todos los bautizados renueven su espíritu de fe y afiancen su compromiso de coherencia evangélica. Como sugiere el \textbf{evangelio} de hoy (\textit{Mc} 1, 12-15), durante los cuarenta días de la Cuaresma los creyentes están llamados a seguir a Cristo al \textquote{desierto}, para afrontar y vencer con él al espíritu del mal. Se trata de una lucha interior, de la que depende el planteamiento concreto de la vida. En efecto, del corazón del hombre brotan sus intenciones y sus acciones (cf. \textit{Mc} 7, 21); por tanto, sólo purificando la conciencia se prepara el camino de la justicia y de la paz, tanto en el plano personal como en el ámbito social.

2. En el actual contexto internacional, se siente con más fuerza la exigencia de purificar la conciencia y convertir el corazón a la paz verdadera. Al respecto, es muy elocuente el ejemplo de Cristo que desenmascara y vence las mentiras de Satanás con la fuerza de la verdad, contenida en la palabra de Dios. En lo más íntimo de cada persona resuenan la voz de Dios y la insidiosa voz del maligno. Esta última trata de engañar al hombre, seduciéndolo con la perspectiva de falsos bienes, para alejarlo del verdadero bien, que consiste precisamente en cumplir la voluntad divina. Pero la oración humilde y confiada, fortalecida por el ayuno, permite superar también las pruebas más duras, e infunde la valentía necesaria para combatir el mal con el bien. La Cuaresma se convierte así en un tiempo de provechoso entrenamiento del espíritu.

3. Amadísimos hermanos y hermanas, invoquemos a la Virgen santísima para que nos guíe a todos a recorrer con generosidad este exigente camino cuaresmal. [A vuestras oraciones quisiera encomendar, de modo especial, los ejercicios espirituales que, a partir de esta tarde, como todos los años, tendré la oportunidad de hacer juntamente con mis más íntimos colaboradores de la Curia romana. Durante esta semana de silencio y oración tendré presentes las necesidades de la Iglesia y las preocupaciones de toda la humanidad, sobre todo por lo que concierne a la paz en Irak y en Tierra Santa.]
\end{body}

\newsection
\subsection{Benedicto XVI, papa}

\subsubsection{Ángelus (2006): Vencer la tentación para ser libres}

\src{5 de marzo de 2006.}

\begin{body}
\ltr{E}{l} miércoles pasado iniciamos la Cuaresma, y hoy celebramos el primer domingo de este tiempo litúrgico, que estimula a los cristianos a comprometerse en un camino de preparación para la Pascua. Hoy el \textbf{evangelio} nos recuerda que Jesús, después de haber sido bautizado en el río Jordán, impulsado por el Espíritu Santo, que se había posado sobre él revelándolo como el Cristo, se retiró durante cuarenta días al desierto de Judá, donde superó las tentaciones de Satanás (cf. \textit{Mc} 1, 12-13). Siguiendo a su Maestro y Señor, también los cristianos entran espiritualmente en el desierto cuaresmal para afrontar junto con él \textquote{el combate contra el espíritu del mal}. La imagen del desierto es una metáfora muy elocuente de la condición humana. El libro del Éxodo narra la experiencia del pueblo de Israel que, habiendo salido de Egipto, peregrinó por el desierto del Sinaí durante cuarenta años antes de llegar a la tierra prometida. A lo largo de aquel largo viaje, los judíos experimentaron toda la fuerza y la insistencia del tentador, que los inducía a perder la confianza en el Señor y a volver atrás; pero, al mismo tiempo, gracias a la mediación de Moisés, aprendieron a escuchar la voz de Dios, que los invitaba a convertirse en su pueblo santo.

Al meditar en esta página bíblica, comprendemos que, para realizar plenamente la vida en la libertad, es preciso superar la prueba que la misma libertad implica, es decir, la tentación. Sólo liberada de la esclavitud de la mentira y del pecado, la persona humana, gracias a la obediencia de la fe, que la abre a la verdad, encuentra el sentido pleno de su existencia y alcanza la paz, el amor y la alegría. Precisamente por eso, la Cuaresma constituye un tiempo favorable para una atenta revisión de vida en el recogimiento, la oración y la penitencia.

[(\ldots) Invoquemos la intercesión materna de la Virgen María a fin de que la Cuaresma sea para todos los cristianos una ocasión de conversión y de impulso aún más valiente hacia la santidad.]
\end{body}


\subsubsection{Ángelus (2009): La kénosis de Cristo}

\src{1 de marzo de 2009.}

\begin{body}
\ltr{H}{oy} es el primer domingo de Cuaresma, y el Evangelio, con el estilo sobrio y conciso de \textbf{san Marcos}, nos introduce en el clima de este tiempo litúrgico: \textquote{El Espíritu impulsó a Jesús al desierto y permaneció en el desierto cuarenta días, siendo tentado por Satanás} (\textit{Mc} 1, 12-13). En Tierra Santa, al oeste del río Jordán y del oasis de Jericó, se encuentra el desierto de Judea, que, por valles pedregosos, superando un desnivel de cerca de mil metros, sube hasta Jerusalén. Después de recibir el bautismo de Juan, Jesús se adentró en aquella soledad conducido por el mismo Espíritu Santo que se había posado sobre él consagrándolo y revelándolo como Hijo de Dios. En el desierto, lugar de la prueba, como muestra la experiencia del pueblo de Israel, aparece con intenso dramatismo la realidad de la kénosis, del vaciamiento de Cristo, que se despojó de la forma de Dios (cf. \textit{Flp} 2, 6-7). Él, que no ha pecado y no puede pecar, se somete a la prueba y por eso puede compadecerse de nuestras flaquezas (cf. \textit{Hb} 4, 15). Se deja tentar por Satanás, el adversario, que desde el principio se opuso al designio salvífico de Dios en favor de los hombres. Casi de pasada, en la brevedad del relato, ante esta figura oscura y tenebrosa que tiene la osadía de tentar al Señor, aparecen los ángeles, figuras luminosas y misteriosas. Los ángeles, dice el evangelio, \textquote{servían} a Jesús (\textit{Mc} 1, 13); son el contrapunto de Satanás. \textquote{Ángel} quiere decir \textquote{enviado}. 

En todo el Antiguo Testamento encontramos estas figuras que, en nombre de Dios, ayudan y guían a los hombres. Basta recordar el libro de Tobías, en el que aparece la figura del ángel Rafael, que ayuda al protagonista en numerosas vicisitudes. La presencia tranquilizadora del ángel del Señor acompaña al pueblo de Israel en todas las circunstancias, tanto en las buenas como en las malas. En el umbral del Nuevo Testamento, Gabriel es enviado a anunciar a Zacarías y a María los acontecimientos felices que constituyen el inicio de nuestra salvación; y un ángel, cuyo nombre no se dice, advierte a José, orientándolo en aquel momento de incertidumbre. Un coro de ángeles lleva a los pastores la buena nueva del nacimiento del Salvador; y, del mismo modo, son también los ángeles quienes anuncian a las mujeres la feliz noticia de su resurrección. Al final de los tiempos, los ángeles acompañarán a Jesús en su venida en la gloria (cf. \textit{Mt} 25, 31). Los ángeles sirven a Jesús, que es ciertamente superior a ellos, y su dignidad se proclama aquí, en el evangelio, de modo claro aunque discreto. En efecto, incluso en la situación de extrema pobreza y humildad, cuando es tentado por Satanás, sigue siendo el Hijo de Dios, el Mesías, el Señor. (\ldots) Quitaríamos una parte notable del Evangelio, si dejáramos de lado a estos seres enviados por Dios, que anuncian su presencia en medio de nosotros y son un signo de ella. Invoquémoslos a menudo, para que nos sostengan en el compromiso de seguir a Jesús hasta identificarnos con él. [\ldots] María, Reina de los ángeles, ruega por nosotros.
\end{body}


\label{b2-03-012009A}
\newpage

\subsubsection{Ángelus (2012): Paciencia y humildad}

\src{Plaza de San Pedro, 26 de febrero de 2012.}

\begin{body}
\ltr{E}{n} este primer domingo de Cuaresma encontramos a Jesús, quien, tras haber recibido el bautismo en el río Jordán por Juan el Bautista (cf. \textit{Mc} 1, 9), sufre la tentación en el desierto (cf. \textit{Mc} 1, 12-13). La narración de \textbf{san Marcos} es concisa, carente de los detalles que leemos en los otros dos evangelios de Mateo y de Lucas. El desierto del que se habla tiene varios significados. Puede indicar el estado de abandono y de soledad, el \textquote{lugar} de la debilidad del hombre donde no existen apoyos ni seguridades, donde la tentación se hace más fuerte. Pero puede también indicar un lugar de refugio y de amparo –como lo fue para el pueblo de Israel en fuga de la esclavitud egipcia– en el que se puede experimentar de modo particular la presencia de Dios. Jesús \textquote{se quedó en el desierto cuarenta días, siendo tentado por Satanás} (\textit{Mc} 1, 13). San León Magno comenta que \textquote{el Señor quiso sufrir el ataque del tentador para defendernos con su ayuda y para instruirnos con su ejemplo} (\textit{Tractatus} XXXIX, 3 De ieiunio quadragesimae: CCL 138/a, Turnholti 1973, 214-215).

¿Qué puede enseñarnos este episodio? Como leemos en el libro de la \textit{Imitación de Cristo}, \textquote{el hombre jamás está del todo exento de las tentaciones mientras vive\ldots pero es con la paciencia y con la verdadera humildad como nos haremos más fuertes que cualquier enemigo} (\textit{Liber} I, c. XIII, Ciudad del Vaticano 1982, 37); con la paciencia y la humildad de seguir cada día al Señor, aprendemos a construir nuestra vida no fuera de Él y como si no existiera, sino en Él y con Él, porque es la fuente de la vida verdadera. La tentación de suprimir a Dios, de poner orden solos en uno mismo y en el mundo contando exclusivamente con las propias capacidades, está siempre presente en la historia del hombre.

Jesús proclama que \textquote{se ha cumplido el tiempo y está cerca el reino de Dios} (\textit{Mc} 1, 15), anuncia que en Él sucede algo nuevo: Dios se dirige al hombre de forma insospechada, con una cercanía única y concreta, llena de amor; Dios se encarna y entra en el mundo del hombre para cargar con el pecado, para vencer el mal y volver a llevar al hombre al mundo de Dios. Pero este anuncio se acompaña de la petición de corresponder a un don tan grande. Jesús, en efecto, añade: \textquote{convertíos y creed en el Evangelio} (\textit{Mc} 1, 15); es la invitación a tener fe en Dios y a convertir cada día nuestra vida a su voluntad, orientando hacia el bien cada una de nuestras acciones y pensamientos. El tiempo de Cuaresma es el momento propicio para renovar y fortalecer nuestra relación con Dios a través de la oración diaria, los gestos de penitencia, las obras de caridad fraterna.

Supliquemos con fervor a María santísima que acompañe nuestro camino cuaresmal con su protección y nos ayude a imprimir en nuestro corazón y en nuestra vida las palabras de Jesucristo para convertirnos a Él.
\end{body}

\newsection
\subsection{Francisco, papa}

\subsubsection{Ángelus (2015): Tiempo de combate espiritual}

\src{Plaza de San Pedro, 22 de febrero de 2015.}

\begin{body}
\ltr{E}{l} miércoles pasado, con el rito de la Ceniza, inició la Cuaresma, y hoy es el primer domingo de este tiempo litúrgico que hace referencia a los cuarenta días que Jesús pasó en el desierto, después del bautismo en el río Jordán. Escribe \textbf{san Marcos} en el Evangelio de hoy: \textquote{El Espíritu lo empujó al desierto. Se quedó en el desierto cuarenta días, siendo tentado por Satanás; vivía con las fieras y los ángeles lo servían} (\textit{Mc} 1, 12-13). Con estas escuetas palabras el evangelista describe la prueba que Jesús afrontó voluntariamente, antes de iniciar su misión mesiánica. Es una prueba de la que el Señor sale victorioso y que lo prepara para anunciar el Evangelio del Reino de Dios. Él, en esos cuarenta días de soledad, se enfrentó a Satanás \textquote{cuerpo a cuerpo}, desenmascaró sus tentaciones y lo venció. Y en Él hemos vencido todos, pero a nosotros nos toca proteger esta victoria en nuestra vida diaria.

La Iglesia nos hace recordar ese misterio al inicio de la Cuaresma, porque nos da la perspectiva y el sentido de este tiempo, que es un tiempo de combate –en Cuaresma se debe combatir–, un tiempo de combate espiritual contra el espíritu del mal (cf. \textit{Oración colecta del Miércoles de Ceniza}). Y mientras atravesamos el \textquote{desierto} cuaresmal, mantengamos la mirada dirigida a la Pascua, que es la victoria definitiva de Jesús contra el Maligno, contra el pecado y contra la muerte. He aquí entonces el significado de este primer domingo de Cuaresma: volver a situarnos decididamente en la senda de Jesús, la senda que conduce a la vida. Mirar a Jesús, lo que hizo Jesús, e ir con Él.

Y este camino de Jesús pasa a través del desierto. El desierto es el lugar donde se puede escuchar la voz de Dios y la voz del tentador. En el rumor, en la confusión esto no se puede hacer; se oyen sólo las voces superficiales. En cambio, en el desierto podemos bajar en profundidad, donde se juega verdaderamente nuestro destino, la vida o la muerte. ¿Y cómo escuchamos la voz de Dios? La escuchamos en su Palabra. Por eso es importante conocer las Escrituras, porque de otro modo no sabremos responder a las asechanzas del maligno. Y aquí quisiera volver a mi consejo de leer cada día el Evangelio: cada día leer el Evangelio, meditarlo, un poco, diez minutos; y llevarlo incluso siempre con nosotros: en el bolsillo, en la cartera\ldots Pero tener el Evangelio al alcance de la mano. El desierto cuaresmal nos ayuda a decir no a la mundanidad, a los \textquote{ídolos}, nos ayuda a hacer elecciones valientes conformes al Evangelio y a reforzar la solidaridad con los hermanos.

Entonces entramos en el desierto sin miedo, porque no estamos solos: estamos con Jesús, con el Padre y con el Espíritu Santo. Es más, como lo fue para Jesús, es precisamente el Espíritu Santo quien nos guía por el camino cuaresmal, el mismo Espíritu que descendió sobre Jesús y que recibimos en el Bautismo. La Cuaresma, por ello, es un tiempo propicio que debe conducirnos a tomar cada vez más conciencia de cuánto el Espíritu Santo, recibido en el Bautismo, obró y puede obrar en nosotros. Y al final del itinerario cuaresmal, en la Vigilia pascual, podremos renovar con mayor consciencia la alianza bautismal y los compromisos que de ella derivan.

Que la Virgen santa, modelo de docilidad al Espíritu, nos ayude a dejarnos conducir por Él, que quiere hacer de cada uno de nosotros una \textquote{nueva creatura}.

\txtsmall{[A Ella encomiendo, en especial, esta semana de ejercicios espirituales, que iniciará hoy por la tarde, y en la que participaré juntamente con mis colaboradores de la Curia romana. Rezad para que en este \textquote{desierto} que son los ejercicios espirituales podamos escuchar la voz de Jesús y también corregir tantos defectos que todos nosotros tenemos, y hacer frente a las tentaciones que cada día nos atacan. Os pido, por lo tanto, que nos acompañéis con vuestra oración.]}

\end{body}

\label{b2-03-01-2015A}
\newpage


\subsubsection{Ángelus (2018): Conviértete y cree en el Evangelio}

\src{Plaza de San Pedro, 18 de febrero de 2018.}

\begin{body}
\ltr{E}{n} este primer domingo de Cuaresma, el Evangelio menciona los temas de la tentación, la conversión y la Buena Noticia. Escribe el \textbf{evangelista Marcos}: \textquote{El Espíritu le empuja al desierto, y permaneció en el desierto cuarenta días, siendo tentando por Satanás} (\textit{Mc} 1, 12-13). Jesús va al desierto a prepararse para su misión en el mundo. Él no necesita conversión, pero, en cuanto hombre, debe pasar a través de esta prueba, ya sea por sí mismo, para obedecer a la voluntad del Padre, como por nosotros, para darnos la gracia de vencer las tentaciones. Esta preparación consiste en la lucha contra el espíritu del mal, es decir, contra el diablo. También para nosotros la Cuaresma es un tiempo de \textquote{agonismo} espiritual, de lucha espiritual: estamos llamados a afrontar al maligno mediante la oración para ser capaces, con la ayuda de Dios, de vencerlo en nuestra vida cotidiana. Nosotros lo sabemos, el mal está lamentablemente funcionando en nuestra existencia y entorno a nosotros, donde se manifiestan violencias, rechazo del otro, clausuras, guerras, injusticias. Todas estas son obra del maligno, del mal.

Inmediatamente después de las tentaciones en el desierto, Jesús empieza a predicar el Evangelio, es decir, la Buena Noticia, la segunda palabra. La primera era \textquote{tentación}; la segunda, \textquote{Buena Noticia}. Y esta Buena Noticia exige del hombre conversión –tercera palabra– y fe. Él anuncia: \textquote{El tiempo se ha cumplido y el Reino de Dios está cerca}; después dirige la exhortación: \textquote{convertíos y creed en la Buena Nueva} (\textit{Mc} 1, 15), es decir creed en esta Buena Noticia que el Reino de Dios está cerca. En nuestra vida siempre necesitamos conversión –¡todos los días!–, y la Iglesia nos hace rezar por esto. De hecho, no estamos nunca suficientemente orientados hacia Dios y debemos continuamente dirigir nuestra mente y nuestro corazón a Él. Para hacer esto es necesario tener la valentía de rechazar todo lo que nos lleva fuera del camino, los falsos valores que nos engañan atrayendo nuestro egoísmo de forma sutil. Sin embargo, debemos fiarnos del Señor, de su bondad y de su proyecto de amor para cada uno de nosotros.

La Cuaresma es un tiempo de penitencia, sí, ¡pero no es un tiempo triste! Es un tiempo de penitencia, pero no es un tiempo triste, de luto. Es un compromiso alegre y serio para despojarnos de nuestro egoísmo, de nuestro hombre viejo, y renovarnos según la gracia de nuestro bautismo. Solamente Dios nos puede donar la verdadera felicidad: es inútil que perdamos nuestro tiempo buscándola en otro lugar, en las riquezas, en los placeres, en el poder, en la carrera\ldots El Reino de Dios es la realización de todas nuestras aspiraciones, porque es, al mismo tiempo, salvación del hombre y gloria de Dios.

En este primer domingo de Cuaresma, estamos invitados a escuchar con atención y recoger este llamamiento de Jesús a convertirnos y a creer en el Evangelio. Somos exhortados a iniciar con compromiso el camino hacia la Pascua, para acoger cada vez más la gracia de Dios, que quiere transformar el mundo en un reino de justicia, de paz, de fraternidad.

Que María Santísima nos ayude a vivir esta Cuaresma con fidelidad a la Palabra de Dios y con una oración incesante, como hizo Jesús en el desierto.

¡No es imposible! Se trata de vivir las jornadas con el deseo de acoger el amor que viene de Dios y que quiere transformar nuestra vida y el mundo entero.
\end{body}

\begin{patercite}
\textquote{Convertíos y creed en el Evangelio} (\ldots) La conversión, una palabra que hay que considerar en su extraordinaria seriedad, dándonos cuenta de la sorprendente novedad que implica. En efecto, la llamada a la conversión revela y denuncia la fácil superficialidad que con frecuencia caracteriza nuestra vida. Convertirse significa cambiar de dirección en el camino de la vida: pero no con un pequeño ajuste, sino con un verdadero cambio de sentido. Conversión es ir contracorriente, donde la \textquote{corriente} es el estilo de vida superficial, incoherente e ilusorio que a menudo nos arrastra, nos domina y nos hace esclavos del mal, o en cualquier caso prisioneros de la mediocridad moral. Con la conversión, en cambio, aspiramos a la medida alta de la vida cristiana, nos adherimos al Evangelio vivo y personal, que es Jesucristo. La meta final y el sentido profundo de la conversión es su persona, él es la senda por la que todos están llamados a caminar en la vida, dejándose iluminar por su luz y sostener por su fuerza que mueve nuestros pasos. De este modo la conversión manifiesta su rostro más espléndido y fascinante: no es una simple decisión moral, que rectifica nuestra conducta de vida, sino una elección de fe, que nos implica totalmente en la comunión íntima con la persona viva y concreta de Jesús. Convertirse y creer en el Evangelio no son dos cosas distintas o de alguna manera sólo conectadas entre sí, sino que expresan la misma realidad. La conversión es el \textquote{sí} total de quien entrega su existencia al Evangelio, respondiendo libremente a Cristo, que antes se ha ofrecido al hombre como camino, verdad y vida, como el único que lo libera y lo salva. Este es precisamente el sentido de las primeras palabras con las que, según el evangelista san Marcos, Jesús inicia la predicación del \textquote{Evangelio de Dios}: \textquote{El tiempo se ha cumplido y el reino de Dios está cerca; convertíos y creed en el Evangelio} (\textit{Mc} 1, 15).

\textbf{Benedicto XVI, papa}, \textit{Catequesis}, Audiencia general, 17 de febrero de 2010, parr. 3.
\end{patercite}

\newsection
\section{Temas}

\cceth{La tentación de Jesús}
 
\cceref{CEC 394, 538-540, 2119}

\begin{ccebody}
\n{394} La Escritura atestigua la influencia nefasta de aquel a quien Jesús llama \textquote{homicida desde el principio} (\textit{Jn} 8,44) y que incluso intentó apartarlo de la misión recibida del Padre (cf. \textit{Mt} 4,1-11). \textquote{El Hijo de Dios se manifestó para deshacer las obras del diablo} (\textit{1 Jn} 3,8). La más grave en consecuencias de estas obras ha sido la seducción mentirosa que ha inducido al hombre a desobedecer a Dios.

\ccesec{Las tentaciones de Jesús}

\n{538} Los evangelios hablan de un tiempo de soledad de Jesús en el desierto inmediatamente después de su bautismo por Juan: \textquote{Impulsado por el Espíritu} al desierto, Jesús permanece allí sin comer durante cuarenta días; vive entre los animales y los ángeles le servían (cf. \textit{Mc} 1, 12-13). Al final de este tiempo, Satanás le tienta tres veces tratando de poner a prueba su actitud filial hacia Dios. Jesús rechaza estos ataques que recapitulan las tentaciones de Adán en el Paraíso y las de Israel en el desierto, y el diablo se aleja de él \textquote{hasta el tiempo determinado} (\textit{Lc} 4, 13).

\n{539} Los evangelistas indican el sentido salvífico de este acontecimiento misterioso. Jesús es el nuevo Adán que permaneció fiel allí donde el primero sucumbió a la tentación. Jesús cumplió perfectamente la vocación de Israel: al contrario de los que anteriormente provocaron a Dios durante cuarenta años por el desierto (cf. \textit{Sal} 95, 10), Cristo se revela como el Siervo de Dios totalmente obediente a la voluntad divina. En esto Jesús es vencedor del diablo; él ha \textquote{atado al hombre fuerte} para despojarle de lo que se había apropiado (\textit{Mc} 3, 27). La victoria de Jesús en el desierto sobre el Tentador es un anticipo de la victoria de la Pasión, suprema obediencia de su amor filial al Padre.

\n{540} La tentación de Jesús manifiesta la manera que tiene de ser Mesías el Hijo de Dios, en oposición a la que le propone Satanás y a la que los hombres (cf. \textit{Mt} 16, 21-23) le quieren atribuir. Por eso Cristo ha vencido al Tentador \textit{en beneficio nuestro}: \textquote{Pues no tenemos un Sumo Sacerdote que no pueda compadecerse de nuestras flaquezas, sino probado en todo igual que nosotros, excepto en el pecado} (\textit{Hb} 4, 15). La Iglesia se une todos los años, durante los cuarenta días de \textit{la Gran Cuaresma}, al Misterio de Jesús en el desierto.

\n{2119} La acción de \textit{tentar a Dios} consiste en poner a prueba, de palabra o de obra, su bondad y su omnipotencia. Así es como Satán quería conseguir de Jesús que se arrojara del templo y obligase a Dios, mediante este gesto, a actuar (cf. \textit{Lc} 4, 9). Jesús le opone las palabras de Dios: \textquote{No tentaréis al Señor, tu Dios} (\textit{Dt} 6, 16). El reto que contiene este tentar a Dios lesiona el respeto y la confianza que debemos a nuestro Creador y Señor. Incluye siempre una duda respecto a su amor, su providencia y su poder (cf. \textit{1 Co} 10, 9; \textit{Ex} 17, 2-7; \textit{Sal} 95, 9).
\end{ccebody}

\cceth{\textquote{No nos dejes caer en la tentación}} 

\cceref{CEC 2846-2849}

\begin{ccebody}
\n{2846} Esta petición llega a la raíz de la anterior, porque nuestros pecados son los frutos del consentimiento a la tentación. Pedimos a nuestro Padre que no nos \textquote{deje caer} en ella. Traducir en una sola palabra el texto griego es difícil: significa \textquote{no permitas entrar en} (cf. \textit{Mt} 26, 41), \textquote{no nos dejes sucumbir a la tentación}. \textquote{Dios ni es tentado por el mal ni tienta a nadie} (\textit{St} 1, 13), al contrario, quiere librarnos del mal. Le pedimos que no nos deje tomar el camino que conduce al pecado, pues estamos empeñados en el combate \textquote{entre la carne y el Espíritu}. Esta petición implora el Espíritu de discernimiento y de fuerza.

\n{2847} El Espíritu Santo nos hace \textit{discernir} entre la prueba, necesaria para el crecimiento del hombre interior (cf. \textit{Lc} 8, 13-15; \textit{Hch} 14, 22; \textit{2 Tm} 3, 12) en orden a una \textquote{virtud probada} (\textit{Rm} 5, 3-5), y la tentación que conduce al pecado y a la muerte (cf. \textit{St} 1, 14-15). También debemos distinguir entre \textquote{ser tentado} y \textquote{consentir} en la tentación. Por último, el discernimiento desenmascara la mentira de la tentación: aparentemente su objeto es \textquote{bueno, seductor a la vista, deseable} (\textit{Gn} 3, 6), mientras que, en realidad, su fruto es la muerte.

\ccecite{\textquote{Dios no quiere imponer el bien, quiere seres libres [\ldots] En algo la tentación es buena. Todos, menos Dios, ignoran lo que nuestra alma ha recibido de Dios, incluso nosotros. Pero la tentación lo manifiesta para enseñarnos a conocernos, y así, descubrirnos nuestra miseria, y obligarnos a dar gracias por los bienes que la tentación nos ha manifestado} (Orígenes, \textit{De oratione}, 29, 15 y 17).}

\n{2848} \textquote{No entrar en la tentación} implica una \textit{decisión del corazón:} \textquote{Porque donde esté tu tesoro, allí también estará tu corazón [\ldots] Nadie puede servir a dos señores} (\textit{Mt} 6, 21-24). \textquote{Si vivimos según el Espíritu, obremos también según el Espíritu} (\textit{Ga} 5, 25). El Padre nos da la fuerza para este \textquote{dejarnos conducir} por el Espíritu Santo. \textquote{No habéis sufrido tentación superior a la medida humana. Y fiel es Dios que no permitirá que seáis tentados sobre vuestras fuerzas. Antes bien, con la tentación os dará modo de poderla resistir con éxito} (\textit{1 Co} 10, 13).

\n{2849} Pues bien, este combate y esta victoria sólo son posibles con la oración. Por medio de su oración, Jesús es vencedor del Tentador, desde el principio (cf. \textit{Mt} 4, 11) y en el último combate de su agonía (cf. \textit{Mt} 26, 36-44). En esta petición a nuestro Padre, Cristo nos une a su combate y a su agonía. La vigilancia del corazón es recordada con insistencia en comunión con la suya (cf. \textit{Mc} 13, 9. 23. 33-37; 14, 38; \textit{Lc} 12, 35-40). La vigilancia es \textquote{guarda del corazón}, y Jesús pide al Padre que \textquote{nos guarde en su Nombre} (\textit{Jn} 17, 11). El Espíritu Santo trata de despertarnos continuamente a esta vigilancia (cf. \textit{1 Co} 16, 13; \textit{Col} 4, 2; \textit{1 Ts} 5, 6; \textit{1 Pe} 5, 8). Esta petición adquiere todo su sentido dramático referida a la tentación final de nuestro combate en la tierra; pide la \textit{perseverancia final}. \textquote{Mira que vengo como ladrón. Dichoso el que esté en vela} (\textit{Ap} 16, 15).
\end{ccebody}

\newpage

\cceth{La Alianza con Noé} 
\cceref{CEC 56-58, 71}

\begin{ccebody}
\n{56} Una vez rota la unidad del género humano por el pecado, Dios decide desde el comienzo salvar a la humanidad a través de una serie de etapas. La alianza con Noé después del diluvio (cf. \textit{Gn} 9,9) expresa el principio de la Economía divina con las \textquote{naciones}, es decir con los hombres agrupados \textquote{según sus países, cada uno según su lengua, y según sus clanes} (\textit{Gn} 10,5; cf. \textit{Gn} 10,20-31).

\n{57} Este orden a la vez cósmico, social y religioso de la pluralidad de las naciones (cf. \textit{Hch} 17,26-27), está destinado a limitar el orgullo de una humanidad caída que, unánime en su perversidad (cf. \textit{Sb} 10,5), quisiera hacer por sí misma su unidad a la manera de Babel (cf. \textit{Gn} 11,4-6). Pero, a causa del pecado (cf. \textit{Rm} 1,18-25), el politeísmo, así como la idolatría de la nación y de su jefe, son una amenaza constante de vuelta al paganismo para esta economía aún no definitiva.

\n{58} La alianza con Noé permanece en vigor mientras dura el tiempo de las naciones (cf. \textit{Lc} 21,24), hasta la proclamación universal del Evangelio. La Biblia venera algunas grandes figuras de las \textquote{naciones}, como \textquote{Abel el justo}, el rey-sacerdote Melquisedec (cf. \textit{Gn} 14,18), figura de Cristo (cf. \textit{Hb} 7,3), o los justos \textquote{Noé, Daniel y Job} (\textit{Ez} 14,14). De esta manera, la Escritura expresa qué altura de santidad pueden alcanzar los que viven según la alianza de Noé en la espera de que Cristo \textquote{reúna en uno a todos los hijos de Dios dispersos} (\textit{Jn} 11,52).

\n{71} \textit{Dios selló con Noé una alianza eterna entre Él y todos los seres vivientes (cf.} \textit{Gn} 9,16). Esta alianza durará tanto como dure el mundo.
\end{ccebody}

\cceth{El Arca de Noé prefigura la Iglesia y el Bautismo} 

\cceref{CEC 845, 1094, 1219}

\begin{ccebody}
\n{845} El Padre quiso convocar a toda la humanidad en la Iglesia de su Hijo para reunir de nuevo a todos sus hijos que el pecado había dispersado y extraviado. La Iglesia es el lugar donde la humanidad debe volver a encontrar su unidad y su salvación. Ella es el \textquote{mundo reconciliado} (San Agustín, \textit{Sermo} 96, 7-9). Es, además, este barco que \textit{pleno dominicae crucis velo Sancti Spiritus flatu in hoc bene navigat mundo} – \textquote{con su velamen que es la cruz de Cristo, empujado por el Espíritu Santo, navega bien en este mundo} (San Ambrosio, \textit{De virginitate} 18, 119); según otra imagen estimada por los Padres de la Iglesia, está prefigurada por el Arca de Noé que es la única que salva del diluvio (cf. \textit{1 Pe} 3, 20-21).

\n{1094} Sobre esta armonía de los dos Testamentos (cf. DV 14-16) se articula la catequesis pascual del Señor (cf. \textit{Lc} 24,13-49), y luego la de los Apóstoles y de los Padres de la Iglesia. Esta catequesis pone de manifiesto lo que permanecía oculto bajo la letra del Antiguo Testamento: el misterio de Cristo. Es llamada catequesis \textquote{tipológica}, porque revela la novedad de Cristo a partir de \textquote{figuras} (tipos) que lo anunciaban en los hechos, las palabras y los símbolos de la primera Alianza. Por esta relectura en el Espíritu de Verdad a partir de Cristo, las figuras son explicadas (cf. \textit{2 Co} 3, 14-16). Así, el diluvio y el arca de Noé prefiguraban la salvación por el Bautismo (cf. \textit{1 Pe} 3, 21), y lo mismo la nube, y el paso del mar Rojo; el agua de la roca era la figura de los dones espirituales de Cristo (cf. \textit{1 Co} 10,1-6); el maná del desierto prefiguraba la Eucaristía \textquote{el verdadero Pan del Cielo} (\textit{Jn} 6,32).

\n{1219} La Iglesia ha visto en el arca de Noé una prefiguración de la salvación por el bautismo. En efecto, por medio de ella \textquote{unos pocos, es decir, ocho personas, fueron salvados a través del agua} (\textit{1 Pe} 3,20):

\ccecite{\textquote{¡Oh Dios!, que incluso en las aguas torrenciales del diluvio prefiguraste el nacimiento de la nueva humanidad, de modo que una misma agua pusiera fin al pecado y diera origen a la santidad} (\textit{Vigilia Pascual, Bendición del agua: Misal Romano}).}
\end{ccebody}

\cceth{Alianza y sacramentos (especialmente el Bautismo)} 

\cceref{CEC 1116, 1129, 1222}

\begin{ccebody}
\n{1116} Los sacramentos, como \textquote{fuerzas que brotan} del Cuerpo de Cristo (cf. \textit{Lc} 5,17; 6,19; 8,46) siempre vivo y vivificante, y como acciones del Espíritu Santo que actúa en su Cuerpo que es la Iglesia, son \textquote{las obras maestras de Dios} en la nueva y eterna Alianza.

\n{1129} La Iglesia afirma que para los creyentes los sacramentos de la Nueva Alianza son \textit{necesarios para la salvación} (cf. Concilio de Trento: DS 1604). La \textquote{gracia sacramental} es la gracia del Espíritu Santo dada por Cristo y propia de cada sacramento. El Espíritu cura y transforma a los que lo reciben conformándolos con el Hijo de Dios. El fruto de la vida sacramental consiste en que el Espíritu de adopción deifica (cf. \textit{2 Pe} 1,4) a los fieles uniéndolos vitalmente al Hijo único, el Salvador.

\n{1222} Finalmente, el Bautismo es prefigurado en el paso del Jordán, por el que el pueblo de Dios recibe el don de la tierra prometida a la descendencia de Abraham, imagen de la vida eterna. La promesa de esta herencia bienaventurada se cumple en la nueva Alianza.
\end{ccebody}

\cceth{Dios nos salva por medio del Bautismo} 

\cceref{CEC 1257, 1811}

\begin{ccebody}
\ccesec{La necesidad del Bautismo}

\n{1257} El Señor mismo afirma que el Bautismo es necesario para la salvación (cf. \textit{Jn} 3,5). Por ello mandó a sus discípulos a anunciar el Evangelio y bautizar a todas las naciones (cf. \textit{Mt} 28, 19-20; cf. DS 1618; LG 14; AG 5). El Bautismo es necesario para la salvación en aquellos a los que el Evangelio ha sido anunciado y han tenido la posibilidad de pedir este sacramento (cf. \textit{Mc} 16,16). La Iglesia no conoce otro medio que el Bautismo para asegurar la entrada en la bienaventuranza eterna; por eso está obligada a no descuidar la misión que ha recibido del Señor de hacer \textquote{renacer del agua y del Espíritu} a todos los que pueden ser bautizados. \textit{Dios ha vinculado la salvación al sacramento del Bautismo, sin embargo, Él no queda sometido a sus sacramentos}.

\n{1811} Para el hombre herido por el pecado no es fácil guardar el equilibrio moral. El don de la salvación por Cristo nos otorga la gracia necesaria para perseverar en la búsqueda de las virtudes. Cada cual debe pedir siempre esta gracia de luz y de fortaleza, recurrir a los sacramentos, cooperar con el Espíritu Santo, seguir sus invitaciones a amar el bien y guardarse del mal.
\end{ccebody}
	%\chapter{Domingo II de Cuaresma (B)}

\section{Lecturas}

\rtitle{PRIMERA LECTURA}

\rbook{Del libro del Génesis} \rred{22, 1-2. 9a. 10-13. 15-18}

\rtheme{El sacrificio de Abrahán, nuestro padre en la fe}

\begin{scripture}
En aquellos días, Dios puso a prueba a Abrahán.

Le dijo:

\>{¡Abrahán!}.

Él respondió:

\>{Aquí estoy}.

Dios dijo:

\>{Toma a tu hijo único, al que amas, a Isaac, y vete a la tierra de Moria y ofrécemelo allí en holocausto en uno de los montes que yo te indicaré}.

Cuando llegaron al sitio que le había dicho Dios, Abrahán levantó allí el altar y apiló la leña, luego ató a su hijo Isaac y lo puso sobre el altar, encima de la leña.

Entonces Abrahán alargó la mano y tomó el cuchillo para degollar a su hijo.

Pero el ángel del Señor le gritó desde el cielo:

\>{¡Abrahán, Abrahán!}.

Él contestó:

\>{Aquí estoy}.

El ángel le ordenó:

\>{No alargues la mano contra el muchacho ni le hagas nada. \\Ahora he comprobado que temes a Dios, porque no te has reservado a tu hijo, a tu único hijo}.


Abrahán levantó los ojos y vio un carnero enredado por los cuernos en la maleza. Se acercó, tomó el carnero y lo ofreció en holocausto en lugar de su hijo.

El ángel del Señor llamó a Abrahán por segunda vez desde el cielo y le dijo:

\>{Juro por mí mismo, oráculo del Señor: por haber hecho esto, por no haberte reservado tu hijo, tu hijo único, te colmaré de bendiciones y multiplicaré a tus descendientes como las estrellas del cielo y como la arena de la playa. Tus descendientes conquistarán las puertas de sus enemigos. Todas las naciones de la tierra se bendecirán con tu descendencia, porque has escuchado mi voz}.
\end{scripture}

\rtitle{SALMO RESPONSORIAL}

\rbook{Salmo} \rred{115, 10 y 15. 16-17. 18-19}

\rtheme{Caminaré en presencia del Señor en el país de los vivos}

\begin{psbody}
Tenía fe, aun cuando dije:
\textquote{¡Qué desgraciado soy!}.
Mucho le cuesta al Señor
la muerte de sus fieles. 

Señor, yo soy tu siervo,
siervo tuyo, hijo de tu esclava:
rompiste mis cadenas.

Te ofreceré un sacrificio de alabanza,
invocando tu nombre, Señor. 

Cumpliré al Señor mis votos
en presencia de todo el pueblo,
en el atrio de la casa del Señor,
en medio de ti, Jerusalén. 
\end{psbody}

\rtitle{SEGUNDA LECTURA}

\rbook{De la carta del apóstol san Pablo a los Romanos} \rred{8, 31b-34}

\rtheme{Dios no se reservó a su propio Hijo}

\begin{scripture}
Hermanos:

Si Dios está con nosotros, ¿quién estará contra nosotros?

El que no se reservó a su propio Hijo, sino que lo entregó por todos nosotros, ¿cómo no nos dará todo con él? ¿Quién acusará a los elegidos de Dios? Dios es el que justifica. ¿Quién condenará? ¿Acaso Cristo Jesús, que murió, más todavía, resucitó y está a la derecha de Dios y que además intercede por nosotros?
\end{scripture}

\rtitle{EVANGELIO}

\rbook{Del Santo Evangelio según san Marcos} \rred{9, 2-10}

\rtheme{Este es mi Hijo, el amado}

\begin{scripture}
En aquel tiempo, Jesús tomó consigo a Pedro, a Santiago y a Juan, sube aparte con ellos solos a un monte alto, y se transfiguró delante de ellos. Sus vestidos se volvieron de un blanco deslumbrador, como no puede dejarlos ningún batanero del mundo.

Se les aparecieron Elías y Moisés, conversando con Jesús.

Entonces Pedro tomó la palabra y dijo a Jesús:

\>{Maestro, ¡qué bueno es que estemos aquí! Vamos a hacer tres tiendas, una para ti, otra para Moisés y otra para Elías}.

No sabía qué decir, pues estaban asustados.

Se formó una nube que los cubrió y salió una voz de la nube:

\>{Este es mi Hijo, el amado; escuchadlo}.

De pronto, al mirar alrededor, no vieron a nadie más que a Jesús, solo con ellos.

Cuando bajaban del monte, les ordenó que no contasen a nadie lo que habían visto hasta que el Hijo del hombre resucitara de entre los muertos.

Esto se les quedó grabado y discutían qué quería decir aquello de resucitar de entre los muertos.
\end{scripture}

\begin{patercite}
En la Eucaristía Jesús nos da, bajo las especies del pan y del vino, su carne vivificada por el Espíritu Santo y vivificadora de nuestra carne con el fin de hacernos participar con todo nuestro ser, espíritu y cuerpo, en su resurrección y en su condición de gloria. A este respecto, san Ireneo de Lyon enseña: \textquote{Porque de la misma manera que el pan, que proviene de la tierra, después de recibir la invocación de Dios, ya no es un pan ordinario, sino la Eucaristía, constituida de dos cosas: una celeste, otra terrestre, así nuestros cuerpos, al recibir la Eucaristía ya no son corruptibles, puesto que tienen la esperanza de la resurrección} (\textit{Adversus haereses}, IV, 18, 4-5).


(\ldots) El credo cristiano (...) culmina en la proclamación de la resurrección de los muertos al fin de los tiempos, y en la vida eterna» (Catecismo de la Iglesia católica, n. 988). Con la encarnación el Verbo de Dios asumió la carne humana (cf. Jn 1, 14), haciéndola partícipe, por su muerte y resurrección, de su misma gloria de Unigénito del Padre. Mediante los dones del Espíritu y de la carne de Cristo glorificada en la Eucaristía, Dios Padre infunde en todo el ser del hombre y, en cierto modo, en el cosmos mismo el deseo de ese destino.

\textbf{San Juan Pablo II, papa}, \textit{Catequesis}, Audiencia general, 4 de noviembre de 1998, cf. nn. 4-5. 
\end{patercite}

\newsection
\section{Comentarios Patrísticos}

\subsection{San Cirilo de Alejandría, obispo}

\ptheme{Hablaban de la muerte que Jesús iba a consumar en Jerusalén}

\src{ Homilía 9 en la transfiguración del Señor: \\PG 77, 1011-1014.}

\begin{body}
\ltr{J}{esús} subió a una montaña con sus tres discípulos preferidos. Allí se transfiguró en un resplandor tan extraordinario y divino, que su vestido parecía hecho de luz. Se les aparecieron también Moisés y Elías conversando con Jesús: hablaban de su muerte, que iba a consumar en Jerusalén, o sea, del misterio de aquella salvación que había de operarse mediante su cuerpo, de aquella pasión –repito– que habría de consumarse en la cruz. Pues la verdad es que la ley de Moisés y los vaticinios de los santos profetas preanunciaron el misterio de Cristo: las losas de la ley lo describían como en imagen y veladamente; los profetas, en cambio, lo predicaron en distintas ocasiones y de muchas maneras, diciendo que en el momento oportuno aparecería en forma humana y aceptaría morir en la cruz por la salvación y la vida de todos.

Y el hecho de que estuviesen allí presentes Moisés y Elías conversando con Jesús, quería indicar que la ley y los profetas son como los dos aliados de nuestro Señor Jesucristo, presentado por ellos como Dios a través de las cosas que habían preanunciado y que concordaban entre sí. En efecto, no disuenan de la ley los vaticinios de los profetas: y, a mi modo de ver, de esto hablaban Moisés y Elías, el más grande de los profetas.

Habiéndose aparecido, no se mantuvieron en silencio, sino que hablaban de la gloria que el mismo Jesús iba a consumar en Jerusalén, a saber, de la pasión y de la cruz y, en ellas, vislumbraban también la resurrección. Pensando quizá el bienaventurado Pedro que había llegado el tiempo del reinado de Dios, gustoso se quedaría a vivir en la montaña; de hecho, y sin saber lo que decía, propone la construcción de tres tiendas. Pero aún no había llegado el fin de los tiempos, ni en la presente vida entrarán los santos a participar de la esperanza a ellos prometida. Dice, en efecto, Pablo: \textit{El trasformará nuestra condición humilde, según el modelo de su condición gloriosa}, es decir, de la condición gloriosa de Cristo.

Ahora bien, estando estos planes todavía en sus comienzos, sin haber llegado aún a su culminación, sería una incongruencia que Cristo, que por amor había venido al mundo, abandonase el proyecto de padecer voluntariamente por él. Conservó, pues, aquella naturaleza infraceleste, con la que padeció la muerte según la carne y la borró por su resurrección de entre los muertos.

Por lo demás y al margen de este admirable y arcano espectáculo de la gloria de Cristo, ocurrió además otro hecho útil y necesario para consolidar la fe en Cristo, no sólo de los discípulos, sino también de nosotros mismos. Allí, en lo alto, resonó efectivamente la voz del Padre que decía: \textit{Este es mi Hijo, el amado, mi predilecto. Escuchadlo}.
\end{body}

\begin{patercite}(\ldots) Éste es el misterio, saludable para nosotros, que ahora se ha cumplido en la montaña, ya que ahora nos reúne la muerte y, al mismo tiempo, la festividad de Cristo. Por esto, para que podamos penetrar, junto con los elegidos entre los discípulos inspirados por Dios, el sentido profundo de estos inefables y sagrados misterios, escuchemos la voz divina y sagrada que nos llama con insistencia desde lo alto, desde la cumbre de la montaña. Debemos apresurarnos a ir hacia allí –así me atrevo a decirlo– como Jesús, que allí en el cielo es nuestro guía y precursor, con quien brillaremos con nuestra mirada espiritualizada, renovados en cierta manera en los trazos de nuestra alma, hechos conformes a su imagen, y, como él, transfigurados continuamente y hechos partícipes de la naturaleza divina, y dispuestos para los dones celestiales.

Corramos hacia allí, animosos y alegres, y penetremos en la intimidad de la nube, a imitación de Moisés y Elías, o de Santiago y Juan. Seamos como Pedro, arrebatado por la visión y aparición divina, transfigurado por aquella hermosa transfiguración, desasido del mundo, abstraído de la tierra; despojémonos de lo carnal, dejemos lo creado y volvámonos al Creador, al que Pedro, fuera de sí, dijo: \textit{Señor, ¡qué bien se está aquí!} Ciertamente, Pedro, en verdad qué bien se está aquí con Jesús; aquí nos quedaríamos para siempre. ¿Hay algo más dichoso, más elevado, más importante que estar con Dios, ser hechos conformes con él, vivir en la luz? Cada uno de nosotros, por el hecho de tener a Dios en sí y de ser transfigurado en su imagen divina, tiene derecho a exclamar con alegría: \textit{¡Qué bien se está aquí!} donde todo es resplandeciente, donde está el gozo, la felicidad y la alegría, donde el corazón disfruta de absoluta tranquilidad, serenidad y dulzura, donde vemos a (Cristo) Dios, donde él, junto con el Padre, pone su morada y dice, al entrar: \textit{Hoy ha sido la salvación de esta casa,} donde con Cristo se hallan acumulados los tesoros de los bienes eternos, donde hallamos reproducidas, como en un espejo, las imágenes de las realidades futuras.

\textbf{Anastasio Sinaíta}, \textit{Sermón} en el día de la Transfiguración del Señor, 6-10: \textquote{Mélanges d’archeologie et d’histoire} 67 [1955], 241-244 (Breviario, 6 de agosto).
\end{patercite}

\newsection
\subsection{San Juan Pablo II, papa}

\ptheme{Dios cumple la promesa entregando a su propio Hijo}

\src{Homilía durante las celebraciones en recuerdo de \\Abraham \textquote{Padre de todos los creyentes} \\23 de febrero del 2000.}

\begin{body}
\ltr[1. «] Yo soy el Señor que te saqué de Ur de los caldeos, para darte esta tierra en propiedad. (\ldots) Aquel día firmó el Señor una alianza con Abram, diciendo: “A tu descendencia he dado esta tierra, desde el río de Egipto hasta el gran río, el río Éufrates”» (\textit{Gn} 15, 7. 18).

Antes de que Moisés oyera en el monte Sinaí las conocidas palabras de Yahveh: \textquote{Yo soy el Señor, tu Dios, que te he sacado del país de Egipto, de la situación de esclavitud} (\textit{Ex} 20, 2), el patriarca Abraham ya había escuchado estas otras palabras: \textquote{Yo soy el Señor que te saqué de Ur de los caldeos}. Por consiguiente, debemos dirigirnos con el pensamiento hacia ese lugar tan importante en la historia del pueblo de Dios, para buscar en él \textit{los inicios de la alianza de Dios con el hombre}. Precisamente por ello, en este año del gran jubileo, mientras con el corazón nos remontamos hasta los orígenes de la alianza de Dios con la humanidad, \textit{nuestra mirada se vuelve hacia Abraham}, hacia el lugar donde escuchó la llamada de Dios y respondió a ella con la obediencia de la fe. Juntamente con nosotros, también los judíos y los musulmanes contemplan la figura de Abraham como un modelo de sumisión incondicional a la voluntad de Dios (cf. \textit{Nostra aetate}, 3).

El autor de la carta a los Hebreos escribe: \textquote{Por la fe, Abraham, al ser llamado por Dios, obedeció y salió para el lugar que había de recibir en herencia, y salió sin saber a dónde iba} (\textit{Hb} 11, 8). Abraham, a quien el Apóstol llama \textquote{nuestro Padre en la fe} (cf. \textit{Rm} 4, 11-16), creyó en Dios, \textit{se fió de él}, que lo llamaba. \textit{Creyó en la promesa}. Dios dijo a Abraham: \textquote{Sal de tu tierra, y de tu patria, y de la casa de tu padre, a la tierra que yo te mostraré. De ti haré una nación grande y te bendeciré. Engrandeceré tu nombre; y serás tú una bendición. (\ldots) Por ti serán bendecidos todos los linajes de la tierra} (\textit{Gn} 12, 1-3). ¿Estamos, acaso, hablando de la ruta de una de las múltiples emigraciones típicas de una época en la que la ganadería era una forma fundamental de vida económica? Es probable. Pero, con toda seguridad, \textit{no sólo se trató de esto.} En la historia de Abraham, con el que comenzó la historia de la salvación, ya podemos percibir otro significado de la llamada y de la promesa. La tierra hacia la que se encamina el hombre guiado por la voz de Dios \textit{no pertenece exclusivamente a la geografía de este mundo}. Abraham, el creyente que acoge la invitación de Dios, es el que se pone en camino hacia una tierra prometida que no es de aquí abajo.

2. En la carta a los Hebreos leemos: \textquote{Por la fe, Abraham, sometido a la prueba, presentó a Isaac como ofrenda, y el que había recibido las promesas, ofrecía a su unigénito, respecto del cual se le había dicho: Por Isaac tendrás descendencia} (\textit{Hb} 11, 17-18). \textit{He aquí el culmen de la fe de Abraham}. Fue puesto a prueba por el Dios en quien había depositado su confianza, por el Dios del que había recibido la promesa relativa al futuro lejano: \textquote{Por Isaac tendrás descendencia} (\textit{Hb} 11, 18). Pero es invitado a ofrecer en sacrifico a Dios precisamente a ese Isaac, su único hijo, a quien estaba vinculada toda su esperanza, de acuerdo con la promesa divina. ¿Cómo podrá cumplirse la promesa que Dios le hizo de una descendencia numerosa si Isaac, su único hijo, debe ser ofrecido en sacrificio?

Por la fe, Abraham sale victorioso de esta prueba, una prueba dramática, que comprometía directamente su fe. En efecto, como escribe el autor de la carta a los Hebreos, \textquote{pensaba que Dios era poderoso aun para resucitarlo de entre los muertos} (\textit{Hb} 11, 19). Incluso en el instante, humanamente trágico, en que estaba a punto de infligir el golpe mortal a su hijo, Abraham no dejó de creer. Más aún, su fe en la promesa alcanzó entonces su culmen. Pensaba: \textquote{Dios es poderoso aun para resucitarlo de entre los muertos}. Eso pensaba este padre probado, humanamente hablando, por encima de toda medida. Y su fe, su abandono total en Dios, no lo defraudó. Está escrito: \textquote{Por eso lo recobró} (\textit{Hb} 11, 19). Recobró a Isaac, puesto que creyó en Dios plenamente y de forma incondicional.

El autor de la carta a los Hebreos parece expresar aquí algo más: toda la experiencia de Abraham le resulta \textit{una analogía del evento salvífico de la muerte y la resurrección de Cristo}. Este hombre, que está en el origen de nuestra fe, forma parte del eterno designio divino. Según una tradición, el lugar donde Abraham estuvo a punto de sacrificar a su propio hijo es el mismo sobre el que otro padre, el Padre eterno, aceptaría la ofrenda de su Hijo unigénito, Jesucristo. Así, el sacrificio de Abraham se presenta como anuncio profético del sacrificio de Cristo. \textquote{Porque tanto amó Dios al mundo –escribe san Juan– que le dio a su Hijo unigénito} (\textit{Jn} 3, 16). En cierto sentido, el patriarca Abraham, nuestro padre en la fe, sin saberlo, introduce a todos los creyentes en el plan eterno de Dios, en el que se realiza la redención del mundo.

3. Un día Cristo afirmó: \textquote{En verdad, en verdad os digo: antes de que Abraham existiera, Yo Soy} (\textit{Jn} 8, 58) y estas palabras despertaron el asombro de los oyentes, que objetaron: \textquote{¿Aún no tienes cincuenta años y has visto a Abraham?} (\textit{Jn} 8, 57). Los que reaccionaban así razonaban de modo puramente humano, y por eso no aceptaron lo que Cristo les decía. \textquote{¿Eres tú acaso más grande que nuestro padre Abraham, que murió? También los profetas murieron. ¿Por quién te tienes a ti mismo?} (\textit{Jn} 8, 53). Jesús les replicó: \textquote{Vuestro padre Abraham se regocijó pensando en ver mi día; lo vio y se alegró} (\textit{Jn} 8, 56). La vocación de Abraham se presenta completamente orientada hacia el día del que habla Cristo. Aquí no valen los cálculos humanos; \textit{es preciso aplicar el metro de Dios}. Sólo entonces podemos comprender el significado exacto de la obediencia de Abraham, que \textquote{creyó, esperando contra toda esperanza} (\textit{Rm} 4, 18). Esperó que se iba a convertir en padre de numerosas naciones, y hoy seguramente se alegra con nosotros porque la promesa de Dios se cumple a lo largo de los siglos, de generación en generación.

El hecho de haber creído, esperando contra toda esperanza, \textquote{le fue reputado como justicia} (\textit{Rm} 4, 22), no sólo en consideración a él, sino también a todos nosotros, sus descendientes en la fe. Nosotros \textquote{creemos en aquel que resucitó de entre los muertos a Jesús, Señor nuestro} (\textit{Rm} 4, 24), que murió por nuestros pecados y resucitó para nuestra justificación (cf. \textit{Rm} 4, 25). Esto no lo sabía Abraham; sin embargo, por la obediencia de la fe, se dirigía hacia el cumplimiento de todas las promesas divinas, impulsado por la esperanza de que se realizarían. Y ¿existe promesa más grande que la que se cumplió en el misterio pascual de Cristo? Realmente, en la fe de Abraham Dios todopoderoso selló una alianza eterna con el género humano, y Jesucristo es el cumplimiento definitivo de esa alianza. El Hijo unigénito del Padre, de su misma naturaleza, se hizo hombre para introducirnos, mediante la humillación de la cruz y la gloria de la resurrección, en la tierra de salvación que Dios, rico en misericordia, prometió a la humanidad desde el inicio.

4. El modelo insuperable del pueblo redimido, en camino hacia el cumplimiento de esta promesa universal, es María, \textquote{la que creyó que se cumplirían las cosas que le fueron dichas de parte del Señor} (\textit{Lc} 1, 45).

María, hija de Abraham por la fe, además de serlo por la carne, compartió personalmente su experiencia. También ella, como Abraham, aceptó la inmolación de su Hijo, pero mientras que a Abraham no se le pidió el sacrificio efectivo de Isaac, Cristo bebió el cáliz del sufrimiento hasta la última gota. Y María participó personalmente en la prueba de su Hijo, creyendo y esperando de pie junto a la cruz (cf. \textit{Jn} 19, 25).

Era el epílogo de una larga espera. María, formada en la meditación de las páginas proféticas, presagiaba lo que le esperaba y, al alabar la misericordia de Dios, fiel a su pueblo de generación en generación, expresó su adhesión personal al plan divino de salvación; y, en particular, dio su \textquote{sí} al acontecimiento central de aquel plan, el sacrificio del Niño que llevaba en su seno. Como Abraham, aceptó el sacrificio de su Hijo.

Hoy nosotros unimos nuestra voz a la suya, y con ella, la Virgen Hija de Sión, proclamamos que Dios se acordó de su misericordia, \textquote{como lo había prometido a nuestros padres, en favor de Abraham y su descendencia por siempre} (\textit{Lc} 1, 55).
\end{body}



\newsection
\section{Homilías}

\subsection{San Pablo VI, papa}

\subsubsection{Homilía (1967): ¿Conoces realmente a Jesús?}

\src{19 de febrero de 1967.}

\begin{body}
\homsec{El acontecimiento luminoso del Tabor}

[\ldots]

\ltr{L}{os} heraldos del Evangelio, los obispos y, en primer lugar, el Papa, tienen la obligación de anunciar y difundir la palabra de Dios, de explicarla y comentarla.

Repasemos juntos, con espíritu atento, el pasaje de San Mateo\anote{id8} que nos acaba de presentar la Liturgia. Es la historia de la Transfiguración del Señor. Una página de la historia de Cristo, entre las más bellas, espléndidas y misteriosas.

Jesús, de noche, en un monte, al aire libre, quizás durante la primavera, con tres de sus discípulos: Pedro, Juan y Santiago. Mientras estos, cansados de la subida, se detienen a descansar en la hierba, Jesús se aleja un poco para atender la oración, como hacía siempre durante la noche: \textquote{Erat pernoctans in oratione Dei}, nos recuerda San Lucas.

En la oscuridad profunda, en cierto punto, los tres durmientes son despertados por un deslumbrante destello de luz. Y aquí, asombrados, ven a Jesús –San Marcos da algunos detalles– brillando como el sol, mientras su ropa es blanca como la nieve.

Sol y nieve. Es la fiesta de la luz. En ese triunfo los discípulos ven a dos excelentes figuras del Antiguo Testamento, Moisés y Elías, conversando con Jesús.

San Pedro no puede resistir la alegría y el entusiasmo. Luego de exclamar: \textquote{¡Qué bueno es estar aquí !}, propone levantar tres tiendas para una estadía permanente de los tres Personajes.

Pero, al mismo tiempo, los tres Apóstoles ven una nube blanca que se forma para envolver todo el cuadro beatífico: y desde la nube escuchan una voz poderosa que exclama: \textquote{Este es mi Hijo amado, escúchadlo}.

Pedro, Juan y Santiago están aterrorizados y ya no se atreven a mirar hacia arriba. Unos momentos después se sienten conmovidos. Él es todavía y siempre Jesús, pero desprovisto del prodigioso esplendor de hace un momento; los invita a bajar de la montaña; y les prohíbe contar lo sucedido hasta que –otro motivo de asombro para los Apóstoles– el Hijo del Hombre (era el título que Jesús se dio a sí mismo) haya resucitado de entre los muertos.

\homsec{El pleno conocimiento de Jesús}

Podría escribirse un volumen para ilustrar este rasgo del Evangelio. Pero hoy el Santo Padre quiere proponer sólo algunos temas de importancia más inmediata.

¿Qué problema plantea el episodio de la Transfiguración? Se puede condensar en una pregunta que cada uno querrá hacerse: ¿realmente conoces a Jesús? Es decir, ¿tienes un conocimiento real, positivo y concreto de él?

¿Realmente podrías decir quién es? ¿Lo tienes presente en tu alma?

Existe el peligro, dada la debilidad de la naturaleza humana, de insistir en respuestas y títulos correctos, sí, pero no siempre completos. Un cristiano, sin embargo, debe saber responder más y mejor, yendo más allá de lo que resulta de un interés, de una noticia superficial.

Mientras tanto: esta misma pregunta recorre toda la historia evangélica, de principio a fin.

\homsec{Sabiduría, bodad y amor de Cristo}

¿Quién es Jesús? preguntan sus contemporáneos. Hay varias respuestas: el hijo de María, el hijo del carpintero, un profeta, el Mesías. Esta diversidad de nombres persiste: incluso sobre ella se construye un proceso: la Pasión de Jesús. En la noche terrible, después de ser capturado en Getsemaní, Caifás, el Sumo Sacerdote, pregunta a Cristo si es el Hijo de Dios. Jesús responde: \textquote{Sí lo soy}. Más tarde es Pilato quien le pregunta si es Rey: idéntica respuesta afirmativa. De ahí la condenación, por la cual, en la Cruz, se coloca el signo con la motivación de la sentencia: \textquote{Jesús Nazareno, Rey de los Judíos}.

Después de hechos tan excepcionales y terribles, es lógico que los fieles se pregunten si conocen a Jesús.

Para facilitar la respuesta, pensemos en dos tipos de argumentos. El primero proviene del mismo Jesús. Contemplando cómo se presenta y se revela a sí mismo, debe notarse una especie de gradualidad. El Salvador del mundo se nos aparece en la pobreza, en la humildad, quitando a su alrededor todo aparato, toda pompa y todo signo de su Divinidad. Quería comenzar su vida terrena, en secreto, introduciéndose en la humanidad sin acontecimientos extraordinarios; y vivió durante muchos años como un trabajador pobre. No puede haber una humildad más profunda. Y quien no acepte esta presentación se escandalizará y no comprenderá el resto de la vida y la revelación de Cristo. Parecería, por tanto, que no quiere hacer notar su presencia. Esto explica por qué pasan tantos a su lado y no sienten su llamada.

Ahora bien, esta revelación humana, sensible, caracterizada por la pobreza no está sola. Jesús se presentó ante todos, pero a algunos, a los que se le acercaron y le siguieron, otorgó otras manifestaciones de sí mismo: la sabiduría, su palabra maravillosa. Por ejemplo, quedaron impresionados los enviados de los enemigos del Divino Maestro, que un día quisieron tenderle una trampa. Les da vergüenza oírle hablar. En otra ocasión una mujer, después de haberlo escuchado, alza la voz entre la multitud exclamando; \textquote{Bienaventurada la que te generó, porque nunca nadie ha hablado tan bien como Tú enseñas}.

Junto a la revelación de la sabiduría, la del poder: los milagros. Hay muchos, asombrosos: todos los tenemos presentes. Ciertamente, ningún hombre podría hacer tales maravillas.

En tercer lugar, y en un grado aún mayor: Jesús se revela a sí mismo. Está en la bondad. Quien se acerca tiene la emoción y el encanto de esta bondad incomparable. \textquote{Venid a mí todos los que estáis cansados; y yo os restauraré}. Y el perdón a los pecadores, el amor a los niños, a los pobres, a los que sufren. Todos, ahora y siempre, pueden hacer el experimento de pasar junto a Jesús y captar su luz penetrante, en el perfecto conocimiento de las almas. \textquote{Sciebat quid esset in homine}. Sabía lo que había dentro de los corazones y derramaba en ellos su bondad.

Finalmente, –mientras más se estrecha el cerco de los que conocen la suprema aparición celestial–, Jesús también se revela como realmente es. He aquí la Transfiguración. En él palpita no solo una vida humana, sino la vida divina. \textquote{Este es mi Hijo amado}. Es el Hijo de Dios hecho hombre. Precisamente este aspecto se volverá, al parecer, normal después de la muerte y resurrección del Señor. ¿Habéis conocido alguna vez al Señor así?

\homsec{Abrir el alma a la fe y a la gracia}

Ahora tenemos que examinar un segundo orden de elementos que condicionan nuestro conocimiento de Jesús, el cual depende de nuestra disposición: la de abrir los ojos, el corazón, el alma. Si acudimos a él con el corazón cerrado, con los ojos cerrados, con prejuicios e incredulidad preestablecida, Él no se mostrará. La luz pasará cerca de nosotros y nos quedaremos ciegos, indiferentes.

Por tanto, debemos abrir los ojos. Todo el mundo tiene que hacerlo. El Redentor no vino para una categoría específica, por ejemplo, para los sabios. Se mostró al mundo, a toda la humanidad: y ésta estaría, de por sí, en grado de captar los rayos de su rostro divino. La realidad nos muestra en cambio que, lamentablemente, \textquote{no omnes oboediunt Evangelio}: no todos obedecieron al Evangelio, como dice San Pablo. Algunos miran y no ven: siguen siendo extraños y débiles ante la Revelación.

Por eso es necesario abrir la mente al conocimiento de Jesús, y esta invitación explícita no parece exagerada, ya que nunca poseemos suficiente conocimiento de Jesús. Siempre somos ignorantes, porque lo que podemos aprender de Jesús es tan grande e infinito que nuestras pobres facultades, aunque fuéramos teólogos consumados, deberían ser consideradas mezquinas e insuficientes.

Entonces, ¿qué debemos hacer?

Primero, educarnos a nosotros mismos; atesorar la palabra del Señor difundida en la predicación sagrada, en la catequesis, en libros adecuados.

\homsec{La Transfiguración final}

Jesús no se reveló tanto por la vía de los ojos, sino por la escucha que debemos prestarle. El Evangelio nos lo recuerda: \textquote{Ipsum audite}: debes escucharlo a Él. Y de nuevo \textquote{Fides ex auditu}: la fe, es decir, el conocimiento misterioso de Jesús, lo tendremos solamente si recibimos la gracia de escucharlo.

En consecuencia, no solo debemos ser buenos oyentes, sino deseosos de aprender, porque la palabra de Jesús es el mismo Jesús, es la Palabra de Dios, que viene de manera intencional, misericordiosa, muy amplia, a nuestras almas, para que allí se reciba su palabra y se convierta en la norma de nuestra vida.

Lo segundo es amar a Jesús, quien lo ama lo conocerá de la manera más válida. Él mismo lo afirmó: \textquote{Qui diligit me, diligetur a Patre meo; et ego diligam eum et manifestabo ei meipsum} – Si alguno me ama, me abriré a él, me daré a conocer a él. Son las experiencias espirituales las que a menudo tienen una certeza mucho mayor que los silogismos de nuestro razonamiento. Este regalo es ofrecido a todas las almas; los que realmente quieran estar con Cristo podrán poseerlo.

He aquí el voto del Papa, queridísimos hijos y aquí no estamos tanto en el anuncio como en el deseo: que un día todos veamos a nuestro Salvador en su plenitud de vida, en su humanidad, que es igual a la nuestra, en su Divinidad que le viene del Padre. En Él veremos al Dios vivo. Ese encuentro bendito, esa transfiguración final, será nuestra gloria y felicidad eterna: nuestro Paraíso. ¡Amén!
\end{body}


\newsection
\subsection{San Juan Pablo II, papa}

\subsubsection{Homilía (1979): Dios está con nosotros}

\src{Visita Pastoral a la Parroquia Romana de San Basilio. \\11 de marzo de 1979.}

\begin{body}
 \ltr[\ldots]{E}{ste} es un encuentro en la fe, cuyo contenido nos precisa la Palabra de Dios en la liturgia de hoy. Contenido fuerte, profundo y esencial. Escuchando la \textbf{Carta de San Pablo a los Romanos}, encontramos inmediatamente la realidad-clave de la fe. \textquote{Si Dios está por nosotros, ¿quién contra nosotros? El que no perdonó a su propio Hijo, antes lo entregó por todos nosotros, ¿cómo no nos ha de dar con Él todas las cosas? ¿Quién acusará a los elegidos de Dios? Siendo Dios quien justifica. ¿quién condenará? ¿Acaso Cristo Jesús, el que murió, aún más, el que resucitó? Él, que está a la diestra de Dios, y que intercede por nosotros} (\textit{Rm} 8, 31-34). ¡Dios está con nosotros! ¡Dios con el hombre! Con la humanidad. La prueba única y completa de esto es y permanece siempre ésta: \textquote{no perdonó a su propio Hijo, antes le entregó por todos nosotros} (\textit{Rm} 8, 32).

Para poner más de relieve aún esta verdad, la liturgia hace referencia en el libro del \textbf{Génesis}, al sacrificio de Isaac. Cuando Dios pidió a Abraham esta ofrenda, quería preparar en cierto modo la conciencia del pueblo elegido para el sacrificio que después realizaría su Hijo. Dios perdonó a Isaac y perdonó también el corazón de su padre Abraham. ¡Pero no ha perdonado al propio Hijo! Abraham fue \textquote{padre de nuestra fe}, porque con la disposición al sacrificio de su hijo Isaac, preanunció el sacrificio de Cristo, que constituye un momento cumbre en los caminos de la fe de toda la humanidad. Todos somos conscientes de ello. Esta conciencia vivifica nuestras almas, particularmente durante la Cuaresma. Esta conciencia plasma nuestra vida cristiana desde las raíces más profundas. La plasma desde el principio al fin.

Dios está con nosotros a través de la cruz de su Hijo. Y ésta es también la fuente primera de nuestra fuerza espiritual. Cuando el Apóstol pregunta: \textquote{Si Dios está por nosotros, ¿quién contra nosotros?}, con esta pregunta abraza a todo y a todos los que puedan ser un peligro para nuestro espíritu, para nuestra salvación. \textquote{¿Quién condenará? Cristo Jesús, el que murió, aún más, el que resucitó, el que está sentado a la diestra de Dios, es quien intercede por nosotros} (\textit{Rm} 8, 34). De la fe en Cristo, en su cruz y resurrección, nace la esperanza. ¡Gran confianza! Sea ésta nuestra fuerza, particularmente en los momentos difíciles de la vida.

Mi pensamiento y mi palabra se dirigen de modo especial a todos los que se encuentran en dificultades de diverso género: a quienes sufren en el cuerpo y en el espíritu; a quienes sufren pruebas de carácter social, como experiencias negativas en el trabajo, o malentendidos de familia: a los jóvenes que acaso están pasando un momento de crisis: a quienes afrontan con tesón dificultades de naturaleza pastoral, como la incomprensión o la tibieza ante los valores espirituales y la resistencia al Espíritu Santo en Cristo. Todos tienen derecho a esperar.

En el \textbf{Evangelio} de hoy encontramos una manifestación especial de la esperanza que nace de la fe en Jesucristo. Precisamente en el tiempo de Cuaresma la Iglesia nos lee de nuevo el Evangelio de la Transfiguración del Señor. En efecto, este acontecimiento tuvo lugar a fin de preparar a los Apóstoles a las pruebas difíciles de Getsemaní, de la pasión, de la humillación de la flagelación, de la coronación de espinas, del vía crucis, del Calvario. En esta perspectiva Jesús quería demostrar a sus Apóstoles más íntimos el esplendor de la gloria que refulge en El, la que el Padre le confirma con la voz de lo alto, revelando su filiación divina y su misión: \textquote{Este es mi Hijo amado, en quien tengo mi complacencia: escuchadle} (\textit{Mt} 17, 5).

El esplendor de la gloria de la Transfiguración abraza casi toda la Antigua Alianza y llega a los ojos llenos de estupor de los Apóstoles, que se convertirían en maestros de esa fe que hace nacer la esperanza: de aquellos Apóstoles que deberían anunciar todo el misterio de Cristo. \textquote{Señor, ¡qué bien estamos aquí!} (\textit{Mt} 17, 4), exclaman Pedro, Santiago y Juan, como si quisieran decir: ¡Tú eres la encarnación de la esperanza que anhelan el alma y el cuerpo humanos! ¡Esperanza que es más fuerte que la cruz y que el Calvario! Esperanza que disipa las tinieblas de nuestra existencia, del pecado y de la muerte.

¡Qué bien estamos aquí: contigo!

Sea vuestra parroquia, y cada vez lo sea más, el lugar, la comunidad donde los hombres, profundizando por medio de la fe en el misterio de Cristo, adquieran más confianza, más conciencia del valor y del sentido de la vida, y repitan a Cristo: \textquote{¡Qué bien estamos aquí!}: contigo. Aquí, en este templo. Ante este tabernáculo. Y no sólo aquí, sino acaso en una cama de hospital; acaso en los puestos de trabajo; a la mesa en la comunidad de la familia. En todas partes.

[...]
\end{body}

\newpage 

\subsubsection{Homilía (1982): Tres montañas}

\src{Visita pastoral a la parroquia romana \\de la Inmaculada Concepción alla Cervelletta. \\7 de marzo de 1982.}

\begin{body}
\ltr[1. ]{L}{a} liturgia del segundo domingo de Cuaresma es, en cierto sentido, la liturgia de las tres montañas. En la primera escuchamos, como relata el libro del \textbf{Génesis}, las palabras que Dios dirigió a Abraham: \textquote{Toma a tu hijo, tu único hijo a quien amas, Isaac, ve al territorio de Moria y ofrécelo en holocausto en el monte que Yo te indicaré} (\textit{Gn} 22, 2).

La prueba de Abraham. \textquote{Dios puso a prueba a Abraham} (\textit{Gn} 22, 1). Esta fue la prueba de su fe. En el lugar indicado, Abraham construyó el altar, colocó la madera sobre él y colocó a su hijo Isaac sobre la madera: el hijo unigénito. El hijo de la promesa. El hijo de la esperanza. Abraham estaba listo para ofrecerlo como holocausto a Dios, para derramar su sangre y quemar su cuerpo en la hoguera.

En el momento decisivo recibió la prohibición de Dios: \textquote{¡No extiendas tu mano contra el muchacho y no le hagas daño! ahora sé que temes a Dios porque no te has reservado a tu hijo, tu único hijo} (\textit{Gn} 22, 12). En la zarza cercana, Abraham encontró un carnero y lo ofreció sobre el altar preparado. La prueba de fe se cumplió. La gran prueba. La dura prueba. Adecuada a la gran promesa. Dios renovó su promesa ante Abraham, después de haberlo sometido a la prueba: \textquote{Haré muy numerosa tu descendencia, como las estrellas del cielo y como la arena que está a la orilla del mar} (\textit{Gn} 22, 17). La descendencia no tanto según la carne como según el espíritu. Los descendientes de Abraham en la fe son, en cierto sentido, seguidores de las tres grandes religiones monoteístas del mundo: el judaísmo, el cristianismo, el islam. \textquote{Todas las naciones de la tierra serán bendecidas por tu descendencia, porque obedeciste a mi voz} (\textit{Gn} 22, 18).

Los descendientes de la fe de Abraham creen que Dios tiene el poder de probar al hombre. Tiene derecho a ofrecer de su espíritu.

2. La liturgia del segundo domingo de Cuaresma nos lleva a otra montaña, a Galilea. Más allá de la llanura de Galilea, se eleva majestuoso el \textbf{monte Tabor}: el monte de la Transfiguración, según la tradición cristiana. En este monte, Jesús de Nazaret, que vino entre los descendientes de Abraham como el Mesías enviado por Dios, fue transformado milagrosamente ante los ojos de sus Apóstoles: Pedro, Santiago y Juan. A los ojos de los Apóstoles, se manifestó transfigurado en gloria y, junto a él, Moisés y Elías. Al milagro del oído se añadió el milagro de la visión. Oyeron la voz que salió de la nube: \textquote{Este es mi Hijo amado: escuchadlo} (\textit{Mc} 9, 7); las mismas palabras que ya había oído Juan el Bautista cerca del Jordán, con motivo de la primera venida de Jesucristo, después de su bautismo. La teofanía del monte Tabor tiene un carácter pascual. Anuncia la gloria de Cristo resucitado. Al mismo tiempo, prepara a los Apóstoles para la muerte del Cordero de Dios, para la Teofanía del Gólgota.

3. El apóstol Pablo nos lleva al monte Gólgota, el tercer monte, con las palabras de la \textbf{carta a los Romanos}. La Teofanía del Gólgota se indica con las siguientes palabras: \textquote{Si Dios está por nosotros, ¿quién estará contra nosotros? El que no se reservó ni a su propio Hijo, sino que lo entregó por todos nosotros} (\textit{Rm} 8, 31-32). Sabemos que el Padre entregó a su Hijo en el Gólgota; sabemos que este es el nombre de esa colina fuera de los muros de Jerusalén, en la cual Dios \textquote{no se reservó ni a su propio Hijo} (\textit{Rm} 8, 32). Y a través de esto, demostró \textquote{estar con nosotros hasta el final}. \textquote{¿Cómo no nos dará todo junto con él?}, pregunta el Apóstol (\textit{Rm} 8, 32).

Este Dios, que no permitió que Abraham sacrificara a su hijo Isaac en la muerte, no perdonó a su propio Hijo. ¿Acaso haciendo esto no confirmó nuestra elección hasta el final? \textquote{¿Quién acusará a los elegidos de Dios?}, pregunta el Apóstol (\textit{Rm} 8, 33). Él mismo tomó la causa de la justificación del hombre en sus propias manos\ldots \textquote{Es Dios quien justifica} (\textit{Rm} 8, 33). Y si es así, ¿quién puede condenar al hombre? (cf. \textit{Rm} 8, 34). Una sentencia así solo podía ser dictada por Cristo, quien en el Gólgota conocía el peso de los pecados de los hombres. Pero en el Gólgota Jesucristo sufrió la muerte por nosotros \textquote{en verdad –escribe el Apóstol– \ldots resucitó, está a la diestra de Dios e intercede por nosotros} (\textit{Rm} 8, 34).

4. La liturgia dominical de hoy nos invita a subir a una montaña, lugar de la teofanía de la antigua y la nueva alianza. En estos montes estamos invitados, según el espíritu de la Cuaresma, a meditar sobre las grandes obras de Dios (cf. \textit{Hch} 2, 11): los misterios de nuestra redención, de nuestra justificación en Cristo.

En estos montes nos conviene aprender estos misterios, asimilarlos con el corazón y el alma, moldear nuestro espíritu, transformarlo según el aspecto que Cristo le da. Este domingo de Cuaresma nos enseña que estamos llamados a una gran transformación espiritual. Debemos participar en la Transfiguración de Cristo, así como sus discípulos en el monte Tabor. Tenemos que prepararnos para la Santa Pascua. El maestro de esta actitud, a través de la cual Cristo desciende a nuestro corazón, realizando una transformación y conversión, es Abraham: el Padre de todos los creyentes.

5. De hecho, las palabras del \textbf{salmista} parecen resonar en nuestro corazón: \textquote{Tenía fe aun cuando dije: ¡Qué desgraciado soy!} (\textit{Sal} 115 [116], 10). ¿Acaso no se sentía Abraham igualmente \textquote{desgraciado} cuando fue a la montaña indicada por Dios para sacrificar a su propio hijo? ¿Acaso no fue solo la fe lo que le permitió repetir: \textquote{Mucho le cuesta al Señor la muerte de sus fieles} (\textit{Sal} 115 [116], 15)? De Abraham la familia humana comienza a aprender la fe, que se manifiesta en la actitud interior del espíritu humano: se manifiesta en el sacrificio del corazón Jesucristo es el Maestro definitivo y perfecto de tal actitud: \textquote{consumator fidei nostrae} (cf. \textit{Hb} 12, 2).

6. El fruto de la liturgia del segundo domingo de Cuaresma debe ser la disposición a ofrecer los sacrificios espirituales en los que se manifiesta nuestra fe. Es lo que pedimos en las palabras del \textbf{Salmo}: \textquote{Señor, yo soy tu siervo, siervo tuyo, hijo de tu esclava; rompiste mis cadenas. Te ofreceré un sacrificio de alabanza invocando el nombre del Señor. Cumpliré al Señor mis votos en presencia de todo el pueblo} (\textit{Sal} 115 [116], 16-18). 

7.  \txtsmall{[Con este espíritu, vosotros, parroquianos de la parroquia de la \textquote{Inmaculada Concepción en la Cervelletta} de Tor Sapienza, os habéis reunido hoy con vuestro Obispo\ldots]}

8. Queridos hermanos y hermanas, (\ldots) junto con vosotros hoy hice una visita al monte de la fe de Abraham, al monte de la Transfiguración en Galilea y al monte Gólgota. Siguiendo el espíritu de la liturgia de Cuaresma hemos conocido la grandeza de nuestra Redención y de la Justificación en el Sacrificio de Cristo.

Que nuestra fe madure en el mismo espíritu: a través de las obras de todas las horas, a través de las pruebas de la vida diaria y, a veces, a través de las grandes pruebas y experiencias, en las que el espíritu humano es probado como oro acrisolado al fuego.

Para nosotros, redimidos y justificados en la Sangre de Cristo, ninguna prueba o experiencia cierra la perspectiva de la vida. Al contrario, la revelan aún más profundamente en Dios.

Aprendemos esta perspectiva, ofreciendo los sacrificios espirituales de todo lo que compone nuestra vida.

Que la participación en la Eucaristía nos una –cada vez, y particularmente hoy– en esta comunidad, a la que el Padre revela y entrega a su Hijo: \textquote{Este es mi Hijo amado, escúchadlo} (\textit{Mc} 9, 7).

Amén.
\end{body}

\img{holy_trinity}

\label{b2-03-02-1982H}
\newpage 


\subsubsection{Homilía (1985): Comprender la gravedad del mal}

\src{Celebración eucarística en la parroquia romana de San Tarcisio al IV Miglio. \\3 de marzo de 1985.}

\begin{body}
\textquote{[Dios], que no escatimó ni a su propio Hijo, sino que lo entregó por todos nosotros\ldots} (\textit{Rm} 8, 32).

\ltr[1. ]{E}{l} período de Cuaresma, más que cualquier otro, pone ante los ojos de nuestra fe y de nuestra conciencia esta verdad, esta imagen de Dios, que da a su Hijo en sacrificio por los pecados del hombre. Sobre la cruz. En la muerte. No lo perdona, sino que lo da. Dios, en quien la justicia y la misericordia se encuentran y se compenetran de una manera maravillosa. Es estrictamente justo ante el pecado. Es infinitamente misericordioso ante los pecadores. Por tanto, \textquote{no perdona} al Hijo. Y el Hijo \textquote{no se reserva} a sí mismo. Se entrega en sacrificio como \textquote{víctima divina} de la justicia y la misericordia.

2. La liturgia de la Iglesia en Cuaresma está dirigida a ese Dios, Padre e Hijo, en particular el domingo de hoy. Esto ya lo demuestra la \textbf{primera lectura} del libro del \textbf{Génesis} donde –en el sacrificio de Abraham– encontramos una \textquote{prefiguración}, es decir, una figura y un anticipo, en cierto sentido un signo remoto, de ese misterio inescrutable de la cruz. Este sacrificio de Abraham es sólo una prueba de fe para aquel a quien el apóstol llamó \textquote{padre de nuestra fe} (cf. \textit{Rm} 4, 11). Abraham, por la fe, llegó a tener un descendiente y heredero en Isaac. Y, sin embargo, a través de la fe, basada en una rigurosa obediencia a Dios, estaba dispuesto a ofrecer ese primogénito y único hijo como sacrificio a Dios.

Dentro de estos límites, Abraham, el padre, tiene cierto parecido con Dios-Padre, e Isaac, el hijo, es una imagen de Cristo-Hijo. Sin embargo, solo dentro de estos límites. Como parte de una prueba de obediencia y sinceridad de intención. De hecho, en última instancia, Dios no permite que Abraham sacrifique a Isaac. \textquote{No extiendas tu mano contra el muchacho –dice– ¡y no lo lastimes! Ahora sé que temes a Dios} (temer significa tener fe) \textquote{porque no te has reservado a tu hijo, a tu único hijo} (\textit{Gn} 22, 12). Y Abraham ofrece un cordero como sacrificio en lugar de su hijo.

3. En cambio, a su propio Hijo \textquote{Dios no lo perdonó, sino que lo entregó por todos nosotros}. Así lo anuncia \textbf{san Pablo} escribiendo a los \textbf{Romanos}. Y en este contexto plantea una serie de cuestiones fundamentales. En primer lugar: \textquote{Si Dios está por nosotros, ¿quién estará contra nosotros?} (\textit{Rm} 8, 31). Y, al dar a su propio Hijo, Dios revela que está con nosotros. Nos revela que está dispuesto a perdonarnos todo: si nos ofrece a su Hijo en holocausto, \textquote{¿cómo no nos dará todo junto con él?} (\textit{Rm} 8, 32).

Dios, Padre del Hijo crucificado, es Dios \textquote{rico en misericordia} (\textit{Ef} 2, 4). Y al mismo tiempo un Dios sumamente justo, que se encargó personalmente del problema de la justificación del hombre, del hombre pecador.

Y por eso \textbf{el apóstol} pregunta: \textquote{¿Quién acusará a los elegidos de Dios? Es Dios quien justifica} (\textit{Rm} 8, 33). Sabemos que si Él mismo justifica, significa que no quiere acusar. Quiere salvar. No quiere condenar. \textquote{¿Quién condenará?} pregunta el apóstol. \textquote{¿Acaso Cristo Jesús, que murió, más aún, que resucitó, está a la diestra de Dios e intercede por nosotros?} (\textit{Rm} 8, 34).

4. La liturgia de Cuaresma contiene en sí misma una llamada radical para cada uno de nosotros. ¡Meditemos a fondo el problema del pecado! ¡Reflexionemos a fondo sobre el problema de la culpa del hombre ante Dios! Este problema ha sido algo empañado y degradado en la conciencia contemporánea. La Cuaresma es el momento de una conversión particular. Convertirse significa descubrir la malicia del pecado. Redescubrirla en la propia conciencia. Poner en marcha todos los criterios humanos para este fin. Pero los criterios humanos no son suficientes aquí. La maldad del pecado se revela en su plenitud sólo cuando lo pensamos a la luz del misterio del Padre, \textquote{que no escatimó ni a su propio Hijo}. Sólo entonces comprendemos la profundidad del mal, cuando la necesidad de la justificación del pecado por Dios mismo se vuelve clara para nosotros. ¡Sólo entonces nos acercamos a la cruz de Cristo, para que se demuestre la infinitud del amor misericordioso, que cumple toda medida de justicia y de juicio.

5. La liturgia de Cuaresma contiene esta invitación en sí misma. La encíclica \textit{Dives in misericordia}, que se puede leer y meditar como comentario sobre la liturgia de Cuaresma, corresponde a esta invitación.

Este período nos introduce gradualmente en el corazón mismo del misterio pascual. Por eso también el \textbf{Evangelio} dominical de hoy presenta la Transfiguración de Cristo en el monte Tabor.

El Dios de Abraham no aceptó el sacrificio de la vida de Isaac. El Dios y Padre de Jesucristo, en cambio, acoge el sacrificio de la vida de su Hijo. El Padre y el Hijo, en este sacrificio, comienzan a cumplir la Justificación del hombre.

Para preparar a los apóstoles para la horrible muerte de Cristo en la cruz, Dios les permite saborear, casi como un anticipo, la gloria de su resurrección en la Transfiguración en el monte Tabor. Allí, desde el centro de la nube luminosa, se oye la voz del Padre (como después del Bautismo en el Jordán): \textquote{Este es mi Hijo amado: escuchadlo} (\textit{Mc} 9, 7). La muerte en la cruz será una prueba terrible y el despojo del Hijo de Dios, pero al mismo tiempo se convertirá en el comienzo de una nueva vida. Cristo retornará en la gloria del Padre.


6. \txtsmall{[He aquí los rasgos principales de la liturgia cuaresmal, que quise meditar con vosotros, queridos hermanos y hermanas de la parroquia de San Tarcisio.

A todos los aquí presentes (\ldots) A todos ustedes, queridos fieles, mis pensamientos afectuosos y de bendición. A vuestras familias, jóvenes, ancianos, niños, enfermos, todos aquellos que trabajan con buena voluntad en el territorio parroquial por una convivencia humana y civil más justa y serena, para alejar cualquier forma de peligro u ofensa a la dignidad de las personas y el bien común. ¡Que se consolide cada vez más un diálogo fructífero de resultados concretos en el contexto de estas necesidades esenciales de promoción humana!]}

7. La comunidad parroquial está dotada por Dios de fuerzas sobrenaturales que le permiten actuar como levadura en todo el ámbito de su territorio para una continua elevación de la vida moral del entorno, a pesar de las dificultades.

Sé cómo vuestra comunidad se centra mucho en los valores de la liturgia, la catequesis, la evangelización. ¡Muy bien! Os animo a continuar, con un compromiso renovado. Y saber esperar los frutos con paciencia. El \textbf{Evangelio}, y en particular la liturgia de hoy, nos enseñan que de alguna manera es necesario \textquote{morir} para dar vida. Siguiendo el ejemplo de nuestro Señor, imitemos su sacrificio, con la certeza de que los resultados llegarán.

El sacrificio evangélico por los hermanos no nos empobrece; al contrario, nos hace crecer. No nos quita nuestra dignidad o nuestros intereses genuinos; al contrario, nos despoja del \textquote{hombre viejo} y fortalece al \textquote{hombre nuevo} en nosotros. No tengamos miedo de seguir el ejemplo de Nuestro Señor, de Nuestra Señora y de los santos en esto, y nuestra acción será extraordinariamente fructífera.

 \txtsmall{[Entre los santos tenéis como modelo, de manera especial, a san Tarcisio, a quien está dedicada vuestra parroquia. Dio su vida por la santísima Eucaristía, porque sabía que el pan eucarístico es fuente de vida.]} Que el alimento divino que el Padre nos ofrece en Cristo sustente también vuestra acción al servicio del Señor y por el bien de los hermanos.

8. ¡Queridos hermanos y hermanas!

Deseo ardientemente que  \txtsmall{[esta visita a vuestra parroquia y]} la meditación común sobre el misterio de la muerte salvadora del Hijo de Dios despierten en vosotros una vida de profunda fe.

\textquote{Si Dios está con nosotros, ¿quién estará contra nosotros?}.

Y Dios está con nosotros. De hecho, \textquote{no se reservó ni a su propio Hijo}. Y nosotros, ¿respondemos a eso? ¿Estamos con Dios en lo más profundo de nuestros pensamientos, nuestras obras y nuestras conciencias?

¿Estamos con Dios como él \textquote{pide}? ¿Él, \textquote{que dio a su propio Hijo por todos nosotros}?

¿Estamos con Dios?
\end{body}



\label{b2-03-02-1985H}
\newpage 


\subsubsection{Homilía (1988): La Cruz y la Resurrección}

\src{Visita pastoral a la parroquia romana \\de la Resurrección de Nuestro Señor Jesucristo en Torre Nova. \\28 de febrero de 1988.}

\begin{body}
1. \textquote{Este es mi Hijo amado; ¡escuchadlol!} (\textit{Mc} 9, 7).

\homsec{La teofanía del Monte de la Transfiguración.} 

\ltr{L}{as} palabras que llegan a los oídos de los apóstoles Pedro, Santiago y Juan deslumbrados por la visión, son las palabras del Padre. En ellas se revela a sí mismo y a su Hijo. Así fue también con ocasión del bautismo de Jesús en el río Jordán. Ahora la situación es diferente y el tiempo es diferente.

Entonces Juan el Bautista había indicado \textquote{el cordero de Dios\ldots que quita el pecado del mundo} (\textit{Jn} 1, 29). Ahora los caminos de Jesús se han acercado al tiempo en que los pecados del mundo deben ser realmente quitados por medio de Él. Este será el momento del despojo, el momento de la elevación en el monte Gólgota, el momento que, desde el punto de vista humano, constituye la más profunda humillación: la \textquote{kénosis} de la cruz.

Cuando terminó la Transfiguración, Jesús ordenó a los apóstoles que no se lo mencionaran a nadie, \textquote{hasta que el Hijo del Hombre resucitara de entre los muertos} (\textit{Mc} 9, 9). Sin embargo, se preguntaban \textquote{qué significaba resucitar de entre los muertos} (\textit{Mc} 9, 10). Ni sabían ni previeron que esto tenía que hacerse al precio de la cruz y de la muerte.

2. La liturgia del segundo domingo de Cuaresma nos prepara para el misterio de la cruz de Cristo en el Gólgota, para los acontecimientos pascuales y, en primer lugar, nos conduce al Monte de la Transfiguración, de la misma manera que Cristo preparó a sus apóstoles. Y no solamente Cristo, sino también al Padre Celestial. La liturgia del domingo de hoy se perfila para nosotros en su conjunto como centrada en el misterio del Padre, el mismo Padre que \textquote{tanto amó al mundo que le dio a su Hijo unigénito} (\textit{Jn} 3, 16).

En el sacrificio de Abraham descrito en el libro del \textbf{Génesis}, encontramos la figura que presagia este misterio del Padre y del Hijo. Incluso en ese hecho hay un cerro ubicado en el territorio de Moria, sobre el cual Abraham sube con su hijo Isaac: el único hijo de la promesa.

Dios pidió el sacrificio de este hijo, tan esperado antes, y por tanto tan amado: de este hijo al que se unieron todas las esperanzas de Abraham. Sin embargo, cuando Dios pidió tal sacrificio, Abraham no dudó en hacerlo. Estaba dispuesto a sacrificar a su único hijo.

Cuando Abraham levantó su cuchillo para realizar el gesto de \textquote{sacrificar a su hijo} (\textit{Gn} 22, 10) –Dios a través del ángel– tomó su mano. Aceptó el sacrificio del corazón y no permitió la inmolación de su hijo. \textquote{Ahora sé que temes a Dios y no te reservaste a tu hijo, tu único hijo} (\textit{Gn} 22, 12). Podríamos añadir: \textquote{Amas a Dios más que a tu hijo}.

3. Cuán cerca estamos, en este punto, del misterio del Padre celestial, de este Padre, \textquote{que no se reservó ni a su propio Hijo, sino que lo entregó por todos nosotros}, como escribe el apóstol Pablo en su \textbf{carta a los Romanos} (\textit{Rm} 8, 32). Y añade \textquote{¿cómo no nos va a dar todo junto con él?} (\textit{Rm} 8, 32). El sacrificio de la cruz es el sacrificio de la satisfacción y de la expiación. En él está contenida la redención y la remisión de los pecados. El Apóstol penetra desde muchos puntos de vista en el significado y los frutos de este sacrificio cuando insiste en preguntar: \textquote{¿Quién acusará a los elegidos de Dios? Es Dios quien justifica. ¿Quién condenará? ¿Acaso Cristo Jesús, que murió, o más bien, que resucitó, está a la diestra de Dios e intercede por nosotros?} (\textit{Rm} 8, 33-34).

4. Se puede decir que estas preguntas, tan características de Pablo, nos introducen en toda la perspectiva del misterio pascual. El Padre celestial no perdonó a su Hijo unigénito, no lo salvó de la muerte en la cruz, en la que consistía el sacrificio. Pero ese sacrificio, que el Hijo sufrió voluntariamente, se convirtió en la fuente de nuestra justificación. Fuimos comprados a un alto precio (cf. \textit{1 Cor} 7, 23). Al participar en la redención mediante la cruz de Cristo, también hemos sido llamados a participar en la nueva vida revelada mediante la resurrección.

Se puede decir que aquí se abren verdaderas profundidades, \textquote{profundidades insondables} ante nuestro espíritu humano. La Cuaresma es el tiempo de la valentía espiritual, de adentrarse en estas profundidades de las que emerge la verdad definitiva sobre Dios y el hombre. La verdad que verdaderamente nos libera. Todo cristiano y toda comunidad cristiana está llamada a ello en el período actual. Vuestra parroquia también está llamada a ello \txtsmall{[de una manera particular con motivo de la visita de hoy del Obispo de Roma].}

5. Sin dudarlo, queridos hermanos y hermanas, responded a esta llamada, haciendo sin dudarlo, según el espíritu de Cuaresma, ese éxodo espiritual del pecado y del egoísmo, para progresar en la fe, que es escucha y obediencia, libre asentimiento y abandono confiado. Cuando el cristiano sigue la verdad, alcanza la sabiduría del corazón y, humilde y arrepentido, se abre al Redentor, recibiendo su consoladora bendición junto con el perdón.

[\ldots]

6. [\ldots] Rezo para que la luz de Cristo transfigurado ilumine vuestras mentes y corazones para llevaros a vivir con él el misterio pascual de su muerte y resurrección, os exhorto a ser siempre auténticos testigos del Evangelio, de la plenitud de la ley y del cumplimiento de las Profecías.

Por último, dirijo mi cordial saludo a todos los hermanos y hermanas de la parroquia de la Resurrección, y a cada uno de vosotros, queridos míos, deseo recordaros el deber de comprometerse apostólicamente para que, después de haber aceptado la palabra que es Espíritu y vida, hagáis presente, en los lugares donde el Señor os ha colocado, su plan de salvación. Y esto ocurrirá, si os apoyáis en la absolución de los pecados en el confesionario y os alimentáis del pan eucarístico del altar, para ser cada vez más efectivamente piedras vivas de ese edificio espiritual, que es la casa del Padre y la morada de todo hombre.

\txtsmall{[\ldots]}

7. Finalmente, volvamos a la meditación de las palabras del \textbf{salmista}, escuchadas en la liturgia de hoy: \textquote{Tenía fe aun cuando dije: ¡qué desgraciado soy!} (\textit{Sal} 116 [115], 10). ¿No habla el salmista aquí también de la fe de Abraham, el cual \textquote{creyó esperando contra toda esperanza} (\textit{Rm} 4, 18) y así se convirtió en padre de todos los creyentes (cf. \textit{Rm} 4, 11, 16)?

Abraham dice: \textquote{Cumpliré al Señor mis votos} (\textit{Sal} 116 [115], 14). Cumpliré\ldots \textquote{Señor, yo soy tu siervo} (\textit{Sal} 116 [115], 16). Abraham, el siervo del Dios de la Alianza. Abraham, el amigo de Dios, Abraham, la imagen del Padre, que \textquote{tanto amó al mundo que le entregó a su Hijo unigénito} (\textit{Jn} 3, 16). No tanto a Isaac, sino a Cristo. \textquote{Si Dios está con nosotros, ¿quién estará contra nosotros?} (cf. \textit{Rm} 8, 31). Abraham, figura del Padre. De este Padre que, en el Monte de la Transfiguración se revela en la voz: \textquote{Este es mi Hijo amado, escuchadlo} (\textit{Mc} 9, 7).

Con estas palabras prepara a los Apóstoles para el misterio que estará contenido en los acontecimientos de la Pascua en Jerusalén. \textquote{¡Escuchadlo a él!}. Aquí está la obediencia de la fe de todos los discípulos, –los cuales, por esta virtud conforman la descendencia de Abraham– basada de generación en generación en la elocuencia de la cruz y de la resurrección, en la que el Hijo ha revelado plenamente el amor del Padre.
\end{body}

\img{cross_tau}

\label{b2-03-02-1988H}

\newpage 
\subsubsection{Homilía (1991): Cruz que transforma}

\src{Visita pastoral a la parroquia romana de la Santísima Trinidad. \\Domingo 24 de febrero de 1991.}

\begin{body}

\ltr[1. «]{J}{esús} tomó consigo a Pedro, Santiago y Juan y los llevó a un monte alto, a un lugar apartado\ldots y se transfiguró delante de ellos» (\textit{Mc} 9, 2).

Queridos hermanos y hermanas, en el itinerario penitencial de la Cuaresma, la liturgia de hoy nos invita a hacer una pausa para contemplar la divina Transfiguración de Cristo.

Este es un hecho clave no solo en la experiencia terrena de Jesús, el Siervo obediente y sufriente que va a Jerusalén para cumplir, con su sacrificio pascual, la misión que le ha confiado el Padre, sino también para la experiencia de fe de los discípulos, que caminan con él hacia la misma meta, y de toda la comunidad de creyentes que \textquote{entre las persecuciones del mundo y los consuelos de Dios prosigue su peregrinaje terrenal} (San Agustín, \textit{De Civitate Dei}, XVIII, 51, 2) hacia la eterna Pascua.

2. Jesús, por tanto, va camino de Jerusalén, donde tendrá que \textquote{sufrir mucho, ser reprobado por los ancianos, los sumos sacerdotes y los escribas, luego ser matado y, a los tres días, resucitar} (\textit{Mc} 8, 31). Allí, de hecho, se cumplirán las antiguas profecías que predijeron la venida del Mesías, no como un gobernante poderoso o agitador político, sino como un humilde y manso Siervo de Dios y de los hombres, que tendrá que dar su vida en sacrificio, pasando por el camino de la persecución, el sufrimiento y la muerte.

Jesús tiene ante sí una meta difícil hacia la que la voluntad de Dios lo empuja y su vocación de \textquote{Siervo} lo dirige, y al mismo tiempo predice su trágico y glorioso epílogo. Su humanidad, para pasar la prueba, debe ser \textquote{confirmada} por el amor poderoso del Padre y consolada por la solidaridad de los discípulos que caminan con él. Y así les introduce en la comprensión de lo que está por realizarse, para que se hagan sus \textquote{compañeros} en el camino que tendrá que seguir hasta el final. En efecto, ellos parecen dispuestos a seguirlo con palabras, pero, ante los hechos, se retiran temerosos y escandalizados.

Hay una pausa en este camino hacia la Cruz. Jesús, con sus más fieles discípulos, sube al monte y allí, por un momento, les hace vislumbrar su destino último: la gloriosa Resurrección. Pero también les anticipa que primero es necesario seguirlo por el camino de la Pasión y la Cruz.

3. Queridos hermanos y hermanas (\ldots), sentiros también desafiados por el acontecimiento de la Transfiguración y por la invitación divina que os urge a seguir a Cristo. La \textquote{palabra de la Cruz} debe transformar no sólo a vosotros, sino a toda la Iglesia (\ldots), que vive el tiempo favorable de la Cuaresma, como momento fuerte de ese camino de fe y renovación  \txtsmall{[que quiere ser el Sínodo pastoral diocesano].}

Es muy importante que el itinerario espiritual y pastoral caracterice de forma indeleble la existencia de la fe personal y de toda la comunidad eclesial. Lo que hay que poner en primer lugar y que hay que tener siempre presente, para no perderse y no tomar el camino equivocado, es la escucha de Dios. Sólo pasando \textquote{por la muerte podemos alcanzar el triunfo de la resurrección} (cf. \textit{Prefacio}). Escuchar es lo que define al discípulo y lo convierte en servidor de la verdad y del amor de Dios, manifestado en plenitud en Cristo Jesús: \textquote{Presta atención y ven a mí –te dice por medio del profeta– y vivirás. Estableceré contigo una alianza eterna} (\textit{Is} 55, 3).

Sin duda el camino es difícil; pide disponibilidad, coraje, renuncia, para poder hacer de la propia vida, como Cristo hizo de la suya, un \textquote{don} de amor al Padre y a los hermanos. Sólo así podremos hacernos aptos, por la fuerza del Espíritu, para anunciar el \textquote{Evangelio de la Cruz} y realizar esa \textquote{nueva evangelización} que tiene su centro y eje en Cristo crucificado y resucitado. El anuncio, del que los discípulos son portadores, es exigente, difícil de comprender y sobre todo de acoger y vivir. Pero no están solos: están en comunión unos con otros y con Cristo, que murió y resucitó y ahora está glorioso a la diestra del Padre que intercede por ellos (cf. \textit{Rm} 8, 34b). Esta certeza que se basa en la fe, mientras nos consuela en medio de las dificultades, ¡nos empuja, como hijos de Abraham, a esperar contra toda esperanza!

4. Precisamente para que esta esperanza no disminuya, sino que crezca día a día, en el camino de la escucha y el anuncio de la \textquote{palabra de la Cruz}, es indispensable de vez en cuando, queridos hermanos y hermanas, subir al monte con Jesús y estar con él: es decir, estar más atentos a la voz de Dios y dejarse envolver y transformar por el Espíritu. En una palabra: ¡la experiencia de la contemplación y la oración es necesaria! \textquote{La oración, de hecho, es un bien supremo. Es\ldots una comunión íntima con Dios. Así como los ojos del cuerpo al ver la luz están iluminados por él, así también el alma que se extiende hacia Dios es iluminada por la luz inefable de la oración} (San Juan Crisóstomo, \textit{Homilía} 6, \textquote{\textit{De Oratione}}).

Esto no hemos de hacerlo, ciertamente, para escapar de la dureza de la vida cotidiana y huir de los gravosos compromisos del servicio al hombre, sino para disfrutar de la familiaridad con Dios, para luego reanudar con renovado vigor el fatigoso camino de la Cruz, que conduce a la resurrección.

5. Estos son los principales pensamientos que nos llegan de la liturgia de este segundo domingo de Cuaresma y que quería meditar con vosotros, queridos hermanos y hermanas (\ldots).

A todos los aquí presentes \ldots les dirijo mi cordial saludo y mis mejores deseos para todos\ldots [\ldots] Queridos amigos, os animo a continuar con renovado compromiso en este camino, por una convivencia humana y civil más justa, más serena y más fraterna y por una continua elevación de la vida espiritual y moral del entorno en el que están llamados a vivir y a dar testimonio de vuestra fe. Que el Señor Jesús, que hoy os invita a \textquote{transfiguraros}, os ayude a transformar y mejorar vuestra vida a la luz resplandeciente de su gracia.

\newpage 
6. Sí, queridos hermanos y hermanas, caminad juntos ante el Señor Dios y en fidelidad a Cristo, no solo en este tiempo de Cuaresma, sino a lo largo de vuestra vida.

Así, vuestro barrio y la ciudad de [Roma] se convertirán verdaderamente en \textquote{la tierra de los vivos}; es decir, la tierra en la que Dios habita y continúa revelándose en su Hijo y en la que florecen la verdad, la esperanza, el amor y la paz.

¡Amén!
\end{body}

\label{b2-03-02-1991H}
%\newpage 

\subsubsection{Homilía (1994): Amor que destruye el pecado}

\src{Visita pastoral a la parroquia romana de San Alejandro. \\27 de febrero de 1994.}

\begin{body}
\ltr{H}{emos} escuchado la Palabra de Dios, tratemos de resumir lo que hemos escuchado. Me vienen a la mente las palabras de San Juan: \textquote{Tanto amó Dios al mundo que dio a su Hijo} (\textit{Jn} 3, 16). Esto es lo que quiere decirnos la liturgia de hoy. Especialmente con la \textbf{primera lectura} que habla de Abraham. Abraham estaba dispuesto a dar a su único hijo. Fue una gran profecía.

No conocemos sus palabras, no ha escrito libros, pero este gesto de estar dispuesto a entregar a su único hijo, Isaac, en holocausto a Dios es ya una grandísima profecía que anticipa todo el misterio pascual. ¿Qué significa el evangelio de hoy para nosotros? Dios se prepara para dar a su único Hijo primogénito, Jesús hecho hombre, para darlo como sacrificio por todos los pecados del mundo.

Dios no permitió que Abraham ofreciera a su hijo Isaac, pero no renunció a dar a su único Hijo, Jesús, el cual se prepara para este holocausto de la Semana Santa, del Sagrado Triduo, y también prepara a sus Apóstoles. Para ello los lleva junto con él al monte Tabor, y allí el Padre les muestra cómo Jesús es su amado. \textquote{He aquí mi Hijo, mi Hijo amado} (cf. \textit{Mc} 9, 7). Lo manifiesta a los dos Testamentos: lo hace ante los profetas, lo hace ante Moisés, Elías, y evidentemente lo hace ante estos tres Apóstoles escogidos para ser testigos: Pedro, Santiago y Juan.

Jesús se apareció a sus Apóstoles transfigurado, elevado al cielo en su gloriosa figura. Se dice: \textquote{transfiguración}, se trata de una figura celeste después de la figura terrenal. La figura celestial de Jesús apareció precisamente en el monte Tabor. Los Apóstoles se asombran y dicen: \textquote{¡Qué bueno es que estemos aquí!}. Y Jesús les dice algunas palabras un tanto enigmáticas. Les dice que no le digan a nadie lo que vieron antes de la Resurrección.

Los apóstoles se preguntan qué significa la resurrección, qué significa ser resucitado. ¿Significa estar muerto primero? Jesús, con estas enigmáticas palabras, ya anunciaba la Semana Santa, el Viernes Santo, la Pascua.

Así la Iglesia hoy, con estas estupendas lecturas, nos prepara para la solemnidad pascual. Hace esto todos los años. Cuando era joven, me preguntaba por qué había estas lecturas el segundo domingo de Cuaresma, especialmente el pasaje evangélico de la Transfiguración. Hoy comprendo bien que esto se deba al misterio pascual y a la preparación pascual en Cuaresma.

Entonces, ¿qué dice San Pablo en la \textbf{segunda lectura}? San Pablo nos habla casi todos los domingos. Nos dice: ¿quién nos separará del amor de Dios manifestado en Cristo Jesús?

[\ldots]

Esto le da a San Pablo la oportunidad de preguntarse: ¿quién nos separará del amor de Cristo? \ldots El amor de Cristo es más fuerte. Por eso, queridos amigos, durante la Cuaresma debemos recordar siempre, cada año, renovar la conciencia de que el amor de Cristo es más fuerte que todo. Pablo se pregunta: ¿quién nos separará? ¿El pecado? El pecado no es nada delante de Él. Sí, es una falta, pesa en la conciencia del hombre, pero ante la Resurrección, especialmente la Pasión, la Cruz de Cristo, el amor de Cristo, el pecado es derrotado. Podemos eliminarlo, podemos superarlo, podemos pedir perdón.

Y este es el mensaje continuo de Cuaresma. Se repite todos los años, a todos y cada uno. Se repite con la fuerza de nuestros santos, apóstoles y mártires, testigos de la Transfiguración. Estamos llamados a transfigurarnos durante la Cuaresma, a ser semejantes al glorioso Jesús.

Nosotros también estamos llamados a la gloria, a participar de su gloria. Esto es lo que nos dice la bella liturgia de este domingo.

 \txtsmall{[Estoy muy feliz de estar aquí y traer este mensaje, en este maravilloso entorno, en esta basílica paleocristiana vinculada a la memoria de San Alejandro, y a estas catacumbas, que son todas testigos del heroísmo de la vida cristiana de tantos desconocidos como Alejandro. Heroísmo de la vida cristiana que fue posible para ellos y será posible también para nosotros porque la gracia de Dios es siempre más fuerte\ldots]}

 \txtsmall{[\ldots]}

Se perciben muchas fuerzas vivas trabajando por la Resurrección, por la Transfiguración del mundo, por un mundo mejor. Este trabajo es muy necesario. Hay que cambiar el mundo: no podemos permanecer en nuestra antigua forma, no transfigurarnos. El mundo necesita una transfiguración profunda, la que viene de Jesús.

¡Queridos amigos, estas son las pocas palabras que quería deciros en este encuentro que nos prepara para el Ofertorio y la Santísima Comunión Eucarística con Cristo!

¡Alabado sea Jesucristo!
\end{body}

\label{b2-03-02-1994H}
\newpage 

\subsubsection{Homilía (1997): Amor revelado en la Cruz}

\src{Visita Pastoral a la Parroquia Romana de la Santa Cruz.\\23 de febrero de 1997.}

\begin{body}
\textquote{Este es mi Hijo amado: escuchadlo} (\textit{Mc} 9, 7). 

\ltr{H}{oy,} en el marco de la transfiguración del Señor, volvemos a escuchar estas palabras, que resonaron en el momento del bautismo de Jesús en el Jordán (cf. \textit{Mt} 3, 17). \textquote{Jesús se llevó a Pedro, a Santiago y a Juan (\ldots), y se transfiguró delante de ellos (\ldots). Se les aparecieron Elías y Moisés conversando con Jesús (\ldots). Pedro (\ldots) le dijo a Jesús: \textquote{Maestro. ¡Qué bien se está aquí! Vamos a hacer tres chozas, una para ti, otra para Moisés y otra para Elías}} (\textit{Mc} 9, 2-5). En ese preciso instante se oyó una voz: \textquote{Este es mi Hijo amado; escuchadlo} (\textit{Mc} 9, 7).

No duró mucho esa extraordinaria manifestación de la filiación divina de Jesús. Cuando los Apóstoles alzaron nuevamente su mirada, no vieron más que a Jesús, el cual, \textquote{cuando bajaban de la montaña –prosigue el evangelista– (\ldots), les mandó: \textquote{No contéis a nadie lo que habéis visto hasta que el Hijo del hombre resucite de entre los muertos}} (\textit{Mc} 9, 9).

Así, en este segundo domingo de Cuaresma, escuchamos junto con los Apóstoles el anuncio de la Resurrección. Lo escuchamos mientras nos encaminamos con ellos hacia Jerusalén, donde reviviremos el misterio de la pasión y muerte del Señor. En efecto, el ayuno y la penitencia de este tiempo sagrado se orientan precisamente hacia este acontecimiento-clave de toda la economía salvífica.

La transfiguración del Señor, que según la tradición tuvo lugar en el monte Tabor, sitúa en primer plano la persona y la obra de Dios Padre, presente junto al Hijo de modo invisible pero real. Esto explica el hecho de que, en el trasfondo del evangelio de la Transfiguración, la liturgia de hoy sitúa un importante episodio del Antiguo Testamento, en el que se pone de relieve de modo particular la paternidad.

En efecto, la \textbf{primera lectura}, tomada del libro del \textbf{Génesis}, nos recuerda el sacrificio de Abraham. Éste tenía un hijo, Isaac, que había nacido en su vejez. Era el hijo de la promesa. Pero un día Abraham recibe de Dios la orden de ofrecerlo en sacrificio. El anciano patriarca se encuentra ante la perspectiva de un sacrificio que para él, padre, es seguramente el mayor que se pueda imaginar. A pesar de ello, no duda ni un instante y, después de haber preparado lo necesario, parte con Isaac hacia el lugar establecido. Construye un altar, coloca la leña y, una vez atado el muchacho, toma el cuchillo para inmolarlo. Sólo entonces lo detiene una orden de lo alto: \textquote{No alargues tu mano contra tu hijo, ni le hagas nada, que ahora ya sé que tú eres temeroso de Dios, ya que no me has negado tu hijo, tu único hijo} (\textit{Gn} 22, 12).

Es conmovedor este acontecimiento, en el que la fe y el abandono de un padre en las manos de Dios alcanzan la cima. Con razón san Pablo llama a Abraham \textquote{padre de todos los creyentes} (\textit{Rm} 4, 11. 17). Tanto la religión judía como la cristiana hacen referencia a su fe. El Corán destaca también la figura de Abraham. La fe del padre de los creyentes es un espejo en el que se refleja el misterio de Dios, misterio de amor que une al Padre y al Hijo.

[\ldots]

\textquote{El que no perdonó a su propio Hijo, sino que lo entregó a la muerte por nosotros, ¿cómo no nos dará todo con él?} (\textit{Rm} 8, 32). Estas palabras de \textbf{san Pablo} en la \textbf{carta a los Romanos} nos introducen en el tema fundamental de la liturgia de hoy: el misterio del amor divino revelado en el sacrificio de la cruz.

El sacrificio de Isaac anticipa el de Cristo: el Padre no escatimó a su propio Hijo, sino que lo entregó para la salvación del mundo. Él, que detuvo el brazo de Abraham en el momento en que estaba a punto de inmolar a Isaac, no dudó en sacrificar a su propio Hijo por nuestra redención. De ese modo, el sacrifico de Abraham pone de relieve que nunca y en ningún lugar se deben realizar sacrificios humanos, porque el único sacrificio verdadero y perfecto es el del Hijo unigénito y eterno de Dios vivo. Jesús, que por nosotros y por nuestra salvación nació de María virgen, se inmoló voluntariamente una vez para siempre, como víctima de expiación por nuestros pecados, obteniéndonos así la salvación total y definitiva (cf. \textit{Hb} 10, 5-10). Después del sacrifico del Hijo de Dios, no se necesita ninguna otra expiación humana, puesto que su sacrificio en la cruz abarca y supera todos los demás sacrificios que el hombre podría ofrecer a Dios. Aquí nos encontramos en el centro del misterio pascual.

Desde el Tabor, el monte de la transfiguración, el itinerario cuaresmal nos lleva hasta el Gólgota, el monte del sacrifico supremo del único sacerdote de la alianza nueva y eterna. Dicho sacrificio encierra la mayor fuerza de transformación del hombre y de la historia. Asumiendo en sí mismo todas las consecuencias del mal y del pecado, Jesús resucitará al tercer día y saldrá de esa dramática experiencia como vencedor de la muerte, del infierno y de Satanás. La Cuaresma nos prepara para participar personalmente en este gran misterio de la fe, que celebraremos en el triduo de la pasión, la muerte y la resurrección de Cristo.

Pidamos al Señor la gracia de prepararnos de modo conveniente: \textquote{Jesús, Hijo amado del Padre, haz que te escuchemos y te sigamos hasta el Calvario, hasta la cruz, para poder participar contigo en la gloria de la resurrección}. Amén.
\end{body}


\label{b2-03-02-1997A}
\newpage 


\subsubsection{Ángelus (1997): Subida al Tabor}

\src{23 de febrero de 1997.}

\begin{body}
\ltr{E}{n} este segundo domingo de Cuaresma, la liturgia nos presenta la Transfiguración en el monte Tabor. Es la revelación de la gloria, que precede la prueba suprema de la cruz y anticipa la victoria de la resurrección.

Pedro, Santiago y Juan fueron testigos de este evento extraordinario. El evangelio de hoy relata que Jesús los llamó aparte y los llevó consigo \textquote{a un monte alto} (\textit{Mc} 9, 2).

La subida de los discípulos al Tabor nos impulsa a reflexionar sobre el itinerario penitencial de estos días. También la Cuaresma es un camino de subida. Es una invitación a redescubrir el silencio pacificador y regenerador de la meditación. Se trata de un esfuerzo de purificación del corazón, para liberarlo del pecado que pesa sobre él. Ciertamente se trata de un camino arduo, pero que orienta hacia una meta rica en belleza, esplendor y alegría.

En la Transfiguración se oye la voz del Padre celestial: \textquote{Este es mi Hijo amado, escuchadlo} (\textit{Mc} 9, 7). Estas palabras encierran todo el programa de la Cuaresma: debemos ponernos a la escucha de Jesús. Él nos revela al Padre, porque, como Hijo eterno, es \textquote{imagen de Dios invisible} (\textit{Col} 1, 15). Pero, al mismo tiempo, como verdadero \textquote{Hijo del hombre}, revela lo que sabemos, revela el hombre al hombre (cf. \textit{Gaudium et spes}, 22). Por tanto, ¡no tengamos miedo a Cristo! Al elevarnos a la altura de su vida divina, no nos aleja de nuestra humanidad sino que, por el contrario, nos humaniza, dando sentido pleno a nuestra existencia personal y social. [A este redescubrimiento cada vez más vivo de Jesús también nos impulsa la perspectiva del gran jubileo, que en este primer año de preparación inmediata se centra principalmente en] la contemplación de Cristo: una contemplación que debe alimentarse del Evangelio y de la oración, y que siempre tiene que ir acompañada por una conversión auténtica y por el redescubrimiento constante de la caridad como ley de vida diaria.

Queridos hermanos, contemplemos a María, la Virgen a la escucha, siempre dispuesta a acoger y conservar en su corazón cada una de las palabras de su Hijo divino (cf. \textit{Lc} 2, 51). El evangelio la define \textquote{feliz porque ha creído que se cumplirían las cosas que le fueron dichas de parte del Señor} (\textit{Lc} 1, 45). La Madre celestial de Dios nos ayude a entrar en sintonía profunda con la palabra de Dios, para que Cristo se convierta en luz y guía de toda nuestra vida.
\end{body}

\label{b2-03-02-1997A}
\newpage 

\subsubsection{Homilía (2000): Paternidad por la fe}

\src{Celebración Eucarística en el Jubileo de los Artesanos. \\19 de marzo del 2000.}

\begin{body}
\ltr{D}{ios,} \textquote{que no perdonó a su propio Hijo, sino que lo entregó a la muerte por todos nosotros, ¿cómo no nos dará todo con él?} (\textit{Rm} 8, 32).

El apóstol Pablo, en la \textbf{carta a los Romanos}, formula esta pregunta, en la que destaca con claridad el tema central de la liturgia de este día: el misterio de la paternidad de Dios. En el pasaje evangélico es el mismo Padre eterno quien se presenta a nosotros cuando, desde la nube luminosa que envuelve a Jesús y a los Apóstoles en el monte de la Transfiguración, hace oír su voz, que exhorta: \textquote{Éste es mi Hijo amado, escuchadlo} (\textit{Mc} 9, 7). Pedro, Santiago y Juan intuyen –luego lo comprenderán mejor– que Dios les ha hablado revelándose a sí mismo y el misterio de su realidad más íntima.

Después de la resurrección, ellos, junto con los demás Apóstoles, llevarán al mundo este impresionante anuncio: en su Hijo encarnado Dios se ha acercado a todo hombre como Padre misericordioso. En Cristo todo ser humano es envuelto por el abrazo tierno y fuerte de un Padre. [\ldots] Cristo es el Hijo amado del Padre. Es, sobre todo, la palabra \textquote{amado} la que, respondiendo a nuestros interrogantes, descorre en cierto modo el velo que oculta el misterio de la paternidad divina. En efecto, nos da a conocer el amor infinito del Padre al Hijo y, al mismo tiempo, nos revela su \textquote{pasión} por el hombre, por cuya salvación no duda en entregar a este Hijo tan amado. Todo ser humano puede saber ya que en Jesús, Verbo encarnado, es objeto de un amor ilimitado por parte del Padre celestial.

Una contribución ulterior al conocimiento de este misterio nos la da la \textbf{primera lectura}, tomada del libro del \textbf{Génesis}. Dios pide a Abraham el sacrificio de su hijo: \textquote{Toma a tu hijo único, al que quieres, a Isaac, y vete al país de Moria y ofrécemelo en sacrificio, sobre uno de los montes que yo te indicaré} (\textit{Gn} 22, 2). Con el corazón destrozado, Abraham se dispone a cumplir la orden de Dios. Pero, cuando está a punto de clavar a su hijo el cuchillo del sacrificio, el Señor lo detiene y, por medio de un ángel, le dice: \textquote{No alargues la mano contra tu hijo ni le hagas nada. Ahora sé que temes a Dios, porque no te has reservado a tu hijo, tu único hijo} (\textit{Gn} 22, 12).

A través de las vicisitudes de una paternidad humana sometida a una prueba dramática, se revela otra paternidad, basada en la fe. Precisamente en virtud del extraordinario testimonio de fe dado en aquella circunstancia, Abraham obtiene la promesa de una descendencia numerosa: \textquote{Todos los pueblos del mundo se bendecirán con tu descendencia, porque me has obedecido} (\textit{Gn} 22, 18). Gracias a su fe incondicional en la palabra de Dios, Abraham se convierte en padre de todos los creyentes.

Dios Padre \textquote{no perdonó a su propio Hijo, sino que lo entregó a la muerte por nosotros} (\textit{Rm} 8, 32). Abraham, con su disponibilidad a inmolar a Isaac, anuncia el sacrificio de Cristo por la salvación del mundo. La ejecución efectiva del sacrificio, que le fue ahorrada a Abraham, se realizará con Jesucristo. Él mismo informa a los Apóstoles: al bajar del monte de la Transfiguración, les prohíbe que cuenten lo que han visto antes de que el Hijo del hombre resucite de entre los muertos. El evangelista añade: \textquote{Esto se les quedó grabado y discutían qué querría decir aquello de resucitar de entre los muertos} (\textit{Mc} 9, 10).

Los discípulos intuyen que Jesús es el Mesías y que en él se realiza la salvación. Pero no logran comprender por qué habla de pasión y de muerte: no aceptan que el amor de Dios pueda esconderse detrás de la cruz. Y, sin embargo, donde los hombres verán sólo una muerte, Dios manifestará su gloria, resucitando a su Hijo; donde los hombres pronunciarán palabras de condena, Dios realizará su misterio de salvación y amor al género humano.

Ésta es la lección que cada generación cristiana debe volver a aprender. Cada generación, ¡también la nuestra! Aquí radica la razón de ser de nuestro camino de conversión en este tiempo singular de gracia. [El jubileo ilumina toda la vida y la experiencia de los hombres.] Incluso la fatiga y el cansancio del trabajo diario reciben de la fe en Cristo muerto y resucitado una nueva luz de esperanza. Aparecen como elementos significativos del designio de salvación que el Padre celestial está realizando mediante la cruz de su Hijo.

\txtsmall{[Apoyados en esta certeza, queridos artesanos, podéis fortalecer y concretar los valores que desde siempre caracterizan vuestra actividad: el perfil cualitativo, el espíritu de iniciativa, la promoción de las capacidades artísticas, la libertad y la cooperación, la relación correcta entre tecnología y ambiente, el arraigo familiar y las buenas relaciones de vecindad. La civilización artesana ha sabido crear, en el pasado, grandes ocasiones de encuentro entre los pueblos, y ha transmitido a las épocas sucesivas síntesis admirables de cultura y fe.

El misterio de la vida de Nazaret, del que san José, patrono de la Iglesia y vuestro protector, fue custodio fiel y testigo sabio, es el icono de esta admirable síntesis entre vida de fe y trabajo humano, entre crecimiento personal y compromiso de solidaridad.

Amadísimos artesanos, habéis venido hoy para celebrar vuestro jubileo. Que la luz del Evangelio ilumine cada vez más vuestra experiencia laboral diaria. El jubileo os ofrece la ocasión de encontraros con Jesús, José y María, entrando en su casa y en el humilde taller de Nazaret.

En la singular escuela de la Sagrada Familia se aprenden las realidades esenciales de la vida y se profundiza el significado del seguimiento de Jesús. Nazaret enseña a superar la tensión aparente entre la vida activa y la contemplativa; invita a crecer en el amor a la verdad divina que irradia la humanidad de Cristo y a prestar con valentía el exigente servicio de la tutela de Cristo presente en todo hombre (cf. \textit{Redemptoris custos}, 27).

Crucemos, por tanto, en una peregrinación espiritual, el umbral de la casa de Nazaret, el humilde hogar que tendré la alegría de visitar, Dios mediante, la próxima semana, durante mi peregrinación jubilar a Tierra Santa.

Contemplemos a María, testigo del cumplimiento de la promesa hecha por el Señor \textquote{en favor de Abraham y su descendencia por siempre} (\textit{Lc} 1, 54-55).

Que ella, junto con José, su casto esposo, os ayude, queridos artesanos, a permanecer en constante escucha de Dios, uniendo oración y trabajo. Ellos os sostengan en vuestros propósitos jubilares de renovada fidelidad cristiana y hagan que vuestras manos prolonguen, en cierto modo, la obra creadora y providente de Dios.

La Sagrada Familia, lugar de entendimiento y amor, os ayude a realizar gestos de solidaridad, paz y perdón. Así, seréis heraldos del amor infinito de Dios Padre, rico en misericordia y bondad para con todos.]}

Amén.
\end{body}

\begin{patercite}
La Transfiguración del Señor, (\ldots)  proyecta una luz deslumbrante sobre nuestra vida diaria y nos lleva a dirigir la mente al destino inmortal que este hecho esconde.

En la cima del Tabor, durante unos instantes, Cristo levanta el velo que oculta el resplandor de su divinidad y se manifiesta a los testigos elegidos como es realmente, el Hijo de Dios, \textquote{el esplendor de la gloria del Padre y la imagen de su substancia} (cf. \textit{Heb} 1, 5); pero al mismo tiempo desvela el destino trascendente de nuestra naturaleza humana que Él ha tomado para salvarnos, destinada también ésta (por haber sido redimida por su sacrificio de amor irrevocable) a participar en la plenitud de la vida, en la \textquote{herencia de los santos en la luz} (\textit{Col} 1, 12).

Ese cuerpo que se transfigura ante los ojos atónitos de los Apóstoles es el cuerpo de Cristo nuestro hermano, pero es también nuestro cuerpo destinado a la gloria; la luz que le inunda es y será también nuestra parte de herencia y de esplendor.

Estamos llamados a condividir tan gran gloria, porque somos \textquote{partícipes de la divina naturaleza} (\textit{2 Pe} 1. 4).

Nos espera una suerte incomparable, en el caso de que hayamos hecho honor a nuestra vocación cristiana y hayamos vivido con la lógica
consecuencia de palabras y comportamiento, a que nos obligan los compromisos de nuestro bautismo.

\textbf{San Pablo VI, papa}, \textit{Ángelus}, 6 de agosto de 1978.

\tiny{(*) Esta alocución que el Papa había preparado para dirigirla a los files no pudo ser leída por él debido a que su estado de salud se agravó. Pablo VI descansó en la paz del Señor ese mismo día, domingo 6 de agosto, fiesta de la Transfiguración del Señor a las 21,40 horas.}
\end{patercite}
\label{b2-03-02-2000H}

\newsection
\subsection{Benedicto XVI, papa}

\subsubsection{Ángelus (2006): Escuchar a Dios}

\src{12 de marzo del 2006.}

\begin{body}
\ltr[(\ldots)]{E}{n} el tiempo de Cuaresma estamos llamados de un modo particular a la escucha del Señor, que siempre nos habla, pero espera de nosotros mayor atención \ldots 

Nos lo recuerda también la \textbf{página evangélica} de este domingo, que propone de nuevo la narración de la transfiguración de Cristo en el monte Tabor. Mientras estaban atónitos en presencia del Señor transfigurado, que conversaba con Moisés y Elías, Pedro, Santiago y Juan fueron envueltos repentinamente por una nube, de la que salió una voz que proclamó: \textquote{Este es mi Hijo amado; escuchadlo} (\textit{Mc} 9, 7). 

Cuando se tiene la gracia de vivir una fuerte experiencia de Dios, es como si se viviera algo semejante a lo que les sucedió a los discípulos durante la Transfiguración: por un momento se gusta anticipadamente algo de lo que constituirá la bienaventuranza del paraíso. En general, se trata de breves experiencias que Dios concede a veces, especialmente con vistas a duras pruebas. Pero a nadie se le concede vivir \textquote{en el Tabor} mientras está en esta tierra. En efecto, la existencia humana es un camino de fe y, como tal, transcurre más en la penumbra que a plena luz, con momentos de oscuridad e, incluso, de tinieblas. Mientras estamos aquí, nuestra relación con Dios se realiza más en la escucha que en la visión; y la misma contemplación se realiza, por decirlo así, con los ojos cerrados, gracias a la luz interior encendida en nosotros por la palabra de Dios.

También la Virgen María, aun siendo entre todas las criaturas humanas la más cercana a Dios, caminó día a día como en una peregrinación de la fe (cf. \textit{Lumen gentium}, 58), conservando y meditando constantemente en su corazón las palabras que Dios le dirigía, ya sea a través de las Sagradas Escrituras o bien mediante los acontecimientos de la vida de su Hijo, en los que reconocía y acogía la misteriosa voz del Señor. 

He aquí, pues, el don y el compromiso de cada uno de nosotros durante el tiempo cuaresmal: escuchar a Cristo, como María. Escucharlo en su palabra, custodiada en la Sagrada Escritura. Escucharlo en los acontecimientos mismos de nuestra vida, tratando de leer en ellos los mensajes de la Providencia. Por último, escucharlo en los hermanos, especialmente en los pequeños y en los pobres, para los cuales Jesús mismo pide nuestro amor concreto. Escuchar a Cristo y obedecer su voz: este es el camino real, el único que conduce a la plenitud de la alegría y del amor.
\end{body}


\label{b2-03-02-2006A}

\begin{patercite}
El evento de la Transfiguración del Señor nos ofrece un mensaje de esperanza ---así seremos nosotros, con Él---: nos invita a encontrar a Jesús, para estar al servicio de los hermanos.

La ascensión de los discípulos al monte Tabor nos induce a reflexionar sobre la importancia de separarse de las cosas mundanas, para cumplir un camino hacia lo alto y contemplar a Jesús. Se trata de ponernos a la escucha atenta y orante del Cristo, el Hijo amado del Padre, buscando momentos de oración que permiten la acogida dócil y alegre de la Palabra de Dios. En esta ascensión espiritual, en esta separación de las cosas mundanas, estamos llamados a redescubrir el silencio pacificador y regenerador de la meditación del Evangelio, de la lectura de la Biblia, que conduce hacia una meta rica de belleza, de esplendor y de alegría. Y cuando nosotros nos ponemos así, con la Biblia en la mano, en silencio, comenzamos a escuchar esta belleza interior, esta alegría que genera la Palabra de Dios en nosotros.

Al finalizar la experiencia maravillosa de la Transfiguración, los discípulos bajaron del monte (cf. v. 9) con ojos y corazón transfigurados por el encuentro con el Señor. Es el recorrido que podemos hacer también nosotros. El redescubrimiento cada vez más vivo de Jesús no es fin en sí mismo, pero nos lleva a \textquote{bajar del monte}, cargados con la fuerza del Espíritu divino, para decidir nuevos pasos de conversión y para testimoniar constantemente la caridad, como ley de vida cotidiana. Transformados por la presencia de Cristo y del ardor de su palabra, seremos signo concreto del amor vivificante de Dios para todos nuestros hermanos, especialmente para quien sufre, para los que se encuentran en soledad y abandono, para los enfermos y para la multitud de hombres y de mujeres que, en distintas partes del mundo, son humillados por la injusticia, la prepotencia y la violencia. En la Transfiguración se oye la voz del Padre celeste que dice: \textquote{Este es mi hijo amado, ¡escuchadle!} (v. 5). Miremos a María, la \emph{Virgen de la escucha}, siempre
preparada a acoger y custodiar en el corazón cada palabra del Hijo divino (cf. \emph{Lucas} 1, 51). Quiera nuestra Madre y Madre de Dios
ayudarnos a entrar en sintonía con la Palabra de Dios, para que Cristo se convierta en luz y guía de toda nuestra vida. [\ldots]

\textbf{Francisco, papa}, \textit{Ángelus}, 6 de agosto del 2017, parr. 2-3.
\end{patercite}
\newpage


\subsubsection{Ángelus (2009): Una experiencia de oración}

\src{8 de marzo del 2009.}

\begin{body}
\ltr{J}{esús} [como nos muestra el \textbf{Evangelio} de hoy] llevó a los apóstoles Pedro, Santiago y Juan, solos a un monte alto, en un lugar apartado, y mientras oraba se \textquote{transfiguró}: su rostro y su persona se volvieron luminosos, resplandecientes.

La liturgia vuelve a proponer este célebre episodio precisamente hoy, segundo domingo de Cuaresma (cf. \textit{Mc} 9, 2-10). Jesús quería que sus discípulos, de modo especial los que tendrían la responsabilidad de guiar a la Iglesia naciente, experimentaran directamente su gloria divina, para afrontar el escándalo de la cruz. En efecto, cuando llegue la hora de la traición y Jesús se retire a rezar a Getsemaní, tomará consigo a los mismos Pedro, Santiago y Juan, pidiéndoles que velen y oren con él (cf. \textit{Mt} 26, 38). Ellos no lo lograrán, pero la gracia de Cristo los sostendrá y les ayudará a creer en la resurrección.

Quiero subrayar que la Transfiguración de Jesús fue esencialmente una experiencia de oración (cf. \textit{Lc} 9, 28-29). En efecto, la oración alcanza su culmen, y por tanto se convierte en fuente de luz interior, cuando el espíritu del hombre se adhiere al de Dios y sus voluntades se funden como formando una sola cosa. Cuando Jesús subió al monte, se sumergió en la contemplación del designio de amor del Padre, que lo había mandado al mundo para salvar a la humanidad. Junto a Jesús aparecieron Elías y Moisés, para significar que las Sagradas Escrituras concordaban en anunciar el misterio de su Pascua, es decir, que Cristo debía sufrir y morir para entrar en su gloria (cf. \textit{Lc} 24, 26. 46). En aquel momento Jesús vio perfilarse ante él la cruz, el extremo sacrificio necesario para liberarnos del dominio del pecado y de la muerte. Y en su corazón, una vez más, repitió su \textquote{Amén}. Dijo \textquote{sí}, \textquote{heme aquí}, \textquote{hágase, oh Padre, tu voluntad de amor}. Y, como había sucedido después del bautismo en el Jordán, llegaron del cielo los signos de la complacencia de Dios Padre: la luz, que transfiguró a Cristo, y la voz que lo proclamó \textquote{Hijo amado} (\textit{Mc} 9, 7).

Juntamente con el ayuno y las obras de misericordia, la oración forma la estructura fundamental de nuestra vida espiritual. Queridos hermanos y hermanas, os exhorto a encontrar en este tiempo de Cuaresma momentos prolongados de silencio, posiblemente de retiro, para revisar vuestra vida a la luz del designio de amor del Padre celestial. En esta escucha más intensa de Dios dejaos guiar por la Virgen María, maestra y modelo de oración. Ella, incluso en la densa oscuridad de la pasión de Cristo, no perdió la luz de su Hijo divino, sino que la custodió en su alma. Por eso, la invocamos como Madre de la confianza y de la esperanza.
\end{body}

\label{b2-03-02-2009A}
\newpage

\subsubsection{Homilía (2012): Sin cruz no hay salvación}

\src{Visita Pastoral a la Parroquia Romana \\de San Juan Bautista de la Salle en Torrino. \\4 de marzo del 2012.}

\begin{body}
\ltr{L}{a} liturgia de este día nos prepara sea para el misterio de la Pasión –como escuchamos en la primera lectura– sea para la alegría de la Resurrección.

La \textbf{primera lectura} nos refiere el episodio en el que Dios pone a prueba a Abrahán (cf. \textit{Gn} 22, 1-18). Abrahán tenía un hijo único, Isaac, que le nació en la vejez. Era el hijo de la promesa, el hijo que debería llevar luego la salvación también a los pueblos. Pero un día Abrahán recibe de Dios la orden de ofrecerlo en sacrificio. El anciano patriarca se encuentra ante la perspectiva de un sacrificio que para él, padre, es ciertamente el mayor que se pueda imaginar. Sin embargo, no duda ni siquiera un instante y, después de preparar lo necesario, parte junto con Isaac hacia el lugar establecido. Y podemos imaginar esta caminata hacia la cima del monte, lo que sucedió en su corazón y en el corazón de su hijo. Construye un altar, coloca la leña y, después de atar al muchacho, aferra el cuchillo para inmolarlo. Abrahán se fía de Dios hasta tal punto que está dispuesto incluso a sacrificar a su propio hijo y, juntamente con el hijo, su futuro, porque sin ese hijo la promesa de la tierra no servía para nada, acabaría en la nada. Y sacrificando a su hijo se sacrifica a sí mismo, todo su futuro, toda la promesa. Es realmente un acto de fe radicalísimo. En ese momento lo detiene una orden de lo alto: Dios no quiere la muerte, sino la vida; el verdadero sacrificio no da muerte, sino que es la vida, y la obediencia de Abrahán se convierte en fuente de una inmensa bendición hasta hoy. Dejemos esto, pero podemos meditar este misterio.

En la \textbf{segunda lectura}, san Pablo afirma que Dios mismo realizó un sacrificio: nos dio a su propio Hijo, lo donó en la cruz para vencer el pecado y la muerte, para vencer al maligno y para superar toda la malicia que existe en el mundo. Y esta extraordinaria misericordia de Dios suscita la admiración del Apóstol y una profunda confianza en la fuerza del amor de Dios a nosotros; de hecho, san Pablo afirma: \textquote{[Dios], que no se reservó a su propio Hijo, sino que lo entregó por todos nosotros, ¿cómo no nos dará todo con él?} (\textit{Rm} 8, 32). Si Dios se da a sí mismo en el Hijo, nos da todo. Y san Pablo insiste en la potencia del sacrificio redentor de Cristo contra cualquier otro poder que pueda amenazar nuestra vida. Se pregunta: \textquote{¿Quién acusará a los elegidos de Dios? Dios es el que justifica. ¿Quién condenará? ¿Acaso Cristo Jesús, que murió; más todavía, resucitó y está a la derecha de Dios y que además intercede por nosotros?} (\textit{Rm} 8, 33-34). Nosotros estamos en el corazón de Dios; esta es nuestra gran confianza. Esto crea amor y en el amor vamos hacia Dios. Si Dios ha entregado a su propio Hijo por todos nosotros, nadie podrá acusarnos, nadie podrá condenarnos, nadie podrá separarnos de su inmenso amor. Precisamente el sacrificio supremo de amor en la cruz, que el Hijo de Dios aceptó y eligió voluntariamente, se convierte en fuente de nuestra justificación, de nuestra salvación. Y pensemos que en la Sagrada Eucaristía siempre está presente este acto del Señor, que en su corazón permanece por toda la eternidad, y este acto de su corazón nos atrae, nos une a él.

Por último, el \textbf{Evangelio} nos habla del episodio de la Transfiguración (cf. \textit{Mc} 9, 2-10): Jesús se manifiesta en su gloria antes del sacrificio de la cruz y Dios Padre lo proclama su Hijo predilecto, el amado, e invita a los discípulos a escucharlo. Jesús sube a un monte alto y toma consigo a tres apóstoles –Pedro, Santiago y Juan–, que estarán especialmente cercanos a él en la agonía extrema, en otro monte, el de los Olivos. Poco tiempo antes el Señor había anunciado su pasión y Pedro no había logrado comprender por qué el Señor, el Hijo de Dios, hablaba de sufrimiento, de rechazo, de muerte, de cruz; más aún, se había opuesto decididamente a esta perspectiva. Ahora Jesús toma consigo a los tres discípulos para ayudarlos a comprender que el camino para llegar a la gloria, el camino del amor luminoso que vence las tinieblas, pasa por la entrega total de sí mismo, pasa por el escándalo de la cruz. Y el Señor debe volver a llevarnos siempre con Él, para que empecemos al menos a comprender que ese es el camino necesario. La transfiguración es un momento anticipado de luz que nos ayuda también a nosotros a contemplar la pasión de Jesús con una mirada de fe. La pasión de Jesús es un misterio de sufrimiento, pero también es la \textquote{bienaventurada pasión} porque en su núcleo es un misterio de amor extraordinario de Dios; es el éxodo definitivo que nos abre la puerta hacia la libertad y la novedad de la Resurrección, de la salvación del mal. Tenemos necesidad de ella en nuestro camino diario, a menudo marcado también por la oscuridad del mal.

[\ldots]

Por último, quiero recordaros a todos la importancia y la centralidad de la Eucaristía en la vida personal y comunitaria. La santa misa debe estar en el centro de vuestro Domingo, que es preciso redescubrir y vivir como día de Dios y de la comunidad, día en el cual alabar y celebrar a Aquel que murió y resucitó por nuestra salvación, día en el cual vivir juntos en la alegría de una comunidad abierta y dispuesta a acoger a toda persona sola o en dificultades. Reunidos en torno a la Eucaristía, de hecho, percibimos más fácilmente que la misión de toda comunidad cristiana consiste en llevar el mensaje del amor de Dios a todos los hombres. Precisamente por eso es importante que la Eucaristía esté siempre en el corazón de la vida de los fieles, como lo está hoy.

Queridos hermanos y hermanas, desde el Tabor, el monte de la Transfiguración, el itinerario cuaresmal nos conduce hasta el Gólgota, monte del supremo sacrificio de amor del único Sacerdote de la alianza nueva y eterna. En ese sacrificio se encierra la mayor fuerza de transformación del hombre y de la historia. Asumiendo sobre sí todas las consecuencias del mal y del pecado, Jesús resucitó al tercer día como vencedor de la muerte y del Maligno. La Cuaresma nos prepara para participar personalmente en este gran misterio de la fe, que celebraremos en el Triduo de la pasión, muerte y resurrección de Cristo. Encomendemos a la Virgen María nuestro camino cuaresmal, así como el de toda la Iglesia. Ella, que siguió a su Hijo Jesús hasta la cruz, nos ayude a ser discípulos fieles de Cristo, cristianos maduros, para poder participar juntamente con ella en la plenitud de la alegría pascual. Amén.
\end{body}

\begin{patercite}
(\ldots) La apertura del alma a Dios y a su acción en la fe incluye también el elemento de la oscuridad. La relación del ser humano con Dios no cancela la distancia entre Creador y criatura, no elimina cuanto afirma el apóstol Pablo ante las profundidades de la sabiduría de Dios: «¡Qué insondables sus decisiones y qué irrastreables sus caminos!» (\emph{Rm} 11, 33). Pero precisamente quien ---como María--- está totalmente abierto a Dios, llega a aceptar el querer divino, incluso si es misterioso, también si a menudo no corresponde al propio querer y es una espada que traspasa el alma, como dirá proféticamente el anciano Simeón a María, en el momento de la presentación de Jesús en el Templo (cf. \emph{Lc} 2, 35). El camino de fe de Abrahán comprende el momento de alegría por el don del hijo Isaac, pero también el momento de la oscuridad, cuando debe subir al monte Moria para realizar un gesto paradójico: Dios le pide que sacrifique el hijo que le había dado. En el monte el ángel le ordenó: «No alargues la mano contra el muchacho ni le hagas nada. Ahora he comprobado que temes a Dios, porque no te has reservado a tu hijo, a tu único hijo» (\emph{Gn} 22, 12). La plena confianza de Abrahán en el Dios fiel a las promesas no disminuye incluso cuando su palabra es misteriosa y difícil, casi imposible, de acoger. Así es para María; su fe vive la alegría de la Anunciación, pero pasa también a través de la oscuridad de la crucifixión del Hijo para poder llegar a la luz de la Resurrección.

No es distinto incluso para el camino de fe de cada uno de nosotros:encontramos momentos de luz, pero hallamos también momentos en los que Dios parece ausente, su silencio pesa en nuestro corazón y su voluntad no corresponde a la nuestra, a aquello que nosotros quisiéramos. Pero cuanto más nos abrimos a Dios, acogemos el don de la fe, ponemos totalmente en Él nuestra confianza ---como Abrahán y como María---, tanto más Él nos hace capaces, con su presencia, de vivir cada situación de la vida en la paz y en la certeza de su fidelidad y de su amor. Sin embargo, esto implica salir de uno mismo y de los propios proyectos para que la Palabra de Dios sea la lámpara que guíe nuestros pensamientos y nuestras acciones.
	
	\textbf{Benedicto XVI, papa}, \textit{Catequesis}, Audiencia general,  19 de diciembre de 2012, parr. 6-7.
\end{patercite}

\label{b2-03-02-2012H}
\newpage

\subsubsection{Ángelus (2012): Una luz para superar las pruebas}

\src{4 de marzo del 2012.}

\begin{body}
\ltr{E}{ste} domingo, el segundo de Cuaresma, se caracteriza por ser el domingo de la Transfiguración de Cristo. De hecho, durante la Cuaresma, la liturgia, después de habernos invitado a seguir a Jesús en el desierto, para afrontar y superar con él las tentaciones, nos propone subir con él al \textquote{monte} de la oración, para contemplar en su rostro humano la luz gloriosa de Dios. Los evangelistas Mateo, \textbf{Marcos} y Lucas atestiguan de modo concorde el episodio de la transfiguración de Cristo. Los elementos esenciales son dos: en primer lugar, Jesús sube con sus discípulos Pedro, Santiago y Juan a un monte alto, y allí \textquote{se transfiguró delante de ellos} (\textit{Mc} 9, 2), su rostro y sus vestidos irradiaron una luz brillante, mientras que junto a él aparecieron Moisés y Elías; y, en segundo lugar, una nube envolvió la cumbre del monte y de ella salió una voz que decía: \textquote{Este es mi Hijo amado, escuchadlo} (\textit{Mc} 9, 7). Por lo tanto, la luz y la voz: la luz divina que resplandece en el rostro de Jesús, y la voz del Padre celestial que da testimonio de él y manda escucharlo.

El misterio de la Transfiguración no se debe separar del contexto del camino que Jesús está recorriendo. Ya se ha dirigido decididamente hacia el cumplimiento de su misión, a sabiendas de que, para llegar a la resurrección, tendrá que pasar por la pasión y la muerte de cruz. De esto les ha hablado abiertamente a sus discípulos, los cuales sin embargo no han entendido; más aun, han rechazado esta perspectiva porque no piensan como Dios, sino como los hombres (cf. \textit{Mt} 16, 23). Por eso Jesús lleva consigo a tres de ellos al monte y les revela su gloria divina, esplendor de Verdad y de Amor. Jesús quiere que esta luz ilumine sus corazones cuando pasen por la densa oscuridad de su pasión y muerte, cuando el escándalo de la cruz sea insoportable para ellos. Dios es luz, y Jesús quiere dar a sus amigos más íntimos la experiencia de esta luz, que habita en él. Así, después de este episodio, él será en ellos una luz interior, capaz de protegerlos de los asaltos de las tinieblas. Incluso en la noche más oscura, Jesús es la luz que nunca se apaga. San Agustín resume este misterio con una expresión muy bella. Dice: \textquote{Lo que para los ojos del cuerpo es el sol que vemos, lo es [Cristo] para los ojos del corazón} (\textit{Sermo} 78, 2: PL 38, 490).

Queridos hermanos y hermanas, todos necesitamos luz interior para superar las pruebas de la vida. Esta luz viene de Dios, y nos la da Cristo, en quien habita la plenitud de la divinidad (cf. \textit{Col} 2, 9). Subamos con Jesús al monte de la oración y, contemplando su rostro lleno de amor y de verdad, dejémonos colmar interiormente de su luz. Pidamos a la Virgen María, nuestra guía en el camino de la fe, que nos ayude a vivir esta experiencia en el tiempo de la Cuaresma, encontrando cada día algún momento para orar en silencio y para escuchar la Palabra de Dios.
\end{body}

\newsection
\subsection{Francisco, papa}

\subsubsection{Ángelus (2015): Escuchar al Hijo}

\src{Plaza de San Pedro. \\1 de marzo del 2015.}

\begin{body}
\ltr{E}{l} domingo pasado la liturgia nos presentó a Jesús tentado por Satanás en el desierto, pero victorioso en la tentación. A la luz de este Evangelio, hemos tomado nuevamente conciencia de nuestra condición de pecadores, pero también de la victoria sobre el mal donada a quienes inician el camino de conversión y que, como Jesús, quieren hacer la voluntad del Padre. En este segundo domingo de Cuaresma, la Iglesia nos indica la meta de este itinerario de conversión, es decir, la participación en la gloria de Cristo, que resplandece en el rostro del Siervo obediente, muerto y resucitado por nosotros.

El \textbf{pasaje evangélico} narra el acontecimiento de la Transfiguración, que se sitúa en la cima del ministerio público de Jesús. Él está en camino hacia Jerusalén, donde se cumplirán las profecías del \textquote{Siervo de Dios} y se consumará su sacrificio redentor. La multitud no entendía esto: ante las perspectivas de un Mesías que contrasta con sus expectativas terrenas, lo abandonaron. Pero ellos pensaban que el Mesías sería un liberador del dominio de los romanos, un liberador de la patria, y esta perspectiva de Jesús no les gusta y lo abandonan. Incluso los Apóstoles no entienden las palabras con las que Jesús anuncia el cumplimiento de su misión en la pasión gloriosa, ¡no comprenden! Jesús entonces toma la decisión de mostrar a Pedro, Santiago y Juan una anticipación de su gloria, la que tendrá después de la resurrección, para confirmarlos en la fe y alentarlos a seguirlo por la senda de la prueba, por el camino de la Cruz. Y, así, sobre un monte alto, inmerso en oración, se transfigura delante de ellos: su rostro y toda su persona irradian una luz resplandeciente. Los tres discípulos están asustados, mientras una nube los envuelve y desde lo alto resuena –como en el Bautismo en el Jordán– la voz del Padre: \textquote{Este es mi Hijo amado; escuchadlo} (\textit{Mc} 9, 7). Jesús es el Hijo hecho Siervo, enviado al mundo para realizar a través de la Cruz el proyecto de la salvación, para salvarnos a todos nosotros. Su adhesión plena a la voluntad del Padre hace su humanidad transparente a la gloria de Dios, que es el Amor.

Jesús se revela así como el icono perfecto del Padre, la irradiación de su gloria. Es el cumplimiento de la revelación; por eso junto a Él transfigurado aparecen Moisés y Elías, que representan la Ley y los Profetas, para significar que todo termina y comienza en Jesús, en su pasión y en su gloria.

La consigna para los discípulos y para nosotros es esta: \textquote{¡Escuchadlo!}. Escuchad a Jesús. Él es el Salvador: seguidlo. Escuchar a Cristo, en efecto, lleva a asumir la lógica de su misterio pascual, ponerse en camino con Él para hacer de la propia vida un don de amor para los demás, en dócil obediencia a la voluntad de Dios, con una actitud de desapego de las cosas mundanas y de libertad interior. Es necesario, en otras palabras, estar dispuestos a \textquote{perder la propia vida} (cf. \textit{Mc} 8, 35), entregándola a fin de que todos los hombres se salven: así, nos encontraremos en la felicidad eterna. El camino de Jesús nos lleva siempre a la felicidad, ¡no lo olvidéis! El camino de Jesús nos lleva siempre a la felicidad. Habrá siempre una cruz en medio, pruebas, pero al final nos lleva siempre a la felicidad. Jesús no nos engaña, nos prometió la felicidad y nos la dará si vamos por sus caminos.

Con Pedro, Santiago y Juan subamos también nosotros hoy al monte de la Transfiguración y permanezcamos en contemplación del rostro de Jesús, para acoger su mensaje y traducirlo en nuestra vida; para que también nosotros podamos ser transfigurados por el Amor. En realidad, el amor es capaz de transfigurar todo. ¡El amor transfigura todo! ¿Creéis en esto? Que la Virgen María, que ahora invocamos con la oración del Ángelus, nos sostenga en este camino.
\end{body}

\img{cross_quedlinburg}

\label{b2-03-02-2015A}
\newpage

\subsubsection{Homilía (2018): Nos prepara para las pruebas}

\src{Visita pastoral a la parroquia romana \\de San Gelasio I, papa, en Ponte Mammolo.\\25 de febrero del 2018.}

\begin{body}
\ltr{J}{esús} se deja ver a los Apóstoles como es en el cielo: glorioso, luminoso, triunfante, vencedor. Y esto lo hace para prepararles a soportar la Pasión, el escándalo de la cruz, porque ellos no podían entender que Jesús hubiera muerto como un criminal, no podían entenderlo. Ellos pensaban que Jesús fuera un libertador, pero como son los libertadores terrenales, los que ganan en la batalla, los que son siempre triunfadores. Y el camino de Jesús es otro: Jesús triunfa a través de la humillación, la humillación de la cruz. Pero puesto que esto hubiera sido un escándalo para ellos, Jesús les hace ver lo que viene después, lo que hay después de la cruz, lo que nos espera a todos nosotros. Esta gloria y este cielo.

¡Y eso es muy hermoso! Es muy hermoso porque Jesús –y esto escuchadlo bien– nos prepara siempre para la prueba. En un modo o en otro, pero este es el mensaje: nos prepara siempre. Nos da la fuerza para ir adelante en los momentos de prueba y vencerlos con su fuerza. Jesús no nos deja solos en las pruebas de la vida: siempre nos prepara, nos ayuda, como ha preparado a estos [los discípulos], con la visión de su gloria. Y así ellos después recordaron esto [el momento] para soportar el peso de la humillación.

Esto es lo primero que nos enseña la Iglesia: Jesús nos prepara siempre para las pruebas y en las pruebas está con nosotros, no nos deja solos. Nunca. Lo segundo, podemos tomarlo de las palabras de Dios: \textquote{Este es mi Hijo, el amado. Escuchadle}. Este es el mensaje que el Padre da a los Apóstoles. El mensaje de Jesús es prepararlos, haciéndoles ver su gloria; el mensaje del Padre es: \textquote{Escuchadle}. No hay un momento en la vida que no se pueda vivir plenamente escuchando a Jesús. En los momentos hermosos, deteneos y escuchad a Jesús; en los momentos malos, deteneos y escuchad a Jesús. Este es el camino. Él nos dirá lo que tenemos que hacer. Siempre. Y vamos adelante en esta Cuaresma con estas dos cosas: en las pruebas, recordad la gloria de Jesús, es decir, lo que nos espera; que Jesús está presente siempre, con esa gloria para darnos fuerza.

Y durante toda la vida, escuchad a Jesús, lo que nos dice Jesús: en el Evangelio, en la liturgia, siempre nos habla; o en el corazón.

En la vida cotidiana, tal vez tengamos problemas, o tengamos que resolver muchas cosas. Hagámonos esta pregunta: ¿Qué nos dice Jesús hoy? Y busquemos escuchar la voz de Jesús, la inspiración desde dentro. Y así seguimos el consejo del Padre: \textquote{Este es mi Hijo, el amado. Escuchadle}. Será la Virgen la que te dé el segundo consejo en Caná, en Galilea, cuando se produce el milagro del agua [trasformada] en vino. ¿Qué dice la Virgen? \textquote{Haced lo que Él diga}. Escuchar a Jesús y hacer lo que Él dice: este es el camino seguro. Ir adelante con el recuerdo de la gloria de Jesús, con este consejo: escuchar a Jesús y hacer lo que Él nos dice.
\end{body}

\begin{patercite}
\textbf{Jesucristo, el amor de Dios encarnado}

(\ldots) La verdadera originalidad del Nuevo Testamento no consiste en nuevas ideas, sino en la figura misma de Cristo, que da carne y sangre a los conceptos: un realismo inaudito. Tampoco en el Antiguo Testamento la novedad bíblica consiste simplemente en nociones abstractas, sino en la actuación imprevisible y, en cierto sentido inaudita, de Dios. Este actuar de Dios adquiere ahora su forma dramática, puesto que, en Jesucristo, el propio Dios va tras la \textquote{oveja perdida}, la humanidad doliente y extraviada. Cuando Jesús habla en sus parábolas del pastor que va tras la oveja descarriada, de la mujer que busca el dracma, del padre que sale al encuentro del hijo pródigo y lo abraza, no se trata sólo de meras palabras, sino que es la explicación de su propio ser y actuar. En su muerte en la cruz se realiza ese ponerse Dios contra sí mismo, al entregarse para dar nueva vida al hombre y salvarlo: esto es amor en su forma más radical. Poner la mirada en el costado traspasado de Cristo, del que habla Juan (cf. 19, 37), ayuda a comprender lo que [significa que] \textquote{Dios es amor} (\textit{1 Jn} 4, 8). Es allí, en la cruz, donde puede contemplarse esta verdad. Y a partir de allí se debe definir ahora qué es el amor. Y, desde esa mirada, el cristiano encuentra la orientación de su vivir y de su amar.

Jesús ha perpetuado este acto de entrega mediante la institución de la Eucaristía durante la Última Cena. Ya en aquella hora, Él anticipa su muerte y resurrección, dándose a sí mismo a sus discípulos en el pan y en el vino, su cuerpo y su sangre como nuevo maná (cf. \textit{Jn} 6, 31-33). Si el mundo antiguo había soñado que, en el fondo, el verdadero alimento del hombre ---aquello por lo que el hombre vive--- era el \textit{Logos}, la sabiduría eterna, ahora este \textit{Logos} se ha hecho para nosotros verdadera comida, como amor. La Eucaristía nos adentra en el acto oblativo de Jesús. No recibimos solamente de modo pasivo el \textit{Logos} encarnado, sino que nos implicamos en la dinámica de su entrega. La imagen de las nupcias entre Dios e Israel se hace realidad de un modo antes inconcebible: lo que antes era estar frente a Dios, se transforma ahora en unión por la participación en la entrega de Jesús, en su cuerpo y su sangre. La \textquote{mística} del Sacramento, que se basa en el abajamiento de Dios hacia nosotros, tiene otra dimensión de gran alcance y que lleva mucho más alto de lo que cualquier elevación mística del hombre podría alcanzar.

\textbf{Benedicto XVI, papa}, \textit{Deus Caritas est}, nn. 12-13.
\end{patercite}
\label{b2-03-02-2018H}
\newpage


\subsubsection{Ángelus (2018): Anticipo del cielo}

\src{Plaza de San Pedro. \\25 de febrero de 2018.}

\begin{body}
\ltr{E}{l} \textbf{Evangelio} hoy, segundo domingo de Cuaresma, nos invita a contemplar la transfiguración de Jesús (cf. \textit{Mc} 9, 2-10). Este episodio está ligado a lo que sucedió seis días antes, cuando Jesús había desvelado a sus discípulos que en Jerusalén debería \textquote{sufrir mucho y ser reprobado por los ancianos, los sumos sacerdotes y los escribas, ser matado y resucitado a los tres días} (\textit{Mc} 8, 31). Este anuncio había puesto en crisis a Pedro y a todo el grupo de discípulos, que rechazaban la idea de que Jesús terminara rechazado por los jefes del pueblo y después matado. Ellos, de hecho, esperaban a un Mesías poderoso, fuerte, dominador; en cambio, Jesús se presenta como humilde, como manso, siervo de Dios, siervo de los hombres, que deberá entregar su vida en sacrificio, pasando por el camino de la persecución, del sufrimiento y de la muerte.

Pero, ¿cómo poder seguir a un Maestro y Mesías cuya vivencia terrenal terminaría de ese modo? Así pensaban ellos. Y la respuesta llega precisamente de la transfiguración. ¿Qué es la transfiguración de Jesús? Es una aparición pascual anticipada. Jesús toma consigo a los tres discípulos Pedro, Santiago y Juan y \textquote{los lleva, a ellos solos, a parte, a un monte alto} (\textit{Mc} 9, 2); y allí, por un momento, les muestra su gloria, gloria de Hijo de Dios. Este evento de la transfiguración permite así a los discípulos afrontar la pasión de Jesús de un modo positivo, sin ser arrastrados. Lo vieron como será después de la pasión, glorioso. Y así Jesús les prepara para la prueba. La transfiguración ayuda a los discípulos, y también a nosotros, a entender que la pasión de Cristo es un misterio de sufrimiento, pero es sobre todo un regalo de amor, de amor infinito por parte de Jesús.

El evento de Jesús transfigurándose sobre el monte nos hace entender mejor también su resurrección. Para entender el misterio de la cruz es necesario saber con antelación que el que sufre y que es glorificado no es solamente un hombre, sino el Hijo de Dios, que con su amor fiel hasta la muerte nos ha salvado. El padre renueva así su declaración mesiánica sobre el Hijo, ya hecha en la orilla del Jordán después del bautismo y exhorta: \textquote{Escuchadle} (\textit{Mc} 9, 7).

Los discípulos están llamados a seguir al Maestro con confianza, con esperanza, a pesar de su muerte; la divinidad de Jesús debe manifestarse precisamente en la cruz, precisamente en su morir \textquote{de aquel modo}, tanto que el evangelista Marcos pone en la boca del centurión la profesión de fe: \textquote{Verdaderamente este hombre era el Hijo de Dios} (\textit{Mc} 15, 39). Nos dirigimos ahora en oración a la Virgen María, la criatura humana transfigurada interiormente por la gracia de Cristo. Nos encomendamos confiados a su maternal ayuda para proseguir con fe y generosidad el camino de la Cuaresma.
\end{body}

\newsection
\section{Temas}

\cceth{La Transfiguración} 
\cceref{CEC 554-556, 568}

\begin{ccebody}
\ccesec{Una visión anticipada del Reino: La Transfiguración.}

\n{554} A partir del día en que Pedro confesó que Jesús es el Cristo, el Hijo de Dios vivo, el Maestro \textquote{comenzó a mostrar a sus discípulos que él debía ir a Jerusalén, y sufrir [\ldots] y ser condenado a muerte y resucitar al tercer día} (\textit{Mt} 16, 21): Pedro rechazó este anuncio (cf. \textit{Mt} 16, 22-23), los otros no lo comprendieron mejor (cf. \textit{Mt} 17, 23; \textit{Lc} 9, 45). En este contexto se sitúa el episodio misterioso de la Transfiguración de Jesús (cf. \textit{Mt} 17, 1-8 par.; \textit{2 P} 1, 16-18), sobre una montaña, ante tres testigos elegidos por él: Pedro, Santiago y Juan. El rostro y los vestidos de Jesús se pusieron fulgurantes como la luz, Moisés y Elías aparecieron y le \textquote{hablaban de su partida, que estaba para cumplirse en Jerusalén} (\textit{Lc} 9, 31). Una nube les cubrió y se oyó una voz desde el cielo que decía: \textquote{Este es mi Hijo, mi elegido; escuchadle} (\textit{Lc} 9, 35).

\n{555} Por un instante, Jesús muestra su gloria divina, confirmando así la confesión de Pedro. Muestra también que para \textquote{entrar en su gloria} (\textit{Lc} 24, 26), es necesario pasar por la Cruz en Jerusalén. Moisés y Elías habían visto la gloria de Dios en la Montaña; la Ley y los profetas habían anunciado los sufrimientos del Mesías (cf. \textit{Lc} 24, 27). La Pasión de Jesús es la voluntad por excelencia del Padre: el Hijo actúa como siervo de Dios (cf. \textit{Is} 42, 1). La nube indica la presencia del Espíritu Santo: \textit{Tota Trinitas apparuit: Pater in voce; Filius in homine, Spiritus in nube clara} – \textquote{Apareció toda la Trinidad: el Padre en la voz, el Hijo en el hombre, el Espíritu en la nube luminosa} (Santo Tomás de Aquino, \textit{S. th.} 3, q. 45, a. 4, ad 2):

\ccecite{\textquote{En el monte te transfiguraste, Cristo Dios, y tus discípulos contemplaron tu gloria, en cuanto podían comprenderla. Así, cuando te viesen crucificado, entenderían que padecías libremente, y anunciarían al mundo que tú eres en verdad el resplandor del Padre} (\textit{Liturgia bizantina, Himno Breve de la festividad de la Transfiguración del Señor}).}

\n{556} En el umbral de la vida pública se sitúa el Bautismo; en el de la Pascua, la Transfiguración. Por el bautismo de Jesús \textquote{fue manifestado el misterio de la primera regeneración}: nuestro Bautismo; la Transfiguración \textquote{es el sacramento de la segunda regeneración}: nuestra propia resurrección (Santo Tomás de Aquino, \textit{S.Th.}, 3, q. 45, a. 4, ad 2). Desde ahora nosotros participamos en la Resurrección del Señor por el Espíritu Santo que actúa en los sacramentos del Cuerpo de Cristo. La Transfiguración nos concede una visión anticipada de la gloriosa venida de Cristo \textquote{el cual transfigurará este miserable cuerpo nuestro en un cuerpo glorioso como el suyo} (\textit{Flp} 3, 21). Pero ella nos recuerda también que \textquote{es necesario que pasemos por muchas tribulaciones para entrar en el Reino de Dios} (\textit{Hch} 14, 22):

\ccecite{\textquote{Pedro no había comprendido eso cuando deseaba vivir con Cristo en la montaña (cf. \textit{Lc} 9, 33). Te ha reservado eso, oh Pedro, para después de la muerte. Pero ahora, él mismo dice: Desciende para penar en la tierra, para servir en la tierra, para ser despreciado y crucificado en la tierra. La Vida desciende para hacerse matar; el Pan desciende para tener hambre; el Camino desciende para fatigarse andando; la Fuente desciende para sentir la sed; y tú, ¿vas a negarte a sufrir?} (San Agustín, \textit{Sermo}, 78, 6: PL 38, 492-493).}

\n{568} \textit{La Transfiguración de Cristo tiene por finalidad fortalecer la fe de los apóstoles ante la proximidad de la Pasión: la subida a un \textquote{monte alto} prepara la subida al Calvario. Cristo, Cabeza de la Iglesia, manifiesta lo que su cuerpo contiene e irradia en los sacramentos: \textquote{la esperanza de la gloria}} (\textit{Col} 1, 27). (cf. San León Magno, \textit{Sermo} 51, 3: PL 54, 310C). 
\end{ccebody}

\cceth{La obediencia de Abrahán} 
\cceref{CEC 59, 145-146, 2570-2572}

\begin{ccebody}
\ccesec{Dios elige a Abraham}

\n{59} Para reunir a la humanidad dispersa, Dios elige a Abram llamándolo \textquote{fuera de su tierra, de su patria y de su casa} (\textit{Gn} 12,1), para hacer de él \textquote{Abraham}, es decir, \textquote{el padre de una multitud de naciones} (\textit{Gn} 17,5): \textquote{En ti serán benditas todas las naciones de la tierra} (\textit{Gn} 12,3; cf. \textit{Ga} 3,8).

\ccesec{Abraham, \textquote{padre de todos los creyentes}}

\n{145} La carta a los Hebreos, en el gran elogio de la fe de los antepasados, insiste particularmente en la fe de Abraham: \textquote{Por la fe, Abraham obedeció y salió para el lugar que había de recibir en herencia, y salió sin saber a dónde iba} (\textit{Hb} 11,8; cf. \textit{Gn} 12,1-4). Por la fe, vivió como extranjero y peregrino en la Tierra prometida (cf. \textit{Gn} 23,4). Por la fe, a Sara se le otorgó el concebir al hijo de la promesa. Por la fe, finalmente, Abraham ofreció a su hijo único en sacrificio (cf. \textit{Hb} 11,17).

\n{146} Abraham realiza así la definición de la fe dada por la carta a los Hebreos: \textquote{La fe es garantía de lo que se espera; la prueba de las realidades que no se ven} (\textit{Hb} 11,1). \textquote{Creyó Abraham en Dios y le fue reputado como justicia} (\textit{Rm} 4,3; cf. \textit{Gn} 15,6). Y por eso, fortalecido por su fe, Abraham fue hecho \textquote{padre de todos los creyentes} (\textit{Rm} 4,11.18; cf. \textit{Gn} 15, 5).

\ccesec{La Promesa y la oración de la fe}

\n{2570} Cuando Dios lo llama, Abraham se pone en camino \textquote{como se lo había dicho el Señor} (\textit{Gn} 12, 4): todo su corazón \textquote{se somete a la Palabra} y obedece. La escucha del corazón a Dios que llama es esencial a la oración, las palabras tienen un valor relativo. Por eso, la oración de Abraham se expresa primeramente con hechos: hombre de silencio, en cada etapa construye un altar al Señor. Solamente más tarde aparece su primera oración con palabras: una queja velada recordando a Dios sus promesas que no parecen cumplirse (cf. \textit{Gn} 15, 2-3). De este modo surge desde los comienzos uno de los aspectos de la tensión dramática de la oración: la prueba de la fe en Dios que es fiel.

\n{2571} Habiendo creído en Dios (cf. \textit{Gn} 15, 6), marchando en su presencia y en alianza con él (cf. \textit{Gn} 17, 2), el patriarca está dispuesto a acoger en su tienda al Huésped misterioso: es la admirable hospitalidad de Mambré, preludio a la anunciación del verdadero Hijo de la promesa (cf. \textit{Gn} 18, 1-15; \textit{Lc} 1, 26-38). Desde entonces, habiéndole confiado Dios su plan, el corazón de Abraham está en consonancia con la compasión de su Señor hacia los hombres y se atreve a interceder por ellos con una audaz confianza (cf. \textit{Gn} 18, 16-33).

\n{2572} Como última purificación de su fe, se le pide al \textquote{que había recibido las promesas} (\textit{Hb} 11, 17) que sacrifique al hijo que Dios le ha dado. Su fe no vacila: \textquote{Dios proveerá el cordero para el holocausto} (\textit{Gn} 22, 8), \textquote{pensaba que poderoso era Dios aun para resucitar a los muertos} (\textit{Hb} 11, 19). Así, el padre de los creyentes se hace semejante al Padre que no perdonará a su propio Hijo, sino que lo entregará por todos nosotros (cf. \textit{Rm} 8, 32). La oración restablece al hombre en la semejanza con Dios y le hace participar en la potencia del amor de Dios que salva a la multitud (cf. \textit{Rm} 4, 16-21).
\end{ccebody}

\cceth{Las características de la fe} 
\cceref{CEC 153-159}

\begin{ccebody}
\ccesec{La fe es una gracia}

\n{153} Cuando san Pedro confiesa que Jesús es el Cristo, el Hijo de Dios vivo, Jesús le declara que esta revelación no le ha venido \textquote{de la carne y de la sangre, sino de mi Padre que está en los cielos} (\textit{Mt} 16,17; cf. \textit{Ga} 1,15; \textit{Mt} 11,25). La fe es un don de Dios, una virtud sobrenatural infundida por Él. \textquote{Para dar esta respuesta de la fe es necesaria la gracia de Dios, que se adelanta y nos ayuda, junto con los auxilios interiores del Espíritu Santo, que mueve el corazón, lo dirige a Dios, abre los ojos del espíritu y concede \textquote{a todos gusto en aceptar y creer la verdad}} (DV 5).

\ccesec{La fe es un acto humano}

\n{154} Sólo es posible creer por la gracia y los auxilios interiores del Espíritu Santo. Pero no es menos cierto que creer es un acto auténticamente humano. No es contrario ni a la libertad ni a la inteligencia del hombre depositar la confianza en Dios y adherirse a las verdades por Él reveladas. Ya en las relaciones humanas no es contrario a nuestra propia dignidad creer lo que otras personas nos dicen sobre ellas mismas y sobre sus intenciones, y prestar confianza a sus promesas (como, por ejemplo, cuando un hombre y una mujer se casan), para entrar así en comunión mutua. Por ello, es todavía menos contrario a nuestra dignidad \textquote{presentar por la fe la sumisión plena de nuestra inteligencia y de nuestra voluntad al Dios que revela} (Concilio Vaticano I: DS 3008) y entrar así en comunión íntima con Él.

\n{155} En la fe, la inteligencia y la voluntad humanas cooperan con la gracia divina: \textquote{Creer es un acto del entendimiento que asiente a la verdad divina por imperio de la voluntad movida por Dios mediante la gracia} (Santo Tomás de Aquino, \textit{S. Th.}, 2-2, q. 2 a. 9; cf. Concilio Vaticano I: DS 3010).

\newpage
\ccesec{La fe y la inteligencia}

\n{156} El \textit{motivo} de creer no radica en el hecho de que las verdades reveladas aparezcan como verdaderas e inteligibles a la luz de nuestra razón natural. Creemos \textquote{a causa de la autoridad de Dios mismo que revela y que no puede engañarse ni engañarnos}. \textquote{Sin embargo, para que el homenaje de nuestra fe fuese conforme a la razón, Dios ha querido que los auxilios interiores del Espíritu Santo vayan acompañados de las pruebas exteriores de su revelación} (\textit{ibíd.}, DS 3009). Los milagros de Cristo y de los santos (cf. \textit{Mc} 16,20; \textit{Hch} 2,4), las profecías, la propagación y la santidad de la Iglesia, su fecundidad y su estabilidad \textquote{son signos certísimos de la Revelación divina, adaptados a la inteligencia de todos}, motivos de credibilidad que muestran que \textquote{el asentimiento de la fe no es en modo alguno un movimiento ciego del espíritu} (Concilio Vaticano I: DS 3008-3010).

\n{157} La fe es \textit{cierta}, más cierta que todo conocimiento humano, porque se funda en la Palabra misma de Dios, que no puede mentir. Ciertamente las verdades reveladas pueden parecer oscuras a la razón y a la experiencia humanas, pero \textquote{la certeza que da la luz divina es mayor que la que da la luz de la razón natural} (Santo Tomás de Aquino, \textit{S. Th.}, 2-2, q. 171, a. 5, 3). \textquote{Diez mil dificultades no hacen una sola duda} (J. H. Newman, \textit{Apologia pro vita sua,} c. 5).

\n{158} \textquote{La fe \textit{trata de comprender}} (San Anselmo de Canterbury, \textit{Proslogion}, \hyphenation{proemium}: PL 153, 225A) es inherente a la fe que el creyente desee conocer mejor a aquel en quien ha puesto su fe, y comprender mejor lo que le ha sido revelado; un conocimiento más penetrante suscitará a su vez una fe mayor, cada vez más encendida de amor. La gracia de la fe abre \textquote{los ojos del corazón} (\textit{Ef} 1,18) para una inteligencia viva de los contenidos de la Revelación, es decir, del conjunto del designio de Dios y de los misterios de la fe, de su conexión entre sí y con Cristo, centro del Misterio revelado. Ahora bien, \textquote{para que la inteligencia de la Revelación sea más profunda, el mismo Espíritu Santo perfecciona constantemente la fe por medio de sus dones} (DV 5). Así, según el adagio de san Agustín (\textit{Sermo} 43, 7.9: PL 38, 258), \textquote{creo para comprender y comprendo para creer mejor}.

\n{159} \textit{Fe y ciencia}. \textquote{A pesar de que la fe esté por encima de la razón, jamás puede haber contradicción entre ellas. Puesto que el mismo Dios que revela los misterios e infunde la fe otorga al espíritu humano la luz de la razón, Dios no puede negarse a sí mismo ni lo verdadero contradecir jamás a lo verdadero} (Concilio Vaticano I: DS 3017). \textquote{Por eso, la investigación metódica en todas las disciplinas, si se procede de un modo realmente científico y según las normas morales, nunca estará realmente en oposición con la fe, porque las realidades profanas y las realidades de fe tienen su origen en el mismo Dios. Más aún, quien con espíritu humilde y ánimo constante se esfuerza por escrutar lo escondido de las cosas, aun sin saberlo, está como guiado por la mano de Dios, que, sosteniendo todas las cosas, hace que sean lo que son} (GS 36,2).
\end{ccebody}

\newpage
\cceth{Dios manifiesta su Gloria para revelarnos su voluntad} 
\cceref{CEC 2059}

\begin{ccebody}
\n{2059} Las \textquote{diez palabras} son pronunciadas por Dios dentro de una teofanía (\textquote{el Señor os habló cara a cara en la montaña, en medio del fuego}: \textit{Dt} 5, 4). Pertenecen a la revelación que Dios hace de sí mismo y de su gloria. El don de los mandamientos es don de Dios y de su santa voluntad. Dando a conocer su voluntad, Dios se revela a su pueblo.
\end{ccebody}

\cceth{Cristo es para todos nosotros} 
\cceref{CEC 603, 1373, 2634, 2852}

\begin{ccebody}
\n{603} Jesús no conoció la reprobación como si él mismo hubiese pecado (cf. \textit{Jn} 8, 46). Pero, en el amor redentor que le unía siempre al Padre (cf. \textit{Jn} 8, 29), nos asumió desde el alejamiento con relación a Dios por nuestro pecado hasta el punto de poder decir en nuestro nombre en la cruz: \textquote{Dios mío, Dios mío, ¿por qué me has abandonado?} (\textit{Mc} 15, 34; \textit{Sal} 22,2). Al haberle hecho así solidario con nosotros, pecadores, \textquote{Dios no perdonó ni a su propio Hijo, antes bien le entregó por todos nosotros} (\textit{Rm} 8, 32) para que fuéramos \textquote{reconciliados con Dios por la muerte de su Hijo} (\textit{Rm} 5, 10).

\ccesec{La presencia de Cristo por el poder de su Palabra y del Espíritu Santo}

\n{1373} \textquote{Cristo Jesús que murió, resucitó, que está a la derecha de Dios e intercede por nosotros} (\textit{Rm} 8,34), está presente de múltiples maneras en su Iglesia (cf. LG 48): en su Palabra, en la oración de su Iglesia, \textquote{allí donde dos o tres estén reunidos en mi nombre} (\textit{Mt} 18,20), en los pobres, los enfermos, los presos (\textit{Mt} 25,31-46), en los sacramentos de los que Él es autor, en el sacrificio de la misa y en la persona del ministro. Pero, \textquote{\textit{sobre todo}, (está presente) \textit{bajo las especies eucarísticas}} (SC 7).

\ccesec{La oración de intercesión}

\n{2634} La intercesión es una oración de petición que nos conforma muy de cerca con la oración de Jesús. Él es el único intercesor ante el Padre en favor de todos los hombres, de los pecadores en particular (cf. \textit{Rm} 8, 34; \textit{1 Jn} 2, 1; \textit{1 Tm} 2, 5-8). Es capaz de \textquote{salvar perfectamente a los que por Él se llegan a Dios, ya que está siempre vivo para interceder en su favor} (\textit{Hb} 7, 25). El propio Espíritu Santo \textquote{intercede por nosotros [\ldots] y su intercesión a favor de los santos es según Dios} (\textit{Rm} 8, 26-27).

\n{2852} \textquote{Homicida [\ldots] desde el principio [\ldots] mentiroso y padre de la mentira} (\textit{Jn} 8, 44), \textquote{Satanás, el seductor del mundo entero} (\textit{Ap} 12, 9), es aquél por medio del cual el pecado y la muerte entraron en el mundo y, por cuya definitiva derrota toda la creación entera será \textquote{liberada del pecado y de la muerte} (\textit{Plegaria Eucarística IV}, 123: \textit{Misal Romano}). \textquote{Sabemos que todo el que ha nacido de Dios no peca, sino que el Engendrado de Dios le guarda y el Maligno no llega a tocarle. Sabemos que somos de Dios y que el mundo entero yace en poder del Maligno} (\textit{1 Jn} 5, 18-19):

\newpage 
\ccecite{\textquote{El Señor que ha borrado vuestro pecado y perdonado vuestras faltas también os protege y os guarda contra las astucias del Diablo que os combate para que el enemigo, que tiene la costumbre de engendrar la falta, no os sorprenda. Quien confía en Dios, no tema al demonio. \textquote{Si Dios está con nosotros, ¿quién estará contra nosotros?}(\textit{Rm} 8, 31)} (San Ambrosio, \textit{De sacramentis}, 5, 30).}
\end{ccebody}

\begin{patercite}
[\ldots] Tiempo de cuaresma, oh Señor: No permitáis que acudamos a las cisternas agrietadas (\textit{Jer} 2,13), ni que imitemos al siervo infiel, a la virgen necia; no permitáis que el goce de los bienes de la Tierra haga insensible nuestro corazón al lamento de los pobres, de los enfermos, de los niños huérfanos y de los innumerables hermanos nuestros que todavía hoy carecen del mínimo necesario para comer, para cubrir los desnudos miembros, para reunir la familia bajo un mismo techo.	

Las aguas del Jordán descendieron también sobre Vos, oh Jesús, a la vista de la multitud; pero fueron pocos entonces los que pudieron reconoceros; y este misterio de lenta fe o de indiferencia, prolongándose en los siglos, es siempre un motivo de dolor para los que os aman y han recibido la misión de haceros conocer por el mundo. Conceded a los sucesores de los Apóstoles y de los discípulos y cuantos reciben su nombre de Vos y vuestra cruz que puedan llevar adelante la empresa evangelizadora y sostenerla con la oración, con el sufrimiento, y con la íntima fidelidad a vuestra voluntad.

Y así como Vos, inocente Cordero, os presentasteis a Juan en actitud de pecador, atraednos también a nosotros, oh Jesús, a las aguas del Jordán. A ellas queremos acudir para confesar nuestros pecados y purificar nuestras almas. Y así como los cielos abiertos dejaron oír la voz de Vuestro Padre que en Vos, oh Jesús, se complacía, así también superada victoriosamente la prueba, vivido austeramente el período cuaresmal, en los albores de Vuestra resurrección, podamos volver a oír en lo íntimo de nuestras almas la misma voz del Padre celestial, que en nosotros reconoce a sus hijos.

¡Oh santa cuaresma (\ldots)! Ascienda esta oración, en este atardecer de sereno recogimiento, desde cada una de las casas donde se trabaja, se ama y se sufre. Que los ángeles del cielo recojan el suspiro de tantas almas de pequeños inocentes, de jóvenes generosos, de padres de familia trabajadores y sacrificados, y de cuantos sufren en el cuerpo y en el espíritu, para presentarlos luego a Dios. Desde allí descenderán abundantes los dones de las consolaciones celestiales, de los que quiere ser prenda y reflejo nuestra bendición apostólica.

\textbf{San Juan XXIII, papa}, \textit{Radiomensaje}, a todos los fieles con ocasión del inicio de la Cuaresma,  Miércoles de Ceniza, 27 de febrero de 1963, parr. 16-21.
\end{patercite}	
	%\chapter{Domingo III de Cuaresma (B)}

\section{Lecturas}

\rtitle{PRIMERA LECTURA (forma larga)}

\rbook{Del libro del Éxodo} \rred{20, 1-17}

\rtheme{La Ley se dio por medio de Moisés}

\begin{scripture}
En aquellos días, el Señor pronunció estas palabras:

«Yo soy el Señor, tu Dios, que te saqué de la tierra de Egipto, de la casa de esclavitud.

No tendrás otros dioses frente a mí.

No te fabricarás ídolos, ni figura alguna de lo que hay arriba en el cielo, abajo en la tierra, o en el agua debajo de la tierra.

No te postrarás ante ellos, ni les darás culto; porque yo, el Señor, tu Dios, soy un Dios celoso, que castigo el pecado de los padres en los hijos, hasta la tercera y la cuarta generación de los que me odian.

Pero tengo misericordia por mil generaciones de los que me aman y guardan mis preceptos.

No pronunciarás el nombre del Señor, tu Dios, en falso. Porque no dejará el Señor impune a quien pronuncie su nombre en falso.
        
Recuerda el día del sábado para santificarlo. 

Durante seis días trabajarás y harás todas tus tareas, pero el día séptimo es día de descanso, consagrado al Señor, tu Dios. No harás trabajo alguno, ni tú, ni tu hijo, ni tu hija, ni tu esclavo, ni tu esclava, ni tu ganado, ni el emigrante que reside en tus ciudades. Porque en seis días hizo el Señor el cielo, la tierra, el mar y lo que hay en ellos; y el séptimo día descansó. Por eso bendijo el Señor el sábado y lo santificó.

Honra a tu padre y a tu madre, para que se prolonguen tus días en la tierra, que el Señor, tu Dios, te va a dar.

No matarás.

No cometerás adulterio.

No robarás.

No darás falso testimonio contra tu prójimo.

No codiciarás los bienes de tu prójimo. No codiciarás la mujer de tu prójimo, ni su esclavo, ni su esclava, ni su buey, ni su asno, ni nada que sea de tu prójimo».
\end{scripture}


\rtitle{PRIMERA LECTURA (forma breve)}

\rbook{Del libro del Éxodo} \rred{20, 1-3. 7-8. 12-17}

\rtheme{La ley se dio por medio de Moisés}

\begin{scripture}
	En aquellos días, el Señor pronunció estas palabras:
	
	«Yo soy el Señor, tu Dios, que te saqué de la tierra de Egipto, de la casa de esclavitud.
	
	No tendrás otros dioses frente a mí.
	
	No pronunciarás el nombre del Señor, tu Dios, en falso. Porque no dejará el Señor impune a quien pronuncie su nombre en falso.
	
	Recuerda el día del sábado para santificarlo. 
	
	Honra a tu padre y a tu madre, para que se prolonguen tus días en la tierra, que el Señor, tu Dios, te va a dar.
	
	No matarás.
	
	No cometerás adulterio.
	
	No robarás.
	
	No darás falso testimonio contra tu prójimo.
	
	No codiciarás los bienes de tu prójimo. No codiciarás la mujer de tu prójimo, ni su esclavo, ni su esclava, ni su buey, ni su asno, ni nada que sea de tu prójimo».
\end{scripture}

\img{decalog}

\newpage
\rtitle{SALMO RESPONSORIAL}

\rbook{Salmo} \rred{18, 8. 9. 10. 11}

\rtheme{Señor, tú tienes palabras de vida eterna}

\begin{psbody}
La ley del Señor es perfecta
y es descanso del alma;
el precepto del Señor es fiel
e instruye a los ignorantes. 

Los mandatos del Señor son rectos
y alegran el corazón;
la norma del Señor es límpida
y da luz a los ojos. 

El temor del Señor es puro
y eternamente estable;
los mandamientos del Señor son verdaderos
y enteramente justos. 

Más preciosos que el oro,
más que el oro fino;
más dulces que la miel
de un panal que destila. 
\end{psbody}


\rtitle{SEGUNDA LECTURA}

\rbook{De la primera carta del apóstol san Pablo a los Corintios} \rred{1, 22-25}

\rtheme{Predicamos a Cristo crucificado: escándalo para los hombres;
pero para los llamados es sabiduría de Dios}

\begin{scripture}
Hermanos:

Los judíos exigen signos, los griegos buscan sabiduría; pero nosotros predicamos a Cristo crucificado: escándalo para los judíos, necedad para los gentiles; pero para los llamados –judíos o griegos–, un Cristo que es fuerza de Dios y sabiduría de Dios.

Pues lo necio de Dios es más sabio que los hombres; y lo débil de Dios es más fuerte que los hombres.
\end{scripture}

\newpage
\rtitle{EVANGELIO}

\rbook{Del Evangelio según san Juan} \rred{2, 13-25}

\rtheme{Destruid este templo, y en tres días lo levantaré}

\begin{scripture}
Se acercaba la Pascua de los judíos y Jesús subió a Jerusalén. Y encontró en el templo a los vendedores de bueyes, ovejas y palomas, y a los cambistas sentados; y, haciendo un azote de cordeles, los echó a todos del templo, ovejas y bueyes; y a los cambistas les esparció las monedas y les volcó las mesas; y a los que vendían palomas les dijo:

\>{Quitad esto de aquí: no convirtáis en un mercado la casa de mi Padre}.


Sus discípulos se acordaron de lo que está escrito: \textquote{El celo de tu casa me devora}.

Entonces intervinieron los judíos y le preguntaron:

\>{¿Qué signos nos muestras para obrar así?}.

Jesús contestó:

\>{Destruid este templo, y en tres días lo levantaré}.

Los judíos replicaron:

\>{Cuarenta y seis años ha costado construir este templo, ¿y tú lo vas a levantar en tres días?}.

Pero él hablaba del templo de su cuerpo.

Y cuando resucitó de entre los muertos, los discípulos se acordaron de que lo había dicho, y creyeron a la Escritura y a la Palabra que había dicho Jesús.

Mientras estaba en Jerusalén por las fiestas de Pascua, muchos creyeron en su nombre, viendo los signos que hacía; pero Jesús no se confiaba a ellos, porque los conocía a todos y no necesitaba el testimonio de nadie sobre un hombre, porque él sabía lo que hay dentro de cada hombre.
\end{scripture}

\img{cross_romanesque}

\newsection
\section{Comentario Patrístico}

\subsection{San Agustín, obispo}

\ptheme{Somos las piedras vivas con las que se edifica el templo de Dios}
 
\src{Comentario sobre el salmo 130, nn. 1-3: \\CCL 40, 1198-1200.}

\begin{body}
\ltr{C}{on} frecuencia hemos advertido a vuestra Caridad que no hay que considerar los salmos como la voz aislada de un hombre que canta, sino como la voz de todos aquellos que están en el Cuerpo de Cristo. Y como en el Cuerpo de Cristo están todos, habla como un solo hombre, pues él es a la vez uno y muchos. Son muchos considerados aisladamente; son uno en aquel que es uno. Él es también el templo de Dios, del que dice el Apóstol: \textit{El templo de Dios es santo: ese templo sois vosotros}: todos los que creen en Cristo y creyendo, aman. Pues en esto consiste creer en Cristo: en amar a Cristo; no a la manera de los demonios, que creían, pero no amaban. Por eso, a pesar de creer, decían: ¿Qué tenemos nosotros contigo, Hijo de Dios? Nosotros, en cambio, de tal manera creamos que, creyendo en Él, le amemos y no digamos: ¿Qué tenemos nosotros contigo?, sino digamos más bien: \textquote{Te pertenecemos, tú nos has redimido}.

Efectivamente, todos cuantos creen así, son como las piedras vivas con las que se edifica el templo de Dios, y como la madera incorruptible con que se construyó aquella arca que el diluvio no consiguió sumergir. Este es el templo –esto es, los mismos hombres– en que se ruega a Dios y Dios escucha. Sólo al que ora en el templo de Dios se le concede ser escuchado para la vida eterna. Y ora en el templo de Dios el que ora en la paz de la Iglesia, en la unidad del cuerpo de Cristo. Este Cuerpo de Cristo consta de una multitud de creyentes esparcidos por todo el mundo; y por eso es escuchado el que ora en el templo. Ora, pues, en espíritu y en verdad el que ora en la paz de la Iglesia, no en aquel templo que era sólo una figura.

A nivel de figura, el Señor arrojó del templo a los que en el templo buscaban su propio interés, es decir, los que iban al templo a comprar y vender. Ahora bien, si aquel templo era una figura, es evidente que también en el Cuerpo de Cristo –que es el verdadero templo del que el otro era una imagen– existe una mezcolanza de compradores y vendedores, esto es, gente que busca su interés, no el de Jesucristo.

Y puesto que los hombres son vapuleados por sus propios pecados, el Señor hizo un azote de cordeles y arrojó del templo a todos los que buscaban sus intereses, no los de Jesucristo.

Pues bien, la voz de este templo es la que resuena en el salmo. En este templo –y no en el templo material– se ruega a Dios, como os he dicho, y Dios escucha en espíritu y en verdad. Aquel templo era una sombra, figura de lo que había de venir. Por eso aquel templo se derrumbó ya. ¿Quiere decir esto que se derrumbó nuestra casa de oración? De ningún modo. Pues aquel templo que se derrumbó no pudo ser llamado casa de oración, de la que se dijo: \textit{Mi casa es casa de oración, y así la llamarán todos los pueblos}. Y ya habéis oído lo que dice nuestro Señor Jesucristo: \textit{Escrito está: \textquote{Mi casa es casa de oración para todos los pueblos}; pero vosotros la habéis convertido en una \textquote{cueva de bandidos}}.

¿Acaso los que pretendieron convertir la casa de Dios en una cueva de bandidos, consiguieron destruir el templo? Del mismo modo, los que viven mal en la Iglesia católica, en cuanto de ellos depende, quieren convertir la casa de Dios en una cueva de bandidos; pero no por eso destruyen el templo. Pero llegará el día en que, con el azote trenzado con sus pecados, serán arrojados fuera. Por el contrario, este templo de Dios, este Cuerpo de Cristo, esta asamblea de fieles tiene una sola voz y como un solo hombre canta en el salmo. Esta voz la hemos oído en muchos salmos; oigámosla también en éste. Si queremos, es nuestra voz; si queremos, con el oído oímos al cantor, y con el corazón cantamos también nosotros. Pero si no queremos, seremos en aquel templo como los compradores y vendedores, es decir, como los que buscan sus propios intereses: entramos, sí, en la Iglesia, pero no para hacer lo que agrada a los ojos de Dios.
\end{body}


\newsection
\section{Homilías}

\subsection{San Juan Pablo II, papa}

\subsubsection{Homilía (1979): Tengamos celo por la casa}

\src{Visita Pastoral a la Parroquia Romana de \\San José en el Barrio \textquote{Forte Boccea}. \\18 de marzo de 1979.}

\begin{body}
1. \textquote{La casa de mi Padre}.

\ltr{H}{oy} Cristo pronuncia estas palabras en el umbral del templo de Jerusalén. Se presenta sobre este umbral para \textquote{reivindicar} frente a los hombres la casa de su Padre, para reclamar sus derechos sobre esta casa. Los hombres hicieron de ella una plaza de mercado. Cristo los reprende severamente; se pone decididamente contra tales desviaciones. El celo por la casa de Dios lo devora (cf. \textit{Jn} 2, 17), por esto Él no duda en exponerse a la malevolencia de los ancianos del pueblo judío y de todos los que son responsables de lo que se ha hecho contra la casa de su Padre, contra el templo.

Es memorable este acontecimiento. Memorable la escena. Cristo, con las palabras de su ira santa, ha inscrito profundamente en la tradición de la Iglesia la ley de la santidad de la casa de Dios. Pronunciando esas palabras misteriosas que se referían al templo de su cuerpo: \textquote{Destruid este templo, y en tres días lo levantaré} (\textit{Jn} 2, 19), Jesús ha consagrado de una sola vez todos los templos del Pueblo de Dios. Estas palabras adquieren una riqueza de significado totalmente particular en el tiempo de Cuaresma cuando, meditando la pasión de Cristo y su muerte –destrucción del templo de su cuerpo–, nos preparamos a la solemnidad de la Pascua, esto es, al momento en que Jesús se nos revelará todavía en el templo mismo de su cuerpo, levantado de nuevo por el poder de Dios, que quiere construir en él, de generación en generación, el edificio espiritual de la nueva fe, esperanza y caridad [\ldots].

4. La casa es la morada del hombre. Es una condición necesaria para que el hombre pueda venir al mundo, crecer, desarrollarse, para que pueda trabajar, educar, y educarse, para que los hombres puedan constituir esa unión más profunda y más fundamental que se llama \textquote{familia}.

Se construyen las casas para las familias. Después, las mismas familias se construyen en las casas sobre la verdad y el amor. El fundamento primero de esta construcción es la alianza matrimonial, que se expresa en las palabras del sacramento con las que el esposo y la esposa se prometen recíprocamente la unión, el amor, la fidelidad conyugal. Sobre este fundamento se apoya ese edificio espiritual, cuya construcción no puede cesar nunca. Los cónyuges, como padres, deben aplicar constantemente a la propia vida de constructores sabios, la medida de la unión, del amor, de la honestidad y de la fidelidad matrimonial. Deben renovar cada día esa promesa en sus corazones y a veces recordarla también con las palabras. Hoy, [con ocasión de esta visita pastoral,] yo os invito a hacerlo de modo particular, [porque la visita pastoral] debe servir para la renovación de ese templo que formamos todos en Cristo crucificado y resucitado. \textbf{San Pablo} dice que Cristo es \textquote{poder y sabiduría de Dios} (\textit{1 Cor} 1, 24). Sea Él vuestro poder y vuestra sabiduría, queridos esposos y padres. Lo sea para todas las familias de esta parroquia. ¡No os privéis de este poder y de esta sabiduría! Consolidaos en ellos. Educad en ellos a vuestros hijos y no permitáis que este poder y esta sabiduría, que es Cristo, les sea quitado un día. Por ningún ambiente y por ninguna institución. No permitáis que alguien pueda destruir ese \textquote{templo} que vosotros construís en vuestros hijos. Este es vuestro deber, pero éste es también vuestro sacrosanto derecho. Y es un derecho que nadie puede violar sin cometer una arbitrariedad.

5. La familia está construida sobre la sabiduría y el poder del mismo Cristo, porque se apoya sobre un sacramento. Y está construida también y se construye constantemente sobre la ley divina, que no puede ser sustituida en modo alguno por cualquier otra ley. ¿Acaso puede un legislador humano abolir los mandamientos que nos recuerda hoy la lectura del \textbf{libro del Éxodo}: \textquote{No matar, no cometer adulterio, no robar, no decir falsos testimonios} (\textit{Ex} 20, 13-16)? Todos sabemos de memoria el Decálogo. Los diez mandamientos constituyen la concatenación necesaria de la vida humana personal, familiar, social. Si falta esta concatenación, la vida del hombre se hace inhumana. Por esto el deber fundamental de la familia, y después de la escuela, y de todas las instituciones, es la educación y consolidación de la vida humana sobre el fundamento de esta ley, que a nadie es lícito violar.

Así estamos construyendo con Cristo el templo de la vida humana, en el que habita Dios. Construyamos en nosotros la casa del Padre. Que el celo por la construcción de esta casa constituya el núcleo de la vida de todos nosotros aquí presentes (\ldots).
\end{body}

\newpage
\subsubsection{Homilía (1982): Lo que hay en el hombre}

\src{Visita Pastoral a la Parroquia Romana del Santísimo Crucifijo. \\14 de marzo de 1982.}

\begin{body}
\ltr[1. \ldots «]{N}{osotros} predicamos a Cristo crucificado» (\textit{1 Cor} 1, 23). En estas palabras de la \textbf{carta a los Corintios}, Pablo de Tarso pronuncia su mensaje apostólico. \textquote{Predicamos a Cristo crucificado}, que es \textquote{poder y sabiduría de Dios} (\textit{1 Cor} 1, 24). Este mensaje encuentra oposición: para los judíos, que piden milagros, Cristo crucificado es un \textquote{escándalo}; para los griegos, que buscan sabiduría, es \textquote{necedad}. Pablo de Tarso es consciente de la oposición que encuentra su mensaje a los ojos de sus contemporáneos. Sin embargo, lo anuncia con una fuerza mucho mayor, que es la fe: \textquote{Lo que es la locura de Dios es más sabio que los hombres, y lo que es la debilidad de Dios es más fuerte que los hombres} (\textit{1 Cor} 1, 25).

\txtsmall{[Hoy vengo a visitar la parroquia del \textquote{Santísimo Crucifijo}. Lo hago, como obispo de Roma, por amor a vuestra comunidad y con profunda reverencia a Cristo crucificado. ¿No refleja vuestra parroquia, incluso con el mismo nombre, el mensaje de Pablo a los corintios y, por tanto, a todos los cristianos, a todos los hombres? ¡Parroquia del \textquote{Santísimo Crucifijo}!]}

2. \textquote{¡Predicamos a Cristo \ldots}! Este Cristo que conocía y sabe \textquote{lo que hay en cada hombre}. De hecho, en el \textbf{Evangelio} de hoy leemos lo siguiente: \textquote{Mientras él estaba en Jerusalén para la Pascua, durante la fiesta muchos, viendo las señales que estaba haciendo, creyeron en su nombre. Sin embargo, Jesús no se fiaba de ellos porque conocía a todos y no necesitaba que nadie le diera testimonio de otro, más bien sabía lo que hay en cada hombre} (\textit{Jn} 2, 23-25).

Así fue durante la vida terrenal de Jesús. Desde entonces, muchos otros todavía \textquote{creyeron en su nombre}. Aquí en [Roma] muchos creen en Jesucristo. También hay muchos en esta parroquia. Quizás incluso aquellos que aun sin saberlo, creen de alguna manera; incluso aquellos que piensan que no creen. A veces hacemos preguntas a los hombres sobre su fe, incluso se hacen preguntas especiales. Y obtenemos algunas respuestas, ciertamente sinceras.

Sin embargo, en última instancia, sólo él, Cristo, sabe \textquote{lo que hay en cada hombre}. Él sabe esto con la ciencia que solo le pertenece. Ciencia divina y humana, ciencia del Evangelio y de la Redención. Él lo sabe, porque nos ha redimido a cada uno de nosotros. De hecho, fuimos comprados a un precio muy alto (cf. \textit{1 Cor} 6, 20; 7, 23). Y por eso \textquote{predicamos a Cristo crucificado}. Lo predicamos continuamente, sin descanso. También lo predicamos este domingo de Cuaresma, aquí, en esta parroquia.

Es necesario que el hombre, mirándose profundamente en su interior, piense en lo que hay en él; quizás paz de conciencia o quizás inquietud, el peso de los pecados, el peso de una gran responsabilidad, el remordimiento.

Sin embargo, al mismo tiempo, todos deben mirar el Crucifijo y pensar que para él también existe siempre el \textquote{precio alto}. De hecho, ¡a tal precio somos comprados a través de la Cruz!

3. La palabra de este domingo nos recuerda el \textbf{Decálogo}, la ley de Dios dada a Israel a través de Moisés en el monte Sinaí; y dada a todos los hombres. Conocemos estos mandamientos. Muchos los repiten a diario en sus oraciones. ¡El cielo desearía que todos lo hicieran! Es un muy buen hábito. Repitámoslos ahora, como están escritos en el \textbf{libro del Éxodo}, para reconfirmar y renovar lo que recordamos. Los mandamientos fueron dados durante la salida de Israel de Egipto, por obra de Dios; por tanto, las primeras palabras recuerdan este episodio.

\begin{bodyprose}
«Yo soy el Señor, tu Dios, que te saqué de la tierra de Egipto, de la condición de esclavitud:
   No tendrás otros dioses frente a mí \ldots.
   No tomarás el nombre del Señor tu Dios en vano \ldots.

Acuérdate del día de reposo para santificarlo \ldots, 
   aquí decimos: \textit{Acuérdate de santificar las fiestas}.

Honra a tu padre y a tu madre \ldots.
   No mates.

No cometas adulterio.
   No robes.
   No des falso testimonio contra tu prójimo.
   No codicies la casa de tu vecino. 
   No desees a la mujer de tu prójimo, ni a su esclavo, ni a su esclava, 
   ni a su buey, ni a su asno, ni nada que sea de tu prójimo». 
(\textit{Ex} 20, 2-3. 7-8. 17).
\end{bodyprose}

Pronunciamos el último mandamiento con dos fórmulas. El primero: no desear a la mujer de los demás, y el segundo: no desear las cosas de los demás.

¿Fueron todos estos mandamientos grabados solo en las dos tablas que recibió Moisés, e Israel los guardó como la cosa más sagrada en el Arca de la Alianza? ¡No solo! Estos mandamientos están, al mismo tiempo, inscritos en el corazón, en la conciencia de todo hombre.

¿Por qué Dios nos dio a su Hijo Unigénito, como recuerda la liturgia de hoy en el canto al Evangelio? Para que el grabado de los mandamientos divinos no se borre de la conciencia humana; para que el hombre pueda conocer y practicar estos mandamientos, y así tener \textquote{vida eterna}.

A un joven que le pregunta a Jesús: \textquote{¿Qué debo hacer de bueno para obtener la vida eterna?}, El Maestro responde: \textquote{Guarda los mandamientos}. \textquote{¿Cuales?}. Jesús enumera los mismos que recibió Moisés en el monte Sinaí en la alianza antigua (cf. \textit{Mt} 19, 16-22).

4. Jesucristo sabe \textquote{lo que hay en cada hombre}; sabe que los mandamientos del Padre están inscritos en su corazón.

En el \textbf{evangelio} de hoy, Cristo se muestra severo con los que violan el mandamiento del culto y la adoración debidos a Dios mismo: un mandamiento escrito más en la conciencia que en la simple ley.

De hecho, esos vendedores y cambistas quizás estaban de acuerdo con la ley humana, pero Cristo es el que sabe \textquote{lo que hay en cada hombre} y al mismo tiempo lo devora el celo por la casa de Dios (cf. \textit{Jn} 2, 17).

Conduciendo al hombre por la senda de los mandamientos, le enseña no solo a cumplir la ley de Dios, sino también a comprender cada vez mejor y amar esta ley cada vez más profundamente, como afirma el \textbf{Salmo Responsorial} de la Santa Misa.

En la medida en que el hombre comprende los mandamientos divinos, se da cuenta de la gran ayuda que éstos suponen en la vida personal, familiar y social. Son verdaderamente el camino del hombre; son para el hombre.

\begin{bodyprose}
La ley del Señor es perfecta
   y es descanso del alma;
   el precepto del Señor es fiel
   e instruye a los ignorantes.

Los mandatos del Señor son rectos
   y alegran el corazón;
   la norma del Señor es límpida
   y da luz a los ojos.

El temor del Señor es puro
   y eternamente estable;
   los mandamientos del Señor son verdaderos
   y enteramente justos.

Más preciosos que el oro,
   más que el oro fino;
   más dulces que la miel
   de un panal que destila.
(Sal 18 [19], 8-11).
\end{bodyprose}

Valdría la pena detenerse más en estos versículos del Salmo. Entonces veremos mejor cuál es el camino que conduce al amor a los mandamientos divinos, en particular al mayor mandamiento del Evangelio, a ese poder y ese amor divino que Cristo crucificado se ha hecho por nosotros.

¿Acaso no es la Cruz la conciencia suprema de la humanidad? ¿Acaso no es la Cruz la voz misma de Dios hablando con más fuerza que las propias conciencias humanas? ¿Voz que habla de manera particular cuando las diferentes \textquote{medidas humanas} disminuyen esta conciencia y la sofocan? Por tanto, el \textbf{Apóstol} tiene razón cuando clama: \textquote{Predicamos a Cristo crucificado\ldots poder de Dios y sabiduría de Dios}.

5. Meditando sobre la ley divina, sobre la conciencia humana y sobre la cruz de Cristo, la liturgia cuaresmal de hoy nos prepara para el misterio pascual.

Después de la expulsión de los comerciantes y cambistas, algunos judíos se dirigieron a Jesús con esta pregunta: \textquote{¿Qué signo nos muestras para hacer estas cosas? Jesús les respondió: Destruid este templo y en tres días lo levantaré. Entonces los judíos le dijeron: Este templo tardó cuarenta y seis años en construirse, ¿y en tres días lo levantarás? Pero Él hablaba del templo de su cuerpo. Entonces, cuando resucitó de entre los muertos, sus discípulos se acordaron de lo que había dicho, y creyeron en la Escritura y en la palabra de Jesús} (\textit{Jn} 2, 18-22).

6. ¡Queridos hermanos y hermanas! Acoged esta meditación que pronuncio, siguiendo las palabras de la liturgia de hoy, para venerar a Cristo crucificado en la parroquia [del \textquote{Santísimo Crucifijo}].

7. Desde este altar deseo ahora dirigir mi cordial saludo a todos los fieles presentes y a toda la familia parroquial (\ldots) Recuerdo a todos con cariño y ofrezco mis oraciones por todos. (\ldots) 

(\ldots) [Os invito a] profundizar la fe de manera global y exhaustiva para vivirla con coherencia y valentía en la sociedad moderna. Participad en las actividades parroquiales con espíritu de auténtica dedicación, para ser y sentiros cada vez más cristianos convencidos, felices y fervientes, abiertos a la caridad y la ayuda mutua. En particular, me gustaría recomendar la participación en la Santa Misa dominical. Haced el propósito de no fallar nunca. El cristiano es el hombre de la Santa Misa, porque comprende que en ella Cristo renueva su sacrificio redentor por él.

Termino con el más sincero deseo de que en esta parroquia la gente no deje nunca de anunciar a Cristo Crucificado. Que este anuncio sea para toda la comunidad, para cada uno y para todos, \textquote{poder de Dios y sabiduría de Dios} y dé fruto abundante en las conciencias humanas, a pesar de las diversas oposiciones que encuentra en el mundo contemporáneo. De hecho, las encontró no sólo entre los \textquote{judíos} y los \textquote{griegos}, de los que escribe el \textbf{Apóstol}; sino también en el mundo contemporáneo. Pero esto no nos desanima de cara a nuestra misión de anunciar a Cristo, Cristo crucificado.
\end{body}

\newpage
\subsubsection{Homilía (1985): Conversión y amor}

\src{Celebración Eucarística en la Parroquia de \\Nuestra Señora de Bonaria ad Ostia Lido.\\10 de marzo de 1985.}

\begin{body}

\ltr[1. ]{Y}{o} soy el Señor, tu Dios, que te saqué de la tierra de Egipto, de la casa de esclavitud: 

\begin{bodyprose}
No tomarás el nombre del Señor en vano \ldots
   Acuérdate del día de reposo para santificarlo \ldots
   Honra a tu padre y a tu madre \ldots

No mates.
   No cometas adulterio.
   No robes.

No des falso testimonio contra tu prójimo.
   No codicies la casa de tu vecino.
   No desees a la mujer de tu prójimo \ldots ni nada que sea de tu prójimo
(\textit{Ex} 20, 2-3. 7-8. 12-17).
\end{bodyprose}

2. Hoy, tercer domingo de Cuaresma, la Iglesia todavía proclama en este pasaje del \textbf{libro del Éxodo}, que contiene el \textquote{Decálogo}: los diez mandamientos proclamados e impuestos a los hijos e hijas de Israel al pie del monte Sinaí. Ésta es la ley divina que determina los principios fundamentales del comportamiento humano, las principales reglas de la moral, según las cuales las obras humanas adquieren el carácter de bien o mal moral. La observancia de estas normas, de estos mandamientos, imprime el signo del bien en nuestras obras, hace al hombre bueno. La infracción, la transgresión de ellos imprime en nuestras obras el signo del mal: hace malo al hombre. Este bien y este mal conciernen al hombre en su propia humanidad. A través del bien moral, el hombre, como hombre, se vuelve y es bueno. A través del mal moral, el hombre, como hombre, se vuelve y es malo.

Por tanto, es el problema fundamental desde el punto de vista del propio valor esencial del hombre. La ley moral permanece estrictamente relacionada con este valor. Conectada, se puede decir, con la dignidad del hombre, y unida a la dignidad de toda convivencia de los hombres entre sí. La ley moral tiene un significado tanto personal como social. Es Dios quien proclama esta Ley: el Decálogo manifiesta así la Providencia de Dios, su preocupación paterna por el bien fundamental del hombre. Él es el Dios que sacó a los hijos de Israel de la tierra de Egipto, de la condición de esclavitud.

3. Recordemos el Decálogo durante la Cuaresma, porque es el período en el que Jesucristo desafía de manera particular a la conciencia humana. Desde el comienzo de estas jornadas cuaresmales hemos sentido la llamada a la conversión, a la reconciliación con Dios, llamada que encuentra su fundamento objetivo en el Decálogo. Convertirse significa romper con el mal, romper con el pecado, fortalecerse nuevamente en el bien y consolidar el comportamiento en él. Sabemos que Jesucristo ha reconfirmado plenamente los mandamientos divinos del monte Sinaí. Dio instrucciones a los hombres para que los observaran. Indicó que la observancia de los mandamientos es la condición fundamental de la reconciliación con Dios, la condición fundamental para el logro de la salvación eterna.

Por eso la liturgia de hoy proclama: \textquote{¡Señor, tú tienes palabras de vida eterna!} \textquote{La ley del Señor es perfecta\ldots Los mandatos del Señor son rectos, y alegran el corazón\ldots los mandamientos del Señor son verdaderos y enteramente justos. Más preciosos que el oro, más que el oro fino} (\textit{Sal} 19, 8-11). El período de Cuaresma es el tiempo en el que debemos volver a los mandamientos de Dios. A su luz debemos examinar nuestra conciencia, para que no crezca sobre ella una capa de pecado e iniquidad.

4. Jesucristo reconfirma la ley divina de la antigua Alianza, proclamada en el Sinaí. Pero, al mismo tiempo, la misión con la que se dirige a la humanidad va más allá. \textquote{Dios\ldots amó tanto al mundo hasta entregar a su único Hijo; todo el que cree en Él tiene vida eterna} (cf. \textit{Jn} 3, 16). La fe en Cristo es todavía algo más que la pura obediencia a la Ley, aunque sea dictada por el sincero \textquote{temor del Señor} del que habla el salmo (cf. \textit{Sal} 112). Creer en Dios significa afrontar el mismo amor con el que Dios amó al mundo. El amor de Dios se expresa en el hecho de que dio a su único Hijo.

Por eso todo el orden moral de la nueva alianza alcanza su cúspide y su centro en el mandamiento del amor. Hacemos, sí, la voluntad de Dios, observando todos los mandamientos divinos; pero a Dios que se ha revelado en Cristo como amor, ¡solo podemos responder por medio del amor! Por tanto, nuestro examen de conciencia cuaresmal debe centrarse en las exigencias del amor a Dios y al prójimo. Esta es, al mismo tiempo, la principal vía de conversión que Cristo espera de nosotros. Es una conversión constante y continua. Así como debemos orar continuamente, también debemos convertirnos constantemente.

5. Por tanto, la Iglesia, en el período de Cuaresma, no se limita a proclamar la ley divina: el Decálogo del Monte Sinaí. Ella también proclama para nosotros, junto con \textbf{San Pablo}, a Cristo crucificado en el monte Calvario.

Una vez los judíos pidieron milagros y los griegos buscaron sabiduría (cf. \textit{1 Co} 1, 22). Los contemporáneos se comportan como los judíos y griegos de la época apostólica. De hecho, sus solicitudes van mucho más allá. A veces, la crítica y la oposición a la enseñanza de Dios, a los mandamientos divinos es mucho más dura. Sin embargo, al mismo tiempo, la Iglesia permanece fiel a la indicación del apóstol: \textquote{Predicamos a Cristo crucificado} (\textit{1 Co} 1, 23). ¡En él está la respuesta a todo! Toda crítica, toda oposición a la doctrina divina palidece ante la elocuencia de Cristo crucificado. La cruz del Calvario \textquote{es más sabia que los hombres} y \textquote{es más fuerte que los hombres}, como escribe San Pablo (\textit{1 Co} 1, 25).

6. Hoy, tercer domingo de Cuaresma (\ldots), juntos \textquote{predicamos a Cristo crucificado}, y juntos lo profesamos, tal como la Iglesia  lo ha predicado y profesado durante casi dos mil años, heredera de la fe de los santos apóstoles Pedro y Pablo.

[\ldots]

(\ldots) La misa dominical, queridos hermanos y hermanas: esta es la base de todo, y debo pediros que no la descuidéis, que seáis más asiduos a ella, que sintáis, todos los domingos y fiestas, que el Señor viene a vuestro encuentro, os reúne en torno a la mesa doble de la palabra y del cuerpo de Cristo. Sin esta referencia constante a la mesa de Cristo no es posible construir una vida verdaderamente cristiana. Toda la fuerza misionera de una parroquia y toda su capacidad o esperanza para la formación de los jóvenes proviene de esta fuente constante y rica de la presencia de Cristo en la misa dominical.

\txtsmall{[Deseo alentar todos vuestros trabajos, deseo subrayar los resultados significativos de vuestra presencia cristiana en un área que comenzó a formarse con grandes privaciones y sacrificios. Os acordáis bien del primer cuartel, la iglesia en el sótano, la afluencia de gente en dificultad para trabajar, familias y personas casi expulsadas de las afueras de la gran metrópoli Habéis recorrido un largo camino desde entonces; continuemos ahora con todos nuestros esfuerzos para construir juntos la comunidad espiritual de fe, la parroquia del servicio interior a las conciencias y almas, la comunidad educadora en el Espíritu de Cristo, una comunidad de hermanos que irradian el Evangelio, una familia de creyentes y testigos, guiados e iluminados por la presencia bendita y la protección de la Virgen.]}

8. \textquote{Yo soy el Señor, tu Dios, que te saqué de la tierra de Egipto, de la casa de esclavitud}. Dios, que liberó a Israel de Egipto, libera constantemente al hombre: lo libera del pecado. Hacia esta liberación conduce el camino de los mandamientos divinos: el Decálogo y el mandamiento del amor. En este camino encontramos a Cristo que en el \textbf{Evangelio} de hoy dice: \textquote{Destruid este templo y en tres días lo levantaré} (\textit{Jn} 2, 19). Y lo dice \textquote{del templo de su cuerpo} (\textit{Jn} 2, 21), lo dice de la resurrección.

En el período de Cuaresma escudriñamos nuestra conciencia a la luz de los mandamientos de Dios, para librarnos del pecado. Y renovamos en nosotros la esperanza ligada a la resurrección de Cristo, en la que se encierra el principio de la liberación completa del mal, el pecado y la muerte. La liberación de la tierra de Egipto, de la condición de esclavitud, de hecho no estaba lejos de anunciar la liberación, que es compartida por nosotros en Jesucristo.
\end{body}

\newpage
\subsubsection{Homilía (1988): Ídolos contemporáneos}

\src{Visita a la Parroquia de San Dámaso en Monteverde. \\6 de marzo de 1988.}

\begin{body}
1. \textquote{Jesús\ldots sabía lo que hay dentro de cada hombre} (\textit{Jn} 2, 25).

\ltr{L}{a} liturgia del tercer domingo de Cuaresma nos invita a seguir esta \textquote{sabiduría}. La \textquote{sabiduría} de Dios sobre el corazón humano está profundamente inscrita en los acontecimientos del Sinaí que nos refiere el \textbf{libro del Éxodo}. Aquí, habla a Israel, al pueblo elegido, el mismo Dios que \textquote{lo sacó de la tierra de Egipto} (\textit{Ex} 20, 5). Cuando dice: \textquote{No mates. No cometas adulterio. No robes. No des falso testimonio contra tu prójimo. No desees\ldots} (\textit{Ex} 20, 12-17). Dios sabe que en el corazón del hombre se esconde una \textquote{inclinación}, una predisposición a cada uno de estos pecados, a todas las facetas del mal. Incluso una inclinación al crimen. El Dios de nuestros padres sabe todo esto desde el principio, desde el tiempo del árbol del conocimiento del bien y del mal y desde el tiempo del primer pecado. Desde entonces, el hombre, cediendo a la tentación del Maligno, por primera vez se ha creído que podía ser \textquote{como Dios} (cf. \textit{Gn} 3, 5), y ha descendido por la senda del pecado.

2. Sin embargo, en este hombre permanecía una misteriosa necesidad de buscar \textquote{dioses} fuera del único Dios verdadero. El pueblo que estaba al pie del monte Sinaí, aunque elegido por el Dios verdadero, también mostró esta propensión: \textquote{a tener otros dioses} (cf. \textit{Ex} 20, 3). Durante los días en que Moisés se quedó con Dios en el monte Sinaí, recibiendo de él las tablas de la ley divina –o Decálogo– su pueblo también se hizo a sí mismo un \textquote{dios} en forma de \textquote{becerro fundido} (\textit{Ex} 32, 4). Y en esta forma superficial y falseada dio rienda suelta a la perenne necesidad del corazón humano de volver a Dios, poniendo \textquote{un dios de oro} en el lugar del Dios verdadero. Este hecho debe hacernos meditar mucho, cuánto espacio, de hecho, el \textbf{libro del Éxodo} dedica a este problema: \textquote{No tendrás otros dioses fuera de mí. No te fabricarás ídolos ni imagen alguna\ldots (como ese ‘becerro fundido’, por ejemplo)\ldots No te inclinarás ante ellos ni les servirás\ldots} (\textit{Ex} 20, 3-5).

Ese problema, ¿era solo actual en aquellos tiempos lejanos? ¿O no es siempre actual, aunque sea de otras formas? El hombre contemporáneo ciertamente ya no adora a los \textquote{ídolos} como lo hacían los antiguos paganos. Hoy, sin embargo, el hombre hace otra cosa con esa necesidad muy profunda de su ser humano, con la necesidad de \textquote{trascendencia} (como solemos decir hoy). Y aunque no reemplaza materialmente al Dios verdadero con un \textquote{becerro fundido}, hay algún otro \textquote{ídolo} contemporáneo que se traga las energías más profundas de su alma.

A menudo estos \textquote{ídolos} contemporáneos son de naturaleza sutil, conectados con el progreso del pensamiento, con el refinamiento de las propensiones humanas, con el estilo de civilización que exalta un programa de vida que prescinde de Dios: como si él no existiera.

3. Dios, que habla en el \textbf{libro del Éxodo} se llama a sí mismo: \textquote{Dios celoso} (\textit{Ex} 20, 5). ¡Sí! Dios es \textquote{celoso}, de un \textquote{celo} divino por el hombre. Celoso de esta criatura, en la que imprimió su imagen y semejanza desde el principio, y en cuya forma corporal inspiró el alma inmortal. ¡Sí! Dios es \textquote{celoso} de lo que existe de él en el hombre, y que no puede satisfacerse de otra manera que sólo en él y para él. \textquote{No tendrás otros dioses frente a mí\ldots Amarás al Señor tu Dios con todo tu corazón, con toda tu alma y con todas tus fuerzas\ldots}. De lo contrario, tú, hombre, ¡no te encontrarás a ti mismo! ¡Te perderás! ¡Sí! Dios es \textquote{celoso} del hombre así como Cristo estaba \textquote{celoso} de la santidad de la casa de Dios en Jerusalén. El \textbf{evangelio} de hoy nos recuerda esto: \textquote{No hagáis de la casa de mi Padre un mercado} (\textit{Jn} 2, 16). Entonces \textquote{recordaron los discípulos lo que está escrito: \textquote{el celo por tu casa me devora}} (\textit{Jn} 2, 17). Cristo expulsó a los comerciantes del templo, así como Moisés, al pie del Sinaí, había \textquote{disipado} a los idólatras.

4. El mensaje central de este domingo de Cuaresma nos manda a seguir esta \textquote{sabiduría} de Dios sobre el hombre, que se ha revelado plenamente en Cristo.

¿Quién es el Dios \textquote{celoso}? ¿Celoso de los \textquote{celos} divinos? Es ese Dios que amó al mundo. Con amor eterno amó al hombre en el mundo. Y sabiendo \textquote{lo que hay en cada hombre} y de lo que es capaz su corazón dividido en el conocimiento del bien y del mal, este Dios \textquote{ha dado a su Hijo unigénito}. El don del Hijo de la misma sustancia que el Padre es la vara de medir del amor de Dios por el mundo: ¡por el hombre que está en el mundo! Sólo en este Hijo, sólo por él, puede el hombre alcanzar la vida eterna. Y tenerla. Sólo este Dios y nadie más ha inscrito en las profundidades del alma humana la inmortalidad, llamándola a la existencia.

Dios da a la humanidad a su Hijo consustancial , como redentor del mundo, porque conoce plenamente \textquote{lo que hay en cada hombre}. Él solo. Porque solo Él es el creador del hombre. Y Él es un amante del hombre (\textquote{filo-anthropos}).

5. El apóstol Pablo es plenamente consciente de esta \textquote{sabiduría} de Dios, y de este misterio divino que se reveló plenamente en Cristo: crucificado y resucitado. Cristo crucificado: el que se puso en el lugar del templo de Jerusalén cuando dijo \textquote{destruid este templo y en tres días lo levantaré} (\textit{Jn} 2, 19). Habló de su muerte y resurrección al tercer día. Pablo, aún siendo un enemigo acérrimo, se encontró con el Resucitado cerca de Damasco y, a la luz de la resurrección, creyó en el poder de su cruz. De hecho, escribió a los \textbf{Corintios}: \textquote{Predicamos a Cristo crucificado \ldots poder de Dios y sabiduría de Dios} (\textit{1 Cor} 1, 23-24). ¡Sí! Es poder de Dios. He aquí que: \textquote{Él hace volver a levantar el templo de su cuerpo torturado\ldots}.

Es sabiduría. Sí, es la sabiduría de Dios: él sabe –hasta el fondo– \textquote{lo que hay en cada hombre}. El hombre no se conoce a sí mismo si no participa de esta \textquote{sabiduría} de la cruz y la resurrección. Es al mismo tiempo la \textquote{sabiduría} del \textquote{amor con que tanto amó Dios al mundo} (\textit{Jn} 3, 16). Esta \textquote{sabiduría} es poder. Sólo ella es el poder del hombre. Sólo ella es capaz de transformar profundamente el corazón humano.

[\ldots]

7. \txtsmall{(\ldots) A todos los presentes, y a sus seres queridos, les dirijo mis pensamientos afectuosos y mis mejores deseos. (\ldots)} Como es sabido, la participación en la liturgia y los sacramentos ocupan un lugar central en la vida parroquial. De esta fuente de vida sobrenatural nace la comunidad; de ella brota la linfa vital que sostiene la fe y el fervor de cada creyente. En este sentido, quisiera llamar la atención sobre la importancia de la práctica del sacramento de la Reconciliación, especialmente en este tiempo de Cuaresma. La fuerza espiritual que se desprende de este sacramento para la vida cristiana es inconmensurable: de hecho nos acerca a la santidad de Dios; nos permite encontrar la paz interior cuando estamos atribulados por el pecado, y recuperar la alegría perdida, haciéndonos sentir acogidos íntimamente por el abrazo misericordioso de Dios.

\txtsmall{[Deseo expresar mi satisfacción por las iniciativas que la parroquia promueve a favor de los jóvenes, especialmente de aquellos que se encuentran en situaciones difíciles, debido al uso de drogas y la consiguiente marginación; animo a las personas a continuar las actividades culturales a favor de aquellos que deseen ser educados en la doctrina de la fe, que lleguen a perfeccionar y completar las lecciones catequéticas impartidas a los jóvenes en preparación para la Primera Comunión y la Confirmación. Agradezco también a todos los que colaboran en la pastoral parroquial\ldots Por último, mi aplauso a quienes de manera encomiable dedican su tiempo libre y su energía a ayudar y cuidar a los ancianos o enfermos en casa. ¡Que el Señor recompense su generosa entrega y su solidaridad evangélica!]}

8. El \textbf{salmista} proclama: \textquote{El temor del Señor es puro, y eternamente estable; los mandamientos del Señor son verdaderos y enteramente justos. Más preciosos que el oro, más que el oro fino} (\textit{Sal} 19 [18], 10-11). ¡Oremos para tener temor de Dios! A veces falta en el hombre de nuestra época. ¡Sí! Oremos para que nos sea concedido este temor, que es \textquote{el principio de la sabiduría}. Aprendamos esta sabiduría, la sabiduría más profunda y definitiva, que se manifiesta en la cruz de Cristo, a través de su resurrección. Para que no nos sorprenda el \textquote{juicio divino}, que siempre es \textquote{justo}. Dios sabe lo que hay en cada hombre. No necesita el testimonio de nadie. Acojamos sólo este único testimonio: es el testimonio de la cruz y resurrección de Cristo.

Amén.
\end{body}

\newpage
\subsubsection{Homilía (1991): Verdadero templo}

\src{Visita Pastoral a la Parroquia Romana de Todos los Santos.\\3 de marzo de 1991.}

\begin{body}
\textquote{Destruid este templo y en tres días lo levantaré} (\textit{Jn} 2, 19).

\ltr[1. ]{Q}{ueridos} hermanos y hermanas, el \textbf{pasaje evangélico} de la liturgia de hoy nos proyecta hacia la Pascua del Señor. Nos revela su significado, especialmente en lo que respecta a las relaciones de fidelidad y servicio que Dios pide a quienes, muertos y resucitados con Cristo, forman el pueblo de la Nueva Alianza.

Jesús sube a Jerusalén y, como todo israelita piadoso, va al templo a orar, pero lo encuentra transformado en un \textquote{mercado}. Sobre todo, se da cuenta de que el culto que allí se celebra, como ya habían denunciado los profetas, ya no se inspira en la fidelidad a la \textquote{ley de la Alianza}, sino que ha degenerado en actos formalistas y externos, desligados de la vida.

Con un gesto profético, que escandaliza a los judíos allí reunidos para la fiesta de Pascua, expulsa a los comerciantes y reafirma con fuerza el destino original del templo, como \textquote{casa de Dios}.

Jesús, \textquote{Hijo} de Dios que vino a ocuparse de las cosas que conciernen al Padre (cf. \textit{Lc} 2, 49), se siente herido por cómo se ha deshonrado el culto exigido a los verdaderos adoradores. Pero los judíos, cegados por su incredulidad, piden una señal autorizada de confirmación de sus palabras y del hecho realizado. Y Jesús lo da con un anuncio a primera vista incomprensible y muy diferente a sus expectativas, pero que luego quedará claro para los discípulos: \textquote{Destruid este templo y en tres días lo levantaré}. De hecho, como señala el evangelista, no hablaba del templo de piedra, sino del tempo de su Cuerpo. De este modo se nos introduce en la comprensión de otra gran verdad que recuerda la resurrección de Cristo y su significado salvador en la vida de quienes creen en él.

2. En esta perspectiva, la humanidad de Cristo, glorificada con la resurrección, se convierte en el verdadero templo de Dios, \textquote{en el que habita la plenitud de la divinidad} (\textit{Col} 2, 9); se convierte en el único \textquote{lugar} que Dios ha elegido para hacerse presente y revelarse al hombre, con su santidad y su misericordia.

El edificio en el que nos reunimos para darle a Dios el culto que le agrada es importante, pero sigue siendo secundario. Lo esencial es la experiencia de Cristo resucitado, posible gracias a la acción del Espíritu que emana de su Pascua. Es Él la \textquote{nueva ley}, escrita en el corazón de los creyentes, que los guía al conocimiento de toda la verdad, los capacita para realizar el culto espiritual que involucra toda su existencia y los empuja a testificar y servir al hombre.

Lo mismo debe suceder también para vuestra comunidad [y para la Iglesia de Roma, que vive este tiempo propicio de Cuaresma en un clima de renovación espiritual al que la llama el Sínodo pastoral diocesano].

La Cuaresma, de hecho, es un tiempo de \textquote{iluminación}: la palabra divina conduce gradualmente al pueblo de Dios a descubrir y recordar la riqueza del misterio pascual. Será el mismo Espíritu, fuente de sabiduría y santidad, dado a los fieles a través de los sacramentos pascuales del bautismo y la confirmación, que les permitirá \textquote{recordar} el pleno significado de las palabras de Cristo y entrar en una comunión más intensa con él, especialmente en la Eucaristía.

3. Queridos hermanos y hermanas [de la parroquia de Todos los Santos], las palabras que se acaban de proclamar [en este templo restaurado y renovado] están dirigidas a cada uno de vosotros. Os invitan a experimentar los frutos de la \textquote{nueva ley} ya consagrarse al servicio del Reino.

Haced de Cristo el centro de vuestras vidas y el corazón de vuestro apostolado: esta es la llamada misionera que os anima; \txtsmall{[este es el programa apostólico que guió a Don Orione y que aún hoy conserva toda su actualidad. Han pasado más de 80 años desde que Pío X envió al apóstol de la caridad de Porta San Giovanni en 1908. El Pontífice lo envió como misionero a la \textquote{Patagonia romana}. Desde entonces vuestra parroquia ha crecido mucho y se han multiplicado sus actividades pastorales y caritativas. Siguiendo los pasos del Fundador y de sus hijos espirituales que trabajaron aquí y siguen trabajando,]} vosotros queréis ser los apóstoles de la hora presente, amando a Dios y amando a vuestros hermanos: es más, amando a Dios sin reservas para poder servir al prójimo sin cesar.

(\ldots) El anuncio del Evangelio debe llevarse a todos, para que el anuncio de la muerte y resurrección del Señor resuene en todos los rincones y en todas las casas del barrio. Sin embargo, este anuncio sólo puede ser proclamado de manera creíble por una comunidad viva y unida, humilde y valiente, fiel al designio divino y al servicio de los más pobres.

4. \txtsmall{[Precisamente para animaros a continuar en este esfuerzo de renovación espiritual y evangelización he querido visitar vuestra parroquia, ferviente y llena de iniciativas.]} (\ldots) Os saludo a todos vosotros, queridos fieles\ldots y a vuestros familiares, especialmente a quienes están enfermos, a los ancianos y a los que sufren\ldots

[\ldots]

Finalmente, expreso mi exhortación a todos a perseverar en el compromiso de conversión personal y atención a los hermanos. Así, la comunidad cristiana, de la que cada uno es parte viva, será un centro de animación de la paz y la alegría que emana del Redentor.

Que la ayuda maternal de María, Madre de la Divina Providencia, y la intercesión [del Beato Luis Orione] os sostengan en esta misión.

Que Cristo sea en vosotros fuente de vida nueva. Que Cristo sea vuestra vida y vuestro gozo.

Amén.
\end{body}


%1994
\subsubsection{Homilía (1994): El cimiento de la Verdad}

\src{Visita Pastoral a la Parroquia Romana de San Bernardo de Claraval.\\6 de marzo de 1994.}

\begin{body}
Hoy la Palabra de Dios en la liturgia nos habla del templo. 

\txtsmall{[Tenéis la alegría de tener un nuevo templo, una nueva iglesia parroquial construida a costa de muchos esfuerzos y sacrificios. Los niños de la Primera Comunión cantaron sobre este tema, ilustrando con sus manos y voces cómo se ha colocado ladrillo a ladrillo para construir esta iglesia. Me alegro con vosotros, con los diseñadores, con los constructores, con toda vuestra comunidad, con vuestro párroco y con vuestro clero. Es una gran alegría para mí también. Esta nueva iglesia pertenece a la Iglesia de Roma y así, a través de ella, se consolida la Iglesia de Roma. No habiendo podido venir en noviembre, mi alegría por esta visita de hoy es aún mayor.]}

\ltr{L}{a} liturgia nos habla de un templo metafórico. No es solo el templo construido con piedras, sino también el templo construido por personas. Por eso centramos nuestra reflexión en la familia, porque la antigua tradición de la Iglesia de los Padres llamaba a la familia iglesia doméstica. Está formada por padres, madres, hijos e hijas. Dios habita entre ellos y quiere encontrar su hogar en este templo viviente. Por tanto, la familia es una Iglesia doméstica, pero para serlo debe estar basada en un fundamento sólido que es la Verdad, como recordaba en la Encíclica \textit{Veritatis Splendor}.

El poder de la verdad nos permite construir un templo dentro de nosotros mismos y entre nosotros, construirlo en la familia, en la sociedad, en toda la humanidad.

Hay quienes quisieran construirlo sin la Verdad y contra la Verdad. Sobre todo, se ataca el templo, la familia, iglesia doméstica. Se quiere quitar el fundamento de la Verdad, aprovechando las debilidades humanas, afirmando la legitimidad de los divorcios, las separaciones y todo lo que esté en contra de la vida de los no nacidos y los ancianos. Se intenta afirmar esto, contra el fundamento y contra el precepto que claramente está a favor de la vida. La vida es sagrada.

Lo mismo ocurre con el intento de legitimar a las familias falsas formadas por dos hombres o dos mujeres. Respetamos a todos los hombres y a todas las mujeres, pero construir una familia sobre esta base es incorrecto y peligroso.

[Durante este Año de la Familia,] toda la Iglesia debe ser muy prudente y muy valiente en la defensa de la verdadera familia. Debemos tener una gran comprensión de todas las debilidades humanas, como lo hizo Cristo, pero para la familia, entendida como el principio de construcción de la sociedad, debemos ser intrépidos e intransigentes. También intento serlo, aunque el Papa es por naturaleza un hombre dulce, no severo y rígido. Pero es necesario ser estricto con los principios. La construcción se basa en la verdad y por tanto en los preceptos. La Iglesia nos recuerda hoy el \textbf{Decálogo}, los Diez Mandamientos, que son las piedras inmóviles. Ninguno de ellos puede ser abolido, todas estas piedras deben mantenerse firmes. Así se construye la Iglesia.

Junto al pasaje del Antiguo Testamento que nos recuerda los Diez Mandamientos, está la Lectura del \textbf{Evangelio} en la que Cristo, en el templo de Jerusalén, dice que el edificio será destruido, previendo la catástrofe del año setenta. El templo construido de piedras se puede destruir, dice Jesús, también podéis destruir el templo de mi cuerpo, pero en tres días lo reconstruiré. Se trata de la promesa de la Resurrección.

Cristo, muerto y resucitado, viene a ser para nosotros la primera piedra, la piedra angular de toda la construcción de la Iglesia doméstica y de toda la humanidad. El ministerio de la Iglesia católica es precisamente construir la vida humana sobre la piedra que es Jesús. \textbf{San Pablo} nos dice que Jesús crucificado es nuestra sabiduría y nuestra fuerza. Es una paradoja, pero de este Crucifijo viene toda nuestra fuerza, la fuerza de los que sufren y de todos aquellos que no quieren equivocarse en la vida, que quieren seguir el camino recto, que quieren construir y no destruir.

En esta Cuaresma, con gran energía, presentamos a Cristo crucificado como nuestra sabiduría, como nuestra fuerza ante todo el mundo. Miradlo a Él, no lo dejéis de lado. Si queréis excluir el nombre de Dios, toda vuestra construcción será en vano. Se destruirá a sí misma.

Hay un cierto drama en la Liturgia de la Palabra durante el tiempo de Cuaresma, más que en otros períodos: el Decálogo, el discurso de Jesús en el templo, las palabras de Pablo.

A través de estas palabras debemos encontrar siempre la serenidad: Dios es más fuerte que las debilidades humanas y que las desviaciones humanas. Dios es siempre más fuerte, siempre tendrá la última palabra. Debemos tener el espíritu de San Bernardo, que fue un doctor mariano y en María, incluso en la turbulenta época medieval, siempre supo encontrar la serenidad mariana.

Sigamos el ejemplo del Patrón de vuestra parroquia y busquemos la serenidad en María para resolver los conflictos de nuestro tiempo.
\end{body}

\img{two_testaments}

\newpage
\subsubsection{Ángelus (1997): Vendedores del templo de nuestra época}

\src{2 de marzo de 1997.}

\begin{body}
\ltr[1. ]{E}{n} el \textbf{evangelio} de este tercer domingo de Cuaresma, \textbf{san Juan} relata que Jesús, al encontrar en el templo de Jerusalén a vendedores y cambistas, hizo un azote de cordeles y los arrojó con palabras encendidas: \textquote{¡Quitad esto de aquí: no convirtáis en un mercado la casa de mi Padre!} (\textit{Jn} 2, 16). La actitud \textquote{severa} del Señor parecería estar en contraste con la mansedumbre habitual con la que se acerca a los pecadores, cura a los enfermos, acoge a los pequeños y a los débiles. Sin embargo, observando con atención, la mansedumbre y la severidad son expresiones del mismo amor, que sabe ser, según la necesidad, tierno y exigente. El amor auténtico va acompañado siempre por la verdad.

Ciertamente, el celo y el amor de Jesús a la casa del Padre no se limitan a un templo de piedra. El mundo entero pertenece a Dios, y no se ha de profanar. Con el gesto profético que nos refiere el texto evangélico de hoy, Cristo nos pone en guardia contra la tentación de \textquote{comerciar} incluso con la religión, supeditándola a intereses mundanos o, de cualquier modo, ajenos a ella.

Cristo alza su voz también contra los \textquote{vendedores del templo} de nuestra época, es decir, contra cuantos convierten el mercado en su \textquote{religión} hasta ofender, en nombre del \textquote{dios-poder y del dios-dinero}, la dignidad de la persona humana con abusos de todo tipo. Pensemos, por ejemplo, en la falta de respeto a la vida, hecha objeto a veces de peligrosos experimentos; pensemos en la contaminación ecológica, la comercialización del sexo, el tráfico de drogas y la explotación de los pobres y los niños.

2. La \textbf{página evangélica} también tiene un significado más específico, que remite al misterio de Cristo y anuncia la alegría de la Pascua. Respondiendo a quienes le pedían que confirmara con un \textquote{signo} su profecía, Jesús lanza una especie de desafío: \textquote{Destruid este templo, y en tres días lo levantaré} (\textit{Jn} 2, 19). El mismo evangelista advierte que hablaba de su cuerpo, aludiendo a su futura resurrección. Así, la humanidad de Cristo se presenta como el verdadero \textquote{templo}, la casa viva de Dios. Será \textquote{destruida} en el Gólgota, pero inmediatamente volverá a ser \textquote{reconstruida} en la gloria, para transformarse en morada espiritual de cuantos acogen el mensaje evangélico y se dejan plasmar por el Espíritu de Dios.

3. Que la Virgen nos ayude a acoger las palabras de su Hijo divino. La misión de María consiste, precisamente, en llevarnos a él, repitiéndonos la invitación que hizo a los sirvientes en Caná: \textquote{Haced lo que él os diga} (\textit{Jn} 2, 5). Escuchemos su voz materna. María sabe bien que las exigencias del Evangelio, incluso cuando son pesadas y duras, constituyen el secreto de la verdadera libertad y de nuestra felicidad auténtica.
\end{body}


%1997
\subsubsection{Homilía (1997): La Resurrección}

\src{Visita Pastoral a la Parroquia Romana de San Juliano Mártir. \\2 de marzo de 1997.}

\begin{body}
1. \textquote{Señor, tú tienes palabras de vida eterna} (cf. \textit{Jn} 6, 68).

\ltr{E}{l} \textbf{Salmo responsorial} que acabamos de proclamar nos lleva al corazón del mensaje de la liturgia de hoy. El poder de la Palabra divina se manifestó por primera vez en la creación del mundo, cuando Dios dijo: \textquote{\textit{Fiat}} (cf. \textit{Gn} 1, 3), llamando a la existencia a todas las criaturas. Pero las lecturas bíblicas de este tercer domingo de Cuaresma destacan otra dimensión del poder de la Palabra de Dios: la que se refiere al orden moral.

Dios entregó al pueblo elegido el Decálogo en el monte Sinaí, montaña que reviste singular valor simbólico en la historia de la salvación. \txtsmall{[Precisamente por esto, con ocasión del gran jubileo de año 2000, se ha propuesto un encuentro en ese monte (cf. \textit{Tertio millennio adveniente}, 53).]} La \textbf{primera lectura} de hoy, tomada del \textbf{libro del Éxodo}, desarrolla de modo particular los primeros tres mandamientos dados a Israel, esto es, los de la que se suele llamar \textquote{primera tabla}: \textquote{Yo soy el Señor, tu Dios (\ldots). No tendrás otros dioses frente a mí (\ldots). No pronunciarás el nombre del Señor, tu Dios, en falso (\ldots). Guardarás el sábado para santificarlo} (\textit{Ex} 20, 2. 7-8).

2. Es fundamental el primer mandamiento, en el que se afirma solemnemente la unicidad de Dios: no hay otras divinidades, además de Él. En la ley dada a Moisés, se manifiesta el Dios invisible, que ninguna imagen realizada por las manos del hombre puede representar dignamente. Con la encarnación del Verbo, Dios se hizo hombre, y así el Dios invisible se hizo visible y, desde ese momento, la humanidad puede contemplar su gloria. La cuestión de la representación artística de Dios fue examinada detenidamente en el segundo concilio de Nicea, y se aclaró entonces que, dado que el Dios invisible se había hecho hombre en la Encarnación, su reproducción artística era legítima para los cristianos.

Al primer mandamiento está muy unido el segundo, que no sólo quiere condenar el abuso del nombre de Dios, sino que también tiene como finalidad advertir que no se siga la idolatría difundida en las religiones paganas. De la misma forma, por lo que concierne al tercer mandamiento: \textquote{Guarda el sábado para santificarlo} (\textit{Ex} 20, 8), la normativa es detallada y se remonta al modelo originario del descanso, del que dio ejemplo Dios al término de la creación. En cambio, se describen de manera sintética los mandamientos de la que se suele llamar \textquote{segunda tabla}.

3. \textquote{Señor, tú tienes palabras de vida eterna}. Las palabras que Dios pronuncia en el Antiguo Testamento encuentran pleno cumplimiento en Cristo, Palabra de Dios encarnada. En la antigua alianza, el poder creador de Dios en el ámbito moral se expresó en el Decálogo; en la nueva alianza, en cambio, Cristo es la actuación plena de ese poder; por tanto, no es una ley escrita, sino la persona misma del Salvador.

Se trata de una verdad que san Pablo expresa con eficacia al escribir a los Gálatas y a los Romanos: a la justificación mediante la observancia de la ley contrapone la Justificación mediante la fe en Cristo. Hoy, en cambio, en la \textbf{segunda lectura}, tomada de la \textbf{primera carta a los Corintios}, leemos estas palabras: \textquote{Nosotros predicamos a Cristo crucificado: escándalo para los judíos, necedad para los griegos; pero para los llamados en Cristo –judíos o griegos– fuerza de Dios y sabiduría de Dios} (\textit{1 Co} 1, 22-24).

El poder y la sabiduría que Dios manifestó al crear el mundo y al hombre, hecho \textquote{a su imagen y semejanza} (cf. \textit{Gn} 1, 26), se expresan plenamente en el orden moral. Por tanto, está al servicio del bien del hombre y de la sociedad humana. Esto lo confirma el Nuevo Testamento que determina con claridad el papel de la moral al servicio de la salvación eterna del hombre.

Precisamente por esto, en la aclamación antes del Evangelio acabamos de proclamar las palabras que Jesús pronunció en el diálogo con Nicodemo: \textquote{Tanto amó Dios al mundo, que entregó a su Hijo único. Todo el que cree en él, tiene vida eterna} (\textit{Jn} 3, 16). No sólo los mandamientos, sino sobre todo el Verbo eterno, que se hizo hombre, es la fuente de la vida eterna. [\ldots]

5. \textquote{Él hablaba del templo de su cuerpo} (\textit{Jn} 2, 21).

En el \textbf{evangelio} hemos releído el episodio de la expulsión de los vendedores del templo. La descripción de san Juan es viva y elocuente: por una parte está Jesús que, \textquote{haciendo un azote de cordeles, los echó a todos del templo, con sus ovejas y bueyes} (\textit{Jn} 2, 14-15), y por otra están los judíos, en particular los fariseos. El contraste es fuerte, hasta el punto de que algunos de los presentes preguntan a Jesús: \textquote{¿Qué signos nos muestras para obrar así?} (\textit{Jn} 2, 18).

\textquote{Destruid este templo y en tres días lo levantaré} (\textit{Jn} 2, 19), responde Cristo. La gente replica: \textquote{Cuarenta y seis años ha costado construir este templo, ¿y tú lo vas a levantar en tres días?} (\textit{Jn} 2, 20). No habían comprendido –anota san Juan– que el Señor estaba hablando del templo vivo de su cuerpo que, durante los acontecimientos pascuales, sería destruido con la muerte en la cruz, pero que resucitaría al tercer día. \textquote{Y cuando resucitó de entre los muertos –escribe el evangelista–, los discípulos se acordaron de que lo había dicho, y dieron fe a la Escritura y a la palabra que había dicho Jesús} (\textit{Jn} 2, 22).

El acontecimiento pascual da significado auténtico a todos los elementos presentes en las lecturas de hoy. En la Pascua se revela plenamente el poder del Verbo encarnado, poder del Hijo eterno de Dios, que se hizo hombre por nosotros y por nuestra salvación.

\textquote{Señor, tú tienes palabras de vida eterna}. Creemos que tú eres verdaderamente el Hijo de Dios. Y te damos gracias por habernos hecho partícipes de tu misma vida divina. Amén.
\end{body}

\subsubsection{Homilía (2000): La Resurrección, signo de fidelidad}

\src{Misa en la Basílica del Santo Sepulcro de Jerusalén, nn. 1. 3-5. \\26 de marzo del 2000.}

\begin{body}
1. \txtsmall{[Siguiendo el camino de la historia de la salvación, tal como se narra en el Símbolo de los Apóstoles, mi peregrinación jubilar me ha traído a Tierra Santa. De Nazaret, donde Jesús fue concebido en el seno de la Virgen María por obra del Espíritu Santo, he llegado a Jerusalén, donde \textquote{padeció bajo el poder de Poncio Pilato, fue crucificado, muerto y sepultado}. Aquí, en la basílica del Santo Sepulcro, me arrodillo ante el lugar de su sepultura: \textquote{He aquí el lugar donde lo pusieron} (\textit{Mc} 16, 6).]}

\ltr{L}{a} tumba está vacía. Es un testigo silencioso del acontecimiento central de la historia humana: la resurrección de nuestro Señor Jesucristo. Durante casi dos mil años la tumba vacía ha dado testimonio de la victoria de la Vida sobre la muerte. Con los Apóstoles y los evangelistas, con la Iglesia de todos los tiempos y lugares, también nosotros damos testimonio y proclamamos: \textquote{¡Cristo resucitó! Una vez resucitado de entre los muertos, ya no muere más; la muerte no tiene ya señorío sobre él} (cf. \textit{Rm} 6, 9). \textquote{Mors et vita duello conflixere mirando; dux vitae mortuus, regnat vivus} (\textit{Secuencia pascual latina Victimae paschali}). El Señor de la vida estaba muerto; ahora reina, victorioso sobre la muerte, fuente de vida eterna para todos los creyentes.

3. \textquote{Destruid este templo y en tres días lo levantaré} (\textit{Jn} 2, 19). El evangelista \textbf{san Juan} nos narra que, después de la resurrección de Jesús de entre los muertos, los discípulos recordaron estas palabras y creyeron (cf. \textit{Jn} 2, 22). Jesús las pronunció a fin de que fueran un signo para sus discípulos. Cuando fue al templo con sus discípulos, expulsó a los cambistas y a los vendedores del lugar santo (cf. \textit{Jn} 2, 15). En el momento en que los presentes protestaron, preguntándole: \textquote{¿Qué señal nos muestras para obrar así?}, Jesús les replicó: \textquote{Destruid este templo y en tres días lo levantaré}. El evangelista anota que \textquote{él hablaba del templo de su cuerpo} (\textit{Jn} 2, 18-21).

La profecía encerrada en las palabras de Jesús se cumplió en la Pascua, cuando \textquote{al tercer día resucitó de entre los muertos}. La resurrección de nuestro Señor Jesucristo es el signo de que el Padre eterno es fiel a su promesa y hace nacer nueva vida de la muerte: \textquote{la resurrección del cuerpo y la vida eterna}. \txtsmall{[El misterio se refleja claramente en esta antigua iglesia de la \textit{Anástasis}, que contiene tanto el sepulcro vacío, signo de la Resurrección, como el Gólgota, lugar de la crucifixión.]} 

La buena nueva de la Resurrección no puede separarse nunca del misterio de la cruz. \textbf{San Pablo} nos lo dice en la \textbf{segunda lectura} de hoy: \textquote{Nosotros predicamos a Cristo crucificado} (\textit{1 Co} 1, 23). Cristo, que se ofreció a sí mismo como sacrificio vespertino en el altar de la cruz (cf. \textit{Sal} 141, 2), se revela ahora como \textquote{fuerza de Dios y sabiduría de Dios} (\textit{1 Co} 1, 24). Y en su resurrección, los hijos y las hijas de Adán han sido hechos partícipes de su vida divina, que tenía desde toda la eternidad, con el Padre, en el Espíritu Santo.

4. [\ldots] Mediante el Decálogo y la ley moral inscrita en el corazón del hombre (cf. \textit{Rm} 2, 15), Dios desafía radicalmente la libertad de cada hombre y cada mujer. Responder a la voz de Dios que resuena en lo más profundo de nuestra conciencia y elegir el bien es la opción más sublime de la libertad humana. Equivale, realmente, a elegir entre la vida y la muerte (cf. \textit{Dt} 30, 15). Caminando por la senda de la Alianza con Dios santísimo, el pueblo se convierte en heraldo y testigo de la promesa, la promesa de una auténtica liberación y de la plenitud de vida.

La resurrección de Jesús es el sello definitivo de todas las promesas de Dios, el lugar de nacimiento de una humanidad nueva y resucitada, la prenda de una historia caracterizada por los dones mesiánicos de paz y alegría espiritual. En el alba de un nuevo milenio, los cristianos pueden y deben mirar al futuro con firme confianza en el poder glorioso del Resucitado de renovar todas las cosas (cf. \textit{Ap} 21, 5). Él es el único que libra a toda la creación de la servidumbre de la corrupción (cf. \textit{Rm} 8, 20). Con su resurrección, abre el camino al gran descanso del \textit{sabbath}, el octavo día, cuando la peregrinación de la humanidad llegue a su fin y Dios sea todo en todos (cf. \textit{1 Co} 15, 28).

Aquí, [en el Santo Sepulcro y en el Gólgota,] a la vez que renovamos nuestra profesión de fe en el Señor resucitado, ¿podemos dudar de que con el poder del Espíritu de vida recibiremos la fuerza para superar nuestras divisiones y trabajar juntos a fin de construir un futuro de reconciliación, unidad y paz? Aquí, como en ningún otro lugar de la tierra, oímos una vez más al Señor que dice a sus discípulos: \textquote{¡Ánimo!: yo he vencido al mundo} (\textit{Jn} 16, 33).

5. \textquote{Mors et vita duello conflixere mirando; dux vitae mortuus, regnat vivus}. El Señor resucitado, resplandeciente por la gloria del Espíritu, es la Cabeza de la Iglesia, su Cuerpo místico. Él la sostiene en su misión de proclamar el Evangelio de la salvación a los hombres y mujeres de cada generación, hasta que vuelva en la gloria. \txtsmall{[En este lugar, donde se dio a conocer la Resurrección primero a las mujeres y luego a los Apóstoles,]} invito a todos los miembros de la Iglesia a renovar su obediencia al mandato del Señor de anunciar el Evangelio hasta los confines de la tierra. \txtsmall{[En el alba de un nuevo milenio]} es muy necesario proclamar desde los tejados la buena nueva de que \textquote{tanto amó Dios al mundo, que dio a su Hijo único, para que todo el que crea en él no perezca, sino que tenga vida eterna} (\textit{Jn} 3, 16). \textquote{Señor, (\ldots) tú tienes palabras de vida eterna} (\textit{Jn} 6, 68). \txtsmall{[Hoy, como indigno Sucesor de Pedro, deseo repetir estas palabras mientras celebramos el sacrificio eucarístico en este lugar, el más santo de la tierra.]} Con toda la humanidad redimida, hago mías las palabras que Pedro, el pescador, dirigió a Cristo, Hijo del Dios vivo: \textquote{Señor, ¿a quién iremos? Tú tienes palabras de vida eterna}.

\textit{Christós anésti}! ¡Jesucristo ha resucitado! ¡En verdad, ha resucitado! Amén.
\end{body}

\newsection
\subsection{Benedicto XVI, papa}

\subsubsection{Homilía (2006): Crucificado y Resucitado}

\src{Celebración Eucarística con los Trabajadores en la Fiesta de San José. \\Domingo 19 de marzo del 2006.}

\begin{body}
\ltr{H}{emos} escuchado juntos una famosa y bella página del \textbf{libro del Éxodo}, en la que el autor sagrado narra la entrega del Decálogo a Israel por parte de Dios. Un detalle llama enseguida la atención: la enumeración de los diez mandamientos se introduce con una significativa referencia a la liberación del pueblo de Israel. Dice el texto: \textquote{Yo soy el Señor, tu Dios, que te saqué de Egipto, de la esclavitud} (\textit{Ex} 20, 2). Por tanto, el Decálogo quiere ser una confirmación de la libertad conquistada. En efecto, los mandamientos, si se analizan en profundidad, son el instrumento que el Señor nos da para defender nuestra libertad tanto de los condicionamientos internos de las pasiones como de los abusos externos de los maliciosos. Los \textquote{no} de los mandamientos son otros tantos \textquote{sí} al crecimiento de una libertad auténtica. Conviene subrayar también una segunda dimensión del Decálogo: con la Ley dada por medio de Moisés el Señor revela que quiere establecer con Israel una alianza. Por consiguiente, la Ley, más que una imposición, es un don. Más que mandar lo que el hombre debe hacer, quiere manifestar a todos la elección de Dios: él está de parte del pueblo elegido; lo liberó de la esclavitud y lo rodeó con su bondad misericordiosa. El Decálogo es testimonio de un amor de predilección.

La liturgia de hoy nos ofrece un segundo mensaje: la Ley mosaica se cumplió plenamente en Jesús, que reveló la sabiduría y el amor de Dios mediante el misterio de la cruz, \textquote{escándalo para los judíos, necedad para los griegos –como nos dice san Pablo en la \textbf{segunda lectura}–; pero para los llamados (\ldots), judíos o griegos, fuerza de Dios y sabiduría de Dios} (\textit{1 Co} 1, 23-24). Precisamente a este misterio se refiere la \textbf{página evangélica} que se acaba de proclamar: Jesús expulsa del templo a los vendedores y a los cambistas. El evangelista ofrece la clave de lectura de este significativo episodio en el versículo de un salmo: \textquote{El celo por tu casa me devora} (cf. \textit{Sal} 69, 10). A Jesús lo \textquote{devora} este \textquote{celo} por la \textquote{casa de Dios}, utilizada con un fin diferente de aquel para el que estaba destinada. Ante la petición de los responsables religiosos, que pretenden un signo de su autoridad, en medio del asombro de los presentes, afirma: \textquote{Destruid este templo, y en tres días lo levantaré} (\textit{Jn} 2, 19). Palabras misteriosas, incomprensibles en aquel momento, pero que san Juan vuelve a formular para sus lectores cristianos, observando: \textquote{Él hablaba del templo de su cuerpo} (\textit{Jn} 2, 21).

Sus adversarios destruirán este \textquote{templo}, pero él, al cabo de tres días, lo reconstruirá mediante la resurrección. La muerte dolorosa y \textquote{escandalosa} de Cristo se coronará con el triunfo de su gloriosa resurrección. Mientras en este tiempo cuaresmal nos preparamos para revivir en el triduo pascual este acontecimiento central de nuestra salvación, contemplamos al Crucificado vislumbrando ya en él el resplandor del Resucitado.

\txtsmall{[Queridos hermanos y hermanas, esta celebración eucarística, que a la meditación de los textos litúrgicos del tercer domingo de Cuaresma une el recuerdo de san José, nos ofrece la oportunidad de considerar, a la luz del misterio pascual, otro aspecto importante de la existencia humana. Me refiero a la realidad del trabajo, que hoy está en el centro de cambios rápidos y complejos. En numerosas páginas la Biblia muestra cómo el trabajo pertenece a la condición originaria del hombre. Cuando el Creador plasmó al hombre a su imagen y semejanza, lo invitó a trabajar la tierra (cf. \textit{Gn} 2, 5-6). A causa del pecado de nuestros primeros padres, el trabajo se transformó en fatiga y sudor (cf. \textit{Gn} 3, 6-8), pero el proyecto divino mantiene inalterado su valor. El mismo Hijo de Dios, haciéndose semejante en todo a nosotros, se dedicó durante muchos años a actividades manuales, hasta el punto de que lo conocían como el \textquote{hijo del carpintero} (cf. \textit{Mt} 13, 55). La Iglesia ha mostrado siempre, especialmente durante el último siglo, interés y solicitud por este ámbito de la sociedad, como testimonian las numerosas intervenciones sociales del Magisterio y la acción de múltiples asociaciones de inspiración cristiana, algunas de las cuales han venido hoy aquí a representar a todo el mundo de los trabajadores. Me alegra acogeros, queridos amigos, y os dirijo a cada uno mi cordial saludo (\ldots)}

\txtsmall{El trabajo reviste una importancia primaria para la realización del hombre y el desarrollo de la sociedad, y por eso es preciso que se organice y desarrolle siempre en el pleno respeto de la dignidad humana y al servicio del bien común. Al mismo tiempo, es indispensable que el hombre no se deje dominar por el trabajo, que no lo idolatre, pretendiendo encontrar en él el sentido último y definitivo de la vida. Al respecto, es oportuna la invitación de la primera lectura: \textquote{Fíjate en el sábado para santificarlo. Durante seis días trabaja y haz tus tareas, pero el día séptimo es un día de descanso dedicado al Señor, tu Dios} (\textit{Ex} 20, 8-9). El sábado es día santificado, es decir, consagrado a Dios, en el que el hombre comprende mejor el sentido de su existencia y también de la actividad laboral. Por tanto, se puede afirmar que la enseñanza bíblica sobre el trabajo culmina en el mandamiento del descanso. Al respecto, el \textit{Compendio de la doctrina social de la Iglesia} observa oportunamente: \textquote{El descanso abre al hombre, sujeto a la necesidad del trabajo, la perspectiva de una libertad más plena, la del sábado eterno (cf. \textit{Hb} 4, 9-10). El descanso permite a los hombres recordar y revivir las obras de Dios, desde la creación hasta la Redención, reconocerse a sí mismos como obra suya (cf. \textit{Ef} 2, 10), y dar gracias por su vida y su subsistencia a él, que de ellas es el Autor} (n. 258).}

\txtsmall{La actividad laboral debe contribuir al verdadero bien de la humanidad, permitiendo \textquote{al hombre individual y socialmente cultivar y realizar plenamente su vocación} (\textit{Gaudium et spes}, 35). Para que esto suceda no basta la preparación técnica y profesional, por lo demás necesaria; ni siquiera es suficiente la creación de un orden social justo y atento al bien de todos. Es preciso vivir una espiritualidad que ayude a los creyentes a santificarse a través de su trabajo, imitando a san José, que cada día debió proveer con sus manos a las necesidades de la Sagrada Familia, y por eso la Iglesia lo propone como patrono de los trabajadores. Su testimonio muestra que el hombre es sujeto y protagonista del trabajo. Quisiera encomendarle a él a los jóvenes que con esfuerzo logran insertarse en el mundo del trabajo, a los desempleados y a todos los que sufren las dificultades debidas a la crisis laboral generalizada. Que junto con María, su esposa, san José vele sobre todos los trabajadores y obtenga serenidad y paz para las familias y para toda la humanidad. Que al contemplar a este gran santo, los cristianos aprendan a testimoniar en todos los ámbitos laborales el amor de Cristo, manantial de solidaridad verdadera y de paz estable. Amén.]}
\end{body}
\label{b2-03-03-2009}

\begin{patercite}
El Decálogo nos remite al monte Sinaí, cuando Dios entra de modo particular en la historia del pueblo judío y, a través de este pueblo, en la historia de toda la humanidad, dando las \textquote{Diez Palabras} (\ldots)  Pero ¿qué sentido tienen para nosotros estas Diez Palabras en el actual contexto cultural, en el que se corre el riesgo de que el laicismo y el relativismo se conviertan en los criterios de toda decisión, y en esta sociedad que parece vivir como si Dios no existiese? Nosotros respondemos que Dios nos ha dado los Mandamiento para educarnos en la verdadera libertad y en el amor auténtico, de modo que podamos ser realmente felices. Son un signo del amor de Dios Padre, de su deseo de enseñarnos a distinguir correctamente el bien del mal, lo verdadero de lo falso, lo justo de lo injusto. Todos los pueden comprender y, precisamente porque fijan los valores fundamentales en normas y reglas concretas, al ponerlos en práctica el hombre puede recorrer el camino de la verdadera libertad, que lo consolida en el camino que lleva a la vida y a la felicidad. Al contrario, cuando en su existencia el hombre ignora los Mandamientos, no sólo se aliena de Dios y abandona la alianza con él, sino que también se aleja de la vida y de la felicidad duradera. El hombre abandonado a sí mismo, indiferente hacia Dios, orgulloso de su propia autonomía absoluta, acaba por seguir los ídolos del egoísmo, del poder, del dominio, contaminando las relaciones consigo mismo y con los demás, y recorriendo sendas no de vida, sino de muerte. Las tristes experiencias de la historia, sobre todo del siglo pasado, siguen siendo una advertencia para toda la humanidad.

\textbf{Benedicto XVI, papa}, \textit{Videomensaje},  8 de septiembre del 2012, cf. parr. 2-3.
\end{patercite}

\newpage
\subsubsection{Ángelus (2012): Nuevo culto y nuevo Templo}

\src{11 de marzo del 2012.}

\begin{body}
\ltr{E}{l} \textbf{Evangelio} de este tercer domingo de Cuaresma refiere, en la redacción de \textbf{san Juan}, el célebre episodio en el que Jesús expulsa del templo de Jerusalén a los vendedores de animales y a los cambistas (cf. \textit{Jn} 2, 13-25). El hecho, recogido por todos los evangelistas, tuvo lugar en la proximidad de la fiesta de la Pascua y suscitó gran impresión tanto entre la multitud como entre sus discípulos. ¿Cómo debemos interpretar este gesto de Jesús? En primer lugar, hay que señalar que no provocó ninguna represión de los guardianes del orden público, porque lo vieron como una típica acción profética: de hecho, los profetas, en nombre de Dios, con frecuencia denunciaban los abusos, y a veces lo hacían con gestos simbólicos. El problema, en todo caso, era su autoridad. Por eso los judíos le preguntaron a Jesús: \textquote{¿Qué signos nos muestras para obrar así?} (\textit{Jn} 2, 18); demuéstranos que actúas verdaderamente en nombre de Dios.

La expulsión de los mercaderes del templo también se ha interpretado en sentido político revolucionario, colocando a Jesús en la línea del movimiento de los zelotes. Estos, de hecho, eran \textquote{celosos} de la ley de Dios y estaban dispuestos a usar la violencia para hacer que se cumpliera. En tiempos de Jesús esperaban a un mesías que liberase a Israel del dominio de los romanos. Pero Jesús decepcionó estas expectativas, por lo que algunos discípulos lo abandonaron, y Judas Iscariote incluso lo traicionó. En realidad, es imposible interpretar a Jesús como violento: la violencia es contraria al reino de Dios, es un instrumento del anticristo. La violencia nunca sirve a la humanidad, más aún, la deshumaniza.

Escuchemos entonces las palabras que Jesús dijo al realizar ese gesto: \textquote{Quitad esto de aquí: no convirtáis en un mercado la casa de mi Padre} (\textit{Jn} 2, 16). Sus discípulos se acordaron entonces de lo que está escrito en un Salmo: \textquote{El celo de tu casa me devora} (\textit{Sal} 69, 10). Este Salmo es una invocación de ayuda en una situación de extremo peligro a causa del odio de los enemigos: la situación que Jesús vivirá en su pasión. El celo por el Padre y por su casa lo llevará hasta la cruz: el suyo es el celo del amor que paga en carne propia, no el que querría servir a Dios mediante la violencia. De hecho, el \textquote{signo} que Jesús dará como prueba de su autoridad será precisamente su muerte y resurrección. \textquote{Destruid este templo –dijo–, y en tres días lo levantaré}. Y san Juan observa: \textquote{Él hablaba del templo de su cuerpo} (\textit{Jn} 2, 19. 21). Con la Pascua de Jesús se inicia un nuevo culto, el culto del amor, y un nuevo templo que es él mismo, Cristo resucitado, por el cual cada creyente puede adorar a Dios Padre \textquote{en espíritu y verdad} (\textit{Jn} 4, 23). Queridos amigos, el Espíritu Santo comenzó a construir este nuevo templo en el seno de la Virgen María. Por su intercesión, pidamos que cada cristiano sea piedra viva de este edificio espiritual.
\end{body}

\newsection
\subsection{Francisco, papa}

\subsubsection{Homilía (2015): Invitación a un culto auténtico}

\src{Visita a la Parroquia Romana de Todos los Santos. \\Sábado 7 de marzo del 2015.}

\begin{body}
\ltr{C}{on} ocasión de la fiesta de la Pascua judía, Jesús va a Jerusalén. Al llegar al templo, no encuentra gente que busca a Dios, sino gente que hace sus propios negocios: los mercaderes de animales para la ofrenda de los sacrificios; los cambistas, quienes cambian dinero \textquote{impuro} que lleva la imagen del emperador con monedas aprobadas por la autoridad religiosa para pagar el impuesto anual del templo. ¿Qué encontramos nosotros cuando visitamos, cuando vamos a nuestros templos? Dejo la pregunta. El indigno comercio, fuente de ricas ganancias, provoca la enérgica reacción de Jesús. Él volcó los bancos y esparció el dinero por el piso, echó a los vendedores diciéndoles: \textquote{No convirtáis en un mercado la casa de mi Padre} (\textit{Jn} 2, 16).

Esta expresión no se refiere sólo a los negocios que se realizaban en los patios del templo. Se refiere más bien a un tipo de religiosidad. El gesto de Jesús es un gesto de \textquote{limpieza}, de purificación, y la actitud que Él desautoriza se la puede sacar de los textos proféticos, según los cuales Dios no soporta un culto exterior hecho de sacrificios materiales y basado en el interés personal (cf. \textit{Is} 1, 11-17; \textit{Jer} 7, 2-11). Este gesto es la llamada al culto auténtico, a la correspondencia entre liturgia y vida; una llamada válida para todos los tiempos y también hoy para nosotros. Esa correspondencia entre liturgia y vida. La liturgia no es algo extraño, allá, lejano, y mientras se celebra yo pienso en muchas cosas, o rezo el Rosario. No, no. Hay una correspondencia con la celebración litúrgica que luego llevo a mi vida; y en esto se debe aún ir más adelante, se debe aún recorrer mucho camino.

La constitución conciliar \textit{Sacrosanctum Concilium} define la liturgia como \textquote{la primera y más necesaria fuente en la que los fieles beben el espíritu verdaderamente cristiano} (n. 14). Esto significa reafirmar el vínculo esencial que une la vida del discípulo de Jesús y el culto litúrgico. Esto no es ante todo una doctrina que se debe comprender, o un rito que hay que cumplir; es naturalmente también esto pero de otra forma, es esencialmente distinto: es una fuente de vida y de luz para nuestro camino de fe.

Por lo tanto, la Iglesia nos llama a tener y promover una vida litúrgica auténtica, a fin de que pueda haber sintonía entre lo que la liturgia celebra y lo que nosotros vivimos en nuestra existencia. Se trata de expresar en la vida lo que hemos recibido mediante la fe y lo que hemos celebrado (cf. \textit{Sacrosanctum Concilium}, 10).

El discípulo de Jesús no va a la iglesia sólo para cumplir un precepto, para sentirse bien con un Dios que luego no tiene que \textquote{molestar} demasiado. \textquote{Pero yo, Señor, voy todos los domingos, cumplo\ldots, tú no te metas en mi vida, no me molestes}. Esta es la actitud de muchos católicos, muchos. El discípulo de Jesús va a la iglesia para encontrarse con el Señor y encontrar en su gracia, operante en los sacramentos, la fuerza para pensar y obrar según el Evangelio. Por lo que no podemos ilusionarnos con entrar en la casa del Señor y \textquote{encubrir}, con oraciones y prácticas de devoción, comportamientos contrarios a las exigencias de la justicia, la honradez o la caridad hacia el prójimo. No podemos sustituir con \textquote{honores religiosos} lo que debemos dar al prójimo, postergando una verdadera conversión. El culto, las celebraciones litúrgicas, son el ámbito privilegiado para escuchar la voz del Señor, que guía por el camino de la rectitud y de la perfección cristiana.

Se trata de realizar un itinerario de conversión y de penitencia, para quitar de nuestra vida las escorias del pecado, como hizo Jesús, limpiando el templo de intereses mezquinos. Y la Cuaresma es el tiempo favorable para todo esto, es el tiempo de la renovación interior, de la remisión de los pecados, el tiempo en el que somos llamados a redescubrir el sacramento de la Penitencia y de la Reconciliación, que nos hace pasar de las tinieblas del pecado a la luz de la gracia y de la amistad con Jesús. No hay que olvidar la gran fuerza que tiene este sacramento para la vida cristiana: nos hace crecer en la unión con Dios, nos hace reconquistar la alegría perdida y experimentar el consuelo de sentirnos personalmente acogidos por el abrazo misericordioso de Dios.

\txtsmall{[(\ldots) este templo fue construido gracias al celo apostólico de san Luis Orione. Precisamente aquí, hace cincuenta años, el beato Pablo VI inauguró, en cierto sentido, la reforma litúrgica con la celebración de la misa en la lengua hablada por la gente.]} Os deseo que esta circunstancia reavive en todos vosotros el amor por la casa de Dios. En ella encontráis una gran ayuda espiritual. Aquí podéis experimentar, cada vez que queráis, el poder regenerador de la oración personal y de la oración comunitaria. La escucha de la Palabra de Dios, proclamada en la asamblea litúrgica, os sostiene en el camino de vuestra vida cristiana. Os encontráis entre estos muros no como extraños, sino como hermanos, capaces de darse la mano con gusto, porque os congrega el amor a Cristo, fundamento de la esperanza y del compromiso de cada creyente.

A Él, Jesucristo, Piedra angular, nos estrechamos confiados en esta santa misa, renovando el propósito de comprometernos en favor de la purificación y la limpieza interior de la Iglesia edificio espiritual, del cual cada uno de nosotros es parte viva en virtud del Bautismo. Así sea.
\end{body}


\subsubsection{Homilía (2015): No podemos engañar a Jesús}

\src{Visita Pastoral a la Parroquia Romana de \\Santa María Madre del Redentor en Tor Bella Monaca. \\8 de marzo del 2015.}

\begin{body}
\ltr{E}{n} este \textbf{pasaje del Evangelio} que hemos escuchado, hay dos cosas que me impresionan: una imagen y una palabra. La imagen es la de Jesús con el látigo en la mano que echa fuera a todos los que aprovechaban el Templo para hacer negocios. Estos comerciantes que vendían los animales para los sacrificios, cambiaban las monedas\ldots Estaba lo sagrado –el templo, sagrado– y esto sucio, afuera. Esta es la imagen. Y Jesús toma el látigo y procede, para limpiar un poco el Templo. Y la frase, la palabra, está ahí donde se dice que mucha gente creía en Él, una frase terrible: \textquote{Pero Jesús no se confiaba a ellos, porque los conocía a todos, y no necesitaba el testimonio de nadie sobre un hombre, porque Él sabía lo que hay dentro de cada hombre} (\textit{Jn} 2, 24-25).

Nosotros no podemos engañar a Jesús: Él nos conoce por dentro. No se fiaba. Él, Jesús, no se fiaba. Y esta puede ser una buena pregunta en la mitad de la Cuaresma: ¿Puede fiarse Jesús de mí? ¿Puede fiarse Jesús de mí, o tengo una doble cara? ¿Me presento como católico, como uno cercano a la Iglesia, y luego vivo como un pagano? \textquote{Pero Jesús no lo sabe, nadie va a contárselo}. Él lo sabe. \textquote{Él no tenía necesidad de que alguien diese testimonio; Él, en efecto, conocía lo que había en el hombre}. Jesús conoce todo lo que está dentro de nuestro corazón: no podemos engañar a Jesús. No podemos, ante Él, aparentar ser santos, y cerrar los ojos, actuar así, y luego llevar una vida que no es la que Él quiere. Y Él lo sabe. Y todos sabemos el nombre que Jesús daba a estos con doble cara: hipócritas.

Nos hará bien, hoy, entrar en nuestro corazón y mirar a Jesús. Decirle: \textquote{Señor, mira, hay cosas buenas, pero también hay cosas no buenas. Jesús, ¿te fías de mí? Soy pecador\ldots}. Esto no asusta a Jesús. Si tú le dices: \textquote{Soy un pecador}, no se asusta. Lo que a Él lo aleja es la doble cara: mostrarse justo para cubrir el pecado oculto. \textquote{Pero yo voy a la iglesia, todos los domingos, y yo\ldots}. Sí, podemos decir todo esto. Pero si tu corazón no es justo, si tú no vives la justicia, si tú no amas a los que necesitan amor, si tú no vives según el espíritu de las bienaventuranzas, no eres católico. Eres hipócrita. Primero: ¿Puede Jesús fiarse de mí? En la oración, preguntémosle: \textquote{Señor, ¿Tú te fías de mí?}.

Segundo, el gesto. Cuando entramos en nuestro corazón, encontramos cosas que no funcionan, que no están bien, como Jesús encontró en el Templo esa suciedad del comercio, de los vendedores. También dentro de nosotros hay suciedad, hay pecados de egoísmo, de soberbia, de orgullo, de codicia, de envidia, de celos\ldots ¡tantos pecados! Podemos incluso continuar el diálogo con Jesús: \textquote{Jesús, ¿Tú te fías de mí? Yo quiero que Tú te fíes de mí. Entonces te abro la puerta y tú limpia mi alma}. Y pedir al Señor que así como limpió el Templo, venga a limpiar el alma. E imaginamos que Él viene con un látigo de cuerdas\ldots No, con eso no limpia el alma. ¿Vosotros sabéis cuál es el látigo de Jesús para limpiar nuestra alma? La misericordia. Abrid el corazón a la misericordia de Jesús. Decid: \textquote{Jesús, mira cuánta suciedad. Ven, limpia. Limpia con tu misericordia, con tus palabras dulces; limpia con tus caricias}. Y si abrimos nuestro corazón a la misericordia de Jesús, para que limpie nuestro corazón, nuestra alma, Jesús se fiará de nosotros.
\end{body}

\begin{patercite}
El antiguo Templo estaba edificado por las manos de los hombres: se quería \textquote{dar una casa} a Dios para tener un signo visible de su presencia en medio del pueblo. Con la Encarnación del Hijo de Dios, se cumple la profecía de Natán al rey David (cf. \emph{2 Sam} 7, 1-29): no es el rey, no somos nosotros quienes \textquote{damos una casa a Dios}, sino que es Dios mismo quien \textquote{construye su casa} para venir a habitar entre nosotros, como escribe san Juan en su Evangelio (cf. 1, 14). Cristo es el Templo viviente del Padre, y Cristo mismo edifica su \textquote{casa espiritual}, la Iglesia, hecha no de piedras materiales, sino de \textquote{piedras vivientes}, que somos nosotros. El Apóstol Pablo dice a los cristianos de Éfeso: \textquote{Estáis edificados sobre el cimiento de los apóstoles y profetas, y el mismo Cristo Jesús es la piedra angular. Por Él todo el edificio queda ensamblado, y se va levantado hasta formar un templo consagrado al Señor. Por Él también vosotros entráis con ellos en la construcción, para ser morada de Dios, por el Espíritu} (\emph{Ef} 2, 20-22). ¡Esto es algo bello! Nosotros somos las piedras vivas del edificio de Dios, unidas profundamente a Cristo, que es la piedra de sustentación, y también de sustentación entre nosotros. ¿Qué quiere decir esto? Quiere decir que el templo somos nosotros, nosotros somos la Iglesia viviente, el templo viviente, y cuando estamos juntos entre nosotros está también el Espíritu Santo, que nos ayuda a crecer como Iglesia. Nosotros no estamos aislados, sino que somos pueblo de Dios: ¡ésta es la Iglesia! (\ldots) La Iglesia no es un entramado de cosas y de intereses, sino que es el Templo del Espíritu Santo, el Templo en el que Dios actúa, el Templo en el que cada uno de nosotros, con el don del Bautismo, es piedra viva. Esto nos dice que nadie es inútil en la Iglesia, y si alguien dice a veces a otro: \textquote{Vete a casa, eres inútil}, esto no es verdad, porque nadie es inútil en la Iglesia, ¡todos somos necesarios para construir este Templo! Nadie es secundario. Nadie es el más importante en la Iglesia; todos somos iguales a los ojos de Dios. Alguno de vosotros podría decir: \textquote{Oiga, señor Papa, usted no es igual a nosotros}. Sí: soy como uno de vosotros, todos somos iguales, ¡somos hermanos! Nadie es anónimo: todos formamos y construimos la Iglesia. Esto nos invita también a reflexionar sobre el hecho de que si falta la piedra de nuestra vida cristiana, falta algo a la belleza de la Iglesia. (\ldots) todos debemos llevar a la Iglesia nuestra vida, nuestro corazón, nuestro amor, nuestro pensamiento, nuestro trabajo.

\textbf{Francisco, papa}, \textit{Catequesis}, 26 de junio de 2013, parr. 4-5.
\end{patercite}



\label{b2-03-03-2015}

\newpage
\subsubsection{Ángelus (2015): El templo en el que Dios se da a conocer}

\src{8 de marzo del 2015.}

\begin{body}
\ltr{E}{l} \textbf{Evangelio} de hoy (\textit{Jn} 2, 13-25) nos presenta el episodio de la expulsión de los vendedores del templo. Jesús \textquote{hizo un látigo con cuerdas, los echó a todos del Templo, con ovejas y bueyes} (\textit{Jn} 2, 15), el dinero, todo. Tal gesto suscitó una fuerte impresión en la gente y en los discípulos. Aparece claramente como un gesto profético, tanto que algunos de los presentes le preguntaron a Jesús: \textquote{¿Qué signos nos muestras para obrar así?} (\textit{Jn} 2, 18), ¿quién eres para hacer estas cosas? Muéstranos una señal de que tienes realmente autoridad para hacerlas. Buscaban una señal divina, prodigiosa, que acreditara a Jesús como enviado de Dios. Y Él les respondió: \textquote{Destruid este templo y en tres días lo levantaré} (\textit{Jn} 2, 19). Le replicaron: \textquote{Cuarenta y seis años se ha costado construir este templo, ¿y tú lo vas a levantar en tres días?} (\textit{Jn} 2, 20). No habían comprendido que el Señor se refería al templo vivo de su cuerpo, que sería destruido con la muerte en la cruz, pero que resucitaría al tercer día. Por eso, \textquote{en tres días}. \textquote{Cuando resucitó de entre los muertos –comenta el evangelista–, los discípulos se acordaron de que lo había dicho, y creyeron a la Escritura y a la palabra que había dicho Jesús} (\textit{Jn} 2, 22).

En efecto, este gesto de Jesús y su mensaje profético se comprenden plenamente a la luz de su Pascua. Según el evangelista Juan, este es el primer anuncio de la muerte y resurrección de Cristo: su cuerpo, destruido en la cruz por la violencia del pecado, se convertirá con la Resurrección en lugar de la cita universal entre Dios y los hombres. Cristo resucitado es precisamente el lugar de la cita universal –de todos– entre Dios y los hombres. Por eso su humanidad es el verdadero templo en el que Dios se revela, habla, se lo puede encontrar; y los verdaderos adoradores de Dios no son los custodios del templo material, los detentadores del poder o del saber religioso, sino los que adoran a Dios \textquote{en espíritu y verdad} (\textit{Jn} 4, 23).

En este tiempo de Cuaresma nos estamos preparando para la celebración de la Pascua, en la que renovaremos las promesas de nuestro bautismo. Caminemos en el mundo como Jesús y hagamos de toda nuestra existencia un signo de su amor para nuestros hermanos, especialmente para los más débiles y los más pobres, construyamos para Dios un templo en nuestra vida. Y así lo hacemos \textquote{encontrable} para muchas personas que encontramos en nuestro camino. Si somos testigos de este Cristo vivo, mucha gente encontrará a Jesús en nosotros, en nuestro testimonio. Pero –nos preguntamos, y cada uno de nosotros puede preguntarse–, ¿se siente el Señor verdaderamente como en su casa en mi vida? ¿Le permitimos que haga \textquote{limpieza} en nuestro corazón y expulse a los ídolos, es decir, las actitudes de codicia, celos, mundanidad, envidia, odio, la costumbre de murmurar y \textquote{despellejar} a los demás? ¿Le permito que haga limpieza de todos los comportamientos contra Dios, contra el prójimo y contra nosotros mismos, como hemos escuchado hoy en la \textbf{primera lectura}? Cada uno puede responder a sí mismo, en silencio, en su corazón. \textquote{¿Permito que Jesús haga un poco de limpieza en mi corazón?}. \textquote{Oh padre, tengo miedo de que me reprenda}. Pero Jesús no reprende jamás. Jesús hará limpieza con ternura, con misericordia, con amor. La misericordia es su modo de hacer limpieza. Dejemos –cada uno de nosotros–, dejemos que el Señor entre con su misericordia –no con el látigo, no, sino con su misericordia– para hacer limpieza en nuestros corazones. El látigo de Jesús para nosotros es su misericordia. Abrámosle la puerta, para que haga un poco de limpieza.

Cada Eucaristía que celebramos con fe nos hace crecer como templo vivo del Señor, gracias a la comunión con su Cuerpo crucificado y resucitado. Jesús conoce lo que hay en cada uno de nosotros, y también conoce nuestro deseo más ardiente: el de ser habitados por Él, sólo por Él. Dejémoslo entrar en nuestra vida, en nuestra familia, en nuestro corazón. Que María santísima, morada privilegiada del Hijo de Dios, nos acompañe y nos sostenga en el itinerario cuaresmal, para que redescubramos la belleza del encuentro con Cristo, que nos libera y nos salva.
\end{body}

\begin{patercite}
Cristo Jesús es nuestro sumo sacerdote, y su precioso cuerpo, que inmoló en el ara de la cruz por la salvación de todos los hombres, es nuestro sacrificio. La sangre que se derramó para nuestra redención no fue la de los becerros y los machos cabríos (como en la ley antigua), sino la del inocentísimo Cordero, Cristo Jesús, nuestro salvador. El templo en el que nuestro sumo sacerdote ofrecía el sacrificio no era hecho por manos de hombres, sino que había sido levantado por el solo poder de Dios; pues Cristo derramó su sangre a la vista del mundo: un templo ciertamente edificado por la sola mano de Dios. Y este templo tiene dos partes: una es la tierra, que ahora nosotros habitamos; la otra nos es aún desconocida a nosotros, mortales.

Así, primero, ofreció su sacrificio aquí en la tierra, cuando sufrió la más acerba muerte. Luego, cuando revestido de la nueva vestidura de la inmortalidad entró por su propia sangre en el santuario, o sea, en el cielo, presentó ante el trono del Padre celestial aquella sangre de inmenso valor, que había derramado una vez para siempre en favor de todos los hombres, pecadores.

Este sacrificio resultó tan grato y aceptable a Dios, que así que lo hubo visto, compadecido inmediatamente de nosotros, no pudo menos que otorgar su perdón a todos los verdaderos penitentes.

De este santo y definitivo sacrificio se hacen partícipes todos aquellos que llegaron a tener verdadera contrición y aceptaron la penitencia por sus crímenes, aquellos que con firmeza decidieron no repetir en adelante sus maldades, sino que perseveran con constancia en el inicial propósito de las virtudes. (\ldots)

\textbf{San Juan Fisher}, \textit{Comentario} sobre el Salmo 129, Opera omnia, ed. 1579 p. 1610.
\end{patercite}
\label{b2-03-03-2015A}

\newpage 
\subsubsection{Ángelus (2018): Peligro de instrumentalizar a Dios}

\src{Plaza de San Pedro. \\4 de marzo del 2018.}

\begin{body}
\ltr{E}{l} \textbf{Evangelio} de hoy presenta, en la versión de \textbf{Juan}, el episodio en el que Jesús expulsa a los vendedores del templo de Jerusalén (cf. \textit{Jn} 2, 13-25). Él hizo este gesto ayudándose con un látigo, volcó las mesas y dijo: \textquote{No hagáis de la Casa de mi Padre una casa de mercado} (\textit{Jn} 2, 16). Esta acción decidida, realizada en proximidad de la Pascua, suscitó gran impresión en la multitud y la hostilidad de las autoridades religiosas y de los que se sintieron amenazados en sus intereses económicos. Pero, ¿cómo debemos interpretarla? Ciertamente no era una acción violenta, tanto es verdad que no provocó la intervención de los tutores del orden público: de la policía. ¡No! Sino que fue entendida como una acción típica de los profetas, los cuales a menudo denunciaban, en nombre de Dios, abusos y excesos. La cuestión que se planteaba era la de la autoridad. De hecho los judíos preguntaron a Jesús: \textquote{¿Qué señal nos muestras para obrar así?} (\textit{Jn} 2, 18), es decir ¿qué autoridad tienes para hacer estas cosas? Como pidiendo la demostración de que Él actuaba en nombre de Dios. Para interpretar el gesto de Jesús de purificar la casa de Dios, sus discípulos usaron un texto bíblico tomado del salmo 69: \textquote{El celo por tu casa me devorará} (\textit{Sal} 69, 17); así dice el salmo: \textquote{pues me devora el celo de tu casa}. Este salmo es una invocación de ayuda en una situación de extremo peligro a causa del odio de los enemigos: la situación que Jesús vivirá en su pasión. El celo por el Padre y por su casa lo llevará hasta la cruz: su celo es el del amor que lleva al sacrificio de sí, no el falso que presume de servir a Dios mediante la violencia. De hecho, el \textquote{signo} que Jesús dará como prueba de su autoridad será precisamente su muerte y resurrección: \textquote{Destruid este santuario –dice– y en tres días lo levantaré} (\textit{Jn} 2, 19). Y el evangelista anota: \textquote{Él hablaba del Santuario de su cuerpo} (\textit{Jn} 2, 21). Con la Pascua de Jesús inicia el nuevo culto en el nuevo templo, el culto del amor, y el nuevo templo es Él mismo.

La actitud de Jesús contada en la actual página evangélica, nos exhorta a vivir nuestra vida no en la búsqueda de nuestras ventajas e intereses, sino por la gloria de Dios que es el amor. Somos llamados a tener siempre presentes esas palabras fuertes de Jesús: \textquote{No hagáis de la Casa de mi Padre una casa de mercado} (\textit{Jn} 2, 16). Es muy feo cuando la Iglesia se desliza hacia esta actitud de hacer de la casa de Dios un mercado. Estas palabras nos ayudan a rechazar el peligro de hacer también de nuestra alma, que es la casa de Dios, un lugar de mercado que viva en la continua búsqueda de nuestro interés en vez de en el amor generoso y solidario. Esta enseñanza de Jesús es siempre actual, no solamente para las comunidades eclesiales, sino también para los individuos, para las comunidades civiles y para toda la sociedad. Es común, de hecho, la tentación de aprovechar las buenas actividades, a veces necesarias, para cultivar intereses privados, o incluso ilícitos. Es un peligro grave, especialmente cuando instrumentaliza a Dios mismo y el culto que se le debe a Él, o el servicio al hombre, su imagen. Por eso Jesús esa vez usó \textquote{las maneras fuertes}, para sacudirnos de este peligro mortal. Que la Virgen María nos sostenga en el compromiso de hacer de la Cuaresma una buena ocasión para reconocer a Dios como único Señor de nuestra vida, quitando de nuestro corazón y de nuestras obras todo tipo de idolatría.
\end{body}

\begin{patercite}
Él vino por su benignidad hacia nosotros y se nos hizo visible. Tuvo piedad de nuestra raza y de nuestra debilidad y, compadecido de nuestra corrupción, no soportó que la muerte nos dominase, para que no pereciese lo que había sido creado, con lo que hubiera resultado inútil la obra de su Padre al crear al hombre, y por esto tomó para sí un cuerpo como el nuestro, ya que no se contentó con habitar en un cuerpo ni tampoco en hacerse simplemente visible. (\ldots) 

En el seno de la Virgen, se construyó un templo, es decir, su cuerpo, y lo hizo su propio instrumento, en el que había de darse a conocer y habitar; de este modo, habiendo tomado un cuerpo semejante al de cualquiera de nosotros, ya que todos estaban sujetos a la corrupción de la muerte, lo entregó a la muerte por todos, ofreciéndolo al Padre con un amor sin límites; con ello, al morir en su persona todos los hombres, quedó sin vigor la ley de la corrupción que afectaba a todos, ya que agotó toda la eficacia de la muerte en el cuerpo del Señor; y así ya no le quedó fuerza alguna para ensañarse con los demás hombres, semejantes a él; con ello, también hizo de nuevo incorruptibles a los hombres, que habían caído en la corrupción, y los llamó de muerte a vida, consumiendo totalmente en ellos la muerte, con el cuerpo que había asumido y con el poder de su resurrección, del mismo modo que la paja es consumida por el fuego.

De ahí que el cuerpo que él había tomado, al entregarlo a la muerte como una hostia y víctima limpia de toda mancha, alejó al momento la muerte de todos los hombres, a los que él se había asemejado, ya que se ofreció en lugar de ellos.

De este modo, el Verbo de Dios, superior a todo lo que existe, ofreciendo en sacrificio su cuerpo, templo e instrumento de su divinidad, pagó con su muerte la deuda que habíamos contraído, y, así, el Hijo de Dios, inmune a la corrupción, por la promesa de la resurrección, hizo partícipes de esta misma inmunidad a todos los hombres, con los que se había hecho una misma cosa por su cuerpo semejante al de ellos.

\textbf{San Atanasio, obispo}, \textit{Sermón} sobre la encarnación del Verbo, 8-9: PG 25,110-111 (Liturgia de las Horas, Común de Pastores).
\end{patercite}

\newsection
\section{Temas}

\cceth{Jesús y la Ley} 
\cceref{CEC 459, 577-582}

\begin{ccebody}
\n{459} El Verbo se encarnó \textit{para ser nuestro modelo de santidad}: \textquote{Tomad sobre vosotros mi yugo, y aprended de mí \ldots} (\textit{Mt} 11, 29). \textquote{Yo soy el Camino, la Verdad y la Vida. Nadie va al Padre sino por mí} (\textit{Jn} 14, 6). Y el Padre, en el monte de la Transfiguración, ordena: \textquote{Escuchadle} (\textit{Mc} 9, 7; cf. \textit{Dt} 6, 4-5). Él es, en efecto, el modelo de las bienaventuranzas y la norma de la Ley nueva: \textquote{Amaos los unos a los otros como yo os he amado} (\textit{Jn} 15, 12). Este amor tiene como consecuencia la ofrenda efectiva de sí mismo (cf. \textit{Mc} 8, 34).

\ccesec{Jesús y la Ley}

\n{577} Al comienzo del Sermón de la Montaña, Jesús hace una advertencia solemne presentando la Ley dada por Dios en el Sinaí con ocasión de la Primera Alianza, a la luz de la gracia de la Nueva Alianza:

\ccecite{\textquote{No penséis que he venido a abolir la Ley y los Profetas. No he venido a abolir sino a dar cumplimiento. Sí, os lo aseguro: el cielo y la tierra pasarán antes que pase una ‘i’ o un ápice de la Ley sin que todo se haya cumplido. Por tanto, el que quebrante uno de estos mandamientos menores, y así lo enseñe a los hombres, será el menor en el Reino de los cielos; en cambio el que los observe y los enseñe, ése será grande en el Reino de los cielos} (\textit{Mt} 5, 17-19).}

\n{578} Jesús, el Mesías de Israel, por lo tanto el más grande en el Reino de los cielos, se debía sujetar a la Ley cumpliéndola en su totalidad hasta en sus menores preceptos, según sus propias palabras. Incluso es el único en poderlo hacer perfectamente (cf. \textit{Jn} 8, 46). Los judíos, según su propia confesión, jamás han podido cumplir la Ley en su totalidad, sin violar el menor de sus preceptos (cf. \textit{Jn} 7, 19; \textit{Hch} 13, 38-41; 15, 10). Por eso, en cada fiesta anual de la Expiación, los hijos de Israel piden perdón a Dios por sus transgresiones de la Ley. En efecto, la Ley constituye un todo y, como recuerda Santiago, \textquote{quien observa toda la Ley, pero falta en un solo precepto, se hace reo de todos} (\textit{St} 2, 10; cf. \textit{Ga} 3, 10; 5, 3).

\n{579} Este principio de integridad en la observancia de la Ley, no sólo en su letra sino también en su espíritu, era apreciado por los fariseos. Al subrayarlo para Israel, muchos judíos del tiempo de Jesús fueron conducidos a un celo religioso extremo (cf. \textit{Rm} 10, 2), el cual, si no quería convertirse en una casuística \textquote{hipócrita} (cf. \textit{Mt} 15, 3-7; \textit{Lc} 11, 39-54) no podía más que preparar al pueblo a esta intervención inaudita de Dios que será la ejecución perfecta de la Ley por el único Justo en lugar de todos los pecadores (cf. \textit{Is} 53, 11; \textit{Hb} 9, 15).

\n{580} El cumplimiento perfecto de la Ley no podía ser sino obra del divino Legislador que nació sometido a la Ley en la persona del Hijo (cf. \textit{Ga} 4, 4). En Jesús la Ley ya no aparece grabada en tablas de piedra sino \textquote{en el fondo del corazón} (\textit{Jr} 31, 33) del Siervo, quien, por \textquote{aportar fielmente el derecho} (\textit{Is} 42, 3), se ha convertido en \textquote{la Alianza del pueblo} (\textit{Is} 42, 6). Jesús cumplió la Ley hasta tomar sobre sí mismo \textquote{la maldición de la Ley} (\textit{Ga} 3, 13) en la que habían incurrido los que no \textquote{practican todos los preceptos de la Ley} (\textit{Ga} 3, 10) porque \textquote{ha intervenido su muerte para remisión de las transgresiones de la Primera Alianza} (\textit{Hb} 9, 15).

\n{581} Jesús fue considerado por los judíos y sus jefes espirituales como un \textquote{rabbi} (cf. \textit{Jn} 11, 28; 3, 2; \textit{Mt} 22, 23-24. 34-36). Con frecuencia argumentó en el marco de la interpretación rabínica de la Ley (cf. \textit{Mt} 12, 5; 9, 12; \textit{Mc} 2, 23-27; \textit{Lc} 6, 6-9; \textit{Jn} 7, 22-23). Pero al mismo tiempo, Jesús no podía menos que chocar con los doctores de la Ley porque no se contentaba con proponer su interpretación entre los suyos, sino que \textquote{enseñaba como quien tiene autoridad y no como los escribas} (\textit{Mt} 7, 28-29). La misma Palabra de Dios, que resonó en el Sinaí para dar a Moisés la Ley escrita, es la que en Él se hace oír de nuevo en el Monte de las Bienaventuranzas (cf. \textit{Mt} 5, 1). Esa palabra no revoca la Ley sino que la perfecciona aportando de modo divino su interpretación definitiva: \textquote{Habéis oído también que se dijo a los antepasados [\ldots] pero yo os digo} (\textit{Mt} 5, 33-34). Con esta misma autoridad divina, desaprueba ciertas \textquote{tradiciones humanas} (\textit{Mc} 7, 8) de los fariseos que \textquote{anulan la Palabra de Dios} (\textit{Mc} 7, 13).

\n{582} Yendo más lejos, Jesús da plenitud a la Ley sobre la pureza de los alimentos, tan importante en la vida cotidiana judía, manifestando su sentido \textquote{pedagógico} (cf. \textit{Ga} 3, 24) por medio de una interpretación divina: \textquote{Todo lo que de fuera entra en el hombre no puede hacerle impuro [\ldots] –así declaraba puros todos los alimentos–. Lo que sale del hombre, eso es lo que hace impuro al hombre. Porque de dentro, del corazón de los hombres, salen las intenciones malas} (\textit{Mc} 7, 18-21). Jesús, al dar con autoridad divina la interpretación definitiva de la Ley, se vio enfrentado a algunos doctores de la Ley que no aceptaban su interpretación a pesar de estar garantizada por los signos divinos con que la acompañaba (cf. \textit{Jn} 5, 36; 10, 25. 37-38; 12, 37). Esto ocurre, en particular, respecto al problema del sábado: Jesús recuerda, frecuentemente con argumentos rabínicos (cf. \textit{Mt} 2,25-27; \textit{Jn} 7, 22-24), que el descanso del sábado no se quebranta por el servicio de Dios (cf. \textit{Mt} 12, 5; \textit{Nm} 28, 9) o al prójimo (cf. \textit{Lc} 13, 15-16; 14, 3-4) que realizan sus curaciones.
\end{ccebody}

\cceth{El Templo prefigura a Cristo; Él es el Templo} 
\cceref{CEC 593, 583-586}

\begin{ccebody}
\n{593} \textit{Jesús veneró el Templo subiendo a él en peregrinación en las fiestas judías y amó con gran celo esa morada de Dios entre los hombres. El Templo prefigura su Misterio. Anunciando la destrucción del Templo anuncia su propia muerte y la entrada en una nueva edad de la historia de la salvación, donde su cuerpo será el Templo definitivo}.

\ccesec{Jesús y el Templo}

\n{583} Como los profetas anteriores a Él, Jesús profesó el más profundo respeto al Templo de Jerusalén. Fue presentado en él por José y María cuarenta días después de su nacimiento (\textit{Lc} 2, 22-39). A la edad de doce años, decidió quedarse en el Templo para recordar a sus padres que se debía a los asuntos de su Padre (cf. \textit{Lc} 2, 46-49). Durante su vida oculta, subió allí todos los años al menos con ocasión de la Pascua (cf. \textit{Lc} 2, 41); su ministerio público estuvo jalonado por sus peregrinaciones a Jerusalén con motivo de las grandes fiestas judías (cf. \textit{Jn} 2, 13-14; 5, 1. 14; 7, 1. 10. 14; 8, 2; 10, 22-23).

\n{584} Jesús subió al Templo como al lugar privilegiado para el encuentro con Dios. El Templo era para Él la casa de su Padre, una casa de oración, y se indigna porque el atrio exterior se haya convertido en un mercado (\textit{Mt} 21, 13). Si expulsa a los mercaderes del Templo es por celo hacia las cosas de su Padre: \textquote{No hagáis de la Casa de mi Padre una casa de mercado. Sus discípulos se acordaron de que estaba escrito: \textit{El celo por tu Casa me devorará} (\textit{Sal} 69, 10)} (\textit{Jn} 2, 16-17). Después de su Resurrección, los Apóstoles mantuvieron un respeto religioso hacia el Templo (cf. \textit{Hch} 2, 46; 3, 1; 5, 20. 21).

\n{585} Jesús anunció, no obstante, en el umbral de su Pasión, la ruina de ese espléndido edificio del cual no quedará piedra sobre piedra (cf. \textit{Mt} 24, 1-2). Hay aquí un anuncio de una señal de los últimos tiempos que se van a abrir con su propia Pascua (cf. \textit{Mt} 24, 3; \textit{Lc} 13, 35). Pero esta profecía pudo ser deformada por falsos testigos en su interrogatorio en casa del sumo sacerdote (cf. \textit{Mc} 14, 57-58) y serle reprochada como injuriosa cuando estaba clavado en la cruz (cf. \textit{Mt} 27, 39-40).

\n{586} Lejos de haber sido hostil al Templo (cf. \textit{Mt} 8, 4; 23, 21; \textit{Lc} 17, 14; \textit{Jn} 4, 22) donde expuso lo esencial de su enseñanza (cf. \textit{Jn} 18, 20), Jesús quiso pagar el impuesto del Templo asociándose con Pedro (cf. \textit{Mt} 17, 24-27), a quien acababa de poner como fundamento de su futura Iglesia (cf. \textit{Mt} 16, 18). Aún más, se identificó con el Templo presentándose como la morada definitiva de Dios entre los hombres (cf. \textit{Jn} 2, 21; \textit{Mt} 12, 6). Por eso su muerte corporal (cf. \textit{Jn} 2, 18-22) anuncia la destrucción del Templo que señalará la entrada en una nueva edad de la historia de la salvación: \textquote{Llega la hora en que, ni en este monte, ni en Jerusalén adoraréis al Padre} (\textit{Jn} 4, 21; cf. \textit{Jn} 4, 23-24; \textit{Mt} 27, 51; \textit{Hb} 9, 11; \textit{Ap} 21, 22).
\end{ccebody}

\cceth{La nueva Ley completa la antigua} 
\cceref{CEC 1967-1968}

\begin{ccebody}
\n{1967} La Ley evangélica \textquote{da cumplimiento} (cf. \textit{Mt} 5, 17-19), purifica, supera, y lleva a su perfección la Ley antigua. En las \textquote{Bienaventuranzas} \textit{da cumplimiento a las promesas} divinas elevándolas y ordenándolas al \textquote{Reino de los cielos}. Se dirige a los que están dispuestos a acoger con fe esta esperanza nueva: los pobres, los humildes, los afligidos, los limpios de corazón, los perseguidos a causa de Cristo, trazando así los caminos sorprendentes del Reino.

\n{1968} La Ley evangélica \textit{lleva a plenitud los mandamientos} de la Ley. El Sermón del monte, lejos de abolir o devaluar las prescripciones morales de la Ley antigua, extrae de ella sus virtualidades ocultas y hace surgir de ella nuevas exigencias: revela toda su verdad divina y humana. No añade preceptos exteriores nuevos, pero llega a reformar la raíz de los actos, el corazón, donde el hombre elige entre lo puro y lo impuro (cf. \textit{Mt} 15, 18-19), donde se forman la fe, la esperanza y la caridad, y con ellas las otras virtudes. El Evangelio conduce así la Ley a su plenitud mediante la imitación de la perfección del Padre celestial (cf. \textit{Mt} 5, 48), mediante el perdón de los enemigos y la oración por los perseguidores, según el modelo de la generosidad divina (cf. \textit{Mt} 5, 44).
\end{ccebody}

\cceth{La potencia de Cristo revelada en la cruz} 
\cceref{CEC 272, 550, 853}

\begin{ccebody}
\ccesec{El misterio de la aparente impotencia de Dios}

\n{272} La fe en Dios Padre Todopoderoso puede ser puesta a prueba por la experiencia del mal y del sufrimiento. A veces Dios puede parecer ausente e incapaz de impedir el mal. Ahora bien, Dios Padre ha revelado su omnipotencia de la manera más misteriosa en el anonadamiento voluntario y en la Resurrección de su Hijo, por los cuales ha vencido el mal. Así, Cristo crucificado es \textquote{poder de Dios y sabiduría de Dios. Porque la necedad divina es más sabia que la sabiduría de los hombres, y la debilidad divina, más fuerte que la fuerza de los hombres} (\textit{1 Co} 2, 24-25). En la Resurrección y en la exaltación de Cristo es donde el Padre \textquote{desplegó el vigor de su fuerza} y manifestó \textquote{la soberana grandeza de su poder para con nosotros, los creyentes} (\textit{Ef} 1, 19-22).

\n{550} La venida del Reino de Dios es la derrota del reino de Satanás (cf. \textit{Mt} 12, 26): \textquote{Pero si por el Espíritu de Dios expulso yo los demonios, es que ha llegado a vosotros el Reino de Dios} (\textit{Mt} 12, 28). Los \textit{exorcismos} de Jesús liberan a los hombres del dominio de los demonios (cf. \textit{Lc} 8, 26-39). Anticipan la gran victoria de Jesús sobre \textquote{el príncipe de este mundo} (\textit{Jn} 12, 31). Por la Cruz de Cristo será definitivamente establecido el Reino de Dios: \textit{Regnavit a ligno Deus} – \textquote{Dios reinó desde el madero de la Cruz}, (Venancio Fortunato, \textit{Hymnus \textquote{Vexilla Regis}}: MGH 1/4/1, 34: PL 88, 96).

\n{853} Pero en su peregrinación, la Iglesia experimenta también \textquote{hasta qué punto distan entre sí el mensaje que ella proclama y la debilidad humana de aquellos a quienes se confía el Evangelio} (GS 43, 6). Sólo avanzando por el camino \textquote{de la conversión y la renovación} (LG 8; cf. ibíd.,15) y \textquote{por el estrecho sendero de la cruz} (AG 1) es como el Pueblo de Dios puede extender el reino de Cristo (cf. RM 12-20). En efecto, \textquote{como Cristo realizó la obra de la redención en la pobreza y en la persecución, también la Iglesia está llamada a seguir el mismo camino para comunicar a los hombres los frutos de la salvación} (LG 8).
\end{ccebody}

\img{cross_panyvino}
	
	\iffalse
	
		
	\fi 
	
\end{document}