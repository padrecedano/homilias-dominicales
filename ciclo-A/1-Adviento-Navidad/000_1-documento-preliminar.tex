\chapter{Domingo I de Adviento (A)}

\section{Lecturas}

\rtitle{PRIMERA LECTURA}

\rbook{Del libro del profeta Isaías} \rred{2, 1-5}

\rtheme{Visión de Isaías, hijo de Amós, acerca de Judá y de Jerusalén}

\begin{readprose}
 En los días futuros estará firme

 el monte de la casa del Señor,

 en la cumbre de las montañas,

 más elevado que las colinas.

 Hacia él confluirán todas las naciones,

 caminarán pueblos numerosos y dirán:

 «Venid, subamos al monte del Señor,

 a la casa del Dios de Jacob.

 Él nos instruirá en sus caminos

 y marcharemos por sus sendas;

 porque de Sión saldrá la ley,

 la palabra del Señor de Jerusalén».

 Juzgará entre las naciones,

 será árbitro de pueblos numerosos.

 De las espadas forjarán arados,

 de las lanzas, podaderas.

 No alzará la espada pueblo contra pueblo,

 no se adiestrarán para la guerra.

 Casa de Jacob, venid;

 caminemos a la luz del Señor.
\end{readprose}

\rtitle{SALMO RESPONSORIAL}

\rbook{Salmo} \rred{121, 1-2. 4-9}

\rtheme{Qué alegría cuando me dijeron: \textquote{Vamos a la casa del Señor}}

\begin{psbody}
 ¡Qué alegría cuando me dijeron:
 \textquote{Vamos a la casa del Señor}!
 Ya están pisando nuestros pies
 tus umbrales, Jerusalén.

 Allá suben las tribus,
 las tribus del Señor,
 según la costumbre de Israel,
 a celebrar el nombre del Señor;
 en ella están los tribunales de justicia,
 en el palacio de David.

 Desead la paz a Jerusalén:
 \textquote{Vivan seguros los que te aman,
 haya paz dentro de tus muros,
 seguridad en tus palacios}.

 Por mis hermanos y compañeros,
 voy a decir: \textquote{La paz contigo}.
 Por la casa del Señor,
 nuestro Dios, te deseo todo bien.
\end{psbody}

SEGUNDA LECTURA

De la carta del apóstol san Pablo a los Romanos 13, 11-14

Nuestra salvación está cerca

Hermanos: Comportaos reconociendo el momento en que vivís, pues ya es hora de despertaros del sueño, porque ahora la salvación está más cerca de nosotros que cuando abrazamos la fe. La noche está avanzada, el día está cerca: dejemos, pues, las obras de las tinieblas y pongámonos las armas de la luz.

Andemos como en pleno día, con dignidad. Nada de comilonas y borracheras, nada de lujuria y desenfreno, nada de riñas y envidias. Revestíos más bien del Señor Jesucristo.

EVANGELIO

Del Santo Evangelio según san Mateo 24, 37-44

Vigilemos para estar preparados

En aquel tiempo, dijo Jesús a sus discípulos:

«Cuando venga el Hijo del hombre, pasará como en tiempo de Noé.

En los días antes del diluvio, la gente comía y bebía, se casaban los hombres y las mujeres tomaban esposo, hasta el día en que Noé entró en el arca; y cuando menos lo esperaban llegó el diluvio y se los llevó a todos; lo mismo sucederá cuando venga el Hijo del hombre: dos hombres estarán en el campo, a uno se lo llevarán y a otro lo dejarán; dos mujeres estarán moliendo, a una se la llevarán y a otra la dejarán.

Por tanto, estad en vela, porque no sabéis qué día vendrá vuestro Señor. Comprended que si supiera el dueño de casa a qué hora de la noche viene el ladrón, estaría en vela y no dejaría que abrieran un boquete en su casa. Por eso, estad también vosotros preparados, porque a la hora que menos penséis viene el Hijo del hombre».

\section{Comentario Patrístico}

\subsection{Pascasio Radberto}

Velad, para estar preparados

Exposición sobre el evangelio de san Mateo, Lib. 11, cap. 24:

PL 120, 799-800.

\emph{Velad, porque no sabéis el día ni la hora}. Siendo una recomendación que a todos afecta, la expresa como si solamente se refiriera a los hombres de aquel entonces. Es lo que ocurre con muchos otros pasajes que leemos en las Escrituras. Y de tal modo atañe a todos lo así expresado, que a cada uno le llega el último día y para cada cual es el fin del mundo el momento mismo de su muerte. Por eso es necesario que cada uno parta de este mundo tal cual ha de ser juzgado aquel día. En consecuencia, todo hombre debe cuidar de no dejarse seducir ni abandonar la vigilancia, no sea que el día de la venida del Señor lo encuentre desprevenido.

Y aquel día encontrará desprevenido a quien hallare desprevenido el último día de su vida. Pienso que los apóstoles estaban convencidos de que el Señor no iba a presentarse en sus días para el juicio final; y sin embargo, ¿quién dudará de que ellos cuidaron de no dejarse seducir, de que no abandonaron la vigilancia y de que observaron todo lo que a todos fue recomendado, para que el Señor los hallara preparados? Por esta razón, debemos tener siempre presente una doble venida de Cristo: una, cuando aparezca de nuevo y hayamos de dar cuenta de todos nuestros actos; otra diaria, cuando a todas horas visita nuestras conciencias y viene a nosotros, para que cuando viniere, nos encuentre preparados.

¿De qué me sirve, en efecto, conocer el día del juicio si soy consciente de mis muchos pecados?, ¿conocer si viene o cuándo viene el Señor, si antes no viniere a mi alma y retornare a mi espíritu?, ¿si antes no vive Cristo en mí y me habla? Sólo entonces será su venida un bien para mí, si primero Cristo vive en mí y yo vivo en Cristo. Y sólo entonces vendrá a mí, como en una segunda venida, cuando, muerto para el mundo, pueda en cierto modo hacer mía aquella expresión: \emph{El mundo está crucificado para mí, y yo para el mundo}.

Considera asimismo estas palabras de Cristo: \emph{Porque muchos vendrán usando mi nombre}. Sólo el anticristo y sus secuaces se arrogan falsamente el nombre de Cristo, pero sin las obras de Cristo, sin sus palabras de verdad, sin su sabiduría. En ninguna parte de la Escritura hallarás que el Señor haya usado esta expresión y haya dicho: \emph{Yo soy el Cristo}. Le bastaba mostrar con su doctrina y sus milagros lo que era realmente, pues las obras del Padre que realizaba, la doctrina que enseñaba y su poder gritaban: \emph{Yo} soy \emph{el Cristo} con más eficacia que si mil voces lo pregonaran. Cristo, que yo sepa, jamás se atribuyó verbalmente este título: lo hizo realizando las obras del Padre y enseñando la ley del amor. En cambio, los falsos cristos, careciendo de esta ley del amor, proclamaban de palabra ser lo que no eran.

\textbf{Rorate Caeli}

Rorate Caeli desúper\\ et nubes plúant justum

Ne irascáris Dómine,\\ ne ultra memíneris iniquitátis\\ Ecce cívitas Sancti\\ facta est desérta\\ Sion desérta facta est,\\ Jerúsalem desoláta est.\\ Domus sanctificatiónis tuae et gloriae tuae\\ Ubi laudavérunt Te patres nostri.

Peccávimus et facti sumus\\ tamquam immúndus nos,\\ Et cecídimus quasi fólium univérsi\\ Et iniquitátes nostrae quasi ventus\\ abstulérunt nos\\ Abscondísti fáciem tuam a nobis\\ Et allisísti nos\\ in mánu iniquitátis nostrae.

Víde, Dómine, afflictiónem pópuli tui\\ Et mitte quem missúrus es\\ Emítte Agnum dominatórem terrae\\ De pétra desérti\\ ad montem fíliae Sion\\ Ut áuferat ipse jugum\\ captivitátis nostrae.

Consolámini, consolámini, pópule meus\\ Cito véniet salus tua\\ Quare moeróre consúmeris,\\ quia innovávit te dolor?\\ Salvábo te, noli timére\\ Ego énim sum Dóminus Deus túus\\ Sánctus Israël, Redémptor túus.\strut



Derramad, oh cielos, vuestro rocío de lo alto,\\ y las nubes lluevan al Justo.

No te enfades, Señor,\\ ni te acuerdes de la iniquidad.\\ He aquí que la ciudad del Santuario\\ quedó desierta:\\ Sión quedó desierta;\\ Jerusalén está desolada.\\ La casa de tu santidad y de tu gloria,\\ donde nuestros padres te alabaron.

Pecamos y nos volvimos\\ como los inmundos,\\ Y caímos, todos, como hojas.\\ Y nuestra iniquidades, como un viento,\\ nos dispersaron.\\ Ocultaste de nosotros tu rostro\\ Y nos castigaste\\ por mano de nuestras iniquidades.

¡Mira, Señor, la aflicción de tu pueblo!,\\ Y envíale a Aquel que vas a enviar!\\ Envíale al Cordero dominador de la tierra\\ Del desierto de piedra\\ al monte de la hija de Sión\\ Para que Él retire el yugo\\ de nuestro cautiverio.

Consuélate, consuélate, pueblo mío,\\ ¡En breve ha de llegar tu salvación!\\ ¿Por qué te consumes en la tristeza,\\ por qué tu dolor?\\ ¡Yo te salvaré, no tengas miedo!\\ Porque Yo soy el Señor, tu Dios,\\ El Santo de Israel, tu Redentor.\strut



\section{Homilías}

\subsection{San Juan Pablo II, papa}

\subsubsection{Homilía (1980): Nueva llamada a vestirse de Cristo}

Visita Pastoral a la Parroquia de San Leonardo de Porto Mauricio.

Domingo 30 de noviembre de 1980.

1. Al escuchar las palabras del Evangelio de hoy según Mateo, ante nuestros ojos vienen espontáneamente a la memoria los acontecimientos que durante la semana pasada han sacudido a toda Italia\footnote{Se refiere a un gran terremoto que afectó las regiones italianas de Campania y de Basilicata, desde Potenza a Avellino, hasta el litoral, a los Puertos de Nápoles y Salerno}.\ldots{} Mientras nosotros todos, con espíritu de solidaridad humana, queremos ayudar a nuestros hermanos y compatriotas, arrollados por la desgracia, al mismo tiempo, estos acontecimientos traen ante nuestros ojos, con una particular fuerza comparativa, el cuadro terrible que cada año trazan los \textbf{Evangelios} de este primer domingo de Adviento: anuncios de destrucción y de muerte, en la espera escatológica de la \textquote{venida del Hijo del Hombre} (Mt 24, 39).

2. La historia de los hombres y de las naciones, la historia de toda la humanidad suministra pruebas suficientes para afirmar que en todos los tiempos se han multiplicado desgracias y catástrofes, calamidades naturales, como terremotos, o las causadas por el hombre, como guerras, revoluciones, estragos, homicidios y genocidios. Además, cada uno de nosotros sabe que nuestra existencia terrena lleva a la muerte, llegando así un día a su término. El mundo visible, con todos los bienes y las riquezas que oculta en sí mismo, al fin no es capaz de darnos más que la muerte: el término de la vida.

Esta verdad, aunque nos la recuerda también la liturgia de hoy, primer domingo de Adviento, sin embargo, no es la verdad específica anunciada en este día festivo, y en todo el período de Adviento. No es la palabra principal del \textbf{Evangelio}.

¿Cuál es, pues, la palabra principal? La hemos leído hace poco: la venida del Hijo del Hombre. La palabra principal del Evangelio no es \textquote{la separación}, \textquote{la ausencia}, sino \textquote{la venida} y \textquote{la presencia}. Ni siquiera es la \textquote{muerte}, sino la \textquote{vida}. El Evangelio es la Buena Noticia, porque pronuncia la verdad sobre la vida en el contexto de la muerte.

La venida del Hijo del Hombre es el comienzo de esta Vida. Y de este comienzo nos habla precisamente el Adviento, que responde a la pregunta: ¿cómo debe vivir el hombre en el mundo con la perspectiva de la muerte? El hombre al que, en un abrir y cerrar de ojos, le puede ser quitada la vida, ¿cómo debe vivir en este mundo, para encontrarse con el Hijo del Hombre, cuya venida es el comienzo de la nueva vida, de la vida más potente que la muerte?

4. Nos encontramos, pues, todos en el primer domingo de Adviento. ¿Cuál es esta verdad que nos penetra y vivifica hoy? ¿Qué mensaje nos anuncia la Santa Iglesia, nuestra Madre? Como ya he dicho, no se trata de un mensaje de miedo y de muerte, sino del mensaje de la esperanza y de la llamada.

Tomemos como ejemplo la \textbf{segunda lectura}; he aquí lo que el Apóstol Pablo dice a los romanos de entonces, pero que debemos tomar en serio los romanos de hoy: \textquote{Daos cuenta del momento en que vivís; ya es hora de espabilarse, porque ahora nuestra salvación está más cerca que cuando empezamos a creer. La noche está avanzada, el día se echa encima} (Rm 13, 11-12).

En realidad, al contrario de como podemos ser inducidos a pensar, la salvación está más cercana y no más lejana. Efectivamente, al vivir en una época de secularización, somos testigos de comportamientos de indiferencia religiosa y también de programas e ideologías ateas o incluso antiteístas. Se llegaría a pensar que los indicios humanos desmienten el mensaje de la liturgia de hoy. Ella, en cambio ---aun haciendo referencia también a estos \textquote{indicios humanos}--- proclama, sin embargo, la verdad divina y anuncia el designio divino que no decae jamás, que no cambia aun cuando puedan cambiar los hombres, los programas, los proyectos humanos. Ese designio divino es el designio de la salvación del hombre en Cristo, que, una vez emprendido, perdura, y consiguientemente mira a su cumplimiento.

Pero el hombre puede ser ciego y sordo a todo esto. Puede meterse cada vez más profundamente en la noche, aunque se acerque el día. Puede multiplicar las obras de las tinieblas aunque Cristo le ofrezca \textquote{las armas de la luz}.

Por lo tanto, la invitación apremiante de la liturgia de hoy es la del \textbf{Apóstol}: \textquote{Vestíos del Señor Jesucristo} (Rm 13, 14). Esta expresión es, en cierto sentido, la definición del cristiano. Ser cristiano quiere decir \textquote{vestirse de Cristo}. El Adviento es la nueva llamada a vestirse de Jesucristo.

Dice además el Apóstol: \textquote{Conduzcámonos como en pleno día, con dignidad. Nada de comilonas ni borracheras, nada de lujuria ni desenfreno, nada de riñas ni pendencias\ldots{}, y que el cuidado de vuestro cuerpo no fomente los malos deseos} (Rm 13, 13-14).

5. ¿Qué significa, además, el Adviento? El Adviento es el descubrimiento de una gran aspiración de los hombres y de los pueblos hacia la casa del Señor. No hacia la muerte y la destrucción, sino hacia el encuentro con El.

Y por esto en la liturgia de hoy oímos esta invitación: \textquote{Qué alegría cuando me dijeron: vamos a la casa del Señor}.

Y el mismo \textbf{Salmo responsorial} nos traza, por decirlo así, la imagen de esa casa, de esa ciudad, de ese encuentro: \textquote{Ya están pisando nuestros pies tus umbrales, Jerusalén. Allá suben las tribus, las tribus del Señor. Según la costumbre de Israel, a celebrar el nombre del Señor. En ella están los tribunales de justicia en el palacio de David. Por mis hermanos y compañeros voy a decir: \textquote{La paz contigo}. Por la casa del Señor nuestro Dios, te deseo todo bien} (Sal 121 {[}122{]}).

Sí. El Señor es el Dios de la paz, es el Dios de la Alianza con el hombre. Cuando en la noche de Belén los pobres pastores se pondrán en camino hacia el establo donde se realizará la primera venida del Hijo del Hombre, los conducirá el canto de los ángeles: \textquote{Gloria a Dios en las alturas y paz en la tierra a los hombres de buena voluntad} (Lc 2, 14).

6. Esta visión de la paz divina pertenece a toda la espera mesiánica en la Antigua Alianza. Oímos hoy las palabras de \textbf{Isaías}: \textquote{Será el árbitro de las naciones, el juez de pueblos numerosos. De las espadas forjarán arados; de las lanzas, podaderas. No alzará la espada pueblo contra pueblo, no se adiestrarán para la guerra. Casa de Jacob, ven; caminemos a la luz del Señor} (Is 2, 4-5).

El Adviento trae consigo la invitación a la paz de Dios para todos los hombres. Es necesario que nosotros construyamos esta paz y la reconstruyamos continuamente en nosotros mismos y con los otros: en las familias, en las relaciones con los cercanos, en los ambientes de trabajo, en la vida de toda la sociedad.

Trabajad con espíritu de solidaridad fraterna a fin de que vuestra parroquia crezca cada vez más como comunidad de fieles, de familias, de grupos ---me refiero particularmente a todos vuestros grupos organizados--- en comunión de verdad y de amor. La comunidad parroquial, en efecto, se edifica sobre la Palabra de Dios, transmitida y garantizada por los Pastores, se alimenta por la gracia de los sacramentos, se sostiene por la oración, se une por el vínculo de la caridad fraterna. Que cada uno de sus miembros se sienta vivo, activo, partícipe, corresponsable, implicado en tareas efectivas de evangelización cristiana y de promoción humana. De este modo, vuestra parroquia se convierte en signo e instrumento de la presencia de Cristo en el barrio, irradiación de su amor y de su paz.

Para servir a esta paz de múltiples dimensiones, es necesario escuchar también estas palabras del \textbf{Profeta}: \textquote{Venid, subamos al monte del Señor, a la casa del Dios de Jacob. El nos instruirá en sus caminos y marcharemos por sus sendas, porque de Sión saldrá la ley. de Jerusalén la palabra del Señor} (Is 2, 3).

También para vuestra comunidad eclesial el Adviento es el tiempo en el que se deben aprender de nuevo la ley del Señor y sus palabras. Es el tiempo de una catequesis intensificada. La ley y la palabra del Señor deben penetrar de nuevo en el corazón, deben encontrar de nuevo su confirmación en la vida social. Sirven al bien del hombre. ¡Construyen la paz!

7. Queridos hermanos e hijos: Nos encontramos, pues, de nuevo al comienzo del camino. Ha comenzado de nuevo el Adviento: el tiempo de la gracia, el tiempo de la espera, el tiempo de la venida del Señor, que perdura siempre. Y la vida del hombre se desarrolla en el amor del Señor, a pesar de todas las dolorosas experiencias de la destrucción y de la muerte, hacia la realización final en Dios.

¡El Hijo del Hombre vendrá! Escuchemos estas palabras con la esperanza, no con el miedo, aunque estén llenas de una profunda seriedad.

Velad\ldots{} y estad preparados, porque no sabéis en qué día vendrá el Hijo del Hombre. ¡Ven, Señor Jesús! ¡Marana tha!


\subsubsection{Homilía (1998)}


Basílica de San Pedro\\ Domingo 29 de noviembre de 1998. Convocación del Año Santo.


1. \textquote{Vayamos jubilosos al encuentro del Señor} (\emph{Estribillo del Salmo responsorial}).

Son las palabras del Salmo responsorial de esta liturgia del primer domingo de Adviento, tiempo litúrgico que renueva año tras año la espera de la venida de Cristo. En estos años que estamos viviendo en la perspectiva del tercer milenio, el Adviento ha cobrado una dimensión nueva y singular. \emph{Tertio millennio adveniente}: el año 1998, que está a punto de terminar, y el año próximo 1999 nos acercan al umbral de un nuevo siglo y de un nuevo milenio.

\textquote{En el umbral} ha comenzado también esta celebración: en el umbral de la basílica vaticana, ante la puerta santa, con la entrega y la lectura de la \emph{bula de convocación} del gran jubileo del año 2000.

\textquote{Vayamos jubilosos al encuentro del Señor} es un estribillo que está perfectamente en armonía con el jubileo. Es, por decir así, un \textquote{estribillo jubilar}, según la etimología de la palabra latina \emph{iubilar}, que encierra una referencia al júbilo. ¡Vayamos, pues, con alegría! Caminemos jubilosos y vigilantes a la espera del tiempo que recuerda la venida de Dios en la carne humana, tiempo que llegó a su plenitud cuando en la cueva de Belén nació Cristo. Entonces se cumplió el tiempo de la espera.

Viviendo el Adviento, esperamos un acontecimiento que se sitúa en la historia y a la vez la trasciende. Al igual que los demás años, tendrá lugar en la noche de la Navidad del Señor. A la cueva de Belén acudirán los pastores; más tarde, irán los Magos de Oriente. Unos y otros simbolizan, en cierto sentido, a toda la familia humana. La exhortación que resuena en la liturgia de hoy: \textquote{Vayamos jubilosos al encuentro del Señor} se difunde en todos los países, en todos los continentes, en todos los pueblos y naciones. La voz de la liturgia, es decir, la voz de la Iglesia, resuena por doquier e invita a todos al gran jubileo.

2. Estos últimos tres años que preceden al 2000 forman un tiempo de espera muy intenso, orientado a la meditación sobre el significado del inminente evento espiritual y sobre su necesaria preparación. El contenido de esa preparación sigue el modelo trinitario, que se repite al final de toda plegaria litúrgica. Así pues, vayamos jubilosos \emph{hacia el Padre}, \emph{por el camino que es nuestro Señor Jesucristo}, el cual vive y reina con él \emph{en la unidad del Espíritu Santo}.

Por eso, el primer año lo dedicamos al Hijo; el segundo, al Espíritu Santo; y el que comienza hoy ---el último antes del gran jubileo--- será \emph{el año del Padre}. Invitados por el Padre, vayamos a él mediante el Hijo, en el Espíritu Santo. Este trienio de preparación inmediata para el nuevo milenio, por su carácter trinitario, no sólo nos habla de Dios en sí mismo, como misterio inefable de vida y santidad, sino también de \emph{Dios que viene a nuestro encuentro}.

3. Por este motivo, el estribillo \textquote{Vayamos jubilosos \emph{al encuentro del Señor}} resulta tan adecuado. Nosotros podemos encontrar a Dios, porque él ha venido a nuestro encuentro. Lo ha hecho, como el padre de la parábola del hijo pródigo (cf. \emph{Lc} 15, 11-32), porque es rico en misericordia, \emph{dives in misericordia}, y quiere salir a nuestro encuentro sin importarle de qué parte venimos o a dónde lleva nuestro camino. Dios viene a nuestro encuentro, tanto si lo hemos buscado como si lo hemos ignorado, e incluso si lo hemos evitado. Él sale el primero a nuestro encuentro, con los brazos abiertos, como un padre amoroso y misericordioso.

Si Dios se pone en movimiento para salir a nuestro encuentro, ¿podremos nosotros volverle la espalda? Pero no podemos ir solos al encuentro con el Padre. Debemos ir en compañía de cuantos forman parte de \textquote{la familia de Dios}. Para prepararnos convenientemente al jubileo debemos disponernos a acoger a todas las personas. Todos son nuestros hermanos y hermanas, porque son hijos del mismo Padre celestial.

En esta perspectiva, podemos leer la bimilenaria historia de la Iglesia. Es consolador constatar cómo la Iglesia, en este paso del segundo al tercer milenio, está experimentando un nuevo impulso misionero. Lo ponen de manifiesto los Sínodos continentales que se están celebrando estos años, incluido el actual para Australia y Oceanía. Y también lo confirman los informes que llegan al Comit é para el gran jubileo sobre las iniciativas puestas en marcha por las Iglesias locales como preparación para ese histórico acontecimiento.

Quisiera saludar, en particular, al cardenal presidente del comité, al secretario general y a sus colaboradores. Mi saludo se extiende también a los cardenales, a los obispos y a los sacerdotes aquí presentes, así como a todos vosotros, queridos hermanos y hermanas, que participáis en esta solemne liturgia. Saludo en especial al clero, a los religiosos, a las religiosas y a los laicos comprometidos de Roma, que, junto con el cardenal vicario y los obispos auxiliares, están aquí esta mañana para inaugurar la última fase de la misión ciudadana, dirigida a los ambientes de la sociedad.

Es una fase importante, en la que la diócesis realizará una amplia labor de evangelización en todos los ámbitos de vida y de trabajo. Al terminar la santa misa, entregaré a los misioneros la cruz de la misión. Es necesario que Cristo sea anunciado y testimoniado en cada lugar y en cada situación. Invito a todos a sostener con la oración esta gran empresa. En particular, cuento con la aportación de las monjas de clausura, de los enfermos, de las personas ancianas que, a pesar de que les es imposible participar directamente en esta iniciativa apostólica, pueden dar una gran contribución con su oración y con la ofrenda de sus sufrimientos para disponer los corazones a la acogida del anuncio evangélico.

María, que el tiempo de Adviento nos invita a contemplar en espera activa del Redentor, os ayude a todos a ser apóstoles generosos de su Hijo Jesús.

4. En el evangelio de hoy hemos escuchado la invitación del Señor a la \emph{vigilancia}. \textquote{Velad, porque no sabéis qué día vendrá vuestro Señor}. Y a continuación: \textquote{Estad preparados, porque a la hora que menos penséis vendrá el Hijo del hombre} (\emph{Mt} 24, 42.44). La exhortación a velar resuena muchas veces en la liturgia, especialmente en Adviento, tiempo de preparación no sólo para la Navidad, sino también para \emph{la definitiva y gloriosa venida de Cristo al final de los tiempos}. Por eso, tiene un significado marcadamente escatológico e invita al creyente a pasar cada día, cada momento, en presencia de Aquel \textquote{que es, que era y que vendrá} (\emph{Ap} 1, 4), al que pertenece el futuro del mundo y del hombre. Ésta es la esperanza cristiana. Sin esta perspectiva, nuestra existencia se reduciría a un vivir para la muerte.

Cristo es nuestro Redentor: \emph{Redemptor mundi et Redemptor hominis}, Redentor del mundo y Redentor del hombre. Vino a nosotros para ayudarnos a cruzar el umbral que lleva a la puerta de la vida, la \textquote{puerta santa} que es él mismo.

5. Que esta consoladora verdad esté siempre muy presente ante nuestros ojos, mientras caminamos como peregrinos hacia el gran jubileo. Esa verdad constituye la razón última de la alegría a la que nos exhorta la liturgia de hoy: \textquote{Vayamos \emph{jubilosos} al encuentro del Señor}. Creyendo en Cristo crucificado y resucitado, creemos en la resurrección de la carne y en la vida eterna.

\emph{Tertio millennio adveniente}. En esta perspectiva, los años, los siglos y los milenios cobran el sentido definitivo de la existencia que el jubileo del año 2000 quiere manifestarnos.

Contemplando a Cristo, hagamos nuestras las palabras de un antiguo canto popular polaco:

\begin{quote} \textquote{La salvación ha venido por la cruz;\\ éste es un gran misterio.\\ Todo sufrimiento tiene un sentido:\\ lleva a la plenitud de la vida}. \end{quote}

Con esta fe en el corazón, que es la fe de la Iglesia, inauguro hoy, como Obispo de Roma, el tercer año de preparación para el gran jubileo. Lo inauguro en el nombre del Padre celestial, que \textquote{tanto amó (\ldots{}) al mundo que le dio su Hijo único, para que quien cree en él (\ldots{}) tenga la vida eterna} (\emph{Jn} 3, 16).

¡Alabado sea Jesucristo!

\subsubsection{Ángelus (2001)} 

2 de diciembre de 2001.



\emph{Amadísimos hermanos y hermanas:}

1. Con este primer domingo de Adviento comienza un nuevo Año litúrgico. La Iglesia reanuda su camino y nos invita a reflexionar más intensamente en el misterio de Cristo, misterio siempre nuevo que el tiempo no puede agotar. Cristo es el alfa y la omega, el principio y el fin. Gracias a él, la historia de la humanidad avanza como una peregrinación hacia la plenitud del Reino, que él mismo inauguró con su encarnación y su victoria sobre el pecado y la muerte.

Por eso, el Adviento es sinónimo de \emph{esperanza}: no espera vana de un dios sin rostro, sino confianza concreta y cierta en la vuelta de Aquel que ya nos ha visitado, del \textquote{Esposo} que con su sangre ha sellado con la humanidad un pacto de alianza eterna. Es una esperanza que estimula a la \emph{vigilancia}, virtud característica de este singular tiempo litúrgico. Vigilancia en la \emph{oración}, animada por una amorosa espera; vigilancia en el dinamismo de la \emph{caridad concreta}, consciente de que el reino de Dios se acerca donde los hombres aprenden a vivir como hermanos.

2. Con estos sentimientos, la comunidad cristiana entra en el Adviento, manteniendo vigilante su espíritu, para acoger mejor el mensaje de la palabra de Dios. Resuena hoy en la liturgia el célebre y estupendo \emph{oráculo del profeta Isaías}, pronunciado en un momento de crisis de la historia de Israel.

\textquote{Al final de los días ---dice el Señor--- estará firme el monte de la casa del Señor, encumbrado sobre las montañas. Hacia él confluirán los gentiles. (\ldots{}) De las espadas forjarán arados; de las lanzas, podaderas. No alzará la espada pueblo contra pueblo, no se adiestrarán para la guerra} (\emph{Is} 2, 1-5).

Estas palabras contienen una promesa de paz más actual que nunca para la humanidad, y en particular para la Tierra Santa, de donde también hoy, por desgracia, llegan noticias dolorosas y preocupantes. Que las palabras del profeta Isaías inspiren la mente y el corazón de los creyentes y de los hombres de buena voluntad, para que el día de ayuno ---el 14 de diciembre--- y el encuentro de los representantes de las religiones del mundo en Asís ---el 24 de enero del año próximo--- ayuden a crear en el mundo un clima más sereno y solidario.

3. Encomiendo esta invocación de paz a María, Virgen vigilante y Madre de la esperanza. Dentro de algunos días celebraremos con fe renovada la solemnidad de la Inmaculada Concepción. Que ella nos guíe por este camino, ayudando a todo hombre y a toda nación a dirigir la mirada al \textquote{monte del Señor}, imagen del triunfo definitivo de Cristo y de la venida de su reino de paz.

\subsection{Benedicto XVI, papa}

\subsubsection{Homilía (2007): La gran esperanza}

Visita Pastoral al Hospital Romano San Juan Bautista de la Soberana Orden de Malta.

Domingo 2 de diciembre del 2007.

1. \textquote{Vamos alegres al encuentro del Señor}. Estas palabras, que hemos repetido en el estribillo del \textbf{salmo responsorial}, interpretan bien los sentimientos que alberga nuestro corazón hoy, primer domingo de Adviento. La razón por la cual podemos caminar con alegría, como nos ha exhortado el apóstol \textbf{san Pablo}, es que ya está cerca nuestra salvación. El Señor viene. Con esta certeza emprendemos el itinerario del Adviento, preparándonos para celebrar con fe el acontecimiento extraordinario del Nacimiento del Señor.

Durante las próximas semanas, día tras día, la liturgia propondrá a nuestra reflexión textos del Antiguo Testamento, que recuerdan el vivo y constante deseo que animó en el pueblo judío la espera de la venida del Mesías. También nosotros, vigilantes en la oración, tratemos de preparar nuestro corazón para acoger al Salvador, que vendrá a mostrarnos su misericordia y a darnos su salvación.

2. Precisamente porque es tiempo de espera, el Adviento es tiempo de esperanza, y a la esperanza cristiana he querido dedicar mi segunda encíclica, presentada oficialmente anteayer: comienza con las palabras que san Pablo dirigió a los cristianos de Roma: \emph{\textquote{Spe salvi facti sumus}}, \textquote{En esperanza fuimos salvados} (\emph{Rm} 8, 24). En la encíclica escribí, entre otras cosas, que \textquote{nosotros necesitamos tener esperanzas ---más grandes o más pequeñas---, que día a día nos mantengan en camino. Pero sin la gran esperanza, que ha de superar todo lo demás, aquellas no bastan. Esta gran esperanza sólo puede ser Dios, que abraza el universo y que nos puede proponer y dar lo que nosotros por sí solos no podemos alcanzar} (n. 31). Que la certeza de que sólo Dios puede ser nuestra firme esperanza nos anime a todos los que esta mañana nos hemos reunido en esta casa, en la que se lucha contra la enfermedad, sostenidos por la solidaridad.

{[}\ldots{}{]}

4. Queridos hermanos y hermanas, \textquote{que el Dios de la esperanza, que nos colma de todo gozo y paz en la fe por la fuerza del Espíritu Santo, esté con todos vosotros}. Con este deseo, que el sacerdote dirige a la asamblea al inicio de la santa misa, os saludo cordialmente\ldots{}

El saludo más afectuoso es para vosotros, queridos enfermos, y para vuestros familiares, que con vosotros comparten angustias y esperanzas. El Papa está espiritualmente cerca de vosotros y os asegura su oración diaria; os invita a encontrar en Jesús apoyo y consuelo, y a no perder jamás la confianza. La liturgia de Adviento nos repetirá durante las próximas semanas que no nos cansemos de invocarlo; nos exhortará a salir a su encuentro, sabiendo que él mismo viene continuamente a visitarnos. En la prueba y en la enfermedad Dios nos visita misteriosamente y, si nos abandonamos a su voluntad, podemos experimentar la fuerza de su amor.

Los hospitales y las clínicas, precisamente porque en ellos se encuentran personas probadas por el dolor, pueden transformarse en lugares privilegiados para testimoniar el amor cristiano que alimenta la esperanza y suscita propósitos de solidaridad fraterna. En la \textbf{oración colecta} hemos rezado así: \textquote{Dios todopoderoso, aviva en tus fieles, al comenzar el Adviento, el deseo de salir al encuentro de Cristo, acompañados por las buenas obras}. Sí. Abramos el corazón a todas las personas, especialmente a las que atraviesan dificultades, para que, haciendo el bien a cuantos se encuentran en necesidad, nos dispongamos a acoger a Jesús que en ellos viene a visitarnos.

6. [\ldots{}] En cada enfermo, cualquiera que sea, reconoced y servid a Cristo mismo; haced que en vuestros gestos y en vuestras palabras perciba los signos de su amor misericordioso.

Para cumplir bien esta \textquote{misión}, como nos recuerda san Pablo en la \textbf{segunda lectura}, tratad de \textquote{pertrecharos con las armas de la luz} (\emph{Rm} 13, 12), que son la palabra de Dios, los dones del Espíritu, la gracia de los sacramentos, y las virtudes teologales y cardinales; luchad contra el mal y abandonad el pecado, que entenebrece nuestra existencia. Al inicio de un nuevo año litúrgico, renovemos nuestros buenos propósitos de vida evangélica. \textquote{Ya es hora de espabilarse} (\emph{Rm} 13, 11), exhorta el Apóstol; es decir, es hora de convertirse, de despertar del letargo del pecado para disponerse con confianza a acoger al \textquote{Señor que viene}. Por eso, el Adviento es tiempo de oración y de espera vigilante.

7. A la \textquote{vigilancia}, que por lo demás es la palabra clave de todo este período litúrgico, nos exhorta la \textbf{página evangélica} que acabamos de proclamar: \textquote{Estad en vela, porque no sabéis qué día vendrá vuestro Señor} (\emph{Mt} 24, 42). Jesús, que en la Navidad vino a nosotros y volverá glorioso al final de los tiempos, no se cansa de visitarnos continuamente en los acontecimientos de cada día. Nos pide estar atentos para percibir su presencia, su adviento, y nos advierte que lo esperemos vigilando, puesto que su venida no se puede programar o pronosticar, sino que será repentina e imprevisible. Sólo quien está despierto no será tomado de sorpresa. Que no os suceda ---advierte--- lo que pasó en tiempo de Noé, cuando los hombres comían y bebían despreocupadamente, y el diluvio los encontró desprevenidos (cf. \emph{Mt} 24, 37-38). Lo que quiere darnos a entender el Señor con esta recomendación es que no debemos dejarnos absorber por las realidades y preocupaciones materiales hasta el punto de quedar atrapados en ellas. Debemos vivir ante los ojos del Señor con la convicción de que cada día puede hacerse presente. Si vivimos así, el mundo será mejor.

8. \textquote{Estad, pues, en vela\ldots{}}. Escuchemos la invitación de Jesús en el \textbf{Evangelio} y preparémonos para revivir con fe el misterio del nacimiento del Redentor, que ha llenado de alegría el universo; preparémonos para acoger al Señor que viene continuamente a nuestro encuentro en los acontecimientos de la vida, en la alegría y en el dolor, en la salud y en la enfermedad; preparémonos para encontrarlo en su venida última y definitiva.

Su paso es siempre fuente de paz y, si el sufrimiento, herencia de la naturaleza humana, a veces resulta casi insoportable, con la venida del Salvador \textquote{el sufrimiento ---sin dejar de ser sufrimiento--- se convierte a pesar de todo en canto de alabanza} (\emph{Spe salvi}, 37). Confortados por estas palabras, prosigamos la celebración eucarística, invocando sobre los enfermos, sobre sus familiares y sobre cuantos trabajan en este hospital y en toda la Orden de los Caballeros de Malta, la protección materna de María, Virgen de la espera y de la esperanza, así como de la alegría, ya presente en este mundo, porque cuando sentimos la cercanía de Cristo vivo tenemos ya el remedio para el sufrimiento, tenemos ya su alegría. Amén.


\subsubsection{Ángelus (2007)}

Plaza de San Pedro. Domingo 2 de diciembre de 2007.

\emph{Queridos hermanos y hermanas:}

Con este primer domingo de Adviento comienza un nuevo año litúrgico: el pueblo de Dios vuelve a ponerse en camino para vivir el misterio de Cristo en la historia. Cristo es el mismo ayer, hoy y siempre (cf. \emph{Hb} 13, 8); en cambio, la historia cambia y necesita ser evangelizada constantemente; necesita renovarse desde dentro, y la única verdadera novedad es Cristo: él es su realización plena, el futuro luminoso del hombre y del mundo. Jesús, resucitado de entre los muertos, es el Señor al que Dios someterá todos sus enemigos, incluida la misma muerte (cf. \emph{1 Co} 15, 25-28).

Por tanto, el Adviento es el tiempo propicio para reavivar en nuestro corazón la espera de Aquel \textquote{que es, que era y que va a venir} (\emph{Ap} 1, 8). El Hijo de Dios ya vino en Belén hace veinte siglos, viene en cada momento al alma y a la comunidad dispuestas a recibirlo, y de nuevo vendrá al final de los tiempos para \textquote{juzgar a vivos y muertos}. Por eso, el creyente está siempre vigilante, animado por la íntima esperanza de encontrar al Señor, como dice el Salmo: \textquote{Mi alma espera en el Señor, espera en su palabra; mi alma aguarda al Señor, más que el centinela a la aurora} (\emph{Sal} 130, 5-6).

Por consiguiente, este domingo es un día muy adecuado para ofrecer a la Iglesia entera y a todos los hombres de buena voluntad mi segunda encíclica, que quise dedicar precisamente al tema de la esperanza cristiana. Se titula \emph{Spe salvi}, porque comienza con la expresión de san Pablo: \emph{\textquote{Spe salvi factum sumus}}, \textquote{en esperanza fuimos salvados} (\emph{Rm} 8, 24). En este, como en otros pasajes del Nuevo Testamento, la palabra \textquote{esperanza} está íntimamente relacionada con la palabra \textquote{fe}. Es un don que cambia la vida de quien lo recibe, como lo muestra la experiencia de tantos santos y santas.

¿En qué consiste esta esperanza, tan grande y tan \textquote{fiable} que nos hace decir que \emph{en ella} encontramos la \textquote{salvación}? Esencialmente, consiste en el conocimiento de Dios, en el descubrimiento de su corazón de Padre bueno y misericordioso. Jesús, con su muerte en la cruz y su resurrección, nos reveló su rostro, el rostro de un Dios con un amor tan grande que comunica una esperanza inquebrantable, que ni siquiera la muerte puede destruir, porque la vida de quien se pone en manos de este Padre se abre a la perspectiva de la bienaventuranza eterna.

El desarrollo de la ciencia moderna ha marginado cada vez más la fe y la esperanza en la esfera privada y personal, hasta el punto de que hoy se percibe de modo evidente, y a veces dramático, que el hombre y el mundo necesitan a Dios ---¡al verdadero Dios!---; de lo contrario, no tienen esperanza.

No cabe duda de que la ciencia contribuye en gran medida al bien de la humanidad, pero no es capaz de redimirla. El hombre es redimido por el amor, que hace buena y hermosa la vida personal y social. Por eso la gran esperanza, la esperanza plena y definitiva, es garantizada por Dios que es amor, por Dios que en Jesús nos visitó y nos dio la vida, y en él volverá al final de los tiempos.

En Cristo esperamos; es a él a quien aguardamos. Con María, su Madre, la Iglesia va al encuentro del Esposo: lo hace con las obra de caridad, porque la esperanza, como la fe, se manifiesta en el amor. ¡Buen Adviento a todos!


\subsubsection{Ángelus (2010)} \emph{
	
Plaza de San Pedro. Domingo 28 de noviembre de 2010.

\emph{Queridos hermanos y hermanas:}

Hoy, primer domingo de Adviento, la Iglesia inicia un nuevo Año litúrgico, un nuevo camino de fe que, por una parte, conmemora el acontecimiento de Jesucristo, y por otra, se abre a su cumplimiento final. Precisamente de esta doble perspectiva vive el tiempo de Adviento, mirando tanto a la primera venida del Hijo de Dios, cuando nació de la Virgen María, como a su vuelta gloriosa, cuando vendrá a \textquote{juzgar a vivos y muertos}, como decimos en el Credo. Sobre este sugestivo tema de la \textquote{espera} quiero detenerme ahora brevemente, porque se trata de un aspecto profundamente humano, en el que la fe se convierte, por decirlo así, en un todo con nuestra carne y nuestro corazón.

La espera, el esperar, es una dimensión que atraviesa toda nuestra existencia personal, familiar y social. La espera está presente en mil situaciones, desde las más pequeñas y banales hasta las más importantes, que nos implican totalmente y en lo profundo. Pensemos, entre estas, en la espera de un hijo por parte de dos esposos; en la de un pariente o de un amigo que viene a visitarnos de lejos; pensemos, para un joven, en la espera del resultado de un examen decisivo, o de una entrevista de trabajo; en las relaciones afectivas, en la espera del encuentro con la persona amada, de la respuesta a una carta, o de la aceptación de un perdón\ldots{} Se podría decir que el hombre está vivo mientras espera, mientras en su corazón está viva la esperanza. Y al hombre se lo reconoce por sus esperas: nuestra \textquote{estatura} moral y espiritual se puede medir por lo que esperamos, por aquello en lo que esperamos.

Cada uno de nosotros, por tanto, especialmente en este tiempo que nos prepara a la Navidad, puede preguntarse: ¿yo qué espero? En este momento de mi vida, ¿a qué tiende mi corazón? Y esta misma pregunta se puede formular a nivel de familia, de comunidad, de nación. ¿Qué es lo que esperamos juntos? ¿Qué une nuestras aspiraciones?, ¿qué tienen en común? En el tiempo anterior al nacimiento de Jesús, era muy fuerte en Israel la espera del Mesías, es decir, de un Consagrado, descendiente del rey David, que finalmente liberaría al pueblo de toda esclavitud moral y política e instauraría el reino de Dios. Pero nadie habría imaginado nunca que el Mesías pudiese nacer de una joven humilde como era María, prometida del justo José. Ni siquiera ella lo habría pensado nunca, pero en su corazón la espera del Salvador era tan grande, su fe y su esperanza eran tan ardientes, que él pudo encontrar en ella una madre digna. Por lo demás, Dios mismo la había preparado, antes de los siglos. Hay una misteriosa correspondencia entre la espera de Dios y la de María, la criatura \textquote{llena de gracia}, totalmente transparente al designio de amor del Altísimo. Aprendamos de ella, Mujer del Adviento, a vivir los gestos cotidianos con un espíritu nuevo, con el sentimiento de una espera profunda, que sólo la venida de Dios puede colmar.



\subsection{Francisco, papa}

\subsubsection{Homilía (2013)}
Visita Pastoral a la Parroquia Romana de San Cirilo Alejandrino. Domingo 1 de diciembre de 2013}

En la primera lectura, hemos escuchado que el profeta Isaías nos habla de un camino, y dice que al final de los días, al final del camino, el monte del Templo del Señor estará firme en la cima de los montes. Y esto, para decirnos que nuestra vida es un camino: debemos ir por este camino, para llegar al monte del Señor, al encuentro con Jesús. La cosa más importante que le puede suceder a una persona es encontrar a Jesús: este encuentro con Jesús que nos ama, que nos ha salvado, que ha dado su vida por nosotros. Encontrar a Jesús. Y nosotros caminamos para encontrar a Jesús.

Podemos preguntarnos: ¿Cuándo encuentro a Jesús? ¿Sólo al final? ¡No, no! Lo encontramos todos los días. ¿Pero cómo? En la oración, cuando tú rezas, encuentras a Jesús. Cuando recibes la Comunión, encuentras a Jesús, en los Sacramentos. Cuando llevas a bautizar a tu hijo, te encuentras a Jesús, hallas a Jesús. Y vosotros, hoy, que recibís la Confirmación, también vosotros encontraréis a Jesús; luego lo encontraréis en la Comunión. \textquote{Y más tarde, Padre, después de la Confirmación, adiós}, porque dicen que la Confirmación se llama \textquote{el sacramento del ¡adiós!}. ¿Es verdad esto o no? Después de la Confirmación no se va nunca a la iglesia: ¿es verdad o no?\ldots{} ¡Más o menos! Pero también después de la Confirmación, toda la vida, es un encuentro con Jesús: en la oración, cuando vamos a misa y cuando realizamos buenas obras, cuando visitamos a los enfermos, cuando ayudamos a un pobre, cuando pensamos en los demás, cuando no somos egoístas, cuando somos amables\ldots{} en estas cosas encontramos siempre a Jesús. Y el camino de la vida es precisamente este: caminar para encontrar a Jesús.

Hoy, también para mí es una alegría venir a encontrarme con vosotros, porque todos juntos, hoy, en la misa encontraremos a Jesús, y hacemos un tramo del camino juntos.

Recordad siempre esto: la vida es un camino. Es un camino. Un camino para encontrar a Jesús. Al final, y siempre. Un camino donde no encontramos a Jesús, no es un camino cristiano. Es propio del cristiano encontrar siempre a Jesús, mirarle, dejarse mirar por Jesús, porque Jesús nos mira con amor, nos ama mucho, nos quiere mucho y nos mira siempre. Encontrar a Jesús es también dejarte mirar por Él. \textquote{Pero, Padre, tú sabes ---alguno de vosotros podría decirme---, tú sabes que este camino, para mí, es un camino difícil, porque yo soy muy pecador, he cometido muchos pecados\ldots{} ¿cómo puedo encontrar a Jesús?}. Pero tú sabes que las personas a las que Jesús mayormente buscaba eran los más pecadores; y le reñían por esto, y la gente ---las personas que se creían justas--- decía: pero éste, éste no es un verdadero profeta, ¡mira la buena compañía que tiene! Estaba con los pecadores\ldots{} Y Él decía: He venido por quienes tienen necesidad de salud, necesidad de curación, y Jesús cura nuestros pecados. En el camino, nosotros ---todos pecadores, todos, todos somos pecadores--- incluso cuando nos equivocamos, cuando cometemos un pecado, cuando pecamos, Jesús viene y nos perdona. Este perdón que recibimos en la Confesión es un encuentro con Jesús. Siempre encontramos a Jesús.

Y así vamos por la vida, como dice el profeta, al monte, hasta el día que tendrá lugar el encuentro definitivo, cuando contemplemos esa mirada tan bella de Jesús, tan hermosa. Ésta es la vida cristiana: caminar, seguir adelante, unidos como hermanos, queriéndose uno a otro. Encontrar a Jesús. ¿Estáis de acuerdo, vosotros, los nueve? ¿Queréis encontrar a Jesús en vuestra vida? ¿Sí? Esto es importante en la vida cristiana. Vosotros, hoy, con el sello del Espíritu Santo, tendréis más fuerza para este camino, para encontrar a Jesús. ¡Sed valientes, no tengáis miedo! La vida es este camino. Y el regalo más hermoso es encontrar a Jesús. ¡Adelante, ánimo!

Y ahora, sigamos adelante con el Sacramento de la Confirmación.

\subsubsection{Ángelus (2013): En camino}

Plaza de San Pedro. Domingo 1 de diciembre del 2013.

Comenzamos hoy, primer domingo de Adviento, un nuevo año litúrgico, es decir \emph{un nuevo camino del Pueblo de Dios} con Jesucristo, nuestro Pastor, que nos guía en la historia hacia la realización del Reino de Dios. Por ello este día tiene un atractivo especial, nos hace experimentar un sentimiento profundo del sentido de la historia. Redescubrimos la belleza de estar todos en camino: la Iglesia, con su vocación y misión, y toda la humanidad, los pueblos, las civilizaciones, las culturas, todos en camino a través de los senderos del tiempo.

¿En camino hacia dónde? ¿Hay una meta común? ¿Y cuál es esta meta? El Señor nos responde a través del profeta \textbf{Isaías}, y dice así: \textquote{En los días futuros estará firme el monte de la casa del Señor, en la cumbre de las montañas, más elevado que las colinas. Hacia él confluirán todas las naciones, caminarán pueblos numerosos y dirán: \textquote{Venid, subamos al monte del Señor, a la casa del Dios de Jacob. Él nos instruirá en sus caminos y marcharemos por sus sendas}} (2, 2-3). Esto es lo que dice Isaías acerca de la meta hacia la que nos dirigimos. Es \emph{una peregrinación universal hacia una meta común}, que en el Antiguo Testamento es Jerusalén, donde surge el templo del Señor, porque desde allí, de Jerusalén, ha venido la revelación del rostro de Dios y de su ley. La revelación ha encontrado su realización en \emph{Jesucristo}, y Él mismo, el Verbo hecho carne, se ha convertido en el \textquote{templo del Señor}: es Él la guía y al mismo tiempo la meta de nuestra peregrinación, de la peregrinación de todo el Pueblo de Dios; y bajo su luz también los demás pueblos pueden caminar hacia el Reino de la justicia, hacia el Reino de la paz. Dice de nuevo el profeta: \textquote{De las espadas forjarán arados, de las lanzas, podaderas. No alzará la espada pueblo contra pueblo, no se adiestrarán para la guerra} (2, 4).

Me permito repetir esto que dice el profeta, escuchad bien: \textquote{De las espadas forjarán arados, de las lanzas, podaderas. No alzará la espada pueblo contra pueblo, no se adiestrarán para la guerra}. ¿Pero cuándo sucederá esto? Qué hermoso día será ese en el que las armas sean desmontadas, para transformarse en instrumentos de trabajo. ¡Qué hermoso día será ése! ¡Y esto es posible! Apostemos por la esperanza, la esperanza de la paz. Y será posible.

Este camino no se acaba nunca. Así como en la vida de cada uno de nosotros siempre hay necesidad de comenzar de nuevo, de volver a levantarse, de volver a encontrar el sentido de la meta de la propia existencia, de la misma manera para la gran familia humana es necesario renovar siempre el horizonte común hacia el cual estamos encaminados. \emph{¡El horizonte de la esperanza!} Es ese el horizonte para hacer un buen camino. El tiempo de Adviento, que hoy de nuevo comenzamos, nos devuelve el horizonte de la esperanza, una esperanza que no decepciona porque está fundada en la Palabra de Dios. Una esperanza que no decepciona, sencillamente porque el Señor no decepciona jamás. ¡Él es fiel!, ¡Él no decepciona! ¡Pensemos y sintamos esta belleza!

El modelo de esta actitud espiritual, de este modo de ser y de caminar en la vida, es la Virgen María. Una sencilla muchacha de pueblo, que lleva en el corazón toda la esperanza de Dios. En su seno, la esperanza de Dios se hizo carne, se hizo hombre, se hizo historia: Jesucristo. Su \emph{Magníficat} es el cántico del Pueblo de Dios en camino, y de todos los hombres y mujeres que esperan en Dios, en el poder de su misericordia. Dejémonos guiar por Ella, que es madre, es mamá, y sabe cómo guiarnos. Dejémonos guiar por Ella en este tiempo de espera y de vigilancia activa.

\subsubsection{Ángelus (2016): El Señor nos visita}

Plaza de San Pedro. Domingo 27 de noviembre del 2016.

Hoy la Iglesia inicia un nuevo año litúrgico, es decir, un nuevo camino de fe del pueblo de Dios. Y como siempre iniciamos con el Adviento. La página del \textbf{Evangelio} (cf. Mt 24, 37-44) nos presenta uno de los temas más sugestivos del tiempo de Adviento: la visita del Señor a la humanidad. La primera visita ---lo sabemos todos--- se produjo con la Encarnación, el nacimiento de Jesús en la gruta de Belén; la segunda sucede en el presente: el Señor nos visita continuamente cada día, camina a nuestro lado y es una presencia de consolación; y para concluir estará la tercera y última visita, que profesamos cada vez que recitamos el Credo: \textquote{De nuevo vendrá en la gloria para juzgar a vivos y a muertos}. El Señor hoy nos habla de esta última visita suya, la que sucederá al final de los tiempos y nos dice dónde llegará nuestro camino.

La palabra de Dios hace resaltar el contraste entre el desarrollarse normal de las cosas, la rutina cotidiana y la venida repentina del Señor. \textbf{Dice Jesús}: \textquote{Como en los días que precedieron al diluvio, comían, bebían, tomaban mujer o marido, hasta el día en el que entró Noé en el arca, y no se dieron cuenta hasta que vino el diluvio y los arrasó a todos} (vv. 38-39): así dice Jesús. Siempre nos impresiona pensar en las horas que preceden a una gran calamidad: todos están tranquilos, hacen las cosas de siempre sin darse cuenta que su vida está a punto de ser alterada. El Evangelio, ciertamente no quiere darnos miedo, sino abrir nuestro horizonte a la dimensión ulterior, más grande, que por una parte relativiza las cosas de cada día pero al mismo tiempo las hace preciosas, decisivas. La relación con el Dios que viene a visitarnos da a cada gesto, a cada cosa una luz diversa, una profundidad, un valor simbólico.

Desde esta perspectiva llega también una invitación a la sobriedad, a no ser dominados por las cosas de este mundo, por las realidades materiales, sino más bien a gobernarlas. Si por el contrario nos dejamos condicionar y dominar por ellas, no podemos percibir que hay algo mucho más importante: nuestro encuentro final con el Señor, y esto es importante. Ese, ese encuentro. Y las cosas de cada día deben tener ese horizonte, deben ser dirigidas a ese horizonte. Este encuentro con el Señor que viene por nosotros. En aquel momento, como dice el \textbf{Evangelio}, \textquote{estarán dos en el campo: uno es tomado, el otro dejado} (v. 40). Es una invitación a la vigilancia, porque no sabiendo cuando Él vendrá, es necesario estar preparados siempre para partir.

En este tiempo de Adviento estamos llamados a ensanchar los horizontes de nuestro corazón, a dejarnos sorprender por la vida que se presenta cada día con sus novedades. Para hacer esto es necesario aprender a no depender de nuestras seguridades, de nuestros esquemas consolidados, porque el Señor viene a la hora que no nos imaginamos. Viene para presentarnos una dimensión más hermosa y más grande.

Que Nuestra Señora, Virgen del Adviento, nos ayude a no considerarnos propietarios de nuestra vida, a no oponer resistencia cuando el Señor viene para cambiarla, sino a estar preparados para dejarnos visitar por Él, huésped esperado y grato, aunque desarme nuestros planes.

\subsubsection{Ángelus (2019): Velar}

Plaza de San Pedro. Domingo 1 de diciembre de 2019.

Hoy, primer domingo de Adviento, comienza un nuevo año litúrgico. En estas cuatro semanas de Adviento, la liturgia nos lleva a celebrar el nacimiento de Jesús, mientras nos recuerda que Él viene todos los días en nuestras vidas, y que regresará gloriosamente al final de los tiempos. Esta certeza nos lleva a mirar al futuro con confianza, como nos invita el profeta Isaías, que con su voz inspirada acompaña todo el camino del Adviento.

En la \textbf{primera lectura} de hoy, Isaías profetiza que \textquote{sucederá en días futuros que el monte de la Casa de Yahveh será asentado en la cima de los montes y se alzará por encima de las colinas. Confluirán a él todas las naciones} (Isaías 2, 2). El templo del Señor en Jerusalén se presenta como el punto de encuentro y de convergencia de todos los pueblos.

Después de la Encarnación del Hijo de Dios, Jesús mismo se reveló como el verdadero templo. Por lo tanto, la maravillosa \textbf{visión de Isaía}s es una promesa divina y nos impulsa a asumir una actitud de peregrinación, de camino hacia Cristo, sentido y fin de toda la historia. Los que tienen hambre y sed de justicia sólo pueden encontrarla a través de los caminos del Señor, mientras que el mal y el pecado provienen del hecho de que los individuos y los grupos sociales prefieren seguir caminos dictados por intereses egoístas, que causan conflictos y guerras.

El Adviento es el tiempo para acoger la venida de Jesús, que viene como mensajero de paz para mostrarnos los caminos de Dios.

En el \textbf{Evangelio} de hoy, Jesús nos exhorta a estar preparados para su venida: \textquote{Velad, pues, porque no sabéis qué día vendrá vuestro Señor} (Mateo 24, 42). Velar no significa tener los ojos materialmente abiertos, sino tener el corazón libre y orientado en la dirección correcta, es decir, dispuesto a dar y servir. ¡Eso es velar! El sueño del que debemos despertar está constituido por la indiferencia, por la vanidad, por la incapacidad de establecer relaciones verdaderamente humanas, por la incapacidad de hacerse cargo de nuestro hermano aislado, abandonado o enfermo.

La espera de la venida de Jesús debe traducirse, por tanto, en un compromiso de vigilancia. Se trata sobre todo de maravillarse de la acción de Dios, de sus sorpresas y de darle primacía. Vigilancia significa también, concretamente, estar atento al prójimo en dificultades, dejarse interpelar por sus necesidades, sin esperar a que nos pida ayuda, sino aprendiendo a prevenir, a anticipar, como Dios siempre hace con nosotros.

Que María, Virgen vigilante y Madre de la esperanza, nos guíe en este camino, ayudándonos a dirigir la mirada hacia el \textquote{monte del Señor}, imagen de Jesucristo, que atrae a todos los hombres y todos los pueblos.


\section{Temas}

Tribulación final y venida de Cristo en gloria

CEC 668-677, 769:

\textbf{668} \textquote{Cristo murió y volvió a la vida para eso, para ser Señor de muertos y vivos} (\emph{Rm} 14, 9). La Ascensión de Cristo al Cielo significa su participación, en su humanidad, en el poder y en la autoridad de Dios mismo. Jesucristo es Señor: posee todo poder en los cielos y en la tierra. El está \textquote{por encima de todo principado, potestad, virtud, dominación} porque el Padre \textquote{bajo sus pies sometió todas las cosas} (\emph{Ef} 1, 20-22). Cristo es el Señor del cosmos (cf. \emph{Ef} 4, 10; \emph{1 Co} 15, 24. 27-28) y de la historia. En Él, la historia de la humanidad e incluso toda la Creación encuentran su recapitulación (\emph{Ef} 1, 10), su cumplimiento transcendente.

\textbf{669} Como Señor, Cristo es también la cabeza de la Iglesia que es su Cuerpo (cf. \emph{Ef} 1, 22). Elevado al cielo y glorificado, habiendo cumplido así su misión, permanece en la tierra en su Iglesia. La Redención es la fuente de la autoridad que Cristo, en virtud del Espíritu Santo, ejerce sobre la Iglesia (cf. \emph{Ef} 4, 11-13). \textquote{La Iglesia, o el reino de Cristo presente ya en misterio} (LG 3), \textquote{constituye el germen y el comienzo de este Reino en la tierra} (LG 5).

\textbf{670} Desde la Ascensión, el designio de Dios ha entrado en su consumación. Estamos ya en la \textquote{última hora} (\emph{1 Jn} 2, 18; cf. \emph{1 P} 4, 7). \textquote{El final de la historia ha llegado ya a nosotros y la renovación del mundo está ya decidida de manera irrevocable e incluso de alguna manera real está ya por anticipado en este mundo. La Iglesia, en efecto, ya en la tierra, se caracteriza por una verdadera santidad, aunque todavía imperfecta} (LG 48). El Reino de Cristo manifiesta ya su presencia por los signos milagrosos (cf. \emph{Mc} 16, 17-18) que acompañan a su anuncio por la Iglesia (cf. \emph{Mc} 16, 20).

\textbf{\ldots{} esperando que todo le sea sometido}

\textbf{671} El Reino de Cristo, presente ya en su Iglesia, sin embargo, no está todavía acabado \textquote{con gran poder y gloria} (\emph{Lc} 21, 27; cf. \emph{Mt} 25, 31) con el advenimiento del Rey a la tierra. Este Reino aún es objeto de los ataques de los poderes del mal (cf. \emph{2 Ts} 2, 7), a pesar de que estos poderes hayan sido vencidos en su raíz por la Pascua de Cristo. Hasta que todo le haya sido sometido (cf. \emph{1 Co} 15, 28), y \textquote{mientras no [\ldots{}] haya nuevos cielos y nueva tierra, en los que habite la justicia, la Iglesia peregrina lleva en sus sacramentos e instituciones, que pertenecen a este tiempo, la imagen de este mundo que pasa. Ella misma vive entre las criaturas que gimen en dolores de parto hasta ahora y que esperan la manifestación de los hijos de Dios} (LG 48). Por esta razón los cristianos piden, sobre todo en la Eucaristía (cf. \emph{1 Co} 11, 26), que se apresure el retorno de Cristo (cf. \emph{2 P} 3, 11-12) cuando suplican: \textquote{Ven, Señor Jesús} (\emph{Ap} 22, 20; cf. \emph{1 Co} 16, 22; \emph{Ap} 22, 17-20).



\textbf{672} Cristo afirmó antes de su Ascensión que aún no era la hora del establecimiento glorioso del Reino mesiánico esperado por Israel (cf. \emph{Hch} 1, 6-7) que, según los profetas (cf. \emph{Is} 11, 1-9), debía traer a todos los hombres el orden definitivo de la justicia, del amor y de la paz. El tiempo presente, según el Señor, es el tiempo del Espíritu y del testimonio (cf. \emph{Hch} 1, 8), pero es también un tiempo marcado todavía por la \textquote{tribulación} (\emph{1 Co} 7, 26) y la prueba del mal (cf. \emph{Ef} 5, 16) que afecta también a la Iglesia (cf. \emph{1 P} 4, 17) e inaugura los combates de los últimos días (\emph{1 Jn} 2, 18; 4, 3; \emph{1 Tm} 4, 1). Es un tiempo de espera y de vigilia (cf. \emph{Mt} 25, 1-13; \emph{Mc} 13, 33-37).

\textbf{El glorioso advenimiento de Cristo, esperanza de Israel}

\textbf{673} Desde la Ascensión, el advenimiento de Cristo en la gloria es inminente (cf. \emph{Ap} 22, 20) aun cuando a nosotros no nos \textquote{toca conocer el tiempo y el momento que ha fijado el Padre con su autoridad} (\emph{Hch} 1, 7; cf. \emph{Mc} 13, 32). Este acontecimiento escatológico se puede cumplir en cualquier momento (cf. \emph{Mt} 24, 44: \emph{1 Ts} 5, 2), aunque tal acontecimiento y la prueba final que le ha de preceder estén \textquote{retenidos} en las manos de Dios (cf. \emph{2 Ts} 2, 3-12).

\textbf{674} La venida del Mesías glorioso, en un momento determinado de la historia (cf. \emph{Rm} 11, 31), se vincula al reconocimiento del Mesías por \textquote{todo Israel} (\emph{Rm} 11, 26; \emph{Mt} 23, 39) del que \textquote{una parte está endurecida} (\emph{Rm} 11, 25) en \textquote{la incredulidad} (\emph{Rm} 11, 20) respecto a Jesús. San Pedro dice a los judíos de Jerusalén después de Pentecostés: \textquote{Arrepentíos, pues, y convertíos para que vuestros pecados sean borrados, a fin de que del Señor venga el tiempo de la consolación y envíe al Cristo que os había sido destinado, a Jesús, a quien debe retener el cielo hasta el tiempo de la restauración universal, de que Dios habló por boca de sus profetas} (\emph{Hch} 3, 19-21). Y san Pablo le hace eco: \textquote{si su reprobación ha sido la reconciliación del mundo ¿qué será su readmisión sino una resurrección de entre los muertos?} (\emph{Rm} 11, 5). La entrada de \textquote{la plenitud de los judíos} (\emph{Rm} 11, 12) en la salvación mesiánica, a continuación de \textquote{la plenitud de los gentiles} (Rm 11, 25; cf. Lc 21, 24), hará al pueblo de Dios \textquote{llegar a la plenitud de Cristo} (\emph{Ef} 4, 13) en la cual \textquote{Dios será todo en nosotros} (\emph{1 Co} 15, 28).

\textbf{La última prueba de la Iglesia}

\textbf{675} Antes del advenimiento de Cristo, la Iglesia deberá pasar por una prueba final que sacudirá la fe de numerosos creyentes (cf. \emph{Lc} 18, 8; \emph{Mt} 24, 12). La persecución que acompaña a su peregrinación sobre la tierra (cf. \emph{Lc} 21, 12; \emph{Jn} 15, 19-20) desvelará el \textquote{misterio de iniquidad} bajo la forma de una impostura religiosa que proporcionará a los hombres una solución aparente a sus problemas mediante el precio de la apostasía de la verdad. La impostura religiosa suprema es la del Anticristo, es decir, la de un seudo-mesianismo en que el hombre se glorifica a sí mismo colocándose en el lugar de Dios y de su Mesías venido en la carne (cf. \emph{2 Ts} 2, 4-12; \emph{1Ts} 5, 2-3;2 \emph{Jn} 7; \emph{1 Jn} 2, 18.22).



\textbf{676} Esta impostura del Anticristo aparece esbozada ya en el mundo cada vez que se pretende llevar a cabo la esperanza mesiánica en la historia, lo cual no puede alcanzarse sino más allá del tiempo histórico a través del juicio escatológico: incluso en su forma mitigada, la Iglesia ha rechazado esta falsificación del Reino futuro con el nombre de milenarismo (cf. DS 3839), sobre todo bajo la forma política de un mesianismo secularizado, \textquote{intrínsecamente perverso} (cf. Pío XI, carta enc. \emph{Divini Redemptoris}, condenando \textquote{los errores presentados bajo un falso sentido místico de esta especie de falseada redención de los más humildes}; GS 20-21).

\textbf{677} La Iglesia sólo entrará en la gloria del Reino a través de esta última Pascua en la que seguirá a su Señor en su muerte y su Resurrección (cf. \emph{Ap} 19, 1-9). El Reino no se realizará, por tanto, mediante un triunfo histórico de la Iglesia (cf. \emph{Ap} 13, 8) en forma de un proceso creciente, sino por una victoria de Dios sobre el último desencadenamiento del mal (cf. \emph{Ap} 20, 7-10) que hará descender desde el cielo a su Esposa (cf. \emph{Ap} 21, 2-4). El triunfo de Dios sobre la rebelión del mal tomará la forma de Juicio final (cf. \emph{Ap} 20, 12) después de la última sacudida cósmica de este mundo que pasa (cf. \emph{2 P} 3, 12-13).

\textbf{La Iglesia, consumada en la gloria}

\textbf{769} La Iglesia \textquote{sólo llegará a su perfección en la gloria del cielo} (LG 48), cuando Cristo vuelva glorioso. Hasta ese día, \textquote{la Iglesia avanza en su peregrinación a través de las persecuciones del mundo y de los consuelos de Dios} (San Agustín, \emph{De civitate Dei} 18, 51; cf. LG 8). Aquí abajo, ella se sabe en exilio, lejos del Señor (cf. \emph{2Co} 5, 6; LG 6), y aspira al advenimiento pleno del Reino, \textquote{y espera y desea con todas sus fuerzas reunirse con su Rey en la gloria} (LG 5). La consumación de la Iglesia en la gloria, y a través de ella la del mundo, no sucederá sin grandes pruebas. Solamente entonces, \textquote{todos los justos descendientes de Adán, \textquote{desde Abel el justo hasta el último de los elegidos} se reunirán con el Padre en la Iglesia universal} (LG 2).

\textquote{¡Ven, Señor Jesús!}

CEC 451, 671, 1130, 1403, 2817:

\textbf{451} La oración cristiana está marcada por el título \textquote{Señor}, ya sea en la invitación a la oración \textquote{el Señor esté con vosotros}, o en su conclusión \textquote{por Jesucristo nuestro Señor} o incluso en la exclamación llena de confianza y de esperanza: \emph{Maran atha} (\textquote{¡el Señor viene!}) o \emph{Marana tha} (\textquote{¡Ven, Señor!}) (\emph{1 Co} 16, 22): \textquote{¡Amén! ¡ven, Señor Jesús!} (\emph{Ap} 22, 20).



\textbf{\ldots{} esperando que todo le sea sometido}

\textbf{671} El Reino de Cristo, presente ya en su Iglesia, sin embargo, no está todavía acabado \textquote{con gran poder y gloria} (\emph{Lc} 21, 27; cf. \emph{Mt} 25, 31) con el advenimiento del Rey a la tierra. Este Reino aún es objeto de los ataques de los poderes del mal (cf. \emph{2 Ts} 2, 7), a pesar de que estos poderes hayan sido vencidos en su raíz por la Pascua de Cristo. Hasta que todo le haya sido sometido (cf. \emph{1 Co} 15, 28), y \textquote{mientras no [\ldots{}] haya nuevos cielos y nueva tierra, en los que habite la justicia, la Iglesia peregrina lleva en sus sacramentos e instituciones, que pertenecen a este tiempo, la imagen de este mundo que pasa. Ella misma vive entre las criaturas que gimen en dolores de parto hasta ahora y que esperan la manifestación de los hijos de Dios} (LG 48). Por esta razón los cristianos piden, sobre todo en la Eucaristía (cf. \emph{1 Co} 11, 26), que se apresure el retorno de Cristo (cf. \emph{2 P} 3, 11-12) cuando suplican: \textquote{Ven, Señor Jesús} (\emph{Ap} 22, 20; cf. \emph{1 Co} 16, 22; \emph{Ap} 22, 17-20).

\textbf{Sacramentos de la vida eterna}

\textbf{1130} La Iglesia celebra el Misterio de su Señor \textquote{hasta que él venga} y \textquote{Dios sea todo en todos} (\emph{1 Co} 11, 26; 15, 28). Desde la era apostólica, la liturgia es atraída hacia su término por el gemido del Espíritu en la Iglesia: \emph{¡Marana tha!} (\emph{1 Co} 16,22). La liturgia participa así en el deseo de Jesús: \textquote{Con ansia he deseado comer esta Pascua con vosotros [\ldots{}] hasta que halle su cumplimiento en el Reino de Dios} (\emph{Lc} 22,15-16). En los sacramentos de Cristo, la Iglesia recibe ya las arras de su herencia, participa ya en la vida eterna, aunque \textquote{aguardando la feliz esperanza y la manifestación de la gloria del Gran Dios y Salvador nuestro Jesucristo} (\emph{Tt} 2,13). \textquote{El Espíritu y la Esposa dicen: ¡Ven! [\ldots{}] ¡Ven, Señor Jesús!} (\emph{Ap} 22,17.20).

\begin{quote} Santo Tomás resume así las diferentes dimensiones del signo sacramental: \textquote{\emph{Unde sacramentum est signum rememorativum eius quod praecessit, scilicet passionis Christi; et desmonstrativum eius quod in nobis efficitur per Christi passionem, scilicet gratiae; et prognosticum, id est, praenuntiativum futurae gloriae}} (\textquote{Por eso el sacramento es un signo que rememora lo que sucedió, es decir, la pasión de Cristo; es un signo que demuestra lo que se realiza en nosotros en virtud de la pasión de Cristo, es decir, la gracia; y es un signo que anticipa, es decir, que preanuncia la gloria venidera}) (\emph{Summa theologiae} 3, q. 60, a. 3, c.) \end{quote}

\textbf{1403} En la última Cena, el Señor mismo atrajo la atención de sus discípulos hacia el cumplimiento de la Pascua en el Reino de Dios: \textquote{Y os digo que desde ahora no beberé de este fruto de la vid hasta el día en que lo beba con vosotros, de nuevo, en el Reino de mi Padre} (\emph{Mt} 26,29; cf. \emph{Lc} 22,18; \emph{Mc} 14,25). Cada vez que la Iglesia celebra la Eucaristía recuerda esta promesa y su mirada se dirige hacia \textquote{el que viene} (\emph{Ap} 1,4). En su oración, implora su venida: \emph{Marana tha} (\emph{1 Co} 16,22), \textquote{Ven, Señor Jesús} (\emph{Ap} 22,20), \textquote{que tu gracia venga y que este mundo pase} (\emph{Didaché} 10,6).

\textbf{Venga a nosotros tu Reino}

\textbf{2817} Esta petición es el \emph{Marana Tha}, el grito del Espíritu y de la Esposa: \textquote{Ven, Señor Jesús}:

\begin{quote} \textquote{Incluso aunque esta oración no nos hubiera mandado pedir el advenimiento del Reino, habríamos tenido que expresar esta petición, dirigiéndonos con premura a la meta de nuestras esperanzas. Las almas de los mártires, bajo el altar, invocan al Señor con grandes gritos: \textquote{¿Hasta cuándo, Dueño santo y veraz, vas a estar sin hacer justicia por nuestra sangre a los habitantes de la tierra?} (\emph{Ap} 6, 10). En efecto, los mártires deben alcanzar la justicia al fin de los tiempos. Señor, ¡apresura, pues, la venida de tu Reino!} (Tertuliano, \emph{De oratione}, 5, 2-4). \end{quote}

La vigilancia humilde del corazón

CEC 2729-2733:

\textbf{Frente a las dificultades de la oración}

\textbf{2729} La dificultad habitual de la oración es la \emph{distracción}. En la oración vocal, la distracción puede referirse a las palabras y al sentido de estas. La distracción, de un modo más profundo, puede referirse a Aquél al que oramos, tanto en la oración vocal (litúrgica o personal), como en la meditación y en la oración contemplativa. Dedicarse a perseguir las distracciones es caer en sus redes; basta con volver a nuestro corazón: la distracción descubre al que ora aquello a lo que su corazón está apegado. Esta humilde toma de conciencia debe empujar al orante a ofrecerse al Señor para ser purificado. El combate se decide cuando se elige a quién se desea servir (cf. \emph{Mt} 6,21.24).

\textbf{2730} Mirado positivamente, el combate contra el ánimo posesivo y dominador es la vigilancia, la sobriedad del corazón. Cuando Jesús insiste en la vigilancia, es siempre en relación a Él, a su Venida, al último día y al \textquote{hoy}. El esposo viene en mitad de la noche; la luz que no debe apagarse es la de la fe: \textquote{Dice de ti mi corazón: busca su rostro} (\emph{Sal} 27, 8).

\textbf{2731} Otra dificultad, especialmente para los que quieren sinceramente orar, es la \emph{sequedad}. Forma parte de la oración en la que el corazón está desprendido, sin gusto por los pensamientos, recuerdos y sentimientos, incluso espirituales. Es el momento en que la fe es más pura, la fe que se mantiene firme junto a Jesús en su agonía y en el sepulcro. \textquote{El grano de trigo, si [\ldots{}] muere, da mucho fruto} (\emph{Jn} 12, 24). Si la sequedad se debe a falta de raíz, porque la Palabra ha caído sobre roca, no hay éxito en el combate sin una mayor conversión (cf. \emph{Lc} 8, 6. 13).



\textbf{Frente a las tentaciones en la oración}

\textbf{2732} La tentación más frecuente, la más oculta, es nuestra \emph{falta de fe}. Esta se expresa menos en una incredulidad declarada que en unas preferencias de hecho. Cuando se empieza a orar, se presentan como prioritarios mil trabajos y cuidados que se consideran más urgentes; una vez más, es el momento de la verdad del corazón y de su más profundo deseo. Mientras tanto, nos volvemos al Señor como nuestro único recurso; pero ¿alguien se lo cree verdaderamente? Consideramos a Dios como asociado a la alianza con nosotros, pero nuestro corazón continúa en la arrogancia. En cualquier caso, la falta de fe revela que no se ha alcanzado todavía la disposición propia de un corazón humilde: \textquote{Sin mí, no podéis hacer nada} (\emph{Jn} 15, 5).

\textbf{2733} Otra tentación a la que abre la puerta la presunción es la \emph{acedia}. Los Padres espirituales entienden por ella una forma de aspereza o de desabrimiento debidos a la pereza, al relajamiento de la ascesis, al descuido de la vigilancia, a la negligencia del corazón. \textquote{El espíritu [\ldots{}] está pronto pero la carne es débil} (\emph{Mt} 26, 41). Cuanto más alto es el punto desde el que alguien toma decisiones, tanto mayor es la dificultad. El desaliento, doloroso, es el reverso de la presunción. Quien es humilde no se extraña de su miseria; ésta le lleva a una mayor confianza, a mantenerse firme en la constancia.

El que quisiere ver cuánto ha aprovechado en este camino de Dios, mire cuánto crece cada día en humildad interior y exterior. ¿Cómo sufre las injusticias de los otros? ¿Cómo sabe dar pasada a las flaquezas ajenas? ¿Cómo acude a las necesidades de sus prójimos? ¿Cómo se compadece y no se indigna contra los defectos ajenos? ¿Cómo sabe esperar en Dios en el tiempo de la tribulación? ¿Cómo rige su lengua? ¿Cómo guarda su corazón? ¿Cómo trae domada su carne con todos sus apetitos y sentidos? ¿ Cómo se sabe valer en las prosperidades y adversidades? ¿Cómo se repara y provee en todas las cosas con gravedad y discreción?

Y, sobre todo esto, mire si está muerto el amor de la honra, y del regalo, y del mundo, y según lo que en esto hubiere aprovechado o desaprovechado, así se juzgue, y no según lo que siente o no siente de Dios. Y por esto siempre ha de tener él un ojo, y el más principal en la mortificación, y el otro en la oración, porque esa misma mortificación no se puede perfectamente alcanzar sin el socorro de la oración.

\textbf{San Pedro de Alcántara}, \emph{Tratado sobre la Oración,} capítulo 5.

\chapter{Domingo II de Adviento (A)}

\section{Lecturas}

PRIMERA LECTURA

Del libro del profeta Isaías 11, 1-10

Con equidad dará sentencia al pobre

Esto dice el Señor:

«Aquel día brotará un renuevo del tronco de Jesé, y de su raíz florecerá
un vástago.

Sobre él se posará el espíritu del Señor: espíritu de sabiduría y
entendimiento, espíritu de consejo y fortaleza, espíritu de ciencia y
temor del Señor.

Lo inspirará el temor del Señor. No juzgará por apariencias ni
sentenciará de oídas; juzgará a los pobres con justicia, sentenciará con
rectitud a los sencillos de la tierra; pero golpeará al violento con la
vara de su boca, y con el soplo de sus labios hará morir al malvado.

La justicia será ceñidor de su cintura, y la lealtad, cinturón de sus
caderas.

Habitará el lobo con el cordero, el leopardo se tumbará con el cabrito,
el ternero y el león pacerán juntos: un muchacho será su pastor.

La vaca pastará con el oso, sus crías se tumbarán juntas; el león como
el buey, comerá paja.

El niño de pecho retoza junto al escondrijo de la serpiente, y el recién
destetado extiende la mano hacia la madriguera del áspid.

Nadie causará daño ni estrago por todo mi monte santo: porque está lleno
el país del conocimiento del Señor, como las aguas colman el mar.

Aquel día, la raíz de Jesé será elevada como enseña de los pueblos: se
volverán hacia ella las naciones y será gloriosa su morada.»

SALMO RESPONSORIAL

Salmo 71, 1-2. 7-8. 12-13. 17

Que en sus días florezca la justicia, y la paz abunde eternamente

℣. Dios mío, confía tu juicio al rey,

tu justicia al hijo de reyes,

para que rija a tu pueblo con justicia,

a tus humildes con rectitud. ℟.

℣. En sus días florezca la justicia

y la paz hasta que falte la luna;

domine de mar a mar,

del Gran Río al confín de la tierra. ℟.

℣. Él librará al pobre que clamaba,

al afligido que no tenía protector;

él se apiadará del pobre y del indigente,

y salvará la vida de los pobres. ℟.

℣. Que su nombre sea eterno,

y su fama dure como el sol;

él sea la bendición de todos los pueblos,

y lo proclamen dichoso todas las razas de la tierra. ℟.

SEGUNDA LECTURA

De la carta del apóstol san Pablo a los Romanos 15, 4-9

Cristo salvó a todos los hombres

Hermanos:

Todo lo que se escribió en el pasado, se escribió para enseñanza
nuestra, a fin de que a través de nuestra paciencia y del consuelo que
dan las Escrituras mantengamos la esperanza.

Que el Dios de la paciencia y del consuelo os conceda tener entre
vosotros los mismos sentimientos, según Cristo Jesús; de este modo,
unánimes, a una voz, glorificaréis al Dios y Padre de nuestro Señor
Jesucristo.

Por eso, acogeos mutuamente, como Cristo os acogió para gloria de Dios.
Es decir, Cristo se hizo servidor de la circuncisión en atención a la
fidelidad de Dios, para llevar a cumplimiento las promesas hechas a los
patriarcas y, en cuanto a los gentiles, para que glorifiquen a Dios por
su misericordia; como está escrito:

Por esto te alabaré entre los gentiles y cantaré para tu nombre.

EVANGELIO

Del Santo Evangelio según san Mateo 3, 1-12

Haced penitencia porque se acerca el Reino de los Cielos

Por aquellos días, Juan el Bautista se presentó en el desierto de Judea,
predicando:

\textquote{Convertíos, porque está cerca el reino de los cielos}.

Este es el que anunció el profeta Isaías diciendo:

«Voz del que grita en el desierto:

``Preparad el camino del Señor,

allanad sus senderos''».

Juan llevaba un vestido de piel de camello, con una correa de cuero a la
cintura, y se alimentaba de saltamontes y miel silvestre.

Y acudía a él toda la gente de Jerusalén, de Judea y de la comarca del
Jordán; confesaban sus pecados y él los bautizaba en el Jordán.

Al ver que muchos fariseos y saduceos venían a que los bautizara, les
dijo:

«¡Raza de víboras!, ¿quién os ha enseñado a escapar del castigo
inminente?

Dad el fruto que pide la conversión.

Y no os hagáis ilusiones, pensando: \textquote{Tenemos por padre a Abrahán},
pues os digo que Dios es capaz de sacar hijos de Abrahán de estas
piedras.

Ya toca el hacha la raíz de los árboles, y todo árbol que no dé buen
fruto será talado y echado al fuego.

Yo os bautizo con agua para que os convirtáis; pero el que viene detrás
de mí es más fuerte que yo y no merezco ni llevarle las sandalias.

Él os bautizará con Espíritu Santo y fuego.

Él tiene el bieldo en la mano: aventará su parva, reunirá su trigo en el
granero y quemará la paja en una hoguera que no se
apaga».

\section{Comentario Patrístico}



\subsection{San Agustín, obispo}

Convertíos, porque está cerca el Reino de los cielos

Sermón 109, 1: PL 38, 636.

Hemos escuchado el evangelio y en el evangelio al Señor descubriendo la ceguera de quienes son capaces de interpretar el aspecto del cielo, pero son incapaces de discernir el tiempo de la fe en un reino de los cielos que está ya llegando. Les decía esto a los judíos, pero sus palabras nos afectan también a nosotros. Y el mismo Jesucristo comenzó así la predicación de su evangelio: \emph{Convertíos, porque está cerca el Reino de los cielos}. Igualmente, Juan el Bautista, su Precursor, comenzó así: \emph{Convertíos, porque está cerca el Reino de los cielos}. Y ahora corrige el Señor a los que se niegan a convertirse, próximo ya el Reino de los cielos. \emph{El Reino de los cielos} ---como él mismo dice--- \emph{no vendrá espectacularmente}. Y añade: \emph{El Reino de Dios está dentro de vosotros}.

Que cada cual reciba con prudencia las admoniciones del preceptor, si no quiere perder la hora de misericordia del Salvador, misericordia que se otorga en la presente coyuntura, en que al género humano se le ofrece el perdón. Precisamente al hombre se le brinda el perdón para que se convierta y no haya a quien condenar. Eso lo ha de decidir Dios cuando llegue el fin del mundo; pero de momento nos hallamos en el tiempo de la fe. Si el fin del mundo encontrará o no aquí a alguno de nosotros, lo ignoro; posiblemente no encuentre a ninguno. Lo cierto es que el tiempo de cada uno de nosotros está cercano, pues somos mortales. Andamos en medio de peligros. Nos asustan más las caídas que si fuésemos de vidrio. ¿Y hay algo más frágil que un vaso de cristal? Y sin embargo se conserva y dura siglos. Y aunque pueda temerse la caída de un vaso de cristal, no hay miedo de que le afecte la vejez o la fiebre.

Somos, por tanto, más frágiles que el cristal porque debido indudablemente a nuestra propia fragilidad, cada día nos acecha el temor de los numerosos y continuos accidentes inherentes a la condición humana; y aunque estos temores no lleguen a materializarse, el tiempo corre: y el hombre que puede evitar un golpe, ¿podrá también evitar la muerte? Y si logra sustraerse a los peligros exteriores, ¿logrará evitar asimismo los que vienen de dentro? Unas veces son los virus que se multiplican en el interior del hombre, otras es la enfermedad que súbitamente se abate sobre nosotros; y aun cuando logre verse libre de estas taras, acabará finalmente por llegarle la vejez, sin moratoria posible.

\section{Homilías}

\subsection{San Juan Pablo II, papa}

\subsubsection{Homilía: Renovar la espera}

Visita Pastoral a la Parroquia Romana Santa Rosa de Viterbo.

Domingo 6 de diciembre de 1998.

1. \textquote{Preparad el camino del Señor} (\emph{Mt 3,3}). Estas palabras, tomadas del libro del profeta Isaías (cf. \emph{Is 40,3}), las pronunció san Juan Bautista, a quien Jesús mismo definió en una ocasión el más grande entre los nacidos de mujer (cf. \emph{Mt 11,11}). El \textbf{evangelista san Mateo} lo presenta como el Precursor, es decir, el que recibió la misión de \textquote{preparar el camino} al Mesías.

Su apremiante exhortación a la penitencia y a la conversión sigue resonando en el mundo e impulsa a los creyentes (\ldots{}) a acoger dignamente al Señor que viene\ldots{}

Amadísimos hermanos y hermanas, preparémonos para el encuentro con Cristo. Preparémosle el camino en nuestro corazón y en nuestras comunidades. La figura del \textbf{Bautista}, que viste con pobreza y se alimenta con langostas y miel silvestre, constituye un fuerte llamamiento a la vigilancia y a la espera del Salvador.

2. \textquote{Aquel día, brotará un renuevo del tronco de Jesé} (\emph{Is 11,1}). En el tiempo del Adviento, la liturgia pone de relieve otra gran figura: el profeta \textbf{Isaías}, que, en el seno del pueblo elegido, mantuvo viva la expectativa, llena de esperanza, en la venida del Salvador prometido. Como hemos escuchado en la primera lectura, Isaías describe al Mesías como un vástago que sale del antiguo tronco de Jesé. El Espíritu de Dios se posará plenamente sobre él y su reino se caracterizará por el restablecimiento de la justicia y la consolidación de la paz universal.

También nosotros necesitamos renovar esta espera confiada en el Señor. Escuchemos las palabras del \textbf{profeta}. Nos invitan a aguardar con esperanza la instauración definitiva del reino de Dios, que él describe con imágenes muy poéticas, capaces de poner de relieve el triunfo de la justicia y la paz por obra del Mesías. \textquote{Habitarán el lobo y el cordero, (\ldots{}) el novillo y el león pacerán juntos, y un niño pequeño los pastoreará} (\emph{Is 11,6}). Se trata de expresiones simbólicas, que anticipan la realidad de una reconciliación universal. En esta obra de renovación cósmica todos estamos llamados a colaborar, sostenidos por la certeza de que un día toda la creación se someterá completamente al señorío universal de Cristo.

5. \textquote{Acogeos mutuamente como os acogió Cristo} (\emph{Rm 15,7}). San \textbf{Pablo}, indicándonos el sentido profundo del Adviento, manifiesta la necesidad de la acogida y la fraternidad en cada familia y en cada comunidad. Acoger a Cristo y abrir el corazón a los hermanos es nuestro compromiso diario, al que nos impulsa el clima espiritual de este tiempo litúrgico.

El Apóstol prosigue: \textquote{El Dios de la paciencia y del consuelo os conceda tener los unos para con los otros los mismos sentimientos, según Cristo Jesús, para que unánimes, a una voz, glorifiquéis al Dios y Padre de nuestro Señor Jesucristo} (\emph{Rm 15,5-6}). Que el Adviento y la próxima celebración del nacimiento de Jesús refuercen en cada creyente este sentido de unidad y comunión.

Que María, la Virgen de la escucha y la acogida, nos acompañe en el itinerario del Adviento, y nos guíe para ser testigos creíbles y generosos del amor salvífico de Dios. Amén.

\subsection{Benedicto XVI, Papa}

\subsubsection{Ángelus: Palabras saludables}

Plaza de San Pedro. Domingo 9 de diciembre del 2007.

[\ldots{}] Hoy, segundo domingo de Adviento, se nos presenta la figura austera del \textbf{Precursor}, que el \textbf{evangelista san Mateo} introduce así: \textquote{Por aquel tiempo, Juan Bautista se presentó en el desierto de Judea predicando: \textquote{Convertíos, porque está cerca el reino de los cielos}} (\emph{Mt} 3, 1-2). Tenía la misión de preparar y allanar el sendero al Mesías, exhortando al pueblo de Israel a arrepentirse de sus pecados y corregir toda injusticia. Con palabras exigentes, Juan Bautista anunciaba el juicio inminente: \textquote{El árbol que no da fruto será talado y echado al fuego} (\emph{Mt} 3, 10). Sobre todo ponía en guardia contra la hipocresía de quien se sentía seguro por el mero hecho de pertenecer al pueblo elegido: ante Dios ---decía--- nadie tiene títulos para enorgullecerse, sino que debe dar \textquote{frutos dignos de conversión} (\emph{Mt} 3, 8).

Mientras prosigue el camino del Adviento, mientras nos preparamos para celebrar el Nacimiento de Cristo, resuena en nuestras comunidades esta exhortación de \textbf{Juan Bautista} a la conversión. Es una invitación apremiante a abrir el corazón y acoger al Hijo de Dios que viene a nosotros para manifestar el juicio divino. El Padre ---escribe el evangelista san Juan--- no juzga a nadie, sino que ha dado al Hijo el poder de juzgar, porque es Hijo del hombre (cf. \emph{Jn} 5, 22. 27). Hoy, en el presente, es cuando se juega nuestro destino futuro; con el comportamiento concreto que tenemos en esta vida decidimos nuestro destino eterno. En el ocaso de nuestros días en la tierra, en el momento de la muerte, seremos juzgados según nuestra semejanza o desemejanza con el Niño que está a punto de nacer en la pobre cueva de Belén, puesto que él es el criterio de medida que Dios ha dado a la humanidad.

El Padre celestial, que en el nacimiento de su Hijo unigénito nos manifestó su amor misericordioso, nos llama a seguir sus pasos convirtiendo, como él, nuestra existencia en un don de amor. Y los frutos del amor son los \textquote{frutos dignos de conversión} a los que hacía referencia san \textbf{Juan Bautista} cuando, con palabras tajantes, se dirigía a los fariseos y a los saduceos que acudían entre la multitud a su bautismo.

Mediante el \textbf{Evangelio}, Juan Bautista sigue hablando a lo largo de los siglos a todas las generaciones. Sus palabras claras y duras resultan muy saludables para nosotros, hombres y mujeres de nuestro tiempo, en el que, por desgracia, también el modo de vivir y percibir la Navidad muy a menudo sufre las consecuencias de una mentalidad materialista. La \textquote{voz} del gran profeta nos pide que preparemos el camino del Señor que viene, en los desiertos de hoy, desiertos exteriores e interiores, sedientos del agua viva que es Cristo.

Que la Virgen María nos guíe a una auténtica conversión del corazón, a fin de que podamos realizar las opciones necesarias para sintonizar nuestra mentalidad con el Evangelio.

\subsubsection{Ángelus: Voz de Dios en el desierto del mundo}

Plaza de San Pedro.

Domingo 5 de diciembre del 2010.

El \textbf{Evangelio} de este segundo domingo de Adviento (\emph{Mt} 3, 1-12) nos presenta la figura de san Juan Bautista, el cual, según una célebre profecía de Isaías (cf. 40, 3), se retiró al desierto de Judea y, con su predicación, llamó al pueblo a convertirse para estar preparado para la inminente venida del Mesías. San Gregorio Magno comenta que el Bautista \textquote{predica la recta fe y las obras buenas\ldots{} para que la fuerza de la gracia penetre, la luz de la verdad resplandezca, los caminos hacia Dios se enderecen y nazcan en el corazón pensamientos honestos tras la escucha de la Palabra que guía hacia el bien} (\emph{Hom. in Evangelia,} XX, 3: CCL 141, 155). El precursor de Jesús, situado entre la Antigua y la Nueva Alianza, es como una estrella que precede la salida del Sol, de Cristo, es decir, de Aquel sobre el cual ---según otra profecía de Isaías--- \textquote{reposará el espíritu del Señor: espíritu de sabiduría e inteligencia, espíritu de consejo y fortaleza, espíritu de ciencia y temor del Señor} (\emph{Is} 11, 2).

En el tiempo de Adviento, también nosotros estamos llamados a escuchar la voz de Dios, que resuena en el desierto del mundo a través de las Sagradas Escrituras, especialmente cuando se predican con la fuerza del Espíritu Santo. De hecho, la fe se fortalece cuanto más se deja iluminar por la Palabra divina, por \textquote{todo cuanto ---como nos recuerda el \textbf{apóstol san Pablo}--- fue escrito en el pasado\ldots{} para enseñanza nuestra, para que con la paciencia y el consuelo que dan las Escrituras mantengamos la esperanza} (\emph{Rm} 15, 4). El modelo de la escucha es la Virgen María: \textquote{Contemplando en la Madre de Dios una existencia totalmente modelada por la Palabra, también nosotros nos sentimos llamados a entrar en el misterio de la fe, con la que Cristo viene a habitar en nuestra vida}. San Ambrosio nos recuerda que \textquote{todo cristiano que cree, concibe en cierto sentido y engendra al Verbo de Dios en sí mismo} (\emph{Verbum Domini,} 28).

Queridos amigos, \textquote{nuestra salvación se basa en una venida}, escribió Romano Guardini (\emph{La santa notte. Dall'Avvento all'Epifania}, Brescia 1994, p. 13). \textquote{El Salvador vino por la libertad de Dios\ldots{} Así la decisión de la fe consiste\ldots{} en acoger a Aquel que se acerca} (\emph{ib}., p. 14). \textquote{El Redentor ---añade--- viene a cada hombre: en sus alegrías y penas, en sus conocimientos claros, en sus dudas y tentaciones, en todo lo que constituye su naturaleza y su vida} (\emph{ib}., p. 15).

A la Virgen María, en cuyo seno habitó el Hijo del Altísimo, {[}y que el miércoles próximo, 8 de diciembre, celebraremos en la solemnidad de la Inmaculada Concepción,{]} pedimos que nos sostenga en este camino espiritual, para acoger con fe y con amor la venida del Salvador.


\subsection{Francisco, papa}

\subsubsection{Ángelus: El Reino ha llegado ya}

Plaza de San Pedro. Domingo 4 de diciembre del 2016.

En el \textbf{Evangelio} de este segundo domingo de Adviento resuena la invitación de Juan Bautista: \textquote{¡Convertíos porque el reino de los cielos está cerca!} (Mt 3,2). Con estas palabras Jesús dará inicio a su misión en Galilea (cfr Mt 4,17); y tal será también el anuncio que deberán llevar los discípulos en su primera experiencia misionera (cfr Mt 10,7).

El evangelista Mateo quiere así presentar a \textbf{Juan} como el que prepara el camino al Cristo que viene, y los discípulos como los continuadores de la predicación de Jesús. Se trata del mismo anuncio alegre: ¡viene el reino de Dios, es más, está cerca, está en medio de nosotros! Esta palabra es muy importante: \textquote{el reino de Dios está en medio de vosotros}, dice Jesús. Y Juan anuncia esto que Jesús luego dirá: \textquote{El reino de Dios ha venido, ha llegado, está en medio de vosotros}. Este es el mensaje central de toda misión cristiana. Cuando un misionero va, un cristiano va a anunciar a Jesús, no va a hacer proselitismo como si fuera un hincha que busca más seguidores para su equipo. No, va simplemente a anunciar: \textquote{¡El reino de Dios está en medio de vosotros!}. Y así el misionero prepara el camino a Jesús, que encuentra a su pueblo.

¿Pero qué es este reino de Dios, reino de los cielos? Son sinónimos. Nosotros pensamos enseguida en algo que se refiere al más allá: la vida eterna. Cierto, esto es verdad, el reino de Dios se extenderá sin fin más allá de la vida terrena, pero la buena noticia que Jesús nos trae ---y que Juan anticipa--- es que el reino de Dios no tenemos que esperarlo en el futuro: se ha acercado, de alguna manera está ya presente y podemos experimentar desde ahora el poder espiritual. Dios viene a establecer su señorío en la historia, en nuestra vida de cada día; y allí donde esta viene acogida con fe y humildad brotan el amor, la alegría y la paz.

La condición para entrar a formar parte de este reino es cumplir un cambio en nuestra vida, es decir, convertirnos. Convertirnos cada día, un paso adelante cada día. Se trata de dejar los caminos, cómodos pero engañosos, de los ídolos de este mundo: el éxito a toda costa, el poder a costa de los más débiles, la sed de riquezas, el placer a cualquier precio. Y de abrir sin embargo el camino al Señor que viene: Él no nos quita nuestra libertad, sino que nos da la verdadera felicidad. Con el nacimiento de Jesús en Belén, es Dios mismo que viene a habitar en medio de nosotros para librarnos del egoísmo, del pecado y de la corrupción, de estas actitudes que son del diablo: buscar éxito a toda costa, el poder a costa de los más débiles, tener sed de riquezas y buscar el placer a cualquier precio.

La Navidad es un día de gran alegría también exterior, pero es sobre todo un evento religioso por lo que es necesaria una preparación espiritual. En este tiempo de Adviento, dejémonos guiar por la \textbf{exhortación del Bautista}: \textquote{Preparad el camino al Señor, allanad sus senderos} (v. 3).

Nosotros preparamos el camino del Señor y allanamos sus senderos cuando examinamos nuestra conciencia, cuando escrutamos nuestras actitudes, cuando con sinceridad y confianza confesamos nuestros pecados en el sacramento de la penitencia. En este sacramento experimentamos en nuestro corazón la cercanía del reino de Dios y su salvación.

La salvación de Dios es trabajo de un amor más grande que nuestro pecado; solamente el amor de Dios puede cancelar el pecado y liberar del mal, y solamente el amor de Dios puede orientarnos sobre el camino del bien. Que la Virgen María nos ayude a prepararnos al encuentro con este Amor cada vez más grande que en la noche de Navidad se ha hecho pequeño pequeño, como una semilla caída en la tierra, la semilla del reino de Dios.



\section{Temas}

Los profetas y la espera del Mesías

CEC 522, 711-716, 722:

\textbf{Los misterios de la infancia y de la vida oculta de Jesús: Los preparativos}

\textbf{522} La venida del Hijo de Dios a la tierra es un acontecimiento tan inmenso que Dios quiso prepararlo durante siglos. Ritos y sacrificios, figuras y símbolos de la \textquote{Primera Alianza} (\emph{Hb} 9,15), todo lo hace converger hacia Cristo; anuncia esta venida por boca de los profetas que se suceden en Israel. Además, despierta en el corazón de los paganos una espera, aún confusa, de esta venida.

\textbf{La espera del Mesías y de su Espíritu}

\textbf{711} \textquote{He aquí que yo lo renuevo} (\emph{Is} 43, 19): dos líneas proféticas se van a perfilar, una se refiere a la espera del Mesías, la otra al anuncio de un Espíritu nuevo, y las dos convergen en el pequeño Resto, el pueblo de los Pobres (cf. \emph{So} 2, 3), que aguardan en la esperanza la \textquote{consolación de Israel} y \textquote{la redención de Jerusalén} (cf. \emph{Lc} 2, 25. 38).

Ya se ha dicho cómo Jesús cumple las profecías que a Él se refieren. A continuación se describen aquéllas en que aparece sobre todo la relación del Mesías y de su Espíritu.

\textbf{712} Los rasgos del rostro del \emph{Mesías} esperado comienzan a aparecer en el Libro del Emmanuel (cf. \emph{Is} 6, 12) (cuando \textquote{Isaías vio [\ldots{}] la gloria} de Cristo \emph{Jn} 12, 41), especialmente en \emph{Is} 11, 1-2:

«Saldrá un vástago del tronco de Jesé, y un retoño de sus raíces brotará. Reposará sobre él el Espíritu del Señor: espíritu de sabiduría e inteligencia, espíritu de consejo y de fortaleza, espíritu de ciencia y temor del Señor».

\textbf{713} Los rasgos del Mesías se revelan sobre todo en los Cantos del Siervo (cf. \emph{Is} 42, 1-9; cf. \emph{Mt} 12, 18-21; \emph{Jn} 1, 32-34; y también \emph{Is} 49, 1-6; cf. \emph{Mt} 3, 17; \emph{Lc} 2, 32, y por último \emph{Is} 50, 4-10 y 52, 13-53, 12). Estos cantos anuncian el sentido de la Pasión de Jesús, e indican así cómo enviará el Espíritu Santo para vivificar a la multitud: no desde fuera, sino desposándose con nuestra \textquote{condición de esclavos} (\emph{Flp} 2, 7). Tomando sobre sí nuestra muerte, puede comunicarnos su propio Espíritu de vida.

\textbf{714} Por eso Cristo inaugura el anuncio de la Buena Nueva haciendo suyo este pasaje de Isaías (\emph{Lc} 4, 18-19; cf. \emph{Is} 61, 1-2):

«El Espíritu del Señor está sobre mí, porque me ha ungido. Me ha enviado a anunciar a los pobres la Buena Nueva, a proclamar la liberación a los cautivos y la vista a los ciegos, para dar la libertad a los oprimidos y proclamar un año de gracia del Señor».

\textbf{715} Los textos proféticos que se refieren directamente al envío del Espíritu Santo son oráculos en los que Dios habla al corazón de su Pueblo en el lenguaje de la Promesa, con los acentos del \textquote{amor y de la fidelidad} (cf. \emph{Ez} 11, 19; 36, 25-28; 37, 1-14; \emph{Jr} 31, 31-34; y \emph{Jl} 3, 1-5), cuyo cumplimiento proclamará San Pedro la mañana de Pentecostés (cf. \emph{Hch} 2, 17-21). Según estas promesas, en los \textquote{últimos tiempos}, el Espíritu del Señor renovará el corazón de los hombres grabando en ellos una Ley nueva; reunirá y reconciliará a los pueblos dispersos y divididos; transformará la primera creación y Dios habitará en ella con los hombres en la paz.

\textbf{716} El Pueblo de los \textquote{pobres} (cf. \emph{So} 2, 3; \emph{Sal} 22, 27; 34, 3; \emph{Is} 49, 13; 61, 1; etc.), los humildes y los mansos, totalmente entregados a los designios misteriosos de Dios, los que esperan la justicia, no de los hombres sino del Mesías, todo esto es, finalmente, la gran obra de la Misión escondida del Espíritu Santo durante el tiempo de las Promesas para preparar la venida de Cristo. Esta es la calidad de corazón del Pueblo, purificado e iluminado por el Espíritu, que se expresa en los Salmos. En estos pobres, el Espíritu prepara para el Señor \textquote{un pueblo bien dispuesto} (cf. \emph{Lc} 1, 17).

\textbf{722} El Espíritu Santo \emph{preparó} a María con su gracia. Convenía que fuese \textquote{llena de gracia} la Madre de Aquel en quien \textquote{reside toda la plenitud de la divinidad corporalmente} (\emph{Col} 2, 9). Ella fue concebida sin pecado, por pura gracia, como la más humilde de todas las criaturas, la más capaz de acoger el don inefable del Omnipotente. Con justa razón, el ángel Gabriel la saluda como la \textquote{Hija de Sión}: \textquote{Alégrate} (cf. \emph{So} 3, 14; \emph{Za} 2, 14). Cuando ella lleva en sí al Hijo eterno, hace subir hasta el cielo con su cántico al Padre, en el Espíritu Santo, la acción de gracias de todo el pueblo de Dios y, por tanto, de la Iglesia (cf. \emph{Lc} 1, 46-55).

La misión de Juan Bautista

CEC 523, 717-720:

\textbf{523} \emph{San Juan Bautista} es el precursor (cf. \emph{Hch} 13, 24) inmediato del Señor, enviado para prepararle el camino (cf. \emph{Mt} 3, 3). \textquote{Profeta del Altísimo} (\emph{Lc} 1, 76), sobrepasa a todos los profetas (cf. \emph{Lc} 7, 26), de los que es el último (cf. \emph{Mt} 11, 13), e inaugura el Evangelio (cf. \emph{Hch} 1, 22; \emph{Lc} 16,16); desde el seno de su madre (cf. \emph{Lc} 1,41) saluda la venida de Cristo y encuentra su alegría en ser \textquote{el amigo del esposo} (\emph{Jn} 3, 29) a quien señala como \textquote{el Cordero de Dios que quita el pecado del mundo} (\emph{Jn} 1, 29). Precediendo a Jesús \textquote{con el espíritu y el poder de Elías} (\emph{Lc} 1, 17), da testimonio de él mediante su predicación, su bautismo de conversión y finalmente con su martirio (cf. \emph{Mc} 6, 17-29).

\textbf{El Espíritu de Cristo en la plenitud de los tiempos. Juan, Precursor, Profeta y Bautista}

\textbf{717} \textquote{Hubo un hombre, enviado por Dios, que se llamaba Juan} (\emph{Jn} 1, 6). Juan fue \textquote{lleno del Espíritu Santo ya desde el seno de su madre} (\emph{Lc} 1, 15. 41) por obra del mismo Cristo que la Virgen María acababa de concebir del Espíritu Santo. La \textquote{Visitación} de María a Isabel se convirtió así en \textquote{visita de Dios a su pueblo} (\emph{Lc} 1, 68).

\textbf{718} Juan es \textquote{Elías que debe venir} (\emph{Mt} 17, 10-13): El fuego del Espíritu lo habita y le hace correr delante {[}como \textquote{precursor}{]} del Señor que viene. En Juan el Precursor, el Espíritu Santo culmina la obra de \textquote{preparar al Señor un pueblo bien dispuesto} (\emph{Lc} 1, 17).

\textbf{719} Juan es \textquote{más que un profeta} (\emph{Lc} 7, 26). En él, el Espíritu Santo consuma el \textquote{hablar por los profetas}. Juan termina el ciclo de los profetas inaugurado por Elías (cf. \emph{Mt} 11, 13-14). Anuncia la inminencia de la consolación de Israel, es la \textquote{voz} del Consolador que llega (\emph{Jn} 1, 23; cf. \emph{Is} 40, 1-3). Como lo hará el Espíritu de Verdad, \textquote{vino como testigo para dar testimonio de la luz} (\emph{Jn} 1, 7; cf. \emph{Jn} 15, 26; 5, 33). Con respecto a Juan, el Espíritu colma así las \textquote{indagaciones de los profetas} y la ansiedad de los ángeles (\emph{1 P} 1, 10-12): \textquote{Aquél sobre quien veas que baja el Espíritu y se queda sobre él, ése es el que bautiza con el Espíritu Santo. Y yo lo he visto y doy testimonio de que éste es el Hijo de Dios [\ldots{}] He ahí el Cordero de Dios} (\emph{Jn} 1, 33-36).

\textbf{720} En fin, con Juan Bautista, el Espíritu Santo, inaugura, prefigurándolo, lo que realizará con y en Cristo: volver a dar al hombre la \textquote{semejanza} divina. El bautismo de Juan era para el arrepentimiento, el del agua y del Espíritu será un nuevo nacimiento (cf. \emph{Jn} 3, 5).

La conversión de los bautizados

CEC 1427-1429:

\textbf{1427} Jesús llama a la conversión. Esta llamada es una parte esencial del anuncio del Reino: \textquote{El tiempo se ha cumplido y el Reino de Dios está cerca; convertíos y creed en la Buena Nueva} (\emph{Mc} 1,15). En la predicación de la Iglesia, esta llamada se dirige primeramente a los que no conocen todavía a Cristo y su Evangelio. Así, el Bautismo es el lugar principal de la conversión primera y fundamental. Por la fe en la Buena Nueva y por el Bautismo (cf. \emph{Hch} 2,38) se renuncia al mal y se alcanza la salvación, es decir, la remisión de todos los pecados y el don de la vida nueva.

\textbf{1428} Ahora bien, la llamada de Cristo a la conversión sigue resonando en la vida de los cristianos. Esta \emph{segunda conversión} es una tarea ininterrumpida para toda la Iglesia que \textquote{recibe en su propio seno a los pecadores} y que siendo \textquote{santa al mismo tiempo que necesitada de purificación constante, busca sin cesar la penitencia y la renovación} (\href{http://www.vatican.va/archive/hist_councils/ii_vatican_council/documents/vat-ii_const_19641121_lumen-gentium_sp.html}{LG} 8). Este esfuerzo de conversión no es sólo una obra humana. Es el movimiento del \textquote{corazón contrito} (\emph{Sal} 51,19), atraído y movido por la gracia (cf. \emph{Jn} 6,44; 12,32) a responder al amor misericordioso de Dios que nos ha amado primero (cf. \emph{1 Jn} 4,10).

\textbf{1429} De ello da testimonio la conversión de san Pedro tras la triple negación de su Maestro. La mirada de infinita misericordia de Jesús provoca las lágrimas del arrepentimiento (\emph{Lc} 22,61) y, tras la resurrección del Señor, la triple afirmación de su amor hacia él (cf. \emph{Jn} 21,15-17). La segunda conversión tiene también una dimensión \emph{comunitaria}. Esto aparece en la llamada del Señor a toda la Iglesia: \textquote{¡Arrepiéntete!} (\emph{Ap} 2,5.16).

San Ambrosio dice acerca de las dos conversiones que, \textquote{en la Iglesia, existen el agua y las lágrimas: el agua del Bautismo y las lágrimas de la Penitencia} (\emph{Epistula extra collectionem} 1 {[}41{]}, 12).

Mas yo, tirándome debajo de una higuera, no sé cómo, solté la rienda a las lágrimas, brotando dos ríos de mis ojos, sacrificio tuyo aceptable. Y aunque no con estas palabras, pero sí con el mismo sentido, te dije muchas cosas como éstas: \emph{¡Y tú, Señor, hasta cuándo! ¡Hasta cuándo, Señor, has de estar irritado!} No te acuerdes más de nuestras maldades pasadas. Me sentía aún cautivo de ellas y lanzaba voces lastimeras: \textquote{¿Hasta cuándo, hasta cuándo, ¡mañana!, ¡mañana!? ¿Por qué no hoy? ¿Por qué no poner fin a mis torpezas ahora mismo?}.

Decía estas cosas y lloraba con muy dolorosa contrición de mi corazón. Pero he aquí que oigo de la casa vecina una voz, como de niño o niña, que decía cantando y repetía muchas veces: \textquote{\emph{Toma y lee, toma y lee}} (tolle lege, tolle lege).

\ldots{} Así que, apresurado, volví al lugar donde estaba sentado Alipio y yo había dejado el códice del Apóstol al levantarme de allí. Lo tomé, lo abrí y leí en silencio el primer capítulo que se me vino a los ojos, que decía: \emph{No en comilonas y embriagueces, no en lechos y en liviandades, no en contiendas y emulaciones sino revestíos de nuestro Señor Jesucristo y no cuidéis de la carne con demasiados deseos}.

No quise leer más, ni era necesario tampoco, pues al punto que di fin a la sentencia, como si se hubiera infiltrado en mi corazón una luz de seguridad, se disiparon todas las tinieblas de mis dudas.

Después entramos a ver a mi madre, indicándoselo, y se llenó de gozo; le contamos el modo como había sucedido, y saltaba de alegría y cantaba victoria, por lo cual te bendecía a ti, que eres poderoso para darnos más de lo que pedimos o entendemos, porque veía que le habías concedido, respecto de mí, mucho más de lo que constantemente te pedía con sollozos y lágrimas piadosas.

\textbf{San Agustín}, \emph{Confesiones ,} Libro VIII, capítulo 12.

\chapter{***}
\chapter{Domingo III de Adviento (A)}

\section{Lecturas}

PRIMERA LECTURA

Del libro del profeta Isaías 35, 1-6a. 10

Dios vendrá y nos salvará

El desierto y el yermo se regocijarán,

se alegrará la estepa y florecerá,

germinará y florecerá como flor de narciso,

festejará con gozo y cantos de júbilo.

Le ha sido dada la gloria del Líbano,

el esplendor del Carmelo y del Sarón.

Contemplarán la gloria del Señor,

la majestad de nuestro Dios.

Fortaleced las manos débiles,

afianzad las rodillas vacilantes;

decid a los inquietos:

«Sed fuertes, no temáis.

¡He aquí vuestro Dios! Llega el desquite,

la retribución de Dios.

Viene en persona y os salvará».

Entonces se despegarán los ojos de los ciegos,

los oídos de los sordos se abrirán;

entonces saltará el cojo como un ciervo.

Retornan los rescatados del Señor.

Llegarán a Sión con cantos de júbilo:

alegría sin límite en sus rostros.

Los dominan el gozo y la alegría.

Quedan atrás la pena y la aflicción.

SALMO RESPONSORIAL

Salmo 145, 6c-7. 8-9a. 9bc-10

Ven, Señor, a salvarnos

℣. El Señor mantiene su fidelidad perpetuamente,

hace justicia a los oprimidos,

da pan a los hambrientos.

El Señor liberta a los cautivos. ℟.

℣. El Señor abre los ojos al ciego,

el Señor endereza a los que ya se doblan,

el Señor ama a los justos.

El Señor guarda a los peregrinos. ℟.

℣. Sustenta al huérfano y a la viuda

y trastorna el camino de los malvados.

El Señor reina eternamente,

tu Dios, Sión, de edad en edad. ℟.

SEGUNDA LECTURA

De la carta del apóstol Santiago 5, 7-10

Manteneos firmes, porque la venida del Señor está cerca

Hermanos, esperad con paciencia hasta la venida del Señor. Mirad: el
labrador aguarda el fruto precioso de la tierra, esperando con paciencia
hasta que recibe la lluvia temprana y la tardía.

Esperad con paciencia también vosotros, y fortaleced vuestros corazones,
porque la venida del Señor está cerca.

Hermanos, no os quejéis los unos de los otros, para que no seáis
condenados; mirad: el juez está ya a las puertas.

Hermanos, tomad como modelo de resistencia y de paciencia a los profetas
que hablaron en nombre del Señor.

EVANGELIO

Del Santo Evangelio según san Mateo 11, 2-11

¿Eres tú el que ha de venir o tenemos que esperar a otro?

En aquel tiempo, Juan, que había oído en la cárcel las obras del Mesías,
mandó a sus discípulos a preguntarle:

\textquote{¿Eres tú el que ha de venir o tenemos que esperar a otro?}.

Jesús les respondió:

«Id a anunciar a Juan lo que estáis viendo y oyendo:

los ciegos ven y los cojos andan;

los leprosos quedan limpios y los sordos oyen;

los muertos resucitan

y los pobres son evangelizados.

¡Y bienaventurado el que no se escandalice de mí!».

Al irse ellos, Jesús se puso a hablar a la gente sobre Juan:

«¿Qué salisteis a contemplar en el desierto, una caña sacudida por el
viento? ¿O qué salisteis a ver, un hombre vestido con lujo? Mirad, los
que visten con lujo habitan en los palacios. Entonces, ¿a qué
salisteis?, ¿a ver a un profeta?

Sí, os digo, y más que profeta. Este es de quien está escrito:

\textquote{Yo envío a mi mensajero delante de ti, el cual preparará tu camino ante ti}.

En verdad os digo que no ha nacido de mujer uno más grande que Juan el
Bautista; aunque el más pequeño en el reino de los cielos es más grande
que él.»



\section{Comentario Patrístico}

\subsection{San Ambrosio, obispo}

¿Eres tú el que ha de venir o tenemos que esperar a otro?

Comentario sobre el evangelio de san Lucas,

Lib. 5, 93-95. 99-102. 109: CCL 14, 165-166. 167-168. 171-177.

\emph{Juan envió a dos de sus discípulos a preguntar a Jesús: \textquote{¿Eres tú el que ha de venir o tenemos que esperar a otro?}}. No es sencilla la comprensión de estas sencillas palabras, o de lo contrario este texto estaría en contradicción con lo dicho anteriormente. ¿Cómo, en efecto, puede Juan afirmar aquí que desconoce a quien anteriormente había reconocido por revelación de Dios Padre? ¿Cómo es que entonces conoció al que previamente desconocía mientras que ahora parece desconocer al que ya antes conocía? \emph{Yo} ---dice--- \emph{no lo conocía, pero el que me envió a bautizar con agua me dijo: \textquote{Aquel sobre quien veas bajar el Espíritu Santo\ldots{}}}. Y Juan dio fe al oráculo, reconoció al revelado, adoró al bautizado y profetizó al enviado Y concluye: \emph{Y yo lo he visto, y he dado testimonio de que éste es el elegido de Dios}. ¿Cómo, pues, aceptar siquiera la posibilidad de que un profeta tan grande haya podido equivocarse, hasta el punto de no considerar aún como Hijo de Dios a aquel de quien había afirmado: \emph{Éste es el que quita el pecado del mundo?}

Así pues, ya que la interpretación literal es contradictoria, busquemos el sentido espiritual. Juan --lo hemos dicho ya-- era tipo de la ley, precursora de Cristo. Y es correcto afirmar que la ley --aherrojada materialmente como estaba en los corazones de los sin fe, como en cárceles privadas de la luz eterna, y constreñida por entrañas fecundas en sufrimientos e insensatez-- era incapaz de llevar a pleno cumplimiento el testimonio de la divina economía sin la garantía del evangelio. Por eso, envía Juan a Cristo dos de sus discípulos, para conseguir un suplemento de sabiduría, dado que Cristo es la plenitud de la ley.

Además, sabiendo el Señor que nadie puede tener una fe plena sin el evangelio ---ya que si la fe comienza en el antiguo Testamento no se consuma sino en el nuevo---, a la pregunta sobre su propia identidad, responde no con palabras, sino con hechos. \emph{Id} ---dice--- \emph{a anunciar a Juan lo que estáis viendo y oyendo: los ciegos ven y los inválidos andan; los leprosos quedan limpios y los sordos oyen; los muertos resucitan y a los pobres se les anuncia la buena noticia}. Y sin embargo, estos ejemplos aducidos por el Señor no son aún los definitivos: la plenificación de la fe es la cruz del Señor, su muerte, su sepultura. Por eso, completa sus anteriores afirmaciones añadiendo: \emph{¡Y dichoso el que no se sienta defraudado por mí!} Es verdad que la cruz se presta a ser motivo de escándalo incluso para los elegidos, pero no lo es menos que no existe mayor testimonio de una persona divina, nada hay más sobrehumano que la íntegra oblación de uno solo por la salvación del mundo; este solo hecho lo acredita plenamente como Señor. Por lo demás, así es cómo Juan lo designa: \emph{Este es el Cordero de Dios, que quita el pecado del mundo}. En realidad, esta respuesta no va únicamente dirigida a aquellos dos hombres, discípulos de Juan: va dirigida a todos nosotros, para que creamos en Cristo en base a los hechos.

\emph{Entonces, ¿a qué salisteis?, ¿a ver a un profeta? Sí, os digo, y más que profeta}. Pero, ¿cómo es que querían ver a Juan en el desierto, si estaba encerrado en la cárcel? El Señor propone a nuestra imitación a aquel que le había preparado el camino no sólo precediéndolo en el nacimiento según la carne y anunciándolo con la fe, sino también anticipándosele con su gloriosa pasión. Más que profeta, sí, ya que es él quien cierra la serie de los profetas; más que profeta, ya que muchos desearon ver a quien éste profetizó, a quien éste contempló, a quien éste bautizó.

\section{Homilías}

\subsection{San Juan Pablo II, papa}


\subsubsection{Homilía (1983): La fuente de la alegría}

Visita a la Parroquia romana de San Camilo de Lellis.

Domingo 11 de diciembre de 1983.

1. El tercer domingo de Adviento es una apremiante invitación a la alegría. Precisamente, por las primeras palabras del texto latino de la \textquote{\textbf{antífona de entrada}}, este domingo se llama domingo de \emph{Gaudete} (cf. Flp 4, 4. 5).

El \textbf{profeta Isaías} invita a la naturaleza misma a manifestar con exuberante regocijo signos de júbilo: \textquote{El desierto y el yermo se regocijarán, se alegrará la estepa y florecerá} (Is 35, 1), porque muy pronto \textquote{contemplarán la gloria del Señor} (Is 35, 2).

Es la alegría del Adviento que, en el fiel, está acompañada por la humilde e intensa invocación a Dios: \textquote{¡Ven!}. Es la súplica ardiente que se convierte en la respuesta del \textbf{Salmo responsorial} de la liturgia de hoy: \textquote{¡Ven, Señor, a salvarnos!}.

2. La alegría del Adviento, típica de este domingo, encuentra su fuente en la respuesta que recibieron de Cristo \textbf{los mensajeros} a quienes envió \textbf{Juan el Bautista}. Este, mientras estaba en la cárcel, habiendo oído hablar de las obras de Jesús, le envió sus discípulos con la pregunta crucial, que esperaba una respuesta definitiva: \textquote{¿Eres tú el que ha de venir, o tenemos que esperar a otro?} (Mt 11, 3).

Y esta fue la respuesta de Cristo: \textquote{Id a anunciar a Juan lo que estáis viendo y oyendo: los ciegos ven y los cojos andan; los leprosos quedan limpios y los sordos oyen; los muertos resucitan y a los pobres se les anuncia la Buena Noticia. ¡Y bienaventurado el que no se escandalice de mí!} (Mt 11, 4-6).

Jesús de Nazaret, en su solemne respuesta a Juan el Bautista, se remite, evidentemente, al cumplimiento de las promesas mesiánicas. Promesas que se encuentran profetizadas en el libro de Isaías, que acabamos de escuchar en la \textbf{primera lectura}:

\textquote{He aquí a vuestro Dios\ldots{} Viene en persona y os salvará. Entonces se despegarán los ojos de los ciegos, los oídos de los sordos se abrirán; entonces saltará el cojo como un ciervo, la lengua del mudo cantará. Porque han brotado aguas en el desierto, torrentes en la estepa\ldots{} Lo cruzará una calzada que llamarán Vía Sacra\ldots{} Por ella volverán los rescatados del Señor. Llegarán a Sión con cantos de júbilo: alegría sin límite en sus rostros. Los dominan el gozo y la alegría. Quedan atrás la pena y la aflicción} (Is 35, 4-10).

Así pues, esto responde Cristo a Juan el Bautista: ¿Acaso no se están cumpliendo las promesas mesiánicas? Por lo tanto, ¡ha llegado el tiempo del primer Adviento!

Nosotros hemos dejado atrás ya este tiempo y, al mismo tiempo, estamos siempre en él. Efectivamente, la liturgia lo hace presente cada año. Y esta es la fuente de nuestra alegría.

3. Esta alegría del Adviento tiene una fuente propia más profunda. El hecho de que en Jesús se han cumplido las promesas mesiánicas es la demostración de que Dios es fiel a su palabra. Verdaderamente podemos repetir con el \textbf{Salmista}: \textquote{El Señor mantiene su fidelidad perpetuamente} (Sal 146, 6). El Adviento nos recuerda cada año el cumplimiento de las promesas mesiánicas que se refieren a Cristo, con el fin de orientar nuestras almas hacia estas promesas, cuya realización hemos recibido en Cristo y por Cristo. Estas promesas conducen al hombre a su destino último.

En Cristo y por Cristo \textquote{el Señor mantiene su fidelidad perpetuamente}. En él y por él se abre, de generación en generación, el segundo Adviento, que es el \textquote{tiempo de la Iglesia}. Por Cristo la Iglesia vive el Adviento de cada día, esto es, la propia fe en la fidelidad de Dios, que \textquote{mantiene su fidelidad perpetuamente}. El Adviento vuelve a confirmar de este modo en la vida de la Iglesia la dimensión escatológica de la esperanza. Por esto \textbf{Santiago} nos recomienda: \textquote{Tened paciencia, hermanos, hasta la venida del Señor} (St 5, 7).

4. En esta perspectiva, la liturgia de este tercer domingo de Adviento no es solo una invitación a la alegría, sino también a la valentía. Si, de hecho, debemos regocijarnos en la serena esperanza de la plenitud futura de los bienes mesiánicos, también debemos pasar con valentía por medio y por encima de la realidad temporal y transitoria, con la mirada y el interés dirigidos a lo que es eterno e inmutable.

Esta valentía nace de la esperanza cristiana y, en cierto sentido, es la misma esperanza cristiana. La invitación a la valentía resuena en la \textbf{profecía del libro de Isaías}: \textquote{Fortaleced las manos débiles, afianzad las rodillas vacilantes, decid a los cobardes de corazón: sed fuertes, no temáis} (Is 35, 3).

El Adviento, como dimensión estable de nuestra existencia cristiana, se manifiesta en esta esperanza, que comporta, al mismo tiempo, la valentía \textquote{escatológica} de la fe.

Esta valentía --fuerza de la fe-- es, como dice el profeta, magnanimidad. Y es, a la vez, paciencia. Es similar a la paciencia del labrador que \textquote{aguarda pacientemente el fruto valioso de la tierra mientras recibe la lluvia temprana y tardía} (St 5, 7). Y añade Santiago: \textquote{Tened paciencia también vosotros, manteneos firmes, porque la venida del Señor está cerca} (St 5, 8).

El \textbf{Evangelio} de este domingo nos presenta un ejemplo espléndido de esta paciente magnanimidad: Juan el Bautista. Jesús habla de él a la multitud en términos elogiosos: \textquote{¿Qué salisteis a contemplar en el desierto? ¿Una caña sacudida por el viento?\ldots{} ¿Un profeta? Sí, os digo, y más que un profeta} (Mt 11, 7-9). Y añade: \textquote{No ha nacido de mujer uno más grande que Juan el Bautista, aunque el más pequeño en el reino de los cielos es más grande que él} (Mt 11, 11).

La fe magnánima y la valentía paciente de la esperanza nos abren a todos el camino para el Reino de los cielos.

5. Esta fe magnánima y esta valentía paciente quiero desear hoy a todos los fieles. {[}\ldots{}{]}

6. Queridos hermanos y hermanas \ldots{} que a través de esta celebración se renueve en todos vosotros la invitación a la alegría del Adviento que resuena en la liturgia de este domingo. Que se renueve, al mismo tiempo, la invitación a la esperanza magnánima, que tiene su fuente en la valentía sobrenatural de la fe.

Cultivemos pacientemente la tierra de nuestra vida, como el labrador que \textquote{aguarda con paciencia el fruto valioso de la tierra}. ¡Este fruto se manifestará en la venida del Señor!

Amén.

\subsubsection{Homilía (1986): Dialoguemos con Cristo}

Visita a la Parroquia romana de Santa María \textquote{Regina Mundi}.

14 de diciembre de 1986.

1. \textquote{¿Eres tú el que ha de venir?} (Mt 11, 3).

En la liturgia del este tercer domingo de Adviento resuena nuevamente esta \textbf{pregunta dirigida a Jesús por Juan el Bautista, a través de sus discípulos}: \textquote{¿Eres tú quien debe venir o debemos esperar a otro?}. En ese momento, Juan ya estaba en prisión y su actividad a orillas del río Jordán había sido brutalmente interrumpida. De hecho, estaba diciendo la verdad a todos; no había dudado en decirlo incluso al rey adúltero. Por eso terminó en prisión. Desde la prisión, dirigió una pregunta a Jesús, que puede sorprendernos en boca de Juan.

En el Jordán, de hecho, ya había \textbf{dado testimonio de Cristo} al proclamar: \textquote{Este es el Cordero de Dios, el que quita el pecado del mundo} (Jn 1, 29). ¿Por qué, entonces, tal pregunta de quien reconoció la plenitud de los tiempos? No porque hubiera ninguna duda en él sobre el Redentor, a quien indicó como aquel que traía el perdón de Dios que tanto tiempo había esperado e invocado, sino porque había una cierta sorpresa en el Bautista. De hecho, él, que estaba experimentando la condición del prisionero destinado a morir, en cierto sentido está desconcertado de que Jesús traiga el juicio de Dios de una manera tan humilde e indefensa: comprendiendo con el delicado poder del amor lo que el Bautista había predicho con palabras fuertes: el Mesías \textquote{tiene el bieldo en la mano: aventará su parva, reunirá su trigo en el granero y quemará la paja en una hoguera que no se apaga} (Mt 3, 12).

2. Jesús primero da la \textbf{respuesta}; luego, dirigiéndose a las personas presentes, manifiesta la grandeza de un hombre que lo anhelaba con todas las energías de la mente y el corazón.

Para los discípulos de Juan, Jesús se refiere a algunas palabras del profeta Isaías sobre el Mesías, afirmando que esa descripción del tiempo mesiánico, como una era de salud, paz y alegría, se había cumplido en su obra salvadora.

Estos textos de Isaías (Is 26, 19; 29, 18-19; 35, 5ss) son bien conocidos y a menudo se repiten en la liturgia. En ellos, este gran profeta predijo lo que, con la venida de Cristo, ahora está sucediendo ante los ojos de todos. La Palabra se hizo carne, el anuncio se hizo realidad. Aquí: \textquote{Los ciegos recuperan la vista, los lisiados caminan, los leprosos sanan, los sordos recuperan su audición, los muertos resucitan, la Buena Nueva se predica a los pobres}. Por lo tanto, Jesús afirma que todas estas \textquote{señales mesiánicas} son confirmadas por su actividad, y concluye diciendo: \textquote{Ve y dile a Juan lo que oyes y ves} (Mt 11, 5. 4).

3. En la \textbf{primera parte del pasaje del Evangelio de hoy}, Jesús responde a la pregunta del Bautista, refiriéndose al testimonio de la Escritura: lo que el Antiguo Testamento anunció del Mesías se realiza en él, en Jesús de Nazaret. En la \textbf{segunda parte} de la perícopa, el Salvador de testimonio de Juan. Este es un \textquote{profeta} \ldots{} incluso \textquote{más que un profeta} (Mt 11, 9).

Al afirmar esto, Jesús apela a la memoria de aquellos que lo escuchaban en ese momento: cuando fueron al Jordán para escuchar a Juan el Bautista no encontraron allí ni un bastón golpeado por el viento, ni un hombre envuelto en ropas suntuosas, como ¡los que están en los palacios de los reyes! (cf. Mt 11, 7-8). Allí encontraron a un profeta: un hombre que decía la verdad, y solo la verdad, en el nombre de Dios mismo. Esta verdad fue a veces dura y exigente. En el nombre del Reino, que se acercaba, en nombre del juicio de Dios, Juan exhortó a la penitencia a cuantos acudieron a él para escucharlo, para que al reconocerse a sí mismos como pecadores pudieran recibir al Mesías con un alma contrita.

Quizás, como ya he mencionado, el propio Juan estaba un poco sorprendido de que las palabras y acciones de Jesús no fueran tan severas como las suyas. Pero, en primer lugar, Cristo es una buena noticia de salvación y revelación del amor misericordioso de Dios. Sin embargo, Juan no solo fue un profeta, servidor de la Palabra de Dios para ser proclamada a los hombres como un anuncio de paz, sino que también fue el Precursor, el enviado a \textquote{preparar el camino} a Cristo e indicarlo como una verdad gozosa ofrecida al hombre. Por esta razón, Jesús exclama: \textquote{En verdad os digo que entre los nacidos de mujer no ha surgido uno mayor que Juan el Bautista} (Mt 11, 11).

4. A estas palabras, Jesús agrega otras que hacen reflexionar: \textquote{Sin embargo, el más pequeño en el Reino de los cielos es más grande que él}. Porque el Padre se complace plenamente en todos los que nacen de nuevo en el Espíritu, y los hijos de adopción son elevados a una familiaridad con Dios comparable a la que Juan pudo disfrutar en la vida terrenal.

De hecho, debido a esta última frase, podemos afirmar que esta \textbf{segunda intervención} de Cristo no se refiere solo a Juan el Bautista, sino, en cierto sentido, a todo hombre. De hecho, desde que el reino de Dios vino al mundo, todos estamos sujetos a la medida de gracia que éste constituye para el hombre. E incluso un profeta tan grande como Juan, que sabía cómo reconocer al Mesías a orillas del Jordán, es valorado de acuerdo con esta medida en el reino de Dios.

5. Hoy, tercer domingo de Adviento, la Iglesia recuerda estas palabras cuando en toda la liturgia resuena la verdad exultante de la \textbf{cercanía de Dios}: \textquote{¡El Señor está cerca!} (Flp 4, 5; Jn 5, 8).

Realmente el Redentor está cerca. Él ha puesto su morada entre los hombres y puede ser encontrado en el camino de la experiencia humana, no solo nos enseña, sino que conversa con nosotros de una manera fraterna y santa: nos invita a un diálogo de salvación.

El hombre de nuestro tiempo es muy sensible al \textquote{diálogo}, una palabra que se ha utilizado con frecuencia, pero de la que a veces se abusa en nuestros días. Sin embargo, la Iglesia ha insertado esta palabra en su lenguaje contemporáneo. Mi predecesor, el Papa Pablo VI, ha dedicado gran parte de su primera encíclica a este tema. Y, en particular, explicó en qué consiste el \textquote{diálogo de salvación}. Después de recordar que esta conversación salvífica se abrió espontáneamente por la iniciativa amorosa y benéfica de Dios, él enseña que nuestro hablar con el hombre también debe ser movido por un amor ferviente, libre y sin exclusiones.

Él dice: \textquote{El diálogo de la salvación no obligó físicamente a nadie a acogerlo; fue un formidable requerimiento de amor, el cual si bien constituía una tremenda responsabilidad en aquellos a quienes se dirigió (Mt 11, 21), les dejó, sin embargo, libres para acogerlo o rechazarlo, adaptando inclusive la medida (Mt 12, 38ss.) y la fuerza probativa de los milagros (Mt 13, 13ss.) a las exigencias y disposiciones espirituales de sus oyentes, para que les fuese fácil un asentimiento libre a la divina revelación sin perder, por otro lado, el mérito de tal asentimiento} (Ecclesiam Suam, III, 36).

6. El hombre también se acerca a Dios a través del diálogo, a través del diálogo con los hombres, cuando busca la verdad y la justicia en este camino. Se podría decir que el diálogo es parte del espíritu del Adviento, de la actitud del Adviento. El \textbf{Evangelio de hoy} lo pone de relieve con mucha fuerza.

El hombre puede, de hecho debe, hacer \textbf{preguntas a Cristo}, incluso en la etapa actual de la historia: \textquote{¿Eres tú el que ha de venir?}. La respuesta de Cristo, preservada y ofrecida por la Iglesia, por un lado, será similar a la que recibieron los discípulos de Juan. Por otro lado, se adaptará a los problemas de la era en que vivimos. Precisamente de esta manera, el Concilio Vaticano II dio esta respuesta a la gente hoy.

Cristo es aquel en quien los \textquote{signos mesiánicos} continúan encontrando confirmación. El que indica los caminos de la salvación. El diálogo de salvación, por lo tanto, puede y debe tenerse. A través de él, nuestra fe se profundiza y, al mismo tiempo, se convierte en una exhortación a actuar con madurez y responsabilidad.

7. Sin embargo, recordemos que el centro de este diálogo, si es un verdadero diálogo de salvación, se encuentra en la oración. Incluso en la oración y, en primer lugar, en ella, el hombre \textquote{plantea preguntas}. Y en la oración, especialmente en la oración, encuentra la respuesta. La oración, más que cualquier otra realidad, pertenece al espíritu del Adviento, forma de la manera más completa la actitud del Adviento: \textquote{el Señor está cerca}.

8. ¡Queridos hermanos y hermanas! Sed cada vez más una comunidad de oración, que especialmente con la liturgia se injerta en la comunión divina, manteniendo el deseo de Dios despierto y constante. De esta manera, madurará en vosotros una comunidad de vida redimida, que se sentirá empujada a practicar la caridad, que Cristo mismo nos recomendó (cf. Jn 13, 35), promoviendo al hombre según el plan de Dios.

La existencia de una parroquia se desarrolla plenamente cuando avanza en la fe y progresa en la santidad de las buenas obras. No dudéis en poneros en el camino que indica la misericordia de Cristo, imitando su ejemplo, preservando su enseñanza y procediendo con seguridad en su paz. La parroquia es una casa de hermanos, que se hace bella y acogedora por la caridad.

[\ldots{}]

10. \textquote{¡El Señor está cerca!} He aquí que en la pequeñez de la casa de Nazaret, la humilde Virgen, llamada María, escucha las palabras de la Anunciación: \textquote{concebirás un hijo, lo darás a luz y lo llamarás Jesús. Será grande y será llamado Hijo del Altísimo, el Señor Dios le dará el trono de David su padre y reinará para siempre sobre la casa de Jacob y su reinado no tendrá fin} (Lc 1, 31-33).

Estas palabras se han hecho realidad. La Virgen Madre entró en la historia del Adviento como la Sierva del Señor que dice: \textquote{Hágase en mi según tu palabra} (Lc 1, 38).

{[}\ldots{}{]} La Iglesia, y vuestra parroquia en particular, llama a esta humilde Sierva del Señor \textquote{Reina del mundo}. Y la invoca en todas las necesidades de los hombres de hoy. María, nuestra Madre y Reina del mundo, ayuda y protege a todos los hijos de esta parroquia.

\subsubsection{Homilía (1989): ¿Cómo reconocer al Mesías?}

Visita a la Parroquia de San León I.

Domingo 17 de diciembre del 1989.

1. \textquote{¿Eres tú el que vendrá?} (Mt 11, 3).

La \textbf{pregunta de los discípulos de Juan el Bautista} recorre todo el pasaje del Evangelio de este tercer domingo de Adviento, acertadamente llamado \textquote{de la alegría} (\emph{Gaudete}).

El precursor, que terminó en prisión por dar un valiente testimonio de la verdad, había recibido el eco de las palabras y gestos de Jesús: eran palabras y gestos que contradecían las expectativas de un Mesías político y violento, como era el esperado por muchos en Israel. Esto les había hecho dudar; algunos incluso estaban escandalizados.

\textquote{Eres tú\ldots{}. ¿O debemos esperar a otro?} La pregunta de los discípulos de Juan conserva toda su actualidad. Esta pregunta se la hace también hoy el nuevo Israel, la Iglesia, que espera la venida, especialmente la última, de su Señor. Se la hacen también de modo particular muchos hombres desanimados y desconcertados, que con un corazón sincero, buscan el camino de la salvación.

2. La respuesta a esta pregunta no se puede dar en base a una simple lógica humana. La identificación del Mesías-salvador no es fácil. Solo pueden reconocerlo aquellos que tienen oídos para escuchar las palabras de Cristo y ojos para ver sus obras a la luz de la fe y como cumplimiento del plan salvador de Dios, ya anunciado por los profetas.

\textbf{Isaías} había ofrecido un resumen particularmente elocuente de este proyecto de salvación. Jesús se refiere a la profecía de Isaías para revelar su verdadera identidad mesiánica y su misión: \textquote{Los ciegos recuperan la vista, los lisiados caminan, los leprosos sanan, los sordos oyen, los muertos resucitan, la Buena Noticia es anunciada a los pobres}.

Con sus palabras, por lo tanto, Jesús ofrece las \textquote{señales} de la venida del Reino prometido y presenta las credenciales de su misión. Con su predicación, lleva a cumplimiento la liberación ya anunciada. De hecho, él es el Mesías-Siervo, salvador de todo hombre y de todos los hombres, de los pobres, de los que sufren, de los marginados sobre todo. A través de sus palabras y sus gestos, el Reino de Dios \textquote{viene} para la salvación y la alegría de los pobres que reconocen en él \textquote{la gloria y la magnificencia} de Dios y son así salvados. Para ellos, la liberación será un \textquote{nuevo éxodo}, un \textquote{camino sagrado} que se recorrerá a partir de ahora en una actitud de conversión y fidelidad a su palabra, como Buena Noticia de esperanza y de alegría.

3. La \textbf{respuesta dada por Jesús} es un motivo de confianza y estímulo para la vida de toda persona de buena voluntad. Y lo es, en particular, para esta comunidad parroquial\ldots{} Aquí también hay enfermos, marginados, pobres, a quienes Jesús quiere traer la Buena Noticia de la salvación. Lo hace a través de sus ministros: su párroco\ldots{} y sus sacerdotes colaboradores, a quienes extiendo mi cordial saludo. Lo hace a través del {[}Obispo{]} \ldots{} Lo hace a través de la persona del Papa, que ha venido entre vosotros para expresar su afecto y su preocupación pastoral por los problemas que afligen a su comunidad.

Animo a todos {[}los fieles{]} a asumir su propia parte de responsabilidad y a perseverar en esta colaboración, sabiendo que Dios os recompensará. Insto a dar prioridad, a la luz del pasaje del \textbf{Evangelio} escuchado hace un momento, a la atención a los últimos. Es a ellos en primer lugar, a los \textquote{pobres}, a quienes el Señor Jesús quiere traer las Buena Nueva a través de vosotros. Él estará a vuestro lado, en el cumplimiento de esta tarea, apoyándoos con su gracia\ldots{}

4. La situación de la ciudad en la que vivimos\ldots{}, a veces se presenta, para usar las palabras del \textbf{profeta Isaías}, como un \textquote{desierto}, una \textquote{tierra árida}, difícil de cultivar y resistente a la siembra del Evangelio. Hay {[}en nuestras ciudades{]} \textquote{cansados de corazón}, que han perdido el camino de la verdad y de la vida; muchas \textquote{manos débiles}, incapaces de hacer el bien; muchas \textquote{rodillas vacilantes} en el camino de seguir a Cristo. Con las palabras del profeta, Dios nos invita a no desanimarnos y nos exhorta a esperar: \textquote{Ánimo, no tengas miedo; aquí está tu Dios \ldots{} ¡Viene a salvarte!}.

Sí, hermanos y hermanas, viene el Señor; de hecho está aquí entre nosotros. Los signos de su presencia salvadora ya son visibles en nuestra ciudad. De hecho, hay muchas iniciativas puestas en marcha en la comunidad eclesial para anunciar la Buena Nueva del Evangelio a los pobres. Hay muchas obras de caridad y servicio creadas para ofrecer a los enfermos, los que sufren y los marginados una liberación integral.

Sin embargo, la salvación que Jesús viene a traer es un don que todavía no ha llegado a todos; muchos \textquote{pobres} que tenemos con nosotros aún no han aceptado el anuncio de las Buena Nueva y aún no han sido liberados del pecado ni de todo aquello que los humilla y los excluye de una coexistencia humana fraternal y solidaria; muchos \textquote{escandalizados} se han distanciado de Cristo y de la Iglesia.

Por lo tanto, es necesario aumentar la misión de evangelización y promoción humana, para abrir las puertas del Reino de Dios a todos, para que entre Jesucristo\ldots{}

6. {[}\ldots{}{]} Los creyentes saben que la salvación ofrecida por Cristo no termina en una dimensión exclusivamente terrenal y temporal, es trascendente, escatológica y tendrá su cumplimiento definitivo en el segundo advenimiento del Señor. Ciertamente tiene su comienzo aquí y ahora, pero solo al final tendrá su realización completa.

Por esta razón, la Iglesia \ldots{} participa activamente en la promoción del hombre. Lo hace con esa actitud de esperanza y paciencia activa, como nos pedía \textbf{Santiago en la segunda lectura} de la Misa, confiando en que el desierto florecerá y la tierra dará sus frutos.

Lo hace tratando de acompañar sus obras con el testimonio valiente y coherente de la vida. Como \textbf{Juan el Bautista}.

Lo hace dirigiendo los esfuerzos de todos aquellos en la verdadera dirección, y hay muchos que, como individuos o en grupos, trabajan por la justicia, la solidaridad y la paz. Para que todos los hombres, pero especialmente los pobres y los oprimidos, vean la gloria del Señor, la magnificencia de nuestro Dios y se salven.

\textquote{¡Ánimo, no temas, ---os repite hoy la Iglesia con el \textbf{profeta Isaías}--- aquí está tu Dios \ldots{} Él viene a salvarte}.

Sí, ¡ven, Señor Jesús, ven y sálvanos!

\subsubsection{Homilía (1992): liberación definitiva y verdadera}

Visita a la Parroquia de San Hugo, Obispo.

Domingo 13 de diciembre del 1992.

¡Queridos hermanos y hermanas de la parroquia de San Hugo!

1. \textquote{Fortaleced vuestros corazones} (St 5, 8). Con el tercer domingo de Adviento, que estamos celebrando, hemos llegado al \textquote{corazón} de ese itinerario espiritual que nos llevará al pie de la Santa Gruta, para contemplar, adorar y dar gracias al Verbo de Dios, que se ha hecho hombre para la salvación de toda la humanidad. Y la liturgia de hoy, como para apoyarnos en el exigente viaje de preparación y conversión, está impregnada de una invitación a confiar y esperar. La expectativa del creyente, de hecho, no es en vano y la promesa de Dios es verdadera.

El \textbf{apóstol Santiago} nos recordó esto en la segunda lectura: \textquote{Fortaleced vuestros corazones, porque la venida del Señor está cerca} (St 5, 8). Sus palabras se hacen eco de las del \textbf{profeta Isaías}, dirigidas al pueblo judío durante el duro exilio en la tierra de Babilonia. \textquote{Sed fuertes, no temáis. ¡He aquí vuestro Dios!} (Is 35, 4). Al igual que en tiempos de Moisés, Dios intervino para liberar a su pueblo de la esclavitud egipcia y, a través del desierto, llevarlos a la tierra prometida, así también ahora está dispuesto a hacer maravillas en favor de su pueblo, restaurando su libertad. \textquote{Él viene a salvarnos} (Is 36, 4).

\textbf{Isaías} describe un camino llano, por medio del cual los deportados regresarán exultantes: verán la gloria y la magnificencia del Señor. Los desanimados y desconcertados no deben desesperarse porque, como dice el profeta, \textquote{tu Dios \ldots{} viene a salvarte}; y agrega: \textquote{Entonces se abrirán los ojos de los ciegos y se abrirán los oídos de los sordos. Entonces el cojo saltará como un ciervo, la lengua del mudo gritará de alegría} (Is 35, 5-6). En esta página, tan rica en simbolismo, la Iglesia ve una clara profecía mesiánica, que supera y perfecciona la inmediatamente histórica. De hecho, para el hombre, la verdadera y definitiva liberación de toda esclavitud y opresión es solo la lograda por Jesús, en el misterio pascual de su muerte y resurrección.

2. \textquote{¿Eres tú el que ha de venir?} (Mt 11, 3), los discípulos de Juan el Bautista, encarcelados por el rey perseguidor Herodes Antipas, le \textbf{preguntan al Mesías}. \textquote{¿Eres tú quien tiene que venir, o tenemos que esperar a otro?} Una vez más, el Precursor abre el camino al Señor y le ofrece otra oportunidad para manifestarse a los hombres. \textbf{Jesús responde} con las mismas palabras de Isaías: \textquote{Ve e informa a Juan lo que oís y veis: los ciegos recuperan la vista, los lisiados caminan, los leprosos se curan, los sordos oyen, los muertos resucitan y a los pobres es anunciada la Buena Noticia} (Mt 11, 5).

Días antes, en la sinagoga de Nazaret, Jesús se había aplicado otro pasaje del profeta Isaías: \textquote{El Espíritu del Señor está sobre mí; por eso me ha consagrado con la unción y me ha enviado a anunciar el Evangelio a los pobres, a proclamar la liberación a los prisioneros y la vista a los ciegos; a liberar a los oprimidos} (Lc 4,18). El Redentor hace referencia a la autoridad del gran profeta del Antiguo Testamento para demostrar su mesianismo. Y esta vez lo hace para eliminar cualquier duda tanto de los discípulos de Juan como de las multitudes que ahora lo siguen asiduamente considerándolo el Maestro. Para luego \textbf{dar testimonio del Precursor}, que ha terminado su predicación, pero aún no su misión, expresa un elogio sin paralelo hacia él. Lo define \textquote{más que un profeta}, lo indica como el mensajero enviado para preparar el camino al Mesías, lo compara con Elías y resume su elogio con esta afirmación solemne: \textquote{En verdad te digo: entre los nacidos de mujer no hay uno mayor que Juan el Bautista} (Mt 11, 11).

3. \textquote{El más pequeño en el reino de los cielos es más grande que él} (Mt 11, 11). Sin embargo, la alabanza extraordinaria es seguida por una \textbf{nota aparentemente misteriosa}: \textquote{El más pequeño en el reino de los cielos es más grande que él}. Puede parecer contradictorio y, en cambio, expresa una verdad fundamental. De hecho, el Señor tenía la intención de contrastar el tiempo de preparación de la salvación, concluido y casi simbolizado por el Bautista, con el de su realización definitiva, inaugurada por el mismo Cristo. Con el advenimiento del Redentor, la espera termina y la salvación destinada para cada hombre se inaugura, sin restricciones. Esto explica la paradoja de las palabras de Jesús sobre Juan el Bautista. Los prodigiosos signos de curación hechos por Cristo en los enfermos adquieren así un valor simbólico precioso, es decir, Cristo trae consigo el auténtico don de sanar las almas y de otorgarnos una nueva vida. Las curaciones de Jesús son por tanto signos de salvación eterna.

4. Queridos hermanos y hermanas, aquí está el profundo significado de la Navidad del Señor para la cual nos estamos preparando. Jesús aparece ante el mundo como un niño pequeño; a través de la pobreza, la simplicidad y la humildad de su nacimiento, quiere llevarnos a todos a comprender su arcano plan salvífico. Después de dar el ejemplo, Jesús predicó los caminos del reino divino; después de entregarse a sí mismo para redimir a la humanidad, Él, resucitado, funda la Iglesia al confiarle la verdad eterna que se transmitirá y la gracia renovadora que se difundirá gratuitamente. Desde entonces, el pueblo de Dios, rico en carismas y ministerios puestos al servicio de la única fe y el único Señor, se extiende sobre la tierra en múltiples comunidades particulares, diócesis y parroquias precisamente para proclamar y presenciar este mensaje de salvación del cual es depositario.

El pueblo de Dios es muy consciente de que debe continuar la obra redentora del Salvador entre los hombres, proclamando su Evangelio a toda criatura. Es en la parroquia donde se genera la nueva existencia cristiana a través de la gracia bautismal; en ella participamos en la existencia divina a través de los sacramentos; en ella crecemos en la fe gracias a una catequesis permanente, en ella se cultivan las vocaciones al orden sagrado, el matrimonio y a la vida consagrada. Es en la parroquia donde florece la caridad para todos. De hecho, la comunidad parroquial está llamada a ser una \textquote{escuela de caridad} privilegiada, donde se aprende a acoger y amar a cada persona sin discriminación, distinción o preferencias, ofreciendo el don de las obras de misericordia a los más necesitados. He aquí un resumen del papel de la parroquia en la comunidad cristiana, comenzando con los primitivos cristianos, alrededor de los Apóstoles, hasta hoy con esta parroquia de san Hugo, recientemente inaugurada.

{[}\ldots{}{]}

6. \textquote{La venida del Señor está cerca}.

Sí, queridos hermanos y hermanas, la venida del Señor está cerca porque la Navidad, el nacimiento de Jesús en el vientre virginal de María, está a las puertas; pero también está cerca porque la vida, incluso la más larga, está destinada a terminar en el tiempo para abrirse a la eternidad.

El tiempo es corto (cf. 1 Cor 7, 29). He aquí, ahora el momento favorable, ahora el día de salvación (cf. 2 Cor 6, 2).

Aprovechemos el tiempo, como un tesoro, por el bien de nuestras almas.

Jesús viene ¡Ven, Señor, a salvarnos!

Amén.

\subsubsection{Homilía (1995): fidelidad perpetua}

Visita a la Parroquia de Santa María, Reina de los Apóstoles.

Domingo 17 de diciembre de 1995.

Rorate caeli desuper!

Queridos hermanos y hermanas!

1. Las lecturas, que hemos escuchado en la liturgia de hoy, ilustran cómo la realidad del Adviento ya está inscrita en la misma experiencia de la naturaleza. El Adviento, de hecho, es el tiempo de espera. Santiago habla del agricultor que \textquote{espera pacientemente el precioso fruto de la tierra hasta que recibe las lluvias de otoño y las de primavera} (St 5, 7). Estas palabras se pueden relacionar de alguna manera con las del \textbf{profeta Isaías}, proclamadas en la \textbf{primera lectura}: \textquote{El desierto y el yermo se regocijarán, se alegrará la estepa y florecerá, germinará y florecerá como flor de narciso, festejará con gozo y cantos de júbilo} (Is 35, 1-2). Para los israelitas, que vivían al borde del desierto, la expectativa de la cosecha era motivo de especial preocupación. Después de todo, ¿no es este el contenido de la invocación de Adviento: \textquote{Rorate caeli desuper!}? La expectativa del Mesías es, por lo tanto, similar a la del agricultor: \textquote{Et nubes pluant Iustum}, \textquote{Cielos, destilad desde lo alto la justicia, las nubes la derramen, se abra la tierra y brote la salvación, y con ella germine la justicia}. (cf. Is 45, 8)

2. En este contexto de expectativa ansiosa, la liturgia de hoy nos lleva a afirmar una vez más que el hombre y Dios están en el centro del Adviento: el hombre que espera la venida de Dios y Dios que viene al encuentro del hombre. El contenido de la expectativa del hombre es la salvación que solo le puede llegar de Dios. El Mesías prometido, que viene a la tierra en la noche de Belén, es el Salvador del mundo, es el que libera al hombre del mal, orientándolo hacia el bien y la felicidad.

El \textbf{Salmo Responsorial} canta la fidelidad de Dios, que es perpetua: Él hace justicia a los oprimidos, da pan a los hambrientos de pan, libera a los prisioneros, devuelve la vista a los ciegos, levanta a los que han caído, ama a los justos, protege a los extranjeros, vela por los huérfanos y las viudas (cf. Sal 145,7-10).

3. Las palabras del salmista están relacionadas con lo que expresó el \textbf{profeta Isaías en la primera lectura}: \textquote{Entonces se abrirán los ojos de los ciegos y se abrirán los oídos de los sordos. Entonces el cojo saltará como un ciervo, la lengua del mudo gritará de alegría} (Is 35, 5-6). Son signos de una gran conversión, de un retorno, que se logrará con la obra del Redentor. El Profeta anuncia: \textquote{Los redimidos del Señor volverán y vendrán a Sión con alegría; la felicidad perenne brillará en su cabeza; la alegría y la felicidad los seguirán y la tristeza y las lágrimas huirán} (Is 35, 10).

Y cuando los \textbf{discípulos de Juan el Bautista fueron a Cristo para preguntarle}: \textquote{¿Eres tú quien debe venir o debemos esperar a otro?}, Jesús responde: \textquote{Id y decidle a Juan lo que veis y oís: los ciegos recuperan la vista, los lisiados caminan, los leprosos son sanados, los sordos recuperan su audición, los muertos resucitan, la Buena Nueva se predica a los pobres y bienaventurado el que no escandaliza de mi} (Mt 11, 3-6). Por lo tanto, Jesús de Nazaret confirma inequívocamente que él mismo es el cumplimiento de las expectativas mesiánicas de Israel. De esta manera, actúa como mediador entre las expectativas del hombre y la voluntad eterna de Dios de corresponder plenamente a las necesidades de la humanidad.

4. Refiriéndose al mensaje de Juan el Bautista y la consiguiente respuesta, Jesús habla a la multitud de la persona del Bautista. El bautista no es un hombre que duda. La pregunta que hizo surge de la profundidad de su vocación profética y tiende a obtener de Cristo mismo la confirmación de esa verdad divina de la que había dado testimonio a orillas del Jordán: la verdad confirmada definitivamente con el sacrificio de su propia vida.

Y \textbf{Jesús da testimonio de la misión especial del Bautista}, como si quisiera pagar una \textquote{deuda de gratitud} hacia el Precursor. \textquote{En verdad os digo que no ha nacido de mujer uno más grande que Juan el Bautista} (Mt 11, 11). La multitud se encontró no solo con un profeta, sino \textquote{más que un profeta} (cf. Mt 11, 9). Con estas palabras, Cristo da testimonio de Juan y, en cierto sentido, imprime un sello mesiánico en toda su misión profética.

5. La figura de Juan el Bautista se repite varias veces en las lecturas del Adviento y confiere un significado especial a la liturgia de este período. Sí, el Adviento es tiempo de espera de la Navidad del Señor, de su entrada en la existencia terrena en un clima de alegría y de paz. Juan el Bautista, en un cierto sentido, hace revivir, con treinta años de diferencia, la experiencia del Adviento, en el momento en el que Jesús de Nazaret empieza su vida pública. Precisamente, la realización concreta de su misión salvífica es la que manifiesta el sentido definitivo de la Noche de Navidad.

El Mesías cumplirá la \textbf{profecía de Isaías} con su misión y continuará repitiendo todo lo que dijo a los enviados de Juan: \textquote{Bienaventurado el que no se escandalice de mi} (Mt 11,6). Hoy, Cristo repite lo mismo a los hombres del siglo XX que ahora está llegando a su fin; a nosotros, reunidos en este templo; a la Iglesia y a toda la humanidad. A medida que avanzamos hacia el final del segundo milenio cristiano, estas palabras continúan resonando con particular claridad y reviviendo los corazones de los hombres en este punto de inflexión de nuestra época.

6. ¡Queridos hermanos y hermanas de la parroquia de Santa Maria Regina Apostolorum! Hoy el profeta Isaías nos dirige una invitación apremiante: \textquote{¡Ánimo! no tengas miedo; he aquí tu Dios \ldots{} Él viene a salvarte} (Is 35, 4). Os repito estas palabras, ahora unos días antes de la gran solemnidad de Navidad, a vosotros que vivís en esta parroquia \ldots{}

{[}\ldots{}{]}

8. Queridos hermanos y hermanas \ldots{} Se puede decir que cada templo es un signo del Adviento, de ese Adviento que está inscrito en toda la creación. En la casa del Señor, ubicada entre los hombres, este encuentro se lleva a cabo de manera sacramental.

¡Espero que este templo sea para vosotros un lugar de encuentros frecuentes con Dios! Venid aquí, dentro de estos muros, para hacerle, como los discípulos de Juan, preguntas a Cristo. Saldréis de aquí tranquilizados, llevando con vosotros la respuesta dada por Jesús a los enviados de Juan: \textquote{Bienaventurado el que no se escandaliza de mi}.

9. ¡Espero que este templo esté al servicio de vuestra fe, vuestra esperanza y vuestra caridad! Preparaos, aquí en la tierra, para el encuentro con Dios, el destino definitivo de cada hombre. \textbf{Santiago} nos exhortaba así: \textquote{Tened paciencia, hermanos \ldots{} fortaleced vuestros corazones, porque la venida del Señor está cerca \ldots{} he aquí, el juez está a la puerta} (St 5, 7-10). Este juez es el Salvador del mundo y su juicio es un juicio salvador.

Que este templo os ayude a poneros en contacto con el Dios que juzga mediante la verdad de la salvación. De hecho, Dios \textquote{quiere que todos los hombres se salven y lleguen al conocimiento de la verdad} (1 Tim 2, 4). ¡Que este deseo de Dios esté presente también en este templo como una señal de bendición divina para todos los que entren en él!

Amén.



\subsubsection{Homilía (1998): El motivo de nuestra alegría}

Visita pastoral a la Parroquia romana de Santa Julia Billiart.

Domingo 13 de diciembre de 1998.

1. \textquote{Alegraos siempre en el Señor; os lo repito: alegraos. El Señor está cerca} (\emph{Antífona de entrada}).

De esta apremiante invitación a la alegría, que caracteriza la liturgia de hoy, recibe su nombre el tercer domingo de Adviento, llamado tradicionalmente \emph{domingo \textquote{Gaudete}}. En efecto, ésta es la primera palabra en latín de la misa de hoy: \textquote{\emph{Gaudete}}, es decir, alegraos porque el Señor está cerca.

El \textbf{texto evangélico} nos ayuda a comprender el motivo de nuestra alegría, subrayando el gran misterio de salvación que se realiza en Navidad. El evangelista san Mateo nos habla de Jesús, \textquote{el que ha de venir} (\emph{Mt} 11, 3), que se manifiesta como el Mesías esperado mediante su obra salvífica: \textquote{Los ciegos ven y los cojos andan, (\ldots{}) y se anuncia a los pobres la buena nueva} (\emph{Mt} 11, 5). Viene a consolar, a devolver la serenidad y la esperanza a los que sufren, a los que están cansados y desmoralizados en su vida.

También en nuestros días son numerosos los que están envueltos en las tinieblas de la ignorancia y no han recibido la luz de la fe; son numerosos los cojos, que tienen dificultades para avanzar por los caminos del bien; son numerosos los que se sienten defraudados y desalentados; son numerosos los que están afectados por la lepra del mal y del pecado y esperan la salvación. A todos ellos se dirige la \textquote{buena nueva} del Evangelio, encomendada a la comunidad cristiana. La Iglesia, en el umbral del tercer milenio, proclama con vigor que Cristo es el verdadero liberador del hombre, el que lleva de nuevo a toda la humanidad al abrazo paterno y misericordioso de Dios.

2. \textquote{Sed fuertes, no temáis. Vuestro Dios va a venir a salvaros} (\emph{Is} 35, 4).

[\ldots{}] Con gran afecto, hago mías las palabras del \textbf{profeta Isaías} que acabamos de proclamar: \textquote{Sed fuertes, no temáis. (\ldots{}) El Señor va a venir a salvaros}. Estas palabras expresan mi mejor deseo, que renuevo a todos aquellos con quienes Dios me permite encontrarme en cualquier parte del mundo. Resumen lo que quiero repetiros también a vosotros esta mañana. Mi presencia desea ser una invitación a tener valor, a perseverar dando razón de la esperanza que la fe suscita en cada uno de vosotros.

\textquote{Sed fuertes}. No temáis las dificultades que se han de afrontar en el anuncio del Evangelio. Sostenidos por la gracia del Señor, no os canséis de ser apóstoles de Cristo en nuestra ciudad que, aunque se ciernen sobre ella los numerosos peligros de la secularización típicos de las metrópolis, mantiene firmes sus raíces cristianas, de las que puede recibir la savia espiritual necesaria para responder a los desafíos de nuestro tiempo. Los frutos positivos que la misión ciudadana está produciendo, y por los que damos gracias al Señor, son estímulos para proseguir sin vacilación la obra de la nueva evangelización.

4. \textquote{El Espíritu del Señor \ldots{} me ha enviado para anunciar la Buena Nueva a los pobres}.

Estas palabras del \emph{\textbf{Aleluya}} reflejan bien el clima de la misión ciudadana, que ya ha entrado en su última fase, en la que todos los cristianos son impulsados a llevar el Evangelio a los diversos ambientes de la ciudad\ldots{} Como recuerda la Escritura: Un hermano ayudado por su hermano es como una plaza fuerte (cf. \emph{Pr} 18, 19)» (n. 6).

[\ldots{}] deseo de corazón que todos los cristianos sientan la urgencia de transmitir a los demás, especialmente a los jóvenes, los valores evangélicos que favorecen la instauración de la \textquote{civilización del amor}.

5. \textquote{Tened paciencia (\ldots{}) hasta la venida del Señor} (\emph{St} 5, 7). Al mensaje de alegría, típico de este domingo \textquote{Gaudete}, la liturgia une la \textbf{invitación a la paciencia} y a la espera vigilante, con vistas a la venida del Salvador, ya próxima.

Desde esta perspectiva, es preciso saber aceptar y afrontar con alegría las dificultades y las adversidades, esperando con paciencia al Salvador que viene. Es elocuente el ejemplo del labrador que nos propone \textbf{la carta del apóstol Santiago}: \textquote{aguarda paciente el fruto valioso de la tierra, mientras recibe la lluvia temprana y tardía}. \textquote{Tened paciencia también vosotros ---añade---; manteneos firmes, porque la venida del Señor está cerca} (\emph{St} 5, 7-8).

Abramos nuestro espíritu a esa invitación; avancemos con alegría hacia el misterio de la Navidad. María, que esperó en silencio y orando el nacimiento del Redentor, nos ayude a hacer que nuestro corazón sea una morada para acogerlo dignamente. Amén.



\subsubsection{Homilía (2001): Alegría de la comunión}

Visita pastoral a la Parroquia romana de Santa María Josefa del Corazón de Jesús.

Domingo 16 de diciembre del 2001.

1. \textquote{El desierto y el yermo se regocijarán, se alegrarán el páramo y la estepa} (\emph{Is} 35, 1).

Una insistente invitación a la alegría caracteriza la liturgia de este tercer domingo de Adviento, llamado domingo \textquote{\emph{Gaudete}}, porque precisamente \textquote{\emph{Gaudete}} es la primera palabra de la antífona de entrada. \textquote{Regocijaos}, \textquote{alegraos}. Además de la vigilancia, la oración y la caridad, el Adviento nos invita a la alegría y al gozo, porque ya es inminente el encuentro con el Salvador.

En la \textbf{primera lectura}, que acabamos de escuchar, encontramos un verdadero himno a la alegría. El profeta Isaías anuncia las maravillas que el Señor realizará en favor de su pueblo, liberándolo de la esclavitud y conduciéndolo de nuevo a su patria. Con su venida, se realizará un éxodo nuevo y más importante, que hará revivir plenamente la alegría de la comunión con Dios.

Para los que están desanimados y han perdido la esperanza resuena la \textquote{buena nueva} de la salvación: \textquote{Gozo y alegría seguirán a los rescatados del Señor. Pena y aflicción se alejarán} (cf. \emph{Is} 35, 10).

2. \textquote{Sed fuertes, no temáis. Mirad a vuestro Dios. (\ldots{}) Viene a salvaros} (\emph{Is} 35, 4). ¡Cuánta confianza infunde esta profecía mesiánica, que permite vislumbrar la verdadera y definitiva liberación, realizada por Jesucristo. En efecto, en la \textbf{página evangélica} que ha sido proclamada en nuestra asamblea, Jesús, respondiendo a \textbf{la pregunta} de los discípulos de Juan Bautista, se aplica a sí mismo lo que había afirmado Isaías: él es el Mesías esperado: \textquote{Id a anunciar a Juan -dice- lo que estáis viendo y oyendo: los ciegos ven y los inválidos andan; los leprosos quedan limpios y los sordos oyen; los muertos resucitan, y a los pobres se les anuncia la buena nueva} (\emph{Mt} 11, 4-5).

Aquí radica la razón profunda de nuestra alegría: en Cristo se cumplió el tiempo de la espera. Dios realizó finalmente la salvación para todo hombre y para la humanidad entera. Con esta íntima convicción nos preparamos para celebrar la fiesta de la santa Navidad, acontecimiento extraordinario que vuelve a encender en nuestro corazón la esperanza y el gozo espiritual\ldots{}

6. \textquote{Tened paciencia, hermanos, hasta la venida del Señor} (\emph{St} 5, 7).

El Adviento nos invita a la alegría, pero, al mismo tiempo, nos exhorta a \textbf{esperar con paciencia} la venida ya próxima del Salvador. Nos exhorta a no desalentarnos, superando todo tipo de adversidades, con la certeza de que el Señor no tardará en venir.

Esta paciencia vigilante, como subraya el \textbf{apóstol Santiago} en la segunda lectura, favorece la consolidación de sentimientos fraternos en la comunidad cristiana. Al reconocerse humildes, pobres y necesitados de la ayuda de Dios, los creyentes se unen para acoger a su Mesías que está a punto de venir. Vendrá en el silencio, en la humildad y en la pobreza del pesebre, y a quien le abra el corazón le traerá su alegría.

Por tanto, avancemos con alegría y generosidad hacia la Navidad. Hagamos nuestros los sentimientos de María, que esperó en oración y en silencio al Redentor y preparó con cuidado su nacimiento en Belén. ¡Feliz Navidad!

\subsection{Benedicto XVI, papa}

\subsubsection{Homilía (2007)} 

Visita Pastoral a la Parroquia Romana de Santa María del Rosario en los Mártires Portuenses. Domingo 16 de diciembre de 2007.

\emph{Queridos hermanos y hermanas:}

\textquote{Estad siempre alegres en el Señor. Os lo repito: estad alegres. El Señor está cerca} (\emph{Flp} 4, 4-5).\\ Con esta invitación a la alegría comienza la antífona de entrada de la santa misa en este tercer domingo de Adviento, que precisamente por eso se llama domingo \textquote{\emph{Gaudete}\textquote{. En verdad, todo el Adviento es una invitación a alegrarse, porque }el Señor viene}, porque viene a salvarnos.

Durante estas semanas, casi diariamente, nos consuelan las palabras del profeta Isaías, dirigidas al pueblo judío desterrado en Babilonia después de la destrucción del templo de Jerusalén, el cual había perdido la esperanza de volver a la ciudad santa en ruinas. \textquote{A los que esperan en el Señor él les renovará el vigor ---asegura el profeta---, subirán con alas como de águilas, correrán sin fatigarse y andarán sin cansarse} (\emph{Is} 40, 31). Y también: \textquote{Regocijo y alegría los acompañarán. Pena y aflicción se alejarán} (\emph{Is} 35, 10).

La liturgia de Adviento nos repite constantemente que debemos despertar del sueño de la rutina y de la mediocridad; debemos abandonar la tristeza y el desaliento. Es preciso que se alegre nuestro corazón porque \textquote{el Señor está cerca}.

Hoy tenemos un motivo ulterior para alegrarnos, queridos fieles de la parroquia de \emph{Santa María del Rosario en los Mártires Portuenses,} y es la dedicación de vuestra nueva iglesia parroquial, que surge en el mismo lugar donde mi amado predecesor el siervo de Dios Juan Pablo II celebró, el 8 de noviembre de 1998, la santa misa con ocasión de su visita pastoral a vuestra comunidad.

La solemne liturgia de la dedicación de este templo constituye una ocasión de intenso gozo espiritual para todo el pueblo de Dios que vive en esta zona. Y de buen grado me uno también yo a vuestra satisfacción por tener por fin una iglesia acogedora y funcional. El lugar en que está construida evoca un pasado de testimonios cristianos resplandecientes. En efecto, precisamente aquí, en las cercanías, se encuentran las catacumbas de Generosa, donde según la tradición fueron sepultados tres hermanos, Simplicio, Faustino y Beatriz, víctimas de la persecución desencadenada en el año 303, y cuyos restos mortales fueron conservados, en parte, en Roma en la iglesia de San Nicolás in Carcere y en Monte Savello, y, en parte, en Fulda, Alemania, ciudad que desde el siglo VIII, gracias a que san Bonifacio llevó allí las reliquias, honra a los mártires portuenses como sus copatronos.

A este respecto, saludo al representante del obispo de Fulda, y también a mons. Carlo Liberati, arzobispo-prelado de Pompeya, santuario mariano con el que vuestra parroquia mantiene un hermanamiento espiritual.

La dedicación de esta iglesia parroquial cobra un significado muy particular para vosotros que vivís en este barrio. Los jóvenes mártires que entonces murieron por dar testimonio de Cristo, ¿no son un fuerte estímulo para vosotros, cristianos de hoy, a perseverar en el seguimiento fiel de Jesucristo? Y la protección de la Virgen del Santo Rosario, ¿no os pide ser hombres y mujeres de fe profunda y de oración, como lo fue ella?

También hoy, aunque sea con formas diversas, el mensaje salvífico de Cristo encuentra oposición y los cristianos, de otras maneras y no menos que ayer, están llamados a dar razón de su esperanza, a testimoniar ante el mundo la verdad de Cristo, el único que salva y redime. Por consiguiente, esta nueva iglesia ha de ser un espacio privilegiado para crecer en el conocimiento y en el amor de Cristo, a quien dentro de pocos días acogeremos en la alegría de su nacimiento como Redentor del mundo y Salvador nuestro.

Aprovechando la dedicación de esta nueva y hermosa iglesia, quiero dar las gracias a todos los que han contribuido a construirla. Sé que la diócesis de Roma se está esforzando con empeño, desde hace muchos años, por lograr que en cada barrio de esta ciudad en crecimiento constante haya complejos parroquiales adecuados.

Saludo y expreso mi gratitud, en primer lugar, al cardenal vicario y al obispo auxiliar Ernesto Mandara, secretario de la Obra romana para la conservación de la fe y la provisión de nuevas iglesias en Roma. Os saludo y os manifiesto mi agradecimiento en particular a vosotros, queridos feligreses, que de diversas maneras os habéis comprometido en la realización de este centro parroquial, que se añade a los más de cincuenta que ya funcionan gracias al notable esfuerzo económico de la diócesis, de tantos fieles y ciudadanos de buena voluntad, y a la colaboración de las instituciones públicas. En este domingo, dedicado precisamente al apoyo de esa meritoria obra, pido a todos que prosigan ese compromiso con generosidad.

Asimismo, saludo con afecto a mons. Benedetto Tuzia, obispo auxiliar del sector oeste; a vuestro párroco, don Gerard Charles McCarthy, a quien agradezco de corazón las cordiales palabras que me ha dirigido al inicio de esta solemne celebración. Saludo a sus colaboradores sacerdotes, pertenecientes a la fraternidad sacerdotal de los Misioneros de San Carlos Borromeo, aquí representada por el superior general, mons. Massimo Camisasca, a la que desde 1997 está encomendada la atención pastoral de esta parroquia.

Saludo a las religiosas Oblatas del Amor Divino y a las Misioneras de San Carlos, que con gran entrega realizan su apostolado en esta comunidad, y a todos los grupos de niños, de jóvenes, de familias y de ancianos que animan la vida de la parroquia. También saludo cordialmente a los diversos movimientos eclesiales presentes, entre los cuales están la Juventud ardiente mariana, Comunión y liberación, la Renovación carismática católica, la Fraternidad de Santa María de los ángeles, y el grupo de voluntariado Santa Teresita.

Además, quiero animar a todos los que, juntamente con la \emph{Cáritas} parroquial, tratan de salir al encuentro de las muchas necesidades del barrio, especialmente respondiendo a las expectativas de los más pobres y necesitados. Por último, saludo a las autoridades presentes y a las personalidades que han querido participar en esta asamblea litúrgica.

Queridos amigos, vivimos hoy una jornada que corona los esfuerzos, las fatigas, los sacrificios realizados y el compromiso de la comunidad de formar una comunidad cristiana madura, deseosa de tener un espacio reservado definitivamente al culto de Dios. Esta celebración es muy rica en palabras y símbolos que nos ayudan a comprender el valor profundo de lo que estamos realizando. Por eso, recojamos brevemente la enseñanza que nos dan las lecturas que se acaban de proclamar.

La primera lectura está tomada del libro de Nehemías, un libro que nos narra el restablecimiento de la comunidad judía después del destierro, después de la dispersión y la destrucción de Jerusalén. Por tanto, es el libro de los nuevos orígenes de una comunidad, y está lleno de esperanza, aunque las dificultades eran aún grandísimas. En el centro del pasaje que nos acaban de leer se encuentran dos grandes figuras: un sacerdote, Esdras, y un laico, Nehemías, que son respectivamente la autoridad religiosa y la autoridad civil de aquel tiempo.

El texto describe el momento solemne en que se restablece oficialmente, después de la dispersión, la pequeña comunidad judía; es el momento de volver a proclamar públicamente la ley, que es el fundamento de la vida de esta comunidad, y todo se desarrolla en un clima de sencillez, de pobreza y de esperanza. La escucha de esta proclamación tiene lugar en un clima de gran intensidad espiritual. Algunos comienzan a llorar de alegría por poder escuchar nuevamente con libertad la palabra de Dios, después de la tragedia de la destrucción de Jerusalén, y recomenzar la historia de la salvación. Y Nehemías los exhorta diciendo que es un día de fiesta y que, para tener la fuerza del Señor, es preciso alegrarse, agradeciendo a Dios sus dones. La palabra de Dios es fuerza y alegría.

También en nosotros esta lectura del Antiguo Testamento suscita gran conmoción. En este momento ¡cuántos recuerdos se agolpan en vuestra mente! ¡Cuántos esfuerzos realizados para construir, año tras año, la comunidad! ¡Cuántos sueños, cuántos proyectos, cuántas dificultades! Sin embargo, ahora tenéis la posibilidad de proclamar y escuchar la palabra de Dios en una hermosa iglesia, que favorece el recogimiento y suscita alegría, la alegría de saber que no sólo está presente la palabra de Dios, sino también el Señor mismo; una iglesia que quiere ser una invitación constante a una fe firme y al compromiso de crecer como comunidad unida. Agradezcamos a Dios sus dones y manifestemos nuestra gratitud también a todos los que han sido artífices de la construcción de esta iglesia y de la comunidad viva que en ella se reúne.

En la segunda lectura, tomada del Apocalipsis, se nos narra una visión estupenda. El proyecto de Dios para su Iglesia y para la humanidad entera es una ciudad santa, Jerusalén, que desciende del cielo resplandeciente de gloria divina. El autor la describe como ciudad maravillosa, comparándola con las joyas más preciosas, y por último precisa que se apoya en la persona y en el mensaje de los Apóstoles. Al decir esto, el evangelista san Juan nos sugiere que la comunidad viva es la verdadera nueva Jerusalén, y que la comunidad viva es más sagrada que el templo material que consagramos.

Para construir este templo vivo, esta nueva ciudad de Dios en nuestras ciudades, para construir el templo que sois vosotros, hace falta mucha oración, hace falta aprovechar todas las oportunidades que nos brindan la liturgia, la catequesis y las múltiples actividades pastorales, caritativas, misioneras y culturales, que conservan \textquote{joven} vuestra prometedora parroquia. El cuidado que con razón brindamos al edificio material ---rociándolo con el agua bendita, ungiéndolo con óleo y llenándolo de incienso--- debe ser signo y estímulo de un cuidado más intenso para defender y promover el templo de las personas, formado por vosotros, queridos feligreses.

Por último, la página evangélica que acabamos de escuchar nos narra el diálogo entre Jesús y los suyos, en particular con Pedro. Es una conversación totalmente centrada en la persona del Maestro divino. La gente había intuido algo en él. Algunos pensaban que era Juan Bautista que había vuelto a la vida; otros que Elías había regresado a la tierra; otros, que era el profeta Jeremías. En cualquier caso, la gente pensaba que era una de las grandes personalidades religiosas.

Pedro, en cambio, en nombre de los discípulos que conocen a Jesús de cerca, declara que Jesús es más que un profeta, más que una gran personalidad religiosa de la historia: es el Mesías, el Cristo, el Hijo de Dios vivo. Y Cristo, el Señor, le dice respondiendo solemnemente: \textquote{Tú eres Pedro y sobre esta piedra edificaré mi Iglesia} (\emph{Mt} 16, 18). Pedro, el pobre hombre con todas sus debilidades y con su fe, se convierte en la piedra, asociado precisamente por su fe a Jesús, es la roca sobre la que está fundada la Iglesia.

De ese modo, vemos una vez más cómo Jesucristo es la verdadera roca indefectible sobre la que se apoya nuestra fe, sobre la que se construye toda la Iglesia y, así, también esta parroquia. Y a Jesús lo encontramos en la escucha de la sagrada Escritura; está presente y se hace nuestro alimento en la Eucaristía; vive en la comunidad, en la fe de la comunidad parroquial.

Por consiguiente, en la iglesia edificio y en la Iglesia comunidad, todo habla de Jesús; todo gira en torno a él; todo hace referencia a él. Y Jesús, el Señor, nos reúne en la gran comunidad de la Iglesia de todos los tiempos y de todos los lugares, en comunión con el Sucesor de Pedro como roca de la unidad. La acción de los obispos y de los presbíteros, el compromiso apostólico y misionero de todos los fieles consiste en proclamar y testimoniar con la palabra y con la vida que él, el Hijo de Dios hecho hombre, es nuestro único Salvador.

Pidamos a Jesús que guíe a vuestra comunidad y la haga crecer cada vez más en la fidelidad a su Evangelio; pidámosle que suscite muchas y santas vocaciones sacerdotales, religiosas y misioneras; que suscite en todos los feligreses la disponibilidad a seguir el ejemplo de los santos mártires portuenses.

Pongamos esta oración en las manos maternales de María, Reina del Rosario. Que ella obtenga que se realicen en nosotros, en este día, las palabras finales de la primera lectura: \textquote{Que la alegría del Señor sea nuestra fuerza} (cf. \emph{Ne} 8, 10). Sólo la alegría del Señor y la fuerza de la fe en él pueden hacer fecundo el camino de vuestra parroquia. Así sea.

\subsubsection{Ángelus (2007)}  
	
Plaza de San Pedro.\\ Domingo 16 de diciembre de 2007.


\emph{Queridos hermanos y hermanas:} 

\emph{\textquote{Gaudete in Domino semper}, estad siempre alegres en el Señor} (\emph{Flp} 4, 4). Con estas palabras de san Pablo se inicia la santa misa del III domingo de Adviento, que por eso se llama domingo \emph{\textquote{Gaudete}}. El Apóstol exhorta a los cristianos a alegrarse porque la venida del Señor, es decir, su vuelta gloriosa es segura y no tardará. La Iglesia acoge esta invitación mientras se prepara para celebrar la Navidad, y su mirada se dirige cada vez más a Belén. En efecto, aguardamos con esperanza segura la segunda venida de Cristo, porque hemos conocido la primera.

El misterio de Belén nos revela al Dios-con-nosotros, al Dios cercano a nosotros, no sólo en sentido espacial y temporal; está cerca de nosotros porque, por decirlo así, se ha \textquote{casado} con nuestra humanidad; ha asumido nuestra condición, escogiendo ser en todo como nosotros, excepto en el pecado, para hacer que lleguemos a ser como él.

Por tanto, la alegría cristiana brota de esta certeza: Dios está cerca, está conmigo, está con nosotros, en la alegría y en el dolor, en la salud y en la enfermedad, como amigo y esposo fiel. Y esta alegría permanece también en la prueba, incluso en el sufrimiento; y no está en la superficie, sino en lo más profundo de la persona que se encomienda a Dios y confía en él.

Algunos se preguntan: ¿también hoy es posible esta alegría? La respuesta la dan, con su vida, hombres y mujeres de toda edad y condición social, felices de consagrar su existencia a los demás. En nuestros tiempos, la beata madre Teresa de Calcuta fue testigo inolvidable de la verdadera alegría evangélica. Vivía diariamente en contacto con la miseria, con la degradación humana, con la muerte. Su alma experimentó la prueba de la noche oscura de la fe y, sin embargo, regaló a todos la sonrisa de Dios.

En uno de sus escritos leemos: \textquote{Esperamos con impaciencia el paraíso, donde está Dios, pero ya aquí en la tierra y desde este momento podemos estar en el paraíso. Ser felices con Dios significa: amar como él, ayudar como él, dar como él, servir como él} (\emph{La gioia di darsi agli altri}, Ed. Paoline 1987, p. 143). Sí, la alegría entra en el corazón de quien se pone al servicio de los pequeños y de los pobres. Dios habita en quien ama así, y el alma vive en la alegría.

En cambio, si se hace de la felicidad un ídolo, se equivoca el camino y es verdaderamente difícil encontrar la alegría de la que habla Jesús. Por desgracia, esta es la propuesta de las culturas que ponen la felicidad individual en lugar de Dios, mentalidad que se manifiesta de forma emblemática en la búsqueda del placer a toda costa y en la difusión del uso de drogas como fuga, como refugio en paraísos artificiales, que luego resultan del todo ilusorios.

Queridos hermanos y hermanas, también en Navidad se puede equivocar el camino, confundiendo la verdadera fiesta con una que no abre el corazón a la alegría de Cristo. Que la Virgen María ayude a todos los cristianos, y a los hombres que buscan a Dios, a llegar hasta Belén para encontrar al Niño que nació por nosotros, para la salvación y la felicidad de todos los hombres.


\subsubsection{Homilía (2010)}

Domingo 12 de diciembre de 2010.

\emph{Queridos hermanos y hermanas de la parroquia de San Maximiliano Kolbe:}

Vivid con empeño el camino personal y comunitario de seguimiento del Señor. El Adviento es una fuerte invitación para todos a dejar que Dios entre cada vez más en nuestra vida, en nuestros hogares, en nuestros barrios, en nuestras comunidades, para tener una luz en medio de tantas sombras y de las numerosas pruebas de cada día. Queridos amigos, estoy muy contento de estar entre vosotros hoy para celebrar el día del Señor, el tercer domingo del Adviento, domingo de la alegría. Saludo cordialmente al cardenal vicario, al obispo auxiliar del sector, a vuestro párroco, a quien agradezco las palabras que me ha dirigido en nombre de todos vosotros, y al vicario parroquial. Saludo a cuantos colaboran en las actividades de la parroquia: a los catequistas, a las personas que forman parte de los diversos grupos, así como a los numerosos miembros del Camino Neocatecumenal. Aprecio mucho la elección de dar espacio a la adoración eucarística, y os agradezco las oraciones que me reserváis ante el Santísimo Sacramento. Quiero extender mi saludo a todos los habitantes del barrio, especialmente a los ancianos, a los enfermos, a las personas solas o que atraviesan dificultades. A todos y cada uno los recuerdo en esta misa.

Admiro junto con vosotros esta nueva iglesia y los edificios parroquiales, y con mi presencia deseo alentaros a construir cada vez mejor la Iglesia de piedras vivas que sois vosotros mismos. Conozco las numerosas y significativas obras de evangelización que estáis realizando. Exhorto a todos los fieles a contribuir a la edificación de la comunidad, especialmente en el campo de la catequesis, de la liturgia y de la caridad ---pilares de la vida cristiana--- en comunión con toda la diócesis de Roma. Ninguna comunidad puede vivir como una célula aislada del contexto diocesano; al contrario, debe ser expresión viva de la belleza de la Iglesia que, bajo la guía del obispo ---y, en la parroquia, bajo la guía del párroco, que lo representa---, camina en comunión hacia el reino de Dios. Dirijo un saludo especial a las familias, acompañándolo con el deseo de que realicen plenamente su vocación al amor con generosidad y perseverancia. Aunque se presentaran dificultades en la vida conyugal y en la relación con los hijos, los esposos deben permanecer siempre fieles al fundamental \textquote{sí} que pronunciaron delante de Dios y se dijeron mutuamente en el día de su matrimonio, recordando que la fidelidad a la propia vocación exige valentía, generosidad y sacrificio.

En el seno de vuestra comunidad hay muchas familias venidas del centro y del sur de Italia en busca de trabajo y de mejores condiciones de vida. Con el paso del tiempo, la comunidad ha crecido y en parte se ha transformado, con la llegada de numerosas personas de los países del Este europeo y de otros países. Precisamente a partir de esta situación concreta de la parroquia, esforzaos por crecer cada vez más en la comunión con todos: es importante crear ocasiones de diálogo y favorecer la comprensión mutua entre personas provenientes de culturas, modelos de vida y condiciones sociales diferentes; pero es preciso sobre todo tratar de que participen en la vida cristiana, mediante una pastoral atenta a las necesidades reales de cada uno. Aquí, como en cada parroquia, hay que partir de los \textquote{cercanos} para llegar a los \textquote{lejanos}, para llevar una presencia evangélica a los ambientes de vida y de trabajo. En la parroquia todos deben poder encontrar caminos adecuados de formación y experimentar la dimensión comunitaria, que es una característica fundamental de la vida cristiana. De ese modo se verán alentados a redescubrir la belleza de seguir a Cristo y de formar parte de su Iglesia.

Sabed, pues, hacer comunidad con todos, unidos en la escucha de la Palabra de Dios y en la celebración de los sacramentos, especialmente de la Eucaristía. A este propósito, la verificación pastoral diocesana que se está llevando a cabo, sobre el tema \textquote{Eucaristía dominical y testimonio de la caridad}, es una ocasión propicia para profundizar y vivir mejor estos dos componentes fundamentales de la vida y de la misión de la Iglesia y de todo creyente, es decir, la Eucaristía del domingo y la practica de la caridad. Reunidos en torno a la Eucaristía, sentimos más fácilmente que la misión de toda comunidad cristiana consiste en llevar el mensaje del amor de Dios a todos los hombres. Por eso es importante que la Eucaristía siempre sea el corazón de la vida de los fieles. También quiero dirigiros unas palabras de afecto y de amistad en especial a vosotros, queridos muchachos y jóvenes que me escucháis, y a vuestros coetáneos que viven en esta parroquia. La Iglesia espera mucho de vosotros, de vuestro entusiasmo, de vuestra capacidad de mirar hacia adelante y de vuestro deseo de radicalidad en las opciones de la vida. Sentíos verdaderos protagonistas en la parroquia, poniendo vuestras energías lozanas y toda vuestra vida al servicio de Dios y de los hermanos.

Queridos hermanos y hermanas, la liturgia de hoy ---con las palabras del apóstol Santiago que hemos escuchado--- nos invita no sólo a la alegría sino también a ser constantes y pacientes en la espera del Señor que viene, y a serlo juntos, como comunidad, evitando quejas y juicios (cf. \emph{St} 5, 7-10).

Hemos escuchado en el Evangelio la pregunta de san Juan Bautista que se encuentra en la cárcel; el Bautista, que había anunciado la venida del Juez que cambia el mundo, y ahora siente que el mundo sigue igual. Por eso, pide que pregunten a Jesús: \textquote{¿Eres tú el que ha de venir o debemos esperar a otro? ¿Eres tú o debemos esperar a otro?}. En los últimos dos o tres siglos muchos han preguntado: \textquote{¿Realmente eres tú o hay que cambiar el mundo de modo más radical? ¿Tú no lo haces?}. Y han venido muchos profetas, ideólogos y dictadores que han dicho: \textquote{¡No es él! ¡No ha cambiado el mundo! ¡Somos nosotros!}. Y han creado sus imperios, sus dictaduras, su totalitarismo que cambiaría el mundo. Y lo ha cambiado, pero de modo destructivo. Hoy sabemos que de esas grandes promesas no ha quedado más que un gran vacío y una gran destrucción. No eran ellos.

Y así debemos mirar de nuevo a Cristo y preguntarle: \textquote{¿Eres tú?}. El Señor, con el modo silencioso que le es propio, responde: \textquote{Mirad lo que he hecho. No he hecho una revolución cruenta, no he cambiado el mundo con la fuerza, sino que he encendido muchas luces que forman, a la vez, un gran camino de luz a lo largo de los milenios}.

Comencemos aquí, en nuestra parroquia: san Maximiliano Kolbe, que se ofreció para morir de hambre a fin de salvar a un padre de familia. ¡En qué gran luz se ha convertido! ¡Cuánta luz ha venido de esta figura! Y ha alentado a otros a entregarse, a estar cerca de quienes sufren, de los oprimidos. Pensemos en el padre que era para los leprosos Damián de Veuster, que vivió y murió \emph{con} y \emph{para} los leprosos, y así llevó luz a esa comunidad. Pensemos en la madre Teresa, que dio tanta luz a personas, que, después de una vida sin luz, murieron con una sonrisa, porque las había tocado la luz del amor de Dios.

Y podríamos seguir y veríamos, como dijo el Señor en la respuesta a Juan, que lo que cambia el mundo no es la revolución violenta, ni las grandes promesas, sino la silenciosa luz de la verdad, de la bondad de Dios, que es el signo de su presencia y nos da la certeza de que somos amados hasta el fondo y de que no caemos en el olvido, no somos un producto de la casualidad, sino de una voluntad de amor.

Así podemos vivir, podemos sentir la cercanía de Dios. \textquote{Dios está cerca} dice la primera lectura de hoy; está cerca, pero nosotros a menudo estamos lejos. Acerquémonos, vayamos hacia la presencia de su luz, oremos al Señor y en el contacto de la oración también nosotros seremos luz para los demás.

Precisamente este es el sentido de la iglesia parroquial: entrar aquí, entrar en diálogo, en contacto con Jesús, con el Hijo de Dios, a fin de que nosotros mismos nos convirtamos en una de las luces más pequeñas que él ha encendido y traigamos luz al mundo, que sienta que es redimido.

Nuestro espíritu debe abrirse a esta invitación; así caminemos con alegría al encuentro de la Navidad, imitando a la Virgen María, que esperó en la oración, con íntimo y gozoso temor, el nacimiento del Redentor. Amén.

\subsubsection{Ángelus (2010): ¿Qué es lo que cambia el mundo?}

Visita pastoral parroquia romana de San Maximiliano Kolbe, barrio de Torre Angela.

Domingo 12 de diciembre del 2010.

Queridos hermanos y hermanas de la parroquia de San Maximiliano Kolbe:

Vivid con empeño el camino personal y comunitario de seguimiento del Señor. El Adviento es una fuerte invitación para todos a dejar que Dios entre cada vez más en nuestra vida, en nuestros hogares, en nuestros barrios, en nuestras comunidades, para tener una luz en medio de tantas sombras y de las numerosas pruebas de cada día. Queridos amigos, estoy muy contento de estar entre vosotros hoy para celebrar el día del Señor, el tercer domingo del Adviento, domingo de la \textbf{alegría}\ldots{}

[\ldots{}] La liturgia de hoy ---con las palabras del \textbf{apóstol Santiago} que hemos escuchado--- nos invita no sólo a la alegría sino también a ser constantes y pacientes en la espera del Señor que viene, y a serlo juntos, como comunidad, evitando quejas y juicios (cf. \emph{St} 5, 7-10).

Hemos escuchado en el \textbf{Evangelio} la \textbf{pregunta de san Juan Bautista} que se encuentra en la cárcel; el Bautista, que había anunciado la venida del Juez que cambia el mundo, y ahora siente que el mundo sigue igual. Por eso, pide que pregunten a Jesús: \textquote{¿Eres tú el que ha de venir o debemos esperar a otro? ¿Eres tú o debemos esperar a otro?}. En los últimos dos o tres siglos muchos han preguntado: \textquote{¿Realmente eres tú o hay que cambiar el mundo de modo más radical? ¿Tú no lo haces?}. Y han venido muchos profetas, ideólogos y dictadores que han dicho: \textquote{¡No es él! ¡No ha cambiado el mundo! ¡Somos nosotros!}. Y han creado sus imperios, sus dictaduras, su totalitarismo que cambiaría el mundo. Y lo ha cambiado, pero de modo destructivo. Hoy sabemos que de esas grandes promesas no ha quedado más que un gran vacío y una gran destrucción. No eran ellos.

Y así debemos mirar de nuevo a Cristo y preguntarle: \textquote{¿Eres tú?}. El Señor, con el modo silencioso que le es propio, responde: \textquote{Mirad lo que he hecho. No he hecho una revolución cruenta, no he cambiado el mundo con la fuerza, sino que he encendido muchas luces que forman, a la vez, un gran camino de luz a lo largo de los milenios}.

Comencemos aquí, en nuestra parroquia: san Maximiliano Kolbe, que se ofreció para morir de hambre a fin de salvar a un padre de familia. ¡En qué gran luz se ha convertido! ¡Cuánta luz ha venido de esta figura! Y ha alentado a otros a entregarse, a estar cerca de quienes sufren, de los oprimidos. Pensemos en el padre que era para los leprosos Damián de Veuster, que vivió y murió \emph{con} y \emph{para} los leprosos, y así llevó luz a esa comunidad. Pensemos en la madre Teresa, que dio tanta luz a personas, que, después de una vida sin luz, murieron con una sonrisa, porque las había tocado la luz del amor de Dios.

Y podríamos seguir y veríamos, como dijo el Señor en la respuesta a Juan, que lo que cambia el mundo no es la revolución violenta, ni las grandes promesas, sino la silenciosa luz de la verdad, de la bondad de Dios, que es el signo de su presencia y nos da la certeza de que somos amados hasta el fondo y de que no caemos en el olvido, no somos un producto de la casualidad, sino de una voluntad de amor.

Así podemos vivir, podemos sentir la cercanía de Dios. \textquote{Dios está cerca} dice la \textbf{primera lectura} de hoy; está cerca, pero nosotros a menudo estamos lejos. Acerquémonos, vayamos hacia la presencia de su luz, oremos al Señor y en el contacto de la oración también nosotros seremos luz para los demás.

Precisamente este es el sentido de la iglesia parroquial: entrar aquí, entrar en diálogo, en contacto con Jesús, con el Hijo de Dios, a fin de que nosotros mismos nos convirtamos en una de las luces más pequeñas que él ha encendido y traigamos luz al mundo, que sienta que es redimido.

Nuestro espíritu debe abrirse a esta invitación; así caminemos con alegría al encuentro de la Navidad, imitando a la Virgen María, que esperó en la oración, con íntimo y gozoso temor, el nacimiento del Redentor. Amén.



\subsection{Francisco, papa}

\subsubsection{Ángelus (2013): Es posible recomenzar}

Plaza de San Pedro. Domingo 15 de diciembre del 2013.

¡Gracias! Queridos hermanos y hermanas, ¡buenos días!

Hoy es el tercer domingo de Adviento, llamado también domingo de \emph{Gaudete}, es decir, domingo de la \textbf{alegría}. En la liturgia resuena repetidas veces la invitación a gozar, a alegrarse. ¿Por qué? Porque el Señor está cerca. La Navidad está cercana. El mensaje cristiano se llama \textquote{Evangelio}, es decir, \textquote{buena noticia}, un anuncio de alegría para todo el pueblo; la Iglesia no es un refugio para gente triste, la Iglesia es la casa de la alegría. Y quienes están tristes encuentran en ella la alegría, encuentran en ella la verdadera alegría.

Pero la alegría del Evangelio no es una alegría cualquiera. Encuentra su razón de ser en el saberse acogidos y amados por Dios. Como nos recuerda hoy el \textbf{profeta Isaías} (cf. 35, 1-6a.8a.10), Dios es Aquél que viene a salvarnos, y socorre especialmente a los extraviados de corazón. Su venida en medio de nosotros fortalece, da firmeza, dona valor, hace exultar y florecer el desierto y la estepa, es decir, nuestra vida, cuando se vuelve árida. ¿Cuándo llega a ser árida nuestra vida? Cuando no tiene el agua de la Palabra de Dios y de su Espíritu de amor.

Por más grandes que sean nuestros límites y nuestros extravíos, no se nos permite ser débiles y vacilantes ante las dificultades y ante nuestras debilidades mismas. Al contrario, estamos invitados a robustecer las manos, a fortalecer las rodillas, a tener valor y a no temer, porque nuestro Dios nos muestra siempre la grandeza de su misericordia. Él nos da la fuerza para seguir adelante. Él está siempre con nosotros para ayudarnos a seguir adelante. Es un Dios que nos quiere mucho, nos ama y por ello está con nosotros, para ayudarnos, para robustecernos y seguir adelante.

¡Ánimo! ¡Siempre adelante! Gracias a su ayuda podemos siempre recomenzar de nuevo. ¿Cómo? ¿Recomenzar desde el inicio? Alguien puede decirme: \textquote{No, Padre, yo he hecho muchas cosas\ldots{} Soy un gran pecador, una gran pecadora\ldots{} No puedo recomenzar desde el inicio}. ¡Te equivocas! Tú puedes recomenzar de nuevo. ¿Por qué? Porque Él te espera, Él está cerca de ti, Él te ama, Él es misericordioso, Él te perdona, Él te da la fuerza para recomenzar de nuevo. ¡A todos! Entonces somos capaces de volver a abrir los ojos, de superar tristeza y llanto y entonar un canto nuevo. Esta alegría verdadera permanece también en la prueba, incluso en el sufrimiento, porque no es una alegría superficial, sino que desciende en lo profundo de la persona que se fía de Dios y confía en Él.

La alegría cristiana, al igual que la esperanza, tiene su fundamento en la fidelidad de Dios, en la certeza de que Él mantiene siempre sus promesas. El \textbf{profeta Isaías} exhorta a quienes se equivocaron de camino y están desalentados a confiar en la fidelidad del Señor, porque su salvación no tardará en irrumpir en su vida. Quienes han encontrado a Jesús a lo largo del camino, experimentan en el corazón una serenidad y una alegría de la que nada ni nadie puede privarles. Nuestra alegría es Jesucristo, su amor fiel e inagotable. Por ello, cuando un cristiano llega a estar triste, quiere decir que se ha alejado de Jesús. Entonces, no hay que dejarle solo. Debemos rezar por él, y hacerle sentir el calor de la comunidad.

Que la Virgen María nos ayude a apresurar el paso hacia Belén, para encontrar al Niño que nació por nosotros, por la salvación y la alegría de todos los hombres. A ella le dice el Ángel: \textquote{Alégrate, llena de gracia, el Señor está contigo} (\emph{Lc} 1, 28). Que ella nos conceda vivir la alegría del Evangelio en la familia, en el trabajo, en la parroquia y en cada ambiente. Una alegría íntima, hecha de asombro y ternura. La alegría que experimenta la mamá cuando contempla a su niño recién nacido, y siente que es un don de Dios, un milagro por el cual sólo se puede agradecer.

\subsubsection{Ángelus (2016): Su venida, alegría plena}

Plaza de San Pedro. Domingo 11 de diciembre del 2016.

Hoy celebramos el tercer domingo de Adviento, caracterizado por la invitación de \textbf{san Pablo}: \textquote{Estad siempre alegres en el Señor: os lo repito, estad alegres} (Fil 4, 4-5). No es una alegría superficial o puramente emotiva a la que nos exhorta el apóstol, y ni siquiera una mundana o la alegría del consumismo. No, no es esa, sino que se trata de una alegría más auténtica, de la cual estamos llamados a redescubrir su sabor. El sabor de la verdadera alegría. Es una alegría que toca lo íntimo de nuestro ser, mientras que esperamos a Jesús, que ya ha venido a traer la salvación al mundo, el Mesías prometido, nacido en Belén de la Virgen María. La liturgia de la Palabra nos ofrece el contexto adecuado para comprender y vivir esta alegría. \textbf{Isaías} habla de desierto, de tierra árida, de estepa (cf. 35, 1); el profeta tiene ante sí manos débiles, rodillas vacilantes, corazones perdidos, ciegos, sordos y mudos (cf. vv. 3-6). Es el cuadro de una situación de desolación, de un destino inexorable sin Dios.

Pero finalmente la salvación es anunciada: \textquote{¡Ánimo, no temáis! ---dice el profeta--- [\ldots{}] Mirad que vuestro Dios, [\ldots{}] Él vendrá y os salvará} (cf. Is 35, 4). Y enseguida todo se transforma: el desierto florece, la consolación y la alegría inundan los corazones (cf. vv. 5-6). Estos signos anunciados por Isaías como reveladores de la salvación ya presente, se realizan en Jesús. Él mismo lo afirma \textbf{respondiendo a los mensajeros enviados por Juan Bautista}. ¿Qué dice Jesús a estos mensajeros? \textquote{Los ciegos ven y los cojos andan, los leprosos quedan limpios y los sordos oyen, los muertos resucitan} (Mt 11, 5). No son palabras, son hechos que demuestran cómo la salvación traída por Jesús, aferra a todo el ser humano y le regenera. Dios ha entrado en la historia para liberarnos de la esclavitud del pecado; ha puesto su tienda en medio de nosotros para compartir nuestra existencia, curar nuestras llagas, vendar nuestras heridas y donarnos la vida nueva. La alegría es el fruto de esta intervención de salvación y de amor de Dios.

Estamos llamados a dejarnos llevar por el sentimiento de exultación. Este júbilo, esta alegría\ldots{} Pero un cristiano que no está alegre, algo le falta a este cristiano, ¡o no es cristiano! La alegría del corazón, la alegría dentro que nos lleva adelante y nos da el valor. El Señor viene, viene a nuestra vida como libertador, viene a liberarnos de todas las esclavitudes interiores y exteriores. Es Él quien nos indica el camino de la fidelidad, de la paciencia y de la perseverancia porque, a su llegada, nuestra alegría será plena.

La Navidad está cerca, los signos de su aproximarse son evidentes en nuestras calles y en nuestras casas; también aquí en la Plaza se ha puesto el pesebre con el árbol al lado. Estos signos externos nos invitan a acoger al Señor que siempre viene y llama a nuestra puerta, llama a nuestro corazón, para estar cerca de nosotros. Nos invitan a reconocer sus pasos entre los de los hermanos que pasan a nuestro lado, especialmente los más débiles y necesitados.

Hoy estamos invitados a alegrarnos por la llegada inminente de nuestro Redentor; y estamos llamados a compartir esta alegría con los demás, dando consuelo y esperanza a los pobres, a los enfermos, a las personas solas e infelices. Que la Virgen María, la \textquote{sierva del Señor}, nos ayude a escuchar la voz de Dios en la oración y a servirle con compasión en los hermanos, para llegar preparados a la cita con la Navidad, preparando nuestro corazón para acoger a Jesús.


\subsubsection{Homilía (2019)}

Santa Misa para la Comunidad Católica Filipina. Basílica Vaticana\\ Domingo, 15 de diciembre de 2019.





\emph{Queridos hermanos y hermanas:}

Celebramos hoy el tercer domingo de Adviento. En la primera lectura, el profeta Isaías invita a la tierra entera a alegrarse por la venida del Señor, que trae la salvación a su pueblo. Viene a abrir los ojos a los ciegos y los oídos a los sordos, a curar a los cojos y a los mudos (cf. 35,5-6). La salvación se ofrece a todos, pero el Señor muestra una ternura especial por los más vulnerables, los más frágiles, los más pobres de su pueblo.

De las palabras del salmo responsorial aprendemos que hay otros vulnerables que merecen una mirada de amor especial de Dios: los oprimidos, los hambrientos, los prisioneros, los extranjeros, los huérfanos y las viudas (cf. \emph{Sal} 145,7-9). Son los habitantes de las periferias existenciales de ayer y de hoy.

En Jesucristo el amor salvífico de Dios se hace tangible: \textquote{Los ciegos ven y los cojos andan, los leprosos quedan limpios y los sordos oyen, los muertos resucitan y a los pobres se anuncia la Buena Nueva} (\emph{Mt} 11,5). Estos son los signos que acompañan la realización del Reino de Dios. No toques de trompeta o triunfos militares, no juicios y condenas de pecadores, sino liberación del mal y anuncio de misericordia y de paz.

También este año nos preparamos para celebrar el misterio de la Encarnación, de Emmanuel, el \textquote{Dios con nosotros} que obra maravillas en favor de su pueblo, especialmente de los más pequeños y frágiles. Estas maravillas son los \textquote{signos} de la presencia de su Reino. Y como todavía son muchos los habitantes de las periferias existenciales, debemos pedir al Señor que renueve cada año el milagro de la Navidad, ofreciéndonos nosotros mismos como instrumentos de su amor misericordioso por los más pequeños.

Para prepararnos adecuadamente a esta nueva efusión de gracia, la Iglesia nos brinda el tiempo de Adviento, en el que estamos llamados a despertar la esperanza en nuestros corazones e intensificar nuestra oración. Con este fin, en la riqueza de las diferentes tradiciones, las Iglesias particulares han introducido una variedad de prácticas devocionales.

En Filipinas existe desde hace siglos una novena en preparación para la Santa Navidad llamada \emph{Simbang-Gabi} (misa nocturna). Durante nueve días los fieles filipinos se reúnen al amanecer en sus parroquias para una celebración eucarística especial. En las últimas décadas, gracias a los emigrantes filipinos, esta devoción ha traspasado las fronteras nacionales llegando a muchos otros países. Desde hace años \emph{Simbang-Gabi} también se celebra en la diócesis de Roma, y hoy lo celebramos juntos aquí, en la basílica de San Pedro.

Con esta celebración queremos prepararnos para la Navidad según el espíritu de la Palabra de Dios que hemos escuchado, permaneciendo constantes hasta la venida definitiva del Señor, como nos recomienda el apóstol Santiago (cf. \emph{St} 5,7). Queremos comprometernos a manifestar el amor y la ternura de Dios hacia todos, especialmente hacia los más pequeños. Estamos llamados a ser levadura en una sociedad que a menudo ya no puede saborear la belleza de Dios y experimentar la gracia de su presencia.

Y vosotros, queridos hermanos y hermanas, que habéis dejado vuestra tierra en busca de un futuro mejor, tenéis una misión especial. Que vuestra fe sea \textquote{levadura} en las comunidades parroquiales a las que pertenecéis hoy. Os animo a multiplicar las oportunidades de encuentro para compartir vuestra riqueza cultural y espiritual, al mismo tiempo que os dejáis enriquecer por las experiencias de los demás. Todos estamos invitados a construir juntos esa comunión en la diversidad que es un rasgo distintivo del Reino de Dios, inaugurado por Jesucristo, el Hijo de Dios hecho hombre. Todos estamos llamados a practicar juntos la caridad con los habitantes de las periferias existenciales, poniendo al servicio nuestros diversos dones, para renovar los signos de la presencia del Reino. Todos estamos llamados a anunciar juntos el Evangelio, la Buena Nueva de la salvación, en todas las lenguas, para llegar al mayor número posible de personas.

Que el Santo Niño al que nos disponemos a adorar, envuelto en pobres pañales y recostado en un pesebre, os bendiga y os dé la fuerza para continuar vuestro testimonio con alegría.


\subsubsection{Ángelus (2019): Conversión}

Plaza de San Pedro. Domingo 15 de diciembre de 2019.

En este tercer domingo de Adviento, llamado el \textquote{domingo de la alegría}, la Palabra de Dios nos invita, por una parte, a la \emph{alegría} y, por otra, a la conciencia de que la existencia incluye también momentos de \emph{duda}, en los que es difícil creer. La \emph{alegría} y la \emph{duda} son experiencias que forman parte de nuestras vidas.

A la invitación explícita a la alegría del \textbf{profeta Isaías}: \textquote{Que el desierto y el sequedal se alegren, regocíjese la estepa y florezca como flor} (35, 1), se contrapone en el \textbf{Evangelio} la duda de Juan el Bautista: \textquote{¿Eres tú el que ha de venir, o debemos esperar a otro?} (\emph{Mateo} 11, 3). De hecho, el profeta ve más allá de la situación, tiene ante sí gente desanimada: manos débiles, rodillas vacilantes, corazones intranquilos (cf. \emph{Isaías} 35, 3-4). Es la misma realidad que siempre pone a prueba la fe. Pero el hombre de Dios mira más allá, porque el Espíritu Santo hace que su corazón sienta el poder de su promesa y anuncia la salvación: \textquote{¡Ánimo, no temáis! Mirad que vuestro Dios viene, [\ldots{}] os salvará} (v. 4). Y entonces todo se transforma: el desierto florece, el consuelo y la alegría se apoderan de los perdidos, los cojos, los ciegos, los mudos se curan (cf. vv. 5-6). Esto es lo que sucede con Jesús: \textquote{los ciegos ven y los cojos andan, los leprosos quedan limpios y los sordos oyen, los muertos resucitan y se anuncia a los pobres la Buena Nueva} (\emph{Mateo} 11, 5).

Esta descripción nos muestra que la salvación envuelve al hombre entero y lo regenera. Pero este nuevo nacimiento, con la alegría que lo acompaña, presupone siempre una muerte para nosotros mismos y para el pecado que está dentro de nosotros. De ahí la llamada a la conversión, que es la base de la predicación tanto del Bautista como de Jesús; en particular, se trata de convertir la idea que tenemos de Dios. Y el tiempo de Adviento nos estimula a hacerlo precisamente con \textbf{la pregunta que Juan el Bautista le hace a Jesús}: \textquote{¿Eres tú el que ha de venir, o debemos esperar a otro?} (\emph{Mateo} 11, 3). Pensemos: toda su vida Juan esperó al Mesías; su estilo de vida, su cuerpo mismo, está moldeado por esta espera. Por eso también Jesús lo alaba con estas palabras: \textquote{no ha surgido entre los nacidos de mujer uno mayor que Juan el Bautista} (\emph{Mateo} 11, 11). Sin embargo, él también tuvo que convertirse a Jesús. Como Juan, también nosotros estamos llamados a reconocer el rostro que Dios eligió asumir en Jesucristo, humilde y misericordioso.

El Adviento es un tiempo de gracia. Nos dice que no basta con creer en Dios: es necesario purificar nuestra fe cada día. Se trata de prepararnos para acoger no a un personaje de cuento de hadas, sino al Dios que nos llama, que nos implica y ante el que se impone una elección. El Niño que yace en el pesebre tiene el rostro de nuestros hermanos más necesitados, de los pobres, que \textquote{son los privilegiados de este misterio y, a menudo, aquellos que son más capaces de reconocer la presencia de Dios en medio de nosotros} (Carta Apostólica \emph{Admirabile signum}, 6).

Que la Virgen María nos ayude para que, al acercarnos a la Navidad, no nos dejemos distraer por las cosas externas, sino que hagamos espacio en nuestros corazones a Aquél que ya ha venido y quiere volver a venir para curar nuestras enfermedades y darnos su alegría.



\section{Temas}

El gozo

CEC 30, 163, 301, 736, 1829, 1832, 2015, 2362:

\textbf{30} \textquote{Alégrese el corazón de los que buscan a Dios} (\emph{Sal} 105,3). Si el hombre puede olvidar o rechazar a Dios, Dios no cesa de llamar a todo hombre a buscarle para que viva y encuentre la dicha. Pero esta búsqueda exige del hombre todo el esfuerzo de su inteligencia, la rectitud de su voluntad, \textquote{un corazón recto}, y también el testimonio de otros que le enseñen a buscar a Dios.

\textquote{Tú eres grande, Señor, y muy digno de alabanza: grande es tu poder, y tu sabiduría no tiene medida. Y el hombre, pequeña parte de tu creación, pretende alabarte, precisamente el hombre que, revestido de su condición mortal, lleva en sí el testimonio de su pecado y el testimonio de que tú resistes a los soberbios. A pesar de todo, el hombre, pequeña parte de tu creación, quiere alabarte. Tú mismo le incitas a ello, haciendo que encuentre sus delicias en tu alabanza, porque nos has hecho para ti y nuestro corazón está inquieto mientras no descansa en ti} (San Agustín, \emph{Confessiones,} 1,1,1).

\textbf{La fe, comienzo de la vida eterna}

\textbf{163} La fe nos hace gustar de antemano el gozo y la luz de la visión beatífica, fin de nuestro caminar aquí abajo. Entonces veremos a Dios \textquote{cara a cara} (\emph{1 Co} 13,12), \textquote{tal cual es} (\emph{1 Jn} 3,2). La fe es, pues, ya el comienzo de la vida eterna:

\textquote{Mientras que ahora contemplamos las bendiciones de la fe como reflejadas en un espejo, es como si poseyésemos ya las cosas maravillosas de que nuestra fe nos asegura que gozaremos un día} (San Basilio Magno, \emph{Liber de Spiritu Sancto} 15,36: PG 32, 132; cf. Santo Tomás de Aquino, \emph{S.Th}., 2-2, q.4, a.1, c).

\textbf{Dios mantiene y conduce la creación}

\textbf{301} Realizada la creación, Dios no abandona su criatura a ella misma. No sólo le da el ser y el existir, sino que la mantiene a cada instante en el ser, le da el obrar y la lleva a su término. Reconocer esta dependencia completa con respecto al Creador es fuente de sabiduría y de libertad, de gozo y de confianza:

\textquote{Amas a todos los seres y nada de lo que hiciste aborreces pues, si algo odiases, no lo hubieras creado. Y ¿cómo podría subsistir cosa que no hubieses querido? ¿Cómo se conservaría si no la hubieses llamado? Mas tú todo lo perdonas porque todo es tuyo, Señor que amas la vida} (\emph{Sb} 11, 24-26).

\textbf{736} Gracias a este poder del Espíritu Santo los hijos de Dios pueden dar fruto. El que nos ha injertado en la Vid verdadera hará que demos \textquote{el fruto del Espíritu, que es caridad, alegría, paz, paciencia, afabilidad, bondad, fidelidad, mansedumbre, templanza} (\emph{Ga} 5, 22-23). \textquote{El Espíritu es nuestra Vida}: cuanto más renunciamos a nosotros mismos (cf. \emph{Mt} 16, 24-26), más \textquote{obramos también según el Espíritu} (\emph{Ga} 5, 25):

«Por el Espíritu Santo se nos concede de nuevo la entrada en el paraíso, la posesión del reino de los cielos, la recuperación de la adopción de hijos: se nos da la confianza de invocar a Dios como Padre, la participación de la gracia de Cristo, el podernos llamar hijos de la luz, el compartir la gloria eterna (San Basilio Magno, \emph{Liber de Spiritu Sancto}, 15, 36: PG 32, 132).

\textbf{1829} La caridad tiene por \emph{frutos} el gozo, la paz y la misericordia. Exige la práctica del bien y la corrección fraterna; es benevolencia; suscita la reciprocidad; es siempre desinteresada y generosa; es amistad y comunión:

\textquote{La culminación de todas nuestras obras es el amor. Ese es el fin; para conseguirlo, corremos; hacia él corremos; una vez llegados, en él reposamos} (San Agustín, \emph{In epistulam Ioannis tractatus,} 10, 4).

\textbf{1832} Los \emph{frutos} del Espíritu son perfecciones que forma en nosotros el Espíritu Santo como primicias de la gloria eterna. La tradición de la Iglesia enumera doce: \textquote{caridad, gozo, paz, paciencia, longanimidad, bondad, benignidad, mansedumbre, fidelidad, modestia, continencia, castidad} (\emph{Ga} 5,22-23, vulg.).

\textbf{2015} El camino de la perfección pasa por la cruz. No hay santidad sin renuncia y sin combate espiritual (cf. \emph{2 Tm} 4). El progreso espiritual implica la ascesis y la mortificación que conducen gradualmente a vivir en la paz y el gozo de las bienaventuranzas:

\textquote{El que asciende no termina nunca de subir; y va paso a paso; no se alcanza nunca el final de lo que es siempre susceptible de perfección. El deseo de quien asciende no se detiene nunca en lo que ya le es conocido} (San Gregorio de Nisa, \emph{In Canticum} homilia 8).

\textbf{2362} \textquote{Los actos [\ldots{}] con los que los esposos se unen íntima y castamente entre sí son honestos y dignos, y, realizados de modo verdaderamente humano, significan y fomentan la recíproca donación, con la que se enriquecen mutuamente con alegría y gratitud} (\href{http://www.vatican.va/archive/hist_councils/ii_vatican_council/documents/vat-ii_const_19651207_gaudium-et-spes_sp.html}{\emph{GS}} 49). La sexualidad es fuente de alegría y de agrado:

\textquote{El Creador [\ldots{}] estableció que en esta función {[}de generación{]} los esposos experimentasen un placer y una satisfacción del cuerpo y del espíritu. Por tanto, los esposos no hacen nada malo procurando este placer y gozando de él. Aceptan lo que el Creador les ha destinado. Sin embargo, los esposos deben saber mantenerse en los límites de una justa moderación} (Pío XII, \emph{Discurso a los participantes en el Congreso de la Unión Católica Italiana de especialistas en Obstetricia}, 29 octubre 1951).

La paciencia

CEC 227, 2613, 2665, 2772:

\textbf{227} \emph{Es confiar en Dios en todas las circunstancias}, incluso en la adversidad. Una oración de Santa Teresa de Jesús lo expresa admirablemente:

Nada te turbe, Nada te espante.

Todo se pasa, Dios no se muda.

La paciencia, Todo lo alcanza;

Quien a Dios tiene, Nada le falta:

Sólo Dios basta. (\emph{Poesía}, 30)

\textbf{2613} San Lucas nos ha trasmitido tres \emph{parábolas} principales sobre la oración:

La primera, \textquote{el amigo importuno} (cf. \emph{Lc} 11, 5-13), invita a una oración insistente: \textquote{Llamad y se os abrirá}. Al que ora así, el Padre del cielo \textquote{le dará todo lo que necesite}, y sobre todo el Espíritu Santo que contiene todos los dones.

La segunda, \textquote{la viuda importuna} (cf. \emph{Lc} 18, 1-8), está centrada en una de las cualidades de la oración: es necesario orar siempre, sin cansarse, con la \emph{paciencia} de la fe. \textquote{Pero, cuando el Hijo del hombre venga, ¿encontrará fe sobre la tierra?}.

La tercera parábola, \textquote{el fariseo y el publicano} (cf. \emph{Lc} 18, 9-14), se refiere a la \emph{humildad} del corazón que ora. \textquote{Oh Dios, ten compasión de mí que soy pecador}. La Iglesia no cesa de hacer suya esta oración: ¡\emph{Kyrie eleison!}

\textbf{La oración a Jesús}

\textbf{2665} La oración de la Iglesia, alimentada por la palabra de Dios y por la celebración de la liturgia, nos enseña a orar al Señor Jesús. Aunque esté dirigida sobre todo al Padre, en todas las tradiciones litúrgicas incluye formas de oración dirigidas a Cristo. Algunos salmos, según su actualización en la Oración de la Iglesia, y el Nuevo Testamento ponen en nuestros labios y graban en nuestros corazones las invocaciones de esta oración a Cristo: Hijo de Dios, Verbo de Dios, Señor, Salvador, Cordero de Dios, Rey, Hijo amado, Hijo de la Virgen, Buen Pastor, Vida nuestra, nuestra Luz, nuestra Esperanza, Resurrección nuestra, Amigo de los hombres\ldots{}

\textbf{2772} De esta fe inquebrantable brota la esperanza que suscita cada una de las siete peticiones. Estas expresan los gemidos del tiempo presente, este tiempo de paciencia y de espera durante el cual \textquote{aún no se ha manifestado lo que seremos} (\emph{1 Jn} 3, 2; cf. \emph{Col} 3, 4). La Eucaristía y el Padre Nuestro están orientados hacia la venida del Señor, \textquote{¡hasta que venga!} (\emph{1 Co} 11, 26).

Manifestación de Jesús como el Mesías

CEC 439, 547-550, 1751:

\textbf{439} Numerosos judíos e incluso ciertos paganos que compartían su esperanza reconocieron en Jesús los rasgos fundamentales del mesiánico \textquote{hijo de David} prometido por Dios a Israel (cf. \emph{Mt} 2, 2; 9, 27; 12, 23; 15, 22; 20, 30; 21, 9. 15). Jesús aceptó el título de Mesías al cual tenía derecho (cf. \emph{Jn} 4, 25-26;11, 27), pero no sin reservas porque una parte de sus contemporáneos lo comprendían según una concepción demasiado humana (cf. \emph{Mt} 22, 41-46), esencialmente política (cf. \emph{Jn} 6, 15; \emph{Lc} 24, 21).

\textbf{Los signos del Reino de Dios}

\textbf{547} Jesús acompaña sus palabras con numerosos \textquote{milagros, prodigios y signos} (\emph{Hch} 2, 22) que manifiestan que el Reino está presente en Él. Ellos atestiguan que Jesús es el Mesías anunciado (cf, \emph{Lc} 7, 18-23).

\textbf{548} Los signos que lleva a cabo Jesús testimonian que el Padre le ha enviado (cf. \emph{Jn} 5, 36; 10, 25). Invitan a creer en Jesús (cf. \emph{Jn} 10, 38). Concede lo que le piden a los que acuden a él con fe (cf. \emph{Mc} 5, 25-34; 10, 52). Por tanto, los milagros fortalecen la fe en Aquel que hace las obras de su Padre: éstas testimonian que él es Hijo de Dios (cf. \emph{Jn} 10, 31-38). Pero también pueden ser \textquote{ocasión de escándalo} (\emph{Mt} 11, 6). No pretenden satisfacer la curiosidad ni los deseos mágicos. A pesar de tan evidentes milagros, Jesús es rechazado por algunos (cf. \emph{Jn} 11, 47-48); incluso se le acusa de obrar movido por los demonios (cf. \emph{Mc} 3, 22).

\textbf{549} Al liberar a algunos hombres de los males terrenos del hambre (cf. \emph{Jn} 6, 5-15), de la injusticia (cf. \emph{Lc} 19, 8), de la enfermedad y de la muerte (cf. \emph{Mt} 11,5), Jesús realizó unos signos mesiánicos; no obstante, no vino para abolir todos los males aquí abajo (cf. \emph{Lc} 12, 13. 14; \emph{Jn} 18, 36), sino a liberar a los hombres de la esclavitud más grave, la del pecado (cf. \emph{Jn} 8, 34-36), que es el obstáculo en su vocación de hijos de Dios y causa de todas sus servidumbres humanas.

\textbf{550} La venida del Reino de Dios es la derrota del reino de Satanás (cf. \emph{Mt} 12, 26): \textquote{Pero si por el Espíritu de Dios expulso yo los demonios, es que ha llegado a vosotros el Reino de Dios} (\emph{Mt} 12, 28). Los \emph{exorcismos} de Jesús liberan a los hombres del dominio de los demonios (cf. \emph{Lc} 8, 26-39). Anticipan la gran victoria de Jesús sobre \textquote{el príncipe de este mundo} (\emph{Jn} 12, 31). Por la Cruz de Cristo será definitivamente establecido el Reino de Dios: \emph{Regnavit a ligno Deus} (\textquote{Dios reinó desde el madero de la Cruz}, {[}Venancio Fortunato, \emph{Hymnus \textquote{Vexilla Regis}}: MGH 1/4/1, 34: PL 88, 96{]}).

\textbf{1751} El \emph{objeto} elegido es un bien hacia el cual tiende deliberadamente la voluntad. Es la materia de un acto humano. El objeto elegido especifica moralmente el acto del querer, según que la razón lo reconozca y lo juzgue conforme o no conforme al bien verdadero. Las reglas objetivas de la moralidad enuncian el orden racional del bien y del mal, atestiguado por la conciencia.

\ldots{} hermanas, tengo por mejor que nos pongamos delante del Señor y miremos su misericordia y grandeza y nuestra bajeza, y dénos El lo que quisiere, siquiera haya agua, siquiera sequedad: El sabe mejor lo que nos conviene.\\ Y con esto andaremos descansadas y el demonio no tendrá tanto lugar de hacernos trampantojos.

Entre estas cosas penosas y sabrosas juntamente da nuestro Señor al alma algunas veces unos júbilos y oración extraña, que no sabe entender qué es.\\ Porque si os hiciere esta merced, le alabéis mucho y sepáis que es cosa que pasa, la pongo aquí. Es, a mi parecer, una unión grande de las potencias, sino que las deja nuestro Señor con libertad para que gocen de este gozo, y a los sentidos lo mismo, sin entender qué es lo que gozan y cómo lo gozan.

Parece esto algarabía, y cierto pasa así, que es un gozo tan excesivo del alma, que no querría gozarle a solas, sino decirlo a todos para que la ayudasen a alabar a nuestro Señor, que aquí va todo su movimiento. ¡Oh, qué de fiestas haría y qué de muestras, si pudiese, para que todos entendiesen su gozo! Parece que se ha hallado a sí, y que, como el padre del hijo pródigo, querría convidar a todos y hacer grandes fiestas , por ver su alma en puesto que no puede dudar que está en seguridad, al menos por entonces. Y tengo para mí que es con razón; porque tanto gozo interior de lo muy íntimo del alma, y con tanta paz, y que todo su contento provoca a alabanzas de Dios, no es posible darle el demonio.

Es harto, estando con este gran ímpetu de alegría, que calle y pueda disimular, y no poco penoso. Esto debía sentir San Francisco, cuando le toparon los ladrones, que andaba por el campo dando voces y les dijo que era pregonero del gran Rey , y otros santos que se van a los desiertos por poder pregonar lo que San Francisco estas alabanzas de su Dios.

\textbf{Santa Teresa de Jesús}, \emph{El Castillo Interior,} Moradas Sextas, capítulo 6.

\chapter{Domingo IV de Adviento (A)}

\section{Lecturas}

PRIMERA LECTURA

Del libro del profeta Isaías 7, 10-14

Mirad: la virgen está encinta

El Señor volvió a hablar a Ajaz y le dijo:

«Pide un signo al Señor, tu Dios: en lo hondo del abismo o en lo alto
del cielo».

Respondió Ajaz:

\textquote{No lo pido, no quiero tentar al Señor}.

Entonces dijo Isaías:

«Escucha, casa de David: ¿no os basta cansar a los hombres, que cansáis
incluso a mi Dios? Pues el Señor, por su cuenta, os dará un signo.
Mirad: la virgen está encinta y da a luz un hijo, y le pondrá por nombre
Enmanuel».

SALMO RESPONSORIAL

Salmo 23, 1-2. 3-4ab. 5-6

Va a entrar el Señor; Él es el Rey de la gloria

℣. Del Señor es la tierra y cuanto la llena,

el orbe y todos sus habitantes:

él la fundó sobre los mares,

él la afianzó sobre los ríos. ℟.

℣. ¿Quién puede subir al monte del Señor?

¿Quién puede estar en el recinto sacro?

El hombre de manos inocentes y puro corazón,

que no confía en los ídolos. ℟.

℣. Ese recibirá la bendición del Señor,

le hará justicia el Dios de salvación.

Esta es la generación que busca al Señor,

que busca tu rostro, Dios de Jacob. ℟.

SEGUNDA LECTURA

De la carta del apóstol san Pablo a los Romanos 1, 1-7

Jesucristo, de la estirpe de David, Hijo de Dios

Pablo, siervo de Cristo Jesús, llamado a ser apóstol, escogido para el
Evangelio de Dios, que fue prometido por sus profetas en las Escrituras
Santas y se refiere a su Hijo, nacido de la estirpe de David según la
carne, constituido Hijo de Dios en poder según el Espíritu de santidad
por la resurrección de entre los muertos: Jesucristo nuestro Señor.

Por él hemos recibido la gracia del apostolado, para suscitar la
obediencia de la fe entre todos los gentiles, para gloria de su nombre.
Entre ellos os encontráis también vosotros, llamados de Jesucristo.

A todos los que están en Roma, amados de Dios, llamados santos, gracia y
paz de Dios nuestro Padre y del Señor Jesucristo.

EVANGELIO

Del Santo Evangelio según san Mateo 1, 18-24

Jesús nacerá de María, desposada con José, hijo de David

La generación de Jesucristo fue de esta manera:

María, su madre, estaba desposada con José y, antes de vivir juntos,
resultó que ella esperaba un hijo por obra del Espíritu Santo.

José, su esposo, como era justo y no quería difamarla, decidió
repudiarla en privado. Pero, apenas había tomado esta resolución, se le
apareció en sueños un ángel del Señor que le dijo: «José, hijo de David,
no temas acoger a María, tu mujer, porque la criatura que hay en ella
viene del Espíritu Santo. Dará a luz un hijo y tú le pondrás por nombre
Jesús, porque él salvará a su pueblo de sus pecados».

Todo esto sucedió para que se cumpliese lo que había dicho el Señor por
medio del profeta:

«Mirad: la Virgen concebirá y dará a luz un hijo

y le pondrán por nombre Enmanuel,

que significa \textquote{Dios-con-nosotros}».

Cuando José se despertó, hizo lo que le había mandado el ángel del Señor
y acogió a su mujer.



\section{Comentario Patrístico}

\subsection{San Beda el Venerable, presbítero}

¡Oh grande e insondable misterio!

Homilía 5 en la vigilia de Navidad: CCL 122, 32-36

En breves palabras, pero llenas de verismo, describe el evangelista san Mateo el nacimiento del Señor y Salvador nuestro Jesucristo, por el que el Hijo de Dios, eterno antes del tiempo, apareció en el tiempo Hijo del hombre. Al conducir el evangelista la serie genealógica partiendo de Abrahán para acabar en José, el esposo de María, y enumerar ---según el acostumbrado orden de la humana generación--- la totalidad así de los genitores como de los engendrados, y disponiéndose a hablar del nacimiento de Cristo, subrayó la enorme diferencia existente entre éste y el resto de los nacimientos: los demás nacimientos se producen por la normal unión del hombre y de la mujer mientras que él, por ser Hijo de Dios, vino al mundo por conducto de una Virgen. Y era conveniente bajo todos los aspectos que, al decidir Dios hacerse hombre para salvar a los hombres, no naciera sino de una virgen, pues era inimaginable que una virgen engendrara a ningún otro, sino a uno que, siendo Dios, ella lo procreara como Hijo.

\emph{Mirad} ---dice--- \emph{la Virgen está encinta y dará a luz un hijo, y le pondrá por nombre Emmanuel (que significa \textquote{Dios-con-nosotros})}. El nombre que el profeta da al Salvador, \textquote{\emph{Dios-con-nosotros}}, indica la doble naturaleza de su única persona. En efecto, el que es Dios nacido del Padre antes de los tiempos, es el mismo que, en la plenitud de los tiempos, se convirtió, en el seno materno, en Emmanuel, esto es, en \textquote{\emph{Dios-con-nosotros}}, ya que se dignó asumir la fragilidad de nuestra naturaleza en la unidad de su persona, cuando \emph{la Palabra se hizo carne y acampó entre nosotros}, esto es, cuando de modo admirable comenzó a ser lo que nosotros somos, sin dejar de ser lo que era asumiendo de forma tal nuestra naturaleza que no le obligase a perder lo que él era.

Dio, pues, a luz María a su hijo primogénito, es decir, al hijo de su propia carne; dio a luz al que, antes de la creación, había nacido Dios de Dios, y en la humanidad en que fue creado, superaba ampliamente a toda creatura. Y él \emph{le puso} ---dice--- \emph{por nombre Jesús}.

Jesús es el nombre del hijo que nació de la virgen, nombre que significa ---según la explicación del ángel--- que él iba a \emph{salvar a su pueblo de los pecados}. Y el que salva de los pecados, salvará igualmente de las corruptelas de alma y cuerpo, secuela del pecado.

La palabra \textquote{\emph{Cristo}} connota la dignidad sacerdotal o regia. En la ley, tanto los sacerdotes como los reyes eran llamados \textquote{\emph{cristos}} por el crisma, es decir, por la unción con el óleo sagrado: eran un signo de quien, al manifestarse en el mundo como verdadero Rey y Pontífice, fue ungido \emph{con aceite de júbilo entre todos sus compañeros}.

Debido a esta unción o crisma, se le llama \emph{Cristo}; a los que participan de esta unción, es decir, de esta gracia espiritual, se les llama \textquote{\emph{cristianos}}. Que él, por ser nuestro Salvador, nos salve de los pecados; en cuanto Pontífice, nos reconcilie con Dios Padre; en su calidad de Rey se digne darnos el reino eterno de su Padre, Jesucristo nuestro Señor, que con el Padre y el Espíritu Santo vive y reina y es Dios por todos los siglos de los siglos. Amén.

Durante su vida, que fue una peregrinación en la fe, José, al igual que María, permaneció fiel a la llamada de Dios hasta el final. La vida de ella fue el cumplimiento hasta sus últimas consecuencias de aquel primer \textquote{\emph{fiat}} pronunciado en el momento de la anunciación mientras que José en el momento de su \textquote{anunciación} no pronunció palabra alguna. Simplemente él \textquote{\emph{hizo} como el ángel del Señor le había mandado} (\emph{Mt} 1, 24). \emph{Y este primer \textquote{hizo} es el comienzo del \textquote{camino de José}}. A lo largo de este camino, los Evangelios no citan ninguna palabra dicha por él. Pero el \emph{silencio de José} posee una especial elocuencia: gracias a este silencio se puede leer plenamente la verdad contenida en el juicio que de él da el Evangelio: el \textquote{justo} (\emph{Mt} 1, 19).

Hace falta saber leer esta verdad, porque ella contiene \emph{uno de los testimonios más importantes acerca del hombre y de su vocación}. En el transcurso de las generaciones la Iglesia lee, de modo siempre atento y consciente, dicho testimonio, casi como si sacase del tesoro de esta figura insigne \textquote{lo nuevo y lo viejo} (\emph{Mt} 13, 52).

\textbf{San Juan Pablo II, papa,} \emph{Redemptoris Custos,} 17.



\section{Homilías}

\subsection{San Juan Pablo II, papa}

\subsubsection{Homilía (1983): Nos enseñan a acoger a Cristo}

Visita a la Parroquia romana de San Jorge Mártir.

Domingo 18 de diciembre de 1983.

1. \textquote{¿Quién puede subir al monte del Señor? ¿Quién puede estar en el recinto sacro? El hombre de manos inocentes y puro corazón} (Sal 24, 3-4).

Queridísimos hermanos y hermanas: La liturgia de este domingo cuarto y último de Adviento, insiste sobre el tema de la cercanía, recordando la llegada inminente del que debe venir, y trazando al mismo tiempo las características de quien, con motivo de esta venida, se acerca, a su vez, a Dios.

Desde los primeros versículos, el \textbf{Salmo responsorial} nos lleva a lo alto, a Aquel que es Señor de la tierra, de cuanto la llena, del universo y de sus habitantes. Dios creó todo para regalárselo al hombre, a fin de que este, por la contemplación de lo creado, pueda reconocerlo y acercarse a él.

Según la expresión del Salmista, Dios, precisamente porque trasciende todo el universo material, está \textquote{por encima} del mundo; y así, el acercamiento a él se presenta como un \textquote{subir}. Pero no se trata de un desplazamiento material en el espacio, sino de una apertura, una orientación del espíritu; una actividad \textquote{santa}, propia de los que buscan a Dios, \textquote{el grupo que busca el rostro del Dios de Jacob}.

La liturgia de hoy nos hace ver concretamente las dos figuras a las que les fue dado acercarse más a Aquel que tenía que venir: \textbf{María y José}. Son las dos personas culminantes del tiempo del Adviento, situadas en la etapa de la cercanía más grande de Dios mismo.

2. La figura de \textbf{María}, en la presente liturgia, queda delineada en dos pasajes de la Escritura: en el Antiguo Testamento, como prefiguración, con el texto de Isaías (Is 7, 10-14); y en el Nuevo, como realización, con el texto de Mateo (Mt 1, 18-24).

Los libros del Antiguo Testamento, al describirnos la historia de la salvación, ponen de relieve, paso a paso --como observa el Concilio (Lumen gentium, 55)--, cada vez con más claridad a la Madre del Redentor. Bajo este haz de luz ella queda proféticamente bosquejada en la imagen de la Virgen que concebirá y dará a luz un Hijo, cuyo nombre será Emmanuel, que quiere decir \textquote{Dios con nosotros}. Es apenas una anticipación eficaz para prefigurar un ser sin igual predestinado por Dios, el cual, ya muchos siglos antes, comienza a proyectar hacia nosotros algunos rasgos de su grandeza.

Este texto de \textbf{Isaías}, durante el curso de los siglos, se lee y entiende en la Iglesia a la luz de la revelación posterior. Lo que en el Antiguo Testamento, con sus aperturas mesiánicas, era un comienzo, se convierte en claridad en el Nuevo Testamento. \textbf{San Mateo} reconoce en las palabras de Isaías a la mujer que, por obra del Espíritu Santo, concibió virginalmente, con exclusión de intervención de varón.

Jesús es Aquel que salvará al pueblo de sus pecados. Y ella, María, es la Madre de Jesús. El Hijo de Dios \textquote{viene} a su seno para hacerse hombre. Ella lo acoge. Jamás Dios se acercó tanto al ser humano como en este caso en el que el Hijo y su Madre entran en una relación estrecha.

3. Al mismo tiempo, Mateo procura cuidadosamente hacernos partícipes de la acogida consciente y amorosa de parte de \textbf{José}. Él, el esposo, que por sí solo no puede explicarse el acontecimiento nuevo que se realizaba ante sus ojos, es iluminado por la intervención del Ángel del Señor sobre la naturaleza de la maternidad de María. \textquote{Lo concebido en ella es obra del Espíritu Santo} (Mt 1, 20).

De esta manera, José es puesto al corriente de los hechos y es llamado a insertarse en el designio salvífico de Dios. Ahora él sabe quién es el Niño que ha de nacer y quién es la Madre. De acuerdo con la invitación del ángel, llevó consigo a su esposa, no la repudió. \textquote{Acogiendo} a María, José acoge también al que en ella ha sido concebido por obra admirable de Dios, para quien nada es imposible.

Concentrándose en estos dos personajes del Adviento, la liturgia, nos conduce directamente al terreno de la Navidad.

4. Llegados a este punto, abramos el oído para escuchar la \textbf{segunda lectura}, tomada de la Carta dirigida por el apóstol san Pablo a los \textbf{Romanos}. Ella nos habla a nosotros, habitantes de la {[}Roma{]} moderna. El apóstol Pablo proclama la venida de Cristo a {[}Roma{]}, a nuestra propia ciudad: es la venida mediante el Evangelio, \textquote{el Evangelio de Dios\ldots{} acerca de su Hijo, nacido del linaje de David según la carne, constituido Hijo de Dios con poder según el Espíritu de santidad por su resurrección de entre los muertos, Jesucristo Señor nuestro, por quien recibimos la gracia y el apostolado} (Rm 1, 1-5).

5. Han pasado casi dos mil años desde que el apóstol escribió estas palabras y hoy siguen siendo actuales. Estas palabras se dirigen a nosotros hoy y no nos queda más que ponernos en actitud de disponibilidad para acoger a Jesucristo por medio del Evangelio que anuncia la Iglesia, del mismo modo que lo acogieron los primeros cristianos \ldots{} Queremos acogerlo, por utilizar la expresión del Apóstol, en toda la verdad de su Divinidad y de su Humanidad.

Recibámoslo la noche de Belén en el conjunto de su misterio pascual. \textquote{Por su resurrección de entre los muertos} Cristo ha sido \textquote{constituido Hijo de Dios con poder según el Espíritu de santidad}. Mediante el misterio pascual se ha revelado plenamente la filiación divina del que nació la noche de Belén.

Acojamos a Cristo Hijo de Dios, Aquel que debe venir; y, al acogerlo, esforcémonos por asemejarnos a María y a José, que fueron los primeros en acogerlo mediante la fe con la fuerza del Espíritu Santo. Efectivamente, en ellos se manifiesta la plena madurez del Adviento.

6. Hoy quiero desear a todos vosotros esa madurez de vida cristiana y esa disponibilidad abierta y generosa para acoger en la riqueza de su verdad al Hijo de Dios nacido según la carne, os la quiero desear hoy a todos \ldots{} Os animo a cada uno de vosotros a seguir adelante.

7. Queridísimos hermanos y hermanas: \textquote{Por él hemos recibido la gracia y el apostolado} \ldots{} Deseo que esta celebración traiga para nosotros la Gracia del Adviento divino.

\textquote{Mirad: la Virgen concebirá y dará a luz un Hijo, y le pondrán por nombre Emmanuel, que significa: Dios con nosotros} (Mt 1, 23). Que la Navidad traiga el cumplimiento del Adviento en cada uno de nosotros \ldots{} Que el Emmanuel, el Dios-con-nosotros, se convierta en la alegría y en la esperanza de todos los corazones humanos.

\subsubsection{Ángelus (1992)} \emph{\textbf{ÁNGELUS\\ }\\ Domingo 20 de diciembre de 1992}

\emph{Amadísimos hermanos y hermanas:}

1. Faltan ya pocos días para la celebración de la Navidad del Señor y queremos vivirlos siguiendo las huellas de María y haciendo nuestros en la medida de lo posible, los sentimientos que ella experimentó en la trémula espera del nacimiento de Jesús.

El evangelista Lucas narra que la Virgen santa y su esposo José se dirigieron de Galilea a Judea para ir a Belén, la ciudad de David, obedeciendo un decreto del emperador romano que ordenaba un censo general del Imperio.

Pero, ¿quién podía reparar en ellos? Pertenecían a la innumerable legión de pobres, a quienes la vida a duras penas regala un rincón para vivir, y que no dejan rastro en las crónicas. De hecho no encontraron acomodo en ningún sitio, a pesar de que llevaban el \textquote{secreto} del mundo.

Podemos intuir cuáles eran los sentimientos de María, totalmente abandonada en las manos del Señor. Ella es la mujer creyente: en la profundidad de su obediencia interior madura la plenitud de los tiempos (cf. \emph{Ga} 4, 4).

2. Por estar enraizada en la fe, la Madre del Verbo hecho hombre \emph{encarna la gran esperanza del mundo}. En ella confluye tanto la espera mesiánica de Israel como el anhelo de salvación de la humanidad entera. En su espíritu resuena el grito de dolor de los que, en toda época de la historia, se sienten abrumados por las dificultades de la vida: los hambrientos y los necesitados, los enfermos y las víctimas del odio y la guerra, los que no tienen hogar ni trabajo y los que viven solos y marginados, los que se sienten aplastados por la violencia y la injusticia o rechazados por la desconfianza y la indiferencia, los desanimados y los defraudados.

Para los hombres de toda raza y cultura, sedientos de amor, de fraternidad y de paz, María se prepara a dar a luz el fruto divino de su vientre. Por más oscuro que pueda parecer el horizonte, hay un alba que despunta. La humanidad, como recuerda san Pablo, gime y \textquote{sufre dolores de parto} (\emph{Rm} 8, 22): en el nacimiento del Hijo de Dios todo renace, todo está llamado a vida nueva.

3. Queridos hermanos y hermanas preparémonos para la Navidad con la fe y la esperanza de María. Dejemos que el mismo amor que vibra en su adhesión al plan divino toque nuestro corazón. La Navidad es tiempo de renovación y fraternidad: miremos a nuestro alrededor, miremos a lo lejos. El hombre que sufre, dondequiera que se encuentre, nos atañe. Allí se encuentra el belén al que debemos dirigirnos, con solidaridad activa, para encontrar de verdad al Redentor que nace en el mundo. Caminemos, por consiguiente, hacia la Noche Santa con María, la Madre del Amor. Con ella esperemos el cumplimiento del misterio de la salvación.


\subsection{Benedicto XVI, papa}

\subsubsection{Ángelus: San José miró el futuro con fe}

Domingo 19 de diciembre del 2010.

En este cuarto domingo de Adviento el \textbf{evangelio de san Mateo} narra cómo sucedió el nacimiento de Jesús situándose desde el punto de vista de san José. Él era el prometido de María, la cual \textquote{antes de empezar a estar juntos ellos, se encontró encinta por obra del Espíritu Santo} (\emph{Mt} 1, 18). El Hijo de Dios, realizando una antigua profecía (cf. \emph{Is} 7, 14), se hace hombre en el seno de una virgen, y ese misterio manifiesta a la vez el amor, la sabiduría y el poder de Dios a favor de la humanidad herida por el pecado. \textbf{San José} se presenta como hombre \textquote{justo} (\emph{Mt} 1, 19), fiel a la ley de Dios, disponible a cumplir su voluntad. Por eso entra en el misterio de la Encarnación después de que un ángel del Señor, apareciéndosele en sueños, le anuncia: \textquote{José, hijo de David, no temas tomar contigo a María, tu mujer, porque lo engendrado en ella es del Espíritu Santo. Dará a luz un hijo y tú le pondrás por nombre Jesús, porque él salvará a su pueblo de sus pecados} (\emph{Mt} 1, 20-21). Abandonando el pensamiento de repudiar en secreto a María, la toma consigo, porque ahora sus ojos ven en ella la obra de Dios.

San Ambrosio comenta que \textquote{en José se dio la amabilidad y la figura del justo, para hacer más digna su calidad de testigo} (\emph{Exp. Ev. sec. Lucam} II, 5: ccl 14, 32-33). Él ---prosigue san Ambrosio--- \textquote{no habría podido contaminar el templo del Espíritu Santo, la Madre del Señor, el seno fecundado por el misterio} (\emph{ib}., II, 6: CCL 14, 33). A pesar de haber experimentado turbación, José actúa \textquote{como le había ordenado el ángel del Señor}, seguro de hacer lo que debía. También poniendo el nombre de \textquote{Jesús} a ese Niño que rige todo el universo, él se inserta en el grupo de los servidores humildes y fieles, parecido a los ángeles y a los profetas, parecido a los mártires y a los apóstoles, como cantan antiguos himnos orientales. San José anuncia los prodigios del Señor, dando testimonio de la virginidad de María, de la acción gratuita de Dios, y custodiando la vida terrena del Mesías. Veneremos, por tanto, al padre legal de Jesús (cf. \emph{\href{http://www.vatican.va/archive/catechism_sp/p122a3p3_sp.html\#II\%20Los\%20misterios\%20de\%20la\%20infancia\%20y\%20de\%20la\%20vida\%20oculta\%20de\%20Jes\%C3\%BAs}{Catecismo de la Iglesia católica},} n. 532), porque en él se perfila el hombre nuevo, que mira con fe y valentía al futuro, no sigue su propio proyecto, sino que se confía totalmente a la infinita misericordia de Aquel que realiza las profecías y abre el tiempo de la salvación.

Queridos amigos, a san José, patrono universal de la Iglesia, deseo confiar a todos los pastores, exhortándolos a ofrecer \textquote{a los fieles cristianos y al mundo entero la humilde y cotidiana propuesta de las palabras y de los gestos de Cristo} (\href{https://www.deiverbum.org/content/benedict-xvi/es/letters/2009/documents/hf_ben-xvi_let_20090616_anno-sacerdotale.html}{Carta de convocatoria del Año sacerdotal}). Que nuestra vida se adhiera cada vez más a la Persona de Jesús, precisamente porque \textquote{el que es la Palabra asume él mismo un cuerpo; viene de Dios como hombre y atrae a sí toda la existencia humana, la lleva al interior de la palabra de Dios} (\emph{Jesús de Nazaret,} Madrid 2007, p. 387). Invoquemos con confianza a la Virgen María, la llena de gracia \textquote{adornada de Dios}, para que, en la Navidad ya inminente, nuestros ojos se abran y vean a Jesús, y el corazón se alegre en este admirable encuentro de amor.

\subsection{Francisco, papa}

\subsubsection{Ángelus: La prueba de José}

Plaza de San Pedro. Domingo 22 de diciembre del 2013.

En este cuarto domingo de Adviento, el Evangelio nos relata los hechos que precedieron el nacimiento de Jesús, y el evangelista Mateo los presenta desde el punto de vista de san José, el prometido esposo de la Virgen María.

\textbf{José} y \textbf{María} vivían en Nazaret; aún no vivían juntos, porque el matrimonio no se había realizado todavía. Mientras tanto, María, después de acoger el anuncio del Ángel, quedó embarazada por obra del Espíritu Santo. Cuando José se dio cuenta del hecho, quedó desconcertado. El Evangelio no explica cuáles fueron sus pensamientos, pero nos dice lo esencial: él busca cumplir la voluntad de Dios y está preparado para la renuncia más radical. En lugar de defenderse y hacer valer sus derechos, José elige una solución que para él representa un enorme sacrificio. Y el Evangelio dice: \textquote{Como era justo y no quería difamarla, decidió repudiarla en privado} (1, 19).

Esta breve frase resume un verdadero drama interior, si pensamos en el amor que José tenía por María. Pero también en esa circunstancia José quiere hacer la voluntad de Dios y decide, seguramente con gran dolor, repudiar a María en privado. Hay que meditar estas palabras para comprender cuál fue la prueba que José tuvo que afrontar los días anteriores al nacimiento de Jesús. Una prueba semejante a la del sacrificio de Abrahán, cuando Dios le pidió el hijo Isaac (cf. \emph{Gn} 22): renunciar a lo más precioso, a la persona más amada.

Pero, como en el caso de Abrahán, el Señor interviene: encontró la fe que buscaba y abre una vía diversa, una vía de amor y de felicidad: \textquote{José ---le dice--- no temas acoger a María, tu mujer, porque la criatura que hay en ella viene del Espíritu Santo} (\emph{Mt} 1, 20).

Este \textbf{Evangelio} nos muestra toda la grandeza del alma de san José. Él estaba siguiendo un buen proyecto de vida, pero Dios reservaba para él otro designio, una misión más grande. José era un hombre que siempre dejaba espacio para escuchar la voz de Dios, profundamente sensible a su secreto querer, un hombre atento a los mensajes que le llegaban desde lo profundo del corazón y desde lo alto. No se obstinó en seguir su proyecto de vida, no permitió que el rencor le envenenase el alma, sino que estuvo disponible para ponerse a disposición de la novedad que se le presentaba de modo desconcertante. Y así, era un hombre bueno. No odiaba, y no permitió que el rencor le envenenase el alma. ¡Cuántas veces a nosotros el odio, la antipatía, el rencor nos envenenan el alma! Y esto hace mal. No permitirlo jamás: él es un ejemplo de esto. Y así, José llegó a ser aún más libre y grande.

Aceptándose según el designio del Señor, José se encuentra plenamente a sí mismo, más allá de sí mismo. Esta libertad de renunciar a lo que es suyo, a la posesión de la propia existencia, y esta plena disponibilidad interior a la voluntad de Dios, nos interpelan y nos muestran el camino.

Nos disponemos entonces a celebrar la Navidad contemplando a María y a José: María, la mujer llena de gracia que tuvo la valentía de fiarse totalmente de la Palabra de Dios; José, el hombre fiel y justo que prefirió creer al Señor en lugar de escuchar las voces de la duda y del orgullo humano. Con ellos, caminamos juntos hacia Belén.

\subsubsection{Ángelus: Respuesta a la cercanía de Dios}

Plaza de San Pedro. Domingo 18 de diciembre del 2016.

La liturgia de hoy, cuarto y último domingo de Adviento, está caracterizada por el tema de la cercanía, la cercanía de Dios a la humanidad. El pasaje del \textbf{Evangelio} (cfr. Mt 1,18-24) nos muestra a las dos personas que más que cualquier otra están envueltas en este misterio de amor: la Virgen \textbf{María} y su esposo \textbf{José}. Misterio de amor, misterio de cercanía de Dios con la humanidad.

\textbf{María} es presentada a la luz de la profecía que dice: \textquote{La Virgen concebirá y dará a luz un hijo} (Mt 1, 23). El evangelista Mateo reconoce que aquello ha acontecido en María, quien ha concebido a Jesús por obra del Espíritu Santo (cfr. v. 18). El hijo de Dios \textquote{viene} en su vientre para convertirse en hombre y Ella lo acoge. Así, de manera única, Dios se ha acercado al ser humano tomando la carne de una mujer: Dios se acercó a nosotros y tomó la carne de una mujer. También a nosotros, de manera diferente, Dios se acerca con su gracia para entrar en nuestra vida y ofrecernos en don a su Hijo. Y nosotros ¿qué hacemos? ¿Lo acogemos, lo dejamos acercarse o lo rechazamos, lo echamos? Como a María, que ofreciéndose libremente al Señor de la historia, se le ha permitido cambiar el destino de la humanidad, así también nosotros, acogiendo a Jesús y tratando de seguirlo cada día, podemos cooperar con su diseño de salvación sobre nosotros mismos y sobre el mundo. Por lo tanto María se nos presenta como el modelo al cual mirar y apoyo sobre el cual contar en nuestra búsqueda de Dios, en nuestra cercanía a Dios, con este dejar que Dios se acerque a nosotros, y en nuestro empeño por construir la civilización del amor.

El otro protagonista del Evangelio de hoy es San \textbf{José}. El evangelista pone en evidencia cómo José por sí solo no pueda darse una explicación del acontecimiento que ve verificarse ante sus ojos, o sea el embarazo de María. Precisamente entonces, en aquel momento de la duda, también del miedo, Dios se le acerca a través de un mensajero suyo y él es iluminado sobre la naturaleza de aquella maternidad: \textquote{porque lo que ha sido engendrado en ella proviene del Espíritu Santo} (v. 20). Así, frente al evento extraordinario, que ciertamente suscita en su corazón tantas interrogantes, José confía totalmente en Dios que se le acerca y, siguiendo su invitación, no repudia a su comprometida sino que la toma consigo y la desposa. Acogiendo a María, José acoge conscientemente y con amor a Aquel que ha sido concebido en ella por obra admirable de Dios, para quien nada es imposible. José, hombre humilde y justo (cfr v. 19), nos enseña a confiarnos siempre en Dios, que se nos acerca: cuando Dios se nos acerca debemos confiarnos. José nos enseña a dejarnos guiar por Él con voluntaria obediencia.

Estas dos figuras, \textbf{María y José}, que han sido los primeros en acoger a Jesús mediante la fe, nos introducen en el misterio de la Navidad. María nos ayuda a colocarnos en actitud de disponibilidad para acoger al Hijo de Dios en nuestra vida concreta, en nuestra carne. José nos insta a buscar siempre la voluntad de Dios y a seguirla con total confianza. \textquote{La Virgen concebirá y dará a luz un hijo a quien pondrás el nombre de Emanuel, que traducido significa: Dios-con-nosotros} (Mt 1,23). ). Así dice el ángel: \textquote{Emanuel se llamará el niño, que significa Dios-con-nosotros} o sea Dios cerca a nosotros. Y a Dios que se acerca yo le abro la puerta --al Señor-- cuando siento una inspiración interior, cuando siento que me pide hacer algo más por los demás, cuando me llama a la oración. Dios-con-nosotros, Dios que se acerca. Que este anuncio de esperanza, que se cumple en Navidad, lleve a cumplimiento la espera de Dios también en cada uno de nosotros, en toda la Iglesia, y en tantos pequeños que el mundo desprecia, pero que Dios ama y a los cuales se acerca.

\subsubsection{Ángelus: La fe inquebrantable de José}

Plaza de San Pedro. Domingo 22 de diciembre del 2019.

En este cuarto y último domingo de Adviento, el \textbf{Evangelio} (cf. \emph{Mateo} 1, 18-24) nos guía hacia la Navidad, a través de la experiencia de san \textbf{José}, una figura aparentemente de segundo plano, pero en cuya actitud está contenida toda la sabiduría cristiana. Él, junto con Juan Bautista y \textbf{María}, es uno de los personajes que la liturgia nos propone para el tiempo de Adviento; y de los tres es el más modesto. El que no predica, no habla, sino que trata de hacer la voluntad de Dios; y lo hace al estilo del Evangelio y de las Bienaventuranzas. Pensemos: \textquote{Bienaventurados los pobres de espíritu, porque de ellos es el Reino de los Cielos} (\emph{Mateo} 5, 3). Y José es pobre porque vive de lo esencial, trabaja, vive del trabajo; es la pobreza típica de quien es consciente de que depende en todo de Dios y pone en Él toda su confianza.

La narración del \textbf{Evangelio de hoy} presenta una situación humanamente incómoda y conflictiva. \textbf{José} y \textbf{María} están comprometidos; todavía no viven juntos, pero ella está esperando un hijo por obra de Dios. José, ante esta sorpresa, naturalmente permanece perturbado pero, en lugar de reaccionar de manera impulsiva y punitiva ―como era costumbre, la ley lo protegía― busca una solución que respete la dignidad y la integridad de su amada María. El Evangelio lo dice así: \textquote{Su marido José, como era justo y no quería ponerla en evidencia, resolvió repudiarla en secreto} (v. 19). José sabía que si denunciaba a su prometida, la expondría a graves consecuencias, incluso a la muerte. Tenía plena confianza en María, a quien eligió como su esposa. No entiende, pero busca otra solución.

Esta circunstancia inexplicable le llevó a cuestionar su compromiso; por eso, con gran sufrimiento, decidió separarse de María sin crear escándalo. Pero \textbf{el Ángel del Señor} interviene para decirle que la solución que él propone no es la deseada por Dios. Por el contrario, el Señor le abrió un nuevo camino, un camino de unión, de amor y de felicidad, y le dijo: \textquote{José, hijo de David, no temas tomar contigo a María tu mujer porque lo engendrado en ella es del Espíritu Santo} (v. 20).

En este punto, José confía totalmente en Dios, obedece las palabras del Ángel y se lleva a María con él. Fue precisamente esta confianza inquebrantable en Dios la que le permitió aceptar una situación humanamente difícil y, en cierto sentido, incomprensible. José entiende, en la fe, que el niño nacido en el seno de María no es su hijo, sino el Hijo de Dios, y él, José, será su guardián, asumiendo plenamente su paternidad terrenal. El ejemplo de este hombre gentil y sabio nos exhorta a levantar la vista, a mirar más allá. Se trata de recuperar la sorprendente lógica de Dios que, lejos de pequeños o grandes cálculos, está hecha de apertura hacia nuevos horizontes, hacia Cristo y Su Palabra.

Que la Virgen María y su casto esposo José nos ayuden a escuchar a Jesús que viene, y que pide ser acogido en nuestros planes y elecciones.

\section{Temas}

Maternidad virginal de María

CEC 496-507, 495:

\textbf{La virginidad de María}

\textbf{496} Desde las primeras formulaciones de la fe (cf. DS 10-64), la Iglesia ha confesado que Jesús fue concebido en el seno de la Virgen María únicamente por el poder del Espíritu Santo, afirmando también el aspecto corporal de este suceso: Jesús fue concebido \textquote{absque semine ex Spiritu Sancto} (Cc Letrán, año 649; DS 503), esto es, sin elemento humano, por obra del Espíritu Santo. Los Padres ven en la concepción virginal el signo de que es verdaderamente el Hijo de Dios el que ha venido en una humanidad como la nuestra:

Así, S. Ignacio de Antioquía (comienzos del siglo II):

\textquote{Estáis firmemente convencidos acerca de que nuestro Señor es verdaderamente de la raza de David según la carne (cf. Rm 1, 3), Hijo de Dios según la voluntad y el poder de Dios (cf. Jn 1, 13), nacido verdaderamente de una virgen, \ldots{} Fue verdaderamente clavado por nosotros en su carne bajo Poncio Pilato \ldots{} padeció verdaderamente, como también resucitó verdaderamente} (Smyrn. 1-2).

\textbf{497} Los relatos evangélicos (cf. Mt 1, 18-25; Lc 1, 26-38) presentan la concepción virginal como una obra divina que sobrepasa toda comprensión y toda posibilidad humanas (cf. Lc 1, 34): \textquote{Lo concebido en ella viene del Espíritu Santo}, dice el ángel a José a propósito de María, su desposada (Mt 1, 20). La Iglesia ve en ello el cumplimiento de la promesa divina hecha por el profeta Isaías: \textquote{He aquí que la virgen concebirá y dará a luz un Hijo} (Is 7, 14 según la traducción griega de Mt 1, 23).

\textbf{498} A veces ha desconcertado el silencio del Evangelio de S. Marcos y de las cartas del Nuevo Testamento sobre la concepción virginal de María. También se ha podido plantear si no se trataría en este caso de leyendas o de construcciones teológicas sin pretensiones históricas. A lo cual hay que responder: La fe en la concepción virginal de Jesús ha encontrado viva oposición, burlas o incomprensión por parte de los no creyentes, judíos y paganos (cf. S. Justino, Dial 99, 7; Orígenes, Cels. 1, 32, 69; entre otros); no ha tenido su origen en la mitología pagana ni en una adaptación de las ideas de su tiempo. El sentido de este misterio no es accesible más que a la fe que lo ve en ese \textquote{nexo que reúne entre sí los misterios} (DS 3016), dentro del conjunto de los Misterios de Cristo, desde su Encarnación hasta su Pascua. S. Ignacio de Antioquía da ya testimonio de este vínculo: \textquote{El príncipe de este mundo ignoró la virginidad de María y su parto, así como la muerte del Señor: tres misterios resonantes que se realizaron en el silencio de Dios} (Eph. 19, 1;cf. 1 Co 2, 8).\textbf{\\ }

\textbf{María, la \textquote{siempre Virgen}}

\textbf{499} La profundización de la fe en la maternidad virginal ha llevado a la Iglesia a confesar la virginidad real y perpetua de María (cf. DS 427) incluso en el parto del Hijo de Dios hecho hombre (cf. DS 291; 294; 442; 503; 571; 1880). En efecto, el nacimiento de Cristo \textquote{lejos de disminuir consagró la integridad virginal} de su madre (LG 57). La liturgia de la Iglesia celebra a María como la \textquote{Aeiparthenos}, la \textquote{siempre-virgen} (cf. LG 52).

\textbf{500} A esto se objeta a veces que la Escritura menciona unos hermanos y hermanas de Jesús (cf. Mc 3, 31-55; 6, 3; 1 Co 9, 5; Ga 1, 19). La Iglesia siempre ha entendido estos pasajes como no referidos a otros hijos de la Virgen María; en efecto, Santiago y José \textquote{hermanos de Jesús} (Mt 13, 55) son los hijos de una María discípula de Cristo (cf. Mt 27, 56) que se designa de manera significativa como \textquote{la otra María} (Mt 28, 1). Se trata de parientes próximos de Jesús, según una expresión conocida del Antiguo Testamento (cf. Gn 13, 8; 14, 16;29, 15; etc.).

\textbf{501} Jesús es el Hijo único de María. Pero la maternidad espiritual de María se extiende (cf. Jn 19, 26-27; Ap 12, 17) a todos los hombres a los cuales, El vino a salvar: \textquote{Dio a luz al Hijo, al que Dios constituyó el mayor de muchos hermanos (Rom 8,29), es decir, de los creyentes, a cuyo nacimiento y educación colabora con amor de madre} (LG 63).

\textbf{La maternidad virginal de María en el designio de Dios}

\textbf{502} La mirada de la fe, unida al conjunto de la Revelación, puede descubrir las razones misteriosas por las que Dios, en su designio salvífico, quiso que su Hijo naciera de una virgen. Estas razones se refieren tanto a la persona y a la misión redentora de Cristo como a la aceptación por María de esta misión para con los hombres.

\textbf{503} La virginidad de María manifiesta la iniciativa absoluta de Dios en la Encarnación. Jesús no tiene como Padre más que a Dios (cf. Lc 2, 48-49). \textquote{La naturaleza humana que ha tomado no le ha alejado jamás de su Padre \ldots{}; consubstancial con su Padre en la divinidad, consubstancial con su Madre en nuestras humanidad, pero propiamente Hijo de Dios en sus dos naturalezas} (Cc. Friul en el año 796: DS 619).

\textbf{504} Jesús fue concebido por obra del Espíritu Santo en el seno de la Virgen María porque El es el \emph{Nuevo Adán} (cf. 1 Co 15, 45) que inaugura la nueva creación: \textquote{El primer hombre, salido de la tierra, es terreno; el segundo viene del cielo} (1 Co 15, 47). La humanidad de Cristo, desde su concepción, está llena del Espíritu Santo porque Dios \textquote{le da el Espíritu sin medida} (Jn 3, 34). De \textquote{su plenitud}, cabeza de la humanidad redimida (cf. Col 1, 18), \textquote{hemos recibido todos gracia por gracia} (Jn 1, 16).

\textbf{505} Jesús, el nuevo Adán, inaugura por su concepción virginal el \emph{nuevo nacimiento} de los hijos de adopción en el Espíritu Santo por la fe \textquote{¿Cómo será eso?} (Lc 1, 34;cf. Jn 3, 9). La participación en la vida divina no nace \textquote{de la sangre, ni de deseo de carne, ni de deseo de hombre, sino de Dios} (Jn 1, 13). La acogida de esta vida es virginal porque toda ella es dada al hombre por el Espíritu. El sentido esponsal de la vocación humana con relación a Dios (cf. 2 Co 11, 2) se lleva a cabo perfectamente en la maternidad virginal de María.

\textbf{506} María es virgen porque su virginidad es \emph{el signo de su fe} \textquote{no adulterada por duda alguna} (LG 63) y de su entrega total a la voluntad de Dios (cf. 1 Co 7, 34-35). Su fe es la que le hace llegar a ser la madre del Salvador: \textquote{Beatior est Maria percipiendo fidem Christi quam concipiendo carnem Christi} (\textquote{Más bienaventurada es María al recibir a Cristo por la fe que al concebir en su seno la carne de Cristo} (S. Agustín, virg. 3).

\textbf{507} María es a la vez virgen y madre porque ella es la figura y la más perfecta realización de la Iglesia (cf. LG 63): \textquote{La Iglesia se convierte en Madre por la palabra de Dios acogida con fe, ya que, por la predicación y el bautismo, engendra para una vida nueva e inmortal a los hijos concebidos por el Espíritu Santo y nacidos de Dios. También ella es virgen que guarda íntegra y pura la fidelidad prometida al Esposo} (LG 64).

\textbf{La maternidad divina de María}

\textbf{495} Llamada en los Evangelios \textquote{la Madre de Jesús} (Jn 2, 1; 19, 25; cf. Mt 13, 55, etc.), María es aclamada bajo el impulso del Espíritu como \textquote{la madre de mi Señor} desde antes del nacimiento de su hijo (cf. Lc 1, 43). En efecto, aquél que ella concibió como hombre, por obra del Espíritu Santo, y que se ha hecho verdaderamente su Hijo según la carne, no es otro que el Hijo eterno del Padre, la segunda persona de la Santísima Trinidad. La Iglesia confiesa que María es verdaderamente \emph{Madre de Dios} {[}\textquote{Theotokos}{]} (cf. DS 251).

María, madre de Dios por obra del Espíritu Santo

CEC 437, 456, 484-486, 721-726:

\textbf{437} El ángel anunció a los pastores el nacimiento de Jesús como el del Mesías prometido a Israel: \textquote{Os ha nacido hoy, en la ciudad de David, un salvador, que es el Cristo Señor} (\emph{Lc} 2, 11). Desde el principio él es \textquote{a quien el Padre ha santificado y enviado al mundo} (\emph{Jn} 10, 36), concebido como \textquote{santo} (\emph{Lc} 1, 35) en el seno virginal de María. José fue llamado por Dios para \textquote{tomar consigo a María su esposa} encinta \textquote{del que fue engendrado en ella por el Espíritu Santo} (\emph{Mt} 1, 20) para que Jesús \textquote{llamado Cristo} nazca de la esposa de José en la descendencia mesiánica de David (\emph{Mt} 1, 16; cf. \emph{Rm} 1, 3; \emph{2 Tm} 2, 8; \emph{Ap} 22, 16).

\textbf{\\ }

\textbf{Por qué el Verbo se hizo carne}

\textbf{456} Con el Credo Niceno-Constantinopolitano respondemos confesando: \textquote{\emph{Por nosotros los hombres y por nuestra salvación} bajó del cielo, y por obra del Espíritu Santo se encarnó de María la Virgen y se hizo hombre} (DS 150).

\textbf{457} El Verbo se encarnó \emph{para salvarnos reconciliándonos con Dios}: \textquote{Dios nos amó y nos envió a su Hijo como propiciación por nuestros pecados} (\emph{1 Jn} 4, 10). \textquote{El Padre envió a su Hijo para ser salvador del mundo} (\emph{1 Jn} 4, 14). \textquote{Él se manifestó para quitar los pecados} (\emph{1 Jn} 3, 5):

\begin{quote} \textquote{Nuestra naturaleza enferma exigía ser sanada; desgarrada, ser restablecida; muerta, ser resucitada. Habíamos perdido la posesión del bien, era necesario que se nos devolviera. Encerrados en las tinieblas, hacía falta que nos llegara la luz; estando cautivos, esperábamos un salvador; prisioneros, un socorro; esclavos, un libertador. ¿No tenían importancia estos razonamientos? ¿No merecían conmover a Dios hasta el punto de hacerle bajar hasta nuestra naturaleza humana para visitarla, ya que la humanidad se encontraba en un estado tan miserable y tan desgraciado?} (San Gregorio de Nisa, \emph{Oratio catechetica}, 15: PG 45, 48B). \end{quote}

\textbf{Concebido por obra y gracia del Espíritu Santo \ldots{}}

\textbf{484} La Anunciación a María inaugura \textquote{la plenitud de los tiempos} (\emph{Ga} 4, 4), es decir, el cumplimiento de las promesas y de los preparativos. María es invitada a concebir a aquel en quien habitará \textquote{corporalmente la plenitud de la divinidad} (\emph{Col} 2, 9). La respuesta divina a su \textquote{¿cómo será esto, puesto que no conozco varón?} (\emph{Lc} 1, 34) se dio mediante el poder del Espíritu: \textquote{El Espíritu Santo vendrá sobre ti} (\emph{Lc} 1, 35).

\textbf{485} La misión del Espíritu Santo está siempre unida y ordenada a la del Hijo (cf. \emph{Jn} 16, 14-15). El Espíritu Santo fue enviado para santificar el seno de la Virgen María y fecundarla por obra divina, él que es \textquote{el Señor que da la vida}, haciendo que ella conciba al Hijo eterno del Padre en una humanidad tomada de la suya.

\textbf{486} El Hijo único del Padre, al ser concebido como hombre en el seno de la Virgen María es \textquote{Cristo}, es decir, el ungido por el Espíritu Santo (cf. \emph{Mt} 1, 20; \emph{Lc} 1, 35), desde el principio de su existencia humana, aunque su manifestación no tuviera lugar sino progresivamente: a los pastores (cf. \emph{Lc} 2,8-20), a los magos (cf. \emph{Mt} 2, 1-12), a Juan Bautista (cf. \emph{Jn} 1, 31-34), a los discípulos (cf. \emph{Jn} 2, 11). Por tanto, toda la vida de Jesucristo manifestará \textquote{cómo Dios le ungió con el Espíritu Santo y con poder} (\emph{Hch} 10, 38).

\textbf{\textquote{Alégrate, llena de gracia}}

\textbf{721} María, la Santísima Madre de Dios, la siempre Virgen, es la obra maestra de la Misión del Hijo y del Espíritu Santo en la Plenitud de los tiempos. Por primera vez en el designio de Salvación y porque su Espíritu la ha preparado, el Padre encuentra la Morada en donde su Hijo y su Espíritu pueden habitar entre los hombres. Por ello, los más bellos textos sobre la Sabiduría, la Tradición de la Iglesia los ha entendido frecuentemente con relación a María (cf. \emph{Pr} 8, 1-9, 6; \emph{Si} 24): María es cantada y representada en la Liturgia como el \textquote{Trono de la Sabiduría}.

En ella comienzan a manifestarse las \textquote{maravillas de Dios}, que el Espíritu va a realizar en Cristo y en la Iglesia:

\textbf{722} El Espíritu Santo \emph{preparó} a María con su gracia. Convenía que fuese \textquote{llena de gracia} la Madre de Aquel en quien \textquote{reside toda la plenitud de la divinidad corporalmente} (\emph{Col} 2, 9). Ella fue concebida sin pecado, por pura gracia, como la más humilde de todas las criaturas, la más capaz de acoger el don inefable del Omnipotente. Con justa razón, el ángel Gabriel la saluda como la \textquote{Hija de Sión}: \textquote{Alégrate} (cf. \emph{So} 3, 14; \emph{Za} 2, 14). Cuando ella lleva en sí al Hijo eterno, hace subir hasta el cielo con su cántico al Padre, en el Espíritu Santo, la acción de gracias de todo el pueblo de Dios y, por tanto, de la Iglesia (cf. \emph{Lc} 1, 46-55).

\textbf{723} En María el Espíritu Santo \emph{realiza} el designio benevolente del Padre. La Virgen concibe y da a luz al Hijo de Dios por obra del Espíritu Santo. Su virginidad se convierte en fecundidad única por medio del poder del Espíritu y de la fe (cf. \emph{Lc} 1, 26-38; \emph{Rm} 4, 18-21; \emph{Ga} 4, 26-28).

\textbf{724} En María, el Espíritu Santo \emph{manifiesta} al Hijo del Padre hecho Hijo de la Virgen. Ella es la zarza ardiente de la teofanía definitiva: llena del Espíritu Santo, presenta al Verbo en la humildad de su carne dándolo a conocer a los pobres (cf. \emph{Lc} 2, 15-19) y a las primicias de las naciones (cf. \emph{Mt} 2, 11).

\textbf{725} En fin, por medio de María, el Espíritu Santo comienza a \emph{poner en comunión} con Cristo a los hombres \textquote{objeto del amor benevolente de Dios} (cf. \emph{Lc} 2, 14), y los humildes son siempre los primeros en recibirle: los pastores, los magos, Simeón y Ana, los esposos de Caná y los primeros discípulos.

\textbf{726} Al término de esta misión del Espíritu, María se convierte en la \textquote{Mujer}, nueva Eva \textquote{madre de los vivientes}, Madre del \textquote{Cristo total} (cf. \emph{Jn} 19, 25-27). Así es como ella está presente con los Doce, que \textquote{perseveraban en la oración, con un mismo espíritu} (\emph{Hch} 1, 14), en el amanecer de los \textquote{últimos tiempos} que el Espíritu va a inaugurar en la mañana de Pentecostés con la manifestación de la Iglesia.

Jesús viene revelado como Salvador a José

CEC 1846:

\textbf{1846} El Evangelio es la revelación, en Jesucristo, de la misericordia de Dios con los pecadores (cf. \emph{Lc} 15). El ángel anuncia a José: \textquote{Tú le pondrás por nombre Jesús, porque él salvará a su pueblo de sus pecados} (\emph{Mt} 1, 21). Y en la institución de la Eucaristía, sacramento de la redención, Jesús dice: \textquote{Esta es mi sangre de la alianza, que va a ser derramada por muchos para remisión de los pecados} (\emph{Mt} 26, 28).

Cristo, el Hijo de Dios en su Resurrección

CEC 445, 648, 695:

\textbf{445} Después de su Resurrección, su filiación divina aparece en el poder de su humanidad glorificada: \textquote{Constituido Hijo de Dios con poder, según el Espíritu de santidad, por su Resurrección de entre los muertos} (\emph{Rm} 1, 4; cf. \emph{Hch} 13, 33). Los apóstoles podrán confesar \textquote{Hemos visto su gloria, gloria que recibe del Padre como Hijo único, lleno de gracia y de verdad } (\emph{Jn} 1, 14).

\textbf{La Resurrección obra de la Santísima Trinidad}

\textbf{648} La Resurrección de Cristo es objeto de fe en cuanto es una intervención transcendente de Dios mismo en la creación y en la historia. En ella, las tres Personas divinas actúan juntas a la vez y manifiestan su propia originalidad. Se realiza por el poder del Padre que \textquote{ha resucitado} (\emph{Hch} 2, 24) a Cristo, su Hijo, y de este modo ha introducido de manera perfecta su humanidad ---con su cuerpo--- en la Trinidad. Jesús se revela definitivamente \textquote{Hijo de Dios con poder, según el Espíritu de santidad, por su resurrección de entre los muertos} (\emph{Rm} 1, 3-4). San Pablo insiste en la manifestación del poder de Dios (cf. \emph{Rm} 6, 4; 2 Co 13, 4; \emph{Flp} 3, 10; \emph{Ef} 1, 19-22; \emph{Hb} 7, 16) por la acción del Espíritu que ha vivificado la humanidad muerta de Jesús y la ha llamado al estado glorioso de Señor.

\textbf{Los símbolos del Espíritu Santo}

\textbf{695} \emph{La unción}. El simbolismo de la unción con el óleo es también significativo del Espíritu Santo, hasta el punto de que se ha convertido en sinónimo suyo (cf. \emph{1 Jn} 2, 20. 27; \emph{2 Co} 1, 21). En la iniciación cristiana es el signo sacramental de la Confirmación, llamada justamente en las Iglesias de Oriente \textquote{Crismación}. Pero para captar toda la fuerza que tiene, es necesario volver a la Unción primera realizada por el Espíritu Santo: la de Jesús. Cristo {[}\textquote{Mesías} en hebreo{]} significa \textquote{Ungido} del Espíritu de Dios. En la Antigua Alianza hubo \textquote{ungidos} del Señor (cf. \emph{Ex} 30, 22-32), de forma eminente el rey David (cf. \emph{1 S} 16, 13). Pero Jesús es el Ungido de Dios de una manera única: la humanidad que el Hijo asume está totalmente \textquote{ungida por el Espíritu Santo}. Jesús es constituido \textquote{Cristo} por el Espíritu Santo (cf. \emph{Lc} 4, 18-19; \emph{Is} 61, 1).

La Virgen María concibe a Cristo del Espíritu Santo, quien por medio del ángel lo anuncia como Cristo en su nacimiento (cf. \emph{Lc} 2,11) e impulsa a Simeón a ir al Templo a ver al Cristo del Señor (cf. \emph{Lc} 2, 26-27); es de quien Cristo está lleno (cf. \emph{Lc} 4, 1) y cuyo poder emana de Cristo en sus curaciones y en sus acciones salvíficas (cf. \emph{Lc} 6, 19; 8, 46). Es él en fin quien resucita a Jesús de entre los muertos (cf. \emph{Rm} 1, 4; 8, 11). Por tanto, constituido plenamente \textquote{Cristo} en su humanidad victoriosa de la muerte (cf. \emph{Hch} 2, 36), Jesús distribuye profusamente el Espíritu Santo hasta que \textquote{los santos} constituyan, en su unión con la humanidad del Hijo de Dios, \textquote{ese Hombre perfecto [\ldots{}] que realiza la plenitud de Cristo} (\emph{Ef} 4, 13): \textquote{el Cristo total} según la expresión de San Agustín (\emph{Sermo} 341, 1, 1: PL 39, 1493; Ibíd., 9, 11: PL 39, 1499).

\textquote{La obediencia de la fe}

CEC 143-149, 494, 2087:

\textbf{143} \emph{Por la fe}, el hombre somete completamente su inteligencia y su voluntad a Dios. Con todo su ser, el hombre da su asentimiento a Dios que revela (cf. DV 5). La sagrada Escritura llama \textquote{obediencia de la fe} a esta respuesta del hombre a Dios que revela (cf. \emph{Rm} 1,5; 16,26).

\textbf{La obediencia de la fe. Abraham, \textquote{padre de todos los creyentes}}

\textbf{144} Obedecer (\emph{ob-audire}) en la fe es someterse libremente a la palabra escuchada, porque su verdad está garantizada por Dios, la Verdad misma. De esta obediencia, Abraham es el modelo que nos propone la Sagrada Escritura. La Virgen María es la realización más perfecta de la misma.

\textbf{145} La carta a los Hebreos, en el gran elogio de la fe de los antepasados, insiste particularmente en la fe de Abraham: \textquote{Por la fe, Abraham obedeció y salió para el lugar que había de recibir en herencia, y salió sin saber a dónde iba} (\emph{Hb} 11,8; cf. \emph{Gn} 12,1-4). Por la fe, vivió como extranjero y peregrino en la Tierra prometida (cf. \emph{Gn} 23,4). Por la fe, a Sara se le otorgó el concebir al hijo de la promesa. Por la fe, finalmente, Abraham ofreció a su hijo único en sacrificio (cf. \emph{Hb} 11,17).

\textbf{146} Abraham realiza así la definición de la fe dada por la carta a los Hebreos: \textquote{La fe es garantía de lo que se espera; la prueba de las realidades que no se ven} (\emph{Hb} 11,1). \textquote{Creyó Abraham en Dios y le fue reputado como justicia} (\emph{Rm} 4,3; cf. \emph{Gn} 15,6). Y por eso, fortalecido por su fe, Abraham fue hecho \textquote{padre de todos los creyentes} (\emph{Rm} 4,11.18; cf. \emph{Gn} 15, 5).

\textbf{147} El Antiguo Testamento es rico en testimonios acerca de esta fe. La carta a los Hebreos proclama el elogio de la fe ejemplar por la que los antiguos \textquote{fueron alabados} (\emph{Hb} 11, 2.39). Sin embargo, \textquote{Dios tenía ya dispuesto algo mejor}: la gracia de creer en su Hijo Jesús, \textquote{el que inicia y consuma la fe} (\emph{Hb} 11,40; 12,2).

\textbf{María : \textquote{Dichosa la que ha creído}}

\textbf{148} La Virgen María realiza de la manera más perfecta la obediencia de la fe. En la fe, María acogió el anuncio y la promesa que le traía el ángel Gabriel, creyendo que \textquote{nada es imposible para Dios} (\emph{Lc} 1,37; cf. \emph{Gn} 18,14) y dando su asentimiento: \textquote{He aquí la esclava del Señor; hágase en mí según tu palabra} (\emph{Lc} 1,38). Isabel la saludó: \textquote{¡Dichosa la que ha creído que se cumplirían las cosas que le fueron dichas de parte del Señor!} (\emph{Lc} 1,45). Por esta fe todas las generaciones la proclamarán bienaventurada (cf. \emph{Lc} 1,48).

\textbf{149} Durante toda su vida, y hasta su última prueba (cf. \emph{Lc} 2,35), cuando Jesús, su hijo, murió en la cruz, su fe no vaciló. María no cesó de creer en el \textquote{cumplimiento} de la palabra de Dios. Por todo ello, la Iglesia venera en María la realización más pura de la fe.

\textbf{\textquote{Hágase en mí según tu palabra \ldots{}}}

\textbf{494} Al anuncio de que ella dará a luz al \textquote{Hijo del Altísimo} sin conocer varón, por la virtud del Espíritu Santo (cf. Lc 1, 28-37), María respondió por \textquote{la obediencia de la fe} (Rm 1, 5), segura de que \textquote{nada hay imposible para Dios}: \textquote{He aquí la esclava del Señor: hágase en mí según tu palabra} (Lc 1, 37-38). Así dando su consentimiento a la palabra de Dios, María llegó a ser Madre de Jesús y, aceptando de todo corazón la voluntad divina de salvación, sin que ningún pecado se lo impidiera, se entregó a sí misma por entero a la persona y a la obra de su Hijo, para servir, en su dependencia y con él, por la gracia de Dios, al Misterio de la Redención (cf. LG 56):

\begin{quote} Ella, en efecto, como dice S. Ireneo, \textquote{por su obediencia fue causa de la salvación propia y de la de todo el género humano}. Por eso, no pocos Padres antiguos, en su predicación, coincidieron con él en afirmar \textquote{el nudo de la desobediencia de Eva lo desató la obediencia de María. Lo que ató la virgen Eva por su falta de fe lo desató la Virgen María por su fe}. Comparándola con Eva, llaman a María `Madre de los vivientes' y afirman con mayor frecuencia: \textquote{la muerte vino por Eva, la vida por María}. (LG. 56). \end{quote}

\textbf{La fe}

\textbf{2087} Nuestra vida moral tiene su fuente en la fe en Dios que nos revela su amor. San Pablo habla de la \textquote{obediencia de la fe} (\emph{Rm} 1, 5; 16, 26) como de la primera obligación. Hace ver en el \textquote{desconocimiento de Dios} el principio y la explicación de todas las desviaciones morales (cf. \emph{Rm} 1, 18-32). Nuestro deber para con Dios es creer en Él y dar testimonio de Él.


\part{Tiempo de Navidad}

\chapter{Introducción}

\textbf{Normativa litúrgica}\footnote{Cf. NUALC nn. 32-38; OLM n. 95}.

En la Iglesia, la celebración más antigua después de la del Misterio Pascual es la memoria del Nacimiento del Señor y sus primeras manifestaciones, que se realiza en el tiempo de Navidad.

El tiempo de Navidad va desde las primeras vísperas de Navidad hasta el domingo después de Epifanía, o después del 6 de enero, inclusive.

La Misa de la vigilia de Navidad se celebra en la tarde del 24 de diciembre, antes o después de las primeras vísperas.

El día de Navidad se pueden celebrar tres Misas, según una antigua tradición de la Iglesia Romana, o sea en la noche, en la aurora y en el día.

El día de Navidad tiene su octava propia dispuesta de la siguiente manera:

a) Domingo dentro de la octava, o en su defecto, el día 30 de diciembre: fiesta de la Sagrada Familia.

b) El 26 de diciembre: fiesta de san Esteban, el primer mártir.

c) El 27 de diciembre: fiesta de san Juan, apóstol y evangelista.

d) El 28 de diciembre: fiesta de los santos Inocentes.

e) El 29, 30, 31 de diciembre son días \textquote{dentro de la octava}.

f) El 1 de enero, octava de Navidad: solemnidad de santa María Madre de Dios, en que se conmemora también la imposición del santo Nombre de Jesús.

El domingo entre el 2 y 5 de enero se llama Domingo 2° después de Navidad.

La Epifanía del Señor se celebra el 6 de enero, a no ser que se transfiera -donde no es de precepto- al domingo situado entre el 2 y el 8 de enero (cf. n. 7).

La fiesta del Bautismo del Señor se celebra el domingo siguiente al 6 de enero.

En la vigilia y en las tres misas de Navidad, las lecturas, tanto las proféticas como las demás, se han tomado de la tradición romana.

El domingo dentro de la Octava de Navidad, fiesta de la Sagrada Familia, el Evangelio es de la infancia de Jesús, las demás lecturas hablan de las virtudes de la vida doméstica.

En la Octava de Navidad y solemnidad de santa María, Madre de Dios, las lecturas tratan de la Virgen, Madre de Dios, y de la imposición del santísimo nombre de Jesús.

El segundo domingo después de Navidad, las lecturas tratan del misterio de la encarnación.

En la Epifanía del Señor, la lectura del Antiguo Testamento y el Evangelio conservan la tradición romana; en la lectura apostólica se lee un texto relativo a la vocación de los paganos a la salvación.

En la fiesta del Bautismo del Señor, los textos se refieren a este misterio.

\textbf{Las celebraciones de la Navidad}\footnote{Cf. Congregación para el Culto Divino, \emph{Directorio Homilético} (2014), nn. 110-119}.

\textquote{En la vigilia y en las tres Misas de Navidad, las lecturas, tanto las proféticas como las demás, se han tomado de la tradición Romana} (OLM 95). Un momento distintivo de la Solemnidad de la Navidad del Señor es la costumbre de celebrar tres misas diferentes: la de medianoche, la de la aurora y la del día. Con la reforma posterior al Concilio Vaticano II se ha añadido una vespertina en la vigilia. A excepción de las comunidades monásticas, no es normal que todos participen en las tres (o cuatro) celebraciones; la mayor parte de los fieles participará en una Liturgia que será su \textquote{Misa de Navidad}. Por ello se ha llevado a cabo una selección de lecturas para cada celebración. No obstante, antes de considerar algunos temas integrales y comunes a los textos litúrgicos y bíblicos, resulta ilustrativo examinar la secuencia de las cuatro misas.

La Navidad es la fiesta de la luz. Es opinión difundida que la celebración del Nacimiento del Señor se fijó a finales de diciembre para dar un valor cristiano a la fiesta pagana del \emph{Sol invictus}. Aunque podría también no ser así. Si ya en la primera parte del siglo III, Tertuliano escribió que en algunos calendarios Cristo fue concebido el 25 de marzo, día que se considera como el primero del año, es posible que la fiesta de la Navidad haya sido calculada a partir de esta fecha. En todo caso, ya desde el siglo IV, muchos Padres reconocen el valor simbólico del hecho de que los días se alargan después de la Fiesta de la Navidad. Las fiestas paganas que exaltan la luz en la oscuridad del invierno no eran extrañas, y las fiestas invernales de la luz aún hoy son celebradas en algunos lugares por los no creyentes. A diferencia de ello, las lecturas y las oraciones de las diversas Liturgias natalicias evidencian el tema de la verdadera Luz que viene a nosotros en Jesucristo. El primer prefacio de Navidad exclama, dirigiéndose a Dios Padre: \textquote{Porque gracias al misterio de la Palabra hecha carne, la luz de tu gloria brilló ante nuestros ojos con nuevo resplandor}. El homileta debería acentuar esta dinámica de la luz en las tinieblas, que inunda estos días gozosos. Presentamos a continuación una síntesis de las características de cada Celebración.

\textbf{La Misa vespertina de la Vigilia}. Aunque la celebración de la Navidad comienza con esta Misa, las oraciones y las lecturas evocan aún un sentido de temblorosa espera; en cierto sentido, esta misa es una síntesis de todo el Tiempo de Adviento. Casi todas las oraciones están conjugadas en futuro: \textquote{Mañana contemplaréis su gloria} (antífona de entrada); \textquote{Concédenos que así como ahora acogemos, gozosos, a tu Hijo como Redentor, lo recibamos también confiados cuando venga como juez} (colecta); \textquote{Mañana quedará borrada la bondad de la tierra} (canto al Evangelio); \textquote{Concédenos, Señor, empezar estas fiestas de Navidad con una entrega digna del santo misterio del nacimiento de tu Hijo en el que has instaurado el principio de nuestra salvación} (oración sobre las ofrendas); \textquote{Se revelará la gloria del Señor} (antífona de comunión). Las lecturas de Isaías en las otras Misas de Navidad describen lo que \emph{está} sucediendo, mientras que el pasaje proclamado en esta Misa cuenta lo que \emph{sucederá}. La segunda lectura y el pasaje evangélico hablan de Jesús como el Hijo de David y de los antepasados humanos que han preparado el camino para su venida. La genealogía del Evangelio de san Mateo, describiendo a grandes rasgos el largo camino de la Historia de la Salvación que conduce al acontecimiento que vamos a celebrar, es similar a las lecturas del Antiguo Testamento de la Vigila Pascual. La letanía de nombres aumenta la sensación de espera. En la Misa de la Vigilia somos un poco como los niños que agarran con fuerza el regalo de Navidad, esperando la palabra que les permita abrirlo.

\textbf{La Misa de medianoche}. En el corazón de la noche, mientras el resto del mundo duerme, los cristianos abren este regalo: el don del Verbo hecho carne. El profeta Isaías anuncia: \textquote{El pueblo que caminaba en tinieblas vio una luz grande}. Continúa refiriéndose a la gloriosa victoria del héroe conquistador que ha quebrantado la vara del opresor y ha tirado al fuego los instrumentos de guerra. Anuncia que el dominio de aquel que reinará será dilatado y con una paz sin límites y, por último, le llena de títulos: \textquote{Maravilla de Consejero, Dios guerrero, Padre perpetuo, Príncipe de la Paz}. El comienzo del Evangelio resalta la eminencia de tal dignatario, mencionando por su nombre al emperador y al gobernador que reinaban cuando Él irrumpe en escena. La narración prosigue con una revelación impresionante: este rey potente ha nacido en un modesto pueblecito de las fronteras del Imperio Romano y su madre \textquote{lo envolvió en pañales y lo acostó en un pesebre, porque no tenían sitio en la posada}. El contraste entre el héroe conquistador descrito por Isaías y el niño indefenso en el establo nos trae a la mente todas las paradojas del Evangelio. El conocimiento de estas paradojas está profundamente arraigado en el corazón de los fieles y los atrae a la Iglesia en el corazón de la noche. La respuesta apropiada es unir nuestro agradecimiento al de los ángeles, cuyo canto resuena en los cielos en esta noche.

\textbf{La Misa de la Aurora}. Las lecturas propuestas para esta Celebración son particularmente concisas. Somos como aquellos que se despertaron en la gélida luz del alba, preguntándose si la aparición angélica en medio de la noche había sido un sueño. Los pastores, con ese innato buen sentido propio de los pobres, piensan entre sí: \textquote{Vamos derechos a Belén, a ver eso que ha pasado y que nos ha comunicado el Señor}. Van corriendo y encuentran exactamente lo que les había anunciado el Ángel: una pobre pareja y su Hijo apenas recién nacido, dormido en un pesebre para los animales. ¿Su reacción a esta escena de humilde pobreza? Vuelven glorificando y alabando a Dios por lo que han visto y oído, y todos los que los escuchan quedan impresionados por lo que les han referido. Los pastores vieron, y también nosotros estamos invitados a ver, algo mucho más trascendente que la escena que nos llena de emoción y que ha sido objeto de tantas representaciones artísticas. Pero esta realidad se puede ver sólo con los ojos de la fe y emerge con la luz del día, en la siguiente Celebración.

\textbf{La Misa del día}. Como un sol resplandeciente ya en lo alto del cielo, el Prólogo del Evangelio de san Juan aclara la identidad del niño del pesebre. El evangelista afirma: \textquote{Y la Palabra se hizo carne, y acampó entre nosotros, y hemos contemplado su gloria: gloria propia del Hijo único del Padre, lleno de gracia y de verdad}. Con anterioridad, como recuerda la segunda lectura, Dios había hablado de muchas maneras por medio de los profetas; pero ahora \textquote{en esta etapa final, nos ha hablado por el Hijo, al que ha nombrado heredero de todo, y por medio del cual ha ido realizando las edades del mundo. Él es reflejo de su gloria \ldots{}} Esta es su grandeza, por la que lo adoran los mismos ángeles. Y aquí está la invitación para que todos se unan a ellos: \textquote{adorad al Señor, porque hoy una gran luz ha bajado a la tierra} (canto al evangelio).

El Verbo se hace carne para redimirnos, gracias a su Sangre derramada, y ensalzarnos con él a la gloria de la Resurrección. Los primeros discípulos reconocieron la relación íntima entre la Encarnación y el Misterio Pascual, como testimonia el himno citado en la carta de san Pablo a los Filipenses (2,5-11). La luz de la Misa de medianoche es la misma luz de la Vigilia Pascual. Las colectas de estas dos grandes Solemnidades comienzan con términos muy similares. En Navidad, el sacerdote dice: \textquote{Oh Dios, que has iluminado esta noche santa con el nacimiento de Cristo, la luz verdadera \ldots{}}; en Pascua: \textquote{Oh Dios, que iluminas esta noche santa con la gloria de la Resurrección del Señor \ldots{}}. La segunda lectura de la Misa de la aurora propone una síntesis admirable de la revelación del Misterio de la Trinidad y de nuestra introducción al mismo a través del Bautismo: \textquote{Cuando se apareció la Bondad de Dios, nuestro Salvador, y su Amor al hombre, \ldots{} sino que según su propia misericordia nos ha salvado: con el baño del segundo nacimiento, y con la renovación por el Espíritu Santo; Dios lo derramó copiosamente sobre nosotros por medio de Jesucristo nuestro Salvador. Así, justificados por su gracia, somos, en esperanza, herederos de la vida eterna}. Las oraciones propias de la Misa del día hablan de Cristo como autor de nuestra generación divina y de cómo su nacimiento manifiesta la reconciliación que nos hace amables a los ojos de Dios. La colecta, una de las más antiguas del tesoro de las oraciones de la Iglesia, expresa sintéticamente \emph{porqué} el Verbo se hace carne: \textquote{Oh Dios, que de modo admirable has creado al hombre a tu imagen y semejanza; y de modo más admirable todavía restableciste su dignidad por Jesucristo; concédenos compartir la vida divina de aquél que hoy se ha dignado compartir con el hombre la condición humana}. Una de las finalidades fundamentales de la homilía es, como afirma el presente Directorio, la de anunciar el Misterio Pascual de Cristo. Los textos de la Navidad ofrecen explícitas oportunidades para hacerlo.

Otra finalidad de la homilía es la de conducir a la comunidad hacia el Sacrificio Eucarístico, en el que el misterio Pascual se hace presente. Es un indicador claro la palabra \textquote{hoy}, a la que recurren con frecuencia los textos litúrgicos de las Misas de Navidad. El Misterio del Nacimiento de Cristo está presente en esta celebración, pero como en su primera venida, solo puede ser percibido con la mirada de la fe. Para los pastores el gran \textquote{signo} fue, simplemente, un pobre niño clocado en el pesebre, aunque en su recuerdo glorificaban y alababan a Dios por lo que habían visto. Con la mirada de la fe tenemos que percibir al mismo Cristo, nacido hoy, bajo los signos del pan y del vino. El \emph{admirabile commercium} del que nos habla la colecta del día de Navidad, según la cual Cristo comparte nuestra humanidad y nosotros su divinidad, se manifiesta de modo particular en la Eucaristía, como sugieren las oraciones de la celebración. En la media noche rezamos así en la oración sobre las ofrendas: \textquote{Acepta, Señor, nuestras ofrendas en esta noche santa, y por este intercambio de dones en el que nos muestras tu divina largueza, haznos partícipes de la divinidad de tu Hijo que, al asumir la naturaleza humana, nos ha unido a la tuya de modo admirable}. Y en la de la aurora: \textquote{Señor, que estas ofrendas sean signo del Misterio de Navidad que estamos celebrando; y así como tu Hijo, hecho hombre, se manifestó como Dios, así nuestras ofrendas de la tierra nos hagan partícipes de los dones del cielo}. Y también, en el prefacio III de Navidad: \textquote{Por él, hoy resplandece ante el mundo el maravilloso intercambio que nos salva: pues al revestirse tu Hijo de nuestra frágil condición no sólo confiere dignidad eterna a la naturaleza humana, sino que por esta unión admirable nos hace a nosotros eternos}.

La referencia a la inmortalidad roza otro tema recurrente en los textos de Navidad: la celebración es sólo una parada momentánea en nuestra peregrinación. El mensaje escatológico, tan evidente en el tiempo de Adviento, también encuentra aquí su expresión. En la colecta de la Vigilia, rezamos: \textquote{\ldots{} que cada año nos alegras con la fiesta esperanzadora de nuestra redención; concédenos que así como ahora acogemos, gozosos, a tu Hijo como Redentor, lo recibamos también confiados cuando venga como juez}. En la segunda lectura de la Misa de medianoche, el Apóstol nos exhorta \textquote{a renunciar a la vida sin religión y a los deseos mundanos, y a llevar ya desde ahora una vida sobria, honrada y religiosa, aguardando la dicha que esperamos: la aparición gloriosa del gran Dios y Salvador nuestro Jesucristo}. Y por último, en la oración después de la comunión de la Misa del día, pedimos que Cristo, autor de nuestra generación divina, nacido en este día, \textquote{nos haga igualmente partícipes del don de su inmortalidad}.

Las lecturas y las oraciones de Navidad ofrecen un rico alimento al pueblo de Dios peregrino en esta vida; revelando a Cristo como Luz del mundo, nos invitan a sumergirnos en el Misterio Pascual de nuestra redención a través del \textquote{hoy} de la Celebración Eucarística. El homileta puede presentar este banquete al pueblo de Dios reunido para celebrar el nacimiento del Señor, exhortándole a imitar a María, la Madre de Jesús, que \textquote{conservaba todas estas cosas, meditándolas en su corazón} (Evangelio, Misa de la aurora).

\textbf{Una reflexión sobre la Navidad}\footnote{Benedicto XVI, papa, \emph{Catequesis,} Audiencia general, 17 de diciembre del 2008}.

Comenzamos precisamente hoy los días del Adviento que nos preparan inmediatamente para el Nacimiento del Señor: estamos en la \emph{Novena de Navidad,} que en muchas comunidades cristianas se celebra con liturgias ricas en texto bíblicos, todos ellos orientados a alimentar la espera del nacimiento del Salvador. En efecto, toda la Iglesia concentra su mirada de fe en esta fiesta, ya cercana, disponiéndose, como cada año, a unirse al canto alegre de los ángeles, que en el corazón de la noche anunciarán a los pastores el extraordinario acontecimiento del nacimiento del Redentor, invitándolos a dirigirse a la cueva de Belén. Allí yace el Emmanuel, el Creador que se ha hecho criatura, envuelto en pañales y acostado en un pobre pesebre (cf. \emph{Lc} 2, 12-14).

La Navidad, por el clima que la caracteriza, es una fiesta universal. De hecho, incluso quien se dice no creyente puede percibir en esta celebración cristiana anual algo extraordinario y trascendente, algo íntimo que habla al corazón. Es la fiesta que canta el don de la vida. El nacimiento de un niño debería ser siempre un acontecimiento que trae alegría: el abrazo de un recién nacido suscita normalmente sentimientos de atención y de solicitud, de conmoción y de ternura.

La Navidad es el encuentro con un recién nacido que llora en una cueva miserable. Contemplándolo en el pesebre, ¿cómo no pensar en tantos niños que también hoy, en muchas regiones del mundo, nacen en una gran pobreza? ¿Cómo no pensar en los recién nacidos que no son acogidos sino rechazados, en los que no logran sobrevivir por falta de cuidados y atenciones? ¿Cómo no pensar también en las familias que quisieran tener la alegría de un hijo y no ven cumplida esta esperanza? Por desgracia, por el impulso de un consumismo hedonista, la Navidad corre el riesgo de perder su significado espiritual para reducirse a una mera ocasión comercial de compras e intercambio de regalos.

Sin embargo, en realidad, las dificultades, las incertidumbres y la misma crisis económica que en estos meses están viviendo tantas familias, y que afecta a toda la humanidad, pueden ser un estímulo para volver a descubrir el calor de la sencillez, la amistad y la solidaridad, valores típicos de la Navidad. Así, sin las incrustaciones consumistas y materialistas, la Navidad puede convertirse en una ocasión para acoger, como regalo personal, el mensaje de esperanza que brota del misterio del nacimiento de Cristo.

Todo esto, sin embargo, no basta para captar en su plenitud el valor de la fiesta a la que nos estamos preparando. Nosotros sabemos que en ella se celebra el acontecimiento central de la historia: la Encarnación del Verbo divino para la redención de la humanidad. San León Magno, en una de sus numerosas homilías navideñas, exclama: \textquote{Exultemos en el Señor, queridos hermanos, y abramos nuestro corazón a la alegría más pura. Porque ha amanecido el día que para nosotros significa la nueva redención, la antigua preparación, la felicidad eterna. Así, en el ciclo anual, se renueva para nosotros el elevado misterio de nuestra salvación, que, prometido al principio y realizado al final de los tiempos, está destinado a durar sin fin} \emph{(Homilía XXII). }

San Pablo comenta muchas veces esta verdad fundamental en sus cartas. Por ejemplo, a los \emph{Gálatas} escribe: \textquote{Al llegar la plenitud de los tiempos, envió Dios a su Hijo, nacido de mujer, nacido bajo la Ley (\ldots{}) para que recibiéramos la filiación adoptiva} \emph{(Ga} 4, 4-5). En la \emph{carta a los Romanos} pone de manifiesto las lógicas y exigentes consecuencias de este acontecimiento salvador: \textquote{Si somos hijos (de Dios), también somos herederos; herederos de Dios y coherederos de Cristo, ya que sufrimos con él, para ser también con él glorificados} \emph{(Rm} 8, 17). Pero es sobre todo san Juan, en el \emph{Prólogo} del cuarto \emph{Evangelio,} quien medita profundamente en el misterio de la Encarnación. Y por eso desde los tiempos más antiguos el \emph{Prólogo} forma parte de la liturgia de la Navidad. En efecto, en él se encuentra la expresión más auténtica y la síntesis más profunda de esta fiesta y del fundamento de su alegría. San Juan escribe: \textquote{\emph{Et Verbum caro factum est et habitavit in nobis}} --- \textquote{Y el Verbo se hizo carne y habitó entre nosotros} \emph{(Jn} 1, 14).

Así pues, en Navidad no nos limitamos a conmemorar el nacimiento de un gran personaje; no celebramos simplemente y en abstracto el misterio del nacimiento del hombre o en general el misterio de la vida; tampoco celebramos sólo el inicio de la nueva estación. En Navidad recordamos algo muy concreto e importante para los hombres, algo esencial para la fe cristiana, una verdad que san Juan resume en estas pocas palabras: \textquote{El Verbo se hizo carne}.

Se trata de un acontecimiento histórico que el evangelista san Lucas se preocupa de situar en un contexto muy determinado: en los días en que César Augusto emanó el decreto para el primer censo, cuando Quirino era ya gobernador de Siria (cf. \emph{Lc} 2, 1-7). Por tanto, en una noche fechada históricamente se verificó el acontecimiento de salvación que Israel esperaba desde hacía siglos. En la oscuridad de la noche de Belén se encendió realmente una gran luz: el Creador del universo se encarnó uniéndose indisolublemente a la naturaleza humana, siendo realmente \textquote{Dios de Dios, luz de luz} y al mismo tiempo hombre, verdadero hombre.

Aquel a quien san Juan llama en griego \textquote{ \emph{ho Logo} s} ---traducido en latín \textquote{Verbum} y en español \textquote{el Verbo} --- significa también \textquote{el Sentido}. Por tanto, la expresión de san Juan se puede entender así: el \textquote{Sentido eterno} del mundo se ha hecho perceptible a nuestros sentidos y a nuestra inteligencia: ahora podemos tocarlo y contemplarlo (cf. \emph{I Jn} 1, 1). El \textquote{Sentido} que se ha hecho carne no es simplemente una idea general inscrita en el mundo; es una \textquote{Palabra} dirigida a nosotros. El \emph{Logos} nos conoce, nos llama, nos guía. No es una ley universal, en la que nosotros desarrollamos algún papel; es una Persona que se interesa por cada persona: es el Hijo del Dios vivo, que se ha hecho hombre en Belén.

A muchos hombres, y de algún modo a todos nosotros, esto parece demasiado hermoso para ser cierto. En efecto, aquí se nos reafirma: sí, existe un sentido, y el sentido no es una protesta impotente contra lo absurdo. El Sentido tiene poder: es Dios. Un Dios bueno, que no se confunde con un poder excelso y lejano, al que nunca se podría llegar, sino un Dios que se ha hecho nuestro prójimo, muy cercano a nosotros, que tiene tiempo para cada uno de nosotros y que ha venido a quedarse con nosotros.

Entonces surge espontáneamente la pregunta: \textquote{¿Cómo es posible algo semejante? ¿Es digno de Dios hacerse niño?}. Para intentar abrir el corazón a esta verdad que ilumina toda la existencia humana, es necesario plegar la mente y reconocer la limitación de nuestra inteligencia. En la cueva de Belén Dios se nos muestra \textquote{niño} humilde para vencer nuestra soberbia. Tal vez nos habríamos rendido más fácilmente frente al poder, frente a la sabiduría; pero él no quiere nuestra rendición; más bien apela a nuestro corazón y a nuestra decisión libre de aceptar su amor. Se ha hecho pequeño para liberarnos de la pretensión humana de grandeza que brota de la soberbia; se ha encarnado libremente para hacernos a nosotros verdaderamente libres, libres de amarlo.

{[}\ldots{}{]} La Navidad es una oportunidad privilegiada para meditar en el sentido y en el valor de nuestra existencia. La proximidad de esta solemnidad nos ayuda a reflexionar, por una parte, en el dramatismo de la historia en la que los hombres, heridos por el pecado, buscan permanentemente la felicidad y el sentido pleno de la vida y de la muerte; y, por otra, nos exhorta a meditar en la bondad misericordiosa de Dios, que ha salido al encuentro del hombre para comunicarle directamente la Verdad que salva y para hacerlo partícipe de su amistad y de su vida.

Preparémonos, por tanto, para la Navidad con humildad y sencillez, disponiéndonos a recibir el don de la luz, la alegría y la paz que irradian de este misterio. Acojamos el Nacimiento de Cristo como un acontecimiento capaz de renovar hoy nuestra vida. Que el encuentro con el Niño Jesús nos haga personas que no piensen sólo en sí mismas, sino que se abran a las expectativas y necesidades de los hermanos. De esta forma nos convertiremos también nosotros en testigos de la luz que la Navidad irradia sobre la humanidad del tercer milenio.

Pidamos a María santísima, tabernáculo del Verbo encarnado, y a san José, testigo silencioso de los acontecimientos de la salvación, que nos comuniquen los sentimientos que ellos tenían mientras esperaban el nacimiento de Jesús, de modo que podamos prepararnos para celebrar santamente la próxima Navidad, en el gozo de la fe y animados por el compromiso de una conversión sincera.

¡Feliz Navidad a todos!

\chapter{Misa de la Vigilia}

\section{Lecturas}

PRIMERA LECTURA

Del libro del profeta Isaías 62, 1-5

El Señor te prefiere a ti

Por amor a Sión no callaré,

por amor de Jerusalén no descansaré,

hasta que rompa la aurora de su justicia,

y su salvación llamee como antorcha.

Los pueblos verán tu justicia,

y los reyes tu gloria;

te pondrán un nombre nuevo,

pronunciado por la boca del Señor.

Serás corona fúlgida en la mano del Señor

y diadema real en la palma de tu Dios.

Ya no te llamarán \textquote{Abandonada},

ni a tu tierra \textquote{Devastada};

a ti te llamarán \textquote{Mi predilecta},

y a tu tierra \textquote{Desposada},

porque el Señor te prefiere a ti,

y tu tierra tendrá un esposo.

Como un joven se desposa con una doncella,

así te desposan tus constructores.

Como se regocija el marido con su esposa, se regocija tu Dios contigo.

SALMO RESPONSORIAL

Salmo 88, 4-5. 16-17. 27 y 29

Cantaré eternamente las misericordias del Señor

℣. «Sellé una alianza con mi elegido,

jurando a David, mi siervo:

Te fundaré un linaje perpetuo,

edificaré tu trono para todas las edades». ℟.

℣. Dichoso el pueblo que sabe aclamarte:

caminará, oh, Señor, a la luz de tu rostro;

tu nombre es su gozo cada día,

tu justicia es su orgullo. ℟.

℣. Él me invocará: ``Tú eres mi padre,

mi Dios, mi Roca salvadora''.

Le mantendré eternamente mi favor,

y mi alianza con él será estable. ℟.

SEGUNDA LECTURA

De los Hechos de los Apóstoles 13, 16-17. 22-25

Testimonio de Pablo sobre Cristo, hijo de David

Cuando Pablo llegó a Antioquía de Pisidia, se puso en pie y, haciendo
seña con la mano de que se callaran, dijo:

«Israelitas y los que teméis a Dios, escuchad:

El Dios de este pueblo, Israel, eligió a nuestros padres y multiplicó al
pueblo cuando vivían como forasteros en Egipto. Los sacó de allí con
brazo poderoso.

Después, les suscitó como rey a David, en favor del cual dio testimonio,
diciendo:

Encontré a David, hijo de Jesé,

hombre conforme a mi corazón,

que cumplirá todos mis preceptos.

Según lo prometido, Dios sacó de su descendencia un salvador para
Israel: Jesús.

Juan predicó a todo Israel un bautismo de conversión antes de que
llegara Jesús; y, cuando Juan estaba para concluir el curso de su vida,
decía:

``Yo no soy quien pensáis, pero, mirad, viene uno detrás de mí a quien
no merezco desatarle las sandalias de los pies''.

EVANGELIO (forma larga)

Del Santo Evangelio según san Mateo 1, 1-25

Genealogía de Jesucristo, hijo de David

Libro del origen de Jesucristo, hijo de David, hijo de Abrahán.

Abrahán engendró a Isaac, Isaac engendró a Jacob, Jacob engendró a Judá
y a sus hermanos. Judá engendró, de Tamar, a Fares y a Zará, Fares
engendró a Esrón, Esrón engendró a Arán, Arán engendró a Aminadab,
Aminadab engendró a Naasón, Naasón engendró a Salmón, Salmón engendró,
de Rajab, a Booz; Booz engendró, de Rut, a Obed; Obed engendró a Jesé,
Jesé engendró a David, el rey.

David, de la mujer de Urías, engendró a Salomón, Salomón engendró a
Roboán, Roboán engendró a Abías, Abías engendró a Asaf, Asaf engendró a
Josafat, Josafat engendró a Jorán, Jorán engendró a Ozías, Ozías
engendró a Joatán, Joatán engendró a Acaz, Acaz engendró a Ezequías,
Ezequías engendró a Manasés, Manasés engendró a Amós, Amós engendró a
Josías; Josías engendró a Jeconías y a sus hermanos, cuando el destierro
de Babilonia.

Después del destierro de Babilonia, Jeconías engendró a Salatiel,
Salatiel engendró a Zorobabel, Zorobabel engendró a Abiud, Abiud
engendró a Eliaquín, Eliaquín engendró a Azor, Azor engendró a Sadoc,
Sadoc engendró a Aquín, Aquín engendró a Eliud, Eliud engendró a
Eleazar, Eleazar engendró a Matán, Matán engendró a Jacob; y Jacob
engendró a José, el esposo de María, de la cual nació Jesús, llamado
Cristo. Así, las generaciones desde Abrahán a David fueron en total
catorce; desde David hasta la deportación a Babilonia, catorce; y desde
la deportación a Babilonia hasta el Cristo, catorce.

La generación de Jesucristo fue de esta manera: María, su madre, estaba
desposada con José y, antes de vivir juntos, resultó que ella esperaba
un hijo por obra del Espíritu Santo.

José, su esposo, como era justo y no quería difamarla, decidió
repudiarla en privado. Pero, apenas había tomado esta resolución, se le
apareció en sueños un ángel del Señor que le dijo:

«José, hijo de David, no temas acoger a María, tu mujer, porque la
criatura que hay en ella viene del Espíritu Santo. Dará a luz un hijo y
tú le pondrás por nombre Jesús, porque él salvará a su pueblo de sus
pecados».

Todo esto sucedió para que se cumpliese lo que había dicho el Señor por
medio del profeta:

«Mirad: la Virgen concebirá y dará a luz un hijo y le pondrán por nombre
Enmanuel, que significa \textquote{Dios-con-nosotros}».

Cuando José se despertó, hizo lo que le había mandado el ángel del Señor
y acogió a su mujer. Y sin haberla conocido, ella dio a luz un hijo al
que puso por nombre Jesús.

EVANGELIO (forma breve)

Del Santo Evangelio según san Mateo 1, 18-25

María dará a luz un hijo y tú le pondrás por nombre Jesús

La generación de Jesucristo fue de esta manera:

María, su madre, estaba desposada con José y, antes de vivir juntos,
resultó que ella esperaba un hijo por obra del Espíritu Santo.

José, su esposo, como era justo y no quería difamarla, decidió
repudiarla en privado. Pero, apenas había tomado esta resolución, se le
apareció en sueños un ángel del Señor que le dijo: «José, hijo de David,
no temas acoger a María, tu mujer, porque la criatura que hay en ella
viene del Espíritu Santo. Dará a luz un hijo y tú le pondrás por nombre
Jesús, porque él salvará a su pueblo de sus pecados».

Todo esto sucedió para que se cumpliese lo que había dicho el Señor por
medio del profeta:

«Mirad: la Virgen concebirá y dará a luz un hijo

y le pondrán por nombre Enmanuel,

que significa \textquote{Dios-con-nosotros}».

Cuando José se despertó, hizo lo que le había mandado el ángel del Señor
y acogió a su mujer.

Y sin haberla conocido, ella dio a luz un hijo al que puso por nombre
Jesús.



\section{Comentario Patrístico}

\subsection{San Pedro Crisólogo, obispo}

Mirad: la virgen concebirá y dará a luz un hijo,

y le pondrá por nombre Emmanuel

Sermón 145: PL 52, 588

Es mi propósito hablaros hoy, hermanos, de cómo nos relata el santo evangelista el misterio de la generación de Cristo. Dice: \emph{El nacimiento de Jesucristo fue de esta manera: La madre de Jesús estaba desposada con José y, antes de vivir juntos, resultó que ella esperaba un hijo, por obra del Espíritu Santo. José, su esposo, que era bueno y no quería denunciarla, decidió repudiarla en secreto}. Pero, ¿cómo se compagina esta bondad con la resolución de no discutir la gravidez de su esposa? Las virtudes no pueden sostenerse separadamente: la equidad desprovista de bondad se convierte en severidad, y la justicia sin piedad se torna crueldad. Con razón, pues, a José se le califica de bueno, porque era piadoso; y de piadoso, por ser bueno. Al pensar piadosamente, se libra de la crueldad; al juzgar benignamente, observó la justicia; al no querer erigirse en acusador, rehuyó la sentencia. Se requemaba el alma del justo perpleja ante la novedad del evento: tenía ante sí una esposa preñada, pero virgen; grávida del don no menos que del pudor; solícita por lo concebido, pero segura de su integridad; revestida de la función maternal, sin perder el decoro virginal.

¿Cuál debía ser la conducta del esposo ante tal situación? ¿Acusarla de infidelidad? No, pues él era testigo de su inocencia. ¿Airear la culpa? Tampoco, pues era él el guardián de su pudor. ¿Inculparla de adulterio? Menos aún, pues estaba plenamente convencido de su virginidad. ¿Qué hacer, entonces? Piensa repudiarla en secreto, pues ni podía ir por ahí aireando lo sucedido ni ocultarlo en la intimidad del hogar. Decide repudiarla en secreto y confía a Dios todo el negocio, ya que nada tiene que comunicar a los hombres.

\emph{José, hijo de David, no tengas reparo en llevarte a María, tu mujer, porque la criatura que hay en ella viene del Espíritu Santo. Dará a luz un hijo y tú le pondrás por nombre Jesús, porque él salvará al pueblo de los pecados}. Veis, hermanos, cómo una sola persona representa toda una raza, veis cómo un solo individuo lleva la representación de toda una estirpe, veis cómo en José se da cita la serie genealógica de David.

\emph{José, hijo de David}. Nacido de la vigésima octava generación, ¿por qué se le llama hijo de David, sino porque en él se desvela el misterio de una estirpe, se cumple la fidelidad de la promesa, y en la carne virginal luce ya el sello de la sobrenatural concepción de un parto celeste? La promesa de Dios Padre hecha a David estaba expresada en estos términos: \emph{El Señor ha jurado a David una promesa que no retractará: \textquote{A uno de tu linaje pondré sobre tu trono}}. Del fruto de tu vientre: sí, fruto de tu vientre, de tu seno, sí, porque el huésped celeste, el supremo morador de tal modo descendió al receptáculo de tus entrañas que ignoró las limitaciones corporales; y de tal modo salió del claustro materno, que dejó intacto el sello de la virginidad, cumpliéndose de esta manera lo que se canta en el Cantar de los cantares: \emph{Eres jardín cerrado, hermana y novia mía; eres jardín cerrado, fuente sellada}.

\emph{La criatura que hay en ella viene del Espíritu Santo}. Concibió la virgen, pero del Espíritu; parió la virgen, pero a aquel que había predicho Isaías: \emph{Mirad: la virgen concebirá y dará a luz un hijo, y le pondrá por nombre Emmanuel (que significa \textquote{Dios-con-nosotros})}.


\section{Homilías}

Las homilías para esta celebración están tomadas de textos de las Padres de la Iglesia que tocan algunos aspectos de la Navidad en particular o relacionados con alguno de los textos bíblicos que se leen en la misma.

Conviene señalar que estas homilías pueden iluminar aspectos de cualquiera de las otras celebraciones durante el tiempo de Navidad.

\subsection{San Bernardo de Claraval, abad}

\subsubsection{Sermón: Habitaré y caminaré con ellos}

Sermón 27, 7.9 sobre el Cantar de los cantares:

Opera omnia. Edit. Cister. 1957, I, 186-188.

Luego que el divino Emmanuel implantó en la tierra el magisterio de la doctrina celeste, luego que por Cristo y en Cristo se nos manifestó la imagen visible de aquella celestial Jerusalén, que es nuestra madre, y el esplendor de su belleza, ¿qué es lo que contemplamos sino a la Esposa en el Esposo, admirando en el mismo y único Señor de la gloria al Esposo que se ciñe la corona y a la Esposa que se adorna de sus joyas? El que bajó es efectivamente el mismo que subió, pues nadie ha subido al cielo sino el que bajó del cielo: el mismo y único Señor, esposo en la cabeza y esposa en el cuerpo. Y no en vano apareció en la tierra el hombre celestial, él que convirtió en celestiales a muchos hombres terrenales haciéndolos semejantes a él, para que se cumpliera lo que dice la Escritura: \emph{Igual que el celestial son los hombres celestiales}.

Desde entonces en la tierra se vive como en el cielo, pues, a imitación de aquella soberana y dichosa criatura, también ésta que viene desde los confines de la tierra a escuchar la sabiduría de Salomón, se une al esposo celeste con un vínculo de casto amor; y si bien no le está todavía como aquélla unida por la visión, está ya desposada por la fe, según la promesa de Dios, que dice por el profeta: \emph{Me casaré contigo en derecho y justicia, en misericordia y compasión, me casaré contigo en fidelidad}. Por eso se afana en conformarse más y más al modelo celestial, aprendiendo de él a ser modesta y sobria, pudorosa y santa, paciente y compasiva, aprendiendo finalmente a ser mansa y humilde de corazón. Con semejante conducta procura agradar, aunque ausente, a aquel a quien los ángeles desean contemplar, a fin de que, inflamada de angélico ardor, se comporte como ciudadana de los santos y miembro de la familia de Dios, se comporte como la amada, se comporte como la esposa.

\emph{Ven, mi elegida, y pondré en ti mi trono}. ¿Por qué te acongojas ahora, alma mía, por qué te me turbas? ¿Crees que podrás disponer en tu interior un lugar para el Señor? Y ¿qué lugar, en nuestro interior, podrá parecernos idóneo para tanta gloria, capaz de tamaña majestad? ¡Ojalá pudiera merecer siquiera adorar al estrado de sus pies! ¡Quién me diera seguir al menos las huellas de cualquier alma santa, que el Señor se escogió como heredad! Mas si él se dignase derramar en mi alma la unción de su misericordia, y dilatarla como se dilata una piel engrasada, de modo que también yo pudiera decir: \emph{Correré por el camino de tus mandatos cuando me ensanches el corazón,} quizá pudiera a mi vez mostrarle en mí mismo, si no una sala grande arreglada con divanes, donde pueda sentarse a la mesa con sus discípulos, sí al menos un lugar donde pueda reclinar su cabeza. Veo en lontananza a aquellas almas realmente dichosas, de las cuales se ha dicho: \emph{Habitaré y caminaré con ellos}.

\subsection{San Juan Pablo II, papa}

\subsubsection{Catequesis: La genealogía de Mateo}

Catequesis en la Audiencia general del 28 de enero de 1987, nn. 5-10.

El Evangelio \emph{según Mateo} completa la narración de Lucas describiendo algunas circunstancias que precedieron al nacimiento de Jesús. Leemos: \textquote{La concepción de Jesucristo fue así: Estando desposada María, su Madre, con José, \emph{antes de que conviviesen} se halló \emph{haber concebido María del Espíritu Santo}. José, su esposo, siendo justo, no quiso denunciarla y resolvió repudiarla en secreto. Mientras reflexionaba sobre esto, he aquí que se le apareció en sueños un ángel del Señor y le dijo: José, hijo de David, no temas recibir en tu casa a María, tu esposa, pues \emph{lo concebido en ella es obra del Espíritu Santo}. Dará a luz un hijo a quien pondrás por nombre Jesús, porque salvará a su pueblo de sus pecados} (\emph{Mt} 1, 18-21 ).

Como se ve, ambos textos del \textquote{Evangelio de la infancia\emph{} concuerdan en la constatación fundamental}: Jesús fue concebido por obra del Espíritu Santo y nació de María Virgen; y son entre sí \emph{complementarios} en el esclarecimiento de las circunstancias de este acontecimiento extraordinario: Lucas respecto a María, Mateo respecto a José.

Para identificar \emph{la fuente de la que deriva el Evangelio de la infancia}, hay que referirse a la frase de San Lucas: \textquote{\emph{María guardaba todo esto} y lo meditaba en su corazón} (\emph{Lc} 2, 19). Lucas lo dice dos veces: después de marchar los pastores de Belén y después del encuentro de Jesús en el templo (cf. 2, 51). El Evangelista mismo nos ofrece los elementos para identificar en la Madre de Jesús una de las fuentes de información utilizadas por él para escribir el \textquote{Evangelio de la infancia}. María, que \textquote{guardó todo esto en su corazón} (cf. \emph{Lc} 2, 19), pudo dar testimonio, después de la muerte y resurrección de Cristo, de lo que se refería a la propia persona y a la función de Madre precisamente en el período apostólico, en el que nacieron los textos del Nuevo Testamento y tuvo origen la primera tradición cristiana.

El testimonio evangélico de \emph{la concepción virginal de Jesús} por parte de María es de gran relevancia teológica. Pues constituye un signo especial \emph{del origen divino del Hijo de María}. El que Jesús no tenga un padre terreno porque ha sido engendrado \textquote{sin intervención de varón}, pone de relieve la verdad de que Él es el Hijo de Dios, de modo que cuando asume la naturaleza humana, su Padre continúa siendo exclusivamente Dios.

La revelación de la intervención del Espíritu Santo \emph{en la concepción de Jesús}, indica \emph{el comienzo} en la historia del hombre de la nueva generación espiritual que tiene un carácter estrictamente sobrenatural (cf. \emph{1 Cor} 15, 45-49). De este modo Dios Uno y Trino \textquote{se comunica} a la criatura mediante el Espíritu Santo. Es el misterio al que se pueden aplicar las palabras del Salmo: \textquote{Envía tu Espíritu, y serán creados, y renovarás la faz de la tierra} (\emph{Sal} 103 {[}104{]}, 30). En la economía de esa comunicación de Sí mismo que Dios hace a la criatura, la concepción virginal de Jesús, que sucedió por obra del Espíritu Santo, es un \emph{acontecimiento central y culminante}. Él \emph{inicia la \textquote{nueva creación}}. Dios entra así en un modo decisivo en la historia para actuar el destino sobrenatural del hombre, o sea, la predestinación de todas las cosas en Cristo. Es \emph{la expresión} definitiva del \emph{Amor salvífico} de Dios al hombre, del que hemos hablado en las catequesis sobre la Providencia.

En la actuación del plan de la salvación hay siempre una participación de la criatura. Así en la concepción de Jesús por obra del Espíritu Santo \emph{María participa} de forma \emph{decisiva}. Iluminada interiormente por el mensaje del ángel sobre su vocación de Madre y sobre la conservación de su virginidad, María \emph{expresa su voluntad y consentimiento} y acepta hacerse el humilde instrumento de la \textquote{virtud del Altísimo}. La acción del Espíritu Santo hace que en María la maternidad y la virginidad estén presentes de un modo que, aunque inaccesible a la mente humana, entre de lleno en el ámbito de la predilección de la omnipotencia de Dios. En María se cumple la gran profecía de Isaías: \textquote{La virgen grávida da a luz} (7, 14; cf. \emph{Mt} 1, 22-23); su virginidad, signo en el Antiguo Testamento de la pobreza y de disponibilidad total al plan de Dios, se convierte en el terreno de la acción excepcional de Dios, que escoge a María para ser Madre del Mesías.

La excepcionalidad de María se deduce también de las genealogías aducidas por Mateo y Lucas.

El Evangelio \emph{según Mateo} comienza, conforme a la costumbre hebrea\emph{, con la genealogía de Jesús} (\emph{Mt} 1, 2-17) y hace un elenco partiendo de Abraham, de las generaciones masculinas. A Mateo de hecho, le importa poner de relieve, mediante la paternidad \emph{legal} de José, la descendencia de Jesús de Abraham y David y, por consiguiente, la legitimidad de su calificación de Mesías. Sin embargo, al final de la serie de los ascendientes leemos: \textquote{Y Jacob engendró a José esposo de María\emph{, de la cual nació Jesús llamado Cristo}} (\emph{Mt} 1, 16). Poniendo el acento en la maternidad de María, el Evangelista implícitamente subraya la verdad del nacimiento virginal: Jesús, como hombre, no tiene padre terreno.

\emph{Según el Evangelio de} Lucas, la genealogía de Jesús (\emph{Lc} 3, 23-38) es ascendente: desde Jesús a través de sus antepasados se remonta \emph{hasta Adán}. El Evangelista ha querido mostrar la vinculación de Jesús \emph{con todo el género humano}. María, como colaboradora de Dios en dar a su Eterno Hijo la naturaleza humana, ha sido el instrumento de la unión de Jesús con toda la humanidad.

\section{Temas}

El Directorio Homilético recoge los temas de la Navidad en un solo grupo, ver página 163.

\chapter{Misa de Medianoche}

\section{Lecturas}

PRIMERA LECTURA

Del libro del profeta Isaías 9, 1-6

Un hijo se nos ha dado

El pueblo que caminaba en tinieblas vio una luz grande;

habitaba en tierra y sombras de muerte, y una luz les brilló.

Acreciste la alegría, aumentaste el gozo;

se gozan en tu presencia, como gozan al segar,

como se alegran al repartirse el botín.

Porque la vara del opresor, el yugo de su carga,

el bastón de su hombro,

los quebrantaste como el día de Madián.

Porque la bota que pisa con estrépito

y la túnica empapada de sangre

serán combustible, pasto del fuego.

Porque un niño nos ha nacido, un hijo se nos ha dado:

lleva a hombros el principado, y es su nombre:

«Maravilla de Consejero, Dios fuerte,

Padre de eternidad, Príncipe de la paz».

Para dilatar el principado, con una paz sin límites,

sobre el trono de David y sobre su reino.

Para sostenerlo y consolidarlo

con la justicia y el derecho, desde ahora y por siempre.

El celo del Señor del universo lo realizará.

SALMO RESPONSORIAL

Salmo 95, 1-2a. 2b-3. 11-12. 13

Hoy nos ha nacido un Salvador: el Mesías, el Señor

℣. Cantad al Señor un cántico nuevo,

cantad al Señor, toda la tierra;

cantad al Señor, bendecid su nombre. ℟.

℣. Proclamad día tras día su victoria.

Contad a los pueblos su gloria,

sus maravillas a todas las naciones. ℟.

℣. Alégrese el cielo, goce la tierra,

retumbe el mar y cuanto lo llena;

vitoreen los campos y cuanto hay en ellos,

aclamen los árboles del bosque. ℟.

℣. Delante del Señor, que ya llega,

ya llega a regir la tierra:

regirá el orbe con justicia

y los pueblos con fidelidad. ℟.

SEGUNDA LECTURA

De la carta del apóstol san Pablo a Tito 2, 11-14

Se ha manifestado la gracia de Dios para todos los hombres

Querido hermano:

Se ha manifestado la gracia de Dios, que trae la salvación para todos
los hombres, enseñándonos a que, renunciando a la impiedad y a los
deseos mundanos, llevemos ya desde ahora una vida sobria, justa y
piadosa, aguardando la dicha que esperamos y la manifestación de la
gloria del gran Dios y Salvador nuestro, Jesucristo, el cual se entregó
por nosotros para rescatarnos de toda iniquidad y purificar para sí un
pueblo de su propiedad, dedicado enteramente a las buenas obras.

EVANGELIO

Del Santo Evangelio según san Lucas 2, 1-14

Hoy os ha nacido un Salvador

Sucedió en aquellos días que salió un decreto del emperador Augusto,
ordenando que se empadronase todo el Imperio.

Este primer empadronamiento se hizo siendo Cirino gobernador de Siria. Y
todos iban a empadronarse, cada cual a su ciudad.

También José, por ser de la casa y familia de David, subió desde la
ciudad de Nazaret, en Galilea, a la ciudad de David, que se llama Belén,
en Judea, para empadronarse con su esposa María, que estaba encinta. Y
sucedió que, mientras estaban allí, le llegó a ella el tiempo del parto
y dio a luz a su hijo primogénito, lo envolvió en pañales y lo recostó
en un pesebre, porque no había sitio para ellos en la posada.

En aquella misma región había unos pastores que pasaban la noche al aire
libre, velando por turno su rebaño.

De repente un ángel del Señor se les presentó; la gloria del Señor los
envolvió de claridad, y se llenaron de gran temor.

El ángel les dijo:

«No temáis, os anuncio una buena noticia que será de gran alegría para
todo el pueblo: hoy, en la ciudad de David, os ha nacido un Salvador, el
Mesías, el Señor. Y aquí tenéis la señal: encontraréis un niño envuelto
en pañales y acostado en un pesebre».

De pronto, en torno al ángel, apareció una legión del ejército
celestial, que alababa a Dios diciendo:

«Gloria a Dios en el cielo, y en la tierra paz a los hombres de buena
voluntad».



\section{Comentario Patrístico}

\subsection{Beato Elredo de Rievaulx, abad}

Hoy nos ha nacido un Salvador

Sermón 1 de la Natividad del Señor: PL 195, 226-227

\emph{Hoy, en la ciudad de David, nos ha nacido un Salvador: El Mesías, el Señor}. La ciudad de que aquí se habla es Belén, a la que debemos acudir corriendo, como lo hicieron los pastores, apenas oído este rumor. Así es como soléis cantar ---en el himno de María, la Virgen---: \textquote{Cantaron gloria a Dios, corrieron a Belén}. \emph{Y aquí tenéis la señal: encontraréis un niño envuelto en pañales y acostado en un pesebre}.

Ved por qué os dije que debéis amar. Teméis al Señor de los ángeles, pero amadle chiquitín; teméis al Señor de la majestad, pero amadle envuelto en pañales; teméis al que reina en el cielo, pero amadle acostado en un pesebre. Y ¿cuál fue la señal que recibieron los pastores? \emph{Encontraréis un niño envuelto en pañales y acostado en un pesebre}. El es el Salvador, él es el Señor. Pero, ¿qué tiene de extraordinario ser envuelto en pañales y yacer en un establo? ¿No son también los demás niños envueltos en pañales? Entonces, ¿qué clase de señal es ésta? Una señal realmente grande, a condición de que sepamos comprenderla. Y la comprendemos si no nos limitamos a escuchar este mensaje de amor, sino que, además, albergamos en nuestro corazón aquella claridad que apareció junto con los ángeles. Y si el ángel se apareció envuelto en claridad, cuando por primera vez anunció este rumor, fue para enseñarnos que sólo escuchan de verdad, los que acogen en su alma la claridad espiritual.

Podríamos decir muchas cosas sobre esta señal, pero como el tiempo corre, insistiré brevemente en este tema. Belén, \textquote{casa del pan}, es la santa Iglesia, en la cual se distribuye el cuerpo de Cristo, a saber, el pan verdadero. El pesebre de Belén se ha convertido en el altar de la Iglesia. En él se alimentan los animales de Cristo. De esta mesa se ha escrito: \emph{Preparas una mesa ante mí}. En este pesebre está Jesús envuelto en pañales. La envoltura de los pañales es la cobertura de los sacramentos. En este pesebre y bajo las especies de pan y vino está el verdadero cuerpo y la sangre de Cristo. En este sacramento creemos que está el mismo Cristo; pero está envuelto en pañales, es decir, invisible bajo los signos sacramentales. No tenemos señal más grande y más evidente del nacimiento de Cristo como el hecho de que cada día sumimos en el altar santo su cuerpo y su sangre; como el comprobar que a diario se inmola por nosotros, el que por nosotros nació una vez de la Virgen.

Apresurémonos, hermanos, al pesebre del Señor; pero antes y en la medida de lo posible, preparémonos con su gracia para este encuentro de suerte que asociados a los ángeles, \emph{con corazón limpio, con una conciencia honrada y con una fe sentida,} cantemos al Señor con toda nuestra vida y toda nuestra conducta: \emph{Gloria a Dios en el cielo, y en la tierra, paz a los hombres que Dios ama}. Por el mismo Jesucristo, nuestro Señor, a quien sea el honor y la gloria por los siglos de los siglos. Amén.

\textquote{No temáis, pues os anuncio una gran alegría. (\ldots{}) Os ha nacido hoy, en la ciudad de David, un salvador} (\emph{Lc} 2, 10-11). El mensaje de la venida de Cristo, que llegó del cielo mediante el anuncio de los ángeles, sigue resonando en esta ciudad, así como en las familias, en los hogares y en las comunidades de todo el mundo. Es una \textquote{gran alegría}, dijeron los ángeles, \textquote{para todo el pueblo}. Este mensaje proclama que el Mesías, el Hijo de Dios e hijo de David nació \textquote{por vosotros}: por ti y por mí, y por todos los hombres y mujeres de todo tiempo y lugar. En el plan de Dios, Belén, \textquote{el menor entre los clanes de Judá} (\emph{Mi} 5, 1) se convirtió en un lugar de gloria imperecedera: el lugar donde, en la plenitud de los tiempos, Dios eligió hacerse hombre, para acabar con el largo reinado del pecado y de la muerte, y para traer vida nueva y abundante a un mundo ya viejo, cansado y oprimido por la desesperación.

Para los hombres y mujeres de todo lugar, Belén está asociada a este alegre mensaje de renacimiento, renovación, luz y libertad. Y, sin embargo, aquí, en medio de nosotros, ¡qué lejos de hacerse realidad parece esa magnífica promesa! ¡Qué distante parece el Reino de amplio dominio y paz, de seguridad, justicia e integridad, que el profeta Isaías anunció, como hemos escuchado en la primera lectura (cf. \emph{Is} 9, 7) y que proclamamos como definitivamente establecido con la venida de Jesucristo, Mesías y Rey!

\textbf{Benedicto XVI, papa,} \emph{Homilía} en Belén, 13 de mayo del 2009.


\section{Homilías}

Las lecturas para esta solemnidad son las mismas en los tres ciclos dominicales. No obstante, las homilías han sido distribuidas en esta obra en tres grupos, tomando en cuenta el ciclo litúrgico correspondiente al año en que fueron pronunciadas. Aquí aparecen las homilías que correspondieron al año A, y las de los años B y C aparecerán en sus respectivos volúmenes.


\subsection{San Juan XXIII, papa}

\subsubsection{Homilía (1962)}

Lunes 24 de diciembre de 1962 \emph{Venerables hermanos y queridos hijos:}

Esta misa de la noche de la Navidad del Señor santifica las más hermosas interioridades del alma, que tienden a lo que es la esencia viva de la unión con Cristo: la religión sincera, liturgia bien comprendida y anhelo de perfección cristiana. Lo advertimos en este momento de tranquilo recogimiento, bajo la mirada del Divino Infante.

En realidad, los grandes problemas de la vida social e individual se acercan a la cuna de Belén, al paso que los ángeles invitan a dar gloria a Dios, gloria a Cristo redentor y salvador, y a excitar gozosamente las buenas voluntades para la celebración de la paz universal.

Gran don, gran riqueza en verdad, es la paz del mundo, que va tras la paz. Lo hemos repetido en el radiomensaje navideño, y Nos satisface dar gracias al Señor por haberlo hecho acoger con buena voluntad de un extremo al otro de la tierra, como confirmación de la luz de esperanza encendida y viva en todas las naciones.

Las súplicas de todos continúan pidiendo la conservación y el perfeccionamiento de este don celestial, al paso que son cada vez más atentos y prudentes todos los movimientos de ideas, palabras y actividades, y se multiplican en todos los campos los esfuerzos y los acuerdos para alejar y superar los obstáculos, conocer y substraer las causas que provocan los conflictos.

Comprendednos, queridos hijos, si hemos preferido, para la misa de Navidad, la sencillez de nuestra capilla privada a las majestuosas bóvedas de los templos romanos, como para dejarnos envolver por el ambiente de las humildes iglesias del campo y de la montaña, de las innumerables instituciones de asistencia social, que son el refugio de la inocencia pobre y abandonada, consuelo y endulzamiento de las lágrimas amargas, reparación de injusticias palmarias y no suficientemente conocidas.

También pensamos en vosotros, queridos enfermos y ancianos, que sufrís dolores y soledad; que vuestro dolor y soledad alcance grandes merecimientos a vosotros y bien a la humanidad.

Hay también circunstancias y situaciones que en esta solemnidad hacen más evidente y agudo el contraste con el gozo de la Navidad. Reclamo eficaz no para disminuir el servicio que hacemos a la verdad y a la justicia, ni para olvidar el inmenso bien realizado por las almas rectas, que tienen como honor la ley divina y el Evangelio; sino para alentar las mejores energías a reparar los errores y a reavivar en el mundo el fervor religioso y las piadosas tradiciones paternas como gozo tranquilo de la Navidad.

Hijos queridos: Junto a la cuna del Niño recién nacido, del Hijo de Dios hecho hombre, todos los hombres que caminan por la tierra piensan con conciencia clara y seria que en la hora suprema se les pedirá cuenta estrecha del don de la vida; y ésta tendrá una sanción definitiva de premio o de castigo, de gloria o de abominación.

En la conciencia de este rendir cuentas es donde se mide la participación de los cristianos y de todos los hombres en el gran misterio que conmemoramos en esta noche; de aquí surge el deseo de que por la luz del Verbo de Dios la civilización humana reciba la llamita que le puede transformar en vivo fulgor, en beneficio de los pueblos.

En torno a la cuna de Jesús sus ángeles cantaron la paz. Y quien creyó en el mensaje celestial y le hizo honor consiguió gloria y alegría. Así ayer; y así será siempre a lo largo de los siglos.

La historia de Cristo es perpetua. Bienaventurado quien la comprende y consigue gracia, fortaleza y bendición. Amén. Amén.


\protect\hyperlink{_ednrefux2a}{*}\emph{ AAS} 55 (1963) 51; \emph{Discorsi-Messaggi-Colloqui del Santo Padre Giovanni XXIII}, vol. V, pp. 63-65.


\subsection{San Pablo VI, papa}

\subsubsection{Homilía (1965)} \emph{Capilla Sixtina\\ Viernes 24 de diciembre de 1965}


Esta santa noche vuelve a proponer a nuestra mente la meditación siempre nueva, siempre sugestiva y, a decir verdad, inagotable, del misterio fundamental de todo el Cristianismo: ¡Dios se hizo hombre!

\textquote{Si alguno -- dice Santo Tomás -- considera con atención y piedad el misterio de la Encarnación, hallará una profundidad de sabiduría tal, que sobrepuja, todo conocimiento humano} (\emph{Contra Gentiles,}4, 54).

En efecto, decir: Dios, es como decir la Grandeza, el poder, la santidad infinita. Decir: el hombre, es como decir la pequeñez, la debilidad, la miseria. Entre estos dos extremos, la distancia parece imposible de salvar, el foso parece imposible de colmar. Y he aquí que en Cristo estos dos conceptos son una sola cosa. La misma persona vive, a la vez, en la naturaleza divina y en la naturaleza humana de Cristo. El Padre de los Cielos puede decir: \textquote{Este es mi hijo bienamado} (\emph{Mt}. 17, 5), como a su vez lo puede decir la Virgen María dirigiéndose al Infante del pesebre que acaba de dar a luz.

Misterio inefable de unión: lo que estaba dividido se reúne, lo que parecía incompatible se acerca, los extremos se funden en uno solo: dos naturalezas -- la humana y la divina -- en una sola persona, la del Hombre-Dios. He aquí toda la teología de la Encarnación, el fundamento y la síntesis de todo el cristianismo.

El prodigio inicial, realizado en Cristo, halla su continuación misteriosa en lo que aquí abajo, hasta el fin de los tiempos, es el \textquote{Cuerpo místico} de Cristo, la gran familia de todos los que creen en El. Porque todo hombre debe unirse a Dios: \textquote{Dios se hizo hombre -- dice magníficamente San Agustín -- para que el hombre se haga Dios}. Tal es el designio divino, revelado en el misterio de Navidad. Y la historia de la Iglesia a través de los siglos, constituye la historia de la realización de tal designio.

En la Encarnación, Dios ha unido el hombre a sí con vínculos tan fuertes que se demuestran superiores a todos los demás, más fuertes que los de la carne y la sangre e incluso los que unen al hombre con lo que le es más valioso en el mundo: la vida. ¿No nos habla acaso todo, aquí en Roma, del valor de los mártires cristianos de los primeros siglos? Hombres, mujeres y también niños dan testimonio ante el verdugo de que separarse de Dios por una abjuración sería para ellos mayor desgracia que perder la vida. La sacrifican para permanecer unidos a Dios.

Cuando la espada del perseguidor romano cesó de herir, las grandes almas cristianas van a buscar a Dios en la soledad. Se abandona la familia, se renuncia a formar una, para unirse mejor a Dios. La corona de la virginidad es ambicionada con el mismo fervor con que se ambiciona la del martirio. La ofrenda cotidiana de sí mismo en la vida monástica tomó el lugar del sacrificio cruento realizado da una sola vez. Y en las mil formas de la vida consagrada, esta unión del hombre con Dios, amado sobre todas las cosas, seguirá manifestándose a través de los siglos hasta nuestros días. La Iglesia suscitará también legiones de santos en el mundo; junto a los mártires, las vírgenes, los doctores, los pontífices y los confesores, ella tendrá la inmensa familia de sus santas mujeres, madres de familia y viudas; en todas las épocas y en todos los países ella suscitará innumerables y fieles ejemplos en muchos hogares cristianos para testimoniar lo que el hombre es capaz de hacer para unirse a Dios, cuando comprende lo que Dios ha hecho para unirse al hombre.

Modelo sublime y principio de la unión del hombre con Dios, la Encarnación se reveló también un maravilloso factor de civilización. ¿Quién como los Apóstoles del Dios encarnado, ha contribuido tanto en el transcurso de los siglos a elevar a los pueblos y a revelarles, además de la grandeza de Dios, su propia dignidad?

La sociedad en la que penetra el fermento cristiano ve elevarse poco a poco su nivel moral y ampliarse su horizonte a las dimensiones del mundo pues la que parecía que sólo incumbía a las relaciones del hombre con Dios se revela el más poderoso factor de unión entre los hombres mismos. El poder de unión de la fe cristiana actúa en el seno de las familias y de los pueblos; derriba las barreras de castas, razas y naciones. La fe que une el hombre a Dios une también a los hombres entre si en un ideal común, en un esfuerzo común, en una esperanza común. ¡Qué motivo ilimitado de meditación! La fe en el Dios encarnado penetra, a lo largo de los siglos, las diversas culturas y las purifica, las enriquece, las transforma. Es la inteligencia humana que se ha superado a si misma, es la filosofía humana que recibió el complemento de las luces divinas como una luz más viva sobre su camino. ¿Y no es acaso también la fe la que inspiró a Miguel Ángel las obras de arte contenidas en esta Capilla, que suscitan la admiración de los hombres de generación en generación?

Pues este enriquecimiento de la cultura es al mismo tiempo un estupendo principio de unión: una civilización cristiana que madura en un país, significa el ingreso de este país en la gran familia donde una misma fe pone en comunión las inteligencias, los corazones y las voluntades. No se terminaría nunca de enumerar los maravillosos desarrollos que jalonan la historia de la civilización. ¿Y todo esto qué es sino, en definitiva, la consecuencia de la Encarnación?

De estos amplios frescos que pueden evocar la historia de la Iglesia, hay que volver al hombre que es su protagonista y su artífice. En el interior del hombre, en su alma, en su psicología, hay que captar las armonías de la fe y de la inteligencia.

La Encarnación puede parecer ante todo, a la inteligencia humana, un peso muy difícil de llevar. Santo Tomás lo dice sin rodeos: de todas las obras divinas es la que más sobrepasa a la razón humana: porque no se puede imaginar -- dice Santo Tomás -- nada más admirable (\emph{Contra Gentiles}, 4, 27). ¿A quién, en efecto, sé le hubiera ocurrido que Dios un día se habría hecho hombre.

Pero esta sublime verdad no encandila al espíritu que la recibe con humildad; antes bien, lo ilumina con la luz nueva y superior. A esta luz el hombre comprende su destino, ve la razón de su existencia, la posibilidad de salir de la miseria, de alcanzar el objetivo de sus esfuerzos. También ve el valor de las creaturas, la ayuda y el obstáculo que éstas pueden constituir para él en su camino hacia Dios. Aquí también, y sobre todo aquí, el misterio de Navidad ejerce su acción unificadora. Y, escrutándolo más profundamente, el creyente no halla por cierto una explicación entre tantas del destino del hombre sino la explicación definitiva: ¡no hay más que un Cristo, no hay más que una salvación! Y tal salvación, lejos de estar reservada a una nación privilegiada, se ofrece a todos. El alma del creyente se siente entonces penetrada por un sentimiento de fraternidad universal; comprende en qué radica la verdadera unidad de destino de la humanidad, que está en el designio de Dios que nos manifestó la Encarnación; comprende el principio fundamental del hombre con Dios y de los hombres entre sí; Navidad se ha vuelto para esa alma lo que es: más que un misterio de unión, un misterio de unidad.

¿Y de dónde procede o dónde tiene su fuente ese misterio? Digámoslo con una palabra que explica todo: es el efecto del amor Este medio divino de unificar al hombre en sí mismo y de unificar al género humano alrededor del Dios hecho hombre, no es y no puede ser una determinación impuesta por la fuerza, a la cual fuera imposible substraerse. Así pues la fe es propuesta y no impuesta. Dios respeta demasiado a su creatura, a la que hizo libre, no esclava. Si la fe y la inteligencia son amigas, ¡cuánto más lo serán la fe y la libertad! ¿Qué valor tendría un amor si fuera una obligación y no una elección?

Así el Infante del pesebre nos revela la última palabra del misterio: Dios se ha encarnado porque amó al hombre y porque quiso salvarlo. Al amor se lo puede aceptar o rechazar. Pero si se lo acepta, produce en el corazón una paz y un gozo indescriptibles: Pax hominibus bonae voluntatis !Quiera Dios, hecho hombre, abrir en esta noche nuestras inteligencias y nuestros corazones para que \textquote{conociendo a Dios visiblemente, seamos atraídos por su intermedio hacia el amor de las cosas invisibles: «ut dum visibiliter Deum cognoscimus, per huno in invisibilium amorem rapiamur!} (Prefacio de Navidad). Amén.

\subsubsection{Homilía (1977): ¿Qué es la Navidad?}

Basílica de San Pedro. Sábado 24 de diciembre de 1977.

¡Hermanos e hijos amadísimos!

Esperáis de nosotros una palabra que resuena ya en vuestros espíritus; el hecho de escucharla una vez más en esta noche y en este lugar os haga reconocer su perenne novedad, su fuerza de verdad, su maravillosa y beatificante alegría. No es nuestra, es celestial. Nuestros labios repiten el \textbf{anuncio del ángel}, que resplandeció en la noche, en Belén, hace 1977 años, y que tras confortar a los humildes y asustados pastores, vigilantes al raso sobre su rebaño, vaticinó el hecho inefable que se estaba realizando en un pesebre cercano: \textquote{Os traigo una buena nueva, una gran alegría, que es para todo el pueblo; pues os ha nacido hoy un Salvador, que es el Mesías, Señor, en la ciudad de David (Belén)} (\emph{Lc} 2, 10-11).

¡Así es, así, hermanos e hijos! Y puesto que es así, queremos extender nuestro grito humilde e impávido a cuantos \textquote{tienen oídos para escuchar} (cf. \emph{Mt} 11, 15). Un hecho y una alegría; ¡he aquí la doble grande noticia!

El hecho parece casi insignificante. Un niño que nace y ¡en qué condiciones tan humillantes! Lo saben nuestros muchachos cuando preparan sus belenes, ingenuos pero auténticos documentos de la realidad evangélica. La realidad evangélica transparenta una concomitante realidad inefable: ese Niño vive de una trascendente filiación divina, \textquote{será llamado Hijo del Altísimo} \emph{(Lc} 1, 32). Hagamos nuestras las expresiones entusiastas de nuestro gran predecesor, San León Magno, que exclama: \textquote{Nuestro Salvador, amadísimos, ha nacido hoy: ¡gocemos! ¡No hay lugar para la tristeza cuando nace la vida que, apagando el temor de la muerte, nos infunde la alegría de la promesa eterna} \emph{(Serm. I de Nativ. Dom)}.

Así que mientras el misterio supremo de la vida trinitaria del Dios único se nos revela en las tres distintas Personas, Padre generante; Hijo engendrado, unidos ambos en el Espíritu Santo, otro misterio llena de maravilla inextinguible nuestra relación religiosa con Dios, abriendo el cielo a la visión de la gloria de la infinita trascendencia divina y, superando en un don de incomparable amor toda distancia, la proximidad, la cercanía de Cristo-Dios hecho hombre nos muestra que El está con nosotros, que está en busca de nosotros: \textquote{Porque se ha manifestado la gracia salutífera de Dios a todos los hombres} (\emph{Tit} 2, 11; 3, 4).

¡Hermanos, hombres todos! ¿Qué es la Navidad sino este acontecimiento histórico, cósmico, sumamente comunitario porque asume proporciones universales y al mismo tiempo incomparablemente íntimo y personal para cada uno de nosotros, pues el Verbo eterno de Dios, en virtud del cual vivimos ya en nuestra existencia natural (cf. \emph{Act} 17, 23-28) ha venido en busca de nosotros? El, eterno, se ha inscrito en el tiempo; El, infinito, se ha como anonadado, \textquote{en la condición de hombre se humilló, hecho obediente hasta la muerte, y muerte de cruz} \emph{(Flp} 2, 6 ss.). Nuestros oídos están habituados a semejante mensaje y nuestros corazones se han hecho sordos a semejante llamada, una llamada de amor: \textquote{tanto amó Dios al mundo\ldots{}} (\emph{Jn} 3, 16); más aún, seamos precisos: cada uno de nosotros puede decir con San Pablo: \textquote{me amó y se entregó por mí} (\emph{Gál} 2, 20).

La Navidad es esta llegada del Verbo de Dios hecho hombre entre nosotros. Cada uno puede decir: ¡por mí! Navidad es este prodigio. Navidad es esta maravilla. Navidad es esta alegría. Nos vienen a los labios las palabras de Pascal: ¡alegría, alegría, alegría, llantos de alegría!

¡Oh! Que esta celebración nocturna de la Natividad de Cristo sea de veras para todos nosotros, para la Iglesia entera, para el mundo, una renovada revelación del misterio inefable de la Encarnación, un manantial de felicidad inagotable! ¡Así sea!

\subsection{San Juan Pablo II, papa}

\subsubsection{Homilía: El Don más grande}

Basílica de San Pedro. Miércoles 24 de diciembre de 1980.

1. Queridos hermanos y hermanas, reunidos en la basílica de San Pedro en Roma, y vosotros todos, los que me escucháis en este momento, desde cualquier parte del globo terrestre.

He aquí que estoy ante vosotros, yo, siervo de Cristo y administrador de los misterios de Dios (cf. \emph{1 Cor} 4, 1), como mensajero de la noche de Belén: \emph{la noche de Belén 1980}.

La noche del nacimiento de Jesucristo, Hijo de Dios, nacido de María Virgen, de la casa de David, de la estirpe de Abraham, padre de nuestra fe, de la generación de los hijos de Adán.

El Hijo de Dios, de la misma sustancia que el Padre, viene al mundo como hombre.

2. Es una noche profunda: \textquote{El pueblo que caminaba en tinieblas vio una luz grande; habitaban tierras de sombras, y una luz les brilló} (palabras del \textbf{Profeta Isaías}, 9, 2).

¿Cómo se cumplen estas palabras en la noche de Belén? He aquí que las tinieblas envuelven la región de Judá y los países cercanos. Sólo en un lugar aparece la luz. Sólo llega a un pequeño grupo de hombres sencillos.

\emph{Son \textbf{los pastores},} que estaban en aquella región \textquote{velando por turno su rebaño} \emph{(Lc} 2, 8).

Solamente en ellos se cumple, esa noche, la profecía de Isaías. \emph{Ven una gran luz:} \textquote{La gloria del Señor los envolvió de claridad y se llenaron de gran temor} \emph{(L}c 2, 9).

Esta luz deslumbra sus ojos y, al mismo tiempo, ilumina sus corazones. He aquí que ellos ya saben: \textquote{Hoy, en la ciudad de David, os ha nacido un Salvador, el Mesías, el Señor} \emph{(Lc} 2, 11). Son los primeros en saberlo. En cambio, hoy lo saben millones de hombres en todo el mundo. \emph{La luz} de la noche de Belén ha llegado a muchos corazones, y sin embargo, al mismo tiempo, permanece la oscuridad. A veces, incluso parece que se hace más intensa\ldots{}

¿Qué puedo pedir en mis plegarías esta noche de Belén 1980, yo, siervo de Cristo y administrador de los misterios de Dios?, ¿qué puedo pedir principalmente, junto con todos vosotros, los que participáis en la luz de esta noche, sino que esta luz \emph{llegue a todas partes,} que encuentre acceso a todos los corazones, que vuelva allá, donde parece que se ha apagado\ldots{}? ¡Que ella \textquote{despierte}!, tal como despertó a los pastores en los campos de las cercanías de Belén.

3. \textquote{Acreciste la alegría, aumentaste el gozo}, palabras del \textbf{Profeta Isaías}.

Los que aquella noche lo acogieron, encontraron \emph{una gran alegría}. La alegría que brota de la luz. La oscuridad del mundo superada por la luz del nacimiento de Dios.

No importa que esta luz, por el momento sea participada, solamente por algunos corazones: que participe de ella la Virgen de Nazaret y su esposo, la Virgen a la que no fue dado traer a su Hijo al mundo bajo el techo de una casa en Belén, \textquote{porque no tenían sitio en la posada}\emph{(Lc} 2, 7). Y participan de esta alegría \emph{los pastores,} iluminados por una gran luz en los campos cerca de la ciudad.

No importa que, en esa primera noche, la noche del nacimiento de Dios, la alegría de este acontecimiento llegue sólo a estos pocos corazones. No importa.

Está \emph{destinada} a \emph{todos los corazones humanos}. ¡Es la alegría del género humano, alegría sobrehumana! ¿Acaso puede haber una alegría mayor que ésta, puede haber una Nueva mejor que ésta: el hombre \emph{ha sido aceptado por Dios para convertirse en hijo} suyo en este Hijo de Dios, que se ha hecho hombre?

Y ésta es una alegría cósmica. Llena a todo el mundo creado: creado por Dios ---mundo que se alejó de Dios a causa del pecado--- y he aquí: restituido de nuevo a Dios mediante el nacimiento de Dios en cuerpo humano.

Es la \emph{alegría cósmica. }

La alegría que llena toda la creación, llamada esta noche a compartirla de nuevo según estas palabras que descienden del cielo: \textquote{Gloria a Dios en el cielo y en la tierra paz a los hombres que Dios ama} (a los hombres de buena voluntad) \emph{(Lc} 2, 14).

Esta noche quiero \emph{estar particularmente cercano a vosotros, a todos vosotros los que sufrís }

y a vosotros, las víctimas del terremoto,

y a vosotros, los que vivís atemorizados por las guerras y las violencias,

y a vosotros, los que os halláis privados de la alegría de esta Santa Misa a medianoche en la Navidad del Señor,

y a vosotros, los que estáis inmovilizados en el lecho del dolor,

y a vosotros, los que habéis caído en la desesperación, en la duda sobre el sentido de la vida y sobre el sentido de todo.

Cercano a todos vosotros.

A vosotros \emph{de modo especial} está destinada esta alegría, que llena los corazones de los pastores de Belén, ella es sobre todo para vosotros. Porque es la alegría de los hombres de buena voluntad, de los que tienen hambre y sed de justicia, de los que lloran, de los que sufren persecución por la justicia.

Que se cumplan en vosotros las palabras del \textbf{Profeta}: \textquote{Acreciste la alegría, aumentaste el gozo\ldots{}} \emph{(Is} 9, 2).

4. \textquote{Se gozan en tu presencia, como se gozan al segar}, palabras de Isaías.

Ciertamente: los hombres sencillos, que viven del trabajo de sus manos, no se presentan ante el recién nacido con las manos vacías. No se presentaron con los corazones vacíos. Llevan los dones.

\emph{Responden con dones al don}.

Queridos hermanos y hermanas, los que estáis reunidos en la basílica de San Pedro y todos los que me escucháis en este momento y en cualquier punto del globo terrestre: ¡en esta noche toda la humanidad ha recibido \emph{el don más grande!} ¡Esta noche cada uno de los hombres recibe el don más grande! Dios mismo se convierte en el don para el hombre. \emph{El hace de sí mismo el \textquote{don}} para la naturaleza humana. ¡Entra en la historia del hombre no sólo ya mediante la palabra que de El viene al hombre, sino mediante el Verbo que se ha hecho carne!

Os pregunto a todos: ¿tenéis conciencia de este don?

Estáis \emph{dispuestos} a responder con el don al don? Tal como los pastores de Belén, que respondieron\ldots{}

Y os deseo desde lo profundo de esta nueva noche de Belén 1980, que aceptéis el don de Dios, que se ha hecho hombre.

¡Os deseo que respondáis con el don al don!

\subsubsection{Homilía: Ha nacido para nosotros}

25 de diciembre de 1998.

1. \textquote{\emph{No temáis, pues os anuncio una gran alegría\ldots{} os ha nacido hoy, en la ciudad de David, un salvador, que es el Cristo Señor}} (\emph{Lc} 2,10-11).

En esta Noche Santa la liturgia nos invita a celebrar con alegría el gran acontecimiento del nacimiento de Jesús en Belén. Como hemos escuchado en el \textbf{Evangelio de Lucas}, viene a la luz en una familia pobre de medios materiales, pero rica de alegría. Nace en un establo, porque para Él no hay lugar en la posada (cf. \emph{Lc} 2,7); es acostado en un pesebre, porque no tiene una cuna; llega al mundo en pleno abandono, ignorándolo todos y, al mismo tiempo, acogido y reconocido en primer lugar por los pastores, que reciben del ángel el anuncio de su nacimiento.

Este acontecimiento esconde un misterio. Lo revelan los coros de los mensajeros celestiales que cantan el nacimiento de Jesús y proclaman \textquote{gloria a Dios en el cielo y en la tierra paz a los hombres que ama el Señor} (\emph{Lc} 2,14). La alabanza a lo largo de los siglos se hace oración que sube del corazón de las multitudes, que en la Noche Santa siguen acogiendo al Hijo de Dios.

2. \emph{Mysterium}: acontecimiento y misterio. Nace un hombre, que es el Hijo eterno del Padre todopoderoso, Creador del cielo y de la tierra: en este acontecimiento extraordinario se revela el misterio de Dios. En la Palabra que se hace hombre se manifiesta el prodigio de Dios encarnado. El misterio ilumina el acontecimiento del nacimiento: un niño es adorado por los pastores en la gruta de Belén. Es \textquote{el Salvador del mundo}, es \textquote{Cristo Señor} (cf. \emph{Lc} 2,11). Sus ojos ven a un recién nacido envuelto en pañales y acostado en un pesebre, y en aquella \textquote{señal}, gracias a la luz interior de la fe, reconocen al Mesías anunciado por los Profetas.

3. Es el Emmanuel, \textquote{Dios-con-nosotros}, que viene a llenar de gracia la tierra. Viene al mundo para transformar la creación. Se hace hombre entre los hombres, para que en Él y por medio de Él todo ser humano pueda renovarse profundamente. Con su nacimiento, nos introduce a todos en la dimensión de la divinidad, concediendo a quien acoge su don con fe la posibilidad de participar de su misma vida divina.

Éste es el significado de la salvación de la que oyen hablar \textbf{los pastores} en la noche de Belén: \textquote{Os ha nacido un Salvador} (\emph{Lc} 2,11). La venida de Cristo entre nosotros es el centro de la historia, que desde entonces adquiere una nueva dimensión. En cierto modo, es Dios mismo que escribe la historia entrando en ella. El acontecimiento de la Encarnación se abre así para abrazar totalmente la historia humana, desde la creación a la parusía. Por esto en la liturgia \textbf{canta toda la creación} expresando su propia alegría: aplauden los ríos; vitorean los campos; se alegran las numerosas islas (cf. \emph{Sal} 98,8; 96,12; 97,1).

Todo ser creado sobre la faz de la tierra acoge este anuncio. En el silencio atónito del universo, resuena con eco cósmico lo que la liturgia pone en boca de la Iglesia: \emph{Christus natus est nobis. Venite adoremus!}

4. Cristo ha nacido para nosotros, ¡venid a adorarlo! Pienso ya en la Navidad del próximo año cuando, si Dios quiere, daré inicio al Gran Jubileo con la apertura de la Puerta Santa. Será un Año Santo verdaderamente grande, porque de manera muy singular se celebrará el bimilenario del acontecimiento-misterio de la Encarnación, con la cual la humanidad alcanzó el culmen de su vocación. Dios se hizo Hombre para hacer al ser humano partícipe de su propia divinidad.

¡Éste es el anuncio de la salvación; éste es el mensaje de la Navidad! La Iglesia lo proclama también, en esta noche, mediante mis palabras, para que lo oigan los pueblos y las naciones de toda la tierra: \emph{Christus natus est nobis} -- Cristo ha nacido para nosotros. \emph{Venite, adoremus!} -- ¡Venid a adorarlo!

\subsubsection{Homilía: Luz de la nueva creación}

Basílica Vaticana. 24 de diciembre del 2001.

1. \emph{\textquote{Populus, quí ambulabat in tenebris, vidit lucem magnam -- El pueblo que caminaba en las tinieblas vio una luz grande}} (\emph{Is} 9, 1).

Todos los años escuchamos estas palabras del \textbf{profeta Isaías}, en el contexto sugestivo de la conmemoración litúrgica del nacimiento de Cristo. Cada año \emph{adquieren un nuevo sabor} y hacen revivir el clima de expectación y de esperanza, de estupor y de gozo, que son típicos de la Navidad.

Al pueblo oprimido y doliente, que caminaba en tinieblas, se le apareció \textquote{una gran luz}. Sí, una luz verdaderamente \textquote{grande}, porque la que irradia de la humildad del pesebre \emph{es la luz de la nueva creación}. Si la primera creación empezó con la luz (cf. \emph{Gn} 1, 3), mucho más resplandeciente y \textquote{grande} es la luz que da comienzo a la nueva creación: ¡es Dios mismo hecho hombre!

La Navidad es acontecimiento de luz, \emph{es la fiesta de la luz}: en el Niño de Belén, la luz originaria vuelve a resplandecer en el cielo de la humanidad y despeja las nubes del pecado. El fulgor del triunfo definitivo de Dios aparece en el horizonte de la historia para proponer a los hombres un nuevo futuro de esperanza.

2. \textquote{\emph{Habitaban tierras de sombras, y una luz les brilló}} (Is 9, 1).

El \textbf{anuncio gozoso} que se acaba de proclamar en nuestra asamblea \emph{vale también para nosotros}, hombres y mujeres en el alba del tercer milenio. La comunidad de los creyentes se reúne en oración para escucharlo en todas las regiones del mundo. Tanto en el frío y la nieve del invierno como en el calor tórrido de los trópicos, \emph{esta noche es Noche Santa para todos}.

Esperado por mucho tiempo, irrumpe por fin el resplandor del nuevo Día. ¡El Mesías ha nacido, el Enmanuel, Dios con nosotros! Ha nacido Aquel que fue preanunciado por los profetas e invocado constantemente por cuantos \textquote{habitaban en tierras de sombras}. En el silencio y la oscuridad de la noche, la luz se hace palabra y mensaje de esperanza.

Pero, ¿no contrasta quizás esta certeza de fe \emph{con la realidad histórica en que vivimos}? Si escuchamos las tristes noticias de las crónicas, estas palabras de luz y esperanza parecen hablar de ensueños. Pero aquí reside precisamente el reto de la fe, que convierte este anuncio en consolador y, al mismo tiempo, exigente. La fe nos hace sentirnos rodeados por el tierno amor de Dios, a la vez que \emph{nos compromete en el amor efectivo a Dios y a los hermanos}.

3. \emph{\textquote{Ha aparecido la gracia de Dios, que trae la salvación para todos los hombres}} (\emph{Tt} 2, 11).

En esta Navidad, nuestros corazones están \emph{preocupados e inquietos} por la persistencia en muchas regiones del mundo de la guerra, de tensiones sociales y de la penuria en que se encuentran muchos seres humanos. Todo buscamos una respuesta que nos tranquilice.

El texto de la \textbf{Carta a Tito} que acabamos de escuchar nos recuerda cómo el nacimiento del Hijo unigénito del Padre \emph{\textquote{trae la salvación}} a todos los rincones del planeta y a cada momento de la historia. Nace para todo hombre y mujer el Niño llamado \emph{\textquote{Maravilla de Consejero, Dios guerrero, Padre perpetuo, Príncipe de la paz}} (\emph{Is} 9, 5). Él tiene la respuesta que puede disipar nuestros miedos y dar nuevo vigor a nuestras esperanzas.

Sí, en esta noche evocadora de recuerdos santos, se hace más firme nuestra confianza en el poder redentor de la \textbf{Palabra hecha carne}. Cuando parecen prevalecer las tinieblas y el mal, Cristo nos repite: ¡no temáis! \emph{Con su venida al mundo, Él ha derrotado el poder del mal}, nos ha liberado de la esclavitud de la muerte y nos ha readmitido al convite de la vida.

Nos toca a nosotros recurrir a la fuerza de su amor victorioso, \emph{haciendo nuestra su lógica de servicio y humildad}. Cada uno de nosotros está llamado a vencer con Él \textquote{el misterio de la iniquidad}, haciéndose testigo de la solidaridad y constructor de la paz. Vayamos, pues, a la gruta de Belén para encontrarlo, pero también para encontrar, en Él, a todos los niños del mundo, a todo hermano lacerado en el cuerpo u oprimido en el espíritu.

4. Los \textbf{pastores} \emph{\textquote{se volvieron dando gloria y alabanza a Dios por lo que habían visto y oído; todo como les habían dicho}} (\emph{Lc} 2, 17).

Al igual que los pastores, también nosotros hemos de sentir en esta noche extraordinaria el deseo de comunicar a los demás la alegría del encuentro con este \emph{\textquote{Niño envuelto en pañales}}, en el cual se revela el poder salvador del Omnipotente. No podemos limitarnos a contemplar extasiados al Mesías que yace en el pesebre, olvidando el compromiso de \emph{ser sus testigos}.

Hemos de volver de prisa a nuestro camino. Debemos volver gozosos de la gruta de Belén para contar por doquier el prodigio del que hemos sido testigos. ¡Hemos encontrado la luz y la vida! En Él se nos ha dado el amor.

5. \emph{\textquote{Un Niño nos ha nacido\ldots{}}}

Te acogemos con alegría, Omnipotente Dios del cielo y de la tierra, que por amor te has hecho Niño \emph{\textquote{en Judea, en la ciudad de David, que se llama Belén}} (cf. \emph{Lc} 2, 4).

Te acogemos agradecidos, nueva Luz que surges en la noche del mundo.

Te acogemos como a nuestro hermano, \textquote{\emph{Príncipe de la paz}}, que has hecho \textquote{\emph{de los dos pueblos una sola cosa}} (\emph{Ef} 2, 14).

Cólmanos de tus dones, Tú que no has desdeñado comenzar la vida humana como nosotros. Haz que seamos hijos de Dios, Tú que por nosotros has querido hacerte hijo del hombre (cf. S. Agustín, \emph{Sermón} 184).

Tú, \textquote{Maravilla de Consejero}, promesa segura de paz; Tú, presencia eficaz del \textquote{Dios poderoso}; Tú, nuestro único Dios, que yaces pobre y humilde en la sombra del pesebre, acógenos al lado de tu cuna.

¡Venid, pueblos de la tierra y abridle las puertas de vuestra historia! Venid a adorar al Hijo de la Virgen María, que ha venido entre nosotros en esta noche preparada por siglos.

Noche de alegría y de luz.

\emph{¡Venite, adoremus!}

\subsubsection{Homilía: ¡Quédate con nosotros!}

Misa de Nochebuena.

Viernes, 24 de diciembre de 2004.

1. \textquote{\emph{Adoro Te devote, latens Deitas}}.

En esta Noche resuenan en mi corazón las primeras palabras del célebre himno eucarístico, que me acompaña día a día en este año dedicado particularmente a la Eucaristía.

En el \emph{\textbf{Hijo de la Virgen}}, \textquote{envuelto en pañales} y \textquote{acostado en un pesebre} (cf. \emph{Lc} 2,12), reconocemos y adoramos \textquote{\emph{el pan bajado del cielo}} (\emph{Jn} 6,41.51), el Redentor venido a la tierra para dar la vida al mundo.

2. \textbf{¡Belén!} La ciudad donde según las Escrituras nació Jesús, en lengua hebrea, significa \textquote{\emph{casa del pan}}. Allí, pues, debía nacer el Mesías, que más tarde diría de sí mismo: \textquote{Yo soy el pan de vida} (\emph{Jn} 6,35.48).

En Belén nació Aquél que, bajo el signo del pan partido, dejaría el memorial de la Pascua. Por esto, la adoración del Niño Jesús, en esta Noche Santa, se convierte en \emph{adoración eucarística}.

3. Te adoramos, Señor, presente realmente en el Sacramento del altar, Pan vivo que das vida al hombre. Te reconocemos como \emph{nuestro único Dios}, frágil Niño que estás indefenso en el pesebre. \textquote{En la plenitud de los tiempos, te hiciste hombre entre los hombres para unir el fin con el principio, es decir, al hombre con Dios} (cf. S. Ireneo, \emph{Adv. haer}., IV,20,4).

Naciste en esta Noche, divino Redentor nuestro, y, por nosotros, peregrino por los \emph{senderos del tiempo}, te hiciste alimento \emph{de vida eterna}.

¡Acuérdate de nosotros, Hijo eterno de Dios, que te encarnaste en el seno de la Virgen María! Te necesita la humanidad entera, marcada por tantas pruebas y dificultades.

¡Quédate con nosotros, Pan vivo bajado del Cielo para nuestra salvación! ¡Quédate con nosotros para siempre! Amén.

\subsection{Benedicto XVI, papa}

\subsubsection{Homilía (2007)}

MISA DE NOCHEBUENA

\textbf{SOLEMNIDAD DE LA NATIVIDAD DEL SEÑOR}

\textbf{\emph{HOMILÍA DEL SANTO PADRE BENEDICTO XVI}}

\emph{Basílica Vaticana\\ 25 de diciembre de 2007}



\emph{Queridos hermanos y hermanas}:

\textquote{A María le llegó el tiempo del parto y dio a luz a su hijo primogénito, lo envolvió en pañales y lo acostó en un pesebre, porque no tenían sitio en la posada} (cf. \emph{Lc} 2,6s). Estas frases, nos llegan al corazón siempre de nuevo. Llegó el momento anunciado por el Ángel en Nazaret: \textquote{Darás a luz un hijo, y le pondrás por nombre Jesús. Será grande, se llamará Hijo del Altísimo} (\emph{Lc} 1,31). Llegó el momento que Israel esperaba desde hacía muchos siglos, durante tantas horas oscuras, el momento en cierto modo esperado por toda la humanidad con figuras todavía confusas: que Dios se preocupase por nosotros, que saliera de su ocultamiento, que el mundo alcanzara la salvación y que Él renovase todo. Podemos imaginar con cuánta preparación interior, con cuánto amor, esperó María aquella hora. El breve inciso, \textquote{lo envolvió en pañales}, nos permite vislumbrar algo de la santa alegría y del callado celo de aquella preparación. Los pañales estaban dispuestos, para que el niño se encontrara bien atendido. Pero en la posada no había sitio. En cierto modo, la humanidad espera a Dios, su cercanía. Pero cuando llega el momento, no tiene sitio para Él. Está tan ocupada consigo misma de forma tan exigente, que necesita todo el espacio y todo el tiempo para sus cosas y ya no queda nada para el otro, para el prójimo, para el pobre, para Dios. Y cuanto más se enriquecen los hombres, tanto más llenan todo de sí mismos y menos puede entrar el otro.

Juan, en su Evangelio, fijándose en lo esencial, ha profundizado en la breve referencia de san Lucas sobre la situación de Belén: \textquote{Vino a su casa, y los suyos no lo recibieron} (1,11). Esto se refiere sobre todo a Belén: el Hijo de David fue a su ciudad, pero tuvo que nacer en un establo, porque en la posada no había sitio para él. Se refiere también a Israel: el enviado vino a los suyos, pero no lo quisieron. En realidad, se refiere a toda la humanidad: Aquel por el que el mundo fue hecho, el Verbo creador primordial entra en el mundo, pero no se le escucha, no se le acoge.

En definitiva, estas palabras se refieren a nosotros, a cada persona y a la sociedad en su conjunto. ¿Tenemos tiempo para el prójimo que tiene necesidad de nuestra palabra, de mi palabra, de mi afecto? ¿Para aquel que sufre y necesita ayuda? ¿Para el prófugo o el refugiado que busca asilo? ¿Tenemos tiempo y espacio para Dios? ¿Puede entrar Él en nuestra vida? ¿Encuentra un lugar en nosotros o tenemos ocupado todo nuestro pensamiento, nuestro quehacer, nuestra vida, con nosotros mismos?

Gracias a Dios, la noticia negativa no es la única ni la última que hallamos en el Evangelio. De la misma manera que en \emph{Lucas} encontramos el amor de su madre María y la fidelidad de san José, la vigilancia de los pastores y su gran alegría, y en \emph{Mateo} encontramos la visita de los sabios Magos, llegados de lejos, así también nos dice \emph{Juan}: \textquote{Pero a cuantos lo recibieron, les da poder para ser hijos de Dios} (\emph{Jn} 1,12). Hay quienes lo acogen y, de este modo, desde fuera, crece silenciosamente, comenzando por el establo, la nueva casa, la nueva ciudad, el mundo nuevo. El mensaje de Navidad nos hace reconocer la oscuridad de un mundo cerrado y, con ello, se nos muestra sin duda una realidad que vemos cotidianamente. Pero nos dice también que Dios no se deja encerrar fuera. Él encuentra un espacio, entrando tal vez por el establo; hay hombres que ven su luz y la transmiten. Mediante la palabra del Evangelio, el Ángel nos habla también a nosotros y, en la sagrada liturgia, la luz del Redentor entra en nuestra vida. Si somos pastores o sabios, la luz y su mensaje nos llaman a ponernos en camino, a salir de la cerrazón de nuestros deseos e intereses para ir al encuentro del Señor y adorarlo. Lo adoramos abriendo el mundo a la verdad, al bien, a Cristo, al servicio de cuantos están marginados y en los cuales Él nos espera.

En algunas representaciones navideñas de la Baja Edad media y de comienzo de la Edad Moderna, el pesebre se representa como edificio más bien desvencijado. Se puede reconocer todavía su pasado esplendor, pero ahora está deteriorado, sus muros en ruinas; se ha convertido justamente en un establo. Aunque no tiene un fundamento histórico, esta interpretación metafórica expresa sin embargo algo de la verdad que se esconde en el misterio de la Navidad. El trono de David, al que se había prometido una duración eterna, está vacío. Son otros los que dominan en Tierra Santa. José, el descendiente de David, es un simple artesano; de hecho, el palacio se ha convertido en una choza. David mismo había comenzado como pastor. Cuando Samuel lo buscó para ungirlo, parecía imposible y contradictorio que un joven pastor pudiera convertirse en el portador de la promesa de Israel. En el establo de Belén, precisamente donde estuvo el punto de partida, vuelve a comenzar la realeza davídica de un modo nuevo: en aquel niño envuelto en pañales y acostado en un pesebre. El nuevo trono desde el cual este David atraerá hacia sí el mundo es la Cruz. El nuevo trono ---la Cruz--- corresponde al nuevo inicio en el establo. Pero justamente así se construye el verdadero palacio davídico, la verdadera realeza. Así, pues, este nuevo palacio no es como los hombres se imaginan un palacio y el poder real. Este nuevo palacio es la comunidad de cuantos se dejan atraer por el amor de Cristo y con Él llegan a ser un solo cuerpo, una humanidad nueva. El poder que proviene de la Cruz, el poder de la bondad que se entrega, ésta es la verdadera realeza. El establo se transforma en palacio; precisamente a partir de este inicio, Jesús edifica la nueva gran comunidad, cuya palabra clave cantan los ángeles en el momento de su nacimiento: \textquote{Gloria a Dios en el cielo y en la tierra paz a los hombres que Dios ama}, hombres que ponen su voluntad en la suya, transformándose en hombres de Dios, hombres nuevos, mundo nuevo.

Gregorio de Nisa ha desarrollado en sus homilías navideñas la misma temática partiendo del mensaje de Navidad en el \emph{Evangelio de Juan: \textquote{}Y puso su morada entre nosotros} (\emph{Jn} 1,14). Gregorio aplica esta palabra de la morada a nuestro cuerpo, deteriorado y débil; expuesto por todas partes al dolor y al sufrimiento. Y la aplica a todo el cosmos, herido y desfigurado por el pecado. ¿Qué habría dicho si hubiese visto las condiciones en las que hoy se encuentra la tierra a causa del abuso de las fuentes de energía y de su explotación egoísta y sin ningún reparo? Anselmo de Canterbury, casi de manera profética, describió con antelación lo que nosotros vemos hoy en un mundo contaminado y con un futuro incierto: \textquote{Todas las cosas se encontraban como muertas, al haber perdido su innata dignidad de servir al dominio y al uso de aquellos que alaban a Dios, para lo que habían sido creadas; se encontraban aplastadas por la opresión y como descoloridas por el abuso que de ellas hacían los servidores de los ídolos, para los que no habían sido creadas} (\emph{PL} 158, 955s). Así, según la visión de Gregorio, el establo del mensaje de Navidad representa la tierra maltratada. Cristo no reconstruye un palacio cualquiera. Él vino para volver a dar a la creación, al cosmos, su belleza y su dignidad: esto es lo que comienza con la Navidad y hace saltar de gozo a los ángeles. La tierra queda restablecida precisamente por el hecho de que se abre a Dios, que recibe nuevamente su verdadera luz y, en la sintonía entre voluntad humana y voluntad divina, en la unificación de lo alto con lo bajo, recupera su belleza, su dignidad. Así, pues, Navidad es la fiesta de la creación renovada. Los Padres interpretan el canto de los ángeles en la Noche santa a partir de este contexto: se trata de la expresión de la alegría porque lo alto y lo bajo, cielo y tierra, se encuentran nuevamente unidos; porque el hombre se ha unido nuevamente a Dios. Para los Padres, forma parte del canto navideño de los ángeles el que ahora ángeles y hombres canten juntos y, de este modo, la belleza del cosmos se exprese en la belleza del canto de alabanza. El canto litúrgico ---siempre según los Padres--- tiene una dignidad particular porque es un cantar junto con los coros celestiales. El encuentro con Jesucristo es lo que nos hace capaces de escuchar el canto de los ángeles, creando así la verdadera música, que acaba cuando perdemos este cantar juntos y este sentir juntos.

En el establo de Belén el cielo y la tierra se tocan. El cielo vino a la tierra. Por eso, de allí se difunde una luz para todos los tiempos; por eso, de allí brota la alegría y nace el canto. Al final de nuestra meditación navideña quisiera citar una palabra extraordinaria de san Agustín. Interpretando la invocación de la oración del Señor: \textquote{Padre nuestro que estás en los cielos}, él se pregunta: ¿qué es esto del cielo? Y ¿dónde está el cielo? Sigue una respuesta sorprendente: Que estás en los cielos significa: en los santos y en los justos. \textquote{En verdad, Dios no se encierra en lugar alguno. Los cielos son ciertamente los cuerpos más excelentes del mundo, pero, no obstante, son cuerpos, y no pueden ellos existir sino en algún espacio; mas, si uno se imagina que el lugar de Dios está en los cielos, como en regiones superiores del mundo, podrá decirse que las aves son de mejor condición que nosotros, porque viven más próximas a Dios. Por otra parte, no está escrito que Dios está cerca de los hombres elevados, o sea de aquellos que habitan en los montes, sino que fue escrito en el Salmo: \textquote{El Señor está cerca de los que tienen el corazón atribulado} (\emph{Sal} 34 {[}33{]}, 19), y la tribulación propiamente pertenece a la humildad. Mas así como el pecador fue llamado \textquote{tierra}, así, por el contrario, el justo puede llamarse \textquote{cielo}} (\emph{Serm. in monte} II 5,17). El cielo no pertenece a la geografía del espacio, sino a la geografía del corazón. Y el corazón de Dios, en la Noche santa, ha descendido hasta un establo: la humildad de Dios es el cielo. Y si salimos al encuentro de esta humildad, entonces tocamos el cielo. Entonces, se renueva también la tierra. Con la humildad de los pastores, pongámonos en camino, en esta Noche santa, hacia el Niño en el establo. Toquemos la humildad de Dios, el corazón de Dios. Entonces su alegría nos alcanzará y hará más luminoso el mundo. Amén.

\subsubsection{Homilía (2010): Un niño que cumple la promesa}

Basílica Vaticana. 24 de diciembre de 2010.

\textquote{Tú eres mi hijo, yo te he engendrado hoy}. La Iglesia comienza la liturgia del Noche Santa con estas palabras del \textbf{\emph{Salmo} segundo}. Ella sabe que estas palabras pertenecían originariamente al rito de la coronación de los reyes de Israel. El rey, que de por sí es un ser humano como los demás hombres, se convierte en \textquote{hijo de Dios} mediante la llamada y la toma de posesión de su cargo: es una especie de adopción por parte de Dios, un acto de decisión, por el que confiere a ese hombre una nueva existencia, lo atrae en su propio ser. La lectura tomada del \textbf{profeta Isaías}, que acabamos de escuchar, presenta de manera todavía más clara el mismo proceso en una situación de turbación y amenaza para Israel: \textquote{Un hijo se nos ha dado: lleva sobre sus hombros el principado} (9,5). La toma de posesión de la función de rey es como un nuevo nacimiento. Precisamente como recién nacido por decisión personal de Dios, como niño procedente de Dios, el rey constituye una esperanza. El futuro recae sobre sus hombros. Él es el portador de la promesa de paz. En la noche de Belén, esta palabra profética se ha hecho realidad de un modo que habría sido todavía inimaginable en tiempos de Isaías. Sí, ahora es realmente un niño el que lleva sobre sus hombros el poder. En Él aparece la nueva realeza que Dios establece en el mundo. Este niño ha nacido realmente de Dios. Es la Palabra eterna de Dios, que une la humanidad y la divinidad. Para este niño valen los títulos de dignidad que el cántico de coronación de Isaías le atribuye: Consejero admirable, Dios poderoso, Padre por siempre, Príncipe de la paz (9,5). Sí, este rey no necesita consejeros provenientes de los sabios del mundo. Él lleva en sí mismo la sabiduría y el consejo de Dios. Precisamente en la debilidad como niño Él es el Dios fuerte, y nos muestra así, frente a los poderes presuntuosos del mundo, la fortaleza propia de Dios.

A decir verdad, las palabras del rito de coronación en Israel eran siempre sólo ritos de esperanza, que preveían a lo lejos un futuro que sería otorgado por Dios. Ninguno de los reyes saludados de este modo se correspondía con lo sublime de dichas palabras. En ellos, todas las palabras sobre la filiación de Dios, sobre su designación como heredero de las naciones, sobre el dominio de las tierras lejanas (\emph{Sal} 2,8), quedaron sólo como referencia a un futuro; casi como carteles que señalan la esperanza, indicaciones que guían hacia un futuro, que en aquel entonces era todavía inconcebible. Por eso, el cumplimiento de la palabra que da comienzo en la noche de Belén es a la vez inmensamente más grande y ---desde el punto de vista del mundo--- más humilde que lo que la palabra profética permitía intuir. Es más grande, porque este niño es realmente Hijo de Dios, verdaderamente \textquote{Dios de Dios, Luz de Luz, engendrado, no creado, de la misma naturaleza del Padre}. Ha quedado superada la distancia infinita entre Dios y el hombre. Dios no solamente se ha inclinado hacia abajo, como dicen los Salmos; Él ha \textquote{descendido} realmente, ha entrado en el mundo, haciéndose uno de nosotros para atraernos a todos a sí. Este niño es verdaderamente el Emmanuel, el Dios-con-nosotros. Su reino se extiende realmente hasta los confines de la tierra. En la magnitud universal de la santa Eucaristía, Él ha hecho surgir realmente islas de paz. En cualquier lugar que se celebra hay una isla de paz, de esa paz que es propia de Dios. Este niño ha encendido en los hombres la luz de la bondad y les ha dado la fuerza de resistir a la tiranía del poder. Él construye su reino desde dentro, partiendo del corazón, en cada generación. Pero también es cierto que no se ha roto la \textquote{vara del opresor}. También hoy siguen marchando con estruendo las botas de los soldados y todavía hoy, una y otra vez, queda la \textquote{túnica empapada de sangre} (\emph{Is} 9,3s). Así, forma parte de esta noche la alegría por la cercanía de Dios. Damos gracias porque el Dios niño se pone en nuestras manos, mendiga, por decirlo así, nuestro amor, infunde su paz en nuestro corazón. Esta alegría, sin embargo, es también una oración: Señor, cumple por entero tu promesa. Quiebra las varas de los opresores. Quema las botas resonantes. Haz que termine el tiempo de las túnicas ensangrentadas. Cumple la promesa: \textquote{La paz no tendrá fin} (\emph{Is} 9,6). Te damos gracias por tu bondad, pero también te pedimos: Muestra tu poder. Erige en el mundo el dominio de tu verdad, de tu amor; el \textquote{reino de justicia, de amor y de paz}.

\textquote{María dio a la luz a su hijo primogénito} (\emph{Lc} 2,7). \textbf{San Lucas} describe con esta frase, sin énfasis alguno, el gran acontecimiento que habían vislumbrado con antelación las palabras proféticas en la historia de Israel. Designa al niño como \textquote{primogénito}. En el lenguaje que se había ido formando en la Sagrada Escritura de la Antigua Alianza, \textquote{primogénito} no significa el primero de otros hijos. \textquote{Primogénito} es un título de honor, independientemente de que después sigan o no otros hermanos y hermanas. Así, en el Libro del \emph{Éxodo} (\emph{Ex} 4,22), Dios llama a Israel \textquote{mi hijo primogénito}, expresando de este modo su elección, su dignidad única, el amor particular de Dios Padre. La Iglesia naciente sabía que esta palabra había recibido una nueva profundidad en Jesús; que en Él se resumen las promesas hechas a Israel. Así, la \emph{Carta a los Hebreos} llama a Jesús simplemente \textquote{el primogénito}, para identificarlo como el Hijo que Dios envía al mundo después de los preparativos en el Antiguo Testamento (cf. \emph{Hb} 1,5-7). El primogénito pertenece de modo particular a Dios, y por eso ---como en muchas religiones--- debía ser entregado de manera especial a Dios y ser rescatado mediante un sacrificio sustitutivo, como relata san Lucas en el episodio de la presentación de Jesús en templo. El primogénito pertenece a Dios de modo particular; está destinado al sacrificio, por decirlo así. El destino del primogénito se cumple de modo único en el sacrificio de Jesús en la cruz. Él ofrece en sí mismo la humanidad a Dios, y une al hombre y a Dios de tal modo que Dios sea todo en todos. San Pablo ha ampliado y profundizado la idea de Jesús como primogénito en las \emph{Cartas a los Colosenses} y \emph{a los Efesios}: Jesús, nos dicen estas Cartas, es el Primogénito de la creación: el verdadero arquetipo del hombre, según el cual Dios ha formado la criatura hombre. El hombre puede ser imagen de Dios, porque Jesús es Dios y Hombre, la verdadera imagen de Dios y el Hombre. Él es el primogénito de los muertos, nos dicen además estas Cartas. En la Resurrección, Él ha desfondado el muro de la muerte para todos nosotros. Ha abierto al hombre la dimensión de la vida eterna en la comunión con Dios. Finalmente, se nos dice: Él es el primogénito de muchos hermanos. Sí, con todo, Él es ahora el primero de más hermanos, es decir, el primero que inaugura para nosotros el estar en comunión con Dios. Crea la verdadera hermandad: no la hermandad deteriorada por el pecado, la de Caín y Abel, de Rómulo y Remo, sino la hermandad nueva en la que somos de la misma familia de Dios. Esta nueva familia de Dios comienza en el momento en el que María envuelve en pañales al \textquote{primogénito} y lo acuesta en el pesebre. Pidámosle: Señor Jesús, tú que has querido nacer como el primero de muchos hermanos, danos la verdadera hermandad. Ayúdanos para que nos parezcamos a ti. Ayúdanos a reconocer tu rostro en el otro que me necesita, en los que sufren o están desamparados, en todos los hombres, y a vivir junto a ti como hermanos y hermanas, para convertirnos en una familia, tu familia.

El \textbf{Evangelio de Navidad} nos relata al final que una multitud de ángeles del ejército celestial alababa a Dios diciendo: \textquote{Gloria a Dios en el cielo, y en la tierra paz a los hombres que Dios ama} (\emph{Lc} 2,14). La Iglesia ha amplificado en el \emph{Gloria} esta alabanza, que los ángeles entonaron ante el acontecimiento de la Noche Santa, haciéndola un himno de alegría sobre la gloria de Dios. \textquote{Por tu gloria inmensa, te damos gracias}. Te damos gracias por la belleza, por la grandeza, por tu bondad, que en esta noche se nos manifiestan. La aparición de la belleza, de lo hermoso, nos hace alegres sin tener que preguntarnos por su utilidad. La gloria de Dios, de la que proviene toda belleza, hace saltar en nosotros el asombro y la alegría. Quien vislumbra a Dios siente alegría, y en esta noche vemos algo de su luz. Pero el mensaje de los ángeles en la Noche Santa habla también de los hombres: \textquote{Paz a los hombres que Dios ama}. La traducción latina de estas palabras, que usamos en la liturgia y que se remonta a Jerónimo, suena de otra manera: \textquote{Paz a los hombres de buena voluntad}. La expresión \textquote{hombres de buena voluntad} ha entrado en el vocabulario de la Iglesia de un modo particular precisamente en los últimos decenios. Pero, ¿cuál es la traducción correcta? Debemos leer ambos textos juntos; sólo así entenderemos la palabra de los ángeles del modo justo. Sería equivocada una interpretación que reconociera solamente el obrar exclusivo de Dios, como si Él no hubiera llamado al hombre a una libre respuesta de amor. Pero sería también errónea una interpretación moralizadora, según la cual, por decirlo así, el hombre podría con su buena voluntad redimirse a sí mismo. Ambas cosas van juntas: gracia y libertad; el amor de Dios, que nos precede, y sin el cual no podríamos amarlo, y nuestra respuesta, que Él espera y que incluso nos ruega en el nacimiento de su Hijo. El entramado de gracia y libertad, de llamada y respuesta, no lo podemos dividir en partes separadas una de otra. Las dos están indisolublemente entretejidas entre sí. Así, esta palabra es promesa y llamada a la vez. Dios nos ha precedido con el don de su Hijo. Una y otra vez, nos precede de manera inesperada. No deja de buscarnos, de levantarnos cada vez que lo necesitamos. No abandona a la oveja extraviada en el desierto en que se ha perdido. Dios no se deja confundir por nuestro pecado. Él siempre vuelve a comenzar con nosotros. No obstante, espera que amemos con Él. Él nos ama para que nosotros podamos convertirnos en personas que aman junto con Él y así haya paz en la tierra.

\textbf{Lucas} no dice que \textbf{los ángeles} cantaran. Él escribe muy sobriamente: el ejército celestial alababa a Dios diciendo: \textquote{Gloria a Dios en el cielo\ldots{}} (\emph{Lc} 2,13s). Pero los hombres siempre han sabido que el hablar de los ángeles es diferente al de los hombres; que precisamente esta noche del mensaje gozoso ha sido un canto en el que ha brillado la gloria sublime de Dios. Por eso, este canto de los ángeles ha sido percibido desde el principio como música que viene de Dios, más aún, como invitación a unirse al canto, a la alegría del corazón por ser amados por Dios. \emph{Cantare amantis est}, dice san Agustín: cantar es propio de quien ama. Así, a lo largo de los siglos, el canto de los ángeles se ha convertido siempre en un nuevo canto de amor y alegría, un canto de los que aman. En esta hora, nosotros nos asociamos llenos de gratitud a este cantar de todos los siglos, que une cielo y tierra, ángeles y hombres. Sí, te damos gracias por tu gloria inmensa. Te damos gracias por tu amor. Haz que seamos cada vez más personas que aman contigo y, por tanto, personas de paz. Amén.

\subsection{Francisco, papa}

\subsubsection{Homilía (2013): Caminar y ver}

Basílica Vaticana. Martes 24 de diciembre del 2013.

1. \textquote{\emph{El pueblo que caminaba en tinieblas vio una luz grande}} (\emph{Is} 9,1).

Esta \textbf{profecía de Isaías} no deja de conmovernos, especialmente cuando la escuchamos en la Liturgia de la Noche de Navidad. No se trata sólo de algo emotivo, sentimental; nos conmueve porque dice la realidad de lo que somos: somos un pueblo en camino, y a nuestro alrededor --y también dentro de nosotros-- hay tinieblas y luces. Y en esta noche, cuando el espíritu de las tinieblas cubre el mundo, se renueva el acontecimiento que siempre nos asombra y sorprende: el pueblo en camino ve una gran luz. Una luz que nos invita a reflexionar en este misterio: misterio de \emph{caminar} y de \emph{ver}.

Caminar. Este verbo nos hace pensar en el curso de la historia, en el largo camino de la historia de la salvación, comenzando por Abrahán, nuestro padre en la fe, a quien el Señor llamó un día a salir de su pueblo para ir a la tierra que Él le indicaría. Desde entonces, nuestra identidad como creyentes es la de peregrinos hacia la tierra prometida. El Señor acompaña siempre esta historia. Él permanece siempre fiel a su alianza y a sus promesas. Porque es fiel, \textquote{Dios es luz sin tiniebla alguna} (\emph{1 Jn} 1,5). Por parte del pueblo, en cambio, se alternan momentos de luz y de tiniebla, de fidelidad y de infidelidad, de obediencia y de rebelión, momentos de pueblo peregrino y momentos de pueblo errante.

También en nuestra historia personal se alternan momentos luminosos y oscuros, luces y sombras. Si amamos a Dios y a los hermanos, caminamos en la luz, pero si nuestro corazón se cierra, si prevalecen el orgullo, la mentira, la búsqueda del propio interés, entonces las tinieblas nos rodean por dentro y por fuera. \textquote{Quien aborrece a su hermano --escribe el apóstol San Juan-- está en las tinieblas, camina en las tinieblas, no sabe adónde va, porque las tinieblas han cegado sus ojos} (\emph{1 Jn} 2,11). Pueblo en camino, sobre todo pueblo peregrino que no quiere ser un pueblo errante.

2. En esta noche, como un haz de luz clarísima, resuena \textbf{el anuncio del Apóstol}: \textquote{\emph{Ha aparecido la gracia de Dios, que trae la salvación para todos los hombres}} (\emph{Tt} 2,11).

La \textbf{gracia} que ha aparecido en el mundo es Jesús, nacido de María Virgen, Dios y hombre verdadero. Ha venido a nuestra historia, ha compartido nuestro camino. Ha venido para librarnos de las tinieblas y darnos la luz. En Él ha aparecido la gracia, la misericordia, la ternura del Padre: Jesús es el Amor hecho carne. No es solamente un maestro de sabiduría, no es un ideal al que tendemos y del que nos sabemos por fuerza distantes, es el sentido de la vida y de la historia que ha puesto su tienda entre nosotros.

3. Los \textbf{pastores} fueron los primeros que vieron esta \textquote{tienda}, que recibieron el anuncio del nacimiento de Jesús. Fueron los primeros porque eran de los últimos, de los marginados. Y fueron los primeros porque estaban en vela aquella noche, guardando su rebaño. Es condición del peregrino velar, y ellos estaban en vela. Con ellos nos quedamos ante el Niño, nos quedamos en silencio. Con ellos damos gracias al Señor por habernos dado a Jesús, y con ellos, desde dentro de nuestro corazón, alabamos su fidelidad: Te bendecimos, Señor, Dios Altísimo, que te has despojado de tu rango por nosotros. Tú eres inmenso, y te has hecho pequeño; eres rico, y te has hecho pobre; eres omnipotente, y te has hecho débil.

Que en esta Noche compartamos \emph{\textbf{la alegría del Evangelio}}: Dios nos ama, nos ama tanto que nos ha dado a su Hijo como nuestro hermano, como luz para nuestras tinieblas. El Señor nos dice una vez más: \textquote{No teman} (\emph{Lc} 2,10). Como dijeron los ángeles a los pastores: \textquote{No teman}. Y también yo les repito a todos: \textquote{No teman}. Nuestro Padre tiene paciencia con nosotros, nos ama, nos da a Jesús como guía en el camino a la tierra prometida. Él es la luz que disipa las tinieblas. Él es la misericordia. Nuestro Padre nos perdona siempre. Y Él es nuestra paz. Amén.

\subsubsection{Urbi et Orbi (2013)} \emph{Miércoles 25 de diciembre de 2013}


\emph{\textquote{Gloria a Dios en el cielo,\\ y en la tierra paz a los hombres que Dios ama }} (\emph{Lc} 2,14).

Queridos hermanos y hermanas de Roma y del mundo entero, ¡buenos días y feliz Navidad!

Hago mías las palabras del cántico de los ángeles, que se aparecieron a los pastores de Belén la noche de la Navidad. Un cántico que une cielo y tierra, elevando al cielo la alabanza y la gloria y saludando a la tierra de los hombres con el deseo de la paz.

Les invito a todos a hacer suyo este cántico, que es el de cada hombre y mujer que vigila en la noche, que espera un mundo mejor, que se preocupa de los otros, intentado hacer humildemente su propio deber.

\emph{Gloria a Dios}.

A esto nos invita la Navidad en primer lugar: a dar gloria a Dios, porque es bueno, fiel, misericordioso. En este día mi deseo es que todos puedan conocer el verdadero rostro de Dios, el Padre que nos ha dado a Jesús. Me gustaría que todos pudieran sentir a Dios cerca, sentirse en su presencia, que lo amen, que lo adoren.

Y que todos nosotros demos gloria a Dios, sobre todo, con la vida, con una vida entregada por amor a Él y a los hermanos.

\emph{Paz a los hombres}.

La verdadera paz -- como sabemos -- no es un equilibrio de fuerzas opuestas. No es pura \textquote{fachada}, que esconde luchas y divisiones. La paz es un compromiso cotidiano, y la paz es también artesanal, que se logra contando con el don de Dios, con la gracia que nos ha dado en Jesucristo.

Viendo al Niño en el Belén, niño de paz, pensemos en los niños que son las víctimas más vulnerables de las guerras, pero pensemos también en los ancianos, en las mujeres maltratadas, en los enfermos\ldots{} ¡Las guerras destrozan tantas vidas y causan tanto sufrimiento!

Demasiadas ha destrozado en los últimos tiempos el conflicto de Siria, generando odios y venganzas. Sigamos rezando al Señor para que el amado pueblo sirio se vea libre de más sufrimientos y las partes en conflicto pongan fin a la violencia y garanticen el acceso a la ayuda humanitaria. Hemos podido comprobar la fuerza de la oración. Y me alegra que hoy se unan a nuestra oración por la paz en Siria creyentes de diversas confesiones religiosas. No perdamos nunca la fuerza de la oración. La fuerza para decir a Dios: Señor, concede tu paz a Siria y al mundo entero. E invito también a los no creyentes a desear la paz, con su deseo, ese deseo que ensancha el corazón: todos unidos, con la oración o con el deseo. Pero todos, por la paz.

Concede la paz, Niño, a la República Centroafricana, a menudo olvidada por los hombres. Pero tú, Señor, no te olvidas de nadie. Y quieres que reine la paz también en aquella tierra, atormentada por una espiral de violencia y de miseria, donde muchas personas carecen de techo, agua y alimento, sin lo mínimo indispensable para vivir. Que se afiance la concordia en Sudán del Sur, donde las tensiones actuales ya han provocado demasiadas víctimas y amenazan la pacífica convivencia de este joven Estado.

Tú, Príncipe de la paz, convierte el corazón de los violentos, allá donde se encuentren, para que depongan las armas y emprendan el camino del diálogo. Vela por Nigeria, lacerada por continuas violencias que no respetan ni a los inocentes e indefensos. Bendice la tierra que elegiste para venir al mundo y haz que lleguen a feliz término las negociaciones de paz entre israelíes y palestinos. Sana las llagas de la querida tierra de Iraq, azotada todavía por frecuentes atentados.

Tú, Señor de la vida, protege a cuantos sufren persecución a causa de tu nombre. Alienta y conforta a los desplazados y refugiados, especialmente en el Cuerno de África y en el este de la República Democrática del Congo. Haz que los emigrantes, que buscan una vida digna, encuentren acogida y ayuda. Que no asistamos de nuevo a tragedias como las que hemos visto este año, con los numerosos muertos en Lampedusa.

Niño de Belén, toca el corazón de cuantos están involucrados en la trata de seres humanos, para que se den cuenta de la gravedad de este delito contra la humanidad. Dirige tu mirada sobre los niños secuestrados, heridos y asesinados en los conflictos armados, y sobre los que se ven obligados a convertirse en soldados, robándoles su infancia.

Señor, del cielo y de la tierra, mira a nuestro planeta, que a menudo la codicia y el egoísmo de los hombres explota indiscriminadamente. Asiste y protege a cuantos son víctimas de los desastres naturales, sobre todo al querido pueblo filipino, gravemente afectado por el reciente tifón.

Queridos hermanos y hermanas, en este mundo, en esta humanidad hoy ha nacido el Salvador, Cristo el Señor. No pasemos de largo ante el Niño de Belén. Dejemos que nuestro corazón se conmueva: no tengamos miedo de esto. No tengamos miedo de que nuestro corazón se conmueva. Tenemos necesidad de que nuestro corazón se conmueva. Dejémoslo que se inflame con la ternura de Dios; necesitamos sus caricias. Las caricias de Dios no producen heridas: las caricias de Dios nos dan paz y fuerza. Tenemos necesidad de sus caricias. El amor de Dios es grande; a Él la gloria por los siglos. Dios es nuestra paz: pidámosle que nos ayude a construirla cada día, en nuestra vida, en nuestras familias, en nuestras ciudades y naciones, en el mundo entero. Dejémonos conmover por la bondad de Dios.

\subsubsection{Homilía (2016): ¿Dónde se manifiesta Dios?}

24 de diciembre del 2016.

\textquote{Ha aparecido la gracia de Dios, que trae la salvación para todos los hombres} (\emph{Tt} 2,11). Las palabras del \textbf{apóstol Pablo} manifiestan el misterio de esta noche santa: ha aparecido la gracia de Dios, su regalo gratuito; en el Niño que se nos ha dado se hace concreto el amor de Dios para con nosotros.

Es una \emph{noche de gloria}, esa gloria proclamada por los ángeles en Belén y también por nosotros en todo el mundo. Es una \emph{noche de alegría}, porque desde hoy y para siempre Dios, el Eterno, el Infinito, es \emph{Dios con nosotros}: no está lejos, no debemos buscarlo en las órbitas celestes o en una idea mística; es cercano, se ha hecho hombre y no se cansará jamás de nuestra humanidad, que ha hecho suya. Es una \emph{noche de luz}: esa luz que, según la \textbf{profecía de Isaías} (cf. 9,1), iluminará a quien camina en tierras de tiniebla, ha aparecido y ha envuelto a \textbf{los pastores} de Belén (cf. \emph{Lc} 2,9).

\textbf{Los pastores} descubren sencillamente que \textquote{un niño nos ha nacido} (\emph{Is} 9,5) y comprenden que toda esta gloria, toda esta alegría, toda esta luz se concentra en un único punto, en ese \emph{signo} que el ángel les ha indicado: \textquote{Encontraréis un niño envuelto en pañales y acostado en un pesebre} (\emph{Lc} 2,12). Este es \emph{el signo de siempre} para encontrar a Jesús. No sólo entonces, sino también hoy. Si queremos celebrar la verdadera Navidad, contemplemos este signo: la sencillez frágil de un niño recién nacido, la dulzura al verlo recostado, la ternura de los pañales que lo cubren. Allí está Dios.

Y con este signo, el \textbf{Evangelio} nos revela una paradoja: habla del emperador, del gobernador, de los grandes de aquel tiempo, pero Dios no se hace presente allí; no aparece en la sala noble de un palacio real, sino en la pobreza de un establo; no en los fastos de la apariencia, sino en la sencillez de la vida; no en el poder, sino en una pequeñez que sorprende. Y para encontrarlo hay que ir allí, donde él está: es necesario reclinarse, abajarse, hacerse pequeño. El Niño que nace nos interpela: nos llama a dejar los engaños de lo efímero para ir a lo esencial, a renunciar a nuestras pretensiones insaciables, a abandonar las insatisfacciones permanentes y la tristeza ante cualquier cosa que siempre nos faltará. Nos hará bien dejar estas cosas para encontrar de nuevo en la sencillez del Niño Dios la paz, la alegría, el sentido luminoso de la vida.

Dejémonos interpelar por el Niño en el pesebre, pero dejémonos interpelar también por los niños que, hoy, no están recostados en una cuna ni acariciados por el afecto de una madre ni de un padre, sino que yacen en los escuálidos \textquote{\emph{pesebres donde se devora su dignidad}}: en el refugio subterráneo para escapar de los bombardeos, sobre las aceras de una gran ciudad, en el fondo de una barcaza repleta de emigrantes. Dejémonos interpelar por los niños a los que no se les deja nacer, por los que lloran porque nadie les sacia su hambre, por los que no tienen en sus manos juguetes, sino armas.

El misterio de la Navidad, que es luz y alegría, interpela y golpea, porque es al mismo tiempo un \emph{misterio de esperanza y de tristeza}. Lleva consigo un \emph{sabor de tristeza}, porque el amor no ha sido acogido, la vida es descartada. Así sucedió a \textbf{José y a María}, que encontraron las puertas cerradas y pusieron a Jesús en un pesebre, \textquote{porque no tenían {[}para ellos{]} sitio en la posada} (v. 7): Jesús nace rechazado por algunos y en la indiferencia de la mayoría. También hoy puede darse la misma indiferencia, cuando Navidad es una fiesta donde los protagonistas somos nosotros en vez de él; cuando las luces del comercio arrinconan en la sombra la luz de Dios; cuando nos afanamos por los regalos y permanecemos insensibles ante quien está marginado. ¡Esta mundanidad nos ha secuestrado la Navidad, es necesario liberarla!

Pero la Navidad tiene sobre todo un \emph{sabor de esperanza} porque, a pesar de nuestras tinieblas, \textbf{la luz de Dios resplandece}. Su luz suave no da miedo; Dios, enamorado de nosotros, nos atrae con su ternura, naciendo pobre y frágil en medio de nosotros, como uno más. Nace en Belén, que significa \textquote{\emph{casa del pan}}. Parece que nos quiere decir que nace como \emph{pan para nosotros} ; viene a la vida para darnos su vida; viene a nuestro mundo para traernos su amor. No viene a devorar y a mandar, sino a nutrir y servir. De este modo hay una línea directa que une el pesebre y la cruz, donde Jesús será \emph{pan partido}: es la línea directa del amor que se da y nos salva, que da luz a nuestra vida, paz a nuestros corazones.

Lo entendieron, en esa noche, \textbf{los pastores}, que estaban entre los marginados de entonces. Pero ninguno está marginado a los ojos de Dios y fueron justamente ellos los invitados a la Navidad. Quien estaba seguro de sí mismo, autosuficiente se quedó en casa entre sus cosas; los pastores en cambio \textquote{fueron corriendo de prisa} (cf. \emph{Lc} 2,16). También nosotros dejémonos interpelar y convocar en esta noche por Jesús, vayamos a él con confianza, desde aquello en lo que nos sentimos marginados, desde nuestros límites, desde nuestros pecados. Dejémonos tocar por la ternura que salva. Acerquémonos a Dios que se hace cercano, detengámonos a mirar el belén, imaginemos el nacimiento de Jesús: la luz y la paz, la pobreza absoluta y el rechazo. Entremos en la verdadera Navidad con los pastores, llevemos a Jesús lo que somos, nuestras marginaciones, nuestras heridas no curadas, nuestros pecados. Así, en Jesús, saborearemos el verdadero espíritu de Navidad: la belleza de ser amados por Dios. Con \textbf{María y José} quedémonos ante el pesebre, ante Jesús que nace como pan para mi vida. Contemplando su amor humilde e infinito, digámosle sencillamente gracias: gracias, porque has hecho todo esto \emph{por mí} .




\subsubsection{Urbi et Orbi (2016)}

\emph{Queridos hermanos y hermanas, feliz Navidad}.

Hoy la Iglesia revive el asombro de la Virgen María, de san José y de los pastores de Belén, contemplando al Niño que ha nacido y que está acostado en el pesebre: Jesús, el Salvador.

En este día lleno de luz, resuena el anuncio del Profeta:

\textquote{Un niño nos ha nacido,\\ un hijo se nos ha dado:\\ lleva a hombros el principado, y es su nombre:\\ Maravilla del Consejero,\\ Dios guerrero,\\ Padre perpetuo,\\ Príncipe de la paz} (\emph{Is} 9, 5).

El poder de un Niño, Hijo de Dios y de María, no es el poder de este mundo, basado en la fuerza y en la riqueza, es el poder del amor. Es el poder que creó el cielo y la tierra, que da vida a cada criatura: a los minerales, a las plantas, a los animales; es la fuerza que atrae al hombre y a la mujer, y hace de ellos una sola carne, una sola existencia; es el poder que regenera la vida, que perdona las culpas, reconcilia a los enemigos, transforma el mal en bien. Es el poder de Dios. Este poder del amor ha llevado a Jesucristo a despojarse de su gloria y a hacerse hombre; y lo conducirá a dar la vida en la cruz y a resucitar de entre los muertos. Es el poder del servicio, que instaura en el mundo el reino de Dios, reino de justicia y de paz.

Por esto el nacimiento de Jesús está acompañado por el canto de los ángeles que anuncian:

\textquote{Gloria a Dios en el cielo,\\ y en la tierra paz a los hombres que Dios ama} (\emph{Lc} 2,14).

Hoy este anuncio recorre toda la tierra y quiere llegar a todos los pueblos, especialmente los golpeados por la guerra y por conflictos violentos, y que sienten fuertemente el deseo de la paz.

Paz a los hombres y a las mujeres de la martirizada Siria, donde demasiada sangre ha sido derramada. Sobre todo en la ciudad de Alepo, escenario, en las últimas semanas, de una de las batallas más atroces, es muy urgente que, respetando el derecho humanitario, se garanticen asistencia y consolación a la extenuada población civil, que se encuentra todavía en una situación desesperada y de gran sufrimiento y miseria. Es hora de que las armas callen definitivamente y la comunidad internacional se comprometa activamente para que se logre una solución negociable y se restablezca la convivencia civil en el País.

Paz para las mujeres y para los hombres de la amada Tierra Santa, elegida y predilecta por Dios. Que los israelíes y los palestinos tengan la valentía y la determinación de escribir una nueva página de la historia, en la que el odio y la venganza cedan el lugar a la voluntad de construir conjuntamente un futuro de recíproca comprensión y armonía. Que puedan recobrar unidad y concordia Irak, Libia, Yemen, donde las poblaciones sufren la guerra y brutales acciones terroristas.

Paz a los hombres y mujeres en las diferentes regiones de África, particularmente en Nigeria, donde el terrorismo fundamentalista explota también a los niños para perpetrar el horror y la muerte. Paz en Sudán del Sur y en la República Democrática del Congo, para que se curen las divisiones y para que todos las personas de buena voluntad se esfuercen para iniciar nuevos caminos de desarrollo y de compartir, prefiriendo la cultura del diálogo a la lógica del enfrentamiento.

Paz a las mujeres y hombres que todavía padecen las consecuencias del conflicto en Ucrania oriental, donde es urgente una voluntad común para llevar alivio a la población y poner en práctica los compromisos asumidos.

Pedimos concordia para el querido pueblo colombiano, que desea cumplir un nuevo y valiente camino de diálogo y de reconciliación. Dicha valentía anime también la amada Venezuela para dar los pasos necesarios con vistas a poner fin a las tensiones actuales y a edificar conjuntamente un futuro de esperanza para la población entera.

Paz a todos los que, en varias zonas, están afrontando sufrimiento a causa de peligros constantes e injusticias persistentes. Que Myanmar pueda consolidar los esfuerzos para favorecer la convivencia pacífica y, con la ayuda de la comunidad internacional, pueda dar la necesaria protección y asistencia humanitaria a los que tienen necesidad extrema y urgente. Que pueda la península coreana ver superadas las tensiones que la atraviesan en un renovado espíritu de colaboración.

Paz a quien ha sido herido o ha perdido a un ser querido debido a viles actos de terrorismo que han sembrado miedo y muerte en el corazón de tantos países y ciudades. Paz ---no de palabra, sino eficaz y concreta--- a nuestros hermanos y hermanas que están abandonados y excluidos, a los que sufren hambre y los que son víctimas de violencia. Paz a los prófugos, a los emigrantes y refugiados, a los que hoy son objeto de la trata de personas. Paz a los pueblos que sufren por las ambiciones económicas de unos pocos y la avaricia voraz del dios dinero que lleva a la esclavitud. Paz a los que están marcados por el malestar social y económico, y a los que sufren las consecuencias de los terremotos u otras catástrofes naturales.

Y paz a los niños, en este día especial en el que Dios se hace niño, sobre todo a los privados de la alegría de la infancia a causa del hambre, de las guerras y del egoísmo de los adultos.

Paz sobre la tierra a todos los hombres de buena voluntad, que cada día trabajan, con discreción y paciencia, en la familia y en la sociedad para construir un mundo más humano y más justo, sostenidos por la convicción de que sólo con la paz es posible un futuro más próspero para todos.

Queridos hermanos y hermanas:

\textquote{Un niño nos ha nacido, un hijo se nos ha dado}: es el \textquote{Príncipe de la paz}. Acojámoslo.

\subsubsection{Homilía (2019): Acoger la gracia de Dios}

24 de diciembre del 2019.

\textquote{El pueblo que caminaba en tinieblas vio una luz grande} (\emph{Is} 9,1). Esta profecía de la \textbf{primera lectura} se realizó en el Evangelio. De hecho, mientras los pastores velaban de noche en sus campos, \textquote{la gloria del Señor los envolvió de claridad} (\emph{Lc} 2,9). En la noche de la tierra apareció una luz del cielo. ¿Qué significa esta luz surgida en la oscuridad? Nos lo sugiere el \textbf{apóstol Pablo}, que nos dijo: \textquote{Se ha manifestado la gracia de Dios}. La gracia de Dios, \textquote{que trae la salvación para todos los hombres} (\emph{Tt} 2,11), ha envuelto al mundo esta noche.

Pero, ¿qué es esta gracia? Es el amor divino, el amor que transforma la vida, renueva la historia, libera del mal, infunde paz y alegría. En esta noche, el amor de Dios se ha mostrado a nosotros: es Jesús. En Jesús, el Altísimo se hizo pequeño para ser amado por nosotros. En Jesús, Dios se hizo Niño, para dejarse abrazar por nosotros. Pero, podemos todavía preguntarnos, ¿por qué san Pablo llama \textquote{gracia} a la venida de Dios al mundo? Para decirnos que es completamente gratuita. Mientras que aquí en la tierra todo parece responder a la lógica de dar para tener, Dios llega gratis. Su amor no es negociable: no hemos hecho nada para merecerlo y nunca podremos recompensarlo.

\emph{Se ha \textbf{manifestado} la gracia de Dios}. En esta noche nos damos cuenta de que, aunque no estábamos a la altura, Él se hizo pequeñez para nosotros; mientras andábamos ocupados en nuestros asuntos, Él vino entre nosotros. La Navidad nos recuerda que Dios sigue amando a cada hombre, incluso al peor. A mí, a ti, a cada uno de nosotros, Él nos dice hoy \emph{:} \textquote{Te amo y siempre te amaré, eres precioso a mis ojos}. Dios no te ama porque piensas correctamente y te comportas bien; Él te ama y basta. Su amor es incondicional, no depende de ti. Puede que tengas ideas equivocadas, que hayas hecho de las tuyas; sin embargo, el Señor no deja de amarte. ¿Cuántas veces pensamos que Dios es bueno si nosotros somos buenos, y que nos castiga si somos malos? Pero no es así. Aun en nuestros pecados continúa amándonos. Su amor no cambia, no es quisquilloso; es fiel, es paciente. Este es el regalo que encontramos en Navidad: descubrimos con asombro que el Señor es toda la gratuidad posible, toda la ternura posible. Su gloria no nos deslumbra, su presencia no nos asusta. Nació pobre de todo, para conquistarnos con la riqueza de su amor.

\emph{Se ha \textbf{manifestado} la gracia de Dios}. Gracia es sinónimo de belleza. En esta noche, redescubrimos en la belleza del amor de Dios, también nuestra belleza, porque somos \emph{los amados de Dios}. En el bien y en el mal, en la salud y en la enfermedad, felices o tristes, a sus ojos nos vemos hermosos: no por lo que hacemos sino por lo que somos. Hay en nosotros una belleza indeleble, intangible; una belleza irreprimible que es el núcleo de nuestro ser. Dios nos lo recuerda hoy, tomando con amor nuestra humanidad y haciéndola suya, \textquote{desposándose con ella} para siempre.

De hecho, la \textbf{\textquote{gran alegría} anunciada a los pastores} esta noche es \textquote{para todo el pueblo}. En aquellos pastores, que ciertamente no eran santos, también estamos nosotros, con nuestras flaquezas y debilidades. Así como los llamó a ellos, Dios también nos llama a nosotros, porque nos ama. Y, en las noches de la vida, a nosotros como a ellos nos dice: \textquote{No temáis} (\emph{Lc} 2,10). ¡Ánimo, no hay que perder la confianza, no hay que perder la esperanza, no hay que pensar que amar es tiempo perdido! En esta noche, el amor venció al miedo, apareció una nueva esperanza, la luz amable de Dios venció la oscuridad de la arrogancia humana. ¡Humanidad, Dios te ama, se hizo hombre por ti, ya no estás sola!

Queridos hermanos y hermanas: ¿Qué hacer ante esta gracia? Una sola cosa: \emph{acoger el don}. Antes de ir en busca de Dios, dejémonos buscar por Él, porque Él nos busca primero. No partamos de nuestras capacidades, sino de su gracia, porque Él es Jesús, el Salvador. Pongamos nuestra mirada en el Niño y dejémonos envolver por su ternura. Ya no tendremos más excusas para no dejarnos amar por Él: Lo que sale mal en la vida, lo que no funciona en la Iglesia, lo que no va bien en el mundo ya no será una justificación. Pasará a un segundo plano, porque frente al amor excesivo de Jesús, que es todo mansedumbre y cercanía, no hay excusas. La pregunta que surge en Navidad es: \textquote{¿Me dejo amar por Dios? ¿Me abandono a su amor que viene a salvarme?}.

Un regalo así, tan grande, merece mucha gratitud. Acoger la gracia es saber \emph{agradecer}. Pero nuestras vidas a menudo transcurren lejos de la gratitud. Hoy es el día adecuado para acercarse al sagrario, al belén, al pesebre, para agradecer. Acojamos el don que es Jesús, para luego \emph{transformarnos en don} como Jesús. Convertirse en don es dar sentido a la vida y es la mejor manera de cambiar el mundo: cambiamos nosotros, cambia la Iglesia, cambia la historia cuando comenzamos a no querer cambiar a los otros, sino a nosotros mismos, haciendo de nuestra vida un don.

Jesús nos lo manifiesta esta noche. No cambió la historia constriñendo a alguien o a fuerza de palabras, sino con el don de su vida. No esperó a que fuéramos buenos para amarnos, sino que se dio a nosotros gratuitamente. Tampoco nosotros podemos esperar que el prójimo cambie para hacerle el bien, que la Iglesia sea perfecta para amarla, que los demás nos tengan consideración para servirlos. Empecemos nosotros. Así es como se acoge el don de la gracia. Y la santidad no es sino custodiar esta gratuidad.

Una hermosa leyenda cuenta que, cuando Jesús nació, los pastores corrían hacia la gruta llevando muchos regalos. Cada uno llevaba lo que tenía: unos, el fruto de su trabajo, otros, algo de valor. Pero mientras todos los pastores se esforzaban, con generosidad, en llevar lo mejor, había uno que no tenía nada. Era muy pobre, no tenía nada que ofrecer. Y mientras los demás competían en presentar sus regalos, él se mantenía apartado, con vergüenza. En un determinado momento, san José y la Virgen se vieron en dificultad para recibir todos los regalos, muchos, sobre todo María, que debía tener en brazos al Niño. Entonces, viendo a aquel pastor con las manos vacías, le pidió que se acercara. Y le puso a Jesús en sus manos. El pastor, tomándolo, se dio cuenta de que había recibido lo que no se merecía, que tenía entre sus brazos el regalo más grande de la historia. Se miró las manos, y esas manos que le parecían siempre vacías se habían convertido en la cuna de Dios. Se sintió amado y, superando la vergüenza, comenzó a mostrar a Jesús a los otros, porque no podía sólo quedarse para él el regalo de los regalos.

Querido hermano, querida hermana: Si tus manos te parecen vacías, si ves tu corazón pobre en amor, esta noche es para ti. \emph{Se ha manifestado la gracia de Dios} para resplandecer en tu vida. Acógela y brillará en ti la luz de la Navidad.


\subsubsection{Urbi et Orbi (2019)} \emph{Balcón central de la Basílica Vaticana\\ Miércoles, 25 de diciembre de 2019}





\textquote{El pueblo que caminaba en tinieblas vio una luz grande} (\emph{Is} 9,1)

\emph{Queridos hermanos y hermanas: ¡Feliz Navidad!}

En el seno de la madre Iglesia, esta noche ha nacido nuevamente el Hijo de Dios hecho hombre. Su nombre es Jesús, que significa Dios salva. El Padre, Amor eterno e infinito, lo envió al mundo no para condenarlo, sino para salvarlo (cf. \emph{Jn} 3,17). El Padre lo dio, con inmensa misericordia. Lo entregó para todos. Lo dio para siempre. Y Él nació, como pequeña llama encendida en la oscuridad y en el frío de la noche.

Aquel Niño, nacido de la Virgen María, es la Palabra de Dios hecha carne. La Palabra que orientó el corazón y los pasos de Abrahán hacia la tierra prometida, y sigue atrayendo a quienes confían en las promesas de Dios. La Palabra que guio a los hebreos en el camino de la esclavitud a la libertad, y continúa llamando a los esclavos de todos los tiempos, también hoy, a salir de sus prisiones. Es Palabra, más luminosa que el sol, encarnada en un pequeño hijo del hombre, Jesús, luz del mundo.

Por esto el profeta exclama: \textquote{El pueblo que caminaba en tinieblas vio una luz grande} (\emph{Is} 9,1). Sí, hay tinieblas en los corazones humanos, pero más grande es la luz de Cristo. Hay tinieblas en las relaciones personales, familiares, sociales, pero más grande es la luz de Cristo. Hay tinieblas en los conflictos económicos, geopolíticos y ecológicos, pero más grande es la luz de Cristo.

Que Cristo sea luz para tantos niños que sufren la guerra y los conflictos en Oriente Medio y en diversos países del mundo. Que sea consuelo para el amado pueblo sirio, que todavía no ve el final de las hostilidades que han desgarrado el país en este decenio. Que remueva las conciencias de los hombres de buena voluntad. Que inspire hoy a los gobernantes y a la comunidad internacional para encontrar soluciones que garanticen la seguridad y la convivencia pacífica de los pueblos de la región y ponga fin a sus sufrimientos. Que sea apoyo para el pueblo libanés, de este modo pueda salir de la crisis actual y descubra nuevamente su vocación de ser un mensaje de libertad y de armoniosa coexistencia para todos.

Que el Señor Jesús sea luz para la Tierra Santa donde Él nació, Salvador del mundo, y donde continúa la espera de tantos que, incluso en la fatiga, pero sin desesperarse, aguardan días de paz, de seguridad y de prosperidad. Que sea consolación para Irak, atravesado por tensiones sociales, y para Yemen, probado por una grave crisis humanitaria.

Que el pequeño Niño de Belén sea esperanza para todo el continente americano, donde diversas naciones están pasando un período de agitaciones sociales y políticas. Que reanime al querido pueblo venezolano, probado largamente por tensiones políticas y sociales, y no le haga faltar el auxilio que necesita. Que bendiga los esfuerzos de cuantos se están prodigando para favorecer la justicia y la reconciliación, y se desvelan para superar las diversas crisis y las numerosas formas de pobreza que ofenden la dignidad de cada persona.

Que el Redentor del mundo sea luz para la querida Ucrania, que aspira a soluciones concretas para alcanzar una paz duradera.

Que el Señor recién nacido sea luz para los pueblos de África, donde perduran situaciones sociales y políticas que a menudo obligan a las personas a emigrar, privándolas de una casa y de una familia. Que haya paz para la población que vive en las regiones orientales de la República Democrática del Congo, martirizada por conflictos persistentes. Que sea consuelo para cuantos son perseguidos a causa de su fe, especialmente los misioneros y los fieles secuestrados, y para cuantos caen víctimas de ataques por parte de grupos extremistas, sobre todo en Burkina Faso, Malí, Níger y Nigeria.

Que el Hijo de Dios, que bajó del cielo a la tierra, sea defensa y apoyo para cuantos, a causa de estas y otras injusticias, deben emigrar con la esperanza de una vida segura. La injusticia los obliga a atravesar desiertos y mares, transformados en cementerios. La injusticia los fuerza a sufrir abusos indecibles, esclavitudes de todo tipo y torturas en campos de detención inhumanos. La injusticia les niega lugares donde podrían tener la esperanza de una vida digna y les hace encontrar muros de indiferencia.

Que el Emmanuel sea luz para toda la humanidad herida. Que ablande nuestro corazón, a menudo endurecido y egoísta, y nos haga instrumentos de su amor. Que, a través de nuestros pobres rostros, regale su sonrisa a los niños de todo el mundo, especialmente a los abandonados y a los que han sufrido a causa de la violencia. Que, a través de nuestros brazos débiles, vista a los pobres que no tienen con qué cubrirse, dé el pan a los hambrientos, cure a los enfermos. Que, por nuestra frágil compañía, esté cerca de las personas ancianas y solas, de los migrantes y de los marginados. Que, en este día de fiesta, conceda su ternura a todos, e ilumine las tinieblas de este mundo.


\section{Temas}

El Directorio Homilético recoge los temas de la Navidad en un solo
grupo, ver página 163.

\chapter{Misa de la Aurora}

\section{Lecturas}

PRIMERA LECTURA

Del libro del profeta Isaías 62, 11-12

Mira a tu salvador, que llega

El Señor hace oír esto

hasta el confín de la tierra:

«Decid a la hija de Sión:

Mira a tu salvador, que llega,

el premio de su victoria lo acompaña,

la recompensa lo precede».

Los llamarán \textquote{Pueblo santo},

\textquote{Redimidos del Señor},

y a ti te llamarán \textquote{Buscada},

\textquote{Ciudad no abandonada}.

SALMO RESPONSORIAL

Salmo 96, 1 y 6. 11-12

Hoy brillará una luz sobre nosotros, porque nos ha nacido el Señor

℣. El Señor reina, la tierra goza,

se alegran las islas innumerables.

Los cielos pregonan su justicia,

y todos los pueblos contemplan su gloria. ℟.

℣. Amanece la luz para el justo,

y la alegría para los rectos de corazón.

Alegraos, justos, con el Señor,

celebrad su santo nombre. ℟.

SEGUNDA LECTURA

De la carta del apóstol san Pablo a Tito 3, 4-7

Según su propia misericordia, nos salvó

Querido hermano:

Cuando se manifestó la bondad de Dios nuestro Salvador y su amor al
hombre, no por las obras de justicia que hubiéramos hecho nosotros,
sino, según su propia misericordia, nos salvó por el baño del nuevo
nacimiento y de la renovación del Espíritu Santo, que derramó
copiosamente sobre nosotros por medio de Jesucristo nuestro Salvador,
para que, justificados por su gracia, seamos, en esperanza, herederos de
la vida eterna.

EVANGELIO

Del Santo Evangelio según san Lucas 2, 15-20

Los pastores encontraron a María y a José y al niño

Sucedió que, cuando los ángeles se marcharon al cielo, los pastores se
decían unos a otros:

«Vayamos, pues, a Belén, y veamos lo que ha sucedido

y que el Señor nos ha comunicado».

Fueron corriendo y encontraron a María y a José, y al niño acostado en
el pesebre. Al verlo, contaron lo que se les había dicho de aquel niño.

Todos los que lo oían se admiraban de lo que les habían dicho los
pastores. María, por su parte, conservaba todas estas cosas,
meditándolas en su corazón. Y se volvieron los pastores dando gloria y
alabanza a Dios por todo lo que habían oído y visto, conforme a lo que
se les había dicho.

\section{Comentario Patrístico}

\subsection{Teodoto de Ancira}

El Dueño de todo vino en forma de siervo

Sermón en la Natividad del Salvador

Edic. Schwartz, ACO t. 3, parte 1, 157-159

El Dueño de todo vino en forma de siervo, revestido de pobreza, para no ahuyentar la presa. Habiendo elegido para nacer la inseguridad de un campo indefenso, nace de una pobrecilla virgen, inmerso en la pobreza, para, en silencio, dar caza al hombre y así salvarlo. Pues de haber nacido en medio del boato, y si se hubiera rodeado de riqueza, los infieles habrían dicho, y con razón, que había sido la abundancia de riqueza la que había operado la transformación de la redondez de la tierra. Y si hubiera elegido Roma, entonces la ciudad más poderosa, hubieran pensado que era el poderío de sus ciudadanos el que había cambiado el mundo.

De haber sido el hijo del emperador, su obra benéfica se habría inscrito en el haber de las influencias. Si hubiera nacido hijo de un legislador, su reforma social se habría atribuido al ordenamiento jurídico. Y ¿qué es lo que hizo? Escogió todo lo vil y pobre, todo lo mediocre e ignorado por la gran masa, a fin de dar a conocer que la divinidad era la única transformadora de la tierra. He aquí por qué eligió una madre pobre, una patria todavía más pobre, y él mismo falto de recursos.

Aprende la lección del pesebre. No habiendo lecho en que acostar al Señor, se le coloca en un pesebre, y la indigencia de lo más imprescindible se convierte en privilegiado anuncio de la profecía. Fue colocado en un pesebre para indicar que iba a convertirse en manjar incluso de los irracionales. En efecto, viviendo en la pobreza y yaciendo en un pesebre, la Palabra e Hijo de Dios atrae a sí tanto a los ricos como a los pobres, a los elocuentes como a los de premiosa palabra.

Fíjate cómo la ausencia de bienes dio cumplimiento a la profecía, y cómo la pobreza ha hecho accesible a todos a aquel que por nosotros se hizo pobre. Nadie tuvo reparo en acudir por temor a las soberbias riquezas de Cristo; nadie sintió bloqueado el acceso por la magnificencia del poder: se mostró cercano y pobre, ofreciéndose a sí mismo por la salvación de todos.

Mediante la corporeidad asumida, el Verbo de Dios se hace presente en el pesebre, para hacer posible que todos, racionales e irracionales, participen del manjar de salvación. Y pienso que esto es lo que ya antes había pregonado el profeta, desvelándonos el misterio de este pesebre: \emph{Conoce el buey a su amo, y el asno, el pesebre de su dueño; Israel no me conoce, mi pueblo no recapacita}. El que es rico, por nosotros se hizo pobre, haciendo fácilmente perceptible a todos la salvación con la fuerza de la divinidad. Refiriéndose a esto decía asimismo el gran Pablo: \emph{Siendo rico, por nosotros se hizo pobre, para que nosotros, con su pobreza, nos hagamos ricos}.

Pero, ¿quién era el que enriquecía?, ¿de qué enriquecía?, y, ¿cómo se hizo él pobre por nosotros? Dime, por favor: ¿quién, siendo rico, se ha hecho pobre con mi pobreza? ¿Quizá el que apareció hecho hombre? Pero éste nunca fue rico, sino que nació pobre de padres pobres. ¿Quién, pues, era rico y con qué nos enriquecía el que por nosotros se hizo pobre? Dios ---dice--- enriquece a la criatura. Es, pues, Dios quien se hizo pobre, haciendo suya la pobreza del que se hacía visible; él es efectivamente rico en su divinidad, y por nosotros se hizo pobre.

¡Cuánta emoción debería acompañarnos mientras colocamos en el belén las montañas, los riachuelos, las ovejas y los pastores! De esta manera recordamos, como lo habían anunciado los profetas, que toda la creación participa en la fiesta de la venida del Mesías. Los ángeles y la estrella son la señal de que también nosotros estamos llamados a ponernos en camino para llegar a la gruta y adorar al Señor.

\textquote{Vayamos, pues, a Belén, y veamos lo que ha sucedido y que el Señor nos ha comunicado} (\emph{Lc} 2,15), así dicen los pastores después del anuncio hecho por los ángeles. Es una enseñanza muy hermosa que se muestra en la sencillez de la descripción. A diferencia de tanta gente que pretende hacer otras mil cosas, los pastores se convierten en los primeros testigos de lo esencial, es decir, de la salvación que se les ofrece. Son los más humildes y los más pobres quienes saben acoger el acontecimiento de la encarnación. A Dios que viene a nuestro encuentro en el Niño Jesús, los pastores responden poniéndose en camino hacia Él, para un encuentro de amor y de agradable asombro. Este encuentro entre Dios y sus hijos, gracias a Jesús, es el que da vida precisamente a nuestra religión y constituye su singular belleza, y resplandece de una manera particular en el pesebre.

Tenemos la costumbre de poner en nuestros belenes muchas figuras simbólicas, sobre todo, las de mendigos y de gente que no conocen otra abundancia que la del corazón. Ellos también están cerca del Niño Jesús por derecho propio, sin que nadie pueda echarlos o alejarlos de una cuna tan improvisada que los pobres a su alrededor no desentonan en absoluto. De hecho, los pobres son los privilegiados de este misterio y, a menudo, aquellos que son más capaces de reconocer la presencia de Dios en medio de nosotros.

Los pobres y los sencillos en el Nacimiento recuerdan que Dios se hace hombre para aquellos que más sienten la necesidad de su amor y piden su cercanía. Jesús, \textquote{manso y humilde de corazón} (\emph{Mt} 11,29), nació pobre, llevó una vida sencilla para enseñarnos a comprender lo esencial y a vivir de ello. Desde el belén emerge claramente el mensaje de que no podemos dejarnos engañar por la riqueza y por tantas propuestas efímeras de felicidad\ldots{}

\textbf{Francisco, papa,} Carta apostólica \emph{Admirabile signum,} nn. 5-6.

\section{Homilías}

Las homilías para esta celebración están tomadas de textos de las Padres de la Iglesia que tocan algunos aspectos de la Navidad en particular o relacionados con alguno de los textos bíblicos que se leen en la misma.

Conviene señalar que estas homilías pueden iluminar aspectos de cualquiera de las otras celebraciones durante el tiempo de Navidad.

\subsection{San Bernardo de Claraval, abad}

\subsubsection{Sermón: Plenitud de los tiempos y de la divinidad}

Sermón 1-2 en la Epifanía del Señor, 1-2: PL 183, 141-143.

\emph{Ha aparecido la bondad de Dios, nuestro Salvador, y su amor al hombre}. Gracias sean dadas a Dios, que ha hecho abundar en nosotros el consuelo en medio de esta peregrinación, de este destierro, de esta miseria.

Antes de que apareciese la humanidad de nuestro Salvador, su bondad se hallaba también oculta, aunque ésta ya existía, pues la misericordia del Señor es eterna. ¿Pero cómo, a pesar de ser tan inmensa, iba a poder ser reconocida? Estaba prometida, pero no se la alcanzaba a ver; por lo que muchos no creían en ella. Efectivamente, \emph{en distintas ocasiones y de muchas maneras habló Dios por los profetas}. Y decía: Yo tengo \emph{designios de paz y no de aflicción}. Pero ¿qué podía responder el hombre que sólo experimentaba la aflicción e ignoraba la paz? ¿Hasta cuándo vais a estar diciendo: \emph{\textquote{Paz, paz}, y no hay paz?} A causa de lo cual \emph{los mensajeros de paz lloraban amargamente,} diciendo: \emph{Señor, ¿quién creyó nuestro anuncio?} Pero ahora los hombres tendrán que creer a sus propios ojos, ya que \emph{los testimonios de Dios se han vuelto absolutamente creíbles}. Pues para que ni una vista perturbada pueda dejar de verlo, \emph{puso su tienda al sol}.

Pero de lo que se trata ahora no es de la promesa de la paz, sino de su envío; no de la dilatación de su entrega, sino de su realidad; no de su anuncio profético, sino de su presencia. Es como si Dios hubiera vaciado sobre la tierra un saco lleno de su misericordia; un saco que habría de desfondarse en la pasión, para que se derramara nuestro precio, oculto en él; un saco pequeño, pero lleno. Ya que \emph{un niño se nos ha dado,} pero \emph{en quien habita toda la plenitud de la divinidad}. Ya que, cuando llegó la plenitud del tiempo, hizo también su aparición la plenitud de la divinidad. Vino en carne mortal para que, al presentarse así ante quienes eran carnales, en la aparición de su humanidad se reconociese su bondad. Porque, cuando se pone de manifiesto la humanidad de Dios, ya no puede mantenerse oculta su bondad. ¿De qué manera podía manifestar mejor su bondad que asumiendo mi carne? La mía, no la de Adán, es decir, no la que Adán tuvo antes del pecado.

¿Hay algo que pueda declarar más inequívocamente la misericordia de Dios que el hecho de haber aceptado nuestra miseria? ¿Qué hay más rebosante de piedad que la Palabra de Dios convertida en tan poca cosa por nosotros? \emph{Señor, ¿qué es el hombre, para que te acuerdes de él, el ser humano, para darle poder?} Que deduzcan de aquí los hombres lo grande que es el cuidado que Dios tiene de ellos; que se enteren de lo que Dios piensa y siente sobre ellos. No te preguntes, tú, que eres hombre, por lo que has sufrido, sino por lo que sufrió él. Deduce de todo lo que sufrió por ti, en cuánto te tasó, y así su bondad se te hará evidente por su humanidad. Cuanto más pequeño se hizo en su humanidad, tanto más grande se reveló en su bondad; y cuanto más se dejó envilecer por mí, tanto más querido me es ahora. \emph{Ha aparecido} ---dice el Apóstol--- \emph{la bondad de Dios, nuestro Salvador, y su amor al hombre}. Grandes y manifiestos son, sin duda, la bondad y el amor de Dios, y gran indicio de bondad reveló quien se preocupó de añadir a la humanidad el nombre de Dios.

\subsection{San Hipólito de Roma, presbítero}

\subsubsection{Tratado: La Palabra hecha carne nos diviniza}

Contra las herejías, Cap. 10, 33-34: PG 16, 3452-3453.

No prestamos nuestra adhesión a discursos vacíos ni nos dejamos seducir por pasajeros impulsos del corazón, como tampoco por el encanto de discursos elocuentes, sino que nuestra fe se apoya en las palabras pronunciadas por el poder divino. Dios se las ha ordenado a su Palabra, y la Palabra las ha pronunciado, tratando con ellas de apartar al hombre de la desobediencia, no dominándolo como a un esclavo por la violencia que coacciona, sino apelando a su libertad y plena decisión.

Fue el Padre quien envió la Palabra, al fin de los tiempos. Quiso que no siguiera hablando por medio de un profeta, ni que se hiciera adivinar mediante anuncios velados; sino que le dijo que se manifestara a rostro descubierto, a fin de que el mundo, al verla, pudiera salvarse.

Sabemos que esta Palabra tomó un cuerpo de la Virgen, y que asumió al hombre viejo, transformándolo. Sabemos que se hizo hombre de nuestra misma condición, porque, si no hubiera sido así, sería inútil que luego nos prescribiera imitarle como maestro. Porque, si este hombre hubiera sido de otra naturaleza, ¿cómo habría de ordenarme las mismas cosas que él hace, a mí, débil por nacimiento, y cómo sería entonces bueno y justo?

Para que nadie pensara que era distinto de nosotros, se sometió a la fatiga, quiso tener hambre y no se negó a pasar sed, tuvo necesidad de descanso y no rechazó el sufrimiento, obedeció hasta la muerte y manifestó su resurrección, ofreciendo en todo esto su humanidad como primicia, para que tú no te descorazones en medio de tus sufrimientos, sino que, aun reconociéndote hombre, aguardes a tu vez lo mismo que Dios dispuso para él.

Cuando contemples ya al verdadero Dios, poseerás un cuerpo inmortal e incorruptible, junto con el alma, y obtendrás el reino de los cielos, porque, sobre la tierra, habrás reconocido al Rey celestial; serás íntimo de Dios, coheredero de Cristo, y ya no serás más esclavo de los deseos, de los sufrimientos y de las enfermedades, porque habrás llegado a ser dios.

Porque todos los sufrimientos que has soportado, por ser hombre, te los ha dado Dios precisamente porque lo eras; pero Dios ha prometido también otorgarte todos sus atributos, una vez que hayas sido divinizado y te hayas vuelto inmortal. Es decir, \emph{conócete a ti mismo} mediante el conocimiento de Dios, que te ha creado, porque conocerlo y ser conocido por él es la suerte de su elegido.

No seáis vuestros propios enemigos, ni os volváis hacia atrás, porque Cristo es \emph{el Dios que está por encima de todo:} él ha ordenado purificar a los hombres del pecado, y él es quien renueva al hombre viejo, al que ha llamado desde el comienzo imagen suya, mostrando, por su impronta en ti, el amor que te tiene. Y, si tú obedeces sus órdenes y te haces buen imitador de este buen maestro, llegarás a ser semejante a él y recompensado por él; porque Dios no es pobre, y te divinizará para su gloria.


\subsection{San Juan Pablo II, papa}

\subsubsection{Catequesis (1983)} \textbf{\emph{AUDIENCIA GENERAL\\ }}\\ \emph{Miércoles 28 de diciembre de 1983}



1. El misterio de Navidad hace resonar en nuestros oídos el canto con que el cielo quiere hacer participar a la tierra en el gran acontecimiento de la Encarnación: \textquote{Gloria a Dios en las alturas y paz en la tierra a los hombres de buena voluntad } (\emph{Lc} 2, 14).

\emph{La paz es anunciada por toda la tierra}. No es una paz que los hombres consigan conquistar con sus fuerzas. \emph{Viene de lo alto} como don maravilloso de Dios a la humanidad. No podemos olvidar que, si todos debemos trabajar para instaurar la paz en el mundo, antes de nada debemos abrirnos al don divino de la paz poniendo toda nuestra confianza en el Señor.

Según el cántico de Navidad, la paz prometida a la tierra \emph{está ligada al amor que Dios trae a los hombres}. Los hombres son llamados \textquote{hombres de buena voluntad} porque ya la buena voluntad divina les pertenece. El nacimiento de Jesús es el testimonio irrefutable y definitivo de esta buena voluntad que jamás será retirada de la humanidad.

Este nacimiento pone de manifiesto \emph{la voluntad divina de reconciliación}: Dios desea reconciliar consigo al mundo pecador, perdonando y cancelando los pecados. Ya en el anuncio del nacimiento el ángel había expresado esta voluntad reconciliadora indicando el nombre que debía llevar el Niño: Jesús, o sea, \textquote{Dios salva}. \textquote{Porque salvará a su pueblo de sus pecados}, comenta el ángel (Mt 1, 21). El nombre revela el destino y la misión del Niño juntamente con su personalidad: es el Dios que salva, el que libera a la humanidad de la esclavitud del pecado y, por ello, restablece las relaciones amistosas del hombre con Dios.

2. El acontecimiento que da a la humanidad un Dios Salvador supera en gran medida las expectativas del pueblo judío. Este pueblo esperaba la salvación, esperaba al Mesías, a un rey ideal del futuro que debía establecer sobre la tierra el reino de Dios. A pesar de que la esperanza judía había puesto muy en lo alto a este Mesías, para ellos no era más que un hombre.

La gran novedad de la venida del Salvador consiste en el hecho de que Él es Dios y hombre a la vez. Lo que el judaísmo no había podido concebir ni esperar, es decir, un Hijo de Dios hecho hombre, se realiza en el misterio de la Encarnación. \emph{El cumplimiento es mucho más maravilloso que la promesa}.

Esta es la razón por la que no podemos medir la grandeza de Jesús sólo con los oráculos proféticos del Antiguo Testamento. Cuando Él realiza estos oráculos se mueve a un nivel trascendente. Todos los tentativos de encerrar a Jesús en los límites de una personalidad humana, no tienen en cuenta lo que hay de esencial en la revelación de la Nueva Alianza: la persona divina del Hijo que se ha hecho hombre o, según la palabra de San Juan, del Verbo que se ha hecho carne y ha venido a habitar entre nosotros (cf. 1, 14). Aquí aparece la grandiosidad generosa del plan divino de salvación. El Padre ha enviado a su Hijo que es Dios como Él. No se ha limitado a enviar a siervos, a hombres que hablasen en su nombre como los Profetas. Ha querido testimoniar a la humanidad el máximo de amor y le ha hecho la sorpresa de darle un Salvador que poseía la omnipotencia divina.

En este Salvador, que es Dios y hombre a la vez, podemos descubrir \emph{la intención de la obra reconciliadora}. El Padre no quiere sólo purificar a la humanidad liberándola del pecado; quiere realizar \emph{la unión más íntima de la divinidad y la humanidad}. En la única persona divina de Jesús, la divinidad y la humanidad están unidas del modo más completo. Él que es perfectamente Dios es perfectamente hombre. Ha realizado en Sí esta unión de la divinidad y la humanidad para poder hacer participar de ella a todos los hombres. Perfectamente hombre, Él, que es Dios, quiere comunicar a sus hermanos humanos una vida divina que les permita ser más perfectamente hombres, reflejando en sí mismos la perfección divina.

3. Un aspecto de la reconciliación merece ser subrayado aquí. Mientras el hombre pecador podía temer para su porvenir las consecuencias de su culpa y esperarse una vida humana disminuida, en cambio recibe de Cristo Salvador \emph{la posibilidad de un completo desarrollo humano}. No sólo es liberado de la esclavitud en la que le aprisionaban sus culpas, sino que puede alcanzar \emph{una perfección humana} superior a la que poseía antes del pecado. Cristo le ofrece una vida humana más abundante y más elevada. Por el hecho de que en Cristo la divinidad no ha comprimido en modo alguno a la humanidad sino que la ha elevado a un grado supremo de desarrollo, con su vida divina comunica a los hombres una vida humana más intensa y completa.

Que Jesús sea el Dios Salvador hecho hombre significa, pues, que ya \emph{en el hombre nada está perdido}. Todo lo que había sido herido, manchado por el pecado, puede revivir y florecer. Esto explica cómo la gracia cristiana favorece el pleno ejercicio de todas las facultades humanas y también la afirmación de toda personalidad, tanto la femenina como la masculina. Reconciliando al hombre con Dios, la religión cristiana tiende a promover todo lo que es humano.

Por tanto, podemos unirnos al canto que resonó en la gruta de Belén y proclamar con los ángeles: \textquote{Gloria a Dios en las alturas y paz en la tierra a los hombres de buena voluntad}.


\section{Temas}

El Directorio Homilético recoge los temas de la Navidad en un solo
grupo, ver página 163.

\chapter{Misa del Día}

\section{Lecturas}

PRIMERA LECTURA

Del libro del profeta Isaías 52, 7-10

Verán los confines de la tierra la salvación de nuestro Dios

¡Qué hermosos son sobre los montes

los pies del mensajero que proclama la paz,

que anuncia la buena noticia,

que pregona la justicia,

que dice a Sión: \textquote{¡Tu Dios reina!}.

Escucha: tus vigías gritan, cantan a coro,

porque ven cara a cara al Señor,

que vuelve a Sión.

Romped a cantar a coro,

ruinas de Jerusalén,

porque el Señor ha consolado a su pueblo,

ha rescatado a Jerusalén.

Ha descubierto el Señor su santo brazo

a los ojos de todas las naciones,

y verán los confines de la tierra

la salvación de nuestro Dios.

SALMO RESPONSORIAL

Salmo 97, 1bcde. 2-3ab. 3cd-4. 5-6

Los confines de la tierra han contemplado la salvación de nuestro Dios

℣. Cantad al Señor un cántico nuevo,

porque ha hecho maravillas:

su diestra le ha dado la victoria,

su santo brazo. ℟.

℣. El Señor da a conocer su salvación.

revela a las naciones su justicia.

Se acordó de su misericordia y su fidelidad

en favor de la casa de Israel. ℟.

℣. Los confines de la tierra han contemplado

la salvación de nuestro Dios.

Aclamad al Señor, tierra entera;

gritad, vitoread, tocad. ℟.

℣. Tañed la cítara para el Señor,

suenen los instrumentos:

con clarines y al son de trompetas,

aclamad al Rey y Señor. ℟.

SEGUNDA LECTURA

De la carta a los Hebreos 1, 1-6

Dios nos ha hablado por el Hijo

En muchas ocasiones y de muchas maneras habló Dios antiguamente a los
padres por los profetas.

En esta etapa final, nos ha hablado por el Hijo, al que ha nombrado
heredero de todo, y por medio del cual ha realizado los siglos.

Él es reflejo de su gloria, impronta de su ser. Él sostiene el universo
con su palabra poderosa. Y, habiendo realizado la purificación de los
pecados, está sentado a la derecha de la Majestad en las alturas; tanto
más encumbrado sobre los ángeles cuanto más sublime es el nombre que ha
heredado.

Pues ¿a qué ángel dijo jamás:

Hijo mío eres tú, yo te he engendrado hoy;

y en otro lugar:

Yo seré para él un padre,

y él será para mí un hijo?

Asimismo, cuando introduce en el mundo al primogénito, dice:

Adórenlo todos los ángeles de Dios.

EVANGELIO (forma larga)

Del Evangelio según san Juan 1, 1-18

El Verbo se hizo carne y habitó entre nosotros

En el principio existía el Verbo, y el Verbo estaba junto a Dios, y el
Verbo era Dios.

Él estaba en el principio junto a Dios.

Por medio de él se hizo todo, y sin él no se hizo nada de cuanto se ha
hecho.

En él estaba la vida, y la vida era la luz de los hombres.

Y la luz brilla en la tiniebla, y la tiniebla no lo recibió.

Surgió un hombre enviado por Dios, que se llamaba Juan:

este venía como testigo, para dar testimonio de la luz, para que todos
creyeran por medio de él.

No era él la luz, sino el que daba testimonio de la luz.

El Verbo era la luz verdadera, que alumbra a todo hombre, viniendo al
mundo.

En el mundo estaba; el mundo se hizo por medio de él, y el mundo no lo
conoció.

Vino a su casa, y los suyos no lo recibieron.

Pero a cuantos lo recibieron, les dio poder de ser hijos de Dios, a los
que creen en su nombre.

Estos no han nacido de sangre, ni de deseo de carne, ni de deseo de
varón, sino que han nacido de Dios.

Y el Verbo se hizo carne y habitó entre nosotros, y hemos contemplado su
gloria: gloria como del Unigénito del Padre, lleno de gracia y de
verdad.

Juan da testimonio de él y grita diciendo: «Este es de quien dije: El
que viene detrás de mí se ha puesto delante de mí, porque existía antes
que yo».

Pues de su plenitud todos hemos recibido, gracia tras gracia.

Porque la ley se dio por medio de Moisés, la gracia y la verdad nos han
llegado por medio de Jesucristo.

A Dios nadie lo ha visto jamás: Dios unigénito, que está en el seno del
Padre, es quien lo ha dado a conocer.

EVANGELIO (forma breve)

Del Evangelio según san Juan 1, 1-5. 9-14

El Verbo se hizo carne y habitó entre nosotros

En el principio existía el Verbo, y el Verbo estaba junto a Dios, y el
Verbo era Dios.

Él estaba en el principio junto a Dios.

Por medio de él se hizo todo, y sin él no se hizo nada de cuanto se ha
hecho.

En él estaba la vida, y la vida era la luz de los hombres.

Y la luz brilla en la tiniebla, y la tiniebla no lo recibió.

El Verbo era la luz verdadera, que alumbra a todo hombre, viniendo al
mundo.

En el mundo estaba; el mundo se hizo por medio de él, y el mundo no lo
conoció.

Vino a su casa, y los suyos no lo recibieron.

Pero a cuantos lo recibieron, les dio poder de ser hijos de Dios, a los
que creen en su nombre.

Estos no han nacido de sangre, ni de deseo de carne, ni de deseo de
varón, sino que han nacido de Dios.

Y el Verbo se hizo carne y habitó entre nosotros, y hemos contemplado su
gloria: gloria como del Unigénito del Padre, lleno de gracia y de
verdad.

\section{Comentario Patrístico}

\subsection{San Basilio Magno, obispo}

El Verbo se hizo carne y puso su morada entre nosotros

Homilía 2, 6; PG 31, 1459-1462. 1471-1474.

Dios en la tierra, Dios en medio de los hombres, no un Dios que consigna la ley entre resplandores de fuego y ruido de trompetas sobre un monte humeante, o en densa nube entre relámpagos y truenos, sembrando el terror entre quienes escuchan; sino un Dios encarnado, que habla a las creaturas de su misma naturaleza con suavidad y dulzura. Un Dios encarnado, que no obra desde lejos o por medio de profetas, sino a través de la humanidad asumida para revestir su persona, para reconducir a sí, en nuestra misma carne hecha suya, a toda la humanidad. ¿Cómo, por medio de uno solo, el resplandor alcanza a todos? ¿Cómo la divinidad reside en la carne? Como el fuego en el hierro: no por transformación, sino por participación. El fuego, efectivamente, no pasa al hierro: permaneciendo donde está, le comunica su virtud; ni por esta comunicación disminuye, sino que invade con lo suyo a quien se comunica. Así el Dios-Verbo, sin jamás separarse de sí mismo \emph{puso su morada en medio de nosotros;} sin sufrir cambio alguno \emph{se hizo carne;} el cielo que lo contenía no quedó privado de él mientras la tierra lo acogió en su seno.

Busca penetrar en el misterio: Dios asume la carne justamente para destruir la muerte oculta en ella. Como los antídotos de un veneno, una vez ingeridos, anulan sus efectos, y como las tinieblas de una casa se disuelven a la luz del sol, la muerte que dominaba sobre la naturaleza humana fue destruida por la presencia de Dios. Y como el hielo permanece sólido en el agua mientras dura la noche y reinan las tinieblas, pero prontamente se diluye al calor del sol, así la muerte reinante hasta la venida de Cristo, apenas resplandeció la gracia de Dios Salvador y surgió el sol de justicia, \emph{fue engullida por la victoria} (1Co 15, 54), no pudiendo coexistir con la Vida. ¡Oh grandeza de la bondad y del amor de Dios por los hombres!

Démosle gloria con los pastores, exultemos con los ángeles \emph{porque hoy ha nacido el Salvador, Cristo el Señor} (Lc 2, 11). Tampoco a nosotros se apareció el Señor en forma de Dios, porque habría asustado a nuestra fragilidad, sino en forma de siervo, para restituir a la libertad a los que estaban en la esclavitud. ¿Quién es tan tibio, tan poco reconocido que no goce, no exulte, no lleve dones? Hoy es fiesta para toda creatura. No haya nadie que no ofrezca algo, nadie se muestre ingrato. Estallemos también nosotros en un canto de exultación.

\section{Homilías}

Las homilías para esta celebración están tomadas de textos de las Padres de la Iglesia que tocan algunos aspectos de la Navidad en particular o relacionados con alguno de los textos bíblicos que se leen en la misma.

Conviene señalar que estas homilías pueden iluminar aspectos de cualquiera de las otras celebraciones durante el tiempo de Navidad.

\subsection{San León Magno, papa}

\subsubsection{Sermón: Nace el Señor, nace la paz}

Sermón 6, 2-3 en la Natividad del Señor: PL 54, 213-216.

Aunque aquella infancia, que la majestad del Hijo de Dios se dignó hacer suya, tuvo como continuación la plenitud de una edad adulta, y, después del triunfo de su pasión y resurrección, todas las acciones de su estado de humildad, que el Señor asumió por nosotros, pertenecen ya al pasado, la festividad de hoy renueva ante nosotros los sagrados comienzos de Jesús, nacido de la Virgen María; de modo que, mientras adoramos el nacimiento de nuestro Salvador, resulta que estamos celebrando nuestro propio comienzo. Efectivamente, la generación de Cristo es el comienzo del pueblo cristiano, y el nacimiento de la cabeza lo es al mismo tiempo del cuerpo.

Aunque cada uno de los que llama el Señor a formar parte de su pueblo sea llamado en un tiempo determinado y aunque todos los hijos de la Iglesia hayan sido llamados cada uno en días distintos, con todo, la totalidad de los fieles, nacida en la fuente bautismal, ha nacido con Cristo en su nacimiento, del mismo modo que ha sido crucificada con Cristo en su pasión, ha sido resucitada en su resurrección y ha sido colocada a la derecha del Padre en su ascensión.

Cualquier hombre que cree --en cualquier parte del mundo--, y se regenera en Cristo, una vez interrumpido el camino de su vieja condición original, pasa a ser un nuevo hombre al renacer; y ya no pertenece a la ascendencia de su padre carnal, sino a la simiente del Salvador, que se hizo precisamente Hijo del hombre, para que nosotros pudiésemos llegar a ser hijos de Dios.

Pues si él no hubiera descendido hasta nosotros revestido de esta humilde condición, nadie hubiera logrado llegar hasta él por sus propios méritos. Por eso, la misma magnitud del beneficio otorgado exige de nosotros una veneración proporcionada a la excelsitud de esta dádiva. Y, como el bienaventurado Apóstol nos enseña, \emph{no hemos recibido el espíritu de este mundo, sino el Espíritu que procede de Dios}, a fin de que conozcamos lo que Dios nos ha otorgado; y el mismo Dios sólo acepta como culto piadoso el ofrecimiento de lo que él nos ha concedido.

¿Y qué podremos encontrar en el tesoro de la divina largueza tan adecuado al honor de la presente festividad como la paz, lo primero que los ángeles pregonaron en el nacimiento del Señor?

La paz es la que engendra los hijos de Dios, alimenta el amor y origina la unidad, es el descanso de los bienaventurados y la mansión de la eternidad. El fin propio de la paz y su fruto específico consiste en que se unan a Dios los que el mismo Señor separa del mundo.

Que los que \emph{no han nacido de sangre, ni de amor carnal, ni de amor humano, sino de Dios,} ofrezcan, por tanto, al Padre la concordia que es propia de hijos pacíficos, y que todos los miembros de la adopción converjan hacia el Primogénito de la nueva creación, que vino a cumplir la voluntad del que le enviaba y no la suya: puesto que la gracia del Padre no adoptó como herederos a quienes se hallaban en discordia e incompatibilidad, sino a quienes amaban y sentían lo mismo. Los que han sido reformados de acuerdo con una sola imagen deben ser concordes en el espíritu.

El nacimiento del Señor es el nacimiento de la paz: y así dice el Apóstol: \emph{El es nuestra paz; él ha hecho de los dos pueblos una sola cosa,} ya que, tanto los judíos como los gentiles, por su medio \emph{podemos acercarnos al Padre con un mismo Espíritu}.

\subsection{San Ambrosio, obispo}

\subsubsection{Comentario: Nació el que es Siervo y Señor a la vez}

Comentario 4-5 sobre el salmo 35: CCL 64, 52-53.

Creo que sobre la pobreza y sufrimientos del Señor hemos aducido testimonios muy válidos de dos santos, de los cuales uno vio y testimonió, mientras que el otro fue elegido tan sólo para testimoniar. Escuchemos todavía nuevos testimonios sobre la condición servil del Señor tomados de estos testigos fiables, o mejor, escuchemos lo que de sí mismo dice el mismo Señor por boca de ambos. Veamos lo que dice: \emph{Habla el Señor, que desde el vientre me formó siervo suyo, para que le trajese a Jacob, para que le reuniese a Israel}. Advirtamos que asumió la condición de siervo para reunir al pueblo.

Estaba yo ---dice--- en las entrañas maternas, y el Señor pronunció mi nombre. Escuchemos cuál es el nombre que el Padre le da: \emph{Mirad: la Virgen concebirá y dará a luz un hijo y le pondrá por nombre Emmanuel, que significa \textquote{Dios-con-nosotros}}. ¿Cuál si no es el nombre de Cristo sino el de \textquote{Hijo de Dios}? Escucha un nuevo texto. Hablando de María a José, también Gabriel había dicho: \emph{Dará a luz un hijo, y tú le pondrás por nombre Jesús}. Escucha ahora la voz de Dios: \emph{Y tú, Belén, tierra de Judá, no eres ni mucho menos la última de las ciudades de Judá: pues de ti saldrá un jefe que será el pastor de mi pueblo}.

Advierte el misterio: del seno de la Virgen nació el que es Siervo y Señor a la vez ---siervo para trabajar, señor para mandar---, a fin de implantar el reinado de Dios en el corazón del hombre. Ambos son uno: no uno del Padre y otro de la Virgen, sino que el mismo que antes de los siglos fue engendrado por el Padre se encarnará más tarde en el seno de la Virgen. Por eso se le llama Siervo y Señor: siervo por nosotros, mas, por la unidad de la naturaleza divina, Dios de Dios, príncipe de príncipe, igual de igual; pues no pudo el Padre engendrar un ser inferior a él y afirmar al mismo tiempo que en el Hijo tiene sus complacencias.

\emph{Gran cosa es para ti} ---dice--- \emph{que seas mi siervo y restablezcas las tribus de Jacob}. Emplea siempre términos adecuados a su dignidad: Gran Dios y gran siervo, pues al encarnarse no perdió los atributos de su grandeza, ya que su grandeza no tiene fin. Así pues, es igual en cuanto Hijo de Dios, asumió la condición de siervo al encarnarse, sufrió la muerte aquel cuya grandeza no tiene fin, porque \emph{el fin de la ley es Cristo, y con eso se justifica a todo el que cree,} para que todos creamos en él y le adoremos con profundo afecto. Bendita servidumbre que a todos nos otorgó la libertad, bendita servidumbre que le valió el \textquote{nombre-sobre-todo-nombre}, bendita humildad que hizo que \emph{al nombre de Jesús toda rodilla se doble en el cielo, en la tierra, en el abismo, y toda lengua proclame: Jesucristo es Señor para gloria de Dios Padre}.

\subsection{San Agustín, obispo}

\subsubsection{Sermón: Saciados con la visión de la Palabra}

Sermón 194, 3-4: PL 38, 1016-1017.

¿Qué ser humano podría conocer todos los tesoros de sabiduría y de ciencia ocultos en Cristo y escondidos en la pobreza de su carne? Porque, \emph{siendo rico, se hizo pobre por vosotros, para enriqueceros con su pobreza}. Pues cuando asumió la condición mortal y experimentó la muerte, se mostró pobre: pero prometió riquezas para más adelante, y no perdió las que le habían quitado.

¡Qué inmensidad la de su dulzura, que escondió para los que lo temen, y llevó a cabo para los que esperan en él!

Nuestros conocimientos son ahora parciales, hasta que se cumpla lo que es perfecto. Y para que nos hagamos capaces de alcanzarlo, él, que era igual al Padre en la forma de Dios, se hizo semejante a nosotros en la forma de siervo, para reformarnos a semejanza de Dios: y, convertido en hijo del hombre --él, que era único Hijo de Dios---, convirtió a muchos hijos de los hombres en hijos de Dios; y, habiendo alimentado a aquellos siervos con su forma visible de siervo, los hizo libres para que contemplasen la forma de Dios.

Pues \emph{ahora somos hijos de Dios y aún no se ha manifestado lo que seremos. Sabemos que, cuando se manifieste, seremos semejantes a él, porque lo veremos tal cual es}. Pues ¿para qué son aquellos tesoros de sabiduría y de ciencia, para qué sirven aquellas riquezas divinas sino para colmarnos? ¿Y para qué la inmensidad de aquella dulzura sino para saciarnos? \emph{Muéstranos al Padre y nos basta}.

Y en algún salmo, uno de nosotros, o en nosotros, o por nosotros, le dice: \emph{Me saciaré cuando se manifieste tu gloria}. Pues él y el Padre son una misma cosa: y quien lo ve a él ve también al Padre. De modo que \emph{el Señor, Dios de los ejércitos, él es el Rey de la gloria}. Volviendo a nosotros, nos mostrará su rostro; y nos salvaremos y quedaremos saciados, y eso nos bastará.

Pero mientras eso no suceda, mientras no nos muestre lo que habrá de bastarnos, mientras no le bebamos como fuente de vida y nos saciemos, mientras tengamos que andar en la fe y peregrinemos lejos de él, mientras tenemos hambre y sed de justicia y anhelamos con inefable ardor la belleza de la forma de Dios, celebremos con devota obsequiosidad el nacimiento de la forma de siervo.

Si no podemos contemplar todavía al que fue engendrado por el Padre antes que el lucero de la mañana, tratemos de acercarnos al que nació de la Virgen en medio de la noche. No comprendemos aún que su \emph{nombre dura como el sol;} reconozcamos que su \emph{tienda} ha sido puesta \emph{en el sol}.

Todavía no podemos contemplar al Único que permanece en su Padre; recordemos al \emph{Esposo que sale de su alcoba}. Todavía no estamos preparados para el banquete de nuestro Padre; reconozcamos al menos el pesebre de nuestro Señor Jesucristo.

\subsection{San Fulberto de Chartres, obispo}

\subsubsection{Carta: El misterio de nuestra salvación}

Carta 5: PL 141, 198-199.

No nos resulta difícil sopesar la diversidad de naturalezas en Cristo. En efecto: uno es el nacimiento o la naturaleza en que, en frase de san Pablo, \emph{nació de una mujer, nació bajo la ley;} otra por la que en el principio estaba junto a Dios; una es la naturaleza por la que, engendrado de la virgen María, vivió humilde en la tierra, y otra por la que, eterno y sin principio, creó el cielo y la tierra; una es la naturaleza en la que se afirma que fue presa de la tristeza, que el cansancio le rindió, que padeció hambre, que lloró, y otra en virtud de la cual curó paralíticos, hizo caminar a los tullidos, dio la vista al ciego de nacimiento, calmó con su imperio las turgentes olas, resucitó muertos.

Siendo así las cosas, es necesario que quien desee llevar el nombre de cristiano con coherencia y sin perjuicio personal, confiese que Cristo, en quien reconocemos dos naturalezas, es a la vez verdadero Dios y hombre verdadero. Así, una vez asegurada la verdad de las dos naturalezas, la fe verdadera no confunda ni divida a Cristo, verdadero en los dolores de su humanidad y verdadero en los poderes de su divinidad. Pues en él la unidad de persona no tolera división y la realidad de la doble naturaleza no admite confusión. En él no subsisten separados Dios y hombre, sino que Cristo es al mismo tiempo Dios y hombre. Efectivamente, Cristo es el mismo Dios que con su divinidad destruyó la muerte; el mismo Hijo de Dios que no podía morir en su divinidad, murió en la carne mortal que el Dios inmortal había asumido; y este mismo Cristo Hijo de Dios, muerto en la carne, resucitó, pues muriendo en la carne, no perdió la inmortalidad de su divinidad.

Sabemos con plena certeza que, siendo pecadores por el primer nacimiento, el segundo nos ha purificado; siendo cautivos por el primer nacimiento, el segundo nos ha liberado; siendo terrenos por el primer nacimiento, el segundo nos hace celestes; siendo carnales por el vicio del primer nacimiento, el beneficio del segundo nacimiento nos hace espirituales; por el primer nacimiento somos hijos de ira, por el segundo nacimiento somos hijos de gracia. Por tanto, todo el que atenta contra la santidad del bautismo, sepa que está ofendiendo al mismo Dios, que dijo: \emph{El que no nazca de agua y Espíritu no puede entrar en el reino de Dios}. Constituye, por tanto, una gracia de la doctrina de la salvación, conocer la profundidad del misterio del bautismo, del que el Apóstol afirma: Si \emph{hemos muerto con Cristo, creemos que también viviremos con él}. Conmorir y ser sepultados con Cristo tiene como meta poder resucitar con él, poder vivir con él.


\subsection{San Juan Pablo II, papa}

\subsubsection{Urbi et Orbi (2001)} \textbf{\emph{MENSAJE URBI ET ORBI\\ DE SU SANTIDAD JUAN PABLO II}}

\emph{Navidad, 25 de diciembre de 2001}



1. \textquote{\emph{Christus est pax nostra}},\\ \textquote{\emph{Cristo es nuestra paz.\\ Él ha hecho de los dos pueblos una sola cosa}} (\emph{Ef} 2, 14).\\ En el alba del nuevo milenio,\\ comenzado con tantas esperanzas,\\ pero ahora amenazado por nubes tenebrosas\\ de violencia y de guerra,\\ las palabras del apóstol Pablo\\ que escuchamos esta Navidad\\ es un rayo de luz penetrante,\\ un clamor de confianza y optimismo.\\ El divino Niño nacido en Belén\\ lleva en sus pequeñas manos, como un don,\\ el secreto de la paz para la humanidad.\\ ¡Él es el Príncipe de la paz!\\ He aquí el gozoso anuncio que se oyó aquella noche en Belén,\\ y que quiero repetir al mundo\\ en este día bendito.\\ Escuchemos una vez más las palabras del ángel:\\ \textquote{\emph{os traigo la buena noticia,\\ la gran alegría para todo el pueblo:\\ hoy, en la ciudad de David,\\ os ha nacido un salvador: el Mesías, el Señor}} (\emph{Lc} 2, 10-11).\\ En el día de hoy, la Iglesia se hace eco de los ángeles,\\ y reitera su extraordinario mensaje,\\ que sorprendió en primer lugar a los pastores\\ en las alturas de Belén.

2. \textquote{\emph{Christus est pax nostra}!}\\ Cristo, el \textquote{\emph{niño envuelto en pañales\\ y acostado en un pesebre}} (\emph{Lc} 2, 10-12),\\ Él es precisamente nuestra paz.\\ Un Niño indefenso, recién nacido en la humildad de una cueva,\\ devuelve la dignidad a cada vida que nace,\\ da esperanza a quien yace en la duda y en el desaliento.\\ Él ha venido para curar a los heridos de la vida\\ y para dar nuevo sentido incluso a la muerte.\\ En aquel Niño, dócil y desvalido,\\ que llora en una gruta fría y destartalada,\\ Dios ha destruido el pecado\\ y ha puesto el germen de una humanidad nueva,\\ llamada a llevar a término\\ el proyecto original de la creación\\ y a transcenderlo con la gracia de la redención.

3. \textquote{\emph{Christus est pax nostra}!}\\ Hombres y mujeres del tercer milenio,\\ vosotros que tenéis hambre de justicia y de paz,\\ ¡acoged el mensaje de Navidad\\ que se propaga hoy por todo el mundo!\\ Jesús ha nacido para consolidar las relaciones\\ entre los hombres y los pueblos,\\ y hacer de todos ellos hermanos en Él.\\ Ha venido para derribar \textquote{el muro que los separaba:\\ el odio} (\emph{Ef} 2, 14),\\ y para hacer de la humanidad una sola familia.\\ Sí, podemos repetir con certeza:\\ ¡Hoy, con el Verbo encarnado, ha nacido la paz!\\ Paz que se ha de implorar,\\ porque sólo Dios es su autor y garante.\\ Paz que se ha de construir\\ en un mundo en el que pueblos y naciones,\\ afectados por tantas y tan diversas dificultades,\\ esperan en una humanidad\\ no sólo globalizada por intereses económicos,\\ sino por el esfuerzo constante\\ en favor de una convivencia más justa y solidaria.

4. Como los pastores, acudamos a Belén,\\ quedémonos en adoración ante la gruta,\\ fijando la mirada en el Redentor recién nacido.\\ En Él podemos reconocer los rasgos\\ de cada pequeño ser humano que viene a la luz,\\ sea cual fuere su raza o nación:\\ es el pequeño palestino y el pequeño israelí;\\ es el bebé estadounidense y el afgano;\\ es el hijo del hutu y el hijo del tutsi\ldots{}\\ es el niño cualquiera, que es alguien para Cristo.\\ Hoy pienso en todos los pequeños del mundo:\\ muchos, demasiados, son los niños\\ que nacen ya condenados a sufrir, sin culpa,\\ las consecuencias de conflictos inhumanos.\\ ¡Salvemos a los niños,\\ para salvar la esperanza de la humanidad!\\ Nos lo pide hoy con fuerza\\ aquel Niño nacido en Belén,\\ el Dios que se hizo hombre,\\ para devolvernos el derecho de esperar.

5. Supliquemos a Cristo el don de la paz\\ para cuantos sufren a causa de conflictos, antiguos y nuevos.\\ Todos los días siento en mi corazón\\ los dramáticos problemas de Tierra Santa;\\ cada día pienso con preocupación\\ en cuantos mueren de hambre y de frío;\\ día tras día me llega, angustiado,\\ el grito de quien, en tantas partes del mundo,\\ invoca una distribución más ecuánime de los recursos\\ y un trabajo dignamente retribuido para todos.\\ ¡Que nadie deje de esperar\\ en el poder del amor de Dios!\\ Que Cristo sea luz y sustento\\ de quien, a veces contracorriente, cree y actúa\\ en favor del encuentro, del diálogo, de la cooperación\\ entre las culturas y las religiones.\\ Que Cristo guíe en la paz los pasos\\ de quien se afana incansablemente\\ por el progreso de la ciencia y la técnica.\\ Que nunca se usen estos grandes dones de Dios\\ contra el respeto y la promoción de la dignidad humana.\\ ¡Que jamás se utilice el nombre santo de Dios\\ para corroborar el odio!\\ ¡Que jamás se haga de Él motivo de intolerancia y violencia!\\ Que el dulce rostro del Niño de Belén\\ recuerde a todos que tenemos un único Padre.

6. \textquote{\emph{Christus est pax nostra}!}\\ Hermanos y hermanas que me escucháis,\\ abrid el corazón a este mensaje de paz,\\ abridlo a Cristo, Hijo de la Virgen María,\\ a Aquel que se ha hecho \textquote{nuestra paz}.\\ Abridlo a Él, que nada nos quita\\ si no es el pecado,\\ y nos da en cambio\\ plenitud de humanidad y de alegría.\\ Y Tú, adorado Niño de Belén,\\ lleva la paz a cada familia y ciudad,\\ a cada nación y continente.\\ ¡Ven, Dios hecho hombre!\\ ¡Ven a ser el corazón del mundo renovado por el amor!\\ ¡Ven especialmente allí donde más peligra\\ la suerte de la humanidad!\\ ¡Ven, y no tardes!\\ ¡Tú eres \textquote{\emph{nuestra paz}}! (\emph{Ef} 2,14).


\subsection{Benedicto XVI, papa}

\subsubsection{Urbi et Orbi (2007): Lux magna que sólo reconocen los pequeños}

Martes 25 de diciembre de 2007.

\textquote{Nos ha amanecido un día sagrado:\\ venid, naciones, adorad al Señor, porque\\ hoy una gran luz ha bajado a la tierra}\\ (Misa del día de Navidad, Aclamación al Evangelio).

Queridos hermanos y hermanas:

\textquote{Nos ha amanecido un día sagrado}. Un día de gran esperanza: hoy el Salvador de la humanidad ha nacido. El nacimiento de un niño trae normalmente una luz de esperanza a quienes lo aguardan ansiosos. Cuando Jesús nació en la gruta de Belén, una \textquote{gran luz} apareció sobre la tierra; una gran esperanza entró en el corazón de cuantos lo esperaban: \textquote{\emph{lux magna}}, canta la liturgia de este día de Navidad.

Ciertamente no fue \textquote{grande} según el mundo, porque, en un primer momento, sólo la vieron María, José y algunos pastores, luego los Magos, el anciano Simeón, la profetisa Ana: aquellos que Dios había escogido. Sin embargo, en lo recóndito y en el silencio de aquella noche santa se encendió para cada hombre una luz espléndida e imperecedera; ha venido al mundo la gran esperanza portadora de felicidad: \textquote{el Verbo se hizo carne y nosotros hemos visto su gloria} (\emph{Jn} 1,14)

\textquote{Dios es luz --afirma san Juan-- y en él no hay tinieblas} (\emph{1} \emph{Jn} 1,5). En el Libro del Génesis leemos que cuando tuvo origen el universo, \textquote{la tierra era un caos informe; sobre la faz del Abismo, la tiniebla}. \textquote{Y dijo Dios: \textquote{que exista la luz}. Y la luz existió} (\emph{Gn} 1,2-3). La Palabra creadora de Dios es Luz, fuente de la vida. Por medio del \emph{Logos} se hizo todo y sin Él no se hizo nada de lo que se ha hecho (cf. \emph{Jn} 1,3). Por eso todas las criaturas son fundamentalmente buenas y llevan en sí la huella de Dios, una chispa de su luz. Sin embargo, cuando Jesús nació de la Virgen María, la Luz misma vino al mundo: \textquote{Dios de Dios, Luz de Luz}, profesamos en el Credo. En Jesús, Dios asumió lo que no era, permaneciendo en lo que era: \textquote{la omnipotencia entró en un cuerpo infantil y no se sustrajo al gobierno del universo} (cf. S. Agustín, \emph{Serm} 184, 1 sobre la Navidad). Aquel que es el creador del hombre se hizo hombre para traer al mundo la paz. Por eso, en la noche de Navidad, el coro de los Ángeles canta: \textquote{Gloria a Dios en el cielo / y en la tierra paz a los hombres que Dios ama} (\emph{Lc} 2,14).

\emph{\textquote{Hoy una gran luz ha bajado a la tierra}}. La Luz de Cristo es portadora de paz. En la Misa de la noche, la liturgia eucarística comenzó justamente con este canto: \textquote{Hoy, desde el cielo, ha descendido la paz sobre nosotros} (\emph{Antífona de entrada}). Más aún, sólo la \textquote{gran} luz que aparece en Cristo puede dar a los hombres la \textquote{verdadera} paz. He aquí por qué cada generación está llamada a acogerla, a acoger al Dios que en Belén se ha hecho uno de nosotros.

La Navidad es esto: acontecimiento histórico y misterio de amor, que desde hace más de dos mil años interpela a los hombres y mujeres de todo tiempo y lugar. Es el día santo en el que brilla la \textquote{gran luz} de Cristo portadora de paz. Ciertamente, para reconocerla, para acogerla, se necesita fe, se necesita humildad. La humildad de María, que ha creído en la palabra del Señor, y que fue la primera que, inclinada ante el pesebre, adoró el Fruto de su vientre; la humildad de José, hombre justo, que tuvo la valentía de la fe y prefirió obedecer a Dios antes que proteger su propia reputación; la humildad de los pastores, de los pobres y anónimos pastores, que acogieron el anuncio del mensajero celestial y se apresuraron a ir a la gruta, donde encontraron al niño recién nacido y, llenos de asombro, lo adoraron alabando a Dios (cf. \emph{Lc} 2,15-20). Los pequeños, los pobres en espíritu: éstos son los protagonistas de la Navidad, tanto ayer como hoy; los protagonistas de siempre de la historia de Dios, los constructores incansables de su Reino de justicia, de amor y de paz.

En el silencio de la noche de Belén Jesús nació y fue acogido por manos solícitas. Y ahora, en esta nuestra Navidad en la que sigue resonando el alegre anuncio de su nacimiento redentor, ¿quién está listo para abrirle las puertas del corazón? Hombres y mujeres de hoy, Cristo viene a traernos la luz también a nosotros, también a nosotros viene a darnos la paz. Pero ¿quién vela en la noche de la duda y la incertidumbre con el corazón despierto y orante? ¿Quién espera la aurora del nuevo día teniendo encendida la llama de la fe? ¿Quién tiene tiempo para escuchar su palabra y dejarse envolver por su amor fascinante? Sí, su mensaje de paz es para todos; viene para ofrecerse a sí mismo a todos como esperanza segura de salvación.

Que la luz de Cristo, que viene a iluminar a todo ser humano, brille por fin y sea consuelo para cuantos viven en las tinieblas de la miseria, de la injusticia, de la guerra; para aquellos que ven negadas aún sus legítimas aspiraciones a una subsistencia más segura, a la salud, a la educación, a un trabajo estable, a una participación más plena en las responsabilidades civiles y políticas, libres de toda opresión y al resguardo de situaciones que ofenden la dignidad humana. Las víctimas de sangrientos conflictos armados, del terrorismo y de todo tipo de violencia, que causan sufrimientos inauditos a poblaciones enteras, son especialmente las categorías más vulnerables, los niños, las mujeres y los ancianos. A su vez, las tensiones étnicas, religiosas y políticas, la inestabilidad, la rivalidad, las contraposiciones, las injusticias y las discriminaciones que laceran el tejido interno de muchos países, exasperan las relaciones internacionales. Y en el mundo crece cada vez más el número de emigrantes, refugiados y deportados, también por causa de frecuentes calamidades naturales, como consecuencia a veces de preocupantes desequilibrios ambientales.

En este día de paz, pensemos sobre todo en donde resuena el fragor de las armas\ldots{} (hace referencia a los principales conflictos en diversas partes del mundo) \ldots{} y en tantas otras situaciones de crisis, desgraciadamente olvidadas con frecuencia. Que el Niño Jesús traiga consuelo a quien vive en la prueba e infunda a los responsables de los gobiernos sabiduría y fuerza para buscar y encontrar soluciones humanas, justas y estables. A la sed de sentido y de valores que hoy se percibe en el mundo; a la búsqueda de bienestar y paz que marca la vida de toda la humanidad; a las expectativas de los pobres, responde Cristo, verdadero Dios y verdadero Hombre, con su Natividad. Que las personas y las naciones no teman reconocerlo y acogerlo: con Él, \textquote{una espléndida luz} alumbra el horizonte de la humanidad; con Él comienza \textquote{un día sagrado} que no conoce ocaso. Que esta Navidad sea realmente para todos un día de alegría, de esperanza y de paz.

\emph{\textquote{Venid, naciones, adorad al Señor}}. Con María, José y los pastores, con los Magos y la muchedumbre innumerable de humildes adoradores del Niño recién nacido, que han acogido el misterio de la Navidad a lo largo de los siglos, dejemos también nosotros, hermanos y hermanas de todos los continentes, que la luz de este día se difunda por todas partes, que entre en nuestros corazones, alumbre y dé calor a nuestros hogares, lleve serenidad y esperanza a nuestras ciudades, y conceda al mundo la paz. Éste es mi deseo para quienes me escucháis. Un deseo que se hace oración humilde y confiada al Niño Jesús, para que su luz disipe las tinieblas de vuestra vida y os llene del amor y de la paz. El Señor, que ha hecho resplandecer en Cristo su rostro de misericordia, os colme con su felicidad y os haga mensajeros de su bondad. ¡Feliz Navidad!

\subsubsection{Urbi et Orbi (2010): ¿Cómo es posible que Dios se haga hombre?}

25 de diciembre de 2010.

\textquote{\emph{Verbum caro factum est}} --- \textquote{El Verbo se hizo carne} (\emph{Jn} 1,14).

Queridos hermanos y hermanas que me escucháis en Roma y en el mundo entero, os anuncio con gozo el mensaje de la Navidad: Dios se ha hecho hombre, ha venido a habitar entre nosotros. Dios no está lejano: está cerca, más aún, es el \textquote{Emmanuel}, el Dios-con-nosotros. No es un desconocido: tiene un rostro, el de Jesús.

Es un mensaje siempre nuevo, siempre sorprendente, porque supera nuestras más audaces esperanzas. Especialmente porque no es sólo un anuncio: es un acontecimiento, un suceso, que testigos fiables han visto, oído y tocado en la persona de Jesús de Nazaret. Al estar con Él, observando lo que hace y escuchando sus palabras, han reconocido en Jesús al Mesías; y, viéndolo resucitado después de haber sido crucificado, han tenido la certeza de que Él, verdadero hombre, era al mismo tiempo verdadero Dios, el Hijo unigénito venido del Padre, lleno de gracia y de verdad (cf. \emph{Jn} 1,14).

\textquote{El Verbo se hizo carne}. Ante esta revelación, vuelve a surgir una vez más en nosotros la pregunta: ¿Cómo es posible? El Verbo y la carne son realidades opuestas; ¿cómo puede convertirse la Palabra eterna y omnipotente en un hombre frágil y mortal? No hay más que una respuesta: el Amor. El que ama quiere compartir con el amado, quiere estar unido a él, y la Sagrada Escritura nos presenta precisamente la gran historia del amor de Dios por su pueblo, que culmina en Jesucristo.

En realidad, Dios no cambia: es fiel a sí mismo. El que ha creado el mundo es el mismo que ha llamado a Abraham y que ha revelado el propio Nombre a Moisés: Yo soy el que soy\ldots{} el Dios de Abraham, Isaac y Jacob\ldots{} Dios misericordioso y piadoso, rico en amor y fidelidad (cf. \emph{Ex} 3,14-15; 34,6). Dios no cambia, desde siempre y por siempre es Amor. Es en sí mismo comunión, unidad en la Trinidad, y cada una de sus obras y palabras tienden a la comunión. La encarnación es la cumbre de la creación. Cuando, por la voluntad del Padre y la acción del Espíritu Santo, se formó en el regazo de María Jesús, Hijo de Dios hecho hombre, la creación alcanzó su cima. El principio ordenador del universo, el \emph{Logos}, comenzó a existir en el mundo, en un tiempo y en un lugar.

\textquote{El Verbo se hizo carne}. La luz de esta verdad se manifiesta a quien la acoge con fe, porque es un misterio de amor. Sólo los que se abren al amor son cubiertos por la luz de la Navidad. Así fue en la noche de Belén, y así también es hoy. La encarnación del Hijo de Dios es un acontecimiento que ha ocurrido en la historia, pero que al mismo tiempo la supera. En la noche del mundo se enciende una nueva luz, que se deja ver por los ojos sencillos de la fe, del corazón manso y humilde de quien espera al Salvador. Si la verdad fuera sólo una fórmula matemática, en cierto sentido se impondría por sí misma. Pero si la Verdad es Amor, pide la fe, el \textquote{sí} de nuestro corazón.

Y, en efecto, ¿qué busca nuestro corazón si no una Verdad que sea Amor? La busca el niño, con sus preguntas tan desarmantes y estimulantes; la busca el joven, necesitado de encontrar el sentido profundo de la propia vida; la busca el hombre y la mujer en su madurez, para orientar y apoyar el compromiso en la familia y en el trabajo; la busca la persona anciana, para dar cumplimiento a la existencia terrenal.

\textquote{El Verbo se hizo carne}. El anuncio de la Navidad es también luz para los pueblos, para el camino conjunto de la humanidad. El \textquote{Emmanuel}, el Dios-con-nosotros, ha venido como Rey de justicia y de paz. Su Reino ---lo sabemos--- no es de este mundo, sin embargo, es más importante que todos los reinos de este mundo. Es como la levadura de la humanidad: si faltara, desaparecería la fuerza que lleva adelante el verdadero desarrollo, el impulso a colaborar por el bien común, al servicio desinteresado del prójimo, a la lucha pacífica por la justicia. Creer en el Dios que ha querido compartir nuestra historia es un constante estímulo a comprometerse en ella, incluso entre sus contradicciones. Es motivo de esperanza para todos aquellos cuya dignidad es ofendida y violada, porque Aquel que ha nacido en Belén ha venido a liberar al hombre de la raíz de toda esclavitud.

{[}Que la luz de la Navidad resplandezca de nuevo \ldots{} (hace referencia a los principales conflictos y sufrimientos presentes en el mundo) \ldots{}

Que el amor del \textquote{Dios con nosotros} otorgue perseverancia a todas las comunidades cristianas que sufren discriminación y persecución, e inspire a los líderes políticos y religiosos a comprometerse por el pleno respeto de la libertad religiosa de todos.

Queridos hermanos y hermanas, \textquote{el Verbo se hizo carne}, ha venido a habitar entre nosotros, es el Emmanuel, el Dios que se nos ha hecho cercano. Contemplemos juntos este gran misterio de amor, dejémonos iluminar el corazón por la luz que brilla en la gruta de Belén. ¡Feliz Navidad a todos!


\section{Temas}

\textquote{¿Por qué el Verbo se hizo carne?}

CEC 456-460, 566:

\textbf{456} Con el Credo Niceno-Constantinopolitano respondemos confesando: \textquote{\emph{Por nosotros los hombres y por nuestra salvación} bajó del cielo, y por obra del Espíritu Santo se encarnó de María la Virgen y se hizo hombre} (DS 150).

\textbf{457} El Verbo se encarnó \emph{para salvarnos reconciliándonos con Dios}: \textquote{Dios nos amó y nos envió a su Hijo como propiciación por nuestros pecados} (\emph{1 Jn} 4, 10). \textquote{El Padre envió a su Hijo para ser salvador del mundo} (\emph{1 Jn} 4, 14). \textquote{Él se manifestó para quitar los pecados} (\emph{1 Jn} 3, 5):

\textquote{Nuestra naturaleza enferma exigía ser sanada; desgarrada, ser restablecida; muerta, ser resucitada. Habíamos perdido la posesión del bien, era necesario que se nos devolviera. Encerrados en las tinieblas, hacía falta que nos llegara la luz; estando cautivos, esperábamos un salvador; prisioneros, un socorro; esclavos, un libertador. ¿No tenían importancia estos razonamientos? ¿No merecían conmover a Dios hasta el punto de hacerle bajar hasta nuestra naturaleza humana para visitarla, ya que la humanidad se encontraba en un estado tan miserable y tan desgraciado?} (San Gregorio de Nisa, \emph{Oratio catechetica}, 15: PG 45, 48B).

\textbf{458} El Verbo se encarnó \emph{para que nosotros conociésemos así el amor de Dios}: \textquote{En esto se manifestó el amor que Dios nos tiene: en que Dios envió al mundo a su Hijo único para que vivamos por medio de él} (\emph{1 Jn} 4, 9). \textquote{Porque tanto amó Dios al mundo que dio a su Hijo único, para que todo el que crea en él no perezca, sino que tenga vida eterna} (\emph{Jn} 3, 16).

\textbf{459} El Verbo se encarnó \emph{para ser nuestro modelo de santidad}: \textquote{Tomad sobre vosotros mi yugo, y aprended de mí \ldots{} } (\emph{Mt} 11, 29). \textquote{Yo soy el Camino, la Verdad y la Vida. Nadie va al Padre sino por mí} (\emph{Jn} 14, 6). Y el Padre, en el monte de la Transfiguración, ordena: \textquote{Escuchadle} (\emph{Mc} 9, 7; cf. \emph{Dt} 6, 4-5). Él es, en efecto, el modelo de las bienaventuranzas y la norma de la Ley nueva: \textquote{Amaos los unos a los otros como yo os he amado} (\emph{Jn} 15, 12). Este amor tiene como consecuencia la ofrenda efectiva de sí mismo (cf. \emph{Mc} 8, 34).

\textbf{460} El Verbo se encarnó \emph{para hacernos \textquote{partícipes de la naturaleza divina}} (\emph{2 P} 1, 4): \textquote{Porque tal es la razón por la que el Verbo se hizo hombre, y el Hijo de Dios, Hijo del hombre: para que el hombre al entrar en comunión con el Verbo y al recibir así la filiación divina, se convirtiera en hijo de Dios} (San Ireneo de Lyon, \emph{Adversus haereses}, 3, 19, 1). \textquote{Porque el Hijo de Dios se hizo hombre para hacernos Dios} (San Atanasio de Alejandría, \emph{De Incarnatione}, 54, 3: PG 25, 192B). \emph{Unigenitus} [\ldots{}] \emph{Dei Filius, suae divinitatis volens nos esse participes, naturam nostram assumpsit, ut homines deos faceret factus homo} (\textquote{El Hijo Unigénito de Dios, queriendo hacernos partícipes de su divinidad, asumió nuestra naturaleza, para que, habiéndose hecho hombre, hiciera dioses a los hombres}) (Santo Tomás de Aquino, \emph{Oficio de la festividad del Corpus}, Of. de Maitines, primer Nocturno, Lectura I).

\textbf{566} \emph{La tentación en el desierto muestra a Jesús, humilde Mesías que triunfa de Satanás mediante su total adhesión al designio de salvación querido por el Padre}.

La Encarnación

CEC 461-463, 470-478:

\textbf{461} Volviendo a tomar la frase de san Juan (\textquote{El Verbo se encarnó}: \emph{Jn} 1, 14), la Iglesia llama \textquote{Encarnación} al hecho de que el Hijo de Dios haya asumido una naturaleza humana para llevar a cabo por ella nuestra salvación. En un himno citado por san Pablo, la Iglesia canta el misterio de la Encarnación:

\textquote{Tened entre vosotros los mismos sentimientos que tuvo Cristo: el cual, siendo de condición divina, no retuvo ávidamente el ser igual a Dios, sino que se despojó de sí mismo tomando condición de siervo, haciéndose semejante a los hombres y apareciendo en su porte como hombre; y se humilló a sí mismo, obedeciendo hasta la muerte y muerte de cruz} (\emph{Flp} 2, 5-8; cf. \emph{Liturgia de las Horas, Cántico de las Primeras Vísperas de Domingos}).

\textbf{462} La carta a los Hebreos habla del mismo misterio:

\textquote{Por eso, al entrar en este mundo, {[}Cristo{]} dice: No quisiste sacrificio y oblación; pero me has formado un cuerpo. Holocaustos y sacrificios por el pecado no te agradaron. Entonces dije: ¡He aquí que vengo [\ldots{}] a hacer, oh Dios, tu voluntad!} (\emph{Hb} 10, 5-7; \emph{Sal} 40, 7-9 {[}LXX{]}).

\textbf{463} La fe en la verdadera encarnación del Hijo de Dios es el signo distintivo de la fe cristiana: \textquote{Podréis conocer en esto el Espíritu de Dios: todo espíritu que confiesa a Jesucristo, venido en carne, es de Dios} (\emph{1 Jn} 4, 2). Esa es la alegre convicción de la Iglesia desde sus comienzos cuando canta \textquote{el gran misterio de la piedad}: \textquote{Él ha sido manifestado en la carne} (\emph{1 Tm} 3, 16).

\textbf{470} Puesto que en la unión misteriosa de la Encarnación \textquote{la naturaleza humana ha sido asumida, no absorbida} (GS 22, 2), la Iglesia ha llegado a confesar con el correr de los siglos, la plena realidad del alma humana, con sus operaciones de inteligencia y de voluntad, y del cuerpo humano de Cristo. Pero paralelamente, ha tenido que recordar en cada ocasión que la naturaleza humana de Cristo pertenece propiamente a la persona divina del Hijo de Dios que la ha asumido. Todo lo que es y hace en ella proviene de \textquote{uno de la Trinidad}. El Hijo de Dios comunica, pues, a su humanidad su propio modo personal de existir en la Trinidad. Así, en su alma como en su cuerpo, Cristo expresa humanamente las costumbres divinas de la Trinidad (cf. \emph{Jn} 14, 9-10):

\textquote{El Hijo de Dios [\ldots{}] trabajó con manos de hombre, pensó con inteligencia de hombre, obró con voluntad de hombre, amó con corazón de hombre. Nacido de la Virgen María, se hizo verdaderamente uno de nosotros, en todo semejante a nosotros, excepto en el pecado} (GS 22, 2).

\textbf{471} Apolinar de Laodicea afirmaba que en Cristo el Verbo había sustituido al alma o al espíritu. Contra este error la Iglesia confesó que el Hijo eterno asumió también un alma racional humana (cf. Dámaso I, Carta a los Obispos Orientales: DS, 149).

\textbf{472} Este alma humana que el Hijo de Dios asumió está dotada de un verdadero conocimiento humano. Como tal, éste no podía ser de por sí ilimitado: se desenvolvía en las condiciones históricas de su existencia en el espacio y en el tiempo. Por eso el Hijo de Dios, al hacerse hombre, quiso progresar \textquote{en sabiduría, en estatura y en gracia} (\emph{Lc} 2, 52) e igualmente adquirir aquello que en la condición humana se adquiere de manera experimental (cf. \emph{Mc} 6, 38; 8, 27; \emph{Jn} 11, 34; etc.). Eso correspondía a la realidad de su anonadamiento voluntario en \textquote{la condición de esclavo} (\emph{Flp} 2, 7).

\textbf{473} Pero, al mismo tiempo, este conocimiento verdaderamente humano del Hijo de Dios expresaba la vida divina de su persona (cf. san Gregorio Magno, carta \emph{Sicut aqua}: DS, 475). \textquote{El Hijo de Dios conocía todas las cosas; y esto por sí mismo, que se había revestido de la condición humana; no por su naturaleza, sino en cuanto estaba unida al Verbo [\ldots{}]. La naturaleza humana, en cuanto estaba unida al Verbo, conocida todas las cosas, incluso las divinas, y manifestaba en sí todo lo que conviene a Dios} (san Máximo el Confesor, \emph{Quaestiones et dubia}, 66: PG 90, 840). Esto sucede ante todo en lo que se refiere al conocimiento íntimo e inmediato que el Hijo de Dios hecho hombre tiene de su Padre (cf. \emph{Mc} 14, 36; \emph{Mt} 11, 27; \emph{Jn}1, 18; 8, 55; etc.). El Hijo, en su conocimiento humano, mostraba también la penetración divina que tenía de los pensamientos secretos del corazón de los hombres (cf. \emph{Mc} 2, 8; \emph{Jn} 2, 25; 6, 61; etc.).

\textbf{474} Debido a su unión con la Sabiduría divina en la persona del Verbo encarnado, el conocimiento humano de Cristo gozaba en plenitud de la ciencia de los designios eternos que había venido a revelar (cf. \emph{Mc} 8,31; 9,31; 10, 33-34; 14,18-20. 26-30). Lo que reconoce ignorar en este campo (cf. \emph{Mc} 13,32), declara en otro lugar no tener misión de revelarlo (cf. \emph{Hch} 1, 7).

\textbf{475} De manera paralela, la Iglesia confesó en el sexto Concilio Ecuménico que Cristo posee dos voluntades y dos operaciones naturales, divinas y humanas, no opuestas, sino cooperantes, de forma que el Verbo hecho carne, en su obediencia al Padre, ha querido humanamente todo lo que ha decidido divinamente con el Padre y el Espíritu Santo para nuestra salvación (cf. Concilio de Constantinopla III, año 681: DS, 556-559). La voluntad humana de Cristo \textquote{sigue a su voluntad divina sin hacerle resistencia ni oposición, sino todo lo contrario, estando subordinada a esta voluntad omnipotente} (\emph{ibíd}., 556).

\textbf{476} Como el Verbo se hizo carne asumiendo una verdadera humanidad, el cuerpo de Cristo era limitado (cf. Concilio de Letrán, año 649: DS, 504). Por eso se puede \textquote{pintar} la faz humana de Jesús (\emph{Ga} 3,2). En el séptimo Concilio ecuménico, la Iglesia reconoció que es legítima su representación en imágenes sagradas (Concilio de Nicea II, año 787: DS, 600-603).

\textbf{477} Al mismo tiempo, la Iglesia siempre ha admitido que, en el cuerpo de Jesús, Dios \textquote{que era invisible en su naturaleza se hace visible} (\emph{Misal Romano}, Prefacio de Navidad). En efecto, las particularidades individuales del cuerpo de Cristo expresan la persona divina del Hijo de Dios. Él ha hecho suyos los rasgos de su propio cuerpo humano hasta el punto de que, pintados en una imagen sagrada, pueden ser venerados porque el creyente que venera su imagen, \textquote{venera a la persona representada en ella} (Concilio de Nicea II: DS, 601).

\textbf{478} Jesús, durante su vida, su agonía y su pasión nos ha conocido y amado a todos y a cada uno de nosotros y se ha entregado por cada uno de nosotros: \textquote{El Hijo de Dios me amó y se entregó a sí mismo por mí} (\emph{Ga} 2, 20). Nos ha amado a todos con un corazón humano. Por esta razón, el sagrado Corazón de Jesús, traspasado por nuestros pecados y para nuestra salvación (cf. \emph{Jn} 19, 34), \textquote{es considerado como el principal indicador y símbolo [\ldots{}] de aquel amor con que el divino Redentor ama continuamente al eterno Padre y a todos los hombres} (Pío XII, Enc. \emph{Haurietis aquas}: DS, 3924; cf. ID. enc. \emph{Mystici Corporis}: ibíd., 3812).

El misterio de la Navidad

CEC 437, 525-526:

\textbf{437} El ángel anunció a los pastores el nacimiento de Jesús como el del Mesías prometido a Israel: \textquote{Os ha nacido hoy, en la ciudad de David, un salvador, que es el Cristo Señor} (\emph{Lc} 2, 11). Desde el principio él es \textquote{a quien el Padre ha santificado y enviado al mundo} (\emph{Jn} 10, 36), concebido como \textquote{santo} (\emph{Lc} 1, 35) en el seno virginal de María. José fue llamado por Dios para \textquote{tomar consigo a María su esposa} encinta \textquote{del que fue engendrado en ella por el Espíritu Santo} (\emph{Mt} 1, 20) para que Jesús \textquote{llamado Cristo} nazca de la esposa de José en la descendencia mesiánica de David (\emph{Mt} 1, 16; cf. \emph{Rm} 1, 3; \emph{2 Tm} 2, 8; \emph{Ap} 22, 16).

\textbf{525} Jesús nació en la humildad de un establo, de una familia pobre (cf. \emph{Lc} 2, 6-7); unos sencillos pastores son los primeros testigos del acontecimiento. En esta pobreza se manifiesta la gloria del cielo (cf. \emph{Lc} 2, 8-20). La Iglesia no se cansa de cantar la gloria de esta noche:

\textquote{Hoy la Virgen da a luz al Transcendente.\\ Y la tierra ofrece una cueva al Inaccesible.\\ Los ángeles y los pastores le alaban.\\ Los magos caminan con la estrella:\\ Porque ha nacido por nosotros,\\ Niño pequeñito\\ el Dios eterno}

(San Romano Melodo, Kontakion, 10)

\textbf{526} \textquote{Hacerse niño} con relación a Dios es la condición para entrar en el Reino (cf. \emph{Mt} 18, 3-4); para eso es necesario abajarse (cf. \emph{Mt} 23, 12), hacerse pequeño; más todavía: es necesario \textquote{nacer de lo alto} (\emph{Jn} 3,7), \textquote{nacer de Dios} (\emph{Jn} 1, 13) para \textquote{hacerse hijos de Dios} (\emph{Jn} 1, 12). El misterio de Navidad se realiza en nosotros cuando Cristo \textquote{toma forma} en nosotros (\emph{Ga}4, 19). Navidad es el misterio de este \textquote{admirable intercambio}:

\textquote{¡Oh admirable intercambio! El Creador del género humano, tomando cuerpo y alma, nace de la Virgen y, hecho hombre sin concurso de varón, nos da parte en su divinidad} (\emph{Solemnidad de la Santísima Virgen María, Madre de Dios,} Antífona de I y II Vísperas: \emph{Liturgia de las Horas}).

Jesús es el Hijo de David

CEC 439, 496, 559, 2616:

\textbf{439} Numerosos judíos e incluso ciertos paganos que compartían su esperanza reconocieron en Jesús los rasgos fundamentales del mesiánico \textquote{hijo de David} prometido por Dios a Israel (cf. \emph{Mt} 2, 2; 9, 27; 12, 23; 15, 22; 20, 30; 21, 9. 15). Jesús aceptó el título de Mesías al cual tenía derecho (cf. \emph{Jn} 4, 25-26;11, 27), pero no sin reservas porque una parte de sus contemporáneos lo comprendían según una concepción demasiado humana (cf. \emph{Mt} 22, 41-46), esencialmente política (cf. \emph{Jn} 6, 15; \emph{Lc} 24, 21).

\textbf{La virginidad de María}

\textbf{496} Desde las primeras formulaciones de la fe (cf. DS 10-64), la Iglesia ha confesado que Jesús fue concebido en el seno de la Virgen María únicamente por el poder del Espíritu Santo, afirmando también el aspecto corporal de este suceso: Jesús fue concebido \emph{absque semine ex Spiritu Sancto} (Concilio de Letrán, año 649; DS, 503), esto es, sin semilla de varón, por obra del Espíritu Santo. Los Padres ven en la concepción virginal el signo de que es verdaderamente el Hijo de Dios el que ha venido en una humanidad como la nuestra:

Así, san Ignacio de Antioquía (comienzos del siglo II): \textquote{Estáis firmemente convencidos acerca de que nuestro Señor es verdaderamente de la raza de David según la carne (cf. \emph{Rm} 1, 3), Hijo de Dios según la voluntad y el poder de Dios (cf. \emph{Jn} 1, 13), nacido verdaderamente de una virgen [\ldots{}] Fue verdaderamente clavado por nosotros en su carne bajo Poncio Pilato [\ldots{}] padeció verdaderamente, como también resucitó verdaderamente} (\emph{Epistula ad Smyrnaeos}, 1-2).

\textbf{La entrada mesiánica de Jesús en Jerusalén}

\textbf{559} ¿Cómo va a acoger Jerusalén a su Mesías? Jesús rehuyó siempre las tentativas populares de hacerle rey (cf. \emph{Jn} 6, 15), pero elige el momento y prepara los detalles de su entrada mesiánica en la ciudad de \textquote{David, su padre} (\emph{Lc} 1,32; cf. \emph{Mt} 21, 1-11). Es aclamado como hijo de David, el que trae la salvación (\textquote{Hosanna} quiere decir \textquote{¡sálvanos!}, \textquote{Danos la salvación!}). Pues bien, el \textquote{Rey de la Gloria} (\emph{Sal} 24, 7-10) entra en su ciudad \textquote{montado en un asno} (\emph{Za} 9, 9): no conquista a la hija de Sión, figura de su Iglesia, ni por la astucia ni por la violencia, sino por la humildad que da testimonio de la Verdad (cf. \emph{Jn} 18, 37). Por eso los súbditos de su Reino, aquel día fueron los niños (cf. \emph{Mt} 21, 15-16; \emph{Sal} 8, 3) y los \textquote{pobres de Dios}, que le aclamaban como los ángeles lo anunciaron a los pastores (cf. \emph{Lc} 19, 38; 2, 14). Su aclamación \textquote{Bendito el que viene en el nombre del Señor} (\emph{Sal} 118, 26), ha sido recogida por la Iglesia en el \emph{Sanctus} de la liturgia eucarística para introducir al memorial de la Pascua del Señor.

\textbf{Jesús escucha la oración}

\textbf{2616} La oración \emph{a Jesús} ya ha sido escuchada por Él durante su ministerio, a través de signos que anticipan el poder de su muerte y de su resurrección: Jesús escucha la oración de fe expresada en palabras (del leproso {[}cf. \emph{Mc} 1, 40-41{]}, de Jairo {[}cf. \emph{Mc} 5, 36{]}, de la cananea {[}cf. \emph{Mc} 7, 29{]}, del buen ladrón {[}cf. \emph{Lc} 23, 39-43{]}), o en silencio (de los portadores del paralítico {[}cf. \emph{Mc} 2, 5{]}, de la hemorroisa {[}cf. \emph{Mc} 5, 28{]} que toca el borde de su manto, de las lágrimas y el perfume de la pecadora {[}cf. \emph{Lc} 7, 37-38{]}). La petición apremiante de los ciegos: \textquote{¡Ten piedad de nosotros, Hijo de David!} (\emph{Mt} 9, 27) o \textquote{¡Hijo de David, Jesús, ten compasión de mí!} (\emph{Mc} 10, 48) ha sido recogida en la tradición de la \emph{Oración a Jesús}: \textquote{Señor Jesucristo, Hijo de Dios, ten piedad de mí, pecador}. Sanando enfermedades o perdonando pecados, Jesús siempre responde a la plegaria del que le suplica con fe: \textquote{Ve en paz, ¡tu fe te ha salvado!}.

San Agustín resume admirablemente las tres dimensiones de la oración de Jesús: \emph{Orat pro nobis ut sacerdos noster, orat in nobis ut caput nostrum, oratur a nobis ut Deus noster. Agnoscamus ergo et in illo voces nostras et voces eius in nobis} (\textquote{Ora por nosotros como sacerdote nuestro; ora en nosotros como cabeza nuestra; a Él se dirige nuestra oración como a Dios nuestro. Reconozcamos, por tanto, en Él nuestras voces; y la voz de Él, en nosotros}) (\emph{Enarratio in Psalmum} 85, 1; cf. I\emph{nstitución general de la Liturgia de las Horas,} 7).

Dios ha dicho todo en su Verbo

CEC 65, 102:

\textbf{65} \textquote{Muchas veces y de muchos modos habló Dios en el pasado a nuestros padres por medio de los profetas; en estos últimos tiempos nos ha hablado por su Hijo} (\emph{Hb} 1,1-2). Cristo, el Hijo de Dios hecho hombre, es la Palabra única, perfecta e insuperable del Padre. En Él lo dice todo, no habrá otra palabra más que ésta. San Juan de la Cruz, después de otros muchos, lo expresa de manera luminosa, comentando \emph{Hb} 1,1-2:

\textquote{Porque en darnos, como nos dio a su Hijo, que es una Palabra suya, que no tiene otra, todo nos lo habló junto y de una vez en esta sola Palabra [\ldots{}]; porque lo que hablaba antes en partes a los profetas ya lo ha hablado todo en Él, dándonos al Todo, que es su Hijo. Por lo cual, el que ahora quisiese preguntar a Dios, o querer alguna visión o revelación, no sólo haría una necedad, sino haría agravio a Dios, no poniendo los ojos totalmente en Cristo, sin querer otra alguna cosa o novedad} (San Juan de la Cruz, \emph{Subida del monte Carmelo} 2,22,3-5: \emph{Biblioteca Mística Carmelitana,} v. 11 (Burgos 1929), p. 184.).

\textbf{102} A través de todas las palabras de la sagrada Escritura, Dios dice sólo una palabra, su Verbo único, en quien él se da a conocer en plenitud (cf. \emph{Hb} 1,1-3):

\textquote{Recordad que es una misma Palabra de Dios la que se extiende en todas las escrituras, que es un mismo Verbo que resuena en la boca de todos los escritores sagrados, el que, siendo al comienzo Dios junto a Dios, no necesita sílabas porque no está sometido al tiempo} (San Agustín, \emph{Enarratio in Psalmum,} 103,4,1).

Cristo encarnado es adorado por los ángeles

CEC 333:

\textbf{333} De la Encarnación a la Ascensión, la vida del Verbo encarnado está rodeada de la adoración y del servicio de los ángeles. Cuando Dios introduce \textquote{a su Primogénito en el mundo, dice: \textquote{adórenle todos los ángeles de Dios}} (\emph{Hb} 1, 6). Su cántico de alabanza en el nacimiento de Cristo no ha cesado de resonar en la alabanza de la Iglesia: \textquote{Gloria a Dios\ldots{}} (\emph{Lc} 2, 14). Protegen la infancia de Jesús (cf. \emph{Mt} 1, 20; 2, 13.19), le sirven en el desierto (cf. \emph{Mc} 1, 12; \emph{Mt} 4, 11), lo reconfortan en la agonía (cf. \emph{Lc} 22, 43), cuando Él habría podido ser salvado por ellos de la mano de sus enemigos (cf. \emph{Mt} 26, 53) como en otro tiempo Israel (cf. \emph{2 M} 10, 29-30; 11,8). Son también los ángeles quienes \textquote{evangelizan} (\emph{Lc} 2, 10) anunciando la Buena Nueva de la Encarnación (cf. \emph{Lc} 2, 8-14), y de la Resurrección (cf. \emph{Mc} 16, 5-7) de Cristo. Con ocasión de la segunda venida de Cristo, anunciada por los ángeles (cf. \emph{Hb} 1, 10-11), éstos estarán presentes al servicio del juicio del Señor (cf. \emph{Mt} 13, 41; 25, 31 ; \emph{Lc} 12, 8-9).

La Encarnación y las imágenes de Cristo

CEC 1159-1162, 2131, 2502:

\textbf{1159} La imagen sagrada, el icono litúrgico, representa principalmente \emph{a} \emph{Cristo}. No puede representar a Dios invisible e incomprensible; la Encarnación del Hijo de Dios inauguró una nueva \textquote{economía} de las imágenes:

\textquote{En otro tiempo, Dios, que no tenía cuerpo ni figura no podía de ningún modo ser representado con una imagen. Pero ahora que se ha hecho ver en la carne y que ha vivido con los hombres, puedo hacer una imagen de lo que he visto de Dios. [\ldots{}] Nosotros sin embargo, revelado su rostro, contemplamos la gloria del Señor} (San Juan Damasceno, \emph{De sacris imaginibus oratio} 1,16).

\textbf{1160} La iconografía cristiana transcribe a través de la imagen el mensaje evangélico que la sagrada Escritura transmite mediante la palabra. Imagen y Palabra se esclarecen mutuamente:

\textquote{Para expresarnos brevemente: conservamos intactas todas las tradiciones de la Iglesia, escritas o no escritas, que nos han sido transmitidas sin alteración. Una de ellas es la representación pictórica de las imágenes, que está de acuerdo con la predicación de la historia evangélica, creyendo que, verdaderamente y no en apariencia, el Dios Verbo se hizo carne, lo cual es tan útil y provechoso, porque las cosas que se esclarecen mutuamente tienen sin duda una significación recíproca} (Concilio de Nicea II, año 787, \emph{Terminus}: COD 111).

\textbf{1161} Todos los signos de la celebración litúrgica hacen referencia a Cristo: también las imágenes sagradas de la Santísima Madre de Dios y de los santos. Significan, en efecto, a Cristo que es glorificado en ellos. Manifiestan \textquote{la nube de testigos} (\emph{Hb} 12,1) que continúan participando en la salvación del mundo y a los que estamos unidos, sobre todo en la celebración sacramental. A través de sus iconos, es el hombre \textquote{a imagen de Dios}, finalmente transfigurado \textquote{a su semejanza} (cf. \emph{Rm} 8,29; \emph{1 Jn} 3,2), quien se revela a nuestra fe, e incluso los ángeles, recapitulados también en Cristo:

\textquote{Siguiendo [\ldots{}] la enseñanza divinamente inspirada de nuestros santos Padres y la Tradición de la Iglesia católica (pues reconocemos ser del Espíritu Santo que habita en ella), definimos con toda exactitud y cuidado que la imagen de la preciosa y vivificante cruz, así como también las venerables y santas imágenes, tanto las pintadas como las de mosaico u otra materia conveniente, se expongan en las santas iglesias de Dios, en los vasos sagrados y ornamentos, en las paredes y en cuadros, en las casas y en los caminos: tanto las imágenes de nuestro Señor Dios y Salvador Jesucristo, como las de nuestra Señora inmaculada la santa Madre de Dios, de los santos ángeles y de todos los santos y justos} (Concilio de Nicea II: DS 600).

\textbf{1162} \textquote{La belleza y el color de las imágenes estimulan mi oración. Es una fiesta para mis ojos, del mismo modo que el espectáculo del campo estimula mi corazón para dar gloria a Dios} (San Juan Damasceno, \emph{De sacris imaginibus oratio} 127). La contemplación de las sagradas imágenes, unida a la meditación de la Palabra de Dios y al canto de los himnos litúrgicos, forma parte de la armonía de los signos de la celebración para que el misterio celebrado se grabe en la memoria del corazón y se exprese luego en la vida nueva de los fieles.

\textbf{2131} Fundándose en el misterio del Verbo encarnado, el séptimo Concilio Ecuménico (celebrado en Nicea el año 787), justificó contra los iconoclastas el culto de las sagradas imágenes: las de Cristo, pero también las de la Madre de Dios, de los ángeles y de todos los santos. El Hijo de Dios, al encarnarse, inauguró una nueva \textquote{economía} de las imágenes.

\textbf{2502} El \emph{arte sacro} es verdadero y bello cuando corresponde por su forma a su vocación propia: evocar y glorificar, en la fe y la adoración, el Misterio trascendente de Dios, Belleza supereminente e invisible de Verdad y de Amor, manifestado en Cristo, \textquote{Resplandor de su gloria e Impronta de su esencia} (\emph{Hb} 1, 3), en quien \textquote{reside toda la Plenitud de la Divinidad corporalmente} (\emph{Col} 2, 9), belleza espiritual reflejada en la Santísima Virgen Madre de Dios, en los Ángeles y los Santos. El arte sacro verdadero lleva al hombre a la adoración, a la oración y al amor de Dios Creador y Salvador, Santo y Santificador.

El Señor vino a ella para hacerse siervo.

El Verbo vino a ella para callar en su seno.

El rayo vino a ella para no hacer ruido.

El pastor vino a ella, y nació el Cordero, que llora dulcemente.

El seno de María ha trastocado los papeles:

Quien creó todo se ha apoderado de él, pero en la pobreza.

El Altísimo vino a ella,

pero entró humildemente.

El esplendor vino a ella, pero vestido con ropas humildes.

Quien todo lo da experimentó el hambre.

Quien da de beber a todos sufrió la sed.

Desnudo salió de ella, quien todo lo reviste»

(\textbf{San Efrén,} Himno \emph{De Nativitate} 11, 6-8).

\chapter{Sagrada Familia (A)}

\section{Lecturas}

PRIMERA LECTURA

Del libro de Ben Sirá 3, 2-6. 12-14

Quien teme al Señor honrará a sus padres

El Señor honra más al padre que a los hijos

y afirma el derecho de la madre sobre ellos.

Quien honra a su padre expía sus pecados,

y quien respeta a su madre es como quien acumula tesoros.

Quien honra a su padre se alegrará de sus hijos

y cuando rece, será escuchado.

Quien respeta a su padre tendrá larga vida,

y quien honra a su madre obedece al Señor.

Hijo, cuida de tu padre en su vejez

y durante su vida no le causes tristeza.

Aunque pierda el juicio, sé indulgente con él

y no lo desprecies aun estando tú en pleno vigor.

Porque la compasión hacia el padre no será olvidada

y te servirá para reparar tus pecados.

SALMO RESPONSORIAL

Salmo 127, 1bc-2. 3. 4-5

Dichosos los que temen al Señor y siguen sus caminos

℣. Dichoso el que teme al Señor

y sigue sus caminos.

Comerás del fruto de tu trabajo,

serás dichoso, te irá bien. ℟.

℣. Tu mujer, como parra fecunda,

en medio de tu casa;

tus hijos, como renuevos de olivo,

alrededor de tu mesa. ℟.

℣. Esta es la bendición del hombre

que teme al Señor.

Que el Señor te bendiga desde Sión,

que veas la prosperidad de Jerusalén

todos los días de tu vida. ℟.

SEGUNDA LECTURA

De la carta del apóstol san Pablo a los Colosenses 3, 12-21

La vida de familia vivida en el Señor

Hermanos:

Como elegidos de Dios, santos y amados, revestíos de compasión
entrañable, bondad, humildad, mansedumbre, paciencia.

Sobrellevaos mutuamente y perdonaos cuando alguno tenga quejas contra
otro.

El Señor os ha perdonado: haced vosotros lo mismo.

Y por encima de todo esto, el amor, que es el vínculo de la unidad
perfecta.

Que la paz de Cristo reine en vuestro corazón: a ella habéis sido
convocados en un solo cuerpo.

Sed también agradecidos. La Palabra de Cristo habite entre vosotros en
toda su riqueza; enseñaos unos a otros con toda sabiduría; exhortaos
mutuamente.

Cantad a Dios, dando gracias de corazón, con salmos, himnos y cánticos
inspirados.

Y todo lo que de palabra o de obra realicéis, sea todo en nombre de
Jesús, dando gracias a Dios Padre por medio de él.

Mujeres, sed sumisas a vuestros maridos, como conviene en el Señor.
Maridos, amad a vuestras mujeres, y no seáis ásperos con ellas.

Hijos, obedeced a vuestros padres en todo, que eso agrada al Señor.
Padres, no exasperéis a vuestros hijos, no sea que pierdan el ánimo.

EVANGELIO

Del Santo Evangelio según san Mateo 2, 13-15. 19-23

Toma al niño y a su madre y huye a Egipto

Cuando se retiraron los magos, el ángel del Señor se apareció en sueños
a José y le dijo:

«Levántate, toma al niño y a su madre y huye a Egipto; quédate allí
hasta que yo te avise, porque Herodes va a buscar al niño para matarlo».

José se levantó, tomó al niño y a su madre, de noche, se fue a Egipto y
se quedó hasta la muerte de Herodes para que se cumpliese lo que dijo el
Señor por medio del profeta: \textquote{De Egipto llamé a mi hijo}.

Cuando murió Herodes, el ángel del Señor se apareció de nuevo en sueños
a José en Egipto y le dijo:

«Levántate, coge al niño y a su madre y vuelve a la tierra de Israel,
porque han muerto los que atentaban contra la vida del niño».

Se levantó, tomó al niño y a su madre y volvió a la tierra de Israel.

Pero al enterarse de que Arquelao reinaba en Judea como sucesor de su
padre Herodes tuvo miedo de ir allá. Y avisado en sueños se retiró a
Galilea y se estableció en una ciudad llamada Nazaret. Así se cumplió lo
dicho por medio de los profetas, que se llamaría nazareno.


\section{Comentario Patrístico}

\subsection{San Juan Crisóstomo, obispo}

Junto al Niño Jesús están María y José

Homilía sobre el día de Navidad: PG 56, 392

Entró Jesús en Egipto para poner fin al llanto de la antigua tristeza; suplantó las plagas por el gozo, y convirtió la noche y las tinieblas en luz de salvación. Entonces fue contaminada el agua del río con la sangre de los tiernos niños. Por eso entró en Egipto el que había convertido el agua en sangre, comunicó a las aguas vivas el poder de aflorar la salvación y las purificó de su fango e impureza con la virtud del Espíritu. Los egipcios fueron afligidos y, enfurecidos, no reconocieron a Dios. Entró, pues, Jesús en Egipto y, colmando las almas religiosas del conocimiento de Dios, dio al río el poder de fecundar una mies de mártires más copiosa que la mies de grano.

¿Qué más diré o cómo seguir hablando? Veo a un artesano y un pesebre; veo a un Niño y los pañales de la cuna, veo el parto de la Virgen carente de lo más imprescindible, todo marcado por la más apremiante necesidad; todo bajo la más absoluta pobreza. ¿Has visto destellos de riqueza en la más extrema pobreza? ¿Cómo, siendo rico, se ha hecho pobre por nuestra causa? ¿Cómo es que no dispuso ni de lecho ni de mantas, sino que fue depositado en un desnudo pesebre? ¡Oh tesoro de riqueza, disimulado bajo la apariencia de pobreza! Yace en el pesebre, y hace temblar el orbe de la tierra; es envuelto en pañales, y rompe las cadenas del pecado; aún no sabe articular palabra, y adoctrina a los Magos induciéndolos a la conversión.

¿Qué más diré o cómo seguir hablando? Ved a un Niño envuelto en pañales y que yace en un pesebre: está con él María, que es Virgen y Madre; le acompañaba José, que es llamado padre.

José era sólo el esposo: fue el Espíritu quien la cubrió con su sombra. Por eso José estaba en un mar de dudas y no sabía cómo llamar al Niño. Esta es la razón por la que, trabajado por la duda, recibe, por medio del ángel, un oráculo del cielo: \emph{José, no tengas reparo en llevarte a tu mujer, pues la criatura que hay en ella viene del Espíritu Santo}. En efecto, el Espíritu Santo cubrió a la Virgen con su sombra. Y ¿por qué nace de la Virgen y conserva intacta su virginidad? Pues porque en otro tiempo el diablo engañó a la virgen Eva; por lo cual a María, que dio a luz siendo virgen, fue Gabriel quien le comunicó la feliz noticia. Es verdad que la seducida Eva dio a luz una palabra que introdujo la muerte; pero no lo es menos que María, acogiendo la alegre noticia, engendró al Verbo en la carne, que nos ha merecido la vida eterna.

\section{Homilías}

Las lecturas para esta solemnidad podrían ser las mismas en los tres ciclos dominicales (hay lecturas opcionales para los años B y C). En esta obra las homilías han sido distribuidas en tres grupos, tomando en cuenta el ciclo litúrgico correspondiente al año en que fueron pronunciadas. Aquí aparecen las homilías que correspondieron al año A, y las de los años B y C aparecerán en sus respectivos volúmenes. Aquéllas homilías podrían también ser iluminadoras para esta año y viceversa.

\subsection{San Juan XXIII, papa}

\subsubsection{Discurso: Que no reine el espíritu mundano}

En la festividad de la Sagrada Familia. Domingo 10 de enero de 1960.

Hoy que la Iglesia pone a la consideración de los fieles el ejemplo de virtud de la Sagrada Familia, nos complacemos en invocar la protección de Jesús, María y José sobre las queridas familias de todos nuestros hijos.

Nos las imaginamos a todas aquí presentes, unidas con Nos en un mismo afecto, y comprendemos los deseos, angustias y temores de cada uno. Nuestro corazón sabe alegrarse con el que se alegra y sufrir con el que sufre (Rom 12,15). Conocemos también las dificultades que hay en las familias, especialmente en las numerosas, cuyos sacrificios suelen ignorarse e incluso, a veces, ni se aprecian.

Sabemos que el espíritu mundano, empleando cada vez mayores incentivos, trata de insinuarse en esta santa institución familiar, que Dios ha querido como custodia y salvaguardia de la dignidad del hombre, del primer despertar de la vida a la juventud impetuosa y de la edad madura a la vejez.

Por tanto, dirigimos, mejor, repetimos a todos la invitación de la liturgia a que miren con segura confianza el ejemplo de la Sagrada Familia que Jesús santificó con inefables virtudes.

El secreto de la verdadera paz, de la mutua y permanente concordia, de la docilidad de los hijos, del florecimiento de las buenas costumbres está en la constante y generosa imitación de la amabilidad, modestia y mansedumbre de la familia de Nazaret, en la que Jesús, Sabiduría eterna del Padre, se nos ofrece junto con María, su madre purísima, y San José, que representa al Padre celestial.

En esta luz todo se transforma en las grandes realidades de la familia cristiana como poco ha hemos puesto de manifiesto en la alocución de la misa de Nochebuena: \textquote{Esponsales iluminados por la luz de lo alto; matrimonio sagrado e inviolable dentro del respeto a sus cuatro notas características: fidelidad, castidad, amor mutuo y santo temor del Señor; espíritu de prudencia y de sacrificio en la educación cuidadosa de los hijos; y siempre, siempre y en toda circunstancia, en disposición de ayudar, de perdonar, de compartir, de otorgar a otros la confianza que nosotros quisiéramos se nos otorgara. Es así como se edifica la casa que jamás se derrumba}.

De nuestro corazón brota el deseo de esta segura esperanza que es garantía de paz inalterable y se une a cada uno de vosotros para acompañarnos en el año nuevo, y que reforzamos con una oración especial que elevamos al cielo fervorosamente con las familias de todos los que nos escuchan, especialmente de aquellas que por falta de medios, de trabajo y de salud sufren dolorosas privaciones.

Nuestro pensamiento se dirige sobre todo a la juventud, esperanza y consuelo de la Iglesia y futuro sostén de la sociedad y más que nada ---ya lo repetimos el pasado año--- a cuantos jóvenes van a formar un hogar y no pueden por dificultades económicas. A todos deseamos una vida llena de la divina gracia, que se afiance en la defensa de los valores espirituales, y llena de la prosperidad y suavidad de los bienes de este mundo.

\subsection{Juan Pablo II, papa}

\subsubsection{Ángelus: Iglesia doméstica}

Domingo 28 de diciembre de 1986.

1. \textquote{Levántate, coge al niño y a su madre y huye a Egipto; quédate allí hasta que yo te avise, porque \emph{Herodes va a buscar al niño para} \emph{matarlo}} (\emph{Mt} 2, 13).

El \textbf{Evangelio} de este domingo de la octava de Navidad nos recuerda cómo la Sagrada Familia fue amenazada durante su estancia en Belén.

Es una amenaza que viene del mundo, que quiere acabar con la vida del Niño.

2. Reunidos hoy para recitar el \textquote{Ángelus}, deseamos junto con toda la Iglesia expresar veneración y amor a esta Familia que, gracias al Hijo de Dios, se hizo la \textquote{\emph{iglesia doméstica}} en la tierra, antes que Él fundase su Iglesia sobre los Apóstoles y sobre Pedro.

Al mismo tiempo, la plegaria de la Iglesia universal y apostólica abraza hoy a todas las familias de la tierra: ¡a todas las \textquote{iglesias domésticas}!

Deseamos hacer frente a todo lo que, en el mundo de hoy, amenaza a la familia \emph{desde dentro y desde fuera}:

¡A lo que amenaza el amor, la fidelidad y la honestidad conyugal, a lo que amenaza la vida!

¡La vida: la gran dignidad de la persona humana!

3. Recemos, pues, con el Apóstol:

¡Familias!: \textquote{¡La palabra de Cristo habite entre vosotros en toda su riqueza!} (\emph{Col} 3, 16).

¡Familias!: \textquote{¡Que \emph{la paz} de Cristo \emph{actúe de árbitro} en vuestro corazón!} (\emph{Col} 3, 15).

\textquote{Sea vuestro uniforme\ldots{} la comprensión. \emph{Sobrellevaos mutuamente} y perdonaos\ldots{} y por encima de todo esto, el amor, que es el ceñidor de la unidad consumada} (\emph{Col} 3, 12-14).

¡Familias! ¡Esposos e hijos! \textquote{\emph{Estad agradecidos}} por el don de la comunidad y de la unión al que Cristo os ha llamado, ofreciéndoos el modelo de la Santísima Familia de Nazaret.

4. Hoy deseo reavivar, junto con todas las familias de Roma y de la Iglesia, esta gracia que han recibido en el santo sacramento del matrimonio, para que obre en ellos de forma eficaz durante todos los días de la vida.

\subsubsection{Ángelus: El primer seminario}

Domingo 31 de diciembre de 1989.

1. La fiesta de hoy nos invita a contemplar la Sagrada Familia de José, María y Jesús, y a admirar su armonioso entendimiento y su perfecto amor. A la luz de ese modelo podemos comprender mejor el valor de la institución familiar y la importancia de su serena convivencia.

Por la narración bíblica de la creación sabemos que la familia ha sido querida por Dios, cuando creó al hombre y la mujer y, bendiciéndolos, les dijo \textquote{Sed fecundos y multiplicaos} (\emph{Gn} 1, 28).

Además, la gracia de Cristo, transmitida mediante el sacramento del matrimonio, hace a las familias capaces de realizar la unión a la que han sido llamadas. En especial las familias cristianas están comprometidas a reproducir el ideal enunciado por Jesús en la oración sacerdotal: \textquote{Como tú, Padre, en mí y yo en ti, que ellos también sean uno en nosotros} (\emph{Jn} 17, 21). Aquel que hizo esta oración obtuvo con su sacrificio un don especial de unidad para todas las familias.

2. El Hijo de Dios se hizo sacerdote en la Encarnación, pero precisamente en virtud de ese ministerio tuvo \emph{necesidad de una educación familiar}. Jesús obedecía a \textbf{María} y a \textbf{José}: \textquote{Vivía sujeto a ellos}, dice el \textbf{Evangelio} (\emph{Lc} 2, 51). Esta sumisión contribuía a la unión del Niño con sus padres y al clima de perfecto entendimiento que reinaba en la casa de Nazaret.

La educación recibida en familia preparó de hecho a Jesús para la misión que debía realizar en la tierra, según la revelación del ángel en el momento de la Anunciación. Fue, por consiguiente, una formación para el cumplimiento de su ministerio sacerdotal, más particularmente para la ofrenda del sacrificio de sí mismo al Padre.

Así queda iluminado \emph{el papel de la familia cristiana en el desarrollo de las vocaciones sacerdotales}. {[}El próximo Sínodo no podrá dejar de considerar este papel, reconocer su importancia y reflexionar sobre los medios adecuados para favorecerlo{]}.

3. La vocación es una llamada que viene del poder soberano y gratuito de Dios. Pero dicha llamaba debe abrirse un camino en el corazón; debe entrar en las profundidades del pensamiento, del sentimiento, de la voluntad del sujeto, para llegar a influir en el comportamiento moral. El joven tiene necesidad de un ambiente familiar así, que lo ayude a tomar conciencia de la llamada y a desarrollar todas sus virtualidades.

Orando hoy por todas las familias del mundo, pediremos en particular a María, Madre de Dios y Madre nuestra, que favorezca el desarrollo de las vocaciones sacerdotales y que bendiga a aquellas familias que se han mostrado disponibles, regalando uno de sus hijos a la Iglesia.

\subsubsection{Ángelus: Redentor de la familia}

Domingo 27 de diciembre de 1992.

1. Hoy la liturgia nos invita a contemplar la Sagrada Familia de Jesús, María y José.

Familia muy singular, por la presencia en ella del Hijo de Dios hecho hombre.

Pero, precisamente por esto, \emph{familia-modelo}, en la que todas las familias del mundo pueden encontrar su ideal seguro y el secreto de su vitalidad.

No es casualidad el hecho de que la fiesta de la Sagrada Familia caiga en un día tan cercano a la Navidad, pues se trata de su desarrollo natural.

Lo es, ante todo porque el Hijo de Dios \emph{quiso tener necesidad,} como todos los niños, \emph{del calor de una familia}.

Y lo es también porque, al venir a salvar al hombre, quiso asumir todas sus dimensiones: tanto la individual como la social. Es el Redentor del hombre, y también el \emph{Redentor de la familia}. Viviendo con María y José devolvió a la familia el esplendor del designio originario de Dios.

2. La experiencia ejemplar de Nazaret nos invita, queridos hermanos y hermanas, a volver a descubrir \emph{el valor fundamental del núcleo familiar}.

La familia es una vocación al amor, una comunidad de personas llamadas a vivir una experiencia específica de comunión (cf. \emph{Familiaris consortio}, 21) dentro del vasto designio de unidad, que Dios estableció para la Iglesia y el mundo, y que tiene su modelo y su fuente en la comunión trinitaria.

Por desgracia, la unidad familiar hoy se halla a menudo amenazada por una cultura hedonista y relativista, que no favorece la indisolubilidad del matrimonio y la acogida de la vida. Quienes sufren las consecuencias son, sobre todo, los más pequeños, pero también se proyectan sus efectos negativos en todo el entramado social, pues se genera frustración, tensión, agresividad, deseos de evasión y, en ocasiones, violencia.

¿Cómo se podrá lograr una convivencia ordenada y pacífica, en una sociedad cada vez más compleja, si no se vuelve a descubrir el valor y la vocación de la familia?

3. A esa urgencia nos invita precisamente la fiesta de hoy, volviéndonos a presentar el ideal de la Sagrada Familia, donde \emph{no faltaba la cruz}, pero \emph{se hacía oración}; donde los afectos eran profundos y puros; donde la esperanza diaria de la vida se suavizaba con el acatamiento sereno de la voluntad de Dios; donde el amor no se cerraba, sino que se proyectaba lejos, en una solidaridad concreta y universal.

La Virgen santa, a quien ahora nos dirigimos con la oración del \emph{Ángelus}, obtenga a las familias cristianas del mundo entero la gracia de ser cada vez más cautivadas por este ideal evangélico, a fin de que se conviertan en fermento auténtico de nueva humanidad.

\subsubsection{Ángelus: Amor auténtico}

Domingo 31 de diciembre de 1995.

1. Hoy la Iglesia celebra la fiesta de la Sagrada Familia, que este año coincide con el último día del año. La liturgia de hoy refiere \textbf{la invitación que el ángel dirigió dos veces a José}: \textquote{Levántate, toma contigo al niño ya su madre y \emph{huye a Egipto} (\ldots{}) porque Herodes va a buscar al niño para matarle} (Mt 2, 13); y, después de la muerte de Herodes: \textquote{Levántate, toma contigo al niño y a su madre, y ponte en camino \emph{de la tierra de Israel}} (\emph{Mt} 2, 20).

En este relato se pueden distinguir \emph{dos momentos decisivos} para la Sagrada Familia: primero, en Belén, cuando el rey Herodes, quiere matar al Niño, porque ve en él un adversario para el trono; y, en Egipto, cuando pasado el peligro, la Sagrada Familia puede volver del destierro a Nazaret. Observamos, ante todo, \emph{la paternal solicitud de Dios} ---la divina solicitud del Padre por el Hijo encarnado--- y, casi como un reflejo, la solicitud humana de José. Junto a él, percibimos la presencia silenciosa y trepidante de María, que en su corazón medita en la solicitud de Dios y en la obediencia diligente de José. A esa solicitud de Dios solemos llamarla divina Providencia; mientras que la solicitud humana se podría definir \emph{providencia humana}. En virtud de esa \emph{providencia}, el padre o la madre se esmeran para evitar todo tipo de mal y garantizar todo el bien posible a los hijos y a la familia.

2. La solicitud de los padres y de las madres debería suscitar en los hijos y en las familias viva gratitud, un sentimiento que constituye también un mandamiento: \textquote{\emph{Honra}}, dice también a los padres: \textquote{\emph{Trata de merecer esa honra}}. Es preciso recordar constantemente la dimensión de la vida familiar, establecida por el cuarto mandamiento del Decálogo. La familia que, por su naturaleza y vocación, es \emph{ambiente de vida y amor}, a menudo se halla sujeta a dolorosas amenazas de todo tipo. Con la familia y en la familia, se encuentra amenazada también la vida de la persona y también de la sociedad.

3. Amadísimos hermanos y hermanas, contemplemos a la Sagrada Familia de Nazaret, ejemplo para todas las familias cristianas y humanas. Ella irradia el auténtico amor-caridad, creando no sólo un elocuente modelo para todas las familias, sino también ofreciendo una garantía de que ese amor puede realizarse en todo el núcleo familiar. En la Sagrada Familia se han de inspirar los novios al prepararse para el matrimonio; y la deben contemplar los esposos al construir su comunidad doméstica. Quiera Dios que en toda casa crezca la fe y reinen el amor, la concordia, la solidaridad, el respeto recíproco y la apertura a la vida.

Que María, \emph{Reina de la familia}, título con el que podríamos de ahora en adelante invocarla en las letanías lauretanas, ayude a las familias de los creyentes a responder cada vez más fielmente a su vocación a fin de que lleguen a ser auténticas \emph{iglesias domésticas}.

\subsubsection{Ángelus: Imagen viva de la Iglesia de Dios}

Domingo 27 de diciembre de 1998.

1. En el clima gozoso de la Navidad, la Iglesia, reviviendo con nueva admiración el misterio del Emmanuel, el Dios con nosotros, nos invita a contemplar hoy a la Sagrada Familia de Nazaret. En la contemplación de este admirable modelo la Iglesia descubre valores que vuelve a proponer a las mujeres y a los hombres de todos los tiempos y de todas las culturas.

\textquote{¡Oh, Familia de Nazaret, imagen viva de la Iglesia de Dios!}. Con estas palabras, la comunidad cristiana reconoce en la comunión familiar de Jesús, María y José, una auténtica \textquote{regla de vida}: cuanto más sepa realizar la Iglesia la \textquote{alianza de amor} que se manifiesta en la Sagrada Familia, tanto más cumplirá su misión de ser levadura, para que \textquote{los hombres constituyan en Cristo una sola familia} (cf. \emph{Ad gentes}, 1).

2. La Sagrada Familia irradia una luz de esperanza también sobre la realidad de la familia de hoy. {[}Consciente de esto, el Consejo pontificio para la familia ya ha comenzado a trabajar para preparar el \emph{III Encuentro mundial de las familias}, que se celebrará en Roma los días 14 y 15 de octubre del año 2000, en el marco del gran jubileo\ldots{} El encuentro anterior tuvo lugar en Río de Janeiro. Y el primero, hace cuatro años, en Roma. El próximo será el tercero{]}.

\ldots{} En Nazaret brotó la primavera de la vida humana del Hijo de Dios, en el instante en que fue concebido por obra del Espíritu Santo en el seno virginal de María. Entre las paredes acogedoras de la casa de Nazaret, se desarrolló en un ambiente de alegría la infancia de Jesús, que \textquote{crecía en edad, en sabiduría y en gracia ante Dios y ante los hombres} (\emph{Lc} 2, 52).

3. Así, el misterio de Nazaret enseña a toda familia a engendrar y educar a sus hijos, cooperando de modo admirable en la obra del Creador y dando al mundo, con cada niño, una nueva sonrisa. En la familia unida los hijos alcanzan la maduración de su existencia, viviendo la experiencia más significativa y rica del amor gratuito, de la fidelidad, del respeto recíproco y de la defensa de la vida.

Ojalá que las familias de hoy contemplen a la Familia de Nazaret a fin de que, imitando el ejemplo de María y José, dedicados amorosamente al cuidado del Verbo encarnado, obtengan indicaciones oportunas para sus opciones diarias de vida.

A la luz de las enseñanzas aprendidas en esa escuela insuperable, todas las familias podrán orientarse en el camino hacia la plena realización del designio de Dios.

\subsubsection{Ángelus: Eligió una familia donde nacer y crecer}

Domingo 30 de diciembre del 2001.

1. Desde la cueva de Belén, donde en la Noche santa nació el Salvador, nuestra mirada se dirige hoy \emph{hacia la humilde casa de Nazaret}, para contemplar a la Sagrada Familia de Jesús, María y José, cuya fiesta celebramos en el clima festivo y familiar de la Navidad.

El Redentor del mundo quiso elegir la familia como lugar donde nacer y crecer, santificando así esta institución fundamental de toda sociedad. El tiempo que pasó en Nazaret, el más largo de su existencia, se halla envuelto por una gran reserva: los evangelistas nos transmiten pocas noticias. Pero si deseamos comprender más profundamente la vida y la misión de Jesús, debemos acercarnos al misterio de la Sagrada Familia de Nazaret para observar y escuchar. La liturgia de hoy nos ofrece una oportunidad providencial.

2. La humilde morada de Nazaret es para todo creyente y, especialmente para las familias cristianas, \emph{una auténtica escuela del Evangelio}. En ella admiramos la realización del proyecto divino de hacer de la familia una \emph{comunidad íntima de vida y amor}; en ella aprendemos que cada hogar cristiano está llamado a ser una pequeña \emph{iglesia doméstica}, donde deben resplandecer las virtudes evangélicas. Recogimiento y oración, comprensión y respeto mutuos, disciplina personal y ascesis comunitaria, espíritu de sacrificio, trabajo y solidaridad son rasgos típicos que hacen de la familia de Nazaret un modelo para todos nuestros hogares.

Quise poner de relieve estos valores en la exhortación apostólica \emph{Familiaris consortio}, cuyo vigésimo aniversario se celebra precisamente este año. \emph{El futuro de la humanidad pasa a través de la familia} que, en nuestro tiempo, ha sido marcada, más que cualquier otra institución, por las profundas y rápidas transformaciones de la cultura y la sociedad. Pero la Iglesia jamás ha dejado de \textquote{hacer sentir su voz y ofrecer su ayuda a todo aquel que, conociendo ya el valor del matrimonio y de la familia, trata de vivirlo fielmente; a todo aquel que, en medio de la incertidumbre o de la ansiedad, busca la verdad; y a todo aquel que se ve injustamente impedido para vivir con libertad el propio proyecto familiar} (\emph{Familiaris consortio}, 1). Es consciente de esta responsabilidad suya y también hoy quiere seguir \textquote{ofreciendo su servicio a todo hombre preocupado por el destino del matrimonio y de la familia} (\emph{ib}.).

3. Para cumplir esta urgente misión, la Iglesia cuenta de modo especial con el testimonio y la aportación de las familias cristianas. Más aún, frente a los peligros y a las dificultades que afronta la institución familiar, invita a un suplemento de audacia espiritual y apostólica, convencida de que las familias están llamadas a ser \textquote{signo de unidad para el mundo} y a testimoniar \textquote{el reino y la paz de Cristo, hacia el cual el mundo entero está en camino} (\emph{ib}., 48).

Que Jesús, María y José bendigan y protejan a todas las familias del mundo, para que en ellas reinen la serenidad y la alegría, la justicia y la paz que Cristo al nacer trajo como don a la humanidad.

\subsection{Benedicto XVI, papa}

\subsubsection{Ángelus: Santificó la realidad de la familia}

Domingo 30 de diciembre del 2007.

Celebramos hoy la fiesta de la Sagrada Familia. Siguiendo los evangelios de san Mateo y san Lucas, fijamos hoy nuestra mirada en Jesús, María y José, y adoramos el misterio de un Dios que quiso nacer de una mujer, la Virgen santísima, y entrar en este mundo por el camino común a todos los hombres. Al hacerlo así, santificó la realidad de la familia, colmándola de la gracia divina y revelando plenamente su vocación y misión.

A la familia dedicó gran atención el concilio Vaticano II. Los cónyuges ---afirma--- \textquote{son testigos, el uno para el otro y ambos para sus hijos, de la fe y del amor de Cristo} (\emph{Lumen gentium}, 35). Así la familia cristiana participa de la vocación profética de la Iglesia: con su estilo de vida \textquote{proclama en voz alta tanto los valores del reino de Dios ya presentes como la esperanza en la vida eterna} (\emph{ib}.).

Como repitió incansablemente mi venerado predecesor Juan Pablo II, el bien de la persona y de la sociedad está íntimamente vinculado a la \textquote{buena salud} de la familia (cf. \emph{Gaudium et spes}, 47). Por eso, la Iglesia está comprometida en defender y promover \textquote{la dignidad natural y el eximio valor} ---son palabras del Concilio--- del matrimonio y de la familia (\emph{ib}.). Con esta finalidad se está llevando a cabo, precisamente hoy, una importante iniciativa en Madrid, a cuyos participantes me dirigiré ahora en lengua española.

[\ldots{}] Al contemplar el misterio del Hijo de Dios que vino al mundo rodeado del afecto de María y de José, invito a las familias cristianas a experimentar la presencia amorosa del Señor en sus vidas. Asimismo, les aliento a que, inspirándose en el amor de Cristo por los hombres, den testimonio ante el mundo de la belleza del amor humano, del matrimonio y la familia. Esta, fundada en la unión indisoluble entre un hombre y una mujer, constituye el ámbito privilegiado en el que la vida humana es acogida y protegida, desde su inicio hasta su fin natural. Por eso, los padres tienen el derecho y la obligación fundamental de educar a sus hijos en la fe y en los valores que dignifican la existencia humana.

Vale la pena trabajar por la familia y el matrimonio porque vale la pena trabajar por el ser humano, el ser más precioso creado por Dios. Me dirijo de modo especial a los niños, para que quieran y recen por sus padres y hermanos; a los jóvenes, para que estimulados por el amor de sus padres, sigan con generosidad su propia vocación matrimonial, sacerdotal o religiosa; a los ancianos y enfermos, para que encuentren la ayuda y comprensión necesarias. Y vosotros, queridos esposos, contad siempre con la gracia de Dios, para que vuestro amor sea cada vez más fecundo y fiel. En las manos de María, \textquote{que con su \textquote{sí} abrió la puerta de nuestro mundo a Dios} (\emph{Spe salvi}, 49), pongo los frutos de esta celebración. Muchas gracias y ¡felices fiestas!

Nos dirigimos ahora a la Virgen santísima, pidiendo por el bien de la familia y por todas las familias del mundo.

\subsubsection{Ángelus: El calor de una familia}

Domingo 26 de diciembre del 2010.

El \emph{\textbf{Evangelio según san Lucas}} narra que los pastores de Belén, después de recibir del ángel el anuncio del nacimiento del Mesías, \textquote{fueron a toda prisa, y encontraron a María y a José, y al niño acostado en el pesebre} (2, 16). Así pues, a los primeros testigos oculares del nacimiento de Jesús se les presentó la escena de una familia: madre, padre e hijo recién nacido. Por eso, el primer domingo después de Navidad, la liturgia nos hace celebrar la fiesta de la Sagrada Familia. Este año tiene lugar precisamente al día siguiente de la Navidad y, prevaleciendo sobre la de san Esteban, nos invita a contemplar este \textquote{icono} en el que el niño Jesús aparece en el centro del afecto y de la solicitud de sus padres. En la pobre cueva de Belén ---escriben los Padres de la Iglesia--- resplandece una luz vivísima, reflejo del profundo misterio que envuelve a ese Niño, y que María y José custodian en su corazón y dejan traslucir en sus miradas, en sus gestos y sobre todo en sus silencios. De hecho, conservan en lo más íntimo las palabras del anuncio del ángel a María: \textquote{El que ha de nacer será llamado Hijo de Dios} (\emph{Lc} 1, 35).

Sin embargo, el nacimiento de todo niño conlleva algo de este misterio. Lo saben muy bien los padres que lo reciben como un don y que, con frecuencia, así se refieren a él. Todos hemos escuchado decir alguna vez a un papá y a una mamá: \textquote{Este niño es un don, un milagro}. En efecto, los seres humanos no viven la procreación meramente como un acto reproductivo, sino que perciben su riqueza, intuyen que cada criatura humana que se asoma a la tierra es el \textquote{signo} por excelencia del Creador y Padre que está en el cielo. ¡Cuán importante es, por tanto, que cada niño, al venir al mundo, sea acogido por el calor de una familia! No importan las comodidades exteriores: Jesús nació en un establo y como primera cuna tuvo un pesebre, pero el amor de María y de José le hizo sentir la ternura y la belleza de ser amados. Esto es lo que necesitan los niños: el amor del padre y de la madre. Esto es lo que les da seguridad y lo que, al crecer, les permite descubrir el sentido de la vida. La Sagrada Familia de Nazaret pasó por muchas pruebas, como la de la \textquote{matanza de los inocentes} ---nos la recuerda el \emph{Evangelio según san Mateo}---, que obligó a José y María a emigrar a Egipto (cf. 2, 13-23). Ahora bien, confiando en la divina Providencia, encontraron su estabilidad y aseguraron a Jesús una infancia serena y una educación sólida.

Queridos amigos, ciertamente la Sagrada Familia es singular e irrepetible, pero al mismo tiempo es \textquote{modelo de vida} para toda familia, porque Jesús, verdadero hombre, quiso nacer en una familia humana y, al hacerlo así, la bendijo y consagró. Encomendemos, por tanto, a la Virgen y a san José a todas las familias, para que no se desalienten ante las pruebas y dificultades, sino que cultiven siempre el amor conyugal y se dediquen con confianza al servicio de la vida y de la educación.

\subsection{Francisco, papa}

\subsubsection{Ángelus: Dios se hizo como nosotros}

Domingo 29 de diciembre del 2013.

En este primer domingo después de Navidad, la Liturgia nos invita a celebrar la fiesta de la Sagrada Familia de Nazaret. En efecto, cada belén nos muestra a Jesús junto a la Virgen y a san José, en la cueva de Belén. Dios quiso nacer en una familia humana, quiso tener una madre y un padre, como nosotros.

Y hoy el Evangelio nos presenta a la Sagrada Familia por el camino doloroso del destierro, en busca de refugio en Egipto. José, María y Jesús experimentan la condición dramática de los refugiados, marcada por miedo, incertidumbre, incomodidades (cf. Mt 2, 13-15.19-23). Lamentablemente, en nuestros días, millones de familias pueden reconocerse en esta triste realidad. Casi cada día la televisión y los periódicos dan noticias de refugiados que huyen del hambre, de la guerra, de otros peligros graves, en busca de seguridad y de una vida digna para sí mismos y para sus familias.

En tierras lejanas, incluso cuando encuentran trabajo, no siempre los refugiados y los inmigrantes encuentran auténtica acogida, respeto, aprecio por los valores que llevan consigo. Sus legítimas expectativas chocan con situaciones complejas y dificultades que a veces parecen insuperables. Por ello, mientras fijamos la mirada en la Sagrada Familia de Nazaret en el momento en que se ve obligada a huir, pensemos en el drama de los inmigrantes y refugiados que son víctimas del rechazo y de la explotación, que son víctimas de la trata de personas y del trabajo esclavo. Pero pensemos también en los demás \textquote{exiliados}: yo les llamaría \textquote{exiliados ocultos}, esos exiliados que pueden encontrarse en el seno de las familias mismas: los ancianos, por ejemplo, que a veces son tratados como presencias que estorban. Muchas veces pienso que un signo para saber cómo va una familia es ver cómo se tratan en ella a los niños y a los ancianos.

Jesús quiso pertenecer a una familia que experimentó estas dificultades, para que nadie se sienta excluido de la cercanía amorosa de Dios. La huida a Egipto causada por las amenazas de Herodes nos muestra que Dios está allí donde el hombre está en peligro, allí donde el hombre sufre, allí donde huye, donde experimenta el rechazo y el abandono; pero Dios está también allí donde el hombre sueña, espera volver a su patria en libertad, proyecta y elige en favor de la vida y la dignidad suya y de sus familiares.

Hoy, nuestra mirada a la Sagrada Familia se deja atraer también por la sencillez de la vida que ella lleva en Nazaret. Es un ejemplo que hace mucho bien a nuestras familias, les ayuda a convertirse cada vez más en una comunidad de amor y de reconciliación, donde se experimenta la ternura, la ayuda mutua y el perdón recíproco. Recordemos las tres palabras clave para vivir en paz y alegría en la familia: permiso, gracias, perdón. Cuando en una familia no se es entrometido y se pide \textquote{permiso}, cuando en una familia no se es egoísta y se aprende a decir \textquote{gracias}, y cuando en una familia uno se da cuenta que hizo algo malo y sabe pedir \textquote{perdón}, en esa familia hay paz y hay alegría. Recordemos estas tres palabras. Pero las podemos repetir todos juntos: permiso, gracias, perdón. (Todos: permiso, gracias, perdón) Desearía alentar también a las familias a tomar conciencia de la importancia que tienen en la Iglesia y en la sociedad. El anuncio del Evangelio, en efecto, pasa ante todo a través de las familias, para llegar luego a los diversos ámbitos de la vida cotidiana.

Invoquemos con fervor a María santísima, la Madre de Jesús y Madre nuestra, y a san José, su esposo. Pidámosle a ellos que iluminen, conforten y guíen a cada familia del mundo, para que puedan realizar con dignidad y serenidad la misión que Dios les ha confiado.

\subsubsection{Ángelus: Obedecieron a Dios}

Domingo, 29 de diciembre de 2019

Hoy es un día hermoso\ldots{} Hoy celebramos la fiesta de la Sagrada Familia de Nazaret. El término \textquote{sagrada} coloca a esta familia en el ámbito de la santidad, que es un don de Dios pero, al mismo tiempo, es una adhesión libre y responsable al plan de Dios. Éste fue el caso de la familia de Nazaret: estaba totalmente a disposición de la voluntad de Dios.

¿Cómo no asombrarse, por ejemplo, de la \textbf{docilidad de María} a la acción del Espíritu Santo que le pide que se convierta en la madre del Mesías? Porque María, como toda joven de su tiempo, estaba a punto de realizar su proyecto de vida, es decir, casarse con José. Pero cuando se dio cuenta de que Dios la llamaba a una misión particular, no dudó en proclamarse su \textquote{esclava} (cf. \emph{Lucas} 1, 38). Jesús exaltará su grandeza no tanto por su papel de madre, sino por su obediencia a Dios. Jesús dijo: \textquote{Dichosos más bien los que oyen la Palabra de Dios y la guardan} (\emph{Lucas} 11, 28), como María. Y cuando no comprende plenamente los acontecimientos que la involucran, María medita en silencio, reflexiona y adora la iniciativa divina. Su presencia al pie de la Cruz consagra esta disponibilidad total.

Luego, en lo que respecta a \textbf{José}, el Evangelio no nos refiere ni una sola palabra: no habla, sino que actúa por obediencia. Es el hombre del silencio, el hombre de la obediencia. La página del \textbf{Evangelio de hoy} (cf. \emph{Mateo} 2, 13-15, 19-23) nos recuerda tres veces esta obediencia del justo José, refiriéndose a su huida a Egipto y a su retorno a la tierra de Israel. Bajo la guía de Dios, representada por el Ángel, José aleja a su familia de la amenaza de Herodes y los salva. De esta manera, la Sagrada Familia se solidariza con todas las familias del mundo que se ven obligadas a exiliarse, se solidariza con todos aquellos que se ven obligados a abandonar su tierra a causa de la represión, la violencia, la guerra.

Finalmente, la tercera persona de la Sagrada Familia: \textbf{Jesús}. Él es la voluntad del Padre: sobre Él, dice san Pablo, no hubo \textquote{sí} y \textquote{no}, sino sólo \textquote{sí} (cf. \emph{2 Corintios} 1, 19). Y esto se manifestó en muchos momentos de su vida terrenal. Por ejemplo, el episodio en el templo en el que, a los padres angustiados que lo buscaban, les respondió: \textquote{¿No sabíais que yo debía estar en la casa de mi Padre?} (\emph{Lucas} 2, 49); o su constante repetición: \textquote{Mi alimento es hacer la voluntad del que me ha enviado} (\emph{Juan} 4, 34); su oración en el Huerto de los Olivos: \textquote{Padre mío, si esto no puede pasar sin que yo lo beba, hágase tu voluntad} (\emph{Mateo} 26, 42). Todos estos acontecimientos son la perfecta realización de las mismas palabras de Cristo que dice: \textquote{Sacrificio y oblación no quisiste [\ldots{}]. Entonces dije: \textquote{¡He aquí que vengo [\ldots{}] a hacer, oh Dios, tu voluntad!}} (\emph{Hebreos} 10, 5-7; \emph{Salmos} 40, 7-9).

\textbf{María, José, Jesús}: la Sagrada Familia de Nazaret que representa una respuesta coral a la voluntad del Padre: los tres miembros de esta familia se ayudan mutuamente a descubrir el plan de Dios. Rezaban, trabajaban, se comunicaban. Y yo me pregunto: ¿tú, en tu familia, sabes cómo comunicarte o eres como esos chicos que en la mesa, cada uno con un teléfono móvil, están chateando? En esa mesa parece que hay un silencio como si estuvieran en misa\ldots{} Pero no se comunican entre ellos. Debemos reanudar el diálogo en la familia: padres, madres, hijos, abuelos y hermanos deben comunicarse entre sí\ldots{} Es una tarea que hay que hacer hoy, precisamente en el Día de la Sagrada Familia. Que la Sagrada Familia sea un modelo para nuestras familias, para que padres e hijos se apoyen mutuamente en la fidelidad al Evangelio, fundamento de la santidad de la familia.

Confiemos a María \textquote{Reina de la Familia} todas las familias del mundo, especialmente las que sufren o están en peligro, e invoquemos sobre ellas su protección materna.


\section{Temas}

La Sagrada Familia

CEC 531-534:

\textbf{Los misterios de la vida oculta de Jesús}

\textbf{531} Jesús compartió, durante la mayor parte de su vida, la condición de la inmensa mayoría de los hombres: una vida cotidiana sin aparente importancia, vida de trabajo manual, vida religiosa judía sometida a la ley de Dios (cf. \emph{Ga} 4, 4), vida en la comunidad. De todo este período se nos dice que Jesús estaba \textquote{sometido} a sus padres y que \textquote{progresaba en sabiduría, en estatura y en gracia ante Dios y los hombres} (\emph{Lc} 2, 51-52).

\textbf{532} Con la sumisión a su madre, y a su padre legal, Jesús cumple con perfección el cuarto mandamiento. Es la imagen temporal de su obediencia filial a su Padre celestial. La sumisión cotidiana de Jesús a José y a María anunciaba y anticipaba la sumisión del Jueves Santo: \textquote{No se haga mi voluntad \ldots{}} (\emph{Lc} 22, 42). La obediencia de Cristo en lo cotidiano de la vida oculta inauguraba ya la obra de restauración de lo que la desobediencia de Adán había destruido (cf. \emph{Rm} 5, 19).

\textbf{533} La vida oculta de Nazaret permite a todos entrar en comunión con Jesús a través de los caminos más ordinarios de la vida humana:

\textquote{Nazaret es la escuela donde empieza a entenderse la vida de Jesús, es la escuela donde se inicia el conocimiento de su Evangelio. [\ldots{}] Su primera lección es el \emph{silencio}. Cómo desearíamos que se renovara y fortaleciera en nosotros el amor al silencio, este admirable e indispensable hábito del espíritu, tan necesario para nosotros. [\ldots{}] Se nos ofrece además una lección de \emph{vida familiar}. Que Nazaret nos enseñe el significado de la familia, su comunión de amor, su sencilla y austera belleza, su carácter sagrado e inviolable. [\ldots{}] Finalmente, aquí aprendemos también la \emph{lección del trabajo}. Nazaret, la casa del \textquote{hijo del Artesano}: cómo deseamos comprender más en este lugar la austera pero redentora ley del trabajo humano y exaltarla debidamente. [\ldots{}] Queremos finalmente saludar desde aquí a todos los trabajadores del mundo y señalarles al gran modelo, al hermano divino} (Pablo VI, \emph{Homilía en el templo de la Anunciación de la Virgen María en Nazaret} (5 de enero de 1964).

\textbf{534} \emph{El hallazgo de Jesús en el Templo} (cf. \emph{Lc} 2, 41-52) es el único suceso que rompe el silencio de los Evangelios sobre los años ocultos de Jesús. Jesús deja entrever en ello el misterio de su consagración total a una misión derivada de su filiación divina: \textquote{¿No sabíais que me debo a los asuntos de mi Padre?} María y José \textquote{no comprendieron} esta palabra, pero la acogieron en la fe, y María \textquote{conservaba cuidadosamente todas las cosas en su corazón}, a lo largo de todos los años en que Jesús permaneció oculto en el silencio de una vida ordinaria.

La familia cristiana, una Iglesia doméstica

CEC 1655-1658, 2204-2206:

\textbf{1655} Cristo quiso nacer y crecer en el seno de la Sagrada Familia de José y de María. La Iglesia no es otra cosa que la \textquote{familia de Dios}. Desde sus orígenes, el núcleo de la Iglesia estaba a menudo constituido por los que, \textquote{con toda su casa}, habían llegado a ser creyentes (cf. \emph{Hch} 18,8). Cuando se convertían deseaban también que se salvase \textquote{toda su casa} (cf. \emph{Hch} 16,31; 11,14). Estas familias convertidas eran islotes de vida cristiana en un mundo no creyente.

\textbf{1656} En nuestros días, en un mundo frecuentemente extraño e incluso hostil a la fe, las familias creyentes tienen una importancia primordial en cuanto faros de una fe viva e irradiadora. Por eso el Concilio Vaticano II llama a la familia, con una antigua expresión, \emph{Ecclesia domestica} (LG 11; cf. FC 21). En el seno de la familia, \textquote{los padres han de ser para sus hijos los primeros anunciadores de la fe con su palabra y con su ejemplo, y han de fomentar la vocación personal de cada uno y, con especial cuidado, la vocación a la vida consagrada} (LG 11).

\textbf{1657} Aquí es donde se ejercita de manera privilegiada el \emph{sacerdocio bautismal} del padre de familia, de la madre, de los hijos, de todos los miembros de la familia, \textquote{en la recepción de los sacramentos, en la oración y en la acción de gracias, con el testimonio de una vida santa, con la renuncia y el amor que se traduce en obras} (LG 10). El hogar es así la primera escuela de vida cristiana y \textquote{escuela del más rico humanismo} (GS 52,1). Aquí se aprende la paciencia y el gozo del trabajo, el amor fraterno, el perdón generoso, incluso reiterado, y sobre todo el culto divino por medio de la oración y la ofrenda de la propia vida.

\textbf{1658} Es preciso recordar asimismo a un gran número de \emph{personas que permanecen solteras} a causa de las concretas condiciones en que deben vivir, a menudo sin haberlo querido ellas mismas. Estas personas se encuentran particularmente cercanas al corazón de Jesús; y, por ello, merecen afecto y solicitud diligentes de la Iglesia, particularmente de sus pastores. Muchas de ellas viven \emph{sin familia humana}, con frecuencia a causa de condiciones de pobreza. Hay quienes viven su situación según el espíritu de las bienaventuranzas sirviendo a Dios y al prójimo de manera ejemplar. A todas ellas es preciso abrirles las puertas de los hogares, \textquote{iglesias domésticas} y las puertas de la gran familia que es la Iglesia. \textquote{Nadie se sienta sin familia en este mundo: la Iglesia es casa y familia de todos, especialmente para cuantos están \textquote{fatigados y agobiados} (\emph{Mt} 11,28)} (FC 85).

\textbf{2204} \textquote{La familia cristiana constituye una revelación y una actuación específicas de la comunión eclesial; por eso [\ldots{}] puede y debe decirse \emph{Iglesia doméstica}} (FC 21, cf. LG 11). Es una comunidad de fe, esperanza y caridad, posee en la Iglesia una importancia singular como aparece en el Nuevo Testamento (cf. \emph{Ef} 5, 21-6, 4; \emph{Col} 3, 18-21; \emph{1 P} 3, 1-7).

\textbf{2205} La familia cristiana es una comunión de personas, reflejo e imagen de la comunión del Padre y del Hijo en el Espíritu Santo. Su actividad procreadora y educativa es reflejo de la obra creadora de Dios. Es llamada a participar en la oración y el sacrificio de Cristo. La oración cotidiana y la lectura de la Palabra de Dios fortalecen en ella la caridad. La familia cristiana es evangelizadora y misionera.

\textbf{2206} Las relaciones en el seno de la familia entrañan una afinidad de sentimientos, afectos e intereses que provienen sobre todo del mutuo respeto de las personas. La familia es una \emph{comunidad privilegiada} llamada a realizar un propósito común de los esposos y una cooperación diligente de los padres en la educación de los hijos (cf. GS 52).

Los deberes de los miembros de la familia

CEC 2214-2233:

\textbf{Deberes de los hijos}

\textbf{2214} La paternidad divina es la fuente de la paternidad humana (cf. \emph{Ef} 3, 14); es el fundamento del honor debido a los padres. El respeto de los hijos, menores o mayores de edad, hacia su padre y hacia su madre (cf. \emph{Pr} 1, 8; \emph{Tb} 4, 3-4), se nutre del afecto natural nacido del vínculo que los une. Es exigido por el precepto divino (cf. \emph{Ex} 20, 12).

\textbf{2215} El respeto a los padres (\emph{piedad filial)} está hecho de \emph{gratitud} para quienes, mediante el don de la vida, su amor y su trabajo, han traído sus hijos al mundo y les han ayudado a crecer en estatura, en sabiduría y en gracia. \textquote{Con todo tu corazón honra a tu padre, y no olvides los dolores de tu madre. Recuerda que por ellos has nacido, ¿cómo les pagarás lo que contigo han hecho?} (\emph{Si} 7, 27-28).

\textbf{2216} El respeto filial se expresa en la docilidad y la \emph{obediencia} verdaderas. \textquote{Guarda, hijo mío, el mandato de tu padre y no desprecies la lección de tu madre [\ldots{}] en tus pasos ellos serán tu guía; cuando te acuestes, velarán por ti; conversarán contigo al despertar} (\emph{Pr} 6, 20-22). \textquote{El hijo sabio ama la instrucción, el arrogante no escucha la reprensión} (\emph{Pr} 13, 1).

\textbf{2217} Mientras vive en el domicilio de sus padres, el hijo debe obedecer a todo lo que éstos dispongan para su bien o el de la familia. \textquote{Hijos, obedeced en todo a vuestros padres, porque esto es grato a Dios en el Señor} (\emph{Col} 3, 20; cf. \emph{Ef} 6, 1). Los niños deben obedecer también las prescripciones razonables de sus educadores y de todos aquellos a quienes sus padres los han confiado. Pero si el niño está persuadido en conciencia de que es moralmente malo obedecer esa orden, no debe seguirla.

Cuando se hacen mayores, los hijos deben seguir respetando a sus padres. Deben prevenir sus deseos, solicitar dócilmente sus consejos y aceptar sus amonestaciones justificadas. La obediencia a los padres cesa con la emancipación de los hijos, pero no el respeto que les es debido, el cual permanece para siempre. Este, en efecto, tiene su raíz en el temor de Dios, uno de los dones del Espíritu Santo.

\textbf{2218} El cuarto mandamiento recuerda a los hijos mayores de edad sus \emph{responsabilidades para con los padres}. En la medida en que ellos pueden, deben prestarles ayuda material y moral en los años de vejez y durante sus enfermedades, y en momentos de soledad o de abatimiento. Jesús recuerda este deber de gratitud (cf. \emph{Mc} 7, 10-12).

\textquote{El Señor glorifica al padre en los hijos, y afirma el derecho de la madre sobre su prole. Quien honra a su padre expía sus pecados; como el que atesora es quien da gloria a su madre. Quien honra a su padre recibirá contento de sus hijos, y en el día de su oración será escuchado. Quien da gloria al padre vivirá largos días, obedece al Señor quien da sosiego a su madre} (\emph{Si} 3, 2-6).

\textquote{Hijo, cuida de tu padre en su vejez, y en su vida no le causes tristeza. Aunque haya perdido la cabeza, sé indulgente, no le desprecies en la plenitud de tu vigor [\ldots{}] Como blasfemo es el que abandona a su padre, maldito del Señor quien irrita a su madre} (\emph{Si} 3, 12-13.16).

\textbf{2219} El respeto filial favorece la armonía de toda la vida familiar; atañe también a las \emph{relaciones entre hermanos y hermanas}. El respeto a los padres irradia en todo el ambiente familiar. \textquote{Corona de los ancianos son los hijos de los hijos} (\emph{Pr} 17, 6). \textquote{{[}Soportaos{]} unos a otros en la caridad, en toda humildad, dulzura y paciencia} (\emph{Ef} 4, 2).

\textbf{2220} Los cristianos están obligados a una especial gratitud para con aquellos de quienes recibieron el don de la fe, la gracia del bautismo y la vida en la Iglesia. Puede tratarse de los padres, de otros miembros de la familia, de los abuelos, de los pastores, de los catequistas, de otros maestros o amigos. \textquote{Evoco el recuerdo [\ldots{}] de la fe sincera que tú tienes, fe que arraigó primero en tu abuela Loida y en tu madre Eunice, y sé que también ha arraigado en ti} (\emph{2 Tm} 1, 5).

\textbf{Deberes de los padres}

\textbf{2221} La fecundidad del amor conyugal no se reduce a la sola procreación de los hijos, sino que debe extenderse también a su educación moral y a su formación espiritual. \emph{El papel de los padres en la educación} \textquote{tiene tanto peso que, cuando falta, difícilmente puede suplirse} (GE 3). El derecho y el deber de la educación son para los padres primordiales e inalienables (cf. FC 36).

\textbf{2222} Los padres deben mirar a sus hijos como a \emph{hijos de Dios} y respetarlos como a \emph{personas humanas}. Han de educar a sus hijos en el cumplimiento de la ley de Dios, mostrándose ellos mismos obedientes a la voluntad del Padre de los cielos.

\textbf{2223} Los padres son los primeros responsables de la educación de sus hijos. Testimonian esta responsabilidad ante todo por la \emph{creación de un hogar}, donde la ternura, el perdón, el respeto, la fidelidad y el servicio desinteresado son norma. La familia es un lugar apropiado para la \emph{educación de las virtudes}. Esta requiere el aprendizaje de la abnegación, de un sano juicio, del dominio de sí, condiciones de toda libertad verdadera. Los padres han de enseñar a los hijos a subordinar las dimensiones \textquote{materiales e instintivas a las interiores y espirituales} (CA 36). Es una grave responsabilidad para los padres dar buenos ejemplos a sus hijos. Sabiendo reconocer ante sus hijos sus propios defectos, se hacen más aptos para guiarlos y corregirlos:

\textquote{El que ama a su hijo, le corrige sin cesar [\ldots{}] el que enseña a su hijo, sacará provecho de él} (\emph{Si} 30, 1-2). \textquote{Padres, no exasperéis a vuestros hijos, sino formadlos más bien mediante la instrucción y la corrección según el Señor} (\emph{Ef} 6, 4).

\textbf{2224} La familia constituye un medio natural para la iniciación del ser humano en la solidaridad y en las responsabilidades comunitarias. Los padres deben enseñar a los hijos a guardarse de los riesgos y las degradaciones que amenazan a las sociedades humanas.

\textbf{2225} Por la gracia del sacramento del matrimonio, los padres han recibido la responsabilidad y el privilegio de \emph{evangelizar a sus hijos}. Desde su primera edad, deberán iniciarlos en los misterios de la fe, de los que ellos son para sus hijos los \textquote{primeros [\ldots{}] heraldos de la fe} (LG 11). Desde su más tierna infancia, deben asociarlos a la vida de la Iglesia. La forma de vida en la familia puede alimentar las disposiciones afectivas que, durante toda la vida, serán auténticos cimientos y apoyos de una fe viva.

\textbf{2226} La \emph{educación en la fe} por los padres debe comenzar desde la más tierna infancia. Esta educación se hace ya cuando los miembros de la familia se ayudan a crecer en la fe mediante el testimonio de una vida cristiana de acuerdo con el Evangelio. La catequesis familiar precede, acompaña y enriquece las otras formas de enseñanza de la fe. Los padres tienen la misión de enseñar a sus hijos a orar y a descubrir su vocación de hijos de Dios (cf. LG 11). La parroquia es la comunidad eucarística y el corazón de la vida litúrgica de las familias cristianas; es un lugar privilegiado para la catequesis de los niños y de los padres.

\textbf{2227} Los hijos, a su vez, contribuyen al \emph{crecimiento de sus padres en la santidad} (cf. GS 48, 4). Todos y cada uno deben otorgarse generosamente y sin cansarse el mutuo perdón exigido por las ofensas, las querellas, las injusticias y las omisiones. El afecto mutuo lo sugiere. La caridad de Cristo lo exige (cf. \emph{Mt} 18, 21-22; \emph{Lc} 17, 4).

\textbf{2228} Durante la infancia, el respeto y el afecto de los padres se traducen ante todo en el cuidado y la atención que consagran para educar a sus hijos, y para \emph{proveer a sus necesidades físicas y espirituales}. En el transcurso del crecimiento, el mismo respeto y la misma dedicación llevan a los padres a enseñar a sus hijos a usar rectamente de su razón y de su libertad.

\textbf{2229} Los padres, como primeros responsables de la educación de sus hijos, tienen el derecho de \emph{elegir para ellos una escuela} que corresponda a sus propias convicciones. Este derecho es fundamental. En cuanto sea posible, los padres tienen el deber de elegir las escuelas que mejor les ayuden en su tarea de educadores cristianos (cf. GE 6). Los poderes públicos tienen el deber de garantizar este derecho de los padres y de asegurar las condiciones reales de su ejercicio.

\textbf{2230} Cuando llegan a la edad correspondiente, los hijos tienen el deber y el derecho de \emph{elegir su profesión y su estado de vida}. Estas nuevas responsabilidades deberán asumirlas en una relación de confianza con sus padres, cuyo parecer y consejo pedirán y recibirán dócilmente. Los padres deben cuidar de no presionar a sus hijos ni en la elección de una profesión ni en la de su futuro cónyuge. Esta indispensable prudencia no impide, sino al contrario, ayudar a los hijos con consejos juiciosos, particularmente cuando éstos se proponen fundar un hogar.

\textbf{2231} Hay quienes no se casan para poder cuidar a sus padres, o sus hermanos y hermanas, para dedicarse más exclusivamente a una profesión o por otros motivos dignos. Estas personas pueden contribuir grandemente al bien de la familia humana.

\textbf{2232} Los vínculos familiares, aunque son muy importantes, no son absolutos. A la par que el hijo crece hacia una madurez y autonomía humanas y espirituales, la vocación singular que viene de Dios se afirma con más claridad y fuerza. Los padres deben respetar esta llamada y favorecer la respuesta de sus hijos para seguirla. Es preciso convencerse de que la vocación primera del cristiano \emph{es seguir a Jesús} (cf. \emph{Mt} 16, 25): \textquote{El que ama a su padre o a su madre más que a mí, no es digno de mí; el que ama a su hijo o a su hija más que a mí, no es digno de mí} (\emph{Mt} 10, 37).

\textbf{2233} Hacerse discípulo de Jesús es aceptar la invitación a pertenecer a la \emph{familia de Dios}, a vivir en conformidad con su manera de vivir: \textquote{El que cumpla la voluntad de mi Padre celestial, éste es mi hermano, mi hermana y mi madre} (\emph{Mt} 12, 49).

Los padres deben acoger y respetar con alegría y acción de gracias el llamamiento del Señor a uno de sus hijos para que le siga en la virginidad por el Reino, en la vida consagrada o en el ministerio sacerdotal.

La huida a Egipto

CEC 333, 530:

\textbf{333} De la Encarnación a la Ascensión, la vida del Verbo encarnado está rodeada de la adoración y del servicio de los ángeles. Cuando Dios introduce \textquote{a su Primogénito en el mundo, dice: \textquote{adórenle todos los ángeles de Dios}} (\emph{Hb} 1, 6). Su cántico de alabanza en el nacimiento de Cristo no ha cesado de resonar en la alabanza de la Iglesia: \textquote{Gloria a Dios\ldots{}} (\emph{Lc} 2, 14). Protegen la infancia de Jesús (cf. \emph{Mt} 1, 20; 2, 13.19), le sirven en el desierto (cf. \emph{Mc} 1, 12; \emph{Mt} 4, 11), lo reconfortan en la agonía (cf. \emph{Lc} 22, 43), cuando Él habría podido ser salvado por ellos de la mano de sus enemigos (cf. \emph{Mt} 26, 53) como en otro tiempo Israel (cf. \emph{2 M} 10, 29-30; 11,8). Son también los ángeles quienes \textquote{evangelizan} (\emph{Lc} 2, 10) anunciando la Buena Nueva de la Encarnación (cf. \emph{Lc} 2, 8-14), y de la Resurrección (cf. \emph{Mc} 16, 5-7) de Cristo. Con ocasión de la segunda venida de Cristo, anunciada por los ángeles (cf. \emph{Hb} 1, 10-11), éstos estarán presentes al servicio del juicio del Señor (cf. \emph{Mt} 13, 41; 25, 31 ; \emph{Lc} 12, 8-9).

\textbf{530} \emph{La Huida a Egipto} y la matanza de los inocentes (cf. \emph{Mt} 2, 13-18) manifiestan la oposición de las tinieblas a la luz: \textquote{Vino a su Casa, y los suyos no lo recibieron} (\emph{Jn} 1, 11). Toda la vida de Cristo estará bajo el signo de la persecución. Los suyos la comparten con él (cf. \emph{Jn} 15, 20). Su vuelta de Egipto (cf. \emph{Mt} 2, 15) recuerda el éxodo (cf. \emph{Os} 11, 1) y presenta a Jesús como el liberador definitivo.



\chapter{Santa María Madre de Dios}

\section{Lecturas}

PRIMERA LECTURA

Del libro de los Números 6, 22-27

Invocarán mi nombre sobre los hijos de Israel y yo los bendeciré

El Señor habló a Moisés:

«Di a Aarón y a sus hijos, esta es la fórmula con la que bendeciréis a
los hijos de Israel:

``El Señor te bendiga y te proteja,

ilumine su rostro sobre ti

y te conceda su favor.

El Señor te muestre su rostro

y te conceda la paz''.

Así invocarán mi nombre sobre los hijos de Israel y yo los bendeciré».

SALMO RESPONSORIAL

Salmo 66, 2-3. 5. 6 y 8

Que Dios tenga piedad y nos bendiga

℣. Que Dios tenga piedad nos bendiga,

ilumine su rostro sobre nosotros;

conozca la tierra tus caminos,

todos los pueblos tu salvación. ℟.

℣. Que canten de alegría las naciones,

porque riges el mundo con justicia

y gobiernas las naciones de la tierra. ℟.

℣. Oh Dios, que te alaben los pueblos,

que todos los pueblos te alaben.

Que Dios nos bendiga; que le teman

todos los confines de la tierra. ℟.

SEGUNDA LECTURA

De la carta del apóstol san Pablo a los Gálatas 4, 4-7

Envió Dios a su Hijo, nacido de mujer

Hermanos:

Cuando llegó la plenitud del tiempo, envió Dios a su Hijo, nacido de
mujer, nacido bajo la ley, para rescatar a los que estaban bajo la ley,
para que recibiéramos la adopción filial.

Como sois hijos, Dios envió a nuestros corazones el Espíritu de su Hijo,
que clama: \emph{\textquote{¡Abba,} Padre!}. Así que ya no eres esclavo, sino
hijo; y si eres hijo, eres también heredero por voluntad de Dios.

EVANGELIO

Del Santo Evangelio según san Lucas 2, 16-21

Encontraron a María y a José y al niño. Y a los ocho días, le pusieron
por nombre Jesús

En aquel tiempo, los pastores fueron corriendo hacia belén y encontraron
a María y a José, y al niño acostado en el pesebre. Al verlo, contaron
lo que se les había dicho de aquel niño.

Todos los que lo oían se admiraban de lo que les habían dicho los
pastores. María, por su parte, conservaba todas estas cosas,
meditándolas en su corazón.

Y se volvieron los pastores dando gloria y alabanza a Dios por todo lo
que habían oído y visto, conforme a lo que se les había dicho.

Cuando se cumplieron los ocho días para circuncidar al niño, le pusieron
por nombre Jesús, como lo había llamado el ángel antes de su concepción.


\section{Comentario Patrístico}

\subsection{San Atanasio de Alejandría, obispo}

La Palabra tomó de María nuestra condición

Carta a Epicteto, 5-9: PG 26, 1058. 1062-1066.

La Palabra \emph{tendió una mano a los hijos de Abrahán,} como afirma el Apóstol, \emph{y por eso tenía que parecerse en todo a sus hermanos} y asumir un cuerpo semejante al nuestro. Por esta razón, en verdad, María está presente en este misterio, para que de ella la Palabra tome un cuerpo, y, como propio, lo ofrezca por nosotros. La Escritura habla del parto y afirma: \emph{Lo envolvió en pañales;} y se proclaman dichosos los pechos que amamantaron al Señor, y, por el nacimiento de este primogénito, fue ofrecido el sacrificio prescrito. El ángel Gabriel había anunciado esta concepción con palabras muy precisas, cuando dijo a María no simplemente \textquote{\emph{lo que nacerá en ti}} ---para que no se creyese que se trataba de un cuerpo introducido desde el exterior---, sino \emph{de} para que creyéramos que aquel que era engendrado en María procedía realmente de ella.

Las cosas sucedieron de esta forma para que la Palabra, tomando nuestra condición y ofreciéndola en sacrificio, la asumiese completamente, y revistiéndonos después a nosotros de su condición, diese ocasión al Apóstol para afirmar lo siguiente: \emph{Esto corruptible tiene que vestirse de incorrupción, y esto mortal tiene que vestirse de inmortalidad}.

Estas cosas no son una ficción, como algunos juzgaron; ¡tal postura es inadmisible! Nuestro Salvador fue verdaderamente hombre, y de él ha conseguido la salvación el hombre entero. Porque de ninguna forma es ficticia nuestra salvación ni afecta sólo al cuerpo, sino que la salvación de todo el hombre, es decir, alma y cuerpo, se ha realizado en aquel que es la Palabra.

Por lo tanto, el cuerpo que el Señor asumió de María era un verdadero cuerpo humano, conforme lo atestiguan las Escrituras; verdadero, digo, porque fue un cuerpo igual al nuestro. Pues María es nuestra hermana, ya que todos nosotros hemos nacido de Adán.

Lo que Juan afirma: \emph{La Palabra se hizo carne,} tiene la misma significación, como se puede concluir de la idéntica forma de expresarse. En san Pablo encontramos escrito: \emph{Cristo se hizo por nosotros un maldito}. Pues al cuerpo humano, por la unión y comunión con la Palabra, se le ha concedido un inmenso beneficio: de mortal se ha hecho inmortal, de animal se ha hecho espiritual, y de terreno ha penetrado las puertas del cielo.

Por otra parte, la Trinidad, también después de la encarnación de la Palabra en María, siempre sigue siendo la Trinidad, no admitiendo ni aumentos ni disminuciones; siempre es perfecta, y en la Trinidad se reconoce una única Deidad, y así la Iglesia confiesa a un único Dios, Padre de la Palabra.

\subsection{San Cirilo de Alejandría, obispo }

Alabanzas de la Madre de Dios

Homilía 4, pronunciada en el Concilio de Éfeso: PG 77, 991. 995-996.

Tengo ante mis ojos la asamblea de los santos padres que, llenos de gozo y fervor, han acudido aquí, respondiendo con prontitud a la invitación de la santa Madre de Dios, la siempre Virgen María. Este espectáculo ha trocado en gozo la gran tristeza que antes me oprimía. Vemos realizadas en esta reunión aquellas hermosas palabras de David, el salmista: \emph{Ved qué dulzura, qué delicia; convivir los hermanos unidos}.

Te saludamos, santa y misteriosa Trinidad, que nos has convocado a todos nosotros en esta iglesia de santa María, Madre de Dios.

Te saludamos, María, Madre de Dios, tesoro digno de ser venerado por todo el orbe, lámpara inextinguible, corona de la virginidad, trono de la recta doctrina, templo indestructible, lugar propio de aquel que no puede ser contenido en lugar alguno, madre y virgen, por quien es llamado bendito, en los santos evangelios, el que viene en nombre del Señor.

Te saludamos, a ti, que encerraste en tu seno virginal a aquel que es inmenso e inabarcable; a ti, por quien la santa Trinidad es adorada y glorificada; por quien la cruz preciosa es celebrada y adorada en todo el orbe; por quien exulta el cielo; por quien se alegran los ángeles y arcángeles; por quien son puestos en fuga los demonios; por quien el diablo tentador cayó del cielo; por quien la criatura, caída en el pecado, es elevada al cielo; por quien la creación, sujeta a la insensatez de la idolatría, llega al conocimiento de la verdad; por quien los creyentes obtienen la gracia del bautismo y el aceite de la alegría; por quien han sido fundamentadas las Iglesias en el orbe de la tierra; por quien todos los hombres son llamados a la conversión.

Y ¿qué más diré? Por ti, el Hijo unigénito de Dios ha iluminado a \emph{los que vivían en tinieblas y en sombra de muerte}; por ti, los profetas anunciaron las cosas futuras; por ti, los apóstoles predicaron la salvación a los gentiles; por ti, los muertos resucitan; por ti, reinan los reyes, por la santísima Trinidad.

¿Quién habrá que sea capaz de cantar como es debido las alabanzas de María? Ella es madre y virgen a la vez; ¡qué cosa tan admirable! Es una maravilla que me llena de estupor. ¿Quién ha oído jamás decir que le esté prohibido al constructor habitar en el mismo templo que él ha construido? ¿Quién podrá tachar de ignominia el hecho de que la sirviente sea adoptada como madre?

Mirad: hoy todo el mundo se alegra; quiera Dios que todos nosotros reverenciemos y adoremos la unidad, que rindamos un culto impregnado de santo temor a la Trinidad indivisa, al celebrar, con nuestras alabanzas, a María siempre Virgen, el templo santo de Dios, y a su Hijo y esposo inmaculado: porque a él pertenece la gloria por los siglos de los siglos. Amén.


Habitaba en la tierra\\ y llenaba los cielos\\ la Palabra de Dios infinita.\\ Su bajada amorosa hasta el hombre no cambió su morada superna.\\ Era el parto divino de Virgen\\ que este canto escuchaba:

Salve, mansión que contiene el Inmenso; Salve, dintel del augusto Misterio.

Salve, de incrédulo equívoco anuncio; Salve, del fiel inequívoco orgullo.

Salve, carroza del Santo que portan querubes;

Salve, sitial del que adoran sin fin serafines.

Salve, tú sola has unido dos cosas opuestas;

Salve, tú sola a la vez eres Virgen y Madre.

Salve, por ti fue borrada la culpa;

Salve, por ti Dios abrió el Paraíso.

Salve, tú llave del Reino de Cristo;

Salve, esperanza de bienes eternos.

Salve, ¡Virgen y Esposa!

(\emph{Akathistos}, 15)

\section{Homilías}

Las lecturas para esta solemnidad son las mismas en los tres ciclos dominicales. No obstante, las homilías han sido distribuidas en esta obra en tres grupos, tomando en cuenta el ciclo litúrgico correspondiente al año en que fueron pronunciadas.

\subsection{San Pablo VI, papa}
\subsubsection{Homilía (1978)}
Basílica de Santa María la Mayor. Domingo 1 de enero de 1978. XI Jornada Mundial de la Paz.

Convocados por la fe en esta basílica ---erigida por nuestro predecesor Sixto III pocos años después del Concilio de Efeso que había proclamado solemnemente en el año 431 a María la \emph{Theotokos,} es decir, Madre de Dios---, unamos en esta celebración la alabanza por los altísimos privilegios concedidos por Dios a la Virgen Madre, juntamente con la reflexión sobre las exigencias cristianas de la paz en el mundo.

En este espléndido templo, expresión singular de la ferviente devoción mariana del pueblo romano, historia y arte se han fundido admirablemente a través de los siglos; este templo, con su belleza clásica y su atractivo misterioso, nos lleva a pensamientos de alegría serena; en los mosaicos, tan antiguos, refulgen las diversas etapas de la historia de la salvación; en lo alto del ábside resplandece la escena sublime de la \textquote{Coronación de María}, obra de Jacopo Torriti; y junto a los recuerdos de la gruta del Pesebre, los Magos adoran al Verbo encarnado, en la composición escultórica de Arnolfo di Cambio.

Hemos querido celebrar la \textquote{Jornada de la Paz} precisamente en este marco estupendo, creado por la piedad de nuestros antepasados; y desde aquí nos proponemos dirigir una vez más a toda la humanidad las palabras suaves y solemnes de la paz.

La Jornada de la Paz no hace referencia a la paz de un día, de un día solo. Al celebrarla en la primera jornada del año civil, aporta siempre algo al año que comienza: una celebración conjunta que es augurio y promesa al comienzo del calendario; pero presenta también un tema que hemos propuesto nosotros y que resulta ocasión y fuente de convergencia de intenciones con dimensiones universales. Convergencia en la oración, para todos los católicos y para todos los cristianos que quieran unirse a la Jornada; convergencia en el estudio y la reflexión, para los responsables de la guía colectiva de la sociedad y para todos los hombres de buena voluntad; convergencia en una acción conjunta, un testimonio presentado así al mundo a través del esfuerzo solidario para defender a todos los habitantes de nuestro planeta, tan gravemente amenazados en nuestros días por \textquote{el carácter absurdo de la guerra moderna}, según hemos subrayado en nuestro reciente Mensaje, y para construir la paz cuya necesidad perentoria la conciencia de la humanidad siente cada vez más.

Cada uno de los temas de las diferentes \textquote{Jornadas de la Paz} completa a los precedentes, al igual que una piedra se añade a las otras para construir una casa: esta casa de la paz, que se funda ---como decía nuestro venerado predecesor Juan XXIII--- sobre cuatro pilares, \textquote{la verdad, la justicia, la solidaridad operante y la libertad} (cf. \emph{Pacem in terris}, 47).

Pero el pensamiento dominante de esta celebración nuestra se presenta espontáneamente en el binomio \textquote{María y la paz}. ¿Acaso no hay relación entre la Maternidad divina de María y la paz que celebramos el mismo día de su fiesta, una relación que no es accidental, sino que extrae su realidad y fruto de todo el patrimonio dogmático, patrístico, teológico y místico de la Iglesia de Cristo? ¿No es verdad que existe una razón histórica que se añade a éstas y nos reúne hoy con vosotros, carísimos hijas e hijos, romanos de nacimiento o de adopción? En efecto, ¿no venís esta mañana a continuar y confirmar con vuestra presencia la práctica profundamente religiosa y filial de vuestros antepasados, diocesanos de esta Iglesia de Roma, que antes de que esa fecha señalase en Occidente el comienzo del año civil eligió ya la octava de Navidad para rendir homenaje especial a la Madre de Dios? Y en torno a vosotros, ¿no está reunida místicamente toda la Iglesia, todo el Pueblo de Dios en esta Patriarcal Basílica, para celebrar a un tiempo la Maternidad de María y la paz, esa paz que su Hijo Jesucristo vino a traer al mundo?

Pero no es preciso ir muy lejos en nuestra reflexión. Si hay correlación entre la maternidad divina de María y la paz, ¿qué relación hay entre esta maternidad y la repulsa de la violencia que figura en el tema elegido para la jornada de este año de 1978? Sí, existe relación. Los estudios teológicos y exegéticos acerca de este argumento se multiplican, lo subrayan cada vez más en la perspectiva que les es propia, y añaden a sus conclusiones la opinión espontánea del pueblo.

Sea que se contemple la violencia ---como lo hemos hecho en nuestro reciente Mensaje para esta Jornada--- bajo el aspecto colectivo internacional, es decir, bajo el de la guerra moderna que amenaza con su \textquote{suprema irracionalidad}, con su \textquote{carácter absurdo}, y con las tristes hipótesis de una guerra espacial; sea que se la considere bajo los aspectos múltiples de la violencia pasional de la delincuencia creciente, o de la violencia civil erigida en sistema, se plantea una pregunta fundamental: ¿cuáles son las causas de tales comportamientos o de las ideas y sentimientos que los inspiran? Repetidas veces hemos recordado estas causas en nuestros Mensajes precedentes, particularmente en los que tratan sobre el desarme y la defensa de la vida. Esta mañana sólo recordaremos una: la sacudida provocada en la sociedad por condiciones de vida deshumanizadoras (cf. \emph{Gaudium et spes}, 27).

Tales condiciones de vida provocan, sobre todo en los jóvenes, frustraciones que desencadenan reacciones de violencia y agresividad contra ciertas estructuras y coyunturas de la sociedad contemporánea que quisiera reducir a los jóvenes a simples instrumentos pasivos. Pero su contestación, instintiva u organizada, se dirige no sólo a las consecuencias de estas situaciones penosas, sino también a \textquote{una sociedad rebosante de bienestar material, satisfecha y gozosa, pero privada de ideales superiores que dan sentido y valor a la vida} (Mensaje de Navidad; \emph{L'Osservatore Romano,} Edición en Lengua Española, 5 de enero de 1969, pág. 2). En una palabra, una sociedad desacralizada; una sociedad sin alma, una sociedad sin amor.

De hecho, ¿quiénes son frecuentemente estos violentos cuyas acciones, provocando temor u horror, hacen necesario el deber de proteger nuestra convivencia humana? Muy a menudo, con demasiada frecuencia, los que llevan a cabo esas acciones intolerables son personas olvidadas, marginadas, despreciadas, personas que no son amadas o, al menos, no se sienten amadas. Ávidas más de tener que de ser; testigos y con frecuencia víctimas de la injusticia de los más fuertes o, en algunos casos bien conocidos, de la \textquote{violencia estructural de algunos regímenes políticos}; ¿cómo pueden no sentirse \textquote{hijos pródigos} en esta sociedad anónima que los ha engendrado y, luego, con frecuencia abandonado, sin baremo fijo de valores, y en resumen, sin brújula ni estrella, sin la estrella de Navidad?

En el secreto de su corazón estos \textquote{huérfanos}, ¿acaso no aspiran desde los fondos de esta sociedad madrastra a una sociedad materna y, en fin, a la maternidad religiosa de la Madre universal, a la maternidad de María?

Las palabras de Cristo en la cruz, \textquote{Mujer, he ahí a tu hijo} (\emph{Jn} 19, 26-27), ¿no es verdad que iban dirigidas a ellos a través de San Juan: \textquote{Madre, he ahí a tus hijos}? ¿Y acaso no era para ellos la frase del Señor moribundo cuando decía: \textquote{Hijos, he ahí a vuestra madre}, una madre que os ama, una madre a la que amar, una madre situada en el vértice de una sociedad del amor? Es decir, Madre de Dios y del Redentor (\emph{Lumen gentium}, 53), del nuevo Adán en el que y por el que todos los hombres son hermanos (cf. \emph{Rom} 8, 29), María, nueva Eva (cf. \emph{Lumen gentium}, 63), se transforma de este modo en la madre de todos los vivientes (cf. \emph{ib.}, 56), nuestra madre amantísima (\emph{ib}., 53). Miembro eminente y plenamente singular de la Iglesia (\emph{ib.}, 53), \emph{es tipo de la} misma (\emph{ib}., 63); es imagen y principio de la Iglesia que habrá de tener su cumplimiento en la vida futura (\emph{ib}., 68).

Aquí se nos presenta una nueva visión que es el reflejo de la Virgen en la Iglesia, como dice San Agustín: María \emph{\textquote{figuram in se sasnctae Ecclesiae demonstrat:} María refleja en sí la figura misma de la Iglesia} \emph{(De Symbolo}, CI; \emph{PL} 661; H. de Lubac, \emph{Méditations sur l'Eglise,} pág. 245).

Madre de Cristo Rey, Príncipe de la Paz (\emph{Is} 9, 6). María se transforma por esto mismo en Reina y Madre de la paz. El Concilio Vaticano II, al enumerar los títulos de María, jamás la separa de la Iglesia.

Así la Iglesia, toda la Iglesia, a ejemplo de María debe vivir también ella cada vez con mayor intensidad la propia maternidad universa (cf. \emph{Lumen gentium},} 64), respecto de toda la familia humana actualmente deshumanizada porque está desacralizada.

\textquote{Madre y Maestra}, la Iglesia de Cristo no pretende construir la paz del mundo sin El o suplantándolo; sino que proclamando el reino de Dios en todas las naciones se propone al mismo tiempo \textquote{descubrir al hombre el sentido de la propia existencia}, sabiendo que \textquote{el que sigue a Cristo, Hombre perfecto, se perfecciona cada vez más en su propia dignidad de hombre} (\emph{Gaudium et spes}, 41).

Y volviendo con el pensamiento a María Reina de la Paz, nos complacemos en recordar que nuestro venerado predecesor el Papa Benedicto XV quiso exaltar este título debido a la Virgen María haciendo esculpir un monumento en su honor en esta misma basílica, al finalizar la primera guerra mundial.

Nadie piense que la paz, de la que María es portadora, se pueda confundir con la debilidad o la insensibilidad de los tímidos o de los viles; recordemos el himno más bello de la liturgia mariana, el \emph{Magnificat,} en el que la voz sonora y valiente de María resuena para dar fortaleza y valor a los promotores de la paz: \textquote{Desplegó el poder de su brazo y dispersó a los que se engríen con los pensamientos de su corazón. Derribó a los potentados de sus tronos y ensalzó a los humildes} (\emph{Lc} 51-52).

Nos proponemos confiar a María la causa de la paz en todo el mundo, y en particular en la querida nación del Líbano, ejemplo de país arrollado por la espiral de la violencia no tanto por causas suyas internas cuanto por el reflejo de las situaciones de esa región que no han encontrado todavía soluciones justas; es decir que, en realidad, ha sido víctima de dichas situaciones. En esta Jornada de la Paz. exhortamos, pues, a los aquí presentes y a todos los fieles a orar por el Líbano a la Virgen \emph{Notre Dame du Liban,} para que acelere la reconciliación de sus hijos \emph{} y el resurgir espiritual y moral, además del material, de la nación.

Que en las esperanzas de paz que comienzan a vislumbrarse en Oriente Medio, la reconciliación de los distintos grupos libaneses y la convivencia serena de la población lleguen a ser factor de reconciliación y de repulsa de la violencia para todos los pueblos de la región.

Al concluir estas reflexiones nuestras, queremos dirigir una llamada apremiante a todos nuestros hijos y a cada uno particularmente.

Procure cada cual aportar su contribución práctica, generosa y auténtica a la paz del mundo, eliminando del corazón en primer lugar toda forma de violencia, todo sentimiento de avasallamiento del hermano. Actuando así os encontraréis ya en el sendero de la paz universal que se funda en la paz efectiva de cada uno.

Si queréis conseguir que la paz reine en todo el mundo, hacedla reinar primero en vuestro corazón, en vuestra familia, en vuestra casa, en vuestro barrio, en vuestra ciudad, en vuestra región, en vuestra nación:

De este modo los demás sentirán incluso el encanto y el gozo de poder vivir en serenidad y de esforzarse para que este inmenso bien sea aspiración, exigencia y patrimonio de todos.

Esto lo queremos decir en particular a vosotros, jóvenes, y a vosotros, muchachos, presentes hoy muy numerosos en esta basílica.

Hemos querido terminar nuestro reciente Mensaje para la jornada de la Paz dirigiéndonos en especial a los jóvenes y a los muchachos de todo el mundo. porque vosotros tenéis esa extraordinaria capacidad de apertura y esa gozosa disponibilidad que por desgracia a veces los adultos han olvidado o perdido.

También vosotros, jóvenes y muchachos, tenéis una palabra que decir y hacer oír a los mayores, una palabra juvenil, nueva, original.

Comunicad esta palabra de paz, este \textquote{no a la violencia} con energía, con fuerza, con la fuerza de vuestro corazón puro, de vuestros oros límpidos, de vuestra alegría de vivir, pero de vivir en un mundo en el que \textquote{se darán el abrazo la justicia y la paz} (\emph{Sal} 84, 11).

En vuestros ideales y en vuestro comportamiento dad siempre la prioridad al amor, es decir, a la comprensión, a la benevolencia, a la solidaridad con los otros.

Reforzad vuestra convicción de paz en la oración personal y comunitaria: en el dialogo y la meditación en los que os esforzáis por conocer cada vez más profundamente a Cristo y por comprender su mensaje con todas sus exigencias: en los sacramentos, y sobre todo en el sacramento de la Eucaristía, en el que el mismo Cristo os da la fe, la esperanza y, ante todo, la caridad; en fin, reforzadla en la devoción filial a la Virgen María.

Si vuestra convicción es sólida y firme. en todas las manifestaciones de vuestra juventud seréis testimonios de la paz y el amor de Cristo que está: en vosotros.

Jóvenes y muchachos, lleváis en vosotros el porvenir del mundo y de la historia. Este mundo será mejor, más fraterno, más justo, si ya desde ahora toda vuestra vida está abierta a la gracia de Cristo, a los ideales de amor y de paz que os enseña el Evangelio.

María, Reina de la Paz, \emph{Salus Populi Romani,} interceda por estas intenciones.

\subsection{San Juan Pablo II, papa}

\subsubsection{Homilía (1981)}

Basílica de San Pedro\\ Jueves 1 de enero de 1981. XIV Jornada Mundial de la Paz.


1. \textquote{\ldots{}al llegar la plenitud de los tiempos, envió Dios a su Hijo, nacido de mujer\ldots{}}. Son palabras de San Pablo, tomadas de la liturgia de hoy \emph{(Gál} 4, 4).

\emph{\textquote{La plenitud de los tiempos\ldots{}}}.

Estas palabras tienen hoy una particular elocuencia, puesto que nos es dado pronunciar por primera vez la nueva fecha, esto es, el nombre del nuevo año solar: 1981. Así sucede cada año el primer día de enero. Pasan los años, cambian las fechas, transcurre el tiempo. Con el tiempo pasa también toda la naturaleza, naciendo, desarrollándose, muriendo. Y pasa también \emph{el hombre}; pero él pasa conscientemente. Tiene \emph{la conciencia de su pasar, la conciencia del tiempo}. Con el metro del tiempo mide la historia del mundo y, sobre todo, la propia historia. No sólo los años, los decenios, los siglos, los milenios, sino también los días, las horas, los minutos, los segundos.

La liturgia de hoy nos dice con las palabras de San Pablo que el tiempo, que es el metro del pasar de los seres humanos en el mundo, está sometido también a otra medida, es decir, \emph{a la medida de la plenitud, que proviene de Dios:} la plenitud del tiempo. Efectivamente, en el tiempo ---en el tiempo humano, terreno---, Dios lleva a cumplimiento su proyecto eterno de amor. Mediante el amor de Dios, \emph{el tiempo} está sometido a la \emph{eternidad} y al \emph{Verbo}.

El Verbo se hizo carne\ldots{} en el tiempo.

Los años que pasan, que terminan el 31 de diciembre y comienzan de nuevo el 1 de enero, pasan en realidad confrontándose con esa plenitud, que proviene de Dios. Pasan frente a la eternidad y al Verbo. Cada año del calendario humano lleva, juntamente con el tiempo, una pequeña parte del \textquote{Kairós} divino. Cada uno comienza, dura y transcurre en relación \emph{a esa plenitud del tiempo que viene de Dios}.

Es preciso darse cuenta de esto, de modo particular hoy, que es el primer día del año nuevo.

2. Qué fuerte y espléndidamente se capta esta realidad, cuando nos damos cuenta de que este primer día del año nuevo \emph{es,} al mismo tiempo, \emph{el día de la octava de Navidad}. El año nuevo nace en el esplendor del misterio en el que se ha revelado la \textquote{plenitud del tiempo}.

\textquote{Dios envió a su Hijo, nacido de mujer}.

Y precisamente hacia esa Mujer, hacia la Madre del Hijo de Dios, hacia la Theotokos se dirigen hoy, al comienzo del año nuevo, de modo especial, el pensamiento y el corazón de la Iglesia. María está presente durante toda la octava; sin embargo, la Iglesia desea venerarla particularmente hoy, con un día dedicado totalmente a Ella: \emph{la solemnidad de Santa María, Madre de Dios}.

A ella, pues, a la Maternidad admirable de la Virgen de Nazaret, ligada a la \textquote{plenitud de los tiempos}, nos dirigimos mediante este comienzo del año, que coincide con el día de hoy.

Y recordamos que es el comienzo del año del Señor 1981, durante el cual resonarán con eco lejano en los siglos las fechas conmemorativas de los dos importantes Concilios de los primeros tiempos de la Iglesia, que permaneció una y única, a pesar de las primeras grandes herejías que surgieron. Efectivamente, en el año 381 tuvo lugar el \emph{primer Concilio de Constantinopla,} que, después del Concilio de Nicea, fue el segundo Concilio Ecuménico de la Iglesia y al cual debemos el \textquote{Credo} que se recita constantemente en la liturgia. Una herencia particular de aquel Concilio es la doctrina \emph{sobre el Espíritu Santo} proclamada así en la liturgia latina: Credo in Spiritum Sanctum Dominum et vivificantem, qui ex Patre Filioque procedit ----(la formulación de la teología oriental, en cambio, dice: qui a Patre per Filium procedit)---. Qui cum Patre et Filio simul adoratur et conglorificatur qui locutus est per prophetas.

Y, luego, \emph{el año 431} (hace 1550 años), se celebró el \emph{Concilio de Efeso,} que confirmó, con inmensa alegría de los participantes, la fe de la Iglesia en la Maternidad Divina de María. Aquel que \textquote{nació de María Virgen}, como hombre, es, al mismo tiempo, el verdadero Hijo de Dios, \textquote{de la misma naturaleza que el Padre}. Aquella, de la cual \textquote{fue concebido por obra del Espíritu Santo} y que lo trajo al mundo la noche de Belén, es verdadera Madre de Dios: \emph{Theotokos}.

Basta recitar con atención las palabras de nuestro Credo, para darse cuenta de cuán profundamente estos dos Concilios, que recordaremos en el curso del año 1981, están orgánicamente ligados el uno al otro \emph{con la profundidad del misterio divino y humano}. Sobre este misterio se construye la fe de la Iglesia.

3. El primer día del año deseamos leer de nuevo en la profundidad de ese misterio el mensaje de la paz, que, de una vez para siempre, se reveló en la noche de Belén: \emph{¡Paz a los hombres de buena voluntad! ¡Paz en la tierra!,} he aquí lo que el misterio del nacimiento de Dios quiere decirnos cada año y lo que la Iglesia pone de relieve también hoy, primer día del año nuevo.

\textquote{Dios envió a su Hijo, nacido de mujer\ldots{}} para que nosotros podamos recibir la filiación adoptiva.

\textquote{Como sois hijos, Dios envió a vuestros corazones al Espíritu de su Hijo, que clama: ¡Abbá! (Padre). Así que ya no eres esclavo, sino hijo\ldots{}} (\emph{Gál} 4, 6-7).

Toda la humanidad desea ardientemente la paz y ve la guerra como el peligro más grande en su existencia terrena. La Iglesia se halla totalmente presente en estos deseos y, al mismo tiempo, en los miedos y en las preocupaciones que agobian a todos los hombres, manifestando estos sentimientos, de modo particular, el primer día del año nuevo.

¿Qué es la paz? ¿Qué puede ser la paz en la tierra, la paz \emph{entre los hombres y los pueblos,} sino \emph{el fruto de la fraternidad,} que se manifieste más fuerte que lo que divide y contrapone recíprocamente a los hombres? De esta fraternidad habla precisamente San Pablo, cuando escribe a los Gálatas: \textquote{Vosotros sois hijos}. Y \emph{si hijos} ---los hijos de Dios en Cristo--- entonces, también \emph{hermanos}.

Y a continuación escribe: \textquote{Así que ya no eres esclavo, sino hijo}. En este contexto se inserta el tema del mensaje elegido para la Jornada de la Paz del primero de enero de 1981. Dice: \emph{\textquote{Para servir a la paz, respeta la libertad}}.

4. ¡Sí! Debemos apelar a la fraternidad, queridos hermanos y hermanas, si queremos superar los monstruosos \emph{mecanismos} que, en la vida y en el desarrollo de las potencias del mundo contemporáneo, \emph{trabajan en favor de la guerra}.

Es necesario que nosotros consideremos a la humanidad como una única gran familia, en la cual todas las clases de personas deben ser reconocidas y acogidas como hermanos. En los umbrales de un nuevo año, dirijamos de modo especial maestro pensamiento y nuestra solicitud a aquellos, entre estos hermanos, que se hallan en particulares situaciones de necesidad y esperan que los ingentes recursos, destinados a construir instrumentos de recíproca destrucción, sean empleados, en cambio, para las urgentes obras de socorro y de mejoramiento de las condiciones de vida.

5. Como es sabido, el 1981 ha sido proclamado por la ONU \textquote{Año Internacional de los minusválidos}. Se trata de millones de personas que tienen enfermedades congénitas, enfermedades crónicas, o que están afectadas por varias formas de deficiencia mental o debilidades sensoriales; estas personas en el curso del año interpelarán de manera más aguda a nuestra conciencia humana y cristiana. Según recientes estadísticas, su número asciende a más de 400 millones. También ellos son hermanos nuestros. Es necesario que su dignidad humana y sus derechos inalienables reciban pleno y efectivo reconocimiento durante todo el arco de su existencia.

En el pasado noviembre, durante la reunión de un grupo de trabajo, la Pontificia Academia de las Ciencias, en su constante obra al servicio de la humanidad mediante la investigación científica, ha profundizado en el estudio de una clase especial de minusválidos, los mentales. La debilidad mental, que afecta a casi al tres por ciento de la población mundial, debe ser tenida en consideración especial, porque constituye el más grave obstáculo para la realización del hombre. La relación del mencionado grupo de trabajo ha puesto de relieve la posibilidad de cuidados preventivos de las causas de la debilidad mental, mediante oportunas terapias. La ciencia y la medicina ofrecen, pues, un mensaje de esperanza y, al mismo tiempo, de empeño para toda la humanidad. Si sólo una parte del \textquote{budget} para la carrera de armamentos fuese destinado a este objetivo, se podrían conseguir éxitos importantes y aliviar la suerte de numerosas personas que sufren.

Al comienzo de este año deseo confiar todas las personas minusválidas a la materna protección de María. En la Pascua de 1971, 4.000 minusválidos mentales, divididos en pequeños grupos acompañados por familiares y educadores, fueron en peregrinación a Lourdes y vivieron días de paz y de serenidad juntamente con los otros peregrinos. Deseo de corazón que, bajo la mirada materna de María, se multipliquen las experiencias de solidaridad humana y cristiana, en una fraternidad renovada que una a los débiles y a los fuertes en el camino común de la vocación divina de la persona humana.

6. Al pensar, en los umbrales de este nuevo año, sobre las más graves necesidades de la humanidad, quisiera llamar también la atención sobre esa parte de la familia humana que se encuentra en extrema necesidad a causa de la situación alimentaria. El hambre y la mal nutrición constituyen hoy, efectivamente, un problema dramático de supervivencia para millones de seres humanos, especialmente de niños en amplias zonas de nuestro globo. Mi pensamiento se dirige particularmente a algunas extensas regiones de África afectadas por la sequía, como el Sahel, y de Asia, damnificadas por calamidades naturales o que deben afrontar una considerable afluencia de refugiados.

Según una relación de la FAO, al menos 26 países africanos han tenido últimamente cosechas inferiores a las del pasado. En algunas partes de ese continente persiste el hambre y se registran carestías periódicas, que causan no pocas víctimas. Según los cálculos de expertos, las reservas mundiales de cereales disminuirán por tercer año consecutivo si continua la tendencia actual. Hago votos de corazón a fin de que todos los responsables, todas las organizaciones y todos los hombres de buena voluntad den su aportación para la realización de medidas que permitan un socorro más efectivo a los hermanos que se encuentran en la indigencia y, a la vez, se cree un sistema más eficaz de seguridad alimentaria. La palabra de Cristo \textquote{Tuve hambre y me disteis de comer}, es una llamada, urgente y particularmente actual, a nuestras responsabilidades.

Son penetrantes las palabras de San Pablo de la liturgia de hoy. Es necesario qué la vida de la gran familia humana en todo el mundo se transforme bajo el signo de la fraternidad universal de los hombres. Efectivamente, somos hijos de Dios: Dios ha enviado a nuestros corazones el Espíritu de su Hijo que clama: Abbá, Padre. Por lo tanto: ¡ninguno es esclavo, sino hijo!

7. Durante el año que acaba de terminar se ha recordado de modo particular la \emph{figura de San Benito,} como Patrono de Europa, en relación con el 1.500 aniversario de su nacimiento.

Al meditar sobre el desarrollo de los acontecimientos más antiguos y sobre los contemporáneos, ha parecido justo proclamar Copatronos de Europa, al final del año, \emph{a los Santos Cirilo y Metodio}, que representan otro gran componente en la misión cristiana y en la obra de la economía de la salvación en nuestro continente. Es la parte ligada a la heredad de la Grecia antigua y del Patriarcado de Constantinopla, desde donde estos dos hermanos fueron enviados en misión a los pueblos de Europa Meridional y Oriental, precisamente a los eslavos. En efecto, \emph{Europa se hizo cristiana} bajo la acción de estos dos componentes.

Nos ha parecido, pues, que, particularmente al final del año en el que se ha entablado el \emph{diálogo} teológico definitivo \emph{entre la Iglesia católica} y \emph{toda la ortodoxia,} tiene una gran elocuencia el haber puesto de relieve la misión de los Santos Cirilo y Metodio. Es la elocuencia de la reconciliación y de la paz, que en todos los caminos de la humanidad debe demostrarse más potente que las fuerzas de la división y de la amenaza recíproca.

8. Termino citando una vez más las palabras de la liturgia de hoy:

\textquote{El Señor tenga piedad y nos bendiga, ilumine su rostro sobre nosotros: \emph{conozca} la tierra \emph{tus caminos,} todos los pueblos tu salvación. El Señor tenga piedad y nos bendiga} (Salmo responsorial).



\subsubsection{Homilía (1999)}

1 de enero de 1999. XXXII Jornada Mundial de la Paz.


1. \emph{Christus heri et hodie, principium et finis, alpha et omega} \ldots{} \textquote{Cristo ayer y hoy, principio y fin, alfa y omega. Suyo es el tiempo y la eternidad. A él la gloria y el poder por los siglos de los siglos} (\emph{Misal romano}, preparación del cirio pascual).

Todos los años, durante la Vigilia pascual, la Iglesia renueva esta solemne aclamación a Cristo, Señor del tiempo. También el último día del año proclamamos esta verdad, en el paso del \textquote{ayer} al \textquote{hoy}: \textquote{ayer}, al dar gracias a Dios por la conclusión del año viejo; \textquote{hoy}, al acoger el año que empieza. \emph{Heri et hodie}. Celebramos a Cristo que, como dice la Escritura, es \textquote{el mismo ayer, hoy y siempre} (\emph{Hb} 13, 8). Él es el Señor de la historia; suyos son los siglos y los milenios.

Al comenzar el año 1999, el último antes del gran jubileo, parece que el misterio de la historia se revela ante nosotros con una profundidad más intensa. Precisamente por eso, la Iglesia ha querido imprimir el signo trinitario de la presencia del Dios vivo sobre el trienio de preparación inmediata para el acontecimiento jubilar.

2. El primer día del nuevo año concluye la Octava de la Navidad del Señor y está dedicado a la santísima Virgen, venerada como Madre de Dios. El evangelio nos recuerda que \textquote{guardaba todas estas cosas y las meditaba en su corazón} (\emph{Lc} 2, 19). Así sucedió en Belén, en el Gólgota, al pie de la cruz, y el día de Pentecostés, cuando el Espíritu Santo descendió al cenáculo.

Y lo mismo sucede también hoy. La Madre de Dios y de los hombres guarda y medita en su corazón todos los problemas de la humanidad, grandes y difíciles. La \emph{Alma Redemptoris Mater} camina con nosotros y nos guía, con ternura materna, hacia el futuro. Así, ayuda a la humanidad a cruzar todos los \textquote{umbrales} de los años, de los siglos y de los milenios, sosteniendo su esperanza en aquel que es el Señor de la historia

3. \emph{Heri et hodie}. Ayer y hoy. \textquote{Ayer} invita a la retrospección. Cuando dirigimos nuestra mirada a los acontecimientos de este siglo que está a punto de terminar, se presentan ante nuestros ojos las dos guerras mundiales: cementerios, tumbas de caídos, familias destruidas, llanto y desesperación, miseria y sufrimiento. ¿Cómo olvidar los campos de muerte, a los hijos de Israel exterminados cruelmente y a los santos mártires: el padre Maximiliano Kolbe, sor Edith Stein y tantos otros?

Sin embargo, nuestro siglo es también el siglo de la \emph{Declaración universal de derechos del hombre}, cuyo 50° aniversario se celebró recientemente. Teniendo presente precisamente este aniversario, en el tradicional Mensaje para la actual \emph{Jornada mundial de la paz}, quise recordar que el secreto de la paz verdadera reside en el respeto de los derechos humanos. \textquote{El reconocimiento de la dignidad innata de todos los miembros de la familia humana (\ldots{}) es el fundamento de la libertad, de la justicia y de la paz en el mundo} (n. 3: \emph{L'Osservatore Romano}, edición en lengua española, 18 de diciembre de 1998, p. 6).

El concilio Vaticano II, el concilio que ha preparado a la Iglesia para entrar en el tercer milenio, reafirmó que el mundo, teatro de la historia del género humano, ha sido liberado de la esclavitud del pecado por Cristo crucificado y resucitado, \textquote{para que se transforme, según el designio de Dios, y llegue a su consumación} (\emph{Gaudium et spes}, 2). Es así como los creyentes miran al mundo de nuestros días, a la vez que avanzan gradualmente hacia el umbral del año 2000.

4. El Verbo eterno, al hacerse hombre, entró en el mundo y lo acogió para redimirlo. Por tanto, el mundo no sólo está marcado por la terrible herencia del pecado; es, ante todo, un mundo salvado por Cristo, el Hijo de Dios, crucificado y resucitado. Jesús es el Redentor del mundo, el Señor de la historia. \emph{Eius sunt tempora et saecula}: suyos son los años y los siglos. Por eso creemos que, al entrar en el tercer milenio junto con Cristo, cooperaremos en la transformación del mundo redimido por él. \emph{Mundus creatus, mundus redemptus}.

Desgraciadamente, la humanidad cede a la influencia del mal de muchos modos. Sin embargo, impulsada por la gracia, se levanta continuamente, y camina hacia el bien guiada por la fuerza de la redención. Camina hacia Cristo, según el proyecto de Dios Padre.

\textquote{Jesucristo es el principio y el fin, el alfa y la omega. Suyo es el tiempo y la eternidad}.

Empecemos este año nuevo en su nombre. Que María nos obtenga la gracia de ser fieles discípulos suyos, para que con palabras y obras lo glorifiquemos y honremos por los siglos de los siglos: \emph{Ipsi gloria et imperium per universa aeternitatis saecula}. Amén.

\subsubsection{Homilía (2002): Madre de la paz}

1 de enero del 2002.

1. \textquote{¡Salve, Madre santa!, Virgen Madre del Rey que gobierna cielo y tierra por los siglos de los siglos} (cf. \emph{Antífona de entrada}).

Con este antiguo saludo, la Iglesia se dirige hoy, octavo día después de la Navidad y primero del año 2002, a María santísima, invocándola como \emph{Madre de Dios}.

El Hijo eterno del Padre tomó en ella nuestra misma carne y, a través de ella, se convirtió en \textquote{hijo de David e hijo de Abraham} (\emph{Mt} 1, 1). Por tanto, María es su verdadera Madre: ¡\emph{Theotókos}, Madre de Dios!

Si Jesús es la vida, María es la Madre de la vida.

Si Jesús es la esperanza, María es la Madre de la esperanza.

Si Jesús es la paz, María es la Madre de la paz, Madre del Príncipe de la paz.

Al entrar en el nuevo año, pidamos a esta Madre santa que nos bendiga. Pidámosle que nos dé a Jesús, \emph{nuestra bendición plena, en quien el Padre ha bendecido de una vez para siempre la historia}, transformándola en historia de salvación.

2. \emph{¡Salve, Madre santa!} Bajo la mirada materna de María se sitúa esta \emph{Jornada mundial de la paz}. Reflexionamos sobre la paz en un clima de preocupación generalizada a causa de los recientes acontecimientos dramáticos que han sacudido el mundo. Pero, aunque pueda parecer humanamente difícil mirar al futuro con optimismo, no debemos ceder a la tentación del desaliento.

Al contrario, debemos trabajar por la paz con valentía, conscientes de que el mal no prevalecerá.

La luz y la esperanza para este compromiso nos vienen de Cristo. El \emph{Niño} nacido en Belén es la Palabra eterna del Padre hecha carne por nuestra salvación, es el \textquote{Dios con nosotros}, que trae consigo \emph{el secreto de la verdadera paz}. Es el \emph{Príncipe de la paz}.

3. Con estos sentimientos, saludo con deferencia a los ilustres señores embajadores ante la Santa Sede que han querido participar en esta solemne celebración. Saludo afectuosamente al presidente del Consejo pontificio Justicia y paz, señor cardenal François Xavier Nguyên Van Thuân, y a todos sus colaboradores, y les agradezco el esfuerzo que realizan a fin de difundir mi Mensaje anual para la Jornada mundial de la paz, que este año tiene como tema: \textquote{No hay paz sin justicia, no hay justicia sin perdón}.

\emph{Justicia y perdón}: estos son los dos \textquote{pilares} de la paz, que he querido poner de relieve. \emph{Entre justicia y perdón no hay contraposición, sino complementariedad}, porque ambos son esenciales para la promoción de la paz. En efecto, esta, mucho más que un cese temporal de las hostilidades, es una profunda cicatrización de las heridas abiertas que rasgan los corazones (cf. \emph{Mensaje}, 3: \emph{L'Osservatore Romano,} edición en lengua española, 14 de diciembre de 2001, p. 7). Sólo el perdón puede apagar la sed de venganza y abrir el corazón a una reconciliación auténtica y duradera entre los pueblos.

4. Dirigimos hoy nuestra mirada al Niño, a quien María estrecha entre sus brazos. En él reconocemos a Aquel en quien la misericordia y la verdad se encuentran, la justicia y la paz se besan (cf. \emph{Sal} 84, 11). En él adoramos al Mesías verdadero, en quien Dios ha conjugado, para nuestra salvación, la verdad y la misericordia, la justicia y el perdón.

En nombre de Dios renuevo mi llamamiento apremiante a todos, creyentes y no creyentes, para que el binomio \textquote{justicia y perdón} caracterice siempre las relaciones entre las personas, entre los grupos sociales y entre los pueblos.

Este llamamiento se dirige, ante todo, a \emph{cuantos creen en Dios}, en particular a las tres grandes religiones que descienden de Abraham, \emph{judaísmo, cristianismo} e \emph{islam}, llamadas a \emph{rechazar siempre con firmeza y decisión la violencia. Nadie, por ningún motivo, puede matar en nombre de Dios, único y misericordioso}. Dios es vida y fuente de la vida. Creer en él significa testimoniar su misericordia y su perdón, evitando instrumentalizar su santo nombre.

Desde diversas partes del mundo se eleva una ferviente invocación de paz; se eleva particularmente de la \emph{Tierra} que Dios bendijo con su Alianza y su Encarnación, y que por eso llamamos \emph{Santa}. \textquote{La voz de la sangre} clama a Dios desde aquella tierra (cf. \emph{Gn} 4, 10); sangre de hermanos derramada por hermanos, que se remontan al mismo patriarca Abraham; hijos, como todos los hombres, del mismo Padre celestial.

5. \emph{¡Salve, Madre santa!} Virgen hija de Sión, ¡cuánto debe sufrir por esta sangre tu corazón de Madre!

El Niño que estrechas contra tu pecho lleva un nombre apreciado por los pueblos de religión bíblica: \emph{Jesús}, que significa \textquote{Dios salva}. Así lo llamó el arcángel antes de que fuera concebido en tu seno (cf. \emph{Lc} 2, 21). En el rostro del Mesías recién nacido reconocemos el rostro de todos tus hijos vilipendiados y explotados. Reconocemos especialmente el rostro de los niños, cualquiera que sea su raza, nación y cultura. Por ellos, oh María, por su futuro, te pedimos que ablandes los corazones endurecidos por el odio, para que se abran al amor, y la venganza ceda finalmente el paso al perdón.

Obtennos, oh Madre, que la verdad de esta afirmación -\textquote{No hay paz sin justicia, no hay justicia sin perdón}- se grabe en el corazón de todos. Así la familia humana podrá encontrar la paz verdadera, que brota del encuentro entre la justicia y la misericordia.

Madre santa, Madre del Príncipe de la paz, ¡ayúdanos!

Madre de la humanidad y Reina de la paz, ¡ruega por nosotros!

\subsubsection{Homilía(2005): Vencer al mal con el bien}

Sábado 1 de enero de 2005.

1. \textquote{¡Salve, Madre santa!, Virgen Madre del Rey, que gobierna cielo y tierra por los siglos de los siglos} (Antífona de entrada).

En el primer día del año, la Iglesia se reúne en oración ante el icono de la Madre de Dios, y honra con alegría a aquella que \emph{dio al mundo el fruto de su vientre, Jesús, el \textquote{Príncipe de la paz}} (\emph{Is} 9, 5).

2. Ya es tradición consolidada celebrar en este mismo día la \emph{Jornada mundial de la paz}. En esta ocasión, me alegra expresar mi más cordial felicitación a los ilustres embajadores del Cuerpo diplomático ante la Santa Sede. Dirijo un saludo especial a los embajadores de los países particularmente afectados durante estos días por el enorme cataclismo que se abatió sobre ellos\ldots{}

3. La Jornada mundial de la paz constituye una invitación a los cristianos y a todos los hombres de buena voluntad a renovar su firme compromiso de \emph{construir la paz}. Esto supone la acogida de una exigencia moral fundamental, expresada muy bien en las palabras de san Pablo: \textquote{No te dejes vencer por el mal; antes bien, vence al mal con el bien} (\emph{Rm} 12, 21).

Ante las numerosas manifestaciones del mal, que por desgracia hieren a la familia humana, la exigencia prioritaria es \emph{promover la paz utilizando medios coherentes}, dando importancia al diálogo, a las obras de justicia, y educando para el perdón (cf. \emph{Mensaje para la Jornada mundial de la paz de 2005}, n. 1).

4. \emph{Vencer el mal con las armas del amor} es el modo como \emph{cada uno puede contribuir a la paz de todos}. A lo largo de esta senda están llamados a caminar tanto los cristianos como los creyentes de las diversas religiones, juntamente con cuantos se reconocen en la \emph{ley moral universal}.

Amadísimos hermanos y hermanas, promover la paz en la tierra es \emph{nuestra misión común}.

Que la Virgen María nos ayude a realizar las palabras del Señor: \textquote{Bienaventurados los que trabajan por la paz, porque ellos serán llamados hijos de Dios} (\emph{Mt} 5, 9).

¡Feliz año nuevo a todos! ¡Alabado sea Jesucristo!

\subsection{Benedicto XVI, papa}

\subsubsection{Homilía (2008): Nos ayuda a conocer a su Hijo}

Martes 1 de enero del 2008.

Hoy comenzamos un año nuevo y nos lleva de la mano la esperanza cristiana. Lo comenzamos invocando sobre él la bendición divina e implorando, por intercesión de María, Madre de Dios, el don de la paz para nuestras familias, para nuestras ciudades y para el mundo entero.

Con este deseo os saludo a todos vosotros, aquí presentes\ldots{}

La paz. En la \textbf{primera lectura}, tomada del libro de los Números, hemos escuchado la invocación: \textquote{El Señor te conceda la paz} (\emph{Nm} 6, 26). El Señor conceda la paz a cada uno de vosotros, a vuestras familias y al mundo entero. Todos aspiramos a vivir en paz, pero la paz verdadera, la que anunciaron los ángeles en la noche de Navidad, no es conquista del hombre o fruto de acuerdos políticos; es ante todo don divino, que es preciso implorar constantemente y, al mismo tiempo, compromiso que es necesario realizar con paciencia, siempre dóciles a los mandatos del Señor.

Este año, en el Mensaje para esta Jornada mundial de la paz puse de relieve la íntima relación que existe entre la familia y la construcción de la paz en el mundo. La familia natural, fundada en el matrimonio entre un hombre y una mujer, es \textquote{cuna de la vida y del amor} y \textquote{la primera e insustituible educadora de la paz}. Precisamente por eso la familia es \textquote{la principal \textquote{agencia} de paz} y \textquote{la negación o restricción de los derechos de la familia, al oscurecer la verdad sobre el hombre, \emph{amenaza los fundamentos mismos de la paz}} (cf. nn. 1-5). Dado que la humanidad es una \textquote{gran familia}, si quiere vivir en paz, no puede por menos de inspirarse en esos valores, sobre los cuales se funda y se apoya la comunidad familiar.

La providencial coincidencia de varias celebraciones nos impulsa este año a un esfuerzo aún mayor para realizar la paz en el mundo. Hace sesenta años, en 1948, la Asamblea general de las Naciones Unidas hizo pública la \textquote{Declaración universal de derechos humanos}. Hace cuarenta años, mi venerado predecesor Pablo VI celebró la primera Jornada mundial de la paz. Este año, además, recordaremos el 25° aniversario de la adopción por parte de la Santa Sede de la \textquote{Carta de los derechos de la familia}. \textquote{A la luz de estas significativas efemérides ---cito aquí lo que escribí precisamente al concluir el Mensaje---, invito a todos los hombres y mujeres a tomar una conciencia más clara de la pertenencia común a la única familia humana y a comprometerse para que la convivencia en la tierra refleje cada vez más esta convicción, de la cual depende la instauración de una paz verdadera y duradera} (\emph{L'Osservatore Romano}, edición en lengua española, 14 de diciembre de 2007, p. 5).

Nuestro pensamiento se dirige ahora, naturalmente, a la Virgen María, a la que hoy invocamos como Madre de Dios. Fue el Papa Pablo VI quien trasladó al día 1 de enero la fiesta de la Maternidad divina de María, que antes caía el 11 de octubre. En efecto, antes de la reforma litúrgica realizada después del concilio Vaticano II, en el primer día del año se celebraba la memoria de la circuncisión de Jesús en el octavo día después de su nacimiento ---como signo de sumisión a la ley, su inserción oficial en el pueblo elegido--- y el domingo siguiente se celebraba la fiesta del nombre de Jesús.

De esas celebraciones encontramos algunas huellas en la página evangélica que acabamos de proclamar, en la que \textbf{san Lucas} refiere que, ocho días después de su nacimiento, el Niño fue circuncidado y le pusieron el nombre de Jesús, \textquote{el que le dio el ángel antes de ser concebido en el seno de su madre} (\emph{Lc} 2, 21). Por tanto, esta solemnidad, además de ser una fiesta mariana muy significativa, conserva también un fuerte contenido cristológico, porque, podríamos decir, antes que a la Madre, atañe precisamente al Hijo, a Jesús, verdadero Dios y verdadero hombre.

Al misterio de la maternidad divina de María, la \emph{Theotokos}, hace referencia el apóstol san Pablo en la \textbf{carta a los Gálatas}. \textquote{Al llegar la plenitud de los tiempos ---escribe--- envió Dios a su Hijo, nacido de mujer, nacido bajo la ley} (\emph{Ga} 4, 4). En pocas palabras se encuentran sintetizados el misterio de la encarnación del Verbo eterno y la maternidad divina de María: el gran privilegio de la Virgen consiste precisamente en ser Madre del Hijo, que es Dios.

Así pues, \textbf{ocho días después de la Navidad}, esta fiesta mariana encuentra su lugar más lógico y adecuado. En efecto, en la noche de Belén, cuando \textquote{dio a luz a su hijo primogénito} (\emph{Lc} 2, 7), se cumplieron las profecías relativas al Mesías. \textquote{Una virgen concebirá y dará a luz un hijo}, había anunciado Isaías (\emph{Is} 7, 14). \textquote{Concebirás en tu seno y darás a luz un hijo} (\emph{Lc} 1, 31), dijo a María el ángel Gabriel. Y también un ángel del Señor ---narra el evangelista san Mateo---, apareciéndose en sueños a José, lo tranquilizó diciéndole: \textquote{No temas tomar contigo a María tu mujer, porque lo engendrado en ella es del Espíritu Santo. Dará a luz un hijo} (\emph{Mt} 1, 20-21).

El título de Madre de Dios es, juntamente con el de Virgen santa, el más antiguo y constituye el fundamento de todos los demás títulos con los que María ha sido venerada y sigue siendo invocada de generación en generación, tanto en Oriente como en Occidente. Al misterio de su maternidad divina hacen referencia muchos himnos y numerosas oraciones de la tradición cristiana, como por ejemplo una antífona mariana del tiempo navideño, el \emph{Alma Redemptoris Mater}, con la que oramos así: \textquote{\emph{Tu quae genuisti, natura mirante, tuum sanctum Genitorem, Virgo prius ac posterius}, \textquote{Tú, ante el asombro de toda la creación, engendraste a tu Creador, Madre siempre virgen}.

Queridos hermanos y hermanas, contemplemos hoy a María, Madre siempre virgen del Hijo unigénito del Padre. Aprendamos de ella a acoger al Niño que por nosotros nació en Belén. Si en el Niño nacido de ella reconocemos al Hijo eterno de Dios y lo acogemos como nuestro único Salvador, podemos ser llamados, y seremos realmente, hijos de Dios: hijos en el Hijo. El Apóstol escribe: \textquote{Envió Dios a su Hijo, nacido de mujer, nacido bajo la ley, para rescatar a los que se hallaban bajo la ley, y para que recibiéramos la filiación adoptiva} (\emph{Ga} 4, 4-5).

El \textbf{evangelista san Lucas} repite varias veces que la Virgen meditaba silenciosamente esos acontecimientos extraordinarios en los que Dios la había implicado. Lo hemos escuchado también en el breve pasaje evangélico que la liturgia nos vuelve a proponer hoy. \textquote{María conservaba todas estas cosas meditándolas en su corazón} (\emph{Lc} 2, 19). El verbo griego usado, \emph{sumbállousa}, en su sentido literal significa \textquote{poner juntamente}, y hace pensar en un gran misterio que es preciso descubrir poco a poco.

El Niño que emite vagidos en el pesebre, aun siendo en apariencia semejante a todos los niños del mundo, al mismo tiempo es totalmente diferente: es el Hijo de Dios, es Dios, verdadero Dios y verdadero hombre. Este misterio ---la encarnación del Verbo y la maternidad divina de María--- es grande y ciertamente no es fácil de comprender con la sola inteligencia humana.

Sin embargo, en la escuela de María podemos captar con el corazón lo que los ojos y la mente por sí solos no logran percibir ni pueden contener. En efecto, se trata de un don tan grande que sólo con la fe podemos acoger, aun sin comprenderlo todo. Y es precisamente en este camino de fe donde María nos sale al encuentro, nos ayuda y nos guía. Ella es madre porque engendró en la carne a Jesús; y lo es porque se adhirió totalmente a la voluntad del Padre. San Agustín escribe: \textquote{Ningún valor hubiera tenido para ella la misma maternidad divina, si no hubiera llevado a Cristo en su corazón, con una suerte mayor que cuando lo concibió en la carne} (\emph{De sancta Virginitate} 3, 3). Y en su corazón María siguió conservando, \textquote{poniendo juntamente}, los acontecimientos sucesivos de los que fue testigo y protagonista, hasta la muerte en la cruz y la resurrección de su Hijo Jesús.

Queridos hermanos y hermanas, sólo conservando en el corazón, es decir, poniendo juntamente y encontrando una unidad de todo lo que vivimos, podemos entrar, siguiendo a María, en el misterio de un Dios que por amor se hizo hombre y nos llama a seguirlo por la senda del amor, un amor que es preciso traducir cada día en un servicio generoso a los hermanos.

Ojalá que el nuevo año, que hoy comenzamos con confianza, sea un tiempo en el que progresemos en ese conocimiento del corazón, que es la sabiduría de los santos. Oremos para que, como hemos escuchado en la \textbf{primera lectura}, el Señor \textquote{ilumine su rostro sobre nosotros} y nos \textquote{sea propicio} (cf. \emph{Nm} 6, 25) y nos bendiga.

Podemos estar seguros de que, si buscamos sin descanso su rostro, si no cedemos a la tentación del desaliento y de la duda, si incluso en medio de las numerosas dificultades que encontramos permanecemos siempre anclados en él, experimentaremos la fuerza de su amor y de su misericordia. El frágil Niño que la Virgen muestra hoy al mundo nos haga agentes de paz, testigos de él, Príncipe de la paz. Amén.

\subsubsection{Homilía (2011): Recibió el don de Dios}

Basílica Vaticana. Sábado 1 de enero del 2011.

Todavía inmersos en el clima espiritual de la Navidad, en la que hemos contemplado el misterio del nacimiento de Cristo, con los mismos sentimientos celebramos hoy a la Virgen María, a quien la Iglesia venera como Madre de Dios, porque dio carne al Hijo del Padre eterno. Las lecturas bíblicas de esta solemnidad ponen el acento principalmente en el Hijo de Dios hecho hombre y en el \textquote{nombre} del Señor. La \textbf{primera lectura} nos presenta la solemne bendición que pronunciaban los sacerdotes sobre los israelitas en las grandes fiestas religiosas: está marcada precisamente por el nombre del Señor, que se repite tres veces, como para expresar la plenitud y la fuerza que deriva de esa invocación. En efecto, este texto de bendición litúrgica evoca la riqueza de gracia y de paz que Dios da al hombre, con una disposición benévola respecto a este, y que se manifiesta con el \textquote{resplandecer} del rostro divino y el \textquote{dirigirlo} hacia nosotros.

La Iglesia vuelve a escuchar hoy estas palabras, mientras pide al Señor que bendiga el nuevo año que acaba de comenzar, con la conciencia de que, ante los trágicos acontecimientos que marcan la historia, ante las lógicas de guerra que lamentablemente todavía no se han superado totalmente, sólo Dios puede tocar profundamente el alma humana y asegurar esperanza y paz a la humanidad. De hecho, ya es una tradición consolidada que en el primer día del año la Iglesia, presente en todo el mundo, eleve una oración coral para invocar la paz. Es bueno iniciar un emprendiendo decididamente la senda de la paz. Hoy, queremos recoger el grito de tantos hombres, mujeres, niños y ancianos víctimas de la guerra, que es el rostro más horrendo y violento de la historia. Hoy rezamos a fin de que la paz, que los ángeles anunciaron a los pastores la noche de Navidad, llegue a todos los rincones del mundo: \textquote{\emph{Super terram pax in hominibus bonae voluntatis}} (\emph{Lc} 2, 14). Por esto, especialmente con nuestra oración, queremos ayudar a todo hombre y a todo pueblo, en particular a cuantos tienen responsabilidades de gobierno, a avanzar de modo cada vez más decidido por el camino de la paz.

En la \textbf{segunda lectura}, san Pablo resume en la adopción filial la obra de salvación realizada por Cristo, en la cual está como engarzada la figura de María. Gracias a ella el Hijo de Dios, \textquote{nacido de mujer} (\emph{Ga} 4, 4), pudo venir al mundo como verdadero hombre, en la plenitud de los tiempos. Ese cumplimiento, esa plenitud, atañe al pasado y a las esperas mesiánicas, que se realizan, pero, al mismo tiempo, también se refiere a la plenitud en sentido absoluto: en el Verbo hecho carne Dios dijo su Palabra última y definitiva. En el umbral de un año nuevo, resuena así la invitación a caminar con alegría hacia la luz del \textquote{sol que nace de lo alto} (\emph{Lc}1, 78), puesto que en la perspectiva cristiana todo el tiempo está habitado por Dios, no hay futuro que no sea en la dirección de Cristo y no existe plenitud fuera de la de Cristo.

El pasaje del \textbf{Evangelio} de hoy termina con la imposición del nombre de Jesús, mientras María participa en silencio, meditando en su corazón sobre el misterio de su Hijo, que de modo completamente singular es don de Dios. Pero el pasaje evangélico que hemos escuchado hace hincapié especialmente en los pastores, que se volvieron \textquote{glorificando y alabando a Dios por todo lo que habían oído y visto} (\emph{Lc} 2, 20). El ángel les había anunciado que en la ciudad de David, es decir, en Belén había nacido el Salvador y que iban a encontrar \emph{la} \emph{señal}: un niño envuelto en pañales y acostado en un pesebre (cf. \emph{Lc} 2, 11-12). Fueron a toda prisa, y encontraron a María y a José, y al Niño. Notemos que el Evangelista habla de la maternidad de María a partir del Hijo, de ese \textquote{niño envuelto en pañales}, porque es él ---el Verbo de Dios (\emph{Jn} 1, 14)--- el punto de referencia, el centro del acontecimiento que está teniendo lugar, y es él quien hace que la maternidad de María se califique como \textquote{divina}.

Esta atención predominante que las lecturas de hoy dedican al \textquote{Hijo}, a Jesús, no reduce el papel de la Madre; más aún, la sitúa en la perspectiva correcta: en efecto, María es verdadera Madre de Dios precisamente en virtud de su relación total con Cristo. Por tanto, glorificando al Hijo se honra a la Madre y honrando a la Madre se glorifica al Hijo. El título de \textquote{Madre de Dios}, que hoy la liturgia pone de relieve, subraya la misión única de la Virgen santísima en la historia de la salvación: misión que está en la base del culto y de la devoción que el pueblo cristiano le profesa. En efecto, María no recibió el don de Dios sólo para ella, sino para llevarlo al mundo: en su virginidad fecunda, Dios dio a los hombres los bienes de la salvación eterna (cf. \emph{Oración Colecta}). Y María ofrece continuamente su mediación al pueblo de Dios peregrino en la historia hacia la eternidad, como en otro tiempo la ofreció a los pastores de Belén. Ella, que dio la vida terrena al Hijo de Dios, sigue dando a los hombres la vida divina, que es Jesús mismo y su Santo Espíritu. Por esto es considerada madre de todo hombre que nace a la Gracia y a la vez se la invoca como Madre de la Iglesia.

En el nombre de María, Madre de Dios y de los hombres, desde el 1 de enero de 1968 se celebra en todo el mundo la Jornada mundial de la paz. La paz es don de Dios, como hemos escuchado en la \textbf{primera lectura}: \textquote{Que el Señor (\ldots{}) te conceda la paz} (\emph{Nm} 6, 26). Es el don mesiánico por excelencia, el primer fruto de la caridad que Jesús nos ha dado; es nuestra reconciliación y pacificación con Dios. La paz también es un valor humano que se ha de realizar en el ámbito social y político, pero hunde sus raíces en el misterio de Cristo (cf. \emph{Gaudium et spes}, 77-90). En esta celebración solemne, con ocasión de la 44ª Jornada mundial de la paz, me alegra dirigir mi deferente saludo a los ilustres embajadores ante la Santa Sede, con mis mejores deseos para su misión. Asimismo, dirijo un saludo cordial y fraterno a mi secretario de Estado y a los demás responsables de los dicasterios de la Curia romana, con un pensamiento particular para el presidente del Consejo pontificio \textquote{Justicia y paz} y sus colaboradores. Deseo manifestarles mi vivo reconocimiento por su compromiso diario en favor de una convivencia pacífica entre los pueblos y de la formación cada vez más sólida de una conciencia de paz en la Iglesia y en el mundo. Desde esta perspectiva, la comunidad eclesial está cada vez más comprometida a actuar, según las indicaciones del Magisterio, para ofrecer un patrimonio espiritual seguro de valores y de principios, en la búsqueda continua de la paz.

En mi \emph{Mensaje} para la Jornada de hoy, que lleva por título \textquote{Libertad religiosa, camino para la paz} he querido recordar que: \textquote{El mundo tiene necesidad de Dios. Tiene necesidad de valores éticos y espirituales, universales y compartidos, y la religión puede contribuir de manera preciosa a su búsqueda, para la construcción de un orden social e internacional justo y pacífico} (n. 15). Por tanto, he subrayado que \textquote{la libertad religiosa (\ldots{}) es un elemento imprescindible de un Estado de derecho; no se puede negar sin dañar al mismo tiempo los demás derechos y libertades fundamentales, pues es su síntesis y su cumbre} (n. 5).

La humanidad no puede mostrarse resignada a la fuerza negativa del egoísmo y de la violencia; no debe acostumbrarse a conflictos que provoquen víctimas y pongan en peligro el futuro de los pueblos. Frente a las amenazadoras tensiones del momento, especialmente frente a las discriminaciones, los abusos y las intolerancias religiosas, que hoy golpean de modo particular a los cristianos (cf. \emph{ib}., 1), dirijo una vez más una apremiante invitación a no ceder al desaliento y a la resignación. Os exhorto a todos a rezar a fin de que lleguen a buen fin los esfuerzos emprendidos desde diversas partes para promover y construir la paz en el mundo. Para esta difícil tarea no bastan las palabras; es preciso el compromiso concreto y constante de los responsables de las naciones, pero sobre todo es necesario que todas las personas actúen animadas por el auténtico espíritu de paz, que siempre hay que implorar de nuevo en la oración y vivir en las relaciones cotidianas, en cada ambiente.

En esta celebración eucarística tenemos delante de nuestros ojos, para nuestra veneración, la imagen de la Virgen del \textquote{Sacro Monte di Viggiano}, tan querida para los habitantes de Basilicata. La Virgen María nos da a su Hijo, nos muestra el rostro de su Hijo, Príncipe de la paz: que ella nos ayude a permanecer en la luz de este rostro, que brilla sobre nosotros (cf. \emph{Nm} 6, 25), para redescubrir toda la ternura de Dios Padre; que ella nos sostenga al invocar al Espíritu Santo, para que renueve la faz de la tierra y transforme los corazones, ablandando su dureza ante la bondad desarmante del Niño, que ha nacido por nosotros. Que la Madre de Dios nos acompañe en este nuevo año; que obtenga para nosotros y para todo el mundo el deseado don de la paz. Amén.


\subsection{Francisco, papa}





\subsubsection{Homilía (2014): Bendición cumplida en María}

Basílica Vaticana. Miércoles 1 de enero de 2014.

La \textbf{primera lectura} que hemos escuchado nos propone una vez más las antiguas palabras de bendición que Dios sugirió a Moisés para que las enseñara a Aarón y a sus hijos: \textquote{Que el Señor te bendiga y te proteja. Que el Señor haga brillar su rostro sobre ti y te muestre su gracia. Que el Señor te descubra su rostro y te conceda la paz} (\emph{Nm} 6,24-25). Es muy significativo escuchar de nuevo esta bendición precisamente al comienzo del nuevo año: ella acompañará nuestro camino durante el tiempo que ahora nos espera. Son palabras de fuerza, de valor, de esperanza. No de una esperanza ilusoria, basada en frágiles promesas humanas; ni tampoco de una esperanza ingenua, que imagina un futuro mejor sólo porque es futuro. Esta esperanza tiene su razón de ser precisamente en la bendición de Dios, una bendición que contiene el mejor de los deseos, el deseo de la Iglesia para todos nosotros, impregnado de la protección amorosa del Señor, de su ayuda providente.

El deseo contenido en esta bendición se ha realizado plenamente en una mujer, María, por haber sido destinada a ser la Madre de Dios, y se ha cumplido en ella antes que en ninguna otra criatura.

Madre de Dios. Este es el título principal y esencial de la Virgen María. Es una cualidad, un cometido, que la fe del pueblo cristiano siempre ha experimentado, en su tierna y genuina devoción por nuestra madre celestial.

Recordemos aquel gran momento de la historia de la Iglesia antigua, el Concilio de Éfeso, en el que fue definida con autoridad la divina maternidad de la Virgen. La verdad sobre la divina maternidad de María encontró eco en Roma, donde poco después se construyó la Basílica de Santa María \textquote{la Mayor}, primer santuario mariano de Roma y de todo occidente, y en el cual se venera la imagen de la Madre de Dios ---la \emph{Theotokos}--- con el título de \emph{Salus populi romani}. Se dice que, durante el Concilio, los habitantes de Éfeso se congregaban a ambos lados de la puerta de la basílica donde se reunían los Obispos, gritando: \textquote{¡Madre de Dios!}. Los fieles, al pedir que se definiera oficialmente este título mariano, demostraban reconocer ya la divina maternidad. Es la actitud espontánea y sincera de los hijos, que conocen bien a su madre, porque la aman con inmensa ternura. Pero es algo más: es el \emph{sensus fidei} del santo pueblo fiel de Dios, que nunca, en su unidad, nunca se equivoca.

María está desde siempre presente en el corazón, en la devoción y, sobre todo, en el camino de fe del pueblo cristiano. \textquote{La Iglesia\ldots{} camina en el tiempo\ldots{} Pero en este camino ---deseo destacarlo enseguida--- procede recorriendo de nuevo el itinerario realizado por la Virgen María} (Juan Pablo II, Enc. \emph{Redemptoris Mater}, 2). Nuestro itinerario de fe es igual al de María, y por eso la sentimos particularmente cercana a nosotros. Por lo que respecta a la fe, que es el quicio de la vida cristiana, la Madre de Dios ha compartido nuestra condición, ha debido caminar por los mismos caminos que recorremos nosotros, a veces difíciles y oscuros, ha debido avanzar en \textquote{la peregrinación de la fe} (Conc. Ecum. Vat. II, Const. \emph{Lumen gentium}, 58).

Nuestro camino de fe está unido de manera indisoluble a María desde el momento en que Jesús, muriendo en la cruz, nos la ha dado como Madre diciendo: \textquote{He ahí a tu madre} (\emph{Jn} 19,27). Estas palabras tienen un valor de testamento y dan al mundo una Madre. Desde ese momento, la Madre de Dios se ha convertido también en nuestra Madre. En aquella hora en la que la fe de los discípulos se agrietaba por tantas dificultades e incertidumbres, Jesús les confió a aquella que fue la primera en creer, y cuya fe no decaería jamás. Y la \textquote{mujer} se convierte en nuestra Madre en el momento en el que pierde al Hijo divino. Y su corazón herido se ensancha para acoger a todos los hombres, buenos y malos, a todos, y los ama como los amaba Jesús. La mujer que en las bodas de Caná de Galilea había cooperado con su fe a la manifestación de las maravillas de Dios en el mundo, en el Calvario mantiene encendida la llama de la fe en la resurrección de su Hijo, y la comunica con afecto materno a los demás. María se convierte así en fuente de esperanza y de verdadera alegría.

La Madre del Redentor nos precede y continuamente nos confirma en la fe, en la vocación y en la misión. Con su ejemplo de humildad y de disponibilidad a la voluntad de Dios nos ayuda a traducir nuestra fe en un anuncio del Evangelio alegre y sin fronteras. De este modo nuestra misión será fecunda, porque está modelada sobre la maternidad de María. A ella confiamos nuestro itinerario de fe, los deseos de nuestro corazón, nuestras necesidades, las del mundo entero, especialmente el hambre y la sed de justicia y de paz y de Dios; y la invocamos todos juntos, y os invito a invocarla tres veces, imitando a aquellos hermanos de Éfeso, diciéndole: ¡Madre de Dios! ¡Madre de Dios! ¡Madre de Dios! ¡Madre de Dios! Amén.

\subsubsection{Ángelus (2014)} \emph{Queridos hermanos y hermanas, ¡buenos día y feliz año!}

Al inicio del nuevo año dirijo a todos vosotros los más cordiales deseos de paz y de todo bien. Mi deseo es el de la Iglesia, el deseo cristiano. No está relacionado con el sentido un poco mágico y un poco fatalista de un nuevo ciclo que inicia. Sabemos que la historia tiene un centro: Jesucristo, encarnado, muerto y resucitado, que vive entre nosotros; tiene un fin: el Reino de Dios, Reino de paz, de justicia, de libertad en el amor; y tiene una fuerza que la mueve hacia ese fin: la fuerza es el Espíritu Santo. Todos nosotros tenemos el Espíritu Santo que hemos recibido en el Bautismo, y Él nos impulsa a seguir adelante por el camino de la vida cristiana, por la senda de la historia, hacia el Reino de Dios.

Este Espíritu es la potencia de amor que fecundó el seno de la Virgen María; y es el mismo que anima los proyectos y las obras de todos los constructores de paz. Donde hay un hombre o una mujer constructor de paz, es precisamente el Espíritu Santo quien le ayuda, le impulsa a construir la paz. Dos caminos que se cruzan hoy: fiesta de María santísima Madre de Dios y Jornada mundial de la paz. Hace ocho días resonaba el anuncio angelical: \textquote{Gloria a Dios y paz a los hombres}; hoy lo acogemos nuevamente de la Madre de Jesús, que \textquote{conservaba todas estas cosas, meditándolas en su corazón} (\emph{Lc} 2, 19), para hacer de ello nuestro compromiso a lo largo del año que comienza.

El tema de esta Jornada mundial de la paz es \textquote{\emph{La fraternidad, fundamento y camino para la paz}}}. Fraternidad: siguiendo la estela de mis Predecesores, a partir de Pablo VI, he desarrollado el tema en un Mensaje, ya difundido y hoy idealmente entrego a todos. En la base está la convicción de que todos somos hijos del único Padre celestial, formamos parte de la misma familia humana y compartimos un destino común. De aquí se deriva para cada uno la responsabilidad de obrar a fin de que el mundo llegue a ser una comunidad de hermanos que se respetan, se aceptan en su diversidad y se cuidan unos a otros. Estamos llamados también a darnos cuenta de las violencias e injusticias presentes en tantas partes del mundo y que no pueden dejarnos indiferentes e inmóviles: se necesita del compromiso de todos para construir una sociedad verdaderamente más justa y solidaria. Ayer recibí una carta de un señor, tal vez uno de vosotros, quien informándome sobre una tragedia familiar, a continuación enumeraba muchas tragedias y guerras de hoy en el mundo, y me preguntaba: ¿qué sucede en el corazón del hombre, que le lleva a hacer todo esto? Y decía, al final: \textquote{Es hora de detenerse}. También yo creo que nos hará bien detenernos en este camino de violencia, y buscar la paz. Hermanos y hermanas, hago mías las palabras de este hombre: ¿qué sucede en el corazón del hombre? ¿Qué sucede en el corazón de la humanidad? ¡Es hora de detenerse!

Desde todos los rincones de la tierra, los creyentes elevan hoy la oración para pedir al Señor el don de la paz y la capacidad de llevarla a cada ambiente. En este primer día del año, que el Señor nos ayude a encaminarnos todos con más firmeza por las sendas de la justicia y de la paz. Y comencemos en casa. Justicia y paz en casa, entre nosotros. Se comienza en casa y luego se sigue adelante, a toda la humanidad. Pero debemos comenzar en casa. Que el Espíritu Santo actúe en nuestro corazón, rompa las cerrazones y las durezas y nos conceda enternecernos ante la debilidad del Niño Jesús. La paz, en efecto, requiere la fuerza de la mansedumbre, la fuerza no violenta de la verdad y del amor.

En las manos de María, Madre del Redentor, ponemos con confianza filial nuestras esperanzas. A ella, que extiende su maternidad a todos los hombres, confiamos el grito de paz de las poblaciones oprimidas por la guerra y la violencia, para que la valentía del diálogo y de la reconciliación predomine sobre las tentaciones de venganza, de prepotencia y corrupción. A ella le pedimos que el Evangelio de la fraternidad, anunciado y testimoniado por la Iglesia, pueda hablar a cada conciencia y derribar los muros que impiden a los enemigos reconocerse hermanos.

\subsubsection{Homilía (2017): No somos huérfanos}

Basílica Vaticana. 1 de enero del 2017.

\textquote{Mientras tanto, María conservaba estas cosas y las meditaba en su corazón} (\emph{Lc} 2, 19). Así \textbf{Lucas} describe la actitud con la que María recibe todo lo que estaban viviendo en esos días. Lejos de querer entender o adueñarse de la situación, María es la mujer que sabe conservar, es decir proteger, \emph{custodiar} en su corazón el paso de Dios en la vida de su Pueblo. Desde sus entrañas aprendió a escuchar el latir del corazón de su Hijo y eso le enseñó, a lo largo de toda su vida, a descubrir el palpitar de Dios en la historia. Aprendió a ser madre y, en ese aprendizaje, le regaló a Jesús la hermosa experiencia de saberse Hijo. En María, el Verbo Eterno no sólo se hizo carne sino que aprendió a reconocer la ternura maternal de Dios. Con María, el Niño-Dios aprendió a escuchar los anhelos, las angustias, los gozos y las esperanzas del Pueblo de la promesa. Con ella se descubrió a sí mismo Hijo del santo Pueblo fiel de Dios.

En los evangelios María aparece como mujer de pocas palabras, sin grandes discursos ni protagonismos pero con una mirada atenta que sabe custodiar la vida y la misión de su Hijo y, por tanto, de todo lo amado por Él. Ha sabido custodiar los albores de la primera comunidad cristiana, y así aprendió a ser madre de una multitud. Ella se ha acercado en las situaciones más diversas para sembrar esperanza. Acompañó las cruces cargadas en el silencio del corazón de sus hijos. Tantas devociones, tantos santuarios y capillas en los lugares más recónditos, tantas imágenes esparcidas por las casas, nos recuerdan esta gran verdad. María, nos dio el calor materno, ese que nos cobija en medio de la dificultad; el calor materno que permite que nada ni nadie apague en el seno de la Iglesia la revolución de la ternura inaugurada por su Hijo. Donde hay madre, hay ternura. Y María con su maternidad nos muestra que la humildad y la ternura no son virtudes de los débiles sino de los fuertes, nos enseña que no es necesario maltratar a otros para sentirse importantes (cf. Exhort. ap. \emph{Evangelii gaudium,} 288). Y desde siempre el santo Pueblo fiel de Dios la ha reconocido y saludado como la Santa Madre de Dios.

Celebrar la maternidad de María como Madre de Dios y madre nuestra, al comenzar un nuevo año, significa recordar una certeza que acompañará nuestros días: somos un pueblo con Madre, no somos huérfanos.

Las madres son el antídoto más fuerte ante nuestras tendencias individualistas y egoístas, ante nuestros encierros y apatías. Una sociedad sin madres no sería solamente una sociedad fría sino una sociedad que ha perdido el corazón, que ha perdido el \textquote{sabor a hogar}. Una sociedad sin madres sería una sociedad sin piedad que ha dejado lugar sólo al cálculo y a la especulación. Porque las madres, incluso en los peores momentos, saben dar testimonio de la ternura, de la entrega incondicional, de la fuerza de la esperanza. He aprendido mucho de esas madres que teniendo a sus hijos presos, o postrados en la cama de un hospital, o sometidos por la esclavitud de la droga, con frio o calor, lluvia o sequía, no se dan por vencidas y siguen peleando para darles a ellos lo mejor. O esas madres que en los campos de refugiados, o incluso en medio de la guerra, logran abrazar y sostener sin desfallecer el sufrimiento de sus hijos. Madres que dejan literalmente la vida para que ninguno de sus hijos se pierda. Donde está la madre hay unidad, hay pertenencia, pertenencia de hijos.

Comenzar el año haciendo memoria de la bondad de Dios en el rostro maternal de María, en el rostro maternal de la Iglesia, en los rostros de nuestras madres, nos protege de la corrosiva enfermedad de \textquote{la orfandad espiritual}, esa orfandad que vive el alma cuando se siente sin madre y le falta la ternura de Dios. Esa orfandad que vivimos cuando se nos va apagando el sentido de pertenencia a una familia, a un pueblo, a una tierra, a nuestro Dios. Esa orfandad que gana espacio en el corazón narcisista que sólo sabe mirarse a sí mismo y a los propios intereses y que crece cuando nos olvidamos que la vida ha sido un regalo ---que se la debemos a otros--- y que estamos invitados a compartirla en esta casa común.

Tal orfandad autorreferencial fue la que llevó a Caín a decir: \textquote{¿Acaso soy yo el guardián de mi hermano?} (\emph{Gn} 4,9), como afirmando: él no me pertenece, no lo reconozco. Tal actitud de orfandad espiritual es un cáncer que silenciosamente corroe y degrada el alma. Y así nos vamos degradando ya que, entonces, nadie nos pertenece y no pertenecemos a nadie: degrado la tierra, porque no me pertenece, degrado a los otros, porque no me pertenecen, degrado a Dios porque no le pertenezco, y finalmente termina degradándonos a nosotros mismos porque nos olvidamos quiénes somos, qué \textquote{apellido} divino tenemos. La pérdida de los lazos que nos unen, típica de nuestra cultura fragmentada y dividida, hace que crezca ese sentimiento de orfandad y, por tanto, de gran vacío y soledad. La falta de contacto físico (y no virtual) va cauterizando nuestros corazones (cf. Carta enc. \emph{Laudato si'}, 49) haciéndolos perder la capacidad de la ternura y del asombro, de la piedad y de la compasión. La orfandad espiritual nos hace perder la memoria de lo que significa ser hijos, ser nietos, ser padres, ser abuelos, ser amigos, ser creyentes. Nos hace perder la memoria del valor del juego, del canto, de la risa, del descanso, de la gratuidad.

Celebrar la fiesta de la Santa Madre de Dios nos vuelve a dibujar en el rostro la sonrisa de sentirnos pueblo, de sentir que nos pertenecemos; de saber que solamente dentro de una comunidad, de una familia, las personas podemos encontrar \textquote{el clima}, \textquote{el calor} que nos permita aprender a crecer humanamente y no como meros objetos invitados a \textquote{consumir y ser consumidos}. Celebrar la fiesta de la Santa Madre de Dios nos recuerda que no somos mercancía intercambiable o terminales receptoras de información. Somos hijos, somos familia, somos Pueblo de Dios.

Celebrar a la Santa Madre de Dios nos impulsa a generar y cuidar lugares comunes que nos den sentido de pertenencia, de arraigo, de hacernos sentir en casa dentro de nuestras ciudades, en comunidades que nos unan y nos ayudan (cf. Carta enc. \emph{Laudato si'}, 151 ).

Jesucristo en el momento de mayor entrega de su vida, en la cruz, no quiso guardarse nada para sí y entregando su vida nos entregó también a su Madre. Le dijo a María: aquí está tu Hijo, aquí están tus hijos. Y nosotros queremos recibirla en nuestras casas, en nuestras familias, en nuestras comunidades, en nuestros pueblos. Queremos encontrarnos con su mirada maternal. Esa mirada que nos libra de la orfandad; esa mirada que nos recuerda que somos hermanos: que yo te pertenezco, que tú me perteneces, que somos de la misma carne. Esa mirada que nos enseña que tenemos que aprender a cuidar la vida de la misma manera y con la misma ternura con la que ella la ha cuidado: sembrando esperanza, sembrando pertenencia, sembrando fraternidad.

Celebrar a la Santa Madre de Dios nos recuerda que tenemos Madre; no somos huérfanos, tenemos una Madre. Confesemos juntos esta verdad. Y los invito a aclamarla de pie (\emph{todos se alzan}) tres veces como lo hicieron los fieles de Éfeso: Santa Madre de Dios, Santa Madre de Dios, Santa Madre de Dios.

\subsubsection{Ángelus (2017)} \emph{Plaza de San Pedro\\ Domingo 1 de enero de 2017}



\emph{Queridos hermanos y hermanas, ¡buenos días!}

Durante los días pasados hemos puesto nuestra mirada adorante sobre el Hijo de Dios, nacido en Belén; hoy, Solemnidad de María Santísima Madre de Dios, dirigimos nuestros ojos a la Madre, pero recibiendo a ambos con su estrecho vínculo. Este vínculo no se agota en el hecho de haber generado y en haber sido generado; Jesús ha \textquote{nacido de mujer} (\emph{Gal} 4, 4) para una misión de salvación y su madre no está excluida de tal misión, es más, está asociada íntimamente. María es consciente de esto, por lo tanto no se cierra a considerar sólo su relación maternal con Jesús, sino que permanece abierta y primorosa en todos los acontecimientos que suceden a su alrededor: conserva y medita, observa y profundiza, como nos recuerda el Evangelio de hoy (cf \emph{Lc} 2, 19). Ha dicho ya su \textquote{sí} y ha dado su disponibilidad para ser incluida en la aplicación del plan de salvación de Dios, que \textquote{dispersó a los que son soberbios en su propio corazón. Derribó a los potentados de sus tronos y exaltó a los humildes. A los hambrientos colmó de bienes y despidió a los ricos sin nada} (\emph{Lc} 1, 51-53). Ahora, silenciosa y atenta, intenta comprender qué quiere Dios de ella día a día. La visita de los pastores le ofrece la ocasión para percibir algún elemento de la voluntad de Dios que se manifiesta en la presencia de estas personas humildes y pobres. El evangelista Lucas nos narra la visita de los pastores a la gruta con un rápido sucederse de verbos que expresan movimiento. Dice así: ellos van sin demora, encuentran al Niño con María y José, lo ven, y cuentan lo que les ha sido dicho por Él, y al final glorifican a Dios (cf \emph{Lc} 2, 16-20). María sigue atentamente esta escena, qué dicen los pastores, qué les ha ocurrido, por qué en ello ya se discierne el movimiento de salvación que surgirá de la obra de Jesús, y se adapta, preparada ante toda petición del Señor. Dios pide a María no sólo ser la madre de su Hijo unigénito, sino también cooperar con el Hijo y por el Hijo en su plan de salvación, para que en ella, humilde sierva, se cumplan las grandes obras de la misericordia divina.

Por ello, mientras, así como los pastores, contemplan el icono del Niño en brazos de su Madre, sentimos crecer en nuestro corazón un sentido de inmenso agradecimiento hacia quien ha dado al mundo al Salvador. Por ello, en el primer día de un año nuevo, le decimos:

Gracias, oh Santa Madre del Hijo de Dios, Jesús, ¡Santa Madre de Dios!\\ Gracias por tu humildad que ha atraído la mirada de Dios;\\ gracias por la fe con la cual has acogido su Palabra;\\ gracias por la valentía con la cual has dicho \textquote{aquí estoy},\\ olvidada de si misma, fascinada por el Amor Santo, convertida en una única cosa junto con su esperanza.\\ Gracias, ¡oh Santa Madre de Dios!\\ Reza por nosotros, peregrinos del tiempo; ayúdanos a caminar por la vía de la paz. Amén.

\subsubsection{Homilía (2020): La Iglesia es Madre, como María}

Basílica Vaticana. 1 de enero del 2020.

\textquote{Cuando llegó la plenitud del tiempo, envió Dios a su Hijo, nacido de mujer} (\emph{Ga} 4,4). \textbf{Nacido de mujer}: así es cómo vino Jesús. No apareció en el mundo como adulto, sino como nos ha dicho el \textbf{Evangelio}, fue \textquote{concebido} en el vientre (\emph{Lc} 2,21): allí hizo suya nuestra humanidad, día tras día, mes tras mes. En el vientre de una mujer, Dios y la humanidad se unieron para no separarse nunca más. También ahora, en el cielo, Jesús vive en la carne que tomó en el vientre de su madre. En Dios está nuestra carne humana.

El primer día del año celebramos estos desposorios entre Dios y el hombre, inaugurados en el vientre de una mujer. En Dios estará para siempre nuestra humanidad y María será la Madre de Dios para siempre. Ella es mujer y madre, esto es lo esencial. De ella, mujer, surgió la salvación y, por lo tanto, no hay salvación sin la mujer. Allí Dios se unió con nosotros y, si queremos unirnos con Él, debemos ir por el mismo camino: a través de María, mujer y madre. Por ello, comenzamos el año bajo el signo de Nuestra Señora, la mujer que tejió la humanidad de Dios. Si queremos tejer con humanidad las tramas de nuestro tiempo, debemos partir de nuevo de la mujer.

\emph{Nacido de mujer}. El renacer de la humanidad comenzó con la mujer. Las mujeres son fuente de vida. Sin embargo, son continuamente ofendidas, golpeadas, violadas, inducidas a prostituirse y a eliminar la vida que llevan en el vientre. Toda violencia infligida a la mujer es una profanación de Dios, nacido de una mujer. La salvación para la humanidad vino del cuerpo de una mujer: de cómo tratamos el cuerpo de la mujer comprendemos nuestro nivel de humanidad. Cuántas veces el cuerpo de la mujer se sacrifica en los altares profanos de la publicidad, del lucro, de la pornografía, explotado como un terreno para utilizar. Debe ser liberado del consumismo, debe ser respetado y honrado. Es la carne más noble del mundo, pues concibió y dio a luz al Amor que nos ha salvado. Hoy, la maternidad también es humillada, porque el único crecimiento que interesa es el económico. Hay madres que se arriesgan a emprender viajes penosos para tratar desesperadamente de dar un futuro mejor al fruto de sus entrañas, y que son consideradas como números que sobrexceden el cupo por personas que tienen el estómago lleno, pero de cosas, y el corazón vacío de amor.

\emph{Nacido de mujer}. Según la narración bíblica, la mujer aparece en el ápice de la creación, como resumen de todo lo creado. De hecho, ella contiene en sí el fin de la creación misma: la generación y protección de la vida, la comunión con todo, el ocuparse de todo. Es lo que hace la Virgen en el \textbf{Evangelio} hoy. \textquote{María, por su parte ―dice el texto―, conservaba todas estas cosas, meditándolas en su corazón} (v. 19). Conservaba todo: la alegría por el nacimiento de Jesús y la tristeza por la hospitalidad negada en Belén; el amor de José y el asombro de los pastores; las promesas y las incertidumbres del futuro. Todo lo tomaba en serio y todo lo ponía en su lugar en su corazón, incluso la adversidad. Porque en su corazón arreglaba cada cosa con amor y confiaba todo a Dios.

En el Evangelio encontramos por segunda vez esta acción de María: al final de la vida oculta de Jesús se dice, en efecto, que \textquote{su madre conservaba todo esto en su corazón} (v. 51). Esta repetición nos hace comprender que conservar en el corazón no es un buen gesto que la Virgen hizo de vez en cuando, sino un hábito. Es propio de la mujer tomarse la vida en serio. La mujer manifiesta que el significado de la vida no es continuar a producir cosas, sino tomar en serio las que ya están. Sólo quien mira con el corazón ve bien, porque sabe \textquote{ver en profundidad} a la persona más allá de sus errores, al hermano más allá de sus fragilidades, la esperanza en medio de las dificultades; ve a Dios en todo.

Al comenzar el nuevo año, preguntémonos: \textquote{¿Sé mirar con el corazón? ¿sé mirar con el corazón a las personas? ¿Me importa la gente con la que vivo, o la destruyo con la murmuración? Y, sobre todo, ¿tengo al Señor en el centro de mi corazón, o tengo otros valores, otros intereses, mi promoción, las riquezas, el poder?}. Sólo si la vida \emph{es importante} para nosotros sabremos \emph{cómo cuidarla} y superar la indiferencia que nos envuelve. Pidamos esta gracia: vivir el año con el deseo de tomar en serio a los demás, de cuidar a los demás. Y si queremos un mundo mejor, que sea una casa de paz y no un patio de batalla, que nos importe la dignidad de toda mujer. De una mujer nació el Príncipe de la paz. La mujer es donante y mediadora de paz y debe ser completamente involucrada en los procesos de toma de decisiones. Porque cuando las mujeres pueden transmitir sus dones, el mundo se encuentra más unido y más en paz. Por lo tanto, una conquista para la mujer es una conquista para toda la humanidad.

\emph{Nacido de mujer}. Jesús, recién nacido, se reflejó en los ojos de una mujer, en el rostro de su madre. De ella recibió las primeras caricias, con ella intercambió las primeras sonrisas. Con ella inauguró la revolución de la ternura. La Iglesia, mirando al niño Jesús, está llamada a continuarla. De hecho, al igual que María, también ella es mujer y madre, la Iglesia es mujer y madre, y en la Virgen encuentra sus rasgos distintivos. La ve inmaculada, y se siente llamada a decir \textquote{no} al pecado y a la mundanidad. La ve fecunda y se siente llamada a anunciar al Señor, a generarlo en las vidas. La ve, madre, y se siente llamada a acoger a cada hombre como a un hijo.

Acercándose a María, la Iglesia se encuentra a sí misma, encuentra su centro, encuentra su unidad. En cambio, el enemigo de la naturaleza humana, el diablo, trata de dividirla, poniendo en primer plano las diferencias, las ideologías, los pensamientos partidistas y los bandos. Pero no podemos entender a la Iglesia si la miramos a partir de sus estructuras, a partir de los programas y tendencias, de las ideologías, de las funcionalidades: percibiremos algo de ella, pero no el corazón de la Iglesia. Porque la Iglesia tiene el corazón de una madre. Y nosotros, hijos, invocamos hoy a la Madre de Dios, que nos reúne como pueblo creyente. Oh Madre, genera en nosotros la esperanza, tráenos la unidad. Mujer de la salvación, te confiamos este año, custódialo en tu corazón. Te aclamamos: ¡Santa Madre de Dios! Todos juntos, por tres veces, aclamemos a la Señora, en pie, Nuestra Señora, la Santa Madre de Dios: {[}con la asamblea{]}: ¡Santa Madre de Dios, Santa Madre de Dios!


\subsubsection{Ángelus (2020)} 

\emph{Plaza de San Pedro\\ Miércoles, 1 de enero de 2020}
LIII Jornada Mundial de la Paz.


\emph{Queridos hermanos y hermanas, ¡buenos días! ¡Y Feliz Año Nuevo!}

Anoche terminamos el año 2019 agradeciendo a Dios por el don del tiempo y por todos sus beneficios. Hoy comenzamos el año 2020 con la misma actitud de \emph{gratitud} y \emph{alabanza}. No se da por sentado que nuestro planeta ha comenzado una nueva vuelta alrededor del sol y que los seres humanos seguiremos viviendo en él. No se da por sentado, al contrario, siempre es un \textquote{milagro} del que sorprenderse y estar agradecido.

El primer día del año la liturgia celebra a la Santa Madre de Dios, María, la Virgen de Nazaret que dio a luz a Jesús, el Salvador. Ese Niño es la \emph{bendición de Dios} para cada hombre y mujer, para la gran familia humana y para el mundo entero. Jesús no eliminó el mal del mundo, sino que lo derrotó en su raíz. Su salvación no es mágica, sino que es una salvación \textquote{paciente}, es decir, implica la paciencia del amor, que se responsabiliza de la iniquidad y le quita su poder. La paciencia del amor: el amor nos hace pacientes. Muchas veces perdemos la paciencia; yo también, y pido disculpas por el mal ejemplo de ayer {[}se refiere a la reacción que tuvo con una persona que le tiró bruscamente del brazo en la plaza de San Pedro{]}. Por eso, contemplando el Pesebre vemos, con los ojos de la fe, el mundo renovado, liberado del dominio del mal y puesto bajo el señorío real de Cristo, el Niño acostado en el pesebre.

Por eso hoy la Madre de Dios nos bendice. ¿Y cómo nos bendice la Virgen? Mostrándonos al Hijo. Lo toma en sus brazos y nos lo muestra, y así nos bendice. Bendice a toda la Iglesia, bendice al mundo entero. Jesús, como cantaban los ángeles en Belén, es la \textquote{alegría de todo el pueblo}, es la gloria de Dios y la paz para la humanidad (cf. \emph{Lucas} 2, 14). Por eso el santo Papa Pablo VI quiso dedicar el primer día del año a la paz ―es la Jornada de la Paz―, a la oración, a la conciencia y a la responsabilidad por la paz. Para este año 2020 el Mensaje es así: la paz es un \emph{camino de esperanza}, un camino en el que se avanza a través del \emph{diálogo}, la \emph{reconciliación} y la \emph{conversión ecológica}.

Por lo tanto, fijemos la mirada en la Madre y en el Hijo que nos muestra. Al comienzo del año, ¡seamos bendecidos! Dejémonos bendecir por la Virgen con su Hijo.

Jesús es la bendición para aquellos que están oprimidos por el yugo de la esclavitud, la esclavitud moral y la esclavitud material. Él libera con amor. A los que han perdido la autoestima por permanecer prisioneros de círculos viciosos, Jesús les dice: el Padre os ama, no os abandona, espera con una paciencia inquebrantable vuestro regreso (cf. \emph{Lucas} 15, 20). A los que son víctimas de la injusticia y la explotación y no ven la salida, Jesús les abre la puerta de la fraternidad, donde pueden encontrar rostros, corazones y manos acogedores, donde pueden compartir la amargura y la desesperación, y recuperar algo de dignidad. A los que están gravemente enfermos y se sienten abandonados y desanimados, Jesús se acerca, toca con ternura las heridas, derrama el aceite del consuelo y transforma la debilidad en fuerza del bien para desatar los nudos más enredados. A los que están encarcelados y son tentados a encerrarse en sí mismos, Jesús les vuelve a abrir un horizonte de esperanza, empezando por un pequeño rayo de luz.

Queridos hermanos y hermanas, bajemos de los pedestales de nuestro orgullo ―todos tenemos la tentación del orgullo― y pidamos la bendición de la Santa Madre de Dios, la humilde Madre de Dios. Ella nos muestra a Jesús: seamos bendecidos, abramos nuestros corazones a su bondad. Así, el año que comienza será un camino de esperanza y paz, no con palabras, sino con gestos cotidianos de diálogo, reconciliación y cuidado de la creación.


\section{Temas}

Jesucristo, verdadero Dios y verdadero hombre

CEC 464-469:

\textbf{464} El acontecimiento único y totalmente singular de la Encarnación del Hijo de Dios no significa que Jesucristo sea en parte Dios y en parte hombre, ni que sea el resultado de una mezcla confusa entre lo divino y lo humano. Él se hizo verdaderamente hombre sin dejar de ser verdaderamente Dios. Jesucristo es verdadero Dios y verdadero hombre. La Iglesia debió defender y aclarar esta verdad de fe durante los primeros siglos frente a unas herejías que la falseaban.

\textbf{465} Las primeras herejías negaron menos la divinidad de Jesucristo que su humanidad verdadera (docetismo gnóstico). Desde la época apostólica la fe cristiana insistió en la verdadera encarnación del Hijo de Dios, \textquote{venido en la carne} (cf. \emph{1 Jn} 4, 2-3; \emph{2 Jn} 7). Pero desde el siglo III, la Iglesia tuvo que afirmar frente a Pablo de Samosata, en un Concilio reunido en Antioquía, que Jesucristo es Hijo de Dios por naturaleza y no por adopción. El primer Concilio Ecuménico de Nicea, en el año 325, confesó en su Credo que el Hijo de Dios es \textquote{engendrado, no creado, \textquote{de la misma substancia} {[}en griego \emph{homousion}{]} que el Padre} y condenó a Arrio que afirmaba que \textquote{el Hijo de Dios salió de la nada} (Concilio de Nicea I: DS 130) y que sería \textquote{de una substancia distinta de la del Padre} (\emph{Ibíd}., 126).

\textbf{466} La herejía nestoriana veía en Cristo una persona humana junto a la persona divina del Hijo de Dios. Frente a ella san Cirilo de Alejandría y el tercer Concilio Ecuménico reunido en Efeso, en el año 431, confesaron que \textquote{el Verbo, al unirse en su persona a una carne animada por un alma racional, se hizo hombre} (Concilio de Efeso: DS, 250). La humanidad de Cristo no tiene más sujeto que la persona divina del Hijo de Dios que la ha asumido y hecho suya desde su concepción. Por eso el concilio de Efeso proclamó en el año 431 que María llegó a ser con toda verdad Madre de Dios mediante la concepción humana del Hijo de Dios en su seno: \textquote{Madre de Dios, no porque el Verbo de Dios haya tomado de ella su naturaleza divina, sino porque es de ella, de quien tiene el cuerpo sagrado dotado de un alma racional [\ldots{}] unido a la persona del Verbo, de quien se dice que el Verbo nació según la carne} (DS 251).

\textbf{467} Los monofisitas afirmaban que la naturaleza humana había dejado de existir como tal en Cristo al ser asumida por su persona divina de Hijo de Dios. Enfrentado a esta herejía, el cuarto Concilio Ecuménico, en Calcedonia, confesó en el año 451:

\textquote{Siguiendo, pues, a los Santos Padres, enseñamos unánimemente que hay que confesar a un solo y mismo Hijo y Señor nuestro Jesucristo: perfecto en la divinidad, y perfecto en la humanidad; verdaderamente Dios y verdaderamente hombre compuesto de alma racional y cuerpo; consubstancial con el Padre según la divinidad, y consubstancial con nosotros según la humanidad, \textquote{en todo semejante a nosotros, excepto en el pecado} (\emph{Hb} 4, 15); nacido del Padre antes de todos los siglos según la divinidad; y por nosotros y por nuestra salvación, nacido en los últimos tiempos de la Virgen María, la Madre de Dios, según la humanidad.

Se ha de reconocer a un solo y mismo Cristo Señor, Hijo único en dos naturalezas, sin confusión, sin cambio, sin división, sin separación. La diferencia de naturalezas de ningún modo queda suprimida por su unión, sino que quedan a salvo las propiedades de cada una de las naturalezas y confluyen en un solo sujeto y en una sola persona} (Concilio de Calcedonia; DS, 301-302).

\textbf{468} Después del Concilio de Calcedonia, algunos concibieron la naturaleza humana de Cristo como una especie de sujeto personal. Contra éstos, el quinto Concilio Ecuménico, en Constantinopla, el año 553 confesó a propósito de Cristo: \textquote{No hay más que una sola hipóstasis {[}o persona{]} [\ldots{}] que es nuestro Señor Jesucristo, \emph{uno de la Trinidad}} (Concilio de Constantinopla II: DS, 424). Por tanto, todo en la humanidad de Jesucristo debe ser atribuido a su persona divina como a su propio sujeto (cf. ya Concilio de Éfeso: DS, 255), no solamente los milagros sino también los sufrimientos (cf. Concilio de Constantinopla II: DS, 424) y la misma muerte: \textquote{El que ha sido crucificado en la carne, nuestro Señor Jesucristo, es verdadero Dios, Señor de la gloria y uno de la Santísima Trinidad} (\emph{ibíd}., 432).

\textbf{469} La Iglesia confiesa así que Jesús es inseparablemente verdadero Dios y verdadero Hombre. Él es verdaderamente el Hijo de Dios que se ha hecho hombre, nuestro hermano, y eso sin dejar de ser Dios, nuestro Señor:

\emph{Id quod fuit remansit et quod non fuit assumpsit} (\textquote{Sin dejar de ser lo que era ha asumido lo que no era}), canta la liturgia romana (\emph{Solemnidad de la Santísima Virgen María, Madre de Dios}, Antífona al \textquote{Benedictus}; cf. san León Magno, \emph{Sermones} 21, 2-3: PL 54, 192). Y la liturgia de san Juan Crisóstomo proclama y canta: \textquote{¡Oh Hijo unigénito y Verbo de Dios! Tú que eres inmortal, te dignaste, para salvarnos, tomar carne de la santa Madre de Dios y siempre Virgen María. Tú, Cristo Dios, sin sufrir cambio te hiciste hombre y, en la cruz, con tu muerte venciste la muerte. Tú, Uno de la Santísima Trinidad, glorificado con el Padre y el Santo Espíritu, ¡sálvanos! (\emph{Oficio Bizantino de las Horas, Himno O' Monogenés}}).

María es la Madre de Dios

CEC 495, 2677:

\textbf{La maternidad divina de María}

\textbf{495} Llamada en los Evangelios \textquote{la Madre de Jesús} (\emph{Jn} 2, 1; 19, 25; cf. \emph{Mt} 13, 55, etc.), María es aclamada bajo el impulso del Espíritu como \textquote{la madre de mi Señor} desde antes del nacimiento de su hijo (cf. \emph{Lc} 1, 43). En efecto, aquél que ella concibió como hombre, por obra del Espíritu Santo, y que se ha hecho verdaderamente su Hijo según la carne, no es otro que el Hijo eterno del Padre, la segunda persona de la Santísima Trinidad. La Iglesia confiesa que María es verdaderamente \emph{Madre de Dios} {[}\emph{Theotokos}{]} (cf. Concilio de Éfeso, año 649: DS, 251).

\textbf{2677} \emph{\textquote{Santa María, Madre de Dios, ruega por nosotros\ldots{} }} Con Isabel, nos maravillamos y decimos: \textquote{¿De dónde a mí que la madre de mi Señor venga a mí?} (\emph{Lc} 1, 43). Porque nos da a Jesús su hijo, María es madre de Dios y madre nuestra; podemos confiarle todos nuestros cuidados y nuestras peticiones: ora por nosotros como oró por sí misma: \textquote{Hágase en mí según tu palabra} (\emph{Lc} 1, 38). Confiándonos a su oración, nos abandonamos con ella en la voluntad de Dios: \textquote{Hágase tu voluntad}.

\emph{\textquote{Ruega por nosotros, pecadores, ahora y en la hora de nuestra muerte}}. Pidiendo a María que ruegue por nosotros, nos reconocemos pecadores y nos dirigimos a la \textquote{Madre de la Misericordia}, a la Toda Santa. Nos ponemos en sus manos \textquote{ahora}, en el hoy de nuestras vidas. Y nuestra confianza se ensancha para entregarle desde ahora, \textquote{la hora de nuestra muerte}. Que esté presente en esa hora, como estuvo en la muerte en Cruz de su Hijo, y que en la hora de nuestro tránsito nos acoja como madre nuestra (cf. \emph{Jn} 19, 27) para conducirnos a su Hijo Jesús, al Paraíso.

nuestra adopción como hijos de Dios

CEC 1, 52, 270, 294, 422, 654, 1709, 2009:

\textbf{1} Dios, infinitamente perfecto y bienaventurado en sí mismo, en un designio de pura bondad ha creado libremente al hombre para hacerle partícipe de su vida bienaventurada. Por eso, en todo tiempo y en todo lugar, se hace cercano del hombre: le llama y le ayuda a buscarle, a conocerle y a amarle con todas sus fuerzas. Convoca a todos los hombres, que el pecado dispersó, a la unidad de su familia, la Iglesia. Para lograrlo, llegada la plenitud de los tiempos, envió a su Hijo como Redentor y Salvador. En Él y por Él, llama a los hombres a ser, en el Espíritu Santo, sus hijos de adopción, y por tanto los herederos de su vida bienaventurada.

\textbf{52} Dios, que \textquote{habita una luz inaccesible} (\emph{1 Tm} 6,16) quiere comunicar su propia vida divina a los hombres libremente creados por él, para hacer de ellos, en su Hijo único, hijos adoptivos (cf. \emph{Ef} 1,4-5). Al revelarse a sí mismo, Dios quiere hacer a los hombres capaces de responderle, de conocerle y de amarle más allá de lo que ellos serían capaces por sus propias fuerzas.

\textbf{\textquote{Te compadeces de todos porque lo puedes todo} (\emph{Sb} 11, 23)}

\textbf{270} Dios es el \emph{Padre} todopoderoso. Su paternidad y su poder se esclarecen mutuamente. Muestra, en efecto, su omnipotencia paternal por la manera como cuida de nuestras necesidades (cf. \emph{Mt} 6,32); por la adopción filial que nos da (\textquote{Yo seré para vosotros padre, y vosotros seréis para mí hijos e hijas, dice el Señor todopoderoso}: \emph{2 Co} 6,18); finalmente, por su misericordia infinita, pues muestra su poder en el más alto grado perdonando libremente los pecados.

\textbf{294} La gloria de Dios consiste en que se realice esta manifestación y esta comunicación de su bondad para las cuales el mundo ha sido creado. Hacer de nosotros \textquote{hijos adoptivos por medio de Jesucristo, según el beneplácito de su voluntad, \emph{para alabanza de la gloria} de su gracia} (\emph{Ef} 1,5-6): \textquote{Porque la gloria de Dios es que el hombre viva, y la vida del hombre es la visión de Dios: si ya la revelación de Dios por la creación procuró la vida a todos los seres que viven en la tierra, cuánto más la manifestación del Padre por el Verbo procurará la vida a los que ven a Dios} (San Ireneo de Lyon, \emph{Adversus haereses}, 4,20,7). El fin último de la creación es que Dios, \textquote{Creador de todos los seres, sea por fin \textquote{todo en todas las cosas} (\emph{1 Co} 15,28), \emph{procurando al mismo tiempo su gloria y nuestra felicidad}} (AG 2).

\textbf{La Buena Nueva: Dios ha enviado a su Hijo}

\textbf{422} \textquote{Pero, al llegar la plenitud de los tiempos, envió Dios a su Hijo, nacido de mujer, nacido bajo la Ley, para rescatar a los que se hallaban bajo la Ley, y para que recibiéramos la filiación adoptiva} (\emph{Ga} 4, 4-5). He aquí \textquote{la Buena Nueva de Jesucristo, Hijo de Dios} (\emph{Mc} 1, 1): Dios ha visitado a su pueblo (cf. \emph{Lc} 1, 68), ha cumplido las promesas hechas a Abraham y a su descendencia (cf. \emph{Lc} 1, 55); lo ha hecho más allá de toda expectativa: Él ha enviado a su \textquote{Hijo amado} (\emph{Mc} 1, 11).

\textbf{654} Hay un doble aspecto en el misterio pascual: por su muerte nos libera del pecado, por su Resurrección nos abre el acceso a una nueva vida. Esta es, en primer lugar, la \emph{justificación} que nos devuelve a la gracia de Dios (cf. \emph{Rm} 4, 25) \textquote{a fin de que, al igual que Cristo fue resucitado de entre los muertos [\ldots{}] así también nosotros vivamos una nueva vida} (\emph{Rm} 6, 4). Consiste en la victoria sobre la muerte y el pecado y en la nueva participación en la gracia (cf. \emph{Ef} 2, 4-5; \emph{1 P} 1, 3). Realiza la \emph{adopción filial} porque los hombres se convierten en hermanos de Cristo, como Jesús mismo llama a sus discípulos después de su Resurrección: \textquote{Id, avisad a mis hermanos} (\emph{Mt} 28, 10; \emph{Jn} 20, 17). Hermanos no por naturaleza, sino por don de la gracia, porque esta filiación adoptiva confiere una participación real en la vida del Hijo único, la que ha revelado plenamente en su Resurrección.

\textbf{1709} El que cree en Cristo es hecho hijo de Dios. Esta adopción filial lo transforma dándole la posibilidad de seguir el ejemplo de Cristo. Le hace capaz de obrar rectamente y de practicar el bien. En la unión con su Salvador, el discípulo alcanza la perfección de la caridad, la santidad. La vida moral, madurada en la gracia, culmina en vida eterna, en la gloria del cielo.

\textbf{2009} La adopción filial, haciéndonos partícipes por la gracia de la naturaleza divina, puede conferirnos, según la justicia gratuita de Dios, \emph{un verdadero mérito}. Se trata de un derecho por gracia, el pleno derecho del amor, que nos hace \textquote{coherederos} de Cristo y dignos de obtener la herencia prometida de la vida eterna (cf. Concilio de Trento: DS 1546). Los méritos de nuestras buenas obras son dones de la bondad divina (cf. Concilio de Trento: DS 1548). \textquote{La gracia ha precedido; ahora se da lo que es debido [\ldots{}] Los méritos son dones de Dios} (San Agustín, \emph{Sermo} 298, 4-5).

Jesús observa la Ley y la perfecciona

CEC 527, 577-582:

\textbf{527} La \emph{Circuncisión} de Jesús, al octavo día de su nacimiento (cf. \emph{Lc} 2, 21) es señal de su inserción en la descendencia de Abraham, en el pueblo de la Alianza, de su sometimiento a la Ley (cf. \emph{Ga} 4, 4) y de su consagración al culto de Israel en el que participará durante toda su vida. Este signo prefigura \textquote{la circuncisión en Cristo} que es el Bautismo (\emph{Col} 2, 11-13).

\textbf{Jesús y la Ley}

\textbf{577} Al comienzo del Sermón de la Montaña, Jesús hace una advertencia solemne presentando la Ley dada por Dios en el Sinaí con ocasión de la Primera Alianza, a la luz de la gracia de la Nueva Alianza:

\textquote{No penséis que he venido a abolir la Ley y los Profetas. No he venido a abolir sino a dar cumplimiento. Sí, os lo aseguro: el cielo y la tierra pasarán antes que pase una \textquote{i} o un ápice de la Ley sin que todo se haya cumplido. Por tanto, el que quebrante uno de estos mandamientos menores, y así lo enseñe a los hombres, será el menor en el Reino de los cielos; en cambio el que los observe y los enseñe, ése será grande en el Reino de los cielos} (\emph{Mt} 5, 17-19).

\textbf{578} Jesús, el Mesías de Israel, por lo tanto el más grande en el Reino de los cielos, se debía sujetar a la Ley cumpliéndola en su totalidad hasta en sus menores preceptos, según sus propias palabras. Incluso es el único en poderlo hacer perfectamente (cf. \emph{Jn} 8, 46). Los judíos, según su propia confesión, jamás han podido cumplir la Ley en su totalidad, sin violar el menor de sus preceptos (cf. \emph{Jn} 7, 19; \emph{Hch} 13, 38-41; 15, 10). Por eso, en cada fiesta anual de la Expiación, los hijos de Israel piden perdón a Dios por sus transgresiones de la Ley. En efecto, la Ley constituye un todo y, como recuerda Santiago, \textquote{quien observa toda la Ley, pero falta en un solo precepto, se hace reo de todos} (\emph{St} 2, 10; cf. \emph{Ga} 3, 10; 5, 3).

\textbf{579} Este principio de integridad en la observancia de la Ley, no sólo en su letra sino también en su espíritu, era apreciado por los fariseos. Al subrayarlo para Israel, muchos judíos del tiempo de Jesús fueron conducidos a un celo religioso extremo (cf. \emph{Rm} 10, 2), el cual, si no quería convertirse en una casuística \textquote{hipócrita} (cf. \emph{Mt} 15, 3-7; \emph{Lc} 11, 39-54) no podía más que preparar al pueblo a esta intervención inaudita de Dios que será la ejecución perfecta de la Ley por el único Justo en lugar de todos los pecadores (cf. \emph{Is} 53, 11; \emph{Hb} 9, 15).

\textbf{580} El cumplimiento perfecto de la Ley no podía ser sino obra del divino Legislador que nació sometido a la Ley en la persona del Hijo (cf. \emph{Ga} 4, 4). En Jesús la Ley ya no aparece grabada en tablas de piedra sino \textquote{en el fondo del corazón} (\emph{Jr} 31, 33) del Siervo, quien, por \textquote{aportar fielmente el derecho} (\emph{Is} 42, 3), se ha convertido en \textquote{la Alianza del pueblo} (\emph{Is} 42, 6). Jesús cumplió la Ley hasta tomar sobre sí mismo \textquote{la maldición de la Ley} (\emph{Ga} 3, 13) en la que habían incurrido los que no \textquote{practican todos los preceptos de la Ley} (\emph{Ga} 3, 10) porque \textquote{ha intervenido su muerte para remisión de las transgresiones de la Primera Alianza} (\emph{Hb} 9, 15).

\textbf{581} Jesús fue considerado por los judíos y sus jefes espirituales como un \textquote{rabbi} (cf. \emph{Jn} 11, 28; 3, 2; \emph{Mt} 22, 23-24, 34-36). Con frecuencia argumentó en el marco de la interpretación rabínica de la Ley (cf. \emph{Mt} 12, 5; 9, 12; \emph{Mc} 2, 23-27; \emph{Lc} 6, 6-9; \emph{Jn} 7, 22-23). Pero al mismo tiempo, Jesús no podía menos que chocar con los doctores de la Ley porque no se contentaba con proponer su interpretación entre los suyos, sino que \textquote{enseñaba como quien tiene autoridad y no como los escribas} (\emph{Mt} 7, 28-29). La misma Palabra de Dios, que resonó en el Sinaí para dar a Moisés la Ley escrita, es la que en Él se hace oír de nuevo en el Monte de las Bienaventuranzas (cf. \emph{Mt} 5, 1). Esa palabra no revoca la Ley sino que la perfecciona aportando de modo divino su interpretación definitiva: \textquote{Habéis oído también que se dijo a los antepasados [\ldots{}] pero yo os digo} (\emph{Mt} 5, 33-34). Con esta misma autoridad divina, desaprueba ciertas \textquote{tradiciones humanas} (\emph{Mc} 7, 8) de los fariseos que \textquote{anulan la Palabra de Dios} (\emph{Mc} 7, 13).

\textbf{582} Yendo más lejos, Jesús da plenitud a la Ley sobre la pureza de los alimentos, tan importante en la vida cotidiana judía, manifestando su sentido \textquote{pedagógico} (cf. \emph{Ga} 3, 24) por medio de una interpretación divina: \textquote{Todo lo que de fuera entra en el hombre no puede hacerle impuro [\ldots{}] ---así declaraba puros todos los alimentos---. Lo que sale del hombre, eso es lo que hace impuro al hombre. Porque de dentro, del corazón de los hombres, salen las intenciones malas} (\emph{Mc} 7, 18-21). Jesús, al dar con autoridad divina la interpretación definitiva de la Ley, se vio enfrentado a algunos doctores de la Ley que no aceptaban su interpretación a pesar de estar garantizada por los signos divinos con que la acompañaba (cf. \emph{Jn} 5, 36; 10, 25. 37-38; 12, 37). Esto ocurre, en particular, respecto al problema del sábado: Jesús recuerda, frecuentemente con argumentos rabínicos (cf. \emph{Mt} 2,25-27; \emph{Jn} 7, 22-24), que el descanso del sábado no se quebranta por el servicio de Dios (cf. \emph{Mt} 12, 5; \emph{Nm} 28, 9) o al prójimo (cf. \emph{Lc} 13, 15-16; 14, 3-4) que realizan sus curaciones.

la Ley nueva nos libera de las restricciones de la Ley antigua

CEC 580, 1972:

\textbf{580} El cumplimiento perfecto de la Ley no podía ser sino obra del divino Legislador que nació sometido a la Ley en la persona del Hijo (cf. \emph{Ga} 4, 4). En Jesús la Ley ya no aparece grabada en tablas de piedra sino \textquote{en el fondo del corazón} (\emph{Jr} 31, 33) del Siervo, quien, por \textquote{aportar fielmente el derecho} (\emph{Is} 42, 3), se ha convertido en \textquote{la Alianza del pueblo} (\emph{Is} 42, 6). Jesús cumplió la Ley hasta tomar sobre sí mismo \textquote{la maldición de la Ley} (\emph{Ga} 3, 13) en la que habían incurrido los que no \textquote{practican todos los preceptos de la Ley} (\emph{Ga} 3, 10) porque \textquote{ha intervenido su muerte para remisión de las transgresiones de la Primera Alianza} (\emph{Hb} 9, 15).

\textbf{1972} La Ley nueva es llamada \emph{ley de amor}, porque hace obrar por el amor que infunde el Espíritu Santo más que por el temor; \emph{ley de gracia}, porque confiere la fuerza de la gracia para obrar mediante la fe y los sacramentos; \emph{ley de libertad} (cf. \emph{St} 1, 25; 2, 12), porque nos libera de las observancias rituales y jurídicas de la Ley antigua, nos inclina a obrar espontáneamente bajo el impulso de la caridad y nos hace pasar de la condición del siervo \textquote{que ignora lo que hace su señor}, a la de amigo de Cristo, \textquote{porque todo lo que he oído a mi Padre os lo he dado a conocer} (\emph{Jn} 15, 15), o también a la condición de hijo heredero (cf. \emph{Ga} 4, 1-7. 21-31; \emph{Rm} 8, 15).

por medio del Espíritu Santo podemos llamar a Dios \textquote{Abba}

CEC 683, 689, 1695, 2766, 2777-2778:

\textbf{683} \textquote{Nadie puede decir: \textquote{¡Jesús es Señor!} sino por influjo del Espíritu Santo} (\emph{1 Co} 12, 3). \textquote{Dios ha enviado a nuestros corazones el Espíritu de su Hijo que clama ¡\emph{Abbá}, Padre!} (\emph{Ga} 4, 6). Este conocimiento de fe no es posible sino en el Espíritu Santo. Para entrar en contacto con Cristo, es necesario primeramente haber sido atraído por el Espíritu Santo. Él es quien nos precede y despierta en nosotros la fe. Mediante el Bautismo, primer sacramento de la fe, la vida, que tiene su fuente en el Padre y se nos ofrece por el Hijo, se nos comunica íntima y personalmente por el Espíritu Santo en la Iglesia:

El Bautismo \textquote{nos da la gracia del nuevo nacimiento en Dios Padre por medio de su Hijo en el Espíritu Santo. Porque los que son portadores del Espíritu de Dios son conducidos al Verbo, es decir al Hijo; pero el Hijo los presenta al Padre, y el Padre les concede la incorruptibilidad. Por tanto, sin el Espíritu no es posible ver al Hijo de Dios, y, sin el Hijo, nadie puede acercarse al Padre, porque el conocimiento del Padre es el Hijo, y el conocimiento del Hijo de Dios se logra por el Espíritu Santo} (San Ireneo de Lyon, \emph{Demonstratio praedicationis apostolicae}, 7: SC 62 41-42).

\textbf{La misión conjunta del Hijo y del Espíritu Santo}

\textbf{689} Aquel al que el Padre ha enviado a nuestros corazones, el Espíritu de su Hijo (cf. \emph{Ga} 4, 6) es realmente Dios. Consubstancial con el Padre y el Hijo, es inseparable de ellos, tanto en la vida íntima de la Trinidad como en su don de amor para el mundo. Pero al adorar a la Santísima Trinidad vivificante, consubstancial e indivisible, la fe de la Iglesia profesa también la distinción de las Personas. Cuando el Padre envía su Verbo, envía también su Aliento: misión conjunta en la que el Hijo y el Espíritu Santo son distintos pero inseparables. Sin ninguna duda, Cristo es quien se manifiesta, Imagen visible de Dios invisible, pero es el Espíritu Santo quien lo revela.

\textbf{1695} \textquote{Justificados [\ldots{}] en el nombre del Señor Jesucristo y en el Espíritu de nuestro Dios} (\emph{1 Co} 6,11.), \textquote{santificados y llamados a ser santos} (\emph{1 Co} 1,2.), los cristianos se convierten en \textquote{el templo [\ldots{}] del Espíritu Santo} (cf. \emph{1 Co} 6,19). Este \textquote{Espíritu del Hijo} les enseña a orar al Padre (\emph{Ga} 4, 6) y, haciéndose vida en ellos, les hace obrar (cf. \emph{Ga} 5, 25) para dar \textquote{los frutos del Espíritu} (\emph{Ga} 5, 22.) por la caridad operante. Sanando las heridas del pecado, el Espíritu Santo nos renueva interiormente mediante una transformación espiritual (cf. \emph{Ef} 4, 23.), nos ilumina y nos fortalece para vivir como \textquote{hijos de la luz} (\emph{Ef} 5, 8.), \textquote{por la bondad, la justicia y la verdad} en todo (\emph{Ef} 5,9.).

\textbf{2766} Pero Jesús no nos deja una fórmula para repetirla de modo mecánico (cf. \emph{Mt} 6, 7; \emph{1 R} 18, 26-29). Como en toda oración vocal, el Espíritu Santo, a través de la Palabra de Dios, enseña a los hijos de Dios a hablar con su Padre. Jesús no sólo nos enseña las palabras de la oración filial, sino que nos da también el Espíritu por el que estas se hacen en nosotros \textquote{espíritu [\ldots{}] y vida} (\emph{Jn} 6, 63). Más todavía: la prueba y la posibilidad de nuestra oración filial es que el Padre \textquote{ha enviado [\ldots{}] a nuestros corazones el Espíritu de su Hijo que clama: \textquote{¡Abbá, Padre!}} (\emph{Ga} 4, 6). Ya que nuestra oración interpreta nuestros deseos ante Dios, es también \textquote{el que escruta los corazones}, el Padre, quien \textquote{conoce cuál es la aspiración del Espíritu, y que su intercesión en favor de los santos es según Dios} (\emph{Rm} 8, 27). La oración al Padre se inserta en la misión misteriosa del Hijo y del Espíritu.

\textbf{Acercarse a Él con toda confianza}

\textbf{2777} En la liturgia romana, se invita a la asamblea eucarística a rezar el Padre Nuestro con una audacia filial; las liturgias orientales usan y desarrollan expresiones análogas: \textquote{Atrevernos con toda confianza}, \textquote{Haznos dignos de}. Ante la zarza ardiendo, se le dijo a Moisés: \textquote{No te acerques aquí. Quita las sandalias de tus pies} (\emph{Ex} 3, 5). Este umbral de la santidad divina, sólo lo podía franquear Jesús, el que \textquote{después de llevar a cabo la purificación de los pecados} (\emph{Hb} 1, 3), nos introduce en presencia del Padre: \textquote{Hénos aquí, a mí y a los hijos que Dios me dio} (\emph{Hb} 2, 13):

\textquote{La conciencia que tenemos de nuestra condición de esclavos nos haría meternos bajo tierra, nuestra condición terrena se desharía en polvo, si la autoridad de nuestro mismo Padre y el Espíritu de su Hijo, no nos empujasen a proferir este grito: \textquote{Abbá, Padre} (\emph{Rm} 8, 15) \ldots{} ¿Cuándo la debilidad de un mortal se atrevería a llamar a Dios Padre suyo, sino solamente cuando lo íntimo del hombre está animado por el Poder de lo alto?} (San Pedro Crisólogo, \emph{Sermón} 71, 3).

\textbf{2778} Este poder del Espíritu que nos introduce en la Oración del Señor se expresa en las liturgias de Oriente y de Occidente con la bella palabra, típicamente cristiana: \emph{parrhesia}, simplicidad sin desviación, conciencia filial, seguridad alegre, audacia humilde, certeza de ser amado (cf. \emph{Ef} 3, 12; \emph{Hb} 3, 6; 4, 16; 10, 19; \emph{1 Jn} 2,28; 3, 21; 5, 14).

El nombre de Jesús

CEC 430-435, 2666-2668, 2812:

\textbf{430} \emph{Jesús} quiere decir en hebreo: \textquote{Dios salva}. En el momento de la anunciación, el ángel Gabriel le dio como nombre propio el nombre de Jesús que expresa a la vez su identidad y su misión (cf. \emph{Lc} 1, 31). Ya que \textquote{¿quién puede perdonar pecados, sino sólo Dios?} (\emph{Mc} 2, 7), es Él quien, en Jesús, su Hijo eterno hecho hombre \textquote{salvará a su pueblo de sus pecados} (\emph{Mt} 1, 21). En Jesús, Dios recapitula así toda la historia de la salvación en favor de los hombres.

\textbf{431} En la historia de la salvación, Dios no se ha contentado con librar a Israel de \textquote{la casa de servidumbre} (\emph{Dt} 5, 6) haciéndole salir de Egipto. Él lo salva además de su pecado. Puesto que el pecado es siempre una ofensa hecha a Dios (cf. \emph{Sal} 51, 6), sólo Él es quien puede absolverlo (cf. \emph{Sal} 51, 12). Por eso es por lo que Israel, tomando cada vez más conciencia de la universalidad del pecado, ya no podrá buscar la salvación más que en la invocación del nombre de Dios Redentor (cf. \emph{Sal} 79, 9).

\textbf{432} El nombre de Jesús significa que el Nombre mismo de Dios está presente en la Persona de su Hijo (cf. \emph{Hch} 5, 41; \emph{3 Jn} 7) hecho hombre para la Redención universal y definitiva de los pecados. Él es el Nombre divino, el único que trae la salvación (cf. \emph{Jn} 3, 18; \emph{Hch} 2, 21) y de ahora en adelante puede ser invocado por todos porque se ha unido a todos los hombres por la Encarnación (cf. \emph{Rm} 10, 6-13) de tal forma que \textquote{no hay bajo el cielo otro nombre dado a los hombres por el que nosotros debamos salvarnos} (\emph{Hch} 4, 12; cf. \emph{Hch} 9, 14; \emph{St} 2, 7).

\textbf{433} El Nombre de Dios Salvador era invocado una sola vez al año por el sumo sacerdote para la expiación de los pecados de Israel, cuando había asperjado el propiciatorio del Santo de los Santos con la sangre del sacrificio (cf. \emph{Lv} 16, 15-16; \emph{Si} 50, 20; \emph{Hb} 9, 7). El propiciatorio era el lugar de la presencia de Dios (cf. \emph{Ex} 25, 22; \emph{Lv} 16, 2; \emph{Nm} 7, 89; \emph{Hb} 9, 5). Cuando san Pablo dice de Jesús que \textquote{Dios lo exhibió como instrumento de propiciación por su propia sangre} (\emph{Rm} 3, 25) significa que en su humanidad \textquote{estaba Dios reconciliando al mundo consigo} (\emph{2 Co} 5, 19).

\textbf{434} La Resurrección de Jesús glorifica el Nombre de Dios \textquote{Salvador} (cf. \emph{Jn} 12, 28) porque de ahora en adelante, el Nombre de Jesús es el que manifiesta en plenitud el poder soberano del \textquote{Nombre que está sobre todo nombre} (\emph{Flp} 2, 9). Los espíritus malignos temen su Nombre (cf. \emph{Hch} 16, 16-18; 19, 13-16) y en su nombre los discípulos de Jesús hacen milagros (cf. \emph{Mc} 16, 17) porque todo lo que piden al Padre en su Nombre, Él se lo concede (\emph{Jn} 15, 16).

\textbf{435} El Nombre de Jesús está en el corazón de la plegaria cristiana. Todas las oraciones litúrgicas se acaban con la fórmula \emph{Per Dominum nostrum Jesum Christum\ldots{}} (\textquote{Por nuestro Señor Jesucristo\ldots{}}). El \textquote{Avemaría} culmina en \textquote{y bendito es el fruto de tu vientre, Jesús}. La oración del corazón, en uso en Oriente, llamada \textquote{oración a Jesús} dice: \textquote{Señor Jesucristo, Hijo de Dios, ten piedad de mí pecador}. Numerosos cristianos mueren, como santa Juana de Arco, teniendo en sus labios una única palabra: \textquote{Jesús}.

\textbf{2666} Pero el Nombre que todo lo contiene es aquel que el Hijo de Dios recibe en su encarnación: JESÚS. El nombre divino es inefable para los labios humanos (cf. \emph{Ex} 3, 14; 33, 19-23), pero el Verbo de Dios, al asumir nuestra humanidad, nos lo entrega y nosotros podemos invocarlo: \textquote{Jesús}, \textquote{YHVH salva} (cf. \emph{Mt} 1, 21). El Nombre de Jesús contiene todo: Dios y el hombre y toda la Economía de la creación y de la salvación. Decir \textquote{Jesús} es invocarlo desde nuestro propio corazón. Su Nombre es el único que contiene la presencia que significa. Jesús es el resucitado, y cualquiera que invoque su Nombre acoge al Hijo de Dios que le amó y se entregó por él (cf. \emph{Rm} 10, 13; \emph{Hch} 2, 21; 3, 15-16; \emph{Ga} 2, 20).

\textbf{2667} Esta invocación de fe bien sencilla ha sido desarrollada en la tradición de la oración bajo formas diversas en Oriente y en Occidente. La formulación más habitual, transmitida por los espirituales del Sinaí, de Siria y del Monte Athos es la invocación: \textquote{Señor Jesucristo, Hijo de Dios, ten piedad de nosotros, pecadores} Conjuga el himno cristológico de \emph{Flp} 2, 6-11 con la petición del publicano y del mendigo ciego (cf. \emph{Lc} 18,13; \emph{Mc} 10, 46-52). Mediante ella, el corazón está acorde con la miseria de los hombres y con la misericordia de su Salvador.

\textbf{2668} La invocación del santo Nombre de Jesús es el camino más sencillo de la oración continua. Repetida con frecuencia por un corazón humildemente atento, no se dispersa en \textquote{palabrerías} (\emph{Mt} 6, 7), sino que \textquote{conserva la Palabra y fructifica con perseverancia} (cf. \emph{Lc} 8, 15). Es posible \textquote{en todo tiempo} porque no es una ocupación al lado de otra, sino la única ocupación, la de amar a Dios, que anima y transfigura toda acción en Cristo Jesús.

\textbf{2812} Finalmente, el Nombre de Dios Santo se nos ha revelado y dado, en la carne, en Jesús, como Salvador (cf. \emph{Mt} 1, 21; \emph{Lc} 1, 31): revelado por lo que Él es, por su Palabra y por su Sacrificio (cf. \emph{Jn} 8, 28; 17, 8; 17, 17-19). Esto es el núcleo de su oración sacerdotal: \textquote{Padre santo \ldots{} por ellos me consagro a mí mismo, para que ellos también sean consagrados en la verdad} (\emph{Jn} 17, 19). Jesús nos \textquote{manifiesta} el Nombre del Padre (\emph{Jn} 17, 6) porque \textquote{santifica} Él mismo su Nombre (cf. \emph{Ez} 20, 39; 36, 20-21). Al terminar su Pascua, el Padre le da el Nombre que está sobre todo nombre: Jesús es Señor para gloria de Dios Padre (cf. \emph{Flp} 2, 9-11).

\chapter{Domingo II de Navidad}

\section{Lecturas}

PRIMERA LECTURA

Del libro de Ben Sirá 24, 1-2. 8-12

La sabiduría de Dios habitó en el pueblo escogido

La sabiduría hace su propia alabanza

encuentra su honor en Dios

y se gloría en medio de su pueblo.

En la asamblea del Altísimo

abre su boca y se gloría ante el Poderoso.

El Creador del universo me dio una orden,

el que me había creado estableció mi morada

y me dijo: ``Pon tu tienda en Jacob,

y fija tu heredad en Israel''.

Desde el principio, antes de los siglos, me creó,

y nunca jamás dejaré de existir.

Ejercí mi ministerio en la Tienda santa delante de él,

y así me establecí en Sión.

En la ciudad amada encontré descanso,

y en Jerusalén reside mi poder.

Arraigué en un pueblo glorioso,

en la porción del Señor, en su heredad.

SALMO RESPONSORIAL

Salmo 147, 12-15. 19-20

El Verbo se hizo carne y habitó entre nosotros

℣. Glorifica al Señor, Jerusalén;

alaba a tu Dios, Sion.

Que ha reforzado los cerrojos de tus puertas,

y ha bendecido a tus hijos dentro de ti. ℟.

℣. Ha puesto paz en tus fronteras,

te sacia con flor de harina.

Él envía su mensaje a la tierra,

y su palabra corre veloz. ℟.

℣. Anuncia su palabra a Jacob,

sus decretos y mandatos a Israel;

con ninguna nación obró así,

ni les dio a conocer sus mandatos. ℟.

SEGUNDA LECTURA

De la carta del apóstol san Pablo a los Efesios 1, 3-6. 15-18

Él nos ha destinado por medio de Jesucristo a ser sus hijos

Bendito sea Dios, Padre de Nuestro Señor Jesucristo,

que nos ha bendecido en Cristo

con toda clase de bendiciones espirituales en los cielos.

Él nos eligió en Cristo antes de la fundación del mundo

para que fuésemos santos e intachables ante él por el amor.

Él nos ha destinado por medio de Jesucristo,

según el beneplácito de su voluntad,

a ser sus hijos,

para alabanza de la gloria de su gracia,

que tan generosamente nos ha concedido en el Amado.

Por eso, habiendo oído hablar de vuestra fe en Cristo y de vuestro amor
a todos los santos, no ceso de dar gracias por vosotros, recordándoos en
mis oraciones, a fin de que el Dios de nuestro Señor Jesucristo, el
Padre de la gloria, os dé espíritu de sabiduría y revelación para
conocerlo, e ilumine los ojos de vuestro corazón para que comprendáis
cuál es la esperanza a la que os llama, cuál la riqueza de gloria que da
en herencia a los santos.

EVANGELIO

Del Evangelio según san Juan 1, 1-18

El Verbo se hizo carne y habitó entre nosotros

En el principio existía el Verbo, y el Verbo estaba junto a Dios, y el
Verbo era Dios.

Él estaba en el principio junto a Dios.

Por medio de él se hizo todo, y sin él no se hizo nada de cuanto se ha
hecho.

En él estaba la vida, y la vida era la luz de los hombres.

Y la luz brilla en la tiniebla, y la tiniebla no lo recibió.

Surgió un hombre enviado por Dios, que se llamaba Juan:

este venía como testigo, para dar testimonio de la luz, para que todos
creyeran por medio de él.

No era él la luz, sino el que daba testimonio de la luz.

El Verbo era la luz verdadera, que alumbra a todo hombre, viniendo al
mundo.

En el mundo estaba; el mundo se hizo por medio de él, y el mundo no lo
conoció.

Vino a su casa, y los suyos no lo recibieron.

Pero a cuantos lo recibieron, les dio poder de ser hijos de Dios, a los
que creen en su nombre.

Estos no han nacido de sangre, ni de deseo de carne, ni de deseo de
varón, sino que han nacido de Dios.

Y el Verbo se hizo carne y habitó entre nosotros, y hemos contemplado su
gloria: gloria como del Unigénito del Padre, lleno de gracia y de
verdad.

Juan da testimonio de él y grita diciendo: «Este es de quien dije: El
que viene detrás de mí se ha puesto delante de mí, porque existía antes
que yo».

Pues de su plenitud todos hemos recibido, gracia tras gracia.

Porque la ley se dio por medio de Moisés, la gracia y la verdad nos han
llegado por medio de Jesucristo.

A Dios nadie lo ha visto jamás: Dios unigénito, que está en el seno del
Padre, es quien lo ha dado a conocer.

\section{Comentario Patrístico}

\subsection{San León Magno, papa}

El nacimiento del Señor es el nacimiento de la paz

Sermón 6, 2-3 en la Natividad del Señor: PL 54, 213-216.

Aunque aquella infancia, que la majestad del Hijo de Dios se dignó hacer suya, tuvo como continuación la plenitud de una edad adulta, y, después del triunfo de su pasión y resurrección, todas las acciones de su estado de humildad, que el Señor asumió por nosotros, pertenecen ya al pasado, la festividad de hoy renueva ante nosotros los sagrados comienzos de Jesús, nacido de la Virgen María; de modo que, mientras adoramos el nacimiento de nuestro Salvador, resulta que estamos celebrando nuestro propio comienzo.

Efectivamente, la generación de Cristo es el comienzo del pueblo cristiano, y el nacimiento de la cabeza lo es al mismo tiempo del cuerpo.

Aunque cada uno de los que llama el Señor a formar parte de su pueblo sea llamado en un tiempo determinado y aunque todos los hijos de la Iglesia hayan sido llamados cada uno en días distintos, con todo, la totalidad de los fieles, nacida en la fuente bautismal, ha nacido con Cristo en su nacimiento, del mismo modo que ha sido crucificada con Cristo en su pasión, ha sido resucitada en su resurrección y ha sido colocada a la derecha del Padre en su ascensión.

Cualquier hombre que cree --en cualquier parte del mundo--, y se regenera en Cristo, una vez interrumpido el camino de su vieja condición original, pasa a ser un nuevo hombre al renacer; y ya no pertenece a la ascendencia de su padre carnal, sino a la simiente del Salvador, que se hizo precisamente Hijo del hombre, para que nosotros pudiésemos llegar a ser hijos de Dios.

Pues si él no hubiera descendido hasta nosotros revestido de esta humilde condición, nadie hubiera logrado llegar hasta él por sus propios méritos.

Por eso, la misma magnitud del beneficio otorgado exige de nosotros una veneración proporcionada a la excelsitud de esta dádiva. Y, como el bienaventurado Apóstol nos enseña, \emph{no hemos recibido el espíritu de este mundo, sino el Espíritu que procede de Dios}, a fin de que conozcamos lo que Dios nos ha otorgado; y el mismo Dios sólo acepta como culto piadoso el ofrecimiento de lo que él nos ha concedido.

¿Y qué podremos encontrar en el tesoro de la divina largueza tan adecuado al honor de la presente festividad como la paz, lo primero que los ángeles pregonaron en el nacimiento del Señor?

La paz es la que engendra los hijos de Dios, alimenta el amor y origina la unidad, es el descanso de los bienaventurados y la mansión de la eternidad. El fin propio de la paz y su fruto específico consiste en que se unan a Dios los que el mismo Señor separa del mundo.

Que los que \emph{no han nacido de sangre, ni de amor carnal, ni de amor humano, sino de Dios,} ofrezcan, por tanto, al Padre la concordia que es propia de hijos pacíficos, y que todos los miembros de la adopción converjan hacia el Primogénito de la nueva creación, que vino a cumplir la voluntad del que le enviaba y no la suya: puesto que la gracia del Padre no adoptó como herederos a quienes se hallaban en discordia e incompatibilidad, sino a quienes amaban y sentían lo mismo. Los que han sido reformados de acuerdo con una sola imagen deben ser concordes en el espíritu.

El nacimiento del Señor es el nacimiento de la paz: y así dice el Apóstol: \emph{El es nuestra paz; él ha hecho de los dos pueblos una sola cosa,} ya que, tanto los judíos como los gentiles, por su medio \emph{podemos acercarnos al Padre con un mismo Espíritu}.

\subsection{San Máximo Confesor}

Misterio siempre nuevo

De las Cinco Centurias, Centuria 1, 8-13: PG 90, 1182-86.

La Palabra de Dios, nacida una vez en la carne (lo que nos indica la querencia de su benignidad y humanidad), vuelve a nacer siempre gustosamente en el espíritu para quienes lo desean; vuelve a hacerse niño, y se vuelve a formar en aquellas virtudes; y la amplitud de su grandeza no disminuye por malevolencia o envidia, sino que se manifiesta a sí mismo en la medida en que sabe que lo puede asimilar el que lo recibe, y así, al mismo tiempo que explora discretamente la capacidad de quienes desean verlo, sigue manteniéndose siempre fuera del alcance de su percepción, a causa de la excelencia del misterio.

Por lo cual, el santo Apóstol, considerando sabiamente la fuerza del misterio, exclama: \emph{Jesucristo es el mismo ayer y hoy y siempre;} ya que entendía el misterio como algo siempre nuevo, al que nunca la comprensión de la mente puede hacer envejecer.

Nace Cristo Dios, hecho hombre mediante la incorporación de una carne dotada de alma inteligente; el mismo que había otorgado a las cosas proceder de la nada. Mientras tanto, brilla en lo alto la estrella del Oriente y conduce a los Magos al lugar en que yace la Palabra encarnada; con lo que muestra que hay en la ley y los profetas una palabra místicamente superior, que dirige a las gentes a la suprema luz del conocimiento.

Así pues, la palabra de la ley y de los profetas, entendida alegóricamente, conduce, como una estrella, al pleno conocimiento de Dios a aquellos que fueron llamados por la fuerza de la gracia, de acuerdo con el designio divino.

Dios se hace efectivamente hombre perfecto, sin alterar nada de lo que es propio de la naturaleza, a excepción del pecado (pues ni el mismo pecado era propio de la naturaleza). Se hace efectivamente hombre perfecto a fin de provocar, con la vista del manjar de su carne, la voracidad insaciable y ávida del dragón infernal; y abatirlo por completo cuando ingiriera una carne que habría de convertírsele en veneno, porque en ella se hallaba oculto el poder de la divinidad. Esta carne sería al mismo tiempo remedio de la naturaleza humana, ya que el mismo poder divino presente en aquélla habría de restituir la naturaleza humana a la gracia primera.

Y así como el dragón, deslizando su veneno en el árbol de la ciencia, había corrompido con su sabor la naturaleza, de la misma manera, al tratar de devorar la carne del Señor, se vio corrompido y destruido por la virtud de la divinidad que en ella residía.

Inmenso misterio de la divina encarnación, que sigue siendo siempre misterio; pues, ¿de qué modo puede la Palabra hecha carne seguir siendo su propia persona esencialmente, siendo así que la misma persona existe al mismo tiempo con todo su ser en Dios Padre? ¿Cómo la Palabra, que es toda ella Dios por naturaleza, se hizo toda ella por naturaleza hombre, sin detrimento de ninguna de las dos naturalezas: ni de la divina, en cuya virtud es Dios, ni de la nuestra, en virtud de la cual se hizo hombre? Sólo la fe capta estos misterios, ella precisamente que es la sustancia y la base de todas aquellas realidades que exceden la percepción y razón de la mente humana en todo su alcance.


\section{Homilías}

Las lecturas para este domingo son las mismas en los tres ciclos dominicales. No obstante, las homilías han sido distribuidas en esta obra en tres grupos, tomando en cuenta el ciclo litúrgico correspondiente al año en que fueron pronunciadas. Aquí aparecen las homilías que correspondieron al año A, y las de los años B y C aparecerán en sus respectivos volúmenes.

\subsection{Benedicto XVI, papa}

\subsubsection{Ángelus: Poner en Dios nuestra esperanza}

Domingo 3 de enero del 2010.

En este domingo ---segundo después de Navidad y primero del año nuevo--- me alegra renovar a todos mi deseo de todo bien en el Señor. No faltan los problemas, en la Iglesia y en el mundo, al igual que en la vida cotidiana de las familias. Pero, gracias a Dios, nuestra esperanza no se basa en pronósticos improbables ni en las previsiones económicas, aunque sean importantes. Nuestra esperanza está en Dios, no en el sentido de una religiosidad genérica, o de un fatalismo disfrazado de fe. Nosotros confiamos en el Dios que en Jesucristo ha revelado de modo completo y definitivo su voluntad de estar con el hombre, de compartir su historia, para guiarnos a todos a su reino de amor y de vida. Y esta gran esperanza anima y a veces corrige nuestras esperanzas humanas.

De esa revelación nos hablan hoy, en la liturgia eucarística, \textbf{tres lecturas bíblicas} de una riqueza extraordinaria: el capítulo 24 del \emph{Libro del Sirácida}, el himno que abre la \emph{Carta a los Efesios} de san Pablo y el prólogo del \emph{Evangelio de san Juan}. Estos textos afirman que Dios no sólo es el creador del universo ---aspecto común también a otras religiones--- sino que es Padre, que \textquote{nos eligió antes de crear el mundo (\ldots{}) predestinándonos a ser sus hijos adoptivos} (\emph{Ef} 1, 4-5) y que por esto llegó hasta el punto inconcebible de hacerse hombre: \textquote{El Verbo se hizo carne y acampó entre nosotros} (\emph{Jn} 1, 14). El misterio de la Encarnación de la Palabra de Dios fue preparado en el Antiguo Testamento, especialmente donde la Sabiduría divina se identifica con la Ley de Moisés. En efecto, la misma Sabiduría afirma: \textquote{El creador del universo me hizo plantar mi tienda, y me dijo: \textquote{Pon tu tienda en Jacob, entra en la heredad de Israel}} (\emph{Si} 24, 8). En Jesucristo, la Ley de Dios se ha hecho testimonio vivo, escrita en el corazón de un hombre en el que, por la acción del Espíritu Santo, reside corporalmente toda la plenitud de la divinidad (cf. \emph{Col} 2, 9).

Queridos amigos, esta es la verdadera razón de la esperanza de la humanidad: la historia tiene un sentido, porque en ella \textquote{habita} la Sabiduría de Dios. Sin embargo, el designio divino no se cumple automáticamente, porque es un proyecto de amor, y el amor genera libertad y pide libertad. Ciertamente, el reino de Dios viene, más aún, ya está presente en la historia y, gracias a la venida de Cristo, ya ha vencido a la fuerza negativa del maligno. Pero cada hombre y cada mujer es responsable de acogerlo en su vida, día tras día. Por eso, también 2010 será un año más o menos \textquote{bueno} en la medida en que cada uno, de acuerdo con sus responsabilidades, sepa colaborar con la gracia de Dios. Por lo tanto, dirijámonos a la Virgen María, para aprender de ella esta actitud espiritual. El Hijo de Dios tomó carne de ella, con su consentimiento. Cada vez que el Señor quiere dar un paso adelante, junto con nosotros, hacia la \textquote{tierra prometida}, llama primero a nuestro corazón; espera, por decirlo así, nuestro \textquote{sí}, tanto en las pequeñas decisiones como en las grandes. Que María nos ayude a aceptar siempre la voluntad de Dios, con humildad y valentía, a fin de que también las pruebas y los sufrimientos de la vida contribuyan a apresurar la venida de su reino de justicia y de paz.

\subsubsection{Ángelus: Entrar en las profundidades de Dios}

Domingo 2 de enero del 2011.

Os renuevo a todos mis mejores deseos para el año nuevo y doy las gracias a cuantos me han enviado mensajes de cercanía espiritual. La liturgia de este domingo vuelve a proponer el \textbf{Prólogo del \emph{Evangelio de san Juan}}, proclamado solemnemente en el día de Navidad. Este admirable texto expresa, en forma de himno, el misterio de la Encarnación, que predicaron los testigos oculares, los Apóstoles, especialmente san Juan, cuya fiesta, no por casualidad, se celebra el 27 de diciembre. Afirma san Cromacio de Aquileya que \textquote{Juan era el más joven de todos los discípulos del Señor; el más joven por edad, pero ya anciano por la fe} (Sermo II, 1 \emph{De Sancto Iohanne Evangelista:} CCL 9a, 101). Cuando leemos: \textquote{En el principio existía el Verbo y el Verbo estaba con Dios, y el Verbo era Dios} (\emph{Jn} 1, 1), el Evangelista ---al que tradicionalmente se compara con un águila--- se eleva por encima de la historia humana escrutando las profundidades de Dios; pero muy pronto, siguiendo a su Maestro, vuelve a la dimensión terrena diciendo: \textquote{Y el Verbo se hizo carne} (\emph{Jn} 1, 14). El Verbo es \textquote{una realidad viva: un Dios que\ldots{} se comunica haciéndose él mismo hombre} (J. Ratzinger, \emph{Teologia della liturgia}, LEV 2010, p. 618). En efecto, atestigua Juan, \textquote{puso su morada entre nosotros, y hemos contemplado su gloria} (\emph{Jn} 1, 14). \textquote{Se rebajó hasta asumir la humildad de nuestra condición ---comenta san León Magno--- sin que disminuyera su majestad} (\emph{Tractatus} XXI, 2: CCL 138, 86-87). Leemos también en el Prólogo: \textquote{De su plenitud hemos recibido todos, gracia por gracia} (\emph{Jn} 1, 16). \textquote{¿Cuál es la primera gracia que hemos recibido? ---se pregunta san Agustín, y responde--- Es la fe}. La segunda gracia, añade en seguida, es \textquote{la vida eterna} (\emph{Tractatus in Ioh}. III, 8.9: ccl 36, 24.25).

\subsection{Francisco, papa}

\subsubsection{Ángelus: Profundizar el sentido de su nacimiento}

Domingo 5 de enero del 2014.

La liturgia de este domingo nos vuelve a proponer, en el \textbf{Prólogo del Evangelio de san Juan}, el significado más profundo del Nacimiento de Jesús. Él es la Palabra de Dios que se hizo hombre y puso su \textquote{tienda}, su morada entre los hombres. Escribe el evangelista: \textquote{El Verbo se hizo carne y habitó entre nosotros} (\emph{Jn} 1, 14). En estas palabras, que no dejan de asombrarnos, está todo el cristianismo. Dios se hizo mortal, frágil como nosotros, compartió nuestra condición humana, excepto en el pecado, pero cargó sobre sí mismo los nuestros, como si fuesen propios. Entró en nuestra historia, llegó a ser plenamente Dios-con-nosotros. El nacimiento de Jesús, entonces, nos muestra que Dios quiso unirse a cada hombre y a cada mujer, a cada uno de nosotros, para comunicarnos su vida y su alegría.

Así Dios es Dios con nosotros, Dios que nos ama, Dios que camina con nosotros. Éste es el mensaje de Navidad: el Verbo se hizo carne. De este modo la Navidad nos revela el amor inmenso de Dios por la humanidad. De aquí se deriva también el entusiasmo, nuestra esperanza de cristianos, que en nuestra pobreza sabemos que somos amados, visitados y acompañados por Dios; y miramos al mundo y a la historia como el lugar donde caminar juntos con Él y entre nosotros, hacia los cielos nuevos y la tierra nueva. Con el nacimiento de Jesús nació una promesa nueva, nació un mundo nuevo, pero también un mundo que puede ser siempre renovado. Dios siempre está presente para suscitar hombres nuevos, para purificar el mundo del pecado que lo envejece, del pecado que lo corrompe. En lo que la historia humana y la historia personal de cada uno de nosotros pueda estar marcada por dificultades y debilidades, la fe en la Encarnación nos dice que Dios es solidario con el hombre y con su historia. Esta proximidad de Dios al hombre, a cada hombre, a cada uno de nosotros, es un don que no se acaba jamás. ¡Él está con nosotros! ¡Él es Dios con nosotros! Y esta cercanía no termina jamás. He aquí el gozoso anuncio de la Navidad: la luz divina, que inundó el corazón de la Virgen María y de san José, y guio los pasos de los pastores y de los magos, brilla también hoy para nosotros.

En el misterio de la Encarnación del Hijo de Dios hay también un aspecto vinculado con la libertad humana, con la libertad de cada uno de nosotros. En efecto, el Verbo de Dios pone su tienda entre nosotros, pecadores y necesitados de misericordia. Y todos nosotros deberíamos apresurarnos a recibir la gracia que Él nos ofrece. En cambio, continúa el Evangelio de san Juan, \textquote{los suyos no lo recibieron} (v. 11). Incluso nosotros muchas veces lo rechazamos, preferimos permanecer en la cerrazón de nuestros errores y en la angustia de nuestros pecados. Pero Jesús no desiste y no deja de ofrecerse a sí mismo y ofrecer su gracia que nos salva. Jesús es paciente, Jesús sabe esperar, nos espera siempre. Esto es un mensaje de esperanza, un mensaje de salvación, antiguo y siempre nuevo. Y nosotros estamos llamados a testimoniar con alegría este mensaje del Evangelio de la vida, del Evangelio de la luz, de la esperanza y del amor. Porque el mensaje de Jesús es éste: vida, luz, esperanza y amor.

Que María, Madre de Dios y nuestra Madre de ternura, nos sostenga siempre, para que permanezcamos fieles a la vocación cristiana y podamos realizar los deseos de justicia y de paz que llevamos en nosotros al inicio de este nuevo año.

\subsubsection{Ángelus: Revelación plena del plan de Dios}

Domingo 5 de enero del 2020.

En este segundo domingo de la Navidad, las lecturas bíblicas nos ayudan a alargar la mirada, para tomar una conciencia plena del significado del nacimiento de Jesús.

El \textbf{comienzo del Evangelio de San Juan} nos muestra una impactante novedad: el Verbo eterno, el Hijo de Dios, \textquote{se hizo carne} (v. 14). No sólo vino a vivir entre la gente, sino que se convirtió en uno del pueblo, ¡uno de nosotros! Después de este acontecimiento, para dirigir nuestras vidas, ya no tenemos sólo una ley, una institución, sino una Persona, una Persona divina, Jesús, que guía nuestras vidas, nos hace ir por el camino porque Él lo hizo antes.

\textbf{San Pablo} bendice a Dios por su plan de amor realizado en Jesucristo (cf. \emph{Efesios} 1, 3-6; 15-18). En este plan, cada uno de nosotros encuentra su vocación fundamental. ¿Y cuál es? Esto es lo que dice Pablo: estamos predestinados a ser hijos de Dios por medio de Jesucristo. El Hijo de Dios se hizo hombre para hacernos a nosotros, hombres, hijos de Dios. Por eso el Hijo eterno se hizo carne: para introducirnos en su relación filial con el Padre.

Así pues, hermanos y hermanas, mientras continuamos contemplando el admirable signo del belén, la liturgia de hoy nos dice que el Evangelio de Cristo no es una fábula, ni un mito, ni un cuento moralizante, no. El Evangelio de Cristo es la plena revelación del plan de Dios, el plan de Dios para el hombre y el mundo. Es un mensaje a la vez sencillo y grandioso, que nos lleva a preguntarnos: ¿qué plan concreto tiene el Señor para mí, actualizando aún hoy su nacimiento entre nosotros?

Es el \textbf{apóstol Pablo} quien nos sugiere la respuesta: \textquote{{[}Dios{]} nos ha elegido [\ldots{}] para ser santos e inmaculados en su presencia, en el amor} (v. 4). Este es el significado de la Navidad. Si el Señor sigue viniendo entre nosotros, si sigue dándonos el don de su Palabra, es para que cada uno de nosotros pueda responder a esta llamada: ser santos en el amor. La santidad pertenece a Dios, es comunión con Él, transparencia de su infinita bondad. La santidad es guardar el don que Dios nos ha dado. Simplemente esto: guardar la gratuidad. En esto consiste ser santo. Por tanto, quien acepta la santidad en sí mismo como un don de gracia, no puede dejar de traducirla en acciones concretas en la vida cotidiana. Este don, esta gracia que Dios me ha dado, la traduzco en una acción concreta en la vida cotidiana, en el encuentro con los demás. Esta caridad, esta misericordia hacia el prójimo, reflejo del amor de Dios, al mismo tiempo purifica nuestro corazón y nos dispone al perdón, haciéndonos \textquote{inmaculados} día tras día. Pero inmaculados no en el sentido de que yo elimino una mancha: inmaculados en el sentido de que Dios entra en nosotros, el don, la gratuidad de Dios entra en nosotros y nosotros lo guardamos y lo damos a los demás.

Que la Virgen María nos ayude a acoger con alegría y gratitud el diseño divino de amor realizado en Jesucristo.



\section{Temas}

Prólogo del Evangelio de Juan

CEC 151, 241, 291, 423, 445, 456-463, 504-505, 526, 1216, 2466, 2787:

\textbf{Creer en Jesucristo, el Hijo de Dios}

\textbf{151} Para el cristiano, creer en Dios es inseparablemente creer en Aquel que él ha enviado, \textquote{su Hijo amado}, en quien ha puesto toda su complacencia (\emph{Mc} 1,11). Dios nos ha dicho que le escuchemos (cf. \emph{Mc} 9,7). El Señor mismo dice a sus discípulos: \textquote{Creed en Dios, creed también en mí} (\emph{Jn} 14,1). Podemos creer en Jesucristo porque es Dios, el Verbo hecho carne: \textquote{A Dios nadie le ha visto jamás: el Hijo único, que está en el seno del Padre, él lo ha contado} (\emph{Jn} 1,18). Porque \textquote{ha visto al Padre} (\emph{Jn} 6,46), él es único en conocerlo y en poderlo revelar (cf. \emph{Mt} 11,27).

\textbf{241} Por eso los Apóstoles confiesan a Jesús como \textquote{el Verbo que en el principio estaba junto a Dios y que era Dios} (\emph{Jn} 1,1), como \textquote{la imagen del Dios invisible} (\emph{Col} 1,15), como \textquote{el resplandor de su gloria y la impronta de su esencia} (\emph{Hb} 1,3).

\textbf{291} \textquote{En el principio existía el Verbo [\ldots{}] y el Verbo era Dios [\ldots{}] Todo fue hecho por él y sin él nada ha sido hecho} (\emph{Jn} 1,1-3). El Nuevo Testamento revela que Dios creó todo por el Verbo Eterno, su Hijo amado. \textquote{En él fueron creadas todas las cosas, en los cielos y en la tierra [\ldots{}] todo fue creado por él y para él, él existe con anterioridad a todo y todo tiene en él su consistencia} (\emph{Col} 1, 16-17). La fe de la Iglesia afirma también la acción creadora del Espíritu Santo: él es el \textquote{dador de vida} (\emph{Símbolo Niceno-Constantinopolitano}), \textquote{el Espíritu Creador} (\emph{Liturgia de las Horas}, Himno \emph{Veni, Creator Spiritus}), la \textquote{Fuente de todo bien} (\emph{Liturgia bizantina}, Tropario de vísperas de Pentecostés).

\textbf{423} Nosotros creemos y confesamos que Jesús de Nazaret, nacido judío de una hija de Israel, en Belén en el tiempo del rey Herodes el Grande y del emperador César Augusto I; de oficio carpintero, muerto crucificado en Jerusalén, bajo el procurador Poncio Pilato, durante el reinado del emperador Tiberio, es el Hijo eterno de Dios hecho hombre, que ha \textquote{salido de Dios} (\emph{Jn} 13, 3), \textquote{bajó del cielo} (\emph{Jn} 3, 13; 6, 33), \textquote{ha venido en carne} (\emph{1 Jn} 4, 2), porque \textquote{la Palabra se hizo carne, y puso su morada entre nosotros, y hemos visto su gloria, gloria que recibe del Padre como Hijo único, lleno de gracia y de verdad [\ldots{}] Pues de su plenitud hemos recibido todos, y gracia por gracia} (\emph{Jn} 1, 14. 16).

\textbf{445} Después de su Resurrección, su filiación divina aparece en el poder de su humanidad glorificada: \textquote{Constituido Hijo de Dios con poder, según el Espíritu de santidad, por su Resurrección de entre los muertos} (\emph{Rm} 1, 4; cf. \emph{Hch} 13, 33). Los apóstoles podrán confesar \textquote{Hemos visto su gloria, gloria que recibe del Padre como Hijo único, lleno de gracia y de verdad} (\emph{Jn} 1, 14).

\textbf{456} Con el Credo Niceno-Constantinopolitano respondemos confesando: \textquote{\emph{Por nosotros los hombres y por nuestra salvación} bajó del cielo, y por obra del Espíritu Santo se encarnó de María la Virgen y se hizo hombre} (DS 150).

\textbf{457} El Verbo se encarnó \emph{para salvarnos reconciliándonos con Dios}: \textquote{Dios nos amó y nos envió a su Hijo como propiciación por nuestros pecados} (\emph{1 Jn} 4, 10). \textquote{El Padre envió a su Hijo para ser salvador del mundo} (\emph{1 Jn} 4, 14). \textquote{Él se manifestó para quitar los pecados} (\emph{1 Jn} 3, 5):

\textquote{Nuestra naturaleza enferma exigía ser sanada; desgarrada, ser restablecida; muerta, ser resucitada. Habíamos perdido la posesión del bien, era necesario que se nos devolviera. Encerrados en las tinieblas, hacía falta que nos llegara la luz; estando cautivos, esperábamos un salvador; prisioneros, un socorro; esclavos, un libertador. ¿No tenían importancia estos razonamientos? ¿No merecían conmover a Dios hasta el punto de hacerle bajar hasta nuestra naturaleza humana para visitarla, ya que la humanidad se encontraba en un estado tan miserable y tan desgraciado?} (San Gregorio de Nisa, \emph{Oratio catechetica}, 15: PG 45, 48B).

\textbf{458} El Verbo se encarnó \emph{para que nosotros conociésemos así el amor de Dios}: \textquote{En esto se manifestó el amor que Dios nos tiene: en que Dios envió al mundo a su Hijo único para que vivamos por medio de él} (\emph{1 Jn} 4, 9). \textquote{Porque tanto amó Dios al mundo que dio a su Hijo único, para que todo el que crea en él no perezca, sino que tenga vida eterna} (\emph{Jn} 3, 16).

\textbf{459} El Verbo se encarnó \emph{para ser nuestro modelo de santidad}: \textquote{Tomad sobre vosotros mi yugo, y aprended de mí \ldots{} } (\emph{Mt} 11, 29). \textquote{Yo soy el Camino, la Verdad y la Vida. Nadie va al Padre sino por mí} (\emph{Jn} 14, 6). Y el Padre, en el monte de la Transfiguración, ordena: \textquote{Escuchadle} (\emph{Mc} 9, 7; cf. \emph{Dt} 6, 4-5). Él es, en efecto, el modelo de las bienaventuranzas y la norma de la Ley nueva: \textquote{Amaos los unos a los otros como yo os he amado} (\emph{Jn} 15, 12). Este amor tiene como consecuencia la ofrenda efectiva de sí mismo (cf. \emph{Mc} 8, 34).

\textbf{460} El Verbo se encarnó \emph{para hacernos \textquote{partícipes de la naturaleza divina}} (\emph{2 P} 1, 4): \textquote{Porque tal es la razón por la que el Verbo se hizo hombre, y el Hijo de Dios, Hijo del hombre: para que el hombre al entrar en comunión con el Verbo y al recibir así la filiación divina, se convirtiera en hijo de Dios} (San Ireneo de Lyon, \emph{Adversus haereses}, 3, 19, 1). \textquote{Porque el Hijo de Dios se hizo hombre para hacernos Dios} (San Atanasio de Alejandría, \emph{De Incarnatione}, 54, 3: PG 25, 192B). \emph{Unigenitus} [\ldots{}] \emph{Dei Filius, suae divinitatis volens nos esse participes, naturam nostram assumpsit, ut homines deos faceret factus homo} (\textquote{El Hijo Unigénito de Dios, queriendo hacernos partícipes de su divinidad, asumió nuestra naturaleza, para que, habiéndose hecho hombre, hiciera dioses a los hombres}) (Santo Tomás de Aquino, \emph{Oficio de la festividad del Corpus}, Of. de Maitines, primer Nocturno, Lectura I).

\textbf{461} Volviendo a tomar la frase de san Juan (\textquote{El Verbo se encarnó}: \emph{Jn} 1, 14), la Iglesia llama \textquote{Encarnación} al hecho de que el Hijo de Dios haya asumido una naturaleza humana para llevar a cabo por ella nuestra salvación. En un himno citado por san Pablo, la Iglesia canta el misterio de la Encarnación:

\textquote{Tened entre vosotros los mismos sentimientos que tuvo Cristo: el cual, siendo de condición divina, no retuvo ávidamente el ser igual a Dios, sino que se despojó de sí mismo tomando condición de siervo, haciéndose semejante a los hombres y apareciendo en su porte como hombre; y se humilló a sí mismo, obedeciendo hasta la muerte y muerte de cruz} (\emph{Flp} 2, 5-8; cf. \emph{Liturgia de las Horas, Cántico de las Primeras Vísperas de Domingos}).

\textbf{462} La carta a los Hebreos habla del mismo misterio:

\textquote{Por eso, al entrar en este mundo, {[}Cristo{]} dice: No quisiste sacrificio y oblación; pero me has formado un cuerpo. Holocaustos y sacrificios por el pecado no te agradaron. Entonces dije: ¡He aquí que vengo [\ldots{}] a hacer, oh Dios, tu voluntad!} (\emph{Hb} 10, 5-7; \emph{Sal} 40, 7-9 {[}LXX{]}).

\textbf{463} La fe en la verdadera encarnación del Hijo de Dios es el signo distintivo de la fe cristiana: \textquote{Podréis conocer en esto el Espíritu de Dios: todo espíritu que confiesa a Jesucristo, venido en carne, es de Dios} (\emph{1 Jn} 4, 2). Esa es la alegre convicción de la Iglesia desde sus comienzos cuando canta \textquote{el gran misterio de la piedad}: \textquote{Él ha sido manifestado en la carne} (\emph{1 Tm} 3, 16).

\textbf{504} Jesús fue concebido por obra del Espíritu Santo en el seno de la Virgen María porque él es el \emph{Nuevo Adán} (cf. \emph{1 Co} 15, 45) que inaugura la nueva creación: \textquote{El primer hombre, salido de la tierra, es terreno; el segundo viene del cielo} (\emph{1 Co} 15, 47). La humanidad de Cristo, desde su concepción, está llena del Espíritu Santo porque Dios \textquote{le da el Espíritu sin medida} (\emph{Jn} 3, 34). De \textquote{su plenitud}, cabeza de la humanidad redimida (cf. \emph{Col} 1, 18), \textquote{hemos recibido todos gracia por gracia} (\emph{Jn} 1, 16).

\textbf{505} Jesús, el nuevo Adán, inaugura por su concepción virginal el nuevo nacimiento de los hijos de adopción en el Espíritu Santo por la fe \textquote{¿Cómo será eso?} (\emph{Lc} 1, 34; cf. \emph{Jn} 3, 9). La participación en la vida divina no nace \textquote{de la sangre, ni de deseo de carne, ni de deseo de hombre, sino de Dios} (\emph{Jn} 1, 13). La acogida de esta vida es virginal porque toda ella es dada al hombre por el Espíritu. El sentido esponsal de la vocación humana con relación a Dios (cf. \emph{2 Co} 11, 2) se lleva a cabo perfectamente en la maternidad virginal de María.

\textbf{526} \textquote{Hacerse niño} con relación a Dios es la condición para entrar en el Reino (cf. \emph{Mt} 18, 3-4); para eso es necesario abajarse (cf. \emph{Mt} 23, 12), hacerse pequeño; más todavía: es necesario \textquote{nacer de lo alto} (\emph{Jn} 3,7), \textquote{nacer de Dios} (\emph{Jn} 1, 13) para \textquote{hacerse hijos de Dios} (\emph{Jn} 1, 12). El misterio de Navidad se realiza en nosotros cuando Cristo \textquote{toma forma} en nosotros (\emph{Ga} 4, 19). Navidad es el misterio de este \textquote{admirable intercambio}:

\textquote{¡Oh admirable intercambio! El Creador del género humano, tomando cuerpo y alma, nace de la Virgen y, hecho hombre sin concurso de varón, nos da parte en su divinidad} (\emph{Solemnidad de la Santísima Virgen María, Madre de Dios,} Antífona de I y II Vísperas: \emph{Liturgia de las Horas}).

\textbf{1216} \textquote{Este baño es llamado \emph{iluminación} porque quienes reciben esta enseñanza (catequética) su espíritu es iluminado} (San Justino, \emph{Apología} 1,61). Habiendo recibido en el Bautismo al Verbo, \textquote{la luz verdadera que ilumina a todo hombre} (\emph{Jn} 1,9), el bautizado, \textquote{tras haber sido iluminado} (\emph{Hb} 10,32), se convierte en \textquote{hijo de la luz} (\emph{1 Ts} 5,5), y en \textquote{luz} él mismo (\emph{Ef} 5,8):

El Bautismo \textquote{es el más bello y magnífico de los dones de Dios [\ldots{}] lo llamamos don, gracia, unción, iluminación, vestidura de incorruptibilidad, baño de regeneración, sello y todo lo más precioso que hay. \emph{Don}, porque es conferido a los que no aportan nada; \emph{gracia}, porque es dado incluso a culpables; \emph{bautismo}, porque el pecado es sepultado en el agua; \emph{unción}, porque es sagrado y real (tales son los que son ungidos); \emph{iluminación}, porque es luz resplandeciente; \emph{vestidura}, porque cubre nuestra vergüenza; \emph{baño}, porque lava; \emph{sello}, porque nos guarda y es el signo de la soberanía de Dios} (San Gregorio Nacianceno, \emph{Oratio} 40,3-4).

\textbf{2466} En Jesucristo la verdad de Dios se manifestó en plenitud. \textquote{Lleno de gracia y de verdad} (\emph{Jn} 1, 14), él es la \textquote{luz del mundo} (\emph{Jn} 8, 12), \emph{la Verdad} (cf. \emph{Jn} 14, 6). El que cree en él, no permanece en las tinieblas (cf. \emph{Jn} 12, 46). El discípulo de Jesús, \textquote{permanece en su palabra}, para conocer \textquote{la verdad que hace libre} (cf. \emph{Jn} 8, 31-32) y que santifica (cf. \emph{Jn} 17, 17). Seguir a Jesús es vivir del \textquote{Espíritu de verdad} (\emph{Jn} 14, 17) que el Padre envía en su nombre (cf. \emph{Jn} 14, 26) y que conduce \textquote{a la verdad completa} (\emph{Jn} 16, 13). Jesús enseña a sus discípulos el amor incondicional de la verdad: \textquote{Sea vuestro lenguaje: \textquote{sí, sí}; \textquote{no, no}} (\emph{Mt} 5, 37).

\textbf{2787} Cuando decimos Padre \textquote{nuestro}, reconocemos ante todo que todas sus promesas de amor anunciadas por los profetas se han cumplido en la \emph{nueva y eterna Alianza} en Cristo: hemos llegado a ser \textquote{su Pueblo} y Él es desde ahora en adelante \textquote{nuestro Dios}. Esta relación nueva es una pertenencia mutua dada gratuitamente: por amor y fidelidad (cf. \emph{Os} 2, 21-22; 6, 1-6) tenemos que responder a la gracia y a la verdad que nos han sido dadas en Jesucristo (cf. \emph{Jn} 1, 17).

Cristo, Sabiduría de Dios

CEC 272, 295, 299, 474, 721, 1831:

\textbf{El misterio de la aparente impotencia de Dios}

\textbf{272} La fe en Dios Padre Todopoderoso puede ser puesta a prueba por la experiencia del mal y del sufrimiento. A veces Dios puede parecer ausente e incapaz de impedir el mal. Ahora bien, Dios Padre ha revelado su omnipotencia de la manera más misteriosa en el anonadamiento voluntario y en la Resurrección de su Hijo, por los cuales ha vencido el mal. Así, Cristo crucificado es \textquote{poder de Dios y sabiduría de Dios. Porque la necedad divina es más sabia que la sabiduría de los hombres, y la debilidad divina, más fuerte que la fuerza de los hombres} (\emph{1 Co} 2, 24-25). En la Resurrección y en la exaltación de Cristo es donde el Padre \textquote{desplegó el vigor de su fuerza} y manifestó \textquote{la soberana grandeza de su poder para con nosotros, los creyentes} (\emph{Ef} 1,19-22).

\textbf{Dios crea por sabiduría y por amor}

\textbf{295} Creemos que Dios creó el mundo según su sabiduría (cf. \emph{Sb} 9,9). Este no es producto de una necesidad cualquiera, de un destino ciego o del azar. Creemos que procede de la voluntad libre de Dios que ha querido hacer participar a las criaturas de su ser, de su sabiduría y de su bondad: \textquote{Porque tú has creado todas las cosas; por tu voluntad lo que no existía fue creado} (\emph{Ap} 4,11). \textquote{¡Cuán numerosas son tus obras, Señor! Todas las has hecho con sabiduría} (\emph{Sal} 104,24). \textquote{Bueno es el Señor para con todos, y sus ternuras sobre todas sus obras} (\emph{Sal} 145,9).

\textbf{Dios crea un mundo ordenado y bueno}

\textbf{299} Porque Dios crea con sabiduría, la creación está ordenada: \textquote{Tú todo lo dispusiste con medida, número y peso} (\emph{Sb} 11,20). Creada en y por el Verbo eterno, \textquote{imagen del Dios invisible} (\emph{Col} 1,15), la creación está destinada, dirigida al hombre, imagen de Dios (cf. \emph{Gn} 1,26), llamado a una relación personal con Dios. Nuestra inteligencia, participando en la luz del Entendimiento divino, puede entender lo que Dios nos dice por su creación (cf. \emph{Sal} 19,2-5), ciertamente no sin gran esfuerzo y en un espíritu de humildad y de respeto ante el Creador y su obra (cf. \emph{Jb} 42,3). Salida de la bondad divina, la creación participa en esa bondad (\textquote{Y vio Dios que era bueno [\ldots{}] muy bueno}: \emph{Gn} 1,4.10.12.18.21.31). Porque la creación es querida por Dios como un don dirigido al hombre, como una herencia que le es destinada y confiada. La Iglesia ha debido, en repetidas ocasiones, defender la bondad de la creación, comprendida la del mundo material (cf. San León Magno, c. \emph{Quam laudabiliter}, DS, 286; Concilio de Braga I: \emph{ibíd}., 455-463; Concilio de Letrán IV: \emph{ibíd.,} 800; Concilio de Florencia: \emph{ibíd.,}1333; Concilio Vaticano I: \emph{ibíd.,} 3002).

\textbf{474} Debido a su unión con la Sabiduría divina en la persona del Verbo encarnado, el conocimiento humano de Cristo gozaba en plenitud de la ciencia de los designios eternos que había venido a revelar (cf. \emph{Mc} 8,31; 9,31; 10, 33-34; 14,18-20. 26-30). Lo que reconoce ignorar en este campo (cf. \emph{Mc} 13,32), declara en otro lugar no tener misión de revelarlo (cf. \emph{Hch} 1, 7).

\textbf{\textquote{Alégrate, llena de gracia}}

\textbf{721} María, la Santísima Madre de Dios, la siempre Virgen, es la obra maestra de la Misión del Hijo y del Espíritu Santo en la Plenitud de los tiempos. Por primera vez en el designio de Salvación y porque su Espíritu la ha preparado, el Padre encuentra la Morada en donde su Hijo y su Espíritu pueden habitar entre los hombres. Por ello, los más bellos textos sobre la Sabiduría, la Tradición de la Iglesia los ha entendido frecuentemente con relación a María (cf. \emph{Pr} 8, 1-9, 6; \emph{Si} 24): María es cantada y representada en la Liturgia como el \textquote{Trono de la Sabiduría}.

En ella comienzan a manifestarse las \textquote{maravillas de Dios}, que el Espíritu va a realizar en Cristo y en la Iglesia:

\textbf{1831} Los siete \emph{dones} del Espíritu Santo son: sabiduría, inteligencia, consejo, fortaleza, ciencia, piedad y temor de Dios. Pertenecen en plenitud a Cristo, Hijo de David (cf. \emph{Is} 11, 1-2). Completan y llevan a su perfección las virtudes de quienes los reciben. Hacen a los fieles dóciles para obedecer con prontitud a las inspiraciones divinas.

\textquote{Tu espíritu bueno me guíe por una tierra llana} (\emph{Sal} 143,10).

\textquote{Todos los que son guiados por el Espíritu de Dios son hijos de Dios [\ldots{}] Y, si hijos, también herederos; herederos de Dios y coherederos de Cristo} (\emph{Rm} 8, 14.17).

Dios nos dona la Sabiduría

CEC 158, 283, 1303, 1831, 2500:

\textbf{158} \textquote{La fe \emph{trata de comprender}} (San Anselmo de Canterbury, \emph{Proslogion}, proemium: PL 153, 225A) es inherente a la fe que el creyente desee conocer mejor a aquel en quien ha puesto su fe, y comprender mejor lo que le ha sido revelado; un conocimiento más penetrante suscitará a su vez una fe mayor, cada vez más encendida de amor. La gracia de la fe abre \textquote{los ojos del corazón} (\emph{Ef} 1,18) para una inteligencia viva de los contenidos de la Revelación, es decir, del conjunto del designio de Dios y de los misterios de la fe, de su conexión entre sí y con Cristo, centro del Misterio revelado. Ahora bien, \textquote{para que la inteligencia de la Revelación sea más profunda, el mismo Espíritu Santo perfecciona constantemente la fe por medio de sus dones} (DV 5). Así, según el adagio de san Agustín (\emph{Sermo} 43,7,9: PL 38, 258), \textquote{creo para comprender y comprendo para creer mejor}.

\textbf{283} La cuestión sobre los orígenes del mundo y del hombre es objeto de numerosas investigaciones científicas que han enriquecido magníficamente nuestros conocimientos sobre la edad y las dimensiones del cosmos, el devenir de las formas vivientes, la aparición del hombre. Estos descubrimientos nos invitan a admirar más la grandeza del Creador, a darle gracias por todas sus obras y por la inteligencia y la sabiduría que da a los sabios e investigadores. Con Salomón, éstos pueden decir: \textquote{Fue él quien me concedió el conocimiento verdadero de cuanto existe, quien me dio a conocer la estructura del mundo y las propiedades de los elementos [\ldots{}] porque la que todo lo hizo, la Sabiduría, me lo enseñó} (\emph{Sb} 7,17-21).

\textbf{1303} Por este hecho, la Confirmación confiere crecimiento y profundidad a la gracia bautismal:

--- nos introduce más profundamente en la filiación divina que nos hace decir \textquote{\emph{Abbá}, Padre} (\emph{Rm} 8,15).;

--- nos une más firmemente a Cristo;

--- aumenta en nosotros los dones del Espíritu Santo;

--- hace más perfecto nuestro vínculo con la Iglesia (cf. LG 11);

--- nos concede una fuerza especial del Espíritu Santo para difundir y defender la fe mediante la palabra y las obras como verdaderos testigos de Cristo, para confesar valientemente el nombre de Cristo y para no sentir jamás vergüenza de la cruz (cf. DS 1319; LG 11,12):

\textquote{Recuerda, pues, que has recibido el signo espiritual, el Espíritu de sabiduría e inteligencia, el Espíritu de consejo y de fortaleza, el Espíritu de conocimiento y de piedad, el Espíritu de temor santo, y guarda lo que has recibido. Dios Padre te ha marcado con su signo, Cristo Señor te ha confirmado y ha puesto en tu corazón la prenda del Espíritu} (San Ambrosio, \emph{De mysteriis} 7,42).

\textbf{1831} Los siete \emph{dones} del Espíritu Santo son: sabiduría, inteligencia, consejo, fortaleza, ciencia, piedad y temor de Dios. Pertenecen en plenitud a Cristo, Hijo de David (cf. \emph{Is} 11, 1-2). Completan y llevan a su perfección las virtudes de quienes los reciben. Hacen a los fieles dóciles para obedecer con prontitud a las inspiraciones divinas.

\textquote{Tu espíritu bueno me guíe por una tierra llana} (\emph{Sal} 143,10).

\textquote{Todos los que son guiados por el Espíritu de Dios son hijos de Dios [\ldots{}] Y, si hijos, también herederos; herederos de Dios y coherederos de Cristo} (\emph{Rm} 8, 14.17)

\textbf{Verdad, belleza y arte sacro}

\textbf{2500} La práctica del bien va acompañada de un placer espiritual gratuito y de belleza moral. De igual modo, la verdad entraña el gozo y el esplendor de la belleza espiritual. La verdad es bella por sí misma. La verdad de la palabra, expresión racional del conocimiento de la realidad creada e increada, es necesaria al hombre dotado de inteligencia, pero la verdad puede también encontrar otras formas de expresión humana, complementarias, sobre todo cuando se trata de evocar lo que ella entraña de indecible, las profundidades del corazón humano, las elevaciones del alma, el Misterio de Dios. Antes de revelarse al hombre en palabras de verdad, Dios se revela a él, mediante el lenguaje universal de la Creación, obra de su Palabra, de su Sabiduría: el orden y la armonía del cosmos, que percibe tanto el niño como el hombre de ciencia, \textquote{pues por la grandeza y hermosura de las criaturas se llega, por analogía, a contemplar a su Autor} (\emph{Sb} 13, 5), \textquote{pues fue el Autor mismo de la belleza quien las creó} (\emph{Sb} 13, 3).

\textquote{La sabiduría es un hálito del poder de Dios, una emanación pura de la gloria del Omnipotente, por lo que nada manchado llega a alcanzarla. Es un reflejo de la luz eterna, un espejo sin mancha de la actividad de Dios, una imagen de su bondad} (\emph{Sb} 7, 25-26). \textquote{La sabiduría es, en efecto, más bella que el Sol, supera a todas las constelaciones; comparada con la luz, sale vencedora, porque a la luz sucede la noche, pero contra la sabiduría no prevalece la maldad} (\emph{Sb} 7, 29-30). \textquote{Yo me constituí en el amante de su belleza} (\emph{Sb} 8, 2).

He manchado mi cuerpo,

he ensuciado mi espíritu,

estoy todo lleno de llagas;

pero tú, oh Cristo médico,

cura mi espíritu y cuerpo con la penitencia,

báñame, purifícame, lávame:

déjame más puro que la nieve\ldots{}

Crucificado por todos,

has ofrecido tu cuerpo y tu sangre, oh Verbo:

el cuerpo para re-plasmarme,

la sangre para lavarme;

y has entregado el espíritu

para portarme, oh Cristo, a tu Engendrador.

Has obrado la salvación

en medio de la tierra.

Por tu voluntad

has sido clavado en el árbol de la Cruz

y el Edén que había sido cerrado, se ha abierto\ldots{}

Sea mi fuente bautismal

la sangre de tu costado,

y bebida el agua de remisión que ha brotado\ldots{}

y sea ungido, bebiendo como crisma y bebida,

tu vivificante palabra, oh Verbo.

(Canon de San Andrés de Creta).

\chapter{Epifanía del Señor}

\section{Lecturas}

PRIMERA LECTURA

Del libro del profeta Isaías 60, 1-6

La Gloria del Señor amanece sobre ti

¡Levántate y resplandece, Jerusalén,

porque llega tu luz;

la gloria del Señor amanece sobre ti!

Las tinieblas cubren la tierra,

la oscuridad los pueblos,

pero sobre ti amanecerá el Señor

y su gloria se verá sobre ti.

Caminarán los pueblos a tu luz,

los reyes al resplandor de tu aurora.

Levanta la vista en torno,

mira: todos esos se han reunido,

vienen hacia ti; llegan tus hijos desde lejos,

a tus hijas las traen en brazos.

Entonces lo verás y estarás radiante;

tu corazón se asombrará, se ensanchará,

porque la opulencia del mar se vuelca sobre ti,

y a ti llegan las riquezas de los pueblos.

Te cubrirá una multitud de camellos,

dromedarios de Madián y de Efá.

Todos los de Saba llegan trayendo oro e incienso,

y proclaman las alabanzas del Señor.

SALMO RESPONSORIAL

Salmo 71, 1-2. 7-8. 10-13

Se postrarán ante ti, Señor, todos los pueblos de la tierra

℣. Dios mío, confía tu juicio al rey,

tu justicia al hijo de reyes,

para que rija a tu pueblo con justicia,

a tus humildes con rectitud. ℟.

℣. En sus días florezca la justicia

y la paz hasta que falte la luna;

domine de mar a mar,

del Gran Río al confín de la tierra. ℟.

℣. Los reyes de Tarsis y de las islas

le paguen tributo.

Los reyes de Saba y de Arabia

le ofrezcan sus dones;

postrense ante él todos los reyes,

y sirvanle todos los pueblos. ℟.

℣. Él librará al pobre que clamaba,

al afligido que no tenía protector;

él se apiadará del pobre y del indigente,

y salvará la vida de los pobres. ℟.

SEGUNDA LECTURA

De la carta del apóstol san Pablo a los Efesios 3, 2-3a. 5-6

Ahora ha sido revelado que también los gentiles son coherederos de la
promesa

Hermanos:

Habéis oído hablar de la distribución de la gracia de Dios que se me ha
dado en favor de vosotros, los gentiles.

Ya que se me dio a conocer por revelación el misterio, que no había sido
manifestado a los hombres en otros tiempos, como ha sido revelado ahora
por el Espíritu a sus santos apóstoles y profetas: que también los
gentiles son coherederos, miembros del mismo cuerpo, y partícipes de la
misma promesa en Jesucristo, por el Evangelio.

EVANGELIO

Del Santo Evangelio según san Mateo 2, 1-12

Venimos a adorar al Rey

Habiendo nacido Jesús en Belén de Judea en tiempos del rey Herodes, unos
magos de Oriente se presentaron en Jerusalén preguntando:

«¿Dónde está el Rey de los judíos que ha nacido? Porque hemos visto
salir su estrella y venimos a adorarlo».

Al enterarse el rey Herodes, se sobresaltó y toda Jerusalén con él;
convocó a los sumos sacerdotes y a los escribas del país, y les preguntó
dónde tenía que nacer el Mesías.

Ellos le contestaron:

«En Belén de Judea, porque así lo ha escrito el profeta:

``Y tú, Belén, tierra de Judá,

no eres ni mucho menos la última

de las poblaciones de Judá,

pues de ti saldrá un jefe

que pastoreará a mi pueblo Israel''».

Entonces Herodes llamó en secreto a los magos para que le precisaran el
tiempo en que había aparecido la estrella, y los mandó a Belén,
diciéndoles:

«Id y averiguad cuidadosamente qué hay del niño y, cuando lo encontréis,
avisadme, para ir yo también a adorarlo».

Ellos, después de oír al rey, se pusieron en camino y, de pronto, la
estrella que habían visto salir comenzó a guiarlos hasta que vino a
pararse encima de donde estaba el niño.

Al ver la estrella, se llenaron de inmensa alegría. Entraron en la casa,
vieron al niño con María, su madre, y cayendo de rodillas lo adoraron;
después, abriendo sus cofres, le ofrecieron regalos: oro, incienso y
mirra.

Y habiendo recibido en sueños un oráculo, para que no volvieran a
Herodes, se retiraron a su tierra por otro
camino.

\section{Comentarios Patrísticos}

\subsection{San León Magno, papa}

Dios ha manifestado su salvación en todo el mundo

Sermón en la Epifanía del Señor 3, 1-3. 5: PL 54, 240-244

La misericordiosa providencia de Dios, que ya había decidido venir en los últimos tiempos en ayuda del mundo que perecía, determinó de antemano la salvación de todos los pueblos en Cristo.

De estos pueblos se trataba en la descendencia innumerable que fue en otro tiempo prometida al santo patriarca Abrahán, descendencia que no sería engendrada por una semilla de carne, sino por fecundidad de la fe, descendencia comparada a la multitud de las estrellas, para que de este modo el padre de todas las naciones esperara una posteridad no terrestre, sino celeste.

Así pues, que todos los pueblos vengan a incorporarse a la familia de los patriarcas, y que los hijos de la promesa reciban la bendición de la descendencia de Abrahán, a la cual renuncian los hijos según la carne. Que todas las naciones, en la persona de los tres Magos, adoren al Autor del universo, y que Dios sea conocido, no ya sólo en Judea, sino también en el mundo entero, para que por doquier \emph{sea grande su nombre en Israel}.

Instruidos en estos misterios de la gracia divina, queridos míos, celebremos con gozo espiritual el día que es el de nuestras primicias y aquél en que comenzó la salvación de los paganos. Demos gracias al Dios misericordioso, quien, según palabras del Apóstol, \emph{nos ha hecho capaces de compartir la herencia del pueblo santo en la luz; él nos ha sacado del dominio de las tinieblas y nos ha trasladado al reino de su Hijo querido}. Porque, como profetizó Isaías, \emph{el pueblo que caminaba en tinieblas vio una luz grande; habitaban en tierra de sombras, y una luz les brilló}. También a propósito de ellos dice el propio Isaías al Señor: \emph{Naciones que no te conocían te invocarán, un pueblo que no te conocía correrá hacia ti}.

Abrahán vio \emph{este día, y se llenó de alegría,} cuando supo que sus hijos según la fe serían benditos en su descendencia, a saber, en Cristo, y él se vio a sí mismo, por su fe, como futuro padre de todos los pueblos, \emph{dando gloria a Dios, al persuadirse de que Dios es capaz de hacer lo que promete}.

También David anunciaba este día en los salmos cuando decía: Todos los pueblos vendrán a postrarse en tu presencia, Señor; bendecirán tu nombre; y también: El Señor da a conocer su victoria, revela a las naciones su justicia.

Esto se ha realizado, lo sabemos, en el hecho de que tres magos, llamados de su lejano país, fueron conducidos por una estrella para conocer y adorar al Rey del cielo y de la tierra. La docilidad de los magos a esta estrella nos indica el modo de nuestra obediencia, para que, en la medida de nuestras posibilidades, seamos servidores de esa gracia que llama a todos los hombres a Cristo.

Animados por este celo, debéis aplicaros, queridos míos, a seros útiles los unos a los otros, a fin de que brilléis como hijos de la luz en el reino de Dios, al cual se llega gracias a la fe recta y a las buenas obras; por nuestro Señor Jesucristo que, con Dios Padre y el Espíritu Santo, vive y reina por los siglos de los siglos. Amén.

\subsection{San Francisco de Sales, obispo}

Regalemos lo más grande al Niño-Dios.

Sermón VIII, 38.

\textquote{Unos magos, venidos de Oriente, llegaron a Jerusalén} (Mt 2,2).

Es una gran fiesta, en la que celebramos que la Iglesia de los Gentiles es aceptada por Cristo y recibida por Cristo. Sí, es una gran fiesta porque los gentiles llegan a Cristo y a la Casa del Pan.

La Epifanía es el día de los dones. Nunca ha recibido Cristo regalos más espléndidos y ahí tenemos la manera de ofrecer nuestros presentes a Dios. Los Magos nos lo pueden enseñar, ya que el primer acto de cada clase sirve de tipo a lo demás. Veamos, pues, las circunstancias: ¿Quién? ¿Qué? ¿A quién? ¿Por qué? ¿Cómo?

¿Quién? Unos Reyes sabios. Antes de haber recibido la fe, ya creían. Reyes piadosos, que observaban las estrellas siguiendo la profecía de Balaam; su devoción se demuestra al dejar sus reinos y al acudir y presentarse intrépidamente al rey Herodes y confesarle ingenuamente su fe.

¿Qué? Oro, incienso y mirra. Las opiniones de los doctores están divididas cuando explican la razón de estos presentes. Strabus dice que trajeron de lo que producía su país de Arabia. Todo agrada a Dios: Abel le daba de sus rebaños y el que no tenía sino una piel de cabra, también podía ofrecérsela. Honra al Señor con tus bienes.

Hay quienes ofrecen al Señor lo que no poseen. Hijo mío, ¿por qué no eres más devoto? Lo seré en mi ancianidad. Pero, ¿sabes tú que llegarás a viejo? Otro dice: Si yo fuese capuchino, ofrecería sacrificios al Señor. Honra al Señor con lo que tienes. Si yo fuese rico\ldots{} yo daría\ldots{} Honra al Señor con tu pobreza. Si yo fuera santo\ldots{} Honra al Señor con tu paciencia, si yo fuera doctor\ldots{}, honra al Señor con tu sencillez\ldots{}

De lo que tienes, el valor de tu ofrenda se mide en relación con lo que posees. San Agustín dice que los Magos le trajeron oro como a rey; incienso como a Dios; mirra como a hombre. ¿A quién? ¡A Cristo nuestro Señor! ¿Por qué? ¡Hemos venido a adorar al Señor! ¿Cómo? ¡Se postraron y le adoraron!

Y no digamos que no tenemos nada muy grande para regalarle. Nada hay suficientemente digno de Dios. Debéis decir: \textquote{Yo quiero, Divino Niño, darte el único bien que poseo: yo mismo, y te ruego que aceptes este don}. Y Él nos responderá: \textquote{Hijo mío, tu regalo no es pequeño sino en tu propia estima}.

La estrella vino a pararse encima de donde estaba el niño. Por lo cual, los magos, al ver la estrella, se llenaron de inmensa alegría. Recibamos también nosotros esa inmensa alegría en nuestros corazones. Es la alegría que los ángeles anuncian a los pastores. Adoremos con los Magos, demos gloria con los pastores, dancemos con los ángeles. \emph{Porque hoy ha nacido un Salvador: el Mesías, el Señor. El Señor es Dios: él nos ilumina,} pero no en la condición divina, para atemorizar nuestra debilidad, sino en la condición de esclavo, para gratificar con la libertad a quienes gemían bajo la esclavitud. ¿Quién es tan insensible, quién tan ingrato, que no se alegre, que no exulte, que no se recree con tales noticias? Esta es una fiesta común a toda la creación: se le otorgan al mundo dones celestiales, el arcángel es enviado a Zacarías y a María, se forma un coro de ángeles, que cantan: \emph{Gloria a Dios en el cielo, y en la tierra, paz a los hombres que Dios ama}.

Las estrellas se descuelgan del cielo, unos Magos abandonan la paganía, la tierra lo recibe en una gruta. Que todos aporten algo, que ningún hombre se muestre desagradecido. Festejemos la salvación del mundo, celebremos el día natalicio de la naturaleza humana. Hoy ha quedado cancelada la deuda de Adán. Ya no se dirá en adelante: \emph{Eres polvo y al polvo volverás,} sino: \textquote{Unido al que viene del cielo, serás admitido en el cielo}. Ya no se dirá más: \emph{Parirás hijos con dolor,} pues es dichosa la que dio a luz al Emmanuel y los pechos que le alimentaron. Precisamente por esto \emph{un niño nos ha nacido, un hijo se nos ha dado: lleva a hombros el principado}.

(\textbf{San Basilio Magno}, \emph{Homilía} sobre la generación de Cristo: PG 31, 1471-1475).

\section{Homilías}

\subsection{San Juan Pablo II, papa}

\subsubsection{Homilía: Guiados por la luz de la fe}

Ordenación episcopal de 11 nuevos obispos.

\emph{Basílica de San Pedro}. Martes 6 de enero de 1981.

1. \textquote{Levántate, brilla, Jerusalén, que llega tu luz; la gloria del Señor amanece sobre ti!} \emph{(Is} 60, 1). Con estas palabras del \textbf{Profeta Isaías} la liturgia de hoy anuncia la celebración de una gran fiesta: \emph{la solemnidad de la Epifanía del Señor,} que es la culminación de la \emph{fiesta de Navidad;} del nacimiento de Dios

Las palabras del Profeta se dirigen a Jerusalén, a la ciudad del Pueblo de Dios, a la ciudad de la elección divina. En esta ciudad la Epifanía debía alcanzar su cénit en los días del misterio pascual del Redentor.

Sin embargo, por el momento, el Redentor es todavía un \emph{niño pequeño}. Yace en una pobre gruta cerca de Belén, y la gruta sirve de refugio para los animales. Allí encontró el primer albergue para Sí mismo sobre esta tierra. Allí le rodearon el amor de la Madre y la solicitud de José de Nazaret. Y allí tuvo lugar también \emph{el comienzo de la Epifanía:} de esa gran luz que debía penetrar los corazones, guiándolos por el camino de la fe hacia Dios, con el cual solamente por esta senda puede encontrarse el hombre: el hombre viviente con el Dios viviente.

Hoy en este camino de la fe vemos a los tres nuevos hombres \emph{que vienen de Oriente,} de fuera de Israel. Son hombres sabios y poderosos, que vienen a Belén conducidos por la estrella en el firmamento celeste y por la luz interna de la fe en la profundidad de sus corazones.

2. En este día, tan solemne, tan elocuente, os presentáis aquí \emph{vosotros,} venerados y queridos hijos, que por el acto de la ordenación debéis venir a ser hermanos nuestros en el Episcopado, en el servicio apostólico de la Iglesia. \emph{Os saludo} cordialmente en esta basílica, la cual se trasladó la luz de la Jerusalén mesiánica juntamente con la persona del Apóstol Pedro, que vino aquí guiado por el Espíritu Santo de acuerdo con la voluntad de Cristo.

Aquí, en este lugar, medito con vosotros las palabras de la liturgia de hoy, en las que se manifiestan la luz de la Epifanía \emph{y la misión nacida} en los corazones de los hombres \emph{por la fe en Jesucristo}. Que esta luz resplandezca sobre vosotros de modo particular en el día de hoy, que brille continuamente en los caminos de vuestra vida y de vuestro ministerio. Que esta luz os guíe ---como la estrella de los Magos--- y os ayude a guiar a los demás de acuerdo con la sustancia de vuestra vocación en el Episcopado.

\textquote{Los obispos ---ha recordado el Concilio Vaticano II--- como sucesores de los Apóstoles, reciben del Señor, a quien ha sido dado todo poder en el cielo y en la tierra, la misión de enseñar a todas las gentes y de predicar el Evangelio a toda criatura, a fin de que todos los hombres consigan la salvación por medio de la fe, del bautismo y del cumplimiento de los mandamientos (cf. \emph{Mt} 28, 18-20; \emph{Mc} 16, 15-16; \emph{Act} 26. 17 ss.). Para cumplir esta misión, Cristo Señor prometió a los Apóstoles el Espíritu Santo, y lo envió desde el cielo el día de Pentecostés, para que, confortados con su virtud, fuesen sus testigos hasta los confines de la tierra ante las gentes, los pueblos y los reyes (cf. \emph{Act} 1, 8; 2, 1 ss.; 9, 15). Este encargo que el Señor confió a los Pastores de su pueblo es un verdadero servicio, que en la Sagrada Escritura se llama con toda propiedad `diaconía', o sea, ministerio (cf. \emph{Act} 1, 17 y 25; 21, 19; \emph{Rom} 11, 13; \emph{1 Tim} 1, 12)} \emph{(Lumen gentium,} 24).

3. Debéis ser, queridos hermanos, confesores de la fe, testigos de la fe, maestros de la fe. \emph{Debéis ser los hombres de la fe}. Contemplad este maravilloso acontecimiento que la solemnidad de hoy presenta a los ojos de nuestra alma.

Un día, después de la venida del Espíritu Santo, se realizó en la comunidad de la Iglesia primitiva un gran cambio. El protagonista de este cambio fue \emph{Pablo de Tarso}. Escuchemos cómo habla en la liturgia de hoy: \textquote{Se me dio a conocer por revelación el misterio\ldots{}: que también los gentiles son coherederos, miembros del mismo cuerpo y partícipes de la promesa en Jesucristo, por el Evangelio} (\emph{Ef} 3, 3. 6).

\emph{Este misterio,} en virtud del cual Pablo, y luego los otros Apóstoles, llevaron la luz del Evangelio más allá de las fronteras del Pueblo de la Antigua Alianza, este misterio \emph{se anuncia ya hoy}. Ya en el momento del nacimiento del Mesías: en su pesebre de Belén, en la coparticipación de la promesa que El ha venido a realizar, son llamados con la luz de la estrella y con la luz de la fe tres hombres que provienen de fuera de Israel.

Estos tres hombres hablan de todos aquellos que deben seguir la misma luz mesiánica, tanto de Oriente como de Occidente, tanto del Norte como del Sur, para encontrar juntamente \textquote{con Abraham, Isaac y Jacob} la promesa del Dios viviente.

Esta \emph{promesa se realiza} hoy ante los ojos de los Magos, tal como se realizó en la noche del nacimiento de Dios ante los ojos de los pastores, cerca de Belén.

¡Oh, cuánto nos dicen hoy las palabras del \textbf{Profeta}, que interpela a Jerusalén: \textquote{Levanta la vista en torno, mira\ldots{} tu corazón se asombrará, se ensanchará} (\emph{Is} 60, 4-5).

4. Queridos hijos y amados hermanos:

Debéis convertiros en testigos singulares de la alegría que siente hoy la Jerusalén del Señor ¡Deben \emph{palpitar y dilatarse vuestros corazones} ante el misterio que contempláis! ¡Ante la luz a la que debéis servir!

¡Qué grande es la fe de los Magos! ¡Qué seguros están de la luz que el Espíritu del Señor encendió en sus corazones! Con cuánta tenacidad la siguen. Con cuánta coherencia buscan al Mesías recién nacido. Y cuando finalmente llegaron a la meta, \textquote{\ldots{}se llenaron de inmensa alegría. Entraron en la casa, vieron al Niño con María, su Madre, y, \emph{cayendo de rodillas, lo adoraron;} después, abriendo sus cofres, le ofrecieron regalos: oro. incienso y mirra} \emph{(Mt} 2, 10-11).

La luz de la fe les permitió escrutar todas las incógnitas. Los caminos incógnitos. las circunstancias incógnitas. Como cuando se hallaron ante el recién Nacido, un recién nacido humano que no tenía casa. Ellos se dieron cuenta de la miseria del lugar. ¡Qué contraste con su posición de hombres instruidos y socialmente influyentes! Y, sin embargo, \textquote{cayendo de rodillas, lo adoraron} (cf. \emph{Mt} 2, 11).

Si este Niño, Cristo, hubiese podido hablar entonces, tal como habló después muchas veces, les debería haber dicho: ¡Hombres, qué grande es vuestra fe! Palabras semejantes a las que una vez, más tarde, escuchó la mujer cananea: \textquote{¡Grande es tu fe!} (cf. \emph{Mt} 15, 28).

5. Queridos hermanos: Dentro de poco, también vosotros os inclinaréis profundamente, y os \emph{postraréis,} y tendidos sobre el pavimento de esta basílica, prepararéis vuestros corazones para la nueva venida del Espíritu Santo, para recibir sus dones divinos. Son \emph{los mismos dones} que iluminaron y robustecieron a los Magos en el camino de Belén, en el encuentro con el recién Nacido y, luego, en el camino de retorno y en toda su vida.

A estos \textbf{dones divinos} ellos respondieron con un don: el oro, el incienso y la mirra, realidades que tienen también su significado simbólico. Teniendo presente ese significado, ofreced hoy vuestros dones, a vosotros mismos en don, y estad dispuestos a ofrecer, durante toda vuestra vida el amor, la oración, el sufrimiento.

Y luego, \emph{levantaos, dirigíos por el camino} por el que os conducirá el Señor, guiándoos por las sendas de vuestra misión y de vuestro ministerio.

¡Levantaos, robusteceos en la fe! Como testigos del ministerio de Dios. Como siervos del Evangelio y dispensadores de la potencia de Cristo. \emph{Y caminad a la luz de la Epifanía,} guiando a los otros a la fe y fortificando en la fe a todos los que encontréis.

Que os acompañe siempre la sabiduría, la humildad y la valentía de los Magos de Oriente.

\subsubsection{Homilía: La luz de la razón y de la fe}

Basílica de San Pedro. 6 de enero de 1999.

1. \textquote{La luz brilla en las tinieblas, pero las tinieblas no la acogieron} (Jn 1, 5).

Toda la liturgia habla hoy de la \emph{\textbf{luz de Cristo}}, de la luz que se encendió en la noche santa. La misma luz que guió a los pastores hasta el portal de Belén indicó el camino, el día de la Epifanía, a los Magos que fueron desde Oriente para adorar al Rey de los judíos, y resplandece para todos los hombres y todos los pueblos que anhelan encontrar a Dios.

En su búsqueda espiritual, el ser humano ya dispone naturalmente de una luz que lo guía: es la razón, gracias a la cual puede orientarse, aunque a tientas (cf. \emph{Hch} 17, 27), hacia su Creador. Pero, dado que es fácil perder el camino, Dios mismo vino en su ayuda con la luz de la revelación, que alcanzó su plenitud en la encarnación del Verbo, Palabra eterna de verdad.

La Epifanía celebra la aparición en el mundo de esta luz divina, con la que Dios salió al encuentro de la débil luz de la razón humana. Así, en la solemnidad de hoy, se propone la íntima relación que existe entre la razón y la fe, las dos alas de que dispone el espíritu humano para elevarse hacia la contemplación de la verdad, como recordé en la reciente encíclica \emph{Fides et ratio}.

2. Cristo no es sólo luz que ilumina el camino del hombre. \emph{También se ha hecho camino} para sus pasos inciertos hacia Dios, fuente de vida. Un día dijo a los Apóstoles: \textquote{Yo soy el camino, la verdad y la vida. Nadie va al Padre sino por mí. Si me conocéis a mí, conoceréis también a mi Padre; desde ahora lo conocéis y lo habéis visto} (\emph{Jn} 14, 6-7). Y, ante la objeción de Felipe, añadió: \textquote{El que me ha visto a mí, ha visto al Padre. (\ldots{}) Yo estoy en el Padre y el Padre está en mí} (\emph{Jn} 14, 9. 11). \emph{La epifanía del Hijo es la epifanía del Padre}.

¿No es éste, en definitiva, el objetivo de la venida de Cristo al mundo? Él mismo afirmó que había venido para \textquote{dar a conocer al Padre}, para \textquote{explicar} a los hombres quién es Dios y para revelar su rostro, su \textquote{nombre} (cf. \emph{Jn} 17, 6). La vida eterna consiste en el encuentro con el Padre (cf. \emph{Jn} 17, 3). Por eso, ¡cuán oportuna es esta reflexión, especialmente durante el año dedicado al Padre!

La Iglesia prolonga en los siglos la misión de su Señor: su compromiso principal consiste en dar a conocer a todos los hombres el rostro del Padre, reflejando la luz de Cristo, \emph{lumen gentium}, luz de amor, de verdad y de paz. Para esto el divino Maestro envió al mundo a los Apóstoles, y envía continuamente, con el mismo Espíritu, a los obispos, sus sucesores\ldots{}

{[}\ldots{}{]}

4. El mundo, en el umbral del tercer milenio, tiene gran necesidad de \emph{experimentar la bondad divina}; de sentir el amor de Dios a toda persona.

También a nuestra época se puede aplicar el oráculo del \textbf{profeta Isaías}, que acabamos de escuchar: \textquote{La oscuridad cubre la tierra, y espesa nube a los pueblos, mas sobre ti amanece el Señor y su gloria sobre ti aparece} (Is 60, 2-3). En el paso, por decirlo así, del segundo al tercer milenio, la Iglesia está llamada a revestirse de luz (cf. Is 60, 1), para resplandecer como una ciudad situada en la cima de un monte: la Iglesia no puede permanecer oculta (cf. \emph{Mt} 5, 14), porque los hombres necesitan recoger su mensaje de luz y esperanza, y glorificar al Padre que está en los cielos (cf. \emph{Mt} 5, 16).

Conscientes de esta tarea apostólica y misionera, que compete a todo el pueblo cristiano, pero especialmente a cuantos el Espíritu Santo ha puesto como obispos para pastorear la Iglesia de Dios (cf. \emph{Hch} 20, 28), vamos como peregrinos a Belén, a fin de unirnos a los Magos de Oriente, mientras ofrecen dones al Rey recién nacido.

Pero el verdadero don es él: Jesús, el don de Dios al mundo. Debemos acogerlo a él, para llevarlo a cuantos encontremos en nuestro camino. Él es para todos la epifanía, la manifestación de Dios, \emph{esperanza del hombre}, de Dios, \emph{liberación del hombre}, de Dios, \emph{salvación del hombre}.

Cristo nació en Belén por nosotros.

Venid, adorémoslo. Amén.

\subsubsection{Homilía: Luz que guía en la noche}

Ordenación episcopal de diez presbíteros. Domingo 6 de enero de 2002.

1. \textquote{\emph{Lumen gentium (\ldots{}) Christus}, Cristo es la luz de los pueblos} (\emph{Lumen gentium}, 1).

El \emph{tema de la luz} domina las solemnidades de la Navidad y de la Epifanía, que antiguamente -y aún hoy en Oriente- estaban unidas en una sola y gran \textquote{fiesta de la luz}. En el clima sugestivo de la Noche santa apareció la luz; nació Cristo, \textquote{luz de los pueblos}. Él es el \textquote{sol que nace de lo alto} (\emph{Lc} 1, 78), el sol que vino al mundo para disipar las tinieblas del mal e inundarlo con el esplendor del amor divino. El evangelista san Juan escribe: \textquote{La luz verdadera, viniendo a este mundo, ilumina a todo hombre} (\emph{Jn} 1, 9).

\textquote{\emph{Deus lux est} --- Dios es luz}, recuerda también san Juan, sintetizando no una teoría gnóstica, sino \textquote{el mensaje que hemos oído de él} (\emph{1 Jn} 1, 5), es decir, de Jesús. En el evangelio recoge las palabras que oyó de los labios del Maestro: \textquote{Yo soy la luz del mundo; el que me siga no caminará en la oscuridad, sino que tendrá la luz de la vida} (\emph{Jn} 8, 12).

Al encarnarse, el Hijo de Dios \emph{se manifestó como luz}. No sólo luz externa, en la historia del mundo, sino también \emph{dentro del hombre}, en su historia personal. Se hizo uno de nosotros, dando sentido y nuevo valor a nuestra existencia terrena. De este modo, respetando plenamente la libertad humana, Cristo se convirtió en \textquote{\emph{lux mundi}, la luz del mundo}. Luz que brilla en las tinieblas (cf. \emph{Jn} 1, 5).

2. Hoy, solemnidad de la Epifanía, que significa \textquote{manifestación}, se propone de nuevo con vigor el tema de la luz. Hoy el Mesías, que se manifestó en Belén a humildes pastores de la región, sigue revelándose como luz de los pueblos de todos los tiempos y de todos los lugares. Para los Magos, que acudieron de Oriente a adorarlo, la luz del \textquote{rey de los judíos que ha nacido} (\emph{Mt} 2, 2) toma la forma de un astro celeste, tan brillante que atrae su mirada y los guía hasta Jerusalén. Así, les hace seguir los indicios de las antiguas profecías mesiánicas: \textquote{De Jacob avanza una estrella, un cetro surge de Israel\ldots{}} (\emph{Nm} 24, 17).

¡Cuán sugestivo es el \emph{símbolo de la estrella}, que aparece en toda la iconografía de la Navidad y de la Epifanía! Aún hoy evoca profundos sentimientos, aunque como tantos otros signos de lo sagrado, a veces corre el riesgo de quedar desvirtuado por el uso consumista que se hace de él. Sin embargo, la estrella que contemplamos en el belén, situada en su contexto original, \emph{también habla a la mente y al corazón del hombre del tercer milenio}. Habla \emph{al hombre secularizado}, suscitando nuevamente en él la nostalgia de su condición de viandante que busca la verdad y \emph{anhela lo absoluto}. La etimología misma del verbo desear -en latín, \emph{desiderare}-- evoca la experiencia de los navegantes, los cuales se orientan en la noche observando los astros, que en latín se llaman \emph{sidera}.

3. ¿Quién no siente la necesidad de una \textquote{estrella} que lo guíe a lo largo de su camino en la tierra? Sienten esta necesidad tanto las personas como las naciones. A fin de satisfacer este anhelo de salvación universal, el Señor se eligió un pueblo que fuera estrella orientadora para \textquote{todos los linajes de la tierra} (\emph{Gn} 12, 3). Con la encarnación de su Hijo, Dios extendió luego su elección a todos los demás pueblos, sin distinción de raza y cultura. Así nació la Iglesia, formada por hombres y mujeres que, \textquote{reunidos en Cristo, son guiados por el Espíritu Santo en su peregrinar hacia el reino del Padre y han recibido el mensaje de la salvación para proponérselo a todos} (\emph{Gaudium et spes}, 1).

Por tanto, para toda la comunidad eclesial resuena el oráculo del \textbf{profeta Isaías,} que hemos escuchado en la \textbf{primera lectura}: \textquote{¡Levántate, brilla (\ldots{}), que llega tu luz; la gloria del Señor amanece sobre ti! (\ldots{}) Y caminarán los pueblos a tu luz; los reyes al resplandor de tu aurora} (\emph{Is} 60, 1. 3)\ldots{}

5. {[}\ldots{}{]} Os repito las palabras del Redentor: \textquote{\emph{Duc in altum}. \textquote{No tengáis miedo a las tinieblas del mundo, porque quien os envía es la \textquote{luz del mundo} (\emph{Jn} 8, 12), \textquote{el lucero radiante del alba}} (\emph{Ap} 22, 16).

Y tú, Jesús, que un día dijiste a tus discípulos: \textquote{Vosotros sois la luz del mundo} (\emph{Mt} 5, 14), haz que \emph{el testimonio evangélico} de estos hermanos nuestros \emph{resplandezca ante los hombres de nuestro tiempo}. Haz eficaz su misión para que cuantos confíes a su cuidado pastoral glorifiquen siempre al Padre que está en los cielos (cf. \emph{Mt} 5, 16).

Madre del Verbo encarnado, Virgen fiel, conserva a estos nuevos obispos bajo tu constante protección, para que sean misioneros valientes del Evangelio; \emph{fiel reflejo del amor de Cristo}, luz de los pueblos y esperanza del mundo.

\subsection{Benedicto XVI, papa}

\subsubsection{Homilía: Una esperanza mayor}

Basílica de San Pedro. Domingo 6 de enero del 2008.

Celebramos hoy a Cristo, luz del mundo, y su manifestación a las naciones. En el día de Navidad el mensaje de la liturgia era: \textquote{\emph{Hodie descendit lux magna super terram}} --- \textquote{Hoy desciende una gran luz a la tierra} (\emph{Misal romano}). En Belén, esta \textquote{gran luz} se presentó a un pequeño grupo de personas, a un minúsculo \textquote{resto de Israel}: a la Virgen María, a su esposo José, y a algunos pastores. Una luz humilde, según el estilo del verdadero Dios. Una llamita encendida en la noche: un frágil niño recién nacido, que da vagidos en el silencio del mundo\ldots{} Pero en torno a ese nacimiento oculto y desconocido resonaba el himno de alabanza de los coros celestiales, que cantaban gloria y paz (cf. \emph{Lc} 2, 13-14).

Así, aquella luz, aun siendo pequeña cuando apareció en la tierra, se proyectaba con fuerza en los cielos. El nacimiento del Rey de los judíos había sido anunciado por una estrella que se podía ver desde muy lejos. Este fue el testimonio de \textquote{algunos Magos} que llegaron desde Oriente a Jerusalén poco después del nacimiento de Jesús, en tiempos del rey Herodes (cf. \emph{Mt} 2, 1-2).

Una vez más, se comunican y se responden el cielo y la tierra, el cosmos y la historia. Las antiguas profecías se cumplen con el lenguaje de los astros. \textquote{De Jacob avanza una estrella, un cetro surge de Israel} (\emph{Nm} 24, 17), había anunciado el vidente pagano Balaam, llamado a maldecir al pueblo de Israel y que, al contrario, lo bendijo porque, como Dios le reveló, \textquote{ese pueblo es bendito} (\emph{Nm} 22, 12).

Cromacio de Aquileya, en su \emph{Comentario al evangelio de san Mateo}, relacionando a Balaam con los Magos, escribe: \textquote{Aquel profetizó que Cristo vendría; estos lo vieron con los ojos de la fe}. Y añade una observación importante: \textquote{Todos vieron la estrella, pero no todos comprendieron su sentido. Del mismo modo, nuestro Señor y Salvador nació para todos, pero no todos lo acogieron} (\emph{ib}., 4, 1-2). Este es, en la perspectiva histórica, el significado del símbolo de la luz aplicado al nacimiento de Cristo: expresa la bendición especial de Dios en favor de la descendencia de Abraham, destinada a extenderse a todos los pueblos de la tierra.

De este modo, el acontecimiento evangélico que recordamos en la Epifanía, la visita de los Magos al Niño Jesús en Belén, nos remite a los orígenes de la historia del pueblo de Dios, es decir, a la llamada de Abraham, que encontramos en el capítulo 12 del libro del Génesis. Los primeros once capítulos son como grandes cuadros que responden a algunas preguntas fundamentales de la humanidad: ¿Cuál es el origen del universo y del género humano? ¿De dónde viene el mal? ¿Por qué hay diversas lenguas y civilizaciones?

Entre los relatos iniciales de la Biblia aparece una primera \textquote{alianza}, establecida por Dios con Noé, después del diluvio. Se trata de una alianza universal, que atañe a toda la humanidad: el nuevo pacto con la familia de Noé es, a la vez, un pacto con \textquote{toda carne} (cf. \emph{Gn} 9, 15). Luego, antes de la llamada de Abraham, se encuentra otro gran cuadro, muy importante para comprender el sentido de la Epifanía: el de la torre de Babel. El texto sagrado afirma que en los orígenes \textquote{todo el mundo tenía un mismo lenguaje e idénticas palabras} (\emph{Gn} 11, 1). Después los hombres dijeron: \textquote{Ea, vamos a edificarnos una ciudad y una torre con la cúspide en los cielos, y hagámonos famosos, por si nos desperdigamos por toda la haz de la tierra} (\emph{Gn} 11, 4). La consecuencia de este pecado de orgullo, análogo al de Adán y Eva, fue la confusión de las lenguas y la dispersión de la humanidad por toda la tierra (cf. \emph{Gn} 11, 7-8). Esto es lo que significa \textquote{Babel}; fue una especie de maldición, semejante a la expulsión del paraíso terrenal.

En este punto se inicia la historia de la bendición, con la llamada de Abraham: comienza el gran plan de Dios para hacer de la humanidad una familia, mediante la alianza con un pueblo nuevo, elegido por él para que sea una bendición en medio de todas las naciones (cf. \emph{Gn} 12, 1-3). Este plan divino se sigue realizando todavía y tuvo su momento culminante en el misterio de Cristo. Desde entonces se iniciaron \textquote{los últimos tiempos}, en el sentido de que el plan fue plenamente revelado y realizado en Cristo, pero debe ser acogido por la historia humana, que sigue siendo siempre historia de fidelidad por parte de Dios y, lamentablemente, también de infidelidad por parte de nosotros los hombres.

La Iglesia misma, depositaria de la bendición, es santa y a la vez está compuesta de pecadores; está marcada por la tensión entre el \textquote{ya} y el \textquote{todavía no}. En la plenitud de los tiempos Jesucristo vino a establecer la alianza: él mismo, verdadero Dios y verdadero hombre, es el Sacramento de la fidelidad de Dios a su plan de salvación para la humanidad entera, para todos nosotros.

La llegada de los Magos de Oriente a Belén, para adorar al Mesías recién nacido, es la señal de la manifestación del Rey universal a los pueblos y a todos los hombres que buscan la verdad. Es el inicio de un movimiento opuesto al de Babel: de la confusión a la comprensión, de la dispersión a la reconciliación. Por consiguiente, descubrimos un vínculo entre la Epifanía y Pentecostés: si el nacimiento de Cristo, la Cabeza, es también el nacimiento de la Iglesia, su cuerpo, en los Magos vemos a los pueblos que se agregan al resto de Israel, anunciando la gran señal de la \textquote{Iglesia políglota} realizada por el Espíritu Santo cincuenta días después de la Pascua.

El amor fiel y tenaz de Dios, que mantiene siempre su alianza de generación en generación. Este es el \textquote{misterio} del que habla \textbf{san Pablo} en sus cartas, también en el pasaje de la \textbf{carta a los Efesios} que se acaba de proclamar. El Apóstol afirma que este misterio le \textquote{fue comunicado por una revelación} (\emph{Ef} 3, 3) y él se encargó de darlo a conocer.

Este \textquote{misterio} de la fidelidad de Dios constituye la esperanza de la historia. Ciertamente, se le oponen fuerzas de división y atropello, que desgarran a la humanidad a causa del pecado y del conflicto de egoísmos. En la historia, la Iglesia está al servicio de este \textquote{misterio} de bendición para la humanidad entera. En este misterio de la fidelidad de Dios, la Iglesia sólo cumple plenamente su misión cuando refleja en sí misma la luz de Cristo Señor, y así sirve de ayuda a los pueblos del mundo por el camino de la paz y del auténtico progreso.

En efecto, sigue siendo siempre válida la palabra de Dios revelada por medio del \textbf{profeta Isaías}: \textquote{La oscuridad cubre la tierra, y espesa nube a los pueblos, mas sobre ti amanece el Señor y su gloria sobre ti aparece} (\emph{Is} 60, 2). Lo que el profeta anuncia a Jerusalén se cumple en la Iglesia de Cristo: \textquote{A tu luz caminarán las naciones, y los reyes al resplandor de tu aurora} (\emph{Is} 60, 3).

Con Jesucristo la bendición de Abraham se extendió a todos los pueblos, a la Iglesia universal como nuevo Israel que acoge en su seno a la humanidad entera. Con todo, también hoy sigue siendo verdad lo que decía \textbf{el profeta}: \textquote{Espesa nube cubre a los pueblos} y nuestra historia. En efecto, no se puede decir que la globalización sea sinónimo de orden mundial; todo lo contrario. Los conflictos por la supremacía económica y el acaparamiento de los recursos energéticos e hídricos, y de las materias primas, dificultan el trabajo de quienes, en todos los niveles, se esfuerzan por construir un mundo justo y solidario.

Es necesaria una esperanza mayor, que permita preferir el bien común de todos al lujo de pocos y a la miseria de muchos. \textquote{Esta gran esperanza sólo puede ser Dios, (\ldots{}) pero no cualquier dios, sino el Dios que tiene un rostro humano} (\emph{Spe salvi}, 31), el Dios que se manifestó en el Niño de Belén y en el Crucificado Resucitado.

Si hay una gran esperanza, se puede perseverar en la sobriedad. Si falta la verdadera esperanza, se busca la felicidad en la embriaguez, en lo superfluo, en los excesos, y los hombres se arruinan a sí mismos y al mundo. La moderación no sólo es una regla ascética, sino también un camino de salvación para la humanidad.

Ya resulta evidente que sólo adoptando un estilo de vida sobrio, acompañado del serio compromiso por una distribución equitativa de las riquezas, será posible instaurar un orden de desarrollo justo y sostenible. Por esto, hacen falta hombres que alimenten una gran esperanza y posean por ello una gran valentía. La valentía de los Magos, que emprendieron un largo viaje siguiendo una estrella, y que supieron arrodillarse ante un Niño y ofrecerle sus dones preciosos. Todos necesitamos esta valentía, anclada en una firme esperanza.

Que nos la obtenga María, acompañándonos en nuestra peregrinación terrena con su protección materna. Amén.

\subsubsection{Homilía: Estrella que nos lleva a Dios}

Basílica Vaticana. Jueves 6 de enero de 2011.

En la solemnidad de la Epifanía la Iglesia sigue contemplando y celebrando el misterio del nacimiento de Jesús salvador. En particular, la fiesta de hoy subraya el destino y el significado universales de este nacimiento. Al hacerse hombre en el seno de María, el Hijo de Dios vino no sólo para el pueblo de Israel, representado por los pastores de Belén, sino también para toda la humanidad, representada por los Magos. Y la Iglesia nos invita hoy a meditar y orar precisamente sobre los Magos y sobre su camino en busca del Mesías (cf. \emph{Mt} 2, 1-12). En el \textbf{Evangelio} hemos escuchado que los Magos, habiendo llegado a Jerusalén desde el Oriente, preguntan: \textquote{¿Dónde está el Rey de los judíos que ha nacido? Hemos visto su estrella en el Oriente y hemos venido a adorarlo} (v. 2). ¿Qué clase de personas eran y qué tipo de estrella era esa? Probablemente eran sabios que escrutaban el cielo, pero no para tratar de \textquote{leer} en los astros el futuro, quizá para obtener así algún beneficio; más bien, eran hombres \textquote{en busca} de algo más, en busca de la verdadera luz, una luz capaz de indicar el camino que es preciso recorrer en la vida. Eran personas que tenían la certeza de que en la creación existe lo que podríamos definir la \textquote{firma} de Dios, una firma que el hombre puede y debe intentar descubrir y descifrar. Tal vez el modo para conocer mejor a estos Magos y entender su deseo de dejarse guiar por los signos de Dios es detenernos a considerar lo que encontraron, en su camino, en la gran ciudad de Jerusalén.

Ante todo encontraron al \textbf{rey Herodes}. Ciertamente, Herodes estaba interesado en el niño del que hablaban los Magos, pero no con el fin de adorarlo, como quiere dar a entender mintiendo, sino para eliminarlo. Herodes es un hombre de poder, que en el otro sólo ve un rival contra el cual luchar. En el fondo, si reflexionamos bien, también Dios le parece un rival, más aún, un rival especialmente peligroso, que querría privar a los hombres de su espacio vital, de su autonomía, de su poder; un rival que señala el camino que hay que recorrer en la vida y así impide hacer todo lo que se quiere. Herodes escucha de sus expertos en las Sagradas Escrituras las palabras del profeta Miqueas (5, 1), pero sólo piensa en el trono. Entonces Dios mismo debe ser ofuscado y las personas deben limitarse a ser simples peones para mover en el gran tablero de ajedrez del poder. Herodes es un personaje que no nos cae simpático y que instintivamente juzgamos de modo negativo por su brutalidad. Pero deberíamos preguntarnos: ¿Hay algo de Herodes también en nosotros? ¿También nosotros, a veces, vemos a Dios como una especie de rival? ¿También nosotros somos ciegos ante sus signos, sordos a sus palabras, porque pensamos que pone límites a nuestra vida y no nos permite disponer de nuestra existencia como nos plazca? Queridos hermanos y hermanas, cuando vemos a Dios de este modo acabamos por sentirnos insatisfechos y descontentos, porque no nos dejamos guiar por Aquel que está en el fundamento de todas las cosas. Debemos alejar de nuestra mente y de nuestro corazón la idea de la rivalidad, la idea de que dar espacio a Dios es un límite para nosotros mismos; debemos abrirnos a la certeza de que Dios es el amor omnipotente que no quita nada, no amenaza; más aún, es el único capaz de ofrecernos la posibilidad de vivir en plenitud, de experimentar la verdadera alegría.

Los Magos, luego, se encuentran con \textbf{los estudiosos, los teólogos, los expertos} que lo saben todo sobre las Sagradas Escrituras, que conocen las posibles interpretaciones, que son capaces de citar de memoria cualquier pasaje y que, por tanto, son una valiosa ayuda para quienes quieren recorrer el camino de Dios. Pero, afirma san Agustín, les gusta ser guías para los demás, indican el camino, pero no caminan, se quedan inmóviles. Para ellos las Escrituras son una especie de atlas que leen con curiosidad, un conjunto de palabras y conceptos que examinar y sobre los cuales discutir doctamente. Pero podemos preguntarnos de nuevo: ¿no existe también en nosotros la tentación de considerar las Sagradas Escrituras, este tesoro riquísimo y vital para la fe la Iglesia, más como un objeto de estudio y de debate de especialistas que como el Libro que nos señala el camino para llegar a la vida? Creo que, como indiqué en la exhortación apostólica \emph{Verbum Domini}, debería surgir siempre de nuevo en nosotros la disposición profunda a ver la palabra de la Biblia, leída en la Tradición viva de la Iglesia (n. 18), como la verdad que nos dice qué es el hombre y cómo puede realizarse plenamente, la verdad que es el camino a recorrer diariamente, junto a los demás, si queremos construir nuestra existencia sobre la roca y no sobre la arena.

Pasemos ahora a \textbf{la estrella}. ¿Qué clase de estrella era la que los Magos vieron y siguieron? A lo largo de los siglos esta pregunta ha sido objeto de debate entre los astrónomos. Kepler, por ejemplo, creía que se trataba de una \textquote{nova} o una \textquote{supernova}, es decir, una de las estrellas que normalmente emiten una luz débil, pero que pueden tener improvisamente una violenta explosión interna que produce una luz excepcional. Ciertamente, son cosas interesantes, pero que no nos llevan a lo que es esencial para entender esa estrella. Debemos volver al hecho de que esos hombres buscaban las huellas de Dios; trataban de leer su \textquote{firma} en la creación; sabían que \textquote{el cielo proclama la gloria de Dios} (\emph{Sal} 19, 2); es decir, tenían la certeza de que es posible vislumbrar a Dios en la creación. Pero, al ser hombres sabios, sabían también que no es con un telescopio cualquiera, sino con los ojos profundos de la razón en busca del sentido último de la realidad y con el deseo de Dios, suscitado por la fe, como es posible encontrarlo, más aún, como resulta posible que Dios se acerque a nosotros. El universo no es el resultado de la casualidad, como algunos quieren hacernos creer. Al contemplarlo, se nos invita a leer en él algo profundo: la sabiduría del Creador, la inagotable fantasía de Dios, su infinito amor a nosotros. No deberíamos permitir que limiten nuestra mente teorías que siempre llegan sólo hasta cierto punto y que ---si las miramos bien--- de ningún modo están en conflicto con la fe, pero no logran explicar el sentido último de la realidad. En la belleza del mundo, en su misterio, en su grandeza y en su racionalidad no podemos menos de leer la racionalidad eterna, y no podemos menos de dejarnos guiar por ella hasta el único Dios, creador del cielo y de la tierra. Si tenemos esta mirada, veremos que el que creó el mundo y el que nació en una cueva en Belén y sigue habitando entre nosotros en la Eucaristía son el mismo Dios vivo, que nos interpela, nos ama y quiere llevarnos a la vida eterna.

Herodes, los expertos en las Escrituras, la estrella. Sigamos el camino de los Magos que llegan a Jerusalén. Sobre la gran ciudad la estrella desaparece, ya no se ve. ¿Qué significa eso? También en este caso debemos leer el signo en profundidad. Para aquellos hombres era lógico buscar al nuevo rey en el palacio real, donde se encontraban los sabios consejeros de la corte. Pero, probablemente con asombro, tuvieron que constatar que aquel recién nacido no se encontraba en los lugares del poder y de la cultura, aunque en esos lugares se daban valiosas informaciones sobre él. En cambio, se dieron cuenta de que a veces el poder, incluso el del conocimiento, obstaculiza el camino hacia el encuentro con aquel Niño. Entonces la estrella los guió a Belén, una pequeña ciudad; los guió hasta los pobres, hasta los humildes, para encontrar al Rey del mundo. Los criterios de Dios son distintos de los de los hombres. Dios no se manifiesta en el poder de este mundo, sino en la humildad de su amor, un amor que pide a nuestra libertad acogerlo para transformarnos y ser capaces de llegar a Aquel que es el Amor. Pero incluso para nosotros las cosas no son tan diferentes de como lo eran para los Magos. Si se nos pidiera nuestro parecer sobre cómo Dios habría debido salvar al mundo, tal vez responderíamos que habría debido manifestar todo su poder para dar al mundo un sistema económico más justo, en el que cada uno pudiera tener todo lo que quisiera. En realidad, esto sería una especie de violencia contra el hombre, porque lo privaría de elementos fundamentales que lo caracterizan. De hecho, no se verían involucrados ni nuestra libertad ni nuestro amor. El poder de Dios se manifiesta de un modo muy distinto: en Belén, donde encontramos la aparente impotencia de su amor. Y es allí a donde debemos ir y es allí donde encontramos la estrella de Dios.

Así resulta muy claro también un último elemento importante del episodio de los Magos: el lenguaje de la creación nos permite recorrer un buen tramo del camino hacia Dios, pero no nos da la luz definitiva. Al final, para los Magos fue indispensable escuchar la voz de las Sagradas Escrituras: sólo ellas podían indicarles el camino. La Palabra de Dios es la verdadera estrella que, en la incertidumbre de los discursos humanos, nos ofrece el inmenso esplendor de la verdad divina. Queridos hermanos y hermanas, dejémonos guiar por la estrella, que es la Palabra de Dios; sigámosla en nuestra vida, caminando con la Iglesia, donde la Palabra ha plantado su tienda. Nuestro camino estará siempre iluminado por una luz que ningún otro signo puede darnos. Y también nosotros podremos convertirnos en estrellas para los demás, reflejo de la luz que Cristo ha hecho brillar sobre nosotros. Amén.

\subsection{Francisco, papa}

\subsubsection{Homilía (2014): No se detuvieron en la oscuridad}

Basílica de San Pedro. Lunes 6 de enero del 2014.

\textquote{Lumen requirunt lumine}. Esta sugerente expresión de un himno litúrgico de la Epifanía se refiere a la experiencia de los Magos: siguiendo una luz, buscan la Luz. La estrella que aparece en el cielo enciende en su mente y en su corazón una luz que los lleva a buscar la gran Luz de Cristo. Los Magos siguen fielmente aquella luz que los ilumina interiormente y encuentran al Señor.

En este recorrido que hacen los Magos de Oriente está simbolizado el destino de todo hombre: nuestra vida es un camino, iluminados por luces que nos permiten entrever el sendero, hasta encontrar la plenitud de la verdad y del amor, que nosotros cristianos reconocemos en Jesús, Luz del mundo. Y todo hombre, como los Magos, tiene a disposición dos grandes \textquote{libros} de los que sacar los signos para orientarse en su peregrinación: el libro de la creación y el libro de las Sagradas Escrituras. Lo importante es estar atentos, vigilantes, escuchar a Dios que nos habla, siempre nos habla. Como dice el Salmo, refiriéndose a la Ley del Señor: \textquote{Lámpara es tu palabra para mis pasos, / luz en mi sendero} (Sal 119,105). Sobre todo, escuchar el Evangelio, leerlo, meditarlo y convertirlo en alimento espiritual nos permite encontrar a Jesús vivo, hacer experiencia de Él y de su amor.

En la \textbf{primera Lectura} resuena, por boca del \textbf{profeta Isaías}, el llamado de Dios a Jerusalén: \textquote{¡Levántate, brilla!} (60,1). Jerusalén está llamada a ser la ciudad de la luz, que refleja en el mundo la luz de Dios y ayuda a los hombres a seguir sus caminos. Ésta es la vocación y la misión del Pueblo de Dios en el mundo. Pero Jerusalén puede desatender esta llamada del Señor. Nos dice el \textbf{Evangelio} que los Magos, cuando llegaron a Jerusalén, de momento perdieron de vista la estrella. No la veían. En especial, su luz falta en el palacio del rey Herodes: aquella mansión es tenebrosa, en ella reinan la oscuridad, la desconfianza, el miedo, la envidia. De hecho, Herodes se muestra receloso e inquieto por el nacimiento de un frágil Niño, al que ve como un rival. En realidad, Jesús no ha venido a derrocarlo a él, ridículo fantoche, sino al Príncipe de este mundo. Sin embargo, el rey y sus consejeros sienten que el entramado de su poder se resquebraja, temen que cambien las reglas de juego, que las apariencias queden desenmascaradas. Todo un mundo edificado sobre el poder, el prestigio, el tener, la corrupción, entra en crisis por un Niño. Y Herodes llega incluso a matar a los niños: \textquote{Tú matas el cuerpo de los niños, porque el temor te ha matado a ti el corazón} -- escribe san Quodvultdeus (Sermón 2 sobre el Símbolo: PL 40, 655). Es así: tenía temor, y por este temor pierde el juicio.

Los Magos consiguieron superar aquel momento crítico de oscuridad en el palacio de Herodes, porque creyeron en las Escrituras, en la palabra de los profetas que señalaba Belén como el lugar donde había de nacer el Mesías. Así escaparon al letargo de la noche del mundo, reemprendieron su camino y de pronto vieron nuevamente la estrella, y el \textbf{Evangelio} dice que se llenaron de \textquote{inmensa alegría} (Mt 2,10). Esa estrella que no se veía en la oscuridad de la mundanidad de aquel palacio.

Un aspecto de la luz que nos guía en el camino de la fe es también la santa \textquote{astucia}. Es también una virtud, la santa \textquote{astucia}. Se trata de esa sagacidad espiritual que nos permite reconocer los peligros y evitarlos. Los Magos supieron usar esta luz de \textquote{astucia} cuando, de regreso a su tierra, decidieron no pasar por el palacio tenebroso de Herodes, sino marchar por otro camino. Estos sabios venidos de Oriente nos enseñan a no caer en las asechanzas de las tinieblas y a defendernos de la oscuridad que pretende cubrir nuestra vida. Ellos, con esta santa \textquote{astucia}, han protegido la fe. Y también nosotros debemos proteger la fe. Protegerla de esa oscuridad. Esa oscuridad que a menudo se disfraza incluso de luz. Porque el demonio, dice san Pablo, muchas veces se viste de ángel de luz. Y entonces es necesaria la santa \textquote{astucia}, para proteger la fe, protegerla de los cantos de las sirenas, que te dicen: \textquote{Mira, hoy debemos hacer esto, aquello\ldots{}} Pero la fe es una gracia, es un don. Y a nosotros nos corresponde protegerla con la santa \textquote{astucia}, con la oración, con el amor, con la caridad. Es necesario acoger en nuestro corazón la luz de Dios y, al mismo tiempo, practicar aquella astucia espiritual que sabe armonizar la sencillez con la sagacidad, como Jesús pide a sus discípulos: \textquote{Sean sagaces como serpientes y simples como palomas} (Mt 10,16).

En esta fiesta de la Epifanía, que nos recuerda la manifestación de Jesús a la humanidad en el rostro de un Niño, sintamos cerca a los Magos, como sabios compañeros de camino. Su ejemplo nos anima a levantar los ojos a la estrella y a seguir los grandes deseos de nuestro corazón. Nos enseñan a no contentarnos con una vida mediocre, de \textquote{poco calado}, sino a dejarnos fascinar siempre por la bondad, la verdad, la belleza\ldots{} por Dios, que es todo eso en modo siempre mayor. Y nos enseñan a no dejarnos engañar por las apariencias, por aquello que para el mundo es grande, sabio, poderoso. No nos podemos quedar ahí. Es necesario proteger la fe. Es muy importante en este tiempo: proteger la fe. Tenemos que ir más allá, más allá de la oscuridad, más allá de la atracción de las sirenas, más allá de la mundanidad, más allá de tantas modernidades que existen hoy, ir hacia Belén, allí donde en la sencillez de una casa de la periferia, entre una mamá y un papá llenos de amor y de fe, resplandece el Sol que nace de lo alto, el Rey del universo. A ejemplo de los Magos, con nuestras pequeñas luces busquemos la Luz y protejamos la fe. Así sea.

\subsubsection{Ángelus (2014): Dios nos amó primero}

Plaza de San Pedro. Lunes 6 de enero de 2014.

Hoy celebramos la Epifanía, es decir la \textquote{manifestación} del Señor. Esta solemnidad está vinculada al relato bíblico de la llegada de los magos de Oriente a Belén para rendir homenaje al Rey de los judíos: un episodio que el Papa Benedicto comentó magníficamente en su libro sobre la infancia de Jesús. Esa fue precisamente la primera \textquote{manifestación} de Cristo a las gentes. Por ello la Epifanía destaca la apertura universal de la salvación traída por Jesús. La Liturgia de este día aclama: \textquote{Te adorarán, Señor, todos los pueblos de la tierra}, porque Jesús vino por todos nosotros, por todos los pueblos, por todos.

En efecto, esta fiesta nos hace ver un doble movimiento: por una parte el movimiento de Dios hacia el mundo, hacia la humanidad ---toda la historia de la salvación, que culmina en Jesús---; y por otra parte el movimiento de los hombres hacia Dios ---pensemos en las religiones, en la búsqueda de la verdad, en el camino de los pueblos hacia la paz, la paz interior, la justicia, la libertad---. Y a este doble movimiento lo mueve una recíproca atracción. Por parte de Dios, ¿qué es lo que lo atrae? Es el amor por nosotros: somos sus hijos, nos ama, y quiere liberarnos del mal, de las enfermedades, de la muerte, y llevarnos a su casa, a su Reino. \textquote{Dios, por pura gracia, nos atrae para unirnos a sí} (Exhort. ap. Evangelii gaudium, 112). Y también por parte nuestra hay un amor, un deseo: el bien siempre nos atrae, la verdad nos atrae, la vida, la felicidad, la belleza nos atrae\ldots{} Jesús es es el punto de encuentro de esta atracción mutua, y de este doble movimiento. Es Dios y hombre: Jesús. Dios y hombre. ¿Pero quien toma la iniciativa? ¡Siempre Dios! El amor de Dios viene siempre antes del nuestro. Él siempre toma la iniciativa. Él nos espera, Él nos invita, la iniciativa es siempre suya. Jesús es Dios que se hizo hombre, se encarnó, nació por nosotros. La nueva estrella que apareció a los magos era el signo del nacimiento de Cristo. Si no hubiesen visto la estrella, esos hombres no se hubiesen puesto en camino. La luz nos precede, la verdad nos precede, la belleza nos precede. Dios nos precede. El profeta Isaías decía que Dios es como la flor del almendro. ¿Por qué? Porque en esa tierra el almendro es primero en florecer. Y Dios siempre precede, siempre nos busca Él primero, Él da el primer paso. Dios nos precede siempre. Su gracia nos precede; y esta gracia apareció en Jesús. Él es la epifanía. Él, Jesucristo, es la manifestación del amor de Dios. Está con nosotros.

La Iglesia está toda dentro de este movimiento de Dios hacia el mundo: su alegría es el Evangelio, es reflejar la luz de Cristo. La Iglesia es el pueblo de aquellos que experimentaron esta atracción y la llevaron dentro, en el corazón y en la vida. \textquote{Me gustaría ---sinceramente---, me gustaría decir a aquellos que se sienten alejados de Dios y de la Iglesia ---decirlo respetuosamente---, decir a aquellos son temerosos e indiferentes: el Señor te llama también a ti, te llama a formar parte de su pueblo y lo hace con gran respeto y amor} (ibid., 113). El Señor te llama. El Señor te busca. El Señor te espera. El Señor no hace proselitismo, da amor, y este amor te busca, te espera, a ti que en este momento no crees o estás alejado. Esto es el amor de Dios.

Pidamos a Dios, para toda la Iglesia, pidamos la alegría de evangelizar, porque ha sido \textquote{enviada por Cristo para manifestar y comunicar a todos los hombres y a todos los pueblos el amor de Dios} (Ad gentes, 10). Que la Virgen María nos ayude a ser todos discípulos-misioneros, pequeñas estrellas que reflejen su luz. Y oremos para que los corazones se abran para acoger el anuncio, y todos los hombres lleguen a ser \textquote{partícipes de la misma promesa en Jesucristo, por el Evangelio} (Ef 3, 6).

\subsubsection{Homilía (2017)}

\emph{Basílica Vaticana\\ Viernes 6 de enero de 2017}


\textquote{¿Dónde está el Rey de los judíos que acaba de nacer? Porque vimos su estrella y hemos venido a adorarlo} (\emph{Mt} 2, 2).

Con estas palabras, los magos, venidos de tierras lejanas, nos dan a conocer el motivo de su larga travesía: adorar al rey recién nacido. Ver y adorar, dos acciones que se destacan en el relato evangélico: vimos una estrella y queremos adorar.

Estos hombres \emph{vieron una estrella} que los puso en movimiento. El descubrimiento de algo inusual que sucedió en el cielo logró desencadenar un sinfín de acontecimientos. No era una estrella que brilló de manera exclusiva para ellos, ni tampoco tenían un ADN especial para descubrirla. Como bien supo decir un padre de la Iglesia, \textquote{los magos no se pusieron en camino porque hubieran visto la estrella, sino que vieron la estrella porque se habían puesto en camino} (cf. San Juan Crisóstomo). Tenían el corazón abierto al horizonte y lograron ver lo que el cielo les mostraba porque había en ellos una inquietud que los empujaba: estaban abiertos a una novedad.

Los magos, de este modo, expresan el retrato del hombre creyente, del hombre que tiene nostalgia de Dios; del que añora su casa, la patria celeste. Reflejan la imagen de todos los hombres que en su vida no han dejado que se les anestesie el corazón.

La santa nostalgia de Dios brota en el corazón creyente pues sabe que el Evangelio no es un acontecimiento del pasado sino del presente. La santa nostalgia de Dios nos permite tener los ojos abiertos frente a todos los intentos reductivos y empobrecedores de la vida. La santa nostalgia de Dios es la memoria creyente que se rebela frente a tantos profetas de desventura. Esa nostalgia es la que mantiene viva la esperanza de la comunidad creyente la cual, semana a semana, implora diciendo: \textquote{Ven, Señor Jesús}.

Precisamente esta nostalgia fue la que empujó al anciano Simeón a ir todos los días al templo, con la certeza de saber que su vida no terminaría sin poder acunar al Salvador. Fue esta nostalgia la que empujó al hijo pródigo a salir de una actitud de derrota y buscar los brazos de su padre. Fue esta nostalgia la que el pastor sintió en su corazón cuando dejó a las noventa y nueve ovejas en busca de la que estaba perdida, y fue también la que experimentó María Magdalena la mañana del domingo para salir corriendo al sepulcro y encontrar a su Maestro resucitado. La nostalgia de Dios nos saca de nuestros encierros deterministas, esos que nos llevan a pensar que nada puede cambiar. La nostalgia de Dios es la actitud que rompe aburridos conformismos e impulsa a comprometerse por ese cambio que anhelamos y necesitamos. La nostalgia de Dios tiene su raíz en el pasado pero no se queda allí: va en busca del futuro. Al igual que los magos, el creyente \textquote{nostalgioso} busca a Dios, empujado por su fe, en los lugares más recónditos de la historia, porque sabe en su corazón que allí lo espera el Señor. Va a la periferia, a la frontera, a los sitios no evangelizados para poder encontrarse con su Señor; y lejos de hacerlo con una postura de superioridad lo hace como un mendicante que no puede ignorar los ojos de aquel para el cual la Buena Nueva es todavía un terreno a explorar.

Como actitud contrapuesta, en el palacio de Herodes ―que distaba muy pocos kilómetros de Belén―, no se habían percatado de lo que estaba sucediendo. Mientras los magos caminaban, Jerusalén dormía. Dormía de la mano de un Herodes quien lejos de estar en búsqueda también dormía. Dormía bajo la anestesia de una conciencia cauterizada. Y quedó desconcertado. Tuvo miedo. Es el desconcierto que, frente a la novedad que revoluciona la historia, se encierra en sí mismo, en sus logros, en sus saberes, en sus éxitos. El desconcierto de quien está sentado sobre la riqueza sin lograr ver más allá. Un desconcierto que brota del corazón de quién quiere controlar todo y a todos. Es el desconcierto del que está inmerso en la cultura del ganar cueste lo que cueste; en esa cultura que sólo tiene espacio para los \textquote{vencedores} y al precio que sea. Un desconcierto que nace del miedo y del temor ante lo que nos cuestiona y pone en riesgo nuestras seguridades y verdades, nuestras formas de aferrarnos al mundo y a la vida. Y Herodes tuvo miedo, y ese miedo lo condujo a buscar seguridad en el crimen: \textquote{\emph{Necas parvulos corpore, quia te necat timor in corde}} (San Quodvultdeus, \emph{Sermo 2 sobre el símbolo}: \emph{PL}, 40, 655). Matas los niños en el cuerpo porque a ti el miedo te mata el corazón.

\emph{Queremos adorar}. Los hombres de Oriente fueron a adorar, y fueron a hacerlo al lugar propio de un rey: el Palacio. Y esto es importante, allí llegaron ellos con su búsqueda, era el lugar indicado: pues es propio de un rey nacer en un palacio, y tener su corte y súbditos. Es signo de poder, de éxito, de vida lograda. Y es de esperar que el rey sea venerado, temido y adulado, sí; pero no necesariamente amado. Esos son los esquemas mundanos, los pequeños ídolos a los que le rendimos culto: el culto al poder, a la apariencia y a la superioridad. Ídolos que solo prometen tristeza, esclavitud, miedo.

Y fue precisamente ahí donde comenzó el camino más largo que tuvieron que andar esos hombres venidos de lejos. Ahí comenzó la osadía más difícil y complicada. Descubrir que lo que ellos buscaban no estaba en el palacio sino que se encontraba en otro lugar, no sólo geográfico sino existencial. Allí no veían la estrella que los conducía a descubrir un Dios que quiere ser amado, y eso sólo es posible bajo el signo de la libertad y no de la tiranía; descubrir que la mirada de este Rey desconocido ―pero deseado― no humilla, no esclaviza, no encierra. Descubrir que la mirada de Dios levanta, perdona, sana. Descubrir que Dios ha querido nacer allí donde no lo esperamos, donde quizá no lo queremos. O donde tantas veces lo negamos. Descubrir que en la mirada de Dios hay espacio para los heridos, los cansados, los maltratados, abandonados: que su fuerza y su poder se llama misericordia. Qué lejos se encuentra, para algunos, Jerusalén de Belén.

Herodes no puede adorar porque no quiso y no pudo cambiar su mirada. No quiso dejar de rendirse culto a sí mismo creyendo que todo comenzaba y terminaba con él. No pudo adorar porque buscaba que lo adorasen. Los sacerdotes tampoco pudieron adorar porque sabían mucho, conocían las profecías, pero no estaban dispuestos ni a caminar ni a cambiar.

Los magos sintieron nostalgia, no querían más de lo mismo. Estaban acostumbrados, habituados y cansados de los Herodes de su tiempo. Pero allí, en Belén, había promesa de novedad, había promesa de gratuidad. Allí estaba sucediendo algo nuevo. Los magos pudieron adorar porque se animaron a caminar y postrándose ante el pequeño, postrándose ante el pobre, postrándose ante el indefenso, postrándose ante el extraño y desconocido Niño de Belén, allí descubrieron la Gloria de Dios.

\subsubsection{Ángelus (2017)} \emph{Queridos hermanos y hermanas, ¡buenos días!}

Hoy, celebramos la Epifanía del Señor, es decir, la manifestación de Jesús que brilla como luz para todas las gentes. Símbolo de esta luz que resplandece en el mundo y quiere iluminar la vida de cada uno es la estrella, que guió a los Magos a Belén. Ellos, dice el Evangelio, vieron \textquote{su estrella} (\emph{Mt} 2, 2) y decidieron seguirla: decidieron dejarse guiar por la estrella de Jesús.

También en nuestra vida existen diversas estrellas, luces que brillan y orientan. Depende de nosotros elegir cuáles seguir. Por ejemplo, hay luces intermitentes, que van y vienen, como las pequeñas satisfacciones de la vida: que aunque buenas, no son suficientes, porque duran poco y no dejan la paz que buscamos. Después están las luces cegadoras del primer plano, del dinero y del éxito, que prometen todo y enseguida: son seductoras, pero con su fuerza ciegan y hacen pasar de los sueños de gloria a la oscuridad más densa. Los Magos, en cambio, invitan a seguir una luz estable, una luz amable, que no se oculta, porque no es de este mundo: viene del cielo y resplandece\ldots{} ¿Dónde? En el corazón.

Esta luz verdadera es la luz del Señor, o mejor dicho, es el Señor mismo. Él es nuestra luz: una luz que no deslumbra, sino que acompaña y dona una alegría única. Esta luz es para todos y llama a cada uno: podemos escuchar así la actual invitación dirigida a nosotros por el profeta Isaías: \textquote{arriba, resplandece, que ha llegado tu luz} (60, 1). Así decía Isaías, profetizando esta alegría de hoy en Jerusalén: \textquote{arriba, resplandece, que ha llegado tu luz}. Al inicio de cada día podemos acoger esta invitación: arriba, resplandece, que ha llegado tu luz, sigue hoy, entre tantas estrellas fugaces en el mundo, la estrella luminosa de Jesús! Siguiéndola, tendremos alegría, como ocurrió a los Magos, que \textquote{al ver la estrella se llenaron de inmensa alegría} (\emph{Mt} 2, 10); porque donde esta Dios hay alegría. Quien ha encontrado a Jesús ha experimentado el milagro de la luz que rasga las tinieblas y conoce esta luz que ilumina y aclara. Querría, con mucho respeto, invitar a todos a no tener miedo de esta luz y a abrirse al Señor. Sobre todo querría decir a quien ha perdido la fuerza de buscar, está cansado, a quien, superado por las oscuridades de la vida, ha apagado el deseo: \textquote{¡Levántate, ánimo, la luz de Jesús sabe vencer las tinieblas más oscuras; levántate, ánimo!}.

Y ¿cómo encontrar esta luz divina? Sigamos el ejemplo de los Magos, que el Evangelio describe siempre en movimiento. Quien quiere la luz, efectivamente, sale de sí y busca: no permanece cerrado, quieto a ver qué cosa sucede al su alrededor, sino pone en juego su propia vida; sale de sí. La vida cristiana es un camino continuo, hecho de esperanza, hecho de búsqueda; un camino que, como aquel de los Magos, prosigue incluso cuando la estrella desaparece momentáneamente de la vista. En este camino hay también insidias que hay que evitar: las charlas superficiales y mundanas, que frenan el paso; los caprichos paralizantes del egoísmo; los agujeros del pesimismo, que atrapa a la esperanza. Estos obstáculos bloquearon a los escribas, de los que habla el Evangelio de hoy. Ellos sabían dónde estaba la luz, pero no se movieron. Cuando Herodes les pregunto: \textquote{¿Dónde nacerá el Mesías?} --- \textquote{¡En Belén!}. Sabían dónde, pero no se movieron. Su conocimiento fue en vano: sabían muchas cosas, pero para nada, todo en vano. No basta saber que Dios ha nacido, si no se hace con Él Navidad en el corazón. Dios ha nacido, sí, pero ¿Ha nacido en tú corazón? ¿Ha nacido en mí corazón? ¿Ha nacido en nuestro corazón? Y así le encontraremos, como los Magos, con María, José, en el establo.

Los Magos lo hicieron: encontraron al Niño, \textquote{postrándose, le adoraron} (v. 11). No le miraron solamente, dijeron solo una oración circunstancial y se fueron, no, sino que le adoraron: entraron en una comunión personal de amor con Jesús. Después le regalaron oro, incienso y mirra, es decir, sus bienes más preciados. Aprendamos de los Magos a no dedicar a Jesús sólo los ratos perdidos de tiempo y algún pensamiento de vez en cuando, de lo contrario no tendremos su luz. Como los Magos, pongámonos en camino, revistámonos de luz siguiendo la estrella de Jesús, y adoremos al Señor con todo nuestro ser.

\subsubsection{Homilía (2020)} {SANTA MISA EN LA SOLEMNIDAD DE LA EPIFANÍA DEL SEÑOR}

CAPILLA PAPAL

\textbf{\emph{HOMILÍA DEL SANTO PADRE FRANCISCO}}

\emph{Basílica Vaticana\\ Lunes, 6 de enero de 2020}



En el Evangelio (\emph{Mt} 2,1-12) hemos escuchado que los Magos comienzan manifestando sus intenciones: \textquote{Hemos visto salir su estrella y venimos a adorarlo} (v. 2). La adoración es la finalidad de su viaje, el objetivo de su camino. De hecho, cuando llegaron a Belén, \textquote{vieron al niño con María, su madre, y cayendo de rodillas lo adoraron} (v. 11). Si perdemos el sentido de la \emph{adoración}, perdemos el sentido de movimiento de la vida cristiana, que es un camino hacia el Señor, no hacia nosotros. Es el riesgo del que nos advierte el Evangelio, presentando, junto a los Reyes Magos, unos personajes que no logran adorar.

En primer lugar, está el rey Herodes, que usa el verbo adorar, pero de manera engañosa. De hecho, le pide a los Reyes Magos que le informen sobre el lugar donde estaba el Niño \textquote{para ir ---dice--- yo también a adorarlo} (v. 8). En realidad, Herodes sólo se adoraba a sí mismo y, por lo tanto, quería deshacerse del Niño con mentiras. ¿Qué nos enseña esto? Que el hombre, cuando no adora a \emph{Dios}, está orientado a adorar su \emph{yo}. E incluso la vida cristiana, sin adorar al Señor, puede convertirse en una forma educada de alabarse a uno mismo y el talento que se tiene: cristianos que no saben adorar, que no saben rezar adorando. Es un riesgo grave: servirnos de Dios en lugar de servir a Dios. Cuántas veces hemos cambiado los intereses del Evangelio por los nuestros, cuántas veces hemos cubierto de religiosidad lo que era cómodo para nosotros, cuántas veces hemos confundido el poder según Dios, que es servir a los demás, con el poder según el mundo, que es servirse a sí mismo.

Además de Herodes, hay otras personas en el Evangelio que no logran adorar: son los jefes de los sacerdotes y los escribas del pueblo. Ellos indican a Herodes con extrema precisión dónde nacería el Mesías: en Belén de Judea (cf. v. 5). Conocen las profecías y las citan exactamente. Saben a dónde ir ---grandes teólogos, grandes---, pero no van. También de esto podemos aprender una lección. En la vida cristiana no es suficiente saber: sin salir de uno mismo, sin encontrar, sin adorar, no se conoce a Dios. La teología y la eficiencia pastoral valen poco o nada si no se doblan las rodillas; si no se hace como los Magos, que no sólo fueron sabios organizadores de un viaje, sino que caminaron y adoraron. Cuando uno adora, se da cuenta de que la fe no se reduce a un conjunto de hermosas doctrinas, sino que es la relación con una Persona viva a quien amar. Conocemos el rostro de Jesús estando cara a cara con Él. Al adorar, descubrimos que la vida cristiana es una historia de amor con Dios, donde las buenas ideas no son suficientes, sino que se necesita ponerlo en primer lugar, como lo hace un enamorado con la persona que ama. Así debe ser la Iglesia, una adoradora enamorada de Jesús, su esposo.

Al inicio del año redescubrimos la adoración como una exigencia de fe. Si sabemos arrodillarnos ante Jesús, venceremos la tentación de ir cada uno por su camino. De hecho, adorar es hacer un éxodo de la esclavitud más grande, la de uno mismo. Adorar es poner al Señor en el centro para no estar más centrados en nosotros mismos. Es poner cada cosa en su lugar, dejando el primer puesto a Dios. Adorar es poner los planes de Dios antes que mi tiempo, que mis derechos, que mis espacios. Es aceptar la enseñanza de la Escritura: \textquote{Al Señor, tu Dios, adorarás} (\emph{Mt} 4,10). Tu Dios: adorar es experimentar que, con Dios, nos pertenecemos recíprocamente. Es darle del \textquote{tú} en la intimidad, es presentarle la vida y permitirle entrar en nuestras vidas. Es hacer descender su consuelo al mundo. Adorar es descubrir que para rezar basta con decir: \textquote{¡Señor mío y Dios mío!} (\emph{Jn} 20,28), y dejarnos llenar de su ternura.

Adorar es encontrarse con Jesús sin la lista de peticiones, pero con la única solicitud de estar con Él. Es descubrir que la alegría y la paz crecen con la alabanza y la acción de gracias. Cuando adoramos, permitimos que Jesús nos sane y nos cambie. Al adorar, le damos al Señor la oportunidad de transformarnos con su amor, de iluminar nuestra oscuridad, de darnos fuerza en la debilidad y valentía en las pruebas. Adorar es ir a lo esencial: es la forma de desintoxicarse de muchas cosas inútiles, de adicciones que adormecen el corazón y aturden la mente. De hecho, al adorar uno aprende a rechazar lo que no debe ser adorado: el dios del dinero, el dios del consumo, el dios del placer, el dios del éxito, nuestro yo erigido en dios. Adorar es hacerse pequeño en presencia del Altísimo, descubrir ante Él que la grandeza de la vida no consiste en tener, sino en amar. Adorar es redescubrirnos hermanos y hermanas frente al misterio del amor que supera toda distancia: es obtener el bien de la fuente, es encontrar en el Dios cercano la valentía para aproximarnos a los demás. Adorar es saber guardar silencio ante la Palabra divina, para aprender a decir palabras que no duelen, sino que consuelan.

La adoración es un gesto de amor que cambia la vida. Es actuar como los Magos: es traer oro al Señor, para decirle que nada es más precioso que Él; es ofrecerle incienso, para decirle que sólo con Él puede elevarse nuestra vida; es presentarle mirra, con la que se ungían los cuerpos heridos y destrozados, para pedirle a Jesús que socorra a nuestro prójimo que está marginado y sufriendo, porque allí está Él. Por lo general, sabemos cómo orar ---le pedimos, le agradecemos al Señor---, pero la Iglesia debe ir aún más allá con la oración de adoración, debemos crecer en la adoración. Es una sabiduría que debemos aprender todos los días. Rezar adorando: la oración de adoración.

Queridos hermanos y hermanas, hoy cada uno de nosotros puede preguntarse: \textquote{¿Soy un adorador cristiano?}. Muchos cristianos que oran no saben adorar. Hagámonos esta pregunta. ¿Encontramos momentos para la adoración en nuestros días y creamos espacios para la adoración en nuestras comunidades? Depende de nosotros, como Iglesia, poner en práctica las palabras que rezamos hoy en el Salmo: \textquote{Señor, que todos los pueblos te adoren}. Al adorar, nosotros también descubriremos, como los Magos, el significado de nuestro camino. Y, como los Magos, experimentaremos una \textquote{inmensa alegría} (\emph{Mt} 2,10).



\subsubsection{Ángelus (2020)}

\emph{Plaza de San Pedro\\ Lunes, 6 de enero de 2020}



\emph{Queridos hermanos y hermanas, ¡buenos días!}

Celebramos la solemnidad de la Epifanía, en memoria de los Magos que vinieron de Oriente a Belén, siguiendo la estrella, para visitar al Mesías recién nacido. Al final del relato evangélico se dice que los Magos \textquote{avisados en sueños que no volvieran donde Herodes, se retiraron a su país por otro camino} (v. 12). Por otro camino.

Estos sabios, procedentes de regiones lejanas, después de haber viajado mucho, encuentran al que querían conocer, después de haberlo buscado durante mucho tiempo, seguramente también con mucho trabajo y vicisitudes. Y cuando finalmente llegan a su destino, se postran ante el Niño, lo adoran, le ofrecen sus preciosos regalos. Después de eso, se pusieron en marcha de nuevo sin demora para volver a su tierra. Pero ese encuentro con el Niño los ha cambiado.

El encuentro con Jesús no detiene a los Reyes Magos, al contrario, les da un nuevo impulso para volver a su país, para contar lo que han visto y la alegría que han sentido. En esto hay una demostración del estilo de Dios, de su modo de manifestarse en la historia. La experiencia de Dios no nos bloquea, sino que nos libera; no nos aprisiona, sino que nos devuelve al camino, nos devuelve a los lugares habituales de nuestra existencia. Los lugares son y serán los mismos, pero nosotros, después del encuentro con Jesús, no somos los mismos que antes. El encuentro con Jesús nos cambia, nos transforma. El evangelista Mateo subraya que los Reyes Magos regresaron \textquote{por otro camino} (v. 12). La advertencia del ángel los lleva a cambiar sus caminos para no encontrarse con Herodes y sus tramas de poder.

Cada experiencia de encuentro con Jesús nos lleva a tomar caminos diferentes, porque de Él proviene una fuerza buena que sana el corazón y nos aparta del mal.

Existe una sabia dinámica entre continuidad y novedad: vuelven \textquote{a su país}, pero \textquote{por otro camino}. Esto indica que somos nosotros los que debemos cambiar, los que debemos transformar nuestra forma de vida, aunque sea en el mismo ambiente de siempre, los que debemos cambiar los criterios de juicio sobre la realidad que nos rodea. Esta es la diferencia entre el verdadero Dios y los ídolos traidores, como el dinero, el poder, el éxito\ldots{}; entre Dios y aquellos que prometen darte estos ídolos, como los magos, los adivinos, los hechiceros. La diferencia es que los ídolos nos atan a sí mismos, nos hacen dependientes de los ídolos, y nosotros tomamos posesión de ellos. El verdadero Dios no nos retiene ni se deja retener por nosotros: nos abre caminos de novedad y de libertad, porque es Padre que está siempre con nosotros para hacernos crecer.

Si te encuentras con Jesús, si tienes un encuentro espiritual con Jesús, recuerda: debes volver a los mismos lugares de siempre, pero de otra manera, con otro estilo. Es así, es el Espíritu Santo, que Jesús nos da, que nos cambia el corazón.

Pidamos a la Santa Virgen que podamos convertirnos en testigos de Cristo allá donde estemos, con una vida nueva, transformada por su amor.



\section{Temas}

La Epifanía del Señor

CEC 528, 724:

\textbf{528} La \emph{Epifanía} es la manifestación de Jesús como Mesías de Israel, Hijo de Dios y Salvador del mundo. Con el bautismo de Jesús en el Jordán y las bodas de Caná (cf. \emph{Solemnidad de la Epifanía del Señor}, Antífona del \textquote{Magnificat} en II Vísperas, LH), la Epifanía celebra la adoración de Jesús por unos \textquote{magos} venidos de Oriente (\emph{Mt} 2, 1) En estos \textquote{magos}, representantes de religiones paganas de pueblos vecinos, el Evangelio ve las primicias de las naciones que acogen, por la Encarnación, la Buena Nueva de la salvación. La llegada de los magos a Jerusalén para \textquote{rendir homenaje al rey de los Judíos} (\emph{Mt} 2, 2) muestra que buscan en Israel, a la luz mesiánica de la estrella de David (cf. \emph{Nm} 24, 17; \emph{Ap} 22, 16) al que será el rey de las naciones (cf. \emph{Nm} 24, 17-19). Su venida significa que los gentiles no pueden descubrir a Jesús y adorarle como Hijo de Dios y Salvador del mundo sino volviéndose hacia los judíos (cf. \emph{Jn} 4, 22) y recibiendo de ellos su promesa mesiánica tal como está contenida en el Antiguo Testamento (cf. \emph{Mt} 2, 4-6). La Epifanía manifiesta que \textquote{la multitud de los gentiles entra en la familia de los patriarcas} (San León Magno, \emph{Sermones}, 23: PL 54, 224B) y adquiere la \emph{israelitica dignitas} (la dignidad israelítica) (Vigilia pascual, Oración después de la tercera lectura: \emph{Misal Romano}).

\textbf{724} En María, el Espíritu Santo \emph{manifiesta} al Hijo del Padre hecho Hijo de la Virgen. Ella es la zarza ardiente de la teofanía definitiva: llena del Espíritu Santo, presenta al Verbo en la humildad de su carne dándolo a conocer a los pobres (cf. \emph{Lc} 2, 15-19) y a las primicias de las naciones (cf. \emph{Mt} 2, 11).

Cristo, luz de las naciones

CEC 280, 529, 748, 1165, 2466, 2715:

\textbf{280} La creación es el fundamento de \textquote{todos los designios salvíficos de Dios}, \textquote{el comienzo de la historia de la salvación} (DCG 51), que culmina en Cristo. Inversamente, el Misterio de Cristo es la luz decisiva sobre el Misterio de la creación; revela el fin en vista del cual, \textquote{al principio, Dios creó el cielo y la tierra} (\emph{Gn} 1,1): desde el principio Dios preveía la gloria de la nueva creación en Cristo (cf. \emph{Rm} 8,18-23).

\textbf{529} \emph{La Presentación de Jesús en el Templo} (cf. \emph{Lc} 2, 22-39) lo muestra como el Primogénito que pertenece al Señor (cf. \emph{Ex} 13,2.12-13). Con Simeón y Ana, toda la expectación de Israel es la que viene al \emph{Encuentro} de su Salvador (la tradición bizantina llama así a este acontecimiento). Jesús es reconocido como el Mesías tan esperado, \textquote{luz de las naciones} y \textquote{gloria de Israel}, pero también \textquote{signo de contradicción}. La espada de dolor predicha a María anuncia otra oblación, perfecta y única, la de la Cruz que dará la salvación que Dios ha preparado \textquote{ante todos los pueblos}.

\textbf{748} \textquote{Cristo es la luz de los pueblos. Por eso, este sacrosanto Sínodo, reunido en el Espíritu Santo, desea vehementemente iluminar a todos los hombres con la luz de Cristo, que resplandece sobre el rostro de la Iglesia (LG 1), anunciando el Evangelio a todas las criaturas}. Con estas palabras comienza la \textquote{Constitución dogmática sobre la Iglesia} del Concilio Vaticano II. Así, el Concilio muestra que el artículo de la fe sobre la Iglesia depende enteramente de los artículos que se refieren a Cristo Jesús. La Iglesia no tiene otra luz que la de Cristo; ella es, según una imagen predilecta de los Padres de la Iglesia, comparable a la luna cuya luz es reflejo del sol.

\textbf{1165} Cuando la Iglesia celebra el Misterio de Cristo, hay una palabra que jalona su oración: \emph{¡Hoy!}, como eco de la oración que le enseñó su Señor (\emph{Mt} 6,11) y de la llamada del Espíritu Santo (\emph{Hb} 3,7-4,11; \emph{Sal} 95,7). Este \textquote{hoy} del Dios vivo al que el hombre está llamado a entrar, es la \textquote{Hora} de la Pascua de Jesús, que atraviesa y guía toda la historia humana:

\textquote{La vida se ha extendido sobre todos los seres y todos están llenos de una amplia luz: el Oriente de los orientes invade el universo, y el que existía \textquote{antes del lucero de la mañana} y antes de todos los astros, inmortal e inmenso, el gran Cristo brilla sobre todos los seres más que el sol. Por eso, para nosotros que creemos en él, se instaura un día de luz, largo, eterno, que no se extingue: la Pascua mística} (Pseudo-Hipólito Romano, \emph{In Sanctum Pascha} 1-2).

\textbf{2466} En Jesucristo la verdad de Dios se manifestó en plenitud. \textquote{Lleno de gracia y de verdad} (\emph{Jn} 1, 14), él es la \textquote{luz del mundo} (\emph{Jn} 8, 12), \emph{la Verdad} (cf. \emph{Jn} 14, 6). El que cree en él, no permanece en las tinieblas (cf. \emph{Jn} 12, 46). El discípulo de Jesús, \textquote{permanece en su palabra}, para conocer \textquote{la verdad que hace libre} (cf. \emph{Jn} 8, 31-32) y que santifica (cf. \emph{Jn} 17, 17). Seguir a Jesús es vivir del \textquote{Espíritu de verdad} (\emph{Jn} 14, 17) que el Padre envía en su nombre (cf. \emph{Jn} 14, 26) y que conduce \textquote{a la verdad completa} (\emph{Jn} 16, 13). Jesús enseña a sus discípulos el amor incondicional de la verdad: \textquote{Sea vuestro lenguaje: \textquote{sí, sí}; \textquote{no, no}} (\emph{Mt} 5, 37).

\textbf{2715} La oración contemplativa es mirada de fe, fijada en Jesús. \textquote{Yo le miro y él me mira}, decía a su santo cura un campesino de Ars que oraba ante el Sagrario (cf. F. Trochu, \emph{Le Curé d'Ars Saint Jean-Marie Vianney}). Esta atención a Él es renuncia a \textquote{mí}. Su mirada purifica el corazón. La luz de la mirada de Jesús ilumina los ojos de nuestro corazón; nos enseña a ver todo a la luz de su verdad y de su compasión por todos los hombres. La contemplación dirige también su mirada a los misterios de la vida de Cristo. Aprende así el \textquote{conocimiento interno del Señor} para más amarle y seguirle (cf. San Ignacio de Loyola, \emph{Exercitia spiritualia,} 104).

La Iglesia, sacramento de la unidad del género humano

CEC 60, 442, 674, 755, 767, 774-776, 781, 831:

\textbf{60} El pueblo nacido de Abraham será el depositario de la promesa hecha a los patriarcas, el pueblo de la elección (cf. \emph{Rm} 11,28), llamado a preparar la reunión un día de todos los hijos de Dios en la unidad de la Iglesia (cf. \emph{Jn} 11,52; 10,16); ese pueblo será la raíz en la que serán injertados los paganos hechos creyentes (cf. \emph{Rm} 11,17-18.24).

\textbf{442} No ocurre así con Pedro cuando confiesa a Jesús como \textquote{el Cristo, el Hijo de Dios vivo} (\emph{Mt} 16, 16) porque Jesús le responde con solemnidad \textquote{\emph{no te ha revelado} esto ni la carne ni la sangre, sino \emph{mi Padre} que está en los cielos} (\emph{Mt} 16, 17). Paralelamente Pablo dirá a propósito de su conversión en el camino de Damasco: \textquote{Cuando Aquel que me separó desde el seno de mi madre y me llamó por su gracia, tuvo a bien revelar en mí a su Hijo para que le anunciase entre los gentiles\ldots{}} (\emph{Ga} 1,15-16). \textquote{Y en seguida se puso a predicar a Jesús en las sinagogas: que él era el Hijo de Dios} (\emph{Hch} 9, 20). Este será, desde el principio (cf. \emph{1 Ts} 1, 10), el centro de la fe apostólica (cf. \emph{Jn} 20, 31) profesada en primer lugar por Pedro como cimiento de la Iglesia (cf. \emph{Mt} 16, 18).

\textbf{674} La venida del Mesías glorioso, en un momento determinado de la historia (cf. \emph{Rm} 11, 31), se vincula al reconocimiento del Mesías por \textquote{todo Israel} (\emph{Rm} 11, 26; \emph{Mt} 23, 39) del que \textquote{una parte está endurecida} (\emph{Rm} 11, 25) en \textquote{la incredulidad} (\emph{Rm} 11, 20) respecto a Jesús. San Pedro dice a los judíos de Jerusalén después de Pentecostés: \textquote{Arrepentíos, pues, y convertíos para que vuestros pecados sean borrados, a fin de que del Señor venga el tiempo de la consolación y envíe al Cristo que os había sido destinado, a Jesús, a quien debe retener el cielo hasta el tiempo de la restauración universal, de que Dios habló por boca de sus profetas} (\emph{Hch} 3, 19-21). Y san Pablo le hace eco: \textquote{si su reprobación ha sido la reconciliación del mundo ¿qué será su readmisión sino una resurrección de entre los muertos?} (\emph{Rm} 11, 5). La entrada de \textquote{la plenitud de los judíos} (\emph{Rm} 11, 12) en la salvación mesiánica, a continuación de \textquote{la plenitud de los gentiles} (Rm 11, 25; cf. Lc 21, 24), hará al pueblo de Dios \textquote{llegar a la plenitud de Cristo} (\emph{Ef} 4, 13) en la cual \textquote{Dios será todo en nosotros} (\emph{1 Co} 15, 28).

\textbf{755} \textquote{La Iglesia es \emph{labranza} o campo de Dios (\emph{1 Co} 3, 9). En este campo crece el antiguo olivo cuya raíz santa fueron los patriarcas y en el que tuvo y tendrá lugar la reconciliación de los judíos y de los gentiles (\emph{Rm} 11, 13-26). El labrador del cielo la plantó como viña selecta (\emph{Mt} 21, 33-43 par.; cf. \emph{Is} 5, 1-7). La verdadera vid es Cristo, que da vida y fecundidad a los sarmientos, es decir, a nosotros, que permanecemos en él por medio de la Iglesia y que sin él no podemos hacer nada (\emph{Jn} 15, 1-5)}. (LG 6).

\textbf{\\ }

\textbf{La Iglesia, manifestada por el Espíritu Santo}

\textbf{767} \textquote{Cuando el Hijo terminó la obra que el Padre le encargó realizar en la tierra, fue enviado el Espíritu Santo el día de Pentecostés para que santificara continuamente a la Iglesia} (LG 4). Es entonces cuando \textquote{la Iglesia se manifestó públicamente ante la multitud; se inició la difusión del Evangelio entre los pueblos mediante la predicación} (AG 4). Como ella es \textquote{convocatoria} de salvación para todos los hombres, la Iglesia es, por su misma naturaleza, misionera enviada por Cristo a todas las naciones para hacer de ellas discípulos suyos (cf. \emph{Mt} 28, 19-20; AG 2,5-6).

\textbf{La Iglesia, sacramento universal de la salvación}

\textbf{774} La palabra griega \emph{mysterion} ha sido traducida en latín por dos términos: \emph{mysterium} y \emph{sacramentum}. En la interpretación posterior, el término \emph{sacramentum} expresa mejor el signo visible de la realidad oculta de la salvación, indicada por el término \emph{mysterium}. En este sentido, Cristo es Él mismo el Misterio de la salvación: \emph{Non est enim aliud Dei mysterium, nisi Christus} (\textquote{No hay otro misterio de Dios fuera de Cristo}; san Agustín, \emph{Epistula} 187, 11, 34). La obra salvífica de su humanidad santa y santificante es el sacramento de la salvación que se manifiesta y actúa en los sacramentos de la Iglesia (que las Iglesias de Oriente llaman también \textquote{los santos Misterios}). Los siete sacramentos son los signos y los instrumentos mediante los cuales el Espíritu Santo distribuye la gracia de Cristo, que es la Cabeza, en la Iglesia que es su Cuerpo. La Iglesia contiene, por tanto, y comunica la gracia invisible que ella significa. En este sentido analógico ella es llamada \textquote{sacramento}.

\textbf{775} \textquote{La Iglesia es en Cristo como un sacramento o signo e instrumento de la unión íntima con Dios y de la unidad de todo el género humano} (LG 1): Ser el \emph{sacramento de la unión íntima de los hombres con Dios} es el primer fin de la Iglesia. Como la comunión de los hombres radica en la unión con Dios, la Iglesia es también el sacramento de la \emph{unidad del género humano}. Esta unidad ya está comenzada en ella porque reúne hombres \textquote{de toda nación, raza, pueblo y lengua} (\emph{Ap} 7, 9); al mismo tiempo, la Iglesia es \textquote{signo e instrumento} de la plena realización de esta unidad que aún está por venir.

\textbf{776} Como sacramento, la Iglesia es instrumento de Cristo. Ella es asumida por Cristo \textquote{como instrumento de redención universal} (LG 9), \textquote{sacramento universal de salvación} (LG 48), por medio del cual Cristo \textquote{manifiesta y realiza al mismo tiempo el misterio del amor de Dios al hombre} (GS 45, 1). Ella \textquote{es el proyecto visible del amor de Dios hacia la humanidad} (Pablo VI, \emph{Discurso a los Padres del Sacro Colegio Cardenalicio}, 22 junio 1973) que quiere \textquote{que todo el género humano forme un único Pueblo de Dios, se una en un único Cuerpo de Cristo, se coedifique en un único templo del Espíritu Santo} (AG 7; cf. LG 17).

\textbf{781} \textquote{En todo tiempo y lugar ha sido grato a Dios el que le teme y practica la justicia. Sin embargo, quiso santificar y salvar a los hombres no individualmente y aislados, sin conexión entre sí, sino hacer de ellos un pueblo para que le conociera de verdad y le sirviera con una vida santa. Eligió, pues, a Israel para pueblo suyo, hizo una alianza con él y lo fue educando poco a poco. Le fue revelando su persona y su plan a lo largo de su historia y lo fue santificando. Todo esto, sin embargo, sucedió como preparación y figura de su alianza nueva y perfecta que iba a realizar en Cristo [\ldots{}], es decir, el Nuevo Testamento en su sangre, convocando a las gentes de entre los judíos y los gentiles para que se unieran, no según la carne, sino en el Espíritu} (LG 9).

\textbf{831} {[}La Iglesia{]} es católica porque ha sido enviada por Cristo en misión a la totalidad del género humano (cf. \emph{Mt} 28, 19):

\textquote{Todos los hombres están invitados al Pueblo de Dios. Por eso este pueblo, uno y único, ha de extenderse por todo el mundo a través de todos los siglos, para que así se cumpla el designio de Dios, que en el principio creó una única naturaleza humana y decidió reunir a sus hijos dispersos [\ldots{}] Este carácter de universalidad, que distingue al pueblo de Dios, es un don del mismo Señor. Gracias a este carácter, la Iglesia Católica tiende siempre y eficazmente a reunir a la humanidad entera con todos sus valores bajo Cristo como Cabeza, en la unidad de su Espíritu} (LG 13).



\chapter{Bautismo del Señor (A)}

\section{Lecturas}

PRIMERA LECTURA

Del libro del profeta Isaías 42, 1-4. 6-7

Mirad a mi siervo, en quien me complazco

Esto dice el Señor:

«Mirad a mi Siervo, a quien sostengo;

mi elegido, en quien me complazco.

He puesto mi espíritu sobre él,

manifestará la justicia a las naciones.

No gritará, no clamará,

no voceará por las calles.

La caña cascada no la quebrará,

la mecha vacilante no la apagará.

Manifestará la justicia con verdad.

No vacilará ni se quebrará,

hasta implantar la justicia en el país.

En su ley esperan las islas.

«Yo, el Señor,

te he llamado en mi justicia,

te cogí de la mano,

te formé e hice de ti alianza de un pueblo

y luz de las naciones,

para que abras los ojos de los ciegos,

saques a los cautivos de la cárcel,

de la prisión a los que habitan en tinieblas».

SALMO RESPONSORIAL

Salmo 28, 1b y 2. 3ac-4. 3b y 9c-10

El Señor bendice a su pueblo con la paz

℣. Hijos de Dios, aclamad al Señor,

aclamad la gloria del nombre del Señor,

postraos ante el Señor en el atrio sagrado. ℟.

℣. La voz del Señor sobre las aguas,

el Señor sobre las aguas torrenciales.

La voz del Señor es potente,

la voz del Señor es magnífica. ℟.

℣. El Dios de la gloria ha tronado.

En su templo un grito unánime: \textquote{¡Gloria!}

El Señor se sienta sobre las aguas del diluvio,

el Señor se sienta como rey eterno. ℟.

SEGUNDA LECTURA

De los Hechos de los Apóstoles 10, 34-38

Ungido por Dios con la fuerza del Espíritu Santo

En aquellos días, Pedro tomó la palabra y dijo:

«Ahora comprendo con toda verdad que Dios no hace acepción de personas,
sino que acepta al que lo teme y practica la justicia, sea de la nación
que sea. Envió su palabra a los hijos de Israel, anunciando la Buena
Nueva de la paz que traería Jesucristo, el Señor de todos.

Vosotros conocéis lo que sucedió en toda Judea, comenzando por Galilea,
después del bautismo que predicó Juan. Me refiero a Jesús de Nazaret,
ungido por Dios con la fuerza del Espíritu Santo, que pasó haciendo el
bien y curando a todos los oprimidos por el diablo, porque Dios estaba
con él».

EVANGELIO

Del Santo Evangelio según san Mateo 3, 13-17

Se bautizó Jesús y vio que el Espíritu de Dios se posaba sobre él

En aquel tiempo, vino Jesús desde Galilea al Jordán y se presentó a Juan
para que lo bautizara.

Pero Juan intentaba disuadirlo diciéndole:

\textquote{Soy yo el que necesito que tú me bautices, ¿y tú acudes a mí?}.

Jesús le contestó:

\textquote{Déjalo ahora. Conviene que así cumplamos toda justicia}.

Entonces Juan se lo permitió. Apenas se bautizó Jesús, salió del agua;
se abrieron los cielos y vio que el Espíritu de Dios bajaba como una
paloma y se posaba sobre él.

Y vino una voz de los cielos que decía:

«Este es mi Hijo amado, en quien me
complazco».

\section{Comentarios Patrísticos}

\subsection{San Gregorio de Neocesarea, obispo}

Vino a nosotros el que es el esplendor de la gloria del Padre

Homilía 4 en la santa Teofanía: PG 10, 1182-1183.

Estando tú presente, me es imposible callar, pues yo soy la voz, y precisamente \emph{la voz que grita en el desierto: preparad el camino del Señor. Soy yo el que necesita que tú me bautices, ¿y tú acudes a mí?} Al nacer, yo hice fecunda la esterilidad de la madre que me engendró, y, cuando todavía era un niño, procuré medicina a la mudez de mi padre, recibiendo de ti, niño, la gracia de hacer milagros.

Por tu parte, nacido de María la Virgen según quisiste y de la manera que tú solo conociste, no menoscabaste su virginidad, sino que la preservaste y se la regalaste junto con el apelativo de Madre. Ni la virginidad obstaculizó tu nacimiento ni el nacimiento lesionó la virginidad, sino que ambas realidades: nacimiento y virginidad ---realidades contradictorias si las hay---, firmaron un pacto, porque para ti, Creador de la naturaleza, esto es fácil y hacedero.

Yo soy solamente hombre, partícipe de la gracia divina; tú, en cambio, eres a la vez Dios y hombre, pues eres benigno y amas con locura el género humano. \emph{Soy yo el que necesita que tú me bautices, ¿y tú acudes a mí?} Tú que eras al principio, y estabas junto a Dios y eras Dios mismo; tú que eres el esplendor de la gloria del Padre; tú que eres la imagen perfecta del padre perfecto; tú que eres \emph{la luz verdadera, que alumbra a todo hombre que viene a este mundo;} tú que para estar en el mundo viniste donde ya estabas; tú que te hiciste carne sin convertirte en carne; tú que acampaste entre nosotros y te hiciste visible a tus siervos en la condición de esclavo; tú que, con tu santo nombre como con un puente, uniste el cielo y la tierra: ¿tú acudes a mí? ¿Tú, tan grande, a un hombre como yo?, ¿el Rey al precursor?, ¿el Señor al siervo?

Pues aunque tú no te hayas avergonzado de nacer en las humildes condiciones de la humanidad, yo no puedo traspasar los límites de la naturaleza. Tengo conciencia del abismo que separa la tierra del Creador. Conozco la diferencia existente entre el polvo de la tierra y el Hacedor. Soy consciente de que la claridad de tu sol de justicia me supera con mucho a mí, que soy la lámpara de tu gracia. Y aun cuando estés revestido de la blanca nube del cuerpo, reconozco no obstante tu dominación. Confieso mi condición servil y proclamo tu magnificencia. Reconozco la perfección de tu dominio, y conozco mi propia abyección y vileza. \emph{No soy digno de desatar la correa de tu sandalia;} ¿cómo, pues, voy a atreverme a tocar la inmaculada coronilla de tu cabeza? ¿Cómo voy a extender sobre ti mi mano derecha, sobre ti que extendiste los cielos como una tienda y cimentaste sobre las aguas la tierra? ¿Cómo abriré mi mano de siervo sobre tu divina cabeza? ¿Cómo lavar al inmaculado y exento de todo pecado? ¿Cómo iluminar a la misma luz? ¿Qué oración pronunciaré sobre ti, sobre ti que acoges incluso las plegarias de los que no te conocen?

\subsection{San Pedro Crisólogo, obispo}

El que por nosotros quiso nacer

no quiso ser ignorado por nosotros

Sermón 160: PL 52, 620-622.

Aunque en el mismo misterio del nacimiento del Señor se dieron insignes testimonios de su divinidad, sin embargo, la solemnidad que celebramos manifiesta y revela de diversas formas que Dios ha asumido un cuerpo humano, para que nuestra inteligencia, ofuscada por tantas obscuridades, no pierda por su ignorancia lo que por gracia ha merecido recibir y poseer.

Pues el que por nosotros quiso nacer no quiso ser ignorado por nosotros; y por esto se manifestó de tal forma que el gran misterio de su bondad no fuera ocasión de un gran error.

Hoy el mago encuentra llorando en la cuna a aquel que, resplandeciente, buscaba en las estrellas. Hoy el mago contempla claramente entre pañales a aquel que, encubierto, buscaba pacientemente en los astros.

Hoy el mago discierne con profundo asombro lo que allí contempla: el cielo en la tierra, la tierra en el cielo; el hombre en Dios, y Dios en el hombre; y a aquel que no puede ser encerrado en todo el universo incluido en un cuerpo de niño. Y, viendo, cree y no duda; y lo proclama con sus dones místicos: el incienso para Dios, el oro para el Rey, y la mirra para el que morirá.

Hoy el gentil, que era el último, ha pasado a ser el primero, pues entonces la fe de los magos consagró la creencia de las naciones.

Hoy Cristo ha entrado en el cauce del Jordán para lavar el pecado del mundo. El mismo Juan atestigua que Cristo ha venido para esto: \emph{Éste es el Cordero de Dios, que quita el pecado del mundo}. Hoy el siervo recibe al Señor, el hombre a Dios, Juan a Cristo; el que no puede dar el perdón recibe a quien se lo concederá.

Hoy, como afirma el profeta, \emph{la voz del Señor sobre las aguas}. ¿Qué voz? \emph{Este es mi Hijo, el amado, mi predilecto}.

Hoy el Espíritu Santo se cierne sobre las aguas en forma de paloma, para que, así como la paloma de Noé anunció el fin del diluvio, de la misma forma ésta fuera signo de que ha terminado el perpetuo naufragio del mundo. Pero a diferencia de aquélla, que sólo llevaba un ramo de olivo caduco, ésta derramará la enjundia completa del nuevo crisma en la cabeza del Autor de la nueva progenie, para que se cumpliera aquello que predijo el profeta: \emph{Por eso el Señor, tu Dios, te ha ungido con aceite de júbilo entre todos tus compañeros}.

Hoy Cristo, al convertir el agua en vino, comienza los signos celestes. Pero el agua había de convertirse en el misterio de la sangre, para que Cristo ofreciese a los que tienen sed la pura bebida del vaso de su cuerpo, y se cumpliese lo que dice el profeta: \emph{Y mi copa rebosa}.

El Hijo llamó a su siervo Juan,

que se acercó y puso su mano derecha

sobre la cabeza de Aquél que lo había creado:

«¿Qué podré decir,

cómo podré yo bautizarte, Señor mío?

Si digo: en el nombre del Padre,

he aquí que tu estás en tu Padre.

Si digo: en el nombre del Hijo:

he aquí que tú eres este Hijo amado.

Y si digo: en el nombre del Espíritu,

este espíritu está contigo.»

\ldots{}

El Hijo que ha creado toda la creación,

ha sido bautizado y ha emergido de las aguas.

He aquí el Esposo.

Olvida tu pueblo y la casa paterna,

porque el rey se alegra de tu belleza.

(De la Liturgia Sirio-Occidental).


\section{Homilías}

\subsection{Juan Pablo II, papa}

\subsubsection{Homilía: Hijos de Dios por el Bautismo}

Domingo 11 de enero de 1981.

\textquote{Este es mi Hijo muy amado, en quien tengo mis complacencias} (\emph{Mt} 3, 17).

Las palabras del \textbf{Evangelio} que acabamos de oír van a realizarse también en estos queridos niños a quienes me dispongo a administrar el bautismo. Jesús es el primogénito de muchos hermanos (cf. \emph{Rom} 8, 29); lo que se realizó en El se repite misteriosamente en cada uno de los que seguimos sus huellas y llevamos su nombre, el nombre de cristianos.

Cuando Cristo entra en el Jordán, se oye la voz del Padre que lo llama predilecto suyo y se da cumplimiento así a la profecía del Siervo de Yavé que Isaías proclama en la primera lectura; y desciende el Espíritu Santo en forma de paloma para dar comienzo visible y solemne a la misión mesiánica del Hijo de Dios. Como en El, así también ha ocurrido en nosotros; así va a suceder en estos pequeños que están aquí ante nosotros, generación nueva del Pueblo de Dios destinada a crecer continuamente en el mundo gracias a las familias cristianas. También sobre ellos el Padre va a dejar oír su voz: \textquote{Este es mi Hijo muy amado, en quien tengo mis complacencias}. El Padre se complace en estos recién nacidos porque verá impresa en su espíritu la huella inmortal de su paternidad, la semejanza íntima y verdadera con su Hijo: hijos en el Hijo. Y al mismo tiempo descenderá el Espíritu Santo invisible y a la vez presente como entonces, para colmar a estas pequeñas almas de la riqueza de sus dones, para convertirlos en morada suya, templos suyos, manifestadores suyos que deberán irradiar su presencia y testimoniarlo a lo largo de la vida, vida que nosotros no sabemos todavía cómo será, pero que El ya ve en toda su plenitud.

Vamos a poner los cimientos de nuevas vidas cristianas amadas del Padre, redimidas por Cristo, marcadas por el Espíritu Santo, objeto de una predilección eterna que se proyecta ya desde ahora hacia el futuro y a la eternidad entera en un Amor sin fin que los abraza desde ahora: \textquote{Este es mi Hijo muy amado, en quien tengo mis complacencias}.

Sobre estos hijos predilectos, que dentro de poco serán los brotes nuevos de la Iglesia, blancos con la inocencia total de la gracia simbolizada en el manto que les impondré luego, fuertes como auténticos atletas con la unción del óleo de los catecúmenos, santos con la santidad misma de Dios, invoco con vosotros la ayuda continua del Señor y hago votos para que sean fieles siempre durante toda la vida a esta nuestra común y estupenda vocación cristiana.

Con tal fin los confío a vosotros, padres cristianos, que con vuestro amor y entrega mutuos les habéis dado la vida, convirtiéndoos en colaboradores de la creación de Dios. Los habéis traído aquí siendo hijos de la naturaleza, y os los lleváis a casa hijos de la gracia. De vosotros depende gran parte de su realización plena según los planes de Dios, ¡a vosotros los confío en nombre de Dios Trinidad!

Y los confío también a vosotros, padrinos y madrinas, con la misma finalidad de que garanticéis su crecimiento cristiano completo.

Sobre todos descienda la bendición del Señor, de la que es prenda y auspicio mi bendición apostólica.

\subsubsection{Homilía: Jesús entre la multitud penitente}

Domingo 10 de enero de 1999.

1. \textquote{Éste es mi Hijo amado, en quien tengo mis complacencias} (Mt 3, 17).

En la fiesta del Bautismo del Señor, que estamos celebrando, resuenan estas palabras solemnes. Nos invitan a revivir el momento en que Jesús, bautizado por Juan, sale de las aguas del río Jordán y Dios Padre lo presenta como su Hijo unigénito, el Cordero que toma sobre sí el pecado del mundo. Se oye una voz del cielo, mientras el Espíritu Santo, en forma de paloma, se posa sobre Jesús, que comienza públicamente su misión salvífica; misión que se caracteriza por el estilo del \emph{siervo humilde y manso}, dispuesto a compartir y entregarse totalmente: \textquote{No gritará, no clamará. (\ldots{}) No quebrará la caña cascada, no apagará el pabilo vacilante. Promoverá fielmente el derecho} (\emph{Is} 42, 2-3).

La liturgia nos hace revivir la sugestiva \textbf{escena evangélica}: entre la multitud penitente que avanza hacia Juan el Bautista para recibir el bautismo está también Jesús. La promesa está a punto de cumplirse y se abre una nueva era para toda la humanidad. Este hombre, que aparentemente no es diferente de todos los demás, en realidad es Dios, que viene a nosotros para dar a cuantos lo reciban el poder de \textquote{convertirse en hijos de Dios, a los que creen en su nombre; los cuales no nacieron de sangre, ni de deseo de hombre, sino que nacieron de Dios} (\emph{Jn} 1, 12-13).

2. \textquote{Éste es mi Hijo amado; escuchadle} (\emph{Aleluya}).

Hoy, este anuncio y esta invitación, llenos de esperanza para la humanidad, resuenan particularmente para los niños que, dentro de poco, mediante el sacramento del bautismo, se convertirán en nuevas criaturas. Al participar en el misterio de la muerte y resurrección de Cristo, se enriquecerán con el don de la fe y se incorporarán al pueblo de la nueva y definitiva alianza, que es la Iglesia. El Padre los hará en Cristo hijos adoptivos suyos, revelándoles un singular proyecto de vida: escuchar como discípulos a su Hijo, para ser llamados y ser realmente sus hijos.

Sobre cada uno de ellos bajará el Espíritu Santo y, como sucedió con nosotros el día de nuestro bautismo, también ellos gozarán de la vida que el Padre da a los creyentes por medio de Jesús, el Redentor del hombre. Esta riqueza tan grande de dones les exigirá, como a todo bautizado, una única tarea, que el apóstol Pablo no se cansa de indicar a los primeros cristianos con las palabras: \textquote{Caminad según el Espíritu} (\emph{Ga} 5, 16), es decir, vivid y obrad constantemente en el amor a Dios.

Expreso mis mejores deseos de que el bautismo, que hoy reciben estos niños, los convierta a lo largo de toda su vida en valientes testigos del Evangelio. Esto será posible gracias a su empeño constante. Pero también será necesaria vuestra labor educativa, queridos padres, que hoy dais gracias a Dios por los dones extraordinarios que concede a estos hijos vuestros, del mismo modo que será necesario el apoyo de sus padrinos y sus madrinas.

3. Amadísimos hermanos y hermanas, aceptad la invitación que la Iglesia os hace: sed sus \textquote{educadores en la fe}, para que se desarrolle en ellos el germen de la vida nueva y llegue a su plena madurez. Ayudadles con vuestras palabras y, sobre todo, con vuestro ejemplo.

Que aprendan pronto de vosotros a amar a Cristo, a invocarlo sin cesar, y a imitarlo con constante adhesión a su llamada. En su nombre habéis recibido, con el símbolo del cirio, la llama de la fe: cuidad de que esté continuamente alimentada, para que cada uno de ellos, conociendo y amando a Jesús, obre siempre según la sabiduría evangélica. De este modo, llegarán a ser verdaderos discípulos del Señor y apóstoles alegres de su Evangelio.

Encomiendo a la Virgen María a cada uno de estos niños y a sus respectivas familias. Que la Virgen ayude a todos a recorrer con fidelidad el camino inaugurado con el sacramento del bautismo.

\subsubsection{Homilía: Cooperadores de la paternidad divina}

Domingo 13 de enero del 2002.

1. \textquote{Este es mi Hijo, el amado, mi predilecto} (Mt 3, 17).

Acabamos de escuchar de nuevo en el \textbf{evangelio} las palabras que resonaron en el cielo cuando Jesús fue bautizado por Juan en el río Jordán. Las pronunció una voz desde lo alto: la voz de Dios Padre. Revelan el misterio que celebramos hoy, el bautismo de Cristo. El Hombre sobre el que desciende, en forma de paloma, el Espíritu Santo es el Hijo de Dios, que tomó de la Virgen María nuestra carne para redimirla del pecado y de la muerte.

¡Grande es este \emph{misterio de salvación}! Misterio en el que se insertan hoy los niños que presentáis, queridos padres, padrinos y madrinas. Al recibir en la Iglesia el sacramento del bautismo, se convierten en hijos de Dios, \emph{hijos en el Hijo}. Es el misterio del \textquote{segundo nacimiento}.

2. Queridos padres, me dirijo con afecto especialmente a vosotros, que habéis dado la vida a estas criaturas, colaborando en la obra de Dios, autor de la vida y, de modo singular, de toda vida humana. Los habéis engendrado y hoy los presentáis a la fuente bautismal, \emph{para que vuelvan a nacer por el agua y por el Espíritu Santo}. La gracia de Cristo transformará su existencia de mortal en inmortal, liberándola del pecado original. Dad gracias al Señor por el don de su nacimiento y del nuevo nacimiento espiritual de hoy.

Pero ¿cuál fuerza permite a estos inocentes e inconscientes niños realizar un \textquote{paso} espiritual tan profundo? Es la \emph{fe}, la fe de la Iglesia, profesada en particular por vosotros, queridos padres, padrinos y madrinas. Precisamente en esta fe son bautizados vuestros hijos. Cristo no realiza el milagro de regenerar al hombre sin la colaboración del hombre mismo, y la primera cooperación de la criatura humana es la fe, con la que, atraída interiormente por Dios, se abandona libremente en sus manos.

Estos niños reciben hoy el bautismo sobre la base de vuestra fe, que dentro de poco os pediré profesar. ¡Cuánto amor, amadísimos hermanos, cuánta responsabilidad implica el gesto que realizaréis en nombre de vuestros hijos!

3. En el futuro, cuando sean capaces de comprender, ellos mismos deberán recorrer, personal y libremente, un camino espiritual que, con la gracia de Dios, los llevará a \emph{confirmar}, en el sacramento de la confirmación, el don que reciben hoy.

Pero ¿podrán abrirse a la fe si los adultos que los rodean no les dan un buen testimonio? Estos niños os necesitan, ante todo, a vosotros, queridos padres; os necesitan también a vosotros, queridos padrinos y madrinas, para aprender a conocer al verdadero Dios, que es amor misericordioso. A vosotros os corresponde introducirlos en este conocimiento, en primer lugar a través del testimonio de vuestro comportamiento en las relaciones con ellos y con los demás, relaciones que se han de caracterizar por la atención, la acogida y el perdón. Comprenderán que Dios es fidelidad si pueden reconocer su reflejo, aunque sea limitado y débil, ante todo en vuestra presencia amorosa.

Es grande la responsabilidad de la cooperación de los padres en el crecimiento espiritual de sus hijos. Eran muy conscientes de esa responsabilidad los beatos esposos Luis y María Beltrame Quattrocchi, a los que recientemente tuve la alegría de elevar al honor de los altares y que os exhorto a conocer mejor y a imitar. Si ya es grande vuestra misión de ser padres \textquote{según la carne}, ¡cuánto más lo es la de \emph{colaborar en la paternidad divina}, dando vuestra contribución para modelar en estas criaturas la imagen misma de Jesús, Hombre perfecto!

4. \emph{Nunca os sintáis solos} en esta misión tan comprometedora. Os conforte, ante todo, la confianza en el ángel de la guarda, al que Dios ha encomendado su singular mensaje de amor para cada uno de vuestros hijos. Además, toda la Iglesia, a la que tenéis la gracia de pertenecer, está comprometida a asistiros: en el cielo velan los santos, en particular aquellos cuyos nombres tienen estos niños y que serán sus \textquote{patronos}. En la tierra está la comunidad eclesial, en la que es posible fortalecer la propia fe y la propia vida cristiana, alimentándola con la oración y los sacramentos. No podréis dar a vuestros hijos lo que vosotros no habéis recibido y asimilado antes.

Además, todos tenemos una Madre según el Espíritu: María santísima. A ella le encomiendo a vuestros hijos, para que lleguen a ser cristianos auténticos; a María os encomiendo también a vosotros, queridos padres, queridos padrinos y madrinas, para que transmitáis siempre a estos niños el amor que necesitan para \emph{crecer} y para \emph{creer}. En efecto, \emph{la vida y la fe caminan juntas}.

Que así sea en la existencia de cada bautizado con la ayuda de Dios.

\subsection{Benedicto XVI, papa}

\subsubsection{Homilía: Se sumergió en nuestra muerte}

Domingo 13 de enero del 2008.

La celebración de hoy es siempre para mí motivo de especial alegría. En efecto, administrar el sacramento del bautismo en el día de la fiesta del Bautismo del Señor es, en realidad, uno de los momentos más expresivos de nuestra fe, en la que podemos ver de algún modo, a través de los signos de la liturgia, el misterio de la vida. En primer lugar, la vida humana, representada aquí en particular por estos trece niños que son el fruto de vuestro amor, queridos padres, a los cuales dirijo mi saludo cordial, extendiéndolo a los padrinos, a las madrinas y a los demás parientes y amigos presentes. Está, luego, el misterio de la vida divina, que hoy Dios dona a estos pequeños mediante el renacimiento por el agua y el Espíritu Santo. Dios es vida, como está representado estupendamente también en algunas pinturas que embellecen esta Capilla Sixtina.

Sin embargo, no debe parecernos fuera de lugar comparar inmediatamente la experiencia de la vida con la experiencia opuesta, es decir, con la realidad de la muerte. Todo lo que comienza en la tierra, antes o después termina, como la hierba del campo, que brota por la mañana y se marchita al atardecer. Pero en el bautismo el pequeño ser humano recibe una vida nueva, la vida de la gracia, que lo capacita para entrar en relación personal con el Creador, y esto para siempre, para toda la eternidad.

Por desgracia, el hombre es capaz de apagar esta nueva vida con su pecado, reduciéndose a una situación que la sagrada Escritura llama \textquote{segunda muerte}. Mientras que en las demás criaturas, que no están llamadas a la eternidad, la muerte significa solamente el fin de la existencia en la tierra, en nosotros el pecado crea una vorágine que amenaza con tragarnos para siempre, si el Padre que está en los cielos no nos tiende su mano.

Este es, queridos hermanos, el misterio del bautismo: Dios ha querido salvarnos yendo él mismo hasta el fondo del abismo de la muerte, con el fin de que todo hombre, incluso el que ha caído tan bajo que ya no ve el cielo, pueda encontrar la mano de Dios a la cual asirse a fin de subir desde las tinieblas y volver a ver la luz para la que ha sido creado. Todos sentimos, todos percibimos interiormente que nuestra existencia es un deseo de vida que invoca una plenitud, una salvación. Esta plenitud de vida se nos da en el bautismo.

Acabamos de oír el \textbf{relato del bautismo de Jesús en el Jordán}. Fue un bautismo diverso del que estos niños van a recibir, pero tiene una profunda relación con él. En el fondo, todo el misterio de Cristo en el mundo se puede resumir con esta palabra: \textquote{bautismo}, que en griego significa \textquote{inmersión}. El Hijo de Dios, que desde la eternidad comparte con el Padre y con el Espíritu Santo la plenitud de la vida, se \textquote{sumergió} en nuestra realidad de pecadores para hacernos participar en su misma vida: se encarnó, nació como nosotros, creció como nosotros y, al llegar a la edad adulta, manifestó su misión iniciándola precisamente con el \textquote{bautismo de conversión}, que recibió de Juan el Bautista. Su primer acto público, como acabamos de escuchar, fue bajar al Jordán, entre los pecadores penitentes, para recibir aquel bautismo. Naturalmente, Juan no quería, pero Jesús insistió, porque esa era la voluntad del Padre (cf. \emph{Mt} 3, 13-15).

¿Por qué el Padre quiso eso? ¿Por qué mandó a su Hijo unigénito al mundo como Cordero para que tomara sobre sí el pecado del mundo? (cf. \emph{Jn} 1, 29). El evangelista narra que, cuando Jesús salió del agua, se posó sobre él el Espíritu Santo en forma de paloma, mientras la voz del Padre desde el cielo lo proclamaba \textquote{Hijo predilecto} (\emph{Mt} 3, 17). Por tanto, desde aquel momento Jesús fue revelado como aquel que venía para bautizar a la humanidad en el Espíritu Santo: venía a traer a los hombres la vida en abundancia (cf. \emph{Jn} 10, 10), la vida eterna, que resucita al ser humano y lo sana en su totalidad, cuerpo y espíritu, restituyéndolo al proyecto originario para el cual fue creado.

El fin de la existencia de Cristo fue precisamente dar a la humanidad la vida de Dios, su Espíritu de amor, para que todo hombre pueda acudir a este manantial inagotable de salvación. Por eso san Pablo escribe a los Romanos que hemos sido bautizados en la muerte de Cristo para tener su misma vida de resucitado (cf. \emph{Rm} 6, 3-4). Y por eso mismo los padres cristianos, como hoy vosotros, tan pronto como les es posible, llevan a sus hijos a la pila bautismal, sabiendo que la vida que les han transmitido invoca una plenitud, una salvación que sólo Dios puede dar. De este modo los padres se convierten en colaboradores de Dios no sólo en la transmisión de la vida física sino también de la vida espiritual a sus hijos.

Queridos padres, juntamente con vosotros doy gracias al Señor por el don de estos niños e invoco su asistencia para que os ayude a educarlos y a insertarlos en el Cuerpo espiritual de la Iglesia. A la vez que les ofrecéis lo que es necesario para el crecimiento y para la salud, vosotros, con la ayuda de los padrinos, os habéis comprometido a desarrollar en ellos la fe, la esperanza y la caridad, las virtudes teologales que son propias de la vida nueva que han recibido con el sacramento del bautismo.

Aseguraréis esto con vuestra presencia, con vuestro afecto; y lo aseguraréis, ante todo y sobre todo, con la oración, presentándolos diariamente a Dios, encomendándolos a él en cada etapa de su existencia. Ciertamente, para crecer sanos y fuertes, estos niños y niñas necesitarán cuidados materiales y muchas atenciones; pero lo que les será más necesario, más aún indispensable, es conocer, amar y servir fielmente a Dios, para tener la vida eterna. Queridos padres, sed para ellos los primeros testigos de una fe auténtica en Dios.

En el rito del bautismo hay un signo elocuente, que expresa precisamente la transmisión de la fe: es la entrega, a cada uno de los bautizandos, de una vela encendida en la llama del cirio pascual: es la luz de Cristo resucitado que os comprometéis a transmitir a vuestros hijos. Así, de generación en generación, los cristianos nos transmitimos la luz de Cristo, de modo que, cuando vuelva, nos encuentre con esta llama ardiendo entre las manos.

Durante el rito, os diré: \textquote{A vosotros, padres y padrinos, se os confía este signo pascual, una llama que debéis alimentar siempre}. Alimentad siempre, queridos hermanos y hermanas, la llama de la fe con la escucha y la meditación de la palabra de Dios y con la Comunión asidua de Jesús Eucaristía.

Que en esta misión estupenda, aunque difícil, os ayuden los santos protectores cuyos nombres recibirán estos trece niños. Que estos santos les ayuden sobre todo a ellos, los bautizandos, a corresponder a vuestra solicitud de padres cristianos. En particular, que la Virgen María los acompañe a ellos y a vosotros, queridos padres, ahora y siempre. Amén.

\subsubsection{Ángelus: Ungido con el Espíritu Santo}

Domingo 13 de enero del 2008.

Con la fiesta del Bautismo del Señor, que celebramos hoy, se concluye el tiempo litúrgico de Navidad. El Niño, a quien los Magos de Oriente vinieron a adorar en Belén, ofreciéndole sus dones simbólicos, lo encontramos ahora adulto, en el momento en que se hace bautizar en el río Jordán por el gran profeta Juan (cf. \emph{Mt} 3, 13). El \textbf{Evangelio} narra que cuando Jesús, recibido el bautismo, salió del agua, se abrieron los cielos y bajó sobre él el Espíritu Santo en forma de paloma (cf. \emph{Mt} 3, 16). Se oyó entonces una voz del cielo que decía: \textquote{Este es mi Hijo amado, en quien me complazco} (\emph{Mt} 3, 17). Esa fue su primera manifestación pública, después de casi treinta años de vida oculta en Nazaret.

Testigos oculares de ese singular acontecimiento fueron, además del Bautista, sus discípulos, algunos de los cuales se convirtieron desde entonces en seguidores de Cristo (cf. \emph{Jn} 1, 35-40). Se trató simultáneamente de cristofanía y teofanía: ante todo, Jesús se manifestó como el \emph{Cristo}, término griego para traducir el hebreo \emph{Mesías}, que significa \textquote{ungido}. Jesús no fue ungido con óleo a la manera de los reyes y de los sumos sacerdotes de Israel, sino con el Espíritu Santo. Al mismo tiempo, junto con el Hijo de Dios aparecieron los signos del Espíritu Santo y del Padre celestial.

¿Cuál es el significado de este acto, que Jesús quiso realizar ---venciendo la resistencia del Bautista--- para obedecer a la voluntad del Padre? (cf. \emph{Mt} 3, 14-15). Su sentido profundo se manifestará sólo al final de la vida terrena de Cristo, es decir, en su muerte y resurrección. Haciéndose bautizar por Juan juntamente con los pecadores, Jesús comenzó a tomar sobre sí el peso de la culpa de toda la humanidad, como Cordero de Dios que \textquote{quita} el pecado del mundo (cf. \emph{Jn} 1, 29). Obra que consumó en la cruz, cuando recibió también su \textquote{bautismo} (cf. \emph{Lc} 12, 50). En efecto, al morir se \textquote{sumergió} en el amor del Padre y derramó el Espíritu Santo, para que los creyentes en él pudieran renacer de aquel manantial inagotable de vida nueva y eterna.

Toda la misión de Cristo se resume en esto: bautizarnos en el Espíritu Santo, para librarnos de la esclavitud de la muerte y \textquote{abrirnos el cielo}, es decir, el acceso a la vida verdadera y plena, que será \textquote{sumergirse siempre de nuevo en la inmensidad del ser, a la vez que estamos desbordados simplemente por la alegría} (\emph{Spe salvi}, 12).

Es lo que sucedió también a los trece niños a los cuales administré el sacramento del bautismo esta mañana en la capilla Sixtina. Invoquemos sobre ellos y sobre sus familiares la protección materna de María santísima. Y oremos por todos los cristianos, para que comprendan cada vez más el don del bautismo y se comprometan a vivirlo con coherencia, testimoniando el amor del Padre y del Hijo y del Espíritu Santo.

\subsubsection{Homilía: Comunión plena con la humanidad}

Domingo 9 de enero del 2011.

Me alegra daros una cordial bienvenida, en particular a vosotros, padres, padrinos y madrinas de los 21 recién nacidos a los que, dentro de poco, tendré la alegría de administrar el sacramento del Bautismo. Como ya es tradición, también este año este rito tiene lugar en la santa Eucaristía con la que celebramos el Bautismo del Señor. Se trata de la fiesta que, en el primer domingo después de la solemnidad de la Epifanía, cierra el tiempo de Navidad con la manifestación del Señor en el Jordán.

Según el relato del \textbf{evangelista san Mateo} (3, 13-17), Jesús fue de Galilea al río Jordán para que lo bautizara Juan; de hecho, acudían de toda Palestina para escuchar la predicación de este gran profeta, el anuncio de la venida del reino de Dios, y para recibir el bautismo, es decir, para someterse a ese signo de penitencia que invitaba a convertirse del pecado. Aunque se llamara bautismo, no tenía el valor sacramental del rito que celebramos hoy; como bien sabéis, con su muerte y resurrección Jesús instituye los sacramentos y hace nacer la Iglesia. El que administraba Juan era un acto penitencial, un gesto que invitaba a la humildad frente a Dios, invitaba a un nuevo inicio: al sumergirse en el agua, el penitente reconocía que había pecado, imploraba de Dios la purificación de sus culpas y se le enviaba a cambiar los comportamientos equivocados, casi como si muriera en el agua y resucitara a una nueva vida.

Por esto, cuando Juan Bautista ve a Jesús que, en fila con los pecadores, va para que lo bautice, se sorprende; al reconocer en él al Mesías, al Santo de Dios, a aquel que no tenía pecado, Juan manifiesta su desconcierto: él mismo, el que bautizaba, habría querido hacerse bautizar por Jesús. Pero Jesús lo exhorta a no oponer resistencia, a aceptar realizar este acto, para hacer lo que es conveniente para \textquote{cumplir toda justicia}. Con esta expresión Jesús manifiesta que vino al mundo para hacer la voluntad de Aquel que lo mandó, para realizar todo lo que el Padre le pide; aceptó hacerse hombre para obedecer al Padre. Este gesto revela ante todo quién es Jesús: es el Hijo de Dios, verdadero Dios como el Padre; es aquel que \textquote{se rebajó} para hacerse uno de nosotros, aquel que se hizo hombre y aceptó humillarse hasta la muerte de cruz (cf. \emph{Flp} 2, 7). El bautismo de Jesús, que hoy recordamos, se sitúa en esta lógica de la humildad y de la solidaridad: es el gesto de quien quiere hacerse en todo uno de nosotros y se pone realmente en la fila con los pecadores; él, que no tiene pecado, deja que lo traten como pecador (cf. \emph{2 Co} 5, 21), para cargar sobre sus hombros el peso de la culpa de toda la humanidad, también de nuestra culpa. Es el \textquote{siervo de Dios} del que nos habló el \textbf{profeta Isaías en la primera lectura} (cf. 42, 1). Lo que dicta su humildad es el deseo de establecer una comunión plena con la humanidad, el deseo de realizar una verdadera solidaridad con el hombre y con su condición. El gesto de Jesús anticipa la cruz, la aceptación de la muerte por los pecados del hombre. Este acto de anonadamiento, con el que Jesús quiere uniformarse totalmente al designio de amor del Padre y asemejarse a nosotros, manifiesta la plena sintonía de voluntad y de fines que existe entre las personas de la santísima Trinidad. Para ese acto de amor, el Espíritu de Dios se manifiesta como paloma y baja sobre él, y en aquel momento el amor que une a Jesús al Padre se testimonia a cuantos asisten al bautismo, mediante una voz desde lo alto que todos oyen. El Padre manifiesta abiertamente a los hombres ---a nosotros--- la comunión profunda que lo une al Hijo: la voz que resuena desde lo alto atestigua que Jesús es obediente en todo al Padre y que esta obediencia es expresión del amor que los une entre sí. Por eso, el Padre se complace en Jesús, porque reconoce en las acciones del Hijo el deseo de seguir en todo su voluntad: \textquote{Este es mi Hijo amado, en quien me complazco} (\emph{Mt} 3, 17). Y esta palabra del Padre alude también, anticipadamente, a la victoria de la resurrección y nos dice cómo debemos vivir para complacer al Padre, comportándonos como Jesús.

Queridos padres, el Bautismo que hoy pedís para vuestros hijos los inserta en este intercambio de amor recíproco que existe en Dios entre el Padre, el Hijo y el Espíritu Santo; por este gesto que voy a realizar, se derrama sobre ellos el amor de Dios, y los inunda con sus dones. Mediante el lavatorio del agua, vuestros hijos son insertados en la vida misma de Jesús, que murió en la cruz para librarnos del pecado y resucitando venció a la muerte. Por eso, inmersos espiritualmente en su muerte y resurrección, son liberados del pecado original e inicia en ellos la vida de la gracia, que es la vida misma de Jesús resucitado. \textquote{Él se entregó por nosotros ---afirma san Pablo--- a fin de rescatarnos de toda iniquidad y formar para sí un pueblo puro que fuese suyo, fervoroso en buenas obras} (\emph{Tt} 2, 14).

Queridos amigos, al darnos la fe, el Señor nos ha dado lo más precioso que existe en la vida, es decir, el motivo más verdadero y más bello por el cual vivir: por gracia hemos creído en Dios, hemos conocido su amor, con el cual quiere salvarnos y librarnos del mal. La fe es el gran don con el que nos da también la vida eterna, la verdadera vida. Ahora vosotros, queridos padres, padrinos y madrinas, pedís a la Iglesia que acoja en su seno a estos niños, que les dé el Bautismo; y esta petición la hacéis en razón del don de la fe que vosotros mismos, a vuestra vez, habéis recibido. Todo cristiano puede repetir con el \textbf{profeta Isaías}: \textquote{El Señor me plasmó desde el seno materno para siervo suyo} (cf. 49, 5); así, queridos padres, vuestros hijos son un don precioso del Señor, el cual se ha reservado para sí su corazón, para poderlo colmar de su amor. Por el sacramento del Bautismo hoy los consagra y los llama a seguir a Jesús, mediante la realización de su vocación personal según el particular designio de amor que el Padre tiene pensado para cada uno de ellos; meta de esta peregrinación terrena será la plena comunión con él en la felicidad eterna.

Al recibir el Bautismo, estos niños obtienen como don un sello espiritual indeleble, el \textquote{carácter}, que marca interiormente para siempre su pertenencia al Señor y los convierte en miembros vivos de su Cuerpo místico, que es la Iglesia. Mientras entran a formar parte del pueblo de Dios, para estos niños comienza hoy un camino que debería ser un camino de santidad y de configuración con Jesús, una realidad que se deposita en ellos como la semilla de un árbol espléndido, que es preciso ayudar a crecer. Por esto, al comprender la grandeza de este don, desde los primeros siglos se ha tenido la solicitud de dar el Bautismo a los niños recién nacidos. Ciertamente, luego será necesaria una adhesión libre y consciente a esta vida de fe y de amor, y por esto es preciso que, tras el Bautismo, sean educados en la fe, instruidos según la sabiduría de la Sagrada Escritura y las enseñanzas de la Iglesia, a fin de que crezca en ellos este germen de la fe que hoy reciben y puedan alcanzar la plena madurez cristiana. La Iglesia, que los acoge entre sus hijos, debe hacerse cargo, juntamente con los padres y los padrinos, de acompañarlos en este camino de crecimiento. La colaboración entre la comunidad cristiana y la familia es más necesaria que nunca en el contexto social actual, en el que la institución familiar se ve amenazada desde varias partes y debe afrontar no pocas dificultades en su misión de educar en la fe. La pérdida de referencias culturales estables y la rápida transformación a la cual está continuamente sometida la sociedad, hacen que el compromiso educativo sea realmente arduo. Por eso, es necesario que las parroquias se esfuercen cada vez más por sostener a las familias, pequeñas iglesias domésticas, en su tarea de transmisión de la fe.

Queridos padres, junto con vosotros doy gracias al Señor por el don del Bautismo de estos hijos vuestros; al elevar nuestra oración por ellos, invocamos el don abundante del Espíritu Santo, que hoy los consagra a imagen de Cristo sacerdote, rey y profeta. Encomendándolos a la intercesión materna de María santísima, pedimos para ellos vida y salud, para que puedan crecer y madurar en la fe, y dar, con su vida, frutos de santidad y de amor. Amén.

\subsubsection{Ángelus: Obra de la Trinidad}

Domingo 9 de enero del 2011.

Hoy la Iglesia celebra el Bautismo del Señor, fiesta que concluye el tiempo litúrgico de la Navidad. Este misterio de la vida de Cristo muestra visiblemente que su venida en la carne es el acto sublime de amor de las tres personas divinas. Podemos decir que desde este solemne acontecimiento la acción creadora, redentora y santificadora de la santísima Trinidad será cada vez más manifiesta en la misión pública de Jesús, en su enseñanza, en sus milagros, en su pasión, muerte y resurrección. En efecto, leemos en el \emph{\textbf{Evangelio según san Mateo}} que \textquote{bautizado Jesús, salió luego del agua; y en esto se abrieron los cielos y vio al Espíritu de Dios que bajaba en forma de paloma y venía sobre él. Y una voz que salía de los cielos decía: \textquote{Este es mi Hijo amado, en quien me complazco}} (3, 16-17). El Espíritu Santo \textquote{mora} en el Hijo y da testimonio de su divinidad, mientras la voz del Padre, proveniente de los cielos, expresa la comunión de amor. \textquote{La conclusión de la escena del bautismo nos dice que Jesús ha recibido esta \textquote{unción} verdadera, que él es el Ungido {[}el Cristo{]} esperado} (\emph{Jesús de Nazaret}, Madrid 2007, p. 49), como confirmación de la profecía de Isaías: \textquote{He aquí mi siervo que yo sostengo, mi elegido en quien se complace mi alma} (\emph{Is} 42, 1). Verdaderamente es el Mesías, el Hijo del Altísimo que, al salir de las aguas del Jordán, establece la regeneración en el Espíritu y da, a quienes lo deseen, la posibilidad de convertirse en hijos de Dios. De hecho, no es casualidad que todo bautizado adquiera el carácter de hijo a partir del \emph{nombre cristiano}, signo inconfundible de que el Espíritu Santo hace nacer \textquote{de nuevo} al hombre del seno de la Iglesia. El beato Antonio Rosmini afirma que \textquote{el bautizado sufre una operación secreta pero potentísima, por la cual es elevado al orden sobrenatural, es puesto en comunicación con Dios} (\emph{Del principio supremo della metodica}\ldots{}, Turín 1857, n. 331). Todo esto se ha verificado de nuevo esta mañana, durante la celebración eucarística en la Capilla Sixtina, donde he conferido el sacramento del Bautismo a veintiún recién nacidos.

Queridos amigos, el Bautismo es el inicio de la vida espiritual, que encuentra su plenitud por medio de la Iglesia. En la hora propicia del sacramento, mientras la comunidad eclesial reza y encomienda a Dios un nuevo hijo, los padres y los padrinos se comprometen a acoger al recién bautizado sosteniéndolo en la formación y en la educación cristiana. Es una gran responsabilidad, que deriva de un gran don. Por esto, deseo alentar a todos los fieles a redescubrir la belleza de ser bautizados y pertenecer así a la gran familia de Dios, y a dar testimonio gozoso de la propia fe, a fin de que esta fe produzca frutos de bien y de concordia.

Lo pedimos por intercesión de la santísima Virgen María, Auxilio de los cristianos, a quien encomendamos a los padres que se están preparando al Bautismo de sus hijos, al igual que a los catequistas. Que toda la comunidad participe de la alegría del renacimiento del agua y del Espíritu Santo.

\subsection{Francisco, papa}

\subsubsection{Homilía (2014): La mejor herencia}

Domingo 12 de enero del 2014.

Jesús no tenía necesidad de ser bautizado, pero los primeros teólogos dicen que, con su cuerpo, con su divinidad, en su bautismo bendijo todas las aguas, para que las aguas tuvieran el poder de dar el Bautismo. Y luego, antes de subir al Cielo, Jesús nos pidió ir por todo el mundo a bautizar. Y desde aquel día hasta el día de hoy, esto ha sido una cadena ininterrumpida: se bautizan a los hijos, y los hijos después a los hijos, y los hijos\ldots{} Y hoy también esta cadena prosigue.

Estos niños son el eslabón de una cadena. Vosotros padres traéis a bautizar al niño o la niña, pero en algunos años serán ellos los que traerán a bautizar a un niño, o un nietecito\ldots{} Así es la cadena de la fe. ¿Qué quiere decir esto? Desearía solamente deciros esto: vosotros sois los que transmitís la fe, los transmisores; vosotros tenéis el deber de transmitir la fe a estos niños. Es la más hermosa herencia que vosotros les dejaréis: la fe. Sólo esto. Llevad hoy a casa este pensamiento. Debemos ser transmisores de la fe. Pensad en esto, pensad siempre cómo transmitir la fe a los niños.

Hoy canta el coro, pero el coro más bello es este de los niños, que hacen ruido\ldots{} Algunos llorarán, porque no están cómodos o porque tienen hambre: si tienen hambre, mamás, dadles de comer, tranquilas, porque ellos son aquí los protagonistas. Y ahora, con esta conciencia de ser transmisores de la fe, continuemos la ceremonia del Bautismo.

\subsubsection{Ángelus (2014): Los cielos se abrieron}

Domingo 12 de enero del 2014.

Hoy es la fiesta del Bautismo del Señor. Esta mañana he bautizado a treinta y dos recién nacidos. Doy gracias con vosotros al Señor por estas criaturas y por cada nueva vida. A mí me gusta bautizar a los niños. ¡Me gusta mucho! Cada niño que nace es un don de alegría y de esperanza, y cada niño que es bautizado es un prodigio de la fe y una fiesta para la familia de Dios.

La página del \textbf{Evangelio de hoy} subraya que, cuando Jesús recibió el bautismo de Juan en el río Jordán, \textquote{se abrieron los cielos} (\emph{Mt} 3, 16). Esto realiza las profecías. En efecto, hay una invocación que la liturgia nos hace repetir en el tiempo de Adviento: \textquote{Ojalá rasgases el cielo y descendieses!} (\emph{Is} 63, 19). Si el cielo permanece cerrado, nuestro horizonte en esta vida terrena es sombrío, sin esperanza. En cambio, celebrando la Navidad, la fe una vez más nos ha dado la certeza de que el cielo se rasgó con la venida de Jesús. Y en el día del bautismo de Cristo contemplamos aún el cielo abierto. La manifestación del Hijo de Dios en la tierra marca el inicio del gran tiempo de la misericordia, después de que el pecado había cerrado el cielo, elevando como una barrera entre el ser humano y su Creador. Con el nacimiento de Jesús, el cielo se abre. Dios nos da en Cristo la garantía de un amor indestructible. Desde que el Verbo se hizo carne es, por lo tanto, posible ver el cielo abierto. Fue posible para los pastores de Belén, para los Magos de Oriente, para el Bautista, para los Apóstoles de Jesús, para san Esteban, el primer mártir, que exclamó: \textquote{Veo los cielos abiertos} (\emph{Hch} 7, 56). Y es posible también para cada uno de nosotros, si nos dejamos invadir por el amor de Dios, que nos es donado por primera vez en el Bautismo. ¡Dejémonos invadir por el amor de Dios! ¡Éste es el gran tiempo de la misericordia! No lo olvidéis: ¡éste es el gran tiempo de la misericordia!

Cuando Jesús recibió el Bautismo de penitencia de Juan el Bautista, solidarizándose con el pueblo penitente ---Él sin pecado y sin necesidad de conversión---, Dios Padre hizo oír su voz desde el cielo: \textquote{Éste es mi Hijo amado, en quien me complazco} (v. 17). Jesús recibió la aprobación del Padre celestial, que lo envió precisamente para que aceptara compartir nuestra condición, nuestra pobreza. Compartir es el auténtico modo de amar. Jesús no se disocia de nosotros, nos considera hermanos y comparte con nosotros. Así, nos hace hijos, juntamente con Él, de Dios Padre. Ésta es la revelación y la fuente del amor auténtico. Y, ¡este es el gran tiempo de la misericordia!

¿No os parece que en nuestro tiempo se necesita un suplemento de fraternidad y de amor? ¿No os parece que todos necesitamos un suplemento de caridad? No esa caridad que se conforma con la ayuda improvisada que no nos involucra, no nos pone en juego, sino la caridad que comparte, que se hace cargo del malestar y del sufrimiento del hermano. ¡Qué buen sabor adquiere la vida cuando dejamos que la inunde el amor de Dios!

Pidamos a la Virgen Santa que nos sostenga con su intercesión en nuestro compromiso de seguir a Cristo por el camino de la fe y de la caridad, la senda trazada por nuestro Bautismo.

Hoy la creación es iluminada, hoy todas las cosas están llenas de júbilo,

los seres celestes y terrestres. Ángeles y hombres se unen,

porque donde está presente el Rey, allí también está su séquito.

En las aguas del Jordán el Rey de los siglos, el Señor,

moldea de nuevo a Adán que se había corrompido,

destruye las cabezas de los dragones allí anidados \ldots{}

Jesús, autor de la vida,

ha venido a destruir la condena de Adán, el primer creado:

él, que no tiene necesidad de purificación, en cuanto Dios,

en el Jordán se purifica en favor del hombre caído,

y matando allí la enemistad, otorga la paz que sobrepasa toda inteligencia.

Un tiempo estéril, amargamente privada de prole,

alégrate hoy, oh Iglesia de Cristo:

porque del agua y del Espíritu

han sido engendrados hijos que con fe aclaman:

No hay santo como nuestro Dios,

y no hay otro justo fuera de ti, Señor.

(De la Liturgia Bizantina).

\subsubsection{Homilía (2017)} FIESTA DEL BAUTISMO DEL SEÑOR\\ CELEBRACIÓN DE LA SANTA MISA Y BAUTISMO DE ALGUNOS NIÑOS

\textbf{\emph{HOMILÍA DEL SANTO PADRE FRANCISCO}}

\emph{Capilla Sixtina\\ Domingo 8 de enero de 2017}


\emph{Queridos padres:}

Vosotros habéis pedido para vuestros niños la fe, la fe que será dada en el Bautismo. La fe: eso significa vida de fe, porque la fe es vivida; caminar por el camino de la fe y dar testimonio de la fe. La fe no es recitar el \textquote{Credo} el domingo, cuando vamos a misa: no es solo esto. La fe es creer lo que es la Verdad: Dios Padre que ha enviado a su Hijo y al Espíritu que nos vivifica. Pero la fe es también encomendarse a Dios, y esto vosotros se lo tenéis que enseñar a ellos, con vuestro ejemplo, con vuestra vida.

Y la fe es luz: en la ceremonia del Bautismo se os dará una vela encendida, como en los primeros tiempos de la Iglesia. Y por esto el Bautismo, en esos tiempos, se llamaba \textquote{iluminación}, porque la fe ilumina el corazón, hace ver las cosas con otra luz. Vosotros habéis pedido la fe: la Iglesia da la fe a vuestros hijos con el Bautismo, y vosotros tenéis el deber de hacerla crecer, cuidarla, y que se convierta en testimonio para todos los demás. Este es el sentido de esta ceremonia. Y solamente quería deciros esto: cuidar la fe, hacerla crecer, que sea testimonio para los demás.

Y después\ldots{} ¡ha comenzado el concierto! {[}los niños lloran{]}: es porque los niños se encuentran en un lugar que no conocen, se han despertado antes de lo normal. Comienza uno, da la nota y después otros \textquote{imitan}\ldots{} Algunos lloran solamente porque ha llorado el otro\ldots{} Jesús hizo lo mismo, ¿sabéis? A mí me gusta pensar que la primera predicación de Jesús en el establo fue un llanto, la primera\ldots{} Y después, como la ceremonia es un poco larga, alguno llora por hambre. Si es así, vosotras madres amamantadles también, sin miedo, con toda normalidad. Como la Virgen amamantaba a Jesús\ldots{}

No olvidéis: habéis pedido la fe, a vosotros la tarea de cuidar la fe, hacerla crecer, que sea testimonio para todos nosotros, para todos nosotros: también para nosotros sacerdotes, obispos, todos. Gracias.

\subsubsection{Ángelus (2017)}

\emph{Queridos hermanos y hermanas, ¡buenos días!}

Hoy, fiesta del Bautismo de Jesús, el Evangelio (\emph{Mt} 3, 13-17) nos presenta la episodio ocurrido a orillas del río Jordán: en medio de la muchedumbre penitente que avanza hacia Juan Bautista para recibir el Bautismo también se encuentra Jesús ---hacía fila---. Juan querría impedírselo diciendo: \textquote{Soy yo el que necesita ser bautizado por ti} (\emph{Mt} 3, 14). En efecto, el Bautista es consciente de la gran distancia que hay entre él y Jesús. Pero Jesús vino precisamente para colmar la distancia entre el hombre y Dios: si Él está completamente de parte de Dios también está completamente de parte del hombre, y reúne aquello que estaba dividido. Por eso pide a Juan que le bautice, para que se cumpla toda justicia (cf. v. 15), es decir, se realice el proyecto del Padre, que pasa a través de la vía de la obediencia y de la solidaridad con el hombre frágil y pecador, la vía de la humildad y de la plena cercanía de Dios a sus hijos. ¡Porque Dios está muy cerca de nosotros, mucho!

En el momento en el que Jesús, bautizado por Juan, sale de las aguas del río Jordán, la voz de Dios Padre se hace oír desde lo alto: \textquote{Este es mi Hijo amado, en quien me complazco} (v. 17). Y al mismo tiempo el Espíritu Santo, en forma de paloma, se posa sobre Jesús, que da públicamente inicio a su misión de salvación; misión caracterizada por un estilo, el estilo del siervo humilde y dócil, dotado sólo de la fuerza de la verdad, como había profetizado Isaías: \textquote{no vociferará ni alzará el tono, {[}\ldots{}{]} la caña quebrada no partirá, y la mecha mortecina no apagará. Lealmente hará justicia} (42, 2-3). Siervo humilde y manso, he aquí el estilo de Jesús, y también el estilo misionero de los discípulos de Cristo: anunciar el Evangelio con docilidad y firmeza, sin gritar, sin regañar a alguien, sino con docilidad y firmeza, sin arrogancia o imposición. La verdadera misión nunca es proselitismo sino atracción a Cristo. ¿Pero cómo? ¿Cómo se hace esta atracción a Cristo? Con el propio testimonio, a partir de la fuerte unión con Él en la oración, en la adoración y en la caridad concreta, que es servicio a Jesús presente en el más pequeño de los hermanos. Imitando a Jesús, pastor bueno y misericordioso, y animados por su gracia, estamos llamados a hacer de nuestra vida un testimonio alegre que ilumina el camino, que lleva esperanza y amor.

Esta fiesta nos hace redescubrir el don y la belleza de ser un pueblo de bautizados, es decir, de pecadores ---todos lo somos--- de pecadores salvados por la gracia de Cristo, inseridos realmente, por obra del Espíritu Santo, en la relación filial de Jesús con el Padre, acogidos en el seno de la madre Iglesia, hechos capaces de una fraternidad que no conoce confines ni barreras.

Que la Virgen María nos ayude a todos nosotros cristianos a conservar una conciencia siempre viva y agradecida de nuestro Bautismo y a recorrer con fidelidad el camino inaugurado por este Sacramento de nuestro renacimiento. Y siempre humildad, docilidad y firmeza.

\subsubsection{Homilía (2020)} \emph{Capilla Sixtina\\ Domingo, 12 de enero de 2020}

\begin{center}\rule{0.5\linewidth}{\linethickness}\end{center}



Como Jesús, que fue a hacerse bautizar, así hacéis vosotros con vuestros hijos.

Jesús responde a Juan: \textquote{Hágase toda justicia} (cf. \emph{Mt} 3,15). Bautizar a un hijo es un acto de justicia para él. ¿Y por qué? Porque nosotros con el Bautismo le damos un tesoro, nosotros con el Bautismo le damos en prenda el \emph{Espíritu Santo}. El niño sale {[}del Bautismo{]} con la fuerza del Espíritu en su interior: el Espíritu que lo defenderá, que lo ayudará, durante toda su vida. Por eso es tan importante bautizarlos cuando son pequeños, para que crezcan con la fuerza del Espíritu Santo.

Este es el mensaje que quisiera daros hoy. Vosotros traéis hoy a vuestros hijos, {[}para que tengan{]} el Espíritu Santo dentro de ellos. Y cuidad de que crezcan con la luz, con la fuerza del Espíritu Santo, a través de la catequesis, la ayuda, la enseñanza, los ejemplos que les daréis en casa\ldots{} Este es el mensaje.

No quisiera deciros nada más importante. Sólo una advertencia. Los niños no están acostumbrados a venir a la Sixtina, ¡es la primera vez! Tampoco están acostumbrados a estar en un ambiente algo caluroso. Y no están acostumbrados a vestirse así para una fiesta tan hermosa como la de hoy. Se sentirán un poco incómodos en algún momento. Y uno empezará a llorar\ldots{} ―¡El concierto no ha empezado todavía!― pero empezará uno, luego otro\ldots{} No os asustéis, dejad que los niños lloren y griten. A lo mejor si tu niño llora y se queja, quizás sea porque tiene demasiado calor: quitadle algo; o porque tiene hambre: dale de mamar, aquí, sí, siempre en paz. Es algo que dije también el año pasado: tienen una dimensión \textquote{coral}: es suficiente que uno dé la primera nota y empiezan todos y habrá un concierto. No os asustéis. Es un sermón muy bonito el de un niño que llora en una iglesia. Haced que esté cómodo y sigamos adelante.

No lo olvidéis: vosotros lleváis el Espíritu Santo a los niños.



\subsubsection{Ángelus (2020)} \emph{Plaza de San Pedro\\ Domingo, 12 de enero de 2020}

\begin{center}\rule{0.5\linewidth}{\linethickness}\end{center}



\emph{Queridos hermanos y hermanas}, ¡buenos días!

Una vez más he tenido la alegría de bautizar a algunos niños en la fiesta de hoy del Bautismo del Señor. Hoy eran treinta y dos. Recemos por ellos y sus familias.

La liturgia de este año nos propone el acontecimiento del bautismo de Jesús según el relato evangélico de Mateo (cf. 3, 13-17). El evangelista describe el diálogo entre Jesús, que pide el bautismo, y Juan el Bautista, que se niega y observa: \textquote{Soy yo el que necesita ser bautizado por ti, ¿y tú vienes a mí?} (v. 14). Esta decisión de Jesús sorprende al Bautista: de hecho, el Mesías no necesita ser purificado, sino que es Él quien purifica. Pero Dios es Santo, sus caminos no son los nuestros, y Jesús es el Camino de Dios, un camino impredecible. Recordemos que Dios es el Dios de las sorpresas.

Juan había declarado que existía una distancia abismal e insalvable entre él y Jesús. \textquote{No soy digno de llevarle las sandalias} (\emph{Mateo} 3, 11), dijo. Pero el Hijo de Dios vino precisamente para salvar esta distancia entre el hombre y Dios. Si Jesús está del lado de Dios, también está del lado del hombre, y reúne lo que estaba dividido. Por eso le respondió a Juan: \textquote{Déjame ahora, pues conviene que así cumplamos toda justicia} (v. 15). El Mesías pide ser bautizado para que se cumpla toda justicia, para que se realice el proyecto del Padre, que pasa por el camino de la obediencia filial y de la solidaridad con el hombre frágil y pecador. Es el camino de la humildad y de la plena cercanía de Dios a sus hijos.

El profeta Isaías proclama también la justicia del Siervo de Dios, que lleva a cabo su misión en el mundo con un estilo contrario al espíritu mundano: \textquote{No vociferará ni alzará el tono, y no hará oír en la calle su voz. Caña quebrada no partirá, y mecha mortecina no apagará} (42, 2-3). Es la actitud de mansedumbre ―es lo que Jesús nos enseña con su humildad, la mansedumbre―, la actitud de sencillez, respeto, moderación y ocultamiento, que se requiere aún hoy de los discípulos del Señor. Cuántos ―es triste decirlo―, cuántos discípulos del Señor alardean como discípulos del Señor. No es un buen discípulo el que alardea de ello. El buen discípulo es el humilde, el manso que hace el bien sin ser visto. En la acción misionera, la comunidad cristiana está llamada a salir al encuentro de los demás siempre proponiendo y no imponiendo, dando testimonio, compartiendo la vida concreta de la gente.

Tan pronto como Jesús fue bautizado en el río Jordán, los cielos se abrieron y el Espíritu Santo descendió sobre él como una paloma, mientras que desde lo alto resonaba una voz que decía: \textquote{Este es mi Hijo amado; en el que me complazco} (\emph{Mateo} 3, 17). En la fiesta del Bautismo de Jesús redescubrimos nuestro bautismo. Así como Jesús es el Hijo amado del Padre, también nosotros, renacidos del agua y del Espíritu Santo, sabemos que somos hijos amados ―¡el Padre nos ama a todos!―, que somos objeto de la satisfacción de Dios, hermanos y hermanas de muchos otros, con una gran misión de testimoniar y anunciar a todos los hombres y mujeres el amor ilimitado del Padre.

Esta fiesta del Bautismo de Jesús nos recuerda nuestro bautismo. Nosotros también renacemos en el bautismo. En el bautismo el Espíritu Santo vino a permanecer en nosotros. Por eso es importante saber la fecha del bautismo. Sabemos la fecha de nuestro nacimiento, pero no siempre sabemos la fecha de nuestro bautismo. Seguramente algunos de vosotros no la saben\ldots{} Una tarea. Cuando regreses a casa pregunta: ¿Cuándo fui bautizada? ¿Cuándo fui bautizado? Y celebra la fecha de tu bautismo en tu corazón cada año. Hazlo. Es también un deber de justicia hacia el Señor que ha sido tan bueno con nosotros.

Que María Santísima nos ayude a comprender cada vez más el don del bautismo y a vivirlo coherentemente en las situaciones cotidianas.

\section{Temas}

El Directorio Homilético no indica temas del Catecismo para esta fiesta. Pero podemos considerar los siguientes:

Juan, precursor, profeta y bautista

CEC 523; 717-720:

\textbf{523} \emph{San Juan Bautista} es el precursor (cf. \emph{Hch} 13, 24) inmediato del Señor, enviado para prepararle el camino (cf. \emph{Mt} 3, 3). \textquote{Profeta del Altísimo} (\emph{Lc} 1, 76), sobrepasa a todos los profetas (cf. \emph{Lc} 7, 26), de los que es el último (cf. \emph{Mt} 11, 13), e inaugura el Evangelio (cf. \emph{Hch} 1, 22; \emph{Lc} 16,16); desde el seno de su madre (cf. \emph{Lc} 1,41) saluda la venida de Cristo y encuentra su alegría en ser \textquote{el amigo del esposo} (\emph{Jn} 3, 29) a quien señala como \textquote{el Cordero de Dios que quita el pecado del mundo} (\emph{Jn} 1, 29). Precediendo a Jesús \textquote{con el espíritu y el poder de Elías} (\emph{Lc} 1, 17), da testimonio de él mediante su predicación, su bautismo de conversión y finalmente con su martirio (cf. \emph{Mc} 6, 17-29).

\textbf{717} \textquote{Hubo un hombre, enviado por Dios, que se llamaba Juan}. (\emph{Jn} 1, 6). Juan fue \textquote{lleno del Espíritu Santo ya desde el seno de su madre} (\emph{Lc} 1, 15. 41) por obra del mismo Cristo que la Virgen María acababa de concebir del Espíritu Santo. La \textquote{Visitación} de María a Isabel se convirtió así en \textquote{visita de Dios a su pueblo} (\emph{Lc} 1, 68).

\textbf{718} Juan es \textquote{Elías que debe venir} (\emph{Mt} 17, 10-13): El fuego del Espíritu lo habita y le hace correr delante {[}como \textquote{precursor}{]} del Señor que viene. En Juan el Precursor, el Espíritu Santo culmina la obra de \textquote{preparar al Señor un pueblo bien dispuesto} (\emph{Lc} 1, 17).

\textbf{719} Juan es \textquote{más que un profeta} (\emph{Lc} 7, 26). En él, el Espíritu Santo consuma el \textquote{hablar por los profetas}. Juan termina el ciclo de los profetas inaugurado por Elías (cf. \emph{Mt} 11, 13-14). Anuncia la inminencia de la consolación de Israel, es la \textquote{voz} del Consolador que llega (\emph{Jn} 1, 23; cf. \emph{Is} 40, 1-3). Como lo hará el Espíritu de Verdad, \textquote{vino como testigo para dar testimonio de la luz} (\emph{Jn} 1, 7; cf. \emph{Jn} 15, 26; 5, 33). Con respecto a Juan, el Espíritu colma así las \textquote{indagaciones de los profetas} y la ansiedad de los ángeles (\emph{1 P} 1, 10-12): \textquote{Aquél sobre quien veas que baja el Espíritu y se queda sobre él, ése es el que bautiza con el Espíritu Santo. Y yo lo he visto y doy testimonio de que éste es el Hijo de Dios [\ldots{}] He ahí el Cordero de Dios} (\emph{Jn} 1, 33-36).

\textbf{720} En fin, con Juan Bautista, el Espíritu Santo, inaugura, prefigurándolo, lo que realizará con y en Cristo: volver a dar al hombre la \textquote{semejanza} divina. El bautismo de Juan era para el arrepentimiento, el del agua y del Espíritu será un nuevo nacimiento (cf. \emph{Jn} 3, 5).

El Bautismo de Jesús

CEC 535-537:

\textbf{535} El comienzo (cf. \emph{Lc} 3, 23) de la vida pública de Jesús es su bautismo por Juan en el Jordán (cf. \emph{Hch} 1, 22). Juan proclamaba \textquote{un bautismo de conversión para el perdón de los pecados} (\emph{Lc} 3, 3). Una multitud de pecadores, publicanos y soldados (cf. \emph{Lc} 3, 10-14), fariseos y saduceos (cf. \emph{Mt} 3, 7) y prostitutas (cf. \emph{Mt} 21, 32) viene a hacerse bautizar por él. \textquote{Entonces aparece Jesús}. El Bautista duda. Jesús insiste y recibe el bautismo. Entonces el Espíritu Santo, en forma de paloma, viene sobre Jesús, y la voz del cielo proclama que él es \textquote{mi Hijo amado} (\emph{Mt} 3, 13-17). Es la manifestación (\textquote{Epifanía}) de Jesús como Mesías de Israel e Hijo de Dios.

\textbf{536} El bautismo de Jesús es, por su parte, la aceptación y la inauguración de su misión de Siervo doliente. Se deja contar entre los pecadores (cf. \emph{Is} 53, 12); es ya \textquote{el Cordero de Dios que quita el pecado del mundo} (\emph{Jn} 1, 29); anticipa ya el \textquote{bautismo} de su muerte sangrienta (cf. \emph{Mc} 10, 38; \emph{Lc} 12, 50). Viene ya a \textquote{cumplir toda justicia} (\emph{Mt} 3, 15), es decir, se somete enteramente a la voluntad de su Padre: por amor acepta el bautismo de muerte para la remisión de nuestros pecados (cf. \emph{Mt} 26, 39). A esta aceptación responde la voz del Padre que pone toda su complacencia en su Hijo (cf. \emph{Lc} 3, 22; \emph{Is} 42, 1). El Espíritu que Jesús posee en plenitud desde su concepción viene a \textquote{posarse} sobre él (\emph{Jn} 1, 32-33; cf. \emph{Is} 11, 2). De él manará este Espíritu para toda la humanidad. En su bautismo, \textquote{se abrieron los cielos} (\emph{Mt} 3, 16) que el pecado de Adán había cerrado; y las aguas fueron santificadas por el descenso de Jesús y del Espíritu como preludio de la nueva creación.

\textbf{537} Por el Bautismo, el cristiano se asimila sacramentalmente a Jesús que anticipa en su bautismo su muerte y su resurrección: debe entrar en este misterio de rebajamiento humilde y de arrepentimiento, descender al agua con Jesús, para subir con él, renacer del agua y del Espíritu para convertirse, en el Hijo, en hijo amado del Padre y \textquote{vivir una vida nueva} (\emph{Rm} 6, 4):

\textquote{Enterrémonos con Cristo por el Bautismo, para resucitar con él; descendamos con él para ser ascendidos con él; ascendamos con él para ser glorificados con él} (San Gregorio Nacianceno, \emph{Oratio} 40, 9: PG 36, 369).

\textquote{Todo lo que aconteció en Cristo nos enseña que después del baño de agua, el Espíritu Santo desciende sobre nosotros desde lo alto del cielo y que, adoptados por la Voz del Padre, llegamos a ser hijos de Dios}. (San Hilario de Poitiers, \emph{In evangelium Matthaei}, 2, 6: PL 9, 927).

El Mesías esperado

CEC 1286:

\textbf{1286} En el Antiguo Testamento, los profetas anunciaron que el Espíritu del Señor reposaría sobre el Mesías esperado (cf. \emph{Is} 11,2) para realizar su misión salvífica (cf. \emph{Lc} 4,16-22; \emph{Is} 61,1). El descenso del Espíritu Santo sobre Jesús en su Bautismo por Juan fue el signo de que Él era el que debía venir, el Mesías, el Hijo de Dios (\emph{Mt} 3,13-17; \emph{Jn} 1,33- 34). Habiendo sido concedido por obra del Espíritu Santo, toda su vida y toda su misión se realizan en una comunión total con el Espíritu Santo que el Padre le da \textquote{sin medida} (\emph{Jn} 3,34).

El Cordero de Dios que quita el pecado del mundo

CEC 608:

\textbf{608} Juan Bautista, después de haber aceptado bautizarle en compañía de los pecadores (cf. \emph{Lc} 3, 21; \emph{Mt} 3, 14-15), vio y señaló a Jesús como el \textquote{Cordero de Dios que quita los pecados del mundo} (\emph{Jn} 1, 29; cf. \emph{Jn} 1, 36). Manifestó así que Jesús es a la vez el Siervo doliente que se deja llevar en silencio al matadero (\emph{Is} 53, 7; cf. \emph{Jr} 11, 19) y carga con el pecado de las multitudes (cf. \emph{Is} 53, 12) y el cordero pascual símbolo de la redención de Israel cuando celebró la primera Pascua (\emph{Ex} 12, 3-14; cf. \emph{Jn} 19, 36; \emph{1 Co} 5, 7). Toda la vida de Cristo expresa su misión: \textquote{Servir y dar su vida en rescate por muchos} (\emph{Mc} 10, 45).

El Hijo amado

CEC 444:

\textbf{444} Los evangelios narran en dos momentos solemnes, el Bautismo y la Transfiguración de Cristo, que la voz del Padre lo designa como su \textquote{Hijo amado} (\emph{Mt} 3, 17; 17, 5). Jesús se designa a sí mismo como \textquote{el Hijo Único de Dios} (\emph{Jn} 3, 16) y afirma mediante este título su preexistencia eterna (cf. \emph{Jn} 10, 36). Pide la fe en \textquote{el Nombre del Hijo Único de Dios} (\emph{Jn} 3, 18). Esta confesión cristiana aparece ya en la exclamación del centurión delante de Jesús en la cruz: \textquote{Verdaderamente este hombre era Hijo de Dios} (\emph{Mc} 15, 39), porque es solamente en el misterio pascual donde el creyente puede alcanzar el sentido pleno del título \textquote{Hijo de Dios}.

Símbolos del Espíritu Santo en el Bautismo: agua, fuego y paloma

CEC 694. 696. 701:

\textbf{694} \emph{El agua}. El simbolismo del agua es significativo de la acción del Espíritu Santo en el Bautismo, ya que, después de la invocación del Espíritu Santo, ésta se convierte en el signo sacramental eficaz del nuevo nacimiento: del mismo modo que la gestación de nuestro primer nacimiento se hace en el agua, así el agua bautismal significa realmente que nuestro nacimiento a la vida divina se nos da en el Espíritu Santo. Pero \textquote{bautizados [\ldots{}] en un solo Espíritu}, también \textquote{hemos bebido de un solo Espíritu} (\emph{1 Co} 12, 13): el Espíritu es, pues, también personalmente el Agua viva que brota de Cristo crucificado (cf. Jn 19, 34; 1 Jn 5, 8) como de su manantial y que en nosotros brota en vida eterna (cf. \emph{Jn} 4, 10-14; 7, 38; \emph{Ex} 17, 1-6; \emph{Is} 55, 1; \emph{Za} 14, 8; \emph{1 Co} 10, 4; \emph{Ap} 21, 6; 22, 17).

\textbf{696} \emph{El fuego}. Mientras que el agua significaba el nacimiento y la fecundidad de la vida dada en el Espíritu Santo, el fuego simboliza la energía transformadora de los actos del Espíritu Santo. El profeta Elías que \textquote{surgió [\ldots{}] como el fuego y cuya palabra abrasaba como antorcha} (\emph{Si} 48, 1), con su oración, atrajo el fuego del cielo sobre el sacrificio del monte Carmelo (cf. \emph{1 R} 18, 38-39), figura del fuego del Espíritu Santo que transforma lo que toca. Juan Bautista, \textquote{que precede al Señor con el espíritu y el poder de Elías} (\emph{Lc} 1, 17), anuncia a Cristo como el que \textquote{bautizará en el Espíritu Santo y el fuego} (\emph{Lc} 3, 16), Espíritu del cual Jesús dirá: \textquote{He venido a traer fuego sobre la tierra y ¡cuánto desearía que ya estuviese encendido!} (\emph{Lc} 12, 49). En forma de lenguas \textquote{como de fuego} se posó el Espíritu Santo sobre los discípulos la mañana de Pentecostés y los llenó de él (\emph{Hch} 2, 3-4). La tradición espiritual conservará este simbolismo del fuego como uno de los más expresivos de la acción del Espíritu Santo (cf. San Juan de la Cruz, \emph{Llama de amor viva}). \textquote{No extingáis el Espíritu} (\emph{1 Ts} 5, 19).

\textbf{701} \emph{La paloma}. Al final del diluvio (cuyo simbolismo se refiere al Bautismo), la paloma soltada por Noé vuelve con una rama tierna de olivo en el pico, signo de que la tierra es habitable de nuevo (cf. \emph{Gn} 8, 8-12). Cuando Cristo sale del agua de su bautismo, el Espíritu Santo, en forma de paloma, baja y se posa sobre él (cf. \emph{Mt} 3, 16 paralelos). El Espíritu desciende y reposa en el corazón purificado de los bautizados. En algunos templos, la Santa Reserva eucarística se conserva en un receptáculo metálico en forma de paloma (el \emph{columbarium}), suspendido por encima del altar. El símbolo de la paloma para sugerir al Espíritu Santo es tradicional en la iconografía cristiana.

El Bautismo de Cristo y el Bautismo de los cristianos

CEC 1223-1225:

\textbf{1223} Todas las prefiguraciones de la Antigua Alianza culminan en Cristo Jesús. Comienza su vida pública después de hacerse bautizar por san Juan el Bautista en el Jordán (cf. \emph{Mt} 3,13) y, después de su Resurrección, confiere esta misión a sus Apóstoles: \textquote{Id, pues, y haced discípulos a todas las gentes bautizándolas en el nombre del Padre y del Hijo y del Espíritu Santo, y enseñándoles a guardar todo lo que yo os he mandado} (\emph{Mt} 28,19-20; cf. \emph{Mc} 16,15-16).

\textbf{1224} Nuestro Señor se sometió voluntariamente al Bautismo de san Juan, destinado a los pecadores, para \textquote{cumplir toda justicia} (\emph{Mt} 3,15). Este gesto de Jesús es una manifestación de su \textquote{anonadamiento} (\emph{Flp} 2,7). El Espíritu que se cernía sobre las aguas de la primera creación desciende entonces sobre Cristo, como preludio de la nueva creación, y el Padre manifiesta a Jesús como su \textquote{Hijo amado} (\emph{Mt} 3,16-17).

\textbf{1225} En su Pascua, Cristo abrió a todos los hombres las fuentes del Bautismo. En efecto, había hablado ya de su pasión que iba a sufrir en Jerusalén como de un \textquote{Bautismo} con que debía ser bautizado (\emph{Mc} 10,38; cf. \emph{Lc} 12,50). La sangre y el agua que brotaron del costado traspasado de Jesús crucificado (cf. \emph{Jn} 19,34) son figuras del Bautismo y de la Eucaristía, sacramentos de la vida nueva (cf. \emph{1 Jn} 5,6-8): desde entonces, es posible \textquote{nacer del agua y del Espíritu} para entrar en el Reino de Dios (\emph{Jn} 3,5).

\textquote{Considera dónde eres bautizado, de dónde viene el Bautismo: de la cruz de Cristo, de la muerte de Cristo. Ahí está todo el misterio: Él padeció por ti. En él eres rescatado, en él eres salvado}. (San Ambrosio, \emph{De sacramentis} 2, 2, 6).

Frutos del Bautismo

CEC 1262-1266

\textbf{La gracia del Bautismo}

\textbf{1262} Los distintos efectos del Bautismo son significados por los elementos sensibles del rito sacramental. La inmersión en el agua evoca los simbolismos de la muerte y de la purificación, pero también los de la regeneración y de la renovación. Los dos efectos principales, por tanto, son la purificación de los pecados y el nuevo nacimiento en el Espíritu Santo (cf. \emph{Hch}2,38; \emph{Jn} 3,5).

\textbf{1263 Para la remisión de los pecados\ldots{}}

Por el Bautismo, \emph{todos los pecados} son perdonados, el pecado original y todos los pecados personales así como todas las penas del pecado (cf. DS 1316). En efecto, en los que han sido regenerados no permanece nada que les impida entrar en el Reino de Dios, ni el pecado de Adán, ni el pecado personal, ni las consecuencias del pecado, la más grave de las cuales es la separación de Dios.

\textbf{1264} No obstante, en el bautizado permanecen ciertas consecuencias temporales del pecado, como los sufrimientos, la enfermedad, la muerte o las fragilidades inherentes a la vida como las debilidades de carácter, etc., así como una inclinación al pecado que la Tradición llama \emph{concupiscencia}, o metafóricamente \emph{fomes peccati}: \textquote{La concupiscencia, dejada para el combate, no puede dañar a los que no la consienten y la resisten con coraje por la gracia de Jesucristo. Antes bien \textquote{el que legítimamente luchare, será coronado} (\emph{2 Tm} 2,5)} (Concilio de Trento: DS 1515).

\textbf{\textquote{Una criatura nueva}}

\textbf{1265} El Bautismo no solamente purifica de todos los pecados, hace también del neófito \textquote{una nueva creatura} (\emph{2 Co} 5,17), un hijo adoptivo de Dios (cf. \emph{Ga} 4,5-7) que ha sido hecho \textquote{partícipe de la naturaleza divina} (\emph{2 P} 1,4), miembro de Cristo (cf. \emph{1 Co} 6,15; 12,27), coheredero con Él (\emph{Rm} 8,17) y templo del Espíritu Santo (cf. \emph{1 Co} 6,19).

\textbf{1266} La Santísima Trinidad da al bautizado \emph{la gracia santificante, la gracia de la justificación} que :

--- le hace capaz de creer en Dios, de esperar en Él y de amarlo mediante las \emph{virtudes teologales};

--- le concede poder vivir y obrar bajo la moción del Espíritu Santo mediante los \emph{dones del Espíritu Santo};

--- le permite crecer en el bien mediante las \emph{virtudes morales}.

Así todo el organismo de la vida sobrenatural del cristiano tiene su raíz en el santo Bautismo.

Bautismo y Transfiguración

CEC 556:

\textbf{556} En el umbral de la vida pública se sitúa el Bautismo; en el de la Pascua, la Transfiguración. Por el bautismo de Jesús \textquote{fue manifestado el misterio de la primera regeneración}: nuestro Bautismo; la Transfiguración \textquote{es el sacramento de la segunda regeneración}: nuestra propia resurrección (Santo Tomás de Aquino, \emph{S.Th}., 3, q. 45, a. 4, ad 2). Desde ahora nosotros participamos en la Resurrección del Señor por el Espíritu Santo que actúa en los sacramentos del Cuerpo de Cristo. La Transfiguración nos concede una visión anticipada de la gloriosa venida de Cristo \textquote{el cual transfigurará este miserable cuerpo nuestro en un cuerpo glorioso como el suyo} (\emph{Flp} 3, 21). Pero ella nos recuerda también que \textquote{es necesario que pasemos por muchas tribulaciones para entrar en el Reino de Dios} (\emph{Hch} 14, 22):

\textquote{Pedro no había comprendido eso cuando deseaba vivir con Cristo en la montaña (cf. \emph{Lc} 9, 33). Te ha reservado eso, oh Pedro, para después de la muerte. Pero ahora, él mismo dice: Desciende para penar en la tierra, para servir en la tierra, para ser despreciado y crucificado en la tierra. La Vida desciende para hacerse matar; el Pan desciende para tener hambre; el Camino desciende para fatigarse andando; la Fuente desciende para sentir la sed; y tú, ¿vas a negarte a sufrir?} (San Agustín, \emph{Sermo}, 78, 6: PL 38, 492-493).

Sobre los beneficios de este libro

En los últimos años he publicado de manera totalmente gratuita 8 volúmenes del Leccionario Bienal en varias plataformas digitales (son cerca de 8 mil páginas de contenido que han supuesto muchas horas de maquetación, revisión y corrección de errores). Nunca he querido vender nada, ni sacar rédito de ello.

El presente libro, al ser impreso, no puede ser gratis. De todos modos, el precio se ha establecido con el menor margen de ganancia posible, porque el mismo no se publica con fines de lucro, sino sobre todo con una finalidad pastoral (quienes no puedan comprarlo podrán descargarlo gratuitamente en formato electrónico).

La mayor parte del precio fijado corresponde a los gastos de impresión, otra parte corresponde a la plataforma que lo distribuye, en este caso Amazon (yo no soy un escritor famoso, ni pretendo serlo, mi obra ha sido de mera recopilación, revisión, traducción, por eso he optado por una plataforma que permita la impresión bajo demanda que se encargue a su vez de la distribución).

Las ganancias que me revengan serán destinadas a la ayuda pastoral de mi Parroquia San Pedro Claver en Pedro Brand, República Dominicana. Es el lugar donde he recibido la fe y en el que la he vivido durante todos estos años unido a una comunidad de hermanos con rostros y vidas concretos. La tierra buena, donde Dios quiso plantarme.

\end{document}
