\chapter{Misa de Medianoche}

\section{Lecturas}

\rtitle{PRIMERA LECTURA}

\rbook{Del libro del profeta Isaías} \rred{9, 1-6}

\rtheme{Un hijo se nos ha dado}

\begin{scripture}
	\begin{readprose}
		El pueblo que caminaba en tinieblas vio una luz grande; habitaba en tierra y sombras de muerte, y una luz les brilló.
		
		Acreciste la alegría, aumentaste el gozo; 
		se gozan en tu presencia, como gozan al segar,
		como se alegran al repartirse el botín.
		
		Porque la vara del opresor, el yugo de su carga,
		el bastón de su hombro,
		los quebrantaste como el día de Madián.
		
		Porque la bota que pisa con estrépito
		y la túnica empapada de sangre	
		serán combustible, pasto del fuego.
		
		Porque un niño nos ha nacido, un hijo se nos ha dado:
		lleva a hombros el principado, y es su nombre:	
		\textquote{Maravilla de Consejero, Dios fuerte, \\Padre de eternidad, Príncipe de la paz}.
		
		Para dilatar el principado, con una paz sin límites,	
		sobre el trono de David y sobre su reino.
		
		Para sostenerlo y consolidarlo
		con la justicia y el derecho, desde ahora y por siempre.
		
		El celo del Señor del universo lo realizará.
	\end{readprose}
\end{scripture}

\rtitle{SALMO RESPONSORIAL}

\rbook{Salmo} \rred{95, 1-2a. 2b-3. 11-12. 13}

\rtheme{Hoy nos ha nacido un Salvador: el Mesías, el Señor}

\begin{psbody}
	Cantad al Señor un cántico nuevo,
	cantad al Señor, toda la tierra;
	cantad al Señor, bendecid su nombre. 
	
	Proclamad día tras día su victoria.
	Contad a los pueblos su gloria,
	sus maravillas a todas las naciones.
	
	Alégrese el cielo, goce la tierra,
	retumbe el mar y cuanto lo llena;
	vitoreen los campos y cuanto hay en ellos,
	aclamen los árboles del bosque.
	
	Delante del Señor, que ya llega,
	ya llega a regir la tierra:
	regirá el orbe con justicia
	y los pueblos con fidelidad.
\end{psbody}


\rtitle{SEGUNDA LECTURA}

\rbook{De la carta del apóstol san Pablo a Tito} \rred{2, 11-14}

\rtheme{Se ha manifestado la gracia de Dios para todos los hombres}

\begin{scripture}
	
	Querido hermano:
	
	Se ha manifestado la gracia de Dios, que trae la salvación para todos los hombres, enseñándonos a que, renunciando a la impiedad y a los deseos mundanos, llevemos ya desde ahora una vida sobria, justa y piadosa, aguardando la dicha que esperamos y la manifestación de la gloria del gran Dios y Salvador nuestro, Jesucristo, el cual se entregó por nosotros para rescatarnos de toda iniquidad y purificar para sí un pueblo de su propiedad, dedicado enteramente a las buenas obras.
\end{scripture}

\rtitle{EVANGELIO}

\rbook{Del Santo Evangelio según san Lucas} \rred{2, 1-14}

\rtheme{Hoy os ha nacido un Salvador}

\begin{scripture}
	Sucedió en aquellos días que salió un decreto del emperador Augusto, ordenando que se empadronase todo el Imperio.
	
	Este primer empadronamiento se hizo siendo Cirino gobernador de Siria. Y todos iban a empadronarse, cada cual a su ciudad.
	
	También José, por ser de la casa y familia de David, subió desde la ciudad de Nazaret, en Galilea, a la ciudad de David, que se llama Belén, en Judea, para empadronarse con su esposa María, que estaba encinta. Y sucedió que, mientras estaban allí, le llegó a ella el tiempo del parto y dio a luz a su hijo primogénito, lo envolvió en pañales y lo recostó en un pesebre, porque no había sitio para ellos en la posada.
	
	En aquella misma región había unos pastores que pasaban la noche al aire libre, velando por turno su rebaño.
	
	De repente un ángel del Señor se les presentó; la gloria del Señor los envolvió de claridad, y se llenaron de gran temor.
	
	El ángel les dijo:
	
	\>{No temáis, os anuncio una buena noticia que será de gran alegría para todo el pueblo: hoy, en la ciudad de David, os ha nacido un Salvador, el Mesías, el Señor. Y aquí tenéis la señal: encontraréis un niño envuelto en pañales y acostado en un pesebre}.
	
	De pronto, en torno al ángel, apareció una legión del ejército celestial, que alababa a Dios diciendo: 
	
	\>{Gloria a Dios en el cielo, y en la tierra paz a los hombres de buena voluntad}.
\end{scripture}

\newsection				

\section{Comentario Patrístico}

\subsection{Beato Elredo de Rievaulx, abad}

\ptheme{Hoy nos ha nacido un Salvador}

\src{Sermón 1 de la Natividad del Señor: \\PL 195, 226-227.\cite{ElredoRievaulx_PL195_226}}

\begin{body}
	\ltr{H}{oy,} \emph{en la ciudad de David, nos ha nacido un Salvador: El Mesías, el Señor}. La ciudad de que aquí se habla es Belén, a la que debemos acudir corriendo, como lo hicieron los pastores, apenas oído este rumor. Así es como soléis cantar ---en el himno de María, la Virgen---: \textquote{Cantaron gloria a Dios, corrieron a Belén}. \emph{Y aquí tenéis la señal: encontraréis un niño envuelto en pañales y acostado en un pesebre}.
	
	Ved por qué os dije que debéis amar. Teméis al Señor de los ángeles, pero amadle chiquitín; teméis al Señor de la majestad, pero amadle envuelto en pañales; teméis al que reina en el cielo, pero amadle acostado en un pesebre. Y ¿cuál fue la señal que recibieron los pastores? \emph{Encontraréis un niño envuelto en pañales y acostado en un pesebre}. El es el Salvador, él es el Señor. Pero, ¿qué tiene de extraordinario ser envuelto en pañales y yacer en un establo? ¿No son también los demás niños envueltos en pañales? Entonces, ¿qué clase de señal es ésta? Una señal realmente grande, a condición de que sepamos comprenderla. Y la comprendemos si no nos limitamos a escuchar este mensaje de amor, sino que, además, albergamos en nuestro corazón aquella claridad que apareció junto con los ángeles. Y si el ángel se apareció envuelto en claridad, cuando por primera vez anunció este rumor, fue para enseñarnos que sólo escuchan de verdad, los que acogen en su alma la claridad espiritual.
	
	Podríamos decir muchas cosas sobre esta señal, pero como el tiempo corre, insistiré brevemente en este tema. Belén, \textquote{casa del pan}, es la santa Iglesia, en la cual se distribuye el cuerpo de Cristo, a saber, el pan verdadero. El pesebre de Belén se ha convertido en el altar de la Iglesia. En él se alimentan los animales de Cristo. De esta mesa se ha escrito: \emph{Preparas una mesa ante mí}. En este pesebre está Jesús envuelto en pañales. La envoltura de los pañales es la cobertura de los sacramentos. En este pesebre y bajo las especies de pan y vino está el verdadero cuerpo y la sangre de Cristo. En este sacramento creemos que está el mismo Cristo; pero está envuelto en pañales, es decir, invisible bajo los signos sacramentales. No tenemos señal más grande y más evidente del nacimiento de Cristo como el hecho de que cada día sumimos en el altar santo su cuerpo y su sangre; como el comprobar que a diario se inmola por nosotros, el que por nosotros nació una vez de la Virgen.
	
	Apresurémonos, hermanos, al pesebre del Señor; pero antes y en la medida de lo posible, preparémonos con su gracia para este encuentro de suerte que asociados a los ángeles, \emph{con corazón limpio, con una conciencia honrada y con una fe sentida,} cantemos al Señor con toda nuestra vida y toda nuestra conducta: \emph{Gloria a Dios en el cielo, y en la tierra, paz a los hombres que Dios ama}. Por el mismo Jesucristo, nuestro Señor, a quien sea el honor y la gloria por los siglos de los siglos. Amén.
\end{body}

\begin{patercite}
	\textquote{No temáis, pues os anuncio una gran alegría. (\ldots{}) Os ha nacido hoy, en la ciudad de David, un salvador} (\emph{Lc} 2, 10-11). El mensaje de la venida de Cristo, que llegó del cielo mediante el anuncio de los ángeles, sigue resonando en esta ciudad, así como en las familias, en los hogares y en las comunidades de todo el mundo. Es una \textquote{gran alegría}, dijeron los ángeles, \textquote{para todo el pueblo}. Este mensaje proclama que el Mesías, el Hijo de Dios e hijo de David nació \textquote{por vosotros}: por ti y por mí, y por todos los hombres y mujeres de todo tiempo y lugar. En el plan de Dios, Belén, \textquote{el menor entre los clanes de Judá} (\emph{Mi} 5, 1) se convirtió en un lugar de gloria imperecedera: el lugar donde, en la plenitud de los tiempos, Dios eligió hacerse hombre, para acabar con el largo reinado del pecado y de la muerte, y para traer vida nueva y abundante a un mundo ya viejo, cansado y oprimido por la desesperación.
	
	Para los hombres y mujeres de todo lugar, Belén está asociada a este alegre mensaje de renacimiento, renovación, luz y libertad. Y, sin embargo, aquí, en medio de nosotros, ¡qué lejos de hacerse realidad parece esa magnífica promesa! ¡Qué distante parece el Reino de amplio dominio y paz, de seguridad, justicia e integridad, que el profeta Isaías anunció, como hemos escuchado en la primera lectura (cf. \emph{Is} 9, 7) y que proclamamos como definitivamente establecido con la venida de Jesucristo, Mesías y Rey!
	
	\textbf{Benedicto XVI, papa,} \emph{Homilía} en Belén, 13 de mayo del 2009\cite{BenedictoXVI_Homilia_20090513}.
\end{patercite}



\newsection			


\section{Homilías}
\homiliasNavidadABC

\subsection{San Juan XXIII, papa}

\subsubsection{Homilía (1962)}

\src{Lunes 24 de diciembre de 1962} 

\begin{body}
	\emph{Venerables hermanos y queridos hijos:}
	
	\ltr{E}{sta} misa de la noche de la Navidad del Señor santifica las más hermosas interioridades del alma, que tienden a lo que es la esencia viva de la unión con Cristo: la religión sincera, liturgia bien comprendida y anhelo de perfección cristiana. Lo advertimos en este momento de tranquilo recogimiento, bajo la mirada del Divino Infante.
	
	En realidad, los grandes problemas de la vida social e individual se acercan a la cuna de Belén, al paso que los ángeles invitan a dar gloria a Dios, gloria a Cristo redentor y salvador, y a excitar gozosamente las buenas voluntades para la celebración de la paz universal.
	
	Gran don, gran riqueza en verdad, es la paz del mundo, que va tras la paz. Lo hemos repetido en el radiomensaje navideño, y Nos satisface dar gracias al Señor por haberlo hecho acoger con buena voluntad de un extremo al otro de la tierra, como confirmación de la luz de esperanza encendida y viva en todas las naciones.
	
	Las súplicas de todos continúan pidiendo la conservación y el perfeccionamiento de este don celestial, al paso que son cada vez más atentos y prudentes todos los movimientos de ideas, palabras y actividades, y se multiplican en todos los campos los esfuerzos y los acuerdos para alejar y superar los obstáculos, conocer y substraer las causas que provocan los conflictos.
	
	Comprendednos, queridos hijos, si hemos preferido, para la misa de Navidad, la sencillez de nuestra capilla privada a las majestuosas bóvedas de los templos romanos, como para dejarnos envolver por el ambiente de las humildes iglesias del campo y de la montaña, de las innumerables instituciones de asistencia social, que son el refugio de la inocencia pobre y abandonada, consuelo y endulzamiento de las lágrimas amargas, reparación de injusticias palmarias y no suficientemente conocidas.
	
	También pensamos en vosotros, queridos enfermos y ancianos, que sufrís dolores y soledad; que vuestro dolor y soledad alcance grandes merecimientos a vosotros y bien a la humanidad.
	
	Hay también circunstancias y situaciones que en esta solemnidad hacen más evidente y agudo el contraste con el gozo de la Navidad. Reclamo eficaz no para disminuir el servicio que hacemos a la verdad y a la justicia, ni para olvidar el inmenso bien realizado por las almas rectas, que tienen como honor la ley divina y el Evangelio; sino para alentar las mejores energías a reparar los errores y a reavivar en el mundo el fervor religioso y las piadosas tradiciones paternas como gozo tranquilo de la Navidad.
	
	Hijos queridos: Junto a la cuna del Niño recién nacido, del Hijo de Dios hecho hombre, todos los hombres que caminan por la tierra piensan con conciencia clara y seria que en la hora suprema se les pedirá cuenta estrecha del don de la vida; y ésta tendrá una sanción definitiva de premio o de castigo, de gloria o de abominación.
	
	En la conciencia de este rendir cuentas es donde se mide la participación de los cristianos y de todos los hombres en el gran misterio que conmemoramos en esta noche; de aquí surge el deseo de que por la luz del Verbo de Dios la civilización humana reciba la llamita que le puede transformar en vivo fulgor, en beneficio de los pueblos.
	
	En torno a la cuna de Jesús sus ángeles cantaron la paz. Y quien creyó en el mensaje celestial y le hizo honor consiguió gloria y alegría. Así ayer; y así será siempre a lo largo de los siglos.
	
	La historia de Cristo es perpetua. Bienaventurado quien la comprende y consigue gracia, fortaleza y bendición. Amén. Amén.
	
	*\emph{ AAS} 55 (1963) 51; \emph{Discorsi-Messaggi-Colloqui del Santo Padre Giovanni XXIII}, vol. V, pp. 63-65.
\end{body}

\newsection			


\subsection{San Pablo VI, papa}

\subsubsection{Homilía (1965)} 

\src{Capilla Sixtina\\ Viernes 24 de diciembre de 1965.}

\begin{body}
	\ltr{E}{sta} santa noche vuelve a proponer a nuestra mente la meditación siempre nueva, siempre sugestiva y, a decir verdad, inagotable, del misterio fundamental de todo el Cristianismo: ¡Dios se hizo hombre!
	
	\textquote{Si alguno --- dice Santo Tomás --- considera con atención y piedad el misterio de la Encarnación, hallará una profundidad de sabiduría tal, que sobrepuja, todo conocimiento humano} (\emph{Contra Gentiles,}4, 54).
	
	En efecto, decir: Dios, es como decir la Grandeza, el poder, la santidad infinita. Decir: el hombre, es como decir la pequeñez, la debilidad, la miseria. Entre estos dos extremos, la distancia parece imposible de salvar, el foso parece imposible de colmar. Y he aquí que en Cristo estos dos conceptos son una sola cosa. La misma persona vive, a la vez, en la naturaleza divina y en la naturaleza humana de Cristo. El Padre de los Cielos puede decir: \textquote{Este es mi hijo bienamado} (\emph{Mt}. 17, 5), como a su vez lo puede decir la Virgen María dirigiéndose al Infante del pesebre que acaba de dar a luz.
	
	Misterio inefable de unión: lo que estaba dividido se reúne, lo que parecía incompatible se acerca, los extremos se funden en uno solo: dos naturalezas --- la humana y la divina --- en una sola persona, la del Hombre-Dios. He aquí toda la teología de la Encarnación, el fundamento y la síntesis de todo el cristianismo.
	
	El prodigio inicial, realizado en Cristo, halla su continuación misteriosa en lo que aquí abajo, hasta el fin de los tiempos, es el \textquote{Cuerpo místico} de Cristo, la gran familia de todos los que creen en El. Porque todo hombre debe unirse a Dios: \textquote{Dios se hizo hombre --- dice magníficamente San Agustín --- para que el hombre se haga Dios}. Tal es el designio divino, revelado en el misterio de Navidad. Y la historia de la Iglesia a través de los siglos, constituye la historia de la realización de tal designio.
	
	En la Encarnación, Dios ha unido el hombre a sí con vínculos tan fuertes que se demuestran superiores a todos los demás, más fuertes que los de la carne y la sangre e incluso los que unen al hombre con lo que le es más valioso en el mundo: la vida. ¿No nos habla acaso todo, aquí en Roma, del valor de los mártires cristianos de los primeros siglos? Hombres, mujeres y también niños dan testimonio ante el verdugo de que separarse de Dios por una abjuración sería para ellos mayor desgracia que perder la vida. La sacrifican para permanecer unidos a Dios.
	
	Cuando la espada del perseguidor romano cesó de herir, las grandes almas cristianas van a buscar a Dios en la soledad. Se abandona la familia, se renuncia a formar una, para unirse mejor a Dios. La corona de la virginidad es ambicionada con el mismo fervor con que se ambiciona la del martirio. La ofrenda cotidiana de sí mismo en la vida monástica tomó el lugar del sacrificio cruento realizado da una sola vez. Y en las mil formas de la vida consagrada, esta unión del hombre con Dios, amado sobre todas las cosas, seguirá manifestándose a través de los siglos hasta nuestros días. La Iglesia suscitará también legiones de santos en el mundo; junto a los mártires, las vírgenes, los doctores, los pontífices y los confesores, ella tendrá la inmensa familia de sus santas mujeres, madres de familia y viudas; en todas las épocas y en todos los países ella suscitará innumerables y fieles ejemplos en muchos hogares cristianos para testimoniar lo que el hombre es capaz de hacer para unirse a Dios, cuando comprende lo que Dios ha hecho para unirse al hombre.
	
	Modelo sublime y principio de la unión del hombre con Dios, la Encarnación se reveló también un maravilloso factor de civilización. ¿Quién como los Apóstoles del Dios encarnado, ha contribuido tanto en el transcurso de los siglos a elevar a los pueblos y a revelarles, además de la grandeza de Dios, su propia dignidad?
	
	La sociedad en la que penetra el fermento cristiano ve elevarse poco a poco su nivel moral y ampliarse su horizonte a las dimensiones del mundo pues la que parecía que sólo incumbía a las relaciones del hombre con Dios se revela el más poderoso factor de unión entre los hombres mismos. El poder de unión de la fe cristiana actúa en el seno de las familias y de los pueblos; derriba las barreras de castas, razas y naciones. La fe que une el hombre a Dios une también a los hombres entre si en un ideal común, en un esfuerzo común, en una esperanza común. ¡Qué motivo ilimitado de meditación! La fe en el Dios encarnado penetra, a lo largo de los siglos, las diversas culturas y las purifica, las enriquece, las transforma. Es la inteligencia humana que se ha superado a si misma, es la filosofía humana que recibió el complemento de las luces divinas como una luz más viva sobre su camino. ¿Y no es acaso también la fe la que inspiró a Miguel Ángel las obras de arte contenidas en esta Capilla, que suscitan la admiración de los hombres de generación en generación?
	
	Pues este enriquecimiento de la cultura es al mismo tiempo un estupendo principio de unión: una civilización cristiana que madura en un país, significa el ingreso de este país en la gran familia donde una misma fe pone en comunión las inteligencias, los corazones y las voluntades. No se terminaría nunca de enumerar los maravillosos desarrollos que jalonan la historia de la civilización. ¿Y todo esto qué es sino, en definitiva, la consecuencia de la Encarnación?
	
	De estos amplios frescos que pueden evocar la historia de la Iglesia, hay que volver al hombre que es su protagonista y su artífice. En el interior del hombre, en su alma, en su psicología, hay que captar las armonías de la fe y de la inteligencia.
	
	La Encarnación puede parecer ante todo, a la inteligencia humana, un peso muy difícil de llevar. Santo Tomás lo dice sin rodeos: de todas las obras divinas es la que más sobrepasa a la razón humana: porque no se puede imaginar --- dice Santo Tomás --- nada más admirable (\emph{Contra Gentiles}, 4, 27). ¿A quién, en efecto, sé le hubiera ocurrido que Dios un día se habría hecho hombre.
	
	Pero esta sublime verdad no encandila al espíritu que la recibe con humildad; antes bien, lo ilumina con la luz nueva y superior. A esta luz el hombre comprende su destino, ve la razón de su existencia, la posibilidad de salir de la miseria, de alcanzar el objetivo de sus esfuerzos. También ve el valor de las creaturas, la ayuda y el obstáculo que éstas pueden constituir para él en su camino hacia Dios. Aquí también, y sobre todo aquí, el misterio de Navidad ejerce su acción unificadora. Y, escrutándolo más profundamente, el creyente no halla por cierto una explicación entre tantas del destino del hombre sino la explicación definitiva: ¡no hay más que un Cristo, no hay más que una salvación! Y tal salvación, lejos de estar reservada a una nación privilegiada, se ofrece a todos. El alma del creyente se siente entonces penetrada por un sentimiento de fraternidad universal; comprende en qué radica la verdadera unidad de destino de la humanidad, que está en el designio de Dios que nos manifestó la Encarnación; comprende el principio fundamental del hombre con Dios y de los hombres entre sí; Navidad se ha vuelto para esa alma lo que es: más que un misterio de unión, un misterio de unidad.
	
	¿Y de dónde procede o dónde tiene su fuente ese misterio? Digámoslo con una palabra que explica todo: es el efecto del amor Este medio divino de unificar al hombre en sí mismo y de unificar al género humano alrededor del Dios hecho hombre, no es y no puede ser una determinación impuesta por la fuerza, a la cual fuera imposible substraerse. Así pues la fe es propuesta y no impuesta. Dios respeta demasiado a su creatura, a la que hizo libre, no esclava. Si la fe y la inteligencia son amigas, ¡cuánto más lo serán la fe y la libertad! ¿Qué valor tendría un amor si fuera una obligación y no una elección?
	
	Así el Infante del pesebre nos revela la última palabra del misterio: Dios se ha encarnado porque amó al hombre y porque quiso salvarlo. Al amor se lo puede aceptar o rechazar. Pero si se lo acepta, produce en el corazón una paz y un gozo indescriptibles: Pax hominibus bonae voluntatis !Quiera Dios, hecho hombre, abrir en esta noche nuestras inteligencias y nuestros corazones para que \textquote{conociendo a Dios visiblemente, seamos atraídos por su intermedio hacia el amor de las cosas invisibles: «ut dum visibiliter Deum cognoscimus, per huno in invisibilium amorem rapiamur!} (Prefacio de Navidad). Amén.
\end{body}

\subsubsection{Homilía (1977): ¿Qué es la Navidad?}

\src{Basílica de San Pedro. \\Sábado 24 de diciembre de 1977.}

\begin{body}
	¡Hermanos e hijos amadísimos!
	
	\ltr{E}{speráis} de nosotros una palabra que resuena ya en vuestros espíritus; el hecho de escucharla una vez más en esta noche y en este lugar os haga reconocer su perenne novedad, su fuerza de verdad, su maravillosa y beatificante alegría. No es nuestra, es celestial. Nuestros labios repiten el \textbf{anuncio del ángel}, que resplandeció en la noche, en Belén, hace 1977 años, y que tras confortar a los humildes y asustados pastores, vigilantes al raso sobre su rebaño, vaticinó el hecho inefable que se estaba realizando en un pesebre cercano: \textquote{Os traigo una buena nueva, una gran alegría, que es para todo el pueblo; pues os ha nacido hoy un Salvador, que es el Mesías, Señor, en la ciudad de David (Belén)} (\emph{Lc} 2, 10-11).
	
	¡Así es, así, hermanos e hijos! Y puesto que es así, queremos extender nuestro grito humilde e impávido a cuantos \textquote{tienen oídos para escuchar} (cf. \emph{Mt} 11, 15). Un hecho y una alegría; ¡he aquí la doble grande noticia!
	
	El hecho parece casi insignificante. Un niño que nace y ¡en qué condiciones tan humillantes! Lo saben nuestros muchachos cuando preparan sus belenes, ingenuos pero auténticos documentos de la realidad evangélica. La realidad evangélica transparenta una concomitante realidad inefable: ese Niño vive de una trascendente filiación divina, \textquote{será llamado Hijo del Altísimo} \emph{(Lc} 1, 32). Hagamos nuestras las expresiones entusiastas de nuestro gran predecesor, San León Magno, que exclama: \textquote{Nuestro Salvador, amadísimos, ha nacido hoy: ¡gocemos! ¡No hay lugar para la tristeza cuando nace la vida que, apagando el temor de la muerte, nos infunde la alegría de la promesa eterna} \emph{(Serm. I de Nativ. Dom)}.
	
	Así que mientras el misterio supremo de la vida trinitaria del Dios único se nos revela en las tres distintas Personas, Padre generante; Hijo engendrado, unidos ambos en el Espíritu Santo, otro misterio llena de maravilla inextinguible nuestra relación religiosa con Dios, abriendo el cielo a la visión de la gloria de la infinita trascendencia divina y, superando en un don de incomparable amor toda distancia, la proximidad, la cercanía de Cristo-Dios hecho hombre nos muestra que El está con nosotros, que está en busca de nosotros: \textquote{Porque se ha manifestado la gracia salutífera de Dios a todos los hombres} (\emph{Tit} 2, 11; 3, 4).
	
	¡Hermanos, hombres todos! ¿Qué es la Navidad sino este acontecimiento histórico, cósmico, sumamente comunitario porque asume proporciones universales y al mismo tiempo incomparablemente íntimo y personal para cada uno de nosotros, pues el Verbo eterno de Dios, en virtud del cual vivimos ya en nuestra existencia natural (cf. \emph{Act} 17, 23-28) ha venido en busca de nosotros? El, eterno, se ha inscrito en el tiempo; El, infinito, se ha como anonadado, \textquote{en la condición de hombre se humilló, hecho obediente hasta la muerte, y muerte de cruz} (\emph{Flp} 2, 6 ss.). Nuestros oídos están habituados a semejante mensaje y nuestros corazones se han hecho sordos a semejante llamada, una llamada de amor: \textquote{tanto amó Dios al mundo\ldots{}} (\emph{Jn} 3, 16); más aún, seamos precisos: cada uno de nosotros puede decir con San Pablo: \textquote{me amó y se entregó por mí} (\emph{Gál} 2, 20).
	
	La Navidad es esta llegada del Verbo de Dios hecho hombre entre nosotros. Cada uno puede decir: ¡por mí! Navidad es este prodigio. Navidad es esta maravilla. Navidad es esta alegría. Nos vienen a los labios las palabras de Pascal: ¡alegría, alegría, alegría, llantos de alegría!
	
	¡Oh! Que esta celebración nocturna de la Natividad de Cristo sea de veras para todos nosotros, para la Iglesia entera, para el mundo, una renovada revelación del misterio inefable de la Encarnación, un manantial de felicidad inagotable! ¡Así sea!
\end{body}

\newsection		


\subsection{San Juan Pablo II, papa}

\subsubsection{Homilía(1980): El Don más grande}

\src{Basílica de San Pedro. \\Miércoles 24 de diciembre de 1980.}


\begin{body}
	\ltr[1. ]{Q}{ueridos} hermanos y hermanas, reunidos en la basílica de San Pedro en Roma, y vosotros todos, los que me escucháis en este momento, desde cualquier parte del globo terrestre.
	
	He aquí que estoy ante vosotros, yo, siervo de Cristo y administrador de los misterios de Dios (cf. \emph{1 Cor} 4, 1), como mensajero de la noche de Belén: \emph{la noche de Belén 1980}.
	
	La noche del nacimiento de Jesucristo, Hijo de Dios, nacido de María Virgen, de la casa de David, de la estirpe de Abraham, padre de nuestra fe, de la generación de los hijos de Adán.
	
	El Hijo de Dios, de la misma sustancia que el Padre, viene al mundo como hombre.
	
	2. Es una noche profunda: \textquote{El pueblo que caminaba en tinieblas vio una luz grande; habitaban tierras de sombras, y una luz les brilló} (palabras del \textbf{Profeta Isaías}, 9, 2).
	
	¿Cómo se cumplen estas palabras en la noche de Belén? He aquí que las tinieblas envuelven la región de Judá y los países cercanos. Sólo en un lugar aparece la luz. Sólo llega a un pequeño grupo de hombres sencillos.
	
	\emph{Son \textbf{los pastores}}, que estaban en aquella región \textquote{velando por turno su rebaño} \emph{(Lc} 2, 8).
	
	Solamente en ellos se cumple, esa noche, la profecía de Isaías. \emph{Ven una gran luz:} \textquote{La gloria del Señor los envolvió de claridad y se llenaron de gran temor} \emph{(Lc} 2, 9).
	
	Esta luz deslumbra sus ojos y, al mismo tiempo, ilumina sus corazones. He aquí que ellos ya saben: \textquote{Hoy, en la ciudad de David, os ha nacido un Salvador, el Mesías, el Señor} \emph{(Lc} 2, 11). Son los primeros en saberlo. En cambio, hoy lo saben millones de hombres en todo el mundo. \emph{La luz} de la noche de Belén ha llegado a muchos corazones, y sin embargo, al mismo tiempo, permanece la oscuridad. A veces, incluso parece que se hace más intensa\ldots{}
	
	¿Qué puedo pedir en mis plegarias esta noche de Belén 1980, yo, siervo de Cristo y administrador de los misterios de Dios?, ¿qué puedo pedir principalmente, junto con todos vosotros, los que participáis en la luz de esta noche, sino que esta luz \emph{llegue a todas partes,} que encuentre acceso a todos los corazones, que vuelva allá, donde parece que se ha apagado\ldots{}? ¡Que ella \textquote{despierte}!, tal como despertó a los pastores en los campos de las cercanías de Belén.
	
	3. \textquote{Acreciste la alegría, aumentaste el gozo}, palabras del \textbf{Profeta Isaías}.
	
	Los que aquella noche lo acogieron, encontraron \emph{una gran alegría}. La alegría que brota de la luz. La oscuridad del mundo superada por la luz del nacimiento de Dios.
	
	No importa que esta luz, por el momento sea participada, solamente por algunos corazones: que participe de ella la Virgen de Nazaret y su esposo, la Virgen a la que no fue dado traer a su Hijo al mundo bajo el techo de una casa en Belén, \textquote{porque no tenían sitio en la posada}\emph{(Lc} 2, 7). Y participan de esta alegría \emph{los pastores,} iluminados por una gran luz en los campos cerca de la ciudad.
	
	No importa que, en esa primera noche, la noche del nacimiento de Dios, la alegría de este acontecimiento llegue sólo a estos pocos corazones. No importa.
	
	Está \emph{destinada} a \emph{todos los corazones humanos}. ¡Es la alegría del género humano, alegría sobrehumana! ¿Acaso puede haber una alegría mayor que ésta, puede haber una Nueva mejor que ésta: el hombre \emph{ha sido aceptado por Dios para convertirse en hijo} suyo en este Hijo de Dios, que se ha hecho hombre?
	
	Y ésta es una alegría cósmica. Llena a todo el mundo creado: creado por Dios ---mundo que se alejó de Dios a causa del pecado--- y he aquí: restituido de nuevo a Dios mediante el nacimiento de Dios en cuerpo humano.
	
	Es la \emph{alegría cósmica}.
	
	La alegría que llena toda la creación, llamada esta noche a compartirla de nuevo según estas palabras que descienden del cielo: \textquote{Gloria a Dios en el cielo y en la tierra paz a los hombres que Dios ama} (a los hombres de buena voluntad) \emph{(Lc} 2, 14).
	
	Esta noche quiero \emph{estar particularmente cercano a vosotros, a todos vosotros los que sufrís}
	
	y a vosotros, las víctimas del terremoto,
	
	y a vosotros, los que vivís atemorizados por las guerras y las violencias,
	
	y a vosotros, los que os halláis privados de la alegría de esta Santa Misa a medianoche en la Navidad del Señor,
	
	y a vosotros, los que estáis inmovilizados en el lecho del dolor,
	
	y a vosotros, los que habéis caído en la desesperación, en la duda sobre el sentido de la vida y sobre el sentido de todo.
	
	Cercano a todos vosotros.
	
	A vosotros \emph{de modo especial} está destinada esta alegría, que llena los corazones de los pastores de Belén, ella es sobre todo para vosotros. Porque es la alegría de los hombres de buena voluntad, de los que tienen hambre y sed de justicia, de los que lloran, de los que sufren persecución por la justicia.
	
	Que se cumplan en vosotros las palabras del \textbf{Profeta}: \textquote{Acreciste la alegría, aumentaste el gozo\ldots{}} \emph{(Is} 9, 2).
	
	4. \textquote{Se gozan en tu presencia, como se gozan al segar}, palabras de Isaías.
	
	Ciertamente: los hombres sencillos, que viven del trabajo de sus manos, no se presentan ante el recién nacido con las manos vacías. No se presentaron con los corazones vacíos. Llevan los dones.
	
	\emph{Responden con dones al don}.
	
	Queridos hermanos y hermanas, los que estáis reunidos en la basílica de San Pedro y todos los que me escucháis en este momento y en cualquier punto del globo terrestre: ¡en esta noche toda la humanidad ha recibido \emph{el don más grande!} ¡Esta noche cada uno de los hombres recibe el don más grande! Dios mismo se convierte en el don para el hombre. \emph{El hace de sí mismo el \textquote{don}} para la naturaleza humana. ¡Entra en la historia del hombre no sólo ya mediante la palabra que de El viene al hombre, sino mediante el Verbo que se ha hecho carne!
	
	Os pregunto a todos: ¿tenéis conciencia de este don?
	
	Estáis \emph{dispuestos} a responder con el don al don? Tal como los pastores de Belén, que respondieron\ldots{}
	
	Y os deseo desde lo profundo de esta nueva noche de Belén 1980, que aceptéis el don de Dios, que se ha hecho hombre.
	
	¡Os deseo que respondáis con el don al don!
\end{body}

\subsubsection{Homilía (1983):}

\src{Basílica de San Pedro, Santa Navidad, \\24 de diciembre de 1983.}

\begin{body}
	1. \emph{Custos, quid de nocte?} --- Centinela, ¿qué hay de la noche? (cf. \emph{Is} 21, 11).
	
	\ltr[¡]{H}{e} aquí lo que os anuncio de la medianoche! Esta medianoche se mueve de este a oeste. Sigue todos los meridianos. En Oriente ya nos ha precedido, en Occidente está a punto de llegar \ldots{} He aquí que os anuncio la medianoche; en cada lugar y en cada momento en que viaja el globo terrestre, anuncio la Medianoche!
	
	Yo, guardián del Gran Misterio. Yo, obispo de Roma: anuncio la medianoche de Navidad en todas partes. \textquote{Cantad al Señor un cántico nuevo, cantad al Señor toda la tierra} (\emph{Sal} 96, 1).
	
	2. ¡Canta, tierra!
	
	Canta porque fuiste elegida, elegida entre todo el universo. Y todo el universo fue elegido junto a ti.
	
	¡Canta, oh tierra!
	
	\textquote{Alégrese el cielo, goce la tierra, retumbe el mar y cuanto lo llena; vitoreen los campos y cuanto hay en ellos, aclamen los árboles del bosque} (\emph{Sal} 96, 11-12).
	
	Canta, tierra, porque has sido elegida para ser el lugar del nacimiento de Dios en un cuerpo humano. ¡Que se reúna toda la tierra alrededor de esta Medianoche! ¡Que hable la potencia de toda la creación! ¡Que hable por medio de la existencia de todos los mundos creados! ¡Que hable por medio de la lengua de hombre!
	
	3. Y he aquí que un hombre está hablando. Su nombre es Lucas, evangelista. Él dice: \textquote{\ldots{} los días del parto se cumplieron para ella (María). Y dio a luz a su hijo primogénito, lo envolvió en pañales y lo acostó en un pesebre, porque no había lugar para ellos en la posada} (\emph{Lc} 2, 6-7).
	
	De esta manera vino al mundo el hijo de Dios: María era la esposa de José, de la familia de David; de José, que era carpintero en Nazaret. El Niño vino al mundo en Belén porque tanto María como José habían ido allí debido al censo que había ordenado César Augusto.
	
	4. Esto dijo el hombre. Al mismo tiempo, el Ángel del Señor le habla al hombre. Habla a los pastores cuando, en medio de la noche profunda de Belén, \textquote{la gloria del Señor los envolvió en luz}. Y los pastores \textquote{se aterrorizaron} (\emph{Lc} 2, 9). Les dice: \textquote{¡No tengáis miedo! He aquí, os anuncio un gran gozo, que será de todo el pueblo: hoy os ha nacido en la ciudad de David un Salvador, que es Cristo el Señor. Esta será la señal para vosotros: encontraréis a un niño envuelto en pañales, acostado en un pesebre} (\emph{Lc} 2, 10-12).
	
	El hombre y el ángel hablan del mismo hecho y señalan el mismo lugar. El Ángel habla de lo que el hombre no se atreve a decir: el Mesías, que es el Ungido, vino al mundo en Belén, el que viene a visitar a la humanidad en el poder del Espíritu Santo. El Salvador del mundo nació en la tierra en Belén. Él \ldots{} juzgará a la tierra. Él \ldots{} juzgará al mundo con justicia.
	
	Sí, se entregará \textquote{a sí mismo por nosotros, para redimirnos de toda iniquidad y purificar para sí un pueblo de su propiedad \ldots{}} (\emph{Tit} 2, 14). Él se dará a sí mismo por nosotros: ¡aquí está su juicio!
	
	5. \emph{Custos, quid de nocte?} --- Centinela, ¿qué hay de la noche? (cf. \emph{Is} 21, 11).
	
	He aquí que os anuncio la medianoche \ldots{} Desde lo más profundo de la noche de Belén, que es la noche de toda la humanidad que vive en la tierra \ldots{} \textquote{Ha aparecido la gracia de Dios, portadora de salvación para todos los hombres} (\emph{Tit} 2, 11).
	
	¿Qué es la gracia? Gracia y complacencia divina se concentran completamente en este Niño acostado en la cuna. Este Niño es el Hijo Eterno, el Hijo en quien Dios se complace, el Hijo del Amor eterno. Este Niño es el Hijo de María. Es Hijo de hombre y verdadero hombre.
	
	La eterna satisfacción del Padre se concentra en el hombre: ¡aquí está la Gracia! \textquote{Paz en la tierra a los hombres que ama el Señor} (\emph{Lc} 2, 14). Esta complacencia divina con el hombre fue traída a la tierra por el Hijo de María en la noche de Belén. \textquote{Ha aparecido la gracia de Dios} (\emph{Tit} 2, 11). Desde Belén comienza su irradiación sobre el hombre de todos los tiempos.
	
	¿Qué es la Gracia? Es el comienzo de la gloria, de esa gloria que Dios tiene en lo más alto del cielo. Y para esta gloria el hombre fue llamado en Jesucristo. Y esto sucedió en la noche de Belén.
	
	6. Por tanto: ¡regocíjate tierra! ¡Tierra, que eres la morada del hombre! ¡Acoge en ti una vez más el esplendor de la noche del nacimiento divino! ¡Reúnete en torno a este esplendor! ¡Proclama el gozo de la redención a toda la creación! Anuncia al mundo entero la esperanza de su Redención.
	
	\textquote{Vitoreen los campos y cuanto hay en ellos, aclamen los árboles del bosque ante el Señor} (\emph{Sal} 96, 12-13). He aquí que viene. He aquí que ya está entre nosotros: ¡Emmanuel! Todo el poder de la Redención del mundo está en él. ¡Aleluya!
\end{body}

\subsubsection{Homilía (1986):}

\src{Basílica Vaticana, \\Miércoles 24 de diciembre de 1986.}

\begin{body}
	1. \textquote{\emph{¡No tengáis miedo! Os anuncio una gran alegría \ldots{} hoy os ha nacido en la ciudad de David un salvador, que es Cristo el Señor}} (\emph{Lc} 2, 10-11).
	
	\ltr{E}{stamos} aquí reunidos la noche de la vigilia de Navidad para volver a escuchar estas palabras, después de siglos y siglos. Los pastores de los campos de Belén las oyeron por primera vez. Y esta es la razón por la que la asamblea litúrgica de Nochebuena lleva el nombre de \textquote{Misa de los Pastores} en algunos países.
	
	2. {[}Estamos reunidos en la Basílica de San Pedro. No solo las personas aquí presentes participan en la liturgia, sino también muchos de nuestros hermanos y hermanas a quienes este rito solemne se transmite a través de las ondas de la radio y la televisión.{]}
	
	El acontecimiento de la noche de Belén nos une a todos. En momentos sucesivos, marcados por el tiempo que transcurre en la tierra, se desarrolla en todos los lugares de nuestro planeta.
	
	Las estaciones del año y las condiciones climáticas de esta noche santa también son diferentes en las diversas regiones: ocurre tanto en el calor tropical, en el duro invierno nórdico y en las tormentas de nieve. Incluso en condiciones tan diferentes, lo que ocurre en esta hora es siempre el mismo acontecimiento. Y los que anuncian la noche de Belén proclaman la misma \textquote{gran alegría}, aunque sus palabras se escuchen en muchos idiomas diferentes en todo el mundo.
	
	3. Los pastores en los campos de Belén - los primeros testigos del evento - eran hijos de Israel, cuya historia estaba relacionada con la promesa del Mesías. De modo que las palabras que escucharon podrían, y también deberían, despertar su asombro. Pero, al mismo tiempo, no eran palabras incomprensibles para ellos.
	
	Los pastores sabían lo que significaba la palabra \textquote{Mesías}. Durante generaciones, Israel había vivido a la espera del Mesías, del Ungido del Señor. Si el \textquote{Mesías} viene al mundo en la \textquote{ciudad de David} es porque esta circunstancia pertenece a los pronósticos proféticos. La ciudad de David es precisamente Belén.
	
	Además, el Mesías debe haber venido del \textquote{linaje de David}. De la casa y la familia de David eran también José y María, la Madre del Recién Nacido. Y por tanto, debido al censo ordenado por los romanos, tuvieron que ir directamente a Belén, partiendo de Nazaret, donde vivían.
	
	4. Entonces, las palabras escuchadas por los pastores les resultaron comprensibles. En ellos se cumplió la promesa hecha a Israel. Al mismo tiempo, estas palabras deben haberles sorprendido. El ángel dijo: \textquote{Encontraréis a un niño envuelto en pañales, acostado en un pesebre \ldots{}: Esta será para vosotros la señal}.
	
	Los pastores no dudaron de que las palabras que escucharon provenían de Dios, se rindieron ante un \textquote{gran gozo}, y al mismo tiempo demostraron tranquilidad y mesura. Caminaron en la dirección indicada y encontraron todo exactamente como se les dijo.
	
	Se convirtieron en testigos presenciales del acontecimiento, cuya dimensión adecuada sólo es accesible a los \textquote{ojos luminosos} de la fe.
	
	5. Todos nosotros, reunidos religiosamente en tantos lugares de la tierra para renovar y hacer presente, con la liturgia eucarística, el acontecimiento salvífico, que tuvo entre los primeros participantes a los pastores de Belén en esa noche santa; todos nosotros, en breve, nos pondremos de rodillas, cuando resuenan las conocidas palabras del Credo Niceno-Constantinopolitano: Dios de Dios, Luz de luz, de la misma sustancia que el Padre \ldots{} \textquote{Incarnatus est de Spiritu Sancto ex Maria Virgine, et homo factus est}.
	
	6. El misterio de la Encarnación. El misterio de Dios \textquote{hecho hombre} en el Hijo eterno.
	
	Nos arrodillaremos y permaneceremos postrados para manifestar esta realidad inefable. Permaneceremos de rodillas en nombre de todos los hombres, en lugar de toda la creación.
	
	El acontecimiento de la noche de Belén revela ante los ojos de nuestra fe la plenitud definitiva del sentido de la creación, del mundo, del hombre.
	
	7. Y luego un sacerdote o un diácono, ministros de la Eucaristía, se presentará ante cada uno de vosotros y dirá: \textquote{Corpus Christi \ldots{} --- El cuerpo de Cristo}. Y cada uno de vosotros responderá: \textquote{Amén}; la palabra de fe que reconoce, adora, agradece. La palabra que une a los pastores ante el acontecimiento de la noche de Belén: el Verbo se hizo carne, la carne y la sangre de la nueva y eterna alianza de Dios con el hombre.
	
	El acontecimiento de la noche de Belén se convirtió en el comienzo de la nueva comunión, que penetra en el corazón y la historia del hombre en la tierra. \textquote{Gloria a Dios en las alturas y paz en la tierra a los hombres que él ama}.
	
	8. \textquote{Alégrese el cielo, goce la tierra, retumbe el mar y cuanto lo llena; vitoreen los campos y cuanto hay en ellos, aclamen los árboles del bosque. Delante del Señor, que ya llega} (\emph{Sal} 96 (95), 11. 13).
	
	A toda la creación, a todos los que viven esta noche sagrada en Belén: a los hermanos y hermanas esparcidos por el globo terrestre: alegría, paz y bendición. Amén.
\end{body}

\subsubsection{Homilía (1989):}

\src{Basílica de San Pedro, \\Domingo 24 de diciembre de 1989.}

\begin{body}
	1. \textquote{Os anuncio una gran alegría} (\emph{Lc} 2, 10).
	
	\ltr{F}{ue} precisamente a una hora de la noche como esta cuando los pastores de Belén escucharon el anuncio de gran alegría.
	
	A la misma hora nos encontramos todos reunidos aquí {[}en la Basílica de San Pedro{]} para escuchar el anuncio de la misma alegría.
	
	Y así, mientras lo hacemos, la gente se reúne, nuestros hermanos y hermanas se reúnen en muchos lugares del mundo.
	
	{[}A todos, dondequiera que estén reunidos, el Obispo de Roma saluda con las mismas palabras: \textquote{Os anuncio una gran alegría}.
	
	Este saludo mío va para todos los hombres, en todos los continentes.
	
	Va, con especial cariño y con un recuerdo siempre vivo, a las naciones que visité este año, a las multitudes que conocí en esos países: en el Lejano Oriente, en África, en los países nórdicos. A los queridos jóvenes que, en Santiago de Compostela, celebraron conmigo la Jornada Mundial de la Juventud, representando también a todos sus compañeros, de todo el mundo.
	
	Dirijo también este saludo, y de manera especial, a los hombres y mujeres de todas las naciones que, conectados por radio y televisión, escuchan esta Santa Misa de Medianoche y participan, unidos espiritualmente con nosotros y con todos los creyentes del mundo, para el misterio de la natividad del Hijo de Dios en la tierra.{]}
	
	2. Este anuncio va dirigido a los hombres, pero no solo a ellos. La liturgia navideña, a medianoche, también llama a la alegría a todas las criaturas.
	
	\textquote{Alégrese el cielo, goce la tierra, retumbe el mar y cuanto lo llena; vitoreen los campos y cuanto hay en ellos, aclamen los árboles del bosque\ldots{} Cantad al Señor un cántico nuevo, cantad al Señor desde toda la tierra} (\emph{Sal} 97, 11-12. 1).
	
	Por tanto, desde este anuncio de Belén todas las criaturas están llamadas a la alegría. De hecho, quien nace de la Virgen María es \textquote{engendrado antes que todas las criaturas} (cf. \emph{Col} 1, 15). En él y para él todo fue creado. Todo el bien que se encuentra en las criaturas tiene en él su origen y su primer modelo.
	
	Por medio de él, el Padre miró una vez a toda la creación y \textquote{vio que todo era bueno \ldots{} muy bueno} (cf. \emph{Gn} 1, 10. 31).
	
	En esta noche en Belén, todos estamos llamados ---llamados una vez más--- a regocijarnos en la obra de la creación.
	
	3. \textquote{Os anuncio una gran alegría}.
	
	En el momento en que el Hijo, Verbo eterno, primogénito de toda criatura, se presenta en medio de sus criaturas, se reafirma este júbilo por la obra de la creación. Y, al mismo tiempo, se eleva.
	
	La criatura alcanza tal exaltación, que va más allá de su horizonte. Más allá del horizonte de la existencia y el conocimiento.
	
	\textquote{Acreciste la alegría, aumentaste el gozo} (\emph{Is} 9, 2).
	
	Pero esta exaltación lo alcanzan las criaturas en el hombre. Del hombre se dijo al principio que había sido creado a imagen y semejanza de Dios. En la noche de Belén se reconfirma totalmente esta verdad sobre el hombre con mayor fuerza.
	
	\textquote{Porque nos ha nacido un niño, un hijo nos ha sido dado} (\emph{Is} 9, 5).
	
	En la noche de Belén nace el Niño, el niño humano: para María \textquote{se cumplieron los días del parto. Dio a luz a su hijo primogénito, lo envolvió en pañales y lo acostó en un pesebre} (\emph{Lc} 2, 6-7).
	
	El mensajero celestial le dice lo mismo a los pastores; \textquote{Encontraréis a un niño envuelto en pañales, acostado en un pesebre} (\emph{Lc} 2, 12).
	
	4. Aquí está el Niño, el niño humano, el hijo del hombre, como todos los demás nacidos de mujer.
	
	Este niño es el Hijo: \textquote{Se nos ha dado un hijo}. Nos lo dio el Padre. Fue dado a los hombres y al mundo: \textquote{porque tanto amó Dios al mundo que dio a su Hijo unigénito} (\emph{Jn 3, 16}).
	
	\textquote{Un Hijo se nos ha dado}.
	
	En este Hijo eterno, que es de la misma sustancia que el Padre, Dios mismo entra en la historia del hombre y del mundo.
	
	En este Hijo \textquote{apareció \ldots{} la gracia de Dios, portadora de salvación para todos los hombres} (\emph{Tit} 2, 11).
	
	Dios, que creó al hombre a su imagen y semejanza, sabe quién es el hombre. Sabe lo que es el corazón humano, sabe que su corazón está inquieto hasta que descansa en él (cf. San Augustín, \emph{Confesiones}, I, 1: \emph{CSEL} 33, 1).
	
	Y por eso, precisamente por eso, \textquote{se nos ha dado un hijo}. El corazón humano, al llegar al pesebre de Belén, encuentra allí esa paz que sólo se encuentra en Dios. Esta paz está íntimamente ligada a la gloria de Dios, como proclama el mensaje de la noche de Belén.
	
	5. \textquote{Os anuncio una gran alegría \ldots{} hoy nos ha nacido \ldots{} un salvador} (\emph{Lc} 2, 10-11).
	
	Pero, ¿es esta alegría tan pura, tan plena como nos gustaría que fuera?
	
	Sí y no. De hecho, sobre ella se proyecta una sombra de tristeza. El Niño, el Hijo de Dios, nace en un establo, porque no había lugar para él en la posada (cf. \emph{Lc} 2, 7).
	
	El momento de su llegada es al mismo tiempo el momento de la no\emph{-}acogida, del rechazo: \textquote{No había sitio}. Esta sombra de tristeza se alargará. Se espesará hasta el punto del rechazo, a través de la Cruz, en el Gólgota. De esta manera rechazará el hombre al Hijo que nos fue dado por el Padre como signo de su amor.
	
	Jesucristo, \textquote{el cual se entregó por nosotros para rescatarnos de toda iniquidad} (\emph{Tit} 2, 14).
	
	6. Nosotros, reunidos aquí, saludamos, junto con nuestros hermanos y hermanas que están en comunión con nosotros, el nacimiento de Dios con la liturgia del sacrificio eucarístico. Es el sacrificio de nuestra redención. Este sacrificio hace presente la Cruz y la Resurrección: el misterio pascual de Cristo Este misterio tiene su comienzo en la noche de Belén, cuando nos nació un Salvador. ¡El redentor del hombre, el redentor del mundo!
	
	La Iglesia, que esta noche anuncia \textquote{una gran alegría}, sabe que esta alegría proviene totalmente de Dios, es el don de su amor.
	
	También sabe que sólo esta alegría expande el corazón humano a las dimensiones supra-temporales, que Dios mismo ha preparado para el hombre.
	
	Él lo sabe, y por eso repite, incluso en esta noche, frente al mundo: \textquote{Os anuncio una gran alegría. ¡Hoy nació el Salvador!}.
\end{body}

\subsubsection{Homilía (1992):}

\src{Misa en la Noche Santa. Basílica Vaticana, \\Jueves 24 de diciembre de 1992.}

\begin{body}
	1. \textquote{\emph{Gloria a Dios en las alturas y paz en la tierra a los que Dios ama}} (\emph{Lc} 2, 14).
	
	\ltr{E}{sta} es la noche que hemos estado esperando todo el año. En esta noche se cumplen las palabras del \textbf{profeta Isaías} sobre las tinieblas y la luz: \textquote{\emph{Sobre los que habitaban en tierra y en sombras de muerte una luz les brilló}} (\emph{Is} 9, 1).
	
	Esa luz atravesó la noche que había caído sobre Belén de Judea. Gracias a la luz de esa noche, los hombres se vieron inmersos en una luz extraordinaria: eran sobre todo hombres sencillos, \emph{los pastores que custodiaban} su rebaño. La luz brilló en sus almas. \emph{No solo había luz a su alrededor, sino también dentro de ellos.} La luz anunciada por Isaías había entrado en sus corazones. En esa luz, Dios mismo estaba presente. Fue una luz de teofanía.
	
	Como antes Abraham, Moisés y los profetas, ahora también ellos estaban dentro del rayo de la luz de Dios, que los había despertado en la noche y los había impulsado a partir hacia Belén: \textquote{Hoy \emph{nació allí,} en la ciudad de David\emph{,} el \emph{salvador, que es Cristo el Señor}} (\emph{Lc} 2, 11).
	
	2. No dentro de la ciudad, sino fuera de ella. El lugar de nacimiento del Salvador estaba envuelto en la oscuridad de esa noche. Los pastores habían sido advertidos: \textquote{\emph{Encontraréis un niño envuelto en pañales, acostado en un pesebre}} (\emph{Lc} 2, 12). ¿Es posible? ¿Por qué el Salvador del mundo viene a los suyos de esta manera? ¿Por qué, aunque había dejado claro desde el principio que venía, los suyos no lo aceptaban? De hecho, este ya era el caso en Belén.
	
	Los pastores estaban envueltos en una luz de arriba. Cuando se encontraron frente al recién nacido, se \emph{dieron cuenta de que habían llegado al centro de una Teofanía}. La misma certeza será demostrada más tarde también por los Magos que vinieron de Oriente, cuando se encuentren en el umbral de la cabaña. También ellos, como los pastores, entran en el \emph{rayo de la luz divina} que ha venido al mundo. Sobre esa luz no prevalecieron las tinieblas (cf. \emph{Jn} 1, 5). Y no prevalecerán. Como en la noche de Belén, ni la oscuridad de la indigencia, ni la miseria del abandono y la humillación \emph{pudieron sofocar la Luz del Misterio Divino}. He aquí que el Verbo se hizo carne.
	
	3. Como más tarde los Magos de Oriente, esa noche los pastores de Belén llevaron a cabo en sí mismos las palabras del Profeta sobre el pueblo, sobre el pueblo de la Antigua Alianza, de la cual el Mesías, el Salvador del mundo iba a nacer:
	
	He aquí que \textquote{\emph{El pueblo que caminaba en tinieblas vio una gran luz}} (\emph{Is} 9, 1).
	
	La salvación del mundo tiene su origen en Dios mismo, y su comienzo temporal aquí mismo, en medio de este pueblo elegido. Desde aquí debe extenderse por toda la tierra. He aquí, \textquote{el pueblo que andaba en tinieblas} verá una gran luz. \emph{Entre tantas naciones y pueblos de todo el mundo, un solo pueblo de Dios \ldots{}} El espacio del Nacimiento de Dios, que en un principio cubrió de luz los campos de Belén, se encuentra hoy en innumerables lugares de la tierra.
	
	Dondequiera que celebremos, a medianoche, esta liturgia llena de alegría, de cuyo Misterio, esa noche, los pastores participaron en Belén, ciudad de David, se renueva y se hace presente:
	
	\textquote{Acreciste la alegría, aumentaste el gozo} (\emph{Is} 9, 2).
	
	4. \emph{Esta alegría es más fuerte que la pobreza y la miseria.} Esta alegría la conocen los \textquote{pobres de espíritu}. Como entonces los pastores de Belén, así, a través de los siglos y generaciones, tantos y tantos hombres de \textquote{buena voluntad}. ¿De dónde viene esta alegría? ¿No deriva del hecho de que el nacimiento \textquote{de una mujer} (\emph{Gal} 4, 4) del Hijo consustancial al Padre da a todos \emph{la certeza del amor de Dios}? ¿Puede haber una demostración más convincente de que Dios ama al hombre, que ha encontrado su complacencia en los hombres? ¿Puede haber una verificación aún más evidente? Helo aquí, Aquel que es.
	
	Helo aquí Aquel que es, ---no ya en la zarza ardiente, ni en truenos y relámpagos, como en el monte Sinaí---. He aquí Aquel \emph{que es como uno de nosotros: como hombre \ldots{}} como un Niño recién nacido de la Virgen Madre. Encargado al cuidado de María y José.
	
	He aquí, El que es.
	
	5. \textquote{\emph{Natus est nobis \ldots{}}}.
	
	El espacio de la Teofanía de Belén se cumple hasta los confines de la creación. De hecho, va más allá de ellos. Abraza la tierra y, al mismo tiempo, se eleva a aquellas alturas que están llenas de la gloria de Dios.
	
	\textquote{\emph{Gloria a Dios en lo alto de los cielos}} (\emph{Lc} 2, 14).
	
	Ese Dios que amó al mundo, que lo amó hasta dar a su propio Hijo para la salvación del hombre, revela la paz a los hombres: \textquote{\emph{La paz os dejo, mi paz os doy. Os la doy no como os la da el mundo}} (\emph{Jn} 14, 27).
	
	¡Qué difícil es para el mundo asegurar la paz para el hombre, para los hombres, para las naciones, para las épocas históricas!
	
	\textquote{Yo os la doy \ldots{}}: ¡Paz en la tierra a los hombres de buena voluntad!
	
	Pero, ¿puede realmente prevalecer la paz en la tierra cuando falta buena voluntad, cuando a los hombres no les importa si Dios los ama?
	
	Esta noche, \emph{la Iglesia te} mira a ti, Jesucristo, que eres el Dios Fuerte y el príncipe de la paz, y \emph{te pide} paz para toda la humanidad redimida. Esta paz es tu Nombre.
	
	\emph{Erit Iste Pax!}
\end{body}

\subsubsection{Homilía (1995):} 

\src{Basílica Vaticana, Lunes 25 de diciembre de 1995.}

\begin{body}
	1. \textquote{Hoy nos ha nacido el Salvador} (Salmo responsorial).
	
	\ltr{A}{l} \textquote{hoy} del gran misterio de la Encarnación corresponde de manera particular esta hora, en la que celebramos la santa misa llamada de \textquote{medianoche}. Según la tradición, el Hijo de Dios vino al mundo en Belén en medio de la noche.
	
	Leemos en el texto del \textbf{profeta Isaías}: \textquote{El pueblo que caminaba en tinieblas vio una gran luz} (\emph{Is} 9, 1). A este pueblo pertenecían \emph{los pastores de Belén}, que guardaban su rebaño de noche y a quienes, en primer lugar, llegaba la noticia: \textquote{Hoy ha nacido en la ciudad de David un salvador, que es Cristo el Señor} (\emph{Lc} 2, 11). Y fueron los primeros en ir, siguiendo la llamada del ángel, al establo donde nació Jesús:
	
	\textquote{¡Hoy ha nacido Cristo el Señor, el Salvador}! \emph{Esta feliz noticia} invita a toda la creación a \emph{cantar al Señor} \textquote{\emph{un cántico nuevo}}: \textquote{Alégrese el cielo, goce la tierra, retumbe el mar y cuanto lo llena; vitoreen los campos y cuanto hay en ellos, aclamen los árboles del bosque} (\emph{Sal} 95, 11-12).
	
	Por eso, en Nochebuena el mundo entero resuena \emph{con cantos de alegría}, en todos los idiomas del mundo. Son cantos que poseen un encanto singular y contribuyen a crear el ambiente inconfundible de este período del año litúrgico. En verdad, como dice el profeta Isaías, \textquote{acreciste la alegría, aumentaste el gozo} (\emph{Is} 9, 2).
	
	2. \textquote{Hoy ha nacido} (cf. \emph{Lc} 2, 11).
	
	Junto al término \textquote{nació}, \emph{natus est}, encontramos otra expresión en los textos litúrgicos: \emph{apparuit}, \textquote{apareció}, \textquote{se manifestó}. Cuando nace un niño, aparece una nueva persona en el mundo. En referencia al nacimiento en Belén del Hijo de María, \emph{la liturgia habla de} \textquote{\emph{manifestación}} como se subraya especialmente en la \textbf{Carta del Apóstol San Pablo a Tito}: \textquote{Se ha manifestado la gracia de Dios, que trae la salvación para todos los hombres} \emph{(Tit} 2, 11).
	
	\textquote{Un niño nos ha nacido, un hijo se nos ha dado}, está escrito en el texto de \textbf{Isaías} (\emph{Is} 9, 5). \emph{La gracia de Dios apareció en este Niño}, trayendo salvación a todos los hombres. Esta gracia es ante todo Él mismo, el Hijo unigénito del Padre eterno, que en esta hora se hace hombre al nacer de una mujer. Su nacimiento en Belén constituye \emph{el primer momento de la gran revelación de Dios en Cristo}.
	
	Los pastores llegan al establo y encuentran allí \textquote{al Salvador del mundo, que es Cristo Señor} (cf. \emph{Lc} 2, 11). E incluso si sus ojos ven a un recién nacido envuelto en pañales y colocado en un pesebre, en ese \textquote{signo}, gracias a la luz interior de la fe, reconocen al Mesías anunciado por los Profetas. En él se manifiesta el amor de Dios por el hombre, por toda la humanidad. El que nació en la noche de Belén \emph{viene al mundo para entregarse} \textquote{\emph{a sí mismo por nosotros}, para rescatarnos de toda iniquidad y purificar para sí un pueblo de su propiedad, dedicado enteramente a las buenas obras} (\emph{Tit} 2, 14).
	
	3. \textquote{Gloria a Dios en el cielo, y en la tierra paz a los hombres de buena voluntad} (\emph{Lc} 2, 14).
	
	Este himno, que ha entrado firmemente en la tradición litúrgica de la Iglesia, resuena por primera vez en la noche de Belén y habla de un acercamiento singular y extraordinario entre Dios y el hombre. En realidad, \emph{Dios nunca se acercó tanto al hombre como aquella noche en que el unigénito Hijo del Padre se hizo hombre}. Y aunque su nacimiento tuvo lugar en condiciones modestas y pobres ---Jesús nació en la pobreza de un establo, como un vagabundo---, sin embargo estuvo lleno de gloria divina. La gloria, de hecho, no significa sólo esplendor externo; significa ante todo santidad.
	
	La hora del nacimiento del Hijo de Dios en el establo de Belén \emph{es la hora en que la santidad de Dios irrumpe en la historia del mundo.} \textquote{Noche santa}, como anuncia un conocido villancico. Noche que es, al mismo tiempo, el comienzo de la santificación del hombre por obra de Aquel, que es el único \textquote{Santo de Dios}. El himno angelical que acompaña a la Natividad del Señor anuncia precisamente esto.
	
	Al mismo tiempo, proclama la \emph{paz en la tierra}. Pensemos en primer lugar en la paz en el sentido histórico. Así, en la noche del nacimiento del Señor, se renueva en nosotros la esperanza de paz para todos los hombres y para todos los pueblos afectados por la guerra: {[}en los Balcanes, en África{]} y en todos los lugares donde falta la paz.
	
	Pero en la liturgia navideña la palabra \textquote{paz} también tiene otro significado más profundo. Se refiere a la \emph{nueva Alianza de Dios con los hombres}, a su renovación y cumplimiento definitivo. Si la Alianza de Dios con los hombres es una realidad que envuelve toda la historia de la salvación, no podría haber encontrado una expresión más completa que ésta: Dios acogió en sí a la humanidad, asumiéndola en la única Persona del Hijo. Así unió en sí lo divino y lo humano, como fundamento perenne y estable de la paz y de la alianza eterna. Por eso toda la Iglesia entona esta noche un cántico nuevo: \textquote{\emph{¡Gloria a ti, Dios hecho hombre, y paz a los hombres salvados por tu amor!}}.
\end{body}

\subsubsection{Homilía (1998): Ha nacido para nosotros}

\src{25 de diciembre de 1998.}

\begin{body}
	
	1. \textquote{\emph{No temáis, pues os anuncio una gran alegría\ldots{} os ha nacido hoy, en la ciudad de David, un salvador, que es el Cristo Señor}} (\emph{Lc} 2,10-11).
	
	\ltr{E}{n} esta Noche Santa la liturgia nos invita a celebrar con alegría el gran acontecimiento del nacimiento de Jesús en Belén. Como hemos escuchado en el \textbf{Evangelio de Lucas}, viene a la luz en una familia pobre de medios materiales, pero rica de alegría. Nace en un establo, porque para Él no hay lugar en la posada (cf. \emph{Lc} 2,7); es acostado en un pesebre, porque no tiene una cuna; llega al mundo en pleno abandono, ignorándolo todos y, al mismo tiempo, acogido y reconocido en primer lugar por los pastores, que reciben del ángel el anuncio de su nacimiento.
	
	Este acontecimiento esconde un misterio. Lo revelan los coros de los mensajeros celestiales que cantan el nacimiento de Jesús y proclaman \textquote{gloria a Dios en el cielo y en la tierra paz a los hombres que ama el Señor} (\emph{Lc} 2,14). La alabanza a lo largo de los siglos se hace oración que sube del corazón de las multitudes, que en la Noche Santa siguen acogiendo al Hijo de Dios.
	
	2. \emph{Mysterium}: acontecimiento y misterio. Nace un hombre, que es el Hijo eterno del Padre todopoderoso, Creador del cielo y de la tierra: en este acontecimiento extraordinario se revela el misterio de Dios. En la Palabra que se hace hombre se manifiesta el prodigio de Dios encarnado. El misterio ilumina el acontecimiento del nacimiento: un niño es adorado por los pastores en la gruta de Belén. Es \textquote{el Salvador del mundo}, es \textquote{Cristo Señor} (cf. \emph{Lc} 2,11). Sus ojos ven a un recién nacido envuelto en pañales y acostado en un pesebre, y en aquella \textquote{señal}, gracias a la luz interior de la fe, reconocen al Mesías anunciado por los Profetas.
	
	3. Es el Emmanuel, \textquote{Dios-con-nosotros}, que viene a llenar de gracia la tierra. Viene al mundo para transformar la creación. Se hace hombre entre los hombres, para que en Él y por medio de Él todo ser humano pueda renovarse profundamente. Con su nacimiento, nos introduce a todos en la dimensión de la divinidad, concediendo a quien acoge su don con fe la posibilidad de participar de su misma vida divina.
	
	Éste es el significado de la salvación de la que oyen hablar \textbf{los pastores} en la noche de Belén: \textquote{Os ha nacido un Salvador} (\emph{Lc} 2,11). La venida de Cristo entre nosotros es el centro de la historia, que desde entonces adquiere una nueva dimensión. En cierto modo, es Dios mismo que escribe la historia entrando en ella. El acontecimiento de la Encarnación se abre así para abrazar totalmente la historia humana, desde la creación a la parusía. Por esto en la liturgia \textbf{canta toda la creación} expresando su propia alegría: aplauden los ríos; vitorean los campos; se alegran las numerosas islas (cf. \emph{Sal} 98,8; 96,12; 97,1).
	
	Todo ser creado sobre la faz de la tierra acoge este anuncio. En el silencio atónito del universo, resuena con eco cósmico lo que la liturgia pone en boca de la Iglesia: \emph{Christus natus est nobis. Venite adoremus!}
	
	4. Cristo ha nacido para nosotros, ¡venid a adorarlo! Pienso ya en la Navidad del próximo año cuando, si Dios quiere, daré inicio al Gran Jubileo con la apertura de la Puerta Santa. Será un Año Santo verdaderamente grande, porque de manera muy singular se celebrará el bimilenario del acontecimiento-misterio de la Encarnación, con la cual la humanidad alcanzó el culmen de su vocación. Dios se hizo Hombre para hacer al ser humano partícipe de su propia divinidad.
	
	¡Éste es el anuncio de la salvación; éste es el mensaje de la Navidad! La Iglesia lo proclama también, en esta noche, mediante mis palabras, para que lo oigan los pueblos y las naciones de toda la tierra: \emph{Christus natus est nobis} --- Cristo ha nacido para nosotros. \emph{Venite, adoremus!} --- ¡Venid a adorarlo!
\end{body}

\subsubsection{Homilía (2001): Luz de la nueva creación}

\src{Basílica Vaticana. \\24 de diciembre del 2001.}

\begin{body}
	1. \emph{\textquote{Populus, quí ambulabat in tenebris, vidit lucem magnam -- El pueblo que caminaba en las tinieblas vio una luz grande}} (\emph{Is} 9, 1).
	
	\ltr{T}{odos} los años escuchamos estas palabras del \textbf{profeta Isaías}, en el contexto sugestivo de la conmemoración litúrgica del nacimiento de Cristo. Cada año \emph{adquieren un nuevo sabor} y hacen revivir el clima de expectación y de esperanza, de estupor y de gozo, que son típicos de la Navidad.
	
	Al pueblo oprimido y doliente, que caminaba en tinieblas, se le apareció \textquote{una gran luz}. Sí, una luz verdaderamente \textquote{grande}, porque la que irradia de la humildad del pesebre \emph{es la luz de la nueva creación}. Si la primera creación empezó con la luz (cf. \emph{Gn} 1, 3), mucho más resplandeciente y \textquote{grande} es la luz que da comienzo a la nueva creación: ¡es Dios mismo hecho hombre!
	
	La Navidad es acontecimiento de luz, \emph{es la fiesta de la luz}: en el Niño de Belén, la luz originaria vuelve a resplandecer en el cielo de la humanidad y despeja las nubes del pecado. El fulgor del triunfo definitivo de Dios aparece en el horizonte de la historia para proponer a los hombres un nuevo futuro de esperanza.
	
	2. \textquote{\emph{Habitaban tierras de sombras, y una luz les brilló}} (Is 9, 1).
	
	El \textbf{anuncio gozoso} que se acaba de proclamar en nuestra asamblea \emph{vale también para nosotros}, hombres y mujeres en el alba del tercer milenio. La comunidad de los creyentes se reúne en oración para escucharlo en todas las regiones del mundo. Tanto en el frío y la nieve del invierno como en el calor tórrido de los trópicos, \emph{esta noche es Noche Santa para todos}.
	
	Esperado por mucho tiempo, irrumpe por fin el resplandor del nuevo Día. ¡El Mesías ha nacido, el Enmanuel, Dios con nosotros! Ha nacido Aquel que fue preanunciado por los profetas e invocado constantemente por cuantos \textquote{habitaban en tierras de sombras}. En el silencio y la oscuridad de la noche, la luz se hace palabra y mensaje de esperanza.
	
	Pero, ¿no contrasta quizás esta certeza de fe \emph{con la realidad histórica en que vivimos}? Si escuchamos las tristes noticias de las crónicas, estas palabras de luz y esperanza parecen hablar de ensueños. Pero aquí reside precisamente el reto de la fe, que convierte este anuncio en consolador y, al mismo tiempo, exigente. La fe nos hace sentirnos rodeados por el tierno amor de Dios, a la vez que \emph{nos compromete en el amor efectivo a Dios y a los hermanos}.
	
	3. \emph{\textquote{Ha aparecido la gracia de Dios, que trae la salvación para todos los hombres}} (\emph{Tt} 2, 11).
	
	En esta Navidad, nuestros corazones están \emph{preocupados e inquietos} por la persistencia en muchas regiones del mundo de la guerra, de tensiones sociales y de la penuria en que se encuentran muchos seres humanos. Todo buscamos una respuesta que nos tranquilice.
	
	El texto de la \textbf{Carta a Tito} que acabamos de escuchar nos recuerda cómo el nacimiento del Hijo unigénito del Padre \emph{\textquote{trae la salvación}} a todos los rincones del planeta y a cada momento de la historia. Nace para todo hombre y mujer el Niño llamado \emph{\textquote{Maravilla de Consejero, Dios guerrero, Padre perpetuo, Príncipe de la paz}} (\emph{Is} 9, 5). Él tiene la respuesta que puede disipar nuestros miedos y dar nuevo vigor a nuestras esperanzas.
	
	Sí, en esta noche evocadora de recuerdos santos, se hace más firme nuestra confianza en el poder redentor de la \textbf{Palabra hecha carne}. Cuando parecen prevalecer las tinieblas y el mal, Cristo nos repite: ¡no temáis! \emph{Con su venida al mundo, Él ha derrotado el poder del mal}, nos ha liberado de la esclavitud de la muerte y nos ha readmitido al convite de la vida.
	
	Nos toca a nosotros recurrir a la fuerza de su amor victorioso, \emph{haciendo nuestra su lógica de servicio y humildad}. Cada uno de nosotros está llamado a vencer con Él \textquote{el misterio de la iniquidad}, haciéndose testigo de la solidaridad y constructor de la paz. Vayamos, pues, a la gruta de Belén para encontrarlo, pero también para encontrar, en Él, a todos los niños del mundo, a todo hermano lacerado en el cuerpo u oprimido en el espíritu.
	
	4. Los \textbf{pastores} \emph{\textquote{se volvieron dando gloria y alabanza a Dios por lo que habían visto y oído; todo como les habían dicho}} (\emph{Lc} 2, 17).
	
	Al igual que los pastores, también nosotros hemos de sentir en esta noche extraordinaria el deseo de comunicar a los demás la alegría del encuentro con este \emph{\textquote{Niño envuelto en pañales}}, en el cual se revela el poder salvador del Omnipotente. No podemos limitarnos a contemplar extasiados al Mesías que yace en el pesebre, olvidando el compromiso de \emph{ser sus testigos}.
	
	Hemos de volver de prisa a nuestro camino. Debemos volver gozosos de la gruta de Belén para contar por doquier el prodigio del que hemos sido testigos. ¡Hemos encontrado la luz y la vida! En Él se nos ha dado el amor.
	
	5. \emph{\textquote{Un Niño nos ha nacido\ldots{}}}
	
	Te acogemos con alegría, Omnipotente Dios del cielo y de la tierra, que por amor te has hecho Niño \emph{\textquote{en Judea, en la ciudad de David, que se llama Belén}} (cf. \emph{Lc} 2, 4).
	
	Te acogemos agradecidos, nueva Luz que surges en la noche del mundo.
	
	Te acogemos como a nuestro hermano, \textquote{\emph{Príncipe de la paz}}, que has hecho \textquote{\emph{de los dos pueblos una sola cosa}} (\emph{Ef} 2, 14).
	
	Cólmanos de tus dones, Tú que no has desdeñado comenzar la vida humana como nosotros. Haz que seamos hijos de Dios, Tú que por nosotros has querido hacerte hijo del hombre (cf. S. Agustín, \emph{Sermón} 184).
	
	Tú, \textquote{Maravilla de Consejero}, promesa segura de paz; Tú, presencia eficaz del \textquote{Dios poderoso}; Tú, nuestro único Dios, que yaces pobre y humilde en la sombra del pesebre, acógenos al lado de tu cuna.
	
	¡Venid, pueblos de la tierra y abridle las puertas de vuestra historia! Venid a adorar al Hijo de la Virgen María, que ha venido entre nosotros en esta noche preparada por siglos.
	
	Noche de alegría y de luz.
	
	\emph{¡Venite, adoremus!}
\end{body}


\subsubsection{Homilía (2004): ¡Quédate con nosotros!}

\src{Misa de Nochebuena. \\Viernes, 24 de diciembre de 2004.}

\begin{body}
	1. \textquote{\emph{Adoro Te devote, latens Deitas}}.
	
	\ltr{E}{n} esta Noche resuenan en mi corazón las primeras palabras del célebre himno eucarístico, que me acompaña día a día en este año dedicado particularmente a la Eucaristía.
	
	En el \emph{\textbf{Hijo de la Virgen}}, \textquote{envuelto en pañales} y \textquote{acostado en un pesebre} (cf. \emph{Lc} 2,12), reconocemos y adoramos \textquote{\emph{el pan bajado del cielo}} (\emph{Jn} 6,41.51), el Redentor venido a la tierra para dar la vida al mundo.
	
	2. \textbf{¡Belén!} La ciudad donde según las Escrituras nació Jesús, en lengua hebrea, significa \textquote{\emph{casa del pan}}. Allí, pues, debía nacer el Mesías, que más tarde diría de sí mismo: \textquote{Yo soy el pan de vida} (\emph{Jn} 6,35.48).
	
	En Belén nació Aquél que, bajo el signo del pan partido, dejaría el memorial de la Pascua. Por esto, la adoración del Niño Jesús, en esta Noche Santa, se convierte en \emph{adoración eucarística}.
	
	3. Te adoramos, Señor, presente realmente en el Sacramento del altar, Pan vivo que das vida al hombre. Te reconocemos como \emph{nuestro único Dios}, frágil Niño que estás indefenso en el pesebre. \textquote{En la plenitud de los tiempos, te hiciste hombre entre los hombres para unir el fin con el principio, es decir, al hombre con Dios} (cf. S. Ireneo, \emph{Adv. haer}., IV,20,4).
	
	Naciste en esta Noche, divino Redentor nuestro, y, por nosotros, peregrino por los \emph{senderos del tiempo}, te hiciste alimento \emph{de vida eterna}.
	
	¡Acuérdate de nosotros, Hijo eterno de Dios, que te encarnaste en el seno de la Virgen María! Te necesita la humanidad entera, marcada por tantas pruebas y dificultades.
	
	¡Quédate con nosotros, Pan vivo bajado del Cielo para nuestra salvación! ¡Quédate con nosotros para siempre! Amén.
\end{body}

\newsection			

\subsection{Benedicto XVI, papa}

\subsubsection{Homilía (2007)}

\src{Basílica Vaticana \\25 de diciembre del 2007.}

\begin{body}	
	\textquote{A María le llegó el tiempo del parto y dio a luz a su hijo primogénito, lo envolvió en pañales y lo acostó en un pesebre, porque no tenían sitio en la posada} (cf. \emph{Lc} 2,6s). 
	
	\ltr{E}{stas} frases, nos llegan al corazón siempre de nuevo. Llegó el momento anunciado por el Ángel en Nazaret: \textquote{Darás a luz un hijo, y le pondrás por nombre Jesús. Será grande, se llamará Hijo del Altísimo} (\emph{Lc} 1,31). Llegó el momento que Israel esperaba desde hacía muchos siglos, durante tantas horas oscuras, el momento en cierto modo esperado por toda la humanidad con figuras todavía confusas: que Dios se preocupase por nosotros, que saliera de su ocultamiento, que el mundo alcanzara la salvación y que Él renovase todo. Podemos imaginar con cuánta preparación interior, con cuánto amor, esperó María aquella hora. El breve inciso, \textquote{lo envolvió en pañales}, nos permite vislumbrar algo de la santa alegría y del callado celo de aquella preparación. Los pañales estaban dispuestos, para que el niño se encontrara bien atendido. Pero en la posada no había sitio. En cierto modo, la humanidad espera a Dios, su cercanía. Pero cuando llega el momento, no tiene sitio para Él. Está tan ocupada consigo misma de forma tan exigente, que necesita todo el espacio y todo el tiempo para sus cosas y ya no queda nada para el otro, para el prójimo, para el pobre, para Dios. Y cuanto más se enriquecen los hombres, tanto más llenan todo de sí mismos y menos puede entrar el otro.
	
	Juan, en su Evangelio, fijándose en lo esencial, ha profundizado en la breve referencia de san Lucas sobre la situación de Belén: \textquote{Vino a su casa, y los suyos no lo recibieron} (1,11). Esto se refiere sobre todo a Belén: el Hijo de David fue a su ciudad, pero tuvo que nacer en un establo, porque en la posada no había sitio para él. Se refiere también a Israel: el enviado vino a los suyos, pero no lo quisieron. En realidad, se refiere a toda la humanidad: Aquel por el que el mundo fue hecho, el Verbo creador primordial entra en el mundo, pero no se le escucha, no se le acoge.
	
	En definitiva, estas palabras se refieren a nosotros, a cada persona y a la sociedad en su conjunto. ¿Tenemos tiempo para el prójimo que tiene necesidad de nuestra palabra, de mi palabra, de mi afecto? ¿Para aquel que sufre y necesita ayuda? ¿Para el prófugo o el refugiado que busca asilo? ¿Tenemos tiempo y espacio para Dios? ¿Puede entrar Él en nuestra vida? ¿Encuentra un lugar en nosotros o tenemos ocupado todo nuestro pensamiento, nuestro quehacer, nuestra vida, con nosotros mismos?
	
	Gracias a Dios, la noticia negativa no es la única ni la última que hallamos en el Evangelio. De la misma manera que en \emph{Lucas} encontramos el amor de su madre María y la fidelidad de san José, la vigilancia de los pastores y su gran alegría, y en \emph{Mateo} encontramos la visita de los sabios Magos, llegados de lejos, así también nos dice \emph{Juan}: \textquote{Pero a cuantos lo recibieron, les da poder para ser hijos de Dios} (\emph{Jn} 1,12). Hay quienes lo acogen y, de este modo, desde fuera, crece silenciosamente, comenzando por el establo, la nueva casa, la nueva ciudad, el mundo nuevo. El mensaje de Navidad nos hace reconocer la oscuridad de un mundo cerrado y, con ello, se nos muestra sin duda una realidad que vemos cotidianamente. Pero nos dice también que Dios no se deja encerrar fuera. Él encuentra un espacio, entrando tal vez por el establo; hay hombres que ven su luz y la transmiten. Mediante la palabra del Evangelio, el Ángel nos habla también a nosotros y, en la sagrada liturgia, la luz del Redentor entra en nuestra vida. Si somos pastores o sabios, la luz y su mensaje nos llaman a ponernos en camino, a salir de la cerrazón de nuestros deseos e intereses para ir al encuentro del Señor y adorarlo. Lo adoramos abriendo el mundo a la verdad, al bien, a Cristo, al servicio de cuantos están marginados y en los cuales Él nos espera.
	
	En algunas representaciones navideñas de la Baja Edad media y de comienzo de la Edad Moderna, el pesebre se representa como edificio más bien desvencijado. Se puede reconocer todavía su pasado esplendor, pero ahora está deteriorado, sus muros en ruinas; se ha convertido justamente en un establo. Aunque no tiene un fundamento histórico, esta interpretación metafórica expresa sin embargo algo de la verdad que se esconde en el misterio de la Navidad. El trono de David, al que se había prometido una duración eterna, está vacío. Son otros los que dominan en Tierra Santa. José, el descendiente de David, es un simple artesano; de hecho, el palacio se ha convertido en una choza. David mismo había comenzado como pastor. Cuando Samuel lo buscó para ungirlo, parecía imposible y contradictorio que un joven pastor pudiera convertirse en el portador de la promesa de Israel. En el establo de Belén, precisamente donde estuvo el punto de partida, vuelve a comenzar la realeza davídica de un modo nuevo: en aquel niño envuelto en pañales y acostado en un pesebre. El nuevo trono desde el cual este David atraerá hacia sí el mundo es la Cruz. El nuevo trono ---la Cruz--- corresponde al nuevo inicio en el establo. Pero justamente así se construye el verdadero palacio davídico, la verdadera realeza. Así, pues, este nuevo palacio no es como los hombres se imaginan un palacio y el poder real. Este nuevo palacio es la comunidad de cuantos se dejan atraer por el amor de Cristo y con Él llegan a ser un solo cuerpo, una humanidad nueva. El poder que proviene de la Cruz, el poder de la bondad que se entrega, ésta es la verdadera realeza. El establo se transforma en palacio; precisamente a partir de este inicio, Jesús edifica la nueva gran comunidad, cuya palabra clave cantan los ángeles en el momento de su nacimiento: \textquote{Gloria a Dios en el cielo y en la tierra paz a los hombres que Dios ama}, hombres que ponen su voluntad en la suya, transformándose en hombres de Dios, hombres nuevos, mundo nuevo.
	
	Gregorio de Nisa ha desarrollado en sus homilías navideñas la misma temática partiendo del mensaje de Navidad en el \emph{Evangelio de Juan: \textquote{}Y puso su morada entre nosotros} (\emph{Jn} 1,14). Gregorio aplica esta palabra de la morada a nuestro cuerpo, deteriorado y débil; expuesto por todas partes al dolor y al sufrimiento. Y la aplica a todo el cosmos, herido y desfigurado por el pecado. ¿Qué habría dicho si hubiese visto las condiciones en las que hoy se encuentra la tierra a causa del abuso de las fuentes de energía y de su explotación egoísta y sin ningún reparo? Anselmo de Canterbury, casi de manera profética, describió con antelación lo que nosotros vemos hoy en un mundo contaminado y con un futuro incierto: \textquote{Todas las cosas se encontraban como muertas, al haber perdido su innata dignidad de servir al dominio y al uso de aquellos que alaban a Dios, para lo que habían sido creadas; se encontraban aplastadas por la opresión y como descoloridas por el abuso que de ellas hacían los servidores de los ídolos, para los que no habían sido creadas} (\emph{PL} 158, 955s). Así, según la visión de Gregorio, el establo del mensaje de Navidad representa la tierra maltratada. Cristo no reconstruye un palacio cualquiera. Él vino para volver a dar a la creación, al cosmos, su belleza y su dignidad: esto es lo que comienza con la Navidad y hace saltar de gozo a los ángeles. La tierra queda restablecida precisamente por el hecho de que se abre a Dios, que recibe nuevamente su verdadera luz y, en la sintonía entre voluntad humana y voluntad divina, en la unificación de lo alto con lo bajo, recupera su belleza, su dignidad. Así, pues, Navidad es la fiesta de la creación renovada. Los Padres interpretan el canto de los ángeles en la Noche santa a partir de este contexto: se trata de la expresión de la alegría porque lo alto y lo bajo, cielo y tierra, se encuentran nuevamente unidos; porque el hombre se ha unido nuevamente a Dios. Para los Padres, forma parte del canto navideño de los ángeles el que ahora ángeles y hombres canten juntos y, de este modo, la belleza del cosmos se exprese en la belleza del canto de alabanza. El canto litúrgico ---siempre según los Padres--- tiene una dignidad particular porque es un cantar junto con los coros celestiales. El encuentro con Jesucristo es lo que nos hace capaces de escuchar el canto de los ángeles, creando así la verdadera música, que acaba cuando perdemos este cantar juntos y este sentir juntos.
	
	En el establo de Belén el cielo y la tierra se tocan. El cielo vino a la tierra. Por eso, de allí se difunde una luz para todos los tiempos; por eso, de allí brota la alegría y nace el canto. Al final de nuestra meditación navideña quisiera citar una palabra extraordinaria de san Agustín. Interpretando la invocación de la oración del Señor: \textquote{Padre nuestro que estás en los cielos}, él se pregunta: ¿qué es esto del cielo? Y ¿dónde está el cielo? Sigue una respuesta sorprendente: Que estás en los cielos significa: en los santos y en los justos. \textquote{En verdad, Dios no se encierra en lugar alguno. Los cielos son ciertamente los cuerpos más excelentes del mundo, pero, no obstante, son cuerpos, y no pueden ellos existir sino en algún espacio; mas, si uno se imagina que el lugar de Dios está en los cielos, como en regiones superiores del mundo, podrá decirse que las aves son de mejor condición que nosotros, porque viven más próximas a Dios. Por otra parte, no está escrito que Dios está cerca de los hombres elevados, o sea de aquellos que habitan en los montes, sino que fue escrito en el Salmo: \textquote{El Señor está cerca de los que tienen el corazón atribulado} (\emph{Sal} 34 {[}33{]}, 19), y la tribulación propiamente pertenece a la humildad. Mas así como el pecador fue llamado \textquote{tierra}, así, por el contrario, el justo puede llamarse \textquote{cielo}} (\emph{Serm. in monte} II 5,17). El cielo no pertenece a la geografía del espacio, sino a la geografía del corazón. Y el corazón de Dios, en la Noche santa, ha descendido hasta un establo: la humildad de Dios es el cielo. Y si salimos al encuentro de esta humildad, entonces tocamos el cielo. Entonces, se renueva también la tierra. Con la humildad de los pastores, pongámonos en camino, en esta Noche santa, hacia el Niño en el establo. Toquemos la humildad de Dios, el corazón de Dios. Entonces su alegría nos alcanzará y hará más luminoso el mundo. Amén.
\end{body}

\subsubsection{Homilía (2010): Un niño que cumple la promesa}

\src{Basílica Vaticana. \\24 de diciembre de 2010.}

\begin{body}
	\textquote{Tú eres mi hijo, yo te he engendrado hoy}. 
	
	\ltr{L}{a} Iglesia comienza la liturgia del Noche Santa con estas palabras del \textbf{\emph{Salmo} segundo}. Ella sabe que estas palabras pertenecían originariamente al rito de la coronación de los reyes de Israel. El rey, que de por sí es un ser humano como los demás hombres, se convierte en \textquote{hijo de Dios} mediante la llamada y la toma de posesión de su cargo: es una especie de adopción por parte de Dios, un acto de decisión, por el que confiere a ese hombre una nueva existencia, lo atrae en su propio ser. La lectura tomada del \textbf{profeta Isaías}, que acabamos de escuchar, presenta de manera todavía más clara el mismo proceso en una situación de turbación y amenaza para Israel: \textquote{Un hijo se nos ha dado: lleva sobre sus hombros el principado} (9,5). La toma de posesión de la función de rey es como un nuevo nacimiento. Precisamente como recién nacido por decisión personal de Dios, como niño procedente de Dios, el rey constituye una esperanza. El futuro recae sobre sus hombros. Él es el portador de la promesa de paz. En la noche de Belén, esta palabra profética se ha hecho realidad de un modo que habría sido todavía inimaginable en tiempos de Isaías. Sí, ahora es realmente un niño el que lleva sobre sus hombros el poder. En Él aparece la nueva realeza que Dios establece en el mundo. Este niño ha nacido realmente de Dios. Es la Palabra eterna de Dios, que une la humanidad y la divinidad. Para este niño valen los títulos de dignidad que el cántico de coronación de Isaías le atribuye: Consejero admirable, Dios poderoso, Padre por siempre, Príncipe de la paz (9,5). Sí, este rey no necesita consejeros provenientes de los sabios del mundo. Él lleva en sí mismo la sabiduría y el consejo de Dios. Precisamente en la debilidad como niño Él es el Dios fuerte, y nos muestra así, frente a los poderes presuntuosos del mundo, la fortaleza propia de Dios.
	
	A decir verdad, las palabras del rito de coronación en Israel eran siempre sólo ritos de esperanza, que preveían a lo lejos un futuro que sería otorgado por Dios. Ninguno de los reyes saludados de este modo se correspondía con lo sublime de dichas palabras. En ellos, todas las palabras sobre la filiación de Dios, sobre su designación como heredero de las naciones, sobre el dominio de las tierras lejanas (\emph{Sal} 2,8), quedaron sólo como referencia a un futuro; casi como carteles que señalan la esperanza, indicaciones que guían hacia un futuro, que en aquel entonces era todavía inconcebible. Por eso, el cumplimiento de la palabra que da comienzo en la noche de Belén es a la vez inmensamente más grande y ---desde el punto de vista del mundo--- más humilde que lo que la palabra profética permitía intuir. Es más grande, porque este niño es realmente Hijo de Dios, verdaderamente \textquote{Dios de Dios, Luz de Luz, engendrado, no creado, de la misma naturaleza del Padre}. Ha quedado superada la distancia infinita entre Dios y el hombre. Dios no solamente se ha inclinado hacia abajo, como dicen los Salmos; Él ha \textquote{descendido} realmente, ha entrado en el mundo, haciéndose uno de nosotros para atraernos a todos a sí. Este niño es verdaderamente el Emmanuel, el Dios-con-nosotros. Su reino se extiende realmente hasta los confines de la tierra. En la magnitud universal de la santa Eucaristía, Él ha hecho surgir realmente islas de paz. En cualquier lugar que se celebra hay una isla de paz, de esa paz que es propia de Dios. Este niño ha encendido en los hombres la luz de la bondad y les ha dado la fuerza de resistir a la tiranía del poder. Él construye su reino desde dentro, partiendo del corazón, en cada generación. Pero también es cierto que no se ha roto la \textquote{vara del opresor}. También hoy siguen marchando con estruendo las botas de los soldados y todavía hoy, una y otra vez, queda la \textquote{túnica empapada de sangre} (\emph{Is} 9,3s). Así, forma parte de esta noche la alegría por la cercanía de Dios. Damos gracias porque el Dios niño se pone en nuestras manos, mendiga, por decirlo así, nuestro amor, infunde su paz en nuestro corazón. Esta alegría, sin embargo, es también una oración: Señor, cumple por entero tu promesa. Quiebra las varas de los opresores. Quema las botas resonantes. Haz que termine el tiempo de las túnicas ensangrentadas. Cumple la promesa: \textquote{La paz no tendrá fin} (\emph{Is} 9,6). Te damos gracias por tu bondad, pero también te pedimos: Muestra tu poder. Erige en el mundo el dominio de tu verdad, de tu amor; el \textquote{reino de justicia, de amor y de paz}.
	
	\textquote{María dio a la luz a su hijo primogénito} (\emph{Lc} 2,7). \textbf{San Lucas} describe con esta frase, sin énfasis alguno, el gran acontecimiento que habían vislumbrado con antelación las palabras proféticas en la historia de Israel. Designa al niño como \textquote{primogénito}. En el lenguaje que se había ido formando en la Sagrada Escritura de la Antigua Alianza, \textquote{primogénito} no significa el primero de otros hijos. \textquote{Primogénito} es un título de honor, independientemente de que después sigan o no otros hermanos y hermanas. Así, en el Libro del \emph{Éxodo} (\emph{Ex} 4,22), Dios llama a Israel \textquote{mi hijo primogénito}, expresando de este modo su elección, su dignidad única, el amor particular de Dios Padre. La Iglesia naciente sabía que esta palabra había recibido una nueva profundidad en Jesús; que en Él se resumen las promesas hechas a Israel. Así, la \emph{Carta a los Hebreos} llama a Jesús simplemente \textquote{el primogénito}, para identificarlo como el Hijo que Dios envía al mundo después de los preparativos en el Antiguo Testamento (cf. \emph{Hb} 1,5-7). El primogénito pertenece de modo particular a Dios, y por eso ---como en muchas religiones--- debía ser entregado de manera especial a Dios y ser rescatado mediante un sacrificio sustitutivo, como relata san Lucas en el episodio de la presentación de Jesús en templo. El primogénito pertenece a Dios de modo particular; está destinado al sacrificio, por decirlo así. El destino del primogénito se cumple de modo único en el sacrificio de Jesús en la cruz. Él ofrece en sí mismo la humanidad a Dios, y une al hombre y a Dios de tal modo que Dios sea todo en todos. San Pablo ha ampliado y profundizado la idea de Jesús como primogénito en las \emph{Cartas a los Colosenses} y \emph{a los Efesios}: Jesús, nos dicen estas Cartas, es el Primogénito de la creación: el verdadero arquetipo del hombre, según el cual Dios ha formado la criatura hombre. El hombre puede ser imagen de Dios, porque Jesús es Dios y Hombre, la verdadera imagen de Dios y el Hombre. Él es el primogénito de los muertos, nos dicen además estas Cartas. En la Resurrección, Él ha desfondado el muro de la muerte para todos nosotros. Ha abierto al hombre la dimensión de la vida eterna en la comunión con Dios. Finalmente, se nos dice: Él es el primogénito de muchos hermanos. Sí, con todo, Él es ahora el primero de más hermanos, es decir, el primero que inaugura para nosotros el estar en comunión con Dios. Crea la verdadera hermandad: no la hermandad deteriorada por el pecado, la de Caín y Abel, de Rómulo y Remo, sino la hermandad nueva en la que somos de la misma familia de Dios. Esta nueva familia de Dios comienza en el momento en el que María envuelve en pañales al \textquote{primogénito} y lo acuesta en el pesebre. Pidámosle: Señor Jesús, tú que has querido nacer como el primero de muchos hermanos, danos la verdadera hermandad. Ayúdanos para que nos parezcamos a ti. Ayúdanos a reconocer tu rostro en el otro que me necesita, en los que sufren o están desamparados, en todos los hombres, y a vivir junto a ti como hermanos y hermanas, para convertirnos en una familia, tu familia.
	
	El \textbf{Evangelio de Navidad} nos relata al final que una multitud de ángeles del ejército celestial alababa a Dios diciendo: \textquote{Gloria a Dios en el cielo, y en la tierra paz a los hombres que Dios ama} (\emph{Lc} 2,14). La Iglesia ha amplificado en el \emph{Gloria} esta alabanza, que los ángeles entonaron ante el acontecimiento de la Noche Santa, haciéndola un himno de alegría sobre la gloria de Dios. \textquote{Por tu gloria inmensa, te damos gracias}. Te damos gracias por la belleza, por la grandeza, por tu bondad, que en esta noche se nos manifiestan. La aparición de la belleza, de lo hermoso, nos hace alegres sin tener que preguntarnos por su utilidad. La gloria de Dios, de la que proviene toda belleza, hace saltar en nosotros el asombro y la alegría. Quien vislumbra a Dios siente alegría, y en esta noche vemos algo de su luz. Pero el mensaje de los ángeles en la Noche Santa habla también de los hombres: \textquote{Paz a los hombres que Dios ama}. La traducción latina de estas palabras, que usamos en la liturgia y que se remonta a Jerónimo, suena de otra manera: \textquote{Paz a los hombres de buena voluntad}. La expresión \textquote{hombres de buena voluntad} ha entrado en el vocabulario de la Iglesia de un modo particular precisamente en los últimos decenios. Pero, ¿cuál es la traducción correcta? Debemos leer ambos textos juntos; sólo así entenderemos la palabra de los ángeles del modo justo. Sería equivocada una interpretación que reconociera solamente el obrar exclusivo de Dios, como si Él no hubiera llamado al hombre a una libre respuesta de amor. Pero sería también errónea una interpretación moralizadora, según la cual, por decirlo así, el hombre podría con su buena voluntad redimirse a sí mismo. Ambas cosas van juntas: gracia y libertad; el amor de Dios, que nos precede, y sin el cual no podríamos amarlo, y nuestra respuesta, que Él espera y que incluso nos ruega en el nacimiento de su Hijo. El entramado de gracia y libertad, de llamada y respuesta, no lo podemos dividir en partes separadas una de otra. Las dos están indisolublemente entretejidas entre sí. Así, esta palabra es promesa y llamada a la vez. Dios nos ha precedido con el don de su Hijo. Una y otra vez, nos precede de manera inesperada. No deja de buscarnos, de levantarnos cada vez que lo necesitamos. No abandona a la oveja extraviada en el desierto en que se ha perdido. Dios no se deja confundir por nuestro pecado. Él siempre vuelve a comenzar con nosotros. No obstante, espera que amemos con Él. Él nos ama para que nosotros podamos convertirnos en personas que aman junto con Él y así haya paz en la tierra.
	
	\textbf{Lucas} no dice que \textbf{los ángeles} cantaran. Él escribe muy sobriamente: el ejército celestial alababa a Dios diciendo: \textquote{Gloria a Dios en el cielo\ldots{}} (\emph{Lc} 2,13s). Pero los hombres siempre han sabido que el hablar de los ángeles es diferente al de los hombres; que precisamente esta noche del mensaje gozoso ha sido un canto en el que ha brillado la gloria sublime de Dios. Por eso, este canto de los ángeles ha sido percibido desde el principio como música que viene de Dios, más aún, como invitación a unirse al canto, a la alegría del corazón por ser amados por Dios. \emph{Cantare amantis est}, dice san Agustín: cantar es propio de quien ama. Así, a lo largo de los siglos, el canto de los ángeles se ha convertido siempre en un nuevo canto de amor y alegría, un canto de los que aman. En esta hora, nosotros nos asociamos llenos de gratitud a este cantar de todos los siglos, que une cielo y tierra, ángeles y hombres. Sí, te damos gracias por tu gloria inmensa. Te damos gracias por tu amor. Haz que seamos cada vez más personas que aman contigo y, por tanto, personas de paz. Amén.
\end{body}

\newsection			

\subsection{Francisco, papa}

\subsubsection{Homilía (2013): Caminar y ver}

\src{Basílica Vaticana. \\Martes 24 de diciembre del 2013.}

\begin{body}
	1. \textquote{\emph{El pueblo que caminaba en tinieblas vio una luz grande}} (\emph{Is} 9,1).
	
	\ltr{E}{sta} \textbf{profecía de Isaías} no deja de conmovernos, especialmente cuando la escuchamos en la Liturgia de la Noche de Navidad. No se trata sólo de algo emotivo, sentimental; nos conmueve porque dice la realidad de lo que somos: somos un pueblo en camino, y a nuestro alrededor ---y también dentro de nosotros--- hay tinieblas y luces. Y en esta noche, cuando el espíritu de las tinieblas cubre el mundo, se renueva el acontecimiento que siempre nos asombra y sorprende: el pueblo en camino ve una gran luz. Una luz que nos invita a reflexionar en este misterio: misterio de \emph{caminar} y de \emph{ver}.
	
	Caminar. Este verbo nos hace pensar en el curso de la historia, en el largo camino de la historia de la salvación, comenzando por Abrahán, nuestro padre en la fe, a quien el Señor llamó un día a salir de su pueblo para ir a la tierra que Él le indicaría. Desde entonces, nuestra identidad como creyentes es la de peregrinos hacia la tierra prometida. El Señor acompaña siempre esta historia. Él permanece siempre fiel a su alianza y a sus promesas. Porque es fiel, \textquote{Dios es luz sin tiniebla alguna} (\emph{1 Jn} 1,5). Por parte del pueblo, en cambio, se alternan momentos de luz y de tiniebla, de fidelidad y de infidelidad, de obediencia y de rebelión, momentos de pueblo peregrino y momentos de pueblo errante.
	
	También en nuestra historia personal se alternan momentos luminosos y oscuros, luces y sombras. Si amamos a Dios y a los hermanos, caminamos en la luz, pero si nuestro corazón se cierra, si prevalecen el orgullo, la mentira, la búsqueda del propio interés, entonces las tinieblas nos rodean por dentro y por fuera. \textquote{Quien aborrece a su hermano ---escribe el apóstol San Juan--- está en las tinieblas, camina en las tinieblas, no sabe adónde va, porque las tinieblas han cegado sus ojos} (\emph{1 Jn} 2,11). Pueblo en camino, sobre todo pueblo peregrino que no quiere ser un pueblo errante.
	
	2. En esta noche, como un haz de luz clarísima, resuena \textbf{el anuncio del Apóstol}: \textquote{\emph{Ha aparecido la gracia de Dios, que trae la salvación para todos los hombres}} (\emph{Tt} 2,11).
	
	La \textbf{gracia} que ha aparecido en el mundo es Jesús, nacido de María Virgen, Dios y hombre verdadero. Ha venido a nuestra historia, ha compartido nuestro camino. Ha venido para librarnos de las tinieblas y darnos la luz. En Él ha aparecido la gracia, la misericordia, la ternura del Padre: Jesús es el Amor hecho carne. No es solamente un maestro de sabiduría, no es un ideal al que tendemos y del que nos sabemos por fuerza distantes, es el sentido de la vida y de la historia que ha puesto su tienda entre nosotros.
	
	3. Los \textbf{pastores} fueron los primeros que vieron esta \textquote{tienda}, que recibieron el anuncio del nacimiento de Jesús. Fueron los primeros porque eran de los últimos, de los marginados. Y fueron los primeros porque estaban en vela aquella noche, guardando su rebaño. Es condición del peregrino velar, y ellos estaban en vela. Con ellos nos quedamos ante el Niño, nos quedamos en silencio. Con ellos damos gracias al Señor por habernos dado a Jesús, y con ellos, desde dentro de nuestro corazón, alabamos su fidelidad: Te bendecimos, Señor, Dios Altísimo, que te has despojado de tu rango por nosotros. Tú eres inmenso, y te has hecho pequeño; eres rico, y te has hecho pobre; eres omnipotente, y te has hecho débil.
	
	Que en esta Noche compartamos \emph{\textbf{la alegría del Evangelio}}: Dios nos ama, nos ama tanto que nos ha dado a su Hijo como nuestro hermano, como luz para nuestras tinieblas. El Señor nos dice una vez más: \textquote{No teman} (\emph{Lc} 2,10). Como dijeron los ángeles a los pastores: \textquote{No teman}. Y también yo les repito a todos: \textquote{No teman}. Nuestro Padre tiene paciencia con nosotros, nos ama, nos da a Jesús como guía en el camino a la tierra prometida. Él es la luz que disipa las tinieblas. Él es la misericordia. Nuestro Padre nos perdona siempre. Y Él es nuestra paz. Amén.
\end{body}

\subsubsection{Mensaje Urbi et Orbi (2013)} 

\src{Miércoles 25 de diciembre del 2013.}

\begin{body}
	\emph{\textquote{Gloria a Dios en el cielo,\\ y en la tierra paz a los hombres que Dios ama}} (\emph{Lc} 2,14).
	
	Queridos hermanos y hermanas de Roma y del mundo entero, ¡buenos días y feliz Navidad!
	
	\ltr{H}{ago} mías las palabras del cántico de los ángeles, que se aparecieron a los pastores de Belén la noche de la Navidad. Un cántico que une cielo y tierra, elevando al cielo la alabanza y la gloria y saludando a la tierra de los hombres con el deseo de la paz.
	
	Les invito a todos a hacer suyo este cántico, que es el de cada hombre y mujer que vigila en la noche, que espera un mundo mejor, que se preocupa de los otros, intentado hacer humildemente su propio deber.
	
	\emph{Gloria a Dios}.
	
	A esto nos invita la Navidad en primer lugar: a dar gloria a Dios, porque es bueno, fiel, misericordioso. En este día mi deseo es que todos puedan conocer el verdadero rostro de Dios, el Padre que nos ha dado a Jesús. Me gustaría que todos pudieran sentir a Dios cerca, sentirse en su presencia, que lo amen, que lo adoren.
	
	Y que todos nosotros demos gloria a Dios, sobre todo, con la vida, con una vida entregada por amor a Él y a los hermanos.
	
	\emph{Paz a los hombres}.
	
	La verdadera paz --- como sabemos --- no es un equilibrio de fuerzas opuestas. No es pura \textquote{fachada}, que esconde luchas y divisiones. La paz es un compromiso cotidiano, y la paz es también artesanal, que se logra contando con el don de Dios, con la gracia que nos ha dado en Jesucristo.
	
	Viendo al Niño en el Belén, niño de paz, pensemos en los niños que son las víctimas más vulnerables de las guerras, pero pensemos también en los ancianos, en las mujeres maltratadas, en los enfermos\ldots{} ¡Las guerras destrozan tantas vidas y causan tanto sufrimiento!
	
	Demasiadas ha destrozado en los últimos tiempos el conflicto de Siria, generando odios y venganzas. Sigamos rezando al Señor para que el amado pueblo sirio se vea libre de más sufrimientos y las partes en conflicto pongan fin a la violencia y garanticen el acceso a la ayuda humanitaria. Hemos podido comprobar la fuerza de la oración. Y me alegra que hoy se unan a nuestra oración por la paz en Siria creyentes de diversas confesiones religiosas. No perdamos nunca la fuerza de la oración. La fuerza para decir a Dios: Señor, concede tu paz a Siria y al mundo entero. E invito también a los no creyentes a desear la paz, con su deseo, ese deseo que ensancha el corazón: todos unidos, con la oración o con el deseo. Pero todos, por la paz.
	
	Concede la paz, Niño, a la República Centroafricana, a menudo olvidada por los hombres. Pero tú, Señor, no te olvidas de nadie. Y quieres que reine la paz también en aquella tierra, atormentada por una espiral de violencia y de miseria, donde muchas personas carecen de techo, agua y alimento, sin lo mínimo indispensable para vivir. Que se afiance la concordia en Sudán del Sur, donde las tensiones actuales ya han provocado demasiadas víctimas y amenazan la pacífica convivencia de este joven Estado.
	
	Tú, Príncipe de la paz, convierte el corazón de los violentos, allá donde se encuentren, para que depongan las armas y emprendan el camino del diálogo. Vela por Nigeria, lacerada por continuas violencias que no respetan ni a los inocentes e indefensos. Bendice la tierra que elegiste para venir al mundo y haz que lleguen a feliz término las negociaciones de paz entre israelíes y palestinos. Sana las llagas de la querida tierra de Iraq, azotada todavía por frecuentes atentados.
	
	Tú, Señor de la vida, protege a cuantos sufren persecución a causa de tu nombre. Alienta y conforta a los desplazados y refugiados, especialmente en el Cuerno de África y en el este de la República Democrática del Congo. Haz que los emigrantes, que buscan una vida digna, encuentren acogida y ayuda. Que no asistamos de nuevo a tragedias como las que hemos visto este año, con los numerosos muertos en Lampedusa.
	
	Niño de Belén, toca el corazón de cuantos están involucrados en la trata de seres humanos, para que se den cuenta de la gravedad de este delito contra la humanidad. Dirige tu mirada sobre los niños secuestrados, heridos y asesinados en los conflictos armados, y sobre los que se ven obligados a convertirse en soldados, robándoles su infancia.
	
	Señor, del cielo y de la tierra, mira a nuestro planeta, que a menudo la codicia y el egoísmo de los hombres explota indiscriminadamente. Asiste y protege a cuantos son víctimas de los desastres naturales, sobre todo al querido pueblo filipino, gravemente afectado por el reciente tifón.
	
	Queridos hermanos y hermanas, en este mundo, en esta humanidad hoy ha nacido el Salvador, Cristo el Señor. No pasemos de largo ante el Niño de Belén. Dejemos que nuestro corazón se conmueva: no tengamos miedo de esto. No tengamos miedo de que nuestro corazón se conmueva. Tenemos necesidad de que nuestro corazón se conmueva. Dejémoslo que se inflame con la ternura de Dios; necesitamos sus caricias. Las caricias de Dios no producen heridas: las caricias de Dios nos dan paz y fuerza. Tenemos necesidad de sus caricias. El amor de Dios es grande; a Él la gloria por los siglos. Dios es nuestra paz: pidámosle que nos ayude a construirla cada día, en nuestra vida, en nuestras familias, en nuestras ciudades y naciones, en el mundo entero. Dejémonos conmover por la bondad de Dios.
\end{body}

\subsubsection{Homilía (2016): ¿Dónde se manifiesta Dios?}

\src{24 de diciembre del 2016.}

\begin{body}
	\textquote{Ha aparecido la gracia de Dios, que trae la salvación para todos los hombres} (\emph{Tt} 2,11).
	
	\ltr{L}{as} palabras del \textbf{apóstol Pablo} manifiestan el misterio de esta noche santa: ha aparecido la gracia de Dios, su regalo gratuito; en el Niño que se nos ha dado se hace concreto el amor de Dios para con nosotros.
	
	Es una \emph{noche de gloria}, esa gloria proclamada por los ángeles en Belén y también por nosotros en todo el mundo. Es una \emph{noche de alegría}, porque desde hoy y para siempre Dios, el Eterno, el Infinito, es \emph{Dios con nosotros}: no está lejos, no debemos buscarlo en las órbitas celestes o en una idea mística; es cercano, se ha hecho hombre y no se cansará jamás de nuestra humanidad, que ha hecho suya. Es una \emph{noche de luz}: esa luz que, según la \textbf{profecía de Isaías} (cf. 9,1), iluminará a quien camina en tierras de tiniebla, ha aparecido y ha envuelto a \textbf{los pastores} de Belén (cf. \emph{Lc} 2,9).
	
	\textbf{Los pastores} descubren sencillamente que \textquote{un niño nos ha nacido} (\emph{Is} 9,5) y comprenden que toda esta gloria, toda esta alegría, toda esta luz se concentra en un único punto, en ese \emph{signo} que el ángel les ha indicado: \textquote{Encontraréis un niño envuelto en pañales y acostado en un pesebre} (\emph{Lc} 2,12). Este es \emph{el signo de siempre} para encontrar a Jesús. No sólo entonces, sino también hoy. Si queremos celebrar la verdadera Navidad, contemplemos este signo: la sencillez frágil de un niño recién nacido, la dulzura al verlo recostado, la ternura de los pañales que lo cubren. Allí está Dios.
	
	Y con este signo, el \textbf{Evangelio} nos revela una paradoja: habla del emperador, del gobernador, de los grandes de aquel tiempo, pero Dios no se hace presente allí; no aparece en la sala noble de un palacio real, sino en la pobreza de un establo; no en los fastos de la apariencia, sino en la sencillez de la vida; no en el poder, sino en una pequeñez que sorprende. Y para encontrarlo hay que ir allí, donde él está: es necesario reclinarse, abajarse, hacerse pequeño. El Niño que nace nos interpela: nos llama a dejar los engaños de lo efímero para ir a lo esencial, a renunciar a nuestras pretensiones insaciables, a abandonar las insatisfacciones permanentes y la tristeza ante cualquier cosa que siempre nos faltará. Nos hará bien dejar estas cosas para encontrar de nuevo en la sencillez del Niño Dios la paz, la alegría, el sentido luminoso de la vida.
	
	Dejémonos interpelar por el Niño en el pesebre, pero dejémonos interpelar también por los niños que, hoy, no están recostados en una cuna ni acariciados por el afecto de una madre ni de un padre, sino que yacen en los escuálidos \textquote{\emph{pesebres donde se devora su dignidad}}: en el refugio subterráneo para escapar de los bombardeos, sobre las aceras de una gran ciudad, en el fondo de una barcaza repleta de emigrantes. Dejémonos interpelar por los niños a los que no se les deja nacer, por los que lloran porque nadie les sacia su hambre, por los que no tienen en sus manos juguetes, sino armas.
	
	El misterio de la Navidad, que es luz y alegría, interpela y golpea, porque es al mismo tiempo un \emph{misterio de esperanza y de tristeza}. Lleva consigo un \emph{sabor de tristeza}, porque el amor no ha sido acogido, la vida es descartada. Así sucedió a \textbf{José y a María}, que encontraron las puertas cerradas y pusieron a Jesús en un pesebre, \textquote{porque no tenían {[}para ellos{]} sitio en la posada} (v. 7): Jesús nace rechazado por algunos y en la indiferencia de la mayoría. También hoy puede darse la misma indiferencia, cuando Navidad es una fiesta donde los protagonistas somos nosotros en vez de él; cuando las luces del comercio arrinconan en la sombra la luz de Dios; cuando nos afanamos por los regalos y permanecemos insensibles ante quien está marginado. ¡Esta mundanidad nos ha secuestrado la Navidad, es necesario liberarla!
	
	Pero la Navidad tiene sobre todo un \emph{sabor de esperanza} porque, a pesar de nuestras tinieblas, \textbf{la luz de Dios resplandece}. Su luz suave no da miedo; Dios, enamorado de nosotros, nos atrae con su ternura, naciendo pobre y frágil en medio de nosotros, como uno más. Nace en Belén, que significa \textquote{\emph{casa del pan}}. Parece que nos quiere decir que nace como \emph{pan para nosotros} ; viene a la vida para darnos su vida; viene a nuestro mundo para traernos su amor. No viene a devorar y a mandar, sino a nutrir y servir. De este modo hay una línea directa que une el pesebre y la cruz, donde Jesús será \emph{pan partido}: es la línea directa del amor que se da y nos salva, que da luz a nuestra vida, paz a nuestros corazones.
	
	Lo entendieron, en esa noche, \textbf{los pastores}, que estaban entre los marginados de entonces. Pero ninguno está marginado a los ojos de Dios y fueron justamente ellos los invitados a la Navidad. Quien estaba seguro de sí mismo, autosuficiente se quedó en casa entre sus cosas; los pastores en cambio \textquote{fueron corriendo de prisa} (cf. \emph{Lc} 2,16). También nosotros dejémonos interpelar y convocar en esta noche por Jesús, vayamos a él con confianza, desde aquello en lo que nos sentimos marginados, desde nuestros límites, desde nuestros pecados. Dejémonos tocar por la ternura que salva. Acerquémonos a Dios que se hace cercano, detengámonos a mirar el belén, imaginemos el nacimiento de Jesús: la luz y la paz, la pobreza absoluta y el rechazo. Entremos en la verdadera Navidad con los pastores, llevemos a Jesús lo que somos, nuestras marginaciones, nuestras heridas no curadas, nuestros pecados. Así, en Jesús, saborearemos el verdadero espíritu de Navidad: la belleza de ser amados por Dios. Con \textbf{María y José} quedémonos ante el pesebre, ante Jesús que nace como pan para mi vida. Contemplando su amor humilde e infinito, digámosle sencillamente gracias: gracias, porque has hecho todo esto \emph{por mí}.
\end{body}

\subsubsection{Mensaje Urbi et Orbi (2016)}

\begin{body}
	\emph{Queridos hermanos y hermanas, feliz Navidad}.
	
	\ltr{H}{oy} la Iglesia revive el asombro de la Virgen María, de san José y de los pastores de Belén, contemplando al Niño que ha nacido y que está acostado en el pesebre: Jesús, el Salvador.
	
	En este día lleno de luz, resuena el anuncio del Profeta: \textquote{Un niño nos ha nacido,\\ un hijo se nos ha dado:\\ lleva a hombros el principado, y es su nombre:\\ Maravilla del Consejero,\\ Dios guerrero,\\ Padre perpetuo,\\ Príncipe de la paz} (\emph{Is} 9, 5).
	
	El poder de un Niño, Hijo de Dios y de María, no es el poder de este mundo, basado en la fuerza y en la riqueza, es el poder del amor. Es el poder que creó el cielo y la tierra, que da vida a cada criatura: a los minerales, a las plantas, a los animales; es la fuerza que atrae al hombre y a la mujer, y hace de ellos una sola carne, una sola existencia; es el poder que regenera la vida, que perdona las culpas, reconcilia a los enemigos, transforma el mal en bien. Es el poder de Dios. Este poder del amor ha llevado a Jesucristo a despojarse de su gloria y a hacerse hombre; y lo conducirá a dar la vida en la cruz y a resucitar de entre los muertos. Es el poder del servicio, que instaura en el mundo el reino de Dios, reino de justicia y de paz.
	
	Por esto el nacimiento de Jesús está acompañado por el canto de los ángeles que anuncian: \textquote{Gloria a Dios en el cielo,\\ y en la tierra paz a los hombres que Dios ama} (\emph{Lc} 2,14).
	
	Hoy este anuncio recorre toda la tierra y quiere llegar a todos los pueblos, especialmente los golpeados por la guerra y por conflictos violentos, y que sienten fuertemente el deseo de la paz.
	
	Paz a los hombres y a las mujeres de la martirizada Siria, donde demasiada sangre ha sido derramada. Sobre todo en la ciudad de Alepo, escenario, en las últimas semanas, de una de las batallas más atroces, es muy urgente que, respetando el derecho humanitario, se garanticen asistencia y consolación a la extenuada población civil, que se encuentra todavía en una situación desesperada y de gran sufrimiento y miseria. Es hora de que las armas callen definitivamente y la comunidad internacional se comprometa activamente para que se logre una solución negociable y se restablezca la convivencia civil en el País.
	
	Paz para las mujeres y para los hombres de la amada Tierra Santa, elegida y predilecta por Dios. Que los israelíes y los palestinos tengan la valentía y la determinación de escribir una nueva página de la historia, en la que el odio y la venganza cedan el lugar a la voluntad de construir conjuntamente un futuro de recíproca comprensión y armonía. Que puedan recobrar unidad y concordia Irak, Libia, Yemen, donde las poblaciones sufren la guerra y brutales acciones terroristas.
	
	Paz a los hombres y mujeres en las diferentes regiones de África, particularmente en Nigeria, donde el terrorismo fundamentalista explota también a los niños para perpetrar el horror y la muerte. Paz en Sudán del Sur y en la República Democrática del Congo, para que se curen las divisiones y para que todos las personas de buena voluntad se esfuercen para iniciar nuevos caminos de desarrollo y de compartir, prefiriendo la cultura del diálogo a la lógica del enfrentamiento.
	
	Paz a las mujeres y hombres que todavía padecen las consecuencias del conflicto en Ucrania oriental, donde es urgente una voluntad común para llevar alivio a la población y poner en práctica los compromisos asumidos.
	
	Pedimos concordia para el querido pueblo colombiano, que desea cumplir un nuevo y valiente camino de diálogo y de reconciliación. Dicha valentía anime también la amada Venezuela para dar los pasos necesarios con vistas a poner fin a las tensiones actuales y a edificar conjuntamente un futuro de esperanza para la población entera.
	
	Paz a todos los que, en varias zonas, están afrontando sufrimiento a causa de peligros constantes e injusticias persistentes. Que Myanmar pueda consolidar los esfuerzos para favorecer la convivencia pacífica y, con la ayuda de la comunidad internacional, pueda dar la necesaria protección y asistencia humanitaria a los que tienen necesidad extrema y urgente. Que pueda la península coreana ver superadas las tensiones que la atraviesan en un renovado espíritu de colaboración.
	
	Paz a quien ha sido herido o ha perdido a un ser querido debido a viles actos de terrorismo que han sembrado miedo y muerte en el corazón de tantos países y ciudades. Paz ---no de palabra, sino eficaz y concreta--- a nuestros hermanos y hermanas que están abandonados y excluidos, a los que sufren hambre y los que son víctimas de violencia. Paz a los prófugos, a los emigrantes y refugiados, a los que hoy son objeto de la trata de personas. Paz a los pueblos que sufren por las ambiciones económicas de unos pocos y la avaricia voraz del dios dinero que lleva a la esclavitud. Paz a los que están marcados por el malestar social y económico, y a los que sufren las consecuencias de los terremotos u otras catástrofes naturales.
	
	Y paz a los niños, en este día especial en el que Dios se hace niño, sobre todo a los privados de la alegría de la infancia a causa del hambre, de las guerras y del egoísmo de los adultos.
	
	Paz sobre la tierra a todos los hombres de buena voluntad, que cada día trabajan, con discreción y paciencia, en la familia y en la sociedad para construir un mundo más humano y más justo, sostenidos por la convicción de que sólo con la paz es posible un futuro más próspero para todos.
	
	Queridos hermanos y hermanas:
	
	\textquote{Un niño nos ha nacido, un hijo se nos ha dado}: es el \textquote{Príncipe de la paz}. Acojámoslo.
\end{body}

\subsubsection{Homilía (2019): Acoger la gracia de Dios}

\src{24 de diciembre del 2019.}

\begin{body}
	\textquote{El pueblo que caminaba en tinieblas vio una luz grande} (\emph{Is} 9,1). 
	
	\ltr{E}{sta} profecía de la \textbf{primera lectura} se realizó en el Evangelio. De hecho, mientras los pastores velaban de noche en sus campos, \textquote{la gloria del Señor los envolvió de claridad} (\emph{Lc} 2,9). En la noche de la tierra apareció una luz del cielo. ¿Qué significa esta luz surgida en la oscuridad? Nos lo sugiere el \textbf{apóstol Pablo}, que nos dijo: \textquote{Se ha manifestado la gracia de Dios}. La gracia de Dios, \textquote{que trae la salvación para todos los hombres} (\emph{Tt} 2,11), ha envuelto al mundo esta noche.
	
	Pero, ¿qué es esta gracia? Es el amor divino, el amor que transforma la vida, renueva la historia, libera del mal, infunde paz y alegría. En esta noche, el amor de Dios se ha mostrado a nosotros: es Jesús. En Jesús, el Altísimo se hizo pequeño para ser amado por nosotros. En Jesús, Dios se hizo Niño, para dejarse abrazar por nosotros. Pero, podemos todavía preguntarnos, ¿por qué san Pablo llama \textquote{gracia} a la venida de Dios al mundo? Para decirnos que es completamente gratuita. Mientras que aquí en la tierra todo parece responder a la lógica de dar para tener, Dios llega gratis. Su amor no es negociable: no hemos hecho nada para merecerlo y nunca podremos recompensarlo.
	
	\emph{Se ha \textbf{manifestado} la gracia de Dios}. En esta noche nos damos cuenta de que, aunque no estábamos a la altura, Él se hizo pequeñez para nosotros; mientras andábamos ocupados en nuestros asuntos, Él vino entre nosotros. La Navidad nos recuerda que Dios sigue amando a cada hombre, incluso al peor. A mí, a ti, a cada uno de nosotros, Él nos dice hoy \emph{:} \textquote{Te amo y siempre te amaré, eres precioso a mis ojos}. Dios no te ama porque piensas correctamente y te comportas bien; Él te ama y basta. Su amor es incondicional, no depende de ti. Puede que tengas ideas equivocadas, que hayas hecho de las tuyas; sin embargo, el Señor no deja de amarte. ¿Cuántas veces pensamos que Dios es bueno si nosotros somos buenos, y que nos castiga si somos malos? Pero no es así. Aun en nuestros pecados continúa amándonos. Su amor no cambia, no es quisquilloso; es fiel, es paciente. Este es el regalo que encontramos en Navidad: descubrimos con asombro que el Señor es toda la gratuidad posible, toda la ternura posible. Su gloria no nos deslumbra, su presencia no nos asusta. Nació pobre de todo, para conquistarnos con la riqueza de su amor.
	
	\emph{Se ha \textbf{manifestado} la gracia de Dios}. Gracia es sinónimo de belleza. En esta noche, redescubrimos en la belleza del amor de Dios, también nuestra belleza, porque somos \emph{los amados de Dios}. En el bien y en el mal, en la salud y en la enfermedad, felices o tristes, a sus ojos nos vemos hermosos: no por lo que hacemos sino por lo que somos. Hay en nosotros una belleza indeleble, intangible; una belleza irreprimible que es el núcleo de nuestro ser. Dios nos lo recuerda hoy, tomando con amor nuestra humanidad y haciéndola suya, \textquote{desposándose con ella} para siempre.
	
	De hecho, la \textbf{\textquote{gran alegría} anunciada a los pastores} esta noche es \textquote{para todo el pueblo}. En aquellos pastores, que ciertamente no eran santos, también estamos nosotros, con nuestras flaquezas y debilidades. Así como los llamó a ellos, Dios también nos llama a nosotros, porque nos ama. Y, en las noches de la vida, a nosotros como a ellos nos dice: \textquote{No temáis} (\emph{Lc} 2,10). ¡Ánimo, no hay que perder la confianza, no hay que perder la esperanza, no hay que pensar que amar es tiempo perdido! En esta noche, el amor venció al miedo, apareció una nueva esperanza, la luz amable de Dios venció la oscuridad de la arrogancia humana. ¡Humanidad, Dios te ama, se hizo hombre por ti, ya no estás sola!
	
	Queridos hermanos y hermanas: ¿Qué hacer ante esta gracia? Una sola cosa: \emph{acoger el don}. Antes de ir en busca de Dios, dejémonos buscar por Él, porque Él nos busca primero. No partamos de nuestras capacidades, sino de su gracia, porque Él es Jesús, el Salvador. Pongamos nuestra mirada en el Niño y dejémonos envolver por su ternura. Ya no tendremos más excusas para no dejarnos amar por Él: Lo que sale mal en la vida, lo que no funciona en la Iglesia, lo que no va bien en el mundo ya no será una justificación. Pasará a un segundo plano, porque frente al amor excesivo de Jesús, que es todo mansedumbre y cercanía, no hay excusas. La pregunta que surge en Navidad es: \textquote{¿Me dejo amar por Dios? ¿Me abandono a su amor que viene a salvarme?}.
	
	Un regalo así, tan grande, merece mucha gratitud. Acoger la gracia es saber \emph{agradecer}. Pero nuestras vidas a menudo transcurren lejos de la gratitud. Hoy es el día adecuado para acercarse al sagrario, al belén, al pesebre, para agradecer. Acojamos el don que es Jesús, para luego \emph{transformarnos en don} como Jesús. Convertirse en don es dar sentido a la vida y es la mejor manera de cambiar el mundo: cambiamos nosotros, cambia la Iglesia, cambia la historia cuando comenzamos a no querer cambiar a los otros, sino a nosotros mismos, haciendo de nuestra vida un don.
	
	Jesús nos lo manifiesta esta noche. No cambió la historia constriñendo a alguien o a fuerza de palabras, sino con el don de su vida. No esperó a que fuéramos buenos para amarnos, sino que se dio a nosotros gratuitamente. Tampoco nosotros podemos esperar que el prójimo cambie para hacerle el bien, que la Iglesia sea perfecta para amarla, que los demás nos tengan consideración para servirlos. Empecemos nosotros. Así es como se acoge el don de la gracia. Y la santidad no es sino custodiar esta gratuidad.
	
	Una hermosa leyenda cuenta que, cuando Jesús nació, los pastores corrían hacia la gruta llevando muchos regalos. Cada uno llevaba lo que tenía: unos, el fruto de su trabajo, otros, algo de valor. Pero mientras todos los pastores se esforzaban, con generosidad, en llevar lo mejor, había uno que no tenía nada. Era muy pobre, no tenía nada que ofrecer. Y mientras los demás competían en presentar sus regalos, él se mantenía apartado, con vergüenza. En un determinado momento, san José y la Virgen se vieron en dificultad para recibir todos los regalos, muchos, sobre todo María, que debía tener en brazos al Niño. Entonces, viendo a aquel pastor con las manos vacías, le pidió que se acercara. Y le puso a Jesús en sus manos. El pastor, tomándolo, se dio cuenta de que había recibido lo que no se merecía, que tenía entre sus brazos el regalo más grande de la historia. Se miró las manos, y esas manos que le parecían siempre vacías se habían convertido en la cuna de Dios. Se sintió amado y, superando la vergüenza, comenzó a mostrar a Jesús a los otros, porque no podía sólo quedarse para él el regalo de los regalos.
	
	Querido hermano, querida hermana: Si tus manos te parecen vacías, si ves tu corazón pobre en amor, esta noche es para ti. \emph{Se ha manifestado la gracia de Dios} para resplandecer en tu vida. Acógela y brillará en ti la luz de la Navidad.
\end{body}

\subsubsection{Mensaje Urbi et Orbi (2019)} 

\src{Balcón central de la Basílica Vaticana. \\Miércoles 25 de diciembre del 2019.}

\begin{body}
	\textquote{El pueblo que caminaba en tinieblas vio una luz grande} (\emph{Is} 9,1).
	
	\emph{Queridos hermanos y hermanas: ¡Feliz Navidad!}
	
	\ltr{E}{n} el seno de la madre Iglesia, esta noche ha nacido nuevamente el Hijo de Dios hecho hombre. Su nombre es Jesús, que significa Dios salva. El Padre, Amor eterno e infinito, lo envió al mundo no para condenarlo, sino para salvarlo (cf. \emph{Jn} 3,17). El Padre lo dio, con inmensa misericordia. Lo entregó para todos. Lo dio para siempre. Y Él nació, como pequeña llama encendida en la oscuridad y en el frío de la noche.
	
	Aquel Niño, nacido de la Virgen María, es la Palabra de Dios hecha carne. La Palabra que orientó el corazón y los pasos de Abrahán hacia la tierra prometida, y sigue atrayendo a quienes confían en las promesas de Dios. La Palabra que guio a los hebreos en el camino de la esclavitud a la libertad, y continúa llamando a los esclavos de todos los tiempos, también hoy, a salir de sus prisiones. Es Palabra, más luminosa que el sol, encarnada en un pequeño hijo del hombre, Jesús, luz del mundo.
	
	Por esto el profeta exclama: \textquote{El pueblo que caminaba en tinieblas vio una luz grande} (\emph{Is} 9,1). Sí, hay tinieblas en los corazones humanos, pero más grande es la luz de Cristo. Hay tinieblas en las relaciones personales, familiares, sociales, pero más grande es la luz de Cristo. Hay tinieblas en los conflictos económicos, geopolíticos y ecológicos, pero más grande es la luz de Cristo.
	
	Que Cristo sea luz para tantos niños que sufren la guerra y los conflictos en Oriente Medio y en diversos países del mundo. Que sea consuelo para el amado pueblo sirio, que todavía no ve el final de las hostilidades que han desgarrado el país en este decenio. Que remueva las conciencias de los hombres de buena voluntad. Que inspire hoy a los gobernantes y a la comunidad internacional para encontrar soluciones que garanticen la seguridad y la convivencia pacífica de los pueblos de la región y ponga fin a sus sufrimientos. Que sea apoyo para el pueblo libanés, de este modo pueda salir de la crisis actual y descubra nuevamente su vocación de ser un mensaje de libertad y de armoniosa coexistencia para todos.
	
	Que el Señor Jesús sea luz para la Tierra Santa donde Él nació, Salvador del mundo, y donde continúa la espera de tantos que, incluso en la fatiga, pero sin desesperarse, aguardan días de paz, de seguridad y de prosperidad. Que sea consolación para Irak, atravesado por tensiones sociales, y para Yemen, probado por una grave crisis humanitaria.
	
	Que el pequeño Niño de Belén sea esperanza para todo el continente americano, donde diversas naciones están pasando un período de agitaciones sociales y políticas. Que reanime al querido pueblo venezolano, probado largamente por tensiones políticas y sociales, y no le haga faltar el auxilio que necesita. Que bendiga los esfuerzos de cuantos se están prodigando para favorecer la justicia y la reconciliación, y se desvelan para superar las diversas crisis y las numerosas formas de pobreza que ofenden la dignidad de cada persona.
	
	Que el Redentor del mundo sea luz para la querida Ucrania, que aspira a soluciones concretas para alcanzar una paz duradera.
	
	Que el Señor recién nacido sea luz para los pueblos de África, donde perduran situaciones sociales y políticas que a menudo obligan a las personas a emigrar, privándolas de una casa y de una familia. Que haya paz para la población que vive en las regiones orientales de la República Democrática del Congo, martirizada por conflictos persistentes. Que sea consuelo para cuantos son perseguidos a causa de su fe, especialmente los misioneros y los fieles secuestrados, y para cuantos caen víctimas de ataques por parte de grupos extremistas, sobre todo en Burkina Faso, Malí, Níger y Nigeria.
	
	Que el Hijo de Dios, que bajó del cielo a la tierra, sea defensa y apoyo para cuantos, a causa de estas y otras injusticias, deben emigrar con la esperanza de una vida segura. La injusticia los obliga a atravesar desiertos y mares, transformados en cementerios. La injusticia los fuerza a sufrir abusos indecibles, esclavitudes de todo tipo y torturas en campos de detención inhumanos. La injusticia les niega lugares donde podrían tener la esperanza de una vida digna y les hace encontrar muros de indiferencia.
	
	Que el Emmanuel sea luz para toda la humanidad herida. Que ablande nuestro corazón, a menudo endurecido y egoísta, y nos haga instrumentos de su amor. Que, a través de nuestros pobres rostros, regale su sonrisa a los niños de todo el mundo, especialmente a los abandonados y a los que han sufrido a causa de la violencia. Que, a través de nuestros brazos débiles, vista a los pobres que no tienen con qué cubrirse, dé el pan a los hambrientos, cure a los enfermos. Que, por nuestra frágil compañía, esté cerca de las personas ancianas y solas, de los migrantes y de los marginados. Que, en este día de fiesta, conceda su ternura a todos, e ilumine las tinieblas de este mundo.
\end{body}

\newsection

\section{Temas}

\temasNavidad