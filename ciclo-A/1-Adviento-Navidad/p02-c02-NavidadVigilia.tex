\chapter{Misa de la Vigilia}

	\section{Lecturas}

		\rtitle{PRIMERA LECTURA}

		\rbook{Del libro del profeta Isaías} \rred{62, 1-5}

		\rtheme{El Señor te prefiere a ti}
		
		\begin{readprose}
			Por amor a Sión no callaré,
			
			por amor de Jerusalén no descansaré,
			
			hasta que rompa la aurora de su justicia,
			
			y su salvación llamee como antorcha.
			
			Los pueblos verán tu justicia,
			
			y los reyes tu gloria;
			
			te pondrán un nombre nuevo,
			
			pronunciado por la boca del Señor.
			
			Serás corona fúlgida en la mano del Señor
			
			y diadema real en la palma de tu Dios.
			
			Ya no te llamarán \textquote{Abandonada},
			
			ni a tu tierra \textquote{Devastada};
			
			a ti te llamarán \textquote{Mi predilecta},
			
			y a tu tierra \textquote{Desposada},
			
			porque el Señor te prefiere a ti,
			
			y tu tierra tendrá un esposo.
			
			Como un joven se desposa con una doncella,
			
			así te desposan tus constructores.
			
			Como se regocija el marido con su esposa, se regocija tu Dios contigo.
		\end{readprose}
	

		\rtitle{SALMO RESPONSORIAL}

		\rbook{Salmo} \rred{88, 4-5. 16-17. 27 y 29}

		\rtitle{Cantaré eternamente las misericordias del Señor}
				
		\begin{psbody}
			«Sellé una alianza con mi elegido,
			jurando a David, mi siervo:
			Te fundaré un linaje perpetuo,
			edificaré tu trono para todas las edades». 
			
			Dichoso el pueblo que sabe aclamarte:
			caminará, oh, Señor, a la luz de tu rostro;
			tu nombre es su gozo cada día,
			tu justicia es su orgullo.

			Él me invocará: ``Tú eres mi padre,
			mi Dios, mi Roca salvadora''.
			Le mantendré eternamente mi favor,
			y mi alianza con él será estable.
		\end{psbody}
	

		\rtitle{SEGUNDA LECTURA}

		\rbook{De los Hechos de los Apóstoles} \rbook{13, 16-17. 22-25}

		\rtheme{Testimonio de Pablo sobre Cristo, hijo de David}
		
		\begin{scripture}
			Cuando Pablo llegó a Antioquía de Pisidia, se puso en pie y, haciendo seña con la mano de que se callaran, dijo:
			
			«Israelitas y los que teméis a Dios, escuchad:
			
			El Dios de este pueblo, Israel, eligió a nuestros padres y multiplicó al pueblo cuando vivían como forasteros en Egipto. Los sacó de allí con brazo poderoso.
			
			Después, les suscitó como rey a David, en favor del cual dio testimonio, diciendo:
			
			Encontré a David, hijo de Jesé,
			
			hombre conforme a mi corazón,
			
			que cumplirá todos mis preceptos.
			
			Según lo prometido, Dios sacó de su descendencia un salvador para Israel: Jesús.
			
			Juan predicó a todo Israel un bautismo de conversión antes de que llegara Jesús; y, cuando Juan estaba para concluir el curso de su vida, decía:
			
			``Yo no soy quien pensáis, pero, mirad, viene uno detrás de mí a quien no merezco desatarle las sandalias de los pies''.
		\end{scripture}


		\rtitle{EVANGELIO (forma larga)}

		\rbook{Del Santo Evangelio según san Mateo} \rred{1, 1-25}

		\rtheme{Genealogía de Jesucristo, hijo de David}
		
		\begin{scripture}
			Libro del origen de Jesucristo, hijo de David, hijo de Abrahán.
			
			Abrahán engendró a Isaac, Isaac engendró a Jacob, Jacob engendró a Judá y a sus hermanos. Judá engendró, de Tamar, a Fares y a Zará, Fares engendró a Esrón, Esrón engendró a Arán, Arán engendró a Aminadab, Aminadab engendró a Naasón, Naasón engendró a Salmón, Salmón engendró, de Rajab, a Booz; Booz engendró, de Rut, a Obed; Obed engendró a Jesé, Jesé engendró a David, el rey.
			
			David, de la mujer de Urías, engendró a Salomón, Salomón engendró a Roboán, Roboán engendró a Abías, Abías engendró a Asaf, Asaf engendró a Josafat, Josafat engendró a Jorán, Jorán engendró a Ozías, Ozías engendró a Joatán, Joatán engendró a Acaz, Acaz engendró a Ezequías, Ezequías engendró a Manasés, Manasés engendró a Amós, Amós engendró a Josías; Josías engendró a Jeconías y a sus hermanos, cuando el destierro de Babilonia.
			
			Después del destierro de Babilonia, Jeconías engendró a Salatiel, Salatiel engendró a Zorobabel, Zorobabel engendró a Abiud, Abiud engendró a Eliaquín, Eliaquín engendró a Azor, Azor engendró a Sadoc, Sadoc engendró a Aquín, Aquín engendró a Eliud, Eliud engendró a Eleazar, Eleazar engendró a Matán, Matán engendró a Jacob; y Jacob engendró a José, el esposo de María, de la cual nació Jesús, llamado Cristo. Así, las generaciones desde Abrahán a David fueron en total catorce; desde David hasta la deportación a Babilonia, catorce; y desde la deportación a Babilonia hasta el Cristo, catorce.
			
			La generación de Jesucristo fue de esta manera: María, su madre, estaba desposada con José y, antes de vivir juntos, resultó que ella esperaba un hijo por obra del Espíritu Santo.
			
			José, su esposo, como era justo y no quería difamarla, decidió repudiarla en privado. Pero, apenas había tomado esta resolución, se le apareció en sueños un ángel del Señor que le dijo:
			
			«José, hijo de David, no temas acoger a María, tu mujer, porque la criatura que hay en ella viene del Espíritu Santo. Dará a luz un hijo y tú le pondrás por nombre Jesús, porque él salvará a su pueblo de sus pecados».
			
			Todo esto sucedió para que se cumpliese lo que había dicho el Señor por medio del profeta:
			
			«Mirad: la Virgen concebirá y dará a luz un hijo y le pondrán por nombre Enmanuel, que significa \textquote{Dios-con-nosotros}».
			
			Cuando José se despertó, hizo lo que le había mandado el ángel del Señor y acogió a su mujer. Y sin haberla conocido, ella dio a luz un hijo al que puso por nombre Jesús.
		\end{scripture}

		\rtitle{EVANGELIO (forma breve)}

		\rbook{Del Santo Evangelio según san Mateo} \rred{1, 18-25}

		\rtheme{María dará a luz un hijo y tú le pondrás por nombre Jesús}
		
		\begin{scripture}
			La generación de Jesucristo fue de esta manera:
			
			María, su madre, estaba desposada con José y, antes de vivir juntos, resultó que ella esperaba un hijo por obra del Espíritu Santo.
			
			José, su esposo, como era justo y no quería difamarla, decidió repudiarla en privado. Pero, apenas había tomado esta resolución, se le apareció en sueños un ángel del Señor que le dijo: «José, hijo de David, no temas acoger a María, tu mujer, porque la criatura que hay en ella viene del Espíritu Santo. Dará a luz un hijo y tú le pondrás por nombre Jesús, porque él salvará a su pueblo de sus pecados».
			
			Todo esto sucedió para que se cumpliese lo que había dicho el Señor por medio del profeta:
			
			«Mirad: la Virgen concebirá y dará a luz un hijo
			
			y le pondrán por nombre Enmanuel,
			
			que significa \textquote{Dios-con-nosotros}».
			
			Cuando José se despertó, hizo lo que le había mandado el ángel del Señor y acogió a su mujer.
			
			Y sin haberla conocido, ella dio a luz un hijo al que puso por nombre Jesús.
		\end{scripture}
	
	\newsection

	\section{Comentario Patrístico}

		\subsection{San Pedro Crisólogo, obispo}

			\ptheme{Mirad: la virgen concebirá y dará a luz un hijo, y le pondrá por nombre Emmanuel}

			\src{Sermón 145: PL 52, 588}
			
			\begin{body}
				Es mi propósito hablaros hoy, hermanos, de cómo nos relata el santo evangelista el misterio de la generación de Cristo. Dice: \emph{El nacimiento de Jesucristo fue de esta manera: La madre de Jesús estaba desposada con José y, antes de vivir juntos, resultó que ella esperaba un hijo, por obra del Espíritu Santo. José, su esposo, que era bueno y no quería denunciarla, decidió repudiarla en secreto}. Pero, ¿cómo se compagina esta bondad con la resolución de no discutir la gravidez de su esposa? Las virtudes no pueden sostenerse separadamente: la equidad desprovista de bondad se convierte en severidad, y la justicia sin piedad se torna crueldad. Con razón, pues, a José se le califica de bueno, porque era piadoso; y de piadoso, por ser bueno. Al pensar piadosamente, se libra de la crueldad; al juzgar benignamente, observó la justicia; al no querer erigirse en acusador, rehuyó la sentencia. Se requemaba el alma del justo perpleja ante la novedad del evento: tenía ante sí una esposa preñada, pero virgen; grávida del don no menos que del pudor; solícita por lo concebido, pero segura de su integridad; revestida de la función maternal, sin perder el decoro virginal.
				
				¿Cuál debía ser la conducta del esposo ante tal situación? ¿Acusarla de infidelidad? No, pues él era testigo de su inocencia. ¿Airear la culpa? Tampoco, pues era él el guardián de su pudor. ¿Inculparla de adulterio? Menos aún, pues estaba plenamente convencido de su virginidad. ¿Qué hacer, entonces? Piensa repudiarla en secreto, pues ni podía ir por ahí aireando lo sucedido ni ocultarlo en la intimidad del hogar. Decide repudiarla en secreto y confía a Dios todo el negocio, ya que nada tiene que comunicar a los hombres.
				
				\emph{José, hijo de David, no tengas reparo en llevarte a María, tu mujer, porque la criatura que hay en ella viene del Espíritu Santo. Dará a luz un hijo y tú le pondrás por nombre Jesús, porque él salvará al pueblo de los pecados}. Veis, hermanos, cómo una sola persona representa toda una raza, veis cómo un solo individuo lleva la representación de toda una estirpe, veis cómo en José se da cita la serie genealógica de David.
				
				\emph{José, hijo de David}. Nacido de la vigésima octava generación, ¿por qué se le llama hijo de David, sino porque en él se desvela el misterio de una estirpe, se cumple la fidelidad de la promesa, y en la carne virginal luce ya el sello de la sobrenatural concepción de un parto celeste? La promesa de Dios Padre hecha a David estaba expresada en estos términos: \emph{El Señor ha jurado a David una promesa que no retractará: \textquote{A uno de tu linaje pondré sobre tu trono}}. Del fruto de tu vientre: sí, fruto de tu vientre, de tu seno, sí, porque el huésped celeste, el supremo morador de tal modo descendió al receptáculo de tus entrañas que ignoró las limitaciones corporales; y de tal modo salió del claustro materno, que dejó intacto el sello de la virginidad, cumpliéndose de esta manera lo que se canta en el Cantar de los cantares: \emph{Eres jardín cerrado, hermana y novia mía; eres jardín cerrado, fuente sellada}.
				
				\emph{La criatura que hay en ella viene del Espíritu Santo}. Concibió la virgen, pero del Espíritu; parió la virgen, pero a aquel que había predicho Isaías: \emph{Mirad: la virgen concebirá y dará a luz un hijo, y le pondrá por nombre Emmanuel (que significa \textquote{Dios-con-nosotros})}.
			\end{body}

\newsection

	\section{Homilías}

	\rbr{Las homilías para esta celebración están tomadas de textos de las Padres de la Iglesia que tocan algunos aspectos de la Navidad en particular o relacionados con alguno de los textos bíblicos que se leen en la misma. \\Conviene señalar que estas homilías pueden iluminar aspectos de cualquiera de las otras celebraciones durante el tiempo de Navidad.}

		\subsection{San Bernardo de Claraval, abad}

			\subsubsection{Sermón: Habitaré y caminaré con ellos}

				\src{Sermón 27, 7.9 sobre el Cantar de los cantares: \\Opera omnia. Edit. Cister. 1957, I, 186-188.}
				
				\begin{body}
					Luego que el divino Emmanuel implantó en la tierra el magisterio de la doctrina celeste, luego que por Cristo y en Cristo se nos manifestó la imagen visible de aquella celestial Jerusalén, que es nuestra madre, y el esplendor de su belleza, ¿qué es lo que contemplamos sino a la Esposa en el Esposo, admirando en el mismo y único Señor de la gloria al Esposo que se ciñe la corona y a la Esposa que se adorna de sus joyas? El que bajó es efectivamente el mismo que subió, pues nadie ha subido al cielo sino el que bajó del cielo: el mismo y único Señor, esposo en la cabeza y esposa en el cuerpo. Y no en vano apareció en la tierra el hombre celestial, él que convirtió en celestiales a muchos hombres terrenales haciéndolos semejantes a él, para que se cumpliera lo que dice la Escritura: \emph{Igual que el celestial son los hombres celestiales}.
					
					Desde entonces en la tierra se vive como en el cielo, pues, a imitación de aquella soberana y dichosa criatura, también ésta que viene desde los confines de la tierra a escuchar la sabiduría de Salomón, se une al esposo celeste con un vínculo de casto amor; y si bien no le está todavía como aquélla unida por la visión, está ya desposada por la fe, según la promesa de Dios, que dice por el profeta: \emph{Me casaré contigo en derecho y justicia, en misericordia y compasión, me casaré contigo en fidelidad}. Por eso se afana en conformarse más y más al modelo celestial, aprendiendo de él a ser modesta y sobria, pudorosa y santa, paciente y compasiva, aprendiendo finalmente a ser mansa y humilde de corazón. Con semejante conducta procura agradar, aunque ausente, a aquel a quien los ángeles desean contemplar, a fin de que, inflamada de angélico ardor, se comporte como ciudadana de los santos y miembro de la familia de Dios, se comporte como la amada, se comporte como la esposa.
					
					\emph{Ven, mi elegida, y pondré en ti mi trono}. ¿Por qué te acongojas ahora, alma mía, por qué te me turbas? ¿Crees que podrás disponer en tu interior un lugar para el Señor? Y ¿qué lugar, en nuestro interior, podrá parecernos idóneo para tanta gloria, capaz de tamaña majestad? ¡Ojalá pudiera merecer siquiera adorar al estrado de sus pies! ¡Quién me diera seguir al menos las huellas de cualquier alma santa, que el Señor se escogió como heredad! Mas si él se dignase derramar en mi alma la unción de su misericordia, y dilatarla como se dilata una piel engrasada, de modo que también yo pudiera decir: \emph{Correré por el camino de tus mandatos cuando me ensanches el corazón,} quizá pudiera a mi vez mostrarle en mí mismo, si no una sala grande arreglada con divanes, donde pueda sentarse a la mesa con sus discípulos, sí al menos un lugar donde pueda reclinar su cabeza. Veo en lontananza a aquellas almas realmente dichosas, de las cuales se ha dicho: \emph{Habitaré y caminaré con ellos}.
				\end{body}

\newsection			

		\subsection{San Juan Pablo II, papa}

			\subsubsection{Catequesis: La genealogía de Mateo}
			
				\src{Catequesis en la Audiencia general del 28 de enero de 1987, nn. 5-10.}
				
				\begin{body}
					El Evangelio \emph{según Mateo} completa la narración de Lucas describiendo algunas circunstancias que precedieron al nacimiento de Jesús. Leemos: \textquote{La concepción de Jesucristo fue así: Estando desposada María, su Madre, con José, \emph{antes de que conviviesen} se halló \emph{haber concebido María del Espíritu Santo}. José, su esposo, siendo justo, no quiso denunciarla y resolvió repudiarla en secreto. Mientras reflexionaba sobre esto, he aquí que se le apareció en sueños un ángel del Señor y le dijo: José, hijo de David, no temas recibir en tu casa a María, tu esposa, pues \emph{lo concebido en ella es obra del Espíritu Santo}. Dará a luz un hijo a quien pondrás por nombre Jesús, porque salvará a su pueblo de sus pecados} (\emph{Mt} 1, 18-21 ).
					
					Como se ve, ambos textos del \textquote{Evangelio de la infancia concuerdan en la constatación fundamental}: Jesús fue concebido por obra del Espíritu Santo y nació de María Virgen; y son entre sí \emph{complementarios} en el esclarecimiento de las circunstancias de este acontecimiento extraordinario: Lucas respecto a María, Mateo respecto a José.
					
					Para identificar \emph{la fuente de la que deriva el Evangelio de la infancia}, hay que referirse a la frase de San Lucas: \textquote{\emph{María guardaba todo esto} y lo meditaba en su corazón} (\emph{Lc} 2, 19). Lucas lo dice dos veces: después de marchar los pastores de Belén y después del encuentro de Jesús en el templo (cf. 2, 51). El Evangelista mismo nos ofrece los elementos para identificar en la Madre de Jesús una de las fuentes de información utilizadas por él para escribir el \textquote{Evangelio de la infancia}. María, que \textquote{guardó todo esto en su corazón} (cf. \emph{Lc} 2, 19), pudo dar testimonio, después de la muerte y resurrección de Cristo, de lo que se refería a la propia persona y a la función de Madre precisamente en el período apostólico, en el que nacieron los textos del Nuevo Testamento y tuvo origen la primera tradición cristiana.
					
					El testimonio evangélico de \emph{la concepción virginal de Jesús} por parte de María es de gran relevancia teológica. Pues constituye un signo especial \emph{del origen divino del Hijo de María}. El que Jesús no tenga un padre terreno porque ha sido engendrado \textquote{sin intervención de varón}, pone de relieve la verdad de que Él es el Hijo de Dios, de modo que cuando asume la naturaleza humana, su Padre continúa siendo exclusivamente Dios.
					
					La revelación de la intervención del Espíritu Santo \emph{en la concepción de Jesús}, indica \emph{el comienzo} en la historia del hombre de la nueva generación espiritual que tiene un carácter estrictamente sobrenatural (cf. \emph{1 Cor} 15, 45-49). De este modo Dios Uno y Trino \textquote{se comunica} a la criatura mediante el Espíritu Santo. Es el misterio al que se pueden aplicar las palabras del Salmo: \textquote{Envía tu Espíritu, y serán creados, y renovarás la faz de la tierra} (\emph{Sal} 103 {[}104{]}, 30). En la economía de esa comunicación de Sí mismo que Dios hace a la criatura, la concepción virginal de Jesús, que sucedió por obra del Espíritu Santo, es un \emph{acontecimiento central y culminante}. Él \emph{inicia la \textquote{nueva creación}}. Dios entra así en un modo decisivo en la historia para actuar el destino sobrenatural del hombre, o sea, la predestinación de todas las cosas en Cristo. Es \emph{la expresión} definitiva del \emph{Amor salvífico} de Dios al hombre, del que hemos hablado en las catequesis sobre la Providencia.
					
					En la actuación del plan de la salvación hay siempre una participación de la criatura. Así en la concepción de Jesús por obra del Espíritu Santo \emph{María participa} de forma \emph{decisiva}. Iluminada interiormente por el mensaje del ángel sobre su vocación de Madre y sobre la conservación de su virginidad, María \emph{expresa su voluntad y consentimiento} y acepta hacerse el humilde instrumento de la \textquote{virtud del Altísimo}. La acción del Espíritu Santo hace que en María la maternidad y la virginidad estén presentes de un modo que, aunque inaccesible a la mente humana, entre de lleno en el ámbito de la predilección de la omnipotencia de Dios. En María se cumple la gran profecía de Isaías: \textquote{La virgen grávida da a luz} (7, 14; cf. \emph{Mt} 1, 22-23); su virginidad, signo en el Antiguo Testamento de la pobreza y de disponibilidad total al plan de Dios, se convierte en el terreno de la acción excepcional de Dios, que escoge a María para ser Madre del Mesías.
					
					La excepcionalidad de María se deduce también de las genealogías aducidas por Mateo y Lucas.
					
					El Evangelio \emph{según Mateo} comienza, conforme a la costumbre hebrea\emph{, con la genealogía de Jesús} (\emph{Mt} 1, 2-17) y hace un elenco partiendo de Abraham, de las generaciones masculinas. A Mateo de hecho, le importa poner de relieve, mediante la paternidad \emph{legal} de José, la descendencia de Jesús de Abraham y David y, por consiguiente, la legitimidad de su calificación de Mesías. Sin embargo, al final de la serie de los ascendientes leemos: \textquote{Y Jacob engendró a José esposo de María, \emph{de la cual nació Jesús llamado Cristo}} (\emph{Mt} 1, 16). Poniendo el acento en la maternidad de María, el Evangelista implícitamente subraya la verdad del nacimiento virginal: Jesús, como hombre, no tiene padre terreno.
					
					\emph{Según el Evangelio de Lucas}, la genealogía de Jesús (\emph{Lc} 3, 23-38) es ascendente: desde Jesús a través de sus antepasados se remonta \emph{hasta Adán}. El Evangelista ha querido mostrar la vinculación de Jesús \emph{con todo el género humano}. María, como colaboradora de Dios en dar a su Eterno Hijo la naturaleza humana, ha sido el instrumento de la unión de Jesús con toda la humanidad.
				\end{body}

\newsection

	\section{Temas}

		\rbr{El Directorio Homilético recoge los temas de la Navidad en un solo grupo, ver página \pageref{navidad_temas}.}