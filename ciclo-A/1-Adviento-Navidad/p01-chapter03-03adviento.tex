\chapter{Domingo III de Adviento (A)}

	\section{Lecturas}
	
		\rtitle{PRIMERA LECTURA}

		\rbook{Del libro del profeta Isaías 35, 1-6a. 10}

		\rtheme{Dios vendrá y nos salvará}
		
		\begin{readprose}
			El desierto y el yermo se regocijarán, 
			
			se alegrará la estepa y florecerá, 
			
			germinará y florecerá como flor de narciso, 
			
			festejará con gozo y cantos de júbilo. 
			
			Le ha sido dada la gloria del Líbano, 
			
			el esplendor del Carmelo y del Sarón. 
			
			Contemplarán la gloria del Señor,
			
			la majestad de nuestro Dios. 
			
			Fortaleced las manos débiles,
			
			afianzad las rodillas vacilantes;
			
			decid a los inquietos: 
			
			«Sed fuertes, no temáis.
			
			¡He aquí vuestro Dios! Llega el desquite, 
			
			la retribución de Dios. 
			
			Viene en persona y os salvará».
			
			Entonces se despegarán los ojos de los ciegos, 
			
			los oídos de los sordos se abrirán; 
			
			entonces saltará el cojo como un ciervo.
			
			Retornan los rescatados del Señor. 
			
			Llegarán a Sión con cantos de júbilo: 
			
			alegría sin límite en sus rostros. 
			
			Los dominan el gozo y la alegría. 
			
			Quedan atrás la pena y la aflicción.
		\end{readprose}

		\rtitle{SALMO RESPONSORIAL}

		\rbook{Salmo} \rred{145, 6c-7. 8-9a. 9bc-10}

		\rtheme{Ven, Señor, a salvarnos}
		
		\begin{psbody}
			El Señor mantiene su fidelidad perpetuamente, 
			hace justicia a los oprimidos, 
			da pan a los hambrientos. 
			El Señor liberta a los cautivos. 
			
			El Señor abre los ojos al ciego, 
			el Señor endereza a los que ya se doblan, 
			el Señor ama a los justos. 
			El Señor guarda a los peregrinos. 
			
			Sustenta al huérfano y a la viuda 
			y trastorna el camino de los malvados. 
			El Señor reina eternamente, 
			tu Dios, Sión, de edad en edad.
		\end{psbody}

		\rtitle{SEGUNDA LECTURA}

		\rbook{De la carta del apóstol Santiago} \rred{5, 7-10}

		\rtheme{Manteneos firmes, porque la venida del Señor está cerca}
		
		\begin{scripture}
			Hermanos, esperad con paciencia hasta la venida del Señor. Mirad: el labrador aguarda el fruto precioso de la tierra, esperando con paciencia hasta que recibe la lluvia temprana y la tardía. 
			
			Esperad con paciencia también vosotros, y fortaleced vuestros corazones, porque la venida del Señor está cerca. 
			
			Hermanos, no os quejéis los unos de los otros, para que no seáis condenados; mirad: el juez está ya a las puertas. 
			
			Hermanos, tomad como modelo de resistencia y de paciencia a los profetas que hablaron en nombre del Señor.
		\end{scripture}

		\rtitle{EVANGELIO}

		\rbook{Del Santo Evangelio según san Mateo} \rred{11, 2-11}

		\rtheme{¿Eres tú el que ha de venir o tenemos que esperar a otro?}
		
		\begin{scripture}
			En aquel tiempo, Juan, que había oído en la cárcel las obras del Mesías, mandó a sus discípulos a preguntarle:
			
			\textquote{¿Eres tú el que ha de venir o tenemos que esperar a otro?}. 
			
			Jesús les respondió: 
			
			«Id a anunciar a Juan lo que estáis viendo y oyendo:
			
			los ciegos ven y los cojos andan;
			
			los leprosos quedan limpios y los sordos oyen;
			
			los muertos resucitan 
			
			y los pobres son evangelizados. 
			
			¡Y bienaventurado el que no se escandalice de mí!». 
			
			Al irse ellos, Jesús se puso a hablar a la gente sobre Juan: 
			
			«¿Qué salisteis a contemplar en el desierto, una caña sacudida por el viento? ¿O qué salisteis a ver, un hombre vestido con lujo? Mirad, los que visten con lujo habitan en los palacios. Entonces, ¿a qué salisteis?, ¿a ver a un profeta? 
			
			Sí, os digo, y más que profeta. Este es de quien está escrito: 
			
			\textquote{Yo envío a mi mensajero delante de ti, el cual preparará tu camino ante ti}. 
			
			En verdad os digo que no ha nacido de mujer uno más grande que Juan el Bautista; aunque el más pequeño en el reino de los cielos es más grande que él».
		\end{scripture}

	\section{Comentario Patrístico}

		\subsection{San Ambrosio, obispo}

			\ptheme{¿Eres tú el que ha de venir o tenemos que esperar a otro?}

			\src{Comentario sobre el evangelio de san Lucas, \\Lib. 5, 93-95. 99-102. 109: CCL 14, 165-166. 167-168. 171-177.}
			
			\begin{body}
				\emph{Juan envió a dos de sus discípulos a preguntar a Jesús: \textquote{¿Eres tú el que ha de venir o tenemos que esperar a otro?}}. No es sencilla la comprensión de estas sencillas palabras, o de lo contrario este texto estaría en contradicción con lo dicho anteriormente. ¿Cómo, en efecto, puede Juan afirmar aquí que desconoce a quien anteriormente había reconocido por revelación de Dios Padre? ¿Cómo es que entonces conoció al que previamente desconocía mientras que ahora parece desconocer al que ya antes conocía? \emph{Yo} ---dice--- \emph{no lo conocía, pero el que me envió a bautizar con agua me dijo: \textquote{Aquel sobre quien veas bajar el Espíritu Santo\ldots{}}}. Y Juan dio fe al oráculo, reconoció al revelado, adoró al bautizado y profetizó al enviado Y concluye: \emph{Y yo lo he visto, y he dado testimonio de que éste es el elegido de Dios}. ¿Cómo, pues, aceptar siquiera la posibilidad de que un profeta tan grande haya podido equivocarse, hasta el punto de no considerar aún como Hijo de Dios a aquel de quien había afirmado: \emph{Éste es el que quita el pecado del mundo?} 
				
				Así pues, ya que la interpretación literal es contradictoria, busquemos el sentido espiritual. Juan --lo hemos dicho ya-- era tipo de la ley, precursora de Cristo. Y es correcto afirmar que la ley --aherrojada materialmente como estaba en los corazones de los sin fe, como en cárceles privadas de la luz eterna, y constreñida por entrañas fecundas en sufrimientos e insensatez-- era incapaz de llevar a pleno cumplimiento el testimonio de la divina economía sin la garantía del evangelio. Por eso, envía Juan a Cristo dos de sus discípulos, para conseguir un suplemento de sabiduría, dado que Cristo es la plenitud de la ley. 
				
				Además, sabiendo el Señor que nadie puede tener una fe plena sin el evangelio ---ya que si la fe comienza en el antiguo Testamento no se consuma sino en el nuevo---, a la pregunta sobre su propia identidad, responde no con palabras, sino con hechos. \emph{Id} ---dice--- \emph{a anunciar a Juan lo que estáis viendo y oyendo: los ciegos ven y los inválidos andan; los leprosos quedan limpios y los sordos oyen; los muertos resucitan y a los pobres se les anuncia la buena noticia}. Y sin embargo, estos ejemplos aducidos por el Señor no son aún los definitivos: la plenificación de la fe es la cruz del Señor, su muerte, su sepultura. Por eso, completa sus anteriores afirmaciones añadiendo: \emph{¡Y dichoso el que no se sienta defraudado por mí!} Es verdad que la cruz se presta a ser motivo de escándalo incluso para los elegidos, pero no lo es menos que no existe mayor testimonio de una persona divina, nada hay más sobrehumano que la íntegra oblación de uno solo por la salvación del mundo; este solo hecho lo acredita plenamente como Señor. Por lo demás, así es cómo Juan lo designa: \emph{Este es el Cordero de Dios, que quita el pecado del mundo}. En realidad, esta respuesta no va únicamente dirigida a aquellos dos hombres, discípulos de Juan: va dirigida a todos nosotros, para que creamos en Cristo en base a los hechos. 
				
				\emph{Entonces, ¿a qué salisteis?, ¿a ver a un profeta? Sí, os digo, y más que profeta}. Pero, ¿cómo es que querían ver a Juan en el desierto, si estaba encerrado en la cárcel? El Señor propone a nuestra imitación a aquel que le había preparado el camino no sólo precediéndolo en el nacimiento según la carne y anunciándolo con la fe, sino también anticipándosele con su gloriosa pasión. Más que profeta, sí, ya que es él quien cierra la serie de los profetas; más que profeta, ya que muchos desearon ver a quien éste profetizó, a quien éste contempló, a quien éste bautizó.
			\end{body}

	\section{Homilías}
	
		\subsection{San Juan Pablo II, papa}
		
			\subsubsection{Homilía (1980): En medio del Adviento, una pregunta}
			
			\src{Visita Pastoral a la Parroquia Romana de la Natividad de Nuestro Señor.
\\14 de diciembre de 1980.}
			
				\begin{body}
					1. Me alegro por el hecho de que hoy puedo estar en vuestra parroquia. Efectivamente, ya están muy cerca para nosotros las fiestas de la Navidad del Señor, y vuestra parroquia \emph{está dedicada precisamente a la Natividad.} Por eso el período del Adviento se vive en vuestra comunidad de modo particularmente profundo, y me alegro porque puedo participar hoy en este modo vuestro de vivir el Adviento.
					
					2. «¿Eres tú el que ha de venir o tenemos que esperar a otro?» \emph{(Mt} 11, 3). 
					
					Hoy, III domingo de Adviento, la Iglesia repite \textbf{la pregunta} que fue hecha por primera vez a Cristo por los discípulos de Juan Bautista: ¿Eres tú el que ha de venir? 
					
					Así preguntaron los discípulos de aquel que dedicó toda su misión a preparar la venida del Mesías, los discípulos de aquel que «amó y preparó la venida del Señor» hasta la cárcel y hasta la muerte. Ahora sabemos que, cuando sus discípulos presentan esta pregunta a Jesús, Juan Bautista se encuentra ya en la cárcel, de la que no podrá salir más. 
					
					Y \textbf{Jesús responde}, remitiéndose a sus obras y a sus palabras y, a la vez, a la profecía mesiánica de Isaías: «Los ciegos ven y los inválidos andan; los leprosos quedan limpios y los sordos oyen; los muertos resucitan, y a los pobres se les anuncia la Buena Noticia\ldots{...} Id a anunciar a Juan lo que estáis viendo y oyendo» \emph{(Mt} 11, 5. 4). 
					
					En el centro mismo de la liturgia del Adviento nos encontramos, pues, esta pregunta dirigida a Cristo y su respuesta mesiánica. 
					
					Aunque esta pregunta se haya hecho una sola vez, sin embargo nosotros la podemos hacer \emph{siempre de nuevo}. Debe ser hecha. ¡Y en realidad se hace! 
					
					El hombre plantea la pregunta en torno a Cristo. Diversos hombres, desde diversas partes del mundo, desde países y continentes, desde diversas culturas y civilizaciones, plantean la pregunta en torno a Cristo. En este mundo, en el que tanto se ha hecho y se hace siempre para cercar a Cristo con la conjura del silencio, para negar su existencia y misión, o para disminuirlas y deformarlas, retorna siempre de nuevo \emph{la pregunta en torno a Cristo}. Retorna también cuando puede parecer que ya se ha extirpado esencialmente. 
					
					El hombre pregunta: \emph{¿Eres tú, Cristo, el que ha de venir?} ¿Eres tú el que me explicará el sentido definitivo de mi humanidad? ¿El sentido de mi existencia? ¿Eres tú el que me ayudará a plantear y a construir mi vida de hombre desde sus fundamentos? 
					
					Así preguntan los hombres, y \emph{Cristo} constantemente \emph{responde}. Responde como respondió ya a los discípulos de Juan Bautista. Esta pregunta en torno a Cristo es la pregunta de Adviento, y es necesario que nosotros la hagamos dentro de nuestra comunidad cristiana. Hela aquí: 
					
					¿Quién es para mi Jesucristo? 
					
					¿Quién es realmente para mis pensamientos, para mi corazón, para mi actuación? ¿Cómo conozco yo, que soy cristiano y creo en El, y cómo trato de conocer al que confieso? ¿Hablo de El a los otros? ¿Doy testimonio de El, al menos ante los que están más cercanos a mí en la casa paterna, en el ambiente de trabajo, de la universidad o de la escuela, en toda mi vida y en mi conducta? Esta es precisamente la pregunta de Adviento, y es preciso que, basándonos en ella, nos hagamos las referidas, ulteriores preguntas, para que profundicen en nuestra conciencia cristiana y nos preparen así a la venida del Señor. 
					
					3. El Adviento retorna cada año, y cada año se desarrolla en el arco de cuatro semanas, cediendo luego el lugar a la alegría de la Santa Navidad. 
					
					Hay, pues, diversos Advientos: 
					
					Está el adviento del niño inocente y el adviento de la juventud inquieta (frecuentemente crítica); está el adviento de los novios; está el adviento de los esposos, de los padres, de los hombres dedicados a múltiples formas de trabajo y de responsabilidad con frecuencia grave. Finalmente están los advientos de los hombres ancianos, enfermos, de los que sufren, de los abandonados. Este año está el adviento de nuestros compatriotas víctimas de la calamidad del terremoto y que han quedado sin casa.
					
					Hay diversos advientos. Se repiten cada año, \emph{y todos se orientan hacia una dirección única.} Todos nos preparan a la misma realidad. Hoy, en la \textbf{segunda lectura} litúrgica, escuchamos lo que escribe el \textbf{Apóstol Santiago}: «Hermanos, tened paciencia, hasta la venida del Señor. El labrador aguarda paciente el fruto valioso de la tierra mientras recibe la lluvia temprana y tardía. Tened paciencia también vosotros, manteneos firmes, porque la venida del Señor está cerca». Y añade inmediatamente después: «Mirad que el juez está ya a la puerta» (5, 7-9).
					
					Precisamente este reflejo deben tener tales \emph{advientos en nuestros corazones}. Deben parecerse a la espera de la recolección. El labrador aguarda el fruto de la tierra durante todo un año o durante algunos meses. En cambio, la mies de la vida humana se espera durante toda la vida. Y todo adviento es importante. La mies de la tierra se recoge cuando está madura, para utilizarla en satisfacer las necesidades del hombre. \emph{La mies de la vida humana} espera el momento en el que aparecerá en toda la verdad ante Dios y ante Cristo, que es juez de nuestras almas.
					
					La venida de Cristo, la venida de Cristo en Belén anuncia también este \emph{juicio}. ¡Ella dice al hombre por qué le es dado madurar en el curso de todos estos advientos, de los que se compone su vida en la tierra, y cómo debe madurar él!
					
					En el \textbf{Evangelio} de hoy Cristo, ante las muchedumbres reunidas, da el siguiente juicio sobre \emph{Juan Bautista:} «Os aseguro que no ha nacido de mujer uno más grande que Juan el Bautista, aunque el más pequeño en el Reino de los cielos es más grande que él» (\emph{Mt} 11, 11). Mi deseo es que nosotros, queridos hermanos y hermanas, podamos ver el momento en que escuchemos palabras semejantes de nuestro Redentor, como la verdad definitiva sobre nuestra vida.
					
					4. Estoy meditando sobre este mensaje de Adviento, unido a la liturgia de este domingo, juntamente con vosotros, queridos feligreses de la comunidad dedicada a la Natividad del Señor.
					
					Por tanto, es necesario que cada uno lo considere como dirigido a \emph{él mismo} y es necesario también que todos lo acojáis \emph{en vuestra comunidad}.
					
					Efectivamente, la parroquia existe para que los hombres bautizados en la comunidad, esto es, completándose y ayudándose recíprocamente, \emph{se preparen a la venida del Señor}.
					
					A este propósito quisiera preguntar: \emph{¿Cómo se desarrolla y cómo debería desarrollarse} en la comunidad \emph{esta preparación a la venida del Señor?} La respuesta podría ser doble: desde un punto de vista inmediato, se puede decir que esta preparación se realiza siguiendo «en sintonía» la acción pedagógica de la Iglesia en el presente, típico período del Adviento: esto es, acogiendo la renovada invitación a la conversión y meditando el eterno misterio del Hijo de Dios que, encarnándose en el seno purísimo de María, nació en Belén. 
					
					Pero, desde un punto de vista más amplio, no se trata sólo del Adviento de este año, o de la Navidad, para vivir en actitud de fe más viva; se trata también de la cotidiana, constante venida de Cristo en nuestra vida, gracias a una presencia que se alimenta con la catequesis y, sobre todo, con la participación litúrgico-sacramental.
					
					Sé que en vuestra parroquia ésta es \emph{una de las líneas pastorales fundamentales:} efectivamente, se da la catequesis sistemática y permanente, según las diversas edades, y se dedica una atención especial a la sagrada liturgia. En realidad, la vida sacramental, cuando está iluminada por un paralelo y profundo anuncio de Cristo, es el camino más expedito para ir al encuentro de Él. En la oración y, ante todo, en la \emph{participación en la Santa Misa dominical} nos encontramos precisamente con Él. Pensándolo bien, esta participación es la renovación, cada semana, de la conciencia de la «venida del Señor». Si ella faltase, se disiparía esta conciencia, se debilitaría y pronto se destruiría. Por esto quiero dirigir la exhortación del Concilio acerca del permanente valor del domingo como «fiesta» primordial que se debe inculcar a la piedad de los fieles, «a fin de que se reúnan en asamblea para escuchar la Palabra de Dios y participar en la Eucaristía» (cf. \emph{Sacrosanctum Concilium}, 106). 
					
					Pero ---como bien sabemos--- Cristo viene a nosotros también en las personas de los hermanos, especialmente de los más pobres, de los marginados y de los alejados. {[}También a este respecto sé que vuestra comunidad está comprometida según una línea pastoral, que configura una opción precisa y valiente. Sé, por ejemplo, que son muchas las Asociaciones y los grupos eclesiales que practican la acogida evangélica como «una sincera atención para todos los males, las tristezas y las esperanzas del hombre de hoy» (cf. \emph{Relazione pastorale}, pág. 2): bajo la coordinación del consejo pastoral, esta solicitud florece en numerosas obras de asistencia, de promoción y de caridad{]}.
					
					Deseo expresaros públicamente mi aprecio y también mi gratitud por cuanto hacéis en favor de los ancianos, de los jóvenes en dificultad, de los enfermos, de las familias necesitadas, como por el interés y la ayuda que ofrecéis a la misión de Matany en Uganda. 
					
					5. Y ahora permitidme que termine esta meditación sobre el Adviento con las palabras que sugiere el \textbf{Profeta Isaías}: «Fortaleced las manos débiles, robusteced las rodillas vacilantes, decid a los cobardes de corazón: sed fuertes, no temáis. Mirad a vuestro Dios\ldots{} El os salvará» (\emph{Is} 35, 3-4). 
					
					Que nunca falte en vuestra vida, queridos feligreses de la parroquia de la Natividad de Nuestro Señor Jesucristo, esta esperanza que su venida deposita en el corazón de cada hombre y en la que lo confirma saludablemente.
				\end{body}


			\subsubsection{Homilía (1983): La fuente de la alegría}
					\src{Visita a la Parroquia romana de San Camilo de Lellis. \\11 de diciembre de 1983.}
					\begin{body}
						1. El tercer domingo de Adviento es una apremiante invitación a la alegría. Precisamente, por las primeras palabras del texto latino de la \textquote{\textbf{antífona de entrada}}, este domingo se llama domingo de \emph{Gaudete} (cf. Flp 4, 4. 5). 
						
						El \textbf{profeta Isaías} invita a la naturaleza misma a manifestar con exuberante regocijo signos de júbilo: \textquote{El desierto y el yermo se regocijarán, se alegrará la estepa y florecerá} (Is 35, 1), porque muy pronto \textquote{contemplarán la gloria del Señor} (Is 35, 2).
						
						Es la alegría del Adviento que, en el fiel, está acompañada por la humilde e intensa invocación a Dios: \textquote{¡Ven!}. Es la súplica ardiente que se convierte en la respuesta del \textbf{Salmo responsorial} de la liturgia de hoy: \textquote{¡Ven, Señor, a salvarnos!}.
						
						2. La alegría del Adviento, típica de este domingo, encuentra su fuente en la respuesta que recibieron de Cristo \textbf{los mensajeros} a quienes envió \textbf{Juan el Bautista}. Este, mientras estaba en la cárcel, habiendo oído hablar de las obras de Jesús, le envió sus discípulos con la pregunta crucial, que esperaba una respuesta definitiva: \textquote{¿Eres tú el que ha de venir, o tenemos que esperar a otro?} (Mt 11, 3).
						
						Y esta fue la respuesta de Cristo: \textquote{Id a anunciar a Juan lo que estáis viendo y oyendo: los ciegos ven y los cojos andan; los leprosos quedan limpios y los sordos oyen; los muertos resucitan y a los pobres se les anuncia la Buena Noticia. ¡Y bienaventurado el que no se escandalice de mí!} (Mt 11, 4-6).
						
						Jesús de Nazaret, en su solemne respuesta a Juan el Bautista, se remite, evidentemente, al cumplimiento de las promesas mesiánicas. Promesas que se encuentran profetizadas en el libro de Isaías, que acabamos de escuchar en la \textbf{primera lectura}:
						
						\textquote{He aquí a vuestro Dios\ldots{} Viene en persona y os salvará. Entonces se despegarán los ojos de los ciegos, los oídos de los sordos se abrirán; entonces saltará el cojo como un ciervo, la lengua del mudo cantará. Porque han brotado aguas en el desierto, torrentes en la estepa\ldots{} Lo cruzará una calzada que llamarán Vía Sacra\ldots{} Por ella volverán los rescatados del Señor. Llegarán a Sión con cantos de júbilo: alegría sin límite en sus rostros. Los dominan el gozo y la alegría. Quedan atrás la pena y la aflicción} (Is 35, 4-10).
						
						Así pues, esto responde Cristo a Juan el Bautista: ¿Acaso no se están cumpliendo las promesas mesiánicas? Por lo tanto, ¡ha llegado el tiempo del primer Adviento!
						
						Nosotros hemos dejado atrás ya este tiempo y, al mismo tiempo, estamos siempre en él. Efectivamente, la liturgia lo hace presente cada año. Y esta es la fuente de nuestra alegría.
						
						3. Esta alegría del Adviento tiene una fuente propia más profunda. El hecho de que en Jesús se han cumplido las promesas mesiánicas es la demostración de que Dios es fiel a su palabra. Verdaderamente podemos repetir con el \textbf{Salmista}: \textquote{El Señor mantiene su fidelidad perpetuamente} (Sal 146, 6). El Adviento nos recuerda cada año el cumplimiento de las promesas mesiánicas que se refieren a Cristo, con el fin de orientar nuestras almas hacia estas promesas, cuya realización hemos recibido en Cristo y por Cristo. Estas promesas conducen al hombre a su destino último.
						
						En Cristo y por Cristo \textquote{el Señor mantiene su fidelidad perpetuamente}. En él y por él se abre, de generación en generación, el segundo Adviento, que es el \textquote{tiempo de la Iglesia}. Por Cristo la Iglesia vive el Adviento de cada día, esto es, la propia fe en la fidelidad de Dios, que \textquote{mantiene su fidelidad perpetuamente}. El Adviento vuelve a confirmar de este modo en la vida de la Iglesia la dimensión escatológica de la esperanza. Por esto \textbf{Santiago} nos recomienda: \textquote{Tened paciencia, hermanos, hasta la venida del Señor} (St 5, 7).
						
						4. En esta perspectiva, la liturgia de este tercer domingo de Adviento no es solo una invitación a la alegría, sino también a la valentía. Si, de hecho, debemos regocijarnos en la serena esperanza de la plenitud futura de los bienes mesiánicos, también debemos pasar con valentía por medio y por encima de la realidad temporal y transitoria, con la mirada y el interés dirigidos a lo que es eterno e inmutable.
						
						Esta valentía nace de la esperanza cristiana y, en cierto sentido, es la misma esperanza cristiana. La invitación a la valentía resuena en la \textbf{profecía del libro de Isaías}: \textquote{Fortaleced las manos débiles, afianzad las rodillas vacilantes, decid a los cobardes de corazón: sed fuertes, no temáis} (Is 35, 3).
						
						El Adviento, como dimensión estable de nuestra existencia cristiana, se manifiesta en esta esperanza, que comporta, al mismo tiempo, la valentía \textquote{escatológica} de la fe.
						
						Esta valentía --fuerza de la fe-- es, como dice el profeta, magnanimidad. Y es, a la vez, paciencia. Es similar a la paciencia del labrador que \textquote{aguarda pacientemente el fruto valioso de la tierra mientras recibe la lluvia temprana y tardía} (St 5, 7). Y añade Santiago: \textquote{Tened paciencia también vosotros, manteneos firmes, porque la venida del Señor está cerca} (St 5, 8).
						
						El \textbf{Evangelio} de este domingo nos presenta un ejemplo espléndido de esta paciente magnanimidad: Juan el Bautista. Jesús habla de él a la multitud en términos elogiosos: \textquote{¿Qué salisteis a contemplar en el desierto? ¿Una caña sacudida por el viento?\ldots{} ¿Un profeta? Sí, os digo, y más que un profeta} (Mt 11, 7-9). Y añade: \textquote{No ha nacido de mujer uno más grande que Juan el Bautista, aunque el más pequeño en el reino de los cielos es más grande que él} (Mt 11, 11).
						
						La fe magnánima y la valentía paciente de la esperanza nos abren a todos el camino para el Reino de los cielos.
						
						5. Esta fe magnánima y esta valentía paciente quiero desear hoy a todos los fieles. {[}\ldots{}{]}
						
						6. Queridos hermanos y hermanas \ldots{} que a través de esta celebración se renueve en todos vosotros la invitación a la alegría del Adviento que resuena en la liturgia de este domingo. Que se renueve, al mismo tiempo, la invitación a la esperanza magnánima, que tiene su fuente en la valentía sobrenatural de la fe.
						
						Cultivemos pacientemente la tierra de nuestra vida, como el labrador que \textquote{aguarda con paciencia el fruto valioso de la tierra}. ¡Este fruto se manifestará en la venida del Señor!
						
						Amén.
					\end{body}
						
			\subsubsection{Homilía (1986): Dialoguemos con Cristo}
				\src{Visita pastoral a la Parroquia romana de Santa María \textquote{Regina Mundi}. \\14 de diciembre de 1986.}
				
				\begin{body}
					1. \textquote{¿Eres tú el que ha de venir?} (Mt 11, 3).
					
					En la liturgia del este tercer domingo de Adviento resuena nuevamente esta \textbf{pregunta dirigida a Jesús por Juan el Bautista, a través de sus discípulos}: \textquote{¿Eres tú quien debe venir o debemos esperar a otro?}. En ese momento, Juan ya estaba en prisión y su actividad a orillas del río Jordán había sido brutalmente interrumpida. De hecho, estaba diciendo la verdad a todos; no había dudado en decirlo incluso al rey adúltero. Por eso terminó en prisión. Desde la prisión, dirigió una pregunta a Jesús, que puede sorprendernos en boca de Juan.
					
					En el Jordán, de hecho, ya había \textbf{dado testimonio de Cristo} al proclamar: \textquote{Este es el Cordero de Dios, el que quita el pecado del mundo} (Jn 1, 29). ¿Por qué, entonces, tal pregunta de quien reconoció la plenitud de los tiempos? No porque hubiera ninguna duda en él sobre el Redentor, a quien indicó como aquel que traía el perdón de Dios que tanto tiempo había esperado e invocado, sino porque había una cierta sorpresa en el Bautista. De hecho, él, que estaba experimentando la condición del prisionero destinado a morir, en cierto sentido está desconcertado de que Jesús traiga el juicio de Dios de una manera tan humilde e indefensa: comprendiendo con el delicado poder del amor lo que el Bautista había predicho con palabras fuertes: el Mesías \textquote{tiene el bieldo en la mano: aventará su parva, reunirá su trigo en el granero y quemará la paja en una hoguera que no se apaga} (Mt 3, 12).
					
					2. Jesús primero da la \textbf{respuesta}; luego, dirigiéndose a las personas presentes, manifiesta la grandeza de un hombre que lo anhelaba con todas las energías de la mente y el corazón.
					
					Para los discípulos de Juan, Jesús se refiere a algunas palabras del profeta Isaías sobre el Mesías, afirmando que esa descripción del tiempo mesiánico, como una era de salud, paz y alegría, se había cumplido en su obra salvadora.
					
					Estos textos de Isaías (Is 26, 19; 29, 18-19; 35, 5ss) son bien conocidos y a menudo se repiten en la liturgia. En ellos, este gran profeta predijo lo que, con la venida de Cristo, ahora está sucediendo ante los ojos de todos. La Palabra se hizo carne, el anuncio se hizo realidad. Aquí: \textquote{Los ciegos recuperan la vista, los lisiados caminan, los leprosos sanan, los sordos recuperan su audición, los muertos resucitan, la Buena Nueva se predica a los pobres}. Por lo tanto, Jesús afirma que todas estas \textquote{señales mesiánicas} son confirmadas por su actividad, y concluye diciendo: \textquote{Ve y dile a Juan lo que oyes y ves} (Mt 11, 5. 4).
					
					3. En la \textbf{primera parte del pasaje del Evangelio de hoy}, Jesús responde a la pregunta del Bautista, refiriéndose al testimonio de la Escritura: lo que el Antiguo Testamento anunció del Mesías se realiza en él, en Jesús de Nazaret. En la \textbf{segunda parte} de la perícopa, el Salvador de testimonio de Juan. Este es un \textquote{profeta} \ldots{} incluso \textquote{más que un profeta} (Mt 11, 9).
					
					Al afirmar esto, Jesús apela a la memoria de aquellos que lo escuchaban en ese momento: cuando fueron al Jordán para escuchar a Juan el Bautista no encontraron allí ni un bastón golpeado por el viento, ni un hombre envuelto en ropas suntuosas, como ¡los que están en los palacios de los reyes! (cf. Mt 11, 7-8). Allí encontraron a un profeta: un hombre que decía la verdad, y solo la verdad, en el nombre de Dios mismo. Esta verdad fue a veces dura y exigente. En el nombre del Reino, que se acercaba, en nombre del juicio de Dios, Juan exhortó a la penitencia a cuantos acudieron a él para escucharlo, para que al reconocerse a sí mismos como pecadores pudieran recibir al Mesías con un alma contrita.
					
					Quizás, como ya he mencionado, el propio Juan estaba un poco sorprendido de que las palabras y acciones de Jesús no fueran tan severas como las suyas. Pero, en primer lugar, Cristo es una buena noticia de salvación y revelación del amor misericordioso de Dios. Sin embargo, Juan no solo fue un profeta, servidor de la Palabra de Dios para ser proclamada a los hombres como un anuncio de paz, sino que también fue el Precursor, el enviado a \textquote{preparar el camino} a Cristo e indicarlo como una verdad gozosa ofrecida al hombre. Por esta razón, Jesús exclama: \textquote{En verdad os digo que entre los nacidos de mujer no ha surgido uno mayor que Juan el Bautista} (Mt 11, 11).
					
					4. A estas palabras, Jesús agrega otras que hacen reflexionar: \textquote{Sin embargo, el más pequeño en el Reino de los cielos es más grande que él}. Porque el Padre se complace plenamente en todos los que nacen de nuevo en el Espíritu, y los hijos de adopción son elevados a una familiaridad con Dios comparable a la que Juan pudo disfrutar en la vida terrenal.
					
					De hecho, debido a esta última frase, podemos afirmar que esta \textbf{segunda intervención} de Cristo no se refiere solo a Juan el Bautista, sino, en cierto sentido, a todo hombre. De hecho, desde que el reino de Dios vino al mundo, todos estamos sujetos a la medida de gracia que éste constituye para el hombre. E incluso un profeta tan grande como Juan, que sabía cómo reconocer al Mesías a orillas del Jordán, es valorado de acuerdo con esta medida en el reino de Dios.
					
					5. Hoy, tercer domingo de Adviento, la Iglesia recuerda estas palabras cuando en toda la liturgia resuena la verdad exultante de la \textbf{cercanía de Dios}: \textquote{¡El Señor está cerca!} (Flp 4, 5; Jn 5, 8).
					
					Realmente el Redentor está cerca. Él ha puesto su morada entre los hombres y puede ser encontrado en el camino de la experiencia humana, no solo nos enseña, sino que conversa con nosotros de una manera fraterna y santa: nos invita a un diálogo de salvación.
					
					El hombre de nuestro tiempo es muy sensible al \textquote{diálogo}, una palabra que se ha utilizado con frecuencia, pero de la que a veces se abusa en nuestros días. Sin embargo, la Iglesia ha insertado esta palabra en su lenguaje contemporáneo. Mi predecesor, el Papa Pablo VI, ha dedicado gran parte de su primera encíclica a este tema. Y, en particular, explicó en qué consiste el \textquote{diálogo de salvación}. Después de recordar que esta conversación salvífica se abrió espontáneamente por la iniciativa amorosa y benéfica de Dios, él enseña que nuestro hablar con el hombre también debe ser movido por un amor ferviente, libre y sin exclusiones.
					
					Él dice: \textquote{El diálogo de la salvación no obligó físicamente a nadie a acogerlo; fue un formidable requerimiento de amor, el cual si bien constituía una tremenda responsabilidad en aquellos a quienes se dirigió (Mt 11, 21), les dejó, sin embargo, libres para acogerlo o rechazarlo, adaptando inclusive la medida (Mt 12, 38ss.) y la fuerza probativa de los milagros (Mt 13, 13ss.) a las exigencias y disposiciones espirituales de sus oyentes, para que les fuese fácil un asentimiento libre a la divina revelación sin perder, por otro lado, el mérito de tal asentimiento} (Ecclesiam Suam, III, 36).
					
					6. El hombre también se acerca a Dios a través del diálogo, a través del diálogo con los hombres, cuando busca la verdad y la justicia en este camino. Se podría decir que el diálogo es parte del espíritu del Adviento, de la actitud del Adviento. El \textbf{Evangelio de hoy} lo pone de relieve con mucha fuerza.
					
					El hombre puede, de hecho debe, hacer \textbf{preguntas a Cristo}, incluso en la etapa actual de la historia: \textquote{¿Eres tú el que ha de venir?}. La respuesta de Cristo, preservada y ofrecida por la Iglesia, por un lado, será similar a la que recibieron los discípulos de Juan. Por otro lado, se adaptará a los problemas de la era en que vivimos. Precisamente de esta manera, el Concilio Vaticano II dio esta respuesta a la gente hoy.
					
					Cristo es aquel en quien los \textquote{signos mesiánicos} continúan encontrando confirmación. El que indica los caminos de la salvación. El diálogo de salvación, por lo tanto, puede y debe tenerse. A través de él, nuestra fe se profundiza y, al mismo tiempo, se convierte en una exhortación a actuar con madurez y responsabilidad.
					
					7. Sin embargo, recordemos que el centro de este diálogo, si es un verdadero diálogo de salvación, se encuentra en la oración. Incluso en la oración y, en primer lugar, en ella, el hombre \textquote{plantea preguntas}. Y en la oración, especialmente en la oración, encuentra la respuesta. La oración, más que cualquier otra realidad, pertenece al espíritu del Adviento, forma de la manera más completa la actitud del Adviento: \textquote{el Señor está cerca}.
					
					8. ¡Queridos hermanos y hermanas! Sed cada vez más una comunidad de oración, que especialmente con la liturgia se injerta en la comunión divina, manteniendo el deseo de Dios despierto y constante. De esta manera, madurará en vosotros una comunidad de vida redimida, que se sentirá empujada a practicar la caridad, que Cristo mismo nos recomendó (cf. Jn 13, 35), promoviendo al hombre según el plan de Dios.
					
					La existencia de una parroquia se desarrolla plenamente cuando avanza en la fe y progresa en la santidad de las buenas obras. No dudéis en poneros en el camino que indica la misericordia de Cristo, imitando su ejemplo, preservando su enseñanza y procediendo con seguridad en su paz. La parroquia es una casa de hermanos, que se hace bella y acogedora por la caridad.
					
					[\ldots{}]
					
					10. \textquote{¡El Señor está cerca!} He aquí que en la pequeñez de la casa de Nazaret, la humilde Virgen, llamada María, escucha las palabras de la Anunciación: \textquote{concebirás un hijo, lo darás a luz y lo llamarás Jesús. Será grande y será llamado Hijo del Altísimo, el Señor Dios le dará el trono de David su padre y reinará para siempre sobre la casa de Jacob y su reinado no tendrá fin} (Lc 1, 31-33).
					
					Estas palabras se han hecho realidad. La Virgen Madre entró en la historia del Adviento como la Sierva del Señor que dice: \textquote{Hágase en mi según tu palabra} (Lc 1, 38).
					
					{[}\ldots{}{]} La Iglesia, y vuestra parroquia en particular, llama a esta humilde Sierva del Señor \textquote{Reina del mundo}. Y la invoca en todas las necesidades de los hombres de hoy. María, nuestra Madre y Reina del mundo, ayuda y protege a todos los hijos de esta parroquia.
				\end{body}

			\subsubsection{Homilía (1989): ¿Cómo reconocer al Mesías?}
				\src{Visita a la Parroquia de San León I. \\17 de diciembre del 1989.}
				
				\begin{body}
					1. \textquote{¿Eres tú el que vendrá?} (Mt 11, 3).
					
					La \textbf{pregunta de los discípulos de Juan el Bautista} recorre todo el pasaje del Evangelio de este tercer domingo de Adviento, acertadamente llamado \textquote{de la alegría} (\emph{Gaudete}).
					
					El precursor, que terminó en prisión por dar un valiente testimonio de la verdad, había recibido el eco de las palabras y gestos de Jesús: eran palabras y gestos que contradecían las expectativas de un Mesías político y violento, como era el esperado por muchos en Israel. Esto les había hecho dudar; algunos incluso estaban escandalizados.
					
					\textquote{Eres tú\ldots{}. ¿O debemos esperar a otro?} La pregunta de los discípulos de Juan conserva toda su actualidad. Esta pregunta se la hace también hoy el nuevo Israel, la Iglesia, que espera la venida, especialmente la última, de su Señor. Se la hacen también de modo particular muchos hombres desanimados y desconcertados, que con un corazón sincero, buscan el camino de la salvación.
					
					2. La respuesta a esta pregunta no se puede dar en base a una simple lógica humana. La identificación del Mesías-salvador no es fácil. Solo pueden reconocerlo aquellos que tienen oídos para escuchar las palabras de Cristo y ojos para ver sus obras a la luz de la fe y como cumplimiento del plan salvador de Dios, ya anunciado por los profetas.
					
					\textbf{Isaías} había ofrecido un resumen particularmente elocuente de este proyecto de salvación. Jesús se refiere a la profecía de Isaías para revelar su verdadera identidad mesiánica y su misión: \textquote{Los ciegos recuperan la vista, los lisiados caminan, los leprosos sanan, los sordos oyen, los muertos resucitan, la Buena Noticia es anunciada a los pobres}.
					
					Con sus palabras, por lo tanto, Jesús ofrece las \textquote{señales} de la venida del Reino prometido y presenta las credenciales de su misión. Con su predicación, lleva a cumplimiento la liberación ya anunciada. De hecho, él es el Mesías-Siervo, salvador de todo hombre y de todos los hombres, de los pobres, de los que sufren, de los marginados sobre todo. A través de sus palabras y sus gestos, el Reino de Dios \textquote{viene} para la salvación y la alegría de los pobres que reconocen en él \textquote{la gloria y la magnificencia} de Dios y son así salvados. Para ellos, la liberación será un \textquote{nuevo éxodo}, un \textquote{camino sagrado} que se recorrerá a partir de ahora en una actitud de conversión y fidelidad a su palabra, como Buena Noticia de esperanza y de alegría.
					
					3. La \textbf{respuesta dada por Jesús} es un motivo de confianza y estímulo para la vida de toda persona de buena voluntad. Y lo es, en particular, para esta comunidad parroquial\ldots{} Aquí también hay enfermos, marginados, pobres, a quienes Jesús quiere traer la Buena Noticia de la salvación. Lo hace a través de sus ministros: su párroco\ldots{} y sus sacerdotes colaboradores, a quienes extiendo mi cordial saludo. Lo hace a través del {[}Obispo{]} \ldots{} Lo hace a través de la persona del Papa, que ha venido entre vosotros para expresar su afecto y su preocupación pastoral por los problemas que afligen a su comunidad.
					
					Animo a todos {[}los fieles{]} a asumir su propia parte de responsabilidad y a perseverar en esta colaboración, sabiendo que Dios os recompensará. Insto a dar prioridad, a la luz del pasaje del \textbf{Evangelio} escuchado hace un momento, a la atención a los últimos. Es a ellos en primer lugar, a los \textquote{pobres}, a quienes el Señor Jesús quiere traer las Buena Nueva a través de vosotros. Él estará a vuestro lado, en el cumplimiento de esta tarea, apoyándoos con su gracia\ldots{}
					
					4. La situación de la ciudad en la que vivimos\ldots{}, a veces se presenta, para usar las palabras del \textbf{profeta Isaías}, como un \textquote{desierto}, una \textquote{tierra árida}, difícil de cultivar y resistente a la siembra del Evangelio. Hay {[}en nuestras ciudades{]} \textquote{cansados de corazón}, que han perdido el camino de la verdad y de la vida; muchas \textquote{manos débiles}, incapaces de hacer el bien; muchas \textquote{rodillas vacilantes} en el camino de seguir a Cristo. Con las palabras del profeta, Dios nos invita a no desanimarnos y nos exhorta a esperar: \textquote{Ánimo, no tengas miedo; aquí está tu Dios \ldots{} ¡Viene a salvarte!}.
					
					Sí, hermanos y hermanas, viene el Señor; de hecho está aquí entre nosotros. Los signos de su presencia salvadora ya son visibles en nuestra ciudad. De hecho, hay muchas iniciativas puestas en marcha en la comunidad eclesial para anunciar la Buena Nueva del Evangelio a los pobres. Hay muchas obras de caridad y servicio creadas para ofrecer a los enfermos, los que sufren y los marginados una liberación integral.
					
					Sin embargo, la salvación que Jesús viene a traer es un don que todavía no ha llegado a todos; muchos \textquote{pobres} que tenemos con nosotros aún no han aceptado el anuncio de las Buena Nueva y aún no han sido liberados del pecado ni de todo aquello que los humilla y los excluye de una coexistencia humana fraternal y solidaria; muchos \textquote{escandalizados} se han distanciado de Cristo y de la Iglesia.
					
					Por lo tanto, es necesario aumentar la misión de evangelización y promoción humana, para abrir las puertas del Reino de Dios a todos, para que entre Jesucristo\ldots{}
					
					6. {[}\ldots{}{]} Los creyentes saben que la salvación ofrecida por Cristo no termina en una dimensión exclusivamente terrenal y temporal, es trascendente, escatológica y tendrá su cumplimiento definitivo en el segundo advenimiento del Señor. Ciertamente tiene su comienzo aquí y ahora, pero solo al final tendrá su realización completa.
					
					Por esta razón, la Iglesia \ldots{} participa activamente en la promoción del hombre. Lo hace con esa actitud de esperanza y paciencia activa, como nos pedía \textbf{Santiago en la segunda lectura} de la Misa, confiando en que el desierto florecerá y la tierra dará sus frutos.
					
					Lo hace tratando de acompañar sus obras con el testimonio valiente y coherente de la vida. Como \textbf{Juan el Bautista}.
					
					Lo hace dirigiendo los esfuerzos de todos aquellos en la verdadera dirección, y hay muchos que, como individuos o en grupos, trabajan por la justicia, la solidaridad y la paz. Para que todos los hombres, pero especialmente los pobres y los oprimidos, vean la gloria del Señor, la magnificencia de nuestro Dios y se salven.
					
					\textquote{¡Ánimo, no temas, ---os repite hoy la Iglesia con el \textbf{profeta Isaías}--- aquí está tu Dios \ldots{} Él viene a salvarte}.
					
					Sí, ¡ven, Señor Jesús, ven y sálvanos!
				\end{body}
			
			\subsubsection{Homilía (1992): liberación definitiva y verdadera}
			
				\src{Visita pastoral a la Parroquia de San Hugo, Obispo. \\13 de diciembre del 1992.}
				
				\begin{body}
					¡Queridos hermanos y hermanas de la parroquia de San Hugo!
					
					1. \textquote{Fortaleced vuestros corazones} (St 5, 8). Con el tercer domingo de Adviento, que estamos celebrando, hemos llegado al \textquote{corazón} de ese itinerario espiritual que nos llevará al pie de la Santa Gruta, para contemplar, adorar y dar gracias al Verbo de Dios, que se ha hecho hombre para la salvación de toda la humanidad. Y la liturgia de hoy, como para apoyarnos en el exigente viaje de preparación y conversión, está impregnada de una invitación a confiar y esperar. La expectativa del creyente, de hecho, no es en vano y la promesa de Dios es verdadera.
					
					El \textbf{apóstol Santiago} nos recordó esto en la segunda lectura: \textquote{Fortaleced vuestros corazones, porque la venida del Señor está cerca} (St 5, 8). Sus palabras se hacen eco de las del \textbf{profeta Isaías}, dirigidas al pueblo judío durante el duro exilio en la tierra de Babilonia. \textquote{Sed fuertes, no temáis. ¡He aquí vuestro Dios!} (Is 35, 4). Al igual que en tiempos de Moisés, Dios intervino para liberar a su pueblo de la esclavitud egipcia y, a través del desierto, llevarlos a la tierra prometida, así también ahora está dispuesto a hacer maravillas en favor de su pueblo, restaurando su libertad. \textquote{Él viene a salvarnos} (Is 36, 4).
					
					\textbf{Isaías} describe un camino llano, por medio del cual los deportados regresarán exultantes: verán la gloria y la magnificencia del Señor. Los desanimados y desconcertados no deben desesperarse porque, como dice el profeta, \textquote{tu Dios \ldots{} viene a salvarte}; y agrega: \textquote{Entonces se abrirán los ojos de los ciegos y se abrirán los oídos de los sordos. Entonces el cojo saltará como un ciervo, la lengua del mudo gritará de alegría} (Is 35, 5-6). En esta página, tan rica en simbolismo, la Iglesia ve una clara profecía mesiánica, que supera y perfecciona la inmediatamente histórica. De hecho, para el hombre, la verdadera y definitiva liberación de toda esclavitud y opresión es solo la lograda por Jesús, en el misterio pascual de su muerte y resurrección.
					
					2. \textquote{¿Eres tú el que ha de venir?} (Mt 11, 3), los discípulos de Juan el Bautista, encarcelados por el rey perseguidor Herodes Antipas, le \textbf{preguntan al Mesías}. \textquote{¿Eres tú quien tiene que venir, o tenemos que esperar a otro?} Una vez más, el Precursor abre el camino al Señor y le ofrece otra oportunidad para manifestarse a los hombres. \textbf{Jesús responde} con las mismas palabras de Isaías: \textquote{Ve e informa a Juan lo que oís y veis: los ciegos recuperan la vista, los lisiados caminan, los leprosos se curan, los sordos oyen, los muertos resucitan y a los pobres es anunciada la Buena Noticia} (Mt 11, 5).
					
					Días antes, en la sinagoga de Nazaret, Jesús se había aplicado otro pasaje del profeta Isaías: \textquote{El Espíritu del Señor está sobre mí; por eso me ha consagrado con la unción y me ha enviado a anunciar el Evangelio a los pobres, a proclamar la liberación a los prisioneros y la vista a los ciegos; a liberar a los oprimidos} (Lc 4,18). El Redentor hace referencia a la autoridad del gran profeta del Antiguo Testamento para demostrar su mesianismo. Y esta vez lo hace para eliminar cualquier duda tanto de los discípulos de Juan como de las multitudes que ahora lo siguen asiduamente considerándolo el Maestro. Para luego \textbf{dar testimonio del Precursor}, que ha terminado su predicación, pero aún no su misión, expresa un elogio sin paralelo hacia él. Lo define \textquote{más que un profeta}, lo indica como el mensajero enviado para preparar el camino al Mesías, lo compara con Elías y resume su elogio con esta afirmación solemne: \textquote{En verdad te digo: entre los nacidos de mujer no hay uno mayor que Juan el Bautista} (Mt 11, 11).
					
					3. \textquote{El más pequeño en el reino de los cielos es más grande que él} (Mt 11, 11). Sin embargo, la alabanza extraordinaria es seguida por una \textbf{nota aparentemente misteriosa}: \textquote{El más pequeño en el reino de los cielos es más grande que él}. Puede parecer contradictorio y, en cambio, expresa una verdad fundamental. De hecho, el Señor tenía la intención de contrastar el tiempo de preparación de la salvación, concluido y casi simbolizado por el Bautista, con el de su realización definitiva, inaugurada por el mismo Cristo. Con el advenimiento del Redentor, la espera termina y la salvación destinada para cada hombre se inaugura, sin restricciones. Esto explica la paradoja de las palabras de Jesús sobre Juan el Bautista. Los prodigiosos signos de curación hechos por Cristo en los enfermos adquieren así un valor simbólico precioso, es decir, Cristo trae consigo el auténtico don de sanar las almas y de otorgarnos una nueva vida. Las curaciones de Jesús son por tanto signos de salvación eterna.
					
					4. Queridos hermanos y hermanas, aquí está el profundo significado de la Navidad del Señor para la cual nos estamos preparando. Jesús aparece ante el mundo como un niño pequeño; a través de la pobreza, la simplicidad y la humildad de su nacimiento, quiere llevarnos a todos a comprender su arcano plan salvífico. Después de dar el ejemplo, Jesús predicó los caminos del reino divino; después de entregarse a sí mismo para redimir a la humanidad, Él, resucitado, funda la Iglesia al confiarle la verdad eterna que se transmitirá y la gracia renovadora que se difundirá gratuitamente. Desde entonces, el pueblo de Dios, rico en carismas y ministerios puestos al servicio de la única fe y el único Señor, se extiende sobre la tierra en múltiples comunidades particulares, diócesis y parroquias precisamente para proclamar y presenciar este mensaje de salvación del cual es depositario.
					
					El pueblo de Dios es muy consciente de que debe continuar la obra redentora del Salvador entre los hombres, proclamando su Evangelio a toda criatura. Es en la parroquia donde se genera la nueva existencia cristiana a través de la gracia bautismal; en ella participamos en la existencia divina a través de los sacramentos; en ella crecemos en la fe gracias a una catequesis permanente, en ella se cultivan las vocaciones al orden sagrado, el matrimonio y a la vida consagrada. Es en la parroquia donde florece la caridad para todos. De hecho, la comunidad parroquial está llamada a ser una \textquote{escuela de caridad} privilegiada, donde se aprende a acoger y amar a cada persona sin discriminación, distinción o preferencias, ofreciendo el don de las obras de misericordia a los más necesitados. He aquí un resumen del papel de la parroquia en la comunidad cristiana, comenzando con los primitivos cristianos, alrededor de los Apóstoles, hasta hoy con esta parroquia de san Hugo, recientemente inaugurada.
					
					{[}\ldots{}{]}
					
					6. \textquote{La venida del Señor está cerca}.
					
					Sí, queridos hermanos y hermanas, la venida del Señor está cerca porque la Navidad, el nacimiento de Jesús en el vientre virginal de María, está a las puertas; pero también está cerca porque la vida, incluso la más larga, está destinada a terminar en el tiempo para abrirse a la eternidad.
					
					El tiempo es corto (cf. 1 Cor 7, 29). He aquí, ahora el momento favorable, ahora el día de salvación (cf. 2 Cor 6, 2).
					
					Aprovechemos el tiempo, como un tesoro, por el bien de nuestras almas.
					
					Jesús viene ¡Ven, Señor, a salvarnos!
					
					Amén.
				\end{body}					

			\subsubsection{Homilía (1995): Fidelidad perpetua}
			
				\src{Visita pastoral a la Parroquia de Santa María, Reina de los Apóstoles. \\17 de diciembre de 1995.}
				
				\begin{body}
					Rorate caeli desuper!
					
					¡Queridos hermanos y hermanas!
					
					1. Las lecturas, que hemos escuchado en la liturgia de hoy, ilustran cómo la realidad del Adviento ya está inscrita en la misma experiencia de la naturaleza. El Adviento, de hecho, es el tiempo de espera. Santiago habla del agricultor que \textquote{espera pacientemente el precioso fruto de la tierra hasta que recibe las lluvias de otoño y las de primavera} (St 5, 7). Estas palabras se pueden relacionar de alguna manera con las del \textbf{profeta Isaías}, proclamadas en la \textbf{primera lectura}: \textquote{El desierto y el yermo se regocijarán, se alegrará la estepa y florecerá, germinará y florecerá como flor de narciso, festejará con gozo y cantos de júbilo} (Is 35, 1-2). Para los israelitas, que vivían al borde del desierto, la expectativa de la cosecha era motivo de especial preocupación. Después de todo, ¿no es este el contenido de la invocación de Adviento: \textquote{Rorate caeli desuper!}? La expectativa del Mesías es, por lo tanto, similar a la del agricultor: \textquote{Et nubes pluant Iustum}, \textquote{Cielos, destilad desde lo alto la justicia, las nubes la derramen, se abra la tierra y brote la salvación, y con ella germine la justicia} (cf. Is 45, 8).
					
					2. En este contexto de expectativa ansiosa, la liturgia de hoy nos lleva a afirmar una vez más que el hombre y Dios están en el centro del Adviento: el hombre que espera la venida de Dios y Dios que viene al encuentro del hombre. El contenido de la expectativa del hombre es la salvación que solo le puede llegar de Dios. El Mesías prometido, que viene a la tierra en la noche de Belén, es el Salvador del mundo, es el que libera al hombre del mal, orientándolo hacia el bien y la felicidad.
					
					El \textbf{Salmo Responsorial} canta la fidelidad de Dios, que es perpetua: Él hace justicia a los oprimidos, da pan a los hambrientos de pan, libera a los prisioneros, devuelve la vista a los ciegos, levanta a los que han caído, ama a los justos, protege a los extranjeros, vela por los huérfanos y las viudas (cf. Sal 145,7-10).
					
					3. Las palabras del salmista están relacionadas con lo que expresó el \textbf{profeta Isaías en la primera lectura}: \textquote{Entonces se abrirán los ojos de los ciegos y se abrirán los oídos de los sordos. Entonces el cojo saltará como un ciervo, la lengua del mudo gritará de alegría} (Is 35, 5-6). Son signos de una gran conversión, de un retorno, que se logrará con la obra del Redentor. El Profeta anuncia: \textquote{Los redimidos del Señor volverán y vendrán a Sión con alegría; la felicidad perenne brillará en su cabeza; la alegría y la felicidad los seguirán y la tristeza y las lágrimas huirán} (Is 35, 10).
					
					Y cuando los \textbf{discípulos de Juan el Bautista fueron a Cristo para preguntarle}: \textquote{¿Eres tú quien debe venir o debemos esperar a otro?}, Jesús responde: \textquote{Id y decidle a Juan lo que veis y oís: los ciegos recuperan la vista, los lisiados caminan, los leprosos son sanados, los sordos recuperan su audición, los muertos resucitan, la Buena Nueva se predica a los pobres y bienaventurado el que no escandaliza de mi} (Mt 11, 3-6). Por lo tanto, Jesús de Nazaret confirma inequívocamente que él mismo es el cumplimiento de las expectativas mesiánicas de Israel. De esta manera, actúa como mediador entre las expectativas del hombre y la voluntad eterna de Dios de corresponder plenamente a las necesidades de la humanidad.
					
					4. Refiriéndose al mensaje de Juan el Bautista y la consiguiente respuesta, Jesús habla a la multitud de la persona del Bautista. El bautista no es un hombre que duda. La pregunta que hizo surge de la profundidad de su vocación profética y tiende a obtener de Cristo mismo la confirmación de esa verdad divina de la que había dado testimonio a orillas del Jordán: la verdad confirmada definitivamente con el sacrificio de su propia vida.
					
					Y \textbf{Jesús da testimonio de la misión especial del Bautista}, como si quisiera pagar una \textquote{deuda de gratitud} hacia el Precursor. \textquote{En verdad os digo que no ha nacido de mujer uno más grande que Juan el Bautista} (Mt 11, 11). La multitud se encontró no solo con un profeta, sino \textquote{más que un profeta} (cf. Mt 11, 9). Con estas palabras, Cristo da testimonio de Juan y, en cierto sentido, imprime un sello mesiánico en toda su misión profética.
					
					5. La figura de Juan el Bautista se repite varias veces en las lecturas del Adviento y confiere un significado especial a la liturgia de este período. Sí, el Adviento es tiempo de espera de la Navidad del Señor, de su entrada en la existencia terrena en un clima de alegría y de paz. Juan el Bautista, en un cierto sentido, hace revivir, con treinta años de diferencia, la experiencia del Adviento, en el momento en el que Jesús de Nazaret empieza su vida pública. Precisamente, la realización concreta de su misión salvífica es la que manifiesta el sentido definitivo de la Noche de Navidad.
					
					El Mesías cumplirá la \textbf{profecía de Isaías} con su misión y continuará repitiendo todo lo que dijo a los enviados de Juan: \textquote{Bienaventurado el que no se escandalice de mi} (Mt 11,6). Hoy, Cristo repite lo mismo a los hombres del siglo XX que ahora está llegando a su fin; a nosotros, reunidos en este templo; a la Iglesia y a toda la humanidad. A medida que avanzamos hacia el final del segundo milenio cristiano, estas palabras continúan resonando con particular claridad y reviviendo los corazones de los hombres en este punto de inflexión de nuestra época.
					
					6. ¡Queridos hermanos y hermanas de la parroquia de Santa Maria Regina Apostolorum! Hoy el profeta Isaías nos dirige una invitación apremiante: \textquote{¡Ánimo! no tengas miedo; he aquí tu Dios \ldots{} Él viene a salvarte} (Is 35, 4). Os repito estas palabras, ahora unos días antes de la gran solemnidad de Navidad, a vosotros que vivís en esta parroquia \ldots{}
					
					{[}\ldots{}{]}
					
					8. Queridos hermanos y hermanas \ldots{} Se puede decir que cada templo es un signo del Adviento, de ese Adviento que está inscrito en toda la creación. En la casa del Señor, ubicada entre los hombres, este encuentro se lleva a cabo de manera sacramental.
					
					¡Espero que este templo sea para vosotros un lugar de encuentros frecuentes con Dios! Venid aquí, dentro de estos muros, para hacerle, como los discípulos de Juan, preguntas a Cristo. Saldréis de aquí tranquilizados, llevando con vosotros la respuesta dada por Jesús a los enviados de Juan: \textquote{Bienaventurado el que no se escandaliza de mi}.
					
					9. ¡Espero que este templo esté al servicio de vuestra fe, vuestra esperanza y vuestra caridad! Preparaos, aquí en la tierra, para el encuentro con Dios, el destino definitivo de cada hombre. \textbf{Santiago} nos exhortaba así: \textquote{Tened paciencia, hermanos \ldots{} fortaleced vuestros corazones, porque la venida del Señor está cerca \ldots{} he aquí, el juez está a la puerta} (St 5, 7-10). Este juez es el Salvador del mundo y su juicio es un juicio salvador.
					
					Que este templo os ayude a poneros en contacto con el Dios que juzga mediante la verdad de la salvación. De hecho, Dios \textquote{quiere que todos los hombres se salven y lleguen al conocimiento de la verdad} (1 Tim 2, 4). ¡Que este deseo de Dios esté presente también en este templo como una señal de bendición divina para todos los que entren en él!
					
					Amén.
				\end{body}

			\subsubsection{Homilía (1998): El motivo de nuestra alegría}
				\src{Visita pastoral a la Parroquia romana de Santa Julia Billiart. \\13 de diciembre de 1998.}
				
				\begin{body}
					1. \textquote{Alegraos siempre en el Señor; os lo repito: alegraos. El Señor está cerca} (\emph{Antífona de entrada}).
					
					De esta apremiante invitación a la alegría, que caracteriza la liturgia de hoy, recibe su nombre el tercer domingo de Adviento, llamado tradicionalmente \emph{domingo \textquote{Gaudete}}. En efecto, ésta es la primera palabra en latín de la misa de hoy: \textquote{\emph{Gaudete}}, es decir, alegraos porque el Señor está cerca.
					
					El \textbf{texto evangélico} nos ayuda a comprender el motivo de nuestra alegría, subrayando el gran misterio de salvación que se realiza en Navidad. El evangelista san Mateo nos habla de Jesús, \textquote{el que ha de venir} (\emph{Mt} 11, 3), que se manifiesta como el Mesías esperado mediante su obra salvífica: \textquote{Los ciegos ven y los cojos andan, (\ldots{}) y se anuncia a los pobres la buena nueva} (\emph{Mt} 11, 5). Viene a consolar, a devolver la serenidad y la esperanza a los que sufren, a los que están cansados y desmoralizados en su vida.
					
					También en nuestros días son numerosos los que están envueltos en las tinieblas de la ignorancia y no han recibido la luz de la fe; son numerosos los cojos, que tienen dificultades para avanzar por los caminos del bien; son numerosos los que se sienten defraudados y desalentados; son numerosos los que están afectados por la lepra del mal y del pecado y esperan la salvación. A todos ellos se dirige la \textquote{buena nueva} del Evangelio, encomendada a la comunidad cristiana. La Iglesia, en el umbral del tercer milenio, proclama con vigor que Cristo es el verdadero liberador del hombre, el que lleva de nuevo a toda la humanidad al abrazo paterno y misericordioso de Dios.
					
					2. \textquote{Sed fuertes, no temáis. Vuestro Dios va a venir a salvaros} (\emph{Is} 35, 4).
					
					[\ldots{}] Con gran afecto, hago mías las palabras del \textbf{profeta Isaías} que acabamos de proclamar: \textquote{Sed fuertes, no temáis. (\ldots{}) El Señor va a venir a salvaros}. Estas palabras expresan mi mejor deseo, que renuevo a todos aquellos con quienes Dios me permite encontrarme en cualquier parte del mundo. Resumen lo que quiero repetiros también a vosotros esta mañana. Mi presencia desea ser una invitación a tener valor, a perseverar dando razón de la esperanza que la fe suscita en cada uno de vosotros.
					
					\textquote{Sed fuertes}. No temáis las dificultades que se han de afrontar en el anuncio del Evangelio. Sostenidos por la gracia del Señor, no os canséis de ser apóstoles de Cristo en nuestra ciudad que, aunque se ciernen sobre ella los numerosos peligros de la secularización típicos de las metrópolis, mantiene firmes sus raíces cristianas, de las que puede recibir la savia espiritual necesaria para responder a los desafíos de nuestro tiempo. Los frutos positivos que la misión ciudadana está produciendo, y por los que damos gracias al Señor, son estímulos para proseguir sin vacilación la obra de la nueva evangelización.
					
					4. \textquote{El Espíritu del Señor \ldots{} me ha enviado para anunciar la Buena Nueva a los pobres}.
					
					Estas palabras del \emph{\textbf{Aleluya}} reflejan bien el clima de la misión (\ldots{}) en la que todos los cristianos son impulsados a llevar el Evangelio a los diversos ambientes de la ciudad\ldots{} Como recuerda la Escritura: \textquote{Un hermano ayudado por su hermano es como una plaza fuerte} (cf. \emph{Pr} 18, 19).
					
					[\ldots{}] deseo de corazón que todos los cristianos sientan la urgencia de transmitir a los demás, especialmente a los jóvenes, los valores evangélicos que favorecen la instauración de la \textquote{civilización del amor}.
					
					5. \textquote{Tened paciencia (\ldots{}) hasta la venida del Señor} (\emph{St} 5, 7). Al mensaje de alegría, típico de este domingo \textquote{Gaudete}, la liturgia une la \textbf{invitación a la paciencia} y a la espera vigilante, con vistas a la venida del Salvador, ya próxima.
					
					Desde esta perspectiva, es preciso saber aceptar y afrontar con alegría las dificultades y las adversidades, esperando con paciencia al Salvador que viene. Es elocuente el ejemplo del labrador que nos propone \textbf{la carta del apóstol Santiago}: \textquote{aguarda paciente el fruto valioso de la tierra, mientras recibe la lluvia temprana y tardía}. \textquote{Tened paciencia también vosotros ---añade---; manteneos firmes, porque la venida del Señor está cerca} (\emph{St} 5, 7-8).
					
					Abramos nuestro espíritu a esa invitación; avancemos con alegría hacia el misterio de la Navidad. María, que esperó en silencio y orando el nacimiento del Redentor, nos ayude a hacer que nuestro corazón sea una morada para acogerlo dignamente. Amén.
				\end{body}
			
			\subsubsection{Homilía (2001): Alegría de la comunión}
				\src{Visita pastoral a la Parroquia romana de Santa María Josefa del Corazón de Jesús. \\16 de diciembre del 2001.}
				
				\begin{body}
					1. \textquote{El desierto y el yermo se regocijarán, se alegrarán el páramo y la estepa} (\emph{Is} 35, 1).
					
					Una insistente invitación a la alegría caracteriza la liturgia de este tercer domingo de Adviento, llamado domingo \textquote{\emph{Gaudete}}, porque precisamente \textquote{\emph{Gaudete}} es la primera palabra de la antífona de entrada. \textquote{Regocijaos}, \textquote{alegraos}. Además de la vigilancia, la oración y la caridad, el Adviento nos invita a la alegría y al gozo, porque ya es inminente el encuentro con el Salvador.
					
					En la \textbf{primera lectura}, que acabamos de escuchar, encontramos un verdadero himno a la alegría. El profeta Isaías anuncia las maravillas que el Señor realizará en favor de su pueblo, liberándolo de la esclavitud y conduciéndolo de nuevo a su patria. Con su venida, se realizará un éxodo nuevo y más importante, que hará revivir plenamente la alegría de la comunión con Dios.
					
					Para los que están desanimados y han perdido la esperanza resuena la \textquote{buena nueva} de la salvación: \textquote{Gozo y alegría seguirán a los rescatados del Señor. Pena y aflicción se alejarán} (cf. \emph{Is} 35, 10).
					
					2. \textquote{Sed fuertes, no temáis. Mirad a vuestro Dios. (\ldots{}) Viene a salvaros} (\emph{Is} 35, 4). ¡Cuánta confianza infunde esta profecía mesiánica, que permite vislumbrar la verdadera y definitiva liberación, realizada por Jesucristo. En efecto, en la \textbf{página evangélica} que ha sido proclamada en nuestra asamblea, Jesús, respondiendo a \textbf{la pregunta} de los discípulos de Juan Bautista, se aplica a sí mismo lo que había afirmado Isaías: él es el Mesías esperado: \textquote{Id a anunciar a Juan -dice- lo que estáis viendo y oyendo: los ciegos ven y los inválidos andan; los leprosos quedan limpios y los sordos oyen; los muertos resucitan, y a los pobres se les anuncia la buena nueva} (\emph{Mt} 11, 4-5).
					
					Aquí radica la razón profunda de nuestra alegría: en Cristo se cumplió el tiempo de la espera. Dios realizó finalmente la salvación para todo hombre y para la humanidad entera. Con esta íntima convicción nos preparamos para celebrar la fiesta de la santa Navidad, acontecimiento extraordinario que vuelve a encender en nuestro corazón la esperanza y el gozo espiritual\ldots{}
					
					6. \textquote{Tened paciencia, hermanos, hasta la venida del Señor} (\emph{St} 5, 7).
					
					El Adviento nos invita a la alegría, pero, al mismo tiempo, nos exhorta a \textbf{esperar con paciencia} la venida ya próxima del Salvador. Nos exhorta a no desalentarnos, superando todo tipo de adversidades, con la certeza de que el Señor no tardará en venir.
					
					Esta paciencia vigilante, como subraya el \textbf{apóstol Santiago} en la segunda lectura, favorece la consolidación de sentimientos fraternos en la comunidad cristiana. Al reconocerse humildes, pobres y necesitados de la ayuda de Dios, los creyentes se unen para acoger a su Mesías que está a punto de venir. Vendrá en el silencio, en la humildad y en la pobreza del pesebre, y a quien le abra el corazón le traerá su alegría.
					
					Por tanto, avancemos con alegría y generosidad hacia la Navidad. Hagamos nuestros los sentimientos de María, que esperó en oración y en silencio al Redentor y preparó con cuidado su nacimiento en Belén. ¡Feliz Navidad!
				\end{body}

		\subsection{Benedicto XVI, papa}

			\subsubsection{Homilía (2007)} 

				\src{Visita pastoral a la Parroquia Romana de Santa María del Rosario en los Mártires Portuenses. \\16 de diciembre de 2007.}
				
				\begin{body}
					\emph{Queridos hermanos y hermanas:}
					
					\textquote{Estad siempre alegres en el Señor. Os lo repito: estad alegres. El Señor está cerca} (\emph{Flp} 4, 4-5).\\ Con esta invitación a la alegría comienza la antífona de entrada de la santa misa en este tercer domingo de Adviento, que precisamente por eso se llama domingo \textquote{\emph{Gaudete}}. En verdad, todo el Adviento es una invitación a alegrarse, porque \textquote{el Señor viene}, porque viene a salvarnos.
					
					Durante estas semanas, casi diariamente, nos consuelan las palabras del profeta Isaías, dirigidas al pueblo judío desterrado en Babilonia después de la destrucción del templo de Jerusalén, el cual había perdido la esperanza de volver a la ciudad santa en ruinas. \textquote{A los que esperan en el Señor él les renovará el vigor ---asegura el profeta---, subirán con alas como de águilas, correrán sin fatigarse y andarán sin cansarse} (\emph{Is} 40, 31). Y también: \textquote{Regocijo y alegría los acompañarán. Pena y aflicción se alejarán} (\emph{Is} 35, 10).
					
					La liturgia de Adviento nos repite constantemente que debemos despertar del sueño de la rutina y de la mediocridad; debemos abandonar la tristeza y el desaliento. Es preciso que se alegre nuestro corazón porque \textquote{el Señor está cerca}.
					
					Hoy tenemos un motivo ulterior para alegrarnos, queridos fieles de la parroquia de \emph{Santa María del Rosario en los Mártires Portuenses,} y es la dedicación de vuestra nueva iglesia parroquial, que surge en el mismo lugar donde mi amado predecesor el siervo de Dios Juan Pablo II celebró, el 8 de noviembre de 1998, la santa misa con ocasión de su visita pastoral a vuestra comunidad.
					
					La solemne liturgia de la dedicación de este templo constituye una ocasión de intenso gozo espiritual para todo el pueblo de Dios que vive en esta zona. Y de buen grado me uno también yo a vuestra satisfacción por tener por fin una iglesia acogedora y funcional. El lugar en que está construida evoca un pasado de testimonios cristianos resplandecientes. En efecto, precisamente aquí, en las cercanías, se encuentran las catacumbas de Generosa, donde según la tradición fueron sepultados tres hermanos, Simplicio, Faustino y Beatriz, víctimas de la persecución desencadenada en el año 303, y cuyos restos mortales fueron conservados, en parte, en Roma en la iglesia de San Nicolás in Carcere y en Monte Savello, y, en parte, en Fulda, Alemania, ciudad que desde el siglo VIII, gracias a que san Bonifacio llevó allí las reliquias, honra a los mártires portuenses como sus copatronos.
					
					A este respecto, saludo al representante del obispo de Fulda, y también a mons. Carlo Liberati, arzobispo-prelado de Pompeya, santuario mariano con el que vuestra parroquia mantiene un hermanamiento espiritual.
					
					La dedicación de esta iglesia parroquial cobra un significado muy particular para vosotros que vivís en este barrio. Los jóvenes mártires que entonces murieron por dar testimonio de Cristo, ¿no son un fuerte estímulo para vosotros, cristianos de hoy, a perseverar en el seguimiento fiel de Jesucristo? Y la protección de la Virgen del Santo Rosario, ¿no os pide ser hombres y mujeres de fe profunda y de oración, como lo fue ella?
					
					También hoy, aunque sea con formas diversas, el mensaje salvífico de Cristo encuentra oposición y los cristianos, de otras maneras y no menos que ayer, están llamados a dar razón de su esperanza, a testimoniar ante el mundo la verdad de Cristo, el único que salva y redime. Por consiguiente, esta nueva iglesia ha de ser un espacio privilegiado para crecer en el conocimiento y en el amor de Cristo, a quien dentro de pocos días acogeremos en la alegría de su nacimiento como Redentor del mundo y Salvador nuestro.
					
					Aprovechando la dedicación de esta nueva y hermosa iglesia, quiero dar las gracias a todos los que han contribuido a construirla. Sé que la diócesis de Roma se está esforzando con empeño, desde hace muchos años, por lograr que en cada barrio de esta ciudad en crecimiento constante haya complejos parroquiales adecuados.
					
					Saludo y expreso mi gratitud, en primer lugar, al cardenal vicario y al obispo auxiliar Ernesto Mandara, secretario de la Obra romana para la conservación de la fe y la provisión de nuevas iglesias en Roma. Os saludo y os manifiesto mi agradecimiento en particular a vosotros, queridos feligreses, que de diversas maneras os habéis comprometido en la realización de este centro parroquial, que se añade a los más de cincuenta que ya funcionan gracias al notable esfuerzo económico de la diócesis, de tantos fieles y ciudadanos de buena voluntad, y a la colaboración de las instituciones públicas. En este domingo, dedicado precisamente al apoyo de esa meritoria obra, pido a todos que prosigan ese compromiso con generosidad.
					
					Asimismo, saludo con afecto a mons. Benedetto Tuzia, obispo auxiliar del sector oeste; a vuestro párroco, don Gerard Charles McCarthy, a quien agradezco de corazón las cordiales palabras que me ha dirigido al inicio de esta solemne celebración. Saludo a sus colaboradores sacerdotes, pertenecientes a la fraternidad sacerdotal de los Misioneros de San Carlos Borromeo, aquí representada por el superior general, mons. Massimo Camisasca, a la que desde 1997 está encomendada la atención pastoral de esta parroquia.
					
					Saludo a las religiosas Oblatas del Amor Divino y a las Misioneras de San Carlos, que con gran entrega realizan su apostolado en esta comunidad, y a todos los grupos de niños, de jóvenes, de familias y de ancianos que animan la vida de la parroquia. También saludo cordialmente a los diversos movimientos eclesiales presentes, entre los cuales están la Juventud ardiente mariana, Comunión y liberación, la Renovación carismática católica, la Fraternidad de Santa María de los ángeles, y el grupo de voluntariado Santa Teresita.
					
					Además, quiero animar a todos los que, juntamente con la \emph{Cáritas} parroquial, tratan de salir al encuentro de las muchas necesidades del barrio, especialmente respondiendo a las expectativas de los más pobres y necesitados. Por último, saludo a las autoridades presentes y a las personalidades que han querido participar en esta asamblea litúrgica.
					
					Queridos amigos, vivimos hoy una jornada que corona los esfuerzos, las fatigas, los sacrificios realizados y el compromiso de la comunidad de formar una comunidad cristiana madura, deseosa de tener un espacio reservado definitivamente al culto de Dios. Esta celebración es muy rica en palabras y símbolos que nos ayudan a comprender el valor profundo de lo que estamos realizando. Por eso, recojamos brevemente la enseñanza que nos dan las lecturas que se acaban de proclamar.
					
					La primera lectura está tomada del libro de Nehemías, un libro que nos narra el restablecimiento de la comunidad judía después del destierro, después de la dispersión y la destrucción de Jerusalén. Por tanto, es el libro de los nuevos orígenes de una comunidad, y está lleno de esperanza, aunque las dificultades eran aún grandísimas. En el centro del pasaje que nos acaban de leer se encuentran dos grandes figuras: un sacerdote, Esdras, y un laico, Nehemías, que son respectivamente la autoridad religiosa y la autoridad civil de aquel tiempo.
					
					El texto describe el momento solemne en que se restablece oficialmente, después de la dispersión, la pequeña comunidad judía; es el momento de volver a proclamar públicamente la ley, que es el fundamento de la vida de esta comunidad, y todo se desarrolla en un clima de sencillez, de pobreza y de esperanza. La escucha de esta proclamación tiene lugar en un clima de gran intensidad espiritual. Algunos comienzan a llorar de alegría por poder escuchar nuevamente con libertad la palabra de Dios, después de la tragedia de la destrucción de Jerusalén, y recomenzar la historia de la salvación. Y Nehemías los exhorta diciendo que es un día de fiesta y que, para tener la fuerza del Señor, es preciso alegrarse, agradeciendo a Dios sus dones. La palabra de Dios es fuerza y alegría.
					
					También en nosotros esta lectura del Antiguo Testamento suscita gran conmoción. En este momento ¡cuántos recuerdos se agolpan en vuestra mente! ¡Cuántos esfuerzos realizados para construir, año tras año, la comunidad! ¡Cuántos sueños, cuántos proyectos, cuántas dificultades! Sin embargo, ahora tenéis la posibilidad de proclamar y escuchar la palabra de Dios en una hermosa iglesia, que favorece el recogimiento y suscita alegría, la alegría de saber que no sólo está presente la palabra de Dios, sino también el Señor mismo; una iglesia que quiere ser una invitación constante a una fe firme y al compromiso de crecer como comunidad unida. Agradezcamos a Dios sus dones y manifestemos nuestra gratitud también a todos los que han sido artífices de la construcción de esta iglesia y de la comunidad viva que en ella se reúne.
					
					En la segunda lectura, tomada del Apocalipsis, se nos narra una visión estupenda. El proyecto de Dios para su Iglesia y para la humanidad entera es una ciudad santa, Jerusalén, que desciende del cielo resplandeciente de gloria divina. El autor la describe como ciudad maravillosa, comparándola con las joyas más preciosas, y por último precisa que se apoya en la persona y en el mensaje de los Apóstoles. Al decir esto, el evangelista san Juan nos sugiere que la comunidad viva es la verdadera nueva Jerusalén, y que la comunidad viva es más sagrada que el templo material que consagramos.
					
					Para construir este templo vivo, esta nueva ciudad de Dios en nuestras ciudades, para construir el templo que sois vosotros, hace falta mucha oración, hace falta aprovechar todas las oportunidades que nos brindan la liturgia, la catequesis y las múltiples actividades pastorales, caritativas, misioneras y culturales, que conservan \textquote{joven} vuestra prometedora parroquia. El cuidado que con razón brindamos al edificio material ---rociándolo con el agua bendita, ungiéndolo con óleo y llenándolo de incienso--- debe ser signo y estímulo de un cuidado más intenso para defender y promover el templo de las personas, formado por vosotros, queridos feligreses.
					
					Por último, la página evangélica que acabamos de escuchar nos narra el diálogo entre Jesús y los suyos, en particular con Pedro. Es una conversación totalmente centrada en la persona del Maestro divino. La gente había intuido algo en él. Algunos pensaban que era Juan Bautista que había vuelto a la vida; otros que Elías había regresado a la tierra; otros, que era el profeta Jeremías. En cualquier caso, la gente pensaba que era una de las grandes personalidades religiosas.
					
					Pedro, en cambio, en nombre de los discípulos que conocen a Jesús de cerca, declara que Jesús es más que un profeta, más que una gran personalidad religiosa de la historia: es el Mesías, el Cristo, el Hijo de Dios vivo. Y Cristo, el Señor, le dice respondiendo solemnemente: \textquote{Tú eres Pedro y sobre esta piedra edificaré mi Iglesia} (\emph{Mt} 16, 18). Pedro, el pobre hombre con todas sus debilidades y con su fe, se convierte en la piedra, asociado precisamente por su fe a Jesús, es la roca sobre la que está fundada la Iglesia.
					
					De ese modo, vemos una vez más cómo Jesucristo es la verdadera roca indefectible sobre la que se apoya nuestra fe, sobre la que se construye toda la Iglesia y, así, también esta parroquia. Y a Jesús lo encontramos en la escucha de la sagrada Escritura; está presente y se hace nuestro alimento en la Eucaristía; vive en la comunidad, en la fe de la comunidad parroquial.
					
					Por consiguiente, en la iglesia edificio y en la Iglesia comunidad, todo habla de Jesús; todo gira en torno a él; todo hace referencia a él. Y Jesús, el Señor, nos reúne en la gran comunidad de la Iglesia de todos los tiempos y de todos los lugares, en comunión con el Sucesor de Pedro como roca de la unidad. La acción de los obispos y de los presbíteros, el compromiso apostólico y misionero de todos los fieles consiste en proclamar y testimoniar con la palabra y con la vida que él, el Hijo de Dios hecho hombre, es nuestro único Salvador.
					
					Pidamos a Jesús que guíe a vuestra comunidad y la haga crecer cada vez más en la fidelidad a su Evangelio; pidámosle que suscite muchas y santas vocaciones sacerdotales, religiosas y misioneras; que suscite en todos los feligreses la disponibilidad a seguir el ejemplo de los santos mártires portuenses.
					
					Pongamos esta oración en las manos maternales de María, Reina del Rosario. Que ella obtenga que se realicen en nosotros, en este día, las palabras finales de la primera lectura: \textquote{Que la alegría del Señor sea nuestra fuerza} (cf. \emph{Ne} 8, 10). Sólo la alegría del Señor y la fuerza de la fe en él pueden hacer fecundo el camino de vuestra parroquia. Así sea.
				\end{body}

			\subsubsection{Ángelus (2007)}
				\src{Plaza de San Pedro. \\16 de diciembre de 2007.}
				
				\begin{body}
					\emph{Queridos hermanos y hermanas:} 
					
					\emph{\textquote{Gaudete in Domino semper}, estad siempre alegres en el Señor} (\emph{Flp} 4, 4). Con estas palabras de san Pablo se inicia la santa misa del III domingo de Adviento, que por eso se llama domingo \emph{\textquote{Gaudete}}. El Apóstol exhorta a los cristianos a alegrarse porque la venida del Señor, es decir, su vuelta gloriosa es segura y no tardará. La Iglesia acoge esta invitación mientras se prepara para celebrar la Navidad, y su mirada se dirige cada vez más a Belén. En efecto, aguardamos con esperanza segura la segunda venida de Cristo, porque hemos conocido la primera.
					
					El misterio de Belén nos revela al Dios-con-nosotros, al Dios cercano a nosotros, no sólo en sentido espacial y temporal; está cerca de nosotros porque, por decirlo así, se ha \textquote{casado} con nuestra humanidad; ha asumido nuestra condición, escogiendo ser en todo como nosotros, excepto en el pecado, para hacer que lleguemos a ser como él.
					
					Por tanto, la alegría cristiana brota de esta certeza: Dios está cerca, está conmigo, está con nosotros, en la alegría y en el dolor, en la salud y en la enfermedad, como amigo y esposo fiel. Y esta alegría permanece también en la prueba, incluso en el sufrimiento; y no está en la superficie, sino en lo más profundo de la persona que se encomienda a Dios y confía en él.
					
					Algunos se preguntan: ¿también hoy es posible esta alegría? La respuesta la dan, con su vida, hombres y mujeres de toda edad y condición social, felices de consagrar su existencia a los demás. En nuestros tiempos, la beata madre Teresa de Calcuta fue testigo inolvidable de la verdadera alegría evangélica. Vivía diariamente en contacto con la miseria, con la degradación humana, con la muerte. Su alma experimentó la prueba de la noche oscura de la fe y, sin embargo, regaló a todos la sonrisa de Dios.
					
					En uno de sus escritos leemos: \textquote{Esperamos con impaciencia el paraíso, donde está Dios, pero ya aquí en la tierra y desde este momento podemos estar en el paraíso. Ser felices con Dios significa: amar como él, ayudar como él, dar como él, servir como él} (\emph{La gioia di darsi agli altri}, Ed. Paoline 1987, p. 143). Sí, la alegría entra en el corazón de quien se pone al servicio de los pequeños y de los pobres. Dios habita en quien ama así, y el alma vive en la alegría.
					
					En cambio, si se hace de la felicidad un ídolo, se equivoca el camino y es verdaderamente difícil encontrar la alegría de la que habla Jesús. Por desgracia, esta es la propuesta de las culturas que ponen la felicidad individual en lugar de Dios, mentalidad que se manifiesta de forma emblemática en la búsqueda del placer a toda costa y en la difusión del uso de drogas como fuga, como refugio en paraísos artificiales, que luego resultan del todo ilusorios.
					
					Queridos hermanos y hermanas, también en Navidad se puede equivocar el camino, confundiendo la verdadera fiesta con una que no abre el corazón a la alegría de Cristo. Que la Virgen María ayude a todos los cristianos, y a los hombres que buscan a Dios, a llegar hasta Belén para encontrar al Niño que nació por nosotros, para la salvación y la felicidad de todos los hombres.
				\end{body}


			\subsubsection{Homilía (2010)}
				\src{Visita pastoral parroquia romana de San Maximiliano Kolbe, barrio de Torre Angela. \\12 de diciembre del 2010.}
				
				\begin{body}
					\emph{Queridos hermanos y hermanas de la parroquia de San Maximiliano Kolbe:}
					
					Vivid con empeño el camino personal y comunitario de seguimiento del Señor. El Adviento es una fuerte invitación para todos a dejar que Dios entre cada vez más en nuestra vida, en nuestros hogares, en nuestros barrios, en nuestras comunidades, para tener una luz en medio de tantas sombras y de las numerosas pruebas de cada día. Queridos amigos, estoy muy contento de estar entre vosotros hoy para celebrar el día del Señor, el tercer domingo del Adviento, domingo de la alegría. Saludo cordialmente al cardenal vicario, al obispo auxiliar del sector, a vuestro párroco, a quien agradezco las palabras que me ha dirigido en nombre de todos vosotros, y al vicario parroquial. Saludo a cuantos colaboran en las actividades de la parroquia: a los catequistas, a las personas que forman parte de los diversos grupos, así como a los numerosos miembros del Camino Neocatecumenal. Aprecio mucho la elección de dar espacio a la adoración eucarística, y os agradezco las oraciones que me reserváis ante el Santísimo Sacramento. Quiero extender mi saludo a todos los habitantes del barrio, especialmente a los ancianos, a los enfermos, a las personas solas o que atraviesan dificultades. A todos y cada uno los recuerdo en esta misa.
					
					Admiro junto con vosotros esta nueva iglesia y los edificios parroquiales, y con mi presencia deseo alentaros a construir cada vez mejor la Iglesia de piedras vivas que sois vosotros mismos. Conozco las numerosas y significativas obras de evangelización que estáis realizando. Exhorto a todos los fieles a contribuir a la edificación de la comunidad, especialmente en el campo de la catequesis, de la liturgia y de la caridad ---pilares de la vida cristiana--- en comunión con toda la diócesis de Roma. Ninguna comunidad puede vivir como una célula aislada del contexto diocesano; al contrario, debe ser expresión viva de la belleza de la Iglesia que, bajo la guía del obispo ---y, en la parroquia, bajo la guía del párroco, que lo representa---, camina en comunión hacia el reino de Dios. Dirijo un saludo especial a las familias, acompañándolo con el deseo de que realicen plenamente su vocación al amor con generosidad y perseverancia. Aunque se presentaran dificultades en la vida conyugal y en la relación con los hijos, los esposos deben permanecer siempre fieles al fundamental \textquote{sí} que pronunciaron delante de Dios y se dijeron mutuamente en el día de su matrimonio, recordando que la fidelidad a la propia vocación exige valentía, generosidad y sacrificio.
					
					En el seno de vuestra comunidad hay muchas familias venidas del centro y del sur de Italia en busca de trabajo y de mejores condiciones de vida. Con el paso del tiempo, la comunidad ha crecido y en parte se ha transformado, con la llegada de numerosas personas de los países del Este europeo y de otros países. Precisamente a partir de esta situación concreta de la parroquia, esforzaos por crecer cada vez más en la comunión con todos: es importante crear ocasiones de diálogo y favorecer la comprensión mutua entre personas provenientes de culturas, modelos de vida y condiciones sociales diferentes; pero es preciso sobre todo tratar de que participen en la vida cristiana, mediante una pastoral atenta a las necesidades reales de cada uno. Aquí, como en cada parroquia, hay que partir de los \textquote{cercanos} para llegar a los \textquote{lejanos}, para llevar una presencia evangélica a los ambientes de vida y de trabajo. En la parroquia todos deben poder encontrar caminos adecuados de formación y experimentar la dimensión comunitaria, que es una característica fundamental de la vida cristiana. De ese modo se verán alentados a redescubrir la belleza de seguir a Cristo y de formar parte de su Iglesia.
					
					Sabed, pues, hacer comunidad con todos, unidos en la escucha de la Palabra de Dios y en la celebración de los sacramentos, especialmente de la Eucaristía. A este propósito, la verificación pastoral diocesana que se está llevando a cabo, sobre el tema \textquote{Eucaristía dominical y testimonio de la caridad}, es una ocasión propicia para profundizar y vivir mejor estos dos componentes fundamentales de la vida y de la misión de la Iglesia y de todo creyente, es decir, la Eucaristía del domingo y la practica de la caridad. Reunidos en torno a la Eucaristía, sentimos más fácilmente que la misión de toda comunidad cristiana consiste en llevar el mensaje del amor de Dios a todos los hombres. Por eso es importante que la Eucaristía siempre sea el corazón de la vida de los fieles. También quiero dirigiros unas palabras de afecto y de amistad en especial a vosotros, queridos muchachos y jóvenes que me escucháis, y a vuestros coetáneos que viven en esta parroquia. La Iglesia espera mucho de vosotros, de vuestro entusiasmo, de vuestra capacidad de mirar hacia adelante y de vuestro deseo de radicalidad en las opciones de la vida. Sentíos verdaderos protagonistas en la parroquia, poniendo vuestras energías lozanas y toda vuestra vida al servicio de Dios y de los hermanos.
					
					Queridos hermanos y hermanas, la liturgia de hoy ---con las palabras del apóstol Santiago que hemos escuchado--- nos invita no sólo a la alegría sino también a ser constantes y pacientes en la espera del Señor que viene, y a serlo juntos, como comunidad, evitando quejas y juicios (cf. \emph{St} 5, 7-10).
					
					Hemos escuchado en el Evangelio la pregunta de san Juan Bautista que se encuentra en la cárcel; el Bautista, que había anunciado la venida del Juez que cambia el mundo, y ahora siente que el mundo sigue igual. Por eso, pide que pregunten a Jesús: \textquote{¿Eres tú el que ha de venir o debemos esperar a otro? ¿Eres tú o debemos esperar a otro?}. En los últimos dos o tres siglos muchos han preguntado: \textquote{¿Realmente eres tú o hay que cambiar el mundo de modo más radical? ¿Tú no lo haces?}. Y han venido muchos profetas, ideólogos y dictadores que han dicho: \textquote{¡No es él! ¡No ha cambiado el mundo! ¡Somos nosotros!}. Y han creado sus imperios, sus dictaduras, su totalitarismo que cambiaría el mundo. Y lo ha cambiado, pero de modo destructivo. Hoy sabemos que de esas grandes promesas no ha quedado más que un gran vacío y una gran destrucción. No eran ellos.
					
					Y así debemos mirar de nuevo a Cristo y preguntarle: \textquote{¿Eres tú?}. El Señor, con el modo silencioso que le es propio, responde: \textquote{Mirad lo que he hecho. No he hecho una revolución cruenta, no he cambiado el mundo con la fuerza, sino que he encendido muchas luces que forman, a la vez, un gran camino de luz a lo largo de los milenios}.
					
					Comencemos aquí, en nuestra parroquia: san Maximiliano Kolbe, que se ofreció para morir de hambre a fin de salvar a un padre de familia. ¡En qué gran luz se ha convertido! ¡Cuánta luz ha venido de esta figura! Y ha alentado a otros a entregarse, a estar cerca de quienes sufren, de los oprimidos. Pensemos en el padre que era para los leprosos Damián de Veuster, que vivió y murió \emph{con} y \emph{para} los leprosos, y así llevó luz a esa comunidad. Pensemos en la madre Teresa, que dio tanta luz a personas, que, después de una vida sin luz, murieron con una sonrisa, porque las había tocado la luz del amor de Dios.
					
					Y podríamos seguir y veríamos, como dijo el Señor en la respuesta a Juan, que lo que cambia el mundo no es la revolución violenta, ni las grandes promesas, sino la silenciosa luz de la verdad, de la bondad de Dios, que es el signo de su presencia y nos da la certeza de que somos amados hasta el fondo y de que no caemos en el olvido, no somos un producto de la casualidad, sino de una voluntad de amor.
					
					Así podemos vivir, podemos sentir la cercanía de Dios. \textquote{Dios está cerca} dice la primera lectura de hoy; está cerca, pero nosotros a menudo estamos lejos. Acerquémonos, vayamos hacia la presencia de su luz, oremos al Señor y en el contacto de la oración también nosotros seremos luz para los demás.
					
					Precisamente este es el sentido de la iglesia parroquial: entrar aquí, entrar en diálogo, en contacto con Jesús, con el Hijo de Dios, a fin de que nosotros mismos nos convirtamos en una de las luces más pequeñas que él ha encendido y traigamos luz al mundo, que sienta que es redimido.
					
					Nuestro espíritu debe abrirse a esta invitación; así caminemos con alegría al encuentro de la Navidad, imitando a la Virgen María, que esperó en la oración, con íntimo y gozoso temor, el nacimiento del Redentor. Amén.
				\end{body}

			\subsubsection{Ángelus (2010): ¿Qué es lo que cambia el mundo?}
				\src{*************¿?¿¿¿¿¿¿¿¿¿¿¿¿¿¿¿¿¿¿******** \\12 de diciembre del 2010.}
				
				\begin{body}
					Queridos hermanos y hermanas de la parroquia de San Maximiliano Kolbe:
					
					Vivid con empeño el camino personal y comunitario de seguimiento del Señor. El Adviento es una fuerte invitación para todos a dejar que Dios entre cada vez más en nuestra vida, en nuestros hogares, en nuestros barrios, en nuestras comunidades, para tener una luz en medio de tantas sombras y de las numerosas pruebas de cada día. Queridos amigos, estoy muy contento de estar entre vosotros hoy para celebrar el día del Señor, el tercer domingo del Adviento, domingo de la \textbf{alegría}\ldots{}
					
					[\ldots{}] La liturgia de hoy ---con las palabras del \textbf{apóstol Santiago} que hemos escuchado--- nos invita no sólo a la alegría sino también a ser constantes y pacientes en la espera del Señor que viene, y a serlo juntos, como comunidad, evitando quejas y juicios (cf. \emph{St} 5, 7-10).
					
					Hemos escuchado en el \textbf{Evangelio} la \textbf{pregunta de san Juan Bautista} que se encuentra en la cárcel; el Bautista, que había anunciado la venida del Juez que cambia el mundo, y ahora siente que el mundo sigue igual. Por eso, pide que pregunten a Jesús: \textquote{¿Eres tú el que ha de venir o debemos esperar a otro? ¿Eres tú o debemos esperar a otro?}. En los últimos dos o tres siglos muchos han preguntado: \textquote{¿Realmente eres tú o hay que cambiar el mundo de modo más radical? ¿Tú no lo haces?}. Y han venido muchos profetas, ideólogos y dictadores que han dicho: \textquote{¡No es él! ¡No ha cambiado el mundo! ¡Somos nosotros!}. Y han creado sus imperios, sus dictaduras, su totalitarismo que cambiaría el mundo. Y lo ha cambiado, pero de modo destructivo. Hoy sabemos que de esas grandes promesas no ha quedado más que un gran vacío y una gran destrucción. No eran ellos.
					
					Y así debemos mirar de nuevo a Cristo y preguntarle: \textquote{¿Eres tú?}. El Señor, con el modo silencioso que le es propio, responde: \textquote{Mirad lo que he hecho. No he hecho una revolución cruenta, no he cambiado el mundo con la fuerza, sino que he encendido muchas luces que forman, a la vez, un gran camino de luz a lo largo de los milenios}.
					
					Comencemos aquí, en nuestra parroquia: san Maximiliano Kolbe, que se ofreció para morir de hambre a fin de salvar a un padre de familia. ¡En qué gran luz se ha convertido! ¡Cuánta luz ha venido de esta figura! Y ha alentado a otros a entregarse, a estar cerca de quienes sufren, de los oprimidos. Pensemos en el padre que era para los leprosos Damián de Veuster, que vivió y murió \emph{con} y \emph{para} los leprosos, y así llevó luz a esa comunidad. Pensemos en la madre Teresa, que dio tanta luz a personas, que, después de una vida sin luz, murieron con una sonrisa, porque las había tocado la luz del amor de Dios.
					
					Y podríamos seguir y veríamos, como dijo el Señor en la respuesta a Juan, que lo que cambia el mundo no es la revolución violenta, ni las grandes promesas, sino la silenciosa luz de la verdad, de la bondad de Dios, que es el signo de su presencia y nos da la certeza de que somos amados hasta el fondo y de que no caemos en el olvido, no somos un producto de la casualidad, sino de una voluntad de amor.
					
					Así podemos vivir, podemos sentir la cercanía de Dios. \textquote{Dios está cerca} dice la \textbf{primera lectura} de hoy; está cerca, pero nosotros a menudo estamos lejos. Acerquémonos, vayamos hacia la presencia de su luz, oremos al Señor y en el contacto de la oración también nosotros seremos luz para los demás.
					
					Precisamente este es el sentido de la iglesia parroquial: entrar aquí, entrar en diálogo, en contacto con Jesús, con el Hijo de Dios, a fin de que nosotros mismos nos convirtamos en una de las luces más pequeñas que él ha encendido y traigamos luz al mundo, que sienta que es redimido.
					
					Nuestro espíritu debe abrirse a esta invitación; así caminemos con alegría al encuentro de la Navidad, imitando a la Virgen María, que esperó en la oración, con íntimo y gozoso temor, el nacimiento del Redentor. Amén.
				\end{body}

		\subsection{Francisco, papa}

			\subsubsection{Ángelus (2013): Es posible recomenzar}
				\src{Plaza de San Pedro. \\15 de diciembre del 2013.}
				
				\begin{body}
					¡Gracias! Queridos hermanos y hermanas, ¡buenos días!
					
					Hoy es el tercer domingo de Adviento, llamado también domingo de \emph{Gaudete}, es decir, domingo de la \textbf{alegría}. En la liturgia resuena repetidas veces la invitación a gozar, a alegrarse. ¿Por qué? Porque el Señor está cerca. La Navidad está cercana. El mensaje cristiano se llama \textquote{Evangelio}, es decir, \textquote{buena noticia}, un anuncio de alegría para todo el pueblo; la Iglesia no es un refugio para gente triste, la Iglesia es la casa de la alegría. Y quienes están tristes encuentran en ella la alegría, encuentran en ella la verdadera alegría.
					
					Pero la alegría del Evangelio no es una alegría cualquiera. Encuentra su razón de ser en el saberse acogidos y amados por Dios. Como nos recuerda hoy el \textbf{profeta Isaías} (cf. 35, 1-6a.8a.10), Dios es Aquél que viene a salvarnos, y socorre especialmente a los extraviados de corazón. Su venida en medio de nosotros fortalece, da firmeza, dona valor, hace exultar y florecer el desierto y la estepa, es decir, nuestra vida, cuando se vuelve árida. ¿Cuándo llega a ser árida nuestra vida? Cuando no tiene el agua de la Palabra de Dios y de su Espíritu de amor.
					
					Por más grandes que sean nuestros límites y nuestros extravíos, no se nos permite ser débiles y vacilantes ante las dificultades y ante nuestras debilidades mismas. Al contrario, estamos invitados a robustecer las manos, a fortalecer las rodillas, a tener valor y a no temer, porque nuestro Dios nos muestra siempre la grandeza de su misericordia. Él nos da la fuerza para seguir adelante. Él está siempre con nosotros para ayudarnos a seguir adelante. Es un Dios que nos quiere mucho, nos ama y por ello está con nosotros, para ayudarnos, para robustecernos y seguir adelante.
					
					¡Ánimo! ¡Siempre adelante! Gracias a su ayuda podemos siempre recomenzar de nuevo. ¿Cómo? ¿Recomenzar desde el inicio? Alguien puede decirme: \textquote{No, Padre, yo he hecho muchas cosas\ldots{} Soy un gran pecador, una gran pecadora\ldots{} No puedo recomenzar desde el inicio}. ¡Te equivocas! Tú puedes recomenzar de nuevo. ¿Por qué? Porque Él te espera, Él está cerca de ti, Él te ama, Él es misericordioso, Él te perdona, Él te da la fuerza para recomenzar de nuevo. ¡A todos! Entonces somos capaces de volver a abrir los ojos, de superar tristeza y llanto y entonar un canto nuevo. Esta alegría verdadera permanece también en la prueba, incluso en el sufrimiento, porque no es una alegría superficial, sino que desciende en lo profundo de la persona que se fía de Dios y confía en Él.
					
					La alegría cristiana, al igual que la esperanza, tiene su fundamento en la fidelidad de Dios, en la certeza de que Él mantiene siempre sus promesas. El \textbf{profeta Isaías} exhorta a quienes se equivocaron de camino y están desalentados a confiar en la fidelidad del Señor, porque su salvación no tardará en irrumpir en su vida. Quienes han encontrado a Jesús a lo largo del camino, experimentan en el corazón una serenidad y una alegría de la que nada ni nadie puede privarles. Nuestra alegría es Jesucristo, su amor fiel e inagotable. Por ello, cuando un cristiano llega a estar triste, quiere decir que se ha alejado de Jesús. Entonces, no hay que dejarle solo. Debemos rezar por él, y hacerle sentir el calor de la comunidad.
					
					Que la Virgen María nos ayude a apresurar el paso hacia Belén, para encontrar al Niño que nació por nosotros, por la salvación y la alegría de todos los hombres. A ella le dice el Ángel: \textquote{Alégrate, llena de gracia, el Señor está contigo} (\emph{Lc} 1, 28). Que ella nos conceda vivir la alegría del Evangelio en la familia, en el trabajo, en la parroquia y en cada ambiente. Una alegría íntima, hecha de asombro y ternura. La alegría que experimenta la mamá cuando contempla a su niño recién nacido, y siente que es un don de Dios, un milagro por el cual sólo se puede agradecer.
				\end{body}

			\subsubsection{Ángelus (2016): Su venida, alegría plena}
			
				\src{Plaza de San Pedro.\\11 de diciembre del 2016.}
				
				\begin{body}
					Hoy celebramos el tercer domingo de Adviento, caracterizado por la invitación de \textbf{san Pablo}: \textquote{Estad siempre alegres en el Señor: os lo repito, estad alegres} (Fil 4, 4-5). No es una alegría superficial o puramente emotiva a la que nos exhorta el apóstol, y ni siquiera una mundana o la alegría del consumismo. No, no es esa, sino que se trata de una alegría más auténtica, de la cual estamos llamados a redescubrir su sabor. El sabor de la verdadera alegría. Es una alegría que toca lo íntimo de nuestro ser, mientras que esperamos a Jesús, que ya ha venido a traer la salvación al mundo, el Mesías prometido, nacido en Belén de la Virgen María. La liturgia de la Palabra nos ofrece el contexto adecuado para comprender y vivir esta alegría. \textbf{Isaías} habla de desierto, de tierra árida, de estepa (cf. 35, 1); el profeta tiene ante sí manos débiles, rodillas vacilantes, corazones perdidos, ciegos, sordos y mudos (cf. vv. 3-6). Es el cuadro de una situación de desolación, de un destino inexorable sin Dios.
					
					Pero finalmente la salvación es anunciada: \textquote{¡Ánimo, no temáis! ---dice el profeta--- [\ldots{}] Mirad que vuestro Dios, [\ldots{}] Él vendrá y os salvará} (cf. Is 35, 4). Y enseguida todo se transforma: el desierto florece, la consolación y la alegría inundan los corazones (cf. vv. 5-6). Estos signos anunciados por Isaías como reveladores de la salvación ya presente, se realizan en Jesús. Él mismo lo afirma \textbf{respondiendo a los mensajeros enviados por Juan Bautista}. ¿Qué dice Jesús a estos mensajeros? \textquote{Los ciegos ven y los cojos andan, los leprosos quedan limpios y los sordos oyen, los muertos resucitan} (Mt 11, 5). No son palabras, son hechos que demuestran cómo la salvación traída por Jesús, aferra a todo el ser humano y le regenera. Dios ha entrado en la historia para liberarnos de la esclavitud del pecado; ha puesto su tienda en medio de nosotros para compartir nuestra existencia, curar nuestras llagas, vendar nuestras heridas y donarnos la vida nueva. La alegría es el fruto de esta intervención de salvación y de amor de Dios.
					
					Estamos llamados a dejarnos llevar por el sentimiento de exultación. Este júbilo, esta alegría\ldots{} Pero un cristiano que no está alegre, algo le falta a este cristiano, ¡o no es cristiano! La alegría del corazón, la alegría dentro que nos lleva adelante y nos da el valor. El Señor viene, viene a nuestra vida como libertador, viene a liberarnos de todas las esclavitudes interiores y exteriores. Es Él quien nos indica el camino de la fidelidad, de la paciencia y de la perseverancia porque, a su llegada, nuestra alegría será plena.
					
					La Navidad está cerca, los signos de su aproximarse son evidentes en nuestras calles y en nuestras casas; también aquí en la Plaza se ha puesto el pesebre con el árbol al lado. Estos signos externos nos invitan a acoger al Señor que siempre viene y llama a nuestra puerta, llama a nuestro corazón, para estar cerca de nosotros. Nos invitan a reconocer sus pasos entre los de los hermanos que pasan a nuestro lado, especialmente los más débiles y necesitados.
					
					Hoy estamos invitados a alegrarnos por la llegada inminente de nuestro Redentor; y estamos llamados a compartir esta alegría con los demás, dando consuelo y esperanza a los pobres, a los enfermos, a las personas solas e infelices. Que la Virgen María, la \textquote{sierva del Señor}, nos ayude a escuchar la voz de Dios en la oración y a servirle con compasión en los hermanos, para llegar preparados a la cita con la Navidad, preparando nuestro corazón para acoger a Jesús.
				\end{body}
			
			\subsubsection{Homilía (2019)}
			
				\src{Santa Misa para la Comunidad Católica Filipina. Basílica Vaticana. \\15 de diciembre del 2019.}
				
				\begin{body}
					\emph{Queridos hermanos y hermanas:}
					
					Celebramos hoy el tercer domingo de Adviento. En la primera lectura, el profeta Isaías invita a la tierra entera a alegrarse por la venida del Señor, que trae la salvación a su pueblo. Viene a abrir los ojos a los ciegos y los oídos a los sordos, a curar a los cojos y a los mudos (cf. 35,5-6). La salvación se ofrece a todos, pero el Señor muestra una ternura especial por los más vulnerables, los más frágiles, los más pobres de su pueblo.
					
					De las palabras del salmo responsorial aprendemos que hay otros vulnerables que merecen una mirada de amor especial de Dios: los oprimidos, los hambrientos, los prisioneros, los extranjeros, los huérfanos y las viudas (cf. \emph{Sal} 145,7-9). Son los habitantes de las periferias existenciales de ayer y de hoy.
					
					En Jesucristo el amor salvífico de Dios se hace tangible: \textquote{Los ciegos ven y los cojos andan, los leprosos quedan limpios y los sordos oyen, los muertos resucitan y a los pobres se anuncia la Buena Nueva} (\emph{Mt} 11,5). Estos son los signos que acompañan la realización del Reino de Dios. No toques de trompeta o triunfos militares, no juicios y condenas de pecadores, sino liberación del mal y anuncio de misericordia y de paz.
					
					También este año nos preparamos para celebrar el misterio de la Encarnación, de Emmanuel, el \textquote{Dios con nosotros} que obra maravillas en favor de su pueblo, especialmente de los más pequeños y frágiles. Estas maravillas son los \textquote{signos} de la presencia de su Reino. Y como todavía son muchos los habitantes de las periferias existenciales, debemos pedir al Señor que renueve cada año el milagro de la Navidad, ofreciéndonos nosotros mismos como instrumentos de su amor misericordioso por los más pequeños.
					
					Para prepararnos adecuadamente a esta nueva efusión de gracia, la Iglesia nos brinda el tiempo de Adviento, en el que estamos llamados a despertar la esperanza en nuestros corazones e intensificar nuestra oración. Con este fin, en la riqueza de las diferentes tradiciones, las Iglesias particulares han introducido una variedad de prácticas devocionales.
					
					En Filipinas existe desde hace siglos una novena en preparación para la Santa Navidad llamada \emph{Simbang-Gabi} (misa nocturna). Durante nueve días los fieles filipinos se reúnen al amanecer en sus parroquias para una celebración eucarística especial. En las últimas décadas, gracias a los emigrantes filipinos, esta devoción ha traspasado las fronteras nacionales llegando a muchos otros países. Desde hace años \emph{Simbang-Gabi} también se celebra en la diócesis de Roma, y hoy lo celebramos juntos aquí, en la basílica de San Pedro.
					
					Con esta celebración queremos prepararnos para la Navidad según el espíritu de la Palabra de Dios que hemos escuchado, permaneciendo constantes hasta la venida definitiva del Señor, como nos recomienda el apóstol Santiago (cf. \emph{St} 5,7). Queremos comprometernos a manifestar el amor y la ternura de Dios hacia todos, especialmente hacia los más pequeños. Estamos llamados a ser levadura en una sociedad que a menudo ya no puede saborear la belleza de Dios y experimentar la gracia de su presencia.
					
					Y vosotros, queridos hermanos y hermanas, que habéis dejado vuestra tierra en busca de un futuro mejor, tenéis una misión especial. Que vuestra fe sea \textquote{levadura} en las comunidades parroquiales a las que pertenecéis hoy. Os animo a multiplicar las oportunidades de encuentro para compartir vuestra riqueza cultural y espiritual, al mismo tiempo que os dejáis enriquecer por las experiencias de los demás. Todos estamos invitados a construir juntos esa comunión en la diversidad que es un rasgo distintivo del Reino de Dios, inaugurado por Jesucristo, el Hijo de Dios hecho hombre. Todos estamos llamados a practicar juntos la caridad con los habitantes de las periferias existenciales, poniendo al servicio nuestros diversos dones, para renovar los signos de la presencia del Reino. Todos estamos llamados a anunciar juntos el Evangelio, la Buena Nueva de la salvación, en todas las lenguas, para llegar al mayor número posible de personas.
					
					Que el Santo Niño al que nos disponemos a adorar, envuelto en pobres pañales y recostado en un pesebre, os bendiga y os dé la fuerza para continuar vuestro testimonio con alegría.
				\end{body}

			\subsubsection{Ángelus (2019): Conversión}
			
				\src{Plaza de San Pedro. \\15 de diciembre de 2019.}
				
				\begin{body}
					En este tercer domingo de Adviento, llamado el \textquote{domingo de la alegría}, la Palabra de Dios nos invita, por una parte, a la \emph{alegría} y, por otra, a la conciencia de que la existencia incluye también momentos de \emph{duda}, en los que es difícil creer. La \emph{alegría} y la \emph{duda} son experiencias que forman parte de nuestras vidas.
					
					A la invitación explícita a la alegría del \textbf{profeta Isaías}: \textquote{Que el desierto y el sequedal se alegren, regocíjese la estepa y florezca como flor} (35, 1), se contrapone en el \textbf{Evangelio} la duda de Juan el Bautista: \textquote{¿Eres tú el que ha de venir, o debemos esperar a otro?} (\emph{Mateo} 11, 3). De hecho, el profeta ve más allá de la situación, tiene ante sí gente desanimada: manos débiles, rodillas vacilantes, corazones intranquilos (cf. \emph{Isaías} 35, 3-4). Es la misma realidad que siempre pone a prueba la fe. Pero el hombre de Dios mira más allá, porque el Espíritu Santo hace que su corazón sienta el poder de su promesa y anuncia la salvación: \textquote{¡Ánimo, no temáis! Mirad que vuestro Dios viene, [\ldots{}] os salvará} (v. 4). Y entonces todo se transforma: el desierto florece, el consuelo y la alegría se apoderan de los perdidos, los cojos, los ciegos, los mudos se curan (cf. vv. 5-6). Esto es lo que sucede con Jesús: \textquote{los ciegos ven y los cojos andan, los leprosos quedan limpios y los sordos oyen, los muertos resucitan y se anuncia a los pobres la Buena Nueva} (\emph{Mateo} 11, 5).
					
					Esta descripción nos muestra que la salvación envuelve al hombre entero y lo regenera. Pero este nuevo nacimiento, con la alegría que lo acompaña, presupone siempre una muerte para nosotros mismos y para el pecado que está dentro de nosotros. De ahí la llamada a la conversión, que es la base de la predicación tanto del Bautista como de Jesús; en particular, se trata de convertir la idea que tenemos de Dios. Y el tiempo de Adviento nos estimula a hacerlo precisamente con \textbf{la pregunta que Juan el Bautista le hace a Jesús}: \textquote{¿Eres tú el que ha de venir, o debemos esperar a otro?} (\emph{Mateo} 11, 3). Pensemos: toda su vida Juan esperó al Mesías; su estilo de vida, su cuerpo mismo, está moldeado por esta espera. Por eso también Jesús lo alaba con estas palabras: \textquote{no ha surgido entre los nacidos de mujer uno mayor que Juan el Bautista} (\emph{Mateo} 11, 11). Sin embargo, él también tuvo que convertirse a Jesús. Como Juan, también nosotros estamos llamados a reconocer el rostro que Dios eligió asumir en Jesucristo, humilde y misericordioso.
					
					El Adviento es un tiempo de gracia. Nos dice que no basta con creer en Dios: es necesario purificar nuestra fe cada día. Se trata de prepararnos para acoger no a un personaje de cuento de hadas, sino al Dios que nos llama, que nos implica y ante el que se impone una elección. El Niño que yace en el pesebre tiene el rostro de nuestros hermanos más necesitados, de los pobres, que \textquote{son los privilegiados de este misterio y, a menudo, aquellos que son más capaces de reconocer la presencia de Dios en medio de nosotros} (Carta Apostólica \emph{Admirabile signum}, 6).
					
					Que la Virgen María nos ayude para que, al acercarnos a la Navidad, no nos dejemos distraer por las cosas externas, sino que hagamos espacio en nuestros corazones a Aquél que ya ha venido y quiere volver a venir para curar nuestras enfermedades y darnos su alegría.
				\end{body}
			
	\section{Temas}