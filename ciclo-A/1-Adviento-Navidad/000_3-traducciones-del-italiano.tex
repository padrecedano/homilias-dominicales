\chapter{I-Adviento}

\subsubsection{Homilía (1983): }

\src{Visita Pastoral a la Parroquia Romana de San Felipe Neri. \par27 de noviembre de 1983.}

\begin{body} 
	1. "Comportaos reconociendo el momento en que vivís, pues ya es hora de despertaros del sueño" (\emph{Rom} 13, 11).
	
	Con estas palabras, queridos hermanos y hermanas, la liturgia de hoy se dirige a cada uno de nosotros, enseñándonos a acoger la llamada que nos llega desde el comienzo del Adviento. Despertar del sueño significa abrir el corazón a esa realidad divina que está ligada al tiempo humano. Por eso se dice: "la salvación está más cerca".
	
	El Adviento es como una primera dimensión de esta unión de la Realidad divina al tiempo humano. Este vínculo se refleja en el año litúrgico: el domingo primero de Adviento es al mismo tiempo el comienzo del nuevo año litúrgico.
	
	{[}2. Al mismo tiempo comenzamos el Año Santo de la Redención. Este extraordinario Jubileo de la Redención tiene un carácter específico de "adviento": nos prepara para el tercer milenio después de Cristo. De ahí la particular elocuencia del Adviento de este año, que debe expresar esa actitud de la Iglesia, de la que ya hablé en la Bula de Indicación (Juan Pablo II, \href{http://www.vatican.va/content/john-paul-ii/it/jubilee/documents/hf_jp-ii_doc_19830106_bolla-redenzione.html}{\emph{\emph{Aperite portas Redemptori}}}, 7), por la que "se siente especialmente comprometida con la fidelidad a los dones divinos, que tienen su origen en la redención de Cristo, y por medio de los cuales el Espíritu Santo la guía a su desarrollo y renovación, para que pueda siempre convertirse en una esposa más digna de su Señor. Para ello confía en el Espíritu Santo y quiere asociarse a su acción misteriosa de Esposa que invoca la venida de Cristo" (cf. \emph{Ap} 22, 17).
	
	Este carácter particular de "adviento" propio del presente Año Santo, debe ser vivido por la Iglesia "con los mismos sentimientos con los que la Virgen María esperaba el nacimiento del Señor en la humildad de nuestra naturaleza humana". Así como María precedió a la Iglesia en la fe y en el amor en los albores de la era de la Redención, hoy la precede mientras, en este Jubileo, avanza hacia el nuevo milenio de la Redención" (Juan Pablo II, \emph{Aperite portas Redemptori}, 9) .{]}
	
	3. "Reconocer el momento en que vivimos": ¿qué significa? "Vamos alegres al encuentro del Señor".
	
	El Adviento es la perspectiva gozosa de "ir a la casa del Señor" (cf. \emph{Sal} 121, 1): de llegar al final de esta gran "peregrinación" que debe ser la vida terrena. El hombre está llamado a habitar en la "casa del Señor". Allí está su verdadero "hogar". {[}La peregrinación del Año Santo es una figura de nuestro camino hacia la casa del Padre,{]} y el Adviento nos estimula a acelerar este camino con esperanza.
	
	El Adviento aguarda el "día del Señor", es decir, la "hora de la verdad". Es la expectativa de ese día en que "Él será juez entre las naciones y árbitro entre todos los pueblos" (\emph{Is} 2, 4). Esta plenitud de verdad será el principio y fundamento de la paz definitiva y universal, que es el objeto de la esperanza de todos los hombres de buena voluntad.
	
	El Adviento es una confirmación del camino eterno del hombre hacia Dios; es un nuevo comienzo, cada año, de este camino: ¡la vida del hombre no es un camino infranqueable, sino un camino que conduce al encuentro con el Señor! Hay también en esta invocación del primer domingo de Adviento casi un anticipo de esos caminos que en la noche de Belén conducirán a los pastores y a los Reyes Magos de Oriente hacia Jesús el recién nacido.
	
	4. "Reconocer el momento en que vivimos": ¿qué significa? "Revestirnos del Señor Jesucristo" (\emph{Rm} 13, 14):
	
	- el camino del Adviento conduce al interior del hombre, que de diversas formas está cargado de pecado, como atestigua la \textbf{segunda lectura};
	
	- el encuentro antes mencionado no sólo se realiza "desde fuera", sino también "desde dentro", y consiste en una transformación tal del interior del hombre, que corresponde a la santidad de Aquel con quien se encuentra: en eso consiste precisamente "vestirse del Señor Jesucristo";
	
	- el sentido "histórico" del Adviento está penetrado por el sentido "espiritual". De hecho, el Adviento no quiere ser sólo el recuerdo del período histórico que precedió al nacimiento del Salvador, aunque, así entendido, ya tiene en sí mismo un altísimo significado espiritual. Sin embargo, más allá de eso, y de manera más profunda, el Adviento quiere recordarnos que toda la historia del hombre y de cada uno de nosotros debe entenderse como un gran "Adviento", como una expectativa, momento a momento, de la venida del Señor, para que nos encuentre preparados y vigilantes para poder acogerlo dignamente.
	
	5. "Reconocer el momento en que vivimos": significa: "Velad \ldots{} porque no sabéis en qué día vendrá vuestro Señor" (\emph{Mt} 24, 42):
	
	- la unión de Dios, de la realidad divina, con el tiempo humano, por un lado, reafirma la limitación de este tiempo, que tiene fin y, por otra parte, abre este tiempo a la eternidad de Dios y a las "realidades últimas" relacionadas con él;
	
	- el Adviento tiene un significado "escatológico" ya que recuerda nuestros pensamientos y nuestras intenciones hacia realidades futuras. Nos recuerda el fin último de nuestro camino, y nos estimula a involucrarnos en las realidades terrenas sin dejarnos sumergir en ellas, sino al contrario, guiándolas hacia las celestiales; nos exhorta a prepararnos bien para esto último, para que la venida del Señor no nos encuentre desprevenidos y mal dispuestos;
	
	- "Vigilidad": el espíritu del hombre "despertado" a la realidad divina, atraído por ella a sus destinos eternos en Dios, debe animar toda temporalidad con una nueva conciencia.
	
	6. Queridos hermanos y hermanas {[}de la parroquia de San Felipe Neri{]}, mi afectuoso saludo va para todos vosotros, ya que agradezco al Señor por haberme concedido este encuentro. (...)
	
	{[}...{]}
	
	Una comunidad parroquial unida y ferviente puede jugar, con la fuerza del Espíritu Santo, un papel esencial en la reducción de la distancia entre el modelo evangélico que propone al mundo y las condiciones reales del mundo mismo, siempre en cierta medida refractario, mientras estemos aquí abajo, a la llamada evangélica, a la conversión y a la penitencia. Este hecho, sin embargo, lejos de debilitar el testimonio que se da al mundo, debe fortalecerlo cada vez más, en la muy firme convicción de que el mundo, a pesar de todo, tiene una necesidad absoluta de Jesús crucificado y resucitado. El poder de su gracia, especialmente a través del carisma de los laicos cristianos, puede y debe penetrar y animar evangélicamente todos los ambientes seculares de la familia y el trabajo, la escuela, la sociedad y la cultura.
	
	7. En el primer domingo de Adviento (...) espero sinceramente que esta {[}celebración{]} nos permita abrir más los ojos del alma a la realidad divina y, por así decirlo, volver a despertar a ella. Que nos permita transformarnos interiormente y que al mismo tiempo nuestra humanidad se revista del Señor Jesucristo de forma cada vez más madura.
	
	Con nueva alegría, avancemos hacia el encuentro con el Señor que debe venir, como cada año, en la solemnidad de la Navidad; hacia el Señor con quien también debemos encontrarnos al final de nuestros caminos terrenales. De hecho, el Adviento nos recuerda, cada año, que la vida humana no es un camino infranqueable hacia Dios, sino un verdadero camino que Él mismo ha hecho suyo por la Palabra divina. 
\end{body}

\subsubsection{Homilía (1986): }

\src{Santa Misa en el Parque Victoria, Adelaide (Australia), 30 de noviembre de 1986.}

\begin{body} 
	\emph{"¡Qué alegría cuando me dijeron: `Vamos a la casa del Señor'!''} (\emph{Sal} 122 (121), 1).
	
	\emph{Amados hermanos y hermanas en Cristo.}
	
	1. Con estas palabras toda la Iglesia proclama la alegría del Adviento. Hoy es el primer domingo de Adviento. Nos acercamos a la noche en que los pastores, en los campos alrededor de Belén, sintieron la alegría de ser llamados por los ángeles para ir a ver al Señor: "Vamos a Belén, veamos este acontecimiento que el Señor nos ha dado a conocer" (\emph{Lc} 2, 15). Sí, incluso hoy, aquí (...), la Iglesia nos recuerda que el Señor está cerca (cf. \emph{Fil} 4, 5). Y como los pastores en aquella noche espléndida en Belén, también nosotros decimos: ``Iremos a la casa del Señor''.
	
	El Adviento es el tiempo de preparación para la Navidad, para la venida del Salvador. Nos llama a ``ir con gozo a la casa del Señor... a alabar allí el nombre del Señor'' (\emph{Sal} 121, 4). Alabar el nombre del Padre Altísimo, del Hijo y del Espíritu Santo: esta es la primera intención de nuestra celebración eucarística.
	
	2. ``¡Qué alegría cuando me dijeron: `vamos a la casa del Señor'!''. Con la misma alegría que nos comunica la liturgia de Adviento os saludo a todos vosotros aquí reunidos (...) caminando con vosotros hacia la casa del Señor, por los caminos de ese Adviento que es la historia del hombre: el Adviento en el que toda la familia humana y todos los creados esperan la segunda venida de nuestro salvador Jesucristo.
	
	{[}Solo unos pocos años nos separan del final del segundo milenio y el comienzo del tercer milenio de la era cristiana.{]} Este es un tiempo de gracia para la Iglesia. Es un tiempo en el que los seguidores de Jesucristo, en medio de las profundas transformaciones que cambian la cultura y la sociedad, debemos volver a dedicarnos a la vida cristiana. Es un momento en el que el mensaje del Evangelio debe ser proclamado a los hombres y mujeres de esta época con la fuerza de un nuevo Pentecostés. Es un tiempo en el que el mismo Espíritu de verdad habla con claras palabras de vida a la familia humana.
	
	3. En la celebración eucarística de este primer domingo de Adviento, oremos por la realización del proyecto del Padre para la familia humana: "En ese mundo nuevo donde se nos revelará la plenitud de la paz, reúne también a los hombres de toda raza y lengua, en el banquete de la unidad eterna, en los cielos y en la tierra nueva, donde brille la plenitud de tu paz'' (II \emph{Plegaria Eucarística de Reconciliación}).
	
	En otras palabras, oremos para que se haga realidad la visión del profeta Isaías, como decía la \textbf{primera lectura}: ``Todas las naciones confluirán al monte del Señor ... para que nos muestre sus caminos y caminemos por sus sendas'' (\emph{Is} 2, 2-3). El deseo de ese tiempo de gracia y paz está profundamente arraigado en nuestros corazones. ¿Quién no anhela que llegue ese tiempo final cuando "un pueblo ya no alzará la espada contra otro pueblo, ya no practicará el arte de la guerra"? (\emph{Is} 2, 4). De hecho, existe un tiempo de Adviento que es universal y que dura tanto como la historia humana. Hoy meditamos en la visión de \textbf{Isaías} sobre un número incalculable de personas que marchan hacia el monte del Señor: el pueblo de Dios de todas las épocas y de todos los lugares que se reúnen en unión con él y en la unidad entre ellos mismo en la Iglesia. Y deberíamos reflexionar sobre cómo esta visión se materializa en la realidad concreta de la vida {[}actual{]}, en la historia y en la cultura (...). Esta asamblea eucarística en sí misma es un símbolo de la visión del profeta. Sois personas aquí reunidas de "toda raza, lengua y forma de vida", hechos uno en Jesucristo y en su Iglesia.
	
	{[}...{]}
	
	5. La \textbf{palabra de Dios} nos ha llamado a estar atentos y vigilantes, a revestirnos de las armas de Jesucristo: "Ha llegado el momento de despertar del sueño ... La noche está avanzada, el día está cerca: dejemos, pues, las obras de las tinieblas y pongámonos las armas de la luz. Andemos como en pleno día, con dignidad. Nada de comilonas y borracheras, nada de lujuria y desenfreno, nada de riñas y envidias. Revestíos más bien del Señor Jesucristo." (\emph{Rom} 13, 11-14).
	
	Cualquier expresión de hostilidad hacia los demás levanta un muro de tensión entre las personas y revela un corazón de piedra. Cualquier acto de discriminación es un acto de injusticia y una violación de la dignidad de la persona. Siempre que somos intolerantes, cerramos los ojos a la imagen de Dios que está en la otra persona. Hoy, cuando no reconocemos las necesidades de justicia en el mundo, no captamos el sentido de solidaridad universal. Pero cuando hablamos con palabras amables, cuando nos respetamos y nos honramos mutuamente, cuando mostramos una verdadera amistad, cuando ofrecemos hospitalidad, cuando nos esforzamos por comprender las diferencias entre los pueblos, nos convertimos en el signo vivo de que la visión de Isaías se ha hecho realidad, que el reino de Dios ha llegado entre nosotros, que el advenimiento universal de la historia avanza hacia su cumplimiento.
	
	6. La Iglesia hoy nos invita a todos y cada uno de nosotros a emprender con gusto y alegría el camino que Dios ha preparado para todo el género humano. El \textbf{profeta Isaías} habla del camino que sube al monte del Señor, al templo del Dios de Jacob (cf. \emph{Is} 2, 3). Parte de este "ascenso" es la vocación del hombre a buscar una humanidad plena y auténtica, a perfeccionar y desarrollar sus propias cualidades espirituales y físicas en la lucha por dominar el mundo, a través del progreso del conocimiento y mediante su propio esfuerzo. Esto es lo que hace la familia humana a través del progreso cultural (cf. \href{http://www.vatican.va/archive/hist_councils/ii_vatican_council/documents/vat-ii_const_19651207_gaudium-et-spes_it.html}{\emph{\emph{Gaudium et Spes}}}, 53).
	
	Los hombres y mujeres de hoy saben claramente que, hoy más que nunca, están llamados a construir su propio destino en este mundo. Los medios de que disponen son cada vez mayores: un mejor conocimiento de la tierra y sus secretos, un mejor conocimiento de la persona humana y de la actividad humana; una mejor comprensión del curso de la historia y la organización social; y el mundo de las comunicaciones, que brinda a cada vez más personas la oportunidad de participar en el progreso moderno. Un mundo más humano está luchando por nacer. Y, sin embargo, cada vez las mayores esperanzas van acompañadas de inquietantes contradicciones. En lo que respecta al respeto de los derechos humanos fundamentales, las últimas décadas han sido testigos de grandes avances y de una creciente conciencia de la justicia de esta causa. Sin embargo, no podemos pasar por alto el hecho de que nuestro mundo todavía ofrece demasiados ejemplos de gran injusticia y opresión. Donde hay un gran bien por alcanzar, se necesita una gran madurez moral y un gran sentido de la justicia. Sin la visión de la sublime dignidad de la persona humana --- dignidad fundada en la relación única y personal con el Creador y Redentor, dignidad ligada a la naturaleza trascendente, al origen y destino del hombre --- el progreso no tendrá un rumbo seguro. Jesucristo, camino, verdad y vida, nos revela el verdadero sentido de la historia. Nos revela el plan de Dios para la humanidad. Jesús habla de nuestra libertad y nos llama a promover el verdadero progreso humano, dándonos su ley de amor y servicio; "Este es mi mandamiento: amaos los unos a los otros como yo os he amado" (\emph{Jn} 15, 12). El Evangelio purifica y fortalece toda cultura, para que pueda ayudar al hombre "a subir al monte del Señor ... para que nos muestre sus caminos y recorramos sus sendas" (\emph{Is} 2, 3).
	
	7. El llamado de Jesús es claro. Él dice: "Velad". Y de nuevo: "Estad preparados, porque a la hora que no imaginéis, vendrá el Hijo del Hombre" (\emph{Mt} 24, 42, 44). Así exhorta a todos sus seguidores a trabajar por la meta que el Padre se ha marcado: el reino de la justicia, la verdad y la paz. Por lo tanto, insta a los fieles (...) a encontrar un remedio cuando las injusticias puedan dañar la vida de su nación y a garantizar que un nuevo espíritu de reconciliación anime toda la vida nacional. Jesús nos dice que seremos juzgados por la forma en que respondamos a su presencia en los hambrientos, desnudos, enfermos y presos (cf. \emph{Mt} 25, 35-36).
	
	Queridos hermanos y hermanas: estáis llamados a participar con Dios en la construcción de su reino en el corazón de todos (...), corazones de carne y no de piedra. {[}En este día estamos invitados{]} a ver nuestra historia en el contexto del amor eterno de Dios por toda la familia humana, que se manifiesta en la misión salvífica de Jesucristo. Es una historia que aún está evolucionando. Y presenta muchos desafíos (...). Este también es el Adviento, lleno de expectación, que la Iglesia celebra en este período. Nosotros, pueblo peregrino de Dios, caminamos siguiendo a Jesús, que es el camino al Padre. Caminamos con la certeza de que su verdad nos hará libres y que nuestra fuerza proviene de sus palabras y sus sacramentos.
	
	8. Con la mirada puesta en Aquel que ha de venir, miremos, pues, a todos (...) e invoquemos la bendición del Salmo: "Haya paz dentro de tus muros, seguridad en tus palacios \ldots{} Por mis hermanos y compañeros, voy a decir: \textquote{La paz contigo}. Por la casa del Señor, nuestro Dios, te deseo todo bien." (\emph{Sal} 121, 7-9).
	
	{[}Para ti: Adelaide! Para ti: ¡Australia! Para ti: ¡mundo entero!{]}
	
	Y seguimos mirándole a Él, el que ha de venir, el "Príncipe de la paz" (cf. \emph{Is} 9, 6). El \textbf{profeta} dice de él:
	
	"Juzgará entre las naciones,
	
	será árbitro de pueblos numerosos.
	
	De las espadas forjarán arados,
	
	de las lanzas, podaderas.
	
	No alzará la espada pueblo contra pueblo,
	
	no se adiestrarán para la guerra.
	
	Casa de Jacob, venid;
	
	caminemos a la luz del Señor."
	
	(\emph{Is} 2, 4-5).
	
	Esta es la luz del Adviento. Es la luz del Adviento que se despliega ante la familia humana, hasta que el Señor vuelva en gloria: el Adviento de la responsabilidad del hombre por la vida y por el mundo que el Creador ha puesto en sus manos. La luz es la luz del que vendrá, el príncipe de paz, es la luz de Cristo. ¡Que la luz de Cristo brille sobre {[}nuestra tierra{]} (...)! Que la luz de Cristo brille sobre cada uno de vosotros. Amén. 
\end{body}

\subsubsection{Homilía (1989): } Visita Pastoral a la Parroquia de Santo Tomás Apóstol en Castel Fusano.

3 de diciembre de 1989.

\begin{body} 
1. "Vayamos gozosos al encuentro del Señor".

Estas palabras del \textbf{salmo responsorial}, que hemos repetido juntos, pueden considerarse con razón el programa de la Iglesia al comienzo del año litúrgico; especialmente al comienzo del Adviento, que constituye la primera etapa importante del año litúrgico.

Entramos conscientemente en un tiempo "propicio" de salvación: la comunidad cristiana, de hecho, mientras se prepara para conmemorar la primera venida de Cristo "en la humildad de nuestra naturaleza humana" (\emph{Prefacio del Tiempo de Adviento}), es llamada a orientarse, con esperanza confiada y expectación activa, hacia la venida definitiva del Señor "en el esplendor de su gloria" (\emph{Prefacio del Tiempo de Adviento}). Todo ello en dócil apertura de fe a la Palabra de Dios, que ilumina nuestro camino y con una participación activa en los acontecimientos sacramentales, que nos insertan en el misterio de Cristo, "hasta que él venga" (cf. \emph{1 Co} 4, 5).

En el momento en que la Iglesia emprende el itinerario salvífico del año litúrgico, la Palabra de Dios, recién escuchada, pone inmediatamente ante nosotros la meta hacia la que nos dirige el Espíritu: la nueva Jerusalén, símbolo de la comunión plena y definitiva a la que Dios invita y admite a todos aquellos que dicen el "sí" de la fe a Cristo, "maestro de la verdad y fuente de reconciliación" (\emph{Oración colecta}), y se hacen heraldos y testigos entre los hermanos, para que el Dios de la paz sea en todo en todos cuando "venga el Hijo del Hombre" (\emph{Evangelio}).

2. Entonces entendemos las palabras del \textbf{profeta}: "Casa de Jacob, ven, caminemos a la luz del Señor" (\emph{Is} 2, 5).

Esta invitación resuena hoy con acentos de especial relevancia y urgencia para {[}todos nosotros{]} (...) Todos debemos emprender un camino de conversión, superando la tentación de la inercia, la desconfianza y la pasividad.

Muchos, de hecho, que también profesan ser cristianos, viven en una especie de letargo y mediocridad. A menudo, su vida moral contrasta con la fe, que también dicen tener; no pocos limitan su pertenencia a Cristo y a la Iglesia a una práctica religiosa ocasional, compuesta sólo por citas religiosas ocasionales; se alejan de las responsabilidades específicas, contentándose con delegar en los demás la misión evangelizadora propia de cada miembro activo de la comunidad eclesial.

3. A todos ellos quiero repetir el llamamiento que \textbf{san Pablo} dirigió a los primeros fieles de la Iglesia de Roma, y ​​que la liturgia nos ha vuelto a proponer hoy: "Hermanos, es hora de despertarnos del sueño. , porque nuestra salvación está más cerca que cuando empezamos a creer. La noche está avanzada, el día está cerca. Desechemos las obras de las tinieblas y vistámonos las armas de la luz'' (\emph{Rom} 13, 11-12).

Ante la situación de indiferencia e imprevisibilidad que describe el Evangelio y que se refleja en la mentalidad y las costumbres de hoy, la Iglesia (...) {[}en su camino hacia el final del segundo milenio de la era cristiana{]}, no puede ni debe permanecer inerte y pasiva: está impulsada por la voluntad misma de Dios, reconocida y aceptada en unión con su propio Obispo, a tomar la decisión fundamental de renovarse en la vocación y misión que le ha confiado la Providencia.

El Vaticano II volvió a proponer la verdad sobre Dios y el hombre con nueva luz. Aceptar esta luz, dejarse penetrar por ella más íntimamente, anunciarla a todos es un deber de todo cristiano, para que la salvación en Cristo Jesús, que se ofrece a todos, esté más cerca de cada uno. Es un viaje para hacer "juntos". Nadie puede ignorarlo o rehuirlo (...). El mundo, y en particular los hombres que viven en esta ciudad, no podrán ver y recibir la luz de Cristo si sus discípulos son opacos, tibios y pasivos.

{[}...{]}

5. Queridos hermanos y hermanas, quisiera hacer una última reflexión con ustedes, que se me ha sugerido desde el comienzo del año litúrgico.

Los obispos italianos han afirmado en varias ocasiones que el año litúrgico ``constituye el gran camino de fe del pueblo de Dios: toda la comunidad, especialmente en tiempos de gran fuerza, está llamada a redescubrir, celebrar y vivir el don de la salvación. A través de la pedagogía de los ritos y de la oración, todos somos conducidos juntos a la experiencia del misterio pascual de Cristo, que tiene su centro en la Eucaristía'' (Conferencia Episcopal Italiana, \emph{Eucaristía, comunión y comunidad}, 89; cf. etiam Eiusdemn \emph{Evangelización y sacramentos}, 85).

{[}Estamos llamados a establecer{]} en nuestras comunidades itinerarios educativos serios y continuos, en los que se fundan juntos la escucha de la Palabra de Dios, la celebración de los santos misterios y el testimonio, el servicio de la caridad y la promoción del hombre, superando así la fragmentación y el ocasionalismo por un lado y, por otro, el impulso de avanzar por caminos paralelos o divergentes.

El año litúrgico es uno de estos itinerarios, de hecho es el privilegiado de la Iglesia, no solo porque es el más adecuado para todas las edades y diferentes categorías de personas, sino sobre todo porque es el más completo, si se vive con autenticidad y es valorado en todas las posibilidades que ofrecen sus diferentes tiempos y momentos y por la riqueza de los signos litúrgico\emph{-}sacramentales.

La creatividad pastoral, en fidelidad a la Tradición genuina de la Iglesia, podrá promover, especialmente para jóvenes y adultos, otros itinerarios de fe incluidos en el año litúrgico, o en todo caso en sintonía con él, tanto con motivo de la celebración de los sacramentos y, más en general, por una vida cristiana más madura y una fe más consciente y activa.

{[}Confío estos compromisos a ese gran testigo de la fe que fue santo Tomás Apóstol, a quien está dedicada vuestra comunidad, para que toda la Iglesia (...) profese su fe en Cristo resucitado y coseche los frutos de la salvación, para sí misma y para los hombres que viven en la ciudad.

``Casa de Jacob, ven. . . "; {[}parroquia de Santo Tomás Apóstol en Castel Fusano{]}, ``ven, caminemos en la luz del Señor''.

¡Amén!
end{body} 

\subsubsection{Homilía (1992): }

VISITA A LA PARROQUIA DE SAN GERARDO MAIELLA

\emph{\textbf{HOMILIA DE JUAN PABLO II}}

\emph{Domingo 29 de noviembre de 1992}

\begin{body} 
1. \emph{"Vayamos gozosos al encuentro del Señor"} (Salmo responsorial).

¡Queridos hermanos y hermanas (...)! Hoy, primer domingo de Adviento, comienza un nuevo año litúrgico durante el cual la Iglesia recorre y revive espiritualmente las etapas del misterio cristiano.

Este plan divino abarca toda la historia de la humanidad, desde los albores de la creación hasta el día final, cuando todas las cosas serán recapituladas en Cristo (cf. \emph{Ef} 1, 10) y habrá nuevos cielos y nueva tierra (cf. \emph{2 Pt.} 3, 13). El centro de este proyecto está en el misterio de la encarnación del Hijo de Dios.

En un momento preciso, por obra del Espíritu Santo, el Verbo "se hizo carne" en el seno virginal de María y "vino a habitar en nosotros". (cf. \emph{Jn} 1, 12), mostrando la bondad y humanidad de Dios para con los hombres. El Señor, de hecho, no sólo creó al hombre, sino que lo ama con tanta intensidad que lo acoge desde dentro de su propia familia, destinándolo a una gloria sin fin.

De hecho, sostenidos por una certeza tan consoladora vamos con alegría a su encuentro, como nos invita a hacer el \textbf{Salmo Responsorial}.

2. Vamos a encontrarlo en el misterio de la Navidad. Este es el primer sentido del Adviento. Nos conmueve el recuerdo de María y José, que suben de Nazaret de Galilea a Belén de Judea para el censo, y se ven obligados a refugiarse en un lugar destinado a los animales, "porque no había lugar para ellos en la posada" (\emph{Lc} 2, 7).

El Hijo de Dios viene a la luz en la pobreza total: verdadero Dios Salvador, anunciado por los ángeles a los pastores, y verdadero hombre, envuelto en pañales y colocado en un pesebre.

¡Qué sentimientos de ternura, amor y gratitud despierta este extraordinario evento! Sin embargo, también tiene la fuerza de sacudir nuestra conciencia invitándonos a despertar del sueño de la indiferencia y de la habitud.

"Hermanos - nos exhorta el \textbf{apóstol Pablo} - ya es hora de despertar del sueño" (\emph{Rm} 13, 11), Dios nos ha amado hasta darnos a su Hijo único. ¿Acaso un don tan grande no nos obligua a reflexionar y a responderle con la generosidad adecuada? ¿No nos empuja a abandonar las tinieblas del pecado para abrir el espíritu a la luz de la gracia divina? Esto es precisamente a lo que nos invitan las lecturas de la liturgia de hoy.

3. "Venid, subamos al monte del Señor" (\emph{Is} 2, 3).

El texto, tomado del libro del \textbf{profeta Isaías}, se interpreta comúnmente como un anuncio mesiánico. Para el pueblo de Israel forzado al exilio, el profeta predice la reconstrucción del Templo de Jerusalén. Pero sus palabras van más allá de la historia del pueblo judío. Con imágenes como la del cerro más alto de todas las montañas y con el pronóstico de la venida de innumerables naciones al templo del Dios de Jacob, se indica una nueva realidad espiritual, la del pueblo de los redimidos guiados por el Mesías prometido; y la de una nueva Alianza, que transforma profundamente la vida de los hombres, según los ideales de paz y fraternidad, convirtiendo las espadas en arado y las lanzas en podaderas.

4. ¡Queridos hermanos y hermanas! Vuestra comunidad parroquial, una de las más jóvenes de nuestra Iglesia local, está animada por estos ideales de paz y fraternidad, santidad y evangelización. (...)

(...) Gracias a una catequesis capilar, habéis intentado ayudar a los habitantes de todo el barrio a no retirarse al individualismo, sino a crecer como comunidad cristiana solidaria, siguiendo el ejemplo de la Iglesia de los Apóstoles y de las primeras generaciones de creyentes, a través de una atenta escucha de la Palabra, participación en la vida litúrgica de la Comunidad y un intenso esfuerzo de compartir y acogida recíproca. De esta manera, se han desarrollado experiencias comunitarias significativas y formas efectivas de catequesis para adultos, como las Comunidades Neocatecumenales. Muchos jóvenes pertenecen se reúnen también en la Acción Católica y en otros movimientos apostólicos, Grupos de matrimonios cristianos siguen un itinerario formativo común estructurado en encuentros periódicos (...) La parroquia tampoco carece de la sensibilidad misionera que brotó del hermanamiento con una misión africana en Chad.

5. Por todo ello (...) deseo manifestar mi viva satisfacción, reconociendo el compromiso y la generosidad que os animan.

Perseverad, queridos hermanos y hermanas, en el esfuerzo realizado. \emph{Daréis, pues, vida a una nueva evangelización} (...). Todavía hay muchas personas que no conocen el evangelio adecuadamente y esperan el testimonio constante de nuestra vida y la proclamación gozosa de nuestra fe en Cristo.

A vosotros os encomiendo una misión tan exigente: a las familias, a los adultos, a los niños, a los ancianos y especialmente a los jóvenes, y os aseguro a cada uno de vosotros el apoyo de mi cordial oración.

Este inicio del tiempo de Adviento constituye una ocasión propicia para intensificar el ritmo de nuestra vida cristiana.

6. "Vuestro Señor vendrá" (\emph{Mt} 24, 42).

El pasaje del \textbf{Evangelio de Mateo} abarca una parte del discurso de Jesús sobre los últimos acontecimientos, que por eso se llama discurso escatológico.

Jesús anuncia su segunda venida, al final de los tiempos, y nos exhorta a estar alerta y preparados para encontrarnos con Él. Este es el segundo significado del Adviento.

Por las palabras de Jesús, contenidas en este y otros textos, sabemos con certeza que el mundo presente está destinado a terminar, que la historia humana terminará, que para cada uno habrá un juicio, seguido de una recompensa o castigo. A la luz de todo esto es importante escuchar la invitación a velar "porque no sabes en qué día vendrá tu Señor".

7. ¡``Velad, pues''!

La vigilancia evangélica es condición para un buen uso de la vida.

Qué fácil es desperdiciar los dones divinos, distanciarse de Dios con pensamientos y comportamientos, olvidar que la vida pasa.

Las cosas temporales son frágiles y pasajeras, son útiles si se utilizan como medio para crecer en la bondad, para sanar el alma y servir al Señor y a los hermanos con amor; pero si se convierten en el objetivo principal de la vida, vacían a las personas de su núcleo más importante y las convierten en apéndices de las realidades materiales.

Vayamos al encuentro del Señor que viene con buenas obras. "La noche ha avanzado y el día está cerca" (\emph{Rom} 13, 12). El apóstol Pablo nos exhorta a desechar las obras de las tinieblas y revestirnos de las armas de la luz, vestirnos del Señor Jesús y no seguir a la carne en sus deseos desordenados.

Preparémonos con esmero para la Navidad que viene, sobre todo orientando nuestra vida hacia ese Dios con el que el último día nos encontraremos cara a cara, con amor y alegría.

``Prepárate, porque a la hora que no imaginas, vendrá el Hijo del Hombre''.

Por tanto, velad, vestíos de Cristo.

Nuestra salvación está ahora cerca.

¡Amén!
\end{body} 
	
\subsubsection{Homilía (1995): }

Domingo 3 de diciembre de 1995. Canonización de Eugène de Mazenod, fundador de los Misioneros Oblatos de la Inmaculada.

\begin{body} 
1. La venida del \emph{Hijo del Hombre es el tema del Adviento}. Así comienza el tiempo del nuevo Año Litúrgico. Ya miramos hacia la noche de Belén. Pensemos en esa venida del Hijo de Dios que ya pertenece a nuestra historia, de hecho, \emph{de una manera maravillosa la formó} como la historia de los individuos, las naciones y la humanidad. También sabemos con certeza que, después de esa venida, tenemos \emph{ante nosotros para siempre} una segunda venida del Hijo del Hombre, de Cristo. Vivimos en el segundo Adviento, en el Adviento de la historia del mundo, de la historia de la Iglesia, y en la celebración eucarística repetimos todos los días nuestra confiada esperanza de su venida.

{[}...{]}

3. En la liturgia de este primer domingo de Adviento comienza a hablar el \textbf{profeta Isaías}. Escucharemos la palabra inspirada de todo este tiempo. ``Visión de Isaías, hijo de Amós, acerca de Judá y Jerusalén. Al final de los días, la montaña del templo del Señor se elevará en la cumbre de las montañas y será más alta que las colinas; todos los pueblos acudirán a él. Muchos pueblos vendrán y dirán: "Venid, subamos al monte del Señor, al templo del Dios de Jacob, para que él nos muestre sus caminos y podamos caminar por sus sendas". Porque de Sion saldrá la ley, y de Jerusalén la palabra del Señor'' (\emph{Is} 2, 1-3).

A la luz del Espíritu Santo, el Profeta tiene \emph{una visión universalista} y muy aguda de la salvación. Jerusalén, la ciudad ubicada en medio de Israel, Pueblo de elección divina, tiene un gran futuro por delante. Cuando el Profeta dice que "la palabra del Señor saldrá ... de Jerusalén", ya muchos siglos antes de la venida de Cristo, anuncia el alcance de la obra mesiánica.

La mirada de Isaías \emph{enriquece nuestra conciencia del Adviento}. El que ha de venir, que debe revelarse "hasta el fin" en medio de la ciudad santa de Jerusalén, por la palabra de su Evangelio, y especialmente por su cruz y su resurrección, será enviado a todas las naciones del mundo, a toda la humanidad. Será \emph{el Ungido de Dios, el Redentor del hombre}. Su visita no duró mucho, pero la misión que él transmitió a los Apóstoles y a la Iglesia perdurará hasta el final de los siglos. Será mediador entre Dios y los hombres, y en voz alta exhortará a las naciones a la paz, invitando a todos a "forjar de sus espadas arados, de sus lanzas podaderas" (cf. \emph{Is} 2, 4). Así comienza la exhortación de Isaías, dirigida a los pueblos de toda la tierra, a que dirijan la mirada y los pasos hacia Jerusalén.

Esta exhortación tiene su eco en el \textbf{Salmo Responsorial}, \emph{canto de los peregrinos} a la Ciudad Santa. ````Qué alegría cuando me dijeron: Vamos a la casa del Señor''. Ya están pisando nuestros pies tus umbrales, Jerusalén. Allí suben juntas las tribus, las tribus del Señor'''' (\emph{Sal} 122, 1, 4). Y nuevamente: Pide la paz para Jerusalén: ``la paz sea con los que te aman, la paz sea en tus muros, la seguridad en tus baluartes" (\emph{Sal} 122: 6-7).

{[}...{]}

6. El mensaje de Adviento está unido a la venida del Hijo del Hombre, cada vez más cercano. A esta conciencia corresponde \emph{la exhortación a la vigilancia}. En el \textbf{Evangelio de San Mateo}, Jesús dice a los que le escuchan: "Velad, pues, porque no sabéis en qué día vendrá el Señor ... Por tanto, también vosotros estad preparados, porque a la hora que menos penséis vendrá el Hijo del Hombre'' (\emph{Mt} 24, 42, 44). El pasaje de la \textbf{Carta de San Pablo a los Romanos} corresponde de manera excelente a esta exhortación, repetida varias veces en el Evangelio. El Apóstol nos dice cómo podemos ser \emph{"conscientes del momento"} (cf. \emph{Rm} 13,11). \emph{La espera}, volcada hacia el futuro, \emph{siempre} se nos presenta \emph{como un "momento" ya cercano y presente}. En la obra de salvación no se puede dejar nada para después. \emph{¡Cada "hora" importa!} El Apóstol escribe que "nuestra salvación está más cerca ahora que cuando empezamos a creer" (\emph{Rom} 13,11) y compara este momento presente con el amanecer, con el momento culminante del paso entre la noche y el día.

San Pablo traslada el fenómeno que acompaña al despertar de la luz del día al ámbito espiritual. ``La noche está avanzada --- escribe --- el día está cerca. Desechemos, pues, las obras de las tinieblas y vistámonos de las armas de la luz'' (\emph{Rom} 13,12). Después de haber llamado por su nombre las obras de las tinieblas, el Apóstol indica a qué aluden "las armas de la luz": \emph{"vistámonos con las armas de la luz"}, es decir, "vistámonos ... del Señor Jesucristo" (\emph{Rom} 13,14 ). El apóstol nos invita a hacer de Jesucristo la norma de nuestra vida y de nuestras acciones, para que en Él podamos llegar a convertirnos en una nueva creación. Así renovados, podremos renovar el mundo en Cristo, en virtud de la misión, ya injertada en nosotros por el sacramento del Bautismo.

{[}...{]}
\end{body} 
	
\chapter{II-Adviento}

\subsubsection{Homilía (1983): } Visita Pastoral a la Parroquia Romana de Santa Francesca Saverio Cabrini, 4 de diciembre de 1983.

\begin{body} 
\emph{Queridos fieles:}

1. Este segundo domingo de Adviento se desarrolla íntegramente sobre los temas de la venida de Cristo y la preparación necesaria para este maravilloso acontecimiento.

En el centro de la liturgia está la persona de \textbf{Juan el Bautista}. Mateo el evangelista lo describe como un hombre de intensa oración, de austera penitencia, de profunda fe: de hecho, es el último de los profetas del Antiguo Testamento, que actúa como un pasaje al Nuevo, indicando que es Jesús el Mesías esperado por los judíos. En las orillas del río Jordán, Juan el Bautista confiere el bautismo de penitencia: mucha gente "acudía a él desde Jerusalén, de toda Judea y de la zona adyacente al Jordán; y, confesando sus pecados, fueron bautizados por él" (\emph{Mt} 3, 5-6). Este bautismo no es un simple rito de adhesión, sino que indica y exige el arrepentimiento de los pecados y un sentido sincero de expectativa del Mesías.

Y Juan enseña. Predica la conversión: "convertíos, porque el Reino de los cielos está cerca".

Juan enseña. Y de acuerdo con la profecía de Isaías, "endereza los caminos" para el Señor (cf. \emph{Mt} 3, 1-3).

2. Incluso hoy nos resuena esta palabra.

¿Quién es ese Señor que ha de venir? De sus propias palabras podemos calificar la persona, misión y autoridad del Mesías.

Juan el Bautista ante todo enmarca claramente su "persona": "él -\/- dice el Bautista -\/- es más poderoso que yo y ni siquiera soy digno de desatar la correa de sus sandalias" (\emph{Mt} 3, 11). Con estas palabras típicas orientales reconoce la distancia infinita que se abre entre él y el que está por venir, y subraya también su tarea de preparación inmediata para el gran acontecimiento.

Luego indica la misión del Mesías: "Él os bautizará en Espíritu Santo y fuego" (\emph{Mt} 3, 12). Es la primera vez que, después del anuncio del ángel a María, aparece la impresionante dicción "Espíritu Santo", que luego será parte de la enseñanza trinitaria fundamental de Jesús. Juan el Bautista, divinamente iluminado, anuncia que Jesús, el Mesías, continuará impartiendo el Bautismo, pero este rito dará la "gracia" de Dios, el Espíritu Santo, entendido bíblicamente como un "fuego" místico, que anula (quema) el pecado insertándonos en la vida divina misma (enciende la llama de su amor).

Finalmente, el Bautista aclara la autoridad del Mesías: "Tiene el aventador en la mano, limpiará su era y recogerá su grano en el granero, pero quemará la paja con un fuego inextinguible" (\emph{Mt} 3, 12). Según la palabra de la enseñanza de Juan, el que vendrá es el "juez de conciencias"; es decir, es él quien establece lo bueno y lo malo (el grano y la paja), la verdad y el error; es él quien determina qué árboles dan buenos frutos y cuáles dan frutos malos y cuáles deben ser cortados y quemados. Con estas afirmaciones Juan Bautista anuncia la "divinidad" del Mesías, porque sólo Dios puede ser el árbitro supremo del bien, indicar con absoluta certeza el camino positivo de la conducta moral, juzgar las conciencias, premiar o condenar.

De ahí la necesidad de prepararse para la venida del Mesías. Sin duda, la Navidad es un día de gran y serena alegría, incluso desde fuera; pero es sobre todo un acontecimiento sobrenatural y decisivo, para el que es necesaria una seria preparación moral: "¡Preparad el camino del Señor! ¡Enderezad sus senderos!". En palabras de Juan, es toda la herencia profunda de la Antigua Alianza.

3. Pero, al mismo tiempo, se abre ante ellos la Nueva Alianza: en el que ha de venir "todo hombre verá la salvación de Dios" (\emph{Lc} 3, 6). El que viene, Cristo, es enviado "para reuniros para la gloria de Dios" (\emph{Rom} 15, 7). Viene para demostrar la ``veracidad de Dios, para cumplir las promesas de los Padres ..." (\emph{Rom} 15, 8); viene a revelar que el Señor es "el Dios de perseverancia y consolación" (\emph{Rom} 15, 5); viene a "acogeros para la gloria de Dios" (\emph{Rom} 15, 7).

Y, por tanto, Aquel que viene debe ocuparse de que "os acojáis unos a otros" (\emph{Rom} 15, 7). De hecho, indica la conducta moral verdadera y auténtica, que consiste en dar gloria a Dios Padre, siguiendo su ejemplo y con los mismos sentimientos, y en amar al prójimo. San Pablo, escribiendo a los Romanos, tenía en mente tanto a los conversos del judaísmo como a los del paganismo; pero para todos habla del compromiso de ``acoger'': la Palabra de Dios, que viene, debe hacer que ``tengáis los mismos sentimientos los unos hacia los otros, siguiendo a Cristo Jesús ... (cf. \emph{Rm} 15, 5) para que con un solo corazón y una sola voz den gloria a Dios Padre" (cf. \emph{Rm} 15, 6).

4. Así, pues, el "enderezar los caminos", predicado por \textbf{Juan Bautista}, se hace a la luz de la enseñanza de san Pablo en la \textbf{Carta a los Romanos}, acogiendo todo el programa mesiánico del Evangelio: el programa de adoración a Dios -\/- ¡la gloria! -\/- por el amor del hombre, el amor mutuo. Con este espíritu, la Iglesia anuncia el Adviento como dimensión continua de la existencia del hombre hacia Dios: hacia ese Dios "que es, que era y que ha de venir" (\emph{Ap} 1, 4).

Esta dimensión esencial de la existencia cristiana del hombre corresponde a la "preparación" enseñada por la liturgia de hoy. El hombre debe volver siempre al corazón, a la conciencia, para existir en la perspectiva de la "Venida".

Para cumplir con este requisito, el cristiano también debe ser sensible a la acción del Espíritu Santo: el que viene, viene en el Espíritu Santo, como anunció \textbf{Isaías}: "Sobre él reposará el espíritu del Señor, espíritu de sabiduría e inteligencia espíritu de consejo y fortaleza, espíritu de conocimiento y temor del Señor" (\emph{Is} 11, 2). Con el Mesías y con la presencia del Espíritu Santo, la justicia y la paz entran en la historia humana como dones del reino de Dios: abre así la perspectiva de la reconciliación "cósmica" en toda la creación, en el hombre y en el mundo, que había sido perdida debido al pecado. "Ven, Señor, rey de justicia y de paz": rezamos juntos en el \textbf{salmo responsorial}.

5. {[}¡Queridos fieles de la parroquia de Santa Francesca Saverio Cabrini! Estoy muy feliz de estar hoy con ustedes, de encontrarme con ustedes y de haber podido realizar con ustedes esta meditación sobre las lecturas litúrgicas, tan llenas de luz sobrenatural, afirmaciones consoladoras y pautas concretas para la vida diaria.{]}

{[}...{]}

6. (...) "Preparad el camino del Señor". Este mensaje es actual, siempre y para todos. De hecho, todos vivimos en la dimensión del advenimiento de Dios. Nuestra vida es una "preparación" continua.

Finalmente, ruego a la Madre de Aquel que debe venir, para que los bienes mesiánicos del reino de Dios, la justicia y la paz, sean compartidos por vosotros en Jesucristo, su divino Hijo.

Sí, ven, Señor, rey de justicia y paz. ¡Por María! Amén.
\end{body} 
	
\subsubsection{Homilía (1986): }

Visita Pastoral a la Parroquia Romana de Santa María, Reina de los Mártires, 7 de diciembre de 1986.

\begin{body} 
1. \emph{``¡Regem venturum Dominum come adoremus!''.}

La liturgia del Segundo Domingo de Adviento nos permite mirar al Mesías, cuya venida Israel -\/- el pueblo de Dios de la Antigua Alianza -\/- ha esperado. Veamos primero con la mirada profética de \textbf{Isaías} (en el siglo VIII a. C.): ¿quién será el Mesías esperado? Será un gran Maestro, el que vendrá en el poder del Espíritu Santo, lleno de sus dones: ``El Espíritu del Señor reposará sobre él ... " (\emph{Is} 11, 2).

Al mismo tiempo, debe "brotar de la raíz de Jesé", es decir, del linaje de David, como "un brote de su tronco". El Mesías, por tanto, en la profecía de Isaías, aparece como un hombre, descendiente de David, tan animado por la fuerza del Espíritu de Dios, que su misión nos permite profundizar decisivamente en el misterio de este mismo Espíritu. De hecho, como dije en mi encíclica \emph{Dominum et Vivificantem} (n. 15), este texto bíblico "es importante para toda la pneumatología del Antiguo Testamento, porque casi constituye un puente entre el antiguo concepto bíblico de "espíritu", entendido en primer lugar como "aliento carismático", y el "Espíritu" como persona y como don, don para la persona".

Como se desprende de otros pasajes del profeta, el Mesías, el "consagrado y ungido del Espíritu", estará tan lleno de este Espíritu que él mismo, junto con el Padre, tendrá el poder de "enviar el Espíritu" (cf. \emph{Jn} 15, 26; 16, 7), de "conceder este Espíritu a todo el pueblo" \emph{(Dominum et V}ivificantem, 15).

2. A la luz de las palabras del \textbf{salmista} (alrededor del Siglo VII, Salmo 71), el Mesías, que ha de venir, será rey de paz basado en la justicia, llevando la liberación a los "pobres" y a los que padecen la opresión en múltiples formas. En este punto el Salmo se encuentra con el pasaje de Isaías, incluso si el profeta expresa esta verdad sobre el Mesías de otra manera: mientras que Isaías ve aquí al Mesías como un hombre lleno del Espíritu Santo, un "hombre sabio", fuerte y justo, el salmista subraya el "reinado" del Mesías, acentuando así la justicia y eficacia de su gobierno universal y eterno.

Sea como fuere, vemos en estos pasajes de la Escritura la idea de un futuro Mesías liberando a los pobres y oprimidos. Y este es efectivamente uno de los aspectos esenciales de la misión de Cristo. Como se dice, de hecho, en la reciente "Instrucción sobre la libertad y liberación cristianas" (n. 51): "Por la fuerza de su misterio pascual, Cristo nos ha hecho libres. Con su perfecta obediencia en la cruz y con la gloria de la resurrección, el Cordero de Dios quitó el pecado del mundo y nos abrió el camino de la liberación definitiva".

3. Para \textbf{Juan el Bautista} "en el desierto de Judea", el Mesías, cuya venida es inmediatamente precedida y preparada en Israel por el profeta, es el que "bautizará en Espíritu Santo y fuego" (\emph{Mt} 3, 11). En el corazón de la visión del Bautista está la necesidad de la conversión moral debida a la proximidad del reino celestial: "Todo árbol que no da buen fruto es cortado y echado al fuego" (\emph{Mt} 3, 10). Juan advierte a los hombres de su tiempo y de todos los tiempos que es imposible obtener la salvación sin "dar frutos dignos de conversión" (\emph{Mt} 3, 8). Estimula vigorosamente las conciencias a la renovación de las costumbres recordándoles su responsabilidad ante Dios e inspirándoles un sano temor. Aunque tengamos por padre a Abraham -\/- nos recuerda el profeta -\/- esto no es motivo suficiente para estar seguros: para que la misericordia de Dios tenga su eficacia en nosotros, debemos corresponderle con las obras de arrepentimiento, justicia y caridad. Sólo bajo estas condiciones los hombres pueden recibir verdaderamente el "fuego del Espíritu Santo" contenido en el bautismo cristiano.

4. ¿Quién es el Mesías? Para san Pablo, que se refiere a lo que "estaba escrito antes" (\emph{Rm} 15, 4), el Mesías es Cristo, es Él, Cristo el que "os acogió para gloria de Dios ... El que se hizo siervo de los circuncisos en favor de la veracidad de Dios, para cumplir las promesas hechas a los patriarcas ... " (\emph{Rm} 15, 7-8), mientras que a las naciones paganas, que no conocían las profecías del Antiguo Testamento, se manifestó en la gratuidad e imprevisibilidad de su misericordia.

Cristo se ha mostrado de diferentes maneras a ambos, aunque es el mismo "servidor" de todos. A Israel, que estaba esperando al Mesías, se mostró como la plenitud de esa verdad prefigurada por las profecías: por eso mostró fidelidad a Dios. Y a los pueblos paganos, que no lo esperaban, a esos paganos que -\/- como dice Pablo -\/- "no tienen la ley" (\emph{Rm} 2, 14-15), el Mesías se ha mostrado como el cumplimiento superior e inesperado de esa ley que "está escrita en sus corazones": la ley moral de la conciencia natural.

5. Hoy miramos a Cristo Mesías, cuya venida renueva la Iglesia cada año en la liturgia del Adviento; miremos con los ojos del profeta, el salmista, el Bautista y finalmente el Apóstol de los gentiles, que también {[}aquí{]} en Roma proclamó la buena nueva de Cristo. Miremos juntos con los ojos de los hombres {[}del siglo XX, que se acercan al final del segundo milenio después de Cristo{]}. Miremos con los ojos de la fe la acogida de esta invocación de Adviento: "Preparad el camino del Señor, enderezad sus sendas" (\emph{Mt} 3, 3). Esta invocación nunca pierde su relevancia.

Miremos junto con la comunidad, {[}que constituye esta parroquia romana de Santa Maria Regina dei Martiri. Los mártires: aquellos que en diferentes épocas y lugares dieron un testimonio eminente de la venida de Cristo. Vuestra parroquia lleva su nombre. Y lleva el nombre de la Madre de Cristo como Reina de los mártires.{]}

6. (...) Os exhorto a aceptar vuestro compromiso de que la parroquia sea verdadera y cada vez más la "familia de Dios", en la conciencia y el cuidado de ser "comunión", de ser Iglesia. Para ello es necesario fomentar la corresponsabilidad de todos, para que cada uno se sienta desafiado e involucrado según los dones recibidos de Dios y sus propias posibilidades.

7. "¡Regem venturum Dominum come adoremus!" Quiero que {[}esta celebración nos ayude a{]} preparar la venida del Señor, a fin de que vivamos con mayor plenitud en el encuentro con Aquel que es el Rey de la paz y la justicia y que, ungido con el Espíritu Santo en la plenitud de sus dones, viene a "bautizarnos en Espíritu Santo y fuego": para renovar en cada uno la gracia recibida en ese primer sacramento por el cual nos hacemos cristianos . . Para que -\/- todos y cada uno -\/- salgamos al encuentro de Aquel que "os acogió para gloria de Dios". Y apoyándonos en él, "nos acojamos unos a otros" (\emph{Rom} 15, 7) en el espíritu del mandamiento evangélico del amor.

\emph{¡Que la Madre de Cristo, Reina de los mártires, una vez más os acerque a Cristo en el misterio del Nacimiento del Señor!}
\end{body} 
	
\subsubsection{Homilía (1989): }

Canonización del Hermano Mutien-Marie Wiaux, 10 de diciembre de 1989.

\begin{body} 
1. "¡Convertíos porque el Reino de los Cielos está cerca!" (\emph{Mt} 3, 2).

En este tiempo de Adviento, la Iglesia vuelve a dirigirnos la invitación de Juan Bautista. ¡Se lo dirige a cada uno de nosotros! Juan, el precursor, se retiró en el desierto, con austeridad. No tiene otra tarea que preparar el camino del Señor. Se escucha su voz y llega la multitud. Por invitación de Juan, se bautizan reconociendo sus pecados. Su camino es el de la conversión.

Con Juan, estas multitudes preparan el camino del Señor. De hecho, ¡el Reino de los Cielos ya está cerca! Juan anuncia: "El que viene después de mí es más poderoso que yo" (cf. \emph{Mt} 3, 11). El último y más grande de los profetas anuncia al Mesías, el retoño brotado de las raíces de Jesé, que el \textbf{profeta Isaías} estaba esperando.

2. Cuando hoy escuchamos a los profetas, cuando en este día escuchamos a Juan el Bautista, que nos conduce por el camino del Mesías, no percibimos sólo el eco de una palabra antigua. Es la Iglesia de Cristo la que vuelve a cada generación, a esta generación {[}al final del segundo milenio{]}, para decir que el Reino de los Cielos ya está cerca en el Mesías anunciado por los profetas.

Viene entre nosotros Aquel sobre quien ``reposa el Espíritu del Señor, espíritu de sabiduría e inteligencia, espíritu de consejo y fortaleza, espíritu de conocimiento y temor del Señor ... " (\emph{Is} 11, 2).

Si se nos pide conversión, es para acoger al que viene, para recibir los dones de la justicia y la paz, para convertirnos en constructores de paz, ya que, como dice el \textbf{profeta Isaías}: "Juzgará a los pobres con justicia y tomará decisiones justas para los oprimidos de la tierra" (\emph{Is} 11, 4). Para prepararse para su venida, para convertirse en sus discípulos, es necesario dejarse transformar por su justicia y guiarse por su sabiduría. Entendemos que "mediante el espíritu de sabiduría y entendimiento" (\emph{Is} 11, 2), el Mesías descubre nuestro pecado y nos invita a conformarnos a la ley del amor y la verdad de su Reino.

3. El profeta también dice del Mesías que "la justicia será el ceñidor de su cintura" (\emph{Is} 11, 5). Inaugurará un reino de paz: "El lobo morará junto con el cordero ..., el buey y el cachorro de león pastarán juntos y un niño los guiará. (...) No volverán a obrar inicuamente ni saquearán todo mi santo monte" (\emph{Is} 11, 6. 9).

¿Es esto una ilusión, ya que los conflictos interminables en tantas partes del mundo parecen hacer vana esta profecía? ¡Sería así si esta palabra no viniera de Dios, si no se dirigiera a las conciencias, si no fuera ella misma fuente de justicia, justicia según Dios, justicia de Dios! El mensaje de Isaías, el ardiente llamamiento del precursor y luego la misma venida de Jesús dan frutos de paz en el corazón y en las acciones de quienes se dejan convertir.

\textbf{Juan el Bautista} proclama la urgencia de la conversión: "El hacha ya está puesta a la raíz de los árboles" (\emph{Mt} 3, 10). Escuchemos con atención esta advertencia: toda persona está llamada a producir buenos frutos de justicia y paz; cada momento, cada acción tiene su importancia según el Reino que está por venir y cuyo espíritu no debe traicionar. Todo hombre, por modesto que sea, es insustituible en la familia humana; el gesto más humilde inspirado por el amor da gloria a Dios

{[}...{]}

6. (...) Las palabras de \textbf{San Pablo} que la Iglesia nos hace escuchar en este segundo domingo de Adviento nos recuerdan que "todo lo que se escribió en el pasado, se escribió para enseñanza nuestra, a fin de que a través de nuestra paciencia y del consuelo que dan las Escrituras mantengamos la esperanza" (\emph{Rom} 15, 4). {[}Que este tiempo nos ayude a mostrar{]} perseverancia y valentía y a apoyarnos en el poder de la Palabra de Dios para enfrentar los desafíos de cada una de nuestras vidas, los desafíos de cada una de nuestras familias y comunidades.

San Pablo añadía: "Que el Dios de la paciencia y del consuelo os conceda tener entre vosotros los mismos sentimientos, según Cristo Jesús; de este modo, unánimes, a una voz, glorificaréis al Dios y Padre de nuestro Señor Jesucristo" (\emph{Rm} 15, 5). (...) plenamente imbuidos del espíritu de Cristo, recibiremos mejor al Salvador que viene entre nosotros si, unidos, recibimos la Palabra y actuamos en comunión en el mismo Espíritu y compartimos los mismos dones. "Por eso, acogeos mutuamente, como Cristo os acogió para gloria de Dios" (\emph{Rm} 15, 7-9). Vivir el Adviento es preparar los días en los que "florecerá la justicia", los días en los que conoceremos "la abundancia de la paz" (cf. \emph{Sal} 72, 7), los días en los que "todas las razas de la tierra serán bendecidas" (\emph{Sal} 72, 17).

7. Hermanos y hermanas, {[}que este tiempo nos ayude a{]} fortalecer el valor de la esperanza en nosotros. Que se nos conceda allanar el camino del Señor y el camino del hombre, que es el camino de la Iglesia. Que se nos dé la capacidad de producir frutos que expresen una conversión real, de anunciar sin cesar el Reino de los cielos ahora cercano en Aquel que bautiza en el Espíritu Santo (cf. \emph{Mt} 3, 2-11).

Y así podremos "con una sola alma y una sola voz dar gloria a Dios, Padre de nuestro Señor Jesucristo" (cf. \emph{Rm} 15, 6).
\end{body} 
	
\subsubsection{Homilía (1992): }

Visita Pastoral a la Parroquia Romana de Santa María del Buen Consejo, 6 de diciembre de 1992.

\begin{body} 
"Preparad el camino del Señor, enderezad sus sendas" (\emph{Mt} 3, 3).

Queridos hermanos y hermanas (...) El tiempo litúrgico de Adviento llega hoy a su segunda semana, guiándonos hacia el encuentro con el recién nacido de Belén, Jesús Salvador. Acabamos de escuchar un pasaje del \textbf{Evangelio de Mateo}, que nos habla de la predicación de Juan el Bautista, enviado para preparar los caminos del Señor en el corazón de los hombres de su tiempo. La liturgia nos lo presenta en este camino espiritual de espera y oración que es el Adviento, para que también nosotros escuchemos sus llamadas y hagamos nuestra su invitación urgente a la conversión. El austero precursor de Cristo repite hoy en nuestra asamblea litúrgica estas mismas palabras: "¡Arrepentíos, porque el reino de los cielos está cerca!" (\emph{Mt} 3, 2). Aquí, en esta parroquia, en todas las demás parroquias de Roma, de Europa, resuena la misma palabra divina. Convertirnos, abrir el corazón a la fuerza renovadora del Evangelio: este debe ser el programa diario de todo creyente, un programa particularmente exigente y elocuente en el período litúrgico que vivimos. De hecho, este es el significado del tiempo de Adviento: recordar que Jesús vino entre nosotros y por nosotros en la humildad del pesebre. Por tanto, a cada uno de nosotros le corresponde estar dispuesto a acogerle, con espíritu de penitencia, en una vida renovada por una fe auténtica y activa.

2. La Iglesia nos indica hoy como modelo y guía de tal camino espiritual a Juan Bautista (...) Vivía en el desierto en gran penitencia cuando el Espíritu de Dios lo envió a predicar la inminente venida del Mesías al mundo. Luego fue al Jordán, antes de que Jesús bajara allí, y comenzó a predicar la conversión de los corazones y a administrar el bautismo de penitencia. El suyo era un lenguaje franco y directo, casi grosero, que sin embargo la multitud escuchaba con interés. Proclamó que "el hacha ya está puesta a la raíz de los árboles: todo árbol que no da buen fruto es cortado y echado al fuego" (\emph{Mt} 3, 10). Y para ello exhortaba insistentemente a los que acudían a él: "Dad frutos dignos de conversión". Convertirse significa "cambiar de vida", adaptando la propia existencia al don gratuito de la salvación divina. Qué gran lección también para nuestra comunidad eclesial, que hoy se reúne con alegría alrededor del altar del Señor. La figura del Precursor es motivo de una revisión concreta de nuestra existencia cristiana y un estímulo para una adhesión más profunda al Evangelio. Su predicación fue confirmada por el testimonio del martirio. Juan el Bautista fue asesinado por el rey Herodes Antipas por su fidelidad a la palabra de Dios y, después de tantos siglos, la humanidad sigue honrando su fe y su fuerza moral, sellada por el sacrificio de la vida.

3. "¡Convertíos porque el reino de los cielos está cerca!". Esta exhortación de Juan el Bautista resuena una vez más en nuestro espíritu. Nos impulsa a tomar conciencia de la urgencia de adherirse firme y totalmente a Cristo, preparando sus caminos, enderezando sus sendas. La misión de la Iglesia que vive en nuestro tiempo de profundo y rápido cambio social es muy similar a la misión del Precursor. Todos estamos llamados a participar activamente en ella: es la nueva evangelización. "Evangelizar significa actuar como testigos y el imperativo de la evangelización es, por tanto, siempre actual". Cuando hablamos de "nueva evangelización" lo hacemos porque es siempre y en todas partes "nueva". "Jesucristo es el mismo ayer, hoy y por los siglos" (\emph{Hb} 13, 8). Esta "novedad" pertenece a la identidad del Evangelio. Orienta y anima el compromiso evangelizador, que constituye un imperativo continuo y permanente para los testigos de Cristo. Vivimos una época atravesada por fuertes corrientes de "contraevangelización" que, aunque en su expresión más radical se han debilitado considerablemente, no dejan de influir negativamente en el ámbito de los principios y en el de las opciones de vida. La palabra de Dios parece tan abrumada por las llamadas del utilitarismo y el consumismo, del secularismo y del materialismo práctico. Por eso es indispensable que por parte de todos los creyentes en Cristo se renueve y fortalezca la voluntad de ofrecer un testimonio renovado y coherente a favor de Cristo.

4. ¡Queridos hermanos y hermanas! Esta invitación espiritual a trabajar activamente por el reino de los cielos está dirigida especialmente a vosotros (...)

{[}...{]}

6. El Redentor que esperamos en este tiempo de Adviento vendrá trayendo consigo los dones del Espíritu Santo y "hará oír su poderosa voz para el gozo de nuestro corazón" (Antífona de entrada).

Como nos recordó la \textbf{primera lectura} del Libro del \textbf{profeta Isaías}, su venida será una fuente de justicia y paz.

"En aquel día se levantará la raíz de Jesé como estandarte para los pueblos", "será gloriosa su morada" (\emph{Is} 11, 10).

Queridos hermanos y hermanas, que este anuncio de alegría y victoria definitiva sobre el mal y el pecado sostenga vuestro camino espiritual. La alegría del Señor, proclamada en nuestra atenta asamblea litúrgica, se nos ofrece como un don para que, en virtud de la perseverancia y el consuelo que nos llegan de las Escrituras, "mantengamos viva nuestra esperanza" (\emph{Rm} 15, 4).

Miremos y sigamos el ejemplo de María, {[}cuya solemnidad de la Inmaculada Concepción nos estamos preparando para celebrar{]}. ¡Sigamos despertando nuestra confianza a través de la oración incesante y la escucha de Dios! Por tanto, nuestra esperanza nunca fallará.

Caminemos en una vida nueva, teniendo "entre vosotros los mismos sentimientos, según Cristo Jesús" (\emph{Rom} 15, 5). Como san Pablo nos amonesta hoy en la Carta a los Romanos (...). Con esta breve palabra, con esta breve invocación del Adviento quiero terminar.

"Preparemos el camino del Señor". ¡Amén!
\end{body} 
	
\chapter{IV-Adviento}

\subsubsection{Homilía (1983): }

\subsubsection{Homilía (1986): }

Concelebración Eucarística en el Colegio Belga de Roma, 21 de diciembre de 1986.

\begin{body} 
1. "\emph{El hijo que en ella se engendra proviene del Espíritu Santo}"\emph{.}

En este cuarto domingo de Adviento, que nos prepara para la Navidad, contemplamos a María que lleva en ella al Salvador del mundo. La miramos con la mirada silenciosa y admirativa de José.

Desde la Anunciación, la Virgen María guardó dentro de sí un secreto ... que la llenó al mismo tiempo de alegría y temor de Dios, de gratitud y responsabilidad hacia el Hijo del Altísimo que había concebido, de disponibilidad hacia el Espíritu Santo que la había tomado bajo su sombra en una alianza íntima.

Ella no pudo tomar la iniciativa de compartir este secreto divino. Es el ángel del Señor quien se lo revela a José, el justo prometido en matrimonio a la Virgen de Nazaret.

Con José, veneramos a María, la bendita entre todas las mujeres. Adoramos el fruto bendito de su vientre, que ella se prepara para traer al mundo y darlo para la salvación de los hombres, cooperando en la obra de Dios que entregó a su Hijo único al mundo. Releamos a la luz de la fe, como María, como los evangelistas, como la primera comunidad cristiana, la profecía de Isaías: "He aquí que la Virgen concebirá... un hijo al que se le dará el nombre de Emmanuel, Dios con nosotros".

Este tiempo feliz que precede a la natividad fue para María y José el de espera y esperanza, el de maduración y florecimiento de la vida del Niño Jesús, el de la meditación silenciosa de su Madre ante el designio misterioso de Dios, ante la maravilla de su amor.

Queridos hermanos y hermanas, también nosotros damos la bienvenida al Salvador que viene, saltamos con alegría y confianza, ¡presentámosle todas las intenciones de este mundo con el que somos solidarios y que tanto necesita la luz y la salvación de Dios!

{[}...{]}

Que la Santísima Virgen María nos proteja, en la alegría de recibir incesantemente la gracia del Redentor y sobre todo el don de Dios mismo, Emmanuel, que permanece en nosotros por su Palabra, su Espíritu y su Eucaristía.
\end{body} 
	
\subsubsection{Homilía (1989): }

Visita Pastoral a la Parroquia Romana de Santa Ana, 24 de diciembre de 1989.

\begin{body} 
1. "Lo llamarás Jesús, porque él salvará a su pueblo de sus pecados" (\emph{Mt} 1, 21).

En este último domingo de Adviento y Nochebuena, nos reunimos para la Santa Misa, queridos hermanos y hermanas de la parroquia de Santa Ana en el Vaticano. Estamos invitados a reflexionar sobre el misterio del Emmanuel-\/-Dios con nosotros, mientras ya nos preparamos para recibir su venida en la ya inminente liturgia de la noche santa.

Escuchamos con esta expectativa el testimonio de dos personas particularmente cercanas al acontecimiento del nacimiento de Jesús: María y José. El \textbf{evangelio} de hoy nos habla de ellos.

2. La visión de José, narrada por Mateo, debe acercarse a la página lucana del anuncio que el ángel le dio a María. Ambas visiones se despliegan según un modo de proceder, que en la tradición bíblica pretende hacer comprender y aceptar la misión a la que Dios llama: "Darás a luz un hijo... el Espíritu Santo vendrá sobre ti... He aquí la esclava del Señor" (cf. \emph{Lc} 1, 35-38); "No tengas miedo de tomar contigo a María, tu esposa... ella dará a luz un hijo y lo llamarás Jesús... José hizo lo que le dijo el ángel" (cf. \emph{Mt} 1, 20-24).

La visión y el anuncio trazan también al mismo tiempo la línea de fe necesaria para acoger adecuadamente la revelación del misterio de Dios y reconocer a Jesús de Nazaret en su verdad. Por eso el \textbf{pasaje evangélico} que hemos leído, si bien insistimos en la actitud interior de humilde disponibilidad hacia la iniciativa de Dios, vislumbra también con fuerza y ​​claridad las cualidades del Mesías: será el Hijo de David, el anunciado por los profetas, el salvador de su pueblo. Con esta clara perspectiva podremos escuchar mañana el anuncio del Evangelio de Juan, que nos hablará del Verbo hecho carne, "lleno de gracia y de verdad" (\emph{Jn} 1, 14).

{[}...{]}

5. "Va a entrar el Señor; Él es el Rey de la gloria". Así cantamos en el salmo responsorial.

El Señor viene, lo esperamos con gloria y esperanza viva. El mensaje de la liturgia nos ha dicho quién es.

Viene el Rey de la gloria, el que también prometió volver a nosotros al final de los tiempos, "juzgará entre las naciones... será árbitro de pueblos numerosos" (\emph{Is} 2, 4).

Mañana, y ya esta noche, lo contemplaremos manso y humilde, un niño como cualquier otro "nacido de mujer"; seremos invitados a amarlo y reconocerlo, como un día los pastores de Belén. Dará "gracia y paz" a todos los que son "amados por Dios y santos por vocación" (\emph{Rom} 1, 7).

Volveremos a escuchar a los ángeles anunciar "una gran alegría que será para todo el pueblo: hoy nos ha nacido... un Salvador, que es Cristo el Señor" (\emph{Lc} 2, 10-11).

Ésta es nuestra fe; nuestra verdadera alegría se basa en estas palabras y en este anuncio. Esperamos a Dios con nosotros para siempre, y hacia él caminamos en la fe para encontrarlo con el alma pura de los "hombres que él ama" (\emph{Lc} 2, 14).
\end{body} 

\chapter{Navidad: Medianoche}

\subsubsection{Homilía (1983): }

Basílica de San Pedro, Santa Navidad, 24 de diciembre de 1983.

\begin{body} 
1. \emph{Custos, quid de nocte?} -\/- Centinela, ¿qué hay de la noche? (cf. \emph{Is} 21, 11).

¡He aquí que os anuncio de la medianoche! Esta medianoche se mueve de este a oeste. Sigue todos los meridianos. En Oriente ya nos ha precedido, en Occidente está a punto de llegar... He aquí que os anuncio la medianoche; en cada lugar y en cada momento en que viaja el globo terrestre, anuncio la Medianoche!

Yo, guardián del Gran Misterio. Yo, obispo de Roma: anuncio la medianoche de Navidad en todas partes. "Cantad al Señor un cántico nuevo, cantad al Señor toda la tierra" (\emph{Sal} 96, 1).

2. ¡Canta, tierra!

Canta porque fuiste elegida, elegida entre todo el universo. Y todo el universo fue elegido junto a ti.

¡Canta, oh tierra!

"Alégrese el cielo, goce la tierra, retumbe el mar y cuanto lo llena; vitoreen los campos y cuanto hay en ellos, aclamen los árboles del bosque" (\emph{Sal} 96, 11-12).

Canta, tierra, porque has sido elegida para ser el lugar del nacimiento de Dios en un cuerpo humano. ¡Que se reúna toda la tierra alrededor de esta Medianoche! ¡Que hable la potencia de toda la creación! ¡Que hable por medio de la existencia de todos los mundos creados! ¡Que hable por medio de la lengua de hombre!

3. Y he aquí que un hombre está hablando. Su nombre es Lucas, evangelista. Él dice: "... los días del parto se cumplieron para ella (María). Y dio a luz a su hijo primogénito, lo envolvió en pañales y lo acostó en un pesebre, porque no había lugar para ellos en la posada (\emph{Lc} 2, 6-7).

De esta manera vino al mundo el hijo de Dios: María era la esposa de José, de la familia de David; de José, que era carpintero en Nazaret. El Niño vino al mundo en Belén porque tanto María como José habían ido allí debido al censo que había ordenado César Augusto.

4. Esto dijo el hombre. Al mismo tiempo, el Ángel del Señor le habla al hombre. Habla a los pastores cuando, en medio de la noche profunda de Belén, "la gloria del Señor los envolvió en luz". Y los pastores "se aterrorizaron" (\emph{Lc} 2, 9). Les dice: "¡No tengáis miedo! He aquí, os anuncio un gran gozo, que será de todo el pueblo: hoy os ha nacido en la ciudad de David un Salvador, que es Cristo el Señor. Esta será la señal para vosotros: encontraréis a un niño envuelto en pañales, acostado en un pesebre" (\emph{Lc} 2, 10-12).

El hombre y el ángel hablan del mismo hecho y señalan el mismo lugar. El Ángel habla de lo que el hombre no se atreve a decir: el Mesías, que es el Ungido, vino al mundo en Belén, el que viene a visitar a la humanidad en el poder del Espíritu Santo. El Salvador del mundo nació en la tierra en Belén. Él... juzgará a la tierra. Él... juzgará al mundo con justicia.

Sí, se entregará "a sí mismo por nosotros, para redimirnos de toda iniquidad y purificar para sí un pueblo de su propiedad... " (\emph{Tit} 2, 14). Él se dará a sí mismo por nosotros: ¡aquí está su juicio!

5. \emph{Custos, quid de nocte?} -\/- Centinela, ¿qué hay de la noche? (cf. \emph{Is} 21, 11).

He aquí que os anuncio la medianoche... Desde lo más profundo de la noche de Belén, que es la noche de toda la humanidad que vive en la tierra... "Ha aparecido la gracia de Dios, portadora de salvación para todos los hombres" (\emph{Tit} 2, 11).

¿Qué es la gracia? Gracia y complacencia divina se concentran completamente en este Niño acostado en la cuna. Este Niño es el Hijo Eterno, el Hijo en quien Dios se complace, el Hijo del Amor eterno. Este Niño es el Hijo de María. Es Hijo de hombre y verdadero hombre.

La eterna satisfacción del Padre se concentra en el hombre: ¡aquí está la Gracia! "Paz en la tierra a los hombres que ama el Señor" (\emph{Lc} 2, 14). Esta complacencia divina con el hombre fue traída a la tierra por el Hijo de María en la noche de Belén. "Ha aparecido la gracia de Dios" (\emph{Tit} 2, 11). Desde Belén comienza su irradiación sobre el hombre de todos los tiempos.

¿Qué es la Gracia? Es el comienzo de la gloria, de esa gloria que Dios tiene en lo más alto del cielo. Y para esta gloria el hombre fue llamado en Jesucristo. Y esto sucedió en la noche de Belén.

6. Por tanto: ¡regocíjate tierra! ¡Tierra, que eres la morada del hombre! ¡Acoge en ti una vez más el esplendor de la noche del nacimiento divino! ¡Reúnete en torno a este esplendor! ¡Proclama el gozo de la redención a toda la creación! Anuncia al mundo entero la esperanza de su Redención.

"Vitoreen los campos y cuanto hay en ellos, aclamen los árboles del bosque ante el Señor" (\emph{Sal} 96, 12-13). He aquí que viene. He aquí que ya está entre nosotros: ¡Emmanuel! Todo el poder de la Redención del mundo está en él. ¡Aleluya!
\end{body} 
	
\subsubsection{Homilía (1986): }

Basílica Vaticana, Miércoles 24 de diciembre de 1986.

\begin{body} 
1. "\emph{¡No tengáis miedo! Os anuncio una gran alegría... hoy os ha nacido en la ciudad de David un salvador, que es Cristo el Señor}" (\emph{Lc} 2, 10-11).

Estamos aquí reunidos la noche de la vigilia de Navidad para volver a escuchar estas palabras, después de siglos y siglos. Los pastores de los campos de Belén las oyeron por primera vez. Y esta es la razón por la que la asamblea litúrgica de Nochebuena lleva el nombre de "Misa de los Pastores" en algunos países.

2. {[}Estamos reunidos en la Basílica de San Pedro. No solo las personas aquí presentes participan en la liturgia, sino también muchos de nuestros hermanos y hermanas a quienes este rito solemne se transmite a través de las ondas de la radio y la televisión.{]}

El acontecimiento de la noche de Belén nos une a todos. En momentos sucesivos, marcados por el tiempo que transcurre en la tierra, se desarrolla en todos los lugares de nuestro planeta.

Las estaciones del año y las condiciones climáticas de esta noche santa también son diferentes en las diversas regiones: ocurre tanto en el calor tropical, en el duro invierno nórdico y en las tormentas de nieve. Incluso en condiciones tan diferentes, lo que ocurre en esta hora es siempre el mismo acontecimiento. Y los que anuncian la noche de Belén proclaman la misma "gran alegría", aunque sus palabras se escuchen en muchos idiomas diferentes en todo el mundo.

3. Los pastores en los campos de Belén - los primeros testigos del evento - eran hijos de Israel, cuya historia estaba relacionada con la promesa del Mesías. De modo que las palabras que escucharon podrían, y también deberían, despertar su asombro. Pero, al mismo tiempo, no eran palabras incomprensibles para ellos.

Los pastores sabían lo que significaba la palabra "Mesías". Durante generaciones, Israel había vivido a la espera del Mesías, del Ungido del Señor. Si el "Mesías" viene al mundo en la "ciudad de David" es porque esta circunstancia pertenece a los pronósticos proféticos. La ciudad de David es precisamente Belén.

Además, el Mesías debe haber venido del "linaje de David". De la casa y la familia de David eran también José y María, la Madre del Recién Nacido. Y por tanto, debido al censo ordenado por los romanos, tuvieron que ir directamente a Belén, partiendo de Nazaret, donde vivían.

4. Entonces, las palabras escuchadas por los pastores les resultaron comprensibles. En ellos se cumplió la promesa hecha a Israel. Al mismo tiempo, estas palabras deben haberles sorprendido. El ángel dijo: "Encontraréis a un niño envuelto en pañales, acostado en un pesebre...: Esta será para vosotros la señal".

Los pastores no dudaron de que las palabras que escucharon provenían de Dios, se rindieron ante un "gran gozo", y al mismo tiempo demostraron tranquilidad y mesura. Caminaron en la dirección indicada y encontraron todo exactamente como se les dijo.

Se convirtieron en testigos presenciales del acontecimiento, cuya dimensión adecuada sólo es accesible a los "ojos luminosos" de la fe.

5. Todos nosotros, reunidos religiosamente en tantos lugares de la tierra para renovar y hacer presente, con la liturgia eucarística, el acontecimiento salvífico, que tuvo entre los primeros participantes a los pastores de Belén en esa noche santa; todos nosotros, en breve, nos pondremos de rodillas, cuando resuenan las conocidas palabras del Credo Niceno-Constantinopolitano: Dios de Dios, Luz de luz, de la misma sustancia que el Padre... "Incarnatus est de Spiritu Sancto ex Maria Virgine, et homo factus est".

6. El misterio de la Encarnación. El misterio de Dios "hecho hombre" en el Hijo eterno.

Nos arrodillaremos y permaneceremos postrados para manifestar esta realidad inefable. Permaneceremos de rodillas en nombre de todos los hombres, en lugar de toda la creación.

El acontecimiento de la noche de Belén revela ante los ojos de nuestra fe la plenitud definitiva del sentido de la creación, del mundo, del hombre.

7. Y luego un sacerdote o un diácono, ministros de la Eucaristía, se presentará ante cada uno de vosotros y dirá: "Corpus Christi... -\/- El cuerpo de Cristo". Y cada uno de vosotros responderá: "Amén"; la palabra de fe que reconoce, adora, agradece. La palabra que une a los pastores ante el acontecimiento de la noche de Belén: el Verbo se hizo carne, la carne y la sangre de la nueva y eterna alianza de Dios con el hombre.

El acontecimiento de la noche de Belén se convirtió en el comienzo de la nueva comunión, que penetra en el corazón y la historia del hombre en la tierra. "Gloria a Dios en las alturas y paz en la tierra a los hombres que él ama".

8. "Alégrese el cielo, goce la tierra, retumbe el mar y cuanto lo llena; vitoreen los campos y cuanto hay en ellos, aclamen los árboles del bosque. Delante del Señor, que ya llega" (\emph{Sal} 96 (95), 11. 13).

A toda la creación, a todos los que viven esta noche sagrada en Belén: a los hermanos y hermanas esparcidos por el globo terrestre: alegría, paz y bendición. Amén.
\end{body} 
	
\subsubsection{Homilía (1989): }

Basílica de San Pedro, Domingo 24 de diciembre de 1989.

\begin{body} 
1. "Os anuncio una gran alegría" (\emph{Lc} 2, 10).

Fue precisamente a una hora de la noche como esta cuando los pastores de Belén escucharon el anuncio de gran alegría.

A la misma hora nos encontramos todos reunidos aquí {[}en la Basílica de San Pedro{]} para escuchar el anuncio de la misma alegría.

Y así, mientras lo hacemos, la gente se reúne, nuestros hermanos y hermanas se reúnen en muchos lugares del mundo.

{[}A todos, dondequiera que estén reunidos, el Obispo de Roma saluda con las mismas palabras: "Os anuncio una gran alegría".

Este saludo mío va para todos los hombres, en todos los continentes.

Va, con especial cariño y con un recuerdo siempre vivo, a las naciones que visité este año, a las multitudes que conocí en esos países: en el Lejano Oriente, en África, en los países nórdicos. A los queridos jóvenes que, en Santiago de Compostela, celebraron conmigo la Jornada Mundial de la Juventud, representando también a todos sus compañeros, de todo el mundo.

Dirijo también este saludo, y de manera especial, a los hombres y mujeres de todas las naciones que, conectados por radio y televisión, escuchan esta Santa Misa de Medianoche y participan, unidos espiritualmente con nosotros y con todos los creyentes del mundo, para el misterio de la natividad del Hijo de Dios en la tierra.{]}

2. Este anuncio va dirigido a los hombres, pero no solo a ellos. La liturgia navideña, a medianoche, también llama a la alegría a todas las criaturas.

"Alégrese el cielo, goce la tierra, retumbe el mar y cuanto lo llena; vitoreen los campos y cuanto hay en ellos, aclamen los árboles del bosque\ldots{} Cantad al Señor un cántico nuevo, cantad al Señor desde toda la tierra" (\emph{Sal} 97, 11-12. 1).

Por tanto, desde este anuncio de Belén todas las criaturas están llamadas a la alegría. De hecho, quien nace de la Virgen María es "engendrado antes que todas las criaturas" (cf. \emph{Col} 1, 15). En él y para él todo fue creado. Todo el bien que se encuentra en las criaturas tiene en él su origen y su primer modelo.

Por medio de él, el Padre miró una vez a toda la creación y "vio que todo era bueno... muy bueno" (cf. \emph{Gn} 1, 10. 31).

En esta noche en Belén, todos estamos llamados -\/-llamados una vez más-\/- a regocijarnos en la obra de la creación.

3. "Os anuncio una gran alegría".

En el momento en que el Hijo, Verbo eterno, primogénito de toda criatura, se presenta en medio de sus criaturas, se reafirma este júbilo por la obra de la creación. Y, al mismo tiempo, se eleva.

La criatura alcanza tal exaltación, que va más allá de su horizonte. Más allá del horizonte de la existencia y el conocimiento.

"Acreciste la alegría, aumentaste el gozo" (\emph{Is} 9, 2).

Pero esta exaltación lo alcanzan las criaturas en el hombre. Del hombre se dijo al principio que había sido creado a imagen y semejanza de Dios. En la noche de Belén se reconfirma totalmente esta verdad sobre el hombre con mayor fuerza.

"Porque nos ha nacido un niño, un hijo nos ha sido dado" (\emph{Is} 9, 5).

En la noche de Belén nace el Niño, el niño humano: para María "se cumplieron los días del parto. Dio a luz a su hijo primogénito, lo envolvió en pañales y lo acostó en un pesebre" (\emph{Lc} 2, 6-7).

El mensajero celestial le dice lo mismo a los pastores; "Encontraréis a un niño envuelto en pañales, acostado en un pesebre" (\emph{Lc} 2, 12).

4. Aquí está el Niño, el niño humano, el hijo del hombre, como todos los demás nacidos de mujer.

Este niño es el Hijo: "Se nos ha dado un hijo". Nos lo dio el Padre. Fue dado a los hombres y al mundo: "porque tanto amó Dios al mundo que dio a su Hijo unigénito" (\emph{Jn 3, 16} ).

"Un Hijo se nos ha dado".

En este Hijo eterno, que es de la misma sustancia que el Padre, Dios mismo entra en la historia del hombre y del mundo.

En este Hijo "apareció... la gracia de Dios, portadora de salvación para todos los hombres" (\emph{Tit} 2, 11).

Dios, que creó al hombre a su imagen y semejanza, sabe quién es el hombre. Sabe lo que es el corazón humano, sabe que su corazón está inquieto hasta que descansa en él (cf. San Augustín, \emph{Confesiones}, I, 1: \emph{CSEL} 33, 1).

Y por eso, precisamente por eso, "se nos ha dado un hijo". El corazón humano, al llegar al pesebre de Belén, encuentra allí esa paz que sólo se encuentra en Dios. Esta paz está íntimamente ligada a la gloria de Dios, como proclama el mensaje de la noche de Belén.

5. "Os anuncio una gran alegría... hoy nos ha nacido... un salvador" (\emph{Lc} 2, 10-11).

Pero, ¿es esta alegría tan pura, tan plena como nos gustaría que fuera?

Sí y no. De hecho, sobre ella se proyecta una sombra de tristeza. El Niño, el Hijo de Dios, nace en un establo, porque no había lugar para él en la posada (cf. \emph{Lc} 2, 7).

El momento de su llegada es al mismo tiempo el momento de la no\emph{-}acogida, del rechazo: "No había sitio". Esta sombra de tristeza se alargará. Se espesará hasta el punto del rechazo, a través de la Cruz, en el Gólgota. De esta manera rechazará el hombre al Hijo que nos fue dado por el Padre como signo de su amor.

Jesucristo, "el cual se entregó por nosotros para rescatarnos de toda iniquidad" (\emph{Tit} 2, 14).

6. Nosotros, reunidos aquí, saludamos, junto con nuestros hermanos y hermanas que están en comunión con nosotros, el nacimiento de Dios con la liturgia del sacrificio eucarístico. Es el sacrificio de nuestra redención. Este sacrificio hace presente la Cruz y la Resurrección: el misterio pascual de Cristo Este misterio tiene su comienzo en la noche de Belén, cuando nos nació un Salvador. ¡El redentor del hombre, el redentor del mundo!

La Iglesia, que esta noche anuncia "una gran alegría", sabe que esta alegría proviene totalmente de Dios, es el don de su amor.

También sabe que sólo esta alegría expande el corazón humano a las dimensiones supra-temporales, que Dios mismo ha preparado para el hombre.

Él lo sabe, y por eso repite, incluso en esta noche, frente al mundo: "Os anuncio una gran alegría. ¡Hoy nació el Salvador!".
\end{body} 
	
\subsubsection{Homilía (1992): }

Misa en la Noche Santa. Basílica Vaticana, Jueves 24 de diciembre de 1992.

\begin{body} 
1. "\emph{Gloria a Dios en las alturas y paz en la tierra a los que Dios ama}" (\emph{Lc} 2, 14).

Esta es la noche que hemos estado esperando todo el año. En esta noche se cumplen las palabras del \textbf{profeta Isaías} sobre las tinieblas y la luz: "\emph{Sobre los que habitaban en tierra y en sombras de muerte una luz les brilló}" (\emph{Is} 9, 1).

Esa luz atravesó la noche que había caído sobre Belén de Judea. Gracias a la luz de esa noche, los hombres se vieron inmersos en una luz extraordinaria: eran sobre todo hombres sencillos, \emph{los pastores que custodiaban} su rebaño. La luz brilló en sus almas. \emph{No solo había luz a su alrededor, sino también dentro de ellos.} La luz anunciada por Isaías había entrado en sus corazones. En esa luz, Dios mismo estaba presente. Fue una luz de teofanía.

Como antes Abraham, Moisés y los profetas, ahora también ellos estaban dentro del rayo de la luz de Dios, que los había despertado en la noche y los había impulsado a partir hacia Belén: "Hoy \emph{nació allí,} en la ciudad de David\emph{,} el \emph{salvador, que es Cristo el Señor}" (\emph{Lc} 2, 11).

2. No dentro de la ciudad, sino fuera de ella. El lugar de nacimiento del Salvador estaba envuelto en la oscuridad de esa noche. Los pastores habían sido advertidos: "\emph{Encontraréis un niño envuelto en pañales, acostado en un pesebre}" (\emph{Lc} 2, 12). ¿Es posible? ¿Por qué el Salvador del mundo viene a los suyos de esta manera? ¿Por qué, aunque había dejado claro desde el principio que venía, los suyos no lo aceptaban? De hecho, este ya era el caso en Belén.

Los pastores estaban envueltos en una luz de arriba. Cuando se encontraron frente al recién nacido, se \emph{dieron cuenta de que habían llegado al centro de una Teofanía}. La misma certeza será demostrada más tarde también por los Magos que vinieron de Oriente, cuando se encuentren en el umbral de la cabaña. También ellos, como los pastores, entran en el \emph{rayo de la luz divina} que ha venido al mundo. Sobre esa luz no prevalecieron las tinieblas (cf. \emph{Jn} 1, 5). Y no prevalecerán. Como en la noche de Belén, ni la oscuridad de la indigencia, ni la miseria del abandono y la humillación \emph{pudieron sofocar la Luz del Misterio Divino}. He aquí que el Verbo se hizo carne.

3. Como más tarde los Magos de Oriente, esa noche los pastores de Belén llevaron a cabo en sí mismos las palabras del Profeta sobre el pueblo, sobre el pueblo de la Antigua Alianza, de la cual el Mesías, el Salvador del mundo iba a nacer:

He aquí que "\emph{El pueblo que caminaba en tinieblas vio una gran luz}" (\emph{Is} 9, 1).

La salvación del mundo tiene su origen en Dios mismo, y su comienzo temporal aquí mismo, en medio de este pueblo elegido. Desde aquí debe extenderse por toda la tierra. He aquí, "el pueblo que andaba en tinieblas" verá una gran luz. \emph{Entre tantas naciones y pueblos de todo el mundo, un solo pueblo de Dios ...} El espacio del Nacimiento de Dios, que en un principio cubrió de luz los campos de Belén, se encuentra hoy en innumerables lugares de la tierra.

Dondequiera que celebremos, a medianoche, esta liturgia llena de alegría, de cuyo Misterio, esa noche, los pastores participaron en Belén, ciudad de David, se renueva y se hace presente:

"Acreciste la alegría, aumentaste el gozo" (\emph{Is} 9, 2).

4. \emph{Esta alegría es más fuerte que la pobreza y la miseria.} Esta alegría la conocen los "pobres de espíritu". Como entonces los pastores de Belén, así, a través de los siglos y generaciones, tantos y tantos hombres de "buena voluntad". ¿De dónde viene esta alegría? ¿No deriva del hecho de que el nacimiento "de una mujer" (\emph{Gal} 4, 4) del Hijo consustancial al Padre da a todos \emph{la certeza del amor de Dios}? ¿Puede haber una demostración más convincente de que Dios ama al hombre, que ha encontrado su complacencia en los hombres? ¿Puede haber una verificación aún más evidente? Helo aquí, Aquel que es.

Helo aquí Aquel que es, -\/-no ya en la zarza ardiente, ni en truenos y relámpagos, como en el monte Sinaí-\/-. He aquí Aquel \emph{que es como uno de nosotros: como hombre...} como un Niño recién nacido de la Virgen Madre. Encargado al cuidado de María y José.

He aquí, El que es.

5. "\emph{Natus est nobis...} ".

El espacio de la Teofanía de Belén se cumple hasta los confines de la creación. De hecho, va más allá de ellos. Abraza la tierra y, al mismo tiempo, se eleva a aquellas alturas que están llenas de la gloria de Dios.

"\emph{Gloria a Dios en lo alto de los cielos}" (\emph{Lc} 2, 14).

Ese Dios que amó al mundo, que lo amó hasta dar a su propio Hijo para la salvación del hombre, revela la paz a los hombres: "\emph{La paz os dejo, mi paz os doy. Os la doy no como os la da el mundo}" (\emph{Jn} 14, 27).

¡Qué difícil es para el mundo asegurar la paz para el hombre, para los hombres, para las naciones, para las épocas históricas!

"Yo os la doy... ": ¡Paz en la tierra a los hombres de buena voluntad!

Pero, ¿puede realmente prevalecer la paz en la tierra cuando falta buena voluntad, cuando a los hombres no les importa si Dios los ama?

Esta noche, \emph{la Iglesia te} mira a ti, Jesucristo, que eres el Dios Fuerte y el príncipe de la paz, y \emph{te pide} paz para toda la humanidad redimida. Esta paz es tu Nombre.

\emph{Erit Iste Pax!}
\end{body} 
	
\subsubsection{Homilía (1995): } Misa en la Noche Santa. Basílica Vaticana, Lunes 25 de diciembre de 1995.

\begin{body} 
1. "Hoy nos ha nacido el Salvador" (Salmo responsorial).

Al "hoy" del gran misterio de la Encarnación corresponde de manera particular esta hora, en la que celebramos la santa misa llamada de "medianoche". Según la tradición, el Hijo de Dios vino al mundo en Belén en medio de la noche.

Leemos en el texto del \textbf{profeta Isaías}: "El pueblo que caminaba en tinieblas vio una gran luz" (\emph{Is} 9, 1). A este pueblo pertenecían \emph{los pastores de Belén}, que guardaban su rebaño de noche y a quienes, en primer lugar, llegaba la noticia: "Hoy ha nacido en la ciudad de David un salvador, que es Cristo el Señor" (\emph{Lc} 2, 11). Y fueron los primeros en ir, siguiendo la llamada del ángel, al establo donde nació Jesús:

"¡Hoy ha nacido Cristo el Señor, el Salvador"! \emph{Esta feliz noticia} invita a toda la creación a \emph{cantar al Señor} "\emph{un cántico nuevo}": "Alégrese el cielo, goce la tierra, retumbe el mar y cuanto lo llena; vitoreen los campos y cuanto hay en ellos, aclamen los árboles del bosque" (\emph{Sal} 95, 11-12).

Por eso, en Nochebuena el mundo entero resuena \emph{con cantos de alegría}, en todos los idiomas del mundo. Son cantos que poseen un encanto singular y contribuyen a crear el ambiente inconfundible de este período del año litúrgico. En verdad, como dice el profeta Isaías, "acreciste la alegría, aumentaste el gozo" (\emph{Is} 9, 2).

2. "Hoy ha nacido" (cf. \emph{Lc} 2, 11).

Junto al término "nació", \emph{natus est}, encontramos otra expresión en los textos litúrgicos: \emph{apparuit}, "apareció", "se manifestó". Cuando nace un niño, aparece una nueva persona en el mundo. En referencia al nacimiento en Belén del Hijo de María, \emph{la liturgia habla de} "\emph{manifestación}" como se subraya especialmente en la \textbf{Carta del Apóstol San Pablo a Tito}: "Se ha manifestado la gracia de Dios, que trae la salvación para todos los hombres" \emph{(Tit} 2, 11).

"Un niño nos ha nacido, un hijo se nos ha dado", está escrito en el texto de \textbf{Isaías} (\emph{Is} 9, 5). \emph{La gracia de Dios apareció en este Niño}, trayendo salvación a todos los hombres. Esta gracia es ante todo Él mismo, el Hijo unigénito del Padre eterno, que en esta hora se hace hombre al nacer de una mujer. Su nacimiento en Belén constituye \emph{el primer momento de la gran revelación de Dios en Cristo.}

Los pastores llegan al establo y encuentran allí "al Salvador del mundo, que es Cristo Señor" (cf. \emph{Lc} 2, 11). E incluso si sus ojos ven a un recién nacido envuelto en pañales y colocado en un pesebre, en ese "signo", gracias a la luz interior de la fe, reconocen al Mesías anunciado por los Profetas. En él se manifiesta el amor de Dios por el hombre, por toda la humanidad. El que nació en la noche de Belén \emph{viene al mundo para entregarse} "\emph{a sí mismo por nosotros}, para rescatarnos de toda iniquidad y purificar para sí un pueblo de su propiedad, dedicado enteramente a las buenas obras" (\emph{Tit} 2, 14).

3. "Gloria a Dios en el cielo, y en la tierra paz a los hombres de buena voluntad" (\emph{Lc} 2, 14).

Este himno, que ha entrado firmemente en la tradición litúrgica de la Iglesia, resuena por primera vez en la noche de Belén y habla de un acercamiento singular y extraordinario entre Dios y el hombre. En realidad, \emph{Dios nunca se acercó tanto al hombre como aquella noche en que el unigénito Hijo del Padre se hizo hombre}. Y aunque su nacimiento tuvo lugar en condiciones modestas y pobres -\/-Jesús nació en la pobreza de un establo, como un vagabundo-\/-, sin embargo estuvo lleno de gloria divina. La gloria, de hecho, no significa sólo esplendor externo; significa ante todo santidad.

La hora del nacimiento del Hijo de Dios en el establo de Belén \emph{es la hora en que la santidad de Dios irrumpe en la historia del mundo.} "Noche santa", como anuncia un conocido villancico. Noche que es, al mismo tiempo, el comienzo de la santificación del hombre por obra de Aquel, que es el único "Santo de Dios". El himno angelical que acompaña a la Natividad del Señor anuncia precisamente esto.

Al mismo tiempo, proclama la \emph{paz en la tierra}. Pensemos en primer lugar en la paz en el sentido histórico. Así, en la noche del nacimiento del Señor, se renueva en nosotros la esperanza de paz para todos los hombres y para todos los pueblos afectados por la guerra: {[}en los Balcanes, en África{]} y en todos los lugares donde falta la paz.

Pero en la liturgia navideña la palabra "paz" también tiene otro significado más profundo. Se refiere a la \emph{nueva Alianza de Dios con los hombres}, a su renovación y cumplimiento definitivo. Si la Alianza de Dios con los hombres es una realidad que envuelve toda la historia de la salvación, no podría haber encontrado una expresión más completa que ésta: Dios acogió en sí a la humanidad, asumiéndola en la única Persona del Hijo. Así unió en sí lo divino y lo humano, como fundamento perenne y estable de la paz y de la alianza eterna. Por eso toda la Iglesia entona esta noche un cántico nuevo: "\emph{¡Gloria a ti, Dios hecho hombre, y paz a los hombres salvados por tu amor!}".
\end{body} 
	
\chapter{Navidad: Día}

\subsubsection{Urbi et Orbi (1986): }

MENSAJE URBI ET ORBI

DE JUAN PABLO II

NAVIDAD 1986

Desde Asís

\begin{body} 
1. "\emph{Cuán hermosos son sobre los montes los pies del mensajero de la buena nueva que anuncia la paz ... que dice a Sión: Tu Dios reina}" (Is 52, 7).

Sí. Sión, tu Dios reina. Tu Dios admirable aquí está acostado en el pesebre de los animales. ¡Y así comienza a reinar tu Dios, oh Sion!

Tu Dios incomprensible: "Sus pensamientos no son nuestros pensamientos, y nuestros caminos no son sus caminos" (\emph{Is} 55, 8).

Entonces comenzó a reinar bajo el signo de la extrema pobreza: "Se hizo pobre por nosotros, para que nosotros nos hiciéramos ricos con su pobreza" (cf.\emph{2 Co} 8, 9).

2. ¡Oh, qué hermosos son los pies de ese mensajero de la buena nueva que se llama Francisco il Poverello de Asís, de Greccio y de La Verna, Francisco, amante de todas las criaturas; Francisco conquistado por el amor del divino Niño, nacido en la noche de Belén; Francisco en cuyo corazón Cristo comenzó a reinar, para que incluso a través de la pobreza del discípulo comprendamos mejor la pobreza del Maestro y seamos llevados a pensamientos de amor y paz.

Gloria a Dios en las alturas y paz en la tierra a los hombres de buena voluntad; paz a los hombres que Él ama (cf. \emph{Lc} 2, 14). Gloria a Dios ...

3. {[}Escuchemos{]} el mensaje de Francisco, amante del Creador y de toda criatura; de Francisco, heraldo de la Gloria de ese Dios, que "en las alturas de los cielos" es el Amor; de Francisco promotor de la paz en la tierra. {[}Reflexionemos{]} ante el "último e inefable misterio que envuelve nuestra existencia, del que tomamos nuestro origen y hacia el que nos dirigimos" ({\emph{Nostra Aetate}}, 1 ), {[}oremos{]} por la paz en la tierra.

4. {[}...{]} Hemos decidido ser pobres -\/- frente a todos los poderes de esta tierra que devoran incalculables riquezas en armamento, disipan recursos preciosos en bienes superfluos -\/- hemos decidido ser pobres como Cristo, Hijo de Dios y Salvador del mundo; pobres como Francisco, imagen elocuente de Cristo; pobres como tantas grandes almas que han iluminado el camino de la humanidad.

Lo hemos decidido teniendo a nuestra disposición sólo este medio, el medio de la pobreza, y sólo este poder, el poder de la debilidad: sólo la oración y sólo el ayuno.

5. ¿No es necesario que el mundo escuche esta voz? ¿No es necesario que escuche el testimonio de la noche de Belén? ¿Que escuche a Dios nacido en la pobreza? ¿Que escuche a Francisco, heraldo de las ocho bienaventuranzas? ¿No es necesario que el estruendo del odio y el estruendo de las detonaciones mortíferas en muchos lugares de la tierra se callen? ¿No es necesario que Dios finalmente pueda escuchar la voz de nuestro silencio? ¿A través del silencio le llegará la oración y el grito de todos los hombres de bien? ¿El grito de tantos corazones atormentados, la voz de tantos millones de hombres que no tienen voz?

6. Escuchad y comprended todos: Dios que todo lo abraza, Dios en quien "vivimos", nos movemos y existimos (\emph{Hch} 17, 28), ha elegido la tierra como su morada; nació en Belén; ¡ha establecido su Reino en los corazones humanos!

¿Podemos ignorar o distorsionar todo esto? ¿Es lícito destruir la morada de Dios entre los hombres? ¿No es necesario, en cambio, cambiar de raíz los planes del dominio humano sobre la tierra?

7. ¡Hermanos y hermanas de toda la tierra! Si Dios nos amó tanto que se hizo hombre con nosotros, ¿cómo no amarnos unos a otros, hasta el punto de compartir con los demás lo que cada uno posea, para el gozo de todos? Solo el amor que se hace don puede transformar la faz de nuestro planeta, haciendo que las mentes y los corazones se vuelvan a pensamientos de hermandad y de paz.

Hombres y mujeres del mundo, Cristo nos pide que nos amemos unos a otros. Este es el mensaje de Navidad, este es el deseo que dirijo a todos desde el fondo de mi corazón.
\end{body} 
	
\subsubsection{Urbi et Orbi (1989): }

\begin{body} 
1. "... les dio poder de hacerse hijos de Dios" (\emph{Jn} 1, 12)

Es la solemnidad de la Navidad. Los ojos de nuestra alma ven al Niño, colocado en el pesebre. La mirada de nuestra fe se detiene en las palabras del prólogo de Juan. "A los que le recibieron les dio poder de hacerse hijos de Dios".

2. Te bendecimos, Hijo del hombre, que eres el Verbo eterno.

Gloria al Padre que nos dio a ti, el Unigénito.

Gloria al Espíritu, que procede del Padre y de ti, Hijo de Dios.

Gloria al misterio eterno, que todo lo abarca.

En esta noche se acercó al hombre, entró en su vida y en su historia.

Ha cruzado el umbral de nuestra existencia humana.

3. El Niño envuelto en pañales y colocado en un pesebre. El Niño humano indefenso - y al mismo tiempo el Poder, que sobrepasa todo lo que el hombre es, todo lo que puede.

Porque el hombre no puede llegar a ser como Dios con sus propias fuerzas -\/-como ha sido confirmado por la historia, desde el principio-\/-.

Y, al mismo tiempo, el hombre puede llegar a ser como Dios por el poder de Dios.

Este poder está en el Hijo, Verbo eterno, que "se hizo carne y habitó entre nosotros" (\emph{Jn} 1, 14).

Este es el primer día de su morada entre nosotros.

"Él estaba en el mundo, y el mundo fue hecho por él, pero el mundo no lo reconoció.

Vino entre su propia gente, pero los suyos no lo recibieron.

Pero a los que lo recibieron, les dio poder para convertirse en hijos de Dios: quienes ... fueron engendrados por Dios" (\emph{Jn} 1, 10-12).

4. La historia sigue su camino... muchos, incontables hombres, naciones, pueblos, lenguas, razas, culturas ... millones y miles de millones ... y él es el único: ahora colocado como Niño en el pesebre ("no había sitio para él ... en la posada"), y luego en la Cruz.

Él, el único. Y luego, resucitó, él, el único.

¿Cuántos no lo han aceptado? ¿Cuántos no le han acogido?

¿Cuántos saben de él? ¿Cuántos no lo saben?

Quisiéramos calcular con estadísticas humanas hasta dónde llega este poder que está en él: nacido - crucificado - resucitado.

Humanamente, nos gustaría saber cuántos se han convertido, en él y para él, en hijos de Dios, hijos en el Hijo.

Pero los medidores humanos no pueden medir el misterio de Dios.

No pueden medir el don del nacimiento de Dios, que está presente en la historia del hombre y en la historia del mundo, que obra en las almas humanas por la fuerza del Espíritu que da la vida.

5. "Todos los confines de la tierra vieron la salvación de nuestro Dios" (\emph{Sal} 98, 3).

Sí. Vinieron los pastores de Belén y vieron.

Sí. También vinieron los sabios de Oriente, y vieron.

Y vieron el viejo Simeón y la profetisa Ana en el templo de Jerusalén.

¿Con qué mirada te ven, Verbo Encarnado, todos los confines de la tierra?

De hecho, tú has venido para todos. Tú eres la salvación de nuestro Dios, que es para todos y viene a través de ti.

Dios "quiere que todos los hombres se salven y lleguen al conocimiento de la verdad" (\emph{1 Tim} 2, 4).

La verdad y la gracia han venido a través de ti.

Tú eres la verdad.

Tú eres el camino y la vida (cf. \emph{Jn} 14, 6).

Y aunque los tuyos no te han acogido \ldots{} aunque no había lugar para ti en la posada ...

en ti Dios nos ha acogido ... nos ha acogido a todos.

6. En ti, Dios también nos ha acogido a nosotros, {[}hombres y mujeres del segundo milenio, que está por terminar{]}.

No miró nuestras contradicciones, nuestras infidelidades, nuestros desequilibrios.

De hecho, te envió a ti, su Palabra, para sanarnos.

Para decirnos que, por este camino, corremos hacia la autodestrucción.

El mundo aspira a la paz: sin embargo, cada día nuestros hermanos y hermanas mueren en los conflictos en curso, {[}en el Líbano, en Tierra Santa, en América Central;{]} mueren en luchas fratricidas por las supremacías racistas, ideológicas, económicas; mueren de absurda imprudencia.

El mundo aspira a la reconciliación: sin embargo, cada día miles de refugiados son abandonados y rechazados; las minorías étnicas y religiosas son ignoradas en sus necesidades fundamentales; sectores enteros de la población se mantienen al margen de la sociedad en un aislamiento cada vez mayor.

El mundo aspira al equilibrio, interno y externo: sin embargo, el medio ambiente se degrada cada día más por motivos de interés o inconsciencia.

7. El anuncio de la verdad y de la gracia que nos llega en Navidad a través de ti debe tocarnos a todos. Ese anuncio es para nosotros, porque viniste por nosotros, te convertiste en uno de nosotros.

¡Haz que te acojamos, Palabra eterna del Padre!

Que el mundo te acoja.

Suscita en los corazones el rechazo de toda barrera de raza, de ideología, de intolerancia.

Favorece el progreso de las negociaciones en curso sobre control y reducción de armas.

Sostén a quienes se comprometen a superar los conflictos que se han prolongado durante demasiado tiempo {[}en África y Asia{]}, para que los pueblos involucrados en ellos recuperen su libertad y sus derechos, a través de un diálogo leal y confiado.

8. ¡Que te acoja, Verbo Encarnado, también nuestra vieja Europa!

Ella lleva profundamente grabado el estigma de tu Evangelio, del que nació su civilización, su arte, su concepción de la inviolable dignidad del hombre.

Que esta Europa abra sus puertas y su corazón para comprender y acoger las ansiedades, preocupaciones, problemas de las naciones que piden su ayuda.

Que sepa responder con el vigor y la generosidad de sus raíces cristianas a este momento histórico tan particular, verdadero Kairòs providencial, que el mundo vive ahora como liberado de una pesadilla y abierto a una mejor esperanza.

{[}En particular, bendice en esta hora, oh Señor, a la noble tierra de Rumanía, que celebra esta Navidad con temor, en el dolor de tantas vidas humanas trágicamente perdidas y en la alegría de haber retomado el camino de la libertad.{]}

9. Hermanos y hermanas, aquí presentes.

{[}Hermanos y hermanas, que me escuchan en la radio y la televisión, en todos los Continentes,{]} venid a la cuna del Niño indefenso, que es el Poder de Dios, Él nació por nosotros.

Venid \ldots{} y veréis \ldots{} y seréis acogidos, porque hoy se ha manifestado la bondad de Dios y su amor por los hombres.
\end{body} 
	
\subsubsection{Urbi et Orbi (1992): }

\begin{body} 
1. "Gloria a Dios en las alturas y paz en la tierra a los hombres que Él ama" (\emph{Lc} 2, 14).

Este es el mensaje que escuchamos nuevamente a la medianoche, cuando los pastores llegaron a la gruta de Belén.

Y ahora, habiendo llegado al corazón de este día bendito, la Iglesia nos anuncia el Misterio: "El Verbo se hizo carne y habitó entre nosotros" (\emph{Jn} 1, 14).

El Hijo eterno del Padre está en el mundo; el Verbo, por quien todo fue hecho.

Él estaba al principio con Dios, era Dios (cf. \emph{Jn} 1, 1-2). El Padre le ha dicho desde el principio de los siglos: Tú eres mi Hijo, yo te he engendrado en el ``hoy'' eterno y divino (cf. \emph{Hb} 1, 5).

El Verbo -\/- el Hijo: Dios de Dios, Luz de Luz. El Verbo se hizo carne y habitó entre nosotros.

La noche de Belén es el comienzo de su morada entre los hombres. A plena luz del día, la Iglesia proclama el Misterio del Verbo hecho carne.

2. Cur Deus homo? ¿Por qué Dios se hizo hombre?

El hombre pregunta: ¿Por qué? Muéstrame el camino a las profundidades de tu Misterio. El hombre le ha estado haciendo a Dios esta pregunta durante {[}dos mil años{]}.

Pero muchas veces se responde a sí mismo, sin esperar la respuesta de Dios.

Tú, oh Dios, estás por encima de todas las cosas, dice. Tú sólo puedes estar por encima del mundo: Uno y solo en tu infinita Majestad. ¡Dios, quédate solo! ¡No te rebajes a la criatura, no te rebajes al hombre! Así responde el hombre. Y a veces incluso llega a decir: ¡Oh Dios, mantente fuera del mundo! ¡Deja el mundo en manos del hombre solo! Aquí limitas al hombre; aquí no podemos vivir juntos. Y cree que tal respuesta es un signo de progreso y autonomía para la humanidad.

Cur Deus homo? ¿Por qué Dios se hizo hombre? El hombre le hace la pregunta a Dios, pero luego se la responde a sí mismo. Sin embargo, sólo Dios puede señalar el camino a las profundidades de su Misterio.

3. La respuesta de Dios se llama Evangelio.

La respuesta de Dios tiene su inicio en la noche de Belén, para luego convertirse en testimonio de Aquel que nació esa misma noche.

De hecho, tanto amó Dios al mundo que dio a su Hijo para que el hombre no muera, sino que tenga vida eterna en él (cf. \emph{Jn} 3, 16).

4. Hermanos y hermanas, no nos cerremos ante Dios, no impidamos que viva en nosotros Aquel que nació hoy, consustancial con el Padre, el Primogénito de toda criatura.

Él viene a su propiedad, no se lo impidamos.

No pensamos que Dios deba permanecer solo, revestido de inefable Majestad, pero solo, por encima del mundo y fuera de él.

El mundo le pertenece; y, en el mundo, el hombre es el ser más suyo, creado a su imagen y semejanza, imagen de lo Invisible en el mundo visible.

Amor es el nombre que mejor se adapta a la divina Majestad. Y el amor sólo es tal cuando se entrega, cuando se hace don para los demás.

¿Puede el hombre realizarse plenamente a sí mismo sin amor?

¿Qué más puede salvarlo aparte del Amor todopoderoso, revelado en ese Niño indefenso?

¿Quién más puede revelar plenamente al hombre a sí mismo, sino Él?

Su nombre es Jesús, que significa "Dios salva".

5. Queridos hermanos y hermanas, hombres y mujeres de toda la humanidad, Cristo -\/-Dios que salva-\/-, desea encontrarnos.

Él está entre nosotros: acojámosle, abramos a Él nuestro corazón.

Escuchen su voz, ustedes, líderes de las naciones, llamados a gestionar el destino de los pueblos: la solidaridad -\/- ha proclamado Él en silencio en la noche de la esperanza -\/- es el camino de la justicia y la paz.

Tú que sufres en los caminos de la existencia, oprimido por la injusticia y el mal, decepcionado e insatisfecho con todo bienestar transitorio: la vida -\/- anuncia el Verbo hecho carne -\/- se ha manifestado hoy en todo su esplendor.

Es un canto de alegría que acalla el grito amenazador de muerte.

Escucha la voz del amor, dulce y poderosa al mismo tiempo, especialmente tú, que empuñas armas violentas y asesinas.

6. Ante el pesebre, donde el Hombre-Dios llora bajo la mirada ansiosa de María y José, nuestro pensamiento se dirige espontáneamente a nuestros muchos hermanos para quienes la Navidad también está marcada este año por el miedo, la tristeza y el dolor.

{[}Pienso en los niños de Sarajevo, de Banja Luka, de las poblaciones de Bosnia y Herzegovina, rehenes de una violencia planificada e inhumana; en Liberia, devastada y destrozada por luchas locas y fratricidas durante más de tres años; en Somalia, donde afortunadamente, gracias a la ayuda, se enciende la confianza de un futuro mejor.

¿Cómo olvidar en este momento la expectativa de una paz segura y duradera en Angola, en Mozambique?

¿Cómo no preocuparnos por el clima de odio y lucha que en Tierra Santa, suelo santificado por el nacimiento del divino Hacedor de la paz, persiste y aleja aún más las esperanzas suscitadas por el proceso de pacificación iniciado en Madrid?

7. Cur Deus homo?

Aunque oscurecido por las brumas y tormentas de la historia, el camino de la humanidad está iluminado por la respuesta de Dios, que aumenta nuestra esperanza.

Tu amor, oh Verbo encarnado, es más fuerte que el odio, es más fuerte que la muerte misma (cf. \emph{Ct} 8, 6).

¡Sí! Nada puede evitar que vengas a nosotros, incluso en los lugares maltratados del mundo donde la gente todavía mata y el mal parece reinar sin oposición.

Filius datus est nobis!

Tú vienes, oh Señor, a curar las heridas abiertas en el costado de la humanidad.

Ven allí donde el rugido de las armas te impida escuchar siquiera el llanto desconsolado de mujeres y niños, los lamentos de los heridos, las débiles invocaciones de los moribundos.

A veces la tierra parece sorda e impenetrable al Misterio de tu presencia.

Ven, te lo rogamos, para que triunfe tu Amor, don de la paz.

{[}...{]}

8. En el esplendor de este día santo resuena el cántico de alegría celestial: "Gloria a Dios en lo alto del cielo y paz en la tierra a los hombres que Él ama".

Resplandece la victoria del Amor todopoderoso, que llena plenamente todas nuestras expectativas humanas.

Cur Deus homo?

Puer natus est nobis! Filius datus est nobis!

Es la respuesta de Dios, así responde el Verbo Encarnado.

Y su voz llega al hombre cuando él, ante el divino Nacimiento de Belén, deja que Dios hable.

Muéstrame, Señor, el camino a las profundidades de tu Misterio.

¡Muéstrame el camino! Amén.
\end{body} 
	
\subsubsection{Urbi et Orbi (1995): }

\begin{body} 
1. "Tú eres mi hijo; yo te he engendrado hoy" (\emph{Hb} 1, 5).

Las palabras de la liturgia de hoy nos introducen en el misterio del nacimiento eterno, más allá del tiempo, del Hijo de Dios, Hijo consustancial al Padre.

El Evangelio de Juan dice: "En el principio era el Verbo, y el Verbo era con Dios y el Verbo era Dios. Él era en el principio con Dios" (\emph{Jn} 1, 1-2).

Profesamos la misma verdad en el Credo: "Dios de Dios, Luz de Luz, Dios verdadero de Dios verdadero, engendrado, no creado, de la misma sustancia que el Padre; por él fueron creadas todas las cosas. Por nosotros los hombres y por nuestra salvación descendió del cielo, y por obra del Espíritu Santo se encarnó en el seno de la Virgen María y se hizo hombre".

He aquí la alegre noticia del nacimiento del Señor, tal como la transmitieron los evangelistas y la tradición apostólica de la Iglesia.

Hoy queremos anunciarlo "a la Ciudad y al Mundo", Urbi et Orbi.

2. "En el mundo estaba y el mundo por él fue hecho" (\emph{Jn} 1, 10).

Viene a los suyos El que sale a la luz en la noche de Navidad.

¿Por qué viene? Viene a comunicar una "nueva fuerza", un "poder" diferente al del mundo.

Viene pobre en un establo de Belén, con el mayor don: da a los hombres la filiación divina.

A todos los que le acogen les da la "potestad de llegar a ser hijos de Dios" (\emph{Jn} 1, 12), para que en él, Hijo eterno del Padre eterno, "sean engendrados de Dios" (cf. \emph{Jn} 1 , 13).

De hecho, en él, en el Niño de la Noche Santa, habita la vida (cf. \emph{Jn} 1, 4): vida que no conoce la muerte; vida de Dios mismo; vida que, como dice San Juan, es la luz de los hombres.

La luz brilla en las tinieblas, y las tinieblas no la vencieron (cf. \emph{Jn} 1, 4-5).

En la noche de Navidad surge la luz que es Cristo. Brilla y penetra en los corazones de los hombres, injertando en ellos una nueva vida. Enciende en ellos la luz eterna, que siempre ilumina al ser humano incluso cuando la oscuridad de la muerte envuelve su cuerpo.

Por eso "el Verbo se hizo carne y habitó entre nosotros" (\emph{Jn} 1, 14).

3. "Vino a los suyos, pero los suyos no la recibieron" (\emph{Jn} 1,11), recuerda el Prólogo del Evangelio de Juan.

Lucas el evangelista confirma esta verdad y recuerda que "no había lugar para ellos en la posada" (\emph{Lc} 2, 7).

"Para ellos", es decir, para María y José y para el Niño que estaba por nacer.

He aquí un hecho que a menudo se menciona en los villancicos: "Su pueblo no lo aceptó \ldots{} ".

En la gran posada de la comunidad humana, como la pequeña posada de nuestro corazón, ¡cuántos pobres aún hoy, {[}en el umbral del año 2000{]}, llaman a la puerta!

4. Es Navidad: ¡una fiesta de acogida y amor!

{[}¿Encontrarán un lugar en este día las familias desplazadas de Bosnia y Herzegovina, que todavía esperan ansiosamente los frutos de la paz, de esa paz recientemente proclamada? ¿Podrán los refugiados de Ruanda regresar a un país verdaderamente reconciliado? ¿Podrá el pueblo de Burundi redescubrir el camino de la paz fraterna? ¿Tendrá el pueblo de Sri Lanka la oportunidad de mirar juntos, de la mano, hacia un futuro de fraternidad y solidaridad? Por último, ¿se le dará al pueblo iraquí la alegría de recuperar una existencia normal, después de los largos años de embargo?{]}

{[}¿Serán acogidas las poblaciones de Kurdistán, entre las que muchas personas se ven obligadas a afrontar el invierno, una vez más, en la más dura precariedad? ¿Y cómo no pensar en los hermanos y hermanas del sur de Sudán, que todavía sufren violencia armada, alimentada sin descanso?{]}

{[}Por último, no podemos olvidar al pueblo de Argelia, que sigue sufriendo, víctima de atroces juicios.{]}

¡Es en este mundo herido donde irrumpe el Niño Jesús, amoroso y frágil!

Viene a liberar al hombre atrapado en el odio y esclavo de particularismos y divisiones.

Viene a abrir nuevos horizontes.

El Hijo de Dios hace nacer la esperanza de que, a pesar de tantas dificultades graves, la paz finalmente emerja en el horizonte.

También hay signos prometedores de esto en países con problemas como Irlanda del Norte y Oriente Medio.

Es nuestro deseo que los hombres abran su corazón a la Palabra de Dios encarnada en la pobreza de Belén.

5. Este es el Misterio que celebramos hoy: Dios "nos ha hablado por el Hijo" (\emph{Hb} 1, 2).

Muchas veces y de diversas maneras Dios había hablado a través de los Profetas, pero cuando "llegó la plenitud de los tiempos" (\emph{Gál} 4, 4), habló por medio del Hijo.

El Hijo es el reflejo de la gloria del Padre; la irradiación de su sustancia, que todo lo sostiene con el poder de su palabra. Esto es lo que dice el autor de la carta a los Hebreos del Hijo recién nacido de María (cf. \emph{Hb} 1,3). Si por medio de él Dios Padre creó el cosmos, también es el Primogénito y Heredero de toda la creación (cf. \emph{Hb} 1,1-2).

Este pobre Niño, para quien "no había lugar en la posada", a pesar de las apariencias, es el único Heredero de toda la creación.

Vino a compartir su herencia con nosotros, para que nosotros, habiendo llegado a ser hijos por la adopción divina, participemos de la herencia que ha traído consigo al mundo.

Palabra eterna, hoy contemplamos tu gloria, "gloria como del unigénito del Padre, lleno de gracia y de verdad" (\emph{Jn} 1, 14).

Que las buenas nuevas de tu nacimiento, antiguas y siempre nuevas, lleguen a los pueblos y naciones de todos los continentes sobre las olas del éter y traigan la paz al mundo.
\end{body} 


\chapter{S. Familia}

\subsubsection{Homilía (1986): } 30 de noviembre de 1986. Celebración en el Hipódromo ``Belmont'', Perth, Australia.

Esta homilía fue pronunciada el Domigo I de Adviento, pero la celebración estaba dedicada a la Familia, por lo que las palabras del Papa son perfectamente aplicables a la celebración de este día.

\begin{body} 
\emph{``Es hora de despertarnos del sueño, porque nuestra salvación está más cerca ahora ..."} (\emph{Rom} 13, 11).

\emph{Amados hermanos y hermanas en Cristo.}

1. Con estas solemnes palabras, la liturgia de este primer domingo de Adviento conduce a toda la Iglesia a un tiempo de espera y preparación. Es el momento en el que toda comunidad cristiana revive la expectativa que los profetas despertaron en el pueblo de Israel, mientras esperaban ansiosos el cumplimiento de la promesa: "La Virgen concebirá y dará a luz un hijo, al que llamará Emmanuel" (\emph{Is} 7, 14), que significa "Dios con nosotros" (cf. \emph{Mt} 1, 23). Es el tiempo de preparación para la venida del niño, el "Príncipe de la Paz": el infante de Belén, que es al mismo tiempo el Hijo de Dios, y la segunda Persona de la Santísima Trinidad.

(...) La Navidad es un día especial para las familias (...) en muchas otras partes del mundo. La familia en el proyecto de Dios para la humanidad y para la Iglesia (...). El Hijo de Dios, al hacerse hombre, inicia esa familia especial que la Iglesia venera como la Sagrada Familia de Nazaret: Jesús, María y José.

{[}...{]}

3. "La familia es la Iglesia doméstica". El significado de esta idea cristiana tradicional es que la casa es la Iglesia en miniatura. La Iglesia es el sacramento del amor de Dios, es una comunión de fe y de vida. Ella es madre y maestra. Está al servicio de toda la familia humana para caminar hacia su destino final. Al mismo tiempo, la familia es una comunidad de vida y amor. Educa y orienta a sus miembros hacia la plena madurez humana y está al servicio del bien de todos en el camino de la vida. La familia es la "primera y vital célula de la sociedad" ({\emph{Apostolicam Actuositatem}}). El futuro del mundo y de la Iglesia pasa, pues, por la familia.

Por tanto, no es de extrañar que la Iglesia en los últimos tiempos haya prestado mucho cuidado y atención a los problemas que afectan a la vida familiar y al matrimonio. Tampoco es de extrañar que los gobiernos y las organizaciones públicas estén constantemente involucrados en problemas que afectan directa o indirectamente el bienestar institucional del matrimonio y la familia. Y todos pudieron ver que las relaciones saludables en el matrimonio y en la familia son de gran importancia para el crecimiento y el bienestar de la persona humana.

4. Las transformaciones económicas, sociales y culturales que están teniendo lugar en el mundo tienen un gran efecto en la forma en que las personas ven el matrimonio y la familia. Como resultado, muchos esposos no están seguros del significado de su relación y esto les hace sentirse incómodos y angustiados. Por otro lado, muchos otros matrimonios son más fuertes porque, habiendo superado las tensiones del mundo moderno, experimentan mucho más plenamente ese amor especial y la responsabilidad del matrimonio que les hace ver a los hijos como un don especial de Dios para ellos y para la sociedad. Dependiendo de cómo vaya la familia, también lo hará la nación y todo el mundo en el que vivimos.

En cuanto a la familia, la sociedad necesita urgentemente "que todos recuperen la conciencia de la primacía de los valores morales, que son los valores de la persona humana como tal", y también de la "re-comprensión del sentido último de la vida y sus valores fundamentales" (\emph{Familiaris Consortio}, 8). {[}Es necesario{]} saber cómo salvaguardar la familia y la estabilidad del amor conyugal si queremos tener paz y justicia en la tierra.

5. La Iglesia (...) tiene una tarea específica: explicar y promover el plan de Dios para el matrimonio y la familia y ayudar a los esposos y familias a vivir de acuerdo con este plan. La Iglesia se dirige a todas las familias: en primer lugar a aquellas familias cristianas que se esfuerzan por ser cada vez más fieles al designio de Dios, busca fortalecerlas y acompañarlas en su desarrollo. Pero también se dirige, con la compasión del corazón de Jesús, a aquellas familias que se encuentran en dificultades o en situaciones irregulares.

La Iglesia no puede decir que lo malo es bueno, ni puede decir que lo que no es válido es válido. No puede dejar de proclamar la enseñanza de Cristo, incluso cuando esta enseñanza es difícil de aceptar. También sabe que fue enviada a curar, reconciliar, llamar a la conversión, encontrar lo perdido (cf. \emph{Lc} 15, 6). Por tanto, es con inmenso amor y paciencia que la Iglesia busca ayudar a quienes experimentan dificultades para responder a las exigencias del amor conyugal cristiano y de la vida familiar.

La caridad de Cristo solo se puede realizar en la verdad: en la verdad sobre la vida, el amor y la responsabilidad. La Iglesia debe anunciar a Cristo: camino, verdad, vida; y al hacerlo, debe enseñar los valores y principios que corresponden a la llamada del hombre, a la "novedad" de vida en Cristo. La Iglesia a veces es incomprendida y considerada carente de compasión porque apoya el plan creador de Dios para el matrimonio y la familia: su plan para el amor humano y la transmisión de la vida. La Iglesia es siempre la verdadera y fiel amiga de la persona humana en el peregrinaje de la vida. Sabe que la defensa de la ley moral contribuye al establecimiento de una verdadera civilización humana, y desafía constantemente a las personas a no abandonar su responsabilidad personal ante los imperativos éticos y morales (cf. \href{http://www.vatican.va/content/paul-vi/it/encyclicals/documents/hf_p-vi_enc_25071968_humanae-vitae.html}{\emph{\emph{Humanae Vitae}}}, 18).

6. "Venid, subamos al monte del Señor... para que nos muestre sus caminos y recorramos sus sendas" (\emph{Is} 2, 3). Con esta invitación, el profeta Isaías nos dice cómo debemos responder a Dios y cómo esta respuesta también se puede aplicar al plan de Dios para el matrimonio y la familia. A los esposos se les ofrece la gracia y la fuerza del sacramento del matrimonio para que puedan caminar por los caminos del Señor y seguir sus sendas, observando el plan que Cristo confirmó y estableció para la familia. Este plan da testimonio de lo que era en el "principio" (cf. \emph{Mt} 19, 8), lo que Dios quiso desde el principio para el bienestar y la felicidad de la familia. En el plan de Dios, el matrimonio requiere: amor fiel y duradero entre marido y mujer; una comunión indisoluble que "tiene sus raíces en la complementariedad natural que existe entre el hombre y la mujer, y se nutre de la voluntad personal de los esposos de compartir todo el proyecto de vida, lo que tienen y lo que son" (\emph{Familiaris Consortio}, 19); una comunidad de personas en la que el amor entre marido y mujer es plenamente humano, exclusivo y abierto a una nueva vida (cf. \emph{Familiaris consortio}, 29).

El amor conyugal se fortalece con el sacramento del matrimonio para que sea una imagen cada vez más real y eficaz de la unidad que existe entre Cristo y la Iglesia (cf. \emph{Ef} 5, 32).

7. Vosotros sabéis cuánto valor cristiano necesitáis para vivir los mandamientos de Dios en vuestra vida y en vuestra familia. Se trata de la valentía de estar dispuestos cada día a construir el amor, ese amor del que dice san Pablo: "La caridad es paciente, la caridad es bondadosa; la caridad no es envidiosa, no se jacta, no se hincha, no falta el respeto, no busca su interés, no se enoja, no toma en cuenta el mal recibido, no disfruta de la injusticia, pero se complace con la verdad. Todo lo cree ... todo lo soporta. La caridad no se acabará nunca" (\emph{1 Cor} 13, 4-8).

{[}Entonces, ¿puede el Papa venir a Australia y no pedirle a los esposos y familias australianas que reflexionen en sus corazones si están viviendo bien su amor cristiano? ¿Cuán seriamente están comprometidos con la defensa de los valores familiares? ¿Qué tan adecuadas son las políticas para la defensa de estos valores y, por tanto, para la promoción del bien común de toda la nación?{]}

En un mundo cada vez más sensible a los derechos de las mujeres, ¿qué se puede decir sobre los derechos de las mujeres que quieren ser, o necesitan ser, esposas y madres a tiempo completo? ¿Deberían estar agobiadas por un sistema fiscal que las discrimine de aquellas que optan por no quedarse en casa para tener sus propios ingresos? Sin violar la libertad de cualquiera que busque satisfacción en el empleo y las actividades fuera del hogar, ¿no debería apreciarse y apoyarse adecuadamente el trabajo del ama de casa? (cf. \emph{Familiaris Consortio}, 23). Esto es posible cuando las mujeres y los hombres son tratados con pleno respeto de su dignidad personal, por lo que son, más que por lo que hacen.

8. Comprendiendo la importancia esencial de la vida familiar para una sociedad justa y saludable, la Santa Sede ha presentado una Carta de los derechos de la familia basada en los derechos naturales y valores comunes de toda la humanidad. Está dirigida principalmente a gobiernos y organismos internacionales, como "modelo y punto de referencia para la elaboración de legislación y política familiar, y guía para programas de acción" (\emph{\emph{\href{http://www.vatican.va/roman_curia/pontifical_councils/family/documents/rc_pc_family_doc_19831022_family-rights_it.html}{Carta de los derechos de la familia, 22 de octubre de 1983}, Introducción}}).

Entre los diversos principios que la Iglesia apoya firmemente en todas las circunstancias se encuentran los siguientes, sobre los que quiero llamar vuestra atención: el derecho inalienable de los cónyuges "a establecer una familia y a decidir el intervalo entre los nacimientos y el número de hijos a procrear, teniendo plenamente en cuenta sus deberes para con ellos mismos, con los hijos ya nacidos, la familia y la sociedad, en una justa jerarquía de valores y conforme al orden moral objetivo... "; todas las presiones que limitan "la libertad de los padres para decidir sobre sus hijos constituyen una grave ofensa contra la dignidad humana y la justicia"; "las familias tienen derecho a poder contar con una adecuada política familiar por parte de los poderes públicos en los ámbitos jurídico, económico, social y fiscal, sin discriminación de ningún tipo" (\emph{Carta de los derechos de la familia}, artículos 3 y 9) .

9. El orden moral exige que la regla establecida para los procesos de vida por el Creador en el acto de la creación sea respetada siempre y en todas partes. La conocida oposición de la Iglesia a la anticoncepción y la esterilización no es una posición tomada arbitrariamente, ni se basa en una perspectiva parcial de la persona humana. Más bien expresa su visión integral de la persona humana, a quien se le ha dado una vocación no sólo natural y terrena, sino también sobrenatural y eterna (cf. \emph{Humanae vitae}, 7). Además, la comprensión de la Iglesia del valor intrínseco de la vida humana como un regalo irrevocable de Dios explica por qué el Concilio Vaticano II habla de una "misión muy elevada para proteger la vida" y considera el aborto como un "crimen abominable" (\href{http://www.vatican.va/archive/hist_councils/ii_vatican_council/documents/vat-ii_const_19651207_gaudium-et-spes_it.html}{\emph{\emph{Gaudium et Spes}}}, 27. 51). ).

10. El lugar que ocupan los niños en la cultura y la sociedad (...) merece una consideración. Sé que amáis y respetáis a vuestros hijos. Sé que en muchos sentidos las leyes tienen como objetivo salvaguardar su bienestar y su protección. Una sociedad que ama a sus hijos es una sociedad sana y dinámica. En su nombre, os hago un llamamiento a vosotros, padres. Los niños necesitan padres que puedan brindarles un entorno familiar estable. Hacedle saber que necesitan del amor verdadero para sentirse unidos en vuestro amor por los demás y por ellos mismos. Ellos buscan en vosotros amistad y guía. De vosotros, sobre todo, deben aprender a distinguir entre lo justo y lo injusto y saber discernir el bien del mal. Os hago, pues, un llamamiento: no privéis a vuestros hijos de su herencia verdaderamente humana y espiritual. Habladle de Dios, de Jesús, de su amor y de su Evangelio. Enseñadle a amar a Dios y a respetar sus mandamientos con la certeza de que son ante todo hijos suyos. Enseñadle a orar. Ayudadle a convertirse en seres humanos maduros y responsables, ciudadanos honestos de su país. Este es un privilegio maravilloso, un deber importante y una asignación maravillosa que habéis recibido de Dios. Mediante el testimonio de vuestra vida cristiana, guiad a vuestros hijos a ocupar el lugar que les corresponde en la Iglesia de Cristo.

11. ¿Y qué deciros a vosotros, niños y jóvenes, {[}presentes aquí en tan gran número{]}? Amad a vuestros padres; rezad por ellos; dad gracias a Dios todos los días por ellos. Si a veces hay malentendidos entre vosotros, si a veces os resulta difícil obedecerlos, recordad estas palabras de San Pablo: ``Haced todo sin murmuraciones y sin críticas, para que seáis irreprensibles y sencillos, inmaculados, hijos de Dios ... que debes brillar como estrellas en el mundo'' (\emph{Fil} 2, 14-15). Rezad también por vuestros hermanos y hermanas y por todos los niños del mundo, especialmente por los pobres y hambrientos. Orad por los que no conocen a Jesús, por los que están solos y tristes.

A todos los jóvenes católicos (...) se os confía el futuro de la Iglesia en esta tierra. La Iglesia os necesita. Hay mucho que hacer en vuestras parroquias y comunidades locales, al servicio de los pobres y los necesitados, los enfermos y los ancianos, a través de las muchas formas de servicio voluntario. En primer lugar, debéis llevar a Cristo a vuestros amigos. Vuestra generación es el campo, rico para la mies, al que Cristo os envía. Cristo es el camino, la verdad y la vida para vuestra generación y para las generaciones venideras. Vosotros sois la esperanza de la Iglesia para una nueva era de evangelización y servicio. ¡Sed generosos con los demás, sed generosos con Cristo!

12. {[}Queridos padres e hijos, queridas familias ...: el Evangelio del primer domingo de Adviento nos llamaba a ``velar'', porque ``si el dueño de casa lo supiera ... velaría y no permitiría que entren en su casa'' (\emph{Mt} 24, 43). Esta es la exhortación que os repito. ¡Velad! No permitáis que os quiten el bien precioso del fiel amor matrimonial y la vida familiar. No los rechacéis, no penséis que hay una propuesta mejor para la felicidad o la realización humana.

La llamada del Evangelio a "velar" también significa construir en la familia un sentido de responsabilidad. El amor genuino es siempre amor responsable. Los esposos y las esposas se aman verdaderamente cuando son responsables ante Dios y llevan a cabo su plan para el amor y la vida humanos; cuando responden y son responsables unos de otros. La paternidad responsable implica no solo traer hijos al mundo, sino también participar personal y responsablemente en su crecimiento y educación. ¡El verdadero amor en la familia es para siempre! Finalmente, mientras nos esforzamos por ser perfectos en el amor, recordamos las palabras de San Pablo: ``Por lo tanto, desechemos las obras de las tinieblas y vistámonos las armas de la luz... en cambio, vestíos del Señor Jesucristo'' (\emph{Rom} 13, 12. 14).

Queridas familias (...) esta es vuestra vocación y vuestra felicidad hoy y siempre: revestirse del Señor Jesucristo y caminar en su luz. Amén.{]}
\end{body} 

\subsubsection{Homilía (1989): } Misa de fin de año y "Te Deum" de acción de gracias al Señor.

\emph{Domingo 31 de diciembre de 1989}

\begin{body} 
1. "Esta es la bendición del hombre que teme al Señor" (\emph{Salmo responsorial}).

Al cabo de un año más, la Iglesia, "casa" en la que el Verbo hecho hombre se complace en habitar, la familia de Dios que camina en el temor del Señor hacia el cumplimiento de los tiempos, quiere reconocer que ha sido "bendecido" por Dios, con toda bendición espiritual en Cristo Jesús (cf. \emph{Ef} 1, 2).

Al mismo tiempo, siente la necesidad de bendecir y agradecer a aquel de quien proviene todo don perfecto y en quien no hay variación ni sombra de cambio (cf. \emph{St} 1, 16).

Queridos hermanos y hermanas, estamos aquí, esta noche, precisamente para responder a esta necesidad íntima del alma: cantar nuestro "Te Deum" y celebrar la Eucaristía, que es precisamente acción de gracias, por los innumerables beneficios que nos concede la bondad divina en este año que está a punto de terminar...

2. "Esta es la bendición del hombre que teme al Señor". Las palabras del Salmo adquieren hoy un significado más amplio y abren horizontes más amplios. La liturgia de este domingo después de Navidad nos invita, de hecho, a detenernos en la contemplación frente al pesebre, donde nos encontramos con María y José y con el niño Jesús; nos invita a detenernos para aprender la lección de la Sagrada Familia de Nazaret y pedirle a Dios "que las mismas virtudes y el mismo amor puedan florecer en nuestras familias" (\emph{Oratio Collecta}).

Queremos hacerlo con una mirada atenta a la situación y las necesidades de las familias que viven en nuestra ciudad ...

3. ``Levántate, toma al niño y a su madre y huye a Egipto ... " (\emph{Mt} 2, 13).

El pasaje del Evangelio, que acabamos de escuchar, nos presenta un cuadro de la Familia de Nazaret donde no todo es idilio, paz y serenidad. Esta Santa Familia pasa por la prueba de la persecución y las dificultades del exilio. Se ve obligada a huir, a refugiarse, a buscar hospitalidad en otro lugar.

Estos son hechos que no deben sorprendernos. Constituyen una confirmación más de la realidad del misterio de la Encarnación, que estamos celebrando en estos días. Al hacerse hombre, el Hijo de Dios quiso vivir la experiencia concreta de la familia humana y asumir no sólo las alegrías, sino también las pruebas y dificultades: las mismas que muchas familias hoy, incluso en nuestra ciudad, conocen bien y que se intenta remediar con múltiples iniciativas de servicio y soporte.

4. En nuestro tiempo, a las dificultades habituales se han añadido las trampas que la rápida y profunda transformación socio\emph{-}cultural iniciada en las últimas décadas han traído al tejido vital de la familia. Esto constituye hoy un verdadero "desafío" para {[}toda la Iglesia{]} (...)

Es cierto que (...) hay todavía un gran número de familias en las que "el amor se guarda, se revela y se comunica" (\emph{Familiaris Consortio}, 7), pero es igualmente cierto que en la actual revolución social la célula familiar está particularmente en peligro. Las normas éticas y jurídicas, que han regulado su estructura y funciones durante siglos, son a menudo cuestionadas. El laicismo progresista tiende cada vez más a oscurecer e incluso negar esos valores naturales y creaturales de la institución familiar, inscritos en el plan redentor de Dios, reconocidos y potenciados haciendo de la familia, fundada en el sacramento del Matrimonio, una imagen de la Trinidad y una "Iglesia doméstica". Los datos, publicados {[}recientemente{]} (...) son preocupantes: las separaciones matrimoniales aumentan, el número de ``uniones libres'' es cada vez mayor, las tasas de natalidad están disminuyendo, el flagelo del aborto persiste.

Todo esto no puede dejar indiferente a la Iglesia, que ha recibido de Cristo, su Esposo, la misión "de iluminar y consolar a los cristianos y a todos los hombres que se esfuerzan por salvaguardar y promover la dignidad natural y el altísimo valor sagrado del estado matrimonial" \emph{Gaudium et Spes}, 47).

En este sentido, se abre un campo de acción amplio y exigente no solo para la comunidad eclesial (...), sino también para las instituciones públicas, que tienen en el corazón el bien común y la promoción integral de la persona humana.

Sobre todos, en el umbral del nuevo año, invoco las bendiciones del Señor para un renovado impulso en el cumplimiento de su servicio a la Iglesia y a la ciudad y, en particular, en beneficio de la familia, que es la célula fundamental de nuestra sociedad.

5. Es ampliamente reconocido que la crisis actual de la familia a menudo tiene sus raíces en la superficialidad de quienes se comprometen con ella. No pocas veces, de hecho, las parejas jóvenes muestran poca conciencia del significado y valor de la institución familiar, especialmente cuando se lo considera desde la perspectiva de la Revelación. Así sucede que incluso quienes eligen libremente casarse "en el Señor" a veces terminan distanciandose de las cuestiones morales vinculadas a este hecho, exponiéndose a un desorden fácilmente imaginable.

Por tanto, como opción prioritaria, se requiere la pastoral evangelizadora de la familia y, en ella, el compromiso de una preparación más adecuada al matrimonio. Ciertamente, ya se ha hecho mucho en este ámbito en los últimos años. Sin embargo, es necesario incrementar y unificar los esfuerzos, dando vida a itinerarios educativos reales, con herramientas y ayudas adecuadas y, sobre todo, con la implicación de los matrimonios más maduros en la fe y disponibles para esta forma particular de ministerio conyugal.

Un gran aporte a la pastoral familiar vendrá también de un compromiso más marcado con el establecimiento y la animación de "grupos familiares" de espiritualidad y servicio, cada vez más capaces de compartir "con generosidad ... su riqueza espiritual con otras familias" (\emph{Gaudium et Spes}, 48), para construir y expandir la comunidad eclesial, haciendo de la parroquia una" familia de familias" y, por tanto, una verdadera comunidad evangelizadora y testimonial. De hecho, "la evangelización, en el futuro, depende en gran medida de la Iglesia doméstica" (\emph{Familiaris Consortio}, 65).

6. Todo esto será mucho más fácil si las familias cristianas se esfuerzan por vivir la comunión de la que el Espíritu Santo es principio y alimento, que se les da en el sacramento del matrimonio. Una comunión fundada en la escucha de la Palabra de Dios, en la oración común, en el ejercicio de las virtudes cristianas, en primer lugar la caridad, "que es el vínculo de la perfección", según la enseñanza del apóstol Pablo que acabamos de escuchar en la segunda lectura.

Dado que la familia es la primera célula fundamental de la sociedad, es de esperar que ésta sepa aprobar leyes que protejan y promuevan la institución natural de la familia basada en el matrimonio y sus características de singularidad y estabilidad.

7. Hermanos y hermanas, cuando estamos a punto de concluir un año más que nos concede la bondad del Señor, escuchemos la amonestación de San Pablo: "Todo lo que de palabra o de obra realicéis, sea todo en nombre de Jesús, dando gracias a Dios Padre por medio de él".

Sí, al dar gracias a Dios Padre, a través de Cristo, en el Espíritu Santo, nos disponemos a hacer todo en su nombre y para su mayor gloria.

"Y que la paz de Cristo reine en vuestros corazones, porque a ella fuisteis llamados en un solo cuerpo". ¡Amén!

\end{body} 

\chapter{Santa María, Madre de Dios}

\subsubsection{Homilía (1984): } 01A-06-1Enero-1984

SANTA MISA DE LA XVII JORNADA MUNDIAL DE LA PAZ

\emph{\textbf{HOMILIA DE JUAN PABLO II}}

\emph{Solemnidad de María Santísima Madre de Dios}

\emph{Basílica de San Pedro - Domingo 1 de enero de 1984}

1. He aquí, estamos en el umbral del nuevo año {[}1984{]}, y clamamos: "\emph{Dios tenga misericordia de nosotros y nos bendiga}" (\emph{Sal} 67, 2).

Así clama toda la Iglesia en la liturgia del primer día del nuevo año, que es al mismo tiempo el día de la Octava de Navidad.

\emph{A través del misterio del nacimiento} de Dios en el tiempo, a través de los acontecimientos de Belén, nos separamos del año "viejo" y entramos en el año "nuevo". La Octava de Navidad une, por así decirlo, \emph{estas dos orillas del tiempo} humano y la existencia humana en la tierra. De esta manera la Iglesia quiere resaltar el hecho de que nuestra existencia en la tierra, en el mundo visible, está conectada \emph{con el Dios invisible} y que en él "vivimos, nos movemos y existimos" (\emph{Hch} 17, 28).

Más aún: Dios entró en nuestro tiempo humano, porque, hijo de la misma sustancia que el Padre, se hizo hombre por obra del Espíritu Santo y nació en la noche de Belén de la Virgen María. Desde ese momento \emph{nuestro tiempo humano} se ha convertido en su tiempo; por lo tanto, se llena no sólo por la historia del hombre y de la humanidad, sino que es \emph{llenado} también \emph{por el misterio salvífico de la 	Redención}, que opera precisamente en esta historia de la humanidad.

2. Hoy, en el último día dentro de la Octava de Navidad, la atención de la Iglesia, llena de la más alta veneración y amor, se concentra en María, \emph{Madre de Dios ("Theolokos}"), es decir, de aquella que ha dado al hijo de Dios naturaleza humana y vida humana.

Es la solemnidad de María santísima, Madre de Dios, gracias a ella pronunciamos hoy \emph{el nombre de Jesús}, porque ese día se le dio ese nombre al hijo de María.

\emph{También por ella y junto a ella clamamos en nombre de su Hijo} al comienzo del nuevo año: "¡Dios tenga misericordia de nosotros y nos bendiga!". Deseamos con este grito, en unión con la Madre de Dios, \emph{implorar todo el bien} para la gran familia humana, y prevenir el mal, todo mal. Clamemos, por tanto, en el nombre de Jesús, que significa "Salvador", y clamemos en unión con la Madre, a quien la Tradición de la Iglesia llama "la Omnipotencia implorante" ("\emph{Omnipotentia 	Supplex}").

{[}...{]}

3. La \emph{maternidad} siempre se explica en relación \emph{con la 	paternidad.}

Los padres, padre y madre, inician una nueva vida humana en la tierra, colaborando con el poder creativo de Dios mismo.

\emph{La maternidad} de María es \emph{virginal}. Por obra del Espíritu Santo concibió y dio al Hijo de Dios al mundo, "sin conocer varón".

San Pablo explica este misterio de la maternidad divina de María refiriéndose a la \emph{paternidad eterna de Dios:} "Cuando llegó la plenitud de los tiempos, Dios envió a su hijo, nacido de mujer" (\emph{Gal} 4, 4).

La maternidad virginal de la Madre de Dios equivale a la paternidad eterna de Dios, se encuentra, \emph{en cierto sentido}, en el camino de la \emph{misión del Hijo}, que del Padre llega a la humanidad a través de la Madre. Así, la maternidad de María \emph{abre el camino de Dios a 	la humanidad}. Es, en cierto sentido, el punto culminante de este camino.

Sabemos que el camino de esta misión, una vez abierto en la historia de la humanidad, siempre permanece. Siempre permite, a través de la historia de la humanidad, la misión salvífica del Hijo de Dios: \emph{la 	misión}, que se consuma con la cruz y la resurrección. Y junto con la misión del Hijo,permanece en la historia de la humanidad \emph{la 	maternidad salvífica} de su Madre terrena: María de Nazaret.

Veneramos esta maternidad el primer día del nuevo año. En efecto, deseamos que en esta nueva etapa del tiempo humano, María abra el camino de la humanidad a Cristo, tal como lo abrió en la noche del nacimiento de Dios.

4. El misterio de la solemnidad de hoy lleva consigo el siguiente \emph{llamado} a todos los hombres: Mirad, en Jesucristo \emph{todos 	hemos recibido al Padre.}

Cristo en su nacimiento terrenal nos trajo la misma paternidad divina: la dirigió a todos los hombres y la dio a todos como un don indispensable.

De esta paternidad de Dios hacia todos nosotros, \emph{da un testimonio} particularmente elocuente \emph{la maternidad de la Virgen Madre de 	Dios.}

La paternidad de Dios nos dice a todos los hombres \emph{que somos 	hermanos,} y la maternidad de María para toda la humanidad añade a esto un rasgo particular de familiaridad.

Tenemos derecho a pensar y \emph{hablar} de nosotros mismos considerándonos como "la familia humana". Todos somos hermanos y hermanas en esta familia.

¿No dice el Apóstol todo esto claramente en la liturgia de hoy?

- "Dios envió a su hijo, nacido de mujer. . ., para que seamos \emph{adoptados como hijos}" (\emph{Gal} 4, 4-5);

- "La prueba de que sois hijos es que Dios ha enviado a nuestros corazones el Espíritu de su Hijo que clama: Abba, Padre" (\emph{Gal} 4, 6);

- "Así que ya no eres esclavo, sino hijo; y si eres hijo, también eres heredero por voluntad de Dios" (\emph{Gal} 4, 7).

Esta filiación adoptiva de Dios es la gran herencia que nos dejó el nacimiento de Dios. \emph{Es la realidad de la Gracia de la Redención.} Al mismo tiempo, es un punto de referencia fundamental y central para toda la humanidad, para todos los hombres, si es cierto que debemos pensar y hablar \emph{de la fraternidad universal de hombres y pueblos.}

5. \emph{¿Y cuál es la realidad} que nos encontramos en nuestro gran planeta el 1 de enero del año {[}1984{]}? ¿No está acaso \emph{en 	profundo contraste} con la verdad sobre la hermandad universal de hombres y pueblos?

El mundo de hoy está cada vez más marcado por contrastes, entrampado por tensiones, que se manifiestan de manera lacerante y en direcciones cruzadas en las relaciones entre Oriente y Occidente y entre Norte y Sur.

Las relaciones entre Oriente y Occidente han llegado a posiciones radicalmente opuestas, con la interrupción -\/-que todos esperamos sea temporal y lo más breve posible-\/- de las negociaciones sobre las reducciones de armamentos nucleares y convencionales. Mientras tanto, la desconfianza mutua multiplica los efectos nocivos de las luchas ideológicas y exacerba los ya graves conflictos locales, de los que diariamente se ensangrentan diversas naciones, algunas de ellas muy pequeñas.

En la otra dirección entre el Norte y el Sur, la brecha que separa a los países ricos de los países pobres, ya grave desde hace muchos años, se ha ensanchado aún más con la reciente crisis económica. Según los expertos, una desaceleración del uno por ciento en la expansión económica de las naciones más industrializadas correspondería a un empobrecimiento de al menos un uno y medio por ciento en los países en desarrollo. El endeudamiento de estos, que ha alcanzado dimensiones catastróficas, da la medida del agravamiento divergente de estos contrastes económicos. Pero el aspecto más preocupante lo representan los contrastes que de él se derivan en la situación del hombre. En los países ricos, la salud y la nutrición están mejorando, mientras que en los países pobres hay una gran falta de alimentos para la supervivencia y la mortalidad, especialmente la mortalidad infantil, es desenfrenada. Según datos de Unicef, cuarenta mil niños menores de un año morirían cada día en el Tercer Mundo, mientras que la FAO estima que cada día más de quince mil personas morirían de hambre o de mala alimentación.

La amenaza de la catástrofe nuclear y la plaga del hambre aparecen escalofriantes en el horizonte como los jinetes fatales del Apocalipsis: fruto, ambos, de complejos fenómenos de orden económico, político, ideológico y moral, constituyen tantas fuentes de violencia que interactúan constantemente.

6. ¿Cuáles son -\/- nos preguntamos -\/- las causas fundamentales de estos fenómenos?

¿Y por qué el nivel de amenazas y plagas no disminuye, sino que aumenta?

La humanidad se hace estas preguntas con mayor preocupación. Expertos de las distintas ramas del conocimiento intentan explicar los mecanismos específicos que les afectan directa o indirectamente. Sin embargo, \emph{en el fondo de las diversas causas} y complejos mecanismos que acompañan a los procesos de desarrollo y civilización contemporánea, ¿no hay acaso \emph{una causa fundamental y última?}

¿Y acaso esta causa fundamental no está representada por el hecho de que se está perdiendo la conciencia de la hermandad radical de hombres y pueblos?

Todos somos hermanos. Esta \emph{hermandad} está \emph{ligada a la 	filiación común.} Somos hermanos porque somos hijos. Y esta filiación está vinculada \emph{a la paternidad de Dios mismo.} Somos hijos porque tenemos un padre.

Cuanto más perdemos, o intentamos eliminar, la conciencia de esta paternidad, más \emph{dejamos de ser hermanos} y, en consecuencia, la justicia, la paz y el amor social se alejan más de nosotros.

7. \emph{El mensaje} de este año para la Jornada Mundial de la Paz lleva el título: "La paz nace de un corazón nuevo".

Con este mensaje la Sede Apostólica \emph{suma su palabra} a todos aquellos esfuerzos, a veces desesperados, que realizan hombres de buena voluntad en todo el mundo, así como diversos organismos nacionales e internacionales \emph{para asegurar la paz} en el mundo contemporáneo.

Hoy queremos desarrollar plenamente, en cierto sentido, el contenido de este mensaje, a partir de esa luz que la Navidad aporta a la humanidad.

Así, en el curso de este santo sacrificio de Jesucristo y de la Iglesia, \emph{clamamos a Dios} y, al mismo tiempo, \emph{a todos los hombres}, orando por:

\emph{una eficacia renovada de la fraternidad universal} en el corazón de todos los hombres;

\emph{una renovada eficacia de la presencia} del Padre en las diversas dimensiones de la vida y la convivencia.

Solo en un corazón nuevo \emph{esta fuerza puede generar una paz} segura en la tierra.

Con toda humildad y con fe, confiamos el bien de esta paz a la Madre de Cristo.

Sí. ¡Unimos la esperanza de la paz, la justicia y el amor en la tierra con la maternidad de María, la Madre de Dios!

\subsubsection{Homilía (1987): } 01A-06-1Enero-1987

SANTA MISA PARA EL XX DÍA DE PAZ

\emph{\textbf{HOMILIA DE JUAN PABLO II}}

\emph{Solemnidad de María Santísima Madre de Dios}

\emph{Basílica de San Pedro - Jueves 1 de enero de 1987}

1. "Cuando llegó la plenitud de los tiempos, Dios envió a su Hijo ... " (\emph{Gal} 4, 4).

Te saludamos, plenitud de los tiempos, que el Hijo eterno de Dios trajo y cumplió en la historia de la creación, haciéndose hombre.

Te saludamos, plenitud de los tiempos, de la que hoy emerge el nuevo año, según la medida del paso humano.

Te saludamos, \emph{Año del Señor {[}1987{]}}, en el umbral de tus días, semanas y meses.

La Iglesia del Verbo Encarnado te saluda en medio de la gran familia de naciones y pueblos.

La Iglesia os saluda pronunciándoos \emph{la bendición del Dios de la 	Alianza}:

"Que el Señor te bendiga y te proteja. Que el Señor haga resplandecer su rostro sobre ti y tenga piedad de ti. Que el Señor vuelva sobre ti su rostro \emph{y te conceda la paz}" (\emph{Núm} 6, 24-26).

2. "Cuando llegó la plenitud de los tiempos, \emph{Dios envió a su Hijo} ... ".

Te saludamos, Año Nuevo, \emph{en el corazón mismo del misterio de la 	Encarnación}, en el que adoramos al Hijo de Dios hecho carne por nosotros.

Te saludamos, Hijo de la misma sustancia que el Padre Eterno, que vino a nosotros en la plenitud de los tiempos, "\emph{para que recibamos la 	adopción de hijos}" (\emph{Gal} 4, 5).

Te saludamos en tu humanidad, Hijo de Dios, nacido de mujer, así como cada uno de nosotros, hijos humanos, nacimos de mujer.

\emph{Te saludamos en la humanidad de cada hombre} en toda la riqueza y variedad de tribus, naciones y razas, idiomas, culturas y religiones.

\emph{En ti}, Hijo de María, en ti Hijo del hombre, \emph{somos hijos de 	Dios}.

Deseamos celebrar este primer día del nuevo año, junto con la octava de Navidad, \emph{como la solemnidad universal del hombre} en la plenitud de su dignidad humana.

Deseamos celebrar este día, gracias a tu obra, como "\emph{hijos en el 	Hijo}". Viniste "para reunir a los hijos de Dios dispersos" (\emph{Jn} 11, 52). \emph{Eres nuestro hermano y nuestra paz}.

3. "Cuando llegó la plenitud de los tiempos, Dios envió a nuestros corazones \emph{el Espíritu de su Hijo} que clama: ¡Abba, Padre!" (\emph{Gal} 4, 6).

Fuiste tú quien gritó así. Tú, el Hijo. Lo dijiste en momentos de fervor y momentos de desnudez.

Y tú, Hijo de la misma sustancia que el Padre, \emph{nos has enseñado a 	decirlo}; nos animaste a decir junto a ti: "Padre nuestro".

Y aunque no encontremos justificación en nuestra humanidad, nos has dado, en unidad con el Padre, tu Espíritu "que es Señor y dador vida" (Dominum et Vivificantem), \emph{para que podamos decir "Abba, Padre" 	con toda la verdad interior de nuestro corazón}. De hecho, el Espíritu del Hijo fue enviado a nuestros corazones. El Espíritu del Hijo \emph{nos ha formado de} nuevo, desde la raíz misma de nuestra humanidad, de nuestra naturaleza humana, como "hijos en el Hijo".

4. Somos, por tanto \emph{, hijos, no esclavos}. Somos \emph{herederos 	por voluntad de Dios}.

Hoy, al comienzo del nuevo año, deseamos \emph{reconfirmar esta herencia 	universal de todos los hijos e hijas de esta tierra}.

Todos están llamados a la libertad. En el contexto de los tiempos en que vivimos, la Iglesia ha confirmado una vez más la verdad sobre la "\emph{libertad y liberación cristianas}" como fundamento de la justicia y la paz (cf. Congr. Pro doctrina Fidei, \emph{Libertatis Conscientia}, 22 de Marzo de 1986).

El Espíritu del Hijo que el Padre envía incesantemente a nuestros corazones clama constantemente: "Ya no eres esclavo, sino hijo; y si eres hijo, también eres heredero por voluntad de Dios" (\emph{Gal} 4, 7).

5. "Cuando llegó la plenitud de los tiempos, Dios \emph{envió a su Hijo, 	nacido de mujer}".

Durante toda la octava de Navidad, y particularmente hoy, el corazón de la Iglesia late de manera singular por ella, por la \emph{Madre del Hijo 	de Dios}. Por la Madre de Dios.

Hoy se celebra su principal solemnidad. Ella, la Mujer, da el primer testimonio \emph{materno} de la dignidad humana del Hijo de Dios, que nació de ella.

\emph{Ella es su Madre}.

Hoy la vemos en Belén dando la bienvenida a los pastores.

Al octavo día después del nacimiento, cuando se completa el rito de la circuncisión del Antiguo Testamento, ella \emph{le da el nombre al 	Niño}. Y este es el nombre: Jesús, un nombre que habla de la salvación realizada por Dios. Esta salvación es traída por su Hijo. \emph{Jesús 	significa "Salvador"}. Así fue llamado el Hijo de María en el momento de la Anunciación, el día en que fue concebido en su seno. Y así ahora es llamado por ella antes que los hombres.

La dignidad humana del Hijo de Dios se expresa en este nombre. Como hombre, es el Salvador del mundo. Su Madre es la \emph{Madre del 	Salvador}.

6. ``\emph{Alégrate}, llena eres de gracia, el Señor es contigo ... " (\emph{Lc} 1, 28).

Bienaventurada tú que \emph{has creído} ... (cf. \emph{Lc} 1, 45). Creíste en el momento de la Anunciación. Creíste en la noche de Belén. Creíste en el Calvario. Has avanzado en la peregrinación de la fe y has conservado fielmente tu unión con el Hijo, Redentor del mundo (cf. \emph{Lumen gentium}, 58). Así, las generaciones del pueblo de Dios te han visto por toda la tierra. \emph{Así te ha mostrado}, Santísima Virgen, \emph{el Concilio de nuestro siglo}.

La Iglesia fija sus ojos en ti como en su propio modelo. {[}Los fija en particular en este período en el que se prepara para celebrar el advenimiento del tercer milenio de la era cristiana. Para prepararse mejor para ese acontecimiento, la Iglesia vuelve sus ojos hacia ti, que fuiste el instrumento providencial que utilizó el Hijo de Dios para convertirse en Hijo del Hombre y comenzar los nuevos tiempos. Con esta intención quiere celebrar un año especial dedicado a ti, un año mariano, que, a partir del próximo Pentecostés, terminará el año siguiente con la gran fiesta de tu asunción al cielo. Un año que cada diócesis celebrará con iniciativas particulares, encaminadas a profundizar en tu misterio y fomentar la devoción a ti en un compromiso renovado de adhesión a la voluntad de Dios, siguiendo el ejemplo que ofreciste, tú, la sierva del Señor. Estas iniciativas se enmarcarán fructíferamente en el tejido del año litúrgico y en la "geografía" de los santuarios, que la piedad de los fieles te ha elevado, oh Virgen María, en todas las partes de la tierra.{]}

{[}Deseamos, oh María, que brillas en el horizonte del advenimiento de nuestro tiempo, a medida que nos acercamos a la etapa del tercer milenio después de Cristo, profundizar \emph{la conciencia de tu presencia} en el misterio de Cristo y de la Iglesia, como nos enseñó el Concilio. Para ello, el actual Sucesor de Pedro, que te confía su ministerio, tiene la intención de dirigirse en un futuro próximo a sus hermanos en la fe con una \emph{encíclica}, dedicada a ti, Virgen María, don inestimable de Dios a la humanidad.{]}

7. ¡Bienaventurada tú que has creído!

El evangelista dice de ti: "María, por su parte, guardaba todas estas cosas y las meditaba en su corazón" (\emph{Lc} 2, 19).

\emph{¡Eres la memoria de la Iglesia!}

La Iglesia aprende de ti, María, que ser madre significa ser memoria viva, significa "guardar y meditar en el corazón" los acontecimientos de los hombres y de los pueblos; los acontecimientos alegres y dolorosos.

{[}Entre tantos acontecimientos de 1987, deseamos recordar el 600 aniversario del "bautismo de Lituania" a la memoria de la Iglesia, acercándonos con oración a nuestros hermanos y hermanas, que durante muchos siglos han perseverado unidos a Cristo en la fe de la Iglesia.{]}

Y cuántos otros acontecimientos, cuántas esperanzas, pero también cuántas amenazas, cuántas alegrías pero también cuántos sufrimientos ... ¡a veces cuántos grandes sufrimiento! Todos debemos, como Iglesia, guardar y meditar estos eventos en nuestro corazón. Así como la Madre. Tenemos que aprender cada vez más de ti, María, cómo ser Iglesia en este {[}tiempo{]}.

8. \emph{En el umbral del nuevo año, el Obispo de Roma}, abrazando en este sacrificio eucarístico a todas las Iglesias del mundo, unidas en la comunión universal \emph{católica},

- y a todos los amados hermanos \emph{cristianos} que buscan junto con nosotros los caminos de la unidad,

- y a todos los seguidores \emph{de religiones no cristianas},

- y, sin excepción, a todos los \emph{hombres de buena voluntad en} toda la tierra, grita desde la tumba de San Pedro con las palabras de la liturgia: "\emph{Que el Señor nos bendiga y nos proteja}. ¡Que el Señor haga resplandecer su rostro sobre nosotros y tenga misericordia de nosotros y nos dé la paz!" (\emph{Núm} 6, 24-26).

Que {[}1987{]} sea un año en el que la humanidad finalmente deje a un lado las divisiones del pasado; un año en el que, en solidaridad y desarrollo, todo corazón busque la paz.

\subsubsection{Homilía (1990): } 01A-06-1Enero-1990ES

\href{http://www.vatican.va/gmg/years/gmg_1990_it.html}{XXIII JORNADA 	MUNDIAL DE LA PAZ}

\emph{\textbf{HOMILIA DE JUAN PABLO II}}

\emph{Solemnidad de María Santísima Madre de Dios}

\emph{Lunes 1 de enero de 1990}

1. "María, por su parte, guardaba todas estas cosas, meditándolas en su corazón" (\emph{Lc} 2, 19).

El 1 de enero la Iglesia concluye la octava de Navidad, venerando la Maternidad de la Virgen María. Las palabras del Evangelio de Lucas ponen especial énfasis en la dimensión interior de esta Maternidad suya. Estas palabras son muy importantes para la Iglesia de hoy. Durante la octava, la Iglesia meditó sobre el misterio del nacimiento del Hijo de Dios en Belén. Hoy se refiere a la primera que meditó este misterio en su corazón. Ya que, como enseña el Concilio Vaticano II, "María va delante del pueblo de Dios en la peregrinación de la fe, y mantuvo fielmente su unión con el Hijo hasta la cruz" (cf. \emph{Lumen gentium}, 58), peregrinación que comienza en Belén.

Comienza en el Corazón de la Madre y continúa allí sin detenerse. Cada madre vive de una manera particular del recuerdo de haber dado a luz a un hijo. Este nacimiento vive en ella, lo guarda en su corazón. ¿Y qué pensar, entonces, de este nacimiento único en el que el Hijo de Dios vino al mundo?

La Iglesia de hoy se refiere a la dimensión interior de la maternidad, y así venera al mismo tiempo el misterio de la Encarnación y la extraordinaria dignidad de la Virgen-Madre.

2. El misterio de la Encarnación es un principio nuevo en la historia de la salvación. Y también es un nuevo comienzo en la historia del hombre y la creación. El \textbf{apóstol Pablo} define este nuevo principio como "la plenitud de los tiempos". "Cuando llegó la plenitud de los tiempos, Dios envió a su Hijo, nacido de una mujer ... para que seamos adoptados como hijos" (\emph{Gal} 4, 4-5).

Lo que queda en la memoria viva de María -\/- y al mismo tiempo en la memoria viva de la Iglesia -\/- no es un evento único, un evento "cerrado". El nacimiento de Dios está abierto al hombre de todos los tiempos. En él se revela y moldea la adopción como hijos de Dios, que se transmite a todos los seres humanos: "Y el Verbo se hizo carne y habitó entre nosotros ... A cuantos lo acogieron, él les dio poder para ser hijos de Dios" (\emph{Jn} 1, 14. 12). Las palabras del Prólogo de Juan, recordadas durante la octava de Navidad, atestiguan la duración continua del misterio, que comenzó la noche de Belén.

¡Sí! El Hijo de Dios se hizo hombre solo una vez, nació solo una vez de la Virgen María y, sin embargo, la filiación divina es una herencia continua del hombre.

3. El \textbf{apóstol Pablo} todavía habla de esta herencia. Es la obra incesante del Espíritu Santo: fruto de su acción en nosotros. "Y la prueba de que sois hijos es que Dios ha enviado el Espíritu de su Hijo a nuestros corazones que clama: ¡Abba, Padre! Así que ya no eres esclavo, sino hijo; y si eres hijo, también eres heredero por voluntad de Dios" (\emph{Gal} 4, 6-7).

La Iglesia guarda esta herencia, es su custodio y administrador en la tierra. Por eso fija constantemente sus ojos en el misterio de la encarnación. Y desea mirarlo con los ojos de María, participar de su memoria. En ninguna otra criatura está la Navidad tan profundamente inscrita como en ella. De hecho, se identifica con su maternidad. La maternidad humana de esta "Mujer" es, al mismo tiempo, maternidad divina. Aquel que fue traído a la luz por ella es, en realidad, el Hombre-Dios.

María como dice el Concilio Vaticano II "creyendo y obedeciendo, engendró en la tierra al mismo Hijo del Padre, y sin conocer varón, cubierta con la sombra del Espíritu Santo, como una nueva Eva, que presta su fe exenta de toda duda, no a la antigua serpiente, sino al mensajero de Dios, dio a luz al Hijo, a quien Dios constituyó primogénito entre muchos hermanos (cf. \emph{Rm} 8,29), esto es, los fieles, a cuya generación y educación coopera con amor materno" (\emph{Lumen gentium}, 63).

4. Este día de la octava es, por tanto, la fiesta de la herencia divina, en la que participan todos los hombres. La filiación divina, como don del Espíritu Santo en el hombre, impregna toda la herencia de la humanidad, de la naturaleza humana; toda la herencia, de hecho, de la creación misma. De hecho, el hombre fue creado a imagen de Dios y fue colocado en el mundo visible en medio de todas las criaturas.

Si la Iglesia celebra hoy el Día Internacional de la Paz, en la octava de Navidad, es porque hay una profunda lógica de fe en este hecho. De hecho, la paz exige una responsabilidad particular del hombre por toda la creación.

El mensaje pontificio para el nuevo año enfatiza esta responsabilidad en particular: "Paz con Dios creador - Paz con toda la creación". El mensaje del Evangelio de la paz nos recuerda constantemente, y hace referencia una y otra vez al mandamiento de no matar. ¡No mates a otro hombre, no mates desde el momento de su concepción en el vientre de su madre, no mates! No limites la existencia humana en la tierra con el método de lucha: de la violencia, el terrorismo, la guerra, los medios de exterminio masivo. No mates, porque toda vida humana es patrimonio común de todos los hombres.

Y también: no mates, destruyendo tu entorno natural de diferentes formas. Este entorno también pertenece al patrimonio común de todos los hombres, no solo de las generaciones pasadas y contemporáneas, sino también de las futuras. ¡Se un defensor, no un destructor de la vida!

El primer día del nuevo año pide una referencia particular a esta herencia. La herencia de los hijos adoptivos de Dios está íntimamente ligada al imperativo de la paz.

5. {[}Hoy no es sólo el primer día del nuevo año 1990, sino también el de la nueva década. Esta es la última década del siglo XX y, al mismo tiempo, del segundo milenio desde el nacimiento de Cristo.{]}

La Iglesia vuelve a Belén. Allí donde "fueron {[}los pastores{]} y encontraron a María, a José y al niño, que estaba acostado en el pesebre" (\emph{Lc} 2, 16). En el transcurso de los años que siguieron, la Iglesia no cesó de rezar a la Madre de Dios para que estuviera particularmente cerca de ella para recordar el misterio que ella acariciaba y meditaba en su corazón.

{[}En el umbral de la última década de nuestro siglo y del segundo milenio,{]} deseamos participar de manera particular en este recogimiento materno de María sobre el misterio del Hijo nacido, crucificado y resucitado. En él se renueva constantemente la "adopción como hijos" de Dios de todos los hombres. Toda la creación lo espera como herencia terrena del hombre, llamado a la gloria eterna en Cristo.

\subsubsection{Homilía (1993): } MISA EN HONOR A MARÍA SANTA MADRE DE DIOS

XXVI JORNADA MUNDIAL DE LA PAZ

HOMILÍA DE JUAN PABLO II

Viernes 1 de enero de 1993

1. "Le dieron por nombre Jesús" (\emph{Lc} 2, 21). Hoy es el octavo día desde el nacimiento del Hijo de María, en la noche de Belén. Hoy "Jesús recibió su nombre", "como lo llamó el ángel antes de que fuera concebido en el vientre de su madre" (\emph{Lc} 2, 21). "Dios envió a su Hijo, nacido de mujer" (\emph{Gal} 4, 4). El Padre Eterno quiso que este nombre se le impusiera a su Hijo unigénito: Jesús, que significa: "Dios salva". Es un nombre que se usa en Israel, y muchos antes de él lo habían llevado. Sin embargo, sólo el Padre Eterno le dio este nombre al Redentor, y María y José, en el día de la circuncisión, fueron humildes ejecutores de su voluntad. El Padre celestial quería que su Hijo, consustancial con él, Dios de Dios, como Hombre, como Hijo del hombre, llevara el nombre de Jesús. Este nombre significaría para siempre la misión que cumplió en la "plenitud de los tiempos" (cf. \emph{Gal} 4, 4). Dios envió a su Hijo al mundo "para que el mundo sea salvo por él" (cf. \emph{Jn} 3, 17). El nombre de Jesús tiene un carácter universal: es decir, expresa la voluntad salvífica de Dios con respecto al mundo, a la humanidad. "Dios ... quiere que todos los hombres se salven y lleguen al conocimiento de la verdad" (\emph{1 Tim} 2, 4).

2. Salvar significa liberarse del mal, y Jesús nos ordenó orar al Padre por esto: "Líbranos del mal". Unió así nuestra oración a su misión en el mundo; una misión que ya marca el momento de su nacimiento en Belén: ``Natus est nobis Salvator mundi''. ¡Salvar! Liberarse del mal, vencer el mal, todo esto no significa más que dejar lugar para el bien. En el hombre, una vez eliminado el mal, no debe haber vacío: el mal retrocede y desaparece ante el bien. La venida al mundo del Hijo de Dios significa que Dios quiere erradicar definitivamente el mal que hay en la humanidad, introduciendo al hombre en la dimensión divina del bien, como nos enseña el Apóstol, de hecho, en la Carta a los Gálatas: "Dios envió a su Hijo, nacido de mujer, nacido bajo la ley ... para que recibiéramos la adopción como hijos ... así que ya no somos esclavos, sino hijos" (\emph{Gal} 4, 4-7). Eres hijo, es decir, un hombre que en el poder del Espíritu puede clamar al Padre: "¡Abba, Padre!" (\emph{Gal} 4, 6). El Espíritu Santo, de hecho, fue enviado como el Espíritu del Hijo. Los que se convierten en hijos adoptivos de Dios en su Hijo único, se convierten al mismo tiempo en herederos: tienen parte del Bien imperecedero, que es Dios mismo. Toda esta verdad está contenida en el nombre "Jesús": el Salvador.

3. Según las propias palabras de Jesús, la filiación divina está relacionada con la irradiación de la paz. "Bienaventurados los que trabajan por la paz, porque serán llamados hijos de Dios" (\emph{Mt} 5, 9). Hoy, primer día del nuevo año, queremos profesar y anunciar que Jesús es nuestra Paz. Jesús, cuyo nombre significa "Dios salva". El bien de la paz está incluido en esta misión salvadora suya. El mismo Cristo, llamado Príncipe de Paz por el Profeta del Antiguo Testamento, logra la reconciliación entre Dios y la humanidad. Y esta reconciliación constituye la primera dimensión de la paz. En ella encuentra su comienzo y su raíz toda paz que los hijos adoptivos de Dios están llamados a realizar en el mundo y entre los hombres, haciéndose partícipes, incluso "co-arquitectos", de la salvación mesiánica, anunciada en el nombre de Jesús y que él mismo ha traído al mundo: la paz en todas sus dimensiones es un bien que forma parte de la salvación. Constituye un aspecto integral del plan salvífico de Dios ofrecido a la humanidad en Cristo, su Hijo. Por eso quería que el nombre del Redentor fuera "Jesús".

4. "Paz en la tierra a los hombres en quien él se complace ..." (\emph{Lc} 2, 14), "a los hombres de buena voluntad". El mensaje de la noche de Belén habla de una estrecha conexión entre la paz en la tierra y la misión del Salvador. ¿Podría ser de otra manera? Salvar significa liberarse del mal y lo que constituye la antítesis de la paz, ¿no contiene en sí mismo toda la evidencia del mal? {[}Nuestro siglo, el siglo XX, lamentablemente ha resaltado esta evidencia de una manera única a través de las terribles experiencias de las dos guerras mundiales, y también a través de muchos otros conflictos, que, aunque no se definieron como mundiales, fueron sin embargo hechos de guerra, con todo el drama que comportan tales conflictos. Durante la década de 1980, cuando la amenaza de la guerra nuclear se había vuelto extremadamente peligrosa, cristianos y representantes de las otras religiones del mundo se reunieron en Asís para gritar, en el mismo lugar, "líbranos del mal", "dona nobis pacem". ¿Es posible pensar que una oración tan confiada no ha sido escuchada por Aquel que es el Dios de la Paz? Hoy, el horror de la destrucción nuclear parece haberse alejado de la humanidad, pero el bien de la paz aún no se ha consolidado en todas partes. Los acontecimientos recientes ocurridos fuera de Europa y en la propia Europa así lo demuestran. Lamentablemente, incluso en nuestro continente, particularmente en las regiones de los Balcanes, la propagación del mal de la guerra y la violencia destructivas no está disminuyendo. ¿Puede Europa distanciarse de esta situación y no sentirse desafiada por ella? Los discípulos y confesores de Cristo, su Iglesia, no pueden dejar de pensar y trabajar en el espíritu de las ocho Bienaventuranzas: "Bienaventurados los que trabajan por la paz". Precisamente por eso, todas las Conferencias Episcopales de Europa, junto con el Obispo de Roma, han proclamado este 1 de enero como Jornada de oración por la paz en Europa, en particular en los Balcanes. Por eso, como en 1986, volveremos a peregrinar a Asís.{]}

5. ¡Europa! "Que el Señor vuelva sobre ti su rostro y te conceda la paz" (\emph{Núm} 6, 26). Así exclamamos en el nombre de Jesús, es decir, en el nombre de Aquel que es el Salvador del mundo -\/- del hombre -\/- de todos los hombres: de naciones, países y continentes. Él no tiene a su disposición los medios que pueden utilizar los estados y los poderosos de esta tierra. Su poder radica en la pobreza de la noche de Belén y luego en la Cruz del Gólgota. Sin embargo, es un poder que penetra más profundamente. De hecho, sólo este poder puede erradicar el odio, el principal enemigo de la paz, en las profundidades del ser humano. Sólo l es capaz de transformar a los trabajadores de la guerra y la destrucción en constructores de paz, a los que se puede dar el nombre de hijos de Dios.

6. ¡María! Hoy la Iglesia medita sobre el misterio de tu Maternidad. Tú eres la "memoria" de todas las grandes obras de Dios, tú conoces los caminos por los que el Hijo, Verbo consustancial con el Padre, vino al hombre: ¡Cristo, Salvador del mundo! Él es nuestra Paz. María, Madre de la Paz, intercede por nosotros ante él, intercede por nosotros. ¡Amén!

\subsubsection{Homilía (1996): }

XXIX Jornada Mundial de la Paz. Lunes 1 de enero de 1996.

1. "Se le puso el nombre de Jesús ..." (\emph{Lc} 2, 21).

El \textbf{Evangelio} que se acaba de proclamar recuerda que al hijo de María, nacido en Belén, al final de los ocho días prescritos, se le dio el nombre de Jesús, nombre con el que fue llamado por el ángel, antes de ser concebido en el vientre de la Madre (cf. \emph{Lc} 2, 21). Por tanto, fue \emph{el nombre que le dio el Padre celestial}.

\emph{Jesús significa: "Dios salva"}. Con este nombre comenzamos el Año Nuevo: {[}1996{]} desde el nacimiento de Cristo. El hecho de que contamos los años de nuestra era desde el nacimiento de Cristo es muy elocuente. Indica que \emph{Jesús es el centro de la historia}. En Cristo, el Hijo de Dios asumió la naturaleza humana. Y es precisamente el misterio de la Encarnación lo que explica plenamente el significado del nombre de Jesús: "Él (Dios) viene a salvaros" (\emph{Is} 35, 4). El tiempo humano está completamente impregnado del misterio salvífico de Dios. \emph{La historia de la humanidad se ha convertido en la historia de la salvación}.

El primer día del Año Nuevo, combinado con el recuerdo del nombre de Jesús, revela así este profundo significado. Es el día de la octava del nacimiento del Señor, en el que la Iglesia venera especialmente \emph{la maternidad divina de la Madre de Dios}. El primer día del Año Nuevo es su fiesta, la fiesta de la Madre del Dios-Hombre, de la \emph{Theotokos}.

2. El pasaje de la Carta de \textbf{San Pablo a los Gálatas} propuesto por la liturgia de hoy es, en cierto sentido, \emph{el comentario sobre el nombre de Jesús}. El apóstol revela de manera lapidaria todo lo que encierra el significado de este nombre, mostrando \emph{cómo Dios salva}. Por tanto, leemos: "Dios envió a su Hijo, nacido de mujer ..., para que recibiéramos la adopción como hijos" (\emph{Gal} 4, 4-5). La salvación, por tanto, se realiza en la adopción como hijos: en Cristo, el unigénito Hijo de Dios, los hombres se han convertido en hijos adoptivos de Dios, por lo que entendemos cómo el nombre "Jesús" tiene en sí mismo \emph{un dinamismo particular}. Dios no sólo ordena que su Hijo sea llamado por el nombre de Jesús, sino que al mismo tiempo manifiesta en este nombre la profundidad y extensión del misterio de la salvación. \emph{El nombre de Jesús revela el misterio de la adopción como hijos de Dios}. San Pablo añade casi desde una perspectiva profética: "Y la prueba de que sois hijos es que Dios ha enviado a nuestros corazones el Espíritu de su Hijo que grita: "¡\emph{Abba, Padre}!" (\emph{Gal} 4, 6). Jesús nos enseñó a volvernos a Dios diciendo: "¡Padre nuestro!". Pero estas palabras humanas toman del Espíritu Santo, que es el Espíritu del Hijo, su propio poder. Cuando rezamos: "\emph{Abbà}, Padre nuestro", estas palabras humanas nuestras son ante todo \emph{una forma de participar en la vida del Verbo eterno}, Hijo consustancial al Padre. A través de esta "participación", la invocación "¡Abbà, Padre nuestro!" se convierte en expresión de salvación.

Cristo es el Salvador del mundo, porque por él y en él todos los hombres pueden pronunciar esta palabra, palabra que le pertenece plenamente sólo a él, el Hijo eterno. \emph{En él, la filiación divina se ha convertido en nuestra herencia}. Por voluntad de Dios, como hijos adoptivos, somos coherederos con el Hijo eterno, llamados a participar de la vida de Dios, de la felicidad eterna en Él.

3. \emph{El nombre de Jesús, "Dios salva", atestigua que Él es nuestro Salvador}. Las lecturas de la liturgia de hoy nos presentan una vez más \emph{la dimensión universal de la salvación}, a la que están llamados todos los hombres y todos los pueblos, por el misterio de la Encarnación. El Salmo Responsorial lo resalta bien: "Alégrense y regocíjense las naciones, Él juzgará a los pueblos con justicia, gobernará las naciones de la tierra" (\emph{Sal} 66 {[}67{]}, 5). Lo que la \textbf{primera lectura} refiere a los hijos de Israel, el \textbf{Salmo} lo extiende a los pueblos y naciones de la tierra. \emph{La salvación está destinada a toda la humanidad}. Este secreto privilegio no está reservado a una persona o pueblo, sino que está destinado a todos los hombres.

Es una participación que pasa por el \emph{santo temor de Dios}, principio de la sabiduría (cf. \emph{Sal} 110 {[}111{]}, 10). \emph{La venida del Redentor del mundo}, para quienes lo acogen con reverencia y agradecimiento, marca el comienzo de \emph{un nuevo orden}, el orden divino. Con el nacimiento del Hijo de Dios en la naturaleza humana se expresa la voluntad salvífica de Dios; la Divina Providencia se manifiesta y guía el destino del mundo; se anuncia \emph{la justicia definitiva de la historia}, justicia unida a la misericordia. Por eso el salmista proclama: "El Señor tenga misericordia de nosotros y nos bendiga, haga brillar su rostro sobre nosotros; para que sea conocido en la tierra tu camino, tu salvación entre todas las naciones" (\emph{Sal} 66 {[}67{]}, 2-3). El misterio de la Navidad y el nombre mismo de "Jesús" representan así para la humanidad el signo del orden divino, que contiene la historia de la creación y de cada pueblo y nación.

Con razón, por tanto, la Iglesia celebra en este primer día del Año Nuevo \emph{la Jornada Mundial de la Paz} ... "Que el Señor vuelva sobre ti su rostro y te conceda la paz" (\emph{Núm} 6, 26), anuncia hoy la \textbf{primera lectura}. \emph{La paz, signo fundamental de la presencia divina, debe irradiar también en el orden político y en la vida de las comunidades y naciones}. La conocida expresión de Pablo VI: "Desarrollo es el nuevo nombre de la paz" (\emph{Populorum Progressio}, 87), podría ser invertida y formulada de la siguiente manera: la paz es el nuevo nombre del desarrollo y del orden social.

La paz en el lenguaje bíblico indica participación en la salvación que viene de Dios. La paz ya está contenida en el nombre dado al Hijo de Dios ocho días después de su nacimiento. Ese nombre significa salvación de todo mal, especialmente del odio, la guerra y la destrucción. Por tanto, el apóstol Pablo dirá de Cristo: "\emph{Porque él es nuestra paz}" (\emph{Ef} 2, 14).

4. "Muchas veces y de diversas maneras Dios habló a nuestros padres por medio de los profetas; hoy, sin embargo, nos habla por medio del Hijo" (Versículo antes del Evangelio: cf. \emph{Hb} 1, 1-2). En estas palabras encontramos el paso de la Antigua a la Nueva Alianza. Dios nos habló a través del Hijo, a través de su vida y su Evangelio. Nos habló a través de su muerte y resurrección y, \emph{de manera particular, a través de su nombre}: Jesús, que significa "Dios salva". Todo está encerrado en Él: vida, pasión, muerte y resurrección, cruz y gloria. Toda la buena noticia.

El autor de la Carta a los Hebreos nos habla de que este nombre se ha conservado para los "últimos tiempos". Al comienzo del nuevo año somos conscientes de que, en el nombre de Jesús, \emph{los últimos tiempos, el tiempo del cumplimiento} de todas las cosas en Dios, se ha acercado a la humanidad de manera decisiva. Y en virtud de este nombre vamos al encuentro de la meta definitiva del hombre, la definitiva "\emph{plenitud de los tiempos}" (cf. \emph{Gal} 4, 4), a la que Cristo nos conduce por el Espíritu Santo.

Guiados por esta fuerza repetimos: "¡\emph{Abbà}, Padre!" ya aquí, en la tierra, para prepararnos para la realización que, precisamente en el nombre de Jesús, debe manifestarse al final de los tiempos para todo hombre y para toda la familia humana.



\chapter{II Navidad}

\subsubsection{Homilía (1983): }

CELEBRACIÓN DEL TE DEUM DE ACCIÓN DE GRACIAS POR EL FIN DE AÑO

HOMILIA DE JUAN PABLO II

Sábado 31 de diciembre de 1983.

Pequeñas partes de la homilía fueron adaptadas para situarla en un contexto de inicio de año.

"Hijos, esta es la última hora" (\emph{1 Jn} 2, 18).

1. Estamos reunidos aquí, como siempre, en la "Iglesia de Jesús" para prepararnos al encuentro con la última hora del año del Señor {[}1983{]}, y la liturgia dirige nuestro pensamiento hacia Dios, en quien todo lo existente encuentra su comienzo y su fin.

El Evangelio de San Juan nos invita a volvernos a esta Palabra que en un principio estaba con Dios.

He aquí la Palabra eterna: "todo fue hecho por él, y sin él nada se hizo de todo lo que existe" (\emph{Jn} 1, 3).

Por tanto, también este año, que pasa como componente del tiempo humano y del paso cósmico, "se hizo" por medio del Verbo Eterno que "estaba en el principio con Dios" (cf. \emph{Jn} 1, 2) y que era Dios ( cf. \emph{Jn} 1, 1).

{[}Este{]} año del Señor (...), queremos referirlo al principio absoluto. Deseamos redescubrir su lugar en la eternidad que no pasa.

2. "Y el Verbo se hizo carne y habitó entre nosotros" (\emph{Jn} 1, 14). (...) Dios en su Hijo Eterno acogió nuestro tiempo humano y todo el pasado cósmico. Nació la noche de Belén de la Inmaculada Virgen María bajo la protección del carpintero José de Nazaret. Nació en un establo porque "no había lugar para ellos en la posada" (\emph{Lc} 2, 7), porque "su propia gente no lo aceptó" (\emph{Jn} 1, 11).

Todas las decepciones, tristezas y sufrimientos de nuestro mundo humano ya están insertadas de cierta manera en este Nacimiento de Dios en la tierra. Y estarán incluidas para todos los días de la peregrinación terrestre de Jesús de Nazaret hasta llegar a Getsemaní y a la cruz. En unión con él podemos vivir cada una de nuestras acciones a lo largo del tiempo. Podemos caminar los días y las horas de este año con la memoria y el corazón, también, y, especialmente aquellas que más nos hagan sufrir, porque Cristo está presente en ellas de una manera particular. Está presente a través del misterio de la Redención.

3. (...) Jesucristo, crucificado y resucitado, que está en la gloria del Padre, existe simultáneamente en el Cuerpo de su Iglesia.

El Unigénito, lleno de gracia y de verdad, obtiene la gloria del Padre (cf. \emph{Jn} 1, 14) y, al mismo tiempo, por esta gracia y verdad está con nosotros, está en su Iglesia, porque "la gracia y la verdad vino (a nosotros) por Jesucristo" (\emph{Jn} 1, 17).

Esta aceptación de la gracia y la verdad cuando el Verbo se hizo carne determina que el mundo y el hombre están envueltos en el misterio de la Redención. El hombre y el mundo a través de este misterio, de una manera nueva, existen en Dios por la obra de Cristo Redentor.

En {[}este año estamos invitados, junto a toda{]} la Iglesia a sumergirnos de un modo nuevo en "su plenitud" (\emph{Jn} 1, 16): en la plenitud del Redentor del mundo para recibir de esta plenitud "gracia sobre gracia" (\emph{Jn} 1, 16).

{[}...{]}

6. {[}En este año{]} vayamos al encuentro del Señor, dando gloria a Dios, con espíritu de acción de gracias y pidiendo perdón.

Que la gracia y la verdad que "vinieron (a nosotros) por Jesucristo" nos acompañen siempre. En el misterio de la Redención, esta gracia y esta verdad seguirán guiando al hombre y al mundo al encuentro con Aquel "que es, que era y que ha de venir" (\emph{Ap} 1, 8): al encuentro con Dios que es la eternidad y la santidad. Amén.

\chapter{Epifanía}

\subsubsection{Homilía (1984): }

Basílica Vaticana. Jueves 6 de enero de 1984.

1. Hoy, \emph{en el horizonte de la Navidad}, aparecen tres nuevas figuras: \emph{los Magos de Oriente.}

Vienen de lejos, siguiendo la luz de la estrella que se les apareció. Se dirigen a Jerusalén, llegan a la corte de Herodes. Preguntan: "¿Dónde está el \emph{rey de los judíos} que ha nacido? Vimos salir su estrella y venimos a adorarlo" (\emph{Mt} 2, 2).

2. En la liturgia de la Iglesia, la solemnidad de hoy lleva el nombre de \emph{la Epifanía del Señor. Epifanía significa manifestación.}

Esta expresión nos invita a pensar no solo en la estrella que apareció a los ojos de los Magos, no solo en el camino que toman estos hombres de Oriente, siguiendo el signo de la estrella. La epifanía nos invita \emph{a pensar en el camino interior}, en cuyo comienzo se encuentra el misterioso encuentro del intelecto y el corazón humano \emph{con la luz de Dios mismo.}

"La luz ... que ilumina a todo hombre cuando viene al mundo" (cf. \emph{Jn} 1, 9).

Los tres personajes del Oriente seguían esta luz con certeza incluso antes de que apareciera la estrella.

Dios les habló \emph{con la elocuencia de toda la creación}: les dijo que Él es, que Él existe; que es el Creador y Señor del mundo.

En cierto momento, más allá del velo de las criaturas, las acercó más \emph{aún a sí mismo.} Y, al mismo tiempo, comenzó a confiarles \emph{la verdad de su venida al mundo.} Ellos, de alguna manera, se han dado cuenta del plan divino de salvación.

Los magos \emph{respondieron con fe} a esa epifanía interior de Dios.

3. Esta fe les permitió reconocer el significado de la estrella. Esta fe también les ordenó partir. Fueron a Jerusalén, la capital de Israel, donde la verdad sobre la venida del Mesías fue transmitida de generación en generación. Los Profetas lo habían predicado y los libros sagrados habían escrito al respecto.

Dios que habló con la Epifanía en los corazones de los Magos, \emph{había hablado} a través de los siglos al \emph{Pueblo Elegido} y había profetizado la misma verdad sobre su venida.

4. Esta verdad se cumplió la noche del nacimiento de Dios en Belén. Esa \emph{noche ya es la Epifanía de Dios}, que vino: Dios que nació de la Virgen y fue puesto en el pesebre. Dios que \emph{ocultó su venida} en la pobreza de su nacimiento en Belén: he aquí \emph{la Epifanía del divino} ocultamiento.

Solo un \emph{grupo de pastores se} había apresurado a aquel encuentro ...

Pero ahora vienen \emph{los magos.} Dios, que se esconde de los ojos de los hombres que viven cerca de él, se \emph{revela} a los hombres que vienen de lejos.

El profeta dice en Jerusalén: "Caminarán las naciones a tu luz, y los reyes al resplandor de tu alborada. Alza los ojos en torno y mira: todos se reúnen y vienen a ti. Tus hijos vienen de lejos, y tus hijas son llevadas en brazos." (\emph{Is} 60, 3-4).

El Concilio habla de esta fuerza de la siguiente manera:

"Dispuso Dios en su sabiduría revelarse a Sí mismo y dar a conocer el misterio de su voluntad (cf. \emph{Ef} 1, 9), mediante el cual los hombres, por medio de Cristo, Verbo encarnado, tienen acceso al Padre en el Espíritu Santo y se hacen consortes de la naturaleza divina (cf. \emph{Ef} 2, 18; \emph{2 Pe} 1, 4). En consecuencia, por esta revelación, Dios invisible (cf. \emph{Col} 1, 15; \emph{1 Tim} 1, 17) habla a los hombres como amigos (cf. \emph{Ex} 33, 11; \emph{Jn} 15, 14-15), movido por su gran amor y mora con ellos, para invitarlos a la comunicación consigo y recibirlos en su compañía". (\emph{Dei Verbum}, 2).

Los Magos de Oriente llevan consigo esa fuerza interior de la Epifanía que les \emph{permite reconocer al Mesías} en el Niño acostado en el pesebre. Esta fuerza les ordena postrarse ante él y ofrecer los dones: oro, incienso y mirra (cf. \emph{Mt} 2, 11).

Los Magos son al mismo tiempo un presagio de que la fuerza interior de la Epifanía se extenderá ampliamente entre los pueblos de la tierra.

El profeta dice: "Tú entonces al verlo te pondrás radiante, se estremecerá y se ensanchará tu corazón, porque vendrán a ti los tesoros del mar, las riquezas de las naciones vendrán a ti" (\emph{Is} 60, 5).

6. Queridos hermanos (...) \emph{la solemnidad de hoy} manifiesta al Señor al mundo entero, porque su venida es para todos.

{[}Que esta celebración sea para nosotros{]} una nueva \emph{llamada} a someter toda la vida a la fuerza interior de la Epifanía, a través de la cual el Dios infinito confía a cada uno su misterio salvífico en Jesucristo, nacido en la noche de Belén de la Virgen Madre.

Acepta hoy esta llamada que te dirige la Iglesia.

Deja que \emph{este poder divino irradie} en tu corazón como en una Jerusalén interior, a la que la liturgia de hoy dice: "Levántate, revestido de luz, / que viene tu luz, / la gloria del Señor brilla sobre ti" (\emph{Is} 60, 1).

Deja que \emph{el poder salvífico de la divina Epifanía irradie} entre los hombres y pueblos a quienes eres enviado, como testigo de verdad y misericordia.

Verdaderamente: "Los bienes de los pueblos vendrán a ti" (\emph{Is} 60, 5).

Y responde \emph{al don} de la solemnidad de hoy con un incesante y continuo \emph{don}: ofrece oro, incienso y mirra.

Así \emph{la abundancia de la divina Epifanía} quedará en vosotros y se renovará en el camino del servicio apostólico. \emph{Amén.}

\subsubsection{Homilía (1987): }

Martes 6 de enero de 1987.

Ordenación de diez nuevos obispos.

1. \emph{Levántate, Jerusalén}, que viene tu luz. Levántate Jerusalén, vestida de luz (\emph{Is} 60, 1).

Estas palabras del profeta adquieren hoy una relevancia particular. De hecho, los Magos de Oriente llegan a Jerusalén precisamente con esta noticia: "\emph{¡Viene tu luz!}"

¿Dónde buscar el lugar de su nacimiento?

Jerusalén es la ciudad de un gran Rey, más grande que Herodes, y este gobernante temporal, que se sienta en el trono de Israel con el consentimiento de Roma, \emph{no puede ocultar la promesa de un Rey mesiánico}.

Es incapaz de oscurecer su luz.

¿Dónde buscar el lugar de nacimiento del Mesías?

Los escritos del Antiguo Testamento responden con certeza: \emph{en Belén}.

Encuentran un recién nacido. Ningún obstáculo externo puede apagar la luz \emph{que llevan en sus corazones}.

No prestan atención a la pobreza del lugar. Se postran. Le adoran. Ofrecen sus dones.

2. ¡Jerusalén, ha llegado tu luz!

\emph{¡Jerusalén vestida de luz!}

Hombres de lejos, "los reyes de Tarsis y las islas... los reyes de Saba y de Arabia" (\emph{Sal} 72, 10) caminan a tu luz.

Y la luz brilla en la oscuridad.

\emph{¡Jerusalén! El destino que Dios te ha dado es brillar}. Y ninguna oscuridad de la historia del hombre puede quitarte este destino, esta vocación.

Esta es la certeza expresada en nuestro tiempo por los Padres del Concilio Vaticano II, cuando comenzaron su documento sobre la Iglesia con las palabras \emph{Lumen gentium}: "Lumen gentium cum sit Christus".

¿Quién eres tú, Iglesia? ¿Qué dices de ti misma?

"Mi destino es brillar. \emph{¡Brilla con esta luz, que es Cristo!}".

Aquí está la elocuencia de la solemnidad de la Epifanía.

3. Esta elocuencia, esta verdad sobre Jerusalén, esta enseñanza sobre la Iglesia (...) manifiesta \emph{el carácter misionero de la Iglesia}, que se siente irrevocablemente enviada a todos los pueblos y a todos los hombres.

Quizás no apareció la estrella que una vez indujo a los Magos de Oriente a ponerse en el horizonte de tu cielo. Pero en tu alma \emph{hay la misma luz interior que los ha guiado}.

La misma luz te guía a ti también: "\emph{Lumen Gentium}".

4. En este momento (...) me dirijo a cada uno de vosotros con las palabras del profeta:

"Se estremecerá y se ensanchará tu corazón, porque vendrán a ti los tesoros del mar, las riquezas de las naciones vendrán a ti" (\emph{Is} 60, 5).

{[}...{]}

Mira: los magos de Oriente ofrecen los regalos que han traído. Abren sus tesoros y presentan los regalos.

Hoy estás llamado a abrir tu tesoro interior aún más profundamente. Estás llamado a entregarte aún más plenamente a Cristo, que es la Epifanía del Pastor eterno.

A través de tu dedicación, que la luz que es Cristo brille sobre todos aquellos a quienes eres enviado.

\subsubsection{Homilía (1990): } 01A-08-Epifanía-1990ES

Sábado 6 de enero de 1990. Ordenación Episcopal de Doce nuevos Obispos.

"Entraron en la casa; vieron al niño con María su madre y, postrándose, le adoraron; abrieron luego sus cofres y le ofrecieron dones de oro, incienso y mirra" (cf. \emph{Mt} 2, 11).

1. Estas palabras del Evangelio de Mateo contienen como una síntesis del misterio, que se expresa con el sustantivo griego "Epifanía". Los Magos, procedentes de Oriente, colocan los regalos a los pies del Niño de Belén: oro, incienso y mirra.

Estos dones son la respuesta al Don. El Don de arriba les fue anunciado por medio de la estrella brillante en la oscuridad. Los sabios de Oriente siguen la estrella, luego en Jerusalén le piden información a Herodes. Reciben explicaciones que se les dan con las mismas palabras que el Profeta. Van con perseverancia a Belén para recibir el Don de lo alto. Ellos llaman a este Don "el rey de los judíos que ha nacido", mientras que el Profeta lo llama "un jefe que pastoreará al pueblo". Él es el que el Padre ungió con el Espíritu Santo y envió al mundo: el Mesías. El Hijo Unigénito, dado por el Padre.

Los Magos encuentran al Niño en los brazos de la Madre: encuentran al Hijo del hombre. Saben que él es el Don del Padre. Vienen de lejos para acogerlo: acoger el Don en el que el Eterno expresa su amor: "Porque tanto amó Dios al mundo que le dio a su Hijo Unigénito" (\emph{Jn} 3, 16). Los magos de Oriente están entre los primeros en darle la bienvenida. Son los testigos de la divina Epifanía.

2. ¡Queridos hijos y hermanos! {[}Esta solemnidad nos invita a{]} acoger el mismo Don: acoger a Cristo, nacido en Belén, a quien el Padre envió al mundo. El Hijo unigénito que el Padre dio por amor al mundo que es suyo, para que todo el que crea en él no muera, mas tenga vida eterna.

{[}...{]}

3. Respecto a los dones que los Magos de Oriente ofrecieron al Niño recién nacido, el profeta Isaías había dicho las siguientes palabras: "Todos ellos de Sabá vienen portadores de oro e incienso y pregonando alabanzas al Señor" (\emph{Is} 60, 6).

E Isaías había dicho esto mientras tenía ante sus ojos la maravillosa procesión de las naciones, que iban a caminar hacia esa luz, que brillaría sobre Jerusalén: "Alza los ojos en torno y mira: todos se reúnen y vienen a ti. Tus hijos vienen de lejos, y tus hijas son llevadas en brazos. Tú entonces al verlo te pondrás radiante, se estremecerá y se ensanchará tu corazón, porque vendrán a ti los tesoros del mar, las riquezas de las naciones vendrán a ti" (\emph{Is} 60, 4-5).

4. Cada uno de vosotros, queridos hijos, trae a este altar, {[}en la Basílica de San Pedro,{]} su "don propio": el oro, el incienso y la mirra de su propia vida. Este don que brilla en vuestro corazón a través de la luz del Espíritu de la Verdad, que madura en el momento del ofertorio de hoy. Vuestro don debe hoy ser consagrado de nuevo y convertirse en una respuesta particular al Don de la divina Epifanía en Jesucristo.

La Epifanía es la fiesta del intercambio de regalos. Entonces, cada uno de vosotros trae aquí no solo su propio regalo. A través de cada uno de vosotros se expresa "toda la riqueza de la capacidad de los pueblos". (...).

Así, por el don que es propio de cada uno y por "toda la riqueza de la capacidad de los pueblos" que cada uno lleva en su interior, todos crecemos en esta comunión particular que es la Iglesia, el Cuerpo de Cristo, y al mismo tiempo todos formamos y enriquecemos esta comunión.

5. {[}...{]}

Queridos hermanos (...) que el Hijo de Dios nacido de la Virgen acoja vuestros dones, como acogió los dones de los Magos de Oriente, y os ayude a revelar siempre, con la luz y el poder del Espíritu Santo, a todos los hombres, a todos los pueblos y naciones de la tierra, el don del Hijo Eterno dado por el Padre, para que "todo el que crea en él no se pierda, mas tenga la vida eterna".

{[}...{]}

\subsubsection{Homilía (1993): } Miércoles 6 de enero de 1993. Ordenación Episcopal de Once Nuevos Obispos.

\emph{Todos los reyes se postrarán ante él, todas las naciones le servirán} (\emph{Sal} 71, 11).

1. La Constitución Dogmática \emph{Lumen gentium} del Concilio Ecuménico Vaticano II, citando un famoso pasaje de San Juan Crisóstomo, subraya esa universalidad del único pueblo de Dios que brilla de manera especial en la celebración de hoy.

Todos los hombres están llamados a formar parte del nuevo Pueblo de Dios. Por lo cual, este pueblo, sin dejar de ser uno y único, debe extenderse a todo el mundo y en todos los tiempos, para así cumplir el designio de la voluntad de Dios, quien en un principio creó una sola naturaleza humana, y a sus hijos, que estaban dispersos, determinó luego congregarlos (cf. \emph{Jn} 11,52). Para esto envió Dios a su Hijo, a quien constituyó en heredero de todo (cf. \emph{Hb} 1,2), para que sea Maestro, Rey y Sacerdote de todos, Cabeza del pueblo nuevo y universal de los hijos de Dios. Para esto, finalmente, envió Dios al Espíritu de su Hijo, Señor y Vivificador, quien es para toda la Iglesia y para todos y cada uno de los creyentes el principio de asociación y unidad en la doctrina de los Apóstoles, en la mutua unión, en la fracción del pan y en las oraciones (cf. \emph{Hch} 2,42 gr.)''. (\emph{Lumen} gentium, 13).

2. "Entraron en la casa; vieron al niño con María su madre y, postrándose, le adoraron" (\emph{Mt} 2, 11).

En la liturgia de hoy, la Iglesia revive la verdad de estas palabras. De hecho, mientras la noche de Navidad contemplamos el correr presuroso a la gruta de Belén de unos pastores pertenecientes al pueblo de Israel, hoy -\/- solemnidad de la Epifanía -\/- recordamos la llegada de los Magos, que venían del Lejano Oriente para adorar al Rey y Salvador universal en el Niño recién nacido y ofrecerle sus dones.

Los magos vienen preguntando: "¿Dónde está el rey de los judíos que nació?". Y vienen trayendo regalos (\emph{Mt} 2, 2). Cuando llegaron al "lugar donde estaba el niño ... se postraron y lo adoraron. Entonces abrieron sus cofres y le ofrecieron regalos de oro, incienso y mirra" (\emph{Mt} 2, 9. 11).

Habían preguntado al recién nacido "rey de los judíos" (\emph{Mt} 2, 2) y he aquí, ahora está ante ellos el rey no de un solo pueblo, sino de todas las naciones que le "han sido entregadas en posesión" (cf. \emph{Sal} 2, 8). De esta manera se hizo realidad la verdad ya anunciada por el salmista hace mucho tiempo. Los Magos con su gesto de adoración, por tanto, testifican que el niño Jesús no es rey de un solo pueblo, el pueblo de Israel, sino de todos los pueblos de la tierra.

3. Epifanía. La solemnidad de hoy lleva este nombre significativo. Los Magos, que vinieron de Oriente para ofrecer sus dones, se convierten en testigos del don santísimo ofrecido por Dios a los hombres: en el misterio de la Encarnación, Dios Padre ofrece a la humanidad su Hijo único, consustancial con él. En el Verbo hecho carne, el amor mismo de Dios se hizo visible: "Tanto amó Dios al mundo que dio a su Hijo unigénito" (\emph{Jn 3, 16} ). Los Magos, en presencia de este santísimo misterio, se postraron. Aunque sus ojos ven sólo a un niño recién nacido, la luz que los guía hacia adentro les permite reconocer lo que sus ojos no pueden percibir: les permite comprender el don santísimo ofrecido por Dios a la humanidad. Responden a este don santísimo con su don personal, presentando a Jesús oro, incienso y mirra, simples símbolos humanos, que expresan el misterio del rey a quien "todas las naciones han sido entregadas en posesión" (cf. \emph{Sal} 2, 8).

Al mismo tiempo, los dones ponen de manifiesto la profunda verdad de la Encarnación. Recibiendo la naturaleza humana, el Hijo de Dios también quiso compartir la amargura de la existencia terrena, que encontró su punto culminante cuando, en la agonía de la cruz, "le ofrecieron vino mezclado con mirra" (\emph{Mc} 15, 23).

{[}...{]}

{[}Que esta celebración nos revele{]} el don santísimo que, en Jesucristo, Dios concedió al mundo y la humanidad y lleguemos a conocer con qué inmenso amor Dios nos amó, sin dudar en dar por nosotros a su Hijo unigénito.

"A él se postrarán todos los reyes, todas las naciones le servirán". ¡Para siempre! Amén.

\subsubsection{Homilía (1996): } ORDENACIÓN

EN LA FIESTA DE LA EPIFANÍA

\emph{\textbf{HOMILÍA DE JUAN PABLO II}}

\emph{Basílica Vaticana - Sábado 6 de enero de 1996}

1. La fiesta que celebramos hoy, 6 de enero, lleva el nombre de "Epifanía". La palabra griega \emph{epifania} significa "revelación", "manifestación". Se dice en la Carta a Tito: "La gracia de Dios ha sido revelada, portadora de salvación para todos los hombres. . . " (2, 11), y nuevamente: "La bondad de Dios, nuestro Salvador, y su amor por los hombres han sido revelados. . . " (3, 4). \emph{La revelación es precisamente el descubrimiento del misterio de Dios Salvador}. Existe un estrecho vínculo de sentido entre unos y otros, entre la revelación y el misterio de la salvación.

\emph{El Creador le dio al hombre la capacidad de conocer el mundo}, las cosas visibles, los hechos históricos; también le dio la capacidad de penetrar con su propia razón más allá de la superficie de lo sensible. Pero Dios también vino al encuentro del hombre \emph{hablándole directamente}. La Revelación consiste precisamente en esto: \emph{Dios ha hablado al hombre revelándole} lo que sabe y piensa de sí mismo, del hombre, del mundo. Así, gracias a la revelación, conocemos el pensamiento de Dios, lo conocemos \emph{por nuestra razón}, pero \emph{no en virtud de nuestra razón}. Lo que Dios revela lo aceptamos porque \emph{confiamos en Él}. Esta \emph{confianza en nosotros mismos a la autoridad de Dios que se revela} se llama \emph{fe}.

Somos conscientes de que solo Dios mismo puede instruir al hombre sobre las realidades divinas. En la Constitución conciliar \emph{Del Verbum} sobre la Divina Revelación leemos: "Agradó a Dios, en su bondad y sabiduría, revelarse y manifestar el misterio de su voluntad (cf. \emph{Ef} 1, 9) ... Con esta revelación, en efecto, el \emph{Dios invisible} (cf. \emph{Col} 1, 15; \emph{1 Tm} 1,17) \emph{en su inmenso amor habla a los hombres como a amigos} (cf. \emph{Ex} 33, 11; \emph{Jn} 15, 14-15) \emph{y entretiene con ellos} (cf. \emph{Bar} 3, 38), \emph{para invitarlos y admitirlos a la comunión con Él}" (n. 2).

El hecho de que Dios quisiera revelar al hombre la verdad sobre sí mismo, una verdad que es un misterio, atestigua que el hombre es una criatura muy querida por Dios, una criatura hecha a su semejanza, la única en el mundo visible con quien Dios puede diálogar, al que puede confiar la verdad sobre sí mismo y su vida íntima, la verdad de sus misterios divinos.

2. "Hemos visto salir su estrella y hemos venido a adorarle" (\emph{Mt} 2, 2).

Los Magos de Oriente pronunciaron estas palabras en Jerusalén, frente al rey Herodes, quien no solo las entendió de una manera puramente humana, sino incluso con pérfida envidia. Sin embargo, estas palabras resumen \emph{la revelación sobre el nacimiento del Señor}. Los Sabios de Oriente, junto con los pastores de Belén, son los que fueron iniciados por Dios mismo, podríamos decir, fueron iniciados en el misterio de la encarnación del Hijo de Dios. Los pastores ya estaban casi en el lugar, cerca de la "Ciudad de David". En cambio, los magos vinieron de lejos, interpretando las señales que indicaban el momento y el lugar del nacimiento del Salvador. Y una señal particular fue la estrella, que los guió hacia la tierra de Israel: primero a Jerusalén, y luego a Belén.

En el signo visible de la estrella, el Dios invisible les habló. ¿Cómo pudo haber sucedido que, entre tantas estrellas observadas por los Sabios en la bóveda celeste, precisamente aquella estrella les hablara del nacimiento del Hijo de Dios en carne humana? \emph{Esto fue posible solo a través de la fe}. Los magos, habiendo llegado a Jerusalén, buscaron la confirmación de su intuición de los escribas, expertos en la revelación de Dios a Israel. Y obtuvieron la respuesta: el profeta Miqueas había anunciado que el Mesías nacería en Belén (cf. \emph{Mi} 5, 1). Fueron, pues, a Belén y entraron en la casa donde estaba el Niño, junto con su Madre y José; cayeron de rodillas y ofrecieron sus regalos simbólicos. Todo esto testifica que la \emph{fe los había introducido por el camino correcto hasta el centro mismo del misterio del nacimiento del Señor}.

{[}...{]}

4. "Levántate, vístete de luz, porque \emph{tu luz viene, la gloria del Señor brilla sobre ti}... Pueblos caminarán a tu luz, reyes al esplendor de tu ascenso. Levanta tus ojos alrededor y mira: todos estos se han reunido, vienen a ti. Tus hijos vienen de lejos, tus hijas son llevadas en tus brazos" (\emph{Is} 60, 1,3-4).

Los Magos, que llegaron a Belén desde Oriente, constituyen las primicias \emph{de la gran peregrinación de la fe}, que transcurre de generación en generación, acercando a los hombres, pueblos y naciones a Cristo, la luz del mundo. Numerosos pueblos y naciones han participado en esta peregrinación, que se lleva a cabo desde hace casi dos mil años. Y la luz, que se elevó sobre Jerusalén en la plenitud de los tiempos, no se apaga, sino que brilla con un resplandor siempre nuevo. Ilumina el camino de la humanidad en medio de la oscuridad que envuelve la tierra. Y continuamente, durante la noche de la que habla el profeta Isaías (cf. \emph{Is} 60, 2), resuena el grito de los pastores, de los magos, de todos los creyentes de todas las épocas: "Christus apparuit nobis, venite adoremus".

\chapter{Bautismo}

\subsubsection{Homilía (1984): } **** 01A-09-Bautismo-1984

SANTA MISA POR EL BAUTISMO DE 27 NIÑOS

HOMILIA DE JUAN PABLO II

Basílica Vaticana - Domingo 8 de enero de 1984

1. \emph{"¡Gloria y alabanza a tu nombre, oh Señor!".}

La invitación de la liturgia de hoy a glorificar y alabar el nombre del Señor adquiere un significado particular hoy, fiesta del Bautismo de Jesús.

{[}Quería celebrar este "misterio" de la vida de Cristo, confiriendo el Sacramento del Bautismo en esta Basílica a algunos niños y niñas en el Año Jubilar de la Redención, para subrayar que a través de este Sacramento el don de la Redención llega a quienes lo reciben. ., y se les aplica la obra de \emph{salvación} y \emph{santificación}, realizada por Jesús con su ofrenda al Padre celestial.{]}

La palabra de Dios, que hemos escuchado, nos presenta a Jesús de Nazaret como el "siervo de Dios", profetizado en el \emph{\textbf{Libro de Isaías}}; el siervo, objeto de divina elección y complacencia, cumplirá su misión con una actitud de total adhesión a la voluntad del Señor y de ejemplar humildad hacia los hombres; se establecerá como una "alianza de los pueblos", como "luz de las Naciones", es decir, de los pueblos paganos, para dar vista a los ciegos y libertad a los presos.

Este misterioso "Siervo de Dios" es Cristo, que trae la salvación a la humanidad. En el relato del \textbf{Evangelio} que acabamos de escuchar, tan pronto como Juan bautiza a Jesús, los cielos se abren, el Espíritu de Dios desciende sobre Cristo como una paloma y una voz -\/- la del Padre -\/- dice: "Este es mi Hijo amado, en quien me complazco".

La realidad reemplaza ahora a la profecía: la complacencia de Dios por su Siervo es la complacencia del Padre por su Hijo eterno, que asumió la naturaleza humana y que, con un gesto de profunda humildad, pidió a Juan ese bautismo, que era sólo una figura de lo que él mismo instituiría ya no como una preparación para la gracia, sino como un otorgamiento de gracia.

Queridos hermanos y hermanas aquí presentes {[}y, en particular, ¡ustedes que son los padres, madres, padrinos y madrinas de estos niños que pronto recibirán el Bautismo!{]} La Iglesia sigue obedeciendo a lo largo de los siglos las palabras que Jesús dirigió a los Apóstoles: "Se me ha dado toda potestad en el cielo y en la tierra. Id, pues, y haced discípulos de todas las naciones, bautizándolos en el nombre del Padre y del Hijo y del Espíritu Santo" (\emph{Mt} 28, 18-19).

2. {[}Estos hijos, fruto estupendo del amor de sus padres y también del misterioso gesto creador de Dios, nacieron a \emph{la vida natural.} Pero, en unos momentos, serán protagonistas de un \emph{nuevo nacimiento}, el de la vida sobrenatural, merecido para nosotros por Cristo: "En verdad, en verdad os digo -\/- estas son las palabras de Jesús a Nicodemo -\/-, a menos que uno sea nacido de agua y del Espíritu, no puede entrar en el reino de Dios. Lo que es nacido de la carne, carne es, y lo que es nacido del Espíritu, Espíritu es" (\emph{Jn} 3, 5-6).{]}

{[}El bautismo es una regeneración espiritual, un "nacimiento de arriba" (cf. \emph{Jn} 3, 7), no menos verdadero y real que el nacimiento a la vida terrena. Estos bebés a través del Bautismo -\/- como nos enseña la fe cristiana -\/- serán liberados del pecado original, serán santificados interiormente mediante la infusión de la gracia santificante, junto con los dones de las vestiduras de las virtudes teologales de la fe, la esperanza, la caridad y con los dones del Espíritu Santo; serán incorporados y conformados a Cristo y, por tanto, serán en él hijos de Dios, insertados en la Iglesia, cuerpo místico del Verbo Encarnado.{]}

3. {[}Esta vida divina, que les será comunicada por el Sacramento, debe crecer, madurar y perfeccionarse en el camino de su vida: "Los seguidores de Cristo, llamados por Dios no en razón de sus obras, sino en virtud del designio y gracia divinos y justificados en el Señor Jesús, han sido hechos por el bautismo, sacramento de la fe, verdaderos hijos de Dios y partícipes de la divina naturaleza, y, por lo mismo, realmente santos. En consecuencia, es necesario que con la ayuda de Dios conserven y perfeccionen en su vida la santificación que recibieron" \emph{Lumen gentium}, 40).{]}

{[}Es tarea y deber de toda la Iglesia, pero especialmente de los padres, padrinos y madrinas asumir la misión de ayudar a estos nuevos hijos de Dios en su crecimiento sobrenatural, ofreciéndoles cada día el ejemplo constante, generoso y constructivo de una existencia vivida enteramente de acuerdo con las exigencias del Evangelio.{]}

{[}¡Este es el deseo que hoy dirijo a estos pequeños, nuevos cristianos, nuevos miembros de la Iglesia exultantes, \emph{nuevos frutos de la Redención!}{]}

{[}¡Este es el compromiso de responsabilidad cristiana, que confío y encomiendo a todos los que me escuchan, en esta celebración comunitaria del Santo Bautismo!{]}

¡Que así sea!

\subsubsection{Homilía (1987): } 01A-09-Bautismo-1987ES

HOMILIA DE JUAN PABLO II

SANTA MISA POR EL BAUTISMO DE CUARENTA Y NUEVE BEBÉS

Domingo 11 de enero de 1987

\emph{!Queridísimos hermanos!}

Estamos aquí reunidos para celebrar hoy la fiesta litúrgica del Bautismo de Jesús: acontecimiento que, al revelar el misterio de la plena manifestación del Señor como Hijo de Dios ante todo el pueblo, inauguró la vida pública del Redentor. "Este es mi Hijo amado, en quien tengo mi complacencia" (\emph{Mt} 3, 17).

{[}Deseo hacerles llegar a todos mi cordial saludo y expresarles mi profunda alegría por poder conferir el sacramento del bautismo a estos niños. Los saludo, padres, y me alegra que hayan querido compartir con sus hijos, desde los primeros días de su vida, el don de la gracia, la vida divina y su fe. Los saludo a ustedes, padrinos y madrinas, que han aceptado con espíritu cristiano hacerse colaboradores en la formación cristiana y camino espiritual de estos niños, comprometiéndose a ser para ellos testigos de la verdad revelada y de la moral evangélica. Os saludo, como nuevos hermanos e hijos, a todos estos pequeños, a quienes deseo sinceramente una vida y un futuro lleno de felicidad, con el profundo deseo de que la gracia bautismal no falte nunca en su vida y que esta primera llamada a la fe algún día encuentre su feliz desarrollo en el pleno seguimiento de la voz del Señor.{]}

El bautismo ofrecido por Juan fue un signo de penitencia, un testimonio de conversión combinado con el deseo de volver a Dios, purificado en el espíritu, para acoger el cumplimiento de las promesas mesiánicas. A este bautismo, con un gesto que revela una gran humildad, el Hijo de Dios hecho carne se somete en nombre de todo el género humano. Así entra en un bautismo de penitencia que es la remisión de los pecados de todos los hombres, porque, como verdadero hombre, se ha hecho solidario con nosotros. Pero Jesús predice su futuro con este gesto: como "siervo elegido" de Dios (cf. \emph{Is} 42, 1) llevará sobre sí el peso de los pecados de todos y se ofrecerá un día en sacrificio para obtener, lavándonos con su propia sangre, el perdón de toda culpa.

{[}El bautismo que hoy se concederá a estos niños evoca el gesto de Cristo en el Jordán, porque el sacramento que la Iglesia confiere a sus nuevos hijos encuentra su fundamento precisamente en la ofrenda que Jesús entonces hizo de sí mismo. A través del bautismo recibido de Juan, nuestro Salvador quiso prefigurar el sacramento que un día, en la cruz, brotaría de su sacrificio como fuente de regeneración y renovación de vida: el bautismo, por el cual somos sepultados junto con él en la muerte, porque, como Cristo resucitó, también nosotros podemos caminar en una nueva vida (cf. \emph{Rm} 6, 4).{]}

{[}El agua del bautismo liberará a estos pequeños hoy del pecado original, los introducirá en la vida misteriosa escondida en Dios desde la eternidad y revelada en Cristo y en la Iglesia, los hará nuestros hermanos, miembros vivos del mismo cuerpo del Señor. Al sumergirlos en su misterio de muerte y resurrección, Cristo les dará un nuevo nacimiento y les dará su vida divina. Por eso, incluso a vuestros hijos hoy Dios Padre podrá decirles "tú eres mi Hijo", ya que es la vida de Cristo el Hijo de Dios la que les será conferida como don de gracia, una vida que no se originó en ellos por voluntad de la carne, ni por voluntad del hombre, sino sólo por Dios (cf. \emph{Jn} 1, 13).{]}

{[}Preparémonos todos con fe para celebrar este misterio con corazón agradecido hacia el Señor, y pidamos que renueve la gracia bautismal en nosotros, orando por la nueva vida cristiana de estos pequeños.{]}

\subsubsection{Homilía (1990): } 01A-09-Bautismo-1990ES

ADMINISTRACIÓN DEL SACRAMENTO DEL BAUTISMO A ALGUNOS RECIÉN NACIDOS

HOMILIA DE JUAN PABLO II

Fiesta del Bautismo de Jesús

Salón de Bendición - Domingo 7 de enero de 1990

"Este es mi Hijo amado, en quien me complazco" (\emph{Mt} 3, 17).

1. {[}Después de haber celebrado ayer, queridos hermanos y hermanas, la solemnidad de la Epifanía,{]} nos volvemos a encontrar hoy, {[}en este primer domingo del año nuevo,{]} para la fiesta del Bautismo de Jesús en el Jordán. Con este evento se inaugura la misión pública del Mesías, que se manifiesta a todos como "el Hijo amado del Padre" para ser escuchado, acogido, seguido. Jesús permanece con nosotros, en cada uno de nosotros, como nuestro Salvador. Su salvación nos llega a través de la fe y la gracia del Bautismo, el sacramento principal de la Iglesia.{]}

{[}En este contexto litúrgico, hoy es para mí, como cada año, una ocasión renovada de alegría poder acogerlos a ustedes, padres, padrinos y madrinas, por el Bautismo de estos niños (...). Y es también una circunstancia providencial que nos hace reflexionar sobre el sacramento del Bautismo, puerta por la que entramos de lleno en la comunidad eclesial.{]}

De hecho, la Iglesia es consciente de que su misión profética, sacerdotal y real tiene su origen en el Bautismo: de él toma fuerza la tarea de testificar y difundir a todos los hombres, alcanzada por su anuncio misionero, la salvación realizada por Cristo, "el amado Hijo del Padre". Con el Bautismo el cristiano acoge al Salvador y, en virtud del agua y del Espíritu, recorre el camino del amor de Dios, habiéndose renovado profundamente a imagen de Aquel que se manifestó en nuestra carne mortal.

2. Queridos hermanos y hermanas, no olvidéis nunca el don recibido y la alta misión que se os confió el día de vuestro Bautismo. No lo olvidéis especialmente vosotros, padres, padrinos y madrinas de estos niños, llamados por la bondad del Padre celestial a participar de la herencia inmortal de los redimidos.

En ti, estas tiernas criaturas, renovadas por la fuerza del Espíritu, pueden encontrar siempre testigos valientes y verdaderos "padres" en el itinerario de la vida cristiana. El agua del Bautismo los libera hoy de la esclavitud del pecado original y los introduce en la plenitud de la vida de Dios, manifestada en Jesucristo y comunicada por él a su Iglesia a través del misterio de su muerte y resurrección. Precisamente porque están insertados en la Iglesia, estos niños se convierten en miembros vivos del mismo cuerpo de Cristo, nuestros hermanos en la fe, coherederos con nosotros en la salvación y compartiendo desde este momento nuestra misión común en el mundo.

3. La liturgia de hoy, que presenta la teofonía del Jordán, nos muestra al Mesías como el que devuelve la vista a los ciegos y la libertad a los presos. En esta ocasión el Padre proclama solemnemente a Jesús como su "Hijo amado" y sanciona así el paso definitivo del Antiguo al Nuevo Testamento: del bautismo de Juan -\/- signo de penitencia y conversión a la espera del cumplimiento de las promesas mesiánicas -\/- al Bautismo de Jesús, en "Espíritu y fuego" (\emph{Lc} 3, 16). Por tanto, el Espíritu Santo es el autor del Bautismo de los cristianos: de nosotros mismos y también de estos niños. Él es quien hace resonar en nuestro espíritu la Palabra reveladora del Padre: "Este es mi Hijo amado, en quien me complazco".

Es él, el Espíritu Santo, quien abre los ojos del corazón a la Verdad, a toda la Verdad. Es él, el Espíritu Santo, quien impulsa nuestra vida por el camino renovado de la caridad. Es él, el Espíritu Santo, el don extraordinario e inconmensurable que el Padre da a cada uno de estos recién bautizados. Es él, el Espíritu Santo, quien nos reconcilia con la ternura del perdón divino y nos invade totalmente con el poder de la verdad y del amor.

4. Queridos hermanos y hermanas, la celebración de hoy, que es un derramamiento sobreabundante del Espíritu Santo, debe, por tanto, llenarnos de alegría espiritual y empujarnos a renovar nuestra vocación cristiana. En efecto, el Bautismo es vida para transmitir, luz para comunicar. Reafirmamos, por tanto, nuestra fiel adhesión a Jesús, único Salvador que hoy se manifiesta en la plenitud de su misión mesiánica. Acabamos de orar por esa intención en la Oración post-comunión: "Señor (...) imploramos de tu bondad, que, escuchando fielmente a tu Unigénito, de verdad nos llamemos y seamos hijos tuyos. Por Jesucristo, nuestro Señor".

Que María, Madre del Salvador, acompañe los pasos de nuestra vida cristiana en el camino de la Verdad y el Amor y obtenga el don de la perseverancia y la fidelidad para estos hijos, padres, padrinos y madrinas.

\subsubsection{Homilía (1993): } 01A-09-Bautismo-1993ES

VISITA PASTORAL Asís

SANTA MISA DE ASIS EN LA BASÍLICA SUPERIOR

HOMILIA DE JUAN PABLO II

Asís (Perugia) - Domingo 10 de enero de 1993

1. "Domine, murum odii everte, nationes dividentem, et vias concordiae fac hominibus planas" -\/- "Oh Señor, rompe las barreras del odio que dividen a las naciones, abre el camino a la armonía y la paz" (Martes de la tercera semana de Adviento, Invocaciones de Laudes). Queridos hermanos y hermanas, el clamor que hoy elevamos a Dios proviene de la liturgia del Adviento. La oración por la paz en Europa, y en particular en los Balcanes, se plantea en este período en las lenguas de los diferentes pueblos del continente europeo. Junto con los presidentes de los episcopados de toda Europa hemos implorado al Señor la paz. También les hemos pedido a nuestros hermanos cristianos, así como a los hijos de Israel y musulmanes, que oren por esto. Nos encontramos aquí en Asís siguiendo los pasos de San Francisco, que amaba eminentemente a Cristo, a los hombres ya toda la creación. Junto a él revivimos el misterio del Bautismo de Cristo en el Jordán, hecho clave en la misión mesiánica de Jesús de Nazaret.

2. ``En ese momento Jesús de Galilea fue al Jordán donde Juan para ser bautizado por él. Pero Juan quiso impedírselo, diciendo: "Necesito ser bautizado por ti y ¿vienes a mí?". Pero Jesús le dijo: "Déjalo por ahora, porque conviene que así cumplamos toda justicia"" (\emph{Mt} 3, 13-15). El bautismo de penitencia, conferido por Juan en el Jordán, es un signo de la justicia que el hombre espera de Dios, buscándolo con todo su corazón. También es un signo de paz, deseado por todo espíritu humano en todos los pueblos y naciones de la tierra. Y he aquí, encontramos a Jesús de Nazaret en la procesión de hombres que, animados por tal deseo, vienen a recibir el Bautismo de la penitencia, confesando sus pecados. Jesús no tiene pecado, pero sin embargo se inserta entre los pecadores. Este es un hecho muy revelador. "Este es mi Hijo amado, en quien tengo complacencia" (\emph{Mt} 3, 17). Precisamente el Hijo, infinita complacencia del Padre, se inserta entre los pecadores y junto a ellos recibe el Bautismo de la penitencia. "No he venido a llamar a justos, sino a pecadores" (\emph{Mt} 9, 13). Eventualmente, esta tarea lo llevará a la Cruz. Esto es lo que el mismo Juan expresó a orillas del Jordán, cuando dijo: "¡He aquí el Cordero de Dios, he aquí el que quita el pecado del mundo!" (\emph{Jn} 1, 29).

3. Hemos venido aquí hoy, asumiendo los grandes pecados de nuestro tiempo, de nuestro continente. {[}La guerra en curso en los Balcanes constituye una acumulación particular de pecados. Los humanos usan herramientas de destrucción para matar y exterminar a otros de su especie. ¡Qué terribles experiencias de guerras, especialmente en Europa, vivió el siglo XX! Fue un siglo marcado por el odio y el profundo desprecio por la humanidad, el odio y el desprecio que no abandonó ningún medio y método para aniquilar y exterminar al otro. El divino precepto del amor ha sido violado muchas veces y de diversas formas, hasta el punto de llegar incluso a preguntarse con temor si el europeo sería capaz de levantarse de ese abismo al que lo había empujado un loco deseo de poder, de dominación, a expensas de otros: de otros hombres, de otras naciones. Desafortunadamente, una experiencia tan trágica parece haber renacido de alguna manera en los últimos años; continúa extendiéndose directamente en la península de los Balcanes. Ésta es la razón por la que toda Europa se reúne en oración; por eso vinimos en peregrinación a Asís, para invocar a Dios por medio de Cristo: "Romper las barreras del odio ... abrir el camino a la armonía y la paz".{]}

4. Cristo ora junto con nosotros. Entró en la procesión de los pecadores no solo una vez, a orillas del Jordán, donde recibió el bautismo de penitencia de Juan. En cada siglo y en cada generación vuelve para mezclarse en esta procesión en los diversos lugares de la tierra. Cristo es, de hecho, el Redentor del mundo, a quien Dios "hizo pecado en nuestro favor, para que por él seamos justicia de Dios" (\emph{2 Co} 5, 21). De ahí nuestra firme convicción, iluminada por la fe, de que en la tierra atormentada de los hombres y naciones {[}de los pueblos balcánicos{]}, Cristo está presente entre todos los que sufren y sufren una absurda violación de los derechos humanos. Él, Cristo, es siempre testigo y defensor de los derechos humanos: tenía hambre, tenía sed, era un extraño, estaba desnudo, fue torturado, destrozado, violado, ultrajado en la dignidad humana ... (cf. \emph{Mt} 25, 31-46). En él los derechos de la persona no son solo palabras, sino vida: la vida que prevalece sobre la muerte y por la Cruz se afirma en la victoria de la Resurrección. Hoy rezamos junto con él y por él, porque estamos firmemente convencidos de que Él reza incesantemente con nosotros.

5. El Padre se complace en Él. Creemos, por tanto, que en Él y por Él el hombre, incluso el más ultrajado y también el más culpable, es abrazado por el único Amor, más fuerte que todo odio, pecado y maldad inhumana. Él ..., siervo de nuestra justificación, "no quebrará la caña cascada, no apagará la mecha humeante ... no fallará ni caerá, hasta que haya establecido el derecho en la tierra" (\emph{Is} 42, 3-4). El Padre le dice: "Yo te formé y te establecí como alianza de los pueblos y luz de las naciones ..." (\emph{Is} 42, 6). {[}Aquí: los pueblos, las naciones de esa tierra, envueltos en el horrendo conflicto que tiene lugar en los Balcanes, constituyen comunidades unidas por muchos lazos, inscritos no solo en la memoria del pasado, sino también en la esperanza común de un futuro mejor basado sobre los valores de la justicia y la paz. Cada una de estas naciones representa un bien particular, una confirmación de la riqueza multiforme dada por el Creador al hombre y a toda la humanidad. Además, cada nación tiene derecho a la autodeterminación como comunidad. Es un derecho que puede realizarse mediante la propia soberanía política o mediante una federación o confederación con otras naciones. ¿Se podría salvar de una manera u otra entre las naciones de la ex Yugoslavia? Es difícil descartarlo. Sin embargo, la guerra que ha estallado parece haber rechazado esa posibilidad. Y la guerra aún continúa. Hablando humanamente, puede parecer difícil ver el final. Y, sin embargo: "Sanabiles fecit Deus nationes ..." \emph{(Sb} 1, 14).{]}

6. Nos volvemos, por tanto, a Ti, Cristo, Hijo del Dios vivo, Verbo en quien el Padre se complace, y que quiso cumplir la misión de servidor de nuestra Redención. Tú eres la justificación del pecador, de todos los pecadores y malhechores de la historia humana. Tú eres la Alianza con los hombres, la luz de las naciones. Quédate con nosotros. Intercede por nosotros. Ora con nosotros pecadores, para que no prevalezcan las tinieblas. Perdona nuestros pecados, los terribles pecados de los hombres dominados por el odio, así como nosotros perdonamos ... Tratando de romper la espiral del mal ... Destruye tú mismo el odio que divide a las naciones. Allí, donde ahora abunda el pecado, haz que abunden la justicia y el amor, a lo que todo hombre, todo pueblo y nación es llamado en Ti, Príncipe de Paz. En esta hora difícil, nos dirigimos también a tu Santísima Madre, que es también Madre de todos los pueblos, Madre en particular de los pueblos de Europa, que a lo largo de los siglos le han levantado famosos santuarios, que son también hoy el destino de multitudes de peregrinos. En este momento pienso ante todo en el templo mariano más antiguo de Santa Maria Maggiore en Roma, en el "Muro Indestructible" en Ucrania y en esos lugares de devoción en Rusia, donde se venera la imagen de la Madre de Dios bajo la título de Nuestra Señora de Wladimir, de Kazán, de Smolensk. Mi pensamiento también va a los santuarios de Mariapocs en Hungría, de Marija Bistrica en Croacia, de Studenica en Serbia, al santuario nacional de la "Addolorata" en Eslovaquia, a la "Puerta de la Aurora" en Lituania, a los santuarios de Aglona en Letonia., de Marija Pomagaj en Eslovenia, de Czestochowa en Polonia, de Montserrat en España, de Lourdes en Francia, de Fátima en Portugal ... y muchos otros. A María Santísima, tu Madre y Madre nuestra, oh Cristo, toda Europa confía esta oración suya por la paz, utilizando todos los idiomas que se hablan en el Continente en la celebración de hoy.

7. ¡Que se rompan las barreras del odio! ¡Oh Dios de la paz! Endereza los caminos de los hombres, para que sepan volver a vivir juntos como prójimos, hermanos y hermanas, "Hijos del Padre en el Hijo Unigénito" (cf. \emph{Ef} 1, 4-5): en Cristo Jesús nuestra paz auténtica.

\subsubsection{Homilía (1996): } FIESTA DEL BAUTISMO DEL SEÑOR

HOMILIA DI JUAN PABLO II

Capilla Sixtina - Domingo 7 de enero de 1996

\emph{¡Queridos hermanos y hermanas!}

1. La fiesta de hoy del Bautismo de Jesús concluye el tiempo de Navidad, el tiempo litúrgico de las progresivas "manifestaciones" de Jesús: en su nacimiento en Belén, cuando aparece con rostro de Niño, "el primogénito de toda criatura", "imagen visible del Dios invisible" (cf. \emph{Col} 1-15); en la fiesta de la Epifanía, en la que se ofrece como don esperado y buscado por todos los pueblos de la tierra y como luz hacia la que confluye el camino interior de la historia; y finalmente en la celebración de hoy, en la que, en las aguas del Jordán, se solidariza con el hombre, "inclina su cabeza inmaculada ante el Precursor; y, bautizado, libera a la humanidad de la esclavitud, amante de los hombres" (Liturgia bizantina: \emph{EE}, 3038). Así es consagrado Siervo "con unción sacerdotal, profética y real, para que los hombres reconozcan en él al Mesías, enviado para llevar la buena nueva a los pobres" ("Praefatio" in festo Baptismatis Domini).

Estas son las etapas de una manifestación de Cristo que se hace cada vez más interior y profunda.

2. {[}Hoy, de manera muy singular, se traslada a la celebración del Bautismo de estos 20 niños (...). Este sacramento renueva en ellos el don misterioso de la gracia divina, que imprime un \emph{sello indeleble en sus almas}, dando lugar a un nuevo nacimiento: "... de Dios fueron engendrados ... A quienes lo acogieron, les dio poder de convertirse en hijos de Dios ... " (\emph{Jn} 1, 12-13).

La gracia santificante, que elimina el pecado original, infunde en ellos con el Bautismo las virtudes teologales y los dones del Espíritu Santo; \emph{también los inserta en el Cuerpo místico de Cristo}, que es la Iglesia.

¡Qué grande es el Bautismo, el primero de los sacramentos y el más necesario para la salvación! Es \emph{la piedra angular de la vida cristiana}, el umbral de todos los demás sacramentos y de la regeneración a esa vida inmortal de la que también nos hablan los maravillosos frescos de Miguel Ángel, que podemos admirar en esta sugerente Capilla Sixtina.

Es a partir de esta conciencia que la práctica de bautizar a los niños se desarrolló desde el comienzo mismo de su existencia terrenal. Evidentemente, esta práctica presupone que los años siguientes, especialmente los de la infancia, la niñez y la juventud, se configuran entonces como un auténtico \emph{catecumenado}, un itinerario de iniciación a la vida cristiana y de progresiva inserción en la comunidad de los creyentes.{]}

3. {[}Queridos padres, queridos padrinos y madrinas, estos pequeños, a los que hoy se administra el Bautismo, deberán comprender, repensar y apreciar \emph{el don inestimable} del sacramento recibido. Depende de ustedes unirse a ellos, que ahora no saben y no entienden, y ser \emph{sus primeros maestros} en la enseñanza de las verdades cristianas.

\emph{¡Escuchen a estos niños!} La fe viene de la escucha de la Palabra de Dios, y la escucha es una actitud que, como cualquier otra, se aprende ante todo en la familia. Quien ha sido escuchado, sabe escuchar, así como quien ha sido amado puede amar más fácilmente.

Ayuden a estos niños \emph{a crecer fieles al Evangelio}, dispuestos a amar a Dios y a sus hermanos. Guíenlos, con el ejemplo y la palabra, por el camino de la santidad cristiana.

Su misión como padres no se limita únicamente al don de la vida física. Estás llamado a generar a tus hijos también en la fe y en la dimensión del espíritu.

Imitad a la Sagrada Familia de Nazaret e invocad la protección constante de la Santísima Virgen y de San José en vuestros hogares.

En esta solemne ocasión, frente a un número tan significativo de niños que por el bautismo están a punto de convertirse en hijos adoptivos de Dios, parece que volvemos a escuchar las palabras del Padre celestial recién escuchadas en el Evangelio: Cada uno de estos niños y niñas ``es mi hijo amado, en quien me complazco''.

Y para ustedes, padres, padrinos, madrinas, adultos y cristianos conscientes de nuestra vocación, resuena la invitación: "¡Escuchadlo!" (\emph{Mc} 9, 7).

Que María, Madre de Dios y de la Iglesia y todos los santos a los que invocaremos en breve, os ayuden en tan exigente misión.{]}

