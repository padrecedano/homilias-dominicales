\chapter{I-Adviento}

\subsubsection{Homilía (1983): }

Visita Pastoral a la Parroquia Romana de San Felipe Neri.

27 de noviembre de 1983.

1. "Comportaos reconociendo el momento en que vivís, pues ya es hora de
despertaros del sueño" (\emph{Rom} 13, 11).

Con estas palabras, queridos hermanos y hermanas, la liturgia de hoy se
dirige a cada uno de nosotros, enseñándonos a acoger la llamada que nos
llega desde el comienzo del Adviento. Despertar del sueño significa
abrir el corazón a esa realidad divina que está ligada al tiempo humano.
Por eso se dice: "la salvación está más cerca".

El Adviento es como una primera dimensión de esta unión de la Realidad
divina al tiempo humano. Este vínculo se refleja en el año litúrgico: el
domingo primero de Adviento es al mismo tiempo el comienzo del nuevo año
litúrgico.

{[}2. Al mismo tiempo comenzamos el Año Santo de la Redención. Este
extraordinario Jubileo de la Redención tiene un carácter específico de
"adviento": nos prepara para el tercer milenio después de Cristo. De ahí
la particular elocuencia del Adviento de este año, que debe expresar esa
actitud de la Iglesia, de la que ya hablé en la Bula de Indicación (Juan
Pablo II,
\href{http://www.vatican.va/content/john-paul-ii/it/jubilee/documents/hf_jp-ii_doc_19830106_bolla-redenzione.html}{\emph{\emph{Aperite
			portas Redemptori}}}, 7), por la que "se siente especialmente
comprometida con la fidelidad a los dones divinos, que tienen su origen
en la redención de Cristo, y por medio de los cuales el Espíritu Santo
la guía a su desarrollo y renovación, para que pueda siempre convertirse
en una esposa más digna de su Señor. Para ello confía en el Espíritu
Santo y quiere asociarse a su acción misteriosa de Esposa que invoca la
venida de Cristo" (cf. \emph{Ap} 22, 17).

Este carácter particular de "adviento" propio del presente Año Santo,
debe ser vivido por la Iglesia "con los mismos sentimientos con los que
la Virgen María esperaba el nacimiento del Señor en la humildad de
nuestra naturaleza humana". Así como María precedió a la Iglesia en la
fe y en el amor en los albores de la era de la Redención, hoy la precede
mientras, en este Jubileo, avanza hacia el nuevo milenio de la
Redención" (Juan Pablo II, \emph{Aperite portas Redemptori}, 9) .{]}

3. "Reconocer el momento en que vivimos": ¿qué significa? "Vamos alegres
al encuentro del Señor".

El Adviento es la perspectiva gozosa de "ir a la casa del Señor" (cf.
\emph{Sal} 121, 1): de llegar al final de esta gran "peregrinación" que
debe ser la vida terrena. El hombre está llamado a habitar en la "casa
del Señor". Allí está su verdadero "hogar". {[}La peregrinación del Año
Santo es una figura de nuestro camino hacia la casa del Padre,{]} y el
Adviento nos estimula a acelerar este camino con esperanza.

El Adviento aguarda el "día del Señor", es decir, la "hora de la
verdad". Es la expectativa de ese día en que "Él será juez entre las
naciones y árbitro entre todos los pueblos" (\emph{Is} 2, 4). Esta
plenitud de verdad será el principio y fundamento de la paz definitiva y
universal, que es el objeto de la esperanza de todos los hombres de
buena voluntad.

El Adviento es una confirmación del camino eterno del hombre hacia Dios;
es un nuevo comienzo, cada año, de este camino: ¡la vida del hombre no
es un camino infranqueable, sino un camino que conduce al encuentro con
el Señor! Hay también en esta invocación del primer domingo de Adviento
casi un anticipo de esos caminos que en la noche de Belén conducirán a
los pastores y a los Reyes Magos de Oriente hacia Jesús el recién
nacido.

4. "Reconocer el momento en que vivimos": ¿qué significa? "Revestirnos
del Señor Jesucristo" (\emph{Rm} 13, 14):

- el camino del Adviento conduce al interior del hombre, que de diversas
formas está cargado de pecado, como atestigua la \textbf{segunda
	lectura};

- el encuentro antes mencionado no sólo se realiza "desde fuera", sino
también "desde dentro", y consiste en una transformación tal del
interior del hombre, que corresponde a la santidad de Aquel con quien se
encuentra: en eso consiste precisamente "vestirse del Señor Jesucristo";

- el sentido "histórico" del Adviento está penetrado por el sentido
"espiritual". De hecho, el Adviento no quiere ser sólo el recuerdo del
período histórico que precedió al nacimiento del Salvador, aunque, así
entendido, ya tiene en sí mismo un altísimo significado espiritual. Sin
embargo, más allá de eso, y de manera más profunda, el Adviento quiere
recordarnos que toda la historia del hombre y de cada uno de nosotros
debe entenderse como un gran "Adviento", como una expectativa, momento a
momento, de la venida del Señor, para que nos encuentre preparados y
vigilantes para poder acogerlo dignamente.

5. "Reconocer el momento en que vivimos": significa: "Velad \ldots{}
porque no sabéis en qué día vendrá vuestro Señor" (\emph{Mt} 24, 42):

- la unión de Dios, de la realidad divina, con el tiempo humano, por un
lado, reafirma la limitación de este tiempo, que tiene fin y, por otra
parte, abre este tiempo a la eternidad de Dios y a las "realidades
últimas" relacionadas con él;

- el Adviento tiene un significado "escatológico" ya que recuerda
nuestros pensamientos y nuestras intenciones hacia realidades futuras.
Nos recuerda el fin último de nuestro camino, y nos estimula a
involucrarnos en las realidades terrenas sin dejarnos sumergir en ellas,
sino al contrario, guiándolas hacia las celestiales; nos exhorta a
prepararnos bien para esto último, para que la venida del Señor no nos
encuentre desprevenidos y mal dispuestos;

- "Vigilidad": el espíritu del hombre "despertado" a la realidad divina,
atraído por ella a sus destinos eternos en Dios, debe animar toda
temporalidad con una nueva conciencia.

6. Queridos hermanos y hermanas {[}de la parroquia de San Felipe
Neri{]}, mi afectuoso saludo va para todos vosotros, ya que agradezco al
Señor por haberme concedido este encuentro. (...)

{[}...{]}

Una comunidad parroquial unida y ferviente puede jugar, con la fuerza
del Espíritu Santo, un papel esencial en la reducción de la distancia
entre el modelo evangélico que propone al mundo y las condiciones reales
del mundo mismo, siempre en cierta medida refractario, mientras estemos
aquí abajo, a la llamada evangélica, a la conversión y a la penitencia.
Este hecho, sin embargo, lejos de debilitar el testimonio que se da al
mundo, debe fortalecerlo cada vez más, en la muy firme convicción de que
el mundo, a pesar de todo, tiene una necesidad absoluta de Jesús
crucificado y resucitado. El poder de su gracia, especialmente a través
del carisma de los laicos cristianos, puede y debe penetrar y animar
evangélicamente todos los ambientes seculares de la familia y el
trabajo, la escuela, la sociedad y la cultura.

7. En el primer domingo de Adviento (...) espero sinceramente que esta
{[}celebración{]} nos permita abrir más los ojos del alma a la realidad
divina y, por así decirlo, volver a despertar a ella. Que nos permita
transformarnos interiormente y que al mismo tiempo nuestra humanidad se
revista del Señor Jesucristo de forma cada vez más madura.

Con nueva alegría, avancemos hacia el encuentro con el Señor que debe
venir, como cada año, en la solemnidad de la Navidad; hacia el Señor con
quien también debemos encontrarnos al final de nuestros caminos
terrenales. De hecho, el Adviento nos recuerda, cada año, que la vida
humana no es un camino infranqueable hacia Dios, sino un verdadero
camino que Él mismo ha hecho suyo por la Palabra divina.

\subsubsection{Homilía (1986): }

Santa Misa en el Parque Victoria, Adelaide (Australia), 30 de noviembre
de 1986.

\emph{"¡Qué alegría cuando me dijeron: `Vamos a la casa del Señor'!''}
(\emph{Sal} 122 (121), 1).

\emph{Amados hermanos y hermanas en Cristo.}

1. Con estas palabras toda la Iglesia proclama la alegría del Adviento.
Hoy es el primer domingo de Adviento. Nos acercamos a la noche en que
los pastores, en los campos alrededor de Belén, sintieron la alegría de
ser llamados por los ángeles para ir a ver al Señor: "Vamos a Belén,
veamos este acontecimiento que el Señor nos ha dado a conocer"
(\emph{Lc} 2, 15). Sí, incluso hoy, aquí (...), la Iglesia nos recuerda
que el Señor está cerca (cf. \emph{Fil} 4, 5). Y como los pastores en
aquella noche espléndida en Belén, también nosotros decimos: ``Iremos a
la casa del Señor''.

El Adviento es el tiempo de preparación para la Navidad, para la venida
del Salvador. Nos llama a ``ir con gozo a la casa del Señor... a alabar
allí el nombre del Señor'' (\emph{Sal} 121, 4). Alabar el nombre del
Padre Altísimo, del Hijo y del Espíritu Santo: esta es la primera
intención de nuestra celebración eucarística.

2. ``¡Qué alegría cuando me dijeron: `vamos a la casa del Señor'!''. Con
la misma alegría que nos comunica la liturgia de Adviento os saludo a
todos vosotros aquí reunidos (...) caminando con vosotros hacia la casa
del Señor, por los caminos de ese Adviento que es la historia del
hombre: el Adviento en el que toda la familia humana y todos los creados
esperan la segunda venida de nuestro salvador Jesucristo.

{[}Solo unos pocos años nos separan del final del segundo milenio y el
comienzo del tercer milenio de la era cristiana.{]} Este es un tiempo de
gracia para la Iglesia. Es un tiempo en el que los seguidores de
Jesucristo, en medio de las profundas transformaciones que cambian la
cultura y la sociedad, debemos volver a dedicarnos a la vida cristiana.
Es un momento en el que el mensaje del Evangelio debe ser proclamado a
los hombres y mujeres de esta época con la fuerza de un nuevo
Pentecostés. Es un tiempo en el que el mismo Espíritu de verdad habla
con claras palabras de vida a la familia humana.

3. En la celebración eucarística de este primer domingo de Adviento,
oremos por la realización del proyecto del Padre para la familia humana:
"En ese mundo nuevo donde se nos revelará la plenitud de la paz, reúne
también a los hombres de toda raza y lengua, en el banquete de la unidad
eterna, en los cielos y en la tierra nueva, donde brille la plenitud de
tu paz'' (II \emph{Plegaria Eucarística de Reconciliación}).

En otras palabras, oremos para que se haga realidad la visión del
profeta Isaías, como decía la \textbf{primera lectura}: ``Todas las
naciones confluirán al monte del Señor ... para que nos muestre sus
caminos y caminemos por sus sendas'' (\emph{Is} 2, 2-3). El deseo de ese
tiempo de gracia y paz está profundamente arraigado en nuestros
corazones. ¿Quién no anhela que llegue ese tiempo final cuando "un
pueblo ya no alzará la espada contra otro pueblo, ya no practicará el
arte de la guerra"? (\emph{Is} 2, 4). De hecho, existe un tiempo de
Adviento que es universal y que dura tanto como la historia humana. Hoy
meditamos en la visión de \textbf{Isaías} sobre un número incalculable
de personas que marchan hacia el monte del Señor: el pueblo de Dios de
todas las épocas y de todos los lugares que se reúnen en unión con él y
en la unidad entre ellos mismo en la Iglesia. Y deberíamos reflexionar
sobre cómo esta visión se materializa en la realidad concreta de la vida
{[}actual{]}, en la historia y en la cultura (...). Esta asamblea
eucarística en sí misma es un símbolo de la visión del profeta. Sois
personas aquí reunidas de "toda raza, lengua y forma de vida", hechos
uno en Jesucristo y en su Iglesia.

{[}...{]}

5. La \textbf{palabra de Dios} nos ha llamado a estar atentos y
vigilantes, a revestirnos de las armas de Jesucristo: "Ha llegado el
momento de despertar del sueño ... La noche está avanzada, el día está
cerca: dejemos, pues, las obras de las tinieblas y pongámonos las armas
de la luz. Andemos como en pleno día, con dignidad. Nada de comilonas y
borracheras, nada de lujuria y desenfreno, nada de riñas y envidias.
Revestíos más bien del Señor Jesucristo." (\emph{Rom} 13, 11-14).

Cualquier expresión de hostilidad hacia los demás levanta un muro de
tensión entre las personas y revela un corazón de piedra. Cualquier acto
de discriminación es un acto de injusticia y una violación de la
dignidad de la persona. Siempre que somos intolerantes, cerramos los
ojos a la imagen de Dios que está en la otra persona. Hoy, cuando no
reconocemos las necesidades de justicia en el mundo, no captamos el
sentido de solidaridad universal. Pero cuando hablamos con palabras
amables, cuando nos respetamos y nos honramos mutuamente, cuando
mostramos una verdadera amistad, cuando ofrecemos hospitalidad, cuando
nos esforzamos por comprender las diferencias entre los pueblos, nos
convertimos en el signo vivo de que la visión de Isaías se ha hecho
realidad, que el reino de Dios ha llegado entre nosotros, que el
advenimiento universal de la historia avanza hacia su cumplimiento.

6. La Iglesia hoy nos invita a todos y cada uno de nosotros a emprender
con gusto y alegría el camino que Dios ha preparado para todo el género
humano. El \textbf{profeta Isaías} habla del camino que sube al monte
del Señor, al templo del Dios de Jacob (cf. \emph{Is} 2, 3). Parte de
este "ascenso" es la vocación del hombre a buscar una humanidad plena y
auténtica, a perfeccionar y desarrollar sus propias cualidades
espirituales y físicas en la lucha por dominar el mundo, a través del
progreso del conocimiento y mediante su propio esfuerzo. Esto es lo que
hace la familia humana a través del progreso cultural (cf.
\href{http://www.vatican.va/archive/hist_councils/ii_vatican_council/documents/vat-ii_const_19651207_gaudium-et-spes_it.html}{\emph{\emph{Gaudium
			et Spes}}}, 53).

Los hombres y mujeres de hoy saben claramente que, hoy más que nunca,
están llamados a construir su propio destino en este mundo. Los medios
de que disponen son cada vez mayores: un mejor conocimiento de la tierra
y sus secretos, un mejor conocimiento de la persona humana y de la
actividad humana; una mejor comprensión del curso de la historia y la
organización social; y el mundo de las comunicaciones, que brinda a cada
vez más personas la oportunidad de participar en el progreso moderno. Un
mundo más humano está luchando por nacer. Y, sin embargo, cada vez las
mayores esperanzas van acompañadas de inquietantes contradicciones. En
lo que respecta al respeto de los derechos humanos fundamentales, las
últimas décadas han sido testigos de grandes avances y de una creciente
conciencia de la justicia de esta causa. Sin embargo, no podemos pasar
por alto el hecho de que nuestro mundo todavía ofrece demasiados
ejemplos de gran injusticia y opresión. Donde hay un gran bien por
alcanzar, se necesita una gran madurez moral y un gran sentido de la
justicia. Sin la visión de la sublime dignidad de la persona humana
--- dignidad fundada en la relación única y personal con el Creador
y Redentor, dignidad ligada a la naturaleza trascendente, al origen y
destino del hombre --- el progreso no tendrá un rumbo seguro.
Jesucristo, camino, verdad y vida, nos revela el verdadero sentido de la
historia. Nos revela el plan de Dios para la humanidad. Jesús habla de
nuestra libertad y nos llama a promover el verdadero progreso humano,
dándonos su ley de amor y servicio; "Este es mi mandamiento: amaos los
unos a los otros como yo os he amado" (\emph{Jn} 15, 12). El Evangelio
purifica y fortalece toda cultura, para que pueda ayudar al hombre "a
subir al monte del Señor ... para que nos muestre sus caminos y
recorramos sus sendas" (\emph{Is} 2, 3).

7. El llamado de Jesús es claro. Él dice: "Velad". Y de nuevo: "Estad
preparados, porque a la hora que no imaginéis, vendrá el Hijo del
Hombre" (\emph{Mt} 24, 42, 44). Así exhorta a todos sus seguidores a
trabajar por la meta que el Padre se ha marcado: el reino de la
justicia, la verdad y la paz. Por lo tanto, insta a los fieles (...) a
encontrar un remedio cuando las injusticias puedan dañar la vida de su
nación y a garantizar que un nuevo espíritu de reconciliación anime toda
la vida nacional. Jesús nos dice que seremos juzgados por la forma en
que respondamos a su presencia en los hambrientos, desnudos, enfermos y
presos (cf. \emph{Mt} 25, 35-36).

Queridos hermanos y hermanas: estáis llamados a participar con Dios en
la construcción de su reino en el corazón de todos (...), corazones de
carne y no de piedra. {[}En este día estamos invitados{]} a ver nuestra
historia en el contexto del amor eterno de Dios por toda la familia
humana, que se manifiesta en la misión salvífica de Jesucristo. Es una
historia que aún está evolucionando. Y presenta muchos desafíos (...).
Este también es el Adviento, lleno de expectación, que la Iglesia
celebra en este período. Nosotros, pueblo peregrino de Dios, caminamos
siguiendo a Jesús, que es el camino al Padre. Caminamos con la certeza
de que su verdad nos hará libres y que nuestra fuerza proviene de sus
palabras y sus sacramentos.

8. Con la mirada puesta en Aquel que ha de venir, miremos, pues, a todos
(...) e invoquemos la bendición del Salmo: "Haya paz dentro de tus
muros, seguridad en tus palacios \ldots{} Por mis hermanos y compañeros,
voy a decir: \textquote{La paz contigo}. Por la casa
del Señor, nuestro Dios, te deseo todo bien." (\emph{Sal} 121, 7-9).

{[}Para ti: Adelaide! Para ti: ¡Australia! Para ti: ¡mundo entero!{]}

Y seguimos mirándole a Él, el que ha de venir, el "Príncipe de la paz"
(cf. \emph{Is} 9, 6). El \textbf{profeta} dice de él:

"Juzgará entre las naciones,

será árbitro de pueblos numerosos.

De las espadas forjarán arados,

de las lanzas, podaderas.

No alzará la espada pueblo contra pueblo,

no se adiestrarán para la guerra.

Casa de Jacob, venid;

caminemos a la luz del Señor."

(\emph{Is} 2, 4-5).

Esta es la luz del Adviento. Es la luz del Adviento que se despliega
ante la familia humana, hasta que el Señor vuelva en gloria: el Adviento
de la responsabilidad del hombre por la vida y por el mundo que el
Creador ha puesto en sus manos. La luz es la luz del que vendrá, el
príncipe de paz, es la luz de Cristo. ¡Que la luz de Cristo brille sobre
{[}nuestra tierra{]} (...)! Que la luz de Cristo brille sobre cada uno
de vosotros. Amén.


\subsubsection{Homilía (1989): }
Visita Pastoral a la Parroquia de Santo Tomás Apóstol en Castel Fusano.

3 de diciembre de 1989.

1. "Vayamos gozosos al encuentro del Señor".

Estas palabras del \textbf{salmo responsorial}, que hemos repetido
juntos, pueden considerarse con razón el programa de la Iglesia al
comienzo del año litúrgico; especialmente al comienzo del Adviento, que
constituye la primera etapa importante del año litúrgico.

Entramos conscientemente en un tiempo "propicio" de salvación: la
comunidad cristiana, de hecho, mientras se prepara para conmemorar la
primera venida de Cristo "en la humildad de nuestra naturaleza humana"
(\emph{Prefacio del Tiempo de Adviento}), es llamada a orientarse, con
esperanza confiada y expectación activa, hacia la venida definitiva del
Señor "en el esplendor de su gloria" (\emph{Prefacio del Tiempo de
	Adviento}). Todo ello en dócil apertura de fe a la Palabra de Dios, que
ilumina nuestro camino y con una participación activa en los
acontecimientos sacramentales, que nos insertan en el misterio de
Cristo, "hasta que él venga" (cf. \emph{1 Co} 4, 5).

En el momento en que la Iglesia emprende el itinerario salvífico del año
litúrgico, la Palabra de Dios, recién escuchada, pone inmediatamente
ante nosotros la meta hacia la que nos dirige el Espíritu: la nueva
Jerusalén, símbolo de la comunión plena y definitiva a la que Dios
invita y admite a todos aquellos que dicen el "sí" de la fe a Cristo,
"maestro de la verdad y fuente de reconciliación" (\emph{Oración
	colecta}), y se hacen heraldos y testigos entre los hermanos, para que
el Dios de la paz sea en todo en todos cuando "venga el Hijo del Hombre"
(\emph{Evangelio}).

2. Entonces entendemos las palabras del \textbf{profeta}: "Casa de
Jacob, ven, caminemos a la luz del Señor" (\emph{Is} 2, 5).

Esta invitación resuena hoy con acentos de especial relevancia y
urgencia para {[}todos nosotros{]} (...) Todos debemos emprender un
camino de conversión, superando la tentación de la inercia, la
desconfianza y la pasividad.

Muchos, de hecho, que también profesan ser cristianos, viven en una
especie de letargo y mediocridad. A menudo, su vida moral contrasta con
la fe, que también dicen tener; no pocos limitan su pertenencia a Cristo
y a la Iglesia a una práctica religiosa ocasional, compuesta sólo por
citas religiosas ocasionales; se alejan de las responsabilidades
específicas, contentándose con delegar en los demás la misión
evangelizadora propia de cada miembro activo de la comunidad eclesial.

3. A todos ellos quiero repetir el llamamiento que \textbf{san Pablo}
dirigió a los primeros fieles de la Iglesia de Roma, y ​​que la liturgia
nos ha vuelto a proponer hoy: "Hermanos, es hora de despertarnos del
sueño. , porque nuestra salvación está más cerca que cuando empezamos a
creer. La noche está avanzada, el día está cerca. Desechemos las obras
de las tinieblas y vistámonos las armas de la luz'' (\emph{Rom} 13,
11-12).

Ante la situación de indiferencia e imprevisibilidad que describe el
Evangelio y que se refleja en la mentalidad y las costumbres de hoy, la
Iglesia (...) {[}en su camino hacia el final del segundo milenio de la
era cristiana{]}, no puede ni debe permanecer inerte y pasiva: está
impulsada por la voluntad misma de Dios, reconocida y aceptada en unión
con su propio Obispo, a tomar la decisión fundamental de renovarse en la
vocación y misión que le ha confiado la Providencia.

El Vaticano II volvió a proponer la verdad sobre Dios y el hombre con
nueva luz. Aceptar esta luz, dejarse penetrar por ella más íntimamente,
anunciarla a todos es un deber de todo cristiano, para que la salvación
en Cristo Jesús, que se ofrece a todos, esté más cerca de cada uno. Es
un viaje para hacer "juntos". Nadie puede ignorarlo o rehuirlo (...). El
mundo, y en particular los hombres que viven en esta ciudad, no podrán
ver y recibir la luz de Cristo si sus discípulos son opacos, tibios y
pasivos.

{[}...{]}

5. Queridos hermanos y hermanas, quisiera hacer una última reflexión con
ustedes, que se me ha sugerido desde el comienzo del año litúrgico.

Los obispos italianos han afirmado en varias ocasiones que el año
litúrgico ``constituye el gran camino de fe del pueblo de Dios: toda la
comunidad, especialmente en tiempos de gran fuerza, está llamada a
redescubrir, celebrar y vivir el don de la salvación. A través de la
pedagogía de los ritos y de la oración, todos somos conducidos juntos a
la experiencia del misterio pascual de Cristo, que tiene su centro en la
Eucaristía'' (Conferencia Episcopal Italiana, \emph{Eucaristía, comunión
	y comunidad}, 89; cf. etiam Eiusdemn \emph{Evangelización y
	sacramentos}, 85).

{[}Estamos llamados a establecer{]} en nuestras comunidades itinerarios
educativos serios y continuos, en los que se fundan juntos la escucha de
la Palabra de Dios, la celebración de los santos misterios y el
testimonio, el servicio de la caridad y la promoción del hombre,
superando así la fragmentación y el ocasionalismo por un lado y, por
otro, el impulso de avanzar por caminos paralelos o divergentes.

El año litúrgico es uno de estos itinerarios, de hecho es el
privilegiado de la Iglesia, no solo porque es el más adecuado para todas
las edades y diferentes categorías de personas, sino sobre todo porque
es el más completo, si se vive con autenticidad y es valorado en todas
las posibilidades que ofrecen sus diferentes tiempos y momentos y por la
riqueza de los signos litúrgico\emph{-}sacramentales.

La creatividad pastoral, en fidelidad a la Tradición genuina de la
Iglesia, podrá promover, especialmente para jóvenes y adultos, otros
itinerarios de fe incluidos en el año litúrgico, o en todo caso en
sintonía con él, tanto con motivo de la celebración de los sacramentos
y, más en general, por una vida cristiana más madura y una fe más
consciente y activa.

{[}Confío estos compromisos a ese gran testigo de la fe que fue santo
Tomás Apóstol, a quien está dedicada vuestra comunidad, para que toda la
Iglesia (...) profese su fe en Cristo resucitado y coseche los frutos de
la salvación, para sí misma y para los hombres que viven en la ciudad.

``Casa de Jacob, ven. . . "; {[}parroquia de Santo Tomás Apóstol en
Castel Fusano{]}, ``ven, caminemos en la luz del Señor''.

¡Amén!

\subsubsection{Homilía (1992): }

VISITA A LA PARROQUIA DE SAN GERARDO MAIELLA

\emph{\textbf{HOMILIA DE JUAN PABLO II}}

\emph{Domingo 29 de noviembre de 1992}

1. \emph{"Vayamos gozosos al encuentro del Señor"} (Salmo responsorial).

¡Queridos hermanos y hermanas (...)! Hoy, primer domingo de Adviento,
comienza un nuevo año litúrgico durante el cual la Iglesia recorre y
revive espiritualmente las etapas del misterio cristiano.

Este plan divino abarca toda la historia de la humanidad, desde los
albores de la creación hasta el día final, cuando todas las cosas serán
recapituladas en Cristo (cf. \emph{Ef} 1, 10) y habrá nuevos cielos y
nueva tierra (cf. \emph{2 Pt.} 3, 13). El centro de este proyecto está
en el misterio de la encarnación del Hijo de Dios.

En un momento preciso, por obra del Espíritu Santo, el Verbo "se hizo
carne" en el seno virginal de María y "vino a habitar en nosotros". (cf.
\emph{Jn} 1, 12), mostrando la bondad y humanidad de Dios para con los
hombres. El Señor, de hecho, no sólo creó al hombre, sino que lo ama con
tanta intensidad que lo acoge desde dentro de su propia familia,
destinándolo a una gloria sin fin.

De hecho, sostenidos por una certeza tan consoladora vamos con alegría a
su encuentro, como nos invita a hacer el \textbf{Salmo Responsorial}.

2. Vamos a encontrarlo en el misterio de la Navidad. Este es el primer
sentido del Adviento. Nos conmueve el recuerdo de María y José, que
suben de Nazaret de Galilea a Belén de Judea para el censo, y se ven
obligados a refugiarse en un lugar destinado a los animales, "porque no
había lugar para ellos en la posada" (\emph{Lc} 2, 7).

El Hijo de Dios viene a la luz en la pobreza total: verdadero Dios
Salvador, anunciado por los ángeles a los pastores, y verdadero hombre,
envuelto en pañales y colocado en un pesebre.

¡Qué sentimientos de ternura, amor y gratitud despierta este
extraordinario evento! Sin embargo, también tiene la fuerza de sacudir
nuestra conciencia invitándonos a despertar del sueño de la indiferencia
y de la habitud.

"Hermanos - nos exhorta el \textbf{apóstol Pablo} - ya es hora de
despertar del sueño" (\emph{Rm} 13, 11), Dios nos ha amado hasta darnos
a su Hijo único. ¿Acaso un don tan grande no nos obligua a reflexionar y
a responderle con la generosidad adecuada? ¿No nos empuja a abandonar
las tinieblas del pecado para abrir el espíritu a la luz de la gracia
divina? Esto es precisamente a lo que nos invitan las lecturas de la
liturgia de hoy.

3. "Venid, subamos al monte del Señor" (\emph{Is} 2, 3).

El texto, tomado del libro del \textbf{profeta Isaías}, se interpreta
comúnmente como un anuncio mesiánico. Para el pueblo de Israel forzado
al exilio, el profeta predice la reconstrucción del Templo de Jerusalén.
Pero sus palabras van más allá de la historia del pueblo judío. Con
imágenes como la del cerro más alto de todas las montañas y con el
pronóstico de la venida de innumerables naciones al templo del Dios de
Jacob, se indica una nueva realidad espiritual, la del pueblo de los
redimidos guiados por el Mesías prometido; y la de una nueva Alianza,
que transforma profundamente la vida de los hombres, según los ideales
de paz y fraternidad, convirtiendo las espadas en arado y las lanzas en
podaderas.

4. ¡Queridos hermanos y hermanas! Vuestra comunidad parroquial, una de
las más jóvenes de nuestra Iglesia local, está animada por estos ideales
de paz y fraternidad, santidad y evangelización. (...)

(...) Gracias a una catequesis capilar, habéis intentado ayudar a los
habitantes de todo el barrio a no retirarse al individualismo, sino a
crecer como comunidad cristiana solidaria, siguiendo el ejemplo de la
Iglesia de los Apóstoles y de las primeras generaciones de creyentes, a
través de una atenta escucha de la Palabra, participación en la vida
litúrgica de la Comunidad y un intenso esfuerzo de compartir y acogida
recíproca. De esta manera, se han desarrollado experiencias comunitarias
significativas y formas efectivas de catequesis para adultos, como las
Comunidades Neocatecumenales. Muchos jóvenes pertenecen se reúnen
también en la Acción Católica y en otros movimientos apostólicos, Grupos
de matrimonios cristianos siguen un itinerario formativo común
estructurado en encuentros periódicos (...) La parroquia tampoco carece
de la sensibilidad misionera que brotó del hermanamiento con una misión
africana en Chad.

5. Por todo ello (...) deseo manifestar mi viva satisfacción,
reconociendo el compromiso y la generosidad que os animan.

Perseverad, queridos hermanos y hermanas, en el esfuerzo realizado.
\emph{Daréis, pues, vida a una nueva evangelización} (...). Todavía hay
muchas personas que no conocen el evangelio adecuadamente y esperan el
testimonio constante de nuestra vida y la proclamación gozosa de nuestra
fe en Cristo.

A vosotros os encomiendo una misión tan exigente: a las familias, a los
adultos, a los niños, a los ancianos y especialmente a los jóvenes, y os
aseguro a cada uno de vosotros el apoyo de mi cordial oración.

Este inicio del tiempo de Adviento constituye una ocasión propicia para
intensificar el ritmo de nuestra vida cristiana.

6. "Vuestro Señor vendrá" (\emph{Mt} 24, 42).

El pasaje del \textbf{Evangelio de Mateo} abarca una parte del discurso
de Jesús sobre los últimos acontecimientos, que por eso se llama
discurso escatológico.

Jesús anuncia su segunda venida, al final de los tiempos, y nos exhorta
a estar alerta y preparados para encontrarnos con Él. Este es el segundo
significado del Adviento.

Por las palabras de Jesús, contenidas en este y otros textos, sabemos
con certeza que el mundo presente está destinado a terminar, que la
historia humana terminará, que para cada uno habrá un juicio, seguido de
una recompensa o castigo. A la luz de todo esto es importante escuchar
la invitación a velar "porque no sabes en qué día vendrá tu Señor".

7. ¡``Velad, pues''!

La vigilancia evangélica es condición para un buen uso de la vida.

Qué fácil es desperdiciar los dones divinos, distanciarse de Dios con
pensamientos y comportamientos, olvidar que la vida pasa.

Las cosas temporales son frágiles y pasajeras, son útiles si se utilizan
como medio para crecer en la bondad, para sanar el alma y servir al
Señor y a los hermanos con amor; pero si se convierten en el objetivo
principal de la vida, vacían a las personas de su núcleo más importante
y las convierten en apéndices de las realidades materiales.

Vayamos al encuentro del Señor que viene con buenas obras. "La noche ha
avanzado y el día está cerca" (\emph{Rom} 13, 12). El apóstol Pablo nos
exhorta a desechar las obras de las tinieblas y revestirnos de las armas
de la luz, vestirnos del Señor Jesús y no seguir a la carne en sus
deseos desordenados.

Preparémonos con esmero para la Navidad que viene, sobre todo orientando
nuestra vida hacia ese Dios con el que el último día nos encontraremos
cara a cara, con amor y alegría.

``Prepárate, porque a la hora que no imaginas, vendrá el Hijo del Hombre''.

Por tanto, velad, vestíos de Cristo.

Nuestra salvación está ahora cerca.

¡Amén!

\subsubsection{Homilía (1995): }

Domingo 3 de diciembre de 1995. Canonización de Eugène de Mazenod, fundador de los Misioneros Oblatos de la Inmaculada.

1. La venida del \emph{Hijo del Hombre es el tema del Adviento}. Así comienza el tiempo del nuevo Año Litúrgico. Ya miramos hacia la noche de Belén. Pensemos en esa venida del Hijo de Dios que ya pertenece a nuestra historia, de hecho, \emph{de una manera maravillosa la formó} como la historia de los individuos, las naciones y la humanidad. También sabemos con certeza que, después de esa venida, tenemos \emph{ante nosotros para siempre} una segunda venida del Hijo del Hombre, de Cristo. Vivimos en el segundo Adviento, en el Adviento de la historia del mundo, de la historia de la Iglesia, y en la celebración eucarística repetimos todos los días nuestra confiada esperanza de su venida.

{[}...{]}

3. En la liturgia de este primer domingo de Adviento comienza a hablar el \textbf{profeta Isaías}. Escucharemos la palabra inspirada de todo este tiempo. ``Visión de Isaías, hijo de Amós, acerca de Judá y Jerusalén. Al final de los días, la montaña del templo del Señor se elevará en la cumbre de las montañas y será más alta que las colinas; todos los pueblos acudirán a él. Muchos pueblos vendrán y dirán: "Venid, subamos al monte del Señor, al templo del Dios de Jacob, para que él nos muestre sus caminos y podamos caminar por sus sendas". Porque de Sion saldrá la ley, y de Jerusalén la palabra del Señor'' (\emph{Is} 2, 1-3).

A la luz del Espíritu Santo, el Profeta tiene \emph{una visión universalista} y muy aguda de la salvación. Jerusalén, la ciudad ubicada en medio de Israel, Pueblo de elección divina, tiene un gran futuro por delante. Cuando el Profeta dice que "la palabra del Señor saldrá ... de Jerusalén", ya muchos siglos antes de la venida de Cristo, anuncia el alcance de la obra mesiánica.

La mirada de Isaías \emph{enriquece nuestra conciencia del Adviento}. El que ha de venir, que debe revelarse "hasta el fin" en medio de la ciudad santa de Jerusalén, por la palabra de su Evangelio, y especialmente por su cruz y su resurrección, será enviado a todas las naciones del mundo, a toda la humanidad. Será \emph{el Ungido de Dios, el Redentor del hombre}. Su visita no duró mucho, pero la misión que él transmitió a los Apóstoles y a la Iglesia perdurará hasta el final de los siglos. Será mediador entre Dios y los hombres, y en voz alta exhortará a las naciones a la paz, invitando a todos a "forjar de sus espadas arados, de sus lanzas podaderas" (cf. \emph{Is} 2, 4). Así comienza la exhortación de Isaías, dirigida a los pueblos de toda la tierra, a que dirijan la mirada y los pasos hacia Jerusalén.

Esta exhortación tiene su eco en el \textbf{Salmo Responsorial}, \emph{canto de los peregrinos} a la Ciudad Santa. ````Qué alegría cuando me dijeron: Vamos a la casa del Señor''. Ya están pisando nuestros pies tus umbrales, Jerusalén. Allí suben juntas las tribus, las tribus del Señor'''' (\emph{Sal} 122, 1, 4). Y nuevamente: Pide la paz para Jerusalén: ``la paz sea con los que te aman, la paz sea en tus muros, la seguridad en tus baluartes" (\emph{Sal} 122: 6-7).

{[}...{]}

6. El mensaje de Adviento está unido a la venida del Hijo del Hombre, cada vez más cercano. A esta conciencia corresponde \emph{la exhortación a la vigilancia}. En el \textbf{Evangelio de San Mateo}, Jesús dice a los que le escuchan: "Velad, pues, porque no sabéis en qué día vendrá el Señor ... Por tanto, también vosotros estad preparados, porque a la hora que menos penséis vendrá el Hijo del Hombre'' (\emph{Mt} 24, 42, 44). El pasaje de la \textbf{Carta de San Pablo a los Romanos} corresponde de manera excelente a esta exhortación, repetida varias veces en el Evangelio. El Apóstol nos dice cómo podemos ser \emph{"conscientes del momento"} (cf. \emph{Rm} 13,11). \emph{La espera}, volcada hacia el futuro, \emph{siempre} se nos presenta \emph{como un "momento" ya cercano y presente}. En la obra de salvación no se puede dejar nada para después. \emph{¡Cada "hora" importa!} El Apóstol escribe que "nuestra salvación está más cerca ahora que cuando empezamos a creer" (\emph{Rom} 13,11) y compara este momento presente con el amanecer, con el momento culminante del paso entre la noche y el día.

San Pablo traslada el fenómeno que acompaña al despertar de la luz del día al ámbito espiritual. ``La noche está avanzada --- escribe --- el día está cerca. Desechemos, pues, las obras de las tinieblas y vistámonos de las armas de la luz'' (\emph{Rom} 13,12). Después de haber llamado por su nombre las obras de las tinieblas, el Apóstol indica a qué aluden "las armas de la luz": \emph{"vistámonos con las armas de la luz"}, es decir, "vistámonos ... del Señor Jesucristo" (\emph{Rom} 13,14 ). El apóstol nos invita a hacer de Jesucristo la norma de nuestra vida y de nuestras acciones, para que en Él podamos llegar a convertirnos en una nueva creación. Así renovados, podremos renovar el mundo en Cristo, en virtud de la misión, ya injertada en nosotros por el sacramento del Bautismo.

{[}...{]}

\chapter{S. Familia}
\subsubsection{Homilía (1986): }
30 de noviembre de 1986. Celebración en el Hipódromo ``Belmont'', Perth,
Australia.

Esta homilía fue pronunciada el Domigo I de Adviento, pero la celebración estaba dedicada a la Familia, por lo que las palabras del Papa son perfectamente aplicables a la celebración de este día.

\emph{``Es hora de despertarnos del sueño, porque nuestra salvación está
	más cerca ahora ..."} (\emph{Rom} 13, 11).

\emph{Amados hermanos y hermanas en Cristo.}

1. Con estas solemnes palabras, la liturgia de este primer domingo de
Adviento conduce a toda la Iglesia a un tiempo de espera y preparación.
Es el momento en el que toda comunidad cristiana revive la expectativa
que los profetas despertaron en el pueblo de Israel, mientras esperaban
ansiosos el cumplimiento de la promesa: "La Virgen concebirá y dará a
luz un hijo, al que llamará Emmanuel" (\emph{Is} 7, 14), que significa
"Dios con nosotros" (cf. \emph{Mt} 1, 23). Es el tiempo de preparación
para la venida del niño, el "Príncipe de la Paz": el infante de Belén,
que es al mismo tiempo el Hijo de Dios, y la segunda Persona de la
Santísima Trinidad.

(...) La Navidad es un día especial para las familias (...) en muchas
otras partes del mundo. La familia en el proyecto de Dios para la
humanidad y para la Iglesia (...). El Hijo de Dios, al hacerse hombre,
inicia esa familia especial que la Iglesia venera como la Sagrada
Familia de Nazaret: Jesús, María y José.

{[}...{]}

3. "La familia es la Iglesia doméstica". El significado de esta idea
cristiana tradicional es que la casa es la Iglesia en miniatura. La
Iglesia es el sacramento del amor de Dios, es una comunión de fe y de
vida. Ella es madre y maestra. Está al servicio de toda la familia
humana para caminar hacia su destino final. Al mismo tiempo, la familia
es una comunidad de vida y amor. Educa y orienta a sus miembros hacia la
plena madurez humana y está al servicio del bien de todos en el camino
de la vida. La familia es la "primera y vital célula de la sociedad"
(\href{http://www.vatican.va/archive/hist_councils/ii_vatican_council/documents/vat-ii_decree_19651118_apostolicam-actuositatem_it.html}{\emph{\emph{Apostolicam
			Actuositatem}}}). El futuro del mundo y de la Iglesia pasa, pues, por la
familia.

Por tanto, no es de extrañar que la Iglesia en los últimos tiempos haya
prestado mucho cuidado y atención a los problemas que afectan a la vida
familiar y al matrimonio. Tampoco es de extrañar que los gobiernos y las
organizaciones públicas estén constantemente involucrados en problemas
que afectan directa o indirectamente el bienestar institucional del
matrimonio y la familia. Y todos pudieron ver que las relaciones
saludables en el matrimonio y en la familia son de gran importancia para
el crecimiento y el bienestar de la persona humana.

4. Las transformaciones económicas, sociales y culturales que están
teniendo lugar en el mundo tienen un gran efecto en la forma en que las
personas ven el matrimonio y la familia. Como resultado, muchos esposos
no están seguros del significado de su relación y esto les hace sentirse
incómodos y angustiados. Por otro lado, muchos otros matrimonios son más
fuertes porque, habiendo superado las tensiones del mundo moderno,
experimentan mucho más plenamente ese amor especial y la responsabilidad
del matrimonio que les hace ver a los hijos como un don especial de Dios
para ellos y para la sociedad. Dependiendo de cómo vaya la familia,
también lo hará la nación y todo el mundo en el que vivimos.

En cuanto a la familia, la sociedad necesita urgentemente "que todos
recuperen la conciencia de la primacía de los valores morales, que son
los valores de la persona humana como tal", y también de la
"re-comprensión del sentido último de la vida y sus valores
fundamentales"
(\href{http://www.vatican.va/content/john-paul-ii/it/apost_exhortations/documents/hf_jp-ii_exh_19811122_familiaris-consortio.html}{\emph{\emph{Familiaris
			Consortio}}}, 8). {[}Es necesario{]} saber cómo salvaguardar la familia
y la estabilidad del amor conyugal si queremos tener paz y justicia en
la tierra.

5. La Iglesia (...) tiene una tarea específica: explicar y promover el
plan de Dios para el matrimonio y la familia y ayudar a los esposos y
familias a vivir de acuerdo con este plan. La Iglesia se dirige a todas
las familias: en primer lugar a aquellas familias cristianas que se
esfuerzan por ser cada vez más fieles al designio de Dios, busca
fortalecerlas y acompañarlas en su desarrollo. Pero también se dirige,
con la compasión del corazón de Jesús, a aquellas familias que se
encuentran en dificultades o en situaciones irregulares.

La Iglesia no puede decir que lo malo es bueno, ni puede decir que lo
que no es válido es válido. No puede dejar de proclamar la enseñanza de
Cristo, incluso cuando esta enseñanza es difícil de aceptar. También
sabe que fue enviada a curar, reconciliar, llamar a la conversión,
encontrar lo perdido (cf. \emph{Lc} 15, 6). Por tanto, es con inmenso
amor y paciencia que la Iglesia busca ayudar a quienes experimentan
dificultades para responder a las exigencias del amor conyugal cristiano
y de la vida familiar.

La caridad de Cristo solo se puede realizar en la verdad: en la verdad
sobre la vida, el amor y la responsabilidad. La Iglesia debe anunciar a
Cristo: camino, verdad, vida; y al hacerlo, debe enseñar los valores y
principios que corresponden a la llamada del hombre, a la "novedad" de
vida en Cristo. La Iglesia a veces es incomprendida y considerada
carente de compasión porque apoya el plan creador de Dios para el
matrimonio y la familia: su plan para el amor humano y la transmisión de
la vida. La Iglesia es siempre la verdadera y fiel amiga de la persona
humana en el peregrinaje de la vida. Sabe que la defensa de la ley moral
contribuye al establecimiento de una verdadera civilización humana, y
desafía constantemente a las personas a no abandonar su responsabilidad
personal ante los imperativos éticos y morales (cf.
\href{http://www.vatican.va/content/paul-vi/it/encyclicals/documents/hf_p-vi_enc_25071968_humanae-vitae.html}{\emph{\emph{Humanae
			Vitae}}}, 18).

6. "Venid, subamos al monte del Señor... para que nos muestre sus
caminos y recorramos sus sendas" (\emph{Is} 2, 3). Con esta invitación,
el profeta Isaías nos dice cómo debemos responder a Dios y cómo esta
respuesta también se puede aplicar al plan de Dios para el matrimonio y
la familia. A los esposos se les ofrece la gracia y la fuerza del
sacramento del matrimonio para que puedan caminar por los caminos del
Señor y seguir sus sendas, observando el plan que Cristo confirmó y
estableció para la familia. Este plan da testimonio de lo que era en el
"principio" (cf. \emph{Mt} 19, 8), lo que Dios quiso desde el principio
para el bienestar y la felicidad de la familia. En el plan de Dios, el
matrimonio requiere: amor fiel y duradero entre marido y mujer; una
comunión indisoluble que "tiene sus raíces en la complementariedad
natural que existe entre el hombre y la mujer, y se nutre de la voluntad
personal de los esposos de compartir todo el proyecto de vida, lo que
tienen y lo que son" (\emph{Familiaris Consortio}, 19); una comunidad de
personas en la que el amor entre marido y mujer es plenamente humano,
exclusivo y abierto a una nueva vida (cf. \emph{Familiaris consortio},
29).

El amor conyugal se fortalece con el sacramento del matrimonio para que
sea una imagen cada vez más real y eficaz de la unidad que existe entre
Cristo y la Iglesia (cf. \emph{Ef} 5, 32).

7. Vosotros sabéis cuánto valor cristiano necesitáis para vivir los
mandamientos de Dios en vuestra vida y en vuestra familia. Se trata de
la valentía de estar dispuestos cada día a construir el amor, ese amor
del que dice san Pablo: "La caridad es paciente, la caridad es
bondadosa; la caridad no es envidiosa, no se jacta, no se hincha, no
falta el respeto, no busca su interés, no se enoja, no toma en cuenta el
mal recibido, no disfruta de la injusticia, pero se complace con la
verdad. Todo lo cree ... todo lo soporta. La caridad no se acabará
nunca" (\emph{1 Cor} 13, 4-8).

{[}Entonces, ¿puede el Papa venir a Australia y no pedirle a los esposos
y familias australianas que reflexionen en sus corazones si están
viviendo bien su amor cristiano? ¿Cuán seriamente están comprometidos
con la defensa de los valores familiares? ¿Qué tan adecuadas son las
políticas para la defensa de estos valores y, por tanto, para la
promoción del bien común de toda la nación?{]}

En un mundo cada vez más sensible a los derechos de las mujeres, ¿qué se
puede decir sobre los derechos de las mujeres que quieren ser, o
necesitan ser, esposas y madres a tiempo completo? ¿Deberían estar
agobiadas por un sistema fiscal que las discrimine de aquellas que optan
por no quedarse en casa para tener sus propios ingresos? Sin violar la
libertad de cualquiera que busque satisfacción en el empleo y las
actividades fuera del hogar, ¿no debería apreciarse y apoyarse
adecuadamente el trabajo del ama de casa? (cf. \emph{Familiaris
	Consortio}, 23). Esto es posible cuando las mujeres y los hombres son
tratados con pleno respeto de su dignidad personal, por lo que son, más
que por lo que hacen.

8. Comprendiendo la importancia esencial de la vida familiar para una
sociedad justa y saludable, la Santa Sede ha presentado una Carta de los
derechos de la familia basada en los derechos naturales y valores
comunes de toda la humanidad. Está dirigida principalmente a gobiernos y
organismos internacionales, como "modelo y punto de referencia para la
elaboración de legislación y política familiar, y guía para programas de
acción"
(\emph{\emph{\href{http://www.vatican.va/roman_curia/pontifical_councils/family/documents/rc_pc_family_doc_19831022_family-rights_it.html}{Carta
			de los derechos de la familia, 22 de octubre de 1983}, Introducción}}).

Entre los diversos principios que la Iglesia apoya firmemente en todas
las circunstancias se encuentran los siguientes, sobre los que quiero
llamar vuestra atención: el derecho inalienable de los cónyuges "a
establecer una familia y a decidir el intervalo entre los nacimientos y
el número de hijos a procrear, teniendo plenamente en cuenta sus deberes
para con ellos mismos, con los hijos ya nacidos, la familia y la
sociedad, en una justa jerarquía de valores y conforme al orden moral
objetivo... "; todas las presiones que limitan "la libertad de los
padres para decidir sobre sus hijos constituyen una grave ofensa contra
la dignidad humana y la justicia"; "las familias tienen derecho a poder
contar con una adecuada política familiar por parte de los poderes
públicos en los ámbitos jurídico, económico, social y fiscal, sin
discriminación de ningún tipo" (\emph{Carta de los derechos de la
	familia}, artículos 3 y 9) .

9. El orden moral exige que la regla establecida para los procesos de
vida por el Creador en el acto de la creación sea respetada siempre y en
todas partes. La conocida oposición de la Iglesia a la anticoncepción y
la esterilización no es una posición tomada arbitrariamente, ni se basa
en una perspectiva parcial de la persona humana. Más bien expresa su
visión integral de la persona humana, a quien se le ha dado una vocación
no sólo natural y terrena, sino también sobrenatural y eterna (cf.
\emph{Humanae vitae}, 7). Además, la comprensión de la Iglesia del valor
intrínseco de la vida humana como un regalo irrevocable de Dios explica
por qué el Concilio Vaticano II habla de una "misión muy elevada para
proteger la vida" y considera el aborto como un "crimen abominable"
(\href{http://www.vatican.va/archive/hist_councils/ii_vatican_council/documents/vat-ii_const_19651207_gaudium-et-spes_it.html}{\emph{\emph{Gaudium
			et Spes}}}, 27. 51). ).

10. El lugar que ocupan los niños en la cultura y la sociedad (...)
merece una consideración. Sé que amáis y respetáis a vuestros hijos. Sé
que en muchos sentidos las leyes tienen como objetivo salvaguardar su
bienestar y su protección. Una sociedad que ama a sus hijos es una
sociedad sana y dinámica. En su nombre, os hago un llamamiento a
vosotros, padres. Los niños necesitan padres que puedan brindarles un
entorno familiar estable. Hacedle saber que necesitan del amor verdadero
para sentirse unidos en vuestro amor por los demás y por ellos mismos.
Ellos buscan en vosotros amistad y guía. De vosotros, sobre todo, deben
aprender a distinguir entre lo justo y lo injusto y saber discernir el
bien del mal. Os hago, pues, un llamamiento: no privéis a vuestros hijos
de su herencia verdaderamente humana y espiritual. Habladle de Dios, de
Jesús, de su amor y de su Evangelio. Enseñadle a amar a Dios y a
respetar sus mandamientos con la certeza de que son ante todo hijos
suyos. Enseñadle a orar. Ayudadle a convertirse en seres humanos maduros
y responsables, ciudadanos honestos de su país. Este es un privilegio
maravilloso, un deber importante y una asignación maravillosa que habéis
recibido de Dios. Mediante el testimonio de vuestra vida cristiana,
guiad a vuestros hijos a ocupar el lugar que les corresponde en la
Iglesia de Cristo.

11. ¿Y qué deciros a vosotros, niños y jóvenes, {[}presentes aquí en tan
gran número{]}? Amad a vuestros padres; rezad por ellos; dad gracias a
Dios todos los días por ellos. Si a veces hay malentendidos entre
vosotros, si a veces os resulta difícil obedecerlos, recordad estas
palabras de San Pablo: ``Haced todo sin murmuraciones y sin críticas,
para que seáis irreprensibles y sencillos, inmaculados, hijos de Dios
... que debes brillar como estrellas en el mundo'' (\emph{Fil} 2,
14-15). Rezad también por vuestros hermanos y hermanas y por todos los
niños del mundo, especialmente por los pobres y hambrientos. Orad por
los que no conocen a Jesús, por los que están solos y tristes.

A todos los jóvenes católicos (...) se os confía el futuro de la Iglesia
en esta tierra. La Iglesia os necesita. Hay mucho que hacer en vuestras
parroquias y comunidades locales, al servicio de los pobres y los
necesitados, los enfermos y los ancianos, a través de las muchas formas
de servicio voluntario. En primer lugar, debéis llevar a Cristo a
vuestros amigos. Vuestra generación es el campo, rico para la mies, al
que Cristo os envía. Cristo es el camino, la verdad y la vida para
vuestra generación y para las generaciones venideras. Vosotros sois la
esperanza de la Iglesia para una nueva era de evangelización y servicio.
¡Sed generosos con los demás, sed generosos con Cristo!

12. {[}Queridos padres e hijos, queridas familias ...: el Evangelio del
primer domingo de Adviento nos llamaba a ``velar'', porque ``si el dueño
de casa lo supiera ... velaría y no permitiría que entren en su casa''
(\emph{Mt} 24, 43). Esta es la exhortación que os repito. ¡Velad! No
permitáis que os quiten el bien precioso del fiel amor matrimonial y la
vida familiar. No los rechacéis, no penséis que hay una propuesta mejor
para la felicidad o la realización humana.

La llamada del Evangelio a "velar" también significa construir en la
familia un sentido de responsabilidad. El amor genuino es siempre amor
responsable. Los esposos y las esposas se aman verdaderamente cuando son
responsables ante Dios y llevan a cabo su plan para el amor y la vida
humanos; cuando responden y son responsables unos de otros. La
paternidad responsable implica no solo traer hijos al mundo, sino
también participar personal y responsablemente en su crecimiento y
educación. ¡El verdadero amor en la familia es para siempre! Finalmente,
mientras nos esforzamos por ser perfectos en el amor, recordamos las
palabras de San Pablo: ``Por lo tanto, desechemos las obras de las
tinieblas y vistámonos las armas de la luz... en cambio, vestíos del
Señor Jesucristo'' (\emph{Rom} 13, 12. 14).

Queridas familias (...) esta es vuestra vocación y vuestra felicidad hoy
y siempre: revestirse del Señor Jesucristo y caminar en su luz. Amén.{]}