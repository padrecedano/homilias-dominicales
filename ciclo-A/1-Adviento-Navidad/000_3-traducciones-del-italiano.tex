\chapter{I-Adviento}

\subsubsection{Homilía (1992): }

VISITA A LA PARROQUIA DE SAN GERARDO MAIELLA

\emph{\textbf{HOMILIA DE JUAN PABLO II}}

\emph{Domingo 29 de noviembre de 1992}

1. \emph{"Vayamos gozosos al encuentro del Señor"} (Salmo responsorial).

¡Queridos hermanos y hermanas (...)! Hoy, primer domingo de Adviento,
comienza un nuevo año litúrgico durante el cual la Iglesia recorre y
revive espiritualmente las etapas del misterio cristiano.

Este plan divino abarca toda la historia de la humanidad, desde los
albores de la creación hasta el día final, cuando todas las cosas serán
recapituladas en Cristo (cf. \emph{Ef} 1, 10) y habrá nuevos cielos y
nueva tierra (cf. \emph{2 Pt.} 3, 13). El centro de este proyecto está
en el misterio de la encarnación del Hijo de Dios.

En un momento preciso, por obra del Espíritu Santo, el Verbo "se hizo
carne" en el seno virginal de María y "vino a habitar en nosotros". (cf.
\emph{Jn} 1, 12), mostrando la bondad y humanidad de Dios para con los
hombres. El Señor, de hecho, no sólo creó al hombre, sino que lo ama con
tanta intensidad que lo acoge desde dentro de su propia familia,
destinándolo a una gloria sin fin.

De hecho, sostenidos por una certeza tan consoladora vamos con alegría a
su encuentro, como nos invita a hacer el \textbf{Salmo Responsorial}.

2. Vamos a encontrarlo en el misterio de la Navidad. Este es el primer
sentido del Adviento. Nos conmueve el recuerdo de María y José, que
suben de Nazaret de Galilea a Belén de Judea para el censo, y se ven
obligados a refugiarse en un lugar destinado a los animales, "porque no
había lugar para ellos en la posada" (\emph{Lc} 2, 7).

El Hijo de Dios viene a la luz en la pobreza total: verdadero Dios
Salvador, anunciado por los ángeles a los pastores, y verdadero hombre,
envuelto en pañales y colocado en un pesebre.

¡Qué sentimientos de ternura, amor y gratitud despierta este
extraordinario evento! Sin embargo, también tiene la fuerza de sacudir
nuestra conciencia invitándonos a despertar del sueño de la indiferencia
y de la habitud.

"Hermanos - nos exhorta el \textbf{apóstol Pablo} - ya es hora de
despertar del sueño" (\emph{Rm} 13, 11), Dios nos ha amado hasta darnos
a su Hijo único. ¿Acaso un don tan grande no nos obligua a reflexionar y
a responderle con la generosidad adecuada? ¿No nos empuja a abandonar
las tinieblas del pecado para abrir el espíritu a la luz de la gracia
divina? Esto es precisamente a lo que nos invitan las lecturas de la
liturgia de hoy.

3. "Venid, subamos al monte del Señor" (\emph{Is} 2, 3).

El texto, tomado del libro del \textbf{profeta Isaías}, se interpreta
comúnmente como un anuncio mesiánico. Para el pueblo de Israel forzado
al exilio, el profeta predice la reconstrucción del Templo de Jerusalén.
Pero sus palabras van más allá de la historia del pueblo judío. Con
imágenes como la del cerro más alto de todas las montañas y con el
pronóstico de la venida de innumerables naciones al templo del Dios de
Jacob, se indica una nueva realidad espiritual, la del pueblo de los
redimidos guiados por el Mesías prometido; y la de una nueva Alianza,
que transforma profundamente la vida de los hombres, según los ideales
de paz y fraternidad, convirtiendo las espadas en arado y las lanzas en
podaderas.

4. ¡Queridos hermanos y hermanas! Vuestra comunidad parroquial, una de
las más jóvenes de nuestra Iglesia local, está animada por estos ideales
de paz y fraternidad, santidad y evangelización. (...)

(...) Gracias a una catequesis capilar, habéis intentado ayudar a los
habitantes de todo el barrio a no retirarse al individualismo, sino a
crecer como comunidad cristiana solidaria, siguiendo el ejemplo de la
Iglesia de los Apóstoles y de las primeras generaciones de creyentes, a
través de una atenta escucha de la Palabra, participación en la vida
litúrgica de la Comunidad y un intenso esfuerzo de compartir y acogida
recíproca. De esta manera, se han desarrollado experiencias comunitarias
significativas y formas efectivas de catequesis para adultos, como las
Comunidades Neocatecumenales. Muchos jóvenes pertenecen se reúnen
también en la Acción Católica y en otros movimientos apostólicos, Grupos
de matrimonios cristianos siguen un itinerario formativo común
estructurado en encuentros periódicos (...) La parroquia tampoco carece
de la sensibilidad misionera que brotó del hermanamiento con una misión
africana en Chad.

5. Por todo ello (...) deseo manifestar mi viva satisfacción,
reconociendo el compromiso y la generosidad que os animan.

Perseverad, queridos hermanos y hermanas, en el esfuerzo realizado.
\emph{Daréis, pues, vida a una nueva evangelización} (...). Todavía hay
muchas personas que no conocen el evangelio adecuadamente y esperan el
testimonio constante de nuestra vida y la proclamación gozosa de nuestra
fe en Cristo.

A vosotros os encomiendo una misión tan exigente: a las familias, a los
adultos, a los niños, a los ancianos y especialmente a los jóvenes, y os
aseguro a cada uno de vosotros el apoyo de mi cordial oración.

Este inicio del tiempo de Adviento constituye una ocasión propicia para
intensificar el ritmo de nuestra vida cristiana.

6. "Vuestro Señor vendrá" (\emph{Mt} 24, 42).

El pasaje del \textbf{Evangelio de Mateo} abarca una parte del discurso
de Jesús sobre los últimos acontecimientos, que por eso se llama
discurso escatológico.

Jesús anuncia su segunda venida, al final de los tiempos, y nos exhorta
a estar alerta y preparados para encontrarnos con Él. Este es el segundo
significado del Adviento.

Por las palabras de Jesús, contenidas en este y otros textos, sabemos
con certeza que el mundo presente está destinado a terminar, que la
historia humana terminará, que para cada uno habrá un juicio, seguido de
una recompensa o castigo. A la luz de todo esto es importante escuchar
la invitación a velar "porque no sabes en qué día vendrá tu Señor".

7. ¡``Velad, pues''!

La vigilancia evangélica es condición para un buen uso de la vida.

Qué fácil es desperdiciar los dones divinos, distanciarse de Dios con
pensamientos y comportamientos, olvidar que la vida pasa.

Las cosas temporales son frágiles y pasajeras, son útiles si se utilizan
como medio para crecer en la bondad, para sanar el alma y servir al
Señor y a los hermanos con amor; pero si se convierten en el objetivo
principal de la vida, vacían a las personas de su núcleo más importante
y las convierten en apéndices de las realidades materiales.

Vayamos al encuentro del Señor que viene con buenas obras. "La noche ha
avanzado y el día está cerca" (\emph{Rom} 13, 12). El apóstol Pablo nos
exhorta a desechar las obras de las tinieblas y revestirnos de las armas
de la luz, vestirnos del Señor Jesús y no seguir a la carne en sus
deseos desordenados.

Preparémonos con esmero para la Navidad que viene, sobre todo orientando
nuestra vida hacia ese Dios con el que el último día nos encontraremos
cara a cara, con amor y alegría.

``Prepárate, porque a la hora que no imaginas, vendrá el Hijo del Hombre''.

Por tanto, velad, vestíos de Cristo.

Nuestra salvación está ahora cerca.

¡Amén!

\subsubsection{Homilía (1995): }

Domingo 3 de diciembre de 1995. Canonización de Eugène de Mazenod, fundador de los Misioneros Oblatos de la Inmaculada.

1. La venida del \emph{Hijo del Hombre es el tema del Adviento}. Así comienza el tiempo del nuevo Año Litúrgico. Ya miramos hacia la noche de Belén. Pensemos en esa venida del Hijo de Dios que ya pertenece a nuestra historia, de hecho, \emph{de una manera maravillosa la formó} como la historia de los individuos, las naciones y la humanidad. También sabemos con certeza que, después de esa venida, tenemos \emph{ante nosotros para siempre} una segunda venida del Hijo del Hombre, de Cristo. Vivimos en el segundo Adviento, en el Adviento de la historia del mundo, de la historia de la Iglesia, y en la celebración eucarística repetimos todos los días nuestra confiada esperanza de su venida.

{[}...{]}

3. En la liturgia de este primer domingo de Adviento comienza a hablar el \textbf{profeta Isaías}. Escucharemos la palabra inspirada de todo este tiempo. ``Visión de Isaías, hijo de Amós, acerca de Judá y Jerusalén. Al final de los días, la montaña del templo del Señor se elevará en la cumbre de las montañas y será más alta que las colinas; todos los pueblos acudirán a él. Muchos pueblos vendrán y dirán: "Venid, subamos al monte del Señor, al templo del Dios de Jacob, para que él nos muestre sus caminos y podamos caminar por sus sendas". Porque de Sion saldrá la ley, y de Jerusalén la palabra del Señor'' (\emph{Is} 2, 1-3).

A la luz del Espíritu Santo, el Profeta tiene \emph{una visión universalista} y muy aguda de la salvación. Jerusalén, la ciudad ubicada en medio de Israel, Pueblo de elección divina, tiene un gran futuro por delante. Cuando el Profeta dice que "la palabra del Señor saldrá ... de Jerusalén", ya muchos siglos antes de la venida de Cristo, anuncia el alcance de la obra mesiánica.

La mirada de Isaías \emph{enriquece nuestra conciencia del Adviento}. El que ha de venir, que debe revelarse "hasta el fin" en medio de la ciudad santa de Jerusalén, por la palabra de su Evangelio, y especialmente por su cruz y su resurrección, será enviado a todas las naciones del mundo, a toda la humanidad. Será \emph{el Ungido de Dios, el Redentor del hombre}. Su visita no duró mucho, pero la misión que él transmitió a los Apóstoles y a la Iglesia perdurará hasta el final de los siglos. Será mediador entre Dios y los hombres, y en voz alta exhortará a las naciones a la paz, invitando a todos a "forjar de sus espadas arados, de sus lanzas podaderas" (cf. \emph{Is} 2, 4). Así comienza la exhortación de Isaías, dirigida a los pueblos de toda la tierra, a que dirijan la mirada y los pasos hacia Jerusalén.

Esta exhortación tiene su eco en el \textbf{Salmo Responsorial}, \emph{canto de los peregrinos} a la Ciudad Santa. ````Qué alegría cuando me dijeron: Vamos a la casa del Señor''. Ya están pisando nuestros pies tus umbrales, Jerusalén. Allí suben juntas las tribus, las tribus del Señor'''' (\emph{Sal} 122, 1, 4). Y nuevamente: Pide la paz para Jerusalén: ``la paz sea con los que te aman, la paz sea en tus muros, la seguridad en tus baluartes" (\emph{Sal} 122: 6-7).

{[}...{]}

6. El mensaje de Adviento está unido a la venida del Hijo del Hombre, cada vez más cercano. A esta conciencia corresponde \emph{la exhortación a la vigilancia}. En el \textbf{Evangelio de San Mateo}, Jesús dice a los que le escuchan: "Velad, pues, porque no sabéis en qué día vendrá el Señor ... Por tanto, también vosotros estad preparados, porque a la hora que menos penséis vendrá el Hijo del Hombre'' (\emph{Mt} 24, 42, 44). El pasaje de la \textbf{Carta de San Pablo a los Romanos} corresponde de manera excelente a esta exhortación, repetida varias veces en el Evangelio. El Apóstol nos dice cómo podemos ser \emph{"conscientes del momento"} (cf. \emph{Rm} 13,11). \emph{La espera}, volcada hacia el futuro, \emph{siempre} se nos presenta \emph{como un "momento" ya cercano y presente}. En la obra de salvación no se puede dejar nada para después. \emph{¡Cada "hora" importa!} El Apóstol escribe que "nuestra salvación está más cerca ahora que cuando empezamos a creer" (\emph{Rom} 13,11) y compara este momento presente con el amanecer, con el momento culminante del paso entre la noche y el día.

San Pablo traslada el fenómeno que acompaña al despertar de la luz del día al ámbito espiritual. ``La noche está avanzada --- escribe --- el día está cerca. Desechemos, pues, las obras de las tinieblas y vistámonos de las armas de la luz'' (\emph{Rom} 13,12). Después de haber llamado por su nombre las obras de las tinieblas, el Apóstol indica a qué aluden "las armas de la luz": \emph{"vistámonos con las armas de la luz"}, es decir, "vistámonos ... del Señor Jesucristo" (\emph{Rom} 13,14 ). El apóstol nos invita a hacer de Jesucristo la norma de nuestra vida y de nuestras acciones, para que en Él podamos llegar a convertirnos en una nueva creación. Así renovados, podremos renovar el mundo en Cristo, en virtud de la misión, ya injertada en nosotros por el sacramento del Bautismo.

{[}...{]}

\subsubsection{Homilía (1992): }