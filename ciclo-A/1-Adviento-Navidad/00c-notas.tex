	%\nocite{*}

	%Textos que se pueden reutilizar
	\newcommand{\temasBautismo}{\rbr{El Directorio Homilético no indica temas del Catecismo para esta fiesta. Pero podemos considerar los siguientes:}}
	
	\newcommand{\temasNavidad}{\rbr{El Directorio Homilético recoge los temas de la Navidad en un solo grupo, ver página \pageref{navidad_temas}.}}
	
	\newcommand{\homiliasNavidad}{\rbr{Las homilías para esta celebración están tomadas de textos de las Padres de la Iglesia que tocan algunos aspectos de la Navidad en particular o relacionados con alguno de los textos bíblicos que se leen en la misma. \\Conviene señalar que estas homilías pueden iluminar aspectos de cualquiera de las otras celebraciones durante el tiempo de Navidad.}}
	
	\newcommand{\homiliasNavidadABC}{\rbr{Las lecturas para esta solemnidad son las mismas en los tres ciclos dominicales. No obstante, las homilías han sido distribuidas en esta obra en tres grupos, tomando en cuenta el ciclo litúrgico correspondiente al año en que fueron pronunciadas. Aquí aparecen las homilías que correspondieron al año A, y las de los años B y C aparecerán en sus respectivos volúmenes.}}
	
	\newcommand{\homiliasSFamilia}{\rbr{Las lecturas para esta solemnidad podrían ser las mismas en los tres ciclos dominicales (hay lecturas opcionales para los años B y C). En esta obra las homilías han sido distribuidas en tres grupos, tomando en cuenta el ciclo litúrgico correspondiente al año en que fueron pronunciadas. En este volumen aparecen las homilías que correspondieron al año A, mientras que las homilías que correspondieron a los años B y C aparecerán en sus respectivos volúmenes. Aquéllas homilías podrían también ser iluminadoras para esta año y viceversa. En todos los casos, en las homilías podría haber referencia a cualquiera de las lecturas opcionales.}}
	
	\newcommand{\homiliasABC}{\rbr{Las lecturas para este domingo son las mismas en los tres ciclos dominicales. No obstante, las homilías han sido distribuidas en esta obra en tres grupos, tomando en cuenta el ciclo litúrgico correspondiente al año en que fueron pronunciadas. Aquí aparecen las homilías que correspondieron al año A, y las de los años B y C aparecerán en sus respectivos volúmenes.}}

	\newcommand{\lecturasSantaFamilia}{\rbr{Cuando esta fiesta se celebra el 30 de diciembre, por no haber ningún domingo entre los días 25 de diciembre y 1 de enero, antes del Evangelio se ha de elegir una sola lectura.}}
	
	
	\printbibliography
	
\iffalse
\anotecontent{id1}{\label{id1} A partir de esa nueva traducción de la Biblia se publicaron los nuevos Leccionarios de la Misa, que son los leccionarios oficiales desde el mes de septiembre del año 2016.}

\anotecontent{id2}{\label{id2} NUALC n. 40.}

\anotecontent{id3}{\label{id3} Cf. OLM n. 93.}

\anotecontent{id4}{\label{id4} Cf. Congregación para el Culto Divino, \emph{Directorio Homilético} (2014)\emph{,} nn. 78-109.}

\anotecontent{id5}{\label{id5} Cf. Oficio de lecturas, Lunes, I semana de Adviento.}

\anotecontent{id6}{\label{id6} Benedicto XVI, papa, \emph{Homilía}, 28 de noviembre de 2009.}
\fi