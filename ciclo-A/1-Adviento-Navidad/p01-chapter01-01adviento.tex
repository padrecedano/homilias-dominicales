\chapter{Domingo I de Adviento (A)}

	\section{Lecturas}
	
		\rtitle{PRIMERA LECTURA}
		
		\rbook{Del libro del profeta Isaías} \rred{2, 1-5}
		
		\rtheme{Visión de Isaías, hijo de Amós, acerca de Judá y de Jerusalén}
		
		\begin{readprose}
			En los días futuros estará firme
			
			el monte de la casa del Señor, 
			
			en la cumbre de las montañas, 
			
			más elevado que las colinas. 
			
			Hacia él confluirán todas las naciones, 
			
			caminarán pueblos numerosos y dirán: 
			
			«Venid, subamos al monte del Señor, 
			
			a la casa del Dios de Jacob. 
			
			Él nos instruirá en sus caminos 
			
			y marcharemos por sus sendas; 
			
			porque de Sión saldrá la ley, 
			
			la palabra del Señor de Jerusalén». 
			
			Juzgará entre las naciones, 
			
			será árbitro de pueblos numerosos. 
			
			De las espadas forjarán arados, 
			
			de las lanzas, podaderas. 
			
			No alzará la espada pueblo contra pueblo, 
			
			no se adiestrarán para la guerra. 
			
			Casa de Jacob, venid; 
			
			caminemos a la luz del Señor.
		\end{readprose}
	
		\rtitle{SALMO RESPONSORIAL}
		
		\rbook{Salmo} \rred{121, 1-2. 4-9}
		
		\rtheme{Qué alegría cuando me dijeron: \textquote{Vamos a la casa del Señor}}
		
		\begin{psbody}
			¡Qué alegría cuando me dijeron: 
			\textquote{Vamos a la casa del Señor}! 
			Ya están pisando nuestros pies 
			tus umbrales, Jerusalén. 
			
			Allá suben las tribus, 
			las tribus del Señor, 
			según la costumbre de Israel, 
			a celebrar el nombre del Señor; 
			en ella están los tribunales de justicia, 
			en el palacio de David.
			
			Desead la paz a Jerusalén: 
			\textquote{Vivan seguros los que te aman, 
			haya paz dentro de tus muros, 
			seguridad en tus palacios}.
		
			Por mis hermanos y compañeros, 
			voy a decir: \textquote{La paz contigo}. 
			Por la casa del Señor, 
			nuestro Dios, te deseo todo bien.
		\end{psbody}
	
		\rtitle{SEGUNDA LECTURA}
		
		\rbook{De la carta del apóstol san Pablo a los Romanos} \rred{13, 11-14}
		
		\rtheme{Nuestra salvación está cerca}
		
		\begin{scripture}
			
			Hermanos: Comportaos reconociendo el momento en que vivís, pues ya es hora de despertaros del sueño, porque ahora la salvación está más cerca de nosotros que cuando abrazamos la fe. La noche está avanzada, el día está cerca: dejemos, pues, las obras de las tinieblas y pongámonos las armas de la luz. 
			
			Andemos como en pleno día, con dignidad. Nada de comilonas y borracheras, nada de lujuria y desenfreno, nada de riñas y envidias. Revestíos más bien del Señor Jesucristo.
		\end{scripture}
	
		\rtitle{EVANGELIO}
		
		\rbook{Del Santo Evangelio según san Mateo} \rred{24, 37-44}
		
		\rtheme{Vigilemos para estar preparados}
		
		\begin{scripture}
			En aquel tiempo, dijo Jesús a sus discípulos:
			
			«Cuando venga el Hijo del hombre, pasará como en tiempo de Noé. 
			
			En los días antes del diluvio, la gente comía y bebía, se casaban los hombres y las mujeres tomaban esposo, hasta el día en que Noé entró en el arca; y cuando menos lo esperaban llegó el diluvio y se los llevó a todos; lo mismo sucederá cuando venga el Hijo del hombre: dos hombres estarán en el campo, a uno se lo llevarán y a otro lo dejarán; dos mujeres estarán moliendo, a una se la llevarán y a otra la dejarán. 
			
			Por tanto, estad en vela, porque no sabéis qué día vendrá vuestro Señor. Comprended que si supiera el dueño de casa a qué hora de la noche viene el ladrón, estaría en vela y no dejaría que abrieran un boquete en su casa. Por eso, estad también vosotros preparados, porque a la hora que menos penséis viene el Hijo del hombre».
		\end{scripture}
.
\newsection

	\section{Comentario Patrístico}
	
		\subsection{Pascasio Radberto}
		
			\ptheme{Velad, para estar preparados}
			
			\src{Exposición sobre el evangelio de san Mateo, \\Lib. 11, cap. 24: PL 120, 799-800.}
			
			\emph{Velad, porque no sabéis el día ni la hora}. Siendo una recomendación que a todos afecta, la expresa como si solamente se refiriera a los hombres de aquel entonces. Es lo que ocurre con muchos otros pasajes que leemos en las Escrituras. Y de tal modo atañe a todos lo así expresado, que a cada uno le llega el último día y para cada cual es el fin del mundo el momento mismo de su muerte. Por eso es necesario que cada uno parta de este mundo tal cual ha de ser juzgado aquel día. En consecuencia, todo hombre debe cuidar de no dejarse seducir ni abandonar la vigilancia, no sea que el día de la venida del Señor lo encuentre desprevenido. 
			
			Y aquel día encontrará desprevenido a quien hallare desprevenido el último día de su vida. Pienso que los apóstoles estaban convencidos de que el Señor no iba a presentarse en sus días para el juicio final; y sin embargo, ¿quién dudará de que ellos cuidaron de no dejarse seducir, de que no abandonaron la vigilancia y de que observaron todo lo que a todos fue recomendado, para que el Señor los hallara preparados? Por esta razón, debemos tener siempre presente una doble venida de Cristo: una, cuando aparezca de nuevo y hayamos de dar cuenta de todos nuestros actos; otra diaria, cuando a todas horas visita nuestras conciencias y viene a nosotros, para que cuando viniere, nos encuentre preparados.
			
			¿De qué me sirve, en efecto, conocer el día del juicio si soy consciente de mis muchos pecados?, ¿conocer si viene o cuándo viene el Señor, si antes no viniere a mi alma y retornare a mi espíritu?, ¿si antes no vive Cristo en mí y me habla? Sólo entonces será su venida un bien para mí, si primero Cristo vive en mí y yo vivo en Cristo. Y sólo entonces vendrá a mí, como en una segunda venida, cuando, muerto para el mundo, pueda en cierto modo hacer mía aquella expresión: \emph{El mundo está crucificado para mí, y yo para el mundo}.
			
			Considera asimismo estas palabras de Cristo: \emph{Porque muchos vendrán usando mi nombre}. Sólo el anticristo y sus secuaces se arrogan falsamente el nombre de Cristo, pero sin las obras de Cristo, sin sus palabras de verdad, sin su sabiduría. En ninguna parte de la Escritura hallarás que el Señor haya usado esta expresión y haya dicho: \emph{Yo soy el Cristo}. Le bastaba mostrar con su doctrina y sus milagros lo que era realmente, pues las obras del Padre que realizaba, la doctrina que enseñaba y su poder gritaban: \emph{Yo} soy \emph{el Cristo} con más eficacia que si mil voces lo pregonaran. Cristo, que yo sepa, jamás se atribuyó verbalmente este título: lo hizo realizando las obras del Padre y enseñando la ley del amor. En cambio, los falsos cristos, careciendo de esta ley del amor, proclamaban de palabra ser lo que no eran.

\textbf{Rorate Caeli}

Rorate Caeli desúper\\ et nubes plúant justum

Ne irascáris Dómine,\\ ne ultra memíneris iniquitátis\\ Ecce cívitas Sancti\\ facta est desérta\\ Sion desérta facta est,\\ Jerúsalem desoláta est.\\ Domus sanctificatiónis tuae et gloriae tuae\\ Ubi laudavérunt Te patres nostri.

Peccávimus et facti sumus\\ tamquam immúndus nos,\\ Et cecídimus quasi fólium univérsi\\ Et iniquitátes nostrae quasi ventus\\ abstulérunt nos\\ Abscondísti fáciem tuam a nobis\\ Et allisísti nos\\ in mánu iniquitátis nostrae.

Víde, Dómine, afflictiónem pópuli tui\\ Et mitte quem missúrus es\\ Emítte Agnum dominatórem terrae\\ De pétra desérti\\ ad montem fíliae Sion\\ Ut áuferat ipse jugum\\ captivitátis nostrae.

Consolámini, consolámini, pópule meus\\ Cito véniet salus tua\\ Quare moeróre consúmeris,\\ quia innovávit te dolor?\\ Salvábo te, noli timére\\ Ego énim sum Dóminus Deus túus\\ Sánctus Israël, Redémptor túus.\strut


Derramad, oh cielos, vuestro rocío de lo alto,\\ y las nubes lluevan al Justo.

No te enfades, Señor,\\ ni te acuerdes de la iniquidad.\\ He aquí que la ciudad del Santuario\\ quedó desierta:\\ Sión quedó desierta;\\ Jerusalén está desolada.\\ La casa de tu santidad y de tu gloria,\\ donde nuestros padres te alabaron.

Pecamos y nos volvimos\\ como los inmundos,\\ Y caímos, todos, como hojas.\\ Y nuestra iniquidades, como un viento,\\ nos dispersaron.\\ Ocultaste de nosotros tu rostro\\ Y nos castigaste\\ por mano de nuestras iniquidades.

¡Mira, Señor, la aflicción de tu pueblo!,\\ Y envíale a Aquel que vas a enviar!\\ Envíale al Cordero dominador de la tierra\\ Del desierto de piedra\\ al monte de la hija de Sión\\ Para que Él retire el yugo\\ de nuestro cautiverio.

Consuélate, consuélate, pueblo mío,\\ ¡En breve ha de llegar tu salvación!\\ ¿Por qué te consumes en la tristeza,\\ por qué tu dolor?\\ ¡Yo te salvaré, no tengas miedo!\\ Porque Yo soy el Señor, tu Dios,\\ El Santo de Israel, tu Redentor.\strut


\newsection

	\section{Homilías}

		\subsection{San Juan Pablo II, papa}

			\subsubsection{Homilía (1980): Nueva llamada a vestirse de Cristo}
			
				\begin{body}
			
					\src{Visita Pastoral a la Parroquia de San Leonardo de Porto Mauricio. \\30 de noviembre de 1980.}
					
					1. Al escuchar las palabras del Evangelio de hoy según Mateo, ante nuestros ojos vienen espontáneamente a la memoria los acontecimientos que durante la semana pasada han sacudido a toda Italia\footnote{Se refiere a un gran terremoto que afectó las regiones italianas de Campania y de Basilicata, desde Potenza a Avellino, hasta el litoral, a los Puertos de Nápoles y Salerno}.\ldots{} Mientras nosotros todos, con espíritu de solidaridad humana, queremos ayudar a nuestros hermanos y compatriotas, arrollados por la desgracia, al mismo tiempo, estos acontecimientos traen ante nuestros ojos, con una particular fuerza comparativa, el cuadro terrible que cada año trazan los \textbf{Evangelios} de este primer domingo de Adviento: anuncios de destrucción y de muerte, en la espera escatológica de la \textquote{venida del Hijo del Hombre} (Mt 24, 39). 
					
					2. La historia de los hombres y de las naciones, la historia de toda la humanidad suministra pruebas suficientes para afirmar que en todos los tiempos se han multiplicado desgracias y catástrofes, calamidades naturales, como terremotos, o las causadas por el hombre, como guerras, revoluciones, estragos, homicidios y genocidios. Además, cada uno de nosotros sabe que nuestra existencia terrena lleva a la muerte, llegando así un día a su término. El mundo visible, con todos los bienes y las riquezas que oculta en sí mismo, al fin no es capaz de darnos más que la muerte: el término de la vida. 
					
					Esta verdad, aunque nos la recuerda también la liturgia de hoy, primer domingo de Adviento, sin embargo, no es la verdad específica anunciada en este día festivo, y en todo el período de Adviento. No es la palabra principal del \textbf{Evangelio}. 
					
					¿Cuál es, pues, la palabra principal? La hemos leído hace poco: la venida del Hijo del Hombre. La palabra principal del Evangelio no es \textquote{la separación}, \textquote{la ausencia}, sino \textquote{la venida} y \textquote{la presencia}. Ni siquiera es la \textquote{muerte}, sino la \textquote{vida}. El Evangelio es la Buena Noticia, porque pronuncia la verdad sobre la vida en el contexto de la muerte. 
					
					La venida del Hijo del Hombre es el comienzo de esta Vida. Y de este comienzo nos habla precisamente el Adviento, que responde a la pregunta: ¿cómo debe vivir el hombre en el mundo con la perspectiva de la muerte? El hombre al que, en un abrir y cerrar de ojos, le puede ser quitada la vida, ¿cómo debe vivir en este mundo, para encontrarse con el Hijo del Hombre, cuya venida es el comienzo de la nueva vida, de la vida más potente que la muerte? 
					
					4. Nos encontramos, pues, todos en el primer domingo de Adviento. ¿Cuál es esta verdad que nos penetra y vivifica hoy? ¿Qué mensaje nos anuncia la Santa Iglesia, nuestra Madre? Como ya he dicho, no se trata de un mensaje de miedo y de muerte, sino del mensaje de la esperanza y de la llamada. 
					
					Tomemos como ejemplo la \textbf{segunda lectura}; he aquí lo que el Apóstol Pablo dice a los romanos de entonces, pero que debemos tomar en serio los romanos de hoy: \textquote{Daos cuenta del momento en que vivís; ya es hora de espabilarse, porque ahora nuestra salvación está más cerca que cuando empezamos a creer. La noche está avanzada, el día se echa encima} (Rm 13, 11-12). 
					
					En realidad, al contrario de como podemos ser inducidos a pensar, la salvación está más cercana y no más lejana. Efectivamente, al vivir en una época de secularización, somos testigos de comportamientos de indiferencia religiosa y también de programas e ideologías ateas o incluso antiteístas. Se llegaría a pensar que los indicios humanos desmienten el mensaje de la liturgia de hoy. Ella, en cambio ---aun haciendo referencia también a estos \textquote{indicios humanos}--- proclama, sin embargo, la verdad divina y anuncia el designio divino que no decae jamás, que no cambia aun cuando puedan cambiar los hombres, los programas, los proyectos humanos. Ese designio divino es el designio de la salvación del hombre en Cristo, que, una vez emprendido, perdura, y consiguientemente mira a su cumplimiento. 
					
					Pero el hombre puede ser ciego y sordo a todo esto. Puede meterse cada vez más profundamente en la noche, aunque se acerque el día. Puede multiplicar las obras de las tinieblas aunque Cristo le ofrezca \textquote{las armas de la luz}. 
					
					Por lo tanto, la invitación apremiante de la liturgia de hoy es la del \textbf{Apóstol}: \textquote{Vestíos del Señor Jesucristo} (Rm 13, 14). Esta expresión es, en cierto sentido, la definición del cristiano. Ser cristiano quiere decir \textquote{vestirse de Cristo}. El Adviento es la nueva llamada a vestirse de Jesucristo. 
					
					Dice además el Apóstol: \textquote{Conduzcámonos como en pleno día, con dignidad. Nada de comilonas ni borracheras, nada de lujuria ni desenfreno, nada de riñas ni pendencias\ldots{}, y que el cuidado de vuestro cuerpo no fomente los malos deseos} (Rm 13, 13-14). 
					
					5. ¿Qué significa, además, el Adviento? El Adviento es el descubrimiento de una gran aspiración de los hombres y de los pueblos hacia la casa del Señor. No hacia la muerte y la destrucción, sino hacia el encuentro con Él. 
					
					Y por esto en la liturgia de hoy oímos esta invitación: \textquote{Qué alegría cuando me dijeron: vamos a la casa del Señor}. 
					
					Y el mismo \textbf{Salmo responsorial} nos traza, por decirlo así, la imagen de esa casa, de esa ciudad, de ese encuentro: \textquote{Ya están pisando nuestros pies tus umbrales, Jerusalén. Allá suben las tribus, las tribus del Señor. Según la costumbre de Israel, a celebrar el nombre del Señor. En ella están los tribunales de justicia en el palacio de David. Por mis hermanos y compañeros voy a decir: \textquote{La paz contigo}. Por la casa del Señor nuestro Dios, te deseo todo bien} (Sal 121 {[}122{]}). 
					
					Sí. El Señor es el Dios de la paz, es el Dios de la Alianza con el hombre. Cuando en la noche de Belén los pobres pastores se pondrán en camino hacia el establo donde se realizará la primera venida del Hijo del Hombre, los conducirá el canto de los ángeles: \textquote{Gloria a Dios en las alturas y paz en la tierra a los hombres de buena voluntad} (Lc 2, 14). 
					
					6. Esta visión de la paz divina pertenece a toda la espera mesiánica en la Antigua Alianza. Oímos hoy las palabras de \textbf{Isaías}: \textquote{Será el árbitro de las naciones, el juez de pueblos numerosos. De las espadas forjarán arados; de las lanzas, podaderas. No alzará la espada pueblo contra pueblo, no se adiestrarán para la guerra. Casa de Jacob, ven; caminemos a la luz del Señor} (Is 2, 4-5). 
					
					El Adviento trae consigo la invitación a la paz de Dios para todos los hombres. Es necesario que nosotros construyamos esta paz y la reconstruyamos continuamente en nosotros mismos y con los otros: en las familias, en las relaciones con los cercanos, en los ambientes de trabajo, en la vida de toda la sociedad. 
					
					Trabajad con espíritu de solidaridad fraterna a fin de que vuestra parroquia crezca cada vez más como comunidad de fieles, de familias, de grupos ---me refiero particularmente a todos vuestros grupos organizados--- en comunión de verdad y de amor. La comunidad parroquial, en efecto, se edifica sobre la Palabra de Dios, transmitida y garantizada por los Pastores, se alimenta por la gracia de los sacramentos, se sostiene por la oración, se une por el vínculo de la caridad fraterna. Que cada uno de sus miembros se sienta vivo, activo, partícipe, corresponsable, implicado en tareas efectivas de evangelización cristiana y de promoción humana. De este modo, vuestra parroquia se convierte en signo e instrumento de la presencia de Cristo en el barrio, irradiación de su amor y de su paz. 
					
					Para servir a esta paz de múltiples dimensiones, es necesario escuchar también estas palabras del \textbf{Profeta}: \textquote{Venid, subamos al monte del Señor, a la casa del Dios de Jacob. El nos instruirá en sus caminos y marcharemos por sus sendas, porque de Sión saldrá la ley. de Jerusalén la palabra del Señor} (Is 2, 3). 
					
					También para vuestra comunidad eclesial el Adviento es el tiempo en el que se deben aprender de nuevo la ley del Señor y sus palabras. Es el tiempo de una catequesis intensificada. La ley y la palabra del Señor deben penetrar de nuevo en el corazón, deben encontrar de nuevo su confirmación en la vida social. Sirven al bien del hombre. ¡Construyen la paz! 
					
					7. Queridos hermanos e hijos: Nos encontramos, pues, de nuevo al comienzo del camino. Ha comenzado de nuevo el Adviento: el tiempo de la gracia, el tiempo de la espera, el tiempo de la venida del Señor, que perdura siempre. Y la vida del hombre se desarrolla en el amor del Señor, a pesar de todas las dolorosas experiencias de la destrucción y de la muerte, hacia la realización final en Dios. 
					
					¡El Hijo del Hombre vendrá! Escuchemos estas palabras con la esperanza, no con el miedo, aunque estén llenas de una profunda seriedad. 
					
					Velad\ldots{} y estad preparados, porque no sabéis en qué día vendrá el Hijo del Hombre. ¡Ven, Señor Jesús! ¡Marana tha!
				\end{body}
				

			\subsubsection{Homilía (1983):}

				\src{Visita Pastoral a la Parroquia Romana de San Felipe Neri. \\27 de noviembre de 1983.}
				
				\begin{body}
					1. \textquote{Comportaos reconociendo el momento en que vivís, pues ya es hora de despertaros del sueño} (\emph{Rom} 13, 11). 
					
					Con estas palabras, queridos hermanos y hermanas, la liturgia de hoy se dirige a cada uno de nosotros, enseñándonos a acoger la llamada que nos llega desde el comienzo del Adviento. Despertar del sueño significa abrir el corazón a esa realidad divina que está ligada al tiempo humano. Por eso se dice: \textquote{la salvación está más cerca}. 
					
					El Adviento es como una primera dimensión de esta unión de la Realidad divina al tiempo humano. Este vínculo se refleja en el año litúrgico: el domingo primero de Adviento es al mismo tiempo el comienzo del nuevo año litúrgico. 
					
					{[}2. Al mismo tiempo comenzamos el Año Santo de la Redención. Este extraordinario Jubileo de la Redención tiene un carácter específico de \textquote{adviento}: nos prepara para el tercer milenio después de Cristo. De ahí la particular elocuencia del Adviento de este año, que debe expresar esa actitud de la Iglesia, de la que ya hablé en la Bula de Indicación (Juan Pablo II, \emph{Aperite portas Redemptori}, 7), por la que \textquote{se siente especialmente comprometida con la fidelidad a los dones divinos, que tienen su origen en la redención de Cristo, y por medio de los cuales el Espíritu Santo la guía a su desarrollo y renovación, para que pueda siempre convertirse en una esposa más digna de su Señor. Para ello confía en el Espíritu Santo y quiere asociarse a su acción misteriosa de Esposa que invoca la venida de Cristo} (cf. \emph{Ap} 22, 17). 
					
					Este carácter particular de \textquote{adviento} propio del presente Año Santo, debe ser vivido por la Iglesia \textquote{con los mismos sentimientos con los que la Virgen María esperaba el nacimiento del Señor en la humildad de nuestra naturaleza humana. Así como María precedió a la Iglesia en la fe y en el amor en los albores de la era de la Redención, hoy la precede mientras, en este Jubileo, avanza hacia el nuevo milenio de la Redención} (Juan Pablo II, \emph{Aperite portas Redemptori}, 9).{]} 
					
					3. \textquote{Reconocer el momento en que vivimos}: ¿qué significa? \textquote{Vamos alegres al encuentro del Señor}. 
					
					El Adviento es la perspectiva gozosa de \textquote{ir a la casa del Señor} (cf. \emph{Sal} 121, 1): de llegar al final de esta gran \textquote{peregrinación} que debe ser la vida terrena. El hombre está llamado a habitar en la \textquote{casa del Señor}. Allí está su verdadero \textquote{hogar}. {[}La peregrinación del Año Santo es una figura de nuestro camino hacia la casa del Padre,{]} y el Adviento nos estimula a acelerar este camino con esperanza. 
					
					El Adviento aguarda el \textquote{día del Señor}, es decir, la \textquote{hora de la verdad}. Es la expectativa de ese día en que \textquote{Él será juez entre las naciones y árbitro entre todos los pueblos} (\emph{Is} 2, 4). Esta plenitud de verdad será el principio y fundamento de la paz definitiva y universal, que es el objeto de la esperanza de todos los hombres de buena voluntad. 
					
					El Adviento es una confirmación del camino eterno del hombre hacia Dios; es un nuevo comienzo, cada año, de este camino: ¡la vida del hombre no es un camino infranqueable, sino un camino que conduce al encuentro con el Señor! Hay también en esta invocación del primer domingo de Adviento casi un anticipo de esos caminos que en la noche de Belén conducirán a los pastores y a los Reyes Magos de Oriente hacia Jesús el recién nacido. 
					
					4. \textquote{Reconocer el momento en que vivimos}: ¿qué significa? \textquote{Revestirnos del Señor Jesucristo} (\emph{Rm} 13, 14): 
					
					- el camino del Adviento conduce al interior del hombre, que de diversas formas está cargado de pecado, como atestigua la \textbf{segunda lectura}; 
					
					- el encuentro antes mencionado no sólo se realiza \textquote{desde fuera}, sino también \textquote{desde dentro}, y consiste en una transformación tal del interior del hombre, que corresponde a la santidad de Aquel con quien se encuentra: en eso consiste precisamente \textquote{vestirse del Señor Jesucristo}; 
					
					- el sentido \textquote{histórico} del Adviento está penetrado por el sentido \textquote{espiritual}. De hecho, el Adviento no quiere ser sólo el recuerdo del período histórico que precedió al nacimiento del Salvador, aunque, así entendido, ya tiene en sí mismo un altísimo significado espiritual. Sin embargo, más allá de eso, y de manera más profunda, el Adviento quiere recordarnos que toda la historia del hombre y de cada uno de nosotros debe entenderse como un gran \textquote{Adviento}, como una expectativa, momento a momento, de la venida del Señor, para que nos encuentre preparados y vigilantes para poder acogerlo dignamente. 
					
					5. \textquote{Reconocer el momento en que vivimos}: significa: \textquote{Velad \ldots{} porque no sabéis en qué día vendrá vuestro Señor} (\emph{Mt} 24, 42): 
					
					- la unión de Dios, de la realidad divina, con el tiempo humano, por un lado, reafirma la limitación de este tiempo, que tiene fin y, por otra parte, abre este tiempo a la eternidad de Dios y a las \textquote{realidades últimas} relacionadas con él; 
					
					- el Adviento tiene un significado \textquote{escatológico} ya que recuerda nuestros pensamientos y nuestras intenciones hacia realidades futuras. Nos recuerda el fin último de nuestro camino, y nos estimula a involucrarnos en las realidades terrenas sin dejarnos sumergir en ellas, sino al contrario, guiándolas hacia las celestiales; nos exhorta a prepararnos bien para esto último, para que la venida del Señor no nos encuentre desprevenidos y mal dispuestos; 
					
					- \textquote{Vigilidad}: el espíritu del hombre \textquote{despertado} a la realidad divina, atraído por ella a sus destinos eternos en Dios, debe animar toda temporalidad con una nueva conciencia. 
					
					6. Queridos hermanos y hermanas {[}de la parroquia de San Felipe Neri{]}, mi afectuoso saludo va para todos vosotros, ya que agradezco al Señor por haberme concedido este encuentro. (\ldots{}) 
					
					{[} \ldots{}{]} 
					
					Una comunidad parroquial unida y ferviente puede jugar, con la fuerza del Espíritu Santo, un papel esencial en la reducción de la distancia entre el modelo evangélico que propone al mundo y las condiciones reales del mundo mismo, siempre en cierta medida refractario, mientras estemos aquí abajo, a la llamada evangélica, a la conversión y a la penitencia. Este hecho, sin embargo, lejos de debilitar el testimonio que se da al mundo, debe fortalecerlo cada vez más, en la muy firme convicción de que el mundo, a pesar de todo, tiene una necesidad absoluta de Jesús crucificado y resucitado. El poder de su gracia, especialmente a través del carisma de los laicos cristianos, puede y debe penetrar y animar evangélicamente todos los ambientes seculares de la familia y el trabajo, la escuela, la sociedad y la cultura. 
					
					7. En el primer domingo de Adviento (\ldots{}) espero sinceramente que esta {[}celebración{]} nos permita abrir más los ojos del alma a la realidad divina y, por así decirlo, volver a despertar a ella. Que nos permita transformarnos interiormente y que al mismo tiempo nuestra humanidad se revista del Señor Jesucristo de forma cada vez más madura. 
					
					Con nueva alegría, avancemos hacia el encuentro con el Señor que debe venir, como cada año, en la solemnidad de la Navidad; hacia el Señor con quien también debemos encontrarnos al final de nuestros caminos terrenales. De hecho, el Adviento nos recuerda, cada año, que la vida humana no es un camino infranqueable hacia Dios, sino un verdadero camino que Él mismo ha hecho suyo por la Palabra divina.
				\end{body}
			
			\subsubsection{Homilía (1986):}
			
				\src{Santa Misa en el Parque Victoria, Adelaide (Australia), \\30 de noviembre de 1986.}
				
				\begin{body}
					\textquote{\emph{¡Qué alegría cuando me dijeron: \textquote{Vamos a la casa del Señor!}}} (\emph{Sal} 122 (121), 1). 
					
					\emph{Amados hermanos y hermanas en Cristo}. 
					
					1. Con estas palabras toda la Iglesia proclama la alegría del Adviento. Hoy es el primer domingo de Adviento. Nos acercamos a la noche en que los pastores, en los campos alrededor de Belén, sintieron la alegría de ser llamados por los ángeles para ir a ver al Señor: \textquote{Vamos a Belén, veamos este acontecimiento que el Señor nos ha dado a conocer} (\emph{Lc} 2, 15). Sí, incluso hoy, aquí (\ldots{}), la Iglesia nos recuerda que el Señor está cerca (cf. \emph{Fil} 4, 5). Y como los pastores en aquella noche espléndida en Belén, también nosotros decimos: \textquote{Iremos a la casa del Señor}.
					
					El Adviento es el tiempo de preparación para la Navidad, para la venida del Salvador. Nos llama a \textquote{ir con gozo a la casa del Señor \ldots{} a alabar allí el nombre del Señor} (\emph{Sal} 121, 4). Alabar el nombre del Padre Altísimo, del Hijo y del Espíritu Santo: esta es la primera intención de nuestra celebración eucarística.
					
					2. \textquote{¡Qué alegría cuando me dijeron: \textquote{vamos a la casa del Señor}!}. Con la misma alegría que nos comunica la liturgia de Adviento os saludo a todos vosotros aquí reunidos (\ldots{}) caminando con vosotros hacia la casa del Señor, por los caminos de ese Adviento que es la historia del hombre: el Adviento en el que toda la familia humana y todos los creados esperan la segunda venida de nuestro salvador Jesucristo.
					
					{[}Solo unos pocos años nos separan del final del segundo milenio y el comienzo del tercer milenio de la era cristiana.{]} Este es un tiempo de gracia para la Iglesia. Es un tiempo en el que los seguidores de Jesucristo, en medio de las profundas transformaciones que cambian la cultura y la sociedad, debemos volver a dedicarnos a la vida cristiana. Es un momento en el que el mensaje del Evangelio debe ser proclamado a los hombres y mujeres de esta época con la fuerza de un nuevo Pentecostés. Es un tiempo en el que el mismo Espíritu de verdad habla con claras palabras de vida a la familia humana.
					
					3. En la celebración eucarística de este primer domingo de Adviento, oremos por la realización del proyecto del Padre para la familia humana: \textquote{En ese mundo nuevo donde se nos revelará la plenitud de la paz, reúne también a los hombres de toda raza y lengua, en el banquete de la unidad eterna, en los cielos y en la tierra nueva, donde brille la plenitud de tu paz} (II \emph{Plegaria Eucarística de Reconciliación}).
					
					En otras palabras, oremos para que se haga realidad la visión del profeta Isaías, como decía la \textbf{primera lectura}: \textquote{Todas las naciones confluirán al monte del Señor \ldots{} para que nos muestre sus caminos y caminemos por sus sendas} (\emph{Is} 2, 2-3). El deseo de ese tiempo de gracia y paz está profundamente arraigado en nuestros corazones. ¿Quién no anhela que llegue ese tiempo final cuando \textquote{un pueblo ya no alzará la espada contra otro pueblo, ya no practicará el arte de la guerra}? (\emph{Is} 2, 4). De hecho, existe un tiempo de Adviento que es universal y que dura tanto como la historia humana. Hoy meditamos en la visión de \textbf{Isaías} sobre un número incalculable de personas que marchan hacia el monte del Señor: el pueblo de Dios de todas las épocas y de todos los lugares que se reúnen en unión con él y en la unidad entre ellos mismo en la Iglesia. Y deberíamos reflexionar sobre cómo esta visión se materializa en la realidad concreta de la vida {[}actual{]}, en la historia y en la cultura (\ldots{}). Esta asamblea eucarística en sí misma es un símbolo de la visión del profeta. Sois personas aquí reunidas de \textquote{toda raza, lengua y forma de vida}, hechos uno en Jesucristo y en su Iglesia.
					
					{[} \ldots{}{]}
					
					5. La \textbf{palabra de Dios} nos ha llamado a estar atentos y vigilantes, a revestirnos de las armas de Jesucristo: \textquote{Ha llegado el momento de despertar del sueño \ldots{} La noche está avanzada, el día está cerca: dejemos, pues, las obras de las tinieblas y pongámonos las armas de la luz. Andemos como en pleno día, con dignidad. Nada de comilonas y borracheras, nada de lujuria y desenfreno, nada de riñas y envidias. Revestíos más bien del Señor Jesucristo}. (\emph{Rom} 13, 11-14).
					
					Cualquier expresión de hostilidad hacia los demás levanta un muro de tensión entre las personas y revela un corazón de piedra. Cualquier acto de discriminación es un acto de injusticia y una violación de la dignidad de la persona. Siempre que somos intolerantes, cerramos los ojos a la imagen de Dios que está en la otra persona. Hoy, cuando no reconocemos las necesidades de justicia en el mundo, no captamos el sentido de solidaridad universal. Pero cuando hablamos con palabras amables, cuando nos respetamos y nos honramos mutuamente, cuando mostramos una verdadera amistad, cuando ofrecemos hospitalidad, cuando nos esforzamos por comprender las diferencias entre los pueblos, nos convertimos en el signo vivo de que la visión de Isaías se ha hecho realidad, que el reino de Dios ha llegado entre nosotros, que el advenimiento universal de la historia avanza hacia su cumplimiento.
					
					6. La Iglesia hoy nos invita a todos y cada uno de nosotros a emprender con gusto y alegría el camino que Dios ha preparado para todo el género humano. El \textbf{profeta Isaías} habla del camino que sube al monte del Señor, al templo del Dios de Jacob (cf. \emph{Is} 2, 3). Parte de este \textquote{ascenso} es la vocación del hombre a buscar una humanidad plena y auténtica, a perfeccionar y desarrollar sus propias cualidades espirituales y físicas en la lucha por dominar el mundo, a través del progreso del conocimiento y mediante su propio esfuerzo. Esto es lo que hace la familia humana a través del progreso cultural (cf. \emph{Gaudium et Spes}, 53).
					
					Los hombres y mujeres de hoy saben claramente que, hoy más que nunca, están llamados a construir su propio destino en este mundo. Los medios de que disponen son cada vez mayores: un mejor conocimiento de la tierra y sus secretos, un mejor conocimiento de la persona humana y de la actividad humana; una mejor comprensión del curso de la historia y la organización social; y el mundo de las comunicaciones, que brinda a cada vez más personas la oportunidad de participar en el progreso moderno. Un mundo más humano está luchando por nacer. Y, sin embargo, cada vez las mayores esperanzas van acompañadas de inquietantes contradicciones. En lo que respecta al respeto de los derechos humanos fundamentales, las últimas décadas han sido testigos de grandes avances y de una creciente conciencia de la justicia de esta causa. Sin embargo, no podemos pasar por alto el hecho de que nuestro mundo todavía ofrece demasiados ejemplos de gran injusticia y opresión. Donde hay un gran bien por alcanzar, se necesita una gran madurez moral y un gran sentido de la justicia. Sin la visión de la sublime dignidad de la persona humana --- dignidad fundada en la relación única y personal con el Creador y Redentor, dignidad ligada a la naturaleza trascendente, al origen y destino del hombre --- el progreso no tendrá un rumbo seguro. Jesucristo, camino, verdad y vida, nos revela el verdadero sentido de la historia. Nos revela el plan de Dios para la humanidad. Jesús habla de nuestra libertad y nos llama a promover el verdadero progreso humano, dándonos su ley de amor y servicio; \textquote{Este es mi mandamiento: amaos los unos a los otros como yo os he amado} (\emph{Jn} 15, 12). El Evangelio purifica y fortalece toda cultura, para que pueda ayudar al hombre \textquote{a subir al monte del Señor \ldots{} para que nos muestre sus caminos y recorramos sus sendas} (\emph{Is} 2, 3).
					
					7. El llamado de Jesús es claro. Él dice: \textquote{Velad}. Y de nuevo: \textquote{Estad preparados, porque a la hora que no imaginéis, vendrá el Hijo del Hombre} (\emph{Mt} 24, 42, 44). Así exhorta a todos sus seguidores a trabajar por la meta que el Padre se ha marcado: el reino de la justicia, la verdad y la paz. Por lo tanto, insta a los fieles (\ldots{}) a encontrar un remedio cuando las injusticias puedan dañar la vida de su nación y a garantizar que un nuevo espíritu de reconciliación anime toda la vida nacional. Jesús nos dice que seremos juzgados por la forma en que respondamos a su presencia en los hambrientos, desnudos, enfermos y presos (cf. \emph{Mt} 25, 35-36).
					
					Queridos hermanos y hermanas: estáis llamados a participar con Dios en la construcción de su reino en el corazón de todos (\ldots{}), corazones de carne y no de piedra. {[}En este día estamos invitados{]} a ver nuestra historia en el contexto del amor eterno de Dios por toda la familia humana, que se manifiesta en la misión salvífica de Jesucristo. Es una historia que aún está evolucionando. Y presenta muchos desafíos (\ldots{}). Este también es el Adviento, lleno de expectación, que la Iglesia celebra en este período. Nosotros, pueblo peregrino de Dios, caminamos siguiendo a Jesús, que es el camino al Padre. Caminamos con la certeza de que su verdad nos hará libres y que nuestra fuerza proviene de sus palabras y sus sacramentos.
					
					8. Con la mirada puesta en Aquel que ha de venir, miremos, pues, a todos (\ldots{}) e invoquemos la bendición del Salmo: \textquote{Haya paz dentro de tus muros, seguridad en tus palacios \ldots{} Por mis hermanos y compañeros, voy a decir: \textquote{La paz contigo}. Por la casa del Señor, nuestro Dios, te deseo todo bien}. (\emph{Sal} 121, 7-9).
					
					{[}Para ti: Adelaide! Para ti: ¡Australia! Para ti: ¡mundo entero!{]} 
					
					Y seguimos mirándole a Él, el que ha de venir, el \textquote{Príncipe de la paz} (cf. \emph{Is} 9, 6). El \textbf{profeta} dice de él: 
					
					Juzgará entre las naciones, 
					
					será árbitro de pueblos numerosos. 
					
					De las espadas forjarán arados, 
					
					de las lanzas, podaderas. 
					
					No alzará la espada pueblo contra pueblo, 
					
					no se adiestrarán para la guerra. 
					
					Casa de Jacob, venid; 
					
					caminemos a la luz del Señor. 
					
					(\emph{Is} 2, 4-5). 
					
					Esta es la luz del Adviento. Es la luz del Adviento que se despliega ante la familia humana, hasta que el Señor vuelva en gloria: el Adviento de la responsabilidad del hombre por la vida y por el mundo que el Creador ha puesto en sus manos. La luz es la luz del que vendrá, el príncipe de paz, es la luz de Cristo. ¡Que la luz de Cristo brille sobre {[}nuestra tierra{]} (\ldots{})! Que la luz de Cristo brille sobre cada uno de vosotros. Amén. 
				\end{body} 
			
			\subsubsection{Homilía (1989):}
			
			\src{Visita Pastoral a la Parroquia de Santo Tomás Apóstol en Castel Fusano. \\ 3 de diciembre de 1989.}
			
			\begin{body}
				1. \textquote{Vayamos gozosos al encuentro del Señor}. 
				
				Estas palabras del \textbf{salmo responsorial}, que hemos repetido juntos, pueden considerarse con razón el programa de la Iglesia al comienzo del año litúrgico; especialmente al comienzo del Adviento, que constituye la primera etapa importante del año litúrgico. 
				
				Entramos conscientemente en un tiempo \textquote{propicio} de salvación: la comunidad cristiana, de hecho, mientras se prepara para conmemorar la primera venida de Cristo \textquote{en la humildad de nuestra naturaleza humana} (\emph{Prefacio del Tiempo de Adviento}), es llamada a orientarse, con esperanza confiada y expectación activa, hacia la venida definitiva del Señor \textquote{en el esplendor de su gloria} (\emph{Prefacio del Tiempo de Adviento}). Todo ello en dócil apertura de fe a la Palabra de Dios, que ilumina nuestro camino y con una participación activa en los acontecimientos sacramentales, que nos insertan en el misterio de Cristo, \textquote{hasta que él venga} (cf. \emph{1 Co} 4, 5). 
				
				En el momento en que la Iglesia emprende el itinerario salvífico del año litúrgico, la Palabra de Dios, recién escuchada, pone inmediatamente ante nosotros la meta hacia la que nos dirige el Espíritu: la nueva Jerusalén, símbolo de la comunión plena y definitiva a la que Dios invita y admite a todos aquellos que dicen el \textquote{sí} de la fe a Cristo, \textquote{maestro de la verdad y fuente de reconciliación} (\emph{Oración colecta}), y se hacen heraldos y testigos entre los hermanos, para que el Dios de la paz sea en todo en todos cuando \textquote{venga el Hijo del Hombre} (\emph{Evangelio}). 
				
				2. Entonces entendemos las palabras del \textbf{profeta}: \textquote{Casa de Jacob, ven, caminemos a la luz del Señor} (\emph{Is} 2, 5). 
				
				Esta invitación resuena hoy con acentos de especial relevancia y urgencia para {[}todos nosotros{]} (\ldots{}) Todos debemos emprender un camino de conversión, superando la tentación de la inercia, la desconfianza y la pasividad. 
				
				Muchos, de hecho, que también profesan ser cristianos, viven en una especie de letargo y mediocridad. A menudo, su vida moral contrasta con la fe, que también dicen tener; no pocos limitan su pertenencia a Cristo y a la Iglesia a una práctica religiosa ocasional, compuesta sólo por citas religiosas ocasionales; se alejan de las responsabilidades específicas, contentándose con delegar en los demás la misión evangelizadora propia de cada miembro activo de la comunidad eclesial. 
				
				3. A todos ellos quiero repetir el llamamiento que \textbf{san Pablo} dirigió a los primeros fieles de la Iglesia de Roma, y ​​que la liturgia nos ha vuelto a proponer hoy: \textquote{Hermanos, es hora de despertarnos del sueño. , porque nuestra salvación está más cerca que cuando empezamos a creer. La noche está avanzada, el día está cerca. Desechemos las obras de las tinieblas y vistámonos las armas de la luz} (\emph{Rom} 13, 11-12). 
				
				Ante la situación de indiferencia e imprevisibilidad que describe el Evangelio y que se refleja en la mentalidad y las costumbres de hoy, la Iglesia (\ldots{}) {[}en su camino hacia el final del segundo milenio de la era cristiana{]}, no puede ni debe permanecer inerte y pasiva: está impulsada por la voluntad misma de Dios, reconocida y aceptada en unión con su propio Obispo, a tomar la decisión fundamental de renovarse en la vocación y misión que le ha confiado la Providencia. 
				
				El Vaticano II volvió a proponer la verdad sobre Dios y el hombre con nueva luz. Aceptar esta luz, dejarse penetrar por ella más íntimamente, anunciarla a todos es un deber de todo cristiano, para que la salvación en Cristo Jesús, que se ofrece a todos, esté más cerca de cada uno. Es un viaje para hacer \textquote{juntos}. Nadie puede ignorarlo o rehuirlo (\ldots{}). El mundo, y en particular los hombres que viven en esta ciudad, no podrán ver y recibir la luz de Cristo si sus discípulos son opacos, tibios y pasivos. 
				
				{[} \ldots{}{]} 
				
				5. Queridos hermanos y hermanas, quisiera hacer una última reflexión con ustedes, que se me ha sugerido desde el comienzo del año litúrgico.  
				
				Los obispos italianos han afirmado en varias ocasiones que el año litúrgico \textquote{constituye el gran camino de fe del pueblo de Dios: toda la comunidad, especialmente en tiempos de gran fuerza, está llamada a redescubrir, celebrar y vivir el don de la salvación. A través de la pedagogía de los ritos y de la oración, todos somos conducidos juntos a la experiencia del misterio pascual de Cristo, que tiene su centro en la Eucaristía} (Conferencia Episcopal Italiana, \emph{Eucaristía, comunión y comunidad}, 89; cf. etiam Eiusdemn \emph{Evangelización y sacramentos}, 85). 
				
				{[}Estamos llamados a establecer{]} en nuestras comunidades itinerarios educativos serios y continuos, en los que se fundan juntos la escucha de la Palabra de Dios, la celebración de los santos misterios y el testimonio, el servicio de la caridad y la promoción del hombre, superando así la fragmentación y el ocasionalismo por un lado y, por otro, el impulso de avanzar por caminos paralelos o divergentes. 
				
				El año litúrgico es uno de estos itinerarios, de hecho es el privilegiado de la Iglesia, no solo porque es el más adecuado para todas las edades y diferentes categorías de personas, sino sobre todo porque es el más completo, si se vive con autenticidad y es valorado en todas las posibilidades que ofrecen sus diferentes tiempos y momentos y por la riqueza de los signos litúrgico-sacramentales. 
				
				La creatividad pastoral, en fidelidad a la Tradición genuina de la Iglesia, podrá promover, especialmente para jóvenes y adultos, otros itinerarios de fe incluidos en el año litúrgico, o en todo caso en sintonía con él, tanto con motivo de la celebración de los sacramentos y, más en general, por una vida cristiana más madura y una fe más consciente y activa. 
				
				{[}Confío estos compromisos a ese gran testigo de la fe que fue santo Tomás Apóstol, a quien está dedicada vuestra comunidad, para que toda la Iglesia (\ldots{}) profese su fe en Cristo resucitado y coseche los frutos de la salvación, para sí misma y para los hombres que viven en la ciudad. 
				
				\textquote{Casa de Jacob, ven \ldots{}}; {[}parroquia de Santo Tomás Apóstol en Castel Fusano{]}, \textquote{ven, caminemos en la luz del Señor}. 
				
				¡Amén!
			\end{body}
		
			\subsubsection{Homilía (1992):}
			
				\src{Visita Pastoral a la Parroquia de San Gerardo Mayela. \\29 de noviembre de 1992.}
				
				\begin{body}
					1. \textquote{\emph{Vayamos gozosos al encuentro del Señor}} (Salmo responsorial). 
					
					¡Queridos hermanos y hermanas (\ldots{})! Hoy, primer domingo de Adviento, comienza un nuevo año litúrgico durante el cual la Iglesia recorre y revive espiritualmente las etapas del misterio cristiano. 
					
					Este plan divino abarca toda la historia de la humanidad, desde los albores de la creación hasta el día final, cuando todas las cosas serán recapituladas en Cristo (cf. \emph{Ef} 1, 10) y habrá nuevos cielos y nueva tierra (cf. \emph{2 Pt.} 3, 13). El centro de este proyecto está en el misterio de la encarnación del Hijo de Dios. 
					
					En un momento preciso, por obra del Espíritu Santo, el Verbo \textquote{se hizo carne} en el seno virginal de María y \textquote{vino a habitar en nosotros}. (cf. \emph{Jn} 1, 12), mostrando la bondad y humanidad de Dios para con los hombres. El Señor, de hecho, no sólo creó al hombre, sino que lo ama con tanta intensidad que lo acoge desde dentro de su propia familia, destinándolo a una gloria sin fin. 
					
					De hecho, sostenidos por una certeza tan consoladora vamos con alegría a su encuentro, como nos invita a hacer el \textbf{Salmo Responsorial}. 
					
					2. Vamos a encontrarlo en el misterio de la Navidad. Este es el primer sentido del Adviento. Nos conmueve el recuerdo de María y José, que suben de Nazaret de Galilea a Belén de Judea para el censo, y se ven obligados a refugiarse en un lugar destinado a los animales, \textquote{porque no había lugar para ellos en la posada} (\emph{Lc} 2, 7). 
					
					El Hijo de Dios viene a la luz en la pobreza total: verdadero Dios Salvador, anunciado por los ángeles a los pastores, y verdadero hombre, envuelto en pañales y colocado en un pesebre. 
					
					¡Qué sentimientos de ternura, amor y gratitud despierta este extraordinario evento! Sin embargo, también tiene la fuerza de sacudir nuestra conciencia invitándonos a despertar del sueño de la indiferencia y de la habitud. 
					
					\textquote{Hermanos --- nos exhorta el \textbf{apóstol Pablo} --- ya es hora de despertar del sueño} (\emph{Rm} 13, 11), Dios nos ha amado hasta darnos a su Hijo único. ¿Acaso un don tan grande no nos obligua a reflexionar y a responderle con la generosidad adecuada? ¿No nos empuja a abandonar las tinieblas del pecado para abrir el espíritu a la luz de la gracia divina? Esto es precisamente a lo que nos invitan las lecturas de la liturgia de hoy. 
					
					3. \textquote{Venid, subamos al monte del Señor} (\emph{Is} 2, 3). 
					
					El texto, tomado del libro del \textbf{profeta Isaías}, se interpreta comúnmente como un anuncio mesiánico. Para el pueblo de Israel forzado al exilio, el profeta predice la reconstrucción del Templo de Jerusalén. Pero sus palabras van más allá de la historia del pueblo judío. Con imágenes como la del cerro más alto de todas las montañas y con el pronóstico de la venida de innumerables naciones al templo del Dios de Jacob, se indica una nueva realidad espiritual, la del pueblo de los redimidos guiados por el Mesías prometido; y la de una nueva Alianza, que transforma profundamente la vida de los hombres, según los ideales de paz y fraternidad, convirtiendo las espadas en arado y las lanzas en podaderas. 
					
					4. ¡Queridos hermanos y hermanas! Vuestra comunidad parroquial, una de las más jóvenes de nuestra Iglesia local, está animada por estos ideales de paz y fraternidad, santidad y evangelización. (\ldots{}) 
					
					(\ldots{}) Gracias a una catequesis capilar, habéis intentado ayudar a los habitantes de todo el barrio a no retirarse al individualismo, sino a crecer como comunidad cristiana solidaria, siguiendo el ejemplo de la Iglesia de los Apóstoles y de las primeras generaciones de creyentes, a través de una atenta escucha de la Palabra, participación en la vida litúrgica de la Comunidad y un intenso esfuerzo de compartir y acogida recíproca. De esta manera, se han desarrollado experiencias comunitarias significativas y formas efectivas de catequesis para adultos, como las Comunidades Neocatecumenales. Muchos jóvenes pertenecen se reúnen también en la Acción Católica y en otros movimientos apostólicos, Grupos de matrimonios cristianos siguen un itinerario formativo común estructurado en encuentros periódicos (\ldots{}) La parroquia tampoco carece de la sensibilidad misionera que brotó del hermanamiento con una misión africana en Chad. 
					
					5. Por todo ello (\ldots{}) deseo manifestar mi viva satisfacción, reconociendo el compromiso y la generosidad que os animan. 
					
					Perseverad, queridos hermanos y hermanas, en el esfuerzo realizado. \emph{Daréis, pues, vida a una nueva evangelización} (\ldots{}). Todavía hay muchas personas que no conocen el evangelio adecuadamente y esperan el testimonio constante de nuestra vida y la proclamación gozosa de nuestra fe en Cristo. 
					
					A vosotros os encomiendo una misión tan exigente: a las familias, a los adultos, a los niños, a los ancianos y especialmente a los jóvenes, y os aseguro a cada uno de vosotros el apoyo de mi cordial oración. 
					
					Este inicio del tiempo de Adviento constituye una ocasión propicia para intensificar el ritmo de nuestra vida cristiana. 
					
					6. \textquote{Vuestro Señor vendrá} (\emph{Mt} 24, 42). 
					
					El pasaje del \textbf{Evangelio de Mateo} abarca una parte del discurso de Jesús sobre los últimos acontecimientos, que por eso se llama discurso escatológico. 
					
					Jesús anuncia su segunda venida, al final de los tiempos, y nos exhorta a estar alerta y preparados para encontrarnos con Él. Este es el segundo significado del Adviento. 
					
					Por las palabras de Jesús, contenidas en este y otros textos, sabemos con certeza que el mundo presente está destinado a terminar, que la historia humana terminará, que para cada uno habrá un juicio, seguido de una recompensa o castigo. A la luz de todo esto es importante escuchar la invitación a velar \textquote{porque no sabes en qué día vendrá tu Señor}. 
					
					7. ¡\textquote{Velad, pues}! 
					
					La vigilancia evangélica es condición para un buen uso de la vida. 
					
					Qué fácil es desperdiciar los dones divinos, distanciarse de Dios con pensamientos y comportamientos, olvidar que la vida pasa. 
					
					Las cosas temporales son frágiles y pasajeras, son útiles si se utilizan como medio para crecer en la bondad, para sanar el alma y servir al Señor y a los hermanos con amor; pero si se convierten en el objetivo principal de la vida, vacían a las personas de su núcleo más importante y las convierten en apéndices de las realidades materiales. 
					
					Vayamos al encuentro del Señor que viene con buenas obras. \textquote{La noche ha avanzado y el día está cerca} (\emph{Rom} 13, 12). El apóstol Pablo nos exhorta a desechar las obras de las tinieblas y revestirnos de las armas de la luz, vestirnos del Señor Jesús y no seguir a la carne en sus deseos desordenados. 
					
					Preparémonos con esmero para la Navidad que viene, sobre todo orientando nuestra vida hacia ese Dios con el que el último día nos encontraremos cara a cara, con amor y alegría. 
					
					\textquote{Prepárate, porque a la hora que no imaginas, vendrá el Hijo del Hombre}. 
					
					Por tanto, velad, vestíos de Cristo. 
					
					Nuestra salvación está ahora cerca. 
					
					¡Amén!
				\end{body}
		
			\subsubsection{Homilía (1995):}
		
				\src{Canonización de Eugène de Mazenod, fundador de los Misioneros Oblatos de la Inmaculada. \\3 de diciembre de 1995.}
		
				\begin{body}
					\ltr[1. ]{L}{a} venida del \emph{Hijo del Hombre es el tema del Adviento}. Así comienza el tiempo del nuevo Año Litúrgico. Ya miramos hacia la noche de Belén. Pensemos en esa venida del Hijo de Dios que ya pertenece a nuestra historia, de hecho, \emph{de una manera maravillosa la formó} como la historia de los individuos, las naciones y la humanidad. También sabemos con certeza que, después de esa venida, tenemos \emph{ante nosotros para siempre} una segunda venida del Hijo del Hombre, de Cristo. Vivimos en el segundo Adviento, en el Adviento de la historia del mundo, de la historia de la Iglesia, y en la celebración eucarística repetimos todos los días nuestra confiada esperanza de su venida.	 
					
					{[} \ldots{}{]} 
					
					3. En la liturgia de este primer domingo de Adviento comienza a hablar el \textbf{profeta Isaías}. Escucharemos la palabra inspirada de todo este tiempo. \textquote{Visión de Isaías, hijo de Amós, acerca de Judá y Jerusalén. Al final de los días, la montaña del templo del Señor se elevará en la cumbre de las montañas y será más alta que las colinas; todos los pueblos acudirán a él. Muchos pueblos vendrán y dirán: \textquote{Venid, subamos al monte del Señor, al templo del Dios de Jacob, para que él nos muestre sus caminos y podamos caminar por sus sendas}. Porque de Sion saldrá la ley, y de Jerusalén la palabra del Señor} (\emph{Is} 2, 1-3). 
					
					A la luz del Espíritu Santo, el Profeta tiene \emph{una visión universalista} y muy aguda de la salvación. Jerusalén, la ciudad ubicada en medio de Israel, Pueblo de elección divina, tiene un gran futuro por delante. Cuando el Profeta dice que \textquote{la palabra del Señor saldrá \ldots{} de Jerusalén}, ya muchos siglos antes de la venida de Cristo, anuncia el alcance de la obra mesiánica. 
					
					La mirada de Isaías \emph{enriquece nuestra conciencia del Adviento}. El que ha de venir, que debe revelarse \textquote{hasta el fin} en medio de la ciudad santa de Jerusalén, por la palabra de su Evangelio, y especialmente por su cruz y su resurrección, será enviado a todas las naciones del mundo, a toda la humanidad. Será \emph{el Ungido de Dios, el Redentor del hombre}. Su visita no duró mucho, pero la misión que él transmitió a los Apóstoles y a la Iglesia perdurará hasta el final de los siglos. Será mediador entre Dios y los hombres, y en voz alta exhortará a las naciones a la paz, invitando a todos a \textquote{forjar de sus espadas arados, de sus lanzas podaderas} (cf. \emph{Is} 2, 4). Así comienza la exhortación de Isaías, dirigida a los pueblos de toda la tierra, a que dirijan la mirada y los pasos hacia Jerusalén. 
					
					Esta exhortación tiene su eco en el \textbf{Salmo Responsorial}, \emph{canto de los peregrinos} a la Ciudad Santa. \textquote{Qué alegría cuando me dijeron: Vamos a la casa del Señor. Ya están pisando nuestros pies tus umbrales, Jerusalén. Allí suben juntas las tribus, las tribus del Señor} (\emph{Sal} 122, 1, 4). Y nuevamente: Pide la paz para Jerusalén: \textquote{la paz sea con los que te aman, la paz sea en tus muros, la seguridad en tus baluartes} (\emph{Sal} 122, 6-7). 
					
					{[} \ldots{}{]} 
					
					6. El mensaje de Adviento está unido a la venida del Hijo del Hombre, cada vez más cercano. A esta conciencia corresponde \emph{la exhortación a la vigilancia}. En el \textbf{Evangelio de San Mateo}, Jesús dice a los que le escuchan: \textquote{Velad, pues, porque no sabéis en qué día vendrá el Señor \ldots{} Por tanto, también vosotros estad preparados, porque a la hora que menos penséis vendrá el Hijo del Hombre} (\emph{Mt} 24, 42, 44). El pasaje de la \textbf{Carta de San Pablo a los Romanos} corresponde de manera excelente a esta exhortación, repetida varias veces en el Evangelio. El Apóstol nos dice cómo podemos ser \textquote{\emph{conscientes del momento}} (cf. \emph{Rm} 13,11). \emph{La espera}, volcada hacia el futuro, \emph{siempre} se nos presenta \emph{como un \textquote{momento} ya cercano y presente}. En la obra de salvación no se puede dejar nada para después. \emph{¡Cada \textquote{hora} importa!} El Apóstol escribe que \textquote{nuestra salvación está más cerca ahora que cuando empezamos a creer} (\emph{Rom} 13,11) y compara este momento presente con el amanecer, con el momento culminante del paso entre la noche y el día. 
					
					San Pablo traslada el fenómeno que acompaña al despertar de la luz del día al ámbito espiritual. \textquote{La noche está avanzada --- escribe --- el día está cerca. Desechemos, pues, las obras de las tinieblas y vistámonos de las armas de la luz} (\emph{Rom} 13,12). Después de haber llamado por su nombre las obras de las tinieblas, el Apóstol indica a qué aluden \textquote{las armas de la luz}: \textquote{\emph{vistámonos con las armas de la luz}}, es decir, \textquote{vistámonos \ldots{} del Señor Jesucristo} (\emph{Rom} 13,14). El apóstol nos invita a hacer de Jesucristo la norma de nuestra vida y de nuestras acciones, para que en Él podamos llegar a convertirnos en una nueva creación. Así renovados, podremos renovar el mundo en Cristo, en virtud de la misión, ya injertada en nosotros por el sacramento del Bautismo. 
					
					{[} \ldots{}{]}
				\end{body}
		
			\subsubsection{Homilía (1998): }
			
			\src{Basílica de San Pedro. \\29 de noviembre de 1998. Convocación del Año Santo.}
			
			\begin{body}
				1. \textquote{Vayamos jubilosos al encuentro del Señor} (\emph{Estribillo del Salmo responsorial}). 
				
				Son las palabras del Salmo responsorial de esta liturgia del primer domingo de Adviento, tiempo litúrgico que renueva año tras año la espera de la venida de Cristo. En estos años que estamos viviendo en la perspectiva del tercer milenio, el Adviento ha cobrado una dimensión nueva y singular. \emph{Tertio millennio adveniente}: el año 1998, que está a punto de terminar, y el año próximo 1999 nos acercan al umbral de un nuevo siglo y de un nuevo milenio. 
				
				\textquote{En el umbral} ha comenzado también esta celebración: {[}en el umbral de la basílica vaticana, ante la puerta santa, con la entrega y la lectura de la \emph{bula de convocación} del gran jubileo del año 2000{]}. 
				
				\textquote{Vayamos jubilosos al encuentro del Señor} es un estribillo que está perfectamente en armonía con el jubileo. Es, por decir así, un \textquote{estribillo jubilar}, según la etimología de la palabra latina \emph{iubilar}, que encierra una referencia al júbilo. ¡Vayamos, pues, con alegría! Caminemos jubilosos y vigilantes a la espera del tiempo que recuerda la venida de Dios en la carne humana, tiempo que llegó a su plenitud cuando en la cueva de Belén nació Cristo. Entonces se cumplió el tiempo de la espera. 
				
				Viviendo el Adviento, esperamos un acontecimiento que se sitúa en la historia y a la vez la trasciende. Al igual que los demás años, tendrá lugar en la noche de la Navidad del Señor. A la cueva de Belén acudirán los pastores; más tarde, irán los Magos de Oriente. Unos y otros simbolizan, en cierto sentido, a toda la familia humana. La exhortación que resuena en la liturgia de hoy: \textquote{Vayamos jubilosos al encuentro del Señor} se difunde en todos los países, en todos los continentes, en todos los pueblos y naciones. La voz de la liturgia, es decir, la voz de la Iglesia, resuena por doquier e invita a todos al gran jubileo. 
				
				2. Estos últimos tres años que preceden al 2000 forman un tiempo de espera muy intenso, orientado a la meditación sobre el significado del inminente evento espiritual y sobre su necesaria preparación. El contenido de esa preparación sigue el modelo trinitario, que se repite al final de toda plegaria litúrgica. Así pues, vayamos jubilosos \emph{hacia el Padre}, \emph{por el camino que es nuestro Señor Jesucristo}, el cual vive y reina con él \emph{en la unidad del Espíritu Santo}. 
				
				Por eso, el primer año lo dedicamos al Hijo; el segundo, al Espíritu Santo; y el que comienza hoy ---el último antes del gran jubileo--- será \emph{el año del Padre}. Invitados por el Padre, vayamos a él mediante el Hijo, en el Espíritu Santo. Este trienio de preparación inmediata para el nuevo milenio, por su carácter trinitario, no sólo nos habla de Dios en sí mismo, como misterio inefable de vida y santidad, sino también de \emph{Dios que viene a nuestro encuentro}. 
				
				3. Por este motivo, el estribillo \textquote{Vayamos jubilosos \emph{al encuentro del Señor}} resulta tan adecuado. Nosotros podemos encontrar a Dios, porque él ha venido a nuestro encuentro. Lo ha hecho, como el padre de la parábola del hijo pródigo (cf. \emph{Lc} 15, 11-32), porque es rico en misericordia, \emph{dives in misericordia}, y quiere salir a nuestro encuentro sin importarle de qué parte venimos o a dónde lleva nuestro camino. Dios viene a nuestro encuentro, tanto si lo hemos buscado como si lo hemos ignorado, e incluso si lo hemos evitado. Él sale el primero a nuestro encuentro, con los brazos abiertos, como un padre amoroso y misericordioso. 
				
				Si Dios se pone en movimiento para salir a nuestro encuentro, ¿podremos nosotros volverle la espalda? Pero no podemos ir solos al encuentro con el Padre. Debemos ir en compañía de cuantos forman parte de \textquote{la familia de Dios}. Para prepararnos convenientemente al jubileo debemos disponernos a acoger a todas las personas. Todos son nuestros hermanos y hermanas, porque son hijos del mismo Padre celestial. 
				
				En esta perspectiva, podemos leer la bimilenaria historia de la Iglesia. Es consolador constatar cómo la Iglesia, en este paso del segundo al tercer milenio, está experimentando un nuevo impulso misionero. Lo ponen de manifiesto los Sínodos continentales que se están celebrando estos años, incluido el actual para Australia y Oceanía. Y también lo confirman los informes que llegan al Comité para el gran jubileo sobre las iniciativas puestas en marcha por las Iglesias locales como preparación para ese histórico acontecimiento. 
				
				Quisiera saludar, en particular, al cardenal presidente del comité, al secretario general y a sus colaboradores. Mi saludo se extiende también a los cardenales, a los obispos y a los sacerdotes aquí presentes, así como a todos vosotros, queridos hermanos y hermanas, que participáis en esta solemne liturgia. Saludo en especial al clero, a los religiosos, a las religiosas y a los laicos comprometidos de Roma, que, junto con el cardenal vicario y los obispos auxiliares, están aquí esta mañana para inaugurar la última fase de la misión ciudadana, dirigida a los ambientes de la sociedad. 
				
				Es una fase importante, en la que la diócesis realizará una amplia labor de evangelización en todos los ámbitos de vida y de trabajo. Al terminar la santa misa, entregaré a los misioneros la cruz de la misión. Es necesario que Cristo sea anunciado y testimoniado en cada lugar y en cada situación. Invito a todos a sostener con la oración esta gran empresa. En particular, cuento con la aportación de las monjas de clausura, de los enfermos, de las personas ancianas que, a pesar de que les es imposible participar directamente en esta iniciativa apostólica, pueden dar una gran contribución con su oración y con la ofrenda de sus sufrimientos para disponer los corazones a la acogida del anuncio evangélico. 
				
				María, que el tiempo de Adviento nos invita a contemplar en espera activa del Redentor, os ayude a todos a ser apóstoles generosos de su Hijo Jesús. 
				
				4. En el evangelio de hoy hemos escuchado la invitación del Señor a la \emph{vigilancia}. \textquote{Velad, porque no sabéis qué día vendrá vuestro Señor}. Y a continuación: \textquote{Estad preparados, porque a la hora que menos penséis vendrá el Hijo del hombre} (\emph{Mt} 24, 42.44). La exhortación a velar resuena muchas veces en la liturgia, especialmente en Adviento, tiempo de preparación no sólo para la Navidad, sino también para \emph{la definitiva y gloriosa venida de Cristo al final de los tiempos}. Por eso, tiene un significado marcadamente escatológico e invita al creyente a pasar cada día, cada momento, en presencia de Aquel \textquote{que es, que era y que vendrá} (\emph{Ap} 1, 4), al que pertenece el futuro del mundo y del hombre. Ésta es la esperanza cristiana. Sin esta perspectiva, nuestra existencia se reduciría a un vivir para la muerte. 
				
				Cristo es nuestro Redentor: \emph{Redemptor mundi et Redemptor hominis}, Redentor del mundo y Redentor del hombre. Vino a nosotros para ayudarnos a cruzar el umbral que lleva a la puerta de la vida, la \textquote{puerta santa} que es él mismo. 
				
				5. Que esta consoladora verdad esté siempre muy presente ante nuestros ojos, mientras caminamos como peregrinos hacia el gran jubileo. Esa verdad constituye la razón última de la alegría a la que nos exhorta la liturgia de hoy: \textquote{Vayamos \emph{jubilosos} al encuentro del Señor}. Creyendo en Cristo crucificado y resucitado, creemos en la resurrección de la carne y en la vida eterna. 
				
				\emph{Tertio millennio adveniente}. En esta perspectiva, los años, los siglos y los milenios cobran el sentido definitivo de la existencia que el jubileo del año 2000 quiere manifestarnos. 
				
				Contemplando a Cristo, hagamos nuestras las palabras de un antiguo canto popular polaco: 
				
				\begin{quote} \textquote{La salvación ha venido por la cruz;\\ éste es un gran misterio.\\ Todo sufrimiento tiene un sentido:\\ lleva a la plenitud de la vida}. \end{quote} 
				
				Con esta fe en el corazón, que es la fe de la Iglesia, inauguro hoy, como Obispo de Roma, el tercer año de preparación para el gran jubileo. Lo inauguro en el nombre del Padre celestial, que \textquote{tanto amó (\ldots{}) al mundo que le dio su Hijo único, para que quien cree en él (\ldots{}) tenga la vida eterna} (\emph{Jn} 3, 16). 
				
				¡Alabado sea Jesucristo!
				
			\end{body}
	
		\subsubsection{Ángelus (2001): }
			\src{2 de diciembre de 2001.}
			
			\begin{body}
				\emph{Amadísimos hermanos y hermanas:} 
				
				1. Con este primer domingo de Adviento comienza un nuevo Año litúrgico. La Iglesia reanuda su camino y nos invita a reflexionar más intensamente en el misterio de Cristo, misterio siempre nuevo que el tiempo no puede agotar. Cristo es el alfa y la omega, el principio y el fin. Gracias a él, la historia de la humanidad avanza como una peregrinación hacia la plenitud del Reino, que él mismo inauguró con su encarnación y su victoria sobre el pecado y la muerte. 
				
				Por eso, el Adviento es sinónimo de \emph{esperanza}: no espera vana de un dios sin rostro, sino confianza concreta y cierta en la vuelta de Aquel que ya nos ha visitado, del \textquote{Esposo} que con su sangre ha sellado con la humanidad un pacto de alianza eterna. Es una esperanza que estimula a la \emph{vigilancia}, virtud característica de este singular tiempo litúrgico. Vigilancia en la \emph{oración}, animada por una amorosa espera; vigilancia en el dinamismo de la \emph{caridad concreta}, consciente de que el reino de Dios se acerca donde los hombres aprenden a vivir como hermanos. 
				
				2. Con estos sentimientos, la comunidad cristiana entra en el Adviento, manteniendo vigilante su espíritu, para acoger mejor el mensaje de la palabra de Dios. Resuena hoy en la liturgia el célebre y estupendo \emph{oráculo del profeta Isaías}, pronunciado en un momento de crisis de la historia de Israel. 
				
				\textquote{Al final de los días ---dice el Señor--- estará firme el monte de la casa del Señor, encumbrado sobre las montañas. Hacia él confluirán los gentiles. (\ldots{}) De las espadas forjarán arados; de las lanzas, podaderas. No alzará la espada pueblo contra pueblo, no se adiestrarán para la guerra} (\emph{Is} 2, 1-5). 
				
				Estas palabras contienen una promesa de paz más actual que nunca para la humanidad, y en particular para la Tierra Santa, de donde también hoy, por desgracia, llegan noticias dolorosas y preocupantes. Que las palabras del profeta Isaías inspiren la mente y el corazón de los creyentes y de los hombres de buena voluntad, para que el día de ayuno ---el 14 de diciembre--- y el encuentro de los representantes de las religiones del mundo en Asís ---el 24 de enero del año próximo--- ayuden a crear en el mundo un clima más sereno y solidario. 
				
				3. Encomiendo esta invocación de paz a María, Virgen vigilante y Madre de la esperanza. Dentro de algunos días celebraremos con fe renovada la solemnidad de la Inmaculada Concepción. Que ella nos guíe por este camino, ayudando a todo hombre y a toda nación a dirigir la mirada al \textquote{monte del Señor}, imagen del triunfo definitivo de Cristo y de la venida de su reino de paz.
			\end{body}
	
\newsection
	
		\subsection{Benedicto XVI, papa}
	
			\subsubsection{Homilía (2007): La gran esperanza}
			
				\src{Visita Pastoral al Hospital Romano San Juan Bautista de la Soberana Orden de Malta. \\ 2 de diciembre del 2007.} 
				
				\begin{body}
					1. \textquote{Vamos alegres al encuentro del Señor}. Estas palabras, que hemos repetido en el estribillo del \textbf{salmo responsorial}, interpretan bien los sentimientos que alberga nuestro corazón hoy, primer domingo de Adviento. La razón por la cual podemos caminar con alegría, como nos ha exhortado el apóstol \textbf{san Pablo}, es que ya está cerca nuestra salvación. El Señor viene. Con esta certeza emprendemos el itinerario del Adviento, preparándonos para celebrar con fe el acontecimiento extraordinario del Nacimiento del Señor. 
					
					Durante las próximas semanas, día tras día, la liturgia propondrá a nuestra reflexión textos del Antiguo Testamento, que recuerdan el vivo y constante deseo que animó en el pueblo judío la espera de la venida del Mesías. También nosotros, vigilantes en la oración, tratemos de preparar nuestro corazón para acoger al Salvador, que vendrá a mostrarnos su misericordia y a darnos su salvación. 
					
					2. Precisamente porque es tiempo de espera, el Adviento es tiempo de esperanza, y a la esperanza cristiana he querido dedicar mi segunda encíclica, presentada oficialmente anteayer: comienza con las palabras que san Pablo dirigió a los cristianos de Roma: \emph{\textquote{Spe salvi facti sumus}}, \textquote{En esperanza fuimos salvados} (\emph{Rm} 8, 24). En la encíclica escribí, entre otras cosas, que \textquote{nosotros necesitamos tener esperanzas ---más grandes o más pequeñas---, que día a día nos mantengan en camino. Pero sin la gran esperanza, que ha de superar todo lo demás, aquellas no bastan. Esta gran esperanza sólo puede ser Dios, que abraza el universo y que nos puede proponer y dar lo que nosotros por sí solos no podemos alcanzar} (n. 31). Que la certeza de que sólo Dios puede ser nuestra firme esperanza nos anime a todos los que esta mañana nos hemos reunido en esta casa, en la que se lucha contra la enfermedad, sostenidos por la solidaridad. 
					
					{[}\ldots{}{]} 
					
					4. Queridos hermanos y hermanas, \textquote{que el Dios de la esperanza, que nos colma de todo gozo y paz en la fe por la fuerza del Espíritu Santo, esté con todos vosotros}. Con este deseo, que el sacerdote dirige a la asamblea al inicio de la santa misa, os saludo cordialmente \ldots{} 
					
					El saludo más afectuoso es para vosotros, queridos enfermos, y para vuestros familiares, que con vosotros comparten angustias y esperanzas. El Papa está espiritualmente cerca de vosotros y os asegura su oración diaria; os invita a encontrar en Jesús apoyo y consuelo, y a no perder jamás la confianza. La liturgia de Adviento nos repetirá durante las próximas semanas que no nos cansemos de invocarlo; nos exhortará a salir a su encuentro, sabiendo que él mismo viene continuamente a visitarnos. En la prueba y en la enfermedad Dios nos visita misteriosamente y, si nos abandonamos a su voluntad, podemos experimentar la fuerza de su amor. 
					
					Los hospitales y las clínicas, precisamente porque en ellos se encuentran personas probadas por el dolor, pueden transformarse en lugares privilegiados para testimoniar el amor cristiano que alimenta la esperanza y suscita propósitos de solidaridad fraterna. En la \textbf{oración colecta} hemos rezado así: \textquote{Dios todopoderoso, aviva en tus fieles, al comenzar el Adviento, el deseo de salir al encuentro de Cristo, acompañados por las buenas obras}. Sí. Abramos el corazón a todas las personas, especialmente a las que atraviesan dificultades, para que, haciendo el bien a cuantos se encuentran en necesidad, nos dispongamos a acoger a Jesús que en ellos viene a visitarnos. 
					
					6. [\ldots{}] En cada enfermo, cualquiera que sea, reconoced y servid a Cristo mismo; haced que en vuestros gestos y en vuestras palabras perciba los signos de su amor misericordioso. 
					
					Para cumplir bien esta \textquote{misión}, como nos recuerda san Pablo en la \textbf{segunda lectura}, tratad de \textquote{pertrecharos con las armas de la luz} (\emph{Rm} 13, 12), que son la palabra de Dios, los dones del Espíritu, la gracia de los sacramentos, y las virtudes teologales y cardinales; luchad contra el mal y abandonad el pecado, que entenebrece nuestra existencia. Al inicio de un nuevo año litúrgico, renovemos nuestros buenos propósitos de vida evangélica. \textquote{Ya es hora de espabilarse} (\emph{Rm} 13, 11), exhorta el Apóstol; es decir, es hora de convertirse, de despertar del letargo del pecado para disponerse con confianza a acoger al \textquote{Señor que viene}. Por eso, el Adviento es tiempo de oración y de espera vigilante. 
					
					7. A la \textquote{vigilancia}, que por lo demás es la palabra clave de todo este período litúrgico, nos exhorta la \textbf{página evangélica} que acabamos de proclamar: \textquote{Estad en vela, porque no sabéis qué día vendrá vuestro Señor} (\emph{Mt} 24, 42). Jesús, que en la Navidad vino a nosotros y volverá glorioso al final de los tiempos, no se cansa de visitarnos continuamente en los acontecimientos de cada día. Nos pide estar atentos para percibir su presencia, su adviento, y nos advierte que lo esperemos vigilando, puesto que su venida no se puede programar o pronosticar, sino que será repentina e imprevisible. Sólo quien está despierto no será tomado de sorpresa. Que no os suceda ---advierte--- lo que pasó en tiempo de Noé, cuando los hombres comían y bebían despreocupadamente, y el diluvio los encontró desprevenidos (cf. \emph{Mt} 24, 37-38). Lo que quiere darnos a entender el Señor con esta recomendación es que no debemos dejarnos absorber por las realidades y preocupaciones materiales hasta el punto de quedar atrapados en ellas. Debemos vivir ante los ojos del Señor con la convicción de que cada día puede hacerse presente. Si vivimos así, el mundo será mejor. 
					
					8. \textquote{Estad, pues, en vela\ldots{}}. Escuchemos la invitación de Jesús en el \textbf{Evangelio} y preparémonos para revivir con fe el misterio del nacimiento del Redentor, que ha llenado de alegría el universo; preparémonos para acoger al Señor que viene continuamente a nuestro encuentro en los acontecimientos de la vida, en la alegría y en el dolor, en la salud y en la enfermedad; preparémonos para encontrarlo en su venida última y definitiva. 
					
					Su paso es siempre fuente de paz y, si el sufrimiento, herencia de la naturaleza humana, a veces resulta casi insoportable, con la venida del Salvador \textquote{el sufrimiento ---sin dejar de ser sufrimiento--- se convierte a pesar de todo en canto de alabanza} (\emph{Spe salvi}, 37). Confortados por estas palabras, prosigamos la celebración eucarística, invocando sobre los enfermos, sobre sus familiares y sobre cuantos trabajan en este hospital y en toda la Orden de los Caballeros de Malta, la protección materna de María, Virgen de la espera y de la esperanza, así como de la alegría, ya presente en este mundo, porque cuando sentimos la cercanía de Cristo vivo tenemos ya el remedio para el sufrimiento, tenemos ya su alegría. Amén.
				\end{body}
		
			\subsubsection{Ángelus (2007): }
			
			\src{Plaza de San Pedro. 2 de diciembre de 2007.}
			
			\begin{body} 
				\emph{Queridos hermanos y hermanas:} 
				
				Con este primer domingo de Adviento comienza un nuevo año litúrgico: el pueblo de Dios vuelve a ponerse en camino para vivir el misterio de Cristo en la historia. Cristo es el mismo ayer, hoy y siempre (cf. \emph{Hb} 13, 8); en cambio, la historia cambia y necesita ser evangelizada constantemente; necesita renovarse desde dentro, y la única verdadera novedad es Cristo: él es su realización plena, el futuro luminoso del hombre y del mundo. Jesús, resucitado de entre los muertos, es el Señor al que Dios someterá todos sus enemigos, incluida la misma muerte (cf. \emph{1 Co} 15, 25-28). 
				
				Por tanto, el Adviento es el tiempo propicio para reavivar en nuestro corazón la espera de Aquel \textquote{que es, que era y que va a venir} (\emph{Ap} 1, 8). El Hijo de Dios ya vino en Belén hace veinte siglos, viene en cada momento al alma y a la comunidad dispuestas a recibirlo, y de nuevo vendrá al final de los tiempos para \textquote{juzgar a vivos y muertos}. Por eso, el creyente está siempre vigilante, animado por la íntima esperanza de encontrar al Señor, como dice el Salmo: \textquote{Mi alma espera en el Señor, espera en su palabra; mi alma aguarda al Señor, más que el centinela a la aurora} (\emph{Sal} 130, 5-6). 
				
				Por consiguiente, este domingo es un día muy adecuado para ofrecer a la Iglesia entera y a todos los hombres de buena voluntad mi segunda encíclica, que quise dedicar precisamente al tema de la esperanza cristiana. Se titula \emph{Spe salvi}, porque comienza con la expresión de san Pablo: \emph{\textquote{Spe salvi factum sumus}}, \textquote{en esperanza fuimos salvados} (\emph{Rm} 8, 24). En este, como en otros pasajes del Nuevo Testamento, la palabra \textquote{esperanza} está íntimamente relacionada con la palabra \textquote{fe}. Es un don que cambia la vida de quien lo recibe, como lo muestra la experiencia de tantos santos y santas. 
				
				¿En qué consiste esta esperanza, tan grande y tan \textquote{fiable} que nos hace decir que \emph{en ella} encontramos la \textquote{salvación}? Esencialmente, consiste en el conocimiento de Dios, en el descubrimiento de su corazón de Padre bueno y misericordioso. Jesús, con su muerte en la cruz y su resurrección, nos reveló su rostro, el rostro de un Dios con un amor tan grande que comunica una esperanza inquebrantable, que ni siquiera la muerte puede destruir, porque la vida de quien se pone en manos de este Padre se abre a la perspectiva de la bienaventuranza eterna. 
				
				El desarrollo de la ciencia moderna ha marginado cada vez más la fe y la esperanza en la esfera privada y personal, hasta el punto de que hoy se percibe de modo evidente, y a veces dramático, que el hombre y el mundo necesitan a Dios ---¡al verdadero Dios!---; de lo contrario, no tienen esperanza. 
				
				No cabe duda de que la ciencia contribuye en gran medida al bien de la humanidad, pero no es capaz de redimirla. El hombre es redimido por el amor, que hace buena y hermosa la vida personal y social. Por eso la gran esperanza, la esperanza plena y definitiva, es garantizada por Dios que es amor, por Dios que en Jesús nos visitó y nos dio la vida, y en él volverá al final de los tiempos. 
				
				En Cristo esperamos; es a él a quien aguardamos. Con María, su Madre, la Iglesia va al encuentro del Esposo: lo hace con las obra de caridad, porque la esperanza, como la fe, se manifiesta en el amor. ¡Buen Adviento a todos!
			\end{body}
	
		\subsubsection{Ángelus (2010)}
		
			\src{Plaza de San Pedro. \\28 de noviembre de 2010.}
			
			\begin{body}
				\emph{Queridos hermanos y hermanas:} 
				
				Hoy, primer domingo de Adviento, la Iglesia inicia un nuevo Año litúrgico, un nuevo camino de fe que, por una parte, conmemora el acontecimiento de Jesucristo, y por otra, se abre a su cumplimiento final. Precisamente de esta doble perspectiva vive el tiempo de Adviento, mirando tanto a la primera venida del Hijo de Dios, cuando nació de la Virgen María, como a su vuelta gloriosa, cuando vendrá a \textquote{juzgar a vivos y muertos}, como decimos en el Credo. Sobre este sugestivo tema de la \textquote{espera} quiero detenerme ahora brevemente, porque se trata de un aspecto profundamente humano, en el que la fe se convierte, por decirlo así, en un todo con nuestra carne y nuestro corazón. 
				
				La espera, el esperar, es una dimensión que atraviesa toda nuestra existencia personal, familiar y social. La espera está presente en mil situaciones, desde las más pequeñas y banales hasta las más importantes, que nos implican totalmente y en lo profundo. Pensemos, entre estas, en la espera de un hijo por parte de dos esposos; en la de un pariente o de un amigo que viene a visitarnos de lejos; pensemos, para un joven, en la espera del resultado de un examen decisivo, o de una entrevista de trabajo; en las relaciones afectivas, en la espera del encuentro con la persona amada, de la respuesta a una carta, o de la aceptación de un perdón\ldots{} Se podría decir que el hombre está vivo mientras espera, mientras en su corazón está viva la esperanza. Y al hombre se lo reconoce por sus esperas: nuestra \textquote{estatura} moral y espiritual se puede medir por lo que esperamos, por aquello en lo que esperamos. 
				
				Cada uno de nosotros, por tanto, especialmente en este tiempo que nos prepara a la Navidad, puede preguntarse: ¿yo qué espero? En este momento de mi vida, ¿a qué tiende mi corazón? Y esta misma pregunta se puede formular a nivel de familia, de comunidad, de nación. ¿Qué es lo que esperamos juntos? ¿Qué une nuestras aspiraciones?, ¿qué tienen en común? En el tiempo anterior al nacimiento de Jesús, era muy fuerte en Israel la espera del Mesías, es decir, de un Consagrado, descendiente del rey David, que finalmente liberaría al pueblo de toda esclavitud moral y política e instauraría el reino de Dios. Pero nadie habría imaginado nunca que el Mesías pudiese nacer de una joven humilde como era María, prometida del justo José. Ni siquiera ella lo habría pensado nunca, pero en su corazón la espera del Salvador era tan grande, su fe y su esperanza eran tan ardientes, que él pudo encontrar en ella una madre digna. Por lo demás, Dios mismo la había preparado, antes de los siglos. Hay una misteriosa correspondencia entre la espera de Dios y la de María, la criatura \textquote{llena de gracia}, totalmente transparente al designio de amor del Altísimo. Aprendamos de ella, Mujer del Adviento, a vivir los gestos cotidianos con un espíritu nuevo, con el sentimiento de una espera profunda, que sólo la venida de Dios puede colmar.
			\end{body}	
		
\newsection
		
		\subsection{Francisco, papa}
		
			\subsubsection{Homilía (2013):}
			
				\src{Visita Pastoral a la Parroquia Romana de San Cirilo de Alejandría. \\1 de diciembre de 2013.}
				
				\begin{body}
					En la primera lectura, hemos escuchado que el profeta Isaías nos habla de un camino, y dice que al final de los días, al final del camino, el monte del Templo del Señor estará firme en la cima de los montes. Y esto, para decirnos que nuestra vida es un camino: debemos ir por este camino, para llegar al monte del Señor, al encuentro con Jesús. La cosa más importante que le puede suceder a una persona es encontrar a Jesús: este encuentro con Jesús que nos ama, que nos ha salvado, que ha dado su vida por nosotros. Encontrar a Jesús. Y nosotros caminamos para encontrar a Jesús. 
					
					Podemos preguntarnos: ¿Cuándo encuentro a Jesús? ¿Sólo al final? ¡No, no! Lo encontramos todos los días. ¿Pero cómo? En la oración, cuando tú rezas, encuentras a Jesús. Cuando recibes la Comunión, encuentras a Jesús, en los Sacramentos. Cuando llevas a bautizar a tu hijo, te encuentras a Jesús, hallas a Jesús. Y vosotros, hoy, que recibís la Confirmación, también vosotros encontraréis a Jesús; luego lo encontraréis en la Comunión. \textquote{Y más tarde, Padre, después de la Confirmación, adiós}, porque dicen que la Confirmación se llama \textquote{el sacramento del ¡adiós!}. ¿Es verdad esto o no? Después de la Confirmación no se va nunca a la iglesia: ¿es verdad o no?\ldots{} ¡Más o menos! Pero también después de la Confirmación, toda la vida, es un encuentro con Jesús: en la oración, cuando vamos a misa y cuando realizamos buenas obras, cuando visitamos a los enfermos, cuando ayudamos a un pobre, cuando pensamos en los demás, cuando no somos egoístas, cuando somos amables\ldots{} en estas cosas encontramos siempre a Jesús. Y el camino de la vida es precisamente este: caminar para encontrar a Jesús. 
					
					Hoy, también para mí es una alegría venir a encontrarme con vosotros, porque todos juntos, hoy, en la misa encontraremos a Jesús, y hacemos un tramo del camino juntos. 
					
					Recordad siempre esto: la vida es un camino. Es un camino. Un camino para encontrar a Jesús. Al final, y siempre. Un camino donde no encontramos a Jesús, no es un camino cristiano. Es propio del cristiano encontrar siempre a Jesús, mirarle, dejarse mirar por Jesús, porque Jesús nos mira con amor, nos ama mucho, nos quiere mucho y nos mira siempre. Encontrar a Jesús es también dejarte mirar por Él. \textquote{Pero, Padre, tú sabes ---alguno de vosotros podría decirme---, tú sabes que este camino, para mí, es un camino difícil, porque yo soy muy pecador, he cometido muchos pecados\ldots{} ¿cómo puedo encontrar a Jesús?}. Pero tú sabes que las personas a las que Jesús mayormente buscaba eran los más pecadores; y le reñían por esto, y la gente ---las personas que se creían justas--- decía: pero éste, éste no es un verdadero profeta, ¡mira la buena compañía que tiene! Estaba con los pecadores\ldots{} Y Él decía: He venido por quienes tienen necesidad de salud, necesidad de curación, y Jesús cura nuestros pecados. En el camino, nosotros ---todos pecadores, todos, todos somos pecadores--- incluso cuando nos equivocamos, cuando cometemos un pecado, cuando pecamos, Jesús viene y nos perdona. Este perdón que recibimos en la Confesión es un encuentro con Jesús. Siempre encontramos a Jesús. 
					
					Y así vamos por la vida, como dice el profeta, al monte, hasta el día que tendrá lugar el encuentro definitivo, cuando contemplemos esa mirada tan bella de Jesús, tan hermosa. Ésta es la vida cristiana: caminar, seguir adelante, unidos como hermanos, queriéndose uno a otro. Encontrar a Jesús. ¿Estáis de acuerdo, vosotros, los nueve? ¿Queréis encontrar a Jesús en vuestra vida? ¿Sí? Esto es importante en la vida cristiana. Vosotros, hoy, con el sello del Espíritu Santo, tendréis más fuerza para este camino, para encontrar a Jesús. ¡Sed valientes, no tengáis miedo! La vida es este camino. Y el regalo más hermoso es encontrar a Jesús. ¡Adelante, ánimo! 
					
					Y ahora, sigamos adelante con el Sacramento de la Confirmación.
				\end{body}
			
			\subsubsection{Ángelus (2013): En camino}
			
				\src{Plaza de San Pedro. \\1 de diciembre del 2013.}
				
				\begin{body}
					Comenzamos hoy, primer domingo de Adviento, un nuevo año litúrgico, es decir \emph{un nuevo camino del Pueblo de Dios} con Jesucristo, nuestro Pastor, que nos guía en la historia hacia la realización del Reino de Dios. Por ello este día tiene un atractivo especial, nos hace experimentar un sentimiento profundo del sentido de la historia. Redescubrimos la belleza de estar todos en camino: la Iglesia, con su vocación y misión, y toda la humanidad, los pueblos, las civilizaciones, las culturas, todos en camino a través de los senderos del tiempo.
					
					¿En camino hacia dónde? ¿Hay una meta común? ¿Y cuál es esta meta? El Señor nos responde a través del profeta \textbf{Isaías}, y dice así: \textquote{En los días futuros estará firme el monte de la casa del Señor, en la cumbre de las montañas, más elevado que las colinas. Hacia él confluirán todas las naciones, caminarán pueblos numerosos y dirán: \textquote{Venid, subamos al monte del Señor, a la casa del Dios de Jacob. Él nos instruirá en sus caminos y marcharemos por sus sendas}} (2, 2-3). Esto es lo que dice Isaías acerca de la meta hacia la que nos dirigimos. Es \emph{una peregrinación universal hacia una meta común}, que en el Antiguo Testamento es Jerusalén, donde surge el templo del Señor, porque desde allí, de Jerusalén, ha venido la revelación del rostro de Dios y de su ley. La revelación ha encontrado su realización en \emph{Jesucristo}, y Él mismo, el Verbo hecho carne, se ha convertido en el \textquote{templo del Señor}: es Él la guía y al mismo tiempo la meta de nuestra peregrinación, de la peregrinación de todo el Pueblo de Dios; y bajo su luz también los demás pueblos pueden caminar hacia el Reino de la justicia, hacia el Reino de la paz. Dice de nuevo el profeta: \textquote{De las espadas forjarán arados, de las lanzas, podaderas. No alzará la espada pueblo contra pueblo, no se adiestrarán para la guerra} (2, 4).
					
					Me permito repetir esto que dice el profeta, escuchad bien: \textquote{De las espadas forjarán arados, de las lanzas, podaderas. No alzará la espada pueblo contra pueblo, no se adiestrarán para la guerra}. ¿Pero cuándo sucederá esto? Qué hermoso día será ese en el que las armas sean desmontadas, para transformarse en instrumentos de trabajo. ¡Qué hermoso día será ése! ¡Y esto es posible! Apostemos por la esperanza, la esperanza de la paz. Y será posible.
					
					Este camino no se acaba nunca. Así como en la vida de cada uno de nosotros siempre hay necesidad de comenzar de nuevo, de volver a levantarse, de volver a encontrar el sentido de la meta de la propia existencia, de la misma manera para la gran familia humana es necesario renovar siempre el horizonte común hacia el cual estamos encaminados. \emph{¡El horizonte de la esperanza!} Es ese el horizonte para hacer un buen camino. El tiempo de Adviento, que hoy de nuevo comenzamos, nos devuelve el horizonte de la esperanza, una esperanza que no decepciona porque está fundada en la Palabra de Dios. Una esperanza que no decepciona, sencillamente porque el Señor no decepciona jamás. ¡Él es fiel!, ¡Él no decepciona! ¡Pensemos y sintamos esta belleza!
					
					El modelo de esta actitud espiritual, de este modo de ser y de caminar en la vida, es la Virgen María. Una sencilla muchacha de pueblo, que lleva en el corazón toda la esperanza de Dios. En su seno, la esperanza de Dios se hizo carne, se hizo hombre, se hizo historia: Jesucristo. Su \emph{Magníficat} es el cántico del Pueblo de Dios en camino, y de todos los hombres y mujeres que esperan en Dios, en el poder de su misericordia. Dejémonos guiar por Ella, que es madre, es mamá, y sabe cómo guiarnos. Dejémonos guiar por Ella en este tiempo de espera y de vigilancia activa.
				\end{body}
		
			\subsubsection{Ángelus (2016): El Señor nos visita}
			
				\src{Plaza de San Pedro. \\27 de noviembre del 2016.}
			
				\begin{body}
					Hoy la Iglesia inicia un nuevo año litúrgico, es decir, un nuevo camino de fe del pueblo de Dios. Y como siempre iniciamos con el Adviento. La página del \textbf{Evangelio} (cf. Mt 24, 37-44) nos presenta uno de los temas más sugestivos del tiempo de Adviento: la visita del Señor a la humanidad. La primera visita ---lo sabemos todos--- se produjo con la Encarnación, el nacimiento de Jesús en la gruta de Belén; la segunda sucede en el presente: el Señor nos visita continuamente cada día, camina a nuestro lado y es una presencia de consolación; y para concluir estará la tercera y última visita, que profesamos cada vez que recitamos el Credo: \textquote{De nuevo vendrá en la gloria para juzgar a vivos y a muertos}. El Señor hoy nos habla de esta última visita suya, la que sucederá al final de los tiempos y nos dice dónde llegará nuestro camino. 
					
					La palabra de Dios hace resaltar el contraste entre el desarrollarse normal de las cosas, la rutina cotidiana y la venida repentina del Señor. \textbf{Dice Jesús}: \textquote{Como en los días que precedieron al diluvio, comían, bebían, tomaban mujer o marido, hasta el día en el que entró Noé en el arca, y no se dieron cuenta hasta que vino el diluvio y los arrasó a todos} (vv. 38-39): así dice Jesús. Siempre nos impresiona pensar en las horas que preceden a una gran calamidad: todos están tranquilos, hacen las cosas de siempre sin darse cuenta que su vida está a punto de ser alterada. El Evangelio, ciertamente no quiere darnos miedo, sino abrir nuestro horizonte a la dimensión ulterior, más grande, que por una parte relativiza las cosas de cada día pero al mismo tiempo las hace preciosas, decisivas. La relación con el Dios que viene a visitarnos da a cada gesto, a cada cosa una luz diversa, una profundidad, un valor simbólico. 
					
					Desde esta perspectiva llega también una invitación a la sobriedad, a no ser dominados por las cosas de este mundo, por las realidades materiales, sino más bien a gobernarlas. Si por el contrario nos dejamos condicionar y dominar por ellas, no podemos percibir que hay algo mucho más importante: nuestro encuentro final con el Señor, y esto es importante. Ese, ese encuentro. Y las cosas de cada día deben tener ese horizonte, deben ser dirigidas a ese horizonte. Este encuentro con el Señor que viene por nosotros. En aquel momento, como dice el \textbf{Evangelio}, \textquote{estarán dos en el campo: uno es tomado, el otro dejado} (v. 40). Es una invitación a la vigilancia, porque no sabiendo cuando Él vendrá, es necesario estar preparados siempre para partir. 
					
					En este tiempo de Adviento estamos llamados a ensanchar los horizontes de nuestro corazón, a dejarnos sorprender por la vida que se presenta cada día con sus novedades. Para hacer esto es necesario aprender a no depender de nuestras seguridades, de nuestros esquemas consolidados, porque el Señor viene a la hora que no nos imaginamos. Viene para presentarnos una dimensión más hermosa y más grande. 
					
					Que Nuestra Señora, Virgen del Adviento, nos ayude a no considerarnos propietarios de nuestra vida, a no oponer resistencia cuando el Señor viene para cambiarla, sino a estar preparados para dejarnos visitar por Él, huésped esperado y grato, aunque desarme nuestros planes.
				\end{body}
		
			\subsubsection{Ángelus (2019): Velar}
		
			\src{Plaza de San Pedro. 1 de diciembre de 2019.}
			
				\begin{body}
					Hoy, primer domingo de Adviento, comienza un nuevo año litúrgico. En estas cuatro semanas de Adviento, la liturgia nos lleva a celebrar el nacimiento de Jesús, mientras nos recuerda que Él viene todos los días en nuestras vidas, y que regresará gloriosamente al final de los tiempos. Esta certeza nos lleva a mirar al futuro con confianza, como nos invita el profeta Isaías, que con su voz inspirada acompaña todo el camino del Adviento. 
					
					En la \textbf{primera lectura} de hoy, Isaías profetiza que \textquote{sucederá en días futuros que el monte de la Casa de Yahveh será asentado en la cima de los montes y se alzará por encima de las colinas. Confluirán a él todas las naciones} (Isaías 2, 2). El templo del Señor en Jerusalén se presenta como el punto de encuentro y de convergencia de todos los pueblos. 
					
					Después de la Encarnación del Hijo de Dios, Jesús mismo se reveló como el verdadero templo. Por lo tanto, la maravillosa \textbf{visión de Isaía}s es una promesa divina y nos impulsa a asumir una actitud de peregrinación, de camino hacia Cristo, sentido y fin de toda la historia. Los que tienen hambre y sed de justicia sólo pueden encontrarla a través de los caminos del Señor, mientras que el mal y el pecado provienen del hecho de que los individuos y los grupos sociales prefieren seguir caminos dictados por intereses egoístas, que causan conflictos y guerras. 
					
					El Adviento es el tiempo para acoger la venida de Jesús, que viene como mensajero de paz para mostrarnos los caminos de Dios. 
					
					En el \textbf{Evangelio} de hoy, Jesús nos exhorta a estar preparados para su venida: \textquote{Velad, pues, porque no sabéis qué día vendrá vuestro Señor} (Mateo 24, 42). Velar no significa tener los ojos materialmente abiertos, sino tener el corazón libre y orientado en la dirección correcta, es decir, dispuesto a dar y servir. ¡Eso es velar! El sueño del que debemos despertar está constituido por la indiferencia, por la vanidad, por la incapacidad de establecer relaciones verdaderamente humanas, por la incapacidad de hacerse cargo de nuestro hermano aislado, abandonado o enfermo. 
					
					La espera de la venida de Jesús debe traducirse, por tanto, en un compromiso de vigilancia. Se trata sobre todo de maravillarse de la acción de Dios, de sus sorpresas y de darle primacía. Vigilancia significa también, concretamente, estar atento al prójimo en dificultades, dejarse interpelar por sus necesidades, sin esperar a que nos pida ayuda, sino aprendiendo a prevenir, a anticipar, como Dios siempre hace con nosotros. 
					
					Que María, Virgen vigilante y Madre de la esperanza, nos guíe en este camino, ayudándonos a dirigir la mirada hacia el \textquote{monte del Señor}, imagen de Jesucristo, que atrae a todos los hombres y todos los pueblos.
				\end{body}

\newsection
	\section{Temas}

		\cceth{Tribulación final y venida de Cristo en gloria}

\cceref{CEC 668-677, 769}

\begin{ccebody}
\n{668} \textquote{Cristo murió y volvió a la vida para eso, para ser Señor de muertos y vivos} (\emph{Rm} 14, 9). La Ascensión de Cristo al Cielo significa su participación, en su humanidad, en el poder y en la autoridad de Dios mismo. Jesucristo es Señor: posee todo poder en los cielos y en la tierra. El está \textquote{por encima de todo principado, potestad, virtud, dominación} porque el Padre \textquote{bajo sus pies sometió todas las cosas} (\emph{Ef} 1, 20-22). Cristo es el Señor del cosmos (cf. \emph{Ef} 4, 10; \emph{1 Co} 15, 24. 27-28) y de la historia. En Él, la historia de la humanidad e incluso toda la Creación encuentran su recapitulación (\emph{Ef} 1, 10), su cumplimiento transcendente.

\n{669} Como Señor, Cristo es también la cabeza de la Iglesia que es su Cuerpo (cf. \emph{Ef} 1, 22). Elevado al cielo y glorificado, habiendo cumplido así su misión, permanece en la tierra en su Iglesia. La Redención es la fuente de la autoridad que Cristo, en virtud del Espíritu Santo, ejerce sobre la Iglesia (cf. \emph{Ef} 4, 11-13). \textquote{La Iglesia, o el reino de Cristo presente ya en misterio} (LG 3), \textquote{constituye el germen y el comienzo de este Reino en la tierra} (LG 5).

\n{670} Desde la Ascensión, el designio de Dios ha entrado en su consumación. Estamos ya en la \textquote{última hora} (\emph{1 Jn} 2, 18; cf. \emph{1 P} 4, 7). \textquote{El final de la historia ha llegado ya a nosotros y la renovación del mundo está ya decidida de manera irrevocable e incluso de alguna manera real está ya por anticipado en este mundo. La Iglesia, en efecto, ya en la tierra, se caracteriza por una verdadera santidad, aunque todavía imperfecta} (LG 48). El Reino de Cristo manifiesta ya su presencia por los signos milagrosos (cf. \emph{Mc} 16, 17-18) que acompañan a su anuncio por la Iglesia (cf. \emph{Mc} 16, 20).

\ccesec{\ldots{} esperando que todo le sea sometido}

\n{671} El Reino de Cristo, presente ya en su Iglesia, sin embargo, no está todavía acabado \textquote{con gran poder y gloria} (\emph{Lc} 21, 27; cf. \emph{Mt} 25, 31) con el advenimiento del Rey a la tierra. Este Reino aún es objeto de los ataques de los poderes del mal (cf. \emph{2 Ts} 2, 7), a pesar de que estos poderes hayan sido vencidos en su raíz por la Pascua de Cristo. Hasta que todo le haya sido sometido (cf. \emph{1 Co} 15, 28), y \textquote{mientras no [\ldots{}] haya nuevos cielos y nueva tierra, en los que habite la justicia, la Iglesia peregrina lleva en sus sacramentos e instituciones, que pertenecen a este tiempo, la imagen de este mundo que pasa. Ella misma vive entre las criaturas que gimen en dolores de parto hasta ahora y que esperan la manifestación de los hijos de Dios} (LG 48). Por esta razón los cristianos piden, sobre todo en la Eucaristía (cf. \emph{1 Co} 11, 26), que se apresure el retorno de Cristo (cf. \emph{2 P} 3, 11-12) cuando suplican: \textquote{Ven, Señor Jesús} (\emph{Ap} 22, 20; cf. \emph{1 Co} 16, 22; \emph{Ap} 22, 17-20).

\n{672} Cristo afirmó antes de su Ascensión que aún no era la hora del establecimiento glorioso del Reino mesiánico esperado por Israel (cf. \emph{Hch} 1, 6-7) que, según los profetas (cf. \emph{Is} 11, 1-9), debía traer a todos los hombres el orden definitivo de la justicia, del amor y de la paz. El tiempo presente, según el Señor, es el tiempo del Espíritu y del testimonio (cf. \emph{Hch} 1, 8), pero es también un tiempo marcado todavía por la \textquote{tribulación} (\emph{1 Co} 7, 26) y la prueba del mal (cf. \emph{Ef} 5, 16) que afecta también a la Iglesia (cf. \emph{1 P} 4, 17) e inaugura los combates de los últimos días (\emph{1 Jn} 2, 18; 4, 3; \emph{1 Tm} 4, 1). Es un tiempo de espera y de vigilia (cf. \emph{Mt} 25, 1-13; \emph{Mc} 13, 33-37).

\ccesec{El glorioso advenimiento de Cristo, esperanza de Israel}

\n{673} Desde la Ascensión, el advenimiento de Cristo en la gloria es inminente (cf. \emph{Ap} 22, 20) aun cuando a nosotros no nos \textquote{toca conocer el tiempo y el momento que ha fijado el Padre con su autoridad} (\emph{Hch} 1, 7; cf. \emph{Mc} 13, 32). Este acontecimiento escatológico se puede cumplir en cualquier momento (cf. \emph{Mt} 24, 44: \emph{1 Ts} 5, 2), aunque tal acontecimiento y la prueba final que le ha de preceder estén \textquote{retenidos} en las manos de Dios (cf. \emph{2 Ts} 2, 3-12).

\n{674} La venida del Mesías glorioso, en un momento determinado de la historia (cf. \emph{Rm} 11, 31), se vincula al reconocimiento del Mesías por \textquote{todo Israel} (\emph{Rm} 11, 26; \emph{Mt} 23, 39) del que \textquote{una parte está endurecida} (\emph{Rm} 11, 25) en \textquote{la incredulidad} (\emph{Rm} 11, 20) respecto a Jesús. San Pedro dice a los judíos de Jerusalén después de Pentecostés: \textquote{Arrepentíos, pues, y convertíos para que vuestros pecados sean borrados, a fin de que del Señor venga el tiempo de la consolación y envíe al Cristo que os había sido destinado, a Jesús, a quien debe retener el cielo hasta el tiempo de la restauración universal, de que Dios habló por boca de sus profetas} (\emph{Hch} 3, 19-21). Y san Pablo le hace eco: \textquote{si su reprobación ha sido la reconciliación del mundo ¿qué será su readmisión sino una resurrección de entre los muertos?} (\emph{Rm} 11, 5). La entrada de \textquote{la plenitud de los judíos} (\emph{Rm} 11, 12) en la salvación mesiánica, a continuación de \textquote{la plenitud de los gentiles} (Rm 11, 25; cf. Lc 21, 24), hará al pueblo de Dios \textquote{llegar a la plenitud de Cristo} (\emph{Ef} 4, 13) en la cual \textquote{Dios será todo en nosotros} (\emph{1 Co} 15, 28).

\ccesec{La última prueba de la Iglesia}

\n{675} Antes del advenimiento de Cristo, la Iglesia deberá pasar por una prueba final que sacudirá la fe de numerosos creyentes (cf. \emph{Lc} 18, 8; \emph{Mt} 24, 12). La persecución que acompaña a su peregrinación sobre la tierra (cf. \emph{Lc} 21, 12; \emph{Jn} 15, 19-20) desvelará el \textquote{misterio de iniquidad} bajo la forma de una impostura religiosa que proporcionará a los hombres una solución aparente a sus problemas mediante el precio de la apostasía de la verdad. La impostura religiosa suprema es la del Anticristo, es decir, la de un seudo-mesianismo en que el hombre se glorifica a sí mismo colocándose en el lugar de Dios y de su Mesías venido en la carne (cf. \emph{2 Ts} 2, 4-12; \emph{1Ts} 5, 2-3;2 \emph{Jn} 7; \emph{1 Jn} 2, 18.22).

\n{676} Esta impostura del Anticristo aparece esbozada ya en el mundo cada vez que se pretende llevar a cabo la esperanza mesiánica en la historia, lo cual no puede alcanzarse sino más allá del tiempo histórico a través del juicio escatológico: incluso en su forma mitigada, la Iglesia ha rechazado esta falsificación del Reino futuro con el nombre de milenarismo (cf. DS 3839), sobre todo bajo la forma política de un mesianismo secularizado, \textquote{intrínsecamente perverso} (cf. Pío XI, carta enc. \emph{Divini Redemptoris}, condenando \textquote{los errores presentados bajo un falso sentido místico de esta especie de falseada redención de los más humildes}; GS 20-21).

\n{677} La Iglesia sólo entrará en la gloria del Reino a través de esta última Pascua en la que seguirá a su Señor en su muerte y su Resurrección (cf. \emph{Ap} 19, 1-9). El Reino no se realizará, por tanto, mediante un triunfo histórico de la Iglesia (cf. \emph{Ap} 13, 8) en forma de un proceso creciente, sino por una victoria de Dios sobre el último desencadenamiento del mal (cf. \emph{Ap} 20, 7-10) que hará descender desde el cielo a su Esposa (cf. \emph{Ap} 21, 2-4). El triunfo de Dios sobre la rebelión del mal tomará la forma de Juicio final (cf. \emph{Ap} 20, 12) después de la última sacudida cósmica de este mundo que pasa (cf. \emph{2 P} 3, 12-13).

\ccesec{La Iglesia, consumada en la gloria}

\n{769} La Iglesia \textquote{sólo llegará a su perfección en la gloria del cielo} (LG 48), cuando Cristo vuelva glorioso. Hasta ese día, \textquote{la Iglesia avanza en su peregrinación a través de las persecuciones del mundo y de los consuelos de Dios} (San Agustín, \emph{De civitate Dei} 18, 51; cf. LG 8). Aquí abajo, ella se sabe en exilio, lejos del Señor (cf. \emph{2Co} 5, 6; LG 6), y aspira al advenimiento pleno del Reino, \textquote{y espera y desea con todas sus fuerzas reunirse con su Rey en la gloria} (LG 5). La consumación de la Iglesia en la gloria, y a través de ella la del mundo, no sucederá sin grandes pruebas. Solamente entonces, \textquote{todos los justos descendientes de Adán, \textquote{desde Abel el justo hasta el último de los elegidos} se reunirán con el Padre en la Iglesia universal} (LG 2).
\end{ccebody}
		\cceth{¡Ven, Señor Jesús!}

\cceref{CEC 451, 671, 1130, 1403, 2817}

\begin{ccebody}
\n{451} La oración cristiana está marcada por el título \textquote{Señor}, ya sea en la invitación a la oración \textquote{el Señor esté con vosotros}, o en su conclusión \textquote{por Jesucristo nuestro Señor} o incluso en la exclamación llena de confianza y de esperanza: \emph{Maran atha} (\textquote{¡el Señor viene!}) o \emph{Marana tha} (\textquote{¡Ven, Señor!}) (\emph{1 Co} 16, 22): \textquote{¡Amén! ¡ven, Señor Jesús!} (\emph{Ap} 22, 20).


\ccesec{\ldots{} esperando que todo le sea sometido}

\n{671} El Reino de Cristo, presente ya en su Iglesia, sin embargo, no está todavía acabado \textquote{con gran poder y gloria} (\emph{Lc} 21, 27; cf. \emph{Mt} 25, 31) con el advenimiento del Rey a la tierra. Este Reino aún es objeto de los ataques de los poderes del mal (cf. \emph{2 Ts} 2, 7), a pesar de que estos poderes hayan sido vencidos en su raíz por la Pascua de Cristo. Hasta que todo le haya sido sometido (cf. \emph{1 Co} 15, 28), y \textquote{mientras no [\ldots{}] haya nuevos cielos y nueva tierra, en los que habite la justicia, la Iglesia peregrina lleva en sus sacramentos e instituciones, que pertenecen a este tiempo, la imagen de este mundo que pasa. Ella misma vive entre las criaturas que gimen en dolores de parto hasta ahora y que esperan la manifestación de los hijos de Dios} (LG 48). Por esta razón los cristianos piden, sobre todo en la Eucaristía (cf. \emph{1 Co} 11, 26), que se apresure el retorno de Cristo (cf. \emph{2 P} 3, 11-12) cuando suplican: \textquote{Ven, Señor Jesús} (\emph{Ap} 22, 20; cf. \emph{1 Co} 16, 22; \emph{Ap} 22, 17-20).

\ccesec{Sacramentos de la vida eterna}

\n{1130} La Iglesia celebra el Misterio de su Señor \textquote{hasta que él venga} y \textquote{Dios sea todo en todos} (\emph{1 Co} 11, 26; 15, 28). Desde la era apostólica, la liturgia es atraída hacia su término por el gemido del Espíritu en la Iglesia: \emph{¡Marana tha!} (\emph{1 Co} 16,22). La liturgia participa así en el deseo de Jesús: \textquote{Con ansia he deseado comer esta Pascua con vosotros [\ldots{}] hasta que halle su cumplimiento en el Reino de Dios} (\emph{Lc} 22,15-16). En los sacramentos de Cristo, la Iglesia recibe ya las arras de su herencia, participa ya en la vida eterna, aunque \textquote{aguardando la feliz esperanza y la manifestación de la gloria del Gran Dios y Salvador nuestro Jesucristo} (\emph{Tt} 2,13). \textquote{El Espíritu y la Esposa dicen: ¡Ven! [\ldots{}] ¡Ven, Señor Jesús!} (\emph{Ap} 22,17.20).

\begin{quote} 
	Santo Tomás resume así las diferentes dimensiones del signo sacramental: \textquote{\emph{Unde sacramentum est signum rememorativum eius quod praecessit, scilicet passionis Christi; et desmonstrativum eius quod in nobis efficitur per Christi passionem, scilicet gratiae; et prognosticum, id est, praenuntiativum futurae gloriae}} (\textquote{Por eso el sacramento es un signo que rememora lo que sucedió, es decir, la pasión de Cristo; es un signo que demuestra lo que se realiza en nosotros en virtud de la pasión de Cristo, es decir, la gracia; y es un signo que anticipa, es decir, que preanuncia la gloria venidera}) (\emph{Summa theologiae} 3, q. 60, a. 3, c.) 
\end{quote}

\n{1403} En la última Cena, el Señor mismo atrajo la atención de sus discípulos hacia el cumplimiento de la Pascua en el Reino de Dios: \textquote{Y os digo que desde ahora no beberé de este fruto de la vid hasta el día en que lo beba con vosotros, de nuevo, en el Reino de mi Padre} (\emph{Mt} 26,29; cf. \emph{Lc} 22,18; \emph{Mc} 14,25). Cada vez que la Iglesia celebra la Eucaristía recuerda esta promesa y su mirada se dirige hacia \textquote{el que viene} (\emph{Ap} 1,4). En su oración, implora su venida: \emph{Marana tha} (\emph{1 Co} 16,22), \textquote{Ven, Señor Jesús} (\emph{Ap} 22,20), \textquote{que tu gracia venga y que este mundo pase} (\emph{Didaché} 10,6).

\ccesec{Venga a nosotros tu Reino}

\n{2817} Esta petición es el \emph{Marana Tha}, el grito del Espíritu y de la Esposa: \textquote{Ven, Señor Jesús}:

\begin{quote} \textquote{Incluso aunque esta oración no nos hubiera mandado pedir el advenimiento del Reino, habríamos tenido que expresar esta petición, dirigiéndonos con premura a la meta de nuestras esperanzas. Las almas de los mártires, bajo el altar, invocan al Señor con grandes gritos: \textquote{¿Hasta cuándo, Dueño santo y veraz, vas a estar sin hacer justicia por nuestra sangre a los habitantes de la tierra?} (\emph{Ap} 6, 10). En efecto, los mártires deben alcanzar la justicia al fin de los tiempos. Señor, ¡apresura, pues, la venida de tu Reino!} (Tertuliano, \emph{De oratione}, 5, 2-4). \end{quote}
\end{ccebody}
		\cceth{La vigilancia humilde del corazón}

\cceref{CEC 2729-2733}

\ccesec{Frente a las dificultades de la oración}

\begin{ccebody}
	
	\n{2729} La dificultad habitual de la oración es la \emph{distracción}. En la oración vocal, la distracción puede referirse a las palabras y al sentido de estas. La distracción, de un modo más profundo, puede referirse a Aquél al que oramos, tanto en la oración vocal (litúrgica o personal), como en la meditación y en la oración contemplativa. Dedicarse a perseguir las distracciones es caer en sus redes; basta con volver a nuestro corazón: la distracción descubre al que ora aquello a lo que su corazón está apegado. Esta humilde toma de conciencia debe empujar al orante a ofrecerse al Señor para ser purificado. El combate se decide cuando se elige a quién se desea servir (cf. \emph{Mt} 6,21.24).

	\n{2730} Mirado positivamente, el combate contra el ánimo posesivo y dominador es la vigilancia, la sobriedad del corazón. Cuando Jesús insiste en la vigilancia, es siempre en relación a Él, a su Venida, al último día y al \textquote{hoy}. El esposo viene en mitad de la noche; la luz que no debe apagarse es la de la fe: \textquote{Dice de ti mi corazón: busca su rostro} (\emph{Sal} 27, 8).

	\n{2731} Otra dificultad, especialmente para los que quieren sinceramente orar, es la \emph{sequedad}. Forma parte de la oración en la que el corazón está desprendido, sin gusto por los pensamientos, recuerdos y sentimientos, incluso espirituales. Es el momento en que la fe es más pura, la fe que se mantiene firme junto a Jesús en su agonía y en el sepulcro. \textquote{El grano de trigo, si [\ldots{}] muere, da mucho fruto} (\emph{Jn} 12, 24). Si la sequedad se debe a falta de raíz, porque la Palabra ha caído sobre roca, no hay éxito en el combate sin una mayor conversión (cf. \emph{Lc} 8, 6. 13).

	\ccesec{Frente a las tentaciones en la oración}

	\n{2732} La tentación más frecuente, la más oculta, es nuestra \emph{falta de fe}. Esta se expresa menos en una incredulidad declarada que en unas preferencias de hecho. Cuando se empieza a orar, se presentan como prioritarios mil trabajos y cuidados que se consideran más urgentes; una vez más, es el momento de la verdad del corazón y de su más profundo deseo. Mientras tanto, nos volvemos al Señor como nuestro único recurso; pero ¿alguien se lo cree verdaderamente? Consideramos a Dios como asociado a la alianza con nosotros, pero nuestro corazón continúa en la arrogancia. En cualquier caso, la falta de fe revela que no se ha alcanzado todavía la disposición propia de un corazón humilde: \textquote{Sin mí, no podéis hacer nada} (\emph{Jn} 15, 5).

	\n{2733} Otra tentación a la que abre la puerta la presunción es la \emph{acedia}. Los Padres espirituales entienden por ella una forma de aspereza o de desabrimiento debidos a la pereza, al relajamiento de la ascesis, al descuido de la vigilancia, a la negligencia del corazón. \textquote{El espíritu [\ldots{}] está pronto pero la carne es débil} (\emph{Mt} 26, 41). Cuanto más alto es el punto desde el que alguien toma decisiones, tanto mayor es la dificultad. El desaliento, doloroso, es el reverso de la presunción. Quien es humilde no se extraña de su miseria; ésta le lleva a una mayor confianza, a mantenerse firme en la constancia.

	\begin{quote}
		
		El que quisiere ver cuánto ha aprovechado en este camino de Dios, mire cuánto crece cada día en humildad interior y exterior. ¿Cómo sufre las injusticias de los otros? ¿Cómo sabe dar pasada a las flaquezas ajenas? ¿Cómo acude a las necesidades de sus prójimos? ¿Cómo se compadece y no se indigna contra los defectos ajenos? ¿Cómo sabe esperar en Dios en el tiempo de la tribulación? ¿Cómo rige su lengua? ¿Cómo guarda su corazón? ¿Cómo trae domada su carne con todos sus apetitos y sentidos? ¿Cómo se sabe valer en las prosperidades y adversidades? ¿Cómo se repara y provee en todas las cosas con gravedad y discreción?
	
		Y, sobre todo esto, mire si está muerto el amor de la honra, y del regalo, y del mundo, y según lo que en esto hubiere aprovechado o desaprovechado, así se juzgue, y no según lo que siente o no siente de Dios. Y por esto siempre ha de tener él un ojo, y el más principal en la mortificación, y el otro en la oración, porque esa misma mortificación no se puede perfectamente alcanzar sin el socorro de la oración.
	
		\textbf{San Pedro de Alcántara}, \emph{Tratado sobre la Oración,} capítulo 5.
		
	\end{quote}

\end{ccebody}
		%\input{../../CEC/}
		%\input{../../CEC/}
		%\input{../../CEC/}
		%\input{../../CEC/}
		%\input{../../CEC/}		