\chapter{Domingo II de Navidad}

\section{Lecturas}

\rtitle{PRIMERA LECTURA}

\rbook{Del libro de Ben Sirá} \rred{24, 1-2. 8-12}

\rtheme{La sabiduría de Dios habitó en el pueblo escogido}

\begin{scripture}
	\begin{readprose}
		La sabiduría hace su propia alabanza	
		encuentra su honor en Dios	
		y se gloría en medio de su pueblo.
		
		En la asamblea del Altísimo	
		abre su boca y se gloría ante el Poderoso.
		
		\textquote{El Creador del universo me dio una orden, \\el que me había creado estableció mi morada \\y me dijo: \textquote{Pon tu tienda en Jacob, \\y fija tu heredad en Israel}.
			
			Desde el principio, antes de los siglos, me creó,
			y nunca jamás dejaré de existir.
			
			Ejercí mi ministerio en la Tienda santa delante de él,
			y así me establecí en Sión.
			
			En la ciudad amada encontré descanso,
			y en Jerusalén reside mi poder.
			
			Arraigué en un pueblo glorioso,
			en la porción del Señor, en su heredad}.
	\end{readprose}
\end{scripture}

\rtitle{SALMO RESPONSORIAL}

\rbook{Salmo} \rred{147, 12-13. 14-15. 19-20}

\rtheme{El Verbo se hizo carne y habitó entre nosotros}

\begin{psbody}
	Glorifica al Señor, Jerusalén;
	alaba a tu Dios, Sion.
	Que ha reforzado los cerrojos de tus puertas,
	y ha bendecido a tus hijos dentro de ti.
	
	Ha puesto paz en tus fronteras,
	te sacia con flor de harina.
	Él envía su mensaje a la tierra,
	y su palabra corre veloz.
	
	Anuncia su palabra a Jacob,
	sus decretos y mandatos a Israel;
	con ninguna nación obró así,
	ni les dio a conocer sus mandatos.
\end{psbody}


\rtitle{SEGUNDA LECTURA}

\rbook{De la carta del apóstol san Pablo a los Efesios} \rred{1, 3-6. 15-18}

\rtheme{Él nos ha destinado por medio de Jesucristo a ser sus hijos}

\begin{scripture}
	\begin{readprose}
		Bendito sea Dios, Padre de Nuestro Señor Jesucristo,
		que nos ha bendecido en Cristo 
		con toda clase de bendiciones espirituales en los cielos.
		
		Él nos eligió en Cristo antes de la fundación del mundo 
		para que fuésemos santos e intachables ante él por el amor.
		
		Él nos ha destinado por medio de Jesucristo, 
		según el beneplácito de su voluntad, 
		a ser sus hijos, 
		para alabanza de la gloria de su gracia, 
		que tan generosamente nos ha concedido en el Amado.
		
		Por eso, habiendo oído hablar de vuestra fe en Cristo y de vuestro amor a todos los santos, no ceso de dar gracias por vosotros, recordándoos en mis oraciones, a fin de que el Dios de nuestro Señor Jesucristo, el Padre de la gloria, os dé espíritu de sabiduría y revelación para conocerlo, e ilumine los ojos de vuestro corazón para que comprendáis cuál es la esperanza a la que os llama, cuál la riqueza de gloria que da en herencia a los santos.
	\end{readprose}
\end{scripture}


\rtitle{EVANGELIO}

\rbook{Del Evangelio según san Juan} \rred{1, 1-18}

\rtheme{El Verbo se hizo carne y habitó entre nosotros}

\begin{scripture}
	En el principio existía el Verbo, y el Verbo estaba junto a Dios, y el Verbo era Dios.
	
	Él estaba en el principio junto a Dios.
	
	Por medio de él se hizo todo, y sin él no se hizo nada de cuanto se ha hecho.
	
	En él estaba la vida, y la vida era la luz de los hombres.
	
	Y la luz brilla en la tiniebla, y la tiniebla no lo recibió.
	
	Surgió un hombre enviado por Dios, que se llamaba Juan:
	
	este venía como testigo, para dar testimonio de la luz, para que todos creyeran por medio de él.
	
	No era él la luz, sino el que daba testimonio de la luz.
	
	El Verbo era la luz verdadera, que alumbra a todo hombre, viniendo al mundo.
	
	En el mundo estaba; el mundo se hizo por medio de él, y el mundo no lo conoció.
	
	Vino a su casa, y los suyos no lo recibieron.
	
	Pero a cuantos lo recibieron, les dio poder de ser hijos de Dios, a los que creen en su nombre.
	
	Estos no han nacido de sangre, ni de deseo de carne, ni de deseo de varón, sino que han nacido de Dios.
	
	Y el Verbo se hizo carne y habitó entre nosotros, y hemos contemplado su gloria: gloria como del Unigénito del Padre, lleno de gracia y de verdad.
	
	Juan da testimonio de él y grita diciendo: \textquote{Este es de quien dije: El que viene detrás de mí se ha puesto delante de mí, porque existía antes que yo}.
	
	Pues de su plenitud todos hemos recibido, gracia tras gracia.
	
	Porque la ley se dio por medio de Moisés, la gracia y la verdad nos han llegado por medio de Jesucristo.
	
	A Dios nadie lo ha visto jamás: Dios unigénito, que está en el seno del Padre, es quien lo ha dado a conocer.
\end{scripture}


\section{Comentario Patrístico}

%Ya está en Misa del Día
%\subsection{San León Magno, papa}

\ptheme{El nacimiento del Señor es el nacimiento de la paz}

\src{Sermón 6, 2-3 en la Natividad del Señor: \\PL 54, 213-216.\cite{LeonMagno_PL054_0213}}

\begin{body}
	\ltr{A}{unque} aquella infancia, que la majestad del Hijo de Dios se dignó hacer suya, tuvo como continuación la plenitud de una edad adulta, y, después del triunfo de su pasión y resurrección, todas las acciones de su estado de humildad, que el Señor asumió por nosotros, pertenecen ya al pasado, la festividad de hoy renueva ante nosotros los sagrados comienzos de Jesús, nacido de la Virgen María; de modo que, mientras adoramos el nacimiento de nuestro Salvador, resulta que estamos celebrando nuestro propio comienzo.
	
	Efectivamente, la generación de Cristo es el comienzo del pueblo cristiano, y el nacimiento de la cabeza lo es al mismo tiempo del cuerpo.
	
	Aunque cada uno de los que llama el Señor a formar parte de su pueblo sea llamado en un tiempo determinado y aunque todos los hijos de la Iglesia hayan sido llamados cada uno en días distintos, con todo, la totalidad de los fieles, nacida en la fuente bautismal, ha nacido con Cristo en su nacimiento, del mismo modo que ha sido crucificada con Cristo en su pasión, ha sido resucitada en su resurrección y ha sido colocada a la derecha del Padre en su ascensión.
	
	Cualquier hombre que cree ---en cualquier parte del mundo---, y se regenera en Cristo, una vez interrumpido el camino de su vieja condición original, pasa a ser un nuevo hombre al renacer; y ya no pertenece a la ascendencia de su padre carnal, sino a la simiente del Salvador, que se hizo precisamente Hijo del hombre, para que nosotros pudiésemos llegar a ser hijos de Dios.
	
	Pues si él no hubiera descendido hasta nosotros revestido de esta humilde condición, nadie hubiera logrado llegar hasta él por sus propios méritos.
	
	Por eso, la misma magnitud del beneficio otorgado exige de nosotros una veneración proporcionada a la excelsitud de esta dádiva. Y, como el bienaventurado Apóstol nos enseña, \emph{no hemos recibido el espíritu de este mundo, sino el Espíritu que procede de Dios}, a fin de que conozcamos lo que Dios nos ha otorgado; y el mismo Dios sólo acepta como culto piadoso el ofrecimiento de lo que él nos ha concedido.
	
	¿Y qué podremos encontrar en el tesoro de la divina largueza tan adecuado al honor de la presente festividad como la paz, lo primero que los ángeles pregonaron en el nacimiento del Señor?
	
	La paz es la que engendra los hijos de Dios, alimenta el amor y origina la unidad, es el descanso de los bienaventurados y la mansión de la eternidad. El fin propio de la paz y su fruto específico consiste en que se unan a Dios los que el mismo Señor separa del mundo.
	
	Que los que \emph{no han nacido de sangre, ni de amor carnal, ni de amor humano, sino de Dios}, ofrezcan, por tanto, al Padre la concordia que es propia de hijos pacíficos, y que todos los miembros de la adopción converjan hacia el Primogénito de la nueva creación, que vino a cumplir la voluntad del que le enviaba y no la suya: puesto que la gracia del Padre no adoptó como herederos a quienes se hallaban en discordia e incompatibilidad, sino a quienes amaban y sentían lo mismo. Los que han sido reformados de acuerdo con una sola imagen deben ser concordes en el espíritu.
	
	El nacimiento del Señor es el nacimiento de la paz: y así dice el Apóstol: \emph{El es nuestra paz; él ha hecho de los dos pueblos una sola cosa,} ya que, tanto los judíos como los gentiles, por su medio \emph{podemos acercarnos al Padre con un mismo Espíritu}.
\end{body}

%\newsection

\subsection{San Máximo Confesor}

\ptheme{Misterio siempre nuevo}

\src{De las Cinco Centurias, Centuria 1, 8-13: \\PG 90, 1182-86.\cite{MaximoConfesor_PG090_1182}}

\begin{body}
	\ltr{L}{a} Palabra de Dios, nacida una vez en la carne (lo que nos indica la querencia de su benignidad y humanidad), vuelve a nacer siempre gustosamente en el espíritu para quienes lo desean; vuelve a hacerse niño, y se vuelve a formar en aquellas virtudes; y la amplitud de su grandeza no disminuye por malevolencia o envidia, sino que se manifiesta a sí mismo en la medida en que sabe que lo puede asimilar el que lo recibe, y así, al mismo tiempo que explora discretamente la capacidad de quienes desean verlo, sigue manteniéndose siempre fuera del alcance de su percepción, a causa de la excelencia del misterio.
	
	Por lo cual, el santo Apóstol, considerando sabiamente la fuerza del misterio, exclama: \emph{Jesucristo es el mismo ayer y hoy y siempre;} ya que entendía el misterio como algo siempre nuevo, al que nunca la comprensión de la mente puede hacer envejecer.
	
	Nace Cristo Dios, hecho hombre mediante la incorporación de una carne dotada de alma inteligente; el mismo que había otorgado a las cosas proceder de la nada. Mientras tanto, brilla en lo alto la estrella del Oriente y conduce a los Magos al lugar en que yace la Palabra encarnada; con lo que muestra que hay en la ley y los profetas una palabra místicamente superior, que dirige a las gentes a la suprema luz del conocimiento.
	
	Así pues, la palabra de la ley y de los profetas, entendida alegóricamente, conduce, como una estrella, al pleno conocimiento de Dios a aquellos que fueron llamados por la fuerza de la gracia, de acuerdo con el designio divino.
	
	Dios se hace efectivamente hombre perfecto, sin alterar nada de lo que es propio de la naturaleza, a excepción del pecado (pues ni el mismo pecado era propio de la naturaleza). Se hace efectivamente hombre perfecto a fin de provocar, con la vista del manjar de su carne, la voracidad insaciable y ávida del dragón infernal; y abatirlo por completo cuando ingiriera una carne que habría de convertírsele en veneno, porque en ella se hallaba oculto el poder de la divinidad. Esta carne sería al mismo tiempo remedio de la naturaleza humana, ya que el mismo poder divino presente en aquélla habría de restituir la naturaleza humana a la gracia primera.
	
	Y así como el dragón, deslizando su veneno en el árbol de la ciencia, había corrompido con su sabor la naturaleza, de la misma manera, al tratar de devorar la carne del Señor, se vio corrompido y destruido por la virtud de la divinidad que en ella residía.
	
	Inmenso misterio de la divina encarnación, que sigue siendo siempre misterio; pues, ¿de qué modo puede la Palabra hecha carne seguir siendo su propia persona esencialmente, siendo así que la misma persona existe al mismo tiempo con todo su ser en Dios Padre? ¿Cómo la Palabra, que es toda ella Dios por naturaleza, se hizo toda ella por naturaleza hombre, sin detrimento de ninguna de las dos naturalezas: ni de la divina, en cuya virtud es Dios, ni de la nuestra, en virtud de la cual se hizo hombre? Sólo la fe capta estos misterios, ella precisamente que es la sustancia y la base de todas aquellas realidades que exceden la percepción y razón de la mente humana en todo su alcance.
\end{body}



\newsection


\section{Homilías}

\homiliasABC

\subsection{San Juan Pablo II, papa}

\subsubsection{Homilía (1983):}

\src{Celebración del Te Deum de acción de gracias por el fin de año. \\Sábado 31 de diciembre de 1983.}

\rbr{Pequeñas partes de la homilía fueron adaptadas para situarla en un contexto de inicio de año.}

\begin{body}
	\textquote{Hijos, esta es la última hora} (\emph{1 Jn} 2, 18).
	
	\ltr[1. ]{E}{stamos} reunidos aquí, como siempre, en la \textquote{Iglesia de Jesús} para prepararnos al encuentro con la última hora del año del Señor {[}1983{]}, y la liturgia dirige nuestro pensamiento hacia Dios, en quien todo lo existente encuentra su comienzo y su fin.
	
	El Evangelio de San Juan nos invita a volvernos a esta Palabra que en un principio estaba con Dios.
	
	He aquí la Palabra eterna: \textquote{todo fue hecho por él, y sin él nada se hizo de todo lo que existe} (\emph{Jn} 1, 3).
	
	Por tanto, también este año, que pasa como componente del tiempo humano y del paso cósmico, \textquote{se hizo} por medio del Verbo Eterno que \textquote{estaba en el principio con Dios} (cf. \emph{Jn} 1, 2) y que era Dios (cf. \emph{Jn} 1, 1).
	
	{[}Este{]} año del Señor (\ldots{}), queremos referirlo al principio absoluto. Deseamos redescubrir su lugar en la eternidad que no pasa.
	
	2. \textquote{Y el Verbo se hizo carne y habitó entre nosotros} (\emph{Jn} 1, 14). (\ldots{}) Dios en su Hijo Eterno acogió nuestro tiempo humano y todo el pasado cósmico. Nació la noche de Belén de la Inmaculada Virgen María bajo la protección del carpintero José de Nazaret. Nació en un establo porque \textquote{no había lugar para ellos en la posada} (\emph{Lc} 2, 7), porque \textquote{su propia gente no lo aceptó} (\emph{Jn} 1, 11).
	
	Todas las decepciones, tristezas y sufrimientos de nuestro mundo humano ya están insertadas de cierta manera en este Nacimiento de Dios en la tierra. Y estarán incluidas para todos los días de la peregrinación terrestre de Jesús de Nazaret hasta llegar a Getsemaní y a la cruz. En unión con él podemos vivir cada una de nuestras acciones a lo largo del tiempo. Podemos caminar los días y las horas de este año con la memoria y el corazón, también, y, especialmente aquellas que más nos hagan sufrir, porque Cristo está presente en ellas de una manera particular. Está presente a través del misterio de la Redención.
	
	3. (\ldots{}) Jesucristo, crucificado y resucitado, que está en la gloria del Padre, existe simultáneamente en el Cuerpo de su Iglesia.
	
	El Unigénito, lleno de gracia y de verdad, obtiene la gloria del Padre (cf. \emph{Jn} 1, 14) y, al mismo tiempo, por esta gracia y verdad está con nosotros, está en su Iglesia, porque \textquote{la gracia y la verdad vino (a nosotros) por Jesucristo} (\emph{Jn} 1, 17).
	
	Esta aceptación de la gracia y la verdad cuando el Verbo se hizo carne determina que el mundo y el hombre están envueltos en el misterio de la Redención. El hombre y el mundo a través de este misterio, de una manera nueva, existen en Dios por la obra de Cristo Redentor.
	
	En {[}este año estamos invitados, junto a toda{]} la Iglesia a sumergirnos de un modo nuevo en \textquote{su plenitud} (\emph{Jn} 1, 16): en la plenitud del Redentor del mundo para recibir de esta plenitud \textquote{gracia sobre gracia} (\emph{Jn} 1, 16).
	
	{[} \ldots{}{]}
	
	6. {[}En este año{]} vayamos al encuentro del Señor, dando gloria a Dios, con espíritu de acción de gracias y pidiendo perdón.
	
	Que la gracia y la verdad que \textquote{vinieron (a nosotros) por Jesucristo} nos acompañen siempre. En el misterio de la Redención, esta gracia y esta verdad seguirán guiando al hombre y al mundo al encuentro con Aquel \textquote{que es, que era y que ha de venir} (\emph{Ap} 1, 8): al encuentro con Dios que es la eternidad y la santidad. Amén.
\end{body}

\newsection

\subsection{Benedicto XVI, papa}

\subsubsection{Ángelus (2010): Poner en Dios nuestra esperanza}

\src{3 de enero del 2010.}

\begin{body}
	\ltr{E}{n} este domingo ---segundo después de Navidad y primero del año nuevo--- me alegra renovar a todos mi deseo de todo bien en el Señor. No faltan los problemas, en la Iglesia y en el mundo, al igual que en la vida cotidiana de las familias. Pero, gracias a Dios, nuestra esperanza no se basa en pronósticos improbables ni en las previsiones económicas, aunque sean importantes. Nuestra esperanza está en Dios, no en el sentido de una religiosidad genérica, o de un fatalismo disfrazado de fe. Nosotros confiamos en el Dios que en Jesucristo ha revelado de modo completo y definitivo su voluntad de estar con el hombre, de compartir su historia, para guiarnos a todos a su reino de amor y de vida. Y esta gran esperanza anima y a veces corrige nuestras esperanzas humanas.
	
	De esa revelación nos hablan hoy, en la liturgia eucarística, \textbf{tres lecturas bíblicas} de una riqueza extraordinaria: el capítulo 24 del \emph{Libro del Sirácida}, el himno que abre la \emph{Carta a los Efesios} de san Pablo y el prólogo del \emph{Evangelio de san Juan}. Estos textos afirman que Dios no sólo es el creador del universo ---aspecto común también a otras religiones--- sino que es Padre, que \textquote{nos eligió antes de crear el mundo (\ldots{}) predestinándonos a ser sus hijos adoptivos} (\emph{Ef} 1, 4-5) y que por esto llegó hasta el punto inconcebible de hacerse hombre: \textquote{El Verbo se hizo carne y acampó entre nosotros} (\emph{Jn} 1, 14). El misterio de la Encarnación de la Palabra de Dios fue preparado en el Antiguo Testamento, especialmente donde la Sabiduría divina se identifica con la Ley de Moisés. En efecto, la misma Sabiduría afirma: \textquote{El creador del universo me hizo plantar mi tienda, y me dijo: \textquote{Pon tu tienda en Jacob, entra en la heredad de Israel}} (\emph{Si} 24, 8). En Jesucristo, la Ley de Dios se ha hecho testimonio vivo, escrita en el corazón de un hombre en el que, por la acción del Espíritu Santo, reside corporalmente toda la plenitud de la divinidad (cf. \emph{Col} 2, 9).
	
	Queridos amigos, esta es la verdadera razón de la esperanza de la humanidad: la historia tiene un sentido, porque en ella \textquote{habita} la Sabiduría de Dios. Sin embargo, el designio divino no se cumple automáticamente, porque es un proyecto de amor, y el amor genera libertad y pide libertad. Ciertamente, el reino de Dios viene, más aún, ya está presente en la historia y, gracias a la venida de Cristo, ya ha vencido a la fuerza negativa del maligno. Pero cada hombre y cada mujer es responsable de acogerlo en su vida, día tras día. Por eso, también 2010 será un año más o menos \textquote{bueno} en la medida en que cada uno, de acuerdo con sus responsabilidades, sepa colaborar con la gracia de Dios. Por lo tanto, dirijámonos a la Virgen María, para aprender de ella esta actitud espiritual. El Hijo de Dios tomó carne de ella, con su consentimiento. Cada vez que el Señor quiere dar un paso adelante, junto con nosotros, hacia la \textquote{tierra prometida}, llama primero a nuestro corazón; espera, por decirlo así, nuestro \textquote{sí}, tanto en las pequeñas decisiones como en las grandes. Que María nos ayude a aceptar siempre la voluntad de Dios, con humildad y valentía, a fin de que también las pruebas y los sufrimientos de la vida contribuyan a apresurar la venida de su reino de justicia y de paz.
\end{body}

\subsubsection{Ángelus (2011): Entrar en las profundidades de Dios}

\src{2 de enero del 2011.}

\begin{body}
	\ltr{O}{s} renuevo a todos mis mejores deseos para el año nuevo y doy las gracias a cuantos me han enviado mensajes de cercanía espiritual. La liturgia de este domingo vuelve a proponer el \textbf{Prólogo del \emph{Evangelio de san Juan}}, proclamado solemnemente en el día de Navidad. Este admirable texto expresa, en forma de himno, el misterio de la Encarnación, que predicaron los testigos oculares, los Apóstoles, especialmente san Juan, cuya fiesta, no por casualidad, se celebra el 27 de diciembre. Afirma san Cromacio de Aquileya que \textquote{Juan era el más joven de todos los discípulos del Señor; el más joven por edad, pero ya anciano por la fe} (Sermo II, 1 \emph{De Sancto Iohanne Evangelista:} CCL 9a, 101). Cuando leemos: \textquote{En el principio existía el Verbo y el Verbo estaba con Dios, y el Verbo era Dios} (\emph{Jn} 1, 1), el Evangelista ---al que tradicionalmente se compara con un águila--- se eleva por encima de la historia humana escrutando las profundidades de Dios; pero muy pronto, siguiendo a su Maestro, vuelve a la dimensión terrena diciendo: \textquote{Y el Verbo se hizo carne} (\emph{Jn} 1, 14). El Verbo es \textquote{una realidad viva: un Dios que\ldots{} se comunica haciéndose él mismo hombre} (J. Ratzinger, \emph{Teologia della liturgia}, LEV 2010, p. 618). En efecto, atestigua Juan, \textquote{puso su morada entre nosotros, y hemos contemplado su gloria} (\emph{Jn} 1, 14). \textquote{Se rebajó hasta asumir la humildad de nuestra condición ---comenta san León Magno--- sin que disminuyera su majestad} (\emph{Tractatus} XXI, 2: CCL 138, 86-87). Leemos también en el Prólogo: \textquote{De su plenitud hemos recibido todos, gracia por gracia} (\emph{Jn} 1, 16). \textquote{¿Cuál es la primera gracia que hemos recibido? ---se pregunta san Agustín, y responde--- Es la fe}. La segunda gracia, añade en seguida, es \textquote{la vida eterna} (\emph{Tractatus in Ioh}. III, 8.9: ccl 36, 24.25).
\end{body}

\newsection

\subsection{Francisco, papa}

\subsubsection{Ángelus (2014): Profundizar el sentido de su nacimiento}

\src{5 de enero del 2014.}

\begin{body}
	\ltr{L}{a} liturgia de este domingo nos vuelve a proponer, en el \textbf{Prólogo del Evangelio de san Juan}, el significado más profundo del Nacimiento de Jesús. Él es la Palabra de Dios que se hizo hombre y puso su \textquote{tienda}, su morada entre los hombres. Escribe el evangelista: \textquote{El Verbo se hizo carne y habitó entre nosotros} (\emph{Jn} 1, 14). En estas palabras, que no dejan de asombrarnos, está todo el cristianismo. Dios se hizo mortal, frágil como nosotros, compartió nuestra condición humana, excepto en el pecado, pero cargó sobre sí mismo los nuestros, como si fuesen propios. Entró en nuestra historia, llegó a ser plenamente Dios-con-nosotros. El nacimiento de Jesús, entonces, nos muestra que Dios quiso unirse a cada hombre y a cada mujer, a cada uno de nosotros, para comunicarnos su vida y su alegría.
	
	Así Dios es Dios con nosotros, Dios que nos ama, Dios que camina con nosotros. Éste es el mensaje de Navidad: el Verbo se hizo carne. De este modo la Navidad nos revela el amor inmenso de Dios por la humanidad. De aquí se deriva también el entusiasmo, nuestra esperanza de cristianos, que en nuestra pobreza sabemos que somos amados, visitados y acompañados por Dios; y miramos al mundo y a la historia como el lugar donde caminar juntos con Él y entre nosotros, hacia los cielos nuevos y la tierra nueva. Con el nacimiento de Jesús nació una promesa nueva, nació un mundo nuevo, pero también un mundo que puede ser siempre renovado. Dios siempre está presente para suscitar hombres nuevos, para purificar el mundo del pecado que lo envejece, del pecado que lo corrompe. En lo que la historia humana y la historia personal de cada uno de nosotros pueda estar marcada por dificultades y debilidades, la fe en la Encarnación nos dice que Dios es solidario con el hombre y con su historia. Esta proximidad de Dios al hombre, a cada hombre, a cada uno de nosotros, es un don que no se acaba jamás. ¡Él está con nosotros! ¡Él es Dios con nosotros! Y esta cercanía no termina jamás. He aquí el gozoso anuncio de la Navidad: la luz divina, que inundó el corazón de la Virgen María y de san José, y guio los pasos de los pastores y de los magos, brilla también hoy para nosotros.
	
	En el misterio de la Encarnación del Hijo de Dios hay también un aspecto vinculado con la libertad humana, con la libertad de cada uno de nosotros. En efecto, el Verbo de Dios pone su tienda entre nosotros, pecadores y necesitados de misericordia. Y todos nosotros deberíamos apresurarnos a recibir la gracia que Él nos ofrece. En cambio, continúa el Evangelio de san Juan, \textquote{los suyos no lo recibieron} (v. 11). Incluso nosotros muchas veces lo rechazamos, preferimos permanecer en la cerrazón de nuestros errores y en la angustia de nuestros pecados. Pero Jesús no desiste y no deja de ofrecerse a sí mismo y ofrecer su gracia que nos salva. Jesús es paciente, Jesús sabe esperar, nos espera siempre. Esto es un mensaje de esperanza, un mensaje de salvación, antiguo y siempre nuevo. Y nosotros estamos llamados a testimoniar con alegría este mensaje del Evangelio de la vida, del Evangelio de la luz, de la esperanza y del amor. Porque el mensaje de Jesús es éste: vida, luz, esperanza y amor.
	
	Que María, Madre de Dios y nuestra Madre de ternura, nos sostenga siempre, para que permanezcamos fieles a la vocación cristiana y podamos realizar los deseos de justicia y de paz que llevamos en nosotros al inicio de este nuevo año.
\end{body}

\subsubsection{Ángelus (2020): Revelación plena del plan de Dios}

\src{5 de enero del 2020.}

\begin{body}
	\ltr{E}{n} este segundo domingo de la Navidad, las lecturas bíblicas nos ayudan a alargar la mirada, para tomar una conciencia plena del significado del nacimiento de Jesús.
	
	El \textbf{comienzo del Evangelio de San Juan} nos muestra una impactante novedad: el Verbo eterno, el Hijo de Dios, \textquote{se hizo carne} (v. 14). No sólo vino a vivir entre la gente, sino que se convirtió en uno del pueblo, ¡uno de nosotros! Después de este acontecimiento, para dirigir nuestras vidas, ya no tenemos sólo una ley, una institución, sino una Persona, una Persona divina, Jesús, que guía nuestras vidas, nos hace ir por el camino porque Él lo hizo antes.
	
	\textbf{San Pablo} bendice a Dios por su plan de amor realizado en Jesucristo (cf. \emph{Efesios} 1, 3-6; 15-18). En este plan, cada uno de nosotros encuentra su vocación fundamental. ¿Y cuál es? Esto es lo que dice Pablo: estamos predestinados a ser hijos de Dios por medio de Jesucristo. El Hijo de Dios se hizo hombre para hacernos a nosotros, hombres, hijos de Dios. Por eso el Hijo eterno se hizo carne: para introducirnos en su relación filial con el Padre.
	
	Así pues, hermanos y hermanas, mientras continuamos contemplando el admirable signo del belén, la liturgia de hoy nos dice que el Evangelio de Cristo no es una fábula, ni un mito, ni un cuento moralizante, no. El Evangelio de Cristo es la plena revelación del plan de Dios, el plan de Dios para el hombre y el mundo. Es un mensaje a la vez sencillo y grandioso, que nos lleva a preguntarnos: ¿qué plan concreto tiene el Señor para mí, actualizando aún hoy su nacimiento entre nosotros?
	
	Es el \textbf{apóstol Pablo} quien nos sugiere la respuesta: \textquote{{[}Dios{]} nos ha elegido [\ldots{}] para ser santos e inmaculados en su presencia, en el amor} (v. 4). Este es el significado de la Navidad. Si el Señor sigue viniendo entre nosotros, si sigue dándonos el don de su Palabra, es para que cada uno de nosotros pueda responder a esta llamada: ser santos en el amor. La santidad pertenece a Dios, es comunión con Él, transparencia de su infinita bondad. La santidad es guardar el don que Dios nos ha dado. Simplemente esto: guardar la gratuidad. En esto consiste ser santo. Por tanto, quien acepta la santidad en sí mismo como un don de gracia, no puede dejar de traducirla en acciones concretas en la vida cotidiana. Este don, esta gracia que Dios me ha dado, la traduzco en una acción concreta en la vida cotidiana, en el encuentro con los demás. Esta caridad, esta misericordia hacia el prójimo, reflejo del amor de Dios, al mismo tiempo purifica nuestro corazón y nos dispone al perdón, haciéndonos \textquote{inmaculados} día tras día. Pero inmaculados no en el sentido de que yo elimino una mancha: inmaculados en el sentido de que Dios entra en nosotros, el don, la gratuidad de Dios entra en nosotros y nosotros lo guardamos y lo damos a los demás.
	
	Que la Virgen María nos ayude a acoger con alegría y gratitud el diseño divino de amor realizado en Jesucristo.
\end{body}

\newsection

\section{Temas}

%01 Ciclo | 02 Tiempo | 04 Semana

\cceth{Prólogo del Evangelio de Juan}

\cceref{CEC 151, 241, 291, 423, 445, 456-463, 504-505, 526, 1216, 2466, 2787}


\begin{ccebody}
	\ccesec{Creer en Jesucristo, el Hijo de Dios}

\n{151} Para el cristiano, creer en Dios es inseparablemente creer en Aquel que él ha enviado, \textquote{su Hijo amado}, en quien ha puesto toda su complacencia (\emph{Mc} 1,11). Dios nos ha dicho que le escuchemos (cf. \emph{Mc} 9,7). El Señor mismo dice a sus discípulos: \textquote{Creed en Dios, creed también en mí} (\emph{Jn} 14,1). Podemos creer en Jesucristo porque es Dios, el Verbo hecho carne: \textquote{A Dios nadie le ha visto jamás: el Hijo único, que está en el seno del Padre, él lo ha contado} (\emph{Jn} 1,18). Porque \textquote{ha visto al Padre} (\emph{Jn} 6,46), él es único en conocerlo y en poderlo revelar (cf. \emph{Mt} 11,27).
	
	\n{241} Por eso los Apóstoles confiesan a Jesús como \textquote{el Verbo que en el principio estaba junto a Dios y que era Dios} (\emph{Jn} 1,1), como \textquote{la imagen del Dios invisible} (\emph{Col} 1,15), como \textquote{el resplandor de su gloria y la impronta de su esencia} (\emph{Hb} 1,3).
	
	\n{291} \textquote{En el principio existía el Verbo [\ldots{}] y el Verbo era Dios [\ldots{}] Todo fue hecho por él y sin él nada ha sido hecho} (\emph{Jn} 1,1-3). El Nuevo Testamento revela que Dios creó todo por el Verbo Eterno, su Hijo amado. \textquote{En él fueron creadas todas las cosas, en los cielos y en la tierra [\ldots{}] todo fue creado por él y para él, él existe con anterioridad a todo y todo tiene en él su consistencia} (\emph{Col} 1, 16-17). La fe de la Iglesia afirma también la acción creadora del Espíritu Santo: él es el \textquote{dador de vida} (\emph{Símbolo Niceno-Constantinopolitano}), \textquote{el Espíritu Creador} (\emph{Liturgia de las Horas}, Himno \emph{Veni, Creator Spiritus}), la \textquote{Fuente de todo bien} (\emph{Liturgia bizantina}, Tropario de vísperas de Pentecostés).
		
	\n{423} Nosotros creemos y confesamos que Jesús de Nazaret, nacido judío de una hija de Israel, en Belén en el tiempo del rey Herodes el Grande y del emperador César Augusto I; de oficio carpintero, muerto crucificado en Jerusalén, bajo el procurador Poncio Pilato, durante el reinado del emperador Tiberio, es el Hijo eterno de Dios hecho hombre, que ha \textquote{salido de Dios} (\emph{Jn} 13, 3), \textquote{bajó del cielo} (\emph{Jn} 3, 13; 6, 33), \textquote{ha venido en carne} (\emph{1 Jn} 4, 2), porque \textquote{la Palabra se hizo carne, y puso su morada entre nosotros, y hemos visto su gloria, gloria que recibe del Padre como Hijo único, lleno de gracia y de verdad [\ldots{}] Pues de su plenitud hemos recibido todos, y gracia por gracia} (\emph{Jn} 1, 14. 16).
	
		\n{445} Después de su Resurrección, su filiación divina aparece en el poder de su humanidad glorificada: \textquote{Constituido Hijo de Dios con poder, según el Espíritu de santidad, por su Resurrección de entre los muertos} (\emph{Rm} 1, 4; cf. \emph{Hch} 13, 33). Los apóstoles podrán confesar \textquote{Hemos visto su gloria, gloria que recibe del Padre como Hijo único, lleno de gracia y de verdad } (\emph{Jn} 1, 14).
	
	\ccesec{Por qué el Verbo se hizo carne}
	
\n{456} Con el Credo Niceno-Constantinopolitano respondemos confesando: \textquote{\emph{Por nosotros los hombres y por nuestra salvación} bajó del cielo, y por obra del Espíritu Santo se encarnó de María la Virgen y se hizo hombre} (DS 150).

	
	%\n{457} El Verbo se encarnó \emph{para salvarnos reconciliándonos con Dios}: \textquote{Dios nos amó y nos envió a su Hijo como propiciación por nuestros pecados} (\emph{1 Jn} 4, 10). \textquote{El Padre envió a su Hijo para ser salvador del mundo} (\emph{1 Jn} 4, 14). \textquote{Él se manifestó para quitar los pecados} (\emph{1 Jn} 3, 5):
	
\begin{quote} 
	\textquote{Nuestra naturaleza enferma exigía ser sanada; desgarrada, ser restablecida; muerta, ser resucitada. Habíamos perdido la posesión del bien, era necesario que se nos devolviera. Encerrados en las tinieblas, hacía falta que nos llegara la luz; estando cautivos, esperábamos un salvador; prisioneros, un socorro; esclavos, un libertador. ¿No tenían importancia estos razonamientos? ¿No merecían conmover a Dios hasta el punto de hacerle bajar hasta nuestra naturaleza humana para visitarla, ya que la humanidad se encontraba en un estado tan miserable y tan desgraciado?} (San Gregorio de Nisa, \emph{Oratio catechetica}, 15: PG 45, 48B). 
\end{quote}

	\n{458} El Verbo se encarnó \emph{para que nosotros conociésemos así el amor de Dios}: \textquote{En esto se manifestó el amor que Dios nos tiene: en que Dios envió al mundo a su Hijo único para que vivamos por medio de él} (\emph{1 Jn} 4, 9). \textquote{Porque tanto amó Dios al mundo que dio a su Hijo único, para que todo el que crea en él no perezca, sino que tenga vida eterna} (\emph{Jn} 3, 16).
	
	\n{459} El Verbo se encarnó \emph{para ser nuestro modelo de santidad}: \textquote{Tomad sobre vosotros mi yugo, y aprended de mí \ldots{} } (\emph{Mt} 11, 29). \textquote{Yo soy el Camino, la Verdad y la Vida. Nadie va al Padre sino por mí} (\emph{Jn} 14, 6). Y el Padre, en el monte de la Transfiguración, ordena: \textquote{Escuchadle} (\emph{Mc} 9, 7; cf. \emph{Dt} 6, 4-5). Él es, en efecto, el modelo de las bienaventuranzas y la norma de la Ley nueva: \textquote{Amaos los unos a los otros como yo os he amado} (\emph{Jn} 15, 12). Este amor tiene como consecuencia la ofrenda efectiva de sí mismo (cf. \emph{Mc} 8, 34).
	
		\n{460} El Verbo se encarnó \emph{para hacernos \textquote{partícipes de la naturaleza divina}} (\emph{2 P} 1, 4): \textquote{Porque tal es la razón por la que el Verbo se hizo hombre, y el Hijo de Dios, Hijo del hombre: para que el hombre al entrar en comunión con el Verbo y al recibir así la filiación divina, se convirtiera en hijo de Dios} (San Ireneo de Lyon, \emph{Adversus haereses}, 3, 19, 1). \textquote{Porque el Hijo de Dios se hizo hombre para hacernos Dios} (San Atanasio de Alejandría, \emph{De Incarnatione}, 54, 3: PG 25, 192B). \emph{Unigenitus} [\ldots{}] \emph{Dei Filius, suae divinitatis volens nos esse participes, naturam nostram assumpsit, ut homines deos faceret factus homo} (\textquote{El Hijo Unigénito de Dios, queriendo hacernos partícipes de su divinidad, asumió nuestra naturaleza, para que, habiéndose hecho hombre, hiciera dioses a los hombres}) (Santo Tomás de Aquino, \emph{Oficio de la festividad del Corpus}, Of. de Maitines, primer Nocturno, Lectura I).
	
		\n{461} Volviendo a tomar la frase de san Juan (\textquote{El Verbo se encarnó}: \emph{Jn} 1, 14), la Iglesia llama \textquote{Encarnación} al hecho de que el Hijo de Dios haya asumido una naturaleza humana para llevar a cabo por ella nuestra salvación. En un himno citado por san Pablo, la Iglesia canta el misterio de la Encarnación:
	
	\begin{quote}
		\textquote{Tened entre vosotros los mismos sentimientos que tuvo Cristo: el cual, siendo de condición divina, no retuvo ávidamente el ser igual a Dios, sino que se despojó de sí mismo tomando condición de siervo, haciéndose semejante a los hombres y apareciendo en su porte como hombre; y se humilló a sí mismo, obedeciendo hasta la muerte y muerte de cruz} (\emph{Flp} 2, 5-8; cf. \emph{Liturgia de las Horas, Cántico de las Primeras Vísperas de Domingos}).
	\end{quote}
	
		\n{462} La carta a los Hebreos habla del mismo misterio:
	
	\begin{quote}
		\textquote{Por eso, al entrar en este mundo, {[}Cristo{]} dice: No quisiste sacrificio y oblación; pero me has formado un cuerpo. Holocaustos y sacrificios por el pecado no te agradaron. Entonces dije: ¡He aquí que vengo [\ldots{}] a hacer, oh Dios, tu voluntad!} (\emph{Hb} 10, 5-7; \emph{Sal} 40, 7-9 {[}LXX{]}).
	\end{quote}

	
	\n{463} La fe en la verdadera encarnación del Hijo de Dios es el signo distintivo de la fe cristiana: \textquote{Podréis conocer en esto el Espíritu de Dios: todo espíritu que confiesa a Jesucristo, venido en carne, es de Dios} (\emph{1 Jn} 4, 2). Esa es la alegre convicción de la Iglesia desde sus comienzos cuando canta \textquote{el gran misterio de la piedad}: \textquote{Él ha sido manifestado en la carne} (\emph{1 Tm} 3, 16).


		\n{504} Jesús fue concebido por obra del Espíritu Santo en el seno de la Virgen María porque El es el \emph{Nuevo Adán} (cf. 1 Co 15, 45) que inaugura la nueva creación: \textquote{El primer hombre, salido de la tierra, es terreno; el segundo viene del cielo} (1 Co 15, 47). La humanidad de Cristo, desde su concepción, está llena del Espíritu Santo porque Dios \textquote{le da el Espíritu sin medida} (Jn 3, 34). De \textquote{su plenitud}, cabeza de la humanidad redimida (cf. Col 1, 18), \textquote{hemos recibido todos gracia por gracia} (Jn 1, 16).

		\n{505} Jesús, el nuevo Adán, inaugura por su concepción virginal el \emph{nuevo nacimiento} de los hijos de adopción en el Espíritu Santo por la fe \textquote{¿Cómo será eso?} (Lc 1, 34;cf. Jn 3, 9). La participación en la vida divina no nace \textquote{de la sangre, ni de deseo de carne, ni de deseo de hombre, sino de Dios} (Jn 1, 13). La acogida de esta vida es virginal porque toda ella es dada al hombre por el Espíritu. El sentido esponsal de la vocación humana con relación a Dios (cf. 2 Co 11, 2) se lleva a cabo perfectamente en la maternidad virginal de María.

		\n{526} \textquote{Hacerse niño} con relación a Dios es la condición para entrar en el Reino (cf. \emph{Mt} 18, 3-4); para eso es necesario abajarse (cf. \emph{Mt} 23, 12), hacerse pequeño; más todavía: es necesario \textquote{nacer de lo alto} (\emph{Jn} 3,7), \textquote{nacer de Dios} (\emph{Jn} 1, 13) para \textquote{hacerse hijos de Dios} (\emph{Jn} 1, 12). El misterio de Navidad se realiza en nosotros cuando Cristo \textquote{toma forma} en nosotros (\emph{Ga}4, 19). Navidad es el misterio de este \textquote{admirable intercambio}:
	
	\begin{quote}
		\textquote{¡Oh admirable intercambio! El Creador del género humano, tomando cuerpo y alma, nace de la Virgen y, hecho hombre sin concurso de varón, nos da parte en su divinidad} (\emph{Solemnidad de la Santísima Virgen María, Madre de Dios,} Antífona de I y II Vísperas: \emph{Liturgia de las Horas}).
	\end{quote}


	\n{1216} \textquote{Este baño es llamado \emph{iluminación} porque quienes reciben esta enseñanza (catequética) su espíritu es iluminado} (San Justino, \emph{Apología} 1,61). Habiendo recibido en el Bautismo al Verbo, \textquote{la luz verdadera que ilumina a todo hombre} (\emph{Jn} 1,9), el bautizado, \textquote{tras haber sido iluminado} (\emph{Hb} 10,32), se convierte en \textquote{hijo de la luz} (\emph{1 Ts} 5,5), y en \textquote{luz} él mismo (\emph{Ef} 5,8):

\begin{quote}
	El Bautismo \textquote{es el más bello y magnífico de los dones de Dios [\ldots{}] lo llamamos don, gracia, unción, iluminación, vestidura de incorruptibilidad, baño de regeneración, sello y todo lo más precioso que hay. \emph{Don}, porque es conferido a los que no aportan nada; \emph{gracia}, porque es dado incluso a culpables; \emph{bautismo}, porque el pecado es sepultado en el agua; \emph{unción}, porque es sagrado y real (tales son los que son ungidos); \emph{iluminación}, porque es luz resplandeciente; \emph{vestidura}, porque cubre nuestra vergüenza; \emph{baño}, porque lava; \emph{sello}, porque nos guarda y es el signo de la soberanía de Dios} (San Gregorio Nacianceno, \emph{Oratio} 40,3-4).
\end{quote}

	\n{2466} En Jesucristo la verdad de Dios se manifestó en plenitud. \textquote{Lleno de gracia y de verdad} (\emph{Jn} 1, 14), él es la \textquote{luz del mundo} (\emph{Jn} 8, 12), \emph{la Verdad} (cf. \emph{Jn} 14, 6). El que cree en él, no permanece en las tinieblas (cf. \emph{Jn} 12, 46). El discípulo de Jesús, \textquote{permanece en su palabra}, para conocer \textquote{la verdad que hace libre} (cf. \emph{Jn} 8, 31-32) y que santifica (cf. \emph{Jn} 17, 17). Seguir a Jesús es vivir del \textquote{Espíritu de verdad} (\emph{Jn} 14, 17) que el Padre envía en su nombre (cf. \emph{Jn} 14, 26) y que conduce \textquote{a la verdad completa} (\emph{Jn} 16, 13). Jesús enseña a sus discípulos el amor incondicional de la verdad: \textquote{Sea vuestro lenguaje: \textquote{sí, sí}; \textquote{no, no}} (\emph{Mt} 5, 37).

	\n{2787} Cuando decimos Padre \textquote{nuestro}, reconocemos ante todo que todas sus promesas de amor anunciadas por los profetas se han cumplido en la \emph{nueva y eterna Alianza} en Cristo: hemos llegado a ser \textquote{su Pueblo} y Él es desde ahora en adelante \textquote{nuestro Dios}. Esta relación nueva es una pertenencia mutua dada gratuitamente: por amor y fidelidad (cf. \emph{Os} 2, 21-22; 6, 1-6) tenemos que responder a la gracia y a la verdad que nos han sido dadas en Jesucristo (cf. \emph{Jn} 1, 17).		
\end{ccebody}



\cceth{Cristo, Sabiduría de Dios}

\cceref{CEC 272, 295, 299, 474, 721, 1831}


\begin{ccebody}
	\ccesec{El misterio de la aparente impotencia de Dios}

\n{272} La fe en Dios Padre Todopoderoso puede ser puesta a prueba por la experiencia del mal y del sufrimiento. A veces Dios puede parecer ausente e incapaz de impedir el mal. Ahora bien, Dios Padre ha revelado su omnipotencia de la manera más misteriosa en el anonadamiento voluntario y en la Resurrección de su Hijo, por los cuales ha vencido el mal. Así, Cristo crucificado es \textquote{poder de Dios y sabiduría de Dios. Porque la necedad divina es más sabia que la sabiduría de los hombres, y la debilidad divina, más fuerte que la fuerza de los hombres} (\emph{1 Co} 2, 24-25). En la Resurrección y en la exaltación de Cristo es donde el Padre \textquote{desplegó el vigor de su fuerza} y manifestó \textquote{la soberana grandeza de su poder para con nosotros, los creyentes} (\emph{Ef} 1,19-22).
	
	\ccesec{Dios crea por sabiduría y por amor}

\n{295} Creemos que Dios creó el mundo según su sabiduría (cf. \emph{Sb} 9,9). Este no es producto de una necesidad cualquiera, de un destino ciego o del azar. Creemos que procede de la voluntad libre de Dios que ha querido hacer participar a las criaturas de su ser, de su sabiduría y de su bondad: \textquote{Porque tú has creado todas las cosas; por tu voluntad lo que no existía fue creado} (\emph{Ap} 4,11). \textquote{¡Cuán numerosas son tus obras, Señor! Todas las has hecho con sabiduría} (\emph{Sal} 104,24). \textquote{Bueno es el Señor para con todos, y sus ternuras sobre todas sus obras} (\emph{Sal} 145,9).
	
	\ccesec{Dios crea un mundo ordenado y bueno}

\n{299} Porque Dios crea con sabiduría, la creación está ordenada: \textquote{Tú todo lo dispusiste con medida, número y peso} (\emph{Sb} 11,20). Creada en y por el Verbo eterno, \textquote{imagen del Dios invisible} (\emph{Col} 1,15), la creación está destinada, dirigida al hombre, imagen de Dios (cf. \emph{Gn} 1,26), llamado a una relación personal con Dios. Nuestra inteligencia, participando en la luz del Entendimiento divino, puede entender lo que Dios nos dice por su creación (cf. \emph{Sal} 19,2-5), ciertamente no sin gran esfuerzo y en un espíritu de humildad y de respeto ante el Creador y su obra (cf. \emph{Jb} 42,3). Salida de la bondad divina, la creación participa en esa bondad (\textquote{Y vio Dios que era bueno [\ldots{}] muy bueno}: \emph{Gn} 1,4.10.12.18.21.31). Porque la creación es querida por Dios como un don dirigido al hombre, como una herencia que le es destinada y confiada. La Iglesia ha debido, en repetidas ocasiones, defender la bondad de la creación, comprendida la del mundo material (cf. San León Magno, c. \emph{Quam laudabiliter}, DS, 286; Concilio de Braga I: \emph{ibíd}., 455-463; Concilio de Letrán IV: \emph{ibíd.,} 800; Concilio de Florencia: \emph{ibíd.,}1333; Concilio Vaticano I: \emph{ibíd.,} 3002).

	
	\n{474} Debido a su unión con la Sabiduría divina en la persona del Verbo encarnado, el conocimiento humano de Cristo gozaba en plenitud de la ciencia de los designios eternos que había venido a revelar (cf. \emph{Mc} 8,31; 9,31; 10, 33-34; 14,18-20. 26-30). Lo que reconoce ignorar en este campo (cf. \emph{Mc} 13,32), declara en otro lugar no tener misión de revelarlo (cf. \emph{Hch} 1, 7).
	
	\ccesec{\textquote{Alégrate, llena de gracia}}
	
\n{721} María, la Santísima Madre de Dios, la siempre Virgen, es la obra maestra de la Misión del Hijo y del Espíritu Santo en la Plenitud de los tiempos. Por primera vez en el designio de Salvación y porque su Espíritu la ha preparado, el Padre encuentra la Morada en donde su Hijo y su Espíritu pueden habitar entre los hombres. Por ello, los más bellos textos sobre la Sabiduría, la Tradición de la Iglesia los ha entendido frecuentemente con relación a María (cf. \emph{Pr} 8, 1-9, 6; \emph{Si} 24): María es cantada y representada en la Liturgia como el \textquote{Trono de la Sabiduría}.
	
%En ella comienzan a manifestarse las \textquote{maravillas de Dios}, que el Espíritu va a realizar en Cristo y en la Iglesia: %Último párrafo iría con 722 ss
	
	\n{1831} Los siete \emph{dones} del Espíritu Santo son: sabiduría, inteligencia, consejo, fortaleza, ciencia, piedad y temor de Dios. Pertenecen en plenitud a Cristo, Hijo de David (cf. \emph{Is} 11, 1-2). Completan y llevan a su perfección las virtudes de quienes los reciben. Hacen a los fieles dóciles para obedecer con prontitud a las inspiraciones divinas.

\begin{quote}
	\textquote{Tu espíritu bueno me guíe por una tierra llana} (\emph{Sal} 143,10).
	
	\textquote{Todos los que son guiados por el Espíritu de Dios son hijos de Dios [\ldots{}] Y, si hijos, también herederos; herederos de Dios y coherederos de Cristo} (\emph{Rm} 8, 14.17).
\end{quote}

	
\end{ccebody}


\cceth{Dios nos dona la Sabiduría}

\cceref{CEC 158, 283, 1303, 1831, 2500}


\begin{ccebody}
	\n{158} \textquote{La fe \emph{trata de comprender}} (San Anselmo de Canterbury, \emph{Proslogion}, proemium: PL 153, 225A) es inherente a la fe que el creyente desee conocer mejor a aquel en quien ha puesto su fe, y comprender mejor lo que le ha sido revelado; un conocimiento más penetrante suscitará a su vez una fe mayor, cada vez más encendida de amor. La gracia de la fe abre \textquote{los ojos del corazón} (\emph{Ef} 1,18) para una inteligencia viva de los contenidos de la Revelación, es decir, del conjunto del designio de Dios y de los misterios de la fe, de su conexión entre sí y con Cristo, centro del Misterio revelado. Ahora bien, \textquote{para que la inteligencia de la Revelación sea más profunda, el mismo Espíritu Santo perfecciona constantemente la fe por medio de sus dones} (DV 5). Así, según el adagio de san Agustín (\emph{Sermo} 43,7,9: PL 38, 258), \textquote{creo para comprender y comprendo para creer mejor}.
	
	\n{283} La cuestión sobre los orígenes del mundo y del hombre es objeto de numerosas investigaciones científicas que han enriquecido magníficamente nuestros conocimientos sobre la edad y las dimensiones del cosmos, el devenir de las formas vivientes, la aparición del hombre. Estos descubrimientos nos invitan a admirar más la grandeza del Creador, a darle gracias por todas sus obras y por la inteligencia y la sabiduría que da a los sabios e investigadores. Con Salomón, éstos pueden decir: \textquote{Fue él quien me concedió el conocimiento verdadero de cuanto existe, quien me dio a conocer la estructura del mundo y las propiedades de los elementos [\ldots{}] porque la que todo lo hizo, la Sabiduría, me lo enseñó} (\emph{Sb} 7,17-21).
	
	\n{1303} Por este hecho, la Confirmación confiere crecimiento y profundidad a la gracia bautismal:

--- nos introduce más profundamente en la filiación divina que nos hace decir \textquote{\emph{Abbá}, Padre} (\emph{Rm} 8,15).;

--- nos une más firmemente a Cristo;

--- aumenta en nosotros los dones del Espíritu Santo;

--- hace más perfecto nuestro vínculo con la Iglesia (cf. LG 11);

--- nos concede una fuerza especial del Espíritu Santo para difundir y defender la fe mediante la palabra y las obras como verdaderos testigos de Cristo, para confesar valientemente el nombre de Cristo y para no sentir jamás vergüenza de la cruz (cf. DS 1319; LG 11,12):

\begin{quote}
	\textquote{Recuerda, pues, que has recibido el signo espiritual, el Espíritu de sabiduría e inteligencia, el Espíritu de consejo y de fortaleza, el Espíritu de conocimiento y de piedad, el Espíritu de temor santo, y guarda lo que has recibido. Dios Padre te ha marcado con su signo, Cristo Señor te ha confirmado y ha puesto en tu corazón la prenda del Espíritu} (San Ambrosio, \emph{De mysteriis} 7,42).
\end{quote}

	
	\n{1831} Los siete \emph{dones} del Espíritu Santo son: sabiduría, inteligencia, consejo, fortaleza, ciencia, piedad y temor de Dios. Pertenecen en plenitud a Cristo, Hijo de David (cf. \emph{Is} 11, 1-2). Completan y llevan a su perfección las virtudes de quienes los reciben. Hacen a los fieles dóciles para obedecer con prontitud a las inspiraciones divinas.

\begin{quote}
	\textquote{Tu espíritu bueno me guíe por una tierra llana} (\emph{Sal} 143,10).
	
	\textquote{Todos los que son guiados por el Espíritu de Dios son hijos de Dios [\ldots{}] Y, si hijos, también herederos; herederos de Dios y coherederos de Cristo} (\emph{Rm} 8, 14.17).
\end{quote}


	
	\ccesec{Verdad, belleza y arte sacro}

\n{2500} La práctica del bien va acompañada de un placer espiritual gratuito y de belleza moral. De igual modo, la verdad entraña el gozo y el esplendor de la belleza espiritual. La verdad es bella por sí misma. La verdad de la palabra, expresión racional del conocimiento de la realidad creada e increada, es necesaria al hombre dotado de inteligencia, pero la verdad puede también encontrar otras formas de expresión humana, complementarias, sobre todo cuando se trata de evocar lo que ella entraña de indecible, las profundidades del corazón humano, las elevaciones del alma, el Misterio de Dios. Antes de revelarse al hombre en palabras de verdad, Dios se revela a él, mediante el lenguaje universal de la Creación, obra de su Palabra, de su Sabiduría: el orden y la armonía del cosmos, que percibe tanto el niño como el hombre de ciencia, \textquote{pues por la grandeza y hermosura de las criaturas se llega, por analogía, a contemplar a su Autor} (\emph{Sb} 13, 5), \textquote{pues fue el Autor mismo de la belleza quien las creó} (\emph{Sb} 13, 3).

\textquote{La sabiduría es un hálito del poder de Dios, una emanación pura de la gloria del Omnipotente, por lo que nada manchado llega a alcanzarla. Es un reflejo de la luz eterna, un espejo sin mancha de la actividad de Dios, una imagen de su bondad} (\emph{Sb} 7, 25-26). \textquote{La sabiduría es, en efecto, más bella que el Sol, supera a todas las constelaciones; comparada con la luz, sale vencedora, porque a la luz sucede la noche, pero contra la sabiduría no prevalece la maldad} (\emph{Sb} 7, 29-30). \textquote{Yo me constituí en el amante de su belleza} (\emph{Sb} 8, 2).		
\end{ccebody}

\begin{patercite}
	He manchado mi cuerpo,
	
	he ensuciado mi espíritu,
	
	estoy todo lleno de llagas;
	
	pero tú, oh Cristo médico,
	
	cura mi espíritu y cuerpo con la penitencia,
	
	báñame, purifícame, lávame:
	
	déjame más puro que la nieve\ldots{} \\
	
	Crucificado por todos,
	
	has ofrecido tu cuerpo y tu sangre, oh Verbo:
	
	el cuerpo para re-plasmarme,
	
	la sangre para lavarme;
	
	y has entregado el espíritu
	
	para portarme, oh Cristo, a tu Engendrador. \\
	
	Has obrado la salvación
	
	en medio de la tierra.
	
	Por tu voluntad
	
	has sido clavado en el árbol de la Cruz
	
	y el Edén que había sido cerrado, se ha abierto\ldots{} \\
	
	Sea mi fuente bautismal
	
	la sangre de tu costado,
	
	y bebida el agua de remisión que ha brotado\ldots{}
	
	y sea ungido, bebiendo como crisma y bebida,
	
	tu vivificante palabra, oh Verbo. \\
	
	(Canon de San Andrés de Creta).
	%\end{paterprose}
\end{patercite}	