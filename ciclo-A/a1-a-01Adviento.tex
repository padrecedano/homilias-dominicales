\chapter{Domingo I de Adviento (A)}\label{domingo-i-de-adviento-a}

\section{Lecturas}\label{lecturas}

\rtitle{PRIMERA LECTURA}

\rbook{Del libro del profeta Isaías} \rred{2, 1-5}

\rtheme{Visión de Isaías, hijo de Amós, acerca de Judá y de Jerusalén}

\begin{readprose}
En los días futuros estará firme

   el monte de la casa del Señor,

  en la cumbre de las montañas,

  más elevado que las colinas.

Hacia él confluirán todas las naciones,

caminarán pueblos numerosos y dirán:

«Venid, subamos al monte del Señor,

a la casa del Dios de Jacob.

Él nos instruirá en sus caminos

y marcharemos por sus sendas;

porque de Sión saldrá la ley,

la palabra del Señor de Jerusalén».

Juzgará entre las naciones,

será árbitro de pueblos numerosos.

De las espadas forjarán arados,

de las lanzas, podaderas.

No alzará la espada pueblo contra pueblo,

no se adiestrarán para la guerra.

Casa de Jacob, venid;

caminemos a la luz del Señor.
\end{readprose}

\rtitle{SALMO RESPONSORIAL}

\rbook{Salmo} \rred{121, 1-2. 4-9}

\rtheme{Qué alegría cuando me dijeron: \textquote{Vamos a la casa del Señor}}

\begin{psbody}
¡Qué alegría cuando me dijeron:
\textquote{Vamos a la casa del Señor}!
Ya están pisando nuestros pies
tus umbrales, Jerusalén.

Allá suben las tribus,
las tribus del Señor,
según la costumbre de Israel,
a celebrar el nombre del Señor;
en ella están los tribunales de justicia,
en el palacio de David.

Desead la paz a Jerusalén:
\textquote{Vivan seguros los que te aman,
haya paz dentro de tus muros,
seguridad en tus palacios}.

Por mis hermanos y compañeros,
voy a decir: \textquote{La paz contigo}.
Por la casa del Señor,
nuestro Dios, te deseo todo bien.
\end{psbody}

SEGUNDA LECTURA

De la carta del apóstol san Pablo a los Romanos 13, 11-14

Nuestra salvación está cerca

Hermanos: Comportaos reconociendo el momento en que vivís, pues ya es hora de despertaros del sueño, porque ahora la salvación está más cerca de nosotros que cuando abrazamos la fe. La noche está avanzada, el día está cerca: dejemos, pues, las obras de las tinieblas y pongámonos las armas de la luz.

Andemos como en pleno día, con dignidad. Nada de comilonas y borracheras, nada de lujuria y desenfreno, nada de riñas y envidias. Revestíos más bien del Señor Jesucristo.

EVANGELIO

Del Santo Evangelio según san Mateo 24, 37-44

Vigilemos para estar preparados

En aquel tiempo, dijo Jesús a sus discípulos:

«Cuando venga el Hijo del hombre, pasará como en tiempo de Noé.

En los días antes del diluvio, la gente comía y bebía, se casaban los hombres y las mujeres tomaban esposo, hasta el día en que Noé entró en el arca; y cuando menos lo esperaban llegó el diluvio y se los llevó a todos; lo mismo sucederá cuando venga el Hijo del hombre: dos hombres estarán en el campo, a uno se lo llevarán y a otro lo dejarán; dos mujeres estarán moliendo, a una se la llevarán y a otra la dejarán.

Por tanto, estad en vela, porque no sabéis qué día vendrá vuestro Señor. Comprended que si supiera el dueño de casa a qué hora de la noche viene el ladrón, estaría en vela y no dejaría que abrieran un boquete en su casa. Por eso, estad también vosotros preparados, porque a la hora que menos penséis viene el Hijo del hombre».



\subsubsection{Comentario Patrístico}\label{comentario-patruxedstico}

\paragraph{Pascasio Radberto}\label{pascasio-radberto}

Velad, para estar preparados

Exposición sobre el evangelio de san Mateo, Lib. 11, cap. 24:

PL 120, 799-800.

\emph{Velad, porque no sabéis el día ni la hora}. Siendo una recomendación que a todos afecta, la expresa como si solamente se refiriera a los hombres de aquel entonces. Es lo que ocurre con muchos otros pasajes que leemos en las Escrituras. Y de tal modo atañe a todos lo así expresado, que a cada uno le llega el último día y para cada cual es el fin del mundo el momento mismo de su muerte. Por eso es necesario que cada uno parta de este mundo tal cual ha de ser juzgado aquel día. En consecuencia, todo hombre debe cuidar de no dejarse seducir ni abandonar la vigilancia, no sea que el día de la venida del Señor lo encuentre desprevenido.

Y aquel día encontrará desprevenido a quien hallare desprevenido el último día de su vida. Pienso que los apóstoles estaban convencidos de que el Señor no iba a presentarse en sus días para el juicio final; y sin embargo, ¿quién dudará de que ellos cuidaron de no dejarse seducir, de que no abandonaron la vigilancia y de que observaron todo lo que a todos fue recomendado, para que el Señor los hallara preparados? Por esta razón, debemos tener siempre presente una doble venida de Cristo: una, cuando aparezca de nuevo y hayamos de dar cuenta de todos nuestros actos; otra diaria, cuando a todas horas visita nuestras conciencias y viene a nosotros, para que cuando viniere, nos encuentre preparados.

¿De qué me sirve, en efecto, conocer el día del juicio si soy consciente de mis muchos pecados?, ¿conocer si viene o cuándo viene el Señor, si antes no viniere a mi alma y retornare a mi espíritu?, ¿si antes no vive Cristo en mí y me habla? Sólo entonces será su venida un bien para mí, si primero Cristo vive en mí y yo vivo en Cristo. Y sólo entonces vendrá a mí, como en una segunda venida, cuando, muerto para el mundo, pueda en cierto modo hacer mía aquella expresión: \emph{El mundo está crucificado 	para mí, y yo para el mundo.}

Considera asimismo estas palabras de Cristo: \emph{Porque muchos vendrán 	usando mi nombre.} Sólo el anticristo y sus secuaces se arrogan falsamente el nombre de Cristo, pero sin las obras de Cristo, sin sus palabras de verdad, sin su sabiduría. En ninguna parte de la Escritura hallarás que el Señor haya usado esta expresión y haya dicho: \emph{Yo 	soy el Cristo.} Le bastaba mostrar con su doctrina y sus milagros lo que era realmente, pues las obras del Padre que realizaba, la doctrina que enseñaba y su poder gritaban: \emph{Yo} soy \emph{el Cristo} con más eficacia que si mil voces lo pregonaran. Cristo, que yo sepa, jamás se atribuyó verbalmente este título: lo hizo realizando las obras del Padre y enseñando la ley del amor. En cambio, los falsos cristos, careciendo de esta ley del amor, proclamaban de palabra ser lo que no eran.

\textbf{Rorate Caeli}



Rorate Caeli desúper\\ et nubes plúant justum

Ne irascáris Dómine,\\ ne ultra memíneris iniquitátis\\ Ecce cívitas Sancti\\ facta est desérta\\ Sion desérta facta est,\\ Jerúsalem desoláta est.\\ Domus sanctificatiónis tuae et gloriae tuae\\ Ubi laudavérunt Te patres nostri.

Peccávimus et facti sumus\\ tamquam immúndus nos,\\ Et cecídimus quasi fólium univérsi\\ Et iniquitátes nostrae quasi ventus\\ abstulérunt nos\\ Abscondísti fáciem tuam a nobis\\ Et allisísti nos\\ in mánu iniquitátis nostrae.

Víde, Dómine, afflictiónem pópuli tui\\ Et mitte quem missúrus es\\ Emítte Agnum dominatórem terrae\\ De pétra desérti\\ ad montem fíliae Sion\\ Ut áuferat ipse jugum\\ captivitátis nostrae.

Consolámini, consolámini, pópule meus\\ Cito véniet salus tua\\ Quare moeróre consúmeris,\\ quia innovávit te dolor?\\ Salvábo te, noli timére\\ Ego énim sum Dóminus Deus túus\\ Sánctus Israël, Redémptor túus.\strut



Derramad, oh cielos, vuestro rocío de lo alto,\\ y las nubes lluevan al Justo.

No te enfades, Señor,\\ ni te acuerdes de la iniquidad.\\ He aquí que la ciudad del Santuario\\ quedó desierta:\\ Sión quedó desierta;\\ Jerusalén está desolada.\\ La casa de tu santidad y de tu gloria,\\ donde nuestros padres te alabaron.

Pecamos y nos volvimos\\ como los inmundos,\\ Y caímos, todos, como hojas.\\ Y nuestra iniquidades, como un viento,\\ nos dispersaron.\\ Ocultaste de nosotros tu rostro\\ Y nos castigaste\\ por mano de nuestras iniquidades.

¡Mira, Señor, la aflicción de tu pueblo!,\\ Y envíale a Aquel que vas a enviar!\\ Envíale al Cordero dominador de la tierra\\ Del desierto de piedra\\ al monte de la hija de Sión\\ Para que Él retire el yugo\\ de nuestro cautiverio.

Consuélate, consuélate, pueblo mío,\\ ¡En breve ha de llegar tu salvación!\\ ¿Por qué te consumes en la tristeza,\\ por qué tu dolor?\\ ¡Yo te salvaré, no tengas miedo!\\ Porque Yo soy el Señor, tu Dios,\\ El Santo de Israel, tu Redentor.\strut



\subsubsection{Homilías}\label{homiluxedas}

\paragraph{San Juan Pablo II, papa}\label{san-juan-pablo-ii-papa}

\subparagraph{Homilía (1980): Nueva llamada a vestirse de 	Cristo}\label{homiluxeda-nueva-llamada-a-vestirse-de-cristo}

Visita Pastoral a la Parroquia de San Leonardo de Porto Mauricio.


Domingo 30 de noviembre de 1980.

1. Al escuchar las palabras del Evangelio de hoy según Mateo, ante nuestros ojos vienen espontáneamente a la memoria los acontecimientos que durante la semana pasada han sacudido a toda Italia\footnote{Se 	refiere a un gran terremoto que afectó las regiones italianas de 	Campania y de Basilicata, desde Potenza a Avellino, hasta el litoral, 	a los Puertos de Nápoles y Salerno.}\ldots{} Mientras nosotros todos, con espíritu de solidaridad humana, queremos ayudar a nuestros hermanos y compatriotas, arrollados por la desgracia, al mismo tiempo, estos acontecimientos traen ante nuestros ojos, con una particular fuerza comparativa, el cuadro terrible que cada año trazan los \textbf{Evangelios} de este primer domingo de Adviento: anuncios de destrucción y de muerte, en la espera escatológica de la \textquote{venida del Hijo del Hombre} (Mt 24, 39).

2. La historia de los hombres y de las naciones, la historia de toda la humanidad suministra pruebas suficientes para afirmar que en todos los tiempos se han multiplicado desgracias y catástrofes, calamidades naturales, como terremotos, o las causadas por el hombre, como guerras, revoluciones, estragos, homicidios y genocidios. Además, cada uno de nosotros sabe que nuestra existencia terrena lleva a la muerte, llegando así un día a su término. El mundo visible, con todos los bienes y las riquezas que oculta en sí mismo, al fin no es capaz de darnos más que la muerte: el término de la vida.

Esta verdad, aunque nos la recuerda también la liturgia de hoy, primer domingo de Adviento, sin embargo, no es la verdad específica anunciada en este día festivo, y en todo el período de Adviento. No es la palabra principal del \textbf{Evangelio}.

¿Cuál es, pues, la palabra principal? La hemos leído hace poco: la venida del Hijo del Hombre. La palabra principal del Evangelio no es \textquote{la separación}, \textquote{la ausencia}, sino \textquote{la venida} y \textquote{la presencia}. Ni siquiera es la \textquote{muerte}, sino la \textquote{vida}. El Evangelio es la Buena Noticia, porque pronuncia la verdad sobre la vida en el contexto de la muerte.



La venida del Hijo del Hombre es el comienzo de esta Vida. Y de este comienzo nos habla precisamente el Adviento, que responde a la pregunta: ¿cómo debe vivir el hombre en el mundo con la perspectiva de la muerte? El hombre al que, en un abrir y cerrar de ojos, le puede ser quitada la vida, ¿cómo debe vivir en este mundo, para encontrarse con el Hijo del Hombre, cuya venida es el comienzo de la nueva vida, de la vida más potente que la muerte?

4. Nos encontramos, pues, todos en el primer domingo de Adviento. ¿Cuál es esta verdad que nos penetra y vivifica hoy? ¿Qué mensaje nos anuncia la Santa Iglesia, nuestra Madre? Como ya he dicho, no se trata de un mensaje de miedo y de muerte, sino del mensaje de la esperanza y de la llamada.

Tomemos como ejemplo la \textbf{segunda lectura}; he aquí lo que el Apóstol Pablo dice a los romanos de entonces, pero que debemos tomar en serio los romanos de hoy: \textquote{Daos cuenta del momento en que vivís; ya es hora de espabilarse, porque ahora nuestra salvación está más cerca que cuando empezamos a creer. La noche está avanzada, el día se echa encima} (Rm 13, 11-12).

En realidad, al contrario de como podemos ser inducidos a pensar, la salvación está más cercana y no más lejana. Efectivamente, al vivir en una época de secularización, somos testigos de comportamientos de indiferencia religiosa y también de programas e ideologías ateas o incluso antiteístas. Se llegaría a pensar que los indicios humanos desmienten el mensaje de la liturgia de hoy. Ella, en cambio ---aun haciendo referencia también a estos \textquote{indicios humanos}--- proclama, sin embargo, la verdad divina y anuncia el designio divino que no decae jamás, que no cambia aun cuando puedan cambiar los hombres, los programas, los proyectos humanos. Ese designio divino es el designio de la salvación del hombre en Cristo, que, una vez emprendido, perdura, y consiguientemente mira a su cumplimiento.

Pero el hombre puede ser ciego y sordo a todo esto. Puede meterse cada vez más profundamente en la noche, aunque se acerque el día. Puede multiplicar las obras de las tinieblas aunque Cristo le ofrezca \textquote{las armas de la luz}.

Por lo tanto, la invitación apremiante de la liturgia de hoy es la del \textbf{Apóstol}: \textquote{Vestíos del Señor Jesucristo} (Rm 13, 14). Esta expresión es, en cierto sentido, la definición del cristiano. Ser cristiano quiere decir \textquote{vestirse de Cristo}. El Adviento es la nueva llamada a vestirse de Jesucristo.

Dice además el Apóstol: \textquote{Conduzcámonos como en pleno día, con dignidad. Nada de comilonas ni borracheras, nada de lujuria ni desenfreno, nada de riñas ni pendencias..., y que el cuidado de vuestro cuerpo no fomente los malos deseos} (Rm 13, 13-14).

5. ¿Qué significa, además, el Adviento? El Adviento es el descubrimiento de una gran aspiración de los hombres y de los pueblos hacia la casa del Señor. No hacia la muerte y la destrucción, sino hacia el encuentro con El.

Y por esto en la liturgia de hoy oímos esta invitación: \textquote{Qué alegría cuando me dijeron: vamos a la casa del Señor}.

Y el mismo \textbf{Salmo responsorial} nos traza, por decirlo así, la imagen de esa casa, de esa ciudad, de ese encuentro: \textquote{Ya están pisando nuestros pies tus umbrales, Jerusalén. Allá suben las tribus, las tribus del Señor. Según la costumbre de Israel, a celebrar el nombre del Señor. En ella están los tribunales de justicia en el palacio de David. Por mis hermanos y compañeros voy a decir: 'La paz contigo'. Por la casa del Señor nuestro Dios, te deseo todo bien} (Sal 121 {[}122{]}).

Sí. El Señor es el Dios de la paz, es el Dios de la Alianza con el hombre. Cuando en la noche de Belén los pobres pastores se pondrán en camino hacia el establo donde se realizará la primera venida del Hijo del Hombre, los conducirá el canto de los ángeles: \textquote{Gloria a Dios en las alturas y paz en la tierra a los hombres de buena voluntad} (Lc 2, 14).

6. Esta visión de la paz divina pertenece a toda la espera mesiánica en la Antigua Alianza. Oímos hoy las palabras de \textbf{Isaías}: \textquote{Será el árbitro de las naciones, el juez de pueblos numerosos. De las espadas forjarán arados; de las lanzas, podaderas. No alzará la espada pueblo contra pueblo, no se adiestrarán para la guerra. Casa de Jacob, ven; caminemos a la luz del Señor} (Is 2, 4-5).

El Adviento trae consigo la invitación a la paz de Dios para todos los hombres. Es necesario que nosotros construyamos esta paz y la reconstruyamos continuamente en nosotros mismos y con los otros: en las familias, en las relaciones con los cercanos, en los ambientes de trabajo, en la vida de toda la sociedad.

Trabajad con espíritu de solidaridad fraterna a fin de que vuestra parroquia crezca cada vez más como comunidad de fieles, de familias, de grupos ---me refiero particularmente a todos vuestros grupos organizados--- en comunión de verdad y de amor. La comunidad parroquial, en efecto, se edifica sobre la Palabra de Dios, transmitida y garantizada por los Pastores, se alimenta por la gracia de los sacramentos, se sostiene por la oración, se une por el vínculo de la caridad fraterna. Que cada uno de sus miembros se sienta vivo, activo, partícipe, corresponsable, implicado en tareas efectivas de evangelización cristiana y de promoción humana. De este modo, vuestra parroquia se convierte en signo e instrumento de la presencia de Cristo en el barrio, irradiación de su amor y de su paz.

Para servir a esta paz de múltiples dimensiones, es necesario escuchar también estas palabras del \textbf{Profeta}: \textquote{Venid, subamos al monte del Señor, a la casa del Dios de Jacob. El nos instruirá en sus caminos y marcharemos por sus sendas, porque de Sión saldrá la ley. de Jerusalén la palabra del Señor} (Is 2, 3).

También para vuestra comunidad eclesial el Adviento es el tiempo en el que se deben aprender de nuevo la ley del Señor y sus palabras. Es el tiempo de una catequesis intensificada. La ley y la palabra del Señor deben penetrar de nuevo en el corazón, deben encontrar de nuevo su confirmación en la vida social. Sirven al bien del hombre. ¡Construyen la paz!..

7. Queridos hermanos e hijos: Nos encontramos, pues, de nuevo al comienzo del camino. Ha comenzado de nuevo el Adviento: el tiempo de la gracia, el tiempo de la espera, el tiempo de la venida del Señor, que perdura siempre. Y la vida del hombre se desarrolla en el amor del Señor, a pesar de todas las dolorosas experiencias de la destrucción y de la muerte, hacia la realización final en Dios.

¡El Hijo del Hombre vendrá! Escuchemos estas palabras con la esperanza, no con el miedo, aunque estén llenas de una profunda seriedad.

Velad... y estad preparados, porque no sabéis en qué día vendrá el Hijo del Hombre. ¡Ven, Señor Jesús! ¡Marana tha!


\paragraph{Benedicto XVI, papa}\label{benedicto-xvi-papa}

\subparagraph{Homilía (2007): La gran 	esperanza}\label{homiluxeda-la-gran-esperanza}

Visita Pastoral al Hospital Romano San Juan Bautista de la Soberana Orden de Malta.

Domingo 2 de diciembre del 2007.

1. \textquote{Vamos alegres al encuentro del Señor}. Estas palabras, que hemos repetido en el estribillo del \textbf{salmo responsorial}, interpretan bien los sentimientos que alberga nuestro corazón hoy, primer domingo de Adviento. La razón por la cual podemos caminar con alegría, como nos ha exhortado el apóstol \textbf{san Pablo}, es que ya está cerca nuestra salvación. El Señor viene. Con esta certeza emprendemos el itinerario del Adviento, preparándonos para celebrar con fe el acontecimiento extraordinario del Nacimiento del Señor.

Durante las próximas semanas, día tras día, la liturgia propondrá a nuestra reflexión textos del Antiguo Testamento, que recuerdan el vivo y constante deseo que animó en el pueblo judío la espera de la venida del Mesías. También nosotros, vigilantes en la oración, tratemos de preparar nuestro corazón para acoger al Salvador, que vendrá a mostrarnos su misericordia y a darnos su salvación.

2. Precisamente porque es tiempo de espera, el Adviento es tiempo de esperanza, y a la esperanza cristiana he querido dedicar mi segunda encíclica, presentada oficialmente anteayer: comienza con las palabras que san Pablo dirigió a los cristianos de Roma: \emph{\textquote{Spe salvi facti 	sumus}}, \textquote{En esperanza fuimos salvados} (\emph{Rm} 8, 24). En la encíclica escribí, entre otras cosas, que \textquote{nosotros necesitamos tener esperanzas ---más grandes o más pequeñas---, que día a día nos mantengan en camino. Pero sin la gran esperanza, que ha de superar todo lo demás, aquellas no bastan. Esta gran esperanza sólo puede ser Dios, que abraza el universo y que nos puede proponer y dar lo que nosotros por sí solos no podemos alcanzar} (n. 31). Que la certeza de que sólo Dios puede ser nuestra firme esperanza nos anime a todos los que esta mañana nos hemos reunido en esta casa, en la que se lucha contra la enfermedad, sostenidos por la solidaridad.

{[}\ldots{}{]}

4. Queridos hermanos y hermanas, \textquote{que el Dios de la esperanza, que nos colma de todo gozo y paz en la fe por la fuerza del Espíritu Santo, esté con todos vosotros}. Con este deseo, que el sacerdote dirige a la asamblea al inicio de la santa misa, os saludo cordialmente...

El saludo más afectuoso es para vosotros, queridos enfermos, y para vuestros familiares, que con vosotros comparten angustias y esperanzas. El Papa está espiritualmente cerca de vosotros y os asegura su oración diaria; os invita a encontrar en Jesús apoyo y consuelo, y a no perder jamás la confianza. La liturgia de Adviento nos repetirá durante las próximas semanas que no nos cansemos de invocarlo; nos exhortará a salir a su encuentro, sabiendo que él mismo viene continuamente a visitarnos. En la prueba y en la enfermedad Dios nos visita misteriosamente y, si nos abandonamos a su voluntad, podemos experimentar la fuerza de su amor.

Los hospitales y las clínicas, precisamente porque en ellos se encuentran personas probadas por el dolor, pueden transformarse en lugares privilegiados para testimoniar el amor cristiano que alimenta la esperanza y suscita propósitos de solidaridad fraterna. En la \textbf{oración colecta} hemos rezado así: \textquote{Dios todopoderoso, aviva en tus fieles, al comenzar el Adviento, el deseo de salir al encuentro de Cristo, acompañados por las buenas obras}. Sí. Abramos el corazón a todas las personas, especialmente a las que atraviesan dificultades, para que, haciendo el bien a cuantos se encuentran en necesidad, nos dispongamos a acoger a Jesús que en ellos viene a visitarnos.

6. {[}...{]} En cada enfermo, cualquiera que sea, reconoced y servid a Cristo mismo; haced que en vuestros gestos y en vuestras palabras perciba los signos de su amor misericordioso.

Para cumplir bien esta \textquote{misión}, como nos recuerda san Pablo en la \textbf{segunda lectura}, tratad de \textquote{pertrecharos con las armas de la luz} (\emph{Rm} 13, 12), que son la palabra de Dios, los dones del Espíritu, la gracia de los sacramentos, y las virtudes teologales y cardinales; luchad contra el mal y abandonad el pecado, que entenebrece nuestra existencia. Al inicio de un nuevo año litúrgico, renovemos nuestros buenos propósitos de vida evangélica. \textquote{Ya es hora de espabilarse} (\emph{Rm} 13, 11), exhorta el Apóstol; es decir, es hora de convertirse, de despertar del letargo del pecado para disponerse con confianza a acoger al \textquote{Señor que viene}. Por eso, el Adviento es tiempo de oración y de espera vigilante.

7. A la \textquote{vigilancia}, que por lo demás es la palabra clave de todo este período litúrgico, nos exhorta la \textbf{página evangélica} que acabamos de proclamar: \textquote{Estad en vela, porque no sabéis qué día vendrá vuestro Señor} (\emph{Mt} 24, 42). Jesús, que en la Navidad vino a nosotros y volverá glorioso al final de los tiempos, no se cansa de visitarnos continuamente en los acontecimientos de cada día. Nos pide estar atentos para percibir su presencia, su adviento, y nos advierte que lo esperemos vigilando, puesto que su venida no se puede programar o pronosticar, sino que será repentina e imprevisible. Sólo quien está despierto no será tomado de sorpresa. Que no os suceda ---advierte--- lo que pasó en tiempo de Noé, cuando los hombres comían y bebían despreocupadamente, y el diluvio los encontró desprevenidos (cf. \emph{Mt} 24, 37-38). Lo que quiere darnos a entender el Señor con esta recomendación es que no debemos dejarnos absorber por las realidades y preocupaciones materiales hasta el punto de quedar atrapados en ellas. Debemos vivir ante los ojos del Señor con la convicción de que cada día puede hacerse presente. Si vivimos así, el mundo será mejor.

8. \textquote{Estad, pues, en vela...}. Escuchemos la invitación de Jesús en el \textbf{Evangelio} y preparémonos para revivir con fe el misterio del nacimiento del Redentor, que ha llenado de alegría el universo; preparémonos para acoger al Señor que viene continuamente a nuestro encuentro en los acontecimientos de la vida, en la alegría y en el dolor, en la salud y en la enfermedad; preparémonos para encontrarlo en su venida última y definitiva.

Su paso es siempre fuente de paz y, si el sufrimiento, herencia de la naturaleza humana, a veces resulta casi insoportable, con la venida del Salvador \textquote{el sufrimiento ---sin dejar de ser sufrimiento--- se convierte a pesar de todo en canto de alabanza} (\emph{Spe salvi}, 37). Confortados por estas palabras, prosigamos la celebración eucarística, invocando sobre los enfermos, sobre sus familiares y sobre cuantos trabajan en este hospital y en toda la Orden de los Caballeros de Malta, la protección materna de María, Virgen de la espera y de la esperanza, así como de la alegría, ya presente en este mundo, porque cuando sentimos la cercanía de Cristo vivo tenemos ya el remedio para el sufrimiento, tenemos ya su alegría. Amén.

\protect\hypertarget{_Toc448662715}{}{\protect\hypertarget{_Toc448690234}{}{\protect\hypertarget{_Toc448708257}{}{\protect\hypertarget{_Toc448709343}{}{\protect\hypertarget{_Toc449554345}{}{}}}}}

\paragraph{Francisco, papa}\label{francisco-papa}

\subparagraph{Ángelus: En camino}\label{uxe1ngelus-en-camino}

Plaza de San Pedro. Domingo 1 de diciembre del 2013.

Comenzamos hoy, primer domingo de Adviento, un nuevo año litúrgico, es decir \emph{un nuevo camino del Pueblo de Dios} con Jesucristo, nuestro Pastor, que nos guía en la historia hacia la realización del Reino de Dios. Por ello este día tiene un atractivo especial, nos hace experimentar un sentimiento profundo del sentido de la historia. Redescubrimos la belleza de estar todos en camino: la Iglesia, con su vocación y misión, y toda la humanidad, los pueblos, las civilizaciones, las culturas, todos en camino a través de los senderos del tiempo.

¿En camino hacia dónde? ¿Hay una meta común? ¿Y cuál es esta meta? El Señor nos responde a través del profeta \textbf{Isaías}, y dice así: \textquote{En los días futuros estará firme el monte de la casa del Señor, en la cumbre de las montañas, más elevado que las colinas. Hacia él confluirán todas las naciones, caminarán pueblos numerosos y dirán: ``Venid, subamos al monte del Señor, a la casa del Dios de Jacob. Él nos instruirá en sus caminos y marcharemos por sus sendas''} (2, 2-3). Esto es lo que dice Isaías acerca de la meta hacia la que nos dirigimos. Es \emph{una peregrinación universal hacia una meta común}, que en el Antiguo Testamento es Jerusalén, donde surge el templo del Señor, porque desde allí, de Jerusalén, ha venido la revelación del rostro de Dios y de su ley. La revelación ha encontrado su realización en \emph{Jesucristo}, y Él mismo, el Verbo hecho carne, se ha convertido en el \textquote{templo del Señor}: es Él la guía y al mismo tiempo la meta de nuestra peregrinación, de la peregrinación de todo el Pueblo de Dios; y bajo su luz también los demás pueblos pueden caminar hacia el Reino de la justicia, hacia el Reino de la paz. Dice de nuevo el profeta: \textquote{De las espadas forjarán arados, de las lanzas, podaderas. No alzará la espada pueblo contra pueblo, no se adiestrarán para la guerra} (2, 4).

Me permito repetir esto que dice el profeta, escuchad bien: \textquote{De las espadas forjarán arados, de las lanzas, podaderas. No alzará la espada pueblo contra pueblo, no se adiestrarán para la guerra}. ¿Pero cuándo sucederá esto? Qué hermoso día será ese en el que las armas sean desmontadas, para transformarse en instrumentos de trabajo. ¡Qué hermoso día será ése! ¡Y esto es posible! Apostemos por la esperanza, la esperanza de la paz. Y será posible.

Este camino no se acaba nunca. Así como en la vida de cada uno de nosotros siempre hay necesidad de comenzar de nuevo, de volver a levantarse, de volver a encontrar el sentido de la meta de la propia existencia, de la misma manera para la gran familia humana es necesario renovar siempre el horizonte común hacia el cual estamos encaminados. \emph{¡El horizonte de la esperanza!} Es ese el horizonte para hacer un buen camino. El tiempo de Adviento, que hoy de nuevo comenzamos, nos devuelve el horizonte de la esperanza, una esperanza que no decepciona porque está fundada en la Palabra de Dios. Una esperanza que no decepciona, sencillamente porque el Señor no decepciona jamás. ¡Él es fiel!, ¡Él no decepciona! ¡Pensemos y sintamos esta belleza!

El modelo de esta actitud espiritual, de este modo de ser y de caminar en la vida, es la Virgen María. Una sencilla muchacha de pueblo, que lleva en el corazón toda la esperanza de Dios. En su seno, la esperanza de Dios se hizo carne, se hizo hombre, se hizo historia: Jesucristo. Su \emph{Magníficat} es el cántico del Pueblo de Dios en camino, y de todos los hombres y mujeres que esperan en Dios, en el poder de su misericordia. Dejémonos guiar por Ella, que es madre, es mamá, y sabe cómo guiarnos. Dejémonos guiar por Ella en este tiempo de espera y de vigilancia activa.

\subparagraph{Ángelus: El Señor nos 	visita}\label{uxe1ngelus-el-seuxf1or-nos-visita}

Plaza de San Pedro. Domingo 27 de noviembre del 2016.

Hoy la Iglesia inicia un nuevo año litúrgico, es decir, un nuevo camino de fe del pueblo de Dios. Y como siempre iniciamos con el Adviento. La página del \textbf{Evangelio} (cf. Mt 24, 37-44) nos presenta uno de los temas más sugestivos del tiempo de Adviento: la visita del Señor a la humanidad. La primera visita ---lo sabemos todos--- se produjo con la Encarnación, el nacimiento de Jesús en la gruta de Belén; la segunda sucede en el presente: el Señor nos visita continuamente cada día, camina a nuestro lado y es una presencia de consolación; y para concluir estará la tercera y última visita, que profesamos cada vez que recitamos el Credo: \textquote{De nuevo vendrá en la gloria para juzgar a vivos y a muertos}. El Señor hoy nos habla de esta última visita suya, la que sucederá al final de los tiempos y nos dice dónde llegará nuestro camino.

La palabra de Dios hace resaltar el contraste entre el desarrollarse normal de las cosas, la rutina cotidiana y la venida repentina del Señor. \textbf{Dice Jesús}: \textquote{Como en los días que precedieron al diluvio, comían, bebían, tomaban mujer o marido, hasta el día en el que entró Noé en el arca, y no se dieron cuenta hasta que vino el diluvio y los arrasó a todos} (vv. 38-39): así dice Jesús. Siempre nos impresiona pensar en las horas que preceden a una gran calamidad: todos están tranquilos, hacen las cosas de siempre sin darse cuenta que su vida está a punto de ser alterada. El Evangelio, ciertamente no quiere darnos miedo, sino abrir nuestro horizonte a la dimensión ulterior, más grande, que por una parte relativiza las cosas de cada día pero al mismo tiempo las hace preciosas, decisivas. La relación con el Dios que viene a visitarnos da a cada gesto, a cada cosa una luz diversa, una profundidad, un valor simbólico.

Desde esta perspectiva llega también una invitación a la sobriedad, a no ser dominados por las cosas de este mundo, por las realidades materiales, sino más bien a gobernarlas. Si por el contrario nos dejamos condicionar y dominar por ellas, no podemos percibir que hay algo mucho más importante: nuestro encuentro final con el Señor, y esto es importante. Ese, ese encuentro. Y las cosas de cada día deben tener ese horizonte, deben ser dirigidas a ese horizonte. Este encuentro con el Señor que viene por nosotros. En aquel momento, como dice el \textbf{Evangelio}, \textquote{estarán dos en el campo: uno es tomado, el otro dejado} (v. 40). Es una invitación a la vigilancia, porque no sabiendo cuando Él vendrá, es necesario estar preparados siempre para partir.

En este tiempo de Adviento estamos llamados a ensanchar los horizontes de nuestro corazón, a dejarnos sorprender por la vida que se presenta cada día con sus novedades. Para hacer esto es necesario aprender a no depender de nuestras seguridades, de nuestros esquemas consolidados, porque el Señor viene a la hora que no nos imaginamos. Viene para presentarnos una dimensión más hermosa y más grande.

Que Nuestra Señora, Virgen del Adviento, nos ayude a no considerarnos propietarios de nuestra vida, a no oponer resistencia cuando el Señor viene para cambiarla, sino a estar preparados para dejarnos visitar por Él, huésped esperado y grato, aunque desarme nuestros planes.

\subparagraph{Ángelus: Velar}\label{uxe1ngelus-velar}

Plaza de San Pedro. Domingo 1 de diciembre de 2019.

Hoy, primer domingo de Adviento, comienza un nuevo año litúrgico. En estas cuatro semanas de Adviento, la liturgia nos lleva a celebrar el nacimiento de Jesús, mientras nos recuerda que Él viene todos los días en nuestras vidas, y que regresará gloriosamente al final de los tiempos. Esta certeza nos lleva a mirar al futuro con confianza, como nos invita el profeta Isaías, que con su voz inspirada acompaña todo el camino del Adviento.

En la \textbf{primera lectura} de hoy, Isaías profetiza que \textquote{sucederá en días futuros que el monte de la Casa de Yahveh será asentado en la cima de los montes y se alzará por encima de las colinas. Confluirán a él todas las naciones} (Isaías 2, 2). El templo del Señor en Jerusalén se presenta como el punto de encuentro y de convergencia de todos los pueblos.

Después de la Encarnación del Hijo de Dios, Jesús mismo se reveló como el verdadero templo. Por lo tanto, la maravillosa \textbf{visión de 	Isaía}s es una promesa divina y nos impulsa a asumir una actitud de peregrinación, de camino hacia Cristo, sentido y fin de toda la historia. Los que tienen hambre y sed de justicia sólo pueden encontrarla a través de los caminos del Señor, mientras que el mal y el pecado provienen del hecho de que los individuos y los grupos sociales prefieren seguir caminos dictados por intereses egoístas, que causan conflictos y guerras.

El Adviento es el tiempo para acoger la venida de Jesús, que viene como mensajero de paz para mostrarnos los caminos de Dios.

En el \textbf{Evangelio} de hoy, Jesús nos exhorta a estar preparados para su venida: \textquote{Velad, pues, porque no sabéis qué día vendrá vuestro Señor} (Mateo 24, 42). Velar no significa tener los ojos materialmente abiertos, sino tener el corazón libre y orientado en la dirección correcta, es decir, dispuesto a dar y servir. ¡Eso es velar! El sueño del que debemos despertar está constituido por la indiferencia, por la vanidad, por la incapacidad de establecer relaciones verdaderamente humanas, por la incapacidad de hacerse cargo de nuestro hermano aislado, abandonado o enfermo.

La espera de la venida de Jesús debe traducirse, por tanto, en un compromiso de vigilancia. Se trata sobre todo de maravillarse de la acción de Dios, de sus sorpresas y de darle primacía. Vigilancia significa también, concretamente, estar atento al prójimo en dificultades, dejarse interpelar por sus necesidades, sin esperar a que nos pida ayuda, sino aprendiendo a prevenir, a anticipar, como Dios siempre hace con nosotros.

Que María, Virgen vigilante y Madre de la esperanza, nos guíe en este camino, ayudándonos a dirigir la mirada hacia el ``monte del Señor'', imagen de Jesucristo, que atrae a todos los hombres y todos los pueblos.

\protect\hypertarget{_Toc448709347}{}{\protect\hypertarget{_Toc449554349}{}{}}

\subsubsection{Temas}\label{temas}

Tribulación final 
y venida de Cristo en gloria

CEC 668-677, 769:

\textbf{668} "Cristo murió y volvió a la vida para eso, para ser Señor de muertos y vivos" (\emph{Rm} 14, 9). La Ascensión de Cristo al Cielo significa su participación, en su humanidad, en el poder y en la autoridad de Dios mismo. Jesucristo es Señor: posee todo poder en los cielos y en la tierra. El está "por encima de todo principado, potestad, virtud, dominación" porque el Padre "bajo sus pies sometió todas las cosas" (\emph{Ef} 1, 20-22). Cristo es el Señor del cosmos (cf. \emph{Ef} 4, 10; \emph{1 Co} 15, 24. 27-28) y de la historia. En Él, la historia de la humanidad e incluso toda la Creación encuentran su recapitulación (\emph{Ef} 1, 10), su cumplimiento transcendente.

\textbf{669} Como Señor, Cristo es también la cabeza de la Iglesia que es su Cuerpo (cf. \emph{Ef} 1, 22). Elevado al cielo y glorificado, habiendo cumplido así su misión, permanece en la tierra en su Iglesia. La Redención es la fuente de la autoridad que Cristo, en virtud del Espíritu Santo, ejerce sobre la Iglesia (cf. \emph{Ef} 4, 11-13). "La Iglesia, o el reino de Cristo presente ya en misterio" (LG 3), "constituye el germen y el comienzo de este Reino en la tierra" (LG 5).

\textbf{670} Desde la Ascensión, el designio de Dios ha entrado en su consumación. Estamos ya en la "última hora" (\emph{1 Jn} 2, 18; cf. \emph{1 P} 4, 7). "El final de la historia ha llegado ya a nosotros y la renovación del mundo está ya decidida de manera irrevocable e incluso de alguna manera real está ya por anticipado en este mundo. La Iglesia, en efecto, ya en la tierra, se caracteriza por una verdadera santidad, aunque todavía imperfecta" (LG 48). El Reino de Cristo manifiesta ya su presencia por los signos milagrosos (cf. \emph{Mc} 16, 17-18) que acompañan a su anuncio por la Iglesia (cf. \emph{Mc} 16, 20).

\textbf{... esperando que todo le sea sometido}

\textbf{671} El Reino de Cristo, presente ya en su Iglesia, sin embargo, no está todavía acabado "con gran poder y gloria" (\emph{Lc} 21, 27; cf. \emph{Mt} 25, 31) con el advenimiento del Rey a la tierra. Este Reino aún es objeto de los ataques de los poderes del mal (cf. \emph{2 Ts} 2, 7), a pesar de que estos poderes hayan sido vencidos en su raíz por la Pascua de Cristo. Hasta que todo le haya sido sometido (cf. \emph{1 Co} 15, 28), y "mientras no {[}...{]} haya nuevos cielos y nueva tierra, en los que habite la justicia, la Iglesia peregrina lleva en sus sacramentos e instituciones, que pertenecen a este tiempo, la imagen de este mundo que pasa. Ella misma vive entre las criaturas que gimen en dolores de parto hasta ahora y que esperan la manifestación de los hijos de Dios" (LG 48). Por esta razón los cristianos piden, sobre todo en la Eucaristía (cf. \emph{1 Co} 11, 26), que se apresure el retorno de Cristo (cf. \emph{2 P} 3, 11-12) cuando suplican: "Ven, Señor Jesús" (\emph{Ap} 22, 20; cf. \emph{1 Co} 16, 22; \emph{Ap} 22, 17-20).

\textbf{\\ }

\textbf{672} Cristo afirmó antes de su Ascensión que aún no era la hora del establecimiento glorioso del Reino mesiánico esperado por Israel (cf. \emph{Hch} 1, 6-7) que, según los profetas (cf. \emph{Is} 11, 1-9), debía traer a todos los hombres el orden definitivo de la justicia, del amor y de la paz. El tiempo presente, según el Señor, es el tiempo del Espíritu y del testimonio (cf \emph{Hch} 1, 8), pero es también un tiempo marcado todavía por la "tribulación" (\emph{1 Co} 7, 26) y la prueba del mal (cf. \emph{Ef} 5, 16) que afecta también a la Iglesia (cf. \emph{1 P} 4, 17) e inaugura los combates de los últimos días (\emph{1 Jn} 2, 18; 4, 3; \emph{1 Tm} 4, 1). Es un tiempo de espera y de vigilia (cf. \emph{Mt} 25, 1-13; \emph{Mc} 13, 33-37).

\textbf{El glorioso advenimiento de Cristo, esperanza de Israel}

\textbf{673} Desde la Ascensión, el advenimiento de Cristo en la gloria es inminente (cf \emph{Ap} 22, 20) aun cuando a nosotros no nos "toca conocer el tiempo y el momento que ha fijado el Padre con su autoridad" (\emph{Hch} 1, 7; cf. \emph{Mc} 13, 32). Este acontecimiento escatológico se puede cumplir en cualquier momento (cf. \emph{Mt} 24, 44: \emph{1 Ts} 5, 2), aunque tal acontecimiento y la prueba final que le ha de preceder estén "retenidos" en las manos de Dios (cf. \emph{2 	Ts} 2, 3-12).

\textbf{674} La venida del Mesías glorioso, en un momento determinado de la historia (cf. \emph{Rm} 11, 31), se vincula al reconocimiento del Mesías por "todo Israel" (\emph{Rm} 11, 26; \emph{Mt} 23, 39) del que "una parte está endurecida" (\emph{Rm} 11, 25) en "la incredulidad" (\emph{Rm} 11, 20) respecto a Jesús. San Pedro dice a los judíos de Jerusalén después de Pentecostés: "Arrepentíos, pues, y convertíos para que vuestros pecados sean borrados, a fin de que del Señor venga el tiempo de la consolación y envíe al Cristo que os había sido destinado, a Jesús, a quien debe retener el cielo hasta el tiempo de la restauración universal, de que Dios habló por boca de sus profetas" (\emph{Hch} 3, 19-21). Y san Pablo le hace eco: "si su reprobación ha sido la reconciliación del mundo ¿qué será su readmisión sino una resurrección de entre los muertos?" (\emph{Rm} 11, 5). La entrada de "la plenitud de los judíos" (\emph{Rm} 11, 12) en la salvación mesiánica, a continuación de "la plenitud de los gentiles (Rm 11, 25; cf. Lc 21, 24), hará al pueblo de Dios "llegar a la plenitud de Cristo" (\emph{Ef} 4, 13) en la cual "Dios será todo en nosotros" (\emph{1 Co} 15, 28).

\textbf{La última prueba de la Iglesia}

\textbf{675} Antes del advenimiento de Cristo, la Iglesia deberá pasar por una prueba final que sacudirá la fe de numerosos creyentes (cf. \emph{Lc} 18, 8; \emph{Mt} 24, 12). La persecución que acompaña a su peregrinación sobre la tierra (cf. \emph{Lc} 21, 12; \emph{Jn} 15, 19-20) desvelará el "misterio de iniquidad" bajo la forma de una impostura religiosa que proporcionará a los hombres una solución aparente a sus problemas mediante el precio de la apostasía de la verdad. La impostura religiosa suprema es la del Anticristo, es decir, la de un seudo-mesianismo en que el hombre se glorifica a sí mismo colocándose en el lugar de Dios y de su Mesías venido en la carne (cf. \emph{2 Ts} 2, 4-12; \emph{1Ts} 5, 2-3;2 \emph{Jn} 7; \emph{1 Jn} 2, 18.22).

\textbf{\\ }

\textbf{676} Esta impostura del Anticristo aparece esbozada ya en el mundo cada vez que se pretende llevar a cabo la esperanza mesiánica en la historia, lo cual no puede alcanzarse sino más allá del tiempo histórico a través del juicio escatológico: incluso en su forma mitigada, la Iglesia ha rechazado esta falsificación del Reino futuro con el nombre de milenarismo (cf. DS 3839), sobre todo bajo la forma política de un mesianismo secularizado, "intrínsecamente perverso" (cf. Pío XI, carta enc. \emph{Divini Redemptoris}, condenando "los errores presentados bajo un falso sentido místico" "de esta especie de falseada redención de los más humildes"; GS 20-21).

\textbf{677} La Iglesia sólo entrará en la gloria del Reino a través de esta última Pascua en la que seguirá a su Señor en su muerte y su Resurrección (cf. \emph{Ap} 19, 1-9). El Reino no se realizará, por tanto, mediante un triunfo histórico de la Iglesia (cf. \emph{Ap} 13, 8) en forma de un proceso creciente, sino por una victoria de Dios sobre el último desencadenamiento del mal (cf. \emph{Ap} 20, 7-10) que hará descender desde el cielo a su Esposa (cf. \emph{Ap} 21, 2-4). El triunfo de Dios sobre la rebelión del mal tomará la forma de Juicio final (cf. \emph{Ap} 20, 12) después de la última sacudida cósmica de este mundo que pasa (cf. \emph{2 P} 3, 12-13).

\textbf{La Iglesia, consumada en la gloria}

\textbf{769} La Iglesia "sólo llegará a su perfección en la gloria del cielo" (LG 48), cuando Cristo vuelva glorioso. Hasta ese día, "la Iglesia avanza en su peregrinación a través de las persecuciones del mundo y de los consuelos de Dios" (San Agustín, \emph{De civitate Dei} 18, 51; cf. LG 8). Aquí abajo, ella se sabe en exilio, lejos del Señor (cf. \emph{2Co} 5, 6; LG 6), y aspira al advenimiento pleno del Reino, "y espera y desea con todas sus fuerzas reunirse con su Rey en la gloria" (LG 5). La consumación de la Iglesia en la gloria, y a través de ella la del mundo, no sucederá sin grandes pruebas. Solamente entonces, "todos los justos descendientes de Adán, 'desde Abel el justo hasta el último de los elegidos' se reunirán con el Padre en la Iglesia universal" (LG 2).

``¡Ven, Señor Jesús!''

CEC 451, 671, 1130, 1403, 2817:

\textbf{451} La oración cristiana está marcada por el título "Señor", ya sea en la invitación a la oración "el Señor esté con vosotros", o en su conclusión "por Jesucristo nuestro Señor" o incluso en la exclamación llena de confianza y de esperanza: \emph{Maran atha} ("¡el Señor viene!") o \emph{Marana tha} ("¡Ven, Señor!") (\emph{1 Co} 16, 22): "¡Amén! ¡ven, Señor Jesús!" (\emph{Ap} 22, 20).

\textbf{\\ }

\textbf{... esperando que todo le sea sometido}

\textbf{671} El Reino de Cristo, presente ya en su Iglesia, sin embargo, no está todavía acabado "con gran poder y gloria" (\emph{Lc} 21, 27; cf. \emph{Mt} 25, 31) con el advenimiento del Rey a la tierra. Este Reino aún es objeto de los ataques de los poderes del mal (cf. \emph{2 Ts} 2, 7), a pesar de que estos poderes hayan sido vencidos en su raíz por la Pascua de Cristo. Hasta que todo le haya sido sometido (cf. \emph{1 Co} 15, 28), y "mientras no {[}...{]} haya nuevos cielos y nueva tierra, en los que habite la justicia, la Iglesia peregrina lleva en sus sacramentos e instituciones, que pertenecen a este tiempo, la imagen de este mundo que pasa. Ella misma vive entre las criaturas que gimen en dolores de parto hasta ahora y que esperan la manifestación de los hijos de Dios" (LG 48). Por esta razón los cristianos piden, sobre todo en la Eucaristía (cf. \emph{1 Co} 11, 26), que se apresure el retorno de Cristo (cf. \emph{2 P} 3, 11-12) cuando suplican: "Ven, Señor Jesús" (\emph{Ap} 22, 20; cf. \emph{1 Co} 16, 22; \emph{Ap} 22, 17-20).

\textbf{Sacramentos de la vida eterna}

\textbf{1130} La Iglesia celebra el Misterio de su Señor "hasta que él venga" y "Dios sea todo en todos" (\emph{1 Co} 11, 26; 15, 28). Desde la era apostólica, la liturgia es atraída hacia su término por el gemido del Espíritu en la Iglesia: \emph{¡Marana tha!} (\emph{1 Co} 16,22). La liturgia participa así en el deseo de Jesús: "Con ansia he deseado comer esta Pascua con vosotros {[}...{]} hasta que halle su cumplimiento en el Reino de Dios" (\emph{Lc} 22,15-16). En los sacramentos de Cristo, la Iglesia recibe ya las arras de su herencia, participa ya en la vida eterna, aunque "aguardando la feliz esperanza y la manifestación de la gloria del Gran Dios y Salvador nuestro Jesucristo" (\emph{Tt} 2,13). "El Espíritu y la Esposa dicen: ¡Ven! {[}...{]} ¡Ven, Señor Jesús!" (\emph{Ap} 22,17.20).

\begin{quote} 	Santo Tomás resume así las diferentes dimensiones del signo sacramental: 	\textquote{\emph{Unde sacramentum est signum rememorativum eius quod praecessit, scilicet passionis Christi; et desmonstrativum eius quod in nobis efficitur per Christi passionem, scilicet gratiae; et prognosticum, id est, praenuntiativum futurae gloriae}} (\textquote{Por eso el sacramento es un 	signo que rememora lo que sucedió, es decir, la pasión de Cristo; es un 	signo que demuestra lo que se realiza en nosotros en virtud de la pasión 	de Cristo, es decir, la gracia; y es un signo que anticipa, es decir, 	que preanuncia la gloria venidera}) (\emph{Summa theologiae} 3, q. 60, 	a. 3, c.) \end{quote}

\textbf{1403} En la última Cena, el Señor mismo atrajo la atención de sus discípulos hacia el cumplimiento de la Pascua en el Reino de Dios: "Y os digo que desde ahora no beberé de este fruto de la vid hasta el día en que lo beba con vosotros, de nuevo, en el Reino de mi Padre" (\emph{Mt} 26,29; cf. \emph{Lc} 22,18; \emph{Mc} 14,25). Cada vez que la Iglesia celebra la Eucaristía recuerda esta promesa y su mirada se dirige hacia "el que viene" (\emph{Ap} 1,4). En su oración, implora su venida: \emph{Marana tha} (\emph{1 Co} 16,22), "Ven, Señor Jesús" (\emph{Ap} 22,20), "que tu gracia venga y que este mundo pase" (\emph{Didaché} 10,6).

\textbf{\\ }

\textbf{Venga a nosotros tu Reino}

\textbf{2817} Esta petición es el \emph{Marana Tha}, el grito del Espíritu y de la Esposa: ``Ven, Señor Jesús'':

\begin{quote} 	\textquote{Incluso aunque esta oración no nos hubiera mandado pedir el 	advenimiento del Reino, habríamos tenido que expresar esta petición, 	dirigiéndonos con premura a la meta de nuestras esperanzas. Las almas de 	los mártires, bajo el altar, invocan al Señor con grandes gritos: 	``¿Hasta cuándo, Dueño santo y veraz, vas a estar sin hacer justicia por 	nuestra sangre a los habitantes de la tierra?'' (\emph{Ap} 6, 10). En 	efecto, los mártires deben alcanzar la justicia al fin de los tiempos. 	Señor, ¡apresura, pues, la venida de tu Reino!} (Tertuliano, \emph{De oratione}, 5, 2-4). \end{quote}

La vigilancia humilde del corazón

CEC 2729-2733:

\textbf{Frente a las dificultades de la oración}

\textbf{2729} La dificultad habitual de la oración es la \emph{distracción}. En la oración vocal, la distracción puede referirse a las palabras y al sentido de estas. La distracción, de un modo más profundo, puede referirse a Aquél al que oramos, tanto en la oración vocal (litúrgica o personal), como en la meditación y en la oración contemplativa. Dedicarse a perseguir las distracciones es caer en sus redes; basta con volver a nuestro corazón: la distracción descubre al que ora aquello a lo que su corazón está apegado. Esta humilde toma de conciencia debe empujar al orante a ofrecerse al Señor para ser purificado. El combate se decide cuando se elige a quién se desea servir (cf \emph{Mt} 6,21.24).

\textbf{2730} Mirado positivamente, el combate contra el ánimo posesivo y dominador es la vigilancia, la sobriedad del corazón. Cuando Jesús insiste en la vigilancia, es siempre en relación a Él, a su Venida, al último día y al ``hoy''. El esposo viene en mitad de la noche; la luz que no debe apagarse es la de la fe: ``Dice de ti mi corazón: busca su rostro'' (\emph{Sal} 27, 8).

\textbf{2731} Otra dificultad, especialmente para los que quieren sinceramente orar, es la \emph{sequedad}. Forma parte de la oración en la que el corazón está desprendido, sin gusto por los pensamientos, recuerdos y sentimientos, incluso espirituales. Es el momento en que la fe es más pura, la fe que se mantiene firme junto a Jesús en su agonía y en el sepulcro. ``El grano de trigo, si {[}...{]} muere, da mucho fruto'' (\emph{Jn} 12, 24). Si la sequedad se debe a falta de raíz, porque la Palabra ha caído sobre roca, no hay éxito en el combate sin una mayor conversión (cf \emph{Lc} 8, 6. 13).

\textbf{\\ }

\textbf{Frente a las tentaciones en la oración}

\textbf{2732} La tentación más frecuente, la más oculta, es nuestra \emph{falta de fe}. Esta se expresa menos en una incredulidad declarada que en unas preferencias de hecho. Cuando se empieza a orar, se presentan como prioritarios mil trabajos y cuidados que se consideran más urgentes; una vez más, es el momento de la verdad del corazón y de su más profundo deseo. Mientras tanto, nos volvemos al Señor como nuestro único recurso; pero ¿alguien se lo cree verdaderamente? Consideramos a Dios como asociado a la alianza con nosotros, pero nuestro corazón continúa en la arrogancia. En cualquier caso, la falta de fe revela que no se ha alcanzado todavía la disposición propia de un corazón humilde: \textquote{Sin mí, no podéis hacer nada} (\emph{Jn} 15, 5).

\textbf{2733} Otra tentación a la que abre la puerta la presunción es la \emph{acedia}. Los Padres espirituales entienden por ella una forma de aspereza o de desabrimiento debidos a la pereza, al relajamiento de la ascesis, al descuido de la vigilancia, a la negligencia del corazón. ``El espíritu {[}...{]} está pronto pero la carne es débil'' (\emph{Mt} 26, 41). Cuanto más alto es el punto desde el que alguien toma decisiones, tanto mayor es la dificultad. El desaliento, doloroso, es el reverso de la presunción. Quien es humilde no se extraña de su miseria; ésta le lleva a una mayor confianza, a mantenerse firme en la constancia.



El que quisiere ver cuánto ha aprovechado en este camino de Dios, mire cuánto crece cada día en humildad interior y exterior. ¿Cómo sufre las injusticias de los otros? ¿Cómo sabe dar pasada a las flaquezas ajenas? ¿Cómo acude a las necesidades de sus prójimos? ¿Cómo se compadece y no se indigna contra los defectos ajenos? ¿Cómo sabe esperar en Dios en el tiempo de la tribulación? ¿Cómo rige su lengua? ¿Cómo guarda su corazón? ¿Cómo trae domada su carne con todos sus apetitos y sentidos? ¿ Cómo se sabe valer en las prosperidades y adversidades? ¿Cómo se repara y provee en todas las cosas con gravedad y discreción?

Y, sobre todo esto, mire si está muerto el amor de la honra, y del regalo, y del mundo, y según lo que en esto hubiere aprovechado o desaprovechado, así se juzgue, y no según lo que siente o no siente de Dios. Y por esto siempre ha de tener él un ojo, y el más principal en la mortificación, y el otro en la oración, porque esa misma mortificación no se puede perfectamente alcanzar sin el socorro de la oración.

\textbf{San Pedro de Alcántara}, \emph{Tratado sobre la Oración,} capítulo 5.\strut


