\section{Alter} JUAN PABLO II .

\emph{\textbf{AUDIENCIA GENERAL}}\\[2\baselineskip]\emph{Miércoles 30 de diciembre de 1992}



\emph{Queridísimos hermanos y hermanas en el Señor; queridísimos jóvenes:}

1. Hemos celebrado hace algunos días la solemnidad de la Navidad y estamos todavía penetrados por la atmósfera sugestiva de la Noche Santa. Contemplamos asombrados, junto a María santísima y a san José, el misterio del Verbo encarnado.

El nacimiento del Hijo de Dios \textquote{de una mujer} (cf. \emph{Gal} 4, 4) nos hace remontarnos de nuevo al proyecto salvífico: el Altísimo ha querido entrar directamente en la historia de la humanidad y nos ha dado a su Hijo unigénito como Salvador y Redentor.

Eso es la Navidad, \textquote{misterio} providencial de amor, en el que María, escogida como virgen Madre del Emmanuel, se encuentra \emph{asociada a la obra de la redención.} Nos detenemos en estos días a \emph{contemplar a María en Belén.} La Madre, que estrecha entre sus brazos a Jesús, nos ayuda a comprender ante todo que de la gruta, iluminada por la luz divina, viene un \emph{mensaje de verdad:} Dios se ha hecho hombre y, compartiendo nuestra naturaleza, nos habla con el poder de su misericordia salvadora. Sin embargo, \emph{es María quien nos da la Palabra que salva:} ella nos muestra a Jesús, \textquote{la luz del mundo}, que da el verdadero sentido a la vida y el pleno significado a la existencia. ¿Cómo no permanecer sorprendidos y maravillados ante tal misterio? ¿Cómo no abrir el corazón a la venida entre nosotros del Señor de la historia?

2. Queridos jóvenes, que habéis venido de diferentes partes de Europa en nombre de María: La joven de Nazaret, presente silenciosamente en el misterio de la Navidad, está presente también en el corazón de la Iglesia y en el de cada fiel. El \emph{Catecismo de la Iglesia Católica} publicado recientemente afirma, que \textquote{por su total adhesión a la voluntad del Padre, a la obra redentora de su Hijo, a toda moción del Espíritu Santo, la Virgen María es para la Iglesia el modelo de la fe y de la caridad. Por eso es \textquote{miembro muy eminente y del todo singular de la Iglesia} (\emph{Lumen gentium}, 53), incluso constituye \textquote{la figura} [\textquote{tipus}] de la Iglesia (\emph{Lumen gentium}, 63)} (n. 967).

\emph{María es madre:} madre de Cristo y madre nuestra. Su función maternal \textquote{dimana\ldots{} de la superabundancia de los méritos de Cristo; se apoya en la mediación de éste, depende totalmente de ella y de la misma saca todo su poder} (\emph{Lumen gentium}, 60). Con respecto a los creyentes, su función es ser \textquote{nuestra madre en el orden de la gracia} (\emph{ib}., \emph{} 61), y por esto \textquote{es invocada en la Iglesia con los títulos de abogada, auxiliadora, socorro, mediadora} (\emph{ib}., 62). Se trata de una misión providencial que el Señor le ha confiado y que se resume perfectamente en la expresión \emph{Per Mariam ad Iesurm.}

Ésta es, como sabéis bien, la doctrina fundamental de san Luis María Grignon de Monfort, en quien vosotros os inspiráis; y es el ideal que debe impulsar a todos los cristianos. Gracias a la ayuda de la Madre de Dios, el testimonio de los creyentes se hace cada día más coherente y fervoroso, más generoso y más abierto.

3. El concilio Vaticano II, cuyo trigésimo aniversario estamos celebrando, exhortó a los fieles a ofrecer \textquote{súplicas apremiantes a la Madre de Dios y Madre de los hombres para que ella, que ayudó con sus oraciones a la Iglesia naciente, también ahora, ensalzada en el cielo por encima de todos los ángeles y bienaventurados, interceda en la comunión de todos los santos ante su Hijo hasta que todas las familias de los pueblos, tanto los que se honran con el título de cristianos como los que todavía desconocen a su Salvador, lleguen a reunirse felizmente, en paz y concordia, en un solo pueblo de Dios, para gloria de la santísima e indivisible Trinidad} (\emph{Lumen gentium}, 69).

Queridos hermanos y hermanas, de esta profunda riqueza espiritual brota vuestra devoción a María y vuestro compromiso apostólico. Mirad siempre a María como a la estrella segura que os guía en el camino de vuestra vida cristiana.

4\emph{.} Vosotros, queridos jóvenes, que representáis el futuro de la humanidad y la esperanza de la Iglesia, debéis llevar el evangelio de la bondad y de la paz a todos los rincones de los países de donde provenís. En toda Europa aumentan las dificultades y algunas regiones se hallan azotadas por la violencia. Por esto, vuestra misión es una obra de solidaridad espiritual, un servicio a la verdad, que exige un testimonio creíble del mensaje integral de Cristo.

Ante vosotros resplandece \emph{María, la Virgen fiel, la estrella de la evangelización, vuestra madre y modelo.} Acudid a ella todos los días, como lo deseáis hacer hoy.

Con la ayuda de su intercesión maternal, podréis contribuir activamente a la obra de la nueva evangelización, y podréis ser un fermento genuino de vida cristiana y de comunión fiel en vuestras comunidades eclesiales.

5. Queridísimos hermanos, os saludo cordialmente a todos vosotros y a los países de los que provenís. Vuestra presencia aquí es un signo más de la unidad entre las naciones y comunidades cristianas, que se caracteriza por la \textquote{comunicación de bienes} espirituales y materiales con la finalidad de construir un futuro común basado en la justicia y la solidaridad.

Que la Madre del Redentor acompañe vuestra misión de \emph{creyentes y apóstoles del Evangelio.} Con fe y confianza invoquemos su intercesión. Oremos a fin de que obtenga las gracias necesarias para cada uno de nosotros, para toda la humanidad, para todos los que sienten más el peso de la vida y de las adversidades. Pidámosle humildemente, llenos de la alegría de la Navidad, que suscite y mantenga en todos los bautizados una fe convencida y coherente.

De modo especial esforcémonos por escuchar sus enseñanzas y seguir el ejemplo de su vida.

Dirijámonos a ella con las palabras de la antífona del tiempo de Navidad: \textquote{Santa Madre del Redentor, sostiene al pueblo que quiere levantarse. Tú, ante el asombro de toda la creación, has engendrado a tu Creador\ldots{} Ten piedad de nosotros, pecadores}.

\subsubsection{AG 1995} JUAN PABLO II

\textbf{\emph{AUDIENCIA GENERAL\\ }}\\ \emph{Miércoles 20 de diciembre de 1995}

~

\emph{(Lectura:\\ evangelio de san Lucas, capítulo 1,\\ versículos 68-69)}

1. Ya se acerca la Navidad del Señor, para la que nos estamos preparando durante estos días de Adviento. La solemnidad de la Navidad nos trae recuerdos de ternura y bondad, suscitando cada vez nueva atención hacia los valores humanos fundamentales: la familia, la vida, la inocencia, la paz y la gratuidad.

La Navidad es la \emph{fiesta de la familia} que, reunida en torno al belén y al árbol, símbolos navideños tradicionales, se redescubre llamada a ser el santuario de la vida y del amor. La Navidad es la \emph{fiesta de los niños,} porque pone de manifiesto \textquote{el sentido profundo de todo nacimiento humano, y la alegría mesiánica constituye así el fundamento y realización de la alegría por cada niño que nace} (\emph{Evangelium vitae}, 1). La Navidad del Señor lleva a redescubrir, además, el valor de la \emph{inocencia,} invitando a los adultos a aprender de los niños a acercarse con asombro y pureza de corazón a la cuna del Salvador, recién nacido.

La Navidad es la \emph{fiesta de la paz,} porque \textquote{la paz verdadera nos viene del cielo} y \textquote{por toda la tierra los cielos destilan dulzura} (\emph{Liturgia de las Horas,} oficio de lectura de Navidad). Los ángeles cantan en Belén: \textquote{Gloria a Dios en las alturas y paz en la tierra a los hombres que él ama} (\emph{Lc} 2, 14). En este tiempo, que invita a la alegría, ¿cómo no pensar con tristeza en los que, por desgracia, en muchas partes del mundo, se hallan aún inmersos en grandes tragedias? ¿Cuándo podrán celebrar una verdadera Navidad? ¿Cuándo podrá la humanidad vivir la Navidad en un mundo completamente reconciliado? Algunos signos de esperanza, gracias a Dios, nos impulsan a proseguir incansablemente en la búsqueda de la paz.

Mi pensamiento se dirige, naturalmente, a Bosnia, donde el acuerdo logrado, aún con límites comprensibles y con notables sacrificios, constituye un gran paso adelante por el camino de la reconciliación y la paz.

La Navidad es también la \emph{fiesta de los regalos:} me imagino la alegría de los niños, y también de los adultos, que reciben un regalo navideño, al sentirse amados y comprometidos a transformarse ellos mismos en don, como el Niño que la Virgen María nos muestra en el belén.

2. Pero estas consideraciones explican sólo en parte el clima festivo y sugestivo de la Navidad. Como ya es sabido, para los creyentes el auténtico fundamento de la alegría de esta fiesta estriba en el hecho de que \emph{el Verbo eterno,} imagen perfecta del Padre \emph{se ha hecho} \textquote{\emph{carne}}, niño frágil solidario con los hombres débiles y mortales. \emph{En Jesús, Dios mismo se ha acercado y permanece con nosotros,} como don incomparable que es preciso acoger con humildad en nuestra vida.

En el nacimiento del Hijo de Dios del seno virginal de una humilde joven, María de Nazaret, los cristianos reconocen la infinita condescendencia del Altísimo hacia el hombre. Ese acontecimiento, junto con la muerte y resurrección de Cristo, constituye el culmen de la historia.

En la carta del apóstol Pablo a los Filipenses encontramos un himno a Cristo, con el que la Iglesia primitiva expresaba la gratitud y el asombro ante el sublime misterio de Dios que se hace solidario con los hombres: \textquote{(Cristo) siendo de condición divina, no retuvo ávidamente el ser igual a Dios, sino que se despojó de sí mismo tomando condición de siervo, haciéndose semejante a los hombres y apareciendo en su porte como hombre: y se humilló a sí mismo, obedeciendo hasta la muerte y muerte de cruz} (\emph{Flp} 2, 6-8).

En el decurso de los primeros siglos, la Iglesia defendió con especial tenacidad este misterio frente a varias herejías que, al negar, de vez en cuando, la verdadera humanidad de Jesús, su real filiación divina, su divinidad o la unidad de su Persona, tendían a vaciar su excepcional y sorprendente contenido y a desvirtuar el insólito y consolador mensaje que trae al hombre de todos los tiempos.

El \emph{Catecismo de la Iglesia católica} nos recuerda que \textquote{el acontecimiento único y totalmente singular de la encarnación del Hijo de Dios no significa que Jesucristo sea en parte Dios y en parte hombre, ni que sea el resultado de una mezcla confusa entre lo divino y lo humano. Él se hizo verdaderamente hombre sin dejar de ser verdaderamente Dios. Jesucristo es verdadero Dios y verdadero hombre} (n. 464).

3. ¿Qué significado tiene para nosotros el evento extraordinario del nacimiento de Jesucristo? ¿Qué \emph{buena nueva} nos trae? ¿A qué metas nos impulsa? San Lucas, el evangelista de la Navidad, en las palabras inspiradas de Zacarías nos presenta la Encarnación como la \emph{visita de Dios:} \textquote{Bendito el Señor Dios de Israel porque \emph{ha visitado y redimido a su pueblo,} y nos ha suscitado una fuerza salvadora en la casa de David, su siervo} (\emph{Lc} 1, 68-69).

Pero ¿qué efectos produce en el hombre la \emph{visita de Dios}? La sagrada Escritura testimonia que cuando el Señor interviene, trae salvación y alegría, libra de la aflicción, infunde esperanza, mira el destino del que recibe la visita y abre perspectivas nuevas de vida y salvación.

\emph{La Navidad es la visita de Dios por excelencia,} pues en este acontecimiento se hace sumamente cercano al hombre mediante su Hijo único, que manifiesta en el rostro de un niño su ternura hacia los pobres y los pecadores. En el Verbo encarnado se ofrece a los hombres la gracia de la adopción como hijos de Dios. San Lucas se preocupa de mostrar que el evento del nacimiento de Jesús cambia realmente la historia y la vida de los hombres, sobre todo de los que lo acogen con corazón sincero: Isabel, Juan Bautista, los pastores, Simeón, Ana y sobre todo María son testigos de las maravillas que Dios obra con su visita.

En María, de manera especial, el evangelista presenta no sólo un modelo que es necesario seguir para acoger a Dios que sale a nuestro encuentro, sino también las perspectivas exultantes que se abren a quien, habiéndolo acogido, está Llamado a \emph{convertirse,} a su vez, \emph{en instrumento de su visita} y heraldo de su salvación: \textquote{Apenas llegó a mis oídos la voz de tu saludo, saltó de gozo el niño en mi seno}, exclama Isabel dirigiéndose a la Virgen, que le lleva en sí misma la visita de Dios (\emph{Lc} 1, 44). La misma alegría invade a los pastores, que van a Belén por invitación del ángel y encuentran al niño con su Madre: vuelven \textquote{glorificando y alabando a Dios} (\emph{Lc} 2, 20), porque saben que el Señor los ha visitado.

A la luz del misterio que nos disponemos a celebrar, expreso a todos el deseo de que acudamos en esta Navidad, como María, a Cristo que viene a \textquote{visitarnos de lo alto} (\emph{Lc} 1, 78), con corazón abierto y disponible, para convertirnos en instrumentos de la alegre visita de Dios para cuantos encontremos en nuestro camino diario.

\subsubsection{****PARA AÑO B**** Urbi et Orbi 1999} \textbf{\emph{MENSAJE URBI ET ORBI}}

\emph{Navidad, 25 de diciembre de 1999}

~

1. \textquote{\emph{Un niño nos ha nacido.\\ un hijo se nos ha dado}} (Is 9, 5) .\\ Hoy resuena en la Iglesia y en el mundo la \textquote{buena noticia} de la Navidad.\\ Resuena con las palabras del profeta Isaías,\\ llamado por esto el \textquote{evangelista} del Antiguo Testamento,~\\ el cual, hablando del misterio de la redención,\\ parece entrever los acontecimiento de siete siglos después.\\ Palabras inspiradas por Dios, palabras sorprendentes que recorren la historia,~\\ y que hoy, a las puertas del Dos mil, resuenan en toda la tierra\\ anunciando el gran misterio de la Encarnación.

2. \textquote{\emph{Un Niño nos ha nacido}}.\\ Estas palabras proféticas se ven realizadas en la narración del evangelista Lucas,~\\ que describe el \textquote{\emph{acontecimiento}} lleno\\ cada vez más de nueva admiración y esperanza.\\ En la noche de Belén,\\ María dio a luz un Niño, al que puso por nombre Jesús.\\ No había lugar para ellos e la pensión;\\ por esto la Madre alumbró al Hijo\\ en una gruta y lo puso en un pesebre .\\ El evangelista Juan, en el Prólogo de su evangelio,\\ penetra en el \textquote{ \emph{misterio} } de este acontecimiento.\\ Aquel que nace en la gruta es el Hijo eterno de Dios.\\ Es la Palabra, que existía en el principio, la Palabra que estaba junto a Dios,\\ la Palabra que era Dios.\\ Todo lo que ha sido hecho, por medio de la Palabra se hizo (cf. 1,1-3).\\ La Palabra eterna, el Hijo de Dios,\\ tomó la naturaleza humana.\\ Dios Padre \textquote{\emph{tanto amó al mundo\\ que le ha dado su Hijo único}} (Jn 3,16).\\ El profeta Isaías al decir: \textquote{\emph{un hijo se nos ha dado}},\\ revela en toda su plenitud \emph{el misterio de Navidad}:\\ le generación eterna de la Palabra en el Padre,\\ su nacimiento en el tiempo por obra del Espíritu Santo.

3. Se ensancha el círculo del misterio :\\ el evangelista Juan afirma: \textquote{\emph{La Palabra se hizo carne,\\ y puso su Morada entre nosotros} } (Jn 1,14)\\ y añade : \textquote{\emph{a todos tos que la recibieron\\ les dio poder de hacerse hijos de Dios,\\ a los que creen en su nombre} } (ibíd. 1,12).\\ Se ensancha el círculo del misterio:\\ el nacimiento del Hijo de Dios es el don sublime,\\ la gracia más grande en favor del hombre,\\ que la mente humana nunca hubiera podido imaginar.\\ Recordando, en este Día santo,\\ el nacimiento de Cristo,\\ vivimos, junto con este acontecimiento,\\ el \textquote{\emph{misterio de la divina adopción}},\\ por obra de Cristo que viene al mundo.\\ Por eso, la Noche y el Día de Navidad\\ son tenidos como \textquote{sagrados } por los hombres que buscan la verdad.\\ Nosotros, cristianos, los consideramos \textquote{santos } reconociendo en ellos la huella inconfundible de Aquel que es Santo, lleno de misericordia y de bondad.

4. Un motivo más se añade este año\\ para considerar más santo este día de gracia:\emph{\\ es el comienzo del Gran Jubileo}.\\ Esta Noche, antes de la Santa Misa,\\ he abierto la Puerta Santa de esta Basílica.\\ Acto simbólico con el cual se inaugura el Año Jubilar,\\ gesto que pone de relieve con elocuencia singular\\ un elemento ya contenido en el misterio de Navidad:\\ ¡\emph{Jesús}, nacido en la pobreza de Belén,\\ Cristo, \emph{el Hijo eterno} que nos ha sido dado por el Padre,\emph{\\ es}, para nosotros y para todos, \emph{la Puerta}!\emph{\\ la Puerta de nuestra salvación},\emph{\\ la Puerta de la vida},\emph{\\ la Puerta de la paz} !\\ Éste es el mensaje de Navidad y el anuncio del Gran Jubileo.

5. Dirigimos la mirada hacia ti, Cristo,\emph{\\ Puerta de nuestra salvación},\\ y te damos gracias por el bien realizado en los años, siglos y milenios pasados.\\ Debemos confesar, sin embargo, que a veces la humanidad ha buscado fuera de ti la Verdad,\\ que se ha fabricado falsas certezas, ha corrido tras ideologías falaces.\\ A veces el hombre ha excluido del propio respeto y amor\\ a hermanos de otras razas o distintos credos,\\ ha negado los derechos fundamentales a las personas y a las naciones.\\ Pero Tú sigues ofreciendo a todos el Esplendor de la Verdad que salva.\\ Te miramos a Ti, Cristo, \emph{Puerta de la Vida},\\ y te damos gracias por los prodigios\\ con que has enriquecido a cada generación.\\ A veces este mundo a veces no respeta y no ama la vida.\\ Pero Tú no te cansas de amarla,\\ más aún, en el misterio de la Navidad vienes a iluminar las mentes\\ para que los legisladores y los gobernantes, hombres y mujeres de buena voluntad se comprometan a acoger, como don precioso, la vida del hombre.\\ Tú vienes a darnos el Evangelio de la Vida.\\ Fijamos los ojos en Ti, Cristo, \emph{Puerta de la paz},\\ mientras, peregrinos en el tiempo,\\ visitamos tantos lugares del dolor y de la guerra, donde reposan las víctimas\\ de violentos conflictos y de crueles exterminios.\\ Tú, Príncipe de la paz,\\ nos invitas a abandonar el insensato uso de las armas, el recurso a la violencia y al odio que han marcado con la muerte a personas, pueblos y continentes.

6. \textquote{\emph{Un hijo se nos ha dado} }.\\ Tú, Padre, nos \emph{has dado a tu Hijo}.\\ Nos lo das también hoy, al alba del nuevo milenio .\\ Él es la Puerta para nosotros.\\ A través de El entramos en una nueva dimensión\\ y alcanzamos la plenitud del destino de la salvación\\ pensado por ti para todos.\\ Precisamente por esto, Padre, nos has dado a tu Hijo,\\ para que el hombre experimente lo que Tú quieres dar en la eternidad,\\ para que el hombre tenga la fuerza de realizar\\ tu arcano misterio de amor.\\ Cristo, Hijo de la Madre siempre Virgen,\\ luz y esperanza de quienes te buscan, aun sin conocerte y de quienes, conociéndote, te buscan cada vez más; Cristo, ¡Tú eres la Puerta!\\ A través de ti,\\ con la fuerza del Espíritu Santo, queremos entrar en el tercer milenio.\\ Tú, Cristo, eres el mismo ayer, hoy y siempre (cf. Hb 13,8)

\subsubsection{UO 1998} \textbf{\emph{MENSAJE URBI ET ORBI}}

\emph{(Navidad 1998)}

~

1. \emph{\textquote{Regem venturum Dominum, venite, adoremus}}\\ \textquote{Venid, adoremos al Rey, al Señor que ha de venir}.\\ Cuántas veces hemos repetido estas palabras\\ durante el tiempo de Adviento,\\ haciéndonos eco de la esperanza de toda la humanidad.

Proyectado hacia el futuro desde sus más remotos orígenes,\\ el hombre busca a Dios, plenitud de la vida. Desde siempre\\ invoca un Salvador que lo libre del mal y de la muerte,\\ que colme su necesidad innata de felicidad.\\ Ya en el jardín del Edén, después del pecado original,\\ Dios Padre, fiel y misericordioso,\\ había preanunciado un Salvador (cf. \emph{Gn} 3, 15),\\ que habría de restablecer la alianza destruida,\\ instaurando una nueva relación\\ de amistad, de entendimiento y de paz.

2. Este gozoso anuncio, confiado a los hijos de Abraham,\\ desde la época de la salida de Egipto (cf. \emph{Ex} 3, 6-8)\\ ha resonado a lo largo de los siglos\\ como un grito de esperanza en boca de los profetas de Israel,\\ que en diversos momentos han recordado al pueblo:\\ \emph{\textquote{Prope est Dominus: venite, adoremus}}.\\ \textquote{El Señor está cerca: ¡venid a adorarlo!}\\ Venid a adorar al Dios que no abandona\\ a quienes lo buscan con sincero corazón\\ y se esfuerzan en observar su ley.\\ Acoged su mensaje,\\ que conforta los espíritus abatidos y desorientados.\\ \emph{Prope est Dominus}: fiel a la antigua promesa,\\ Dios Padre la cumple ahora en el misterio de la Navidad.

3. Sí, su promesa, que ha alimentado\\ la espera confiada de tantos creyentes\\ se ha hecho don en Belén, en el corazón de la Noche Santa.\\ Nos lo ha recordado ayer la liturgia de la Misa:\\ \emph{\textquote{Hodie scietis quia veniet Dominus,}\\ et mane videbitis gloriam eius}.\\ \textquote{Hoy sabréis que el Señor viene:\\ con el nuevo día veréis su gloria}.\\ Esta noche hemos visto la gloria de Dios,\\ proclamada por el canto gozoso de los ángeles;\\ hemos adorado al Rey, Señor del universo,\\ junto con los pastores que vigilaban sus rebaños.\\ Con los ojos de la fe, también nosotros hemos visto,\\ recostado en un pesebre,\\ al Príncipe de la Paz,\\ y junto a Él, María y José\\ en silenciosa adoración.

4. A la multitud de los ángeles, a los pastores extasiados,\\ nos unimos también nosotros hoy, cantando con júbilo:\\ \textquote{\emph{Chistus natus est nobis: venite, adoremus}}.\\ \textquote{Cristo ha nacido por nosotros: venid, adorémosle}\\ Desde la noche de Belén hasta hoy,\\ la Navidad continúa suscitando himnos de alegría,\\ que expresan la ternura de Dios\\ sembrada en el corazón de los hombres.\\ En todas las lenguas del mundo\\ se celebra el acontecimiento más grande y más humilde:\\ el Emmanuel, Dios con nosotros para siempre.

¡Cuántos cantos sugestivos ha inspirado la Navidad\\ en los pueblos y culturas!\\ ¿Quién desconoce las emociones que evocan?\\ Sus melodías hacen a revivir\\ el misterio de la Noche Santa;\\ atestiguan el encuentro entre el Evangelio y los caminos de los hombres.\\ Sí, la Navidad ha entrado en el corazón de los pueblos,\\ que miran hacia Belén con una admiración común.\\ También la Asamblea General de las Naciones Unidas\\ ha reconocido por unanimidad la pequeña población de Judá (cf. \emph{Mt} 2, 6)\\ como la tierra en la que la celebración del nacimiento de Jesús\\ ofrecerá en el 2000 una ocasión singular\\ para proyectos de esperanza y de paz.

5. ¿Cómo no percibir el clamoroso contraste\\ entre la serenidad de los cantos navideños\\ y los muchos problemas del nuestro momento actual?\\ Conocemos los aspectos preocupantes por las crónicas\\ que aparecen cada día en la televisión y los periódicos\\ pasando de un hemisferio del globo al otro:\\ son situaciones tristísimas, a las que frecuentemente\\ no es ajena la culpa e incluso la malicia humana,\\ impregnada de odio fratricida y de violencia absurda.\\ La luz que viene de Belén\\ nos salve del peligro de resignarnos\\ a un panorama tan desconcertante y atormentado.

Que el anuncio de la Navidad aliente\\ a cuantos se esfuerzan por aliviar\\ la situación penosa del Medio Oriente\\ respetando los compromisos internacionales.\\ Que la Navidad refuerce en el mundo\\ el consenso sobre medidas urgentes y adecuadas\\ para detener la producción y el comercio de armas,\\ para defender la vida humana, para desterrar la pena de muerte,\\ para liberar a los niños y adolescentes de toda forma de explotación,\\ para frenar la mano ensangrentada\\ de los responsables de genocidios y crímenes de guerra,\\ para prestar a las cuestiones del medio ambiente,\\ sobre todo tras las recientes catástrofes naturales,\\ la atención indispensable que merecen\\ a fin de salvaguardar la creación y la dignidad del hombre.

6. La alegría de la Navidad, que canta el nacimiento del Salvador,\\ infunda a todos confianza en la fuerza de la verdad\\ y de la perseverancia paciente en hacer el bien.\\ El mensaje divino de Belén resuena para cada uno de nosotros:\\ \textquote{No temáis, pues os anuncio una gran alegría,\ldots{}\\ os ha nacido hoy, en la ciudad de David,\\ un Salvador, que es el Cristo Señor} (\emph{Lc} 2, 10-11).

Hoy resplandece \emph{Urbi et Orbi},\\ en la ciudad de Roma y en el mundo entero,\\ el rostro de Dios; Jesús nos lo revela\\ como Padre que nos ama.\\ Vosotros que buscáis el sentido de la vida;\\ vosotros que lleváis en el corazón la llama\\ de una esperanza de salvación, de libertad y de paz,\\ venid a ver al Niño que ha nacido de María.\\ Él es Dios, nuestro Salvador,\\ el único digno de tal nombre,\\ el único Señor.\\ Ha nacido por nosotros, venid, ¡adorémosle!

\subsubsection{AG } pandoc -s -r html http://www.vatican.va/content/john-paul-ii/es/messages/urbi/documents/hf_jp-ii_mes_25121997_urbi.html -o aaa.tex

\subsubsection{UO 1997 ***PARA AÑO C*** } \emph{\textbf{MENSAJE URBI ET ORBI\\ DEL SANTO PADRE JUAN PABLO II}}

(\emph{Navidad, 25 de diciembre de 1997})

~

1. \textquote{La tierra ha visto a su Salvador}\\ Hoy, Navidad del Señor, vivimos profundamente\\ la verdad de estas palabras: la tierra ha visto a su Salvador.\\ Lo han visto en primer lugar los pastores de Belén\\ que, al anuncio de los ángeles,\\ se apresuraron con alegría hacia la pobre gruta.\\ Era de noche, noche llena de misterio.\\ ¿Qué vieron sus ojos?\\ Un Niño acostado en un pesebre,\\ con María y José solícitos a su lado.\\ Vieron un niño pero, iluminados por la fe,\\ en aquella frágil criatura reconocieron a Dios hecho hombre,\\ y le ofrecieron sus pobres dones.\\ Iniciaron así, sin darse cuente,\\ aquel canto de alabanza al Emmanuel,\\ Dios venido a habitar entre nosotros,\\ que se extendería de generación en generación.\\ Cántico alegre, que es patrimonio de cuantos, hoy,\\ se dirigen espiritualmente a Belén,\\ para celebrar el nacimiento del Señor,\\ y alaban a Dios por las maravillas que ha realizado.\\ También nosotros nos unimos con fe\\ a este singular encuentro de alabanza\\ que, según la tradición, se renueva cada año en Navidad,\\ aquí, en la Plaza San Pedro, y que concluye con la bendición\\ que el Obispo de Roma imparte \emph{Urbi et Orbi}:

\emph{Urbi}, es decir, a esta Ciudad que, gracias al ministerio\\ de los santos Pedro y Pablo,\\ ha \textquote{visto} de manera singular\\ al Salvador del mundo.

\emph{Et Orbi}, es decir, al mundo entero,\\ en el que se ha difundido ampliamente\\ la Buena Nueva de la salvación,\\ que ha llegado ya hasta los confines extremos de la tierra.\\ La alegría de Navidad ha llegado a ser así\\ patrimonio de innumerables pueblos y naciones.\\ En verdad, \textquote{los confines de la tierra\\ han contemplado la victoria de nuestro Dios} (Sal 97/98,3)

2. A todos, pues, va dirigido el mensaje de la solemnidad de hoy.\\ Todos están llamados a participar\\ de la alegría de la Navidad.\\ \textquote{Aclama al Señor, tierra entera,\\ gritad, vitoread, tocad} (Sal 97/98,4).\\ Día de extraordinaria alegría es la Navidad.\\ Esta alegría ha inundado los corazones humanos\\ y ha tenido múltiples expresiones\\ en la historia y en la cultura de las naciones cristianas:\\ en el canto litúrgico y popular, en la pintura,\\ en la literatura y en cada campo del arte.\\ Para la formación cristiana de generaciones enteras,\\ tienen gran importancia las tradiciones y los cantos,\\ las representaciones sacras y, entre todas, el portal.\\ El cántico de los ángeles en Belén\\ ha encontrado así un eco universal y multiforme\\ en las costumbres, mentalidades y culturas de cada tiempo.\\ Ha encontrado un eco en el corazón de cada creyente.

3. Hoy, día de alegría para todos,\\ día lleno de tantos llamamientos a la paz y la fraternidad,\\ se hacen más intensos e incisivos el clamor y la súplica\\ de los pueblos que anhelan la libertad y la concordia,\\ en situaciones de preocupante violencia étnica y política.\\ Hoy resuena más fuerte la voz\\ de quienes están comprometidos generosamente\\ en derribar barreras de miedo y de agresividad,\\ para promover la comprensión entre hombres\\ de distinto origen, raza y credo religioso.\\ Hoy día nos resultan más dramáticos los sufrimientos\\ de gente que huye a las montañas de su propia tierra\\ o busca atracar a las costas de los Países vecinos,\\ para perseguir la esperanza incluso leve\\ de una vida menos precaria y más segura.\\ Más angustioso es hoy el silencio, lleno de tensiones,\\ de la multitud, cada vez mayor, de nuevos pobres:\\ hombres y mujeres sin trabajo y sin casa,\\ muchachos y niños ofendidos y profanados,\\ adolescentes enrolados en las guerras de los adultos,\\ víctimas jóvenes de la droga\\ o atraídos por mitos falaces.\\ Hoy es Navidad, día de confianza para pueblos por largo tiempo divididos,\\ que finalmente se han vuelto a encontrar y tratar.\\ Son perspectivas a menudo tímidas y frágiles,\\ diálogos lentos y arduos,\\ pero animados por la esperanza\\ de alcanzar finalmente acuerdos\\ respetuosos de los derechos y de los deberes de todos.

4. ¡Es Navidad! Esta humanidad nuestra descarriada,\\ en camino hacia el tercer milenio,\\ te espera, Niño de Belén,\\ que vienes a manifestar el amor del Padre.\\ Tú, Rey de la paz, nos invitas hoy a no tener miedo\\ y abrir nuestros corazones a perspectivas de esperanza.\\ Por esto \textquote{cantemos al Señor un cántico nuevo,\\ porque ha hecho maravillas} (cf. Sal 97/98,1).\\ Este es el mayor prodigio obrado por Dios:\\ El mismo se hizo hombre y nació en la noche de Belén,\\ ofreció por nosotros su vida en la Cruz,\\ resucitó al tercer día según las Escrituras\\ y a través de la Eucaristía permanece con nosotros\\ hasta el fin de los tiempos.\\ En verdad \textquote{\ldots{} la palabra se hizo carne,\\ y acampó entre nosotros} (Jn 1,14).\\ La luz de la fe nos ayuda a reconocer\\ en el Niño recién nacido\\ al Dios eterno e inmortal.\\ Somos testigos de su gloria.\\ De omnipotente como era,\\ se revistió de extrema pobreza.\\ Esta es nuestra fe, la fe de la Iglesia,\\ que nos permite confesar la gloria del Hijo unigénito de Dios,\\ aunque nuestros ojos no vean más que al hombre,\\ un Niño nacido en la gruta de Belén.

Dios hecho hombre yace hoy en el pesebre\\ y el universo lo contempla silenciosamente.\\ ¡Que la humanidad pueda reconocerlo como a su Salvador!

\subsubsection{Adviento en General}

CELEBRACIÓN DE LAS PRIMERAS VÍSPERAS DEL I DOMINGO DE ADVIENTO

\emph{\textbf{HOMILÍA DE SU SANTIDAD BENEDICTO XVI}\\[2\baselineskip]Basílica de San Pedro\\ Domingo 1 de diciembre de 2007}

\emph{Queridos hermanos y hermanas:}

El Adviento es, por excelencia, el tiempo de la esperanza. Cada año, esta actitud fundamental del espíritu se renueva en el corazón de los cristianos que, mientras se preparan para celebrar la gran fiesta del nacimiento de Cristo Salvador, reavivan la esperanza de su vuelta gloriosa al final de los tiempos. La primera parte del Adviento insiste precisamente en la \emph{parusía}, la última venida del Señor. Las antífonas de estas primeras Vísperas, con diversos matices, están orientadas hacia esa perspectiva. La lectura breve, tomada de la primera carta de san Pablo a los Tesalonicenses (\emph{1 Ts} 5, 23-24) hace referencia explícita a la venida final de Cristo, usando precisamente el término griego \emph{parusía} (v. 23). El Apóstol exhorta a los cristianos a ser irreprensibles, pero sobre todo los anima a confiar en Dios, que es \textquote{fiel} (v. 24) y no dejará de realizar la santificación en quienes correspondan a su gracia.

Toda esta liturgia vespertina invita a la esperanza, indicando en el horizonte de la historia la luz del Salvador que viene: \textquote{Aquel día brillará una gran luz} (segunda antífona); \textquote{vendrá el Señor con toda su gloria} (tercera antífona); \textquote{su resplandor ilumina toda la tierra} (antífona del Magníficat). Esta luz, que proviene del futuro de Dios, ya se ha manifestado en la plenitud de los tiempos. Por eso nuestra esperanza no carece de fundamento, sino que se apoya en un acontecimiento que se sitúa en la historia y, al mismo tiempo, supera la historia: el acontecimiento constituido por Jesús de Nazaret. El evangelista san Juan aplica a Jesús el título de \textquote{luz}: es un título que pertenece a Dios. En efecto, en el Credo profesamos que Jesucristo es \textquote{Dios de Dios, Luz de Luz}.

Al tema de la esperanza he dedicado mi segunda encíclica, publicada ayer. Me alegra entregarla idealmente a toda la Iglesia en este primer domingo de Adviento a fin de que, durante la preparación para la santa Navidad, tanto las comunidades como los fieles individualmente puedan leerla y meditarla, de modo que redescubran \emph{la belleza y la profundidad de la esperanza cristiana}. En efecto, la esperanza cristiana está inseparablemente unida al conocimiento del rostro de Dios, el rostro que Jesús, el Hijo unigénito, nos reveló con su encarnación, con su vida terrena y su predicación, y sobre todo con su muerte y resurrección.

La esperanza verdadera y segura está fundamentada en la fe en Dios Amor, Padre misericordioso, que \textquote{tanto amó al mundo que le dio a su Hijo unigénito} (\emph{Jn} 3, 16), para que los hombres, y con ellos todas las criaturas, puedan tener vida en abundancia (cf. \emph{Jn} 10, 10). Por tanto, el Adviento es tiempo favorable para redescubrir una esperanza no vaga e ilusoria, sino cierta y fiable, por estar \textquote{anclada} en Cristo, Dios hecho hombre, roca de nuestra salvación.

Como se puede apreciar en el Nuevo Testamento y en especial en las cartas de los Apóstoles, desde el inicio una nueva esperanza distinguió a los cristianos de las personas que vivían la religiosidad pagana. San Pablo, en su carta a los Efesios, les recuerda que, antes de abrazar la fe en Cristo, estaban \textquote{sin esperanza y sin Dios en este mundo} (\emph{Ef} 2, 12). Esta expresión resulta sumamente actual para el paganismo de nuestros días: podemos referirla en particular al nihilismo contemporáneo, que corroe la esperanza en el corazón del hombre, induciéndolo a pensar que dentro de él y en torno a él reina la nada: nada antes del nacimiento y nada después de la muerte.

En realidad, si falta Dios, falla la esperanza. Todo pierde sentido. Es como si faltara la dimensión de profundidad y todas las cosas se oscurecieran, privadas de su valor simbólico; como si no \textquote{destacaran} de la mera materialidad. Está en juego la relación entre la existencia aquí y ahora y lo que llamamos el \textquote{más allá}. El más allá no es un lugar donde acabaremos después de la muerte, sino la realidad de Dios, la plenitud de vida a la que todo ser humano, por decirlo así, tiende. A esta espera del hombre Dios ha respondido en Cristo con el don de la esperanza.

El hombre es la única criatura libre de decir sí o no a la eternidad, o sea, a Dios. El ser humano puede apagar en sí mismo la esperanza eliminando a Dios de su vida. ¿Cómo puede suceder esto? ¿Cómo puede acontecer que la criatura \textquote{hecha para Dios}, íntimamente orientada a él, la más cercana al Eterno, pueda privarse de esta riqueza?

Dios conoce el corazón del hombre. Sabe que quien lo rechaza no ha conocido su verdadero rostro; por eso no cesa de llamar a nuestra puerta, como humilde peregrino en busca de acogida. El Señor concede un nuevo tiempo a la humanidad precisamente para que todos puedan llegar a conocerlo. Este es también \emph{el sentido de un nuevo año litúrgico que comienza}: es un don de Dios, el cual quiere revelarse de nuevo en el misterio de Cristo, mediante la Palabra y los sacramentos.

Mediante la Iglesia quiere hablar a la humanidad y salvar a los hombres de hoy. Y lo hace saliendo a su encuentro, para \textquote{buscar y salvar lo que estaba perdido} (\emph{Lc} 19, 10). Desde esta perspectiva, la celebración del Adviento es la respuesta de la Iglesia Esposa a la iniciativa continua de Dios Esposo, \textquote{que es, que era y que viene} (\emph{Ap} 1, 8). A la humanidad, que ya no tiene tiempo para él, Dios le ofrece otro tiempo, un nuevo espacio para volver a entrar en sí misma, para ponerse de nuevo en camino, para volver a encontrar el sentido de la esperanza.

He aquí el descubrimiento sorprendente: mi esperanza, nuestra esperanza, está precedida por la espera que Dios cultiva con respecto a nosotros. Sí, Dios nos ama y precisamente por eso espera que volvamos a él, que abramos nuestro corazón a su amor, que pongamos nuestra mano en la suya y recordemos que somos sus hijos.

Esta espera de Dios precede siempre a nuestra esperanza, exactamente como su amor nos abraza siempre primero (cf. \emph{1 Jn} 4, 10). En este sentido, la esperanza cristiana se llama \textquote{teologal}: Dios es su fuente, su apoyo y su término. ¡Qué gran consuelo nos da este misterio! Mi Creador ha puesto en mi espíritu un reflejo de su deseo de vida para todos. Cada hombre está llamado a esperar correspondiendo a lo que Dios espera de él. Por lo demás, la experiencia nos demuestra que eso es precisamente así. ¿Qué es lo que impulsa al mundo sino la confianza que Dios tiene en el hombre? Es una confianza que se refleja en el corazón de los pequeños, de los humildes, cuando a través de las dificultades y las pruebas se esfuerzan cada día por obrar de la mejor forma posible, por realizar un bien que parece pequeño, pero que a los ojos de Dios es muy grande: en la familia, en el lugar de trabajo, en la escuela, en los diversos ámbitos de la sociedad. La esperanza está indeleblemente escrita en el corazón del hombre, porque Dios nuestro Padre es vida, y estamos hechos para la vida eterna y bienaventurada.

Todo niño que nace es signo de la confianza de Dios en el hombre y es una confirmación, al menos implícita, de la esperanza que el hombre alberga en un futuro abierto a la eternidad de Dios. A esta esperanza del hombre respondió Dios naciendo en el tiempo como un ser humano pequeño. San Agustín escribió: \textquote{De no haberse tu Verbo hecho carne y habitado entre nosotros, hubiéramos podido juzgarlo apartado de la naturaleza humana y desesperar de nosotros} (\emph{Confesiones} X, 43, 69, citado en \emph{Spe salvi}, 29).

Dejémonos guiar ahora por Aquella que llevó en su corazón y en su seno al Verbo encarnado. ¡Oh María, Virgen de la espera y Madre de la esperanza, reaviva en toda la Iglesia el espíritu del Adviento, para que la humanidad entera se vuelva a poner en camino hacia Belén, donde vino y de nuevo vendrá a visitarnos el Sol que nace de lo alto (cf. \emph{Lc} 1, 78), Cristo nuestro Dios! Amén.

\subsubsection{Alter Navidad} \emph{Plaza de San Pedro}\\ \emph{Miércoles 18 de diciembre de 2013}


~

\emph{Queridos hermanos y hermanas, ¡buenos días!}

Este encuentro nuestro tiene lugar en el clima espiritual del Adviento, que se hace más intenso por la Novena de la Santa Navidad, que estamos viviendo en estos días y que nos conduce a las fiestas navideñas. Por ello, hoy desearía reflexionar con vosotros sobre el Nacimiento de Jesús, fiesta de la confianza y de la esperanza, que supera la incertidumbre y el pesimismo. Y la razón de nuestra esperanza es ésta: Dios está con nosotros y Dios se fía aún de nosotros. Pero pensad bien en esto: Dios está con nosotros y Dios se fía aún de nosotros. Es generoso este Dios Padre. Él viene a habitar con los hombres, elige la tierra como morada suya para estar junto al hombre y hacerse encontrar allí donde el hombre pasa sus días en la alegría y en el dolor. Por lo tanto, la tierra ya no es sólo un \textquote{valle de lágrimas}, sino el lugar donde Dios mismo puso su tienda, es el lugar del encuentro de Dios con el hombre, de la solidaridad de Dios con los hombres.

Dios quiso compartir nuestra condición humana hasta el punto de hacerse una cosa sola con nosotros en la persona de Jesús, que es verdadero hombre y verdadero Dios. Pero hay algo aún más sorprendente. La presencia de Dios en medio de la humanidad no se realiza en un mundo ideal, idílico, sino en este mundo real, marcado por muchas cosas buenas y malas, marcado por divisiones, maldad, pobreza, prepotencias y guerras. Él eligió habitar nuestra historia así como es, con todo el peso de sus límites y de sus dramas. Actuando así demostró de modo insuperable su inclinación misericordiosa y llena de amor hacia las creaturas humanas. Él es el Dios-con-nosotros; Jesús es Dios-con-nosotros. ¿Creéis vosotros esto? Hagamos juntos esta profesión: Jesús es Dios-con-nosotros. Jesús es Dios-con-nosotros desde siempre y para siempre con nosotros en los sufrimientos y en los dolores de la historia. El nacimiento de Jesús es la manifestación de que Dios \textquote{tomó partido} de una vez para siempre de la parte del hombre, para salvarnos, para levantarnos del polvo de nuestras miserias, de nuestras dificultades, de nuestros pecados.

De aquí viene el gran \textquote{regalo} del Niño de Belén: Él nos trae una energía espiritual, una energía que nos ayuda a no hundirnos en nuestras fatigas, en nuestras desesperaciones, en nuestras tristezas, porque es una energía que caldea y transforma el corazón. El nacimiento de Jesús, en efecto, nos trae la buena noticia de que somos amados inmensamente y singularmente por Dios, y este amor no sólo nos lo da a conocer, sino que nos lo dona, nos lo comunica.

De la contemplación gozosa del misterio del Hijo de Dios nacido por nosotros, podemos sacar dos consideraciones.

La primera es que si en Navidad Dios se revela no como uno que está en lo alto y que domina el universo, sino como Aquél que se abaja, desciende sobre la tierra pequeño y pobre, significa que para ser semejantes a Él no debemos ponernos sobre los demás, sino, es más, abajarnos, ponernos al servicio, hacernos pequeños con los pequeños y pobres con los pobres. Pero es algo feo cuando se ve a un cristiano que no quiere abajarse, que no quiere servir. Un cristiano que se da de importante por todos lados, es feo: ese no es cristiano, ese es pagano. El cristiano sirve, se abaja. Obremos de manera que estos hermanos y hermanas nuestros no se sientan nunca solos.

La segunda consecuencia: si Dios, por medio de Jesús, se implicó con el hombre hasta el punto de hacerse como uno de nosotros, quiere decir que cualquier cosa que hagamos a un hermano o a una hermana la habremos hecho a Él. Nos lo recordó Jesús mismo: quien haya alimentado, acogido, visitado, amado a uno de los más pequeños y de los más pobres entre los hombres, lo habrá hecho al Hijo de Dios.

Encomendémonos a la maternal intercesión de María, Madre de Jesús y nuestra, para que nos ayude en esta Santa Navidad, ya cercana, a reconocer en el rostro de nuestro prójimo, especialmente de las personas más débiles y marginadas, la imagen del Hijo de Dios hecho hombre.

\subsubsection{Navidad, pesebre} \emph{\textbf{AUDIENCIA GENERAL}}

\emph{Aula Pablo VI\\ Miércoles, 18 de diciembre de 2019}


\begin{center}\rule{0.5\linewidth}{\linethickness}\end{center}

~

\emph{Queridos hermanos y hermanas, ¡buenos días!}

Dentro de una semana será Navidad. En estos días, mientras corremos para hacer los preparativos de la fiesta, podemos preguntarnos: \textquote{¿Cómo me preparo para el nacimiento del festejado?} Un modo sencillo pero eficaz de prepararse es \emph{hacer el belén} Este año yo también he seguido este camino: fui a Greccio, donde San Francisco hizo el primer belén, con los lugareños. Y escribí una carta para recordar el significado de esta tradición, lo que significa el belén en el tiempo de Navidad.

En efecto, el pesebre \textquote{es como un Evangelio vivo} (Carta apostólica \emph{Admirabile signum}, 1). Lleva el Evangelio a los lugares donde uno vive: a las casas, a las escuelas, a los lugares de trabajo y de reunión, a los hospitales y a las residencias de ancianos, a las cárceles y a las plazas. Y allí donde vivimos nos recuerda algo esencial: que Dios no permaneció invisible en el cielo, sino que vino a la Tierra, se hizo hombre, un niño. Hacer el pesebre es \emph{celebrar la cercanía de Dios}. Dios siempre estuvo cerca de su pueblo, pero cuando se encarnó y nació, estuvo muy cerca, muy cerca. Hacer el belén es celebrar la cercanía de Dios, es redescubrir que Dios es real, concreto, vivo y palpitante. Dios no es un señor lejano ni un juez distante, sino Amor humilde, descendido hasta nosotros. El Niño en el pesebre nos transmite su ternura. Algunas figuritas representan al Niño con los brazos abiertos, para decirnos que Dios vino a abrazar nuestra humanidad. Entonces es bonito estar delante del pesebre y allí confiar nuestras vidas al Señor, hablarle de las personas y situaciones que nos importan, hacer con Él un balance del año que está llegando a su fin, compartir nuestras expectativas y preocupaciones.

Junto a Jesús vemos a la Virgen y a san José. Podemos imaginar los pensamientos y sentimientos que tuvieron cuando el Niño nació en la pobreza: alegría, pero también consternación. Y también podemos invitar a la Sagrada Familia a nuestra casa, donde hay alegrías y preocupaciones, donde cada día nos levantamos, comemos y dormimos cerca de nuestros seres queridos. El pesebre es un \emph{Evangelio doméstico}. La palabra pesebre significa literalmente \textquote{comedero}, mientras que la ciudad del pesebre, Belén, significa \textquote{casa del pan}. Pesebre y casa del pan: el belén que hacemos en casa, donde compartimos comida y afecto, nos recuerda que Jesús es el alimento, el pan de vida (cf. \emph{Jn} 6,34). Es Él quien alimenta nuestro amor, es Él quien da a nuestras familias la fuerza para seguir adelante y perdonarnos.

El pesebre nos ofrece otra enseñanza de vida. En los ritmos de hoy, a veces frenéticos, \emph{es una invitación a la contemplación}. Nos recuerda la importancia de detenernos. Porque sólo cuando sabemos recogernos podemos acoger lo que cuenta en la vida. Sólo si dejamos el estruendo del mundo fuera de nuestras casas nos abrimos a escuchar a Dios, que habla en silencio. El pesebre es actual, es la actualidad de cada familia. Ayer me regalaron una figura de un belén especial, una pequeña, llamada: \textquote{Dejemos descansar a mamá}. Representaba a la Virgen dormida y a José que hacía que el Niño se durmiera. Cuántos de vosotros tienen que repartir la noche entre marido y mujer por el niño o la niña que llora, llora, llora, llora. \textquote{Dejemos que mamá descanse} es la ternura de una familia, de un matrimonio.

El pesebre es más actual que nunca, cuando cada día se fabrican en el mundo tantas armas y tantas imágenes violentas que entran por los ojos y el corazón. El pesebre es, en cambio, una \emph{imagen artesanal de la paz}. Por eso es un Evangelio vivo.

Queridos hermanos y hermanas, del pesebre podemos sacar también una enseñanza sobre el sentido mismo de la vida. Vemos escenas cotidianas: los pastores con las ovejas, los herreros que baten el yunque, los molineros que hacen pan; a veces se insertan paisajes y situaciones de nuestros territorios. Está bien, porque el pesebre nos recuerda que Jesús viene a nuestra vida concreta. Y esto es importante. Hacer un pequeño belén, en casa, siempre, porque es el recuerdo de Dios que vino entre nosotros, nació entre nosotros, nos acompaña en la vida, es hombre como nosotros, se hizo hombre como nosotros. En la vida diaria ya no estamos solos, Él vive con nosotros. No cambia mágicamente las cosas pero, si lo acogemos, todo puede cambiar. Os deseo, entonces, que hacer el pesebre sea la ocasión de invitar a Jesús a la vida. Cuando hacemos el belén en casa, es como si abriéramos la puerta y dijéramos: \textquote{Jesús, ¡entra!}, es hacer concreta esta cercanía, esta invitación a Jesús para que venga a nuestra vida. Porque si Él habita nuestra vida, la vida renace. Y si la vida renace es de verdad Navidad. ¡Feliz Navidad a todos!

\section{I Adviento}

\subsubsection{Homilía (1998)} I DOMINGO DE ADVIENTO - CONVOCACIÓN DEL AÑO SANTO

\emph{\textbf{HOMILÍA DEL SANTO PADRE JUAN PABLO II\\[2\baselineskip]\hspace*{0.333em}}Basílica de San Pedro\\ Domingo 29 de noviembre de 1998}

~

1.~\textquote{Vayamos jubilosos al encuentro del Señor} (\emph{Estribillo del Salmo responsorial}).

Son las palabras del Salmo responsorial de esta liturgia del primer domingo de Adviento, tiempo litúrgico que renueva año tras año la espera de la venida de Cristo. En estos años que estamos viviendo en la perspectiva del tercer milenio, el Adviento ha cobrado una dimensión nueva y singular. \emph{Tertio millennio adveniente}: el año 1998, que está a punto de terminar, y el año próximo 1999 nos acercan al umbral de un nuevo siglo y de un nuevo milenio.

\textquote{En el umbral} ha comenzado también esta celebración: en el umbral de la basílica vaticana, ante la puerta santa, con la entrega y la lectura de la \emph{bula de convocación} del gran jubileo del año 2000.

\textquote{Vayamos jubilosos al encuentro del Señor} es un estribillo que está perfectamente en armonía con el jubileo. Es, por decir así, un \textquote{estribillo jubilar}, según la etimología de la palabra latina \emph{iubilar}, que encierra una referencia al júbilo. ¡Vayamos, pues, con alegría! Caminemos jubilosos y vigilantes a la espera del tiempo que recuerda la venida de Dios en la carne humana, tiempo que llegó a su plenitud cuando en la cueva de Belén nació Cristo. Entonces se cumplió el tiempo de la espera.

Viviendo el Adviento, esperamos un acontecimiento que se sitúa en la historia y a la vez la trasciende. Al igual que los demás años, tendrá lugar en la noche de la Navidad del Señor. A la cueva de Belén acudirán los pastores; más tarde, irán los Magos de Oriente. Unos y otros simbolizan, en cierto sentido, a toda la familia humana. La exhortación que resuena en la liturgia de hoy: \textquote{Vayamos jubilosos al encuentro del Señor} se difunde en todos los países, en todos los continentes, en todos los pueblos y naciones. La voz de la liturgia, es decir, la voz de la Iglesia, resuena por doquier e invita a todos al gran jubileo.

2.~Estos últimos tres años que preceden al 2000 forman un tiempo de espera muy intenso, orientado a la meditación sobre el significado del inminente evento espiritual y sobre su necesaria preparación. El contenido de esa preparación sigue el modelo trinitario, que se repite al final de toda plegaria litúrgica. Así pues, vayamos jubilosos \emph{hacia el Padre}, \emph{por el camino que es nuestro Señor Jesucristo}, el cual vive y reina con él \emph{en la unidad del Espíritu Santo.}

Por eso, el primer año lo dedicamos al Hijo; el segundo, al Espíritu Santo; y el que comienza hoy ---el último antes del gran jubileo--- será \emph{el año del Padre}. Invitados por el Padre, vayamos a él mediante el Hijo, en el Espíritu Santo. Este trienio de preparación inmediata para el nuevo milenio, por su carácter trinitario, no sólo nos habla de Dios en sí mismo, como misterio inefable de vida y santidad, sino también de \emph{Dios que viene a nuestro encuentro.}

3.~Por este motivo, el estribillo \textquote{Vayamos jubilosos \emph{al encuentro del Señor}} resulta tan adecuado. Nosotros podemos encontrar a Dios, porque él ha venido a nuestro encuentro. Lo ha hecho, como el padre de la parábola del hijo pródigo (cf. \emph{Lc} 15, 11-32), porque es rico en misericordia, \emph{dives in misericordia}, y quiere salir a nuestro encuentro sin importarle de qué parte venimos o a dónde lleva nuestro camino. Dios viene a nuestro encuentro, tanto si lo hemos buscado como si lo hemos ignorado, e incluso si lo hemos evitado. Él sale el primero a nuestro encuentro, con los brazos abiertos, como un padre amoroso y misericordioso.

Si Dios se pone en movimiento para salir a nuestro encuentro, ¿podremos nosotros volverle la espalda? Pero no podemos ir solos al encuentro con el Padre. Debemos ir en compañía de cuantos forman parte de \textquote{la familia de Dios}. Para prepararnos convenientemente al jubileo debemos disponernos a acoger a todas las personas. Todos son nuestros hermanos y hermanas, porque son hijos del mismo Padre celestial.

En esta perspectiva, podemos leer la bimilenaria historia de la Iglesia. Es consolador constatar cómo la Iglesia, en este paso del segundo al tercer milenio, está experimentando un nuevo impulso misionero. Lo ponen de manifiesto los Sínodos continentales que se están celebrando estos años, incluido el actual para Australia y Oceanía. Y también lo confirman los informes que llegan al Comit é para el gran jubileo sobre las iniciativas puestas en marcha por las Iglesias locales como preparación para ese histórico acontecimiento.

Quisiera saludar, en particular, al cardenal presidente del comité, al secretario general y a sus colaboradores. Mi saludo se extiende también a los cardenales, a los obispos y a los sacerdotes aquí presentes, así como a todos vosotros, queridos hermanos y hermanas, que participáis en esta solemne liturgia. Saludo en especial al clero, a los religiosos, a las religiosas y a los laicos comprometidos de Roma, que, junto con el cardenal vicario y los obispos auxiliares, están aquí esta mañana para inaugurar la última fase de la misión ciudadana, dirigida a los ambientes de la sociedad.

Es una fase importante, en la que la diócesis realizará una amplia labor de evangelización en todos los ámbitos de vida y de trabajo. Al terminar la santa misa, entregaré a los misioneros la cruz de la misión. Es necesario que Cristo sea anunciado y testimoniado en cada lugar y en cada situación. Invito a todos a sostener con la oración esta gran empresa. En particular, cuento con la aportación de las monjas de clausura, de los enfermos, de las personas ancianas que, a pesar de que les es imposible participar directamente en esta iniciativa apostólica, pueden dar una gran contribución con su oración y con la ofrenda de sus sufrimientos para disponer los corazones a la acogida del anuncio evangélico.

María, que el tiempo de Adviento nos invita a contemplar en espera activa del Redentor, os ayude a todos a ser apóstoles generosos de su Hijo Jesús.

4.~En el evangelio de hoy hemos escuchado la invitación del Señor a la \emph{vigilancia}. \textquote{Velad, porque no sabéis qué día vendrá vuestro Señor}. Y a continuación: \textquote{Estad preparados, porque a la hora que menos penséis vendrá el Hijo del hombre} (\emph{Mt} 24, 42.44). La exhortación a velar resuena muchas veces en la liturgia, especialmente en Adviento, tiempo de preparación no sólo para la Navidad, sino también para \emph{la definitiva y gloriosa venida de Cristo al final de los tiempos}. Por eso, tiene un significado marcadamente escatológico e invita al creyente a pasar cada día, cada momento, en presencia de Aquel \textquote{que es, que era y que vendrá} (\emph{Ap} 1, 4), al que pertenece el futuro del mundo y del hombre. Ésta es la esperanza cristiana. Sin esta perspectiva, nuestra existencia se reduciría a un vivir para la muerte.

Cristo es nuestro Redentor: \emph{Redemptor mundi et Redemptor hominis}, Redentor del mundo y Redentor del hombre. Vino a nosotros para ayudarnos a cruzar el umbral que lleva a la puerta de la vida, la \textquote{puerta santa} que es él mismo.

5.~Que esta consoladora verdad esté siempre muy presente ante nuestros ojos, mientras caminamos como peregrinos hacia el gran jubileo. Esa verdad constituye la razón última de la alegría a la que nos exhorta la liturgia de hoy: \textquote{Vayamos \emph{jubilosos} al encuentro del Señor}. Creyendo en Cristo crucificado y resucitado, creemos en la resurrección de la carne y en la vida eterna.

\emph{Tertio millennio adveniente}. En esta perspectiva, los años, los siglos y los milenios cobran el sentido definitivo de la existencia que el jubileo del año 2000 quiere manifestarnos.

Contemplando a Cristo, hagamos nuestras las palabras de un antiguo canto popular polaco:

\begin{quote} \textquote{La salvación ha venido por la cruz;\\ éste es un gran misterio.\\ Todo sufrimiento tiene un sentido:\\ lleva a la plenitud de la vida}. \end{quote}

Con esta fe en el corazón, que es la fe de la Iglesia, inauguro hoy, como Obispo de Roma, el tercer año de preparación para el gran jubileo. Lo inauguro en el nombre del Padre celestial, que \textquote{tanto amó (\ldots{}) al mundo que le dio su Hijo único, para que quien cree en él (\ldots{}) tenga la vida eterna} (\emph{Jn} 3, 16).

¡Alabado sea Jesucristo!

\subsubsection{Ángelus (2001)} \textbf{\emph{ÁNGELUS\\[2\baselineskip]}}\emph{Domingo 2 de diciembre de 2001}

~

\emph{Amadísimos hermanos y hermanas:}

1. Con este primer domingo de Adviento comienza un nuevo Año litúrgico. La Iglesia reanuda su camino y nos invita a reflexionar más intensamente en el misterio de Cristo, misterio siempre nuevo que el tiempo no puede agotar. Cristo es el alfa y la omega, el principio y el fin. Gracias a él, la historia de la humanidad avanza como una peregrinación hacia la plenitud del Reino, que él mismo inauguró con su encarnación y su victoria sobre el pecado y la muerte.

Por eso, el Adviento es sinónimo de \emph{esperanza}:~ no espera vana de un dios sin rostro, sino confianza concreta y cierta en la vuelta de Aquel que ya nos ha visitado, del \textquote{Esposo} que con su sangre ha sellado con la humanidad un pacto de alianza eterna. Es una esperanza que estimula a la \emph{vigilancia}, virtud característica de este singular tiempo litúrgico. Vigilancia en la \emph{oración}, animada por una amorosa espera; vigilancia en el dinamismo de la \emph{caridad concreta}, consciente de que el reino de Dios se acerca donde los hombres aprenden a vivir como hermanos.

2. Con estos sentimientos, la comunidad cristiana entra en el Adviento, manteniendo vigilante su espíritu, para acoger mejor el mensaje de la palabra de Dios. Resuena hoy en la liturgia el célebre y estupendo \emph{oráculo del profeta Isaías}, pronunciado en un momento de crisis de la historia de Israel.

\textquote{Al final de los días ---dice el Señor--- estará firme el monte de la casa del Señor, encumbrado sobre las montañas. Hacia él confluirán los gentiles. (\ldots{}) De las espadas forjarán arados; de las lanzas, podaderas. No alzará la espada pueblo contra pueblo, no se adiestrarán para la guerra} (\emph{Is} 2, 1-5).

Estas palabras contienen una promesa de paz más actual que nunca para la humanidad, y en particular para la Tierra Santa, de donde también hoy, por desgracia, llegan noticias dolorosas y preocupantes. Que las palabras del profeta Isaías inspiren la mente y el corazón de los creyentes y de los hombres de buena voluntad, para que el día de ayuno ---el 14 de diciembre--- y el encuentro de los representantes de las religiones del mundo en Asís ---el 24 de enero del año próximo--- ayuden a crear en el mundo un clima más sereno y solidario.

3. Encomiendo esta invocación de paz a María, Virgen vigilante y Madre de la esperanza. Dentro de algunos días celebraremos con fe renovada la solemnidad de la Inmaculada Concepción. Que ella nos guíe por este camino, ayudando a todo hombre y a toda nación a dirigir la mirada al \textquote{monte del Señor}, imagen del triunfo definitivo de Cristo y de la venida de su reino de paz.

\subsubsection{Ángelus (2007)} \emph{\textbf{ÁNGELUS}\\[2\baselineskip]Plaza de San Pedro\\ I domingo de Adviento, 2 de diciembre de 2007}

\emph{Queridos hermanos y hermanas:}

Con este primer domingo de Adviento comienza un nuevo año litúrgico: el pueblo de Dios vuelve a ponerse en camino para vivir el misterio de Cristo en la historia. Cristo es el mismo ayer, hoy y siempre (cf. \emph{Hb} 13, 8); en cambio, la historia cambia y necesita ser evangelizada constantemente; necesita renovarse desde dentro, y la única verdadera novedad es Cristo: él es su realización plena, el futuro luminoso del hombre y del mundo. Jesús, resucitado de entre los muertos, es el Señor al que Dios someterá todos sus enemigos, incluida la misma muerte (cf. \emph{1 Co} 15, 25-28).

Por tanto, el Adviento es el tiempo propicio para reavivar en nuestro corazón la espera de Aquel \textquote{que es, que era y que va a venir} (\emph{Ap} 1, 8). El Hijo de Dios ya vino en Belén hace veinte siglos, viene en cada momento al alma y a la comunidad dispuestas a recibirlo, y de nuevo vendrá al final de los tiempos para \textquote{juzgar a vivos y muertos}. Por eso, el creyente está siempre vigilante, animado por la íntima esperanza de encontrar al Señor, como dice el Salmo: \textquote{Mi alma espera en el Señor, espera en su palabra; mi alma aguarda al Señor, más que el centinela a la aurora} (\emph{Sal} 130, 5-6).

Por consiguiente, este domingo es un día muy adecuado para ofrecer a la Iglesia entera y a todos los hombres de buena voluntad mi segunda encíclica, que quise dedicar precisamente al tema de la esperanza cristiana. Se titula \emph{Spe salvi}, porque comienza con la expresión de san Pablo: \emph{\textquote{Spe salvi factum sumus}}, \textquote{en esperanza fuimos salvados} (\emph{Rm} 8, 24). En este, como en otros pasajes del Nuevo Testamento, la palabra \textquote{esperanza} está íntimamente relacionada con la palabra \textquote{fe}. Es un don que cambia la vida de quien lo recibe, como lo muestra la experiencia de tantos santos y santas.

¿En qué consiste esta esperanza, tan grande y tan \textquote{fiable} que nos hace decir que \emph{en ella} encontramos la \textquote{salvación}? Esencialmente, consiste en el conocimiento de Dios, en el descubrimiento de su corazón de Padre bueno y misericordioso. Jesús, con su muerte en la cruz y su resurrección, nos reveló su rostro, el rostro de un Dios con un amor tan grande que comunica una esperanza inquebrantable, que ni siquiera la muerte puede destruir, porque la vida de quien se pone en manos de este Padre se abre a la perspectiva de la bienaventuranza eterna.

El desarrollo de la ciencia moderna ha marginado cada vez más la fe y la esperanza en la esfera privada y personal, hasta el punto de que hoy se percibe de modo evidente, y a veces dramático, que el hombre y el mundo necesitan a Dios ---¡al verdadero Dios!---; de lo contrario, no tienen esperanza.

No cabe duda de que la ciencia contribuye en gran medida al bien de la humanidad, pero no es capaz de redimirla. El hombre es redimido por el amor, que hace buena y hermosa la vida personal y social. Por eso la gran esperanza, la esperanza plena y definitiva, es garantizada por Dios que es amor, por Dios que en Jesús nos visitó y nos dio la vida, y en él volverá al final de los tiempos.

En Cristo esperamos; es a él a quien aguardamos. Con María, su Madre, la Iglesia va al encuentro del Esposo: lo hace con las obra de caridad, porque la esperanza, como la fe, se manifiesta en el amor. ¡Buen Adviento a todos!

\subsubsection{Ángelus (2010)} \emph{Plaza de San Pedro\\ I Domingo de Adviento, 28 de noviembre de 2010}


~

\emph{Queridos hermanos y hermanas:}

Hoy, primer domingo de Adviento, la Iglesia inicia un nuevo Año litúrgico, un nuevo camino de fe que, por una parte, conmemora el acontecimiento de Jesucristo, y por otra, se abre a su cumplimiento final. Precisamente de esta doble perspectiva vive el tiempo de Adviento, mirando tanto a la primera venida del Hijo de Dios, cuando nació de la Virgen María, como a su vuelta gloriosa, cuando vendrá a \textquote{juzgar a vivos y muertos}, como decimos en el Credo. Sobre este sugestivo tema de la \textquote{espera} quiero detenerme ahora brevemente, porque se trata de un aspecto profundamente humano, en el que la fe se convierte, por decirlo así, en un todo con nuestra carne y nuestro corazón.

La espera, el esperar, es una dimensión que atraviesa toda nuestra existencia personal, familiar y social. La espera está presente en mil situaciones, desde las más pequeñas y banales hasta las más importantes, que nos implican totalmente y en lo profundo. Pensemos, entre estas, en la espera de un hijo por parte de dos esposos; en la de un pariente o de un amigo que viene a visitarnos de lejos; pensemos, para un joven, en la espera del resultado de un examen decisivo, o de una entrevista de trabajo; en las relaciones afectivas, en la espera del encuentro con la persona amada, de la respuesta a una carta, o de la aceptación de un perdón\ldots{} Se podría decir que el hombre está vivo mientras espera, mientras en su corazón está viva la esperanza. Y al hombre se lo reconoce por sus esperas: nuestra \textquote{estatura} moral y espiritual se puede medir por lo que esperamos, por aquello en lo que esperamos.

Cada uno de nosotros, por tanto, especialmente en este tiempo que nos prepara a la Navidad, puede preguntarse: ¿yo qué espero? En este momento de mi vida, ¿a qué tiende mi corazón? Y esta misma pregunta se puede formular a nivel de familia, de comunidad, de nación. ¿Qué es lo que esperamos juntos? ¿Qué une nuestras aspiraciones?, ¿qué tienen en común? En el tiempo anterior al nacimiento de Jesús, era muy fuerte en Israel la espera del Mesías, es decir, de un Consagrado, descendiente del rey David, que finalmente liberaría al pueblo de toda esclavitud moral y política e instauraría el reino de Dios. Pero nadie habría imaginado nunca que el Mesías pudiese nacer de una joven humilde como era María, prometida del justo José. Ni siquiera ella lo habría pensado nunca, pero en su corazón la espera del Salvador era tan grande, su fe y su esperanza eran tan ardientes, que él pudo encontrar en ella una madre digna. Por lo demás, Dios mismo la había preparado, antes de los siglos. Hay una misteriosa correspondencia entre la espera de Dios y la de María, la criatura \textquote{llena de gracia}, totalmente transparente al designio de amor del Altísimo. Aprendamos de ella, Mujer del Adviento, a vivir los gestos cotidianos con un espíritu nuevo, con el sentimiento de una espera profunda, que sólo la venida de Dios puede colmar.

\subsubsection{Homilía (2013)} VISITA PASTORAL A LA PARROQUIA ROMANA DE SAN CIRILO ALEJANDRINO

\textbf{\emph{HOMILÍA DEL SANTO PADRE FRANCISCO}}

\emph{I Domingo de Adviento, 1 de diciembre de 2013}


~

En la primera lectura, hemos escuchado que el profeta Isaías nos habla de un camino, y dice que al final de los días, al final del camino, el monte del Templo del Señor estará firme en la cima de los montes. Y esto, para decirnos que nuestra vida es un camino: debemos ir por este camino, para llegar al monte del Señor, al encuentro con Jesús. La cosa más importante que le puede suceder a una persona es encontrar a Jesús: este encuentro con Jesús que nos ama, que nos ha salvado, que ha dado su vida por nosotros. Encontrar a Jesús. Y nosotros caminamos para encontrar a Jesús.

Podemos preguntarnos: ¿Cuándo encuentro a Jesús? ¿Sólo al final? ¡No, no! Lo encontramos todos los días. ¿Pero cómo? En la oración, cuando tú rezas, encuentras a Jesús. Cuando recibes la Comunión, encuentras a Jesús, en los Sacramentos. Cuando llevas a bautizar a tu hijo, te encuentras a Jesús, hallas a Jesús. Y vosotros, hoy, que recibís la Confirmación, también vosotros encontraréis a Jesús; luego lo encontraréis en la Comunión. \textquote{Y más tarde, Padre, después de la Confirmación, adiós}, porque dicen que la Confirmación se llama \textquote{el sacramento del ¡adiós!}. ¿Es verdad esto o no? Después de la Confirmación no se va nunca a la iglesia: ¿es verdad o no?\ldots{} ¡Más o menos! Pero también después de la Confirmación, toda la vida, es un encuentro con Jesús: en la oración, cuando vamos a misa y cuando realizamos buenas obras, cuando visitamos a los enfermos, cuando ayudamos a un pobre, cuando pensamos en los demás, cuando no somos egoístas, cuando somos amables\ldots{} en estas cosas encontramos siempre a Jesús. Y el camino de la vida es precisamente este: caminar para encontrar a Jesús.

Hoy, también para mí es una alegría venir a encontrarme con vosotros, porque todos juntos, hoy, en la misa encontraremos a Jesús, y hacemos un tramo del camino juntos.

Recordad siempre esto: la vida es un camino. Es un camino. Un camino para encontrar a Jesús. Al final, y siempre. Un camino donde no encontramos a Jesús, no es un camino cristiano. Es propio del cristiano encontrar siempre a Jesús, mirarle, dejarse mirar por Jesús, porque Jesús nos mira con amor, nos ama mucho, nos quiere mucho y nos mira siempre. Encontrar a Jesús es también dejarte mirar por Él. \textquote{Pero, Padre, tú sabes ---alguno de vosotros podría decirme---, tú sabes que este camino, para mí, es un camino difícil, porque yo soy muy pecador, he cometido muchos pecados\ldots{} ¿cómo puedo encontrar a Jesús?}. Pero tú sabes que las personas a las que Jesús mayormente buscaba eran los más pecadores; y le reñían por esto, y la gente ---las personas que se creían justas--- decía: pero éste, éste no es un verdadero profeta, ¡mira la buena compañía que tiene! Estaba con los pecadores\ldots{} Y Él decía: He venido por quienes tienen necesidad de salud, necesidad de curación, y Jesús cura nuestros pecados. En el camino, nosotros ---todos pecadores, todos, todos somos pecadores--- incluso cuando nos equivocamos, cuando cometemos un pecado, cuando pecamos, Jesús viene y nos perdona. Este perdón que recibimos en la Confesión es un encuentro con Jesús. Siempre encontramos a Jesús.

Y así vamos por la vida, como dice el profeta, al monte, hasta el día que tendrá lugar el encuentro definitivo, cuando contemplemos esa mirada tan bella de Jesús, tan hermosa. Ésta es la vida cristiana: caminar, seguir adelante, unidos como hermanos, queriéndose uno a otro. Encontrar a Jesús. ¿Estáis de acuerdo, vosotros, los nueve? ¿Queréis encontrar a Jesús en vuestra vida? ¿Sí? Esto es importante en la vida cristiana. Vosotros, hoy, con el sello del Espíritu Santo, tendréis más fuerza para este camino, para encontrar a Jesús. ¡Sed valientes, no tengáis miedo! La vida es este camino. Y el regalo más hermoso es encontrar a Jesús. ¡Adelante, ánimo!

Y ahora, sigamos adelante con el Sacramento de la Confirmación.

\section{III Adviento} \subsubsection{Homilía (2007)} VISITA PASTORAL A LA PARROQUIA ROMANA\\ DE SANTA MARÍA DEL ROSARIO EN LOS MÁRTIRES PORTUENSES

\emph{\textbf{HOMILÍA DE SU SANTIDAD BENEDICTO XVI}\\[2\baselineskip]III Domingo de Adviento, 16 de diciembre de 2007}

~

\emph{Queridos hermanos y hermanas:}

\textquote{Estad siempre alegres en el Señor. Os lo repito:~estad alegres. El Señor está cerca} (\emph{Flp} 4, 4-5).\\ Con esta invitación a la alegría comienza la antífona de entrada de la santa misa en este tercer domingo de Adviento, que precisamente por eso se llama domingo \textquote{\emph{Gaudete}\textquote{. En verdad, todo el Adviento es una invitación a alegrarse, porque }el Señor viene}, porque viene a salvarnos.

Durante estas semanas, casi diariamente, nos consuelan las palabras del profeta Isaías, dirigidas al pueblo judío desterrado en Babilonia después de la destrucción del templo de Jerusalén, el cual había perdido la esperanza de volver a la ciudad santa en ruinas. \textquote{A los que esperan en el Señor él les renovará el vigor ---asegura el profeta---, subirán con alas como de águilas, correrán sin fatigarse y andarán sin cansarse} (\emph{Is} 40, 31). Y también:~ \textquote{Regocijo y alegría los acompañarán. Pena y aflicción se alejarán} (\emph{Is} 35, 10).

La liturgia de Adviento nos repite constantemente que debemos despertar del sueño de la rutina y de la mediocridad; debemos abandonar la tristeza y el desaliento. Es preciso que se alegre nuestro corazón porque \textquote{el Señor está cerca}.

Hoy tenemos un motivo ulterior para alegrarnos, queridos fieles de la parroquia de \emph{Santa María del Rosario en los Mártires Portuenses,} y es la dedicación de vuestra nueva iglesia parroquial, que surge en el mismo lugar donde mi amado predecesor el siervo de Dios Juan Pablo II celebró, el 8 de noviembre de 1998, la santa misa con ocasión de su visita pastoral a vuestra comunidad.

La solemne liturgia de la dedicación de este templo constituye una ocasión de intenso gozo espiritual para todo el pueblo de Dios que vive en esta zona. Y de buen grado me uno también yo a vuestra satisfacción por tener por fin una iglesia acogedora y funcional. El lugar en que está construida evoca un pasado de testimonios cristianos resplandecientes. En efecto, precisamente aquí, en las cercanías, se encuentran las catacumbas de Generosa, donde según la tradición fueron sepultados tres hermanos, Simplicio, Faustino y Beatriz, víctimas de la persecución desencadenada en el año 303, y cuyos restos mortales fueron conservados, en parte, en Roma en la iglesia de San Nicolás in Carcere y en Monte Savello, y, en parte, en Fulda, Alemania, ciudad que desde el siglo VIII, gracias a que san Bonifacio llevó allí las reliquias, honra a los mártires portuenses como sus copatronos.

A este respecto, saludo al representante del obispo de Fulda, y también a mons. Carlo Liberati, arzobispo-prelado de Pompeya, santuario mariano con el que vuestra parroquia mantiene un hermanamiento espiritual.

La dedicación de esta iglesia parroquial cobra un significado muy particular para vosotros que vivís en este barrio. Los jóvenes mártires que entonces murieron por dar testimonio de Cristo, ¿no son un fuerte estímulo para vosotros, cristianos de hoy, a perseverar en el seguimiento fiel de Jesucristo? Y la protección de la Virgen del Santo Rosario, ¿no os pide ser hombres y mujeres de fe profunda y de oración, como lo fue ella?

También hoy, aunque sea con formas diversas, el mensaje salvífico de Cristo encuentra oposición y los cristianos, de otras maneras y no menos que ayer, están llamados a dar razón de su esperanza, a testimoniar ante el mundo la verdad de Cristo, el único que salva y redime. Por consiguiente, esta nueva iglesia ha de ser un espacio privilegiado para crecer en el conocimiento y en el amor de Cristo, a quien dentro de pocos días acogeremos en la alegría de su nacimiento como Redentor del mundo y Salvador nuestro.

Aprovechando la dedicación de esta nueva y hermosa iglesia, quiero dar las gracias a todos los que han contribuido a construirla. Sé que la diócesis de Roma se está esforzando con empeño, desde hace muchos años, por lograr que en cada barrio de esta ciudad en crecimiento constante haya complejos parroquiales adecuados.

Saludo y expreso mi gratitud, en primer lugar, al cardenal vicario y al obispo auxiliar Ernesto Mandara, secretario de la Obra romana para la conservación de la fe y la provisión de nuevas iglesias en Roma. Os saludo y os manifiesto mi agradecimiento en particular a vosotros, queridos feligreses, que de diversas maneras os habéis comprometido en la realización de este centro parroquial, que se añade a los más de cincuenta que ya funcionan gracias al notable esfuerzo económico de la diócesis, de tantos fieles y ciudadanos de buena voluntad, y a la colaboración de las instituciones públicas. En este domingo, dedicado precisamente al apoyo de esa meritoria obra, pido a todos que prosigan ese compromiso con generosidad.

Asimismo, saludo con afecto a mons. Benedetto Tuzia, obispo auxiliar del sector oeste; a vuestro párroco, don Gerard Charles McCarthy, a quien agradezco de corazón las cordiales palabras que me ha dirigido al inicio de esta solemne celebración. Saludo a sus colaboradores sacerdotes, pertenecientes a la fraternidad sacerdotal de los Misioneros de San Carlos Borromeo, aquí representada por el superior general, mons. Massimo Camisasca, a la que desde 1997 está encomendada la atención pastoral de esta parroquia.

Saludo a las religiosas Oblatas del Amor Divino y a las Misioneras de San Carlos, que con gran entrega realizan su apostolado en esta comunidad, y a todos los grupos de niños, de jóvenes, de familias y de ancianos que animan la vida de la parroquia. También saludo cordialmente a los diversos movimientos eclesiales presentes, entre los cuales están la Juventud ardiente mariana, Comunión y liberación, la Renovación carismática católica, la Fraternidad de Santa María de los ángeles, y el grupo de voluntariado Santa Teresita.

Además, quiero animar a todos los que, juntamente con la \emph{Cáritas} parroquial, tratan de salir al encuentro de las muchas necesidades del barrio, especialmente respondiendo a las expectativas de los más pobres y necesitados. Por último, saludo a las autoridades presentes y a las personalidades que han querido participar en esta asamblea litúrgica.

Queridos amigos, vivimos hoy una jornada que corona los esfuerzos, las fatigas, los sacrificios realizados y el compromiso de la comunidad de formar una comunidad cristiana madura, deseosa de tener un espacio reservado definitivamente al culto de Dios. Esta celebración es muy rica en palabras y símbolos que nos ayudan a comprender el valor profundo de lo que estamos realizando. Por eso, recojamos brevemente la enseñanza que nos dan las lecturas que se acaban de proclamar.

La primera lectura está tomada del libro de Nehemías, un libro que nos narra el restablecimiento de la comunidad judía después del destierro, después de la dispersión y la destrucción de Jerusalén. Por tanto, es el libro de los nuevos orígenes de una comunidad, y está lleno de esperanza, aunque las dificultades eran aún grandísimas. En el centro del pasaje que nos acaban de leer se encuentran dos grandes figuras:~ un sacerdote, Esdras, y un laico, Nehemías, que son respectivamente la autoridad religiosa y la autoridad civil de aquel tiempo.

El texto describe el momento solemne en que se restablece oficialmente, después de la dispersión, la pequeña comunidad judía; es el momento de volver a proclamar públicamente la ley, que es el fundamento de la vida de esta comunidad, y todo se desarrolla en un clima de sencillez, de pobreza y de esperanza. La escucha de esta proclamación tiene lugar en un clima de gran intensidad espiritual. Algunos comienzan a llorar de alegría por poder escuchar nuevamente con libertad la palabra de Dios, después de la tragedia de la destrucción de Jerusalén, y recomenzar la historia de la salvación. Y Nehemías los exhorta diciendo que es un día de fiesta y que, para tener la fuerza del Señor, es preciso alegrarse, agradeciendo a Dios sus dones. La palabra de Dios es fuerza y alegría.

También en nosotros esta lectura del Antiguo Testamento suscita gran conmoción. En este momento ¡cuántos recuerdos se agolpan en vuestra mente! ¡Cuántos esfuerzos realizados para construir, año tras año, la comunidad! ¡Cuántos sueños, cuántos proyectos, cuántas dificultades! Sin embargo, ahora tenéis la posibilidad de proclamar y escuchar la palabra de Dios en una hermosa iglesia, que favorece el recogimiento y suscita alegría, la alegría de saber que no sólo está presente la palabra de Dios, sino también el Señor mismo; una iglesia que quiere ser una invitación constante a una fe firme y al compromiso de crecer como comunidad unida. Agradezcamos a Dios sus dones y manifestemos nuestra gratitud también a todos los que han sido artífices de la construcción de esta iglesia y de la comunidad viva que en ella se reúne.

En la segunda lectura, tomada del Apocalipsis, se nos narra una visión estupenda. El proyecto de Dios para su Iglesia y para la humanidad entera es una ciudad santa, Jerusalén, que desciende del cielo resplandeciente de gloria divina. El autor la describe como ciudad maravillosa, comparándola con las joyas más preciosas, y por último precisa que se apoya en la persona y en el mensaje de los Apóstoles. Al decir esto, el evangelista san Juan nos sugiere que la comunidad viva es la verdadera nueva Jerusalén, y que la comunidad viva es más sagrada que el templo material que consagramos.

Para construir este templo vivo, esta nueva ciudad de Dios en nuestras ciudades, para construir el templo que sois vosotros, hace falta mucha oración, hace falta aprovechar todas las oportunidades que nos brindan la liturgia, la catequesis y las múltiples actividades pastorales, caritativas, misioneras y culturales, que conservan \textquote{joven} vuestra prometedora parroquia. El cuidado que con razón brindamos al edificio material ---rociándolo con el agua bendita, ungiéndolo con óleo y llenándolo de incienso--- debe ser signo y estímulo de un ~cuidado más intenso para defender y promover el templo de las personas, formado por vosotros, queridos feligreses.

Por último, la página evangélica que acabamos de escuchar nos narra el diálogo entre Jesús y los suyos, en particular con Pedro. Es una conversación totalmente centrada en la persona del Maestro divino. La gente había intuido algo en él. Algunos pensaban que era Juan Bautista que había vuelto a la vida; otros que Elías había regresado a la tierra; otros, que era el profeta Jeremías. En cualquier caso, la gente pensaba que era una de las grandes personalidades religiosas.

Pedro, en cambio, en nombre de los discípulos que conocen a Jesús de cerca, declara que Jesús es más que un profeta, más que una gran personalidad religiosa de la historia:~ es el Mesías, el Cristo, el Hijo de Dios vivo. Y Cristo, el Señor, le dice respondiendo solemnemente:~ \textquote{Tú eres Pedro y sobre esta piedra edificaré mi Iglesia} (\emph{Mt} 16, 18). Pedro, el pobre hombre con todas sus debilidades y con su fe, se convierte en la piedra, asociado precisamente por su fe a Jesús, es la roca sobre la que está fundada la Iglesia.

De ese modo, vemos una vez más cómo Jesucristo es la verdadera roca indefectible sobre la que se apoya nuestra fe, sobre la que se construye toda la Iglesia y, así, también esta parroquia. Y a Jesús lo encontramos en la escucha de la sagrada Escritura; está presente y se hace nuestro alimento en la Eucaristía; vive en la comunidad, en la fe de la comunidad parroquial.

Por consiguiente, en la iglesia edificio y en la Iglesia comunidad, todo habla de Jesús; todo gira en torno a él; todo hace referencia a él. Y Jesús, el Señor, nos reúne en la gran comunidad de la Iglesia de todos los tiempos y de todos los lugares, en comunión con el Sucesor de Pedro como roca de la unidad. La acción de los obispos y de los presbíteros, el compromiso apostólico y misionero de todos los fieles consiste en proclamar y testimoniar con la palabra y con la vida que él, el Hijo de Dios hecho hombre, es nuestro único Salvador.

Pidamos a Jesús que guíe a vuestra comunidad y la haga crecer cada vez más en la fidelidad a su Evangelio; pidámosle que suscite muchas y santas vocaciones sacerdotales, religiosas y misioneras; que suscite en todos los feligreses la disponibilidad a seguir el ejemplo de los santos mártires portuenses.

Pongamos esta oración en las manos maternales ~de María, Reina del Rosario. Que ~ella ~obtenga que se realicen en nosotros, ~en ~este ~día, las palabras finales ~de la primera lectura:~ \textquote{Que la alegría del Señor sea nuestra fuerza} (cf. \emph{Ne} 8, 10). Sólo la alegría del Señor y la fuerza de la fe en él pueden hacer fecundo el camino de vuestra parroquia. Así sea.

\subsubsection{Ángelus (2007)} *** \emph{\textbf{ÁNGELUS}\\[2\baselineskip]Plaza de San Pedro\\ III Domingo de Adviento, 16 de diciembre de 2007}

~

\emph{Queridos hermanos y hermanas:}~

\emph{\textquote{Gaudete in Domino semper}, estad siempre alegres en el Señor} (\emph{Flp} 4, 4). Con estas palabras de san Pablo se inicia la santa misa del III domingo de Adviento, que por eso se llama domingo \emph{\textquote{Gaudete}}. El Apóstol exhorta a los cristianos a alegrarse porque la venida del Señor, es decir, su vuelta gloriosa es segura y no tardará. La Iglesia acoge esta invitación mientras se prepara para celebrar la Navidad, y su mirada se dirige cada vez más a Belén. En efecto, aguardamos con esperanza segura la segunda venida de Cristo, porque hemos conocido la primera.

El misterio de Belén nos revela al Dios-con-nosotros, al Dios cercano a nosotros, no sólo en sentido espacial y temporal; está cerca de nosotros porque, por decirlo así, se ha \textquote{casado} con nuestra humanidad; ha asumido nuestra condición, escogiendo ser en todo como nosotros, excepto en el pecado, para hacer que lleguemos a ser como él.

Por tanto, la alegría cristiana brota de esta certeza:~ Dios está cerca, está conmigo, está con nosotros, en la alegría y en el dolor, en la salud y en la enfermedad, como amigo y esposo fiel. Y esta alegría permanece también en la prueba, incluso en el sufrimiento; y no está en la superficie, sino en lo más profundo de la persona que se encomienda a Dios y confía en él.

Algunos se preguntan:~¿también hoy es posible esta alegría? La respuesta la dan, con su vida, hombres y mujeres de toda edad y condición social, felices de consagrar su existencia a los demás. En nuestros tiempos, la beata madre Teresa de Calcuta fue testigo inolvidable de la verdadera alegría evangélica. Vivía diariamente en contacto con la miseria, con la degradación humana, con la muerte. Su alma experimentó la prueba de la noche oscura de la fe y, sin embargo, regaló a todos la sonrisa de Dios.

En uno de sus escritos leemos:~ \textquote{Esperamos con impaciencia el paraíso, donde está Dios, pero ya aquí en la tierra y desde este momento podemos estar en el paraíso. Ser felices con Dios significa:~ amar como él, ayudar como él, dar como él, servir como él} (\emph{La gioia di darsi agli altri}, Ed. Paoline 1987, p. 143). Sí, la alegría entra en el corazón de quien se pone al servicio de los pequeños y de los pobres. Dios habita en quien ama así, y el alma vive en la alegría.

En cambio, si se hace de la felicidad un ídolo, se equivoca el camino y es verdaderamente difícil encontrar la alegría de la que habla Jesús. Por desgracia, esta es la propuesta de las culturas que ponen la felicidad individual en lugar de Dios, mentalidad que se manifiesta de forma emblemática en la búsqueda del placer a toda costa y en la difusión del uso de drogas como fuga, como refugio en paraísos artificiales, que luego resultan del todo ilusorios.

Queridos hermanos y hermanas, también en Navidad se puede equivocar el camino, confundiendo la verdadera fiesta con una que no abre el corazón a la alegría de Cristo. Que la Virgen María ayude a todos los cristianos, y a los hombres que buscan a Dios, a llegar hasta Belén para encontrar al Niño que nació por nosotros, para la salvación y la felicidad de todos los hombres.

\subsubsection{Homilía (2010)} III Domingo de Adviento, 12 de diciembre de 2010


~

\emph{Queridos hermanos y hermanas de la parroquia de San Maximiliano Kolbe:}

Vivid con empeño el camino personal y comunitario de seguimiento del Señor. El Adviento es una fuerte invitación para todos a dejar que Dios entre cada vez más en nuestra vida, en nuestros hogares, en nuestros barrios, en nuestras comunidades, para tener una luz en medio de tantas sombras y de las numerosas pruebas de cada día. Queridos amigos, estoy muy contento de estar entre vosotros hoy para celebrar el día del Señor, el tercer domingo del Adviento, domingo de la alegría. Saludo cordialmente al cardenal vicario, al obispo auxiliar del sector, a vuestro párroco, a quien agradezco las palabras que me ha dirigido en nombre de todos vosotros, y al vicario parroquial. Saludo a cuantos colaboran en las actividades de la parroquia: a los catequistas, a las personas que forman parte de los diversos grupos, así como a los numerosos miembros del Camino Neocatecumenal. Aprecio mucho la elección de dar espacio a la adoración eucarística, y os agradezco las oraciones que me reserváis ante el Santísimo Sacramento. Quiero extender mi saludo a todos los habitantes del barrio, especialmente a los ancianos, a los enfermos, a las personas solas o que atraviesan dificultades. A todos y cada uno los recuerdo en esta misa.

Admiro junto con vosotros esta nueva iglesia y los edificios parroquiales, y con mi presencia deseo alentaros a construir cada vez mejor la Iglesia de piedras vivas que sois vosotros mismos. Conozco las numerosas y significativas obras de evangelización que estáis realizando. Exhorto a todos los fieles a contribuir a la edificación de la comunidad, especialmente en el campo de la catequesis, de la liturgia y de la caridad ---pilares de la vida cristiana--- en comunión con toda la diócesis de Roma. Ninguna comunidad puede vivir como una célula aislada del contexto diocesano; al contrario, debe ser expresión viva de la belleza de la Iglesia que, bajo la guía del obispo ---y, en la parroquia, bajo la guía del párroco, que lo representa---, camina en comunión hacia el reino de Dios. Dirijo un saludo especial a las familias, acompañándolo con el deseo de que realicen plenamente su vocación al amor con generosidad y perseverancia. Aunque se presentaran dificultades en la vida conyugal y en la relación con los hijos, los esposos deben permanecer siempre fieles al fundamental \textquote{sí} que pronunciaron delante de Dios y se dijeron mutuamente en el día de su matrimonio, recordando que la fidelidad a la propia vocación exige valentía, generosidad y sacrificio.

En el seno de vuestra comunidad hay muchas familias venidas del centro y del sur de Italia en busca de trabajo y de mejores condiciones de vida. Con el paso del tiempo, la comunidad ha crecido y en parte se ha transformado, con la llegada de numerosas personas de los países del Este europeo y de otros países. Precisamente a partir de esta situación concreta de la parroquia, esforzaos por crecer cada vez más en la comunión con todos: es importante crear ocasiones de diálogo y favorecer la comprensión mutua entre personas provenientes de culturas, modelos de vida y condiciones sociales diferentes; pero es preciso sobre todo tratar de que participen en la vida cristiana, mediante una pastoral atenta a las necesidades reales de cada uno. Aquí, como en cada parroquia, hay que partir de los \textquote{cercanos} para llegar a los \textquote{lejanos}, para llevar una presencia evangélica a los ambientes de vida y de trabajo. En la parroquia todos deben poder encontrar caminos adecuados de formación y experimentar la dimensión comunitaria, que es una característica fundamental de la vida cristiana. De ese modo se verán alentados a redescubrir la belleza de seguir a Cristo y de formar parte de su Iglesia.

Sabed, pues, hacer comunidad con todos, unidos en la escucha de la Palabra de Dios y en la celebración de los sacramentos, especialmente de la Eucaristía. A este propósito, la verificación pastoral diocesana que se está llevando a cabo, sobre el tema \textquote{Eucaristía dominical y testimonio de la caridad}, es una ocasión propicia para profundizar y vivir mejor estos dos componentes fundamentales de la vida y de la misión de la Iglesia y de todo creyente, es decir, la Eucaristía del domingo y la practica de la caridad. Reunidos en torno a la Eucaristía, sentimos más fácilmente que la misión de toda comunidad cristiana consiste en llevar el mensaje del amor de Dios a todos los hombres. Por eso es importante que la Eucaristía siempre sea el corazón de la vida de los fieles. También quiero dirigiros unas palabras de afecto y de amistad en especial a vosotros, queridos muchachos y jóvenes que me escucháis, y a vuestros coetáneos que viven en esta parroquia. La Iglesia espera mucho de vosotros, de vuestro entusiasmo, de vuestra capacidad de mirar hacia adelante y de vuestro deseo de radicalidad en las opciones de la vida. Sentíos verdaderos protagonistas en la parroquia, poniendo vuestras energías lozanas y toda vuestra vida al servicio de Dios y de los hermanos.

Queridos hermanos y hermanas, la liturgia de hoy ---con las palabras del apóstol Santiago que hemos escuchado--- nos invita no sólo a la alegría sino también a ser constantes y pacientes en la espera del Señor que viene, y a serlo juntos, como comunidad, evitando quejas y juicios (cf. \emph{St} 5, 7-10).

Hemos escuchado en el Evangelio la pregunta de san Juan Bautista que se encuentra en la cárcel; el Bautista, que había anunciado la venida del Juez que cambia el mundo, y ahora siente que el mundo sigue igual. Por eso, pide que pregunten a Jesús: \textquote{¿Eres tú el que ha de venir o debemos esperar a otro? ¿Eres tú o debemos esperar a otro?}. En los últimos dos o tres siglos muchos han preguntado: \textquote{¿Realmente eres tú o hay que cambiar el mundo de modo más radical? ¿Tú no lo haces?}. Y han venido muchos profetas, ideólogos y dictadores que han dicho: \textquote{¡No es él! ¡No ha cambiado el mundo! ¡Somos nosotros!}. Y han creado sus imperios, sus dictaduras, su totalitarismo que cambiaría el mundo. Y lo ha cambiado, pero de modo destructivo. Hoy sabemos que de esas grandes promesas no ha quedado más que un gran vacío y una gran destrucción. No eran ellos.

Y así debemos mirar de nuevo a Cristo y preguntarle: \textquote{¿Eres tú?}. El Señor, con el modo silencioso que le es propio, responde: \textquote{Mirad lo que he hecho. No he hecho una revolución cruenta, no he cambiado el mundo con la fuerza, sino que he encendido muchas luces que forman, a la vez, un gran camino de luz a lo largo de los milenios}.

Comencemos aquí, en nuestra parroquia: san Maximiliano Kolbe, que se ofreció para morir de hambre a fin de salvar a un padre de familia. ¡En qué gran luz se ha convertido! ¡Cuánta luz ha venido de esta figura! Y ha alentado a otros a entregarse, a estar cerca de quienes sufren, de los oprimidos. Pensemos en el padre que era para los leprosos Damián de Veuster, que vivió y murió \emph{con} y \emph{para} los leprosos, y así llevó luz a esa comunidad. Pensemos en la madre Teresa, que dio tanta luz a personas, que, después de una vida sin luz, murieron con una sonrisa, porque las había tocado la luz del amor de Dios.

Y podríamos seguir y veríamos, como dijo el Señor en la respuesta a Juan, que lo que cambia el mundo no es la revolución violenta, ni las grandes promesas, sino la silenciosa luz de la verdad, de la bondad de Dios, que es el signo de su presencia y nos da la certeza de que somos amados hasta el fondo y de que no caemos en el olvido, no somos un producto de la casualidad, sino de una voluntad de amor.

Así podemos vivir, podemos sentir la cercanía de Dios. \textquote{Dios está cerca} dice la primera lectura de hoy; está cerca, pero nosotros a menudo estamos lejos. Acerquémonos, vayamos hacia la presencia de su luz, oremos al Señor y en el contacto de la oración también nosotros seremos luz para los demás.

Precisamente este es el sentido de la iglesia parroquial: entrar aquí, entrar en diálogo, en contacto con Jesús, con el Hijo de Dios, a fin de que nosotros mismos nos convirtamos en una de las luces más pequeñas que él ha encendido y traigamos luz al mundo, que sienta que es redimido.

Nuestro espíritu debe abrirse a esta invitación; así caminemos con alegría al encuentro de la Navidad, imitando a la Virgen María, que esperó en la oración, con íntimo y gozoso temor, el nacimiento del Redentor. Amén.

\subsubsection{Homilía (2019)} SANTA MISA PARA LA COMUNIDAD CATÓLICA FILIPINA

\textbf{\emph{HOMILÍA DEL SANTO PADRE FRANCISCO}}

\emph{Basílica Vaticana\\ Domingo, 15 de diciembre de 2019}



~

\emph{Queridos hermanos y hermanas:}

Celebramos hoy el tercer domingo de Adviento. En la primera lectura, el profeta Isaías invita a la tierra entera a alegrarse por la venida del Señor, que trae la salvación a su pueblo. Viene a abrir los ojos a los ciegos y los oídos a los sordos, a curar a los cojos y a los mudos (cf. 35,5-6). La salvación se ofrece a todos, pero el Señor muestra una ternura especial por los más vulnerables, los más frágiles, los más pobres de su pueblo.

De las palabras del salmo responsorial aprendemos que hay otros vulnerables que merecen una mirada de amor especial de Dios: los oprimidos, los hambrientos, los prisioneros, los extranjeros, los huérfanos y las viudas (cf. \emph{Sal} 145,7-9). Son los habitantes de las periferias existenciales de ayer y de hoy.

En Jesucristo el amor salvífico de Dios se hace tangible: \textquote{Los ciegos ven y los cojos andan, los leprosos quedan limpios y los sordos oyen, los muertos resucitan y a los pobres se anuncia la Buena Nueva} (\emph{Mt} 11,5). Estos son los signos que acompañan la realización del Reino de Dios. No toques de trompeta o triunfos militares, no juicios y condenas de pecadores, sino liberación del mal y anuncio de misericordia y de paz.

También este año nos preparamos para celebrar el misterio de la Encarnación, de Emmanuel, el \textquote{Dios con nosotros} que obra maravillas en favor de su pueblo, especialmente de los más pequeños y frágiles. Estas maravillas son los \textquote{signos} de la presencia de su Reino. Y como todavía son muchos los habitantes de las periferias existenciales, debemos pedir al Señor que renueve cada año el milagro de la Navidad, ofreciéndonos nosotros mismos como instrumentos de su amor misericordioso por los más pequeños.

Para prepararnos adecuadamente a esta nueva efusión de gracia, la Iglesia nos brinda el tiempo de Adviento, en el que estamos llamados a despertar la esperanza en nuestros corazones e intensificar nuestra oración. Con este fin, en la riqueza de las diferentes tradiciones, las Iglesias particulares han introducido una variedad de prácticas devocionales.

En Filipinas existe desde hace siglos una novena en preparación para la Santa Navidad llamada \emph{Simbang-Gabi} (misa nocturna). Durante nueve días los fieles filipinos se reúnen al amanecer en sus parroquias para una celebración eucarística especial. En las últimas décadas, gracias a los emigrantes filipinos, esta devoción ha traspasado las fronteras nacionales llegando a muchos otros países. Desde hace años \emph{Simbang-Gabi} también se celebra en la diócesis de Roma, y hoy lo celebramos juntos aquí, en la basílica de San Pedro.

Con esta celebración queremos prepararnos para la Navidad según el espíritu de la Palabra de Dios que hemos escuchado, permaneciendo constantes hasta la venida definitiva del Señor, como nos recomienda el apóstol Santiago (cf. \emph{St} 5,7). Queremos comprometernos a manifestar el amor y la ternura de Dios hacia todos, especialmente hacia los más pequeños. Estamos llamados a ser levadura en una sociedad que a menudo ya no puede saborear la belleza de Dios y experimentar la gracia de su presencia.

Y vosotros, queridos hermanos y hermanas, que habéis dejado vuestra tierra en busca de un futuro mejor, tenéis una misión especial. Que vuestra fe sea \textquote{levadura} en las comunidades parroquiales a las que pertenecéis hoy. Os animo a multiplicar las oportunidades de encuentro para compartir vuestra riqueza cultural y espiritual, al mismo tiempo que os dejáis enriquecer por las experiencias de los demás. Todos estamos invitados a construir juntos esa comunión en la diversidad que es un rasgo distintivo del Reino de Dios, inaugurado por Jesucristo, el Hijo de Dios hecho hombre. Todos estamos llamados a practicar juntos la caridad con los habitantes de las periferias existenciales, poniendo al servicio nuestros diversos dones, para renovar los signos de la presencia del Reino. Todos estamos llamados a anunciar juntos el Evangelio, la Buena Nueva de la salvación, en todas las lenguas, para llegar al mayor número posible de personas.

Que el Santo Niño al que nos disponemos a adorar, envuelto en pobres pañales y recostado en un pesebre, os bendiga y os dé la fuerza para continuar vuestro testimonio con alegría.

\subsubsection{Ángelus ()}

\section{IV Adviento} \subsubsection{Ángelus (1992)} \emph{\textbf{ÁNGELUS\\ }\\ Domingo 20 de diciembre de 1992}

~

\emph{Amadísimos hermanos y hermanas:}

1. Faltan ya pocos días para la celebración de la Navidad del Señor y queremos vivirlos siguiendo las huellas de María y haciendo nuestros en la medida de lo posible, los sentimientos que ella experimentó en la trémula espera del nacimiento de Jesús.

El evangelista Lucas narra que la Virgen santa y su esposo José se dirigieron de Galilea a Judea para ir a Belén, la ciudad de David, obedeciendo un decreto del emperador romano que ordenaba un censo general del Imperio.

Pero, ¿quién podía reparar en ellos? Pertenecían a la innumerable legión de pobres, a quienes la vida a duras penas regala un rincón para vivir, y que no dejan rastro en las crónicas. De hecho no encontraron acomodo en ningún sitio, a pesar de que llevaban el \textquote{secreto} del mundo.

Podemos intuir cuáles eran los sentimientos de María, totalmente abandonada en las manos del Señor. Ella es la mujer creyente: en la profundidad de su obediencia interior madura la plenitud de los tiempos (cf. \emph{Ga} 4, 4).

2. Por estar enraizada en la fe, la Madre del Verbo hecho hombre \emph{encarna la gran esperanza del mundo.} En ella confluye tanto la espera mesiánica de Israel como el anhelo de salvación de la humanidad entera. En su espíritu resuena el grito de dolor de los que, en toda época de la historia, se sienten abrumados por las dificultades de la vida: los hambrientos y los necesitados, los enfermos y las víctimas del odio y la guerra, los que no tienen hogar ni trabajo y los que viven solos y marginados, los que se sienten aplastados por la violencia y la injusticia o rechazados por la desconfianza y la indiferencia, los desanimados y los defraudados.

Para los hombres de toda raza y cultura, sedientos de amor, de fraternidad y de paz, María se prepara a dar a luz el fruto divino de su vientre. Por más oscuro que pueda parecer el horizonte, hay un alba que despunta. La humanidad, como recuerda san Pablo, gime y \textquote{sufre dolores de parto} (\emph{Rm} 8, 22): en el nacimiento del Hijo de Dios todo renace, todo está llamado a vida nueva.

3. Queridos hermanos y hermanas preparémonos para la Navidad con la fe y la esperanza de María. Dejemos que el mismo amor que vibra en su adhesión al plan divino toque nuestro corazón. La Navidad es tiempo de renovación y fraternidad: miremos a nuestro alrededor, miremos a lo lejos. El hombre que sufre, dondequiera que se encuentre, nos atañe. Allí se encuentra el belén al que debemos dirigirnos, con solidaridad activa, para encontrar de verdad al Redentor que nace en el mundo. Caminemos, por consiguiente, hacia la Noche Santa con María, la Madre del Amor. Con ella esperemos el cumplimiento del misterio de la salvación.

\section{Nochebuena} \subsubsection{Homilía (1962)}

Lunes 24 de diciembre de 1962 \emph{Venerables hermanos y queridos hijos:}

Esta misa de la noche de la Navidad del Señor santifica las más hermosas interioridades del alma, que tienden a lo que es la esencia viva de la unión con Cristo: la religión sincera, liturgia bien comprendida y anhelo de perfección cristiana. Lo advertimos en este momento de tranquilo recogimiento, bajo la mirada del Divino Infante.

En realidad, los grandes problemas de la vida social e individual se acercan a la cuna de Belén, al paso que los ángeles invitan a dar gloria a Dios, gloria a Cristo redentor y salvador, y a excitar gozosamente las buenas voluntades para la celebración de la paz universal.

Gran don, gran riqueza en verdad, es la paz del mundo, que va tras la paz. Lo hemos repetido en el radiomensaje navideño, y Nos satisface dar gracias al Señor por haberlo hecho acoger con buena voluntad de un extremo al otro de la tierra, como confirmación de la luz de esperanza encendida y viva en todas las naciones.

Las súplicas de todos continúan pidiendo la conservación y el perfeccionamiento de este don celestial, al paso que son cada vez más atentos y prudentes todos los movimientos de ideas, palabras y actividades, y se multiplican en todos los campos los esfuerzos y los acuerdos para alejar y superar los obstáculos, conocer y substraer las causas que provocan los conflictos.

Comprendednos, queridos hijos, si hemos preferido, para la misa de Navidad, la sencillez de nuestra capilla privada a las majestuosas bóvedas de los templos romanos, como para dejarnos envolver por el ambiente de las humildes iglesias del campo y de la montaña, de las innumerables instituciones de asistencia social, que son el refugio de la inocencia pobre y abandonada, consuelo y endulzamiento de las lágrimas amargas, reparación de injusticias palmarias y no suficientemente conocidas.

También pensamos en vosotros, queridos enfermos y ancianos, que sufrís dolores y soledad; que vuestro dolor y soledad alcance grandes merecimientos a vosotros y bien a la humanidad.

Hay también circunstancias y situaciones que en esta solemnidad hacen más evidente y agudo el contraste con el gozo de la Navidad. Reclamo eficaz no para disminuir el servicio que hacemos a la verdad y a la justicia, ni para olvidar el inmenso bien realizado por las almas rectas, que tienen como honor la ley divina y el Evangelio; sino para alentar las mejores energías a reparar los errores y a reavivar en el mundo el fervor religioso y~ las piadosas tradiciones paternas como gozo tranquilo de la Navidad.

Hijos queridos: Junto a la cuna del Niño recién nacido, del Hijo de Dios hecho hombre, todos los hombres que caminan por la tierra piensan con conciencia clara y seria que en la hora suprema se les pedirá cuenta estrecha del don de la vida; y ésta tendrá una sanción definitiva de premio o de castigo, de gloria o de abominación.

En la conciencia de este rendir cuentas es donde se mide la participación de los cristianos y de todos los hombres en el gran misterio que conmemoramos en esta noche; de aquí surge el deseo de que por la luz del Verbo de Dios la civilización humana reciba la llamita que le puede transformar en vivo fulgor, en beneficio de los pueblos.

En torno a la cuna de Jesús sus ángeles cantaron la paz. Y quien creyó en el mensaje celestial y le hizo honor consiguió gloria y alegría. Así ayer; y así será siempre a lo largo de los siglos.

La historia de Cristo es perpetua. Bienaventurado quien la comprende y consigue gracia, fortaleza y bendición. Amén. Amén.

\begin{center}\rule{0.5\linewidth}{\linethickness}\end{center}

\protect\hyperlink{_ednrefux2a}{*}\emph{~ AAS} 55 (1963) 51; \emph{Discorsi-Messaggi-Colloqui del Santo Padre Giovanni XXIII}, vol. V, pp. 63-65.

\subsubsection{Homilía (1965)} \emph{Capilla Sixtina\\ Viernes 24 de diciembre de 1965}

~

Esta santa noche vuelve a proponer a nuestra mente la meditación siempre nueva, siempre sugestiva y, a decir verdad, inagotable, del misterio fundamental de todo el Cristianismo: ¡Dios se hizo hombre!

\textquote{Si alguno -- dice Santo Tomás -- considera con atención y piedad el misterio de la Encarnación, hallará una profundidad de sabiduría tal, que sobrepuja, todo conocimiento humano} (\emph{Contra Gentiles,}4, 54).

En efecto, decir: Dios, es como decir la Grandeza, el poder, la santidad infinita. Decir: el hombre, es como decir la pequeñez, la debilidad, la miseria. Entre estos dos extremos, la distancia parece imposible de salvar, el foso parece imposible de colmar. Y he aquí que en Cristo estos dos conceptos son una sola cosa. La misma persona vive, a la vez, en la naturaleza divina y en la naturaleza humana de Cristo. El Padre de los Cielos puede decir: \textquote{Este es mi hijo bienamado} (\emph{Mt}. 17, 5), como a su vez lo puede decir la Virgen María dirigiéndose al Infante del pesebre que acaba de dar a luz.

Misterio inefable de unión: lo que estaba dividido se reúne, lo que parecía incompatible se acerca, los extremos se funden en uno solo: dos naturalezas -- la humana y la divina -- en una sola persona, la del Hombre-Dios. He aquí toda la teología de la Encarnación, el fundamento y la síntesis de todo el cristianismo.

El prodigio inicial, realizado en Cristo, halla su continuación misteriosa en lo que aquí abajo, hasta el fin de los tiempos, es el \textquote{Cuerpo místico} de Cristo, la gran familia de todos los que creen en El. Porque todo hombre debe unirse a Dios: \textquote{Dios se hizo hombre -- dice magníficamente San Agustín -- para que el hombre se haga Dios}. Tal es el designio divino, revelado en el misterio de Navidad. Y la historia de la Iglesia a través de los siglos, constituye la historia de la realización de tal designio.

En la Encarnación, Dios ha unido el hombre a sí con vínculos tan fuertes que se demuestran superiores a todos los demás, más fuertes que los de la carne y la sangre e incluso los que unen al hombre con lo que le es más valioso en el mundo: la vida. ¿No nos habla acaso todo, aquí en Roma, del valor de los mártires cristianos de los primeros siglos? Hombres, mujeres y también niños dan testimonio ante el verdugo de que separarse de Dios por una abjuración sería para ellos mayor desgracia que perder la vida. La sacrifican para permanecer unidos a Dios.

Cuando la espada del perseguidor romano cesó de herir, las grandes almas cristianas van a buscar a Dios en la soledad. Se abandona la familia, se renuncia a formar una, para unirse mejor a Dios. La corona de la virginidad es ambicionada con el mismo fervor con que se ambiciona la del martirio. La ofrenda cotidiana de sí mismo en la vida monástica tomó el lugar del sacrificio cruento realizado da una sola vez. Y en las mil formas de la vida consagrada, esta unión del hombre con Dios, amado sobre todas las cosas, seguirá manifestándose a través de los siglos hasta nuestros días. La Iglesia suscitará también legiones de santos en el mundo; junto a los mártires, las vírgenes, los doctores, los pontífices y los confesores, ella tendrá la inmensa familia de sus santas mujeres, madres de familia y viudas; en todas las épocas y en todos los países ella suscitará innumerables y fieles ejemplos en muchos hogares cristianos para testimoniar lo que el hombre es capaz de hacer para unirse a Dios, cuando comprende lo que Dios ha hecho para unirse al hombre.

Modelo sublime y principio de la unión del hombre con Dios, la Encarnación se reveló también un maravilloso factor de civilización. ¿Quién como los Apóstoles del Dios encarnado, ha contribuido tanto en el transcurso de los siglos a elevar a los pueblos y a revelarles, además de la grandeza de Dios, su propia dignidad?

La sociedad en la que penetra el fermento cristiano ve elevarse poco a poco su nivel moral y ampliarse su horizonte a las dimensiones del mundo pues la que parecía que sólo incumbía a las relaciones del hombre con Dios se revela el más poderoso factor de unión entre los hombres mismos. El poder de unión de la fe cristiana actúa en el seno de las familias y de los pueblos; derriba las barreras de castas, razas y naciones. La fe que une el hombre a Dios une también a los hombres entre si en un ideal común, en un esfuerzo común, en una esperanza común. ¡Qué motivo ilimitado de meditación! La fe en el Dios encarnado penetra, a lo largo de los siglos, las diversas culturas y las purifica, las enriquece, las transforma. Es la inteligencia humana que se ha superado a si misma, es la filosofía humana que recibió el complemento de las luces divinas como una luz más viva sobre su camino. ¿Y no es acaso también la fe la que inspiró a Miguel Ángel las obras de arte contenidas en esta Capilla, que suscitan la admiración de los hombres de generación en generación?

Pues este enriquecimiento de la cultura es al mismo tiempo un estupendo principio de unión: una civilización cristiana que madura en un país, significa el ingreso de este país en la gran familia donde una misma fe pone en comunión las inteligencias, los corazones y las voluntades. No se terminaría nunca de enumerar los maravillosos desarrollos que jalonan la historia de la civilización. ¿Y todo esto qué es sino, en definitiva, la consecuencia de la Encarnación?

De estos amplios frescos que pueden evocar la historia de la Iglesia, hay que volver al hombre que es su protagonista y su artífice. En el interior del hombre, en su alma, en su psicología, hay que captar las armonías de la fe y de la inteligencia.

La Encarnación puede parecer ante todo, a la inteligencia humana, un peso muy difícil de llevar. Santo Tomás lo dice sin rodeos: de todas las obras divinas es la que más sobrepasa a la razón humana: porque no se puede imaginar -- dice Santo Tomás -- nada más admirable (\emph{Contra Gentiles}, 4, 27). ¿A quién, en efecto, sé le hubiera ocurrido que Dios un día se habría hecho hombre.

Pero esta sublime verdad no encandila al espíritu que la recibe con humildad; antes bien, lo ilumina con la luz nueva y superior. A esta luz el hombre comprende su destino, ve la razón de su existencia, la posibilidad de salir de la miseria, de alcanzar el objetivo de sus esfuerzos. También ve el valor de las creaturas, la ayuda y el obstáculo que éstas pueden constituir para él en su camino hacia Dios. Aquí también, y sobre todo aquí, el misterio de Navidad ejerce su acción unificadora. Y, escrutándolo más profundamente, el creyente no halla por cierto una explicación entre tantas del destino del hombre sino la explicación definitiva: ¡no hay más que un Cristo, no hay más que una salvación! Y tal salvación, lejos de estar reservada a una nación privilegiada, se ofrece a todos. El alma del creyente se siente entonces penetrada por un sentimiento de fraternidad universal; comprende en qué radica la verdadera unidad de destino de la humanidad, que está en el designio de Dios que nos manifestó la Encarnación; comprende el principio fundamental del hombre con Dios y de los hombres entre sí; Navidad se ha vuelto para esa alma lo que es: más que un misterio de unión, un misterio de unidad.

¿Y de dónde procede o dónde tiene su fuente ese misterio? Digámoslo con una palabra que explica todo: es el efecto del amor Este medio divino de unificar al hombre en sí mismo y de unificar al género humano alrededor del Dios hecho hombre, no es y no puede ser una determinación impuesta por la fuerza, a la cual fuera imposible substraerse. Así pues la fe es propuesta y no impuesta. Dios respeta demasiado a su creatura, a la que hizo libre, no esclava. Si la fe y la inteligencia son amigas, ¡cuánto más lo serán la fe y la libertad! ¿Qué valor tendría un amor si fuera una obligación y no una elección?

Así el Infante del pesebre nos revela la última palabra del misterio: Dios se ha encarnado porque amó al hombre y porque quiso salvarlo. Al amor se lo puede aceptar o rechazar. Pero si se lo acepta, produce en el corazón una paz y un gozo indescriptibles: Pax hominibus bonae voluntatis !Quiera Dios, hecho hombre, abrir en esta noche nuestras inteligencias y nuestros corazones para que \textquote{conociendo a Dios visiblemente, seamos atraídos por su intermedio hacia el amor de las cosas invisibles: «ut dum visibiliter Deum cognoscimus, per huno in invisibilium amorem rapiamur!} (Prefacio de Navidad). Amén.

\subsubsection{Homilía (2007)}

MISA DE NOCHEBUENA

\textbf{SOLEMNIDAD DE LA NATIVIDAD DEL SEÑOR}

\textbf{\emph{HOMILÍA DEL SANTO PADRE BENEDICTO XVI}}

\emph{Basílica Vaticana\\ 25 de diciembre de 2007}

~

\emph{Queridos hermanos y hermanas}:

\textquote{A María le llegó el tiempo del parto y dio a luz a su hijo primogénito, lo envolvió en pañales y lo acostó en un pesebre, porque no tenían sitio en la posada} (cf. \emph{Lc} 2,6s). Estas frases, nos llegan al corazón siempre de nuevo. Llegó el momento anunciado por el Ángel en Nazaret: \textquote{Darás a luz un hijo, y le pondrás por nombre Jesús. Será grande, se llamará Hijo del Altísimo} (\emph{Lc} 1,31). Llegó el momento que Israel esperaba desde hacía muchos siglos, durante tantas horas oscuras, el momento en cierto modo esperado por toda la humanidad con figuras todavía confusas: que Dios se preocupase por nosotros, que saliera de su ocultamiento, que el mundo alcanzara la salvación y que Él renovase todo. Podemos imaginar con cuánta preparación interior, con cuánto amor, esperó María aquella hora. El breve inciso, \textquote{lo envolvió en pañales}, nos permite vislumbrar algo de la santa alegría y del callado celo de aquella preparación. Los pañales estaban dispuestos, para que el niño se encontrara bien atendido. Pero en la posada no había sitio. En cierto modo, la humanidad espera a Dios, su cercanía. Pero cuando llega el momento, no tiene sitio para Él. Está tan ocupada consigo misma de forma tan exigente, que necesita todo el espacio y todo el tiempo para sus cosas y ya no queda nada para el otro, para el prójimo, para el pobre, para Dios. Y cuanto más se enriquecen los hombres, tanto más llenan todo de sí mismos y menos puede entrar el otro.

Juan, en su Evangelio, fijándose en lo esencial, ha profundizado en la breve referencia de san Lucas sobre la situación de Belén: \textquote{Vino a su casa, y los suyos no lo recibieron} (1,11). Esto se refiere sobre todo a Belén: el Hijo de David fue a su ciudad, pero tuvo que nacer en un establo, porque en la posada no había sitio para él. Se refiere también a Israel: el enviado vino a los suyos, pero no lo quisieron. En realidad, se refiere a toda la humanidad: Aquel por el que el mundo fue hecho, el Verbo creador primordial entra en el mundo, pero no se le escucha, no se le acoge.

En definitiva, estas palabras se refieren a nosotros, a cada persona y a la sociedad en su conjunto. ¿Tenemos tiempo para el prójimo que tiene necesidad de nuestra palabra, de mi palabra, de mi afecto? ¿Para aquel que sufre y necesita ayuda? ¿Para el prófugo o el refugiado que busca asilo? ¿Tenemos tiempo y espacio para Dios? ¿Puede entrar Él en nuestra vida? ¿Encuentra un lugar en nosotros o tenemos ocupado todo nuestro pensamiento, nuestro quehacer, nuestra vida, con nosotros mismos?

Gracias a Dios, la noticia negativa no es la única ni la última que hallamos en el Evangelio. De la misma manera que en \emph{Lucas} encontramos el amor de su madre María y la fidelidad de san José, la vigilancia de los pastores y su gran alegría, y en \emph{Mateo} encontramos la visita de los sabios Magos, llegados de lejos, así también nos dice \emph{Juan}: \textquote{Pero a cuantos lo recibieron, les da poder para ser hijos de Dios} (\emph{Jn} 1,12). Hay quienes lo acogen y, de este modo, desde fuera, crece silenciosamente, comenzando por el establo, la nueva casa, la nueva ciudad, el mundo nuevo. El mensaje de Navidad nos hace reconocer la oscuridad de un mundo cerrado y, con ello, se nos muestra sin duda una realidad que vemos cotidianamente. Pero nos dice también que Dios no se deja encerrar fuera. Él encuentra un espacio, entrando tal vez por el establo; hay hombres que ven su luz y la transmiten. Mediante la palabra del Evangelio, el Ángel nos habla también a nosotros y, en la sagrada liturgia, la luz del Redentor entra en nuestra vida. Si somos pastores o sabios, la luz y su mensaje nos llaman a ponernos en camino, a salir de la cerrazón de nuestros deseos e intereses para ir al encuentro del Señor y adorarlo. Lo adoramos abriendo el mundo a la verdad, al bien, a Cristo, al servicio de cuantos están marginados y en los cuales Él nos espera.

En algunas representaciones navideñas de la Baja Edad media y de comienzo de la Edad Moderna, el pesebre se representa como edificio más bien desvencijado. Se puede reconocer todavía su pasado esplendor, pero ahora está deteriorado, sus muros en ruinas; se ha convertido justamente en un establo. Aunque no tiene un fundamento histórico, esta interpretación metafórica expresa sin embargo algo de la verdad que se esconde en el misterio de la Navidad. El trono de David, al que se había prometido una duración eterna, está vacío. Son otros los que dominan en Tierra Santa. José, el descendiente de David, es un simple artesano; de hecho, el palacio se ha convertido en una choza. David mismo había comenzado como pastor. Cuando Samuel lo buscó para ungirlo, parecía imposible y contradictorio que un joven pastor pudiera convertirse en el portador de la promesa de Israel. En el establo de Belén, precisamente donde estuvo el punto de partida, vuelve a comenzar la realeza davídica de un modo nuevo: en aquel niño envuelto en pañales y acostado en un pesebre. El nuevo trono desde el cual este David atraerá hacia sí el mundo es la Cruz. El nuevo trono ---la Cruz--- corresponde al nuevo inicio en el establo. Pero justamente así se construye el verdadero palacio davídico, la verdadera realeza. Así, pues, este nuevo palacio no es como los hombres se imaginan un palacio y el poder real. Este nuevo palacio es la comunidad de cuantos se dejan atraer por el amor de Cristo y con Él llegan a ser un solo cuerpo, una humanidad nueva. El poder que proviene de la Cruz, el poder de la bondad que se entrega, ésta es la verdadera realeza. El establo se transforma en palacio; precisamente a partir de este inicio, Jesús edifica la nueva gran comunidad, cuya palabra clave cantan los ángeles en el momento de su nacimiento: \textquote{Gloria a Dios en el cielo y en la tierra paz a los hombres que Dios ama}, hombres que ponen su voluntad en la suya, transformándose en hombres de Dios, hombres nuevos, mundo nuevo.

Gregorio de Nisa ha desarrollado en sus homilías navideñas la misma temática partiendo del mensaje de Navidad en el \emph{Evangelio de Juan: \textquote{}Y puso su morada entre nosotros} (\emph{Jn} 1,14). Gregorio aplica esta palabra de la morada a nuestro cuerpo, deteriorado y débil; expuesto por todas partes al dolor y al sufrimiento. Y la aplica a todo el cosmos, herido y desfigurado por el pecado. ¿Qué habría dicho si hubiese visto las condiciones en las que hoy se encuentra la tierra a causa del abuso de las fuentes de energía y de su explotación egoísta y sin ningún reparo? Anselmo de Canterbury, casi de manera profética, describió con antelación lo que nosotros vemos hoy en un mundo contaminado y con un futuro incierto: \textquote{Todas las cosas se encontraban como muertas, al haber perdido su innata dignidad de servir al dominio y al uso de aquellos que alaban a Dios, para lo que habían sido creadas; se encontraban aplastadas por la opresión y como descoloridas por el abuso que de ellas hacían los servidores de los ídolos, para los que no habían sido creadas} (\emph{PL} 158, 955s). Así, según la visión de Gregorio, el establo del mensaje de Navidad representa la tierra maltratada. Cristo no reconstruye un palacio cualquiera. Él vino para volver a dar a la creación, al cosmos, su belleza y su dignidad: esto es lo que comienza con la Navidad y hace saltar de gozo a los ángeles. La tierra queda restablecida precisamente por el hecho de que se abre a Dios, que recibe nuevamente su verdadera luz y, en la sintonía entre voluntad humana y voluntad divina, en la unificación de lo alto con lo bajo, recupera su belleza, su dignidad. Así, pues, Navidad es la fiesta de la creación renovada. Los Padres interpretan el canto de los ángeles en la Noche santa a partir de este contexto: se trata de la expresión de la alegría porque lo alto y lo bajo, cielo y tierra, se encuentran nuevamente unidos; porque el hombre se ha unido nuevamente a Dios. Para los Padres, forma parte del canto navideño de los ángeles el que ahora ángeles y hombres canten juntos y, de este modo, la belleza del cosmos se exprese en la belleza del canto de alabanza. El canto litúrgico ---siempre según los Padres--- tiene una dignidad particular porque es un cantar junto con los coros celestiales. El encuentro con Jesucristo es lo que nos hace capaces de escuchar el canto de los ángeles, creando así la verdadera música, que acaba cuando perdemos este cantar juntos y este sentir juntos.

En el establo de Belén el cielo y la tierra se tocan. El cielo vino a la tierra. Por eso, de allí se difunde una luz para todos los tiempos; por eso, de allí brota la alegría y nace el canto. Al final de nuestra meditación navideña quisiera citar una palabra extraordinaria de san Agustín. Interpretando la invocación de la oración del Señor: \textquote{Padre nuestro que estás en los cielos}, él se pregunta: ¿qué es esto del cielo? Y ¿dónde está el cielo? Sigue una respuesta sorprendente: Que estás en los cielos significa: en los santos y en los justos. \textquote{En verdad, Dios no se encierra en lugar alguno. Los cielos son ciertamente los cuerpos más excelentes del mundo, pero, no obstante, son cuerpos, y no pueden ellos existir sino en algún espacio; mas, si uno se imagina que el lugar de Dios está en los cielos, como en regiones superiores del mundo, podrá decirse que las aves son de mejor condición que nosotros, porque viven más próximas a Dios. Por otra parte, no está escrito que Dios está cerca de los hombres elevados, o sea de aquellos que habitan en los montes, sino que fue escrito en el Salmo: \textquote{El Señor está cerca de los que tienen el corazón atribulado} (\emph{Sal} 34 {[}33{]}, 19), y la tribulación propiamente pertenece a la humildad. Mas así como el pecador fue llamado \textquote{tierra}, así, por el contrario, el justo puede llamarse \textquote{cielo}} (\emph{Serm. in monte} II 5,17). El cielo no pertenece a la geografía del espacio, sino a la geografía del corazón. Y el corazón de Dios, en la Noche santa, ha descendido hasta un establo: la humildad de Dios es el cielo. Y si salimos al encuentro de esta humildad, entonces tocamos el cielo. Entonces, se renueva también la tierra. Con la humildad de los pastores, pongámonos en camino, en esta Noche santa, hacia el Niño en el establo. Toquemos la humildad de Dios, el corazón de Dios. Entonces su alegría nos alcanzará y hará más luminoso el mundo. Amén.

\subsubsection{Urbi et Orbi (2013)} \emph{Miércoles 25 de diciembre de 2013}


~

\emph{\textquote{Gloria a Dios en el cielo,\\ y en la tierra paz a los hombres que Dios ama }} (\emph{Lc} 2,14).

Queridos hermanos y hermanas de Roma y del mundo entero, ¡buenos días y feliz Navidad!

Hago mías las palabras del cántico de los ángeles, que se aparecieron a los pastores de Belén la noche de la Navidad. Un cántico que une cielo y tierra, elevando al cielo la alabanza y la gloria y saludando a la tierra de los hombres con el deseo de la paz.

Les invito a todos a hacer suyo este cántico, que es el de cada hombre y mujer que vigila en la noche, que espera un mundo mejor, que se preocupa de los otros, intentado hacer humildemente su propio deber.

\emph{Gloria a Dios.}

A esto nos invita la Navidad en primer lugar: a dar gloria a Dios, porque es bueno, fiel, misericordioso. En este día mi deseo es que todos puedan conocer el verdadero rostro de Dios, el Padre que nos ha dado a Jesús. Me gustaría que todos pudieran sentir a Dios cerca, sentirse en su presencia, que lo amen, que lo adoren.

Y que todos nosotros demos gloria a Dios, sobre todo, con la vida, con una vida entregada por amor a Él y a los hermanos.

\emph{Paz a los hombres.}

La verdadera paz -- como sabemos -- no es un equilibrio de fuerzas opuestas. No es pura \textquote{fachada}, que esconde luchas y divisiones. La paz es un compromiso cotidiano, y la paz es también~ artesanal, que se logra contando con el don de Dios, con la gracia que nos ha dado en Jesucristo.

Viendo al Niño en el Belén, niño de paz, pensemos en los niños que son las víctimas más vulnerables de las guerras, pero pensemos también en los ancianos, en las mujeres maltratadas, en los enfermos\ldots{} ¡Las guerras destrozan tantas vidas y causan tanto sufrimiento!

Demasiadas ha destrozado en los últimos tiempos el conflicto de Siria, generando odios y venganzas. Sigamos rezando al Señor para que el amado pueblo sirio se vea libre de más sufrimientos y las partes en conflicto pongan fin a la violencia y garanticen el acceso a la ayuda humanitaria. Hemos podido comprobar la fuerza de la oración. Y me alegra que hoy se unan a nuestra oración por la paz en Siria creyentes de diversas confesiones religiosas. No perdamos nunca la fuerza de la oración. La fuerza para decir a Dios: Señor, concede tu paz a Siria y al mundo entero. E invito también a los no creyentes a desear la paz, con su deseo, ese deseo que ensancha el corazón: todos unidos, con la oración o con el deseo. Pero todos, por la paz.

Concede la paz, Niño, a la República Centroafricana, a menudo olvidada por los hombres. Pero tú, Señor, no te olvidas de nadie. Y quieres que reine la paz también en aquella tierra, atormentada por una espiral de violencia y de miseria, donde muchas personas carecen de techo, agua y alimento, sin lo mínimo indispensable para vivir. Que se afiance la concordia en Sudán del Sur, donde las tensiones actuales ya han provocado demasiadas víctimas y amenazan la pacífica convivencia de este joven Estado.

Tú, Príncipe de la paz, convierte el corazón de los violentos, allá donde se encuentren, para que depongan las armas y emprendan el camino del diálogo. Vela por Nigeria, lacerada por continuas violencias que no respetan ni a los inocentes e indefensos. Bendice la tierra que elegiste para venir al mundo y haz que lleguen a feliz término las negociaciones de paz entre israelíes y palestinos. Sana las llagas de la querida tierra de Iraq, azotada todavía por frecuentes atentados.

Tú, Señor de la vida, protege a cuantos sufren persecución a causa de tu nombre. Alienta y conforta a los desplazados y refugiados, especialmente en el Cuerno de África y en el este de la República Democrática del Congo. Haz que los emigrantes, que buscan una vida digna, encuentren acogida y ayuda. Que no asistamos de nuevo a tragedias como las que hemos visto este año, con los numerosos muertos en Lampedusa.

Niño de Belén, toca el corazón de cuantos están involucrados en la trata de seres humanos, para que se den cuenta de la gravedad de este delito contra la humanidad. Dirige tu mirada sobre los niños secuestrados, heridos y asesinados en los conflictos armados, y sobre los que se ven obligados a convertirse en soldados, robándoles su infancia.

Señor, del cielo y de la tierra, mira a nuestro planeta, que a menudo la codicia y el egoísmo de los hombres explota indiscriminadamente. Asiste y protege a cuantos son víctimas de los desastres naturales, sobre todo al querido pueblo filipino, gravemente afectado por el reciente tifón.

Queridos hermanos y hermanas, en este mundo, en esta humanidad hoy ha nacido el Salvador, Cristo el Señor. No pasemos de largo ante el Niño de Belén. Dejemos que nuestro corazón se conmueva: no tengamos miedo de esto. No tengamos miedo de que nuestro corazón se conmueva. Tenemos necesidad de que nuestro corazón se conmueva. Dejémoslo que se inflame con la ternura de Dios; necesitamos sus caricias. Las caricias de Dios no producen heridas: las caricias de Dios nos dan paz y fuerza. Tenemos necesidad de sus caricias. El amor de Dios es grande; a Él la gloria por los siglos. Dios es nuestra paz: pidámosle que nos ayude a construirla cada día, en nuestra vida, en nuestras familias, en nuestras ciudades y naciones, en el mundo entero. Dejémonos conmover por la bondad de Dios.

\begin{center}\rule{0.5\linewidth}{\linethickness}\end{center}

\textbf{\emph{FELICITACIÓN NAVIDEÑA TRAS EL MENSAJE URBI ET ORBI}}

A todos ustedes, queridos hermanos y hermanas, venidos de todas partes del mundo a esta Plaza, y a cuantos desde distintos países se unen a nosotros a través de los medios de comunicación social, les deseo Feliz Navidad.

En este día, iluminado por la esperanza evangélica que proviene de la humilde gruta de Belén, pido para todos ustedes el don navideño de la alegría y de la paz: para los niños y los ancianos, para los jóvenes y las familias, para los pobres y marginados. Que Jesús, que vino a este mundo por nosotros, consuele a los que pasan por la prueba de la enfermedad y el sufrimiento y sostenga a los que se dedican al servicio de los hermanos más necesitados. ¡Feliz Navidad a todos!

\subsubsection{Urbi et Orbi (2016)}

\emph{Queridos hermanos y hermanas, feliz Navidad.}

Hoy la Iglesia revive el asombro de la Virgen María, de san José y de los pastores de Belén, contemplando al Niño que ha nacido y que está acostado en el pesebre: Jesús, el Salvador.

En este día lleno de luz, resuena el anuncio del Profeta:

\textquote{Un niño nos ha nacido,\\ un hijo se nos ha dado:\\ lleva a hombros el principado, y es su nombre:\\ Maravilla del Consejero,\\ Dios guerrero,\\ Padre perpetuo,\\ Príncipe de la paz} (\emph{Is} 9, 5).

El poder de un Niño, Hijo de Dios y de María, no es el poder de este mundo, basado en la fuerza y en la riqueza, es el poder del amor. Es el poder que creó el cielo y la tierra, que da vida a cada criatura: a los minerales, a las plantas, a los animales; es la fuerza que atrae al hombre y a la mujer, y hace de ellos una sola carne, una sola existencia; es el poder que regenera la vida, que perdona las culpas, reconcilia a los enemigos, transforma el mal en bien. Es el poder de Dios. Este poder del amor ha llevado a Jesucristo a despojarse de su gloria y a hacerse hombre; y lo conducirá a dar la vida en la cruz y a resucitar de entre los muertos. Es el poder del servicio, que instaura en el mundo el reino de Dios, reino de justicia y de paz.

Por esto el nacimiento de Jesús está acompañado por el canto de los ángeles que anuncian:

\textquote{Gloria a Dios en el cielo,\\ y en la tierra paz a los hombres que Dios ama} (\emph{Lc} 2,14).

Hoy este anuncio recorre toda la tierra y quiere llegar a todos los pueblos, especialmente los golpeados por la guerra y por conflictos violentos, y que sienten fuertemente el deseo de la paz.

Paz a los hombres y a las mujeres de la martirizada Siria, donde demasiada sangre ha sido derramada. Sobre todo en la ciudad de Alepo, escenario, en las últimas semanas, de una de las batallas más atroces, es muy urgente que, respetando el derecho humanitario, se garanticen asistencia y consolación a la extenuada población civil, que se encuentra todavía en una situación desesperada y de gran sufrimiento y miseria. Es hora de que las armas callen definitivamente y la comunidad internacional se comprometa activamente para que se logre una solución negociable y se restablezca la convivencia civil en el País.

Paz para las mujeres y para los hombres de la amada Tierra Santa, elegida y predilecta por Dios. Que los israelíes y los palestinos tengan la valentía y la determinación de escribir una nueva página de la historia, en la que el odio y la venganza cedan el lugar a la voluntad de construir conjuntamente un futuro de recíproca comprensión y armonía. Que puedan recobrar unidad y concordia Irak, Libia, Yemen, donde las poblaciones sufren la guerra y brutales acciones terroristas.

Paz a los hombres y mujeres en las diferentes regiones de África, particularmente en Nigeria, donde el terrorismo fundamentalista explota también a los niños para perpetrar el horror y la muerte. Paz en Sudán del Sur y en la República Democrática del Congo, para que se curen las divisiones y para que todos las personas de buena voluntad se esfuercen para iniciar nuevos caminos de desarrollo y de compartir, prefiriendo la cultura del diálogo a la lógica del enfrentamiento.

Paz a las mujeres y hombres que todavía padecen las consecuencias del conflicto en Ucrania oriental, donde es urgente una voluntad común para llevar alivio a la población y poner en práctica los compromisos asumidos.

Pedimos concordia para el querido pueblo colombiano, que desea cumplir un nuevo y valiente camino de diálogo y de reconciliación. Dicha valentía anime también la amada Venezuela para dar los pasos necesarios con vistas a poner fin a las tensiones actuales y a edificar conjuntamente un futuro de esperanza para la población entera.

Paz a todos los que, en varias zonas, están afrontando sufrimiento a causa de peligros constantes e injusticias persistentes. Que Myanmar pueda consolidar los esfuerzos para favorecer la convivencia pacífica y, con la ayuda de la comunidad internacional, pueda dar la necesaria protección y asistencia humanitaria a los que tienen necesidad extrema y urgente. Que pueda la península coreana ver superadas las tensiones que la atraviesan en un renovado espíritu de colaboración.

Paz a quien ha sido herido o ha perdido a un ser querido debido a viles actos de terrorismo que han sembrado miedo y muerte en el corazón de tantos países y ciudades. Paz ---no de palabra, sino eficaz y concreta--- a nuestros hermanos y hermanas que están abandonados y excluidos, a los que sufren hambre y los que son víctimas de violencia. Paz a los prófugos, a los emigrantes y refugiados, a los que hoy son objeto de la trata de personas. Paz a los pueblos que sufren por las ambiciones económicas de unos pocos y la avaricia voraz del dios dinero que lleva a la esclavitud. Paz a los que están marcados por el malestar social y económico, y a los que sufren las consecuencias de los terremotos u otras catástrofes naturales.

Y paz a los niños, en este día especial en el que Dios se hace niño, sobre todo a los privados de la alegría de la infancia a causa del hambre, de las guerras y del egoísmo de los adultos.

Paz sobre la tierra a todos los hombres de buena voluntad, que cada día trabajan, con discreción y paciencia, en la familia y en la sociedad para construir un mundo más humano y más justo, sostenidos por la convicción de que sólo con la paz es posible un futuro más próspero para todos.

Queridos hermanos y hermanas:

\textquote{Un niño nos ha nacido, un hijo se nos ha dado}: es el \textquote{Príncipe de la paz}. Acojámoslo.

\subsubsection{Urbi et Orbi (2019)} \emph{Balcón central de la Basílica Vaticana\\ Miércoles, 25 de diciembre de 2019}



~

\textquote{El pueblo que caminaba en tinieblas vio una luz grande} (\emph{Is} 9,1)

\emph{Queridos hermanos y hermanas: ¡Feliz Navidad!}

En el seno de la madre Iglesia, esta noche ha nacido nuevamente el Hijo de Dios hecho hombre. Su nombre es Jesús, que significa Dios salva. El Padre, Amor eterno e infinito, lo envió al mundo no para condenarlo, sino para salvarlo (cf. \emph{Jn} 3,17). El Padre lo dio, con inmensa misericordia. Lo entregó para todos. Lo dio para siempre. Y Él nació, como pequeña llama encendida en la oscuridad y en el frío de la noche.

Aquel Niño, nacido de la Virgen María, es la Palabra de Dios hecha carne. La Palabra que orientó el corazón y los pasos de Abrahán hacia la tierra prometida, y sigue atrayendo a quienes confían en las promesas de Dios. La Palabra que guio a los hebreos en el camino de la esclavitud a la libertad, y continúa llamando a los esclavos de todos los tiempos, también hoy, a salir de sus prisiones. Es Palabra, más luminosa que el sol, encarnada en un pequeño hijo del hombre, Jesús, luz del mundo.

Por esto el profeta exclama: \textquote{El pueblo que caminaba en tinieblas vio una luz grande} (\emph{Is} 9,1). Sí, hay tinieblas en los corazones humanos, pero más grande es la luz de Cristo. Hay tinieblas en las relaciones personales, familiares, sociales, pero más grande es la luz de Cristo. Hay tinieblas en los conflictos económicos, geopolíticos y ecológicos, pero más grande es la luz de Cristo.

Que Cristo sea luz para tantos niños que sufren la guerra y los conflictos en Oriente Medio y en diversos países del mundo. Que sea consuelo para el amado pueblo sirio, que todavía no ve el final de las hostilidades que han desgarrado el país en este decenio. Que remueva las conciencias de los hombres de buena voluntad. Que inspire hoy a los gobernantes y a la comunidad internacional para encontrar soluciones que garanticen la seguridad y la convivencia pacífica de los pueblos de la región y ponga fin a sus sufrimientos. Que sea apoyo para el pueblo libanés, de este modo pueda salir de la crisis actual y descubra nuevamente su vocación de ser un mensaje de libertad y de armoniosa coexistencia para todos.

Que el Señor Jesús sea luz para la Tierra Santa donde Él nació, Salvador del mundo, y donde continúa la espera de tantos que, incluso en la fatiga, pero sin desesperarse, aguardan días de paz, de seguridad y de prosperidad. Que sea consolación para Irak, atravesado por tensiones sociales, y para Yemen, probado por una grave crisis humanitaria.

Que el pequeño Niño de Belén sea esperanza para todo el continente americano, donde diversas naciones están pasando un período de agitaciones sociales y políticas. Que reanime al querido pueblo venezolano, probado largamente por tensiones políticas y sociales, y no le haga faltar el auxilio que necesita. Que bendiga los esfuerzos de cuantos se están prodigando para favorecer la justicia y la reconciliación, y se desvelan para superar las diversas crisis y las numerosas formas de pobreza que ofenden la dignidad de cada persona.

Que el Redentor del mundo sea luz para la querida Ucrania, que aspira a soluciones concretas para alcanzar una paz duradera.

Que el Señor recién nacido sea luz para los pueblos de África, donde perduran situaciones sociales y políticas que a menudo obligan a las personas a emigrar, privándolas de una casa y de una familia. Que haya paz para la población que vive en las regiones orientales de la República Democrática del Congo, martirizada por conflictos persistentes. Que sea consuelo para cuantos son perseguidos a causa de su fe, especialmente los misioneros y los fieles secuestrados, y para cuantos caen víctimas de ataques por parte de grupos extremistas, sobre todo en Burkina Faso, Malí, Níger y Nigeria.

Que el Hijo de Dios, que bajó del cielo a la tierra, sea defensa y apoyo para cuantos, a causa de estas y otras injusticias, deben emigrar con la esperanza de una vida segura. La injusticia los obliga a atravesar desiertos y mares, transformados en cementerios. La injusticia los fuerza a sufrir abusos indecibles, esclavitudes de todo tipo y torturas en campos de detención inhumanos. La injusticia les niega lugares donde podrían tener la esperanza de una vida digna y les hace encontrar muros de indiferencia.

Que el Emmanuel sea luz para toda la humanidad herida. Que ablande nuestro corazón, a menudo endurecido y egoísta, y nos haga instrumentos de su amor. Que, a través de nuestros pobres rostros, regale su sonrisa a los niños de todo el mundo, especialmente a los abandonados y a los que han sufrido a causa de la violencia. Que, a través de nuestros brazos débiles, vista a los pobres que no tienen con qué cubrirse, dé el pan a los hambrientos, cure a los enfermos. Que, por nuestra frágil compañía, esté cerca de las personas ancianas y solas, de los migrantes y de los marginados. Que, en este día de fiesta, conceda su ternura a todos, e ilumine las tinieblas de este mundo.

\section{Misa de la Aurora} \subsubsection{Catequesis (1983)} \textbf{\emph{AUDIENCIA GENERAL\\ }}\\ \emph{Miércoles 28 de diciembre de 1983}

~

1. El misterio de Navidad hace resonar en nuestros oídos el canto con que el cielo quiere hacer participar a la tierra en el gran acontecimiento de la Encarnación: \textquote{Gloria a Dios en las alturas y paz en la tierra a los hombres de buena voluntad } (\emph{Lc} 2, 14).

\emph{La paz es anunciada por toda la tierra}. No es una paz que los hombres consigan conquistar con sus fuerzas. \emph{Viene de lo alto} como don maravilloso de Dios a la humanidad. No podemos olvidar que, si todos debemos trabajar para instaurar la paz en el mundo, antes de nada debemos abrirnos al don divino de la paz poniendo toda nuestra confianza en el Señor.

Según el cántico de Navidad, la paz prometida a la tierra \emph{está ligada al amor que Dios trae a los hombres}. Los hombres son llamados \textquote{hombres de buena voluntad} porque ya la buena voluntad divina les pertenece. El nacimiento de Jesús es el testimonio irrefutable y definitivo de esta buena voluntad que jamás será retirada de la humanidad.

Este nacimiento pone de manifiesto \emph{la voluntad divina de reconciliación}: Dios desea reconciliar consigo al mundo pecador, perdonando y cancelando los pecados. Ya en el anuncio del nacimiento el ángel había expresado esta voluntad reconciliadora indicando el nombre que debía llevar el Niño: Jesús, o sea, \textquote{Dios salva}. \textquote{Porque salvará a su pueblo de sus pecados}, comenta el ángel (Mt 1, 21). El nombre revela el destino y la misión del Niño juntamente con su personalidad: es el Dios que salva, el que libera a la humanidad de la esclavitud del pecado y, por ello, restablece las relaciones amistosas del hombre con Dios.

2. El acontecimiento que da a la humanidad un Dios Salvador supera en gran medida las expectativas del pueblo judío. Este pueblo esperaba la salvación, esperaba al Mesías, a un rey ideal del futuro que debía establecer sobre la tierra el reino de Dios. A pesar de que la esperanza judía había puesto muy en lo alto a este Mesías, para ellos no era más que un hombre.

La gran novedad de la venida del Salvador consiste en el hecho de que Él es Dios y hombre a la vez. Lo que el judaísmo no había podido concebir ni esperar, es decir, un Hijo de Dios hecho hombre, se realiza en el misterio de la Encarnación. \emph{El cumplimiento es mucho más maravilloso que la promesa}.

Esta es la razón por la que no podemos medir la grandeza de Jesús sólo con los oráculos proféticos del Antiguo Testamento. Cuando Él realiza estos oráculos se mueve a un nivel trascendente. Todos los tentativos de encerrar a Jesús en los límites de una personalidad humana, no tienen en cuenta lo que hay de esencial en la revelación de la Nueva Alianza: la persona divina del Hijo que se ha hecho hombre o, según la palabra de San Juan, del Verbo que se ha hecho carne y ha venido a habitar entre nosotros (cf. 1, 14). Aquí aparece la grandiosidad generosa del plan divino de salvación. El Padre ha enviado a su Hijo que es Dios como Él. No se ha limitado a enviar a siervos, a hombres que hablasen en su nombre como los Profetas. Ha querido testimoniar a la humanidad el máximo de amor y le ha hecho la sorpresa de darle un Salvador que poseía la omnipotencia divina.

En este Salvador, que es Dios y hombre a la vez, podemos descubrir \emph{la intención de la obra reconciliadora}. El Padre no quiere sólo purificar a la humanidad liberándola del pecado; quiere realizar \emph{la unión más íntima de la divinidad y la humanidad}. En la única persona divina de Jesús, la divinidad y la humanidad están unidas del modo más completo. Él que es perfectamente Dios es perfectamente hombre. Ha realizado en Sí esta unión de la divinidad y la humanidad para poder hacer participar de ella a todos los hombres. Perfectamente hombre, Él, que es Dios, quiere comunicar a sus hermanos humanos una vida divina que les permita ser más perfectamente hombres, reflejando en sí mismos la perfección divina.

3. Un aspecto de la reconciliación merece ser subrayado aquí. Mientras el hombre pecador podía temer para su porvenir las consecuencias de su culpa y esperarse una vida humana disminuida, en cambio recibe de Cristo Salvador \emph{la posibilidad de un completo desarrollo humano}. No sólo es liberado de la esclavitud en la que le aprisionaban sus culpas, sino que puede alcanzar \emph{una perfección humana} superior a la que poseía antes del pecado. Cristo le ofrece una vida humana más abundante y más elevada. Por el hecho de que en Cristo la divinidad no ha comprimido en modo alguno a la humanidad sino que la ha elevado a un grado supremo de desarrollo, con su vida divina comunica a los hombres una vida humana más intensa y completa.

Que Jesús sea el Dios Salvador hecho hombre significa, pues, que ya \emph{en el hombre nada está perdido}. Todo lo que había sido herido, manchado por el pecado, puede revivir y florecer. Esto explica cómo la gracia cristiana favorece el pleno ejercicio de todas las facultades humanas y también la afirmación de toda personalidad, tanto la femenina como la masculina. Reconciliando al hombre con Dios, la religión cristiana tiende a promover todo lo que es humano.

Por tanto, podemos unirnos al canto que resonó en la gruta de Belén y proclamar con los ángeles: \textquote{Gloria a Dios en las alturas y paz en la tierra a los hombres de buena voluntad}.

\subsubsection{Homilía ()}

\subsubsection{Homilía ()}

\section{Navidad: Día} \subsubsection{Urbi et Orbi (2001)} \textbf{\emph{MENSAJE URBI ET ORBI\\ DE SU SANTIDAD JUAN PABLO II}}

\emph{Navidad, 25 de diciembre de 2001}

~

1. \textquote{\emph{Christus est pax nostra}},\\ \textquote{\emph{Cristo es nuestra paz.\\ Él ha hecho de los dos pueblos una sola cosa}} (\emph{Ef} 2, 14).\\ En el alba del nuevo milenio,\\ comenzado con tantas esperanzas,\\ pero ahora amenazado por nubes tenebrosas\\ de violencia y de guerra,\\ las palabras del apóstol Pablo\\ que escuchamos esta Navidad\\ es un rayo de luz penetrante,\\ un clamor de confianza y optimismo.\\ El divino Niño nacido en Belén\\ lleva en sus pequeñas manos, como un don,\\ el secreto de la paz para la humanidad.\\ ¡Él es el Príncipe de la paz!\\ He aquí el gozoso anuncio que se oyó aquella noche en Belén,\\ y que quiero repetir al mundo\\ en este día bendito.\\ Escuchemos una vez más las palabras del ángel:\\ \textquote{\emph{os traigo la buena noticia,\\ la gran alegría para todo el pueblo:\\ hoy, en la ciudad de David,\\ os ha nacido un salvador: el Mesías, el Señor}} (\emph{Lc} 2, 10-11).\\ En el día de hoy, la Iglesia se hace eco de los ángeles,\\ y reitera su extraordinario mensaje,\\ que sorprendió en primer lugar a los pastores\\ en las alturas de Belén.

2. \textquote{\emph{Christus est pax nostra}!}\\ Cristo, el \textquote{\emph{niño envuelto en pañales\\ y acostado en un pesebre}} (\emph{Lc} 2, 10-12),\\ Él es precisamente nuestra paz.\\ Un Niño indefenso, recién nacido en la humildad de una cueva,\\ devuelve la dignidad a cada vida que nace,\\ da esperanza a quien yace en la duda y en el desaliento.\\ Él ha venido para curar a los heridos de la vida\\ y para dar nuevo sentido incluso a la muerte.\\ En aquel Niño, dócil y desvalido,\\ que llora en una gruta fría y destartalada,\\ Dios ha destruido el pecado\\ y ha puesto el germen de una humanidad nueva,\\ llamada a llevar a término\\ el proyecto original de la creación\\ y a transcenderlo con la gracia de la redención.

3. \textquote{\emph{Christus est pax nostra}!}\\ Hombres y mujeres del tercer milenio,\\ vosotros que tenéis hambre de justicia y de paz,\\ ¡acoged el mensaje de Navidad\\ que se propaga hoy por todo el mundo!\\ Jesús ha nacido para consolidar las relaciones\\ entre los hombres y los pueblos,\\ y hacer de todos ellos hermanos en Él.\\ Ha venido para derribar \textquote{el muro que los separaba:\\ el odio} (\emph{Ef} 2, 14),\\ y para hacer de la humanidad una sola familia.\\ Sí, podemos repetir con certeza:\\ ¡Hoy, con el Verbo encarnado, ha nacido la paz!\\ Paz que se ha de implorar,\\ porque sólo Dios es su autor y garante.\\ Paz que se ha de construir\\ en un mundo en el que pueblos y naciones,\\ afectados por tantas y tan diversas dificultades,\\ esperan en una humanidad\\ no sólo globalizada por intereses económicos,\\ sino por el esfuerzo constante\\ en favor de una convivencia más justa y solidaria.

4. Como los pastores, acudamos a Belén,\\ quedémonos en adoración ante la gruta,\\ fijando la mirada en el Redentor recién nacido.\\ En Él podemos reconocer los rasgos\\ de cada pequeño ser humano que viene a la luz,\\ sea cual fuere su raza o nación:\\ es el pequeño palestino y el pequeño israelí;\\ es el bebé estadounidense y el afgano;\\ es el hijo del hutu y el hijo del tutsi\ldots{}\\ es el niño cualquiera, que es alguien para Cristo.\\ Hoy pienso en todos los pequeños del mundo:\\ muchos, demasiados, son los niños\\ que nacen ya condenados a sufrir, sin culpa,\\ las consecuencias de conflictos inhumanos.\\ ¡Salvemos a los niños,\\ para salvar la esperanza de la humanidad!\\ Nos lo pide hoy con fuerza\\ aquel Niño nacido en Belén,\\ el Dios que se hizo hombre,\\ para devolvernos el derecho de esperar.

5. Supliquemos a Cristo el don de la paz\\ para cuantos sufren a causa de conflictos, antiguos y nuevos.\\ Todos los días siento en mi corazón\\ los dramáticos problemas de Tierra Santa;\\ cada día pienso con preocupación\\ en cuantos mueren de hambre y de frío;\\ día tras día me llega, angustiado,\\ el grito de quien, en tantas partes del mundo,\\ invoca una distribución más ecuánime de los recursos\\ y un trabajo dignamente retribuido para todos.\\ ¡Que nadie deje de esperar\\ en el poder del amor de Dios!\\ Que Cristo sea luz y sustento\\ de quien, a veces contracorriente, cree y actúa\\ en favor del encuentro, del diálogo, de la cooperación\\ entre las culturas y las religiones.\\ Que Cristo guíe en la paz los pasos\\ de quien se afana incansablemente\\ por el progreso de la ciencia y la técnica.\\ Que nunca se usen estos grandes dones de Dios\\ contra el respeto y la promoción de la dignidad humana.\\ ¡Que jamás se utilice el nombre santo de Dios\\ para corroborar el odio!\\ ¡Que jamás se haga de Él motivo de intolerancia y violencia!\\ Que el dulce rostro del Niño de Belén\\ recuerde a todos que tenemos un único Padre.

6. \textquote{\emph{Christus est pax nostra}!}\\ Hermanos y hermanas que me escucháis,\\ abrid el corazón a este mensaje de paz,\\ abridlo a Cristo, Hijo de la Virgen María,\\ a Aquel que se ha hecho \textquote{nuestra paz}.\\ Abridlo a Él, que nada nos quita\\ si no es el pecado,\\ y nos da en cambio\\ plenitud de humanidad y de alegría.\\ Y Tú, adorado Niño de Belén,\\ lleva la paz a cada familia y ciudad,\\ a cada nación y continente.\\ ¡Ven, Dios hecho hombre!\\ ¡Ven a ser el corazón del mundo renovado por el amor!\\ ¡Ven especialmente allí donde más peligra\\ la suerte de la humanidad!\\ ¡Ven, y no tardes!\\ ¡Tú eres \textquote{\emph{nuestra paz}}! (\emph{Ef} 2,14).

\subsubsection{Urbi et Orbi (2007)} \textbf{\emph{MENSAJE URBI ET ORBI}}

\emph{Navidad, martes 25 de diciembre de 2007}

~

\emph{\textquote{Nos ha amanecido un día sagrado:\\ venid, naciones, adorad al Señor, porque\\ hoy una gran luz ha bajado a la tierra}\\ (Misa del día de Navidad, Aclamación al Evangelio).}

Queridos hermanos y hermanas:

\textquote{Nos ha amanecido un día sagrado}. Un día de gran esperanza: hoy el Salvador de la humanidad ha nacido. El nacimiento de un niño trae normalmente una luz de esperanza a quienes lo aguardan ansiosos. Cuando Jesús nació en la gruta de Belén, una \textquote{gran luz} apareció sobre la tierra; una gran esperanza entró en el corazón de cuantos lo esperaban: \textquote{\emph{lux magna}}, canta la liturgia de este día de Navidad.

Ciertamente no fue \textquote{grande} según el mundo, porque, en un primer momento, sólo la vieron María, José y algunos pastores, luego los Magos, el anciano Simeón, la profetisa Ana: aquellos que Dios había escogido. Sin embargo, en lo recóndito y en el silencio de aquella noche santa se encendió para cada hombre una luz espléndida e imperecedera; ha venido al mundo la gran esperanza portadora de felicidad: \textquote{el Verbo se hizo carne y nosotros hemos visto su gloria} (\emph{Jn} 1,14)

\textquote{Dios es luz ---afirma san Juan--- y en él no hay tinieblas} (\emph{1} \emph{Jn} 1,5). En el Libro del Génesis leemos que cuando tuvo origen el universo, \textquote{la tierra era un caos informe; sobre la faz del Abismo, la tiniebla}. \textquote{Y dijo Dios: \textquote{que exista la luz}. Y la luz existió} (\emph{Gn} 1,2-3). La Palabra creadora de Dios es Luz, fuente de la vida. Por medio del \emph{Logos} se hizo todo y sin Él no se hizo nada de lo que se ha hecho (cf. \emph{Jn} 1,3). Por eso todas las criaturas son fundamentalmente buenas y llevan en sí la huella de Dios, una chispa de su luz. Sin embargo, cuando Jesús nació de la Virgen María, la Luz misma vino al mundo: \textquote{Dios de Dios, Luz de Luz}, profesamos en el Credo. En Jesús, Dios asumió lo que no era, permaneciendo en lo que era: \textquote{la omnipotencia entró en un cuerpo infantil y no se sustrajo al gobierno del universo} (cf. S. Agustín, \emph{Serm} 184, 1 sobre la Navidad). Aquel que es el creador del hombre se hizo hombre para traer al mundo la paz. Por eso, en la noche de Navidad, el coro de los Ángeles canta: \textquote{Gloria a Dios en el cielo / y en la tierra paz a los hombres que Dios ama} (\emph{Lc} 2,14).

\emph{\textquote{Hoy una gran luz ha bajado a la tierra}}. La Luz de Cristo es portadora de paz. En la Misa de la noche, la liturgia eucarística comenzó justamente con este canto: \textquote{Hoy, desde el cielo, ha descendido la paz sobre nosotros} (\emph{Antífona de entrada}). Más aún, sólo la \textquote{gran} luz que aparece en Cristo puede dar a los hombres la \textquote{verdadera} paz. He aquí por qué cada generación está llamada a acogerla, a acoger al Dios que en Belén se ha hecho uno de nosotros.

La Navidad es esto: acontecimiento histórico y misterio de amor, que desde hace más de dos mil años interpela a los hombres y mujeres de todo tiempo y lugar. Es el día santo en el que brilla la \textquote{gran luz} de Cristo portadora de paz. Ciertamente, para reconocerla, para acogerla, se necesita fe, se necesita humildad. La humildad de María, que ha creído en la palabra del Señor, y que fue la primera que, inclinada ante el pesebre, adoró el Fruto de su vientre; la humildad de José, hombre justo, que tuvo la valentía de la fe y prefirió obedecer a Dios antes que proteger su propia reputación; la humildad de los pastores, de los pobres y anónimos pastores, que acogieron el anuncio del mensajero celestial y se apresuraron a ir a la gruta, donde encontraron al niño recién nacido y, llenos de asombro, lo adoraron alabando a Dios (cf. \emph{Lc} 2,15-20). Los pequeños, los pobres en espíritu: éstos son los protagonistas de la Navidad, tanto ayer como hoy; los protagonistas de siempre de la historia de Dios, los constructores incansables de su Reino de justicia, de amor y de paz.

En el silencio de la noche de Belén Jesús nació y fue acogido por manos solícitas. Y ahora, en esta nuestra Navidad en la que sigue resonando el alegre anuncio de su nacimiento redentor, ¿quién está listo para abrirle las puertas del corazón? Hombres y mujeres de hoy, Cristo viene a traernos la luz también a nosotros, también a nosotros viene a darnos la paz. Pero ¿quién vela en la noche de la duda y la incertidumbre con el corazón despierto y orante? ¿Quién espera la aurora del nuevo día teniendo encendida la llama de la fe? ¿Quién tiene tiempo para escuchar su palabra y dejarse envolver por su amor fascinante? Sí, su mensaje de paz es para todos; viene para ofrecerse a sí mismo a todos como esperanza segura de salvación.

Que la luz de Cristo, que viene a iluminar a todo ser humano, brille por fin y sea consuelo para cuantos viven en las tinieblas de la miseria, de la injusticia, de la guerra; para aquellos que ven negadas aún sus legítimas aspiraciones a una subsistencia más segura, a la salud, a la educación, a un trabajo estable, a una participación más plena en las responsabilidades civiles y políticas, libres de toda opresión y al resguardo de situaciones que ofenden la dignidad humana. Las víctimas de sangrientos~ conflictos armados, del terrorismo y de todo tipo de violencia, que causan sufrimientos inauditos a poblaciones enteras, son especialmente las categorías más vulnerables, los niños, las mujeres y los ancianos. A su vez, las tensiones étnicas, religiosas y políticas, la inestabilidad, la rivalidad, las contraposiciones, las injusticias y las discriminaciones que laceran el tejido interno de muchos países, exasperan las relaciones internacionales. Y en el mundo crece cada vez más el número de emigrantes, refugiados y deportados, también por causa de frecuentes calamidades naturales, como consecuencia a veces de preocupantes desequilibrios ambientales.

En este día de paz, pensemos sobre todo en donde resuena el fragor de las armas: en las martirizadas tierras del \emph{Darfur}, de \emph{Somalia} y del norte de la \emph{República Democrática del Congo}, en las fronteras de \emph{Eritrea} y \emph{Etiopía}, en todo el \emph{Oriente Medio}, en particular en \emph{Irak}, \emph{Líbano} y \emph{Tierra Santa}, en \emph{Afganistán}, en \emph{Pakistán} y en \emph{Sri Lanka}, en las regiones de los \emph{Balcanes}, y en tantas otras situaciones de crisis, desgraciadamente olvidadas con frecuencia. Que el Niño Jesús traiga consuelo a quien vive en la prueba e infunda a los responsables de los gobiernos sabiduría y fuerza para buscar y encontrar soluciones humanas, justas y estables. A la sed de sentido y de valores que hoy se ~percibe en el mundo; a la búsqueda de bienestar y paz que marca la vida de toda la humanidad; a las expectativas de los pobres, responde Cristo, verdadero Dios y verdadero Hombre, con su Natividad. Que las personas y las naciones no teman reconocerlo y acogerlo: con Él, \textquote{una espléndida luz} alumbra el horizonte de la humanidad; con Él comienza \textquote{un día sagrado} que no conoce ocaso. Que esta Navidad sea realmente para todos un día de alegría, de esperanza y de paz.

\emph{\textquote{Venid, naciones, adorad al Señor}.} Con María, José y los pastores, con los Magos y la muchedumbre innumerable de humildes adoradores del Niño recién nacido, que han acogido el misterio de la Navidad a lo largo de los siglos, dejemos también nosotros, hermanos y hermanas de todos los continentes, que la luz de este día se difunda por todas partes, que entre en nuestros corazones, alumbre y dé calor a nuestros hogares, lleve serenidad y esperanza a nuestras ciudades, y conceda al mundo la paz. Éste es mi deseo para quienes me escucháis. Un deseo que se hace oración humilde y confiada al Niño Jesús, para que su luz disipe las tinieblas de vuestra vida y os llene del amor y de la paz. El Señor, que ha hecho resplandecer en Cristo su rostro de misericordia, os colme con su felicidad y os haga mensajeros de su bondad. ¡Feliz Navidad!

\subsubsection{Urbi et Orbi (2010)} \textbf{\emph{MENSAJE URBI ET ORBI}}

\emph{Navidad, 25 de diciembre de 20}10

~

\textquote{\emph{Verbum caro factum est}} - \textquote{El Verbo se hizo carne} (\emph{Jn} 1,14).

Queridos hermanos y hermanas que me escucháis en Roma y en el mundo entero, os anuncio con gozo el mensaje de la Navidad: Dios se ha hecho hombre, ha venido a habitar entre nosotros. Dios no está lejano: está cerca, más aún, es el \textquote{Emmanuel}, el Dios-con-nosotros. No es un desconocido: tiene un rostro, el de Jesús.

Es un mensaje siempre nuevo, siempre sorprendente, porque supera nuestras más audaces esperanzas. Especialmente porque no es sólo un anuncio: es un acontecimiento, un suceso, que testigos fiables han visto, oído y tocado en la persona de Jesús de Nazaret. Al estar con Él, observando lo que hace y escuchando sus palabras, han reconocido en Jesús al Mesías; y, viéndolo resucitado después de haber sido crucificado, han tenido la certeza de que Él, verdadero hombre, era al mismo tiempo verdadero Dios, el Hijo unigénito venido del Padre, lleno de gracia y de verdad (cf. \emph{Jn} 1,14).

\textquote{El Verbo se hizo carne}. Ante esta revelación, vuelve a surgir una vez más en nosotros la pregunta: ¿Cómo es posible? El Verbo y la carne son realidades opuestas; ¿cómo puede convertirse la Palabra eterna y omnipotente en un hombre frágil y mortal? No hay más que una respuesta: el Amor. El que ama quiere compartir con el amado, quiere estar unido a él, y la Sagrada Escritura nos presenta precisamente la gran historia del amor de Dios por su pueblo, que culmina en Jesucristo.

En realidad, Dios no cambia: es fiel a sí mismo. El que ha creado el mundo es el mismo que ha llamado a Abraham y que ha revelado el propio Nombre a Moisés: Yo soy el que soy\ldots{} el Dios de Abraham, Isaac y Jacob\ldots{} Dios misericordioso y piadoso, rico en amor y fidelidad (cf. \emph{Ex} 3,14-15; 34,6). Dios no cambia, desde siempre y por siempre es Amor. Es en sí mismo comunión, unidad en la Trinidad, y cada una de sus obras y palabras tienden a la comunión. La encarnación es la cumbre de la creación. Cuando, por la voluntad del Padre y la acción del Espíritu Santo, se formó en el regazo de María Jesús, Hijo de Dios hecho hombre, la creación alcanzó su cima. El principio ordenador del universo, el \emph{Logos}, comenzó a existir en el mundo, en un tiempo y en un lugar.

\textquote{El Verbo se hizo carne}. La luz de esta verdad se manifiesta a quien la acoge con fe, porque es un misterio de amor. Sólo los que se abren al amor son cubiertos por la luz de la Navidad. Así fue en la noche de Belén, y así también es hoy. La encarnación del Hijo de Dios es un acontecimiento que ha ocurrido en la historia, pero que al mismo tiempo la supera. En la noche del mundo se enciende una nueva luz, que se deja ver por los ojos sencillos de la fe, del corazón manso y humilde de quien espera al Salvador. Si la verdad fuera sólo una fórmula matemática, en cierto sentido se impondría por sí misma. Pero si la Verdad es Amor, pide la fe, el \textquote{sí} de nuestro corazón.

Y, en efecto, ¿qué busca nuestro corazón si no una Verdad que sea Amor? La busca el niño, con sus preguntas tan desarmantes y estimulantes; la busca el joven, necesitado de encontrar el sentido profundo de la propia vida; la busca el hombre y la mujer en su madurez, para orientar y apoyar el compromiso en la familia y en el trabajo; la busca la persona anciana, para dar cumplimiento a la existencia terrenal.

\textquote{El Verbo se hizo carne}. El anuncio de la Navidad es también luz para los pueblos, para el camino conjunto de la humanidad. El \textquote{Emmanuel}, el Dios-con-nosotros, ha venido como Rey de justicia y de paz. Su Reino ---lo sabemos--- no es de este mundo, sin embargo, es más importante que todos los reinos de este mundo. Es como la levadura de la humanidad: si faltara, desaparecería la fuerza que lleva adelante el verdadero desarrollo, el impulso a colaborar por el bien común, al servicio desinteresado del prójimo, a la lucha pacífica por la justicia. Creer en el Dios que ha querido compartir nuestra historia es un constante estímulo a comprometerse en ella, incluso entre sus contradicciones. Es motivo de esperanza para todos aquellos cuya dignidad es ofendida y violada, porque Aquel que ha nacido en Belén ha venido a liberar al hombre de la raíz de toda esclavitud.

Que la luz de la Navidad resplandezca de nuevo en aquella Tierra donde Jesús ha nacido e inspire a israelitas y palestinos a buscar una convivencia justa y pacífica. Que el anuncio consolador de la llegada del Emmanuel alivie el dolor y conforte en las pruebas a las queridas comunidades cristianas en Irak y en todo el Medio Oriente, dándoles aliento y esperanza para el futuro, y animando a los responsables de las Naciones a una solidaridad efectiva para con ellas. Que se haga esto también en favor de los que todavía sufren por las consecuencias del terremoto devastador y la reciente epidemia de cólera en Haití. Y que tampoco se olvide a los que en Colombia y en Venezuela, como también en Guatemala y Costa Rica, han sido afectados por recientes calamidades naturales.

Que el nacimiento del Salvador abra perspectivas de paz duradera y de auténtico progreso a las poblaciones de Somalia, de Darfur y Costa de Marfil; que promueva la estabilidad política y social en Madagascar; que lleve seguridad y respeto de los derechos humanos en Afganistán y Pakistán; que impulse el diálogo entre Nicaragua y Costa Rica; que favorezca la reconciliación en la Península coreana.

Que la celebración del nacimiento del Redentor refuerce el espíritu de fe, paciencia y fortaleza en los fieles de la Iglesia en la China continental, para que no se desanimen por las limitaciones a su libertad de religión y conciencia y, perseverando en la fidelidad a Cristo y a su Iglesia, mantengan viva la llama de la esperanza. Que el amor del \textquote{Dios con nosotros} otorgue perseverancia a todas las comunidades cristianas que sufren discriminación y persecución, e inspire a los líderes políticos y religiosos a comprometerse por el pleno respeto de la libertad religiosa de todos.

Queridos hermanos y hermanas, \textquote{el Verbo se hizo carne}, ha venido a habitar entre nosotros, es el Emmanuel, el Dios que se nos ha hecho cercano. Contemplemos juntos este gran misterio de amor, dejémonos iluminar el corazón por la luz que brilla en la gruta de Belén. ¡Feliz Navidad a todos! \subsubsection{Homilía ()}

\section{1 de enero} \subsubsection{Homilía (1978)} SOLEMNIDAD DE LA MADRE DE DIOS\\ XI JORNADA MUNDIAL DE LA PAZ

\emph{\textbf{HOMILÍA DE SU SANTIDAD PABLO VI\\ EN LA MISA DE LA PAZ}\\[2\baselineskip]Basílica de Santa María la Mayor\protect\hyperlink{ux2a}{*}\\ Domingo 1 de enero de 1978}

~

Convocados por la fe en esta basílica ---erigida por nuestro predecesor Sixto III pocos años después del Concilio de Efeso que había proclamado solemnemente en el año 431 a María la \emph{Theotokos,} es decir, Madre de Dios---, unamos en esta celebración la alabanza por los altísimos privilegios concedidos por Dios a la Virgen Madre, juntamente con la reflexión sobre las exigencias cristianas de la paz en el mundo.

En este espléndido templo, expresión singular de la ferviente devoción mariana del pueblo romano, historia y arte se han fundido admirablemente a través de los siglos; este templo, con su belleza clásica y su atractivo misterioso, nos lleva a pensamientos de alegría serena; en los mosaicos, tan antiguos, refulgen las diversas etapas de la historia de la salvación; en lo alto del ábside resplandece la escena sublime de la \textquote{Coronación de María}, obra de Jacopo Torriti; y junto a los recuerdos de la gruta del Pesebre, los Magos adoran al Verbo encarnado, en la composición escultórica de Arnolfo di Cambio.

Hemos querido celebrar la \textquote{Jornada de la Paz} precisamente en este marco estupendo, creado por la piedad de nuestros antepasados; y desde aquí nos proponemos dirigir una vez más a toda la humanidad las palabras suaves y solemnes de la paz.

La Jornada de la Paz no hace referencia a la paz de un día, de un día solo. Al celebrarla en la primera jornada del año civil, aporta siempre algo al año que comienza: una celebración conjunta que es augurio y promesa al comienzo del calendario; pero presenta también un tema que hemos propuesto nosotros y que resulta ocasión y fuente de convergencia de intenciones con dimensiones universales. Convergencia en la oración, para todos los católicos y para todos los cristianos que quieran unirse a la Jornada; convergencia en el estudio y la reflexión, para los responsables de la guía colectiva de la sociedad y para todos los hombres de buena voluntad; convergencia en una acción conjunta, un testimonio presentado así al mundo a través del esfuerzo solidario para defender a todos los habitantes de nuestro planeta, tan gravemente amenazados en nuestros días por \textquote{el carácter absurdo de la guerra moderna}, según hemos subrayado en nuestro reciente Mensaje, y para construir la paz cuya necesidad perentoria la conciencia de la humanidad siente cada vez más.

Cada uno de los temas de las diferentes \textquote{Jornadas de la Paz} completa a los precedentes, al igual que una piedra se añade a las otras para construir una casa: esta casa de la paz, que se funda ---como decía nuestro venerado predecesor Juan XXIII--- sobre cuatro pilares, \textquote{la verdad, la justicia, la solidaridad operante y la libertad} (cf. \emph{Pacem in terris}, 47).

Pero el pensamiento dominante de esta celebración nuestra se presenta espontáneamente en el binomio \textquote{María y la paz}. ¿Acaso no hay relación entre la Maternidad divina de María y la paz que celebramos el mismo día de su fiesta, una relación que no es accidental, sino que extrae su realidad y fruto de todo el patrimonio dogmático, patrístico, teológico y místico de la Iglesia de Cristo? ¿No es verdad que existe una razón histórica que se añade a éstas y nos reúne hoy con vosotros, carísimos hijas e hijos, romanos de nacimiento o de adopción? En efecto, ¿no venís esta mañana a continuar y confirmar con vuestra presencia la práctica profundamente religiosa y filial de vuestros antepasados, diocesanos de esta Iglesia de Roma, que antes de que esa fecha señalase en Occidente el comienzo del año civil eligió ya la octava de Navidad para rendir homenaje especial a la Madre de Dios? Y en torno a vosotros, ¿no está reunida místicamente toda la Iglesia, todo el Pueblo de Dios en esta Patriarcal Basílica, para celebrar a un tiempo la Maternidad de María y la paz, esa paz que su Hijo Jesucristo vino a traer al mundo?

Pero no es preciso ir muy lejos en nuestra reflexión. Si hay correlación entre la maternidad divina de María y la paz, ¿qué relación hay entre esta maternidad y la repulsa de la violencia que figura en el tema elegido para la jornada de este año de 1978? Sí, existe relación. Los estudios teológicos y exegéticos acerca de este argumento se multiplican, lo subrayan cada vez más en la perspectiva que les es propia, y añaden a sus conclusiones la opinión espontánea del pueblo.

Sea que se contemple la violencia ---como lo hemos hecho en nuestro reciente Mensaje para esta Jornada--- bajo el aspecto colectivo internacional, es decir, bajo el de la guerra moderna que amenaza con su \textquote{suprema irracionalidad}, con su \textquote{carácter absurdo}, y con las tristes hipótesis de una guerra espacial; sea que se la considere bajo los aspectos múltiples de la violencia pasional de la delincuencia creciente, o de la violencia civil erigida en sistema, se plantea una pregunta fundamental: ¿cuáles son las causas de tales comportamientos o de las ideas y sentimientos que los inspiran? Repetidas veces hemos recordado estas causas en nuestros Mensajes precedentes, particularmente en los que tratan sobre el desarme y la defensa de la vida. Esta mañana sólo recordaremos una: la sacudida provocada en la sociedad por condiciones de vida deshumanizadoras (cf. \emph{Gaudium et spes}, 27).

Tales condiciones de vida provocan, sobre todo en los jóvenes, frustraciones que desencadenan reacciones de violencia y agresividad contra ciertas estructuras y coyunturas de la sociedad contemporánea que quisiera reducir a los jóvenes a simples instrumentos pasivos. Pero su contestación, instintiva u organizada, se dirige no sólo a las consecuencias de estas situaciones penosas, sino también a \textquote{una sociedad rebosante de bienestar material, satisfecha y gozosa, pero privada de ideales superiores que dan sentido y valor a la vida} (Mensaje de Navidad; \emph{L'Osservatore Romano,} Edición en Lengua Española, 5 de enero de 1969, pág. 2). En una palabra, una sociedad desacralizada; una sociedad sin alma, una sociedad sin amor.

De hecho, ¿quiénes son frecuentemente estos violentos cuyas acciones, provocando temor u horror, hacen necesario el deber de proteger nuestra convivencia humana? Muy a menudo, con demasiada frecuencia, los que llevan a cabo esas acciones intolerables son personas olvidadas, marginadas, despreciadas, personas que no son amadas o, al menos, no se sienten amadas. Ávidas más de tener que de ser; testigos y con frecuencia víctimas de la injusticia de los más fuertes o, en algunos casos bien conocidos, de la \textquote{violencia estructural de algunos regímenes políticos}; ¿cómo pueden no sentirse \textquote{hijos pródigos} en esta sociedad anónima que los ha engendrado y, luego, con frecuencia abandonado, sin baremo fijo de valores, y en resumen, sin brújula ni estrella, sin la estrella de Navidad?

En el secreto de su corazón estos \textquote{huérfanos}, ¿acaso no aspiran desde los fondos de esta sociedad madrastra a una sociedad materna y, en fin, a la maternidad religiosa de la Madre universal, a la maternidad de María?

Las palabras de Cristo en la cruz, \textquote{Mujer, he ahí a tu hijo} (\emph{Jn} 19, 26-27), ¿no es verdad que iban dirigidas a ellos a través de San Juan: \textquote{Madre, he ahí a tus hijos}? ¿Y acaso no era para ellos la frase del Señor moribundo cuando decía: \textquote{Hijos, he ahí a vuestra madre}, una madre que os ama, una madre a la que amar, una madre situada en el vértice de una sociedad del amor? Es decir, Madre de Dios y del Redentor (\emph{Lumen gentium}, 53), del nuevo Adán en el que y por el que todos los hombres son hermanos (cf. \emph{Rom} 8, 29), María, nueva Eva (cf. \emph{Lumen gentium}, 63), se transforma de este modo en la madre de todos los vivientes (cf. \emph{ib.,} 56), nuestra madre amantísima (\emph{ib}., 53). Miembro eminente y plenamente singular de la Iglesia (\emph{ib.,} 53), \emph{es tipo de la} misma (\emph{ib.}, 63); es imagen y principio de la Iglesia que habrá de tener su cumplimiento en la vida futura (\emph{ib.}, 68).

Aquí se nos presenta una nueva visión que es el reflejo de la Virgen en la Iglesia, como dice San Agustín: María \emph{\textquote{figuram in se sasnctae Ecclesiae demonstrat:} María refleja en sí la figura misma de la Iglesia} \emph{(De Symbolo,} CI; \emph{PL} 661; H. de Lubac, \emph{Méditations sur l'Eglise,} pág. 245).

Madre de Cristo Rey, Príncipe de la Paz (\emph{Is} 9, 6). María se transforma por esto mismo en Reina y Madre de la paz. El Concilio Vaticano II, al enumerar los títulos de María, jamás la separa de la Iglesia.

Así la Iglesia, toda la Iglesia, a ejemplo de María debe vivir también ella cada vez con mayor intensidad la propia maternidad universa (cf. \emph{Lumen gentium},} 64), respecto de toda la familia humana actualmente deshumanizada porque está desacralizada.

\textquote{Madre y Maestra}, la Iglesia de Cristo no pretende construir la paz del mundo sin El o suplantándolo; sino que proclamando el reino de Dios en todas las naciones se propone al mismo tiempo \textquote{descubrir al hombre el sentido de la propia existencia}, sabiendo que \textquote{el que sigue a Cristo, Hombre perfecto, se perfecciona cada vez más en su propia dignidad de hombre} (\emph{Gaudium et spes}, 41).

Y volviendo con el pensamiento a María Reina de la Paz, nos complacemos en recordar que nuestro venerado predecesor el Papa Benedicto XV quiso exaltar este título debido a la Virgen María haciendo esculpir un monumento en su honor en esta misma basílica, al finalizar la primera guerra mundial.

Nadie piense que la paz, de la que María es portadora, se pueda confundir con la debilidad o la insensibilidad de los tímidos o de los viles; recordemos el himno más bello de la liturgia mariana, el \emph{Magnificat,} en el que la voz sonora y valiente de María resuena para dar fortaleza y valor a los promotores de la paz: \textquote{Desplegó el poder de su brazo y dispersó a los que se engríen con los pensamientos de su corazón. Derribó a los potentados de sus tronos y ensalzó a los humildes} (\emph{Lc} 51-52).

Nos proponemos confiar a María la causa de la paz en todo el mundo, y en particular en la querida nación del Líbano, ejemplo de país arrollado por la espiral de la violencia no tanto por causas suyas internas cuanto por el reflejo de las situaciones de esa región que no han encontrado todavía soluciones justas; es decir que, en realidad, ha sido víctima de dichas situaciones. En esta Jornada de la Paz. exhortamos, pues, a los aquí presentes y a todos los fieles a orar por el Líbano a la Virgen \emph{Notre Dame du Liban,} para que acelere la reconciliación de sus hijos \emph{} y el resurgir espiritual y moral, además del material, de la nación.

Que en las esperanzas de paz que comienzan a vislumbrarse en Oriente Medio, la reconciliación de los distintos grupos libaneses y la convivencia serena de la población lleguen a ser factor de reconciliación y de repulsa de la violencia para todos los pueblos de la región.

Al concluir estas reflexiones nuestras, queremos dirigir una llamada apremiante a todos nuestros hijos y a cada uno particularmente.

Procure cada cual aportar su contribución práctica, generosa y auténtica a la paz del mundo, eliminando del corazón en primer lugar toda forma de violencia, todo sentimiento de avasallamiento del hermano. Actuando así os encontraréis ya en el sendero de la paz universal que se funda en la paz efectiva de cada uno.

Si queréis conseguir que la paz reine en todo el mundo, hacedla reinar primero en vuestro corazón, en vuestra familia, en vuestra casa, en vuestro barrio, en vuestra ciudad, en vuestra región, en vuestra nación:

De este modo los demás sentirán incluso el encanto y el gozo de poder vivir en serenidad y de esforzarse para que este inmenso bien sea aspiración, exigencia y patrimonio de todos.

Esto lo queremos decir en particular a vosotros, jóvenes, y a vosotros, muchachos, presentes hoy muy numerosos en esta basílica.

Hemos querido terminar nuestro reciente Mensaje para la jornada de la Paz dirigiéndonos en especial a los jóvenes y a los muchachos de todo el mundo. porque vosotros tenéis esa extraordinaria capacidad de apertura y esa gozosa disponibilidad que por desgracia a veces los adultos han olvidado o perdido.

También vosotros, jóvenes y muchachos, tenéis una palabra que decir y hacer oír a los mayores, una palabra juvenil, nueva, original.

Comunicad esta palabra de paz, este \textquote{no a la violencia} con energía, con fuerza, con la fuerza de vuestro corazón puro, de vuestros oros límpidos, de vuestra alegría de vivir, pero de vivir en un mundo en el que \textquote{se darán el abrazo la justicia y la paz} (\emph{Sal} 84, 11).

En vuestros ideales y en vuestro comportamiento dad siempre la prioridad al amor, es decir, a la comprensión, a la benevolencia, a la solidaridad con los otros.

Reforzad vuestra convicción de paz en la oración personal y comunitaria: en el dialogo y la meditación en los que os esforzáis por conocer cada vez más profundamente a Cristo y por comprender su mensaje con todas sus exigencias: en los sacramentos, y sobre todo en el sacramento de la Eucaristía, en el que el mismo Cristo os da la fe, la esperanza y, ante todo, la caridad; en fin, reforzadla en la devoción filial a la Virgen María.

Si vuestra convicción es sólida y firme. en todas las manifestaciones de vuestra juventud seréis testimonios de la paz y el amor de Cristo que está: en vosotros.

Jóvenes y muchachos, lleváis en vosotros el porvenir del mundo y de la historia. Este mundo será mejor, más fraterno, más justo, si ya desde ahora toda vuestra vida está abierta a la gracia de Cristo, a los ideales de amor y de paz que os enseña el Evangelio.

María, Reina de la Paz, \emph{Salus Populi Romani,} interceda por estas intenciones.

\subsubsection{Homilía (1981)}

MISA EN LA SOLEMNIDAD DE SANTA MARÍA, MADRE DE DIOS\\ XIV JORNADA MUNDIAL DE LA PAZ

\emph{\textbf{HOMILÍA DEL SANTO PADRE JUAN PABLO II}\\[2\baselineskip]Basílica de San Pedro\\ Jueves 1 de enero de 1981}

~

1. \textquote{\ldots{}al llegar la plenitud de los tiempos, envió Dios a su Hijo, nacido de mujer\ldots{}}. Son palabras de San Pablo, tomadas de la liturgia de hoy \emph{(Gál} 4, 4).

\emph{\textquote{La plenitud de los tiempos\ldots{}}.}

Estas palabras tienen hoy una particular elocuencia, puesto que nos es dado pronunciar por primera vez la nueva fecha, esto es, el nombre del nuevo año solar: 1981. Así sucede cada año el primer día de enero. Pasan los años, cambian las fechas, transcurre el tiempo. Con el tiempo pasa también toda la naturaleza, naciendo, desarrollándose, muriendo. Y pasa también \emph{el hombre}; pero él pasa conscientemente. Tiene \emph{la conciencia de su pasar, la conciencia del tiempo.} Con el metro del tiempo mide la historia del mundo y, sobre todo, la propia historia. No sólo los años, los decenios, los siglos, los milenios, sino también los días, las horas, los minutos, los segundos.

La liturgia de hoy nos dice con las palabras de San Pablo que el tiempo, que es el metro del pasar de los seres humanos en el mundo, está sometido también a otra medida, es decir, \emph{a la medida de la plenitud, que proviene de Dios:} la plenitud del tiempo. Efectivamente, en el tiempo ---en el tiempo humano, terreno---, Dios lleva a cumplimiento su proyecto eterno de amor. Mediante el amor de Dios, \emph{el tiempo} está sometido a la \emph{eternidad} y al \emph{Verbo.}

El Verbo se hizo carne\ldots{} en el tiempo.

Los años que pasan, que terminan el 31 de diciembre y comienzan de nuevo el 1 de enero, pasan en realidad confrontándose con esa plenitud, que proviene de Dios. Pasan frente a la eternidad y al Verbo. Cada año del calendario humano lleva, juntamente con el tiempo, una pequeña parte del \textquote{Kairós} divino. Cada uno comienza, dura y transcurre en relación \emph{a esa plenitud del tiempo que viene de Dios.}

Es preciso darse cuenta de esto, de modo particular hoy, que es el primer día del año nuevo.

2.~Qué fuerte y espléndidamente se capta esta realidad, cuando nos damos cuenta de que este primer día del año nuevo \emph{es,} al mismo tiempo, \emph{el día de la octava de Navidad.} El año nuevo nace en el esplendor del misterio en el que se ha revelado la \textquote{plenitud del tiempo}.

\textquote{Dios envió a su Hijo, nacido de mujer}.

Y~precisamente hacia esa Mujer, hacia la Madre del Hijo de Dios, hacia la Theotokos se dirigen hoy, al comienzo del año nuevo, de modo especial, el pensamiento y el corazón de la Iglesia. María está presente durante toda la octava; sin embargo, la Iglesia desea venerarla particularmente hoy, con un día dedicado totalmente a Ella: \emph{la solemnidad de Santa María, Madre de Dios.}

A ella, pues, a la Maternidad admirable de la Virgen de Nazaret, ligada a la \textquote{plenitud de los tiempos}, nos dirigimos mediante este comienzo del año, que coincide con el día de hoy.

Y~recordamos que es el comienzo del año del Señor 1981, durante el cual resonarán con eco lejano en los siglos las fechas conmemorativas de los dos importantes Concilios de los primeros tiempos de la Iglesia, que permaneció una y única, a pesar de las primeras grandes herejías que surgieron. Efectivamente, en el año 381 tuvo lugar el \emph{primer Concilio de Constantinopla,} que, después del Concilio de Nicea, fue el segundo Concilio Ecuménico de la Iglesia y al cual debemos el \textquote{Credo} que se recita constantemente en la liturgia. Una herencia particular de aquel Concilio es la doctrina \emph{sobre el Espíritu Santo} proclamada así en la liturgia latina: Credo in Spiritum Sanctum Dominum et vivificantem, qui ex Patre Filioque procedit ----(la formulación de la teología oriental, en cambio, dice: qui a Patre per Filium procedit)---. Qui cum Patre et Filio simul adoratur et conglorificatur qui locutus est per prophetas.

Y, luego, \emph{el año 431} (hace 1550 años), se celebró el \emph{Concilio de Efeso,} que confirmó, con inmensa alegría de los participantes, la fe de la Iglesia en la Maternidad Divina de María. Aquel que \textquote{nació de María Virgen}, como hombre, es, al mismo tiempo, el verdadero Hijo de Dios, \textquote{de la misma naturaleza que el Padre}. Aquella, de la cual \textquote{fue concebido por obra del Espíritu Santo} y que lo trajo al mundo la noche de Belén, es verdadera Madre de Dios: \emph{Theotokos.}

Basta recitar con atención las palabras de nuestro Credo, para darse cuenta de cuán profundamente estos dos Concilios, que recordaremos en el curso del año 1981, están orgánicamente ligados el uno al otro \emph{con la profundidad del misterio divino y humano.} Sobre este misterio se construye la fe de la Iglesia.

3.~El primer día del año deseamos leer de nuevo en la profundidad de ese misterio el mensaje de la paz, que, de una vez para siempre, se reveló en la noche de Belén: \emph{¡Paz a los hombres de buena voluntad! ¡Paz en la tierra!,} he aquí lo que el misterio del nacimiento de Dios quiere decirnos cada año y lo que la Iglesia pone de relieve también hoy, primer día del año nuevo.

\textquote{Dios envió a su Hijo, nacido de mujer\ldots{}} para que nosotros podamos recibir la filiación adoptiva.

\textquote{Como sois hijos, Dios envió a vuestros corazones al Espíritu de su Hijo, que clama: ¡Abbá! (Padre). Así que ya no eres esclavo, sino hijo\ldots{}} (\emph{Gál} 4, 6-7).

Toda la humanidad desea ardientemente la paz y ve la guerra como el peligro más grande en su existencia terrena. La Iglesia se halla totalmente presente en estos deseos y, al mismo tiempo, en los miedos y en las preocupaciones que agobian a todos los hombres, manifestando estos sentimientos, de modo particular, el primer día del año nuevo.

¿Qué es la paz? ¿Qué puede ser la paz en la tierra, la paz \emph{entre los hombres y los pueblos,} sino \emph{el fruto de la fraternidad,} que se manifieste más fuerte que lo que divide y contrapone recíprocamente a los hombres? De esta fraternidad habla precisamente San Pablo, cuando escribe a los Gálatas: \textquote{Vosotros sois hijos}. Y \emph{si hijos} ---los hijos de Dios en Cristo--- entonces, también \emph{hermanos.}

Y a continuación escribe: \textquote{Así que ya no eres esclavo, sino hijo}. En este contexto se inserta el tema del mensaje elegido para la Jornada de la Paz del primero de enero de 1981. Dice: \emph{\textquote{Para servir a la paz, respeta la libertad}}.

4. ¡Sí! Debemos apelar a la fraternidad, queridos hermanos y hermanas, si queremos superar los monstruosos \emph{mecanismos} que, en la vida y en el desarrollo de las potencias del mundo contemporáneo, \emph{trabajan en favor de la guerra.}

Es necesario que nosotros consideremos a la humanidad como una única gran familia, en la cual todas las clases de personas deben ser reconocidas y acogidas como hermanos. En los umbrales de un nuevo año, dirijamos de modo especial maestro pensamiento y nuestra solicitud a aquellos, entre estos hermanos, que se hallan en particulares situaciones de necesidad y esperan que los ingentes recursos, destinados a construir instrumentos de recíproca destrucción, sean empleados, en cambio, para las urgentes obras de socorro y de mejoramiento de las condiciones de vida.

5. Como es sabido, el 1981 ha sido proclamado por la ONU \textquote{Año Internacional de los minusválidos}. Se trata de millones de personas que tienen enfermedades congénitas, enfermedades crónicas, o que están afectadas por varias formas de deficiencia mental o debilidades sensoriales; estas personas en el curso del año interpelarán de manera más aguda a nuestra conciencia humana y cristiana. Según recientes estadísticas, su número asciende a más de 400 millones. También ellos son hermanos nuestros. Es necesario que su dignidad humana y sus derechos inalienables reciban pleno y efectivo reconocimiento durante todo el arco de su existencia.

En el pasado noviembre, durante la reunión de un grupo de trabajo, la Pontificia Academia de las Ciencias, en su constante obra al servicio de la humanidad mediante la investigación científica, ha profundizado en el estudio de una clase especial de minusválidos, los mentales. La debilidad mental, que afecta a casi al tres por ciento de la población mundial, debe ser tenida en consideración especial, porque constituye el más grave obstáculo para la realización del hombre. La relación del mencionado grupo de trabajo ha puesto de relieve la posibilidad de cuidados preventivos de las causas de la debilidad mental, mediante oportunas terapias. La ciencia y la medicina ofrecen, pues, un mensaje de esperanza y, al mismo tiempo, de empeño para toda la humanidad. Si sólo una parte del \textquote{budget} para la carrera de armamentos fuese destinado a este objetivo, se podrían conseguir éxitos importantes y aliviar la suerte de numerosas personas que sufren.

Al comienzo de este año deseo confiar todas las personas minusválidas a la materna protección de María. En la Pascua de 1971, 4.000 minusválidos mentales, divididos en pequeños grupos acompañados por familiares y educadores, fueron en peregrinación a Lourdes y vivieron días de paz y de serenidad juntamente con los otros peregrinos. Deseo de corazón que, bajo la mirada materna de María, se multipliquen las experiencias de solidaridad humana y cristiana, en una fraternidad renovada que una a los débiles y a los fuertes en el camino común de la vocación divina de la persona humana.

6. Al pensar, en los umbrales de este nuevo año, sobre las más graves necesidades de la humanidad, quisiera llamar también la atención sobre esa parte de la familia humana que se encuentra en extrema necesidad a causa de la situación alimentaria. El hambre y la mal nutrición constituyen hoy, efectivamente, un problema dramático de supervivencia para millones de seres humanos, especialmente de niños en amplias zonas de nuestro globo. Mi pensamiento se dirige particularmente a algunas extensas regiones de África afectadas por la sequía, como el Sahel, y de Asia, damnificadas por calamidades naturales o que deben afrontar una considerable afluencia de refugiados.

Según una relación de la FAO, al menos 26 países africanos han tenido últimamente cosechas inferiores a las del pasado. En algunas partes de ese continente persiste el hambre y se registran carestías periódicas, que causan no pocas víctimas. Según los cálculos de expertos, las reservas mundiales de cereales disminuirán por tercer año consecutivo si continua la tendencia actual. Hago votos de corazón a fin de que todos los responsables, todas las organizaciones y todos los hombres de buena voluntad den su aportación para la realización de medidas que permitan un socorro más efectivo a los hermanos que se encuentran en la indigencia y, a la vez, se cree un sistema más eficaz de seguridad alimentaria. La palabra de Cristo \textquote{Tuve hambre y me disteis de comer}, es una llamada, urgente y particularmente actual, a nuestras responsabilidades.

Son penetrantes las palabras de San Pablo de la liturgia de hoy. Es necesario qué la vida de la gran familia humana en todo el mundo se transforme bajo el signo de la fraternidad universal de los hombres. Efectivamente, somos hijos de Dios: Dios ha enviado a nuestros corazones el Espíritu de su Hijo que clama: Abbá, Padre. Por lo tanto: ¡ninguno es esclavo, sino hijo!

7. Durante el año que acaba de terminar se ha recordado de modo particular la \emph{figura de San Benito,} como Patrono de Europa, en relación con el 1.500 aniversario de su nacimiento.

Al meditar sobre el desarrollo de los acontecimientos más antiguos y sobre los contemporáneos, ha parecido justo proclamar Copatronos de Europa, al final del año, \emph{a los Santos Cirilo y Metodio}, que representan otro gran componente en la misión cristiana y en la obra de la economía de la salvación en nuestro continente. Es la parte ligada a la heredad de la Grecia antigua y del Patriarcado de Constantinopla, desde donde estos dos hermanos fueron enviados en misión a los pueblos de Europa Meridional y Oriental, precisamente a los eslavos. En efecto, \emph{Europa se hizo cristiana} bajo la acción de estos dos componentes.

Nos ha parecido, pues, que, particularmente al final del año en el que se ha entablado el \emph{diálogo} teológico definitivo \emph{entre la Iglesia católica} y \emph{toda la ortodoxia,} tiene una gran elocuencia el haber puesto de relieve la misión de los Santos Cirilo y Metodio. Es la elocuencia de la reconciliación y de la paz, que en todos los caminos de la humanidad debe demostrarse más potente que las fuerzas de la división y de la amenaza recíproca.

8. Termino citando una vez más las palabras de la liturgia de hoy:

\textquote{El Señor tenga piedad y nos bendiga, ilumine su rostro sobre nosotros: \emph{conozca} la tierra \emph{tus caminos,} todos los pueblos tu salvación. El Señor tenga piedad y nos bendiga} (Salmo responsorial).

\subsubsection{Homilía (1999)} SOLEMNIDAD DE SANTA MARÍA, MADRE DE DIOS\\ XXXII JORNADA MUNDIAL DE LA PAZ

\emph{\textbf{HOMILÍA DEL SANTO PADRE JUAN PABLO II}}\\[2\baselineskip]\emph{1 de enero de 1999}

~

1. \emph{Christus heri et hodie, principium et finis, alpha et omega..}. \textquote{Cristo ayer y hoy, principio y fin, alfa y omega. Suyo es el tiempo y la eternidad. A él la gloria y el poder por los siglos de los siglos} (\emph{Misal romano}, preparación del cirio pascual).

Todos los años, durante la Vigilia pascual, la Iglesia renueva esta solemne aclamación a Cristo, Señor del tiempo. También el último día del año proclamamos esta verdad, en el paso del \textquote{ayer} al \textquote{hoy}: \textquote{ayer}, al dar gracias a Dios por la conclusión del año viejo; \textquote{hoy}, al acoger el año que empieza. \emph{Heri et hodie}. Celebramos a Cristo que, como dice la Escritura, es \textquote{el mismo ayer, hoy y siempre} (\emph{Hb} 13, 8). Él es el Señor de la historia; suyos son los siglos y los milenios.

Al comenzar el año 1999, el último antes del gran jubileo, parece que el misterio de la historia se revela ante nosotros con una profundidad más intensa. Precisamente por eso, la Iglesia ha querido imprimir el signo trinitario de la presencia del Dios vivo sobre el trienio de preparación inmediata para el acontecimiento jubilar.

2. El primer día del nuevo año concluye la Octava de la Navidad del Señor y está dedicado a la santísima Virgen, venerada como Madre de Dios. El evangelio nos recuerda que \textquote{guardaba todas estas cosas y las meditaba en su corazón} (\emph{Lc} 2, 19). Así sucedió en Belén, en el Gólgota, al pie de la cruz, y el día de Pentecostés, cuando el Espíritu Santo descendió al cenáculo.

Y lo mismo sucede también hoy. La Madre de Dios y de los hombres guarda y medita en su corazón todos los problemas de la humanidad, grandes y difíciles. La \emph{Alma Redemptoris Mater} camina con nosotros y nos guía, con ternura materna, hacia el futuro. Así, ayuda a la humanidad a cruzar todos los \textquote{umbrales} de los años, de los siglos y de los milenios, sosteniendo su esperanza en aquel que es el Señor de la historia

3. \emph{Heri et hodie}. Ayer y hoy. \textquote{Ayer} invita a la retrospección. Cuando dirigimos nuestra mirada a los acontecimientos de este siglo que está a punto de terminar, se presentan ante nuestros ojos las dos guerras mundiales: cementerios, tumbas de caídos, familias destruidas, llanto y desesperación, miseria y sufrimiento. ¿Cómo olvidar los campos de muerte, a los hijos de Israel exterminados cruelmente y a los santos mártires: el padre Maximiliano Kolbe, sor Edith Stein y tantos otros?

Sin embargo, nuestro siglo es también el siglo de la \emph{Declaración universal de derechos del hombre}, cuyo 50° aniversario se celebró recientemente. Teniendo presente precisamente este aniversario, en el tradicional Mensaje para la actual \emph{Jornada mundial de la paz}, quise recordar que el secreto de la paz verdadera reside en el respeto de los derechos humanos. \textquote{El reconocimiento de la dignidad innata de todos los miembros de la familia humana (\ldots{}) es el fundamento de la libertad, de la justicia y de la paz en el mundo} (n. 3: \emph{L'Osservatore Romano}, edición en lengua española, 18 de diciembre de 1998, p. 6).

El concilio Vaticano II, el concilio que ha preparado a la Iglesia para entrar en el tercer milenio, reafirmó que el mundo, teatro de la historia del género humano, ha sido liberado de la esclavitud del pecado por Cristo crucificado y resucitado, \textquote{para que se transforme, según el designio de Dios, y llegue a su consumación} (\emph{Gaudium et spes}, 2). Es así como los creyentes miran al mundo de nuestros días, a la vez que avanzan gradualmente hacia el umbral del año 2000.

4. El Verbo eterno, al hacerse hombre, entró en el mundo y lo acogió para redimirlo. Por tanto, el mundo no sólo está marcado por la terrible herencia del pecado; es, ante todo, un mundo salvado por Cristo, el Hijo de Dios, crucificado y resucitado. Jesús es el Redentor del mundo, el Señor de la historia. \emph{Eius sunt tempora et saecula:} suyos son los años y los siglos. Por eso creemos que, al entrar en el tercer milenio junto con Cristo, cooperaremos en la transformación del mundo redimido por él. \emph{Mundus creatus, mundus redemptus}.

Desgraciadamente, la humanidad cede a la influencia del mal de muchos modos. Sin embargo, impulsada por la gracia, se levanta continuamente, y camina hacia el bien guiada por la fuerza de la redención. Camina hacia Cristo, según el proyecto de Dios Padre.

\textquote{Jesucristo es el principio y el fin, el alfa y la omega. Suyo es el tiempo y la eternidad}.

Empecemos este año nuevo en su nombre. Que María nos obtenga la gracia de ser fieles discípulos suyos, para que con palabras y obras lo glorifiquemos y honremos por los siglos de los siglos: \emph{Ipsi gloria et imperium per universa aeternitatis saecula}. Amén.

\subsubsection{Ángelus (2014)} \emph{Queridos hermanos y hermanas, ¡buenos día y feliz año!}

Al inicio del nuevo año dirijo a todos vosotros los más cordiales deseos de paz y de todo bien. Mi deseo es el de la Iglesia, el deseo cristiano. No está relacionado con el sentido un poco mágico y un poco fatalista de un nuevo ciclo que inicia. Sabemos que la historia tiene un centro: Jesucristo, encarnado, muerto y resucitado, que vive entre nosotros; tiene un fin: el Reino de Dios, Reino de paz, de justicia, de libertad en el amor; y tiene una fuerza que la mueve hacia ese fin: la fuerza es el Espíritu Santo. Todos nosotros tenemos el Espíritu Santo que hemos recibido en el Bautismo, y Él nos impulsa a seguir adelante por el camino de la vida cristiana, por la senda de la historia, hacia el Reino de Dios.

Este Espíritu es la potencia de amor que fecundó el seno de la Virgen María; y es el mismo que anima los proyectos y las obras de todos los constructores de paz. Donde hay un hombre o una mujer constructor de paz, es precisamente el Espíritu Santo quien le ayuda, le impulsa a construir la paz. Dos caminos que se cruzan hoy: fiesta de María santísima Madre de Dios y Jornada mundial de la paz. Hace ocho días resonaba el anuncio angelical: \textquote{Gloria a Dios y paz a los hombres}; hoy lo acogemos nuevamente de la Madre de Jesús, que \textquote{conservaba todas estas cosas, meditándolas en su corazón} (\emph{Lc} 2, 19), para hacer de ello nuestro compromiso a lo largo del año que comienza.

El tema de esta Jornada mundial de la paz es \textquote{\emph{La fraternidad, fundamento y camino para la paz}}}. Fraternidad: siguiendo la estela de mis Predecesores, a partir de Pablo VI, he desarrollado el tema en un Mensaje, ya difundido y hoy idealmente entrego a todos. En la base está la convicción de que todos somos hijos del único Padre celestial, formamos parte de la misma familia humana y compartimos un destino común. De aquí se deriva para cada uno la responsabilidad de obrar a fin de que el mundo llegue a ser una comunidad de hermanos que se respetan, se aceptan en su diversidad y se cuidan unos a otros. Estamos llamados también a darnos cuenta de las violencias e injusticias presentes en tantas partes del mundo y que no pueden dejarnos indiferentes e inmóviles: se necesita del compromiso de todos para construir una sociedad verdaderamente más justa y solidaria. Ayer recibí una carta de un señor, tal vez uno de vosotros, quien informándome sobre una tragedia familiar, a continuación enumeraba muchas tragedias y guerras de hoy en el mundo, y me preguntaba: ¿qué sucede en el corazón del hombre, que le lleva a hacer todo esto? Y decía, al final: \textquote{Es hora de detenerse}. También yo creo que nos hará bien detenernos en este camino de violencia, y buscar la paz. Hermanos y hermanas, hago mías las palabras de este hombre: ¿qué sucede en el corazón del hombre? ¿Qué sucede en el corazón de la humanidad? ¡Es hora de detenerse!

Desde todos los rincones de la tierra, los creyentes elevan hoy la oración para pedir al Señor el don de la paz y la capacidad de llevarla a cada ambiente. En este primer día del año, que el Señor nos ayude a encaminarnos todos con más firmeza por las sendas de la justicia y de la paz. Y comencemos en casa. Justicia y paz en casa, entre nosotros. Se comienza en casa y luego se sigue adelante, a toda la humanidad. Pero debemos comenzar en casa. Que el Espíritu Santo actúe en nuestro corazón, rompa las cerrazones y las durezas y nos conceda enternecernos ante la debilidad del Niño Jesús. La paz, en efecto, requiere la fuerza de la mansedumbre, la fuerza no violenta de la verdad y del amor.

En las manos de María, Madre del Redentor, ponemos con confianza filial nuestras esperanzas. A ella, que extiende su maternidad a todos los hombres, confiamos el grito de paz de las poblaciones oprimidas por la guerra y la violencia, para que la valentía del diálogo y de la reconciliación predomine sobre las tentaciones de venganza, de prepotencia y corrupción. A ella le pedimos que el Evangelio de la fraternidad, anunciado y testimoniado por la Iglesia, pueda hablar a cada conciencia y derribar los muros que impiden a los enemigos reconocerse hermanos.

\subsubsection{Ángelus (2017)} \emph{Plaza de San Pedro\\ Domingo 1 de enero de 2017}

~

\emph{Queridos hermanos y hermanas, ¡buenos días!}

Durante los días pasados hemos puesto nuestra mirada adorante sobre el Hijo de Dios, nacido en Belén; hoy, Solemnidad de María Santísima Madre de Dios, dirigimos nuestros ojos a la Madre, pero recibiendo a ambos con su estrecho vínculo. Este vínculo no se agota en el hecho de haber generado y en haber sido generado; Jesús ha \textquote{nacido de mujer} (\emph{Gal} 4, 4) para una misión de salvación y su madre no está excluida de tal misión, es más, está asociada íntimamente. María es consciente de esto, por lo tanto no se cierra a considerar sólo su relación maternal con Jesús, sino que permanece abierta y primorosa en todos los acontecimientos que suceden a su alrededor: conserva y medita, observa y profundiza, como nos recuerda el Evangelio de hoy (cf \emph{Lc} 2, 19). Ha dicho ya su \textquote{sí} y ha dado su disponibilidad para ser incluida en la aplicación del plan de salvación de Dios, que \textquote{dispersó a los que son soberbios en su propio corazón. Derribó a los potentados de sus tronos y exaltó a los humildes. A los hambrientos colmó de bienes y despidió a los ricos sin nada} (\emph{Lc} 1, 51-53). Ahora, silenciosa y atenta, intenta comprender qué quiere Dios de ella día a día. La visita de los pastores le ofrece la ocasión para percibir algún elemento de la voluntad de Dios que se manifiesta en la presencia de estas personas humildes y pobres. El evangelista Lucas nos narra la visita de los pastores a la gruta con un rápido sucederse de verbos que expresan movimiento. Dice así: ellos van sin demora, encuentran al Niño con María y José, lo ven, y cuentan lo que les ha sido dicho por Él, y al final glorifican a Dios (cf \emph{Lc} 2, 16-20). María sigue atentamente esta escena, qué dicen los pastores, qué les ha ocurrido, por qué en ello ya se discierne el movimiento de salvación que surgirá de la obra de Jesús, y se adapta, preparada ante toda petición del Señor. Dios pide a María no sólo ser la madre de su Hijo unigénito, sino también cooperar con el Hijo y por el Hijo en su plan de salvación, para que en ella, humilde sierva, se cumplan las grandes obras de la misericordia divina.

Por ello, mientras, así como los pastores, contemplan el icono del Niño en brazos de su Madre, sentimos crecer en nuestro corazón un sentido de inmenso agradecimiento hacia quien ha dado al mundo al Salvador. Por ello, en el primer día de un año nuevo, le decimos:

Gracias, oh Santa Madre del Hijo de Dios, Jesús, ¡Santa Madre de Dios!\\ Gracias por tu humildad que ha atraído la mirada de Dios;\\ gracias por la fe con la cual has acogido su Palabra;\\ gracias por la valentía con la cual has dicho \textquote{aquí estoy},\\ olvidada de si misma, fascinada por el Amor Santo, convertida en una única cosa junto con su esperanza.\\ Gracias, ¡oh Santa Madre de Dios!\\ Reza por nosotros, peregrinos del tiempo; ayúdanos a caminar por la vía de la paz. Amén.

\subsubsection{Ángelus (2020)} SOLEMNIDAD DE SANTA MARÍA, MADRE DE DIOS\\ LIII JORNADA MUNDIAL DE LA PAZ

\textbf{PAPA FRANCISCO}

\textbf{\emph{ÁNGELUS}}

\emph{Plaza de San Pedro\\ Miércoles, 1 de enero de 2020}

~

\emph{Queridos hermanos y hermanas, ¡buenos días! ¡Y Feliz Año Nuevo!}

Anoche terminamos el año 2019 agradeciendo a Dios por el don del tiempo y por todos sus beneficios. Hoy comenzamos el año 2020 con la misma actitud de \emph{gratitud} y \emph{alabanza}. No se da por sentado que nuestro planeta ha comenzado una nueva vuelta alrededor del sol y que los seres humanos seguiremos viviendo en él. No se da por sentado, al contrario, siempre es un \textquote{milagro} del que sorprenderse y estar agradecido.

El primer día del año la liturgia celebra a la Santa Madre de Dios, María, la Virgen de Nazaret que dio a luz a Jesús, el Salvador. Ese Niño es la \emph{bendición de Dios} para cada hombre y mujer, para la gran familia humana y para el mundo entero. Jesús no eliminó el mal del mundo, sino que lo derrotó en su raíz. Su salvación no es mágica, sino que es una salvación \textquote{paciente}, es decir, implica la paciencia del amor, que se responsabiliza de la iniquidad y le quita su poder. La paciencia del amor: el amor nos hace pacientes. Muchas veces perdemos la paciencia; yo también, y pido disculpas por el mal ejemplo de ayer {[}se refiere a la reacción que tuvo con una persona que le tiró bruscamente del brazo en la plaza de San Pedro{]}. Por eso, contemplando el Pesebre vemos, con los ojos de la fe, el mundo renovado, liberado del dominio del mal y puesto bajo el señorío real de Cristo, el Niño acostado en el pesebre.

Por eso hoy la Madre de Dios nos bendice. ¿Y cómo nos bendice la Virgen? Mostrándonos al Hijo. Lo toma en sus brazos y nos lo muestra, y así nos bendice. Bendice a toda la Iglesia, bendice al mundo entero. Jesús, como cantaban los ángeles en Belén, es la \textquote{alegría de todo el pueblo}, es la gloria de Dios y la paz para la humanidad (cf. \emph{Lucas} 2, 14). Por eso el santo Papa Pablo VI quiso dedicar el primer día del año a la paz ―es la Jornada de la Paz―, a la oración, a la conciencia y a la responsabilidad por la paz. Para este año 2020 el Mensaje es así: la paz es un \emph{camino de esperanza}, un camino en el que se avanza a través del \emph{diálogo}, la \emph{reconciliación} y la \emph{conversión ecológica}.

Por lo tanto, fijemos la mirada en la Madre y en el Hijo que nos muestra. Al comienzo del año, ¡seamos bendecidos! Dejémonos bendecir por la Virgen con su Hijo.

Jesús es la bendición para aquellos que están oprimidos por el yugo de la esclavitud, la esclavitud moral y la esclavitud material. Él libera con amor. A los que han perdido la autoestima por permanecer prisioneros de círculos viciosos, Jesús les dice: el Padre os ama, no os abandona, espera con una paciencia inquebrantable vuestro regreso (cf. \emph{Lucas} 15, 20). A los que son víctimas de la injusticia y la explotación y no ven la salida, Jesús les abre la puerta de la fraternidad, donde pueden encontrar rostros, corazones y manos acogedores, donde pueden compartir la amargura y la desesperación, y recuperar algo de dignidad. A los que están gravemente enfermos y se sienten abandonados y desanimados, Jesús se acerca, toca con ternura las heridas, derrama el aceite del consuelo y transforma la debilidad en fuerza del bien para desatar los nudos más enredados. A los que están encarcelados y son tentados a encerrarse en sí mismos, Jesús les vuelve a abrir un horizonte de esperanza, empezando por un pequeño rayo de luz.

Queridos hermanos y hermanas, bajemos de los pedestales de nuestro orgullo ―todos tenemos la tentación del orgullo― y pidamos la bendición de la Santa Madre de Dios, la humilde Madre de Dios. Ella nos muestra a Jesús: seamos bendecidos, abramos nuestros corazones a su bondad. Así, el año que comienza será un camino de esperanza y paz, no con palabras, sino con gestos cotidianos de diálogo, reconciliación y cuidado de la creación.



\section{6 de enero} \subsubsection{Homilía (2017)} SANTA MISA EN LA SOLEMNIDAD DE LA EPIFANÍA DEL SEÑOR

CAPILLA PAPAL

\textbf{\emph{HOMILÍA DEL SANTO PADRE FRANCISCO}}

\emph{Basílica Vaticana\\ Viernes 6 de enero de 2017}


~

\textquote{¿Dónde está el Rey de los judíos que acaba de nacer? Porque vimos su estrella y hemos venido a adorarlo} (\emph{Mt} 2, 2).

Con estas palabras, los magos, venidos de tierras lejanas, nos dan a conocer el motivo de su larga travesía: adorar al rey recién nacido. Ver y adorar, dos acciones que se destacan en el relato evangélico: vimos una estrella y queremos adorar.

Estos hombres \emph{vieron una estrella} que los puso en movimiento. El descubrimiento de algo inusual que sucedió en el cielo logró desencadenar un sinfín de acontecimientos. No era una estrella que brilló de manera exclusiva para ellos, ni tampoco tenían un ADN especial para descubrirla. Como bien supo decir un padre de la Iglesia, \textquote{los magos no se pusieron en camino porque hubieran visto la estrella, sino que vieron la estrella porque se habían puesto en camino} (cf. San Juan Crisóstomo). Tenían el corazón abierto al horizonte y lograron ver lo que el cielo les mostraba porque había en ellos una inquietud que los empujaba: estaban abiertos a una novedad.

Los magos, de este modo, expresan el retrato del hombre creyente, del hombre que tiene nostalgia de Dios; del que añora su casa, la patria celeste. Reflejan la imagen de todos los hombres que en su vida no han dejado que se les anestesie el corazón.

La santa nostalgia de Dios brota en el corazón creyente pues sabe que el Evangelio no es un acontecimiento del pasado sino del presente. La santa nostalgia de Dios nos permite tener los ojos abiertos frente a todos los intentos reductivos y empobrecedores de la vida. La santa nostalgia de Dios es la memoria creyente que se rebela frente a tantos profetas de desventura. Esa nostalgia es la que mantiene viva la esperanza de la comunidad creyente la cual, semana a semana, implora diciendo: \textquote{Ven, Señor Jesús}.

Precisamente esta nostalgia fue la que empujó al anciano Simeón a ir todos los días al templo, con la certeza de saber que su vida no terminaría sin poder acunar al Salvador. Fue esta nostalgia la que empujó al hijo pródigo a salir de una actitud de derrota y buscar los brazos de su padre. Fue esta nostalgia la que el pastor sintió en su corazón cuando dejó a las noventa y nueve ovejas en busca de la que estaba perdida, y fue también la que experimentó María Magdalena la mañana del domingo para salir corriendo al sepulcro y encontrar a su Maestro resucitado. La nostalgia de Dios nos saca de nuestros encierros deterministas, esos que nos llevan a pensar que nada puede cambiar. La nostalgia de Dios es la actitud que rompe aburridos conformismos e impulsa a comprometerse por ese cambio que anhelamos y necesitamos. La nostalgia de Dios tiene su raíz en el pasado pero no se queda allí: va en busca del futuro. Al igual que los magos, el creyente \textquote{nostalgioso} busca a Dios, empujado por su fe, en los lugares más recónditos de la historia, porque sabe en su corazón que allí lo espera el Señor. Va a la periferia, a la frontera, a los sitios no evangelizados para poder encontrarse con su Señor; y lejos de hacerlo con una postura de superioridad lo hace como un mendicante que no puede ignorar los ojos de aquel para el cual la Buena Nueva es todavía un terreno a explorar.

Como actitud contrapuesta, en el palacio de Herodes ―que distaba muy pocos kilómetros de Belén―, no se habían percatado de lo que estaba sucediendo. Mientras los magos caminaban, Jerusalén dormía. Dormía de la mano de un Herodes quien lejos de estar en búsqueda también dormía. Dormía bajo la anestesia de una conciencia cauterizada. Y quedó desconcertado. Tuvo miedo. Es el desconcierto que, frente a la novedad que revoluciona la historia, se encierra en sí mismo, en sus logros, en sus saberes, en sus éxitos. El desconcierto de quien está sentado sobre la riqueza sin lograr ver más allá. Un desconcierto que brota del corazón de quién quiere controlar todo y a todos. Es el desconcierto del que está inmerso en la cultura del ganar cueste lo que cueste; en esa cultura que sólo tiene espacio para los \textquote{vencedores} y al precio que sea. Un desconcierto que nace del miedo y del temor ante lo que nos cuestiona y pone en riesgo nuestras seguridades y verdades, nuestras formas de aferrarnos al mundo y a la vida. Y Herodes tuvo miedo, y ese miedo lo condujo a buscar seguridad en el crimen: \textquote{\emph{Necas parvulos corpore, quia te necat timor in corde}} (San Quodvultdeus, \emph{Sermo 2 sobre el símbolo}: \emph{PL}, 40, 655). Matas los niños en el cuerpo porque a ti el miedo te mata el corazón.

\emph{Queremos adorar}. Los hombres de Oriente fueron a adorar, y fueron a hacerlo al lugar propio de un rey: el Palacio. Y esto es importante, allí llegaron ellos con su búsqueda, era el lugar indicado: pues es propio de un rey nacer en un palacio, y tener su corte y súbditos. Es signo de poder, de éxito, de vida lograda. Y es de esperar que el rey sea venerado, temido y adulado, sí; pero no necesariamente amado. Esos son los esquemas mundanos, los pequeños ídolos a los que le rendimos culto: el culto al poder, a la apariencia y a la superioridad. Ídolos que solo prometen tristeza, esclavitud, miedo.

Y fue precisamente ahí donde comenzó el camino más largo que tuvieron que andar esos hombres venidos de lejos. Ahí comenzó la osadía más difícil y complicada. Descubrir que lo que ellos buscaban no estaba en el palacio sino que se encontraba en otro lugar, no sólo geográfico sino existencial. Allí no veían la estrella que los conducía a descubrir un Dios que quiere ser amado, y eso sólo es posible bajo el signo de la libertad y no de la tiranía; descubrir que la mirada de este Rey desconocido ―pero deseado― no humilla, no esclaviza, no encierra. Descubrir que la mirada de Dios levanta, perdona, sana. Descubrir que Dios ha querido nacer allí donde no lo esperamos, donde quizá no lo queremos. O donde tantas veces lo negamos. Descubrir que en la mirada de Dios hay espacio para los heridos, los cansados, los maltratados, abandonados: que su fuerza y su poder se llama misericordia. Qué lejos se encuentra, para algunos, Jerusalén de Belén.

Herodes no puede adorar porque no quiso y no pudo cambiar su mirada. No quiso dejar de rendirse culto a sí mismo creyendo que todo comenzaba y terminaba con él. No pudo adorar porque buscaba que lo adorasen. Los sacerdotes tampoco pudieron adorar porque sabían mucho, conocían las profecías, pero no estaban dispuestos ni a caminar ni a cambiar.

Los magos sintieron nostalgia, no querían más de lo mismo. Estaban acostumbrados, habituados y cansados de los Herodes de su tiempo. Pero allí, en Belén, había promesa de novedad, había promesa de gratuidad. Allí estaba sucediendo algo nuevo. Los magos pudieron adorar porque se animaron a caminar y postrándose ante el pequeño, postrándose ante el pobre, postrándose ante el indefenso, postrándose ante el extraño y desconocido Niño de Belén, allí descubrieron la Gloria de Dios.

\subsubsection{Ángelus (2017)} \emph{Queridos hermanos y hermanas, ¡buenos días!}

Hoy, celebramos la Epifanía del Señor, es decir, la manifestación de Jesús que brilla como luz para todas las gentes. Símbolo de esta luz que resplandece en el mundo y quiere iluminar la vida de cada uno es la estrella, que guió a los Magos a Belén. Ellos, dice el Evangelio, vieron \textquote{su estrella} (\emph{Mt} 2, 2) y decidieron seguirla: decidieron dejarse guiar por la estrella de Jesús.

También en nuestra vida existen diversas estrellas, luces que brillan y orientan. Depende de nosotros elegir cuáles seguir. Por ejemplo, hay luces intermitentes, que van y vienen, como las pequeñas satisfacciones de la vida: que aunque buenas, no son suficientes, porque duran poco y no dejan la paz que buscamos. Después están las luces cegadoras del primer plano, del dinero y del éxito, que prometen todo y enseguida: son seductoras, pero con su fuerza ciegan y hacen pasar de los sueños de gloria a la oscuridad más densa. Los Magos, en cambio, invitan a seguir una luz estable, una luz amable, que no se oculta, porque no es de este mundo: viene del cielo y resplandece\ldots{} ¿Dónde? En el corazón.

Esta luz verdadera es la luz del Señor, o mejor dicho, es el Señor mismo. Él es nuestra luz: una luz que no deslumbra, sino que acompaña y dona una alegría única. Esta luz es para todos y llama a cada uno: podemos escuchar así la actual invitación dirigida a nosotros por el profeta Isaías: \textquote{arriba, resplandece, que ha llegado tu luz} (60, 1). Así decía Isaías, profetizando esta alegría de hoy en Jerusalén: \textquote{arriba, resplandece, que ha llegado tu luz}. Al inicio de cada día podemos acoger esta invitación: arriba, resplandece, que ha llegado tu luz, sigue hoy, entre tantas estrellas fugaces en el mundo, la estrella luminosa de Jesús! Siguiéndola, tendremos alegría, como ocurrió a los Magos, que \textquote{al ver la estrella se llenaron de inmensa alegría} (\emph{Mt} 2, 10); porque donde esta Dios hay alegría. Quien ha encontrado a Jesús ha experimentado el milagro de la luz que rasga las tinieblas y conoce esta luz que ilumina y aclara. Querría, con mucho respeto, invitar a todos a no tener miedo de esta luz y a abrirse al Señor. Sobre todo querría decir a quien ha perdido la fuerza de buscar, está cansado, a quien, superado por las oscuridades de la vida, ha apagado el deseo: \textquote{¡Levántate, ánimo, la luz de Jesús sabe vencer las tinieblas más oscuras; levántate, ánimo!}.

Y ¿cómo encontrar esta luz divina? Sigamos el ejemplo de los Magos, que el Evangelio describe siempre en movimiento. Quien quiere la luz, efectivamente, sale de sí y busca: no permanece cerrado, quieto a ver qué cosa sucede al su alrededor, sino pone en juego su propia vida; sale de sí. La vida cristiana es un camino continuo, hecho de esperanza, hecho de búsqueda; un camino que, como aquel de los Magos, prosigue incluso cuando la estrella desaparece momentáneamente de la vista. En este camino hay también insidias que hay que evitar: las charlas superficiales y mundanas, que frenan el paso; los caprichos paralizantes del egoísmo; los agujeros del pesimismo, que atrapa a la esperanza. Estos obstáculos bloquearon a los escribas, de los que habla el Evangelio de hoy. Ellos sabían dónde estaba la luz, pero no se movieron. Cuando Herodes les pregunto: \textquote{¿Dónde nacerá el Mesías?} --- \textquote{¡En Belén!}. Sabían dónde, pero no se movieron. Su conocimiento fue en vano: sabían muchas cosas, pero para nada, todo en vano. No basta saber que Dios ha nacido, si no se hace con Él Navidad en el corazón. Dios ha nacido, sí, pero ¿Ha nacido en tú corazón? ¿Ha nacido en mí corazón? ¿Ha nacido en nuestro corazón? Y así le encontraremos, como los Magos, con María, José, en el establo.

Los Magos lo hicieron: encontraron al Niño, \textquote{postrándose, le adoraron} (v. 11). No le miraron solamente, dijeron solo una oración circunstancial y se fueron, no, sino que le adoraron: entraron en una comunión personal de amor con Jesús. Después le regalaron oro, incienso y mirra, es decir, sus bienes más preciados. Aprendamos de los Magos a no dedicar a Jesús sólo los ratos perdidos de tiempo y algún pensamiento de vez en cuando, de lo contrario no tendremos su luz. Como los Magos, pongámonos en camino, revistámonos de luz siguiendo la estrella de Jesús, y adoremos al Señor con todo nuestro ser.

\subsubsection{Homilía (2020)} {SANTA MISA EN LA SOLEMNIDAD DE LA EPIFANÍA DEL SEÑOR}

CAPILLA PAPAL

\textbf{\emph{HOMILÍA DEL SANTO PADRE FRANCISCO}}

\emph{Basílica Vaticana\\ Lunes, 6 de enero de 2020}

~

En el Evangelio (\emph{Mt} 2,1-12) hemos escuchado que los Magos comienzan manifestando sus intenciones: \textquote{Hemos visto salir su estrella y venimos a adorarlo} (v. 2). La adoración es la finalidad de su viaje, el objetivo de su camino. De hecho, cuando llegaron a Belén, \textquote{vieron al niño con María, su madre, y cayendo de rodillas lo adoraron} (v. 11). Si perdemos el sentido de la \emph{adoración}, perdemos el sentido de movimiento de la vida cristiana, que es un camino hacia el Señor, no hacia nosotros. Es el riesgo del que nos advierte el Evangelio, presentando, junto a los Reyes Magos, unos personajes que no logran adorar.

En primer lugar, está el rey Herodes, que usa el verbo adorar, pero de manera engañosa. De hecho, le pide a los Reyes Magos que le informen sobre el lugar donde estaba el Niño \textquote{para ir ---dice--- yo también a adorarlo} (v. 8). En realidad, Herodes sólo se adoraba a sí mismo y, por lo tanto, quería deshacerse del Niño con mentiras. ¿Qué nos enseña esto? Que el hombre, cuando no adora a \emph{Dios}, está orientado a adorar su \emph{yo}. E incluso la vida cristiana, sin adorar al Señor, puede convertirse en una forma educada de alabarse a uno mismo y el talento que se tiene: cristianos que no saben adorar, que no saben rezar adorando. Es un riesgo grave: servirnos de Dios en lugar de servir a Dios. Cuántas veces hemos cambiado los intereses del Evangelio por los nuestros, cuántas veces hemos cubierto de religiosidad lo que era cómodo para nosotros, cuántas veces hemos confundido el poder según Dios, que es servir a los demás, con el poder según el mundo, que es servirse a sí mismo.

Además de Herodes, hay otras personas en el Evangelio que no logran adorar: son los jefes de los sacerdotes y los escribas del pueblo. Ellos indican a Herodes con extrema precisión dónde nacería el Mesías: en Belén de Judea (cf. v. 5). Conocen las profecías y las citan exactamente. Saben a dónde ir ---grandes teólogos, grandes---, pero no van. También de esto podemos aprender una lección. En la vida cristiana no es suficiente saber: sin salir de uno mismo, sin encontrar, sin adorar, no se conoce a Dios. La teología y la eficiencia pastoral valen poco o nada si no se doblan las rodillas; si no se hace como los Magos, que no sólo fueron sabios organizadores de un viaje, sino que caminaron y adoraron. Cuando uno adora, se da cuenta de que la fe no se reduce a un conjunto de hermosas doctrinas, sino que es la relación con una Persona viva a quien amar. Conocemos el rostro de Jesús estando cara a cara con Él. Al adorar, descubrimos que la vida cristiana es una historia de amor con Dios, donde las buenas ideas no son suficientes, sino que se necesita ponerlo en primer lugar, como lo hace un enamorado con la persona que ama. Así debe ser la Iglesia, una adoradora enamorada de Jesús, su esposo.

Al inicio del año redescubrimos la adoración como una exigencia de fe. Si sabemos arrodillarnos ante Jesús, venceremos la tentación de ir cada uno por su camino. De hecho, adorar es hacer un éxodo de la esclavitud más grande, la de uno mismo. Adorar es poner al Señor en el centro para no estar más centrados en nosotros mismos. Es poner cada cosa en su lugar, dejando el primer puesto a Dios. Adorar es poner los planes de Dios antes que mi tiempo, que mis derechos, que mis espacios. Es aceptar la enseñanza de la Escritura: \textquote{Al Señor, tu Dios, adorarás} (\emph{Mt} 4,10). Tu Dios: adorar es experimentar que, con Dios, nos pertenecemos recíprocamente. Es darle del \textquote{tú} en la intimidad, es presentarle la vida y permitirle entrar en nuestras vidas. Es hacer descender su consuelo al mundo. Adorar es descubrir que para rezar basta con decir: \textquote{¡Señor mío y Dios mío!} (\emph{Jn} 20,28), y dejarnos llenar de su ternura.

Adorar es encontrarse con Jesús sin la lista de peticiones, pero con la única solicitud de estar con Él. Es descubrir que la alegría y la paz crecen con la alabanza y la acción de gracias. Cuando adoramos, permitimos que Jesús nos sane y nos cambie. Al adorar, le damos al Señor la oportunidad de transformarnos con su amor, de iluminar nuestra oscuridad, de darnos fuerza en la debilidad y valentía en las pruebas. Adorar es ir a lo esencial: es la forma de desintoxicarse de muchas cosas inútiles, de adicciones que adormecen el corazón y aturden la mente. De hecho, al adorar uno aprende a rechazar lo que no debe ser adorado: el dios del dinero, el dios del consumo, el dios del placer, el dios del éxito, nuestro yo erigido en dios. Adorar es hacerse pequeño en presencia del Altísimo, descubrir ante Él que la grandeza de la vida no consiste en tener, sino en amar. Adorar es redescubrirnos hermanos y hermanas frente al misterio del amor que supera toda distancia: es obtener el bien de la fuente, es encontrar en el Dios cercano la valentía para aproximarnos a los demás. Adorar es saber guardar silencio ante la Palabra divina, para aprender a decir palabras que no duelen, sino que consuelan.

La adoración es un gesto de amor que cambia la vida. Es actuar como los Magos: es traer oro al Señor, para decirle que nada es más precioso que Él; es ofrecerle incienso, para decirle que sólo con Él puede elevarse nuestra vida; es presentarle mirra, con la que se ungían los cuerpos heridos y destrozados, para pedirle a Jesús que socorra a nuestro prójimo que está marginado y sufriendo, porque allí está Él. Por lo general, sabemos cómo orar ---le pedimos, le agradecemos al Señor---, pero la Iglesia debe ir aún más allá con la oración de adoración, debemos crecer en la adoración. Es una sabiduría que debemos aprender todos los días. Rezar adorando: la oración de adoración.

Queridos hermanos y hermanas, hoy cada uno de nosotros puede preguntarse: \textquote{¿Soy un adorador cristiano?}. Muchos cristianos que oran no saben adorar. Hagámonos esta pregunta. ¿Encontramos momentos para la adoración en nuestros días y creamos espacios para la adoración en nuestras comunidades? Depende de nosotros, como Iglesia, poner en práctica las palabras que rezamos hoy en el Salmo: \textquote{Señor, que todos los pueblos te adoren}. Al adorar, nosotros también descubriremos, como los Magos, el significado de nuestro camino. Y, como los Magos, experimentaremos una \textquote{inmensa alegría} (\emph{Mt} 2,10).



\subsubsection{Ángelus (2020)}

\emph{Plaza de San Pedro\\ Lunes, 6 de enero de 2020}

~

\emph{Queridos hermanos y hermanas, ¡buenos días!}

Celebramos la solemnidad de la Epifanía, en memoria de los Magos que vinieron de Oriente a Belén, siguiendo la estrella, para visitar al Mesías recién nacido. Al final del relato evangélico se dice que los Magos \textquote{avisados en sueños que no volvieran donde Herodes, se retiraron a su país por otro camino} (v. 12). Por otro camino.

Estos sabios, procedentes de regiones lejanas, después de haber viajado mucho, encuentran al que querían conocer, después de haberlo buscado durante mucho tiempo, seguramente también con mucho trabajo y vicisitudes. Y cuando finalmente llegan a su destino, se postran ante el Niño, lo adoran, le ofrecen sus preciosos regalos. Después de eso, se pusieron en marcha de nuevo sin demora para volver a su tierra. Pero ese encuentro con el Niño los ha cambiado.

El encuentro con Jesús no detiene a los Reyes Magos, al contrario, les da un nuevo impulso para volver a su país, para contar lo que han visto y la alegría que han sentido. En esto hay una demostración del estilo de Dios, de su modo de manifestarse en la historia. La experiencia de Dios no nos bloquea, sino que nos libera; no nos aprisiona, sino que nos devuelve al camino, nos devuelve a los lugares habituales de nuestra existencia. Los lugares son y serán los mismos, pero nosotros, después del encuentro con Jesús, no somos los mismos que antes. El encuentro con Jesús nos cambia, nos transforma. El evangelista Mateo subraya que los Reyes Magos regresaron \textquote{por otro camino} (v. 12). La advertencia del ángel los lleva a cambiar sus caminos para no encontrarse con Herodes y sus tramas de poder.

Cada experiencia de encuentro con Jesús nos lleva a tomar caminos diferentes, porque de Él proviene una fuerza buena que sana el corazón y nos aparta del mal.

Existe una sabia dinámica entre continuidad y novedad: vuelven \textquote{a su país}, pero \textquote{por otro camino}. Esto indica que somos nosotros los que debemos cambiar, los que debemos transformar nuestra forma de vida, aunque sea en el mismo ambiente de siempre, los que debemos cambiar los criterios de juicio sobre la realidad que nos rodea. Esta es la diferencia entre el verdadero Dios y los ídolos traidores, como el dinero, el poder, el éxito\ldots{}; entre Dios y aquellos que prometen darte estos ídolos, como los magos, los adivinos, los hechiceros. La diferencia es que los ídolos nos atan a sí mismos, nos hacen dependientes de los ídolos, y nosotros tomamos posesión de ellos. El verdadero Dios no nos retiene ni se deja retener por nosotros: nos abre caminos de novedad y de libertad, porque es Padre que está siempre con nosotros para hacernos crecer.

Si te encuentras con Jesús, si tienes un encuentro espiritual con Jesús, recuerda: debes volver a los mismos lugares de siempre, pero de otra manera, con otro estilo. Es así, es el Espíritu Santo, que Jesús nos da, que nos cambia el corazón.

Pidamos a la Santa Virgen que podamos convertirnos en testigos de Cristo allá donde estemos, con una vida nueva, transformada por su amor.



\subsubsection{Homilía ()}

\subsubsection{Homilía ()}

\section{Bautismo del Señor} \subsubsection{Homilía (2017)} FIESTA DEL BAUTISMO DEL SEÑOR\\ CELEBRACIÓN DE LA SANTA MISA Y BAUTISMO DE ALGUNOS NIÑOS

\textbf{\emph{HOMILÍA DEL SANTO PADRE FRANCISCO}}

\emph{Capilla Sixtina\\ Domingo 8 de enero de 2017}


\emph{Queridos padres:}

Vosotros habéis pedido para vuestros niños la fe, la fe que será dada en el Bautismo. La fe: eso significa vida de fe, porque la fe es vivida; caminar por el camino de la fe y dar testimonio de la fe. La fe no es recitar el \textquote{Credo} el domingo, cuando vamos a misa: no es solo esto. La fe es creer lo que es la Verdad: Dios Padre que ha enviado a su Hijo y al Espíritu que nos vivifica. Pero la fe es también encomendarse a Dios, y esto vosotros se lo tenéis que enseñar a ellos, con vuestro ejemplo, con vuestra vida.

Y la fe es luz: en la ceremonia del Bautismo se os dará una vela encendida, como en los primeros tiempos de la Iglesia. Y por esto el Bautismo, en esos tiempos, se llamaba \textquote{iluminación}, porque la fe ilumina el corazón, hace ver las cosas con otra luz. Vosotros habéis pedido la fe: la Iglesia da la fe a vuestros hijos con el Bautismo, y vosotros tenéis el deber de hacerla crecer, cuidarla, y que se convierta en testimonio para todos los demás. Este es el sentido de esta ceremonia. Y solamente quería deciros esto: cuidar la fe, hacerla crecer, que sea testimonio para los demás.

Y después\ldots{} ¡ha comenzado el concierto! {[}los niños lloran{]}: es porque los niños se encuentran en un lugar que no conocen, se han despertado antes de lo normal. Comienza uno, da la nota y después otros \textquote{imitan}\ldots{} Algunos lloran solamente porque ha llorado el otro\ldots{} Jesús hizo lo mismo, ¿sabéis? A mí me gusta pensar que la primera predicación de Jesús en el establo fue un llanto, la primera\ldots{} Y después, como la ceremonia es un poco larga, alguno llora por hambre. Si es así, vosotras madres amamantadles también, sin miedo, con toda normalidad. Como la Virgen amamantaba a Jesús\ldots{}

No olvidéis: habéis pedido la fe, a vosotros la tarea de cuidar la fe, hacerla crecer, que sea testimonio para todos nosotros, para todos nosotros: también para nosotros sacerdotes, obispos, todos. Gracias.

\subsubsection{Ángelus (2017)}

\emph{Queridos hermanos y hermanas, ¡buenos días!}

Hoy, fiesta del Bautismo de Jesús, el Evangelio (\emph{Mt} 3, 13-17) nos presenta la episodio ocurrido a orillas del río Jordán: en medio de la muchedumbre penitente que avanza hacia Juan Bautista para recibir el Bautismo también se encuentra Jesús ---hacía fila---. Juan querría impedírselo diciendo: \textquote{Soy yo el que necesita ser bautizado por ti} (\emph{Mt} 3, 14). En efecto, el Bautista es consciente de la gran distancia que hay entre él y Jesús. Pero Jesús vino precisamente para colmar la distancia entre el hombre y Dios: si Él está completamente de parte de Dios también está completamente de parte del hombre, y reúne aquello que estaba dividido. Por eso pide a Juan que le bautice, para que se cumpla toda justicia (cf. v. 15), es decir, se realice el proyecto del Padre, que pasa a través de la vía de la obediencia y de la solidaridad con el hombre frágil y pecador, la vía de la humildad y de la plena cercanía de Dios a sus hijos. ¡Porque Dios está muy cerca de nosotros, mucho!

En el momento en el que Jesús, bautizado por Juan, sale de las aguas del río Jordán, la voz de Dios Padre se hace oír desde lo alto: \textquote{Este es mi Hijo amado, en quien me complazco} (v. 17). Y al mismo tiempo el Espíritu Santo, en forma de paloma, se posa sobre Jesús, que da públicamente inicio a su misión de salvación; misión caracterizada por un estilo, el estilo del siervo humilde y dócil, dotado sólo de la fuerza de la verdad, como había profetizado Isaías: \textquote{no vociferará ni alzará el tono, {[}\ldots{}{]} la caña quebrada no partirá, y la mecha mortecina no apagará. Lealmente hará justicia} (42, 2-3). Siervo humilde y manso, he aquí el estilo de Jesús, y también el estilo misionero de los discípulos de Cristo: anunciar el Evangelio con docilidad y firmeza, sin gritar, sin regañar a alguien, sino con docilidad y firmeza, sin arrogancia o imposición. La verdadera misión nunca es proselitismo sino atracción a Cristo. ¿Pero cómo? ¿Cómo se hace esta atracción a Cristo? Con el propio testimonio, a partir de la fuerte unión con Él en la oración, en la adoración y en la caridad concreta, que es servicio a Jesús presente en el más pequeño de los hermanos. Imitando a Jesús, pastor bueno y misericordioso, y animados por su gracia, estamos llamados a hacer de nuestra vida un testimonio alegre que ilumina el camino, que lleva esperanza y amor.

Esta fiesta nos hace redescubrir el don y la belleza de ser un pueblo de bautizados, es decir, de pecadores ---todos lo somos--- de pecadores salvados por la gracia de Cristo, inseridos realmente, por obra del Espíritu Santo, en la relación filial de Jesús con el Padre, acogidos en el seno de la madre Iglesia, hechos capaces de una fraternidad que no conoce confines ni barreras.

Que la Virgen María nos ayude a todos nosotros cristianos a conservar una conciencia siempre viva y agradecida de nuestro Bautismo y a recorrer con fidelidad el camino inaugurado por este Sacramento de nuestro renacimiento. Y siempre humildad, docilidad y firmeza.

\subsubsection{Homilía (2020)} \emph{Capilla Sixtina\\ Domingo, 12 de enero de 2020}

\begin{center}\rule{0.5\linewidth}{\linethickness}\end{center}

~

Como Jesús, que fue a hacerse bautizar, así hacéis vosotros con vuestros hijos.

Jesús responde a Juan: \textquote{Hágase toda justicia} (cf. \emph{Mt} 3,15). Bautizar a un hijo es un acto de justicia para él. ¿Y por qué? Porque nosotros con el Bautismo le damos un tesoro, nosotros con el Bautismo le damos en prenda el \emph{Espíritu Santo}. El niño sale {[}del Bautismo{]} con la fuerza del Espíritu en su interior: el Espíritu que lo defenderá, que lo ayudará, durante toda su vida. Por eso es tan importante bautizarlos cuando son pequeños, para que crezcan con la fuerza del Espíritu Santo.

Este es el mensaje que quisiera daros hoy. Vosotros traéis hoy a vuestros hijos, {[}para que tengan{]} el Espíritu Santo dentro de ellos. Y cuidad de que crezcan con la luz, con la fuerza del Espíritu Santo, a través de la catequesis, la ayuda, la enseñanza, los ejemplos que les daréis en casa\ldots{} Este es el mensaje.

No quisiera deciros nada más importante. Sólo una advertencia. Los niños no están acostumbrados a venir a la Sixtina, ¡es la primera vez! Tampoco están acostumbrados a estar en un ambiente algo caluroso. Y no están acostumbrados a vestirse así para una fiesta tan hermosa como la de hoy. Se sentirán un poco incómodos en algún momento. Y uno empezará a llorar\ldots{} ―¡El concierto no ha empezado todavía!― pero empezará uno, luego otro\ldots{} No os asustéis, dejad que los niños lloren y griten. A lo mejor si tu niño llora y se queja, quizás sea porque tiene demasiado calor: quitadle algo; o porque tiene hambre: dale de mamar, aquí, sí, siempre en paz. Es algo que dije también el año pasado: tienen una dimensión \textquote{coral}: es suficiente que uno dé la primera nota y empiezan todos y habrá un concierto. No os asustéis. Es un sermón muy bonito el de un niño que llora en una iglesia. Haced que esté cómodo y sigamos adelante.

No lo olvidéis: vosotros lleváis el Espíritu Santo a los niños.



\subsubsection{Ángelus (2020)} \emph{Plaza de San Pedro\\ Domingo, 12 de enero de 2020}

\begin{center}\rule{0.5\linewidth}{\linethickness}\end{center}

~

\emph{Queridos hermanos y hermanas}, ¡buenos días!

Una vez más he tenido la alegría de bautizar a algunos niños en la fiesta de hoy del Bautismo del Señor. Hoy eran treinta y dos. Recemos por ellos y sus familias.

La liturgia de este año nos propone el acontecimiento del bautismo de Jesús según el relato evangélico de Mateo (cf. 3, 13-17). El evangelista describe el diálogo entre Jesús, que pide el bautismo, y Juan el Bautista, que se niega y observa: \textquote{Soy yo el que necesita ser bautizado por ti, ¿y tú vienes a mí?} (v. 14). Esta decisión de Jesús sorprende al Bautista: de hecho, el Mesías no necesita ser purificado, sino que es Él quien purifica. Pero Dios es Santo, sus caminos no son los nuestros, y Jesús es el Camino de Dios, un camino impredecible. Recordemos que Dios es el Dios de las sorpresas.

Juan había declarado que existía una distancia abismal e insalvable entre él y Jesús. \textquote{No soy digno de llevarle las sandalias} (\emph{Mateo} 3, 11), dijo. Pero el Hijo de Dios vino precisamente para salvar esta distancia entre el hombre y Dios. Si Jesús está del lado de Dios, también está del lado del hombre, y reúne lo que estaba dividido. Por eso le respondió a Juan: \textquote{Déjame ahora, pues conviene que así cumplamos toda justicia} (v. 15). El Mesías pide ser bautizado para que se cumpla toda justicia, para que se realice el proyecto del Padre, que pasa por el camino de la obediencia filial y de la solidaridad con el hombre frágil y pecador. Es el camino de la humildad y de la plena cercanía de Dios a sus hijos.

El profeta Isaías proclama también la justicia del Siervo de Dios, que lleva a cabo su misión en el mundo con un estilo contrario al espíritu mundano: \textquote{No vociferará ni alzará el tono, y no hará oír en la calle su voz. Caña quebrada no partirá, y mecha mortecina no apagará} (42, 2-3). Es la actitud de mansedumbre ―es lo que Jesús nos enseña con su humildad, la mansedumbre―, la actitud de sencillez, respeto, moderación y ocultamiento, que se requiere aún hoy de los discípulos del Señor. Cuántos ―es triste decirlo―, cuántos discípulos del Señor alardean como discípulos del Señor. No es un buen discípulo el que alardea de ello. El buen discípulo es el humilde, el manso que hace el bien sin ser visto. En la acción misionera, la comunidad cristiana está llamada a salir al encuentro de los demás siempre proponiendo y no imponiendo, dando testimonio, compartiendo la vida concreta de la gente.

Tan pronto como Jesús fue bautizado en el río Jordán, los cielos se abrieron y el Espíritu Santo descendió sobre él como una paloma, mientras que desde lo alto resonaba una voz que decía: \textquote{Este es mi Hijo amado; en el que me complazco} (\emph{Mateo} 3, 17). En la fiesta del Bautismo de Jesús redescubrimos nuestro bautismo. Así como Jesús es el Hijo amado del Padre, también nosotros, renacidos del agua y del Espíritu Santo, sabemos que somos hijos amados ―¡el Padre nos ama a todos!―, que somos objeto de la satisfacción de Dios, hermanos y hermanas de muchos otros, con una gran misión de testimoniar y anunciar a todos los hombres y mujeres el amor ilimitado del Padre.

Esta fiesta del Bautismo de Jesús nos recuerda nuestro bautismo. Nosotros también renacemos en el bautismo. En el bautismo el Espíritu Santo vino a permanecer en nosotros. Por eso es importante saber la fecha del bautismo. Sabemos la fecha de nuestro nacimiento, pero no siempre sabemos la fecha de nuestro bautismo. Seguramente algunos de vosotros no la saben\ldots{} Una tarea. Cuando regreses a casa pregunta: ¿Cuándo fui bautizada? ¿Cuándo fui bautizado? Y celebra la fecha de tu bautismo en tu corazón cada año. Hazlo. Es también un deber de justicia hacia el Señor que ha sido tan bueno con nosotros.

Que María Santísima nos ayude a comprender cada vez más el don del bautismo y a vivirlo coherentemente en las situaciones cotidianas.

\subsubsection{Homilía ()}

\subsubsection{Homilía ()}
