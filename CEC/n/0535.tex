\n{535} El comienzo (cf. \emph{Lc} 3, 23) de la vida pública de Jesús es su bautismo por Juan en el Jordán (cf. \emph{Hch} 1, 22). Juan proclamaba \textquote{un bautismo de conversión para el perdón de los pecados} (\emph{Lc} 3, 3). Una multitud de pecadores, publicanos y soldados (cf. \emph{Lc} 3, 10-14), fariseos y saduceos (cf. \emph{Mt} 3, 7) y prostitutas (cf. \emph{Mt} 21, 32) viene a hacerse bautizar por él. \textquote{Entonces aparece Jesús}. El Bautista duda. Jesús insiste y recibe el bautismo. Entonces el Espíritu Santo, en forma de paloma, viene sobre Jesús, y la voz del cielo proclama que él es \textquote{mi Hijo amado} (\emph{Mt} 3, 13-17). Es la manifestación (\textquote{Epifanía}) de Jesús como Mesías de Israel e Hijo de Dios.