\n{1972} La Ley nueva es llamada \emph{ley de amor}, porque hace obrar por el amor que infunde el Espíritu Santo más que por el temor; \emph{ley de gracia}, porque confiere la fuerza de la gracia para obrar mediante la fe y los sacramentos; \emph{ley de libertad} (cf. \emph{St} 1, 25; 2, 12), porque nos libera de las observancias rituales y jurídicas de la Ley antigua, nos inclina a obrar espontáneamente bajo el impulso de la caridad y nos hace pasar de la condición del siervo \textquote{que ignora lo que hace su señor}, a la de amigo de Cristo, \textquote{porque todo lo que he oído a mi Padre os lo he dado a conocer} (\emph{Jn} 15, 15), o también a la condición de hijo heredero (cf. \emph{Ga} 4, 1-7. 21-31; \emph{Rm} 8, 15).