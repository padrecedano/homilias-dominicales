\n{442} No ocurre así con Pedro cuando confiesa a Jesús como \textquote{el Cristo, el Hijo de Dios vivo} (\emph{Mt} 16, 16) porque Jesús le responde con solemnidad \textquote{\emph{no te ha revelado} esto ni la carne ni la sangre, sino \emph{mi Padre} que está en los cielos} (\emph{Mt} 16, 17). Paralelamente Pablo dirá a propósito de su conversión en el camino de Damasco: \textquote{Cuando Aquel que me separó desde el seno de mi madre y me llamó por su gracia, tuvo a bien revelar en mí a su Hijo para que le anunciase entre los gentiles\ldots{}} (\emph{Ga} 1,15-16). \textquote{Y en seguida se puso a predicar a Jesús en las sinagogas: que él era el Hijo de Dios} (\emph{Hch} 9, 20). Este será, desde el principio (cf. \emph{1 Ts} 1, 10), el centro de la fe apostólica (cf. \emph{Jn} 20, 31) profesada en primer lugar por Pedro como cimiento de la Iglesia (cf. \emph{Mt} 16, 18).