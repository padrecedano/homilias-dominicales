\n{608} Juan Bautista, después de haber aceptado bautizarle en compañía de los pecadores (cf. \emph{Lc} 3, 21; \emph{Mt} 3, 14-15), vio y señaló a Jesús como el \textquote{Cordero de Dios que quita los pecados del mundo} (\emph{Jn} 1, 29; cf. \emph{Jn} 1, 36). Manifestó así que Jesús es a la vez el Siervo doliente que se deja llevar en silencio al matadero (\emph{Is} 53, 7; cf. \emph{Jr} 11, 19) y carga con el pecado de las multitudes (cf. \emph{Is} 53, 12) y el cordero pascual símbolo de la redención de Israel cuando celebró la primera Pascua (\emph{Ex} 12, 3-14; cf. \emph{Jn} 19, 36; \emph{1 Co} 5, 7). Toda la vida de Cristo expresa su misión: \textquote{Servir y dar su vida en rescate por muchos} (\emph{Mc} 10, 45).