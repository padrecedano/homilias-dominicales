	\n{1429} De ello da testimonio la conversión de san Pedro tras la triple negación de su Maestro. La mirada de infinita misericordia de Jesús provoca las lágrimas del arrepentimiento (\emph{Lc} 22,61) y, tras la resurrección del Señor, la triple afirmación de su amor hacia él (cf. \emph{Jn} 21,15-17). La segunda conversión tiene también una dimensión \emph{comunitaria}. Esto aparece en la llamada del Señor a toda la Iglesia: \textquote{¡Arrepiéntete!} (\emph{Ap} 2,5.16).
	
	San Ambrosio dice acerca de las dos conversiones que, \textquote{en la Iglesia, existen el agua y las lágrimas: el agua del Bautismo y las lágrimas de la Penitencia} (\emph{Epistula extra collectionem} 1 {[}41{]}, 12).