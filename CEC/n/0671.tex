%r
\ccesec{\ldots{} esperando que todo le sea sometido}

\n{671} El Reino de Cristo, presente ya en su Iglesia, sin embargo, no está todavía acabado \textquote{con gran poder y gloria} (\emph{Lc} 21, 27; cf. \emph{Mt} 25, 31) con el advenimiento del Rey a la tierra. Este Reino aún es objeto de los ataques de los poderes del mal (cf. \emph{2 Ts} 2, 7), a pesar de que estos poderes hayan sido vencidos en su raíz por la Pascua de Cristo. Hasta que todo le haya sido sometido (cf. \emph{1 Co} 15, 28), y \textquote{mientras no [\ldots{}] haya nuevos cielos y nueva tierra, en los que habite la justicia, la Iglesia peregrina lleva en sus sacramentos e instituciones, que pertenecen a este tiempo, la imagen de este mundo que pasa. Ella misma vive entre las criaturas que gimen en dolores de parto hasta ahora y que esperan la manifestación de los hijos de Dios} (LG 48). Por esta razón los cristianos piden, sobre todo en la Eucaristía (cf. \emph{1 Co} 11, 26), que se apresure el retorno de Cristo (cf. \emph{2 P} 3, 11-12) cuando suplican: \textquote{Ven, Señor Jesús} (\emph{Ap} 22, 20; cf. \emph{1 Co} 16, 22; \emph{Ap} 22, 17-20).