\n{672} Cristo afirmó antes de su Ascensión que aún no era la hora del establecimiento glorioso del Reino mesiánico esperado por Israel (cf. \emph{Hch} 1, 6-7) que, según los profetas (cf. \emph{Is} 11, 1-9), debía traer a todos los hombres el orden definitivo de la justicia, del amor y de la paz. El tiempo presente, según el Señor, es el tiempo del Espíritu y del testimonio (cf. \emph{Hch} 1, 8), pero es también un tiempo marcado todavía por la \textquote{tribulación} (\emph{1 Co} 7, 26) y la prueba del mal (cf. \emph{Ef} 5, 16) que afecta también a la Iglesia (cf. \emph{1 P} 4, 17) e inaugura los combates de los últimos días (\emph{1 Jn} 2, 18; 4, 3; \emph{1 Tm} 4, 1). Es un tiempo de espera y de vigilia (cf. \emph{Mt} 25, 1-13; \emph{Mc} 13, 33-37).