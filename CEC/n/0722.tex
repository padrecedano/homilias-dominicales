	\n{722} El Espíritu Santo \emph{preparó} a María con su gracia. Convenía que fuese \textquote{llena de gracia} la Madre de Aquel en quien \textquote{reside toda la plenitud de la divinidad corporalmente} (\emph{Col} 2, 9). Ella fue concebida sin pecado, por pura gracia, como la más humilde de todas las criaturas, la más capaz de acoger el don inefable del Omnipotente. Con justa razón, el ángel Gabriel la saluda como la \textquote{Hija de Sión}: \textquote{Alégrate} (cf. \emph{So} 3, 14; \emph{Za} 2, 14). Cuando ella lleva en sí al Hijo eterno, hace subir hasta el cielo con su cántico al Padre, en el Espíritu Santo, la acción de gracias de todo el pueblo de Dios y, por tanto, de la Iglesia (cf. \emph{Lc} 1, 46-55).
