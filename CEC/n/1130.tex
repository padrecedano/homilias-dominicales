%
\ccesec{Sacramentos de la vida eterna}

\n{1130} La Iglesia celebra el Misterio de su Señor \textquote{hasta que él venga} y \textquote{Dios sea todo en todos} (\emph{1 Co} 11, 26; 15, 28). Desde la era apostólica, la liturgia es atraída hacia su término por el gemido del Espíritu en la Iglesia: \emph{¡Marana tha!} (\emph{1 Co} 16,22). La liturgia participa así en el deseo de Jesús: \textquote{Con ansia he deseado comer esta Pascua con vosotros [\ldots{}] hasta que halle su cumplimiento en el Reino de Dios} (\emph{Lc} 22,15-16). En los sacramentos de Cristo, la Iglesia recibe ya las arras de su herencia, participa ya en la vida eterna, aunque \textquote{aguardando la feliz esperanza y la manifestación de la gloria del Gran Dios y Salvador nuestro Jesucristo} (\emph{Tt} 2,13). \textquote{El Espíritu y la Esposa dicen: ¡Ven! [\ldots{}] ¡Ven, Señor Jesús!} (\emph{Ap} 22,17.20).

	\begin{quote} 	
		Santo Tomás resume así las diferentes dimensiones del signo sacramental: \textquote{\emph{Unde sacramentum est signum rememorativum eius quod praecessit, scilicet passionis Christi; et desmonstrativum eius quod in nobis efficitur per Christi passionem, scilicet gratiae; et prognosticum, id est, praenuntiativum futurae gloriae}} (\textquote{Por eso el sacramento es un signo que rememora lo que sucedió, es decir, la pasión de Cristo; es un signo que demuestra lo que se realiza en nosotros en virtud de la pasión de Cristo, es decir, la gracia; y es un signo que anticipa, es decir, que preanuncia la gloria venidera}) (\emph{Summa theologiae} 3, q. 60, a. 3, c.) 	
	\end{quote}