\ccesec{\textquote{Alégrate, llena de gracia}}
	
\n{721} María, la Santísima Madre de Dios, la siempre Virgen, es la obra maestra de la Misión del Hijo y del Espíritu Santo en la Plenitud de los tiempos. Por primera vez en el designio de Salvación y porque su Espíritu la ha preparado, el Padre encuentra la Morada en donde su Hijo y su Espíritu pueden habitar entre los hombres. Por ello, los más bellos textos sobre la Sabiduría, la Tradición de la Iglesia los ha entendido frecuentemente con relación a María (cf. \emph{Pr} 8, 1-9, 6; \emph{Si} 24): María es cantada y representada en la Liturgia como el \textquote{Trono de la Sabiduría}.
	
%En ella comienzan a manifestarse las \textquote{maravillas de Dios}, que el Espíritu va a realizar en Cristo y en la Iglesia: