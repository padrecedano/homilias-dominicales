\n{444} Los evangelios narran en dos momentos solemnes, el Bautismo y la Transfiguración de Cristo, que la voz del Padre lo designa como su \textquote{Hijo amado} (\emph{Mt} 3, 17; 17, 5). Jesús se designa a sí mismo como \textquote{el Hijo Único de Dios} (\emph{Jn} 3, 16) y afirma mediante este título su preexistencia eterna (cf. \emph{Jn} 10, 36). Pide la fe en \textquote{el Nombre del Hijo Único de Dios} (\emph{Jn} 3, 18). Esta confesión cristiana aparece ya en la exclamación del centurión delante de Jesús en la cruz: \textquote{Verdaderamente este hombre era Hijo de Dios} (\emph{Mc} 15, 39), porque es solamente en el misterio pascual donde el creyente puede alcanzar el sentido pleno del título \textquote{Hijo de Dios}.