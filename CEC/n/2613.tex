	\n{2613} San Lucas nos ha trasmitido tres \emph{parábolas} principales sobre la oración:
	
	La primera, \textquote{el amigo importuno} (cf. \emph{Lc} 11, 5-13), invita a una oración insistente: \textquote{Llamad y se os abrirá}. Al que ora así, el Padre del cielo \textquote{le dará todo lo que necesite}, y sobre todo el Espíritu Santo que contiene todos los dones.
	
	La segunda, \textquote{la viuda importuna} (cf. \emph{Lc} 18, 1-8), está centrada en una de las cualidades de la oración: es necesario orar siempre, sin cansarse, con la \emph{paciencia} de la fe. \textquote{Pero, cuando el Hijo del hombre venga, ¿encontrará fe sobre la tierra?}.
	
	La tercera parábola, \textquote{el fariseo y el publicano} (cf. \emph{Lc} 18, 9-14), se refiere a la \emph{humildad} del corazón que ora. \textquote{Oh Dios, ten compasión de mí que soy pecador}. La Iglesia no cesa de hacer suya esta oración: ¡\emph{Kyrie eleison!}
