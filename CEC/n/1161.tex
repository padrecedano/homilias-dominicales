	\n{1161} Todos los signos de la celebración litúrgica hacen referencia a Cristo: también las imágenes sagradas de la Santísima Madre de Dios y de los santos. Significan, en efecto, a Cristo que es glorificado en ellos. Manifiestan \textquote{la nube de testigos} (\emph{Hb} 12,1) que continúan participando en la salvación del mundo y a los que estamos unidos, sobre todo en la celebración sacramental. A través de sus iconos, es el hombre \textquote{a imagen de Dios}, finalmente transfigurado \textquote{a su semejanza} (cf. \emph{Rm} 8,29; \emph{1 Jn} 3,2), quien se revela a nuestra fe, e incluso los ángeles, recapitulados también en Cristo:
	
	\begin{quote}
		\textquote{Siguiendo [\ldots{}] la enseñanza divinamente inspirada de nuestros santos Padres y la Tradición de la Iglesia católica (pues reconocemos ser del Espíritu Santo que habita en ella), definimos con toda exactitud y cuidado que la imagen de la preciosa y vivificante cruz, así como también las venerables y santas imágenes, tanto las pintadas como las de mosaico u otra materia conveniente, se expongan en las santas iglesias de Dios, en los vasos sagrados y ornamentos, en las paredes y en cuadros, en las casas y en los caminos: tanto las imágenes de nuestro Señor Dios y Salvador Jesucristo, como las de nuestra Señora inmaculada la santa Madre de Dios, de los santos ángeles y de todos los santos y justos} (Concilio de Nicea II: DS 600).
	\end{quote}
