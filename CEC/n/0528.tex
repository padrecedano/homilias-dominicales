\n{528} La \emph{Epifanía} es la manifestación de Jesús como Mesías de Israel, Hijo de Dios y Salvador del mundo. Con el bautismo de Jesús en el Jordán y las bodas de Caná (cf. \emph{Solemnidad de la Epifanía del Señor}, Antífona del \textquote{Magnificat} en II Vísperas, LH), la Epifanía celebra la adoración de Jesús por unos \textquote{magos} venidos de Oriente (\emph{Mt} 2, 1) En estos \textquote{magos}, representantes de religiones paganas de pueblos vecinos, el Evangelio ve las primicias de las naciones que acogen, por la Encarnación, la Buena Nueva de la salvación. La llegada de los magos a Jerusalén para \textquote{rendir homenaje al rey de los Judíos} (\emph{Mt} 2, 2) muestra que buscan en Israel, a la luz mesiánica de la estrella de David (cf. \emph{Nm} 24, 17; \emph{Ap} 22, 16) al que será el rey de las naciones (cf. \emph{Nm} 24, 17-19). Su venida significa que los gentiles no pueden descubrir a Jesús y adorarle como Hijo de Dios y Salvador del mundo sino volviéndose hacia los judíos (cf. \emph{Jn} 4, 22) y recibiendo de ellos su promesa mesiánica tal como está contenida en el Antiguo Testamento (cf. \emph{Mt} 2, 4-6). La Epifanía manifiesta que \textquote{la multitud de los gentiles entra en la familia de los patriarcas} (San León Magno, \emph{Sermones}, 23: PL 54, 224B) y adquiere la \emph{israelitica dignitas} (la dignidad israelítica) (Vigilia pascual, Oración después de la tercera lectura: \emph{Misal Romano}).