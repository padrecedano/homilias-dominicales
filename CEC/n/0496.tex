	\ccesec{La virginidad de María}
	
	\n{496} Desde las primeras formulaciones de la fe (cf. DS 10-64), la Iglesia ha confesado que Jesús fue concebido en el seno de la Virgen María únicamente por el poder del Espíritu Santo, afirmando también el aspecto corporal de este suceso: Jesús fue concebido \textquote{absque semine ex Spiritu Sancto} (Cc Letrán, año 649; DS 503), esto es, sin elemento humano, por obra del Espíritu Santo. Los Padres ven en la concepción virginal el signo de que es verdaderamente el Hijo de Dios el que ha venido en una humanidad como la nuestra:
	
	\begin{quote}
		Así, S. Ignacio de Antioquía (comienzos del siglo II): \textquote{Estáis firmemente convencidos acerca de que nuestro Señor es verdaderamente de la raza de David según la carne (cf. Rm 1, 3), Hijo de Dios según la voluntad y el poder de Dios (cf. Jn 1, 13), nacido verdaderamente de una virgen, \ldots{} Fue verdaderamente clavado por nosotros en su carne bajo Poncio Pilato \ldots{} padeció verdaderamente, como también resucitó verdaderamente} (Smyrn. 1-2).
	\end{quote}