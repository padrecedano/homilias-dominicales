	\n{333} De la Encarnación a la Ascensión, la vida del Verbo encarnado está rodeada de la adoración y del servicio de los ángeles. Cuando Dios introduce \textquote{a su Primogénito en el mundo, dice: \textquote{adórenle todos los ángeles de Dios}} (\emph{Hb} 1, 6). Su cántico de alabanza en el nacimiento de Cristo no ha cesado de resonar en la alabanza de la Iglesia: \textquote{Gloria a Dios\ldots{}} (\emph{Lc} 2, 14). Protegen la infancia de Jesús (cf. \emph{Mt} 1, 20; 2, 13.19), le sirven en el desierto (cf. \emph{Mc} 1, 12; \emph{Mt} 4, 11), lo reconfortan en la agonía (cf. \emph{Lc} 22, 43), cuando Él habría podido ser salvado por ellos de la mano de sus enemigos (cf. \emph{Mt} 26, 53) como en otro tiempo Israel (cf. \emph{2 M} 10, 29-30; 11,8). Son también los ángeles quienes \textquote{evangelizan} (\emph{Lc} 2, 10) anunciando la Buena Nueva de la Encarnación (cf. \emph{Lc} 2, 8-14), y de la Resurrección (cf. \emph{Mc} 16, 5-7) de Cristo. Con ocasión de la segunda venida de Cristo, anunciada por los ángeles (cf. \emph{Hb} 1, 10-11), éstos estarán presentes al servicio del juicio del Señor (cf. \emph{Mt} 13, 41; 25, 31 ; \emph{Lc} 12, 8-9).