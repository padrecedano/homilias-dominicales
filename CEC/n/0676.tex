\n{676} Esta impostura del Anticristo aparece esbozada ya en el mundo cada vez que se pretende llevar a cabo la esperanza mesiánica en la historia, lo cual no puede alcanzarse sino más allá del tiempo histórico a través del juicio escatológico: incluso en su forma mitigada, la Iglesia ha rechazado esta falsificación del Reino futuro con el nombre de milenarismo (cf. DS 3839), sobre todo bajo la forma política de un mesianismo secularizado, \textquote{intrínsecamente perverso} (cf. Pío XI, carta enc. \emph{Divini Redemptoris}, condenando \textquote{los errores presentados bajo un falso sentido místico de esta especie de falseada redención de los más humildes}; GS 20-21).