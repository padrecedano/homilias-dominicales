\n{294} La gloria de Dios consiste en que se realice esta manifestación y esta comunicación de su bondad para las cuales el mundo ha sido creado. Hacer de nosotros \textquote{hijos adoptivos por medio de Jesucristo, según el beneplácito de su voluntad, \emph{para alabanza de la gloria} de su gracia} (\emph{Ef} 1,5-6): \textquote{Porque la gloria de Dios es que el hombre viva, y la vida del hombre es la visión de Dios: si ya la revelación de Dios por la creación procuró la vida a todos los seres que viven en la tierra, cuánto más la manifestación del Padre por el Verbo procurará la vida a los que ven a Dios} (San Ireneo de Lyon, \emph{Adversus haereses}, 4,20,7). El fin último de la creación es que Dios, \textquote{Creador de todos los seres, sea por fin \textquote{todo en todas las cosas} (\emph{1 Co} 15,28), \emph{procurando al mismo tiempo su gloria y nuestra felicidad}} (AG 2).