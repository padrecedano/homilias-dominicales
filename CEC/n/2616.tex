	\ccesec{Jesús escucha la oración}
	
	\n{2616} La oración \emph{a Jesús} ya ha sido escuchada por Él durante su ministerio, a través de signos que anticipan el poder de su muerte y de su resurrección: Jesús escucha la oración de fe expresada en palabras (del leproso {[}cf. \emph{Mc} 1, 40-41{]}, de Jairo {[}cf. \emph{Mc} 5, 36{]}, de la cananea {[}cf. \emph{Mc} 7, 29{]}, del buen ladrón {[}cf. \emph{Lc} 23, 39-43{]}), o en silencio (de los portadores del paralítico {[}cf. \emph{Mc} 2, 5{]}, de la hemorroisa {[}cf. \emph{Mc} 5, 28{]} que toca el borde de su manto, de las lágrimas y el perfume de la pecadora {[}cf. \emph{Lc} 7, 37-38{]}). La petición apremiante de los ciegos: \textquote{¡Ten piedad de nosotros, Hijo de David!} (\emph{Mt} 9, 27) o \textquote{¡Hijo de David, Jesús, ten compasión de mí!} (\emph{Mc} 10, 48) ha sido recogida en la tradición de la \emph{Oración a Jesús}: \textquote{Señor Jesucristo, Hijo de Dios, ten piedad de mí, pecador}. Sanando enfermedades o perdonando pecados, Jesús siempre responde a la plegaria del que le suplica con fe: \textquote{Ve en paz, ¡tu fe te ha salvado!}.
	
	San Agustín resume admirablemente las tres dimensiones de la oración de Jesús: \emph{Orat pro nobis ut sacerdos noster, orat in nobis ut caput nostrum, oratur a nobis ut Deus noster. Agnoscamus ergo et in illo voces nostras et voces eius in nobis} (\textquote{Ora por nosotros como sacerdote nuestro; ora en nosotros como cabeza nuestra; a Él se dirige nuestra oración como a Dios nuestro. Reconozcamos, por tanto, en Él nuestras voces; y la voz de Él, en nosotros}) (\emph{Enarratio in Psalmum} 85, 1; cf. I\emph{nstitución general de la Liturgia de las Horas,} 7).
