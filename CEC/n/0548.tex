	\n{548} Los signos que lleva a cabo Jesús testimonian que el Padre le ha enviado (cf. \emph{Jn} 5, 36; 10, 25). Invitan a creer en Jesús (cf. \emph{Jn} 10, 38). Concede lo que le piden a los que acuden a él con fe (cf. \emph{Mc} 5, 25-34; 10, 52). Por tanto, los milagros fortalecen la fe en Aquel que hace las obras de su Padre: éstas testimonian que él es Hijo de Dios (cf. \emph{Jn} 10, 31-38). Pero también pueden ser \textquote{ocasión de escándalo} (\emph{Mt} 11, 6). No pretenden satisfacer la curiosidad ni los deseos mágicos. A pesar de tan evidentes milagros, Jesús es rechazado por algunos (cf. \emph{Jn} 11, 47-48); incluso se le acusa de obrar movido por los demonios (cf. \emph{Mc} 3, 22).