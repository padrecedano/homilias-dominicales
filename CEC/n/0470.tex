	\n{470} Puesto que en la unión misteriosa de la Encarnación \textquote{la naturaleza humana ha sido asumida, no absorbida} (GS 22, 2), la Iglesia ha llegado a confesar con el correr de los siglos, la plena realidad del alma humana, con sus operaciones de inteligencia y de voluntad, y del cuerpo humano de Cristo. Pero paralelamente, ha tenido que recordar en cada ocasión que la naturaleza humana de Cristo pertenece propiamente a la persona divina del Hijo de Dios que la ha asumido. Todo lo que es y hace en ella proviene de \textquote{uno de la Trinidad}. El Hijo de Dios comunica, pues, a su humanidad su propio modo personal de existir en la Trinidad. Así, en su alma como en su cuerpo, Cristo expresa humanamente las costumbres divinas de la Trinidad (cf. \emph{Jn} 14, 9-10):
	
	\begin{quote}
		\textquote{El Hijo de Dios [\ldots{}] trabajó con manos de hombre, pensó con inteligencia de hombre, obró con voluntad de hombre, amó con corazón de hombre. Nacido de la Virgen María, se hizo verdaderamente uno de nosotros, en todo semejante a nosotros, excepto en el pecado} (GS 22, 2).
	\end{quote}
	
