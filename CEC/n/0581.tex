\n{581} Jesús fue considerado por los judíos y sus jefes espirituales como un \textquote{rabbi} (cf. \emph{Jn} 11, 28; 3, 2; \emph{Mt} 22, 23-24, 34-36). Con frecuencia argumentó en el marco de la interpretación rabínica de la Ley (cf. \emph{Mt} 12, 5; 9, 12; \emph{Mc} 2, 23-27; \emph{Lc} 6, 6-9; \emph{Jn} 7, 22-23). Pero al mismo tiempo, Jesús no podía menos que chocar con los doctores de la Ley porque no se contentaba con proponer su interpretación entre los suyos, sino que \textquote{enseñaba como quien tiene autoridad y no como los escribas} (\emph{Mt} 7, 28-29). La misma Palabra de Dios, que resonó en el Sinaí para dar a Moisés la Ley escrita, es la que en Él se hace oír de nuevo en el Monte de las Bienaventuranzas (cf. \emph{Mt} 5, 1). Esa palabra no revoca la Ley sino que la perfecciona aportando de modo divino su interpretación definitiva: \textquote{Habéis oído también que se dijo a los antepasados [\ldots{}] pero yo os digo} (\emph{Mt} 5, 33-34). Con esta misma autoridad divina, desaprueba ciertas \textquote{tradiciones humanas} (\emph{Mc} 7, 8) de los fariseos que \textquote{anulan la Palabra de Dios} (\emph{Mc} 7, 13).