\n{1225} En su Pascua, Cristo abrió a todos los hombres las fuentes del Bautismo. En efecto, había hablado ya de su pasión que iba a sufrir en Jerusalén como de un \textquote{Bautismo} con que debía ser bautizado (\emph{Mc} 10,38; cf. \emph{Lc} 12,50). La sangre y el agua que brotaron del costado traspasado de Jesús crucificado (cf. \emph{Jn} 19,34) son figuras del Bautismo y de la Eucaristía, sacramentos de la vida nueva (cf. \emph{1 Jn} 5,6-8): desde entonces, es posible \textquote{nacer del agua y del Espíritu} para entrar en el Reino de Dios (\emph{Jn} 3,5).

\begin{quote}
	\textquote{Considera dónde eres bautizado, de dónde viene el Bautismo: de la cruz de Cristo, de la muerte de Cristo. Ahí está todo el misterio: Él padeció por ti. En él eres rescatado, en él eres salvado}. (San Ambrosio, \emph{De sacramentis} 2, 2, 6).
\end{quote}

