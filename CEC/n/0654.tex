\n{654} Hay un doble aspecto en el misterio pascual: por su muerte nos libera del pecado, por su Resurrección nos abre el acceso a una nueva vida. Esta es, en primer lugar, la \emph{justificación} que nos devuelve a la gracia de Dios (cf. \emph{Rm} 4, 25) \textquote{a fin de que, al igual que Cristo fue resucitado de entre los muertos [\ldots{}] así también nosotros vivamos una nueva vida} (\emph{Rm} 6, 4). Consiste en la victoria sobre la muerte y el pecado y en la nueva participación en la gracia (cf. \emph{Ef} 2, 4-5; \emph{1 P} 1, 3). Realiza la \emph{adopción filial} porque los hombres se convierten en hermanos de Cristo, como Jesús mismo llama a sus discípulos después de su Resurrección: \textquote{Id, avisad a mis hermanos} (\emph{Mt} 28, 10; \emph{Jn} 20, 17). Hermanos no por naturaleza, sino por don de la gracia, porque esta filiación adoptiva confiere una participación real en la vida del Hijo único, la que ha revelado plenamente en su Resurrección.