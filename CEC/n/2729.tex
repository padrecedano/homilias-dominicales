	\n{2729} La dificultad habitual de la oración es la \emph{distracción}. En la oración vocal, la distracción puede referirse a las palabras y al sentido de estas. La distracción, de un modo más profundo, puede referirse a Aquél al que oramos, tanto en la oración vocal (litúrgica o personal), como en la meditación y en la oración contemplativa. Dedicarse a perseguir las distracciones es caer en sus redes; basta con volver a nuestro corazón: la distracción descubre al que ora aquello a lo que su corazón está apegado. Esta humilde toma de conciencia debe empujar al orante a ofrecerse al Señor para ser purificado. El combate se decide cuando se elige a quién se desea servir (cf. \emph{Mt} 6,21.24).