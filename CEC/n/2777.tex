\ccesec{Acercarse a Él con toda confianza}

\n{2777} En la liturgia romana, se invita a la asamblea eucarística a rezar el Padre Nuestro con una audacia filial; las liturgias orientales usan y desarrollan expresiones análogas: \textquote{Atrevernos con toda confianza}, \textquote{Haznos dignos de}. Ante la zarza ardiendo, se le dijo a Moisés: \textquote{No te acerques aquí. Quita las sandalias de tus pies} (\emph{Ex} 3, 5). Este umbral de la santidad divina, sólo lo podía franquear Jesús, el que \textquote{después de llevar a cabo la purificación de los pecados} (\emph{Hb} 1, 3), nos introduce en presencia del Padre: \textquote{Hénos aquí, a mí y a los hijos que Dios me dio} (\emph{Hb} 2, 13):

\begin{quote}
	\textquote{La conciencia que tenemos de nuestra condición de esclavos nos haría meternos bajo tierra, nuestra condición terrena se desharía en polvo, si la autoridad de nuestro mismo Padre y el Espíritu de su Hijo, no nos empujasen a proferir este grito: \textquote{Abbá, Padre} (\emph{Rm} 8, 15) \ldots{} ¿Cuándo la debilidad de un mortal se atrevería a llamar a Dios Padre suyo, sino solamente cuando lo íntimo del hombre está animado por el Poder de lo alto?} (San Pedro Crisólogo, \emph{Sermón} 71, 3).
\end{quote}

