	\n{500} A esto se objeta a veces que la Escritura menciona unos hermanos y hermanas de Jesús (cf. Mc 3, 31-55; 6, 3; 1 Co 9, 5; Ga 1, 19). La Iglesia siempre ha entendido estos pasajes como no referidos a otros hijos de la Virgen María; en efecto, Santiago y José \textquote{hermanos de Jesús} (Mt 13, 55) son los hijos de una María discípula de Cristo (cf. Mt 27, 56) que se designa de manera significativa como \textquote{la otra María} (Mt 28, 1). Se trata de parientes próximos de Jesús, según una expresión conocida del Antiguo Testamento (cf. Gn 13, 8; 14, 16;29, 15; etc.).
