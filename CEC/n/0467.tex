\n{467} Los monofisitas afirmaban que la naturaleza humana había dejado de existir como tal en Cristo al ser asumida por su persona divina de Hijo de Dios. Enfrentado a esta herejía, el cuarto Concilio Ecuménico, en Calcedonia, confesó en el año 451:

\begin{quote}
\textquote{Siguiendo, pues, a los Santos Padres, enseñamos unánimemente que hay que confesar a un solo y mismo Hijo y Señor nuestro Jesucristo: perfecto en la divinidad, y perfecto en la humanidad; verdaderamente Dios y verdaderamente hombre compuesto de alma racional y cuerpo; consubstancial con el Padre según la divinidad, y consubstancial con nosotros según la humanidad, \textquote{en todo semejante a nosotros, excepto en el pecado} (\emph{Hb} 4, 15); nacido del Padre antes de todos los siglos según la divinidad; y por nosotros y por nuestra salvación, nacido en los últimos tiempos de la Virgen María, la Madre de Dios, según la humanidad.
		
Se ha de reconocer a un solo y mismo Cristo Señor, Hijo único en dos naturalezas, sin confusión, sin cambio, sin división, sin separación. La diferencia de naturalezas de ningún modo queda suprimida por su unión, sino que quedan a salvo las propiedades de cada una de las naturalezas y confluyen en un solo sujeto y en una sola persona} (Concilio de Calcedonia; DS, 301-302).
\end{quote}