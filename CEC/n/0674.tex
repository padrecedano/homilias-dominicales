\n{674} La venida del Mesías glorioso, en un momento determinado de la historia (cf. \emph{Rm} 11, 31), se vincula al reconocimiento del Mesías por \textquote{todo Israel} (\emph{Rm} 11, 26; \emph{Mt} 23, 39) del que \textquote{una parte está endurecida} (\emph{Rm} 11, 25) en \textquote{la incredulidad} (\emph{Rm} 11, 20) respecto a Jesús. San Pedro dice a los judíos de Jerusalén después de Pentecostés: \textquote{Arrepentíos, pues, y convertíos para que vuestros pecados sean borrados, a fin de que del Señor venga el tiempo de la consolación y envíe al Cristo que os había sido destinado, a Jesús, a quien debe retener el cielo hasta el tiempo de la restauración universal, de que Dios habló por boca de sus profetas} (\emph{Hch} 3, 19-21). Y san Pablo le hace eco: \textquote{si su reprobación ha sido la reconciliación del mundo ¿qué será su readmisión sino una resurrección de entre los muertos?} (\emph{Rm} 11, 5). La entrada de \textquote{la plenitud de los judíos} (\emph{Rm} 11, 12) en la salvación mesiánica, a continuación de \textquote{la plenitud de los gentiles} (Rm 11, 25; cf. Lc 21, 24), hará al pueblo de Dios \textquote{llegar a la plenitud de Cristo} (\emph{Ef} 4, 13) en la cual \textquote{Dios será todo en nosotros} (\emph{1 Co} 15, 28).