\n{1216} \textquote{Este baño es llamado \emph{iluminación} porque quienes reciben esta enseñanza (catequética) su espíritu es iluminado} (San Justino, \emph{Apología} 1,61). Habiendo recibido en el Bautismo al Verbo, \textquote{la luz verdadera que ilumina a todo hombre} (\emph{Jn} 1,9), el bautizado, \textquote{tras haber sido iluminado} (\emph{Hb} 10,32), se convierte en \textquote{hijo de la luz} (\emph{1 Ts} 5,5), y en \textquote{luz} él mismo (\emph{Ef} 5,8):

\begin{quote}
	El Bautismo \textquote{es el más bello y magnífico de los dones de Dios [\ldots{}] lo llamamos don, gracia, unción, iluminación, vestidura de incorruptibilidad, baño de regeneración, sello y todo lo más precioso que hay. \emph{Don}, porque es conferido a los que no aportan nada; \emph{gracia}, porque es dado incluso a culpables; \emph{bautismo}, porque el pecado es sepultado en el agua; \emph{unción}, porque es sagrado y real (tales son los que son ungidos); \emph{iluminación}, porque es luz resplandeciente; \emph{vestidura}, porque cubre nuestra vergüenza; \emph{baño}, porque lava; \emph{sello}, porque nos guarda y es el signo de la soberanía de Dios} (San Gregorio Nacianceno, \emph{Oratio} 40,3-4).
\end{quote}