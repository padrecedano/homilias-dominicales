	\n{498} A veces ha desconcertado el silencio del Evangelio de San Marcos y de las cartas del Nuevo Testamento sobre la concepción virginal de María. También se ha podido plantear si no se trataría en este caso de leyendas o de construcciones teológicas sin pretensiones históricas. A lo cual hay que responder: La fe en la concepción virginal de Jesús ha encontrado viva oposición, burlas o incomprensión por parte de los no creyentes, judíos y paganos (cf. S. Justino, Dial 99, 7; Orígenes, Cels. 1, 32, 69; entre otros); no ha tenido su origen en la mitología pagana ni en una adaptación de las ideas de su tiempo. El sentido de este misterio no es accesible más que a la fe que lo ve en ese \textquote{nexo que reúne entre sí los misterios} (DS 3016), dentro del conjunto de los Misterios de Cristo, desde su Encarnación hasta su Pascua. San Ignacio de Antioquía da ya testimonio de este vínculo: \textquote{El príncipe de este mundo ignoró la virginidad de María y su parto, así como la muerte del Señor: tres misterios resonantes que se realizaron en el silencio de Dios} (Eph. 19, 1;cf. 1 Co 2, 8).