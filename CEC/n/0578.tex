\n{578} Jesús, el Mesías de Israel, por lo tanto el más grande en el Reino de los cielos, se debía sujetar a la Ley cumpliéndola en su totalidad hasta en sus menores preceptos, según sus propias palabras. Incluso es el único en poderlo hacer perfectamente (cf. \emph{Jn} 8, 46). Los judíos, según su propia confesión, jamás han podido cumplir la Ley en su totalidad, sin violar el menor de sus preceptos (cf. \emph{Jn} 7, 19; \emph{Hch} 13, 38-41; 15, 10). Por eso, en cada fiesta anual de la Expiación, los hijos de Israel piden perdón a Dios por sus transgresiones de la Ley. En efecto, la Ley constituye un todo y, como recuerda Santiago, \textquote{quien observa toda la Ley, pero falta en un solo precepto, se hace reo de todos} (\emph{St} 2, 10; cf. \emph{Ga} 3, 10; 5, 3).