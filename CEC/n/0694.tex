\n{694} \emph{El agua}. El simbolismo del agua es significativo de la acción del Espíritu Santo en el Bautismo, ya que, después de la invocación del Espíritu Santo, ésta se convierte en el signo sacramental eficaz del nuevo nacimiento: del mismo modo que la gestación de nuestro primer nacimiento se hace en el agua, así el agua bautismal significa realmente que nuestro nacimiento a la vida divina se nos da en el Espíritu Santo. Pero \textquote{bautizados [\ldots{}] en un solo Espíritu}, también \textquote{hemos bebido de un solo Espíritu} (\emph{1 Co} 12, 13): el Espíritu es, pues, también personalmente el Agua viva que brota de Cristo crucificado (cf. Jn 19, 34; 1 Jn 5, 8) como de su manantial y que en nosotros brota en vida eterna (cf. \emph{Jn} 4, 10-14; 7, 38; \emph{Ex} 17, 1-6; \emph{Is} 55, 1; \emph{Za} 14, 8; \emph{1 Co} 10, 4; \emph{Ap} 21, 6; 22, 17).