\n{1165} Cuando la Iglesia celebra el Misterio de Cristo, hay una palabra que jalona su oración: \emph{¡Hoy!}, como eco de la oración que le enseñó su Señor (\emph{Mt} 6,11) y de la llamada del Espíritu Santo (\emph{Hb} 3,7-4,11; \emph{Sal} 95,7). Este \textquote{hoy} del Dios vivo al que el hombre está llamado a entrar, es la \textquote{Hora} de la Pascua de Jesús, que atraviesa y guía toda la historia humana:

\begin{quote}
	\textquote{La vida se ha extendido sobre todos los seres y todos están llenos de una amplia luz: el Oriente de los orientes invade el universo, y el que existía \textquote{antes del lucero de la mañana} y antes de todos los astros, inmortal e inmenso, el gran Cristo brilla sobre todos los seres más que el sol. Por eso, para nosotros que creemos en él, se instaura un día de luz, largo, eterno, que no se extingue: la Pascua mística} (Pseudo-Hipólito Romano, \emph{In Sanctum Pascha} 1-2).
\end{quote}