	\n{523} \emph{San Juan Bautista} es el precursor (cf. \emph{Hch} 13, 24) inmediato del Señor, enviado para prepararle el camino (cf. \emph{Mt} 3, 3). \textquote{Profeta del Altísimo} (\emph{Lc} 1, 76), sobrepasa a todos los profetas (cf. \emph{Lc} 7, 26), de los que es el último (cf. \emph{Mt} 11, 13), e inaugura el Evangelio (cf. \emph{Hch} 1, 22; \emph{Lc} 16,16); desde el seno de su madre (cf. \emph{Lc} 1,41) saluda la venida de Cristo y encuentra su alegría en ser \textquote{el amigo del esposo} (\emph{Jn} 3, 29) a quien señala como \textquote{el Cordero de Dios que quita el pecado del mundo} (\emph{Jn} 1, 29). Precediendo a Jesús \textquote{con el espíritu y el poder de Elías} (\emph{Lc} 1, 17), da testimonio de él mediante su predicación, su bautismo de conversión y finalmente con su martirio (cf. \emph{Mc} 6, 17-29).