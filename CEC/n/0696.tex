\n{696} \emph{El fuego}. Mientras que el agua significaba el nacimiento y la fecundidad de la vida dada en el Espíritu Santo, el fuego simboliza la energía transformadora de los actos del Espíritu Santo. El profeta Elías que \textquote{surgió [\ldots{}] como el fuego y cuya palabra abrasaba como antorcha} (\emph{Si} 48, 1), con su oración, atrajo el fuego del cielo sobre el sacrificio del monte Carmelo (cf. \emph{1 R} 18, 38-39), figura del fuego del Espíritu Santo que transforma lo que toca. Juan Bautista, \textquote{que precede al Señor con el espíritu y el poder de Elías} (\emph{Lc} 1, 17), anuncia a Cristo como el que \textquote{bautizará en el Espíritu Santo y el fuego} (\emph{Lc} 3, 16), Espíritu del cual Jesús dirá: \textquote{He venido a traer fuego sobre la tierra y ¡cuánto desearía que ya estuviese encendido!} (\emph{Lc} 12, 49). En forma de lenguas \textquote{como de fuego} se posó el Espíritu Santo sobre los discípulos la mañana de Pentecostés y los llenó de él (\emph{Hch} 2, 3-4). La tradición espiritual conservará este simbolismo del fuego como uno de los más expresivos de la acción del Espíritu Santo (cf. San Juan de la Cruz, \emph{Llama de amor viva}). \textquote{No extingáis el Espíritu} (\emph{1 Ts} 5, 19).