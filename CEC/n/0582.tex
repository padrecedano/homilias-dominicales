\n{582} Yendo más lejos, Jesús da plenitud a la Ley sobre la pureza de los alimentos, tan importante en la vida cotidiana judía, manifestando su sentido \textquote{pedagógico} (cf. \emph{Ga} 3, 24) por medio de una interpretación divina: \textquote{Todo lo que de fuera entra en el hombre no puede hacerle impuro [\ldots{}] ---así declaraba puros todos los alimentos---. Lo que sale del hombre, eso es lo que hace impuro al hombre. Porque de dentro, del corazón de los hombres, salen las intenciones malas} (\emph{Mc} 7, 18-21). Jesús, al dar con autoridad divina la interpretación definitiva de la Ley, se vio enfrentado a algunos doctores de la Ley que no aceptaban su interpretación a pesar de estar garantizada por los signos divinos con que la acompañaba (cf. \emph{Jn} 5, 36; 10, 25. 37-38; 12, 37). Esto ocurre, en particular, respecto al problema del sábado: Jesús recuerda, frecuentemente con argumentos rabínicos (cf. \emph{Mt} 2,25-27; \emph{Jn} 7, 22-24), que el descanso del sábado no se quebranta por el servicio de Dios (cf. \emph{Mt} 12, 5; \emph{Nm} 28, 9) o al prójimo (cf. \emph{Lc} 13, 15-16; 14, 3-4) que realizan sus curaciones.