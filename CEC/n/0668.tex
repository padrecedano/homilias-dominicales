%r
\n{668} \textquote{Cristo murió y volvió a la vida para eso, para ser Señor de muertos y vivos} (\emph{Rm} 14, 9). La Ascensión de Cristo al Cielo significa su participación, en su humanidad, en el poder y en la autoridad de Dios mismo. Jesucristo es Señor: posee todo poder en los cielos y en la tierra. El está \textquote{por encima de todo principado, potestad, virtud, dominación} porque el Padre \textquote{bajo sus pies sometió todas las cosas} (\emph{Ef} 1, 20-22). Cristo es el Señor del cosmos (cf. \emph{Ef} 4, 10; \emph{1 Co} 15, 24. 27-28) y de la historia. En Él, la historia de la humanidad e incluso toda la Creación encuentran su recapitulación (\emph{Ef} 1, 10), su cumplimiento transcendente.