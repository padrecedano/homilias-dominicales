\ccesec{Verdad, belleza y arte sacro}

\n{2500} La práctica del bien va acompañada de un placer espiritual gratuito y de belleza moral. De igual modo, la verdad entraña el gozo y el esplendor de la belleza espiritual. La verdad es bella por sí misma. La verdad de la palabra, expresión racional del conocimiento de la realidad creada e increada, es necesaria al hombre dotado de inteligencia, pero la verdad puede también encontrar otras formas de expresión humana, complementarias, sobre todo cuando se trata de evocar lo que ella entraña de indecible, las profundidades del corazón humano, las elevaciones del alma, el Misterio de Dios. Antes de revelarse al hombre en palabras de verdad, Dios se revela a él, mediante el lenguaje universal de la Creación, obra de su Palabra, de su Sabiduría: el orden y la armonía del cosmos, que percibe tanto el niño como el hombre de ciencia, \textquote{pues por la grandeza y hermosura de las criaturas se llega, por analogía, a contemplar a su Autor} (\emph{Sb} 13, 5), \textquote{pues fue el Autor mismo de la belleza quien las creó} (\emph{Sb} 13, 3).

\textquote{La sabiduría es un hálito del poder de Dios, una emanación pura de la gloria del Omnipotente, por lo que nada manchado llega a alcanzarla. Es un reflejo de la luz eterna, un espejo sin mancha de la actividad de Dios, una imagen de su bondad} (\emph{Sb} 7, 25-26). \textquote{La sabiduría es, en efecto, más bella que el Sol, supera a todas las constelaciones; comparada con la luz, sale vencedora, porque a la luz sucede la noche, pero contra la sabiduría no prevalece la maldad} (\emph{Sb} 7, 29-30). \textquote{Yo me constituí en el amante de su belleza} (\emph{Sb} 8, 2).