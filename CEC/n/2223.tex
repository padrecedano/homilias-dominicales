	\n{2223} Los padres son los primeros responsables de la educación de sus hijos. Testimonian esta responsabilidad ante todo por la \emph{creación de un hogar}, donde la ternura, el perdón, el respeto, la fidelidad y el servicio desinteresado son norma. La familia es un lugar apropiado para la \emph{educación de las virtudes}. Esta requiere el aprendizaje de la abnegación, de un sano juicio, del dominio de sí, condiciones de toda libertad verdadera. Los padres han de enseñar a los hijos a subordinar las dimensiones \textquote{materiales e instintivas a las interiores y espirituales} (CA 36). Es una grave responsabilidad para los padres dar buenos ejemplos a sus hijos. Sabiendo reconocer ante sus hijos sus propios defectos, se hacen más aptos para guiarlos y corregirlos:

	\begin{quote}
		\textquote{El que ama a su hijo, le corrige sin cesar [\ldots{}] el que enseña a su hijo, sacará provecho de él} (\emph{Si} 30, 1-2). \textquote{Padres, no exasperéis a vuestros hijos, sino formadlos más bien mediante la instrucción y la corrección según el Señor} (\emph{Ef} 6, 4).
	\end{quote}
