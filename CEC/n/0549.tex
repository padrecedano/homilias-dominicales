	\n{549} Al liberar a algunos hombres de los males terrenos del hambre (cf. \emph{Jn} 6, 5-15), de la injusticia (cf. \emph{Lc} 19, 8), de la enfermedad y de la muerte (cf. \emph{Mt} 11,5), Jesús realizó unos signos mesiánicos; no obstante, no vino para abolir todos los males aquí abajo (cf. \emph{Lc} 12, 13. 14; \emph{Jn} 18, 36), sino a liberar a los hombres de la esclavitud más grave, la del pecado (cf. \emph{Jn} 8, 34-36), que es el obstáculo en su vocación de hijos de Dios y causa de todas sus servidumbres humanas.