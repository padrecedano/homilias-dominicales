		
	\n{1159} La imagen sagrada, el icono litúrgico, representa principalmente \emph{a} \emph{Cristo}. No puede representar a Dios invisible e incomprensible; la Encarnación del Hijo de Dios inauguró una nueva \textquote{economía} de las imágenes:
	
	\begin{quote}
		\textquote{En otro tiempo, Dios, que no tenía cuerpo ni figura no podía de ningún modo ser representado con una imagen. Pero ahora que se ha hecho ver en la carne y que ha vivido con los hombres, puedo hacer una imagen de lo que he visto de Dios. [\ldots{}] Nosotros sin embargo, revelado su rostro, contemplamos la gloria del Señor} (San Juan Damasceno, \emph{De sacris imaginibus oratio} 1,16).
	\end{quote}
	
