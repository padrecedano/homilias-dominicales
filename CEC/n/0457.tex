\n{457} El Verbo se encarnó \emph{para salvarnos reconciliándonos con Dios}: \textquote{Dios nos amó y nos envió a su Hijo como propiciación por nuestros pecados} (\emph{1 Jn} 4, 10). \textquote{El Padre envió a su Hijo para ser salvador del mundo} (\emph{1 Jn} 4, 14). \textquote{Él se manifestó para quitar los pecados} (\emph{1 Jn} 3, 5):
	
\begin{quote} 
	\textquote{Nuestra naturaleza enferma exigía ser sanada; desgarrada, ser restablecida; muerta, ser resucitada. Habíamos perdido la posesión del bien, era necesario que se nos devolviera. Encerrados en las tinieblas, hacía falta que nos llegara la luz; estando cautivos, esperábamos un salvador; prisioneros, un socorro; esclavos, un libertador. ¿No tenían importancia estos razonamientos? ¿No merecían conmover a Dios hasta el punto de hacerle bajar hasta nuestra naturaleza humana para visitarla, ya que la humanidad se encontraba en un estado tan miserable y tan desgraciado?} (San Gregorio de Nisa, \emph{Oratio catechetica}, 15: PG 45, 48B). 
\end{quote}