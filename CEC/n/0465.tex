\n{465} Las primeras herejías negaron menos la divinidad de Jesucristo que su humanidad verdadera (docetismo gnóstico). Desde la época apostólica la fe cristiana insistió en la verdadera encarnación del Hijo de Dios, \textquote{venido en la carne} (cf. \emph{1 Jn} 4, 2-3; \emph{2 Jn} 7). Pero desde el siglo III, la Iglesia tuvo que afirmar frente a Pablo de Samosata, en un Concilio reunido en Antioquía, que Jesucristo es Hijo de Dios por naturaleza y no por adopción. El primer Concilio Ecuménico de Nicea, en el año 325, confesó en su Credo que el Hijo de Dios es \textquote{engendrado, no creado, \textquote{de la misma substancia} {[}en griego \emph{homousion}{]} que el Padre} y condenó a Arrio que afirmaba que \textquote{el Hijo de Dios salió de la nada} (Concilio de Nicea I: DS 130) y que sería \textquote{de una substancia distinta de la del Padre} (\emph{Ibíd}., 126).