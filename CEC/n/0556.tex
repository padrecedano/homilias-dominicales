\n{556} En el umbral de la vida pública se sitúa el Bautismo; en el de la Pascua, la Transfiguración. Por el bautismo de Jesús \textquote{fue manifestado el misterio de la primera regeneración}: nuestro Bautismo; la Transfiguración \textquote{es el sacramento de la segunda regeneración}: nuestra propia resurrección (Santo Tomás de Aquino, \emph{S.Th}., 3, q. 45, a. 4, ad 2). Desde ahora nosotros participamos en la Resurrección del Señor por el Espíritu Santo que actúa en los sacramentos del Cuerpo de Cristo. La Transfiguración nos concede una visión anticipada de la gloriosa venida de Cristo \textquote{el cual transfigurará este miserable cuerpo nuestro en un cuerpo glorioso como el suyo} (\emph{Flp} 3, 21). Pero ella nos recuerda también que \textquote{es necesario que pasemos por muchas tribulaciones para entrar en el Reino de Dios} (\emph{Hch} 14, 22):

\begin{quote}
	\textquote{Pedro no había comprendido eso cuando deseaba vivir con Cristo en la montaña (cf. \emph{Lc} 9, 33). Te ha reservado eso, oh Pedro, para después de la muerte. Pero ahora, él mismo dice: Desciende para penar en la tierra, para servir en la tierra, para ser despreciado y crucificado en la tierra. La Vida desciende para hacerse matar; el Pan desciende para tener hambre; el Camino desciende para fatigarse andando; la Fuente desciende para sentir la sed; y tú, ¿vas a negarte a sufrir?} (San Agustín, \emph{Sermo}, 78, 6: PL 38, 492-493).
\end{quote}

