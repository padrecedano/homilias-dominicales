\n{468} Después del Concilio de Calcedonia, algunos concibieron la naturaleza humana de Cristo como una especie de sujeto personal. Contra éstos, el quinto Concilio Ecuménico, en Constantinopla, el año 553 confesó a propósito de Cristo: \textquote{No hay más que una sola hipóstasis {[}o persona{]} [\ldots{}] que es nuestro Señor Jesucristo, \emph{uno de la Trinidad}} (Concilio de Constantinopla II: DS, 424). Por tanto, todo en la humanidad de Jesucristo debe ser atribuido a su persona divina como a su propio sujeto (cf. ya Concilio de Éfeso: DS, 255), no solamente los milagros sino también los sufrimientos (cf. Concilio de Constantinopla II: DS, 424) y la misma muerte: \textquote{El que ha sido crucificado en la carne, nuestro Señor Jesucristo, es verdadero Dios, Señor de la gloria y uno de la Santísima Trinidad} (\emph{ibíd}., 432).