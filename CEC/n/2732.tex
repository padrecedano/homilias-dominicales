	\ccesec{Frente a las tentaciones en la oración}

	\n{2732} La tentación más frecuente, la más oculta, es nuestra \emph{falta de fe}. Esta se expresa menos en una incredulidad declarada que en unas preferencias de hecho. Cuando se empieza a orar, se presentan como prioritarios mil trabajos y cuidados que se consideran más urgentes; una vez más, es el momento de la verdad del corazón y de su más profundo deseo. Mientras tanto, nos volvemos al Señor como nuestro único recurso; pero ¿alguien se lo cree verdaderamente? Consideramos a Dios como asociado a la alianza con nosotros, pero nuestro corazón continúa en la arrogancia. En cualquier caso, la falta de fe revela que no se ha alcanzado todavía la disposición propia de un corazón humilde: \textquote{Sin mí, no podéis hacer nada} (\emph{Jn} 15, 5).