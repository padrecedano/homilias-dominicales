	\ccesec{La Resurrección obra de la Santísima Trinidad}
	
	\n{648} La Resurrección de Cristo es objeto de fe en cuanto es una intervención transcendente de Dios mismo en la creación y en la historia. En ella, las tres Personas divinas actúan juntas a la vez y manifiestan su propia originalidad. Se realiza por el poder del Padre que \textquote{ha resucitado} (\emph{Hch} 2, 24) a Cristo, su Hijo, y de este modo ha introducido de manera perfecta su humanidad ---con su cuerpo--- en la Trinidad. Jesús se revela definitivamente \textquote{Hijo de Dios con poder, según el Espíritu de santidad, por su resurrección de entre los muertos} (\emph{Rm} 1, 3-4). San Pablo insiste en la manifestación del poder de Dios (cf. \emph{Rm} 6, 4; 2 Co 13, 4; \emph{Flp} 3, 10; \emph{Ef} 1, 19-22; \emph{Hb} 7, 16) por la acción del Espíritu que ha vivificado la humanidad muerta de Jesús y la ha llamado al estado glorioso de Señor.
