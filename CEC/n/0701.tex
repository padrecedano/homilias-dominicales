\n{701} \emph{La paloma}. Al final del diluvio (cuyo simbolismo se refiere al Bautismo), la paloma soltada por Noé vuelve con una rama tierna de olivo en el pico, signo de que la tierra es habitable de nuevo (cf. \emph{Gn} 8, 8-12). Cuando Cristo sale del agua de su bautismo, el Espíritu Santo, en forma de paloma, baja y se posa sobre él (cf. \emph{Mt} 3, 16 paralelos). El Espíritu desciende y reposa en el corazón purificado de los bautizados. En algunos templos, la Santa Reserva eucarística se conserva en un receptáculo metálico en forma de paloma (el \emph{columbarium}), suspendido por encima del altar. El símbolo de la paloma para sugerir al Espíritu Santo es tradicional en la iconografía cristiana.