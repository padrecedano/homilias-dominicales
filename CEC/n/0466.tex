\n{466} La herejía nestoriana veía en Cristo una persona humana junto a la persona divina del Hijo de Dios. Frente a ella san Cirilo de Alejandría y el tercer Concilio Ecuménico reunido en Efeso, en el año 431, confesaron que \textquote{el Verbo, al unirse en su persona a una carne animada por un alma racional, se hizo hombre} (Concilio de Efeso: DS, 250). La humanidad de Cristo no tiene más sujeto que la persona divina del Hijo de Dios que la ha asumido y hecho suya desde su concepción. Por eso el concilio de Efeso proclamó en el año 431 que María llegó a ser con toda verdad Madre de Dios mediante la concepción humana del Hijo de Dios en su seno: \textquote{Madre de Dios, no porque el Verbo de Dios haya tomado de ella su naturaleza divina, sino porque es de ella, de quien tiene el cuerpo sagrado dotado de un alma racional [\ldots{}] unido a la persona del Verbo, de quien se dice que el Verbo nació según la carne} (DS 251).