	\n{2217} Mientras vive en el domicilio de sus padres, el hijo debe obedecer a todo lo que éstos dispongan para su bien o el de la familia. \textquote{Hijos, obedeced en todo a vuestros padres, porque esto es grato a Dios en el Señor} (\emph{Col} 3, 20; cf. \emph{Ef} 6, 1). Los niños deben obedecer también las prescripciones razonables de sus educadores y de todos aquellos a quienes sus padres los han confiado. Pero si el niño está persuadido en conciencia de que es moralmente malo obedecer esa orden, no debe seguirla.
	
	Cuando se hacen mayores, los hijos deben seguir respetando a sus padres. Deben prevenir sus deseos, solicitar dócilmente sus consejos y aceptar sus amonestaciones justificadas. La obediencia a los padres cesa con la emancipación de los hijos, pero no el respeto que les es debido, el cual permanece para siempre. Este, en efecto, tiene su raíz en el temor de Dios, uno de los dones del Espíritu Santo.