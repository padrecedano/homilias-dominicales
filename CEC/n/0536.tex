\n{536} El bautismo de Jesús es, por su parte, la aceptación y la inauguración de su misión de Siervo doliente. Se deja contar entre los pecadores (cf. \emph{Is} 53, 12); es ya \textquote{el Cordero de Dios que quita el pecado del mundo} (\emph{Jn} 1, 29); anticipa ya el \textquote{bautismo} de su muerte sangrienta (cf. \emph{Mc} 10, 38; \emph{Lc} 12, 50). Viene ya a \textquote{cumplir toda justicia} (\emph{Mt} 3, 15), es decir, se somete enteramente a la voluntad de su Padre: por amor acepta el bautismo de muerte para la remisión de nuestros pecados (cf. \emph{Mt} 26, 39). A esta aceptación responde la voz del Padre que pone toda su complacencia en su Hijo (cf. \emph{Lc} 3, 22; \emph{Is} 42, 1). El Espíritu que Jesús posee en plenitud desde su concepción viene a \textquote{posarse} sobre él (\emph{Jn} 1, 32-33; cf. \emph{Is} 11, 2). De él manará este Espíritu para toda la humanidad. En su bautismo, \textquote{se abrieron los cielos} (\emph{Mt} 3, 16) que el pecado de Adán había cerrado; y las aguas fueron santificadas por el descenso de Jesús y del Espíritu como preludio de la nueva creación.