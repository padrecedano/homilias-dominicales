	\ccesec{La Iglesia, sacramento universal de la salvación}

\n{774} La palabra griega \emph{mysterion} ha sido traducida en latín por dos términos: \emph{mysterium} y \emph{sacramentum}. En la interpretación posterior, el término \emph{sacramentum} expresa mejor el signo visible de la realidad oculta de la salvación, indicada por el término \emph{mysterium}. En este sentido, Cristo es Él mismo el Misterio de la salvación: \emph{Non est enim aliud Dei mysterium, nisi Christus} (\textquote{No hay otro misterio de Dios fuera de Cristo}; san Agustín, \emph{Epistula} 187, 11, 34). La obra salvífica de su humanidad santa y santificante es el sacramento de la salvación que se manifiesta y actúa en los sacramentos de la Iglesia (que las Iglesias de Oriente llaman también \textquote{los santos Misterios}). Los siete sacramentos son los signos y los instrumentos mediante los cuales el Espíritu Santo distribuye la gracia de Cristo, que es la Cabeza, en la Iglesia que es su Cuerpo. La Iglesia contiene, por tanto, y comunica la gracia invisible que ella significa. En este sentido analógico ella es llamada \textquote{sacramento}.
