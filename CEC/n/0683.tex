\n{683} \textquote{Nadie puede decir: \textquote{¡Jesús es Señor!} sino por influjo del Espíritu Santo} (\emph{1 Co} 12, 3). \textquote{Dios ha enviado a nuestros corazones el Espíritu de su Hijo que clama ¡\emph{Abbá}, Padre!} (\emph{Ga} 4, 6). Este conocimiento de fe no es posible sino en el Espíritu Santo. Para entrar en contacto con Cristo, es necesario primeramente haber sido atraído por el Espíritu Santo. Él es quien nos precede y despierta en nosotros la fe. Mediante el Bautismo, primer sacramento de la fe, la vida, que tiene su fuente en el Padre y se nos ofrece por el Hijo, se nos comunica íntima y personalmente por el Espíritu Santo en la Iglesia:

\begin{quote}
	El Bautismo \textquote{nos da la gracia del nuevo nacimiento en Dios Padre por medio de su Hijo en el Espíritu Santo. Porque los que son portadores del Espíritu de Dios son conducidos al Verbo, es decir al Hijo; pero el Hijo los presenta al Padre, y el Padre les concede la incorruptibilidad. Por tanto, sin el Espíritu no es posible ver al Hijo de Dios, y, sin el Hijo, nadie puede acercarse al Padre, porque el conocimiento del Padre es el Hijo, y el conocimiento del Hijo de Dios se logra por el Espíritu Santo} (San Ireneo de Lyon, \emph{Demonstratio praedicationis apostolicae}, 7: SC 62 41-42).
\end{quote}

