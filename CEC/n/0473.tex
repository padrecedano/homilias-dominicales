	\n{473} Pero, al mismo tiempo, este conocimiento verdaderamente humano del Hijo de Dios expresaba la vida divina de su persona (cf. san Gregorio Magno, carta \emph{Sicut aqua}: DS, 475). \textquote{El Hijo de Dios conocía todas las cosas; y esto por sí mismo, que se había revestido de la condición humana; no por su naturaleza, sino en cuanto estaba unida al Verbo [\ldots{}]. La naturaleza humana, en cuanto estaba unida al Verbo, conocida todas las cosas, incluso las divinas, y manifestaba en sí todo lo que conviene a Dios} (san Máximo el Confesor, \emph{Quaestiones et dubia}, 66: PG 90, 840). Esto sucede ante todo en lo que se refiere al conocimiento íntimo e inmediato que el Hijo de Dios hecho hombre tiene de su Padre (cf. \emph{Mc} 14, 36; \emph{Mt} 11, 27; \emph{Jn}1, 18; 8, 55; etc.). El Hijo, en su conocimiento humano, mostraba también la penetración divina que tenía de los pensamientos secretos del corazón de los hombres (cf. \emph{Mc} 2, 8; \emph{Jn} 2, 25; 6, 61; etc.).