\ccesec{Dios crea un mundo ordenado y bueno}

\n{299} Porque Dios crea con sabiduría, la creación está ordenada: \textquote{Tú todo lo dispusiste con medida, número y peso} (\emph{Sb} 11,20). Creada en y por el Verbo eterno, \textquote{imagen del Dios invisible} (\emph{Col} 1,15), la creación está destinada, dirigida al hombre, imagen de Dios (cf. \emph{Gn} 1,26), llamado a una relación personal con Dios. Nuestra inteligencia, participando en la luz del Entendimiento divino, puede entender lo que Dios nos dice por su creación (cf. \emph{Sal} 19,2-5), ciertamente no sin gran esfuerzo y en un espíritu de humildad y de respeto ante el Creador y su obra (cf. \emph{Jb} 42,3). Salida de la bondad divina, la creación participa en esa bondad (\textquote{Y vio Dios que era bueno [\ldots{}] muy bueno}: \emph{Gn} 1,4.10.12.18.21.31). Porque la creación es querida por Dios como un don dirigido al hombre, como una herencia que le es destinada y confiada. La Iglesia ha debido, en repetidas ocasiones, defender la bondad de la creación, comprendida la del mundo material (cf. San León Magno, c. \emph{Quam laudabiliter}, DS, 286; Concilio de Braga I: \emph{ibíd}., 455-463; Concilio de Letrán IV: \emph{ibíd.,} 800; Concilio de Florencia: \emph{ibíd.,}1333; Concilio Vaticano I: \emph{ibíd.,} 3002).
