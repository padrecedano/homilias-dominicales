	\ccesec{La entrada mesiánica de Jesús en Jerusalén}
	
	\n{559} ¿Cómo va a acoger Jerusalén a su Mesías? Jesús rehuyó siempre las tentativas populares de hacerle rey (cf. \emph{Jn} 6, 15), pero elige el momento y prepara los detalles de su entrada mesiánica en la ciudad de \textquote{David, su padre} (\emph{Lc} 1,32; cf. \emph{Mt} 21, 1-11). Es aclamado como hijo de David, el que trae la salvación (\textquote{Hosanna} quiere decir \textquote{¡sálvanos!}, \textquote{Danos la salvación!}). Pues bien, el \textquote{Rey de la Gloria} (\emph{Sal} 24, 7-10) entra en su ciudad \textquote{montado en un asno} (\emph{Za} 9, 9): no conquista a la hija de Sión, figura de su Iglesia, ni por la astucia ni por la violencia, sino por la humildad que da testimonio de la Verdad (cf. \emph{Jn} 18, 37). Por eso los súbditos de su Reino, aquel día fueron los niños (cf. \emph{Mt} 21, 15-16; \emph{Sal} 8, 3) y los \textquote{pobres de Dios}, que le aclamaban como los ángeles lo anunciaron a los pastores (cf. \emph{Lc} 19, 38; 2, 14). Su aclamación \textquote{Bendito el que viene en el nombre del Señor} (\emph{Sal} 118, 26), ha sido recogida por la Iglesia en el \emph{Sanctus} de la liturgia eucarística para introducir al memorial de la Pascua del Señor.