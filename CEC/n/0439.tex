	\n{439} Numerosos judíos e incluso ciertos paganos que compartían su esperanza reconocieron en Jesús los rasgos fundamentales del mesiánico \textquote{hijo de David} prometido por Dios a Israel (cf. \emph{Mt} 2, 2; 9, 27; 12, 23; 15, 22; 20, 30; 21, 9. 15). Jesús aceptó el título de Mesías al cual tenía derecho (cf. \emph{Jn} 4, 25-26;11, 27), pero no sin reservas porque una parte de sus contemporáneos lo comprendían según una concepción demasiado humana (cf. \emph{Mt} 22, 41-46), esencialmente política (cf. \emph{Jn} 6, 15; \emph{Lc} 24, 21).