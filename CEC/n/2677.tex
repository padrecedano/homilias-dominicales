\n{2677} \emph{\textquote{Santa María, Madre de Dios, ruega por nosotros\ldots{} }} Con Isabel, nos maravillamos y decimos: \textquote{¿De dónde a mí que la madre de mi Señor venga a mí?} (\emph{Lc} 1, 43). Porque nos da a Jesús su hijo, María es madre de Dios y madre nuestra; podemos confiarle todos nuestros cuidados y nuestras peticiones: ora por nosotros como oró por sí misma: \textquote{Hágase en mí según tu palabra} (\emph{Lc} 1, 38). Confiándonos a su oración, nos abandonamos con ella en la voluntad de Dios: \textquote{Hágase tu voluntad}.

\textquote{\emph{Ruega por nosotros, pecadores, ahora y en la hora de nuestra muerte}}. Pidiendo a María que ruegue por nosotros, nos reconocemos pecadores y nos dirigimos a la \textquote{Madre de la Misericordia}, a la Toda Santa. Nos ponemos en sus manos \textquote{ahora}, en el hoy de nuestras vidas. Y nuestra confianza se ensancha para entregarle desde ahora, \textquote{la hora de nuestra muerte}. Que esté presente en esa hora, como estuvo en la muerte en Cruz de su Hijo, y que en la hora de nuestro tránsito nos acoja como madre nuestra (cf. \emph{Jn} 19, 27) para conducirnos a su Hijo Jesús, al Paraíso.
