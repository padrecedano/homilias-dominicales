	\ccesec{Los símbolos del Espíritu Santo}
	
	\n{695} \emph{La unción}. El simbolismo de la unción con el óleo es también significativo del Espíritu Santo, hasta el punto de que se ha convertido en sinónimo suyo (cf. \emph{1 Jn} 2, 20. 27; \emph{2 Co} 1, 21). En la iniciación cristiana es el signo sacramental de la Confirmación, llamada justamente en las Iglesias de Oriente \textquote{Crismación}. Pero para captar toda la fuerza que tiene, es necesario volver a la Unción primera realizada por el Espíritu Santo: la de Jesús. Cristo {[}\textquote{Mesías} en hebreo{]} significa \textquote{Ungido} del Espíritu de Dios. En la Antigua Alianza hubo \textquote{ungidos} del Señor (cf. \emph{Ex} 30, 22-32), de forma eminente el rey David (cf. \emph{1 S} 16, 13). Pero Jesús es el Ungido de Dios de una manera única: la humanidad que el Hijo asume está totalmente \textquote{ungida por el Espíritu Santo}. Jesús es constituido \textquote{Cristo} por el Espíritu Santo (cf. \emph{Lc} 4, 18-19; \emph{Is} 61, 1).
	
	La Virgen María concibe a Cristo del Espíritu Santo, quien por medio del ángel lo anuncia como Cristo en su nacimiento (cf. \emph{Lc} 2,11) e impulsa a Simeón a ir al Templo a ver al Cristo del Señor (cf. \emph{Lc} 2, 26-27); es de quien Cristo está lleno (cf. \emph{Lc} 4, 1) y cuyo poder emana de Cristo en sus curaciones y en sus acciones salvíficas (cf. \emph{Lc} 6, 19; 8, 46). Es él en fin quien resucita a Jesús de entre los muertos (cf. \emph{Rm} 1, 4; 8, 11). Por tanto, constituido plenamente \textquote{Cristo} en su humanidad victoriosa de la muerte (cf. \emph{Hch} 2, 36), Jesús distribuye profusamente el Espíritu Santo hasta que \textquote{los santos} constituyan, en su unión con la humanidad del Hijo de Dios, \textquote{ese Hombre perfecto [\ldots{}] que realiza la plenitud de Cristo} (\emph{Ef} 4, 13): \textquote{el Cristo total} según la expresión de San Agustín (\emph{Sermo} 341, 1, 1: PL 39, 1493; Ibíd., 9, 11: PL 39, 1499).
