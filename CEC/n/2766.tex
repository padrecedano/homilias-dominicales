\n{2766} Pero Jesús no nos deja una fórmula para repetirla de modo mecánico (cf. \emph{Mt} 6, 7; \emph{1 R} 18, 26-29). Como en toda oración vocal, el Espíritu Santo, a través de la Palabra de Dios, enseña a los hijos de Dios a hablar con su Padre. Jesús no sólo nos enseña las palabras de la oración filial, sino que nos da también el Espíritu por el que estas se hacen en nosotros \textquote{espíritu [\ldots{}] y vida} (\emph{Jn} 6, 63). Más todavía: la prueba y la posibilidad de nuestra oración filial es que el Padre \textquote{ha enviado [\ldots{}] a nuestros corazones el Espíritu de su Hijo que clama: \textquote{¡Abbá, Padre!}} (\emph{Ga} 4, 6). Ya que nuestra oración interpreta nuestros deseos ante Dios, es también \textquote{el que escruta los corazones}, el Padre, quien \textquote{conoce cuál es la aspiración del Espíritu, y que su intercesión en favor de los santos es según Dios} (\emph{Rm} 8, 27). La oración al Padre se inserta en la misión misteriosa del Hijo y del Espíritu.