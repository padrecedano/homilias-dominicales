	\n{533} La vida oculta de Nazaret permite a todos entrar en comunión con Jesús a través de los caminos más ordinarios de la vida humana:

	\begin{quote}
		\textquote{Nazaret es la escuela donde empieza a entenderse la vida de Jesús, es la escuela donde se inicia el conocimiento de su Evangelio. [\ldots{}] Su primera lección es el \emph{silencio}. Cómo desearíamos que se renovara y fortaleciera en nosotros el amor al silencio, este admirable e indispensable hábito del espíritu, tan necesario para nosotros. [\ldots{}] Se nos ofrece además una lección de \emph{vida familiar}. Que Nazaret nos enseñe el significado de la familia, su comunión de amor, su sencilla y austera belleza, su carácter sagrado e inviolable. [\ldots{}] Finalmente, aquí aprendemos también la \emph{lección del trabajo}. Nazaret, la casa del \textquote{hijo del Artesano}: cómo deseamos comprender más en este lugar la austera pero redentora ley del trabajo humano y exaltarla debidamente. [\ldots{}] Queremos finalmente saludar desde aquí a todos los trabajadores del mundo y señalarles al gran modelo, al hermano divino} (Pablo VI, \emph{Homilía en el templo de la Anunciación de la Virgen María en Nazaret} (5 de enero de 1964).
	\end{quote}