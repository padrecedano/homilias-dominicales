\n{719} Juan es \textquote{más que un profeta} (\emph{Lc} 7, 26). En él, el Espíritu Santo consuma el \textquote{hablar por los profetas}. Juan termina el ciclo de los profetas inaugurado por Elías (cf. \emph{Mt} 11, 13-14). Anuncia la inminencia de la consolación de Israel, es la \textquote{voz} del Consolador que llega (\emph{Jn} 1, 23; cf. \emph{Is} 40, 1-3). Como lo hará el Espíritu de Verdad, \textquote{vino como testigo para dar testimonio de la luz} (\emph{Jn} 1, 7; cf. \emph{Jn} 15, 26; 5, 33). Con respecto a Juan, el Espíritu colma así las \textquote{indagaciones de los profetas} y la ansiedad de los ángeles (\emph{1 P} 1, 10-12): \textquote{Aquél sobre quien veas que baja el Espíritu y se queda sobre él, ése es el que bautiza con el Espíritu Santo. Y yo lo he visto y doy testimonio de que éste es el Hijo de Dios [\ldots{}] He ahí el Cordero de Dios} (\emph{Jn} 1, 33-36).
