	\n{494} Al anuncio de que ella dará a luz al \textquote{Hijo del Altísimo} sin conocer varón, por la virtud del Espíritu Santo (cf. Lc 1, 28-37), María respondió por \textquote{la obediencia de la fe} (Rm 1, 5), segura de que \textquote{nada hay imposible para Dios}: \textquote{He aquí la esclava del Señor: hágase en mí según tu palabra} (Lc 1, 37-38). Así dando su consentimiento a la palabra de Dios, María llegó a ser Madre de Jesús y, aceptando de todo corazón la voluntad divina de salvación, sin que ningún pecado se lo impidiera, se entregó a sí misma por entero a la persona y a la obra de su Hijo, para servir, en su dependencia y con él, por la gracia de Dios, al Misterio de la Redención (cf. LG 56):
	
	\begin{quote} 
		\textquote{Ella, en efecto, como dice S. Ireneo, \textquote{por su obediencia fue causa de la salvación propia y de la de todo el género humano}. Por eso, no pocos Padres antiguos, en su predicación, coincidieron con él en afirmar \textquote{el nudo de la desobediencia de Eva lo desató la obediencia de María. Lo que ató la virgen Eva por su falta de fe lo desató la Virgen María por su fe}. Comparándola con Eva, llaman a María `Madre de los vivientes' y afirman con mayor frecuencia: \textquote{la muerte vino por Eva, la vida por María}. (LG. 56)}. 
	\end{quote}
