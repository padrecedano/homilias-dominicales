	\n{2218} El cuarto mandamiento recuerda a los hijos mayores de edad sus \emph{responsabilidades para con los padres}. En la medida en que ellos pueden, deben prestarles ayuda material y moral en los años de vejez y durante sus enfermedades, y en momentos de soledad o de abatimiento. Jesús recuerda este deber de gratitud (cf. \emph{Mc} 7, 10-12).

	\begin{quote}
		\textquote{El Señor glorifica al padre en los hijos, y afirma el derecho de la madre sobre su prole. Quien honra a su padre expía sus pecados; como el que atesora es quien da gloria a su madre. Quien honra a su padre recibirá contento de sus hijos, y en el día de su oración será escuchado. Quien da gloria al padre vivirá largos días, obedece al Señor quien da sosiego a su madre} (\emph{Si} 3, 2-6).
		
		\textquote{Hijo, cuida de tu padre en su vejez, y en su vida no le causes tristeza. Aunque haya perdido la cabeza, sé indulgente, no le desprecies en la plenitud de tu vigor [\ldots{}] Como blasfemo es el que abandona a su padre, maldito del Señor quien irrita a su madre} (\emph{Si} 3, 12-13.16).
	\end{quote}
