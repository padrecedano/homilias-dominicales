\n{158} \textquote{La fe \emph{trata de comprender}} (San Anselmo de Canterbury, \emph{Proslogion}, proemium: PL 153, 225A) es inherente a la fe que el creyente desee conocer mejor a aquel en quien ha puesto su fe, y comprender mejor lo que le ha sido revelado; un conocimiento más penetrante suscitará a su vez una fe mayor, cada vez más encendida de amor. La gracia de la fe abre \textquote{los ojos del corazón} (\emph{Ef} 1,18) para una inteligencia viva de los contenidos de la Revelación, es decir, del conjunto del designio de Dios y de los misterios de la fe, de su conexión entre sí y con Cristo, centro del Misterio revelado. Ahora bien, \textquote{para que la inteligencia de la Revelación sea más profunda, el mismo Espíritu Santo perfecciona constantemente la fe por medio de sus dones} (DV 5). Así, según el adagio de san Agustín (\emph{Sermo} 43,7,9: PL 38, 258), \textquote{creo para comprender y comprendo para creer mejor}.