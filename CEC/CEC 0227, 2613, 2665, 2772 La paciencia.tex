La paciencia

CEC 227, 2613, 2665, 2772:

\textbf{227} \emph{Es confiar en Dios en todas las circunstancias}, incluso en la adversidad. Una oración de Santa Teresa de Jesús lo expresa admirablemente:

Nada te turbe, Nada te espante.

Todo se pasa, Dios no se muda.

La paciencia, Todo lo alcanza;

Quien a Dios tiene, Nada le falta:

Sólo Dios basta. (\emph{Poesía}, 30)

\textbf{2613} San Lucas nos ha trasmitido tres \emph{parábolas} principales sobre la oración:

La primera, \textquote{el amigo importuno} (cf. \emph{Lc} 11, 5-13), invita a una oración insistente: \textquote{Llamad y se os abrirá}. Al que ora así, el Padre del cielo \textquote{le dará todo lo que necesite}, y sobre todo el Espíritu Santo que contiene todos los dones.

La segunda, \textquote{la viuda importuna} (cf. \emph{Lc} 18, 1-8), está centrada en una de las cualidades de la oración: es necesario orar siempre, sin cansarse, con la \emph{paciencia} de la fe. \textquote{Pero, cuando el Hijo del hombre venga, ¿encontrará fe sobre la tierra?}.

La tercera parábola, \textquote{el fariseo y el publicano} (cf. \emph{Lc} 18, 9-14), se refiere a la \emph{humildad} del corazón que ora. \textquote{Oh Dios, ten compasión de mí que soy pecador}. La Iglesia no cesa de hacer suya esta oración: ¡\emph{Kyrie eleison!}

\textbf{La oración a Jesús}

\textbf{2665} La oración de la Iglesia, alimentada por la palabra de Dios y por la celebración de la liturgia, nos enseña a orar al Señor Jesús. Aunque esté dirigida sobre todo al Padre, en todas las tradiciones litúrgicas incluye formas de oración dirigidas a Cristo. Algunos salmos, según su actualización en la Oración de la Iglesia, y el Nuevo Testamento ponen en nuestros labios y graban en nuestros corazones las invocaciones de esta oración a Cristo: Hijo de Dios, Verbo de Dios, Señor, Salvador, Cordero de Dios, Rey, Hijo amado, Hijo de la Virgen, Buen Pastor, Vida nuestra, nuestra Luz, nuestra Esperanza, Resurrección nuestra, Amigo de los hombres\ldots{}

\textbf{2772} De esta fe inquebrantable brota la esperanza que suscita cada una de las siete peticiones. Estas expresan los gemidos del tiempo presente, este tiempo de paciencia y de espera durante el cual \textquote{aún no se ha manifestado lo que seremos} (\emph{1 Jn} 3, 2; cf. \emph{Col} 3, 4). La Eucaristía y el Padre Nuestro están orientados hacia la venida del Señor, \textquote{¡hasta que venga!} (\emph{1 Co} 11, 26).