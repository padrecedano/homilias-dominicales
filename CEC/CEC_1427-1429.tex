\cceth{La conversión de los bautizados}

\cceref{CEC 1427-1429}

\begin{ccebody}
	\n{1427} Jesús llama a la conversión. Esta llamada es una parte esencial del anuncio del Reino: \textquote{El tiempo se ha cumplido y el Reino de Dios está cerca; convertíos y creed en la Buena Nueva} (\emph{Mc} 1,15). En la predicación de la Iglesia, esta llamada se dirige primeramente a los que no conocen todavía a Cristo y su Evangelio. Así, el Bautismo es el lugar principal de la conversión primera y fundamental. Por la fe en la Buena Nueva y por el Bautismo (cf. \emph{Hch} 2,38) se renuncia al mal y se alcanza la salvación, es decir, la remisión de todos los pecados y el don de la vida nueva.
	
	\n{1428} Ahora bien, la llamada de Cristo a la conversión sigue resonando en la vida de los cristianos. Esta \emph{segunda conversión} es una tarea ininterrumpida para toda la Iglesia que \textquote{recibe en su propio seno a los pecadores} y que siendo \textquote{santa al mismo tiempo que necesitada de purificación constante, busca sin cesar la penitencia y la renovación} (LG 8). Este esfuerzo de conversión no es sólo una obra humana. Es el movimiento del \textquote{corazón contrito} (\emph{Sal} 51,19), atraído y movido por la gracia (cf. \emph{Jn} 6,44; 12,32) a responder al amor misericordioso de Dios que nos ha amado primero (cf. \emph{1 Jn} 4,10).
	
	\n{1429} De ello da testimonio la conversión de san Pedro tras la triple negación de su Maestro. La mirada de infinita misericordia de Jesús provoca las lágrimas del arrepentimiento (\emph{Lc} 22,61) y, tras la resurrección del Señor, la triple afirmación de su amor hacia él (cf. \emph{Jn} 21,15-17). La segunda conversión tiene también una dimensión \emph{comunitaria}. Esto aparece en la llamada del Señor a toda la Iglesia: \textquote{¡Arrepiéntete!} (\emph{Ap} 2,5.16).
	
	San Ambrosio dice acerca de las dos conversiones que, \textquote{en la Iglesia, existen el agua y las lágrimas: el agua del Bautismo y las lágrimas de la Penitencia} (\emph{Epistula extra collectionem} 1 {[}41{]}, 12).
	
	Mas yo, tirándome debajo de una higuera, no sé cómo, solté la rienda a las lágrimas, brotando dos ríos de mis ojos, sacrificio tuyo aceptable. Y aunque no con estas palabras, pero sí con el mismo sentido, te dije muchas cosas como éstas: \emph{¡Y tú, Señor, hasta cuándo! ¡Hasta cuándo, Señor, has de estar irritado!} No te acuerdes más de nuestras maldades pasadas. Me sentía aún cautivo de ellas y lanzaba voces lastimeras: \textquote{¿Hasta cuándo, hasta cuándo, ¡mañana!, ¡mañana!? ¿Por qué no hoy? ¿Por qué no poner fin a mis torpezas ahora mismo?}.
	
	Decía estas cosas y lloraba con muy dolorosa contrición de mi corazón. Pero he aquí que oigo de la casa vecina una voz, como de niño o niña, que decía cantando y repetía muchas veces: \textquote{\emph{Toma y lee, toma y lee}} (tolle lege, tolle lege).
	
	\ldots{} Así que, apresurado, volví al lugar donde estaba sentado Alipio y yo había dejado el códice del Apóstol al levantarme de allí. Lo tomé, lo abrí y leí en silencio el primer capítulo que se me vino a los ojos, que decía: \emph{No en comilonas y embriagueces, no en lechos y en liviandades, no en contiendas y emulaciones sino revestíos de nuestro Señor Jesucristo y no cuidéis de la carne con demasiados deseos}.
	
	No quise leer más, ni era necesario tampoco, pues al punto que di fin a la sentencia, como si se hubiera infiltrado en mi corazón una luz de seguridad, se disiparon todas las tinieblas de mis dudas.
	
	Después entramos a ver a mi madre, indicándoselo, y se llenó de gozo; le contamos el modo como había sucedido, y saltaba de alegría y cantaba victoria, por lo cual te bendecía a ti, que eres poderoso para darnos más de lo que pedimos o entendemos, porque veía que le habías concedido, respecto de mí, mucho más de lo que constantemente te pedía con sollozos y lágrimas piadosas.
	
	\textbf{San Agustín}, \emph{Confesiones ,} Libro VIII, capítulo 12.
	
\end{ccebody}