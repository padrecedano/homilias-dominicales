%01 Ciclo | 02 Tiempo | 01 Semana

\cceth{¿Por qué el Verbo se hizo carne?}

\cceref{CEC 456-460, 566}


\begin{ccebody}
	\ccesec{Por qué el Verbo se hizo carne}
	
\n{456} Con el Credo Niceno-Constantinopolitano respondemos confesando: \textquote{\emph{Por nosotros los hombres y por nuestra salvación} bajó del cielo, y por obra del Espíritu Santo se encarnó de María la Virgen y se hizo hombre} (DS 150).

	
	\n{457} El Verbo se encarnó \emph{para salvarnos reconciliándonos con Dios}: \textquote{Dios nos amó y nos envió a su Hijo como propiciación por nuestros pecados} (\emph{1 Jn} 4, 10). \textquote{El Padre envió a su Hijo para ser salvador del mundo} (\emph{1 Jn} 4, 14). \textquote{Él se manifestó para quitar los pecados} (\emph{1 Jn} 3, 5):
	
\begin{quote} 
	\textquote{Nuestra naturaleza enferma exigía ser sanada; desgarrada, ser restablecida; muerta, ser resucitada. Habíamos perdido la posesión del bien, era necesario que se nos devolviera. Encerrados en las tinieblas, hacía falta que nos llegara la luz; estando cautivos, esperábamos un salvador; prisioneros, un socorro; esclavos, un libertador. ¿No tenían importancia estos razonamientos? ¿No merecían conmover a Dios hasta el punto de hacerle bajar hasta nuestra naturaleza humana para visitarla, ya que la humanidad se encontraba en un estado tan miserable y tan desgraciado?} (San Gregorio de Nisa, \emph{Oratio catechetica}, 15: PG 45, 48B). 
\end{quote}
	
	\n{458} El Verbo se encarnó \emph{para que nosotros conociésemos así el amor de Dios}: \textquote{En esto se manifestó el amor que Dios nos tiene: en que Dios envió al mundo a su Hijo único para que vivamos por medio de él} (\emph{1 Jn} 4, 9). \textquote{Porque tanto amó Dios al mundo que dio a su Hijo único, para que todo el que crea en él no perezca, sino que tenga vida eterna} (\emph{Jn} 3, 16).
	
	\n{459} El Verbo se encarnó \emph{para ser nuestro modelo de santidad}: \textquote{Tomad sobre vosotros mi yugo, y aprended de mí \ldots{} } (\emph{Mt} 11, 29). \textquote{Yo soy el Camino, la Verdad y la Vida. Nadie va al Padre sino por mí} (\emph{Jn} 14, 6). Y el Padre, en el monte de la Transfiguración, ordena: \textquote{Escuchadle} (\emph{Mc} 9, 7; cf. \emph{Dt} 6, 4-5). Él es, en efecto, el modelo de las bienaventuranzas y la norma de la Ley nueva: \textquote{Amaos los unos a los otros como yo os he amado} (\emph{Jn} 15, 12). Este amor tiene como consecuencia la ofrenda efectiva de sí mismo (cf. \emph{Mc} 8, 34).
	
		\n{460} El Verbo se encarnó \emph{para hacernos \textquote{partícipes de la naturaleza divina}} (\emph{2 P} 1, 4): \textquote{Porque tal es la razón por la que el Verbo se hizo hombre, y el Hijo de Dios, Hijo del hombre: para que el hombre al entrar en comunión con el Verbo y al recibir así la filiación divina, se convirtiera en hijo de Dios} (San Ireneo de Lyon, \emph{Adversus haereses}, 3, 19, 1). \textquote{Porque el Hijo de Dios se hizo hombre para hacernos Dios} (San Atanasio de Alejandría, \emph{De Incarnatione}, 54, 3: PG 25, 192B). \emph{Unigenitus} [\ldots{}] \emph{Dei Filius, suae divinitatis volens nos esse participes, naturam nostram assumpsit, ut homines deos faceret factus homo} (\textquote{El Hijo Unigénito de Dios, queriendo hacernos partícipes de su divinidad, asumió nuestra naturaleza, para que, habiéndose hecho hombre, hiciera dioses a los hombres}) (Santo Tomás de Aquino, \emph{Oficio de la festividad del Corpus}, Of. de Maitines, primer Nocturno, Lectura I).
	
	\n{566} \emph{La tentación en el desierto muestra a Jesús, humilde Mesías que triunfa de Satanás mediante su total adhesión al designio de salvación querido por el Padre}.			
\end{ccebody}


\cceth{La Encarnación}

\cceref{CEC 461-463, 470-478}

\begin{ccebody}
		\n{461} Volviendo a tomar la frase de san Juan (\textquote{El Verbo se encarnó}: \emph{Jn} 1, 14), la Iglesia llama \textquote{Encarnación} al hecho de que el Hijo de Dios haya asumido una naturaleza humana para llevar a cabo por ella nuestra salvación. En un himno citado por san Pablo, la Iglesia canta el misterio de la Encarnación:
	
	\begin{quote}
		\textquote{Tened entre vosotros los mismos sentimientos que tuvo Cristo: el cual, siendo de condición divina, no retuvo ávidamente el ser igual a Dios, sino que se despojó de sí mismo tomando condición de siervo, haciéndose semejante a los hombres y apareciendo en su porte como hombre; y se humilló a sí mismo, obedeciendo hasta la muerte y muerte de cruz} (\emph{Flp} 2, 5-8; cf. \emph{Liturgia de las Horas, Cántico de las Primeras Vísperas de Domingos}).
	\end{quote}
	
		\n{462} La carta a los Hebreos habla del mismo misterio:
	
	\begin{quote}
		\textquote{Por eso, al entrar en este mundo, {[}Cristo{]} dice: No quisiste sacrificio y oblación; pero me has formado un cuerpo. Holocaustos y sacrificios por el pecado no te agradaron. Entonces dije: ¡He aquí que vengo [\ldots{}] a hacer, oh Dios, tu voluntad!} (\emph{Hb} 10, 5-7; \emph{Sal} 40, 7-9 {[}LXX{]}).
	\end{quote}

	
	\n{463} La fe en la verdadera encarnación del Hijo de Dios es el signo distintivo de la fe cristiana: \textquote{Podréis conocer en esto el Espíritu de Dios: todo espíritu que confiesa a Jesucristo, venido en carne, es de Dios} (\emph{1 Jn} 4, 2). Esa es la alegre convicción de la Iglesia desde sus comienzos cuando canta \textquote{el gran misterio de la piedad}: \textquote{Él ha sido manifestado en la carne} (\emph{1 Tm} 3, 16).

	
		\n{470} Puesto que en la unión misteriosa de la Encarnación \textquote{la naturaleza humana ha sido asumida, no absorbida} (GS 22, 2), la Iglesia ha llegado a confesar con el correr de los siglos, la plena realidad del alma humana, con sus operaciones de inteligencia y de voluntad, y del cuerpo humano de Cristo. Pero paralelamente, ha tenido que recordar en cada ocasión que la naturaleza humana de Cristo pertenece propiamente a la persona divina del Hijo de Dios que la ha asumido. Todo lo que es y hace en ella proviene de \textquote{uno de la Trinidad}. El Hijo de Dios comunica, pues, a su humanidad su propio modo personal de existir en la Trinidad. Así, en su alma como en su cuerpo, Cristo expresa humanamente las costumbres divinas de la Trinidad (cf. \emph{Jn} 14, 9-10):
	
	\begin{quote}
		\textquote{El Hijo de Dios [\ldots{}] trabajó con manos de hombre, pensó con inteligencia de hombre, obró con voluntad de hombre, amó con corazón de hombre. Nacido de la Virgen María, se hizo verdaderamente uno de nosotros, en todo semejante a nosotros, excepto en el pecado} (GS 22, 2).
	\end{quote}
	


		\n{471} Apolinar de Laodicea afirmaba que en Cristo el Verbo había sustituido al alma o al espíritu. Contra este error la Iglesia confesó que el Hijo eterno asumió también un alma racional humana (cf. Dámaso I, Carta a los Obispos Orientales: DS, 149).
	
		\n{472} Este alma humana que el Hijo de Dios asumió está dotada de un verdadero conocimiento humano. Como tal, éste no podía ser de por sí ilimitado: se desenvolvía en las condiciones históricas de su existencia en el espacio y en el tiempo. Por eso el Hijo de Dios, al hacerse hombre, quiso progresar \textquote{en sabiduría, en estatura y en gracia} (\emph{Lc} 2, 52) e igualmente adquirir aquello que en la condición humana se adquiere de manera experimental (cf. \emph{Mc} 6, 38; 8, 27; \emph{Jn} 11, 34; etc.). Eso correspondía a la realidad de su anonadamiento voluntario en \textquote{la condición de esclavo} (\emph{Flp} 2, 7).
	
		\n{473} Pero, al mismo tiempo, este conocimiento verdaderamente humano del Hijo de Dios expresaba la vida divina de su persona (cf. san Gregorio Magno, carta \emph{Sicut aqua}: DS, 475). \textquote{El Hijo de Dios conocía todas las cosas; y esto por sí mismo, que se había revestido de la condición humana; no por su naturaleza, sino en cuanto estaba unida al Verbo [\ldots{}]. La naturaleza humana, en cuanto estaba unida al Verbo, conocida todas las cosas, incluso las divinas, y manifestaba en sí todo lo que conviene a Dios} (san Máximo el Confesor, \emph{Quaestiones et dubia}, 66: PG 90, 840). Esto sucede ante todo en lo que se refiere al conocimiento íntimo e inmediato que el Hijo de Dios hecho hombre tiene de su Padre (cf. \emph{Mc} 14, 36; \emph{Mt} 11, 27; \emph{Jn}1, 18; 8, 55; etc.). El Hijo, en su conocimiento humano, mostraba también la penetración divina que tenía de los pensamientos secretos del corazón de los hombres (cf. \emph{Mc} 2, 8; \emph{Jn} 2, 25; 6, 61; etc.).
	
	\n{474} Debido a su unión con la Sabiduría divina en la persona del Verbo encarnado, el conocimiento humano de Cristo gozaba en plenitud de la ciencia de los designios eternos que había venido a revelar (cf. \emph{Mc} 8,31; 9,31; 10, 33-34; 14,18-20. 26-30). Lo que reconoce ignorar en este campo (cf. \emph{Mc} 13,32), declara en otro lugar no tener misión de revelarlo (cf. \emph{Hch} 1, 7).
	
		\n{475} De manera paralela, la Iglesia confesó en el sexto Concilio Ecuménico que Cristo posee dos voluntades y dos operaciones naturales, divinas y humanas, no opuestas, sino cooperantes, de forma que el Verbo hecho carne, en su obediencia al Padre, ha querido humanamente todo lo que ha decidido divinamente con el Padre y el Espíritu Santo para nuestra salvación (cf. Concilio de Constantinopla III, año 681: DS, 556-559). La voluntad humana de Cristo \textquote{sigue a su voluntad divina sin hacerle resistencia ni oposición, sino todo lo contrario, estando subordinada a esta voluntad omnipotente} (\emph{ibíd}., 556).
	
		\n{476} Como el Verbo se hizo carne asumiendo una verdadera humanidad, el cuerpo de Cristo era limitado (cf. Concilio de Letrán, año 649: DS, 504). Por eso se puede \textquote{pintar} la faz humana de Jesús (\emph{Ga} 3,2). En el séptimo Concilio ecuménico, la Iglesia reconoció que es legítima su representación en imágenes sagradas (Concilio de Nicea II, año 787: DS, 600-603).
	
		\n{477} Al mismo tiempo, la Iglesia siempre ha admitido que, en el cuerpo de Jesús, Dios \textquote{que era invisible en su naturaleza se hace visible} (\emph{Misal Romano}, Prefacio de Navidad). En efecto, las particularidades individuales del cuerpo de Cristo expresan la persona divina del Hijo de Dios. Él ha hecho suyos los rasgos de su propio cuerpo humano hasta el punto de que, pintados en una imagen sagrada, pueden ser venerados porque el creyente que venera su imagen, \textquote{venera a la persona representada en ella} (Concilio de Nicea II: DS, 601).

		\n{478} Jesús, durante su vida, su agonía y su pasión nos ha conocido y amado a todos y a cada uno de nosotros y se ha entregado por cada uno de nosotros: \textquote{El Hijo de Dios me amó y se entregó a sí mismo por mí} (\emph{Ga} 2, 20). Nos ha amado a todos con un corazón humano. Por esta razón, el sagrado Corazón de Jesús, traspasado por nuestros pecados y para nuestra salvación (cf. \emph{Jn} 19, 34), \textquote{es considerado como el principal indicador y símbolo [\ldots{}] de aquel amor con que el divino Redentor ama continuamente al eterno Padre y a todos los hombres} (Pío XII, Enc. \emph{Haurietis aquas}: DS, 3924; cf. ID. enc. \emph{Mystici Corporis}: ibíd., 3812).	
\end{ccebody}


\cceth{El misterio de la Navidad}

\cceref{CEC 437, 525-526}

\begin{ccebody}
		\n{437} El ángel anunció a los pastores el nacimiento de Jesús como el del Mesías prometido a Israel: \textquote{Os ha nacido hoy, en la ciudad de David, un salvador, que es el Cristo Señor} (\emph{Lc} 2, 11). Desde el principio él es \textquote{a quien el Padre ha santificado y enviado al mundo} (\emph{Jn} 10, 36), concebido como \textquote{santo} (\emph{Lc} 1, 35) en el seno virginal de María. José fue llamado por Dios para \textquote{tomar consigo a María su esposa} encinta \textquote{del que fue engendrado en ella por el Espíritu Santo} (\emph{Mt} 1, 20) para que Jesús \textquote{llamado Cristo} nazca de la esposa de José en la descendencia mesiánica de David (\emph{Mt} 1, 16; cf. \emph{Rm} 1, 3; \emph{2 Tm} 2, 8; \emph{Ap} 22, 16).
	
		\n{525} Jesús nació en la humildad de un establo, de una familia pobre (cf. \emph{Lc} 2, 6-7); unos sencillos pastores son los primeros testigos del acontecimiento. En esta pobreza se manifiesta la gloria del cielo (cf. \emph{Lc} 2, 8-20). La Iglesia no se cansa de cantar la gloria de esta noche:
	
	\begin{quote}
		\textquote{Hoy la Virgen da a luz al Transcendente.\\ Y la tierra ofrece una cueva al Inaccesible.\\ Los ángeles y los pastores le alaban.\\ Los magos caminan con la estrella:\\ Porque ha nacido por nosotros,\\ Niño pequeñito\\ el Dios eterno}
		
		(San Romano Melodo, Kontakion, 10)
	\end{quote}

	
		\n{526} \textquote{Hacerse niño} con relación a Dios es la condición para entrar en el Reino (cf. \emph{Mt} 18, 3-4); para eso es necesario abajarse (cf. \emph{Mt} 23, 12), hacerse pequeño; más todavía: es necesario \textquote{nacer de lo alto} (\emph{Jn} 3,7), \textquote{nacer de Dios} (\emph{Jn} 1, 13) para \textquote{hacerse hijos de Dios} (\emph{Jn} 1, 12). El misterio de Navidad se realiza en nosotros cuando Cristo \textquote{toma forma} en nosotros (\emph{Ga}4, 19). Navidad es el misterio de este \textquote{admirable intercambio}:
	
	\begin{quote}
		\textquote{¡Oh admirable intercambio! El Creador del género humano, tomando cuerpo y alma, nace de la Virgen y, hecho hombre sin concurso de varón, nos da parte en su divinidad} (\emph{Solemnidad de la Santísima Virgen María, Madre de Dios,} Antífona de I y II Vísperas: \emph{Liturgia de las Horas}).
	\end{quote}
			
\end{ccebody}


\cceth{Jesús es el Hijo de David}

\cceref{CEC 439, 496, 559, 2616}


\begin{ccebody}
		\n{439} Numerosos judíos e incluso ciertos paganos que compartían su esperanza reconocieron en Jesús los rasgos fundamentales del mesiánico \textquote{hijo de David} prometido por Dios a Israel (cf. \emph{Mt} 2, 2; 9, 27; 12, 23; 15, 22; 20, 30; 21, 9. 15). Jesús aceptó el título de Mesías al cual tenía derecho (cf. \emph{Jn} 4, 25-26;11, 27), pero no sin reservas porque una parte de sus contemporáneos lo comprendían según una concepción demasiado humana (cf. \emph{Mt} 22, 41-46), esencialmente política (cf. \emph{Jn} 6, 15; \emph{Lc} 24, 21).
	
		\ccesec{La virginidad de María}
	
	\n{496} Desde las primeras formulaciones de la fe (cf. DS 10-64), la Iglesia ha confesado que Jesús fue concebido en el seno de la Virgen María únicamente por el poder del Espíritu Santo, afirmando también el aspecto corporal de este suceso: Jesús fue concebido \textquote{absque semine ex Spiritu Sancto} (Cc Letrán, año 649; DS 503), esto es, sin elemento humano, por obra del Espíritu Santo. Los Padres ven en la concepción virginal el signo de que es verdaderamente el Hijo de Dios el que ha venido en una humanidad como la nuestra:
	
	\begin{quote}
		Así, S. Ignacio de Antioquía (comienzos del siglo II): \textquote{Estáis firmemente convencidos acerca de que nuestro Señor es verdaderamente de la raza de David según la carne (cf. Rm 1, 3), Hijo de Dios según la voluntad y el poder de Dios (cf. Jn 1, 13), nacido verdaderamente de una virgen, \ldots{} Fue verdaderamente clavado por nosotros en su carne bajo Poncio Pilato \ldots{} padeció verdaderamente, como también resucitó verdaderamente} (Smyrn. 1-2).
	\end{quote}
	
		\ccesec{La entrada mesiánica de Jesús en Jerusalén}
	
	\n{559} ¿Cómo va a acoger Jerusalén a su Mesías? Jesús rehuyó siempre las tentativas populares de hacerle rey (cf. \emph{Jn} 6, 15), pero elige el momento y prepara los detalles de su entrada mesiánica en la ciudad de \textquote{David, su padre} (\emph{Lc} 1,32; cf. \emph{Mt} 21, 1-11). Es aclamado como hijo de David, el que trae la salvación (\textquote{Hosanna} quiere decir \textquote{¡sálvanos!}, \textquote{Danos la salvación!}). Pues bien, el \textquote{Rey de la Gloria} (\emph{Sal} 24, 7-10) entra en su ciudad \textquote{montado en un asno} (\emph{Za} 9, 9): no conquista a la hija de Sión, figura de su Iglesia, ni por la astucia ni por la violencia, sino por la humildad que da testimonio de la Verdad (cf. \emph{Jn} 18, 37). Por eso los súbditos de su Reino, aquel día fueron los niños (cf. \emph{Mt} 21, 15-16; \emph{Sal} 8, 3) y los \textquote{pobres de Dios}, que le aclamaban como los ángeles lo anunciaron a los pastores (cf. \emph{Lc} 19, 38; 2, 14). Su aclamación \textquote{Bendito el que viene en el nombre del Señor} (\emph{Sal} 118, 26), ha sido recogida por la Iglesia en el \emph{Sanctus} de la liturgia eucarística para introducir al memorial de la Pascua del Señor.
	
		\ccesec{Jesús escucha la oración}
	
	\n{2616} La oración \emph{a Jesús} ya ha sido escuchada por Él durante su ministerio, a través de signos que anticipan el poder de su muerte y de su resurrección: Jesús escucha la oración de fe expresada en palabras (del leproso {[}cf. \emph{Mc} 1, 40-41{]}, de Jairo {[}cf. \emph{Mc} 5, 36{]}, de la cananea {[}cf. \emph{Mc} 7, 29{]}, del buen ladrón {[}cf. \emph{Lc} 23, 39-43{]}), o en silencio (de los portadores del paralítico {[}cf. \emph{Mc} 2, 5{]}, de la hemorroisa {[}cf. \emph{Mc} 5, 28{]} que toca el borde de su manto, de las lágrimas y el perfume de la pecadora {[}cf. \emph{Lc} 7, 37-38{]}). La petición apremiante de los ciegos: \textquote{¡Ten piedad de nosotros, Hijo de David!} (\emph{Mt} 9, 27) o \textquote{¡Hijo de David, Jesús, ten compasión de mí!} (\emph{Mc} 10, 48) ha sido recogida en la tradición de la \emph{Oración a Jesús}: \textquote{Señor Jesucristo, Hijo de Dios, ten piedad de mí, pecador}. Sanando enfermedades o perdonando pecados, Jesús siempre responde a la plegaria del que le suplica con fe: \textquote{Ve en paz, ¡tu fe te ha salvado!}.
	
	San Agustín resume admirablemente las tres dimensiones de la oración de Jesús: \emph{Orat pro nobis ut sacerdos noster, orat in nobis ut caput nostrum, oratur a nobis ut Deus noster. Agnoscamus ergo et in illo voces nostras et voces eius in nobis} (\textquote{Ora por nosotros como sacerdote nuestro; ora en nosotros como cabeza nuestra; a Él se dirige nuestra oración como a Dios nuestro. Reconozcamos, por tanto, en Él nuestras voces; y la voz de Él, en nosotros}) (\emph{Enarratio in Psalmum} 85, 1; cf. I\emph{nstitución general de la Liturgia de las Horas,} 7).

\end{ccebody}


\cceth{Dios ha dicho todo en su Verbo}

\cceref{CEC 65, 102}


\begin{ccebody}
		\n{65} \textquote{Muchas veces y de muchos modos habló Dios en el pasado a nuestros padres por medio de los profetas; en estos últimos tiempos nos ha hablado por su Hijo} (\emph{Hb} 1,1-2). Cristo, el Hijo de Dios hecho hombre, es la Palabra única, perfecta e insuperable del Padre. En Él lo dice todo, no habrá otra palabra más que ésta. San Juan de la Cruz, después de otros muchos, lo expresa de manera luminosa, comentando \emph{Hb} 1,1-2:
	
	\begin{quote}
		\textquote{Porque en darnos, como nos dio a su Hijo, que es una Palabra suya, que no tiene otra, todo nos lo habló junto y de una vez en esta sola Palabra [\ldots{}]; porque lo que hablaba antes en partes a los profetas ya lo ha hablado todo en Él, dándonos al Todo, que es su Hijo. Por lo cual, el que ahora quisiese preguntar a Dios, o querer alguna visión o revelación, no sólo haría una necedad, sino haría agravio a Dios, no poniendo los ojos totalmente en Cristo, sin querer otra alguna cosa o novedad} (San Juan de la Cruz, \emph{Subida del monte Carmelo} 2,22,3-5: \emph{Biblioteca Mística Carmelitana,} v. 11 (Burgos 1929), p. 184.).
	\end{quote}
	

	
		\n{102} A través de todas las palabras de la sagrada Escritura, Dios dice sólo una palabra, su Verbo único, en quien él se da a conocer en plenitud (cf. \emph{Hb} 1,1-3):
	
	\begin{quote}
		\textquote{Recordad que es una misma Palabra de Dios la que se extiende en todas las escrituras, que es un mismo Verbo que resuena en la boca de todos los escritores sagrados, el que, siendo al comienzo Dios junto a Dios, no necesita sílabas porque no está sometido al tiempo} (San Agustín, \emph{Enarratio in Psalmum,} 103,4,1).
	\end{quote}	
\end{ccebody}


\cceth{Cristo encarnado es adorado por los ángeles}

\cceref{CEC 333}


\begin{ccebody}
		\n{333} De la Encarnación a la Ascensión, la vida del Verbo encarnado está rodeada de la adoración y del servicio de los ángeles. Cuando Dios introduce \textquote{a su Primogénito en el mundo, dice: \textquote{adórenle todos los ángeles de Dios}} (\emph{Hb} 1, 6). Su cántico de alabanza en el nacimiento de Cristo no ha cesado de resonar en la alabanza de la Iglesia: \textquote{Gloria a Dios\ldots{}} (\emph{Lc} 2, 14). Protegen la infancia de Jesús (cf. \emph{Mt} 1, 20; 2, 13.19), le sirven en el desierto (cf. \emph{Mc} 1, 12; \emph{Mt} 4, 11), lo reconfortan en la agonía (cf. \emph{Lc} 22, 43), cuando Él habría podido ser salvado por ellos de la mano de sus enemigos (cf. \emph{Mt} 26, 53) como en otro tiempo Israel (cf. \emph{2 M} 10, 29-30; 11,8). Son también los ángeles quienes \textquote{evangelizan} (\emph{Lc} 2, 10) anunciando la Buena Nueva de la Encarnación (cf. \emph{Lc} 2, 8-14), y de la Resurrección (cf. \emph{Mc} 16, 5-7) de Cristo. Con ocasión de la segunda venida de Cristo, anunciada por los ángeles (cf. \emph{Hb} 1, 10-11), éstos estarán presentes al servicio del juicio del Señor (cf. \emph{Mt} 13, 41; 25, 31 ; \emph{Lc} 12, 8-9).
\end{ccebody}


\cceth{La Encarnación y las imágenes de Cristo}

\cceref{CEC 1159-1162, 2131, 2502}

\begin{ccebody}
			
	\n{1159} La imagen sagrada, el icono litúrgico, representa principalmente \emph{a} \emph{Cristo}. No puede representar a Dios invisible e incomprensible; la Encarnación del Hijo de Dios inauguró una nueva \textquote{economía} de las imágenes:
	
	\begin{quote}
		\textquote{En otro tiempo, Dios, que no tenía cuerpo ni figura no podía de ningún modo ser representado con una imagen. Pero ahora que se ha hecho ver en la carne y que ha vivido con los hombres, puedo hacer una imagen de lo que he visto de Dios. [\ldots{}] Nosotros sin embargo, revelado su rostro, contemplamos la gloria del Señor} (San Juan Damasceno, \emph{De sacris imaginibus oratio} 1,16).
	\end{quote}
	

	
		\n{1160} La iconografía cristiana transcribe a través de la imagen el mensaje evangélico que la sagrada Escritura transmite mediante la palabra. Imagen y Palabra se esclarecen mutuamente:

	\begin{quote}
		\textquote{Para expresarnos brevemente: conservamos intactas todas las tradiciones de la Iglesia, escritas o no escritas, que nos han sido transmitidas sin alteración. Una de ellas es la representación pictórica de las imágenes, que está de acuerdo con la predicación de la historia evangélica, creyendo que, verdaderamente y no en apariencia, el Dios Verbo se hizo carne, lo cual es tan útil y provechoso, porque las cosas que se esclarecen mutuamente tienen sin duda una significación recíproca} (Concilio de Nicea II, año 787, \emph{Terminus}: COD 111).
	\end{quote}

	
		\n{1161} Todos los signos de la celebración litúrgica hacen referencia a Cristo: también las imágenes sagradas de la Santísima Madre de Dios y de los santos. Significan, en efecto, a Cristo que es glorificado en ellos. Manifiestan \textquote{la nube de testigos} (\emph{Hb} 12,1) que continúan participando en la salvación del mundo y a los que estamos unidos, sobre todo en la celebración sacramental. A través de sus iconos, es el hombre \textquote{a imagen de Dios}, finalmente transfigurado \textquote{a su semejanza} (cf. \emph{Rm} 8,29; \emph{1 Jn} 3,2), quien se revela a nuestra fe, e incluso los ángeles, recapitulados también en Cristo:
	
	\begin{quote}
		\textquote{Siguiendo [\ldots{}] la enseñanza divinamente inspirada de nuestros santos Padres y la Tradición de la Iglesia católica (pues reconocemos ser del Espíritu Santo que habita en ella), definimos con toda exactitud y cuidado que la imagen de la preciosa y vivificante cruz, así como también las venerables y santas imágenes, tanto las pintadas como las de mosaico u otra materia conveniente, se expongan en las santas iglesias de Dios, en los vasos sagrados y ornamentos, en las paredes y en cuadros, en las casas y en los caminos: tanto las imágenes de nuestro Señor Dios y Salvador Jesucristo, como las de nuestra Señora inmaculada la santa Madre de Dios, de los santos ángeles y de todos los santos y justos} (Concilio de Nicea II: DS 600).
	\end{quote}

	
		\n{1162} \textquote{La belleza y el color de las imágenes estimulan mi oración. Es una fiesta para mis ojos, del mismo modo que el espectáculo del campo estimula mi corazón para dar gloria a Dios} (San Juan Damasceno, \emph{De sacris imaginibus oratio} 127). La contemplación de las sagradas imágenes, unida a la meditación de la Palabra de Dios y al canto de los himnos litúrgicos, forma parte de la armonía de los signos de la celebración para que el misterio celebrado se grabe en la memoria del corazón y se exprese luego en la vida nueva de los fieles.
	
		\n{2131} Fundándose en el misterio del Verbo encarnado, el séptimo Concilio Ecuménico (celebrado en Nicea el año 787), justificó contra los iconoclastas el culto de las sagradas imágenes: las de Cristo, pero también las de la Madre de Dios, de los ángeles y de todos los santos. El Hijo de Dios, al encarnarse, inauguró una nueva \textquote{economía} de las imágenes.
	
		\n{2502} El \emph{arte sacro} es verdadero y bello cuando corresponde por su forma a su vocación propia: evocar y glorificar, en la fe y la adoración, el Misterio trascendente de Dios, Belleza supereminente e invisible de Verdad y de Amor, manifestado en Cristo, \textquote{Resplandor de su gloria e Impronta de su esencia} (\emph{Hb} 1, 3), en quien \textquote{reside toda la Plenitud de la Divinidad corporalmente} (\emph{Col} 2, 9), belleza espiritual reflejada en la Santísima Virgen Madre de Dios, en los Ángeles y los Santos. El arte sacro verdadero lleva al hombre a la adoración, a la oración y al amor de Dios Creador y Salvador, Santo y Santificador.	
\end{ccebody}

\begin{patercite}
	El Señor vino a ella para hacerse siervo.
	
	El Verbo vino a ella para callar en su seno.
	
	El rayo vino a ella para no hacer ruido.
	
	El pastor vino a ella, y nació el Cordero, que llora dulcemente. \\
	
	
	El seno de María ha trastocado los papeles:
	
	Quien creó todo se ha apoderado de él, pero en la pobreza.
	
	El Altísimo vino a ella,
	
	pero entró humildemente. \\
	
	
	El esplendor vino a ella, pero vestido con ropas humildes.
	
	Quien todo lo da experimentó el hambre.
	
	Quien da de beber a todos sufrió la sed.
	
	Desnudo salió de ella, quien todo lo reviste»
	
	(\textbf{San Efrén,} Himno \emph{De Nativitate} 11, 6-8).				
\end{patercite}	