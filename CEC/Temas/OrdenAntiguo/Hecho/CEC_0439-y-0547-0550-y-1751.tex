Manifestación de Jesús como el Mesías

CEC 439, 547-550, 1751:

\textbf{439} Numerosos judíos e incluso ciertos paganos que compartían su esperanza reconocieron en Jesús los rasgos fundamentales del mesiánico \textquote{hijo de David} prometido por Dios a Israel (cf. \emph{Mt} 2, 2; 9, 27; 12, 23; 15, 22; 20, 30; 21, 9. 15). Jesús aceptó el título de Mesías al cual tenía derecho (cf. \emph{Jn} 4, 25-26;11, 27), pero no sin reservas porque una parte de sus contemporáneos lo comprendían según una concepción demasiado humana (cf. \emph{Mt} 22, 41-46), esencialmente política (cf. \emph{Jn} 6, 15; \emph{Lc} 24, 21).

\textbf{Los signos del Reino de Dios}

\textbf{547} Jesús acompaña sus palabras con numerosos \textquote{milagros, prodigios y signos} (\emph{Hch} 2, 22) que manifiestan que el Reino está presente en Él. Ellos atestiguan que Jesús es el Mesías anunciado (cf, \emph{Lc} 7, 18-23).

\textbf{548} Los signos que lleva a cabo Jesús testimonian que el Padre le ha enviado (cf. \emph{Jn} 5, 36; 10, 25). Invitan a creer en Jesús (cf. \emph{Jn} 10, 38). Concede lo que le piden a los que acuden a él con fe (cf. \emph{Mc} 5, 25-34; 10, 52). Por tanto, los milagros fortalecen la fe en Aquel que hace las obras de su Padre: éstas testimonian que él es Hijo de Dios (cf. \emph{Jn} 10, 31-38). Pero también pueden ser \textquote{ocasión de escándalo} (\emph{Mt} 11, 6). No pretenden satisfacer la curiosidad ni los deseos mágicos. A pesar de tan evidentes milagros, Jesús es rechazado por algunos (cf. \emph{Jn} 11, 47-48); incluso se le acusa de obrar movido por los demonios (cf. \emph{Mc} 3, 22).

\textbf{549} Al liberar a algunos hombres de los males terrenos del hambre (cf. \emph{Jn} 6, 5-15), de la injusticia (cf. \emph{Lc} 19, 8), de la enfermedad y de la muerte (cf. \emph{Mt} 11,5), Jesús realizó unos signos mesiánicos; no obstante, no vino para abolir todos los males aquí abajo (cf. \emph{Lc} 12, 13. 14; \emph{Jn} 18, 36), sino a liberar a los hombres de la esclavitud más grave, la del pecado (cf. \emph{Jn} 8, 34-36), que es el obstáculo en su vocación de hijos de Dios y causa de todas sus servidumbres humanas.

\textbf{550} La venida del Reino de Dios es la derrota del reino de Satanás (cf. \emph{Mt} 12, 26): \textquote{Pero si por el Espíritu de Dios expulso yo los demonios, es que ha llegado a vosotros el Reino de Dios} (\emph{Mt} 12, 28). Los \emph{exorcismos} de Jesús liberan a los hombres del dominio de los demonios (cf. \emph{Lc} 8, 26-39). Anticipan la gran victoria de Jesús sobre \textquote{el príncipe de este mundo} (\emph{Jn} 12, 31). Por la Cruz de Cristo será definitivamente establecido el Reino de Dios: \emph{Regnavit a ligno Deus} (\textquote{Dios reinó desde el madero de la Cruz}, {[}Venancio Fortunato, \emph{Hymnus \textquote{Vexilla Regis}}: MGH 1/4/1, 34: PL 88, 96{]}).

\textbf{1751} El \emph{objeto} elegido es un bien hacia el cual tiende deliberadamente la voluntad. Es la materia de un acto humano. El objeto elegido especifica moralmente el acto del querer, según que la razón lo reconozca y lo juzgue conforme o no conforme al bien verdadero. Las reglas objetivas de la moralidad enuncian el orden racional del bien y del mal, atestiguado por la conciencia.
