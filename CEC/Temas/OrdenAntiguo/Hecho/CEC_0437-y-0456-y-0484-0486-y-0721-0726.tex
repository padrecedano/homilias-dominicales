María, madre de Dios por obra del Espíritu Santo

CEC 437, 456, 484-486, 721-726:

\n{437} El ángel anunció a los pastores el nacimiento de Jesús como el del Mesías prometido a Israel: \textquote{Os ha nacido hoy, en la ciudad de David, un salvador, que es el Cristo Señor} (\emph{Lc} 2, 11). Desde el principio él es \textquote{a quien el Padre ha santificado y enviado al mundo} (\emph{Jn} 10, 36), concebido como \textquote{santo} (\emph{Lc} 1, 35) en el seno virginal de María. José fue llamado por Dios para \textquote{tomar consigo a María su esposa} encinta \textquote{del que fue engendrado en ella por el Espíritu Santo} (\emph{Mt} 1, 20) para que Jesús \textquote{llamado Cristo} nazca de la esposa de José en la descendencia mesiánica de David (\emph{Mt} 1, 16; cf. \emph{Rm} 1, 3; \emph{2 Tm} 2, 8; \emph{Ap} 22, 16).

\textbf{\\ }

\ccesec{Por qué el Verbo se hizo carne}

\n{456} Con el Credo Niceno-Constantinopolitano respondemos confesando: \textquote{\emph{Por nosotros los hombres y por nuestra salvación} bajó del cielo, y por obra del Espíritu Santo se encarnó de María la Virgen y se hizo hombre} (DS 150).

\n{457} El Verbo se encarnó \emph{para salvarnos reconciliándonos con Dios}: \textquote{Dios nos amó y nos envió a su Hijo como propiciación por nuestros pecados} (\emph{1 Jn} 4, 10). \textquote{El Padre envió a su Hijo para ser salvador del mundo} (\emph{1 Jn} 4, 14). \textquote{Él se manifestó para quitar los pecados} (\emph{1 Jn} 3, 5):

\begin{quote} \textquote{Nuestra naturaleza enferma exigía ser sanada; desgarrada, ser restablecida; muerta, ser resucitada. Habíamos perdido la posesión del bien, era necesario que se nos devolviera. Encerrados en las tinieblas, hacía falta que nos llegara la luz; estando cautivos, esperábamos un salvador; prisioneros, un socorro; esclavos, un libertador. ¿No tenían importancia estos razonamientos? ¿No merecían conmover a Dios hasta el punto de hacerle bajar hasta nuestra naturaleza humana para visitarla, ya que la humanidad se encontraba en un estado tan miserable y tan desgraciado?} (San Gregorio de Nisa, \emph{Oratio catechetica}, 15: PG 45, 48B). \end{quote}

\ccesec{Concebido por obra y gracia del Espíritu Santo \ldots{}}

\n{484} La Anunciación a María inaugura \textquote{la plenitud de los tiempos} (\emph{Ga} 4, 4), es decir, el cumplimiento de las promesas y de los preparativos. María es invitada a concebir a aquel en quien habitará \textquote{corporalmente la plenitud de la divinidad} (\emph{Col} 2, 9). La respuesta divina a su \textquote{¿cómo será esto, puesto que no conozco varón?} (\emph{Lc} 1, 34) se dio mediante el poder del Espíritu: \textquote{El Espíritu Santo vendrá sobre ti} (\emph{Lc} 1, 35).

\n{485} La misión del Espíritu Santo está siempre unida y ordenada a la del Hijo (cf. \emph{Jn} 16, 14-15). El Espíritu Santo fue enviado para santificar el seno de la Virgen María y fecundarla por obra divina, él que es \textquote{el Señor que da la vida}, haciendo que ella conciba al Hijo eterno del Padre en una humanidad tomada de la suya.

\n{486} El Hijo único del Padre, al ser concebido como hombre en el seno de la Virgen María es \textquote{Cristo}, es decir, el ungido por el Espíritu Santo (cf. \emph{Mt} 1, 20; \emph{Lc} 1, 35), desde el principio de su existencia humana, aunque su manifestación no tuviera lugar sino progresivamente: a los pastores (cf. \emph{Lc} 2,8-20), a los magos (cf. \emph{Mt} 2, 1-12), a Juan Bautista (cf. \emph{Jn} 1, 31-34), a los discípulos (cf. \emph{Jn} 2, 11). Por tanto, toda la vida de Jesucristo manifestará \textquote{cómo Dios le ungió con el Espíritu Santo y con poder} (\emph{Hch} 10, 38).

\ccesec{\textquote{Alégrate, llena de gracia}}

\n{721} María, la Santísima Madre de Dios, la siempre Virgen, es la obra maestra de la Misión del Hijo y del Espíritu Santo en la Plenitud de los tiempos. Por primera vez en el designio de Salvación y porque su Espíritu la ha preparado, el Padre encuentra la Morada en donde su Hijo y su Espíritu pueden habitar entre los hombres. Por ello, los más bellos textos sobre la Sabiduría, la Tradición de la Iglesia los ha entendido frecuentemente con relación a María (cf. \emph{Pr} 8, 1-9, 6; \emph{Si} 24): María es cantada y representada en la Liturgia como el \textquote{Trono de la Sabiduría}.

En ella comienzan a manifestarse las \textquote{maravillas de Dios}, que el Espíritu va a realizar en Cristo y en la Iglesia:

\n{722} El Espíritu Santo \emph{preparó} a María con su gracia. Convenía que fuese \textquote{llena de gracia} la Madre de Aquel en quien \textquote{reside toda la plenitud de la divinidad corporalmente} (\emph{Col} 2, 9). Ella fue concebida sin pecado, por pura gracia, como la más humilde de todas las criaturas, la más capaz de acoger el don inefable del Omnipotente. Con justa razón, el ángel Gabriel la saluda como la \textquote{Hija de Sión}: \textquote{Alégrate} (cf. \emph{So} 3, 14; \emph{Za} 2, 14). Cuando ella lleva en sí al Hijo eterno, hace subir hasta el cielo con su cántico al Padre, en el Espíritu Santo, la acción de gracias de todo el pueblo de Dios y, por tanto, de la Iglesia (cf. \emph{Lc} 1, 46-55).

\n{723} En María el Espíritu Santo \emph{realiza} el designio benevolente del Padre. La Virgen concibe y da a luz al Hijo de Dios por obra del Espíritu Santo. Su virginidad se convierte en fecundidad única por medio del poder del Espíritu y de la fe (cf. \emph{Lc} 1, 26-38; \emph{Rm} 4, 18-21; \emph{Ga} 4, 26-28).

\n{724} En María, el Espíritu Santo \emph{manifiesta} al Hijo del Padre hecho Hijo de la Virgen. Ella es la zarza ardiente de la teofanía definitiva: llena del Espíritu Santo, presenta al Verbo en la humildad de su carne dándolo a conocer a los pobres (cf. \emph{Lc} 2, 15-19) y a las primicias de las naciones (cf. \emph{Mt} 2, 11).

\n{725} En fin, por medio de María, el Espíritu Santo comienza a \emph{poner en comunión} con Cristo a los hombres \textquote{objeto del amor benevolente de Dios} (cf. \emph{Lc} 2, 14), y los humildes son siempre los primeros en recibirle: los pastores, los magos, Simeón y Ana, los esposos de Caná y los primeros discípulos.

\n{726} Al término de esta misión del Espíritu, María se convierte en la \textquote{Mujer}, nueva Eva \textquote{madre de los vivientes}, Madre del \textquote{Cristo total} (cf. \emph{Jn} 19, 25-27). Así es como ella está presente con los Doce, que \textquote{perseveraban en la oración, con un mismo espíritu} (\emph{Hch} 1, 14), en el amanecer de los \textquote{últimos tiempos} que el Espíritu va a inaugurar en la mañana de Pentecostés con la manifestación de la Iglesia.