Maternidad virginal de María

CEC 496-507, 495:

\ccsec{La virginidad de María}

\n{496} Desde las primeras formulaciones de la fe (cf. DS 10-64), la Iglesia ha confesado que Jesús fue concebido en el seno de la Virgen María únicamente por el poder del Espíritu Santo, afirmando también el aspecto corporal de este suceso: Jesús fue concebido \textquote{absque semine ex Spiritu Sancto} (Cc Letrán, año 649; DS 503), esto es, sin elemento humano, por obra del Espíritu Santo. Los Padres ven en la concepción virginal el signo de que es verdaderamente el Hijo de Dios el que ha venido en una humanidad como la nuestra:

\begin{quote}
	Así, S. Ignacio de Antioquía (comienzos del siglo II): \textquote{Estáis firmemente convencidos acerca de que nuestro Señor es verdaderamente de la raza de David según la carne (cf. Rm 1, 3), Hijo de Dios según la voluntad y el poder de Dios (cf. Jn 1, 13), nacido verdaderamente de una virgen, \ldots{} Fue verdaderamente clavado por nosotros en su carne bajo Poncio Pilato \ldots{} padeció verdaderamente, como también resucitó verdaderamente} (Smyrn. 1-2).
\end{quote}

\n{497} Los relatos evangélicos (cf. Mt 1, 18-25; Lc 1, 26-38) presentan la concepción virginal como una obra divina que sobrepasa toda comprensión y toda posibilidad humanas (cf. Lc 1, 34): \textquote{Lo concebido en ella viene del Espíritu Santo}, dice el ángel a José a propósito de María, su desposada (Mt 1, 20). La Iglesia ve en ello el cumplimiento de la promesa divina hecha por el profeta Isaías: \textquote{He aquí que la virgen concebirá y dará a luz un Hijo} (Is 7, 14 según la traducción griega de Mt 1, 23).

\n{498} A veces ha desconcertado el silencio del Evangelio de S. Marcos y de las cartas del Nuevo Testamento sobre la concepción virginal de María. También se ha podido plantear si no se trataría en este caso de leyendas o de construcciones teológicas sin pretensiones históricas. A lo cual hay que responder: La fe en la concepción virginal de Jesús ha encontrado viva oposición, burlas o incomprensión por parte de los no creyentes, judíos y paganos (cf. S. Justino, Dial 99, 7; Orígenes, Cels. 1, 32, 69; entre otros); no ha tenido su origen en la mitología pagana ni en una adaptación de las ideas de su tiempo. El sentido de este misterio no es accesible más que a la fe que lo ve en ese \textquote{nexo que reúne entre sí los misterios} (DS 3016), dentro del conjunto de los Misterios de Cristo, desde su Encarnación hasta su Pascua. S. Ignacio de Antioquía da ya testimonio de este vínculo: \textquote{El príncipe de este mundo ignoró la virginidad de María y su parto, así como la muerte del Señor: tres misterios resonantes que se realizaron en el silencio de Dios} (Eph. 19, 1;cf. 1 Co 2, 8).

\ccesec{María, la \textquote{siempre Virgen}}

\n{499} La profundización de la fe en la maternidad virginal ha llevado a la Iglesia a confesar la virginidad real y perpetua de María (cf. DS 427) incluso en el parto del Hijo de Dios hecho hombre (cf. DS 291; 294; 442; 503; 571; 1880). En efecto, el nacimiento de Cristo \textquote{lejos de disminuir consagró la integridad virginal} de su madre (LG 57). La liturgia de la Iglesia celebra a María como la \textquote{Aeiparthenos}, la \textquote{siempre-virgen} (cf. LG 52).

\n{500} A esto se objeta a veces que la Escritura menciona unos hermanos y hermanas de Jesús (cf. Mc 3, 31-55; 6, 3; 1 Co 9, 5; Ga 1, 19). La Iglesia siempre ha entendido estos pasajes como no referidos a otros hijos de la Virgen María; en efecto, Santiago y José \textquote{hermanos de Jesús} (Mt 13, 55) son los hijos de una María discípula de Cristo (cf. Mt 27, 56) que se designa de manera significativa como \textquote{la otra María} (Mt 28, 1). Se trata de parientes próximos de Jesús, según una expresión conocida del Antiguo Testamento (cf. Gn 13, 8; 14, 16;29, 15; etc.).

\n{501} Jesús es el Hijo único de María. Pero la maternidad espiritual de María se extiende (cf. Jn 19, 26-27; Ap 12, 17) a todos los hombres a los cuales, El vino a salvar: \textquote{Dio a luz al Hijo, al que Dios constituyó el mayor de muchos hermanos (Rom 8,29), es decir, de los creyentes, a cuyo nacimiento y educación colabora con amor de madre} (LG 63).

\ccesec{La maternidad virginal de María en el designio de Dios}

\n{502} La mirada de la fe, unida al conjunto de la Revelación, puede descubrir las razones misteriosas por las que Dios, en su designio salvífico, quiso que su Hijo naciera de una virgen. Estas razones se refieren tanto a la persona y a la misión redentora de Cristo como a la aceptación por María de esta misión para con los hombres.

\n{503} La virginidad de María manifiesta la iniciativa absoluta de Dios en la Encarnación. Jesús no tiene como Padre más que a Dios (cf. Lc 2, 48-49). \textquote{La naturaleza humana que ha tomado no le ha alejado jamás de su Padre \ldots{}; consubstancial con su Padre en la divinidad, consubstancial con su Madre en nuestras humanidad, pero propiamente Hijo de Dios en sus dos naturalezas} (Cc. Friul en el año 796: DS 619).

\n{504} Jesús fue concebido por obra del Espíritu Santo en el seno de la Virgen María porque El es el \emph{Nuevo Adán} (cf. 1 Co 15, 45) que inaugura la nueva creación: \textquote{El primer hombre, salido de la tierra, es terreno; el segundo viene del cielo} (1 Co 15, 47). La humanidad de Cristo, desde su concepción, está llena del Espíritu Santo porque Dios \textquote{le da el Espíritu sin medida} (Jn 3, 34). De \textquote{su plenitud}, cabeza de la humanidad redimida (cf. Col 1, 18), \textquote{hemos recibido todos gracia por gracia} (Jn 1, 16).

\n{505} Jesús, el nuevo Adán, inaugura por su concepción virginal el \emph{nuevo nacimiento} de los hijos de adopción en el Espíritu Santo por la fe \textquote{¿Cómo será eso?} (Lc 1, 34;cf. Jn 3, 9). La participación en la vida divina no nace \textquote{de la sangre, ni de deseo de carne, ni de deseo de hombre, sino de Dios} (Jn 1, 13). La acogida de esta vida es virginal porque toda ella es dada al hombre por el Espíritu. El sentido esponsal de la vocación humana con relación a Dios (cf. 2 Co 11, 2) se lleva a cabo perfectamente en la maternidad virginal de María.

\n{506} María es virgen porque su virginidad es \emph{el signo de su fe} \textquote{no adulterada por duda alguna} (LG 63) y de su entrega total a la voluntad de Dios (cf. 1 Co 7, 34-35). Su fe es la que le hace llegar a ser la madre del Salvador: \textquote{Beatior est Maria percipiendo fidem Christi quam concipiendo carnem Christi} (\textquote{Más bienaventurada es María al recibir a Cristo por la fe que al concebir en su seno la carne de Cristo} (S. Agustín, virg. 3).

\n{507} María es a la vez virgen y madre porque ella es la figura y la más perfecta realización de la Iglesia (cf. LG 63): \textquote{La Iglesia se convierte en Madre por la palabra de Dios acogida con fe, ya que, por la predicación y el bautismo, engendra para una vida nueva e inmortal a los hijos concebidos por el Espíritu Santo y nacidos de Dios. También ella es virgen que guarda íntegra y pura la fidelidad prometida al Esposo} (LG 64).

\ccesec{La maternidad divina de María}

\n{495} Llamada en los Evangelios \textquote{la Madre de Jesús} (Jn 2, 1; 19, 25; cf. Mt 13, 55, etc.), María es aclamada bajo el impulso del Espíritu como \textquote{la madre de mi Señor} desde antes del nacimiento de su hijo (cf. Lc 1, 43). En efecto, aquél que ella concibió como hombre, por obra del Espíritu Santo, y que se ha hecho verdaderamente su Hijo según la carne, no es otro que el Hijo eterno del Padre, la segunda persona de la Santísima Trinidad. La Iglesia confiesa que María es verdaderamente \emph{Madre de Dios} {[}\textquote{Theotokos}{]} (cf. DS 251).
