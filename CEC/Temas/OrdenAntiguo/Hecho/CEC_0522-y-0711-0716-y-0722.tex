\cceth{Los profetas y la espera del Mesías}

\cceref{CEC 522, 711-716, 722}

\begin{ccebody}
	
	\ccesec{Los misterios de la infancia y de la vida oculta de Jesús: Los preparativos}
	
	\n{522} La venida del Hijo de Dios a la tierra es un acontecimiento tan inmenso que Dios quiso prepararlo durante siglos. Ritos y sacrificios, figuras y símbolos de la \textquote{Primera Alianza} (\emph{Hb} 9,15), todo lo hace converger hacia Cristo; anuncia esta venida por boca de los profetas que se suceden en Israel. Además, despierta en el corazón de los paganos una espera, aún confusa, de esta venida.
	
	\ccesec{La espera del Mesías y de su Espíritu}
	
	\n{711} \textquote{He aquí que yo lo renuevo} (\emph{Is} 43, 19): dos líneas proféticas se van a perfilar, una se refiere a la espera del Mesías, la otra al anuncio de un Espíritu nuevo, y las dos convergen en el pequeño Resto, el pueblo de los Pobres (cf. \emph{So} 2, 3), que aguardan en la esperanza la \textquote{consolación de Israel} y \textquote{la redención de Jerusalén} (cf. \emph{Lc} 2, 25. 38).
	
	Ya se ha dicho cómo Jesús cumple las profecías que a Él se refieren. A continuación se describen aquéllas en que aparece sobre todo la relación del Mesías y de su Espíritu.
	
	\n{712} Los rasgos del rostro del \emph{Mesías} esperado comienzan a aparecer en el Libro del Emmanuel (cf. \emph{Is} 6, 12) (cuando \textquote{Isaías vio [\ldots{}] la gloria} de Cristo \emph{Jn} 12, 41), especialmente en \emph{Is} 11, 1-2:
	
	\begin{cceprose}
			«Saldrá un vástago del tronco de Jesé, 
			y un retoño de sus raíces brotará. 			
			Reposará sobre él el Espíritu del Señor: 			
			espíritu de sabiduría e inteligencia, 			
			espíritu de consejo y de fortaleza, 
			espíritu de ciencia y temor del Señor».
	\end{cceprose}
	

	
	\n{713} Los rasgos del Mesías se revelan sobre todo en los Cantos del Siervo (cf. \emph{Is} 42, 1-9; cf. \emph{Mt} 12, 18-21; \emph{Jn} 1, 32-34; y también \emph{Is} 49, 1-6; cf. \emph{Mt} 3, 17; \emph{Lc} 2, 32, y por último \emph{Is} 50, 4-10 y 52, 13-53, 12). Estos cantos anuncian el sentido de la Pasión de Jesús, e indican así cómo enviará el Espíritu Santo para vivificar a la multitud: no desde fuera, sino desposándose con nuestra \textquote{condición de esclavos} (\emph{Flp} 2, 7). Tomando sobre sí nuestra muerte, puede comunicarnos su propio Espíritu de vida.
	
	\n{714} Por eso Cristo inaugura el anuncio de la Buena Nueva haciendo suyo este pasaje de Isaías (\emph{Lc} 4, 18-19; cf. \emph{Is} 61, 1-2):
	
	«El Espíritu del Señor está sobre mí, porque me ha ungido. Me ha enviado a anunciar a los pobres la Buena Nueva, a proclamar la liberación a los cautivos y la vista a los ciegos, para dar la libertad a los oprimidos y proclamar un año de gracia del Señor».
	
	\n{715} Los textos proféticos que se refieren directamente al envío del Espíritu Santo son oráculos en los que Dios habla al corazón de su Pueblo en el lenguaje de la Promesa, con los acentos del \textquote{amor y de la fidelidad} (cf. \emph{Ez} 11, 19; 36, 25-28; 37, 1-14; \emph{Jr} 31, 31-34; y \emph{Jl} 3, 1-5), cuyo cumplimiento proclamará San Pedro la mañana de Pentecostés (cf. \emph{Hch} 2, 17-21). Según estas promesas, en los \textquote{últimos tiempos}, el Espíritu del Señor renovará el corazón de los hombres grabando en ellos una Ley nueva; reunirá y reconciliará a los pueblos dispersos y divididos; transformará la primera creación y Dios habitará en ella con los hombres en la paz.
	
	\n{716} El Pueblo de los \textquote{pobres} (cf. \emph{So} 2, 3; \emph{Sal} 22, 27; 34, 3; \emph{Is} 49, 13; 61, 1; etc.), los humildes y los mansos, totalmente entregados a los designios misteriosos de Dios, los que esperan la justicia, no de los hombres sino del Mesías, todo esto es, finalmente, la gran obra de la Misión escondida del Espíritu Santo durante el tiempo de las Promesas para preparar la venida de Cristo. Esta es la calidad de corazón del Pueblo, purificado e iluminado por el Espíritu, que se expresa en los Salmos. En estos pobres, el Espíritu prepara para el Señor \textquote{un pueblo bien dispuesto} (cf. \emph{Lc} 1, 17).
	
	\n{722} El Espíritu Santo \emph{preparó} a María con su gracia. Convenía que fuese \textquote{llena de gracia} la Madre de Aquel en quien \textquote{reside toda la plenitud de la divinidad corporalmente} (\emph{Col} 2, 9). Ella fue concebida sin pecado, por pura gracia, como la más humilde de todas las criaturas, la más capaz de acoger el don inefable del Omnipotente. Con justa razón, el ángel Gabriel la saluda como la \textquote{Hija de Sión}: \textquote{Alégrate} (cf. \emph{So} 3, 14; \emph{Za} 2, 14). Cuando ella lleva en sí al Hijo eterno, hace subir hasta el cielo con su cántico al Padre, en el Espíritu Santo, la acción de gracias de todo el pueblo de Dios y, por tanto, de la Iglesia (cf. \emph{Lc} 1, 46-55).

\end{ccebody}