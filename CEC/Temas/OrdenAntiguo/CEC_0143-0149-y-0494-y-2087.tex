\textquote{La obediencia de la fe}

CEC 143-149, 494, 2087:

\n{143} \emph{Por la fe}, el hombre somete completamente su inteligencia y su voluntad a Dios. Con todo su ser, el hombre da su asentimiento a Dios que revela (cf. DV 5). La sagrada Escritura llama \textquote{obediencia de la fe} a esta respuesta del hombre a Dios que revela (cf. \emph{Rm} 1,5; 16,26).

\ccesec{La obediencia de la fe. Abraham, \textquote{padre de todos los creyentes}}

\n{144} Obedecer (\emph{ob-audire}) en la fe es someterse libremente a la palabra escuchada, porque su verdad está garantizada por Dios, la Verdad misma. De esta obediencia, Abraham es el modelo que nos propone la Sagrada Escritura. La Virgen María es la realización más perfecta de la misma.

\n{145} La carta a los Hebreos, en el gran elogio de la fe de los antepasados, insiste particularmente en la fe de Abraham: \textquote{Por la fe, Abraham obedeció y salió para el lugar que había de recibir en herencia, y salió sin saber a dónde iba} (\emph{Hb} 11,8; cf. \emph{Gn} 12,1-4). Por la fe, vivió como extranjero y peregrino en la Tierra prometida (cf. \emph{Gn} 23,4). Por la fe, a Sara se le otorgó el concebir al hijo de la promesa. Por la fe, finalmente, Abraham ofreció a su hijo único en sacrificio (cf. \emph{Hb} 11,17).

\n{146} Abraham realiza así la definición de la fe dada por la carta a los Hebreos: \textquote{La fe es garantía de lo que se espera; la prueba de las realidades que no se ven} (\emph{Hb} 11,1). \textquote{Creyó Abraham en Dios y le fue reputado como justicia} (\emph{Rm} 4,3; cf. \emph{Gn} 15,6). Y por eso, fortalecido por su fe, Abraham fue hecho \textquote{padre de todos los creyentes} (\emph{Rm} 4,11.18; cf. \emph{Gn} 15, 5).

\n{147} El Antiguo Testamento es rico en testimonios acerca de esta fe. La carta a los Hebreos proclama el elogio de la fe ejemplar por la que los antiguos \textquote{fueron alabados} (\emph{Hb} 11, 2.39). Sin embargo, \textquote{Dios tenía ya dispuesto algo mejor}: la gracia de creer en su Hijo Jesús, \textquote{el que inicia y consuma la fe} (\emph{Hb} 11,40; 12,2).

\ccesec{María: \textquote{Dichosa la que ha creído}}

\n{148} La Virgen María realiza de la manera más perfecta la obediencia de la fe. En la fe, María acogió el anuncio y la promesa que le traía el ángel Gabriel, creyendo que \textquote{nada es imposible para Dios} (\emph{Lc} 1,37; cf. \emph{Gn} 18,14) y dando su asentimiento: \textquote{He aquí la esclava del Señor; hágase en mí según tu palabra} (\emph{Lc} 1,38). Isabel la saludó: \textquote{¡Dichosa la que ha creído que se cumplirían las cosas que le fueron dichas de parte del Señor!} (\emph{Lc} 1,45). Por esta fe todas las generaciones la proclamarán bienaventurada (cf. \emph{Lc} 1,48).

\n{149} Durante toda su vida, y hasta su última prueba (cf. \emph{Lc} 2,35), cuando Jesús, su hijo, murió en la cruz, su fe no vaciló. María no cesó de creer en el \textquote{cumplimiento} de la palabra de Dios. Por todo ello, la Iglesia venera en María la realización más pura de la fe.

\ccesec{\textquote{Hágase en mí según tu palabra \ldots{}}}

\n{494} Al anuncio de que ella dará a luz al \textquote{Hijo del Altísimo} sin conocer varón, por la virtud del Espíritu Santo (cf. Lc 1, 28-37), María respondió por \textquote{la obediencia de la fe} (Rm 1, 5), segura de que \textquote{nada hay imposible para Dios}: \textquote{He aquí la esclava del Señor: hágase en mí según tu palabra} (Lc 1, 37-38). Así dando su consentimiento a la palabra de Dios, María llegó a ser Madre de Jesús y, aceptando de todo corazón la voluntad divina de salvación, sin que ningún pecado se lo impidiera, se entregó a sí misma por entero a la persona y a la obra de su Hijo, para servir, en su dependencia y con él, por la gracia de Dios, al Misterio de la Redención (cf. LG 56):

\begin{quote} Ella, en efecto, como dice S. Ireneo, \textquote{por su obediencia fue causa de la salvación propia y de la de todo el género humano}. Por eso, no pocos Padres antiguos, en su predicación, coincidieron con él en afirmar \textquote{el nudo de la desobediencia de Eva lo desató la obediencia de María. Lo que ató la virgen Eva por su falta de fe lo desató la Virgen María por su fe}. Comparándola con Eva, llaman a María `Madre de los vivientes' y afirman con mayor frecuencia: \textquote{la muerte vino por Eva, la vida por María}. (LG. 56). \end{quote}

\ccesec{La fe}

\n{2087} Nuestra vida moral tiene su fuente en la fe en Dios que nos revela su amor. San Pablo habla de la \textquote{obediencia de la fe} (\emph{Rm} 1, 5; 16, 26) como de la primera obligación. Hace ver en el \textquote{desconocimiento de Dios} el principio y la explicación de todas las desviaciones morales (cf. \emph{Rm} 1, 18-32). Nuestro deber para con Dios es creer en Él y dar testimonio de Él.
