\cceth{La conversión de los bautizados}

\cceref{CEC 1427-1429}

\begin{ccebody}
	\n{1427} Jesús llama a la conversión. Esta llamada es una parte esencial del anuncio del Reino: \textquote{El tiempo se ha cumplido y el Reino de Dios está cerca; convertíos y creed en la Buena Nueva} (\emph{Mc} 1,15). En la predicación de la Iglesia, esta llamada se dirige primeramente a los que no conocen todavía a Cristo y su Evangelio. Así, el Bautismo es el lugar principal de la conversión primera y fundamental. Por la fe en la Buena Nueva y por el Bautismo (cf. \emph{Hch} 2,38) se renuncia al mal y se alcanza la salvación, es decir, la remisión de todos los pecados y el don de la vida nueva.
	
	\n{1428} Ahora bien, la llamada de Cristo a la conversión sigue resonando en la vida de los cristianos. Esta \emph{segunda conversión} es una tarea ininterrumpida para toda la Iglesia que \textquote{recibe en su propio seno a los pecadores} y que siendo \textquote{santa al mismo tiempo que necesitada de purificación constante, busca sin cesar la penitencia y la renovación} (LG 8). Este esfuerzo de conversión no es sólo una obra humana. Es el movimiento del \textquote{corazón contrito} (\emph{Sal} 51,19), atraído y movido por la gracia (cf. \emph{Jn} 6,44; 12,32) a responder al amor misericordioso de Dios que nos ha amado primero (cf. \emph{1 Jn} 4,10).
	
	\n{1429} De ello da testimonio la conversión de san Pedro tras la triple negación de su Maestro. La mirada de infinita misericordia de Jesús provoca las lágrimas del arrepentimiento (\emph{Lc} 22,61) y, tras la resurrección del Señor, la triple afirmación de su amor hacia él (cf. \emph{Jn} 21,15-17). La segunda conversión tiene también una dimensión \emph{comunitaria}. Esto aparece en la llamada del Señor a toda la Iglesia: \textquote{¡Arrepiéntete!} (\emph{Ap} 2,5.16).
	
	San Ambrosio dice acerca de las dos conversiones que, \textquote{en la Iglesia, existen el agua y las lágrimas: el agua del Bautismo y las lágrimas de la Penitencia} (\emph{Epistula extra collectionem} 1 {[}41{]}, 12).
\end{ccebody}