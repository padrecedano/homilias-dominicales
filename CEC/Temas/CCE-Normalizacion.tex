	%Navidad
	
	
	
	\n{456} Con el Credo Niceno-Constantinopolitano respondemos confesando: \textquote{\emph{Por nosotros los hombres y por nuestra salvación} bajó del cielo, y por obra del Espíritu Santo se encarnó de María la Virgen y se hizo hombre} (DS 150).
	
	\n{457} El Verbo se encarnó \emph{para salvarnos reconciliándonos con Dios}: \textquote{Dios nos amó y nos envió a su Hijo como propiciación por nuestros pecados} (\emph{1 Jn} 4, 10). \textquote{El Padre envió a su Hijo para ser salvador del mundo} (\emph{1 Jn} 4, 14). \textquote{Él se manifestó para quitar los pecados} (\emph{1 Jn} 3, 5):
	
	\textquote{Nuestra naturaleza enferma exigía ser sanada; desgarrada, ser restablecida; muerta, ser resucitada. Habíamos perdido la posesión del bien, era necesario que se nos devolviera. Encerrados en las tinieblas, hacía falta que nos llegara la luz; estando cautivos, esperábamos un salvador; prisioneros, un socorro; esclavos, un libertador. ¿No tenían importancia estos razonamientos? ¿No merecían conmover a Dios hasta el punto de hacerle bajar hasta nuestra naturaleza humana para visitarla, ya que la humanidad se encontraba en un estado tan miserable y tan desgraciado?} (San Gregorio de Nisa, \emph{Oratio catechetica}, 15: PG 45, 48B).
	
	\n{458} El Verbo se encarnó \emph{para que nosotros conociésemos así el amor de Dios}: \textquote{En esto se manifestó el amor que Dios nos tiene: en que Dios envió al mundo a su Hijo único para que vivamos por medio de él} (\emph{1 Jn} 4, 9). \textquote{Porque tanto amó Dios al mundo que dio a su Hijo único, para que todo el que crea en él no perezca, sino que tenga vida eterna} (\emph{Jn} 3, 16).
	
	\n{459} El Verbo se encarnó \emph{para ser nuestro modelo de santidad}: \textquote{Tomad sobre vosotros mi yugo, y aprended de mí \ldots{} } (\emph{Mt} 11, 29). \textquote{Yo soy el Camino, la Verdad y la Vida. Nadie va al Padre sino por mí} (\emph{Jn} 14, 6). Y el Padre, en el monte de la Transfiguración, ordena: \textquote{Escuchadle} (\emph{Mc} 9, 7; cf. \emph{Dt} 6, 4-5). Él es, en efecto, el modelo de las bienaventuranzas y la norma de la Ley nueva: \textquote{Amaos los unos a los otros como yo os he amado} (\emph{Jn} 15, 12). Este amor tiene como consecuencia la ofrenda efectiva de sí mismo (cf. \emph{Mc} 8, 34).
	
	\n{460} El Verbo se encarnó \emph{para hacernos \textquote{partícipes de la naturaleza divina}} (\emph{2 P} 1, 4): \textquote{Porque tal es la razón por la que el Verbo se hizo hombre, y el Hijo de Dios, Hijo del hombre: para que el hombre al entrar en comunión con el Verbo y al recibir así la filiación divina, se convirtiera en hijo de Dios} (San Ireneo de Lyon, \emph{Adversus haereses}, 3, 19, 1). \textquote{Porque el Hijo de Dios se hizo hombre para hacernos Dios} (San Atanasio de Alejandría, \emph{De Incarnatione}, 54, 3: PG 25, 192B). \emph{Unigenitus} [\ldots{}] \emph{Dei Filius, suae divinitatis volens nos esse participes, naturam nostram assumpsit, ut homines deos faceret factus homo} (\textquote{El Hijo Unigénito de Dios, queriendo hacernos partícipes de su divinidad, asumió nuestra naturaleza, para que, habiéndose hecho hombre, hiciera dioses a los hombres}) (Santo Tomás de Aquino, \emph{Oficio de la festividad del Corpus}, Of. de Maitines, primer Nocturno, Lectura I).
	
	\n{566} \emph{La tentación en el desierto muestra a Jesús, humilde Mesías que triunfa de Satanás mediante su total adhesión al designio de salvación querido por el Padre}.
	
	La Encarnación
	
	CEC 461-463, 470-478:
	
	\n{461} Volviendo a tomar la frase de san Juan (\textquote{El Verbo se encarnó}: \emph{Jn} 1, 14), la Iglesia llama \textquote{Encarnación} al hecho de que el Hijo de Dios haya asumido una naturaleza humana para llevar a cabo por ella nuestra salvación. En un himno citado por san Pablo, la Iglesia canta el misterio de la Encarnación:
	
	\textquote{Tened entre vosotros los mismos sentimientos que tuvo Cristo: el cual, siendo de condición divina, no retuvo ávidamente el ser igual a Dios, sino que se despojó de sí mismo tomando condición de siervo, haciéndose semejante a los hombres y apareciendo en su porte como hombre; y se humilló a sí mismo, obedeciendo hasta la muerte y muerte de cruz} (\emph{Flp} 2, 5-8; cf. \emph{Liturgia de las Horas, Cántico de las Primeras Vísperas de Domingos}).
	
	\n{462} La carta a los Hebreos habla del mismo misterio:
	
	\begin{quote}
		\textquote{Por eso, al entrar en este mundo, {[}Cristo{]} dice: No quisiste sacrificio y oblación; pero me has formado un cuerpo. Holocaustos y sacrificios por el pecado no te agradaron. Entonces dije: ¡He aquí que vengo [\ldots{}] a hacer, oh Dios, tu voluntad!} (\emph{Hb} 10, 5-7; \emph{Sal} 40, 7-9 {[}LXX{]}).
	\end{quote}
	
	
	
	\n{463} La fe en la verdadera encarnación del Hijo de Dios es el signo distintivo de la fe cristiana: \textquote{Podréis conocer en esto el Espíritu de Dios: todo espíritu que confiesa a Jesucristo, venido en carne, es de Dios} (\emph{1 Jn} 4, 2). Esa es la alegre convicción de la Iglesia desde sus comienzos cuando canta \textquote{el gran misterio de la piedad}: \textquote{Él ha sido manifestado en la carne} (\emph{1 Tm} 3, 16).
	
	\n{470} Puesto que en la unión misteriosa de la Encarnación \textquote{la naturaleza humana ha sido asumida, no absorbida} (GS 22, 2), la Iglesia ha llegado a confesar con el correr de los siglos, la plena realidad del alma humana, con sus operaciones de inteligencia y de voluntad, y del cuerpo humano de Cristo. Pero paralelamente, ha tenido que recordar en cada ocasión que la naturaleza humana de Cristo pertenece propiamente a la persona divina del Hijo de Dios que la ha asumido. Todo lo que es y hace en ella proviene de \textquote{uno de la Trinidad}. El Hijo de Dios comunica, pues, a su humanidad su propio modo personal de existir en la Trinidad. Así, en su alma como en su cuerpo, Cristo expresa humanamente las costumbres divinas de la Trinidad (cf. \emph{Jn} 14, 9-10):
	
	\textquote{El Hijo de Dios [\ldots{}] trabajó con manos de hombre, pensó con inteligencia de hombre, obró con voluntad de hombre, amó con corazón de hombre. Nacido de la Virgen María, se hizo verdaderamente uno de nosotros, en todo semejante a nosotros, excepto en el pecado} (GS 22, 2).
	
	\n{471} Apolinar de Laodicea afirmaba que en Cristo el Verbo había sustituido al alma o al espíritu. Contra este error la Iglesia confesó que el Hijo eterno asumió también un alma racional humana (cf. Dámaso I, Carta a los Obispos Orientales: DS, 149).
	
	\n{472} Este alma humana que el Hijo de Dios asumió está dotada de un verdadero conocimiento humano. Como tal, éste no podía ser de por sí ilimitado: se desenvolvía en las condiciones históricas de su existencia en el espacio y en el tiempo. Por eso el Hijo de Dios, al hacerse hombre, quiso progresar \textquote{en sabiduría, en estatura y en gracia} (\emph{Lc} 2, 52) e igualmente adquirir aquello que en la condición humana se adquiere de manera experimental (cf. \emph{Mc} 6, 38; 8, 27; \emph{Jn} 11, 34; etc.). Eso correspondía a la realidad de su anonadamiento voluntario en \textquote{la condición de esclavo} (\emph{Flp} 2, 7).
	
	\n{473} Pero, al mismo tiempo, este conocimiento verdaderamente humano del Hijo de Dios expresaba la vida divina de su persona (cf. san Gregorio Magno, carta \emph{Sicut aqua}: DS, 475). \textquote{El Hijo de Dios conocía todas las cosas; y esto por sí mismo, que se había revestido de la condición humana; no por su naturaleza, sino en cuanto estaba unida al Verbo [\ldots{}]. La naturaleza humana, en cuanto estaba unida al Verbo, conocida todas las cosas, incluso las divinas, y manifestaba en sí todo lo que conviene a Dios} (san Máximo el Confesor, \emph{Quaestiones et dubia}, 66: PG 90, 840). Esto sucede ante todo en lo que se refiere al conocimiento íntimo e inmediato que el Hijo de Dios hecho hombre tiene de su Padre (cf. \emph{Mc} 14, 36; \emph{Mt} 11, 27; \emph{Jn}1, 18; 8, 55; etc.). El Hijo, en su conocimiento humano, mostraba también la penetración divina que tenía de los pensamientos secretos del corazón de los hombres (cf. \emph{Mc} 2, 8; \emph{Jn} 2, 25; 6, 61; etc.).
	
	\n{474} Debido a su unión con la Sabiduría divina en la persona del Verbo encarnado, el conocimiento humano de Cristo gozaba en plenitud de la ciencia de los designios eternos que había venido a revelar (cf. \emph{Mc} 8,31; 9,31; 10, 33-34; 14,18-20. 26-30). Lo que reconoce ignorar en este campo (cf. \emph{Mc} 13,32), declara en otro lugar no tener misión de revelarlo (cf. \emph{Hch} 1, 7).
	
	\n{475} De manera paralela, la Iglesia confesó en el sexto Concilio Ecuménico que Cristo posee dos voluntades y dos operaciones naturales, divinas y humanas, no opuestas, sino cooperantes, de forma que el Verbo hecho carne, en su obediencia al Padre, ha querido humanamente todo lo que ha decidido divinamente con el Padre y el Espíritu Santo para nuestra salvación (cf. Concilio de Constantinopla III, año 681: DS, 556-559). La voluntad humana de Cristo \textquote{sigue a su voluntad divina sin hacerle resistencia ni oposición, sino todo lo contrario, estando subordinada a esta voluntad omnipotente} (\emph{ibíd}., 556).
	
	\n{476} Como el Verbo se hizo carne asumiendo una verdadera humanidad, el cuerpo de Cristo era limitado (cf. Concilio de Letrán, año 649: DS, 504). Por eso se puede \textquote{pintar} la faz humana de Jesús (\emph{Ga} 3,2). En el séptimo Concilio ecuménico, la Iglesia reconoció que es legítima su representación en imágenes sagradas (Concilio de Nicea II, año 787: DS, 600-603).
	
	\n{477} Al mismo tiempo, la Iglesia siempre ha admitido que, en el cuerpo de Jesús, Dios \textquote{que era invisible en su naturaleza se hace visible} (\emph{Misal Romano}, Prefacio de Navidad). En efecto, las particularidades individuales del cuerpo de Cristo expresan la persona divina del Hijo de Dios. Él ha hecho suyos los rasgos de su propio cuerpo humano hasta el punto de que, pintados en una imagen sagrada, pueden ser venerados porque el creyente que venera su imagen, \textquote{venera a la persona representada en ella} (Concilio de Nicea II: DS, 601).
	
	\n{478} Jesús, durante su vida, su agonía y su pasión nos ha conocido y amado a todos y a cada uno de nosotros y se ha entregado por cada uno de nosotros: \textquote{El Hijo de Dios me amó y se entregó a sí mismo por mí} (\emph{Ga} 2, 20). Nos ha amado a todos con un corazón humano. Por esta razón, el sagrado Corazón de Jesús, traspasado por nuestros pecados y para nuestra salvación (cf. \emph{Jn} 19, 34), \textquote{es considerado como el principal indicador y símbolo [\ldots{}] de aquel amor con que el divino Redentor ama continuamente al eterno Padre y a todos los hombres} (Pío XII, Enc. \emph{Haurietis aquas}: DS, 3924; cf. ID. enc. \emph{Mystici Corporis}: ibíd., 3812).
	
	El misterio de la Navidad
	
	CEC 437, 525-526:
	
	\n{437} El ángel anunció a los pastores el nacimiento de Jesús como el del Mesías prometido a Israel: \textquote{Os ha nacido hoy, en la ciudad de David, un salvador, que es el Cristo Señor} (\emph{Lc} 2, 11). Desde el principio él es \textquote{a quien el Padre ha santificado y enviado al mundo} (\emph{Jn} 10, 36), concebido como \textquote{santo} (\emph{Lc} 1, 35) en el seno virginal de María. José fue llamado por Dios para \textquote{tomar consigo a María su esposa} encinta \textquote{del que fue engendrado en ella por el Espíritu Santo} (\emph{Mt} 1, 20) para que Jesús \textquote{llamado Cristo} nazca de la esposa de José en la descendencia mesiánica de David (\emph{Mt} 1, 16; cf. \emph{Rm} 1, 3; \emph{2 Tm} 2, 8; \emph{Ap} 22, 16).
	
	\n{525} Jesús nació en la humildad de un establo, de una familia pobre (cf. \emph{Lc} 2, 6-7); unos sencillos pastores son los primeros testigos del acontecimiento. En esta pobreza se manifiesta la gloria del cielo (cf. \emph{Lc} 2, 8-20). La Iglesia no se cansa de cantar la gloria de esta noche:
	
	\begin{quote}
		\textquote{Hoy la Virgen da a luz al Transcendente.\\ Y la tierra ofrece una cueva al Inaccesible.\\ Los ángeles y los pastores le alaban.\\ Los magos caminan con la estrella:\\ Porque ha nacido por nosotros,\\ Niño pequeñito\\ el Dios eterno}
		
		(San Romano Melodo, Kontakion, 10)
	\end{quote}
	
	
	
	\n{526} \textquote{Hacerse niño} con relación a Dios es la condición para entrar en el Reino (cf. \emph{Mt} 18, 3-4); para eso es necesario abajarse (cf. \emph{Mt} 23, 12), hacerse pequeño; más todavía: es necesario \textquote{nacer de lo alto} (\emph{Jn} 3,7), \textquote{nacer de Dios} (\emph{Jn} 1, 13) para \textquote{hacerse hijos de Dios} (\emph{Jn} 1, 12). El misterio de Navidad se realiza en nosotros cuando Cristo \textquote{toma forma} en nosotros (\emph{Ga}4, 19). Navidad es el misterio de este \textquote{admirable intercambio}:
	
	\begin{quote}
		\textquote{¡Oh admirable intercambio! El Creador del género humano, tomando cuerpo y alma, nace de la Virgen y, hecho hombre sin concurso de varón, nos da parte en su divinidad} (\emph{Solemnidad de la Santísima Virgen María, Madre de Dios,} Antífona de I y II Vísperas: \emph{Liturgia de las Horas}).
	\end{quote}
	
	
	
	Jesús es el Hijo de David
	
	CEC 439, 496, 559, 2616:
	
	\n{439} Numerosos judíos e incluso ciertos paganos que compartían su esperanza reconocieron en Jesús los rasgos fundamentales del mesiánico \textquote{hijo de David} prometido por Dios a Israel (cf. \emph{Mt} 2, 2; 9, 27; 12, 23; 15, 22; 20, 30; 21, 9. 15). Jesús aceptó el título de Mesías al cual tenía derecho (cf. \emph{Jn} 4, 25-26;11, 27), pero no sin reservas porque una parte de sus contemporáneos lo comprendían según una concepción demasiado humana (cf. \emph{Mt} 22, 41-46), esencialmente política (cf. \emph{Jn} 6, 15; \emph{Lc} 24, 21).
	
	\ccesec{La virginidad de María}
	
	\n{496} Desde las primeras formulaciones de la fe (cf. DS 10-64), la Iglesia ha confesado que Jesús fue concebido en el seno de la Virgen María únicamente por el poder del Espíritu Santo, afirmando también el aspecto corporal de este suceso: Jesús fue concebido \emph{absque semine ex Spiritu Sancto} (Concilio de Letrán, año 649; DS, 503), esto es, sin semilla de varón, por obra del Espíritu Santo. Los Padres ven en la concepción virginal el signo de que es verdaderamente el Hijo de Dios el que ha venido en una humanidad como la nuestra:
	
	Así, san Ignacio de Antioquía (comienzos del siglo II): \textquote{Estáis firmemente convencidos acerca de que nuestro Señor es verdaderamente de la raza de David según la carne (cf. \emph{Rm} 1, 3), Hijo de Dios según la voluntad y el poder de Dios (cf. \emph{Jn} 1, 13), nacido verdaderamente de una virgen [\ldots{}] Fue verdaderamente clavado por nosotros en su carne bajo Poncio Pilato [\ldots{}] padeció verdaderamente, como también resucitó verdaderamente} (\emph{Epistula ad Smyrnaeos}, 1-2).
	
	\ccesec{La entrada mesiánica de Jesús en Jerusalén}
	
	\n{559} ¿Cómo va a acoger Jerusalén a su Mesías? Jesús rehuyó siempre las tentativas populares de hacerle rey (cf. \emph{Jn} 6, 15), pero elige el momento y prepara los detalles de su entrada mesiánica en la ciudad de \textquote{David, su padre} (\emph{Lc} 1,32; cf. \emph{Mt} 21, 1-11). Es aclamado como hijo de David, el que trae la salvación (\textquote{Hosanna} quiere decir \textquote{¡sálvanos!}, \textquote{Danos la salvación!}). Pues bien, el \textquote{Rey de la Gloria} (\emph{Sal} 24, 7-10) entra en su ciudad \textquote{montado en un asno} (\emph{Za} 9, 9): no conquista a la hija de Sión, figura de su Iglesia, ni por la astucia ni por la violencia, sino por la humildad que da testimonio de la Verdad (cf. \emph{Jn} 18, 37). Por eso los súbditos de su Reino, aquel día fueron los niños (cf. \emph{Mt} 21, 15-16; \emph{Sal} 8, 3) y los \textquote{pobres de Dios}, que le aclamaban como los ángeles lo anunciaron a los pastores (cf. \emph{Lc} 19, 38; 2, 14). Su aclamación \textquote{Bendito el que viene en el nombre del Señor} (\emph{Sal} 118, 26), ha sido recogida por la Iglesia en el \emph{Sanctus} de la liturgia eucarística para introducir al memorial de la Pascua del Señor.
	
	\ccesec{Jesús escucha la oración}
	
	\n{2616} La oración \emph{a Jesús} ya ha sido escuchada por Él durante su ministerio, a través de signos que anticipan el poder de su muerte y de su resurrección: Jesús escucha la oración de fe expresada en palabras (del leproso {[}cf. \emph{Mc} 1, 40-41{]}, de Jairo {[}cf. \emph{Mc} 5, 36{]}, de la cananea {[}cf. \emph{Mc} 7, 29{]}, del buen ladrón {[}cf. \emph{Lc} 23, 39-43{]}), o en silencio (de los portadores del paralítico {[}cf. \emph{Mc} 2, 5{]}, de la hemorroisa {[}cf. \emph{Mc} 5, 28{]} que toca el borde de su manto, de las lágrimas y el perfume de la pecadora {[}cf. \emph{Lc} 7, 37-38{]}). La petición apremiante de los ciegos: \textquote{¡Ten piedad de nosotros, Hijo de David!} (\emph{Mt} 9, 27) o \textquote{¡Hijo de David, Jesús, ten compasión de mí!} (\emph{Mc} 10, 48) ha sido recogida en la tradición de la \emph{Oración a Jesús}: \textquote{Señor Jesucristo, Hijo de Dios, ten piedad de mí, pecador}. Sanando enfermedades o perdonando pecados, Jesús siempre responde a la plegaria del que le suplica con fe: \textquote{Ve en paz, ¡tu fe te ha salvado!}.
	
	San Agustín resume admirablemente las tres dimensiones de la oración de Jesús: \emph{Orat pro nobis ut sacerdos noster, orat in nobis ut caput nostrum, oratur a nobis ut Deus noster. Agnoscamus ergo et in illo voces nostras et voces eius in nobis} (\textquote{Ora por nosotros como sacerdote nuestro; ora en nosotros como cabeza nuestra; a Él se dirige nuestra oración como a Dios nuestro. Reconozcamos, por tanto, en Él nuestras voces; y la voz de Él, en nosotros}) (\emph{Enarratio in Psalmum} 85, 1; cf. I\emph{nstitución general de la Liturgia de las Horas,} 7).
	
	Dios ha dicho todo en su Verbo
	
	CEC 65, 102:
	
	\n{65} \textquote{Muchas veces y de muchos modos habló Dios en el pasado a nuestros padres por medio de los profetas; en estos últimos tiempos nos ha hablado por su Hijo} (\emph{Hb} 1,1-2). Cristo, el Hijo de Dios hecho hombre, es la Palabra única, perfecta e insuperable del Padre. En Él lo dice todo, no habrá otra palabra más que ésta. San Juan de la Cruz, después de otros muchos, lo expresa de manera luminosa, comentando \emph{Hb} 1,1-2:
	
	\begin{quote}
		\textquote{Porque en darnos, como nos dio a su Hijo, que es una Palabra suya, que no tiene otra, todo nos lo habló junto y de una vez en esta sola Palabra [\ldots{}]; porque lo que hablaba antes en partes a los profetas ya lo ha hablado todo en Él, dándonos al Todo, que es su Hijo. Por lo cual, el que ahora quisiese preguntar a Dios, o querer alguna visión o revelación, no sólo haría una necedad, sino haría agravio a Dios, no poniendo los ojos totalmente en Cristo, sin querer otra alguna cosa o novedad} (San Juan de la Cruz, \emph{Subida del monte Carmelo} 2,22,3-5: \emph{Biblioteca Mística Carmelitana,} v. 11 (Burgos 1929), p. 184.).
	\end{quote}
	
	
	
	\n{102} A través de todas las palabras de la sagrada Escritura, Dios dice sólo una palabra, su Verbo único, en quien él se da a conocer en plenitud (cf. \emph{Hb} 1,1-3):
	
	\textquote{Recordad que es una misma Palabra de Dios la que se extiende en todas las escrituras, que es un mismo Verbo que resuena en la boca de todos los escritores sagrados, el que, siendo al comienzo Dios junto a Dios, no necesita sílabas porque no está sometido al tiempo} (San Agustín, \emph{Enarratio in Psalmum,} 103,4,1).
	
	Cristo encarnado es adorado por los ángeles
	
	CEC 333:
	
	\n{333} De la Encarnación a la Ascensión, la vida del Verbo encarnado está rodeada de la adoración y del servicio de los ángeles. Cuando Dios introduce \textquote{a su Primogénito en el mundo, dice: \textquote{adórenle todos los ángeles de Dios}} (\emph{Hb} 1, 6). Su cántico de alabanza en el nacimiento de Cristo no ha cesado de resonar en la alabanza de la Iglesia: \textquote{Gloria a Dios\ldots{}} (\emph{Lc} 2, 14). Protegen la infancia de Jesús (cf. \emph{Mt} 1, 20; 2, 13.19), le sirven en el desierto (cf. \emph{Mc} 1, 12; \emph{Mt} 4, 11), lo reconfortan en la agonía (cf. \emph{Lc} 22, 43), cuando Él habría podido ser salvado por ellos de la mano de sus enemigos (cf. \emph{Mt} 26, 53) como en otro tiempo Israel (cf. \emph{2 M} 10, 29-30; 11,8). Son también los ángeles quienes \textquote{evangelizan} (\emph{Lc} 2, 10) anunciando la Buena Nueva de la Encarnación (cf. \emph{Lc} 2, 8-14), y de la Resurrección (cf. \emph{Mc} 16, 5-7) de Cristo. Con ocasión de la segunda venida de Cristo, anunciada por los ángeles (cf. \emph{Hb} 1, 10-11), éstos estarán presentes al servicio del juicio del Señor (cf. \emph{Mt} 13, 41; 25, 31 ; \emph{Lc} 12, 8-9).
	
	La Encarnación y las imágenes de Cristo
	
	CEC 1159-1162, 2131, 2502:
	
	\n{1159} La imagen sagrada, el icono litúrgico, representa principalmente \emph{a} \emph{Cristo}. No puede representar a Dios invisible e incomprensible; la Encarnación del Hijo de Dios inauguró una nueva \textquote{economía} de las imágenes:
	
	\begin{quote}
		\textquote{En otro tiempo, Dios, que no tenía cuerpo ni figura no podía de ningún modo ser representado con una imagen. Pero ahora que se ha hecho ver en la carne y que ha vivido con los hombres, puedo hacer una imagen de lo que he visto de Dios. [\ldots{}] Nosotros sin embargo, revelado su rostro, contemplamos la gloria del Señor} (San Juan Damasceno, \emph{De sacris imaginibus oratio} 1,16).
	\end{quote}

	
	\n{1160} La iconografía cristiana transcribe a través de la imagen el mensaje evangélico que la sagrada Escritura transmite mediante la palabra. Imagen y Palabra se esclarecen mutuamente:
	
	\begin{quote}
		\textquote{Para expresarnos brevemente: conservamos intactas todas las tradiciones de la Iglesia, escritas o no escritas, que nos han sido transmitidas sin alteración. Una de ellas es la representación pictórica de las imágenes, que está de acuerdo con la predicación de la historia evangélica, creyendo que, verdaderamente y no en apariencia, el Dios Verbo se hizo carne, lo cual es tan útil y provechoso, porque las cosas que se esclarecen mutuamente tienen sin duda una significación recíproca} (Concilio de Nicea II, año 787, \emph{Terminus}: COD 111).
	\end{quote}
	
	\n{1161} Todos los signos de la celebración litúrgica hacen referencia a Cristo: también las imágenes sagradas de la Santísima Madre de Dios y de los santos. Significan, en efecto, a Cristo que es glorificado en ellos. Manifiestan \textquote{la nube de testigos} (\emph{Hb} 12,1) que continúan participando en la salvación del mundo y a los que estamos unidos, sobre todo en la celebración sacramental. A través de sus iconos, es el hombre \textquote{a imagen de Dios}, finalmente transfigurado \textquote{a su semejanza} (cf. \emph{Rm} 8,29; \emph{1 Jn} 3,2), quien se revela a nuestra fe, e incluso los ángeles, recapitulados también en Cristo:
	
	\begin{quote}
		\textquote{Siguiendo [\ldots{}] la enseñanza divinamente inspirada de nuestros santos Padres y la Tradición de la Iglesia católica (pues reconocemos ser del Espíritu Santo que habita en ella), definimos con toda exactitud y cuidado que la imagen de la preciosa y vivificante cruz, así como también las venerables y santas imágenes, tanto las pintadas como las de mosaico u otra materia conveniente, se expongan en las santas iglesias de Dios, en los vasos sagrados y ornamentos, en las paredes y en cuadros, en las casas y en los caminos: tanto las imágenes de nuestro Señor Dios y Salvador Jesucristo, como las de nuestra Señora inmaculada la santa Madre de Dios, de los santos ángeles y de todos los santos y justos} (Concilio de Nicea II: DS 600).
	\end{quote}
	
	\n{1162} \textquote{La belleza y el color de las imágenes estimulan mi oración. Es una fiesta para mis ojos, del mismo modo que el espectáculo del campo estimula mi corazón para dar gloria a Dios} (San Juan Damasceno, \emph{De sacris imaginibus oratio} 127). La contemplación de las sagradas imágenes, unida a la meditación de la Palabra de Dios y al canto de los himnos litúrgicos, forma parte de la armonía de los signos de la celebración para que el misterio celebrado se grabe en la memoria del corazón y se exprese luego en la vida nueva de los fieles.
	
	\n{2131} Fundándose en el misterio del Verbo encarnado, el séptimo Concilio Ecuménico (celebrado en Nicea el año 787), justificó contra los iconoclastas el culto de las sagradas imágenes: las de Cristo, pero también las de la Madre de Dios, de los ángeles y de todos los santos. El Hijo de Dios, al encarnarse, inauguró una nueva \textquote{economía} de las imágenes.
	
	\n{2502} El \emph{arte sacro} es verdadero y bello cuando corresponde por su forma a su vocación propia: evocar y glorificar, en la fe y la adoración, el Misterio trascendente de Dios, Belleza supereminente e invisible de Verdad y de Amor, manifestado en Cristo, \textquote{Resplandor de su gloria e Impronta de su esencia} (\emph{Hb} 1, 3), en quien \textquote{reside toda la Plenitud de la Divinidad corporalmente} (\emph{Col} 2, 9), belleza espiritual reflejada en la Santísima Virgen Madre de Dios, en los Ángeles y los Santos. El arte sacro verdadero lleva al hombre a la adoración, a la oración y al amor de Dios Creador y Salvador, Santo y Santificador.
	
	\begin{patercite}
		El Señor vino a ella para hacerse siervo.
	
		El Verbo vino a ella para callar en su seno.
		
		El rayo vino a ella para no hacer ruido.
		
		El pastor vino a ella, y nació el Cordero, que llora dulcemente.
		
		El seno de María ha trastocado los papeles:
		
		Quien creó todo se ha apoderado de él, pero en la pobreza.
		
		El Altísimo vino a ella,
		
		pero entró humildemente.
		
		El esplendor vino a ella, pero vestido con ropas humildes.
		
		Quien todo lo da experimentó el hambre.
		
		Quien da de beber a todos sufrió la sed.
		
		Desnudo salió de ella, quien todo lo reviste»
		
		(\n{San Efrén,} Himno \emph{De Nativitate} 11, 6-8).				
	\end{patercite}	

		

	
	
	%S Familia
	La Sagrada Familia
	
	CEC 531-534:
	
	\n{Los misterios de la vida oculta de Jesús}
	
	\n{531} Jesús compartió, durante la mayor parte de su vida, la condición de la inmensa mayoría de los hombres: una vida cotidiana sin aparente importancia, vida de trabajo manual, vida religiosa judía sometida a la ley de Dios (cf. \emph{Ga} 4, 4), vida en la comunidad. De todo este período se nos dice que Jesús estaba \textquote{sometido} a sus padres y que \textquote{progresaba en sabiduría, en estatura y en gracia ante Dios y los hombres} (\emph{Lc} 2, 51-52).
	
	\n{532} Con la sumisión a su madre, y a su padre legal, Jesús cumple con perfección el cuarto mandamiento. Es la imagen temporal de su obediencia filial a su Padre celestial. La sumisión cotidiana de Jesús a José y a María anunciaba y anticipaba la sumisión del Jueves Santo: \textquote{No se haga mi voluntad \ldots{}} (\emph{Lc} 22, 42). La obediencia de Cristo en lo cotidiano de la vida oculta inauguraba ya la obra de restauración de lo que la desobediencia de Adán había destruido (cf. \emph{Rm} 5, 19).
	
	\n{533} La vida oculta de Nazaret permite a todos entrar en comunión con Jesús a través de los caminos más ordinarios de la vida humana:
	
	\textquote{Nazaret es la escuela donde empieza a entenderse la vida de Jesús, es la escuela donde se inicia el conocimiento de su Evangelio. [\ldots{}] Su primera lección es el \emph{silencio}. Cómo desearíamos que se renovara y fortaleciera en nosotros el amor al silencio, este admirable e indispensable hábito del espíritu, tan necesario para nosotros. [\ldots{}] Se nos ofrece además una lección de \emph{vida familiar}. Que Nazaret nos enseñe el significado de la familia, su comunión de amor, su sencilla y austera belleza, su carácter sagrado e inviolable. [\ldots{}] Finalmente, aquí aprendemos también la \emph{lección del trabajo}. Nazaret, la casa del \textquote{hijo del Artesano}: cómo deseamos comprender más en este lugar la austera pero redentora ley del trabajo humano y exaltarla debidamente. [\ldots{}] Queremos finalmente saludar desde aquí a todos los trabajadores del mundo y señalarles al gran modelo, al hermano divino} (Pablo VI, \emph{Homilía en el templo de la Anunciación de la Virgen María en Nazaret} (5 de enero de 1964).
	
	\n{534} \emph{El hallazgo de Jesús en el Templo} (cf. \emph{Lc} 2, 41-52) es el único suceso que rompe el silencio de los Evangelios sobre los años ocultos de Jesús. Jesús deja entrever en ello el misterio de su consagración total a una misión derivada de su filiación divina: \textquote{¿No sabíais que me debo a los asuntos de mi Padre?} María y José \textquote{no comprendieron} esta palabra, pero la acogieron en la fe, y María \textquote{conservaba cuidadosamente todas las cosas en su corazón}, a lo largo de todos los años en que Jesús permaneció oculto en el silencio de una vida ordinaria.
	
	La familia cristiana, una Iglesia doméstica
	
	CEC 1655-1658, 2204-2206:
	
	\n{1655} Cristo quiso nacer y crecer en el seno de la Sagrada Familia de José y de María. La Iglesia no es otra cosa que la \textquote{familia de Dios}. Desde sus orígenes, el núcleo de la Iglesia estaba a menudo constituido por los que, \textquote{con toda su casa}, habían llegado a ser creyentes (cf. \emph{Hch} 18,8). Cuando se convertían deseaban también que se salvase \textquote{toda su casa} (cf. \emph{Hch} 16,31; 11,14). Estas familias convertidas eran islotes de vida cristiana en un mundo no creyente.
	
	\n{1656} En nuestros días, en un mundo frecuentemente extraño e incluso hostil a la fe, las familias creyentes tienen una importancia primordial en cuanto faros de una fe viva e irradiadora. Por eso el Concilio Vaticano II llama a la familia, con una antigua expresión, \emph{Ecclesia domestica} (LG 11; cf. FC 21). En el seno de la familia, \textquote{los padres han de ser para sus hijos los primeros anunciadores de la fe con su palabra y con su ejemplo, y han de fomentar la vocación personal de cada uno y, con especial cuidado, la vocación a la vida consagrada} (LG 11).
	
	\n{1657} Aquí es donde se ejercita de manera privilegiada el \emph{sacerdocio bautismal} del padre de familia, de la madre, de los hijos, de todos los miembros de la familia, \textquote{en la recepción de los sacramentos, en la oración y en la acción de gracias, con el testimonio de una vida santa, con la renuncia y el amor que se traduce en obras} (LG 10). El hogar es así la primera escuela de vida cristiana y \textquote{escuela del más rico humanismo} (GS 52,1). Aquí se aprende la paciencia y el gozo del trabajo, el amor fraterno, el perdón generoso, incluso reiterado, y sobre todo el culto divino por medio de la oración y la ofrenda de la propia vida.
	
	\n{1658} Es preciso recordar asimismo a un gran número de \emph{personas que permanecen solteras} a causa de las concretas condiciones en que deben vivir, a menudo sin haberlo querido ellas mismas. Estas personas se encuentran particularmente cercanas al corazón de Jesús; y, por ello, merecen afecto y solicitud diligentes de la Iglesia, particularmente de sus pastores. Muchas de ellas viven \emph{sin familia humana}, con frecuencia a causa de condiciones de pobreza. Hay quienes viven su situación según el espíritu de las bienaventuranzas sirviendo a Dios y al prójimo de manera ejemplar. A todas ellas es preciso abrirles las puertas de los hogares, \textquote{iglesias domésticas} y las puertas de la gran familia que es la Iglesia. \textquote{Nadie se sienta sin familia en este mundo: la Iglesia es casa y familia de todos, especialmente para cuantos están \textquote{fatigados y agobiados} (\emph{Mt} 11,28)} (FC 85).
	
	\n{2204} \textquote{La familia cristiana constituye una revelación y una actuación específicas de la comunión eclesial; por eso [\ldots{}] puede y debe decirse \emph{Iglesia doméstica}} (FC 21, cf. LG 11). Es una comunidad de fe, esperanza y caridad, posee en la Iglesia una importancia singular como aparece en el Nuevo Testamento (cf. \emph{Ef} 5, 21-6, 4; \emph{Col} 3, 18-21; \emph{1 P} 3, 1-7).
	
	\n{2205} La familia cristiana es una comunión de personas, reflejo e imagen de la comunión del Padre y del Hijo en el Espíritu Santo. Su actividad procreadora y educativa es reflejo de la obra creadora de Dios. Es llamada a participar en la oración y el sacrificio de Cristo. La oración cotidiana y la lectura de la Palabra de Dios fortalecen en ella la caridad. La familia cristiana es evangelizadora y misionera.
	
	\n{2206} Las relaciones en el seno de la familia entrañan una afinidad de sentimientos, afectos e intereses que provienen sobre todo del mutuo respeto de las personas. La familia es una \emph{comunidad privilegiada} llamada a realizar un propósito común de los esposos y una cooperación diligente de los padres en la educación de los hijos (cf. GS 52).
	
	Los deberes de los miembros de la familia
	
	CEC 2214-2233:
	
	\ccesec{Deberes de los hijos}
	
	\n{2214} La paternidad divina es la fuente de la paternidad humana (cf. \emph{Ef} 3, 14); es el fundamento del honor debido a los padres. El respeto de los hijos, menores o mayores de edad, hacia su padre y hacia su madre (cf. \emph{Pr} 1, 8; \emph{Tb} 4, 3-4), se nutre del afecto natural nacido del vínculo que los une. Es exigido por el precepto divino (cf. \emph{Ex} 20, 12).
	
	\n{2215} El respeto a los padres (\emph{piedad filial)} está hecho de \emph{gratitud} para quienes, mediante el don de la vida, su amor y su trabajo, han traído sus hijos al mundo y les han ayudado a crecer en estatura, en sabiduría y en gracia. \textquote{Con todo tu corazón honra a tu padre, y no olvides los dolores de tu madre. Recuerda que por ellos has nacido, ¿cómo les pagarás lo que contigo han hecho?} (\emph{Si} 7, 27-28).
	
	\n{2216} El respeto filial se expresa en la docilidad y la \emph{obediencia} verdaderas. \textquote{Guarda, hijo mío, el mandato de tu padre y no desprecies la lección de tu madre [\ldots{}] en tus pasos ellos serán tu guía; cuando te acuestes, velarán por ti; conversarán contigo al despertar} (\emph{Pr} 6, 20-22). \textquote{El hijo sabio ama la instrucción, el arrogante no escucha la reprensión} (\emph{Pr} 13, 1).
	
	\n{2217} Mientras vive en el domicilio de sus padres, el hijo debe obedecer a todo lo que éstos dispongan para su bien o el de la familia. \textquote{Hijos, obedeced en todo a vuestros padres, porque esto es grato a Dios en el Señor} (\emph{Col} 3, 20; cf. \emph{Ef} 6, 1). Los niños deben obedecer también las prescripciones razonables de sus educadores y de todos aquellos a quienes sus padres los han confiado. Pero si el niño está persuadido en conciencia de que es moralmente malo obedecer esa orden, no debe seguirla.
	
	Cuando se hacen mayores, los hijos deben seguir respetando a sus padres. Deben prevenir sus deseos, solicitar dócilmente sus consejos y aceptar sus amonestaciones justificadas. La obediencia a los padres cesa con la emancipación de los hijos, pero no el respeto que les es debido, el cual permanece para siempre. Este, en efecto, tiene su raíz en el temor de Dios, uno de los dones del Espíritu Santo.
	
	\n{2218} El cuarto mandamiento recuerda a los hijos mayores de edad sus \emph{responsabilidades para con los padres}. En la medida en que ellos pueden, deben prestarles ayuda material y moral en los años de vejez y durante sus enfermedades, y en momentos de soledad o de abatimiento. Jesús recuerda este deber de gratitud (cf. \emph{Mc} 7, 10-12).
	
	\textquote{El Señor glorifica al padre en los hijos, y afirma el derecho de la madre sobre su prole. Quien honra a su padre expía sus pecados; como el que atesora es quien da gloria a su madre. Quien honra a su padre recibirá contento de sus hijos, y en el día de su oración será escuchado. Quien da gloria al padre vivirá largos días, obedece al Señor quien da sosiego a su madre} (\emph{Si} 3, 2-6).
	
	\textquote{Hijo, cuida de tu padre en su vejez, y en su vida no le causes tristeza. Aunque haya perdido la cabeza, sé indulgente, no le desprecies en la plenitud de tu vigor [\ldots{}] Como blasfemo es el que abandona a su padre, maldito del Señor quien irrita a su madre} (\emph{Si} 3, 12-13.16).
	
	\n{2219} El respeto filial favorece la armonía de toda la vida familiar; atañe también a las \emph{relaciones entre hermanos y hermanas}. El respeto a los padres irradia en todo el ambiente familiar. \textquote{Corona de los ancianos son los hijos de los hijos} (\emph{Pr} 17, 6). \textquote{{[}Soportaos{]} unos a otros en la caridad, en toda humildad, dulzura y paciencia} (\emph{Ef} 4, 2).
	
	\n{2220} Los cristianos están obligados a una especial gratitud para con aquellos de quienes recibieron el don de la fe, la gracia del bautismo y la vida en la Iglesia. Puede tratarse de los padres, de otros miembros de la familia, de los abuelos, de los pastores, de los catequistas, de otros maestros o amigos. \textquote{Evoco el recuerdo [\ldots{}] de la fe sincera que tú tienes, fe que arraigó primero en tu abuela Loida y en tu madre Eunice, y sé que también ha arraigado en ti} (\emph{2 Tm} 1, 5).
	
	\ccesec{Deberes de los padres}
	
	\n{2221} La fecundidad del amor conyugal no se reduce a la sola procreación de los hijos, sino que debe extenderse también a su educación moral y a su formación espiritual. \emph{El papel de los padres en la educación} \textquote{tiene tanto peso que, cuando falta, difícilmente puede suplirse} (GE 3). El derecho y el deber de la educación son para los padres primordiales e inalienables (cf. FC 36).
	
	\n{2222} Los padres deben mirar a sus hijos como a \emph{hijos de Dios} y respetarlos como a \emph{personas humanas}. Han de educar a sus hijos en el cumplimiento de la ley de Dios, mostrándose ellos mismos obedientes a la voluntad del Padre de los cielos.
	
	\n{2223} Los padres son los primeros responsables de la educación de sus hijos. Testimonian esta responsabilidad ante todo por la \emph{creación de un hogar}, donde la ternura, el perdón, el respeto, la fidelidad y el servicio desinteresado son norma. La familia es un lugar apropiado para la \emph{educación de las virtudes}. Esta requiere el aprendizaje de la abnegación, de un sano juicio, del dominio de sí, condiciones de toda libertad verdadera. Los padres han de enseñar a los hijos a subordinar las dimensiones \textquote{materiales e instintivas a las interiores y espirituales} (CA 36). Es una grave responsabilidad para los padres dar buenos ejemplos a sus hijos. Sabiendo reconocer ante sus hijos sus propios defectos, se hacen más aptos para guiarlos y corregirlos:
	
	\textquote{El que ama a su hijo, le corrige sin cesar [\ldots{}] el que enseña a su hijo, sacará provecho de él} (\emph{Si} 30, 1-2). \textquote{Padres, no exasperéis a vuestros hijos, sino formadlos más bien mediante la instrucción y la corrección según el Señor} (\emph{Ef} 6, 4).
	
	\n{2224} La familia constituye un medio natural para la iniciación del ser humano en la solidaridad y en las responsabilidades comunitarias. Los padres deben enseñar a los hijos a guardarse de los riesgos y las degradaciones que amenazan a las sociedades humanas.
	
	\n{2225} Por la gracia del sacramento del matrimonio, los padres han recibido la responsabilidad y el privilegio de \emph{evangelizar a sus hijos}. Desde su primera edad, deberán iniciarlos en los misterios de la fe, de los que ellos son para sus hijos los \textquote{primeros [\ldots{}] heraldos de la fe} (LG 11). Desde su más tierna infancia, deben asociarlos a la vida de la Iglesia. La forma de vida en la familia puede alimentar las disposiciones afectivas que, durante toda la vida, serán auténticos cimientos y apoyos de una fe viva.
	
	\n{2226} La \emph{educación en la fe} por los padres debe comenzar desde la más tierna infancia. Esta educación se hace ya cuando los miembros de la familia se ayudan a crecer en la fe mediante el testimonio de una vida cristiana de acuerdo con el Evangelio. La catequesis familiar precede, acompaña y enriquece las otras formas de enseñanza de la fe. Los padres tienen la misión de enseñar a sus hijos a orar y a descubrir su vocación de hijos de Dios (cf. LG 11). La parroquia es la comunidad eucarística y el corazón de la vida litúrgica de las familias cristianas; es un lugar privilegiado para la catequesis de los niños y de los padres.
	
	\n{2227} Los hijos, a su vez, contribuyen al \emph{crecimiento de sus padres en la santidad} (cf. GS 48, 4). Todos y cada uno deben otorgarse generosamente y sin cansarse el mutuo perdón exigido por las ofensas, las querellas, las injusticias y las omisiones. El afecto mutuo lo sugiere. La caridad de Cristo lo exige (cf. \emph{Mt} 18, 21-22; \emph{Lc} 17, 4).
	
	\n{2228} Durante la infancia, el respeto y el afecto de los padres se traducen ante todo en el cuidado y la atención que consagran para educar a sus hijos, y para \emph{proveer a sus necesidades físicas y espirituales}. En el transcurso del crecimiento, el mismo respeto y la misma dedicación llevan a los padres a enseñar a sus hijos a usar rectamente de su razón y de su libertad.
	
	\n{2229} Los padres, como primeros responsables de la educación de sus hijos, tienen el derecho de \emph{elegir para ellos una escuela} que corresponda a sus propias convicciones. Este derecho es fundamental. En cuanto sea posible, los padres tienen el deber de elegir las escuelas que mejor les ayuden en su tarea de educadores cristianos (cf. GE 6). Los poderes públicos tienen el deber de garantizar este derecho de los padres y de asegurar las condiciones reales de su ejercicio.
	
	\n{2230} Cuando llegan a la edad correspondiente, los hijos tienen el deber y el derecho de \emph{elegir su profesión y su estado de vida}. Estas nuevas responsabilidades deberán asumirlas en una relación de confianza con sus padres, cuyo parecer y consejo pedirán y recibirán dócilmente. Los padres deben cuidar de no presionar a sus hijos ni en la elección de una profesión ni en la de su futuro cónyuge. Esta indispensable prudencia no impide, sino al contrario, ayudar a los hijos con consejos juiciosos, particularmente cuando éstos se proponen fundar un hogar.
	
	\n{2231} Hay quienes no se casan para poder cuidar a sus padres, o sus hermanos y hermanas, para dedicarse más exclusivamente a una profesión o por otros motivos dignos. Estas personas pueden contribuir grandemente al bien de la familia humana.
	
	\n{2232} Los vínculos familiares, aunque son muy importantes, no son absolutos. A la par que el hijo crece hacia una madurez y autonomía humanas y espirituales, la vocación singular que viene de Dios se afirma con más claridad y fuerza. Los padres deben respetar esta llamada y favorecer la respuesta de sus hijos para seguirla. Es preciso convencerse de que la vocación primera del cristiano \emph{es seguir a Jesús} (cf. \emph{Mt} 16, 25): \textquote{El que ama a su padre o a su madre más que a mí, no es digno de mí; el que ama a su hijo o a su hija más que a mí, no es digno de mí} (\emph{Mt} 10, 37).
	
	\n{2233} Hacerse discípulo de Jesús es aceptar la invitación a pertenecer a la \emph{familia de Dios}, a vivir en conformidad con su manera de vivir: \textquote{El que cumpla la voluntad de mi Padre celestial, éste es mi hermano, mi hermana y mi madre} (\emph{Mt} 12, 49).
	
	Los padres deben acoger y respetar con alegría y acción de gracias el llamamiento del Señor a uno de sus hijos para que le siga en la virginidad por el Reino, en la vida consagrada o en el ministerio sacerdotal.
	
	La huida a Egipto
	
	CEC 333, 530:
	
	\n{333} De la Encarnación a la Ascensión, la vida del Verbo encarnado está rodeada de la adoración y del servicio de los ángeles. Cuando Dios introduce \textquote{a su Primogénito en el mundo, dice: \textquote{adórenle todos los ángeles de Dios}} (\emph{Hb} 1, 6). Su cántico de alabanza en el nacimiento de Cristo no ha cesado de resonar en la alabanza de la Iglesia: \textquote{Gloria a Dios\ldots{}} (\emph{Lc} 2, 14). Protegen la infancia de Jesús (cf. \emph{Mt} 1, 20; 2, 13.19), le sirven en el desierto (cf. \emph{Mc} 1, 12; \emph{Mt} 4, 11), lo reconfortan en la agonía (cf. \emph{Lc} 22, 43), cuando Él habría podido ser salvado por ellos de la mano de sus enemigos (cf. \emph{Mt} 26, 53) como en otro tiempo Israel (cf. \emph{2 M} 10, 29-30; 11,8). Son también los ángeles quienes \textquote{evangelizan} (\emph{Lc} 2, 10) anunciando la Buena Nueva de la Encarnación (cf. \emph{Lc} 2, 8-14), y de la Resurrección (cf. \emph{Mc} 16, 5-7) de Cristo. Con ocasión de la segunda venida de Cristo, anunciada por los ángeles (cf. \emph{Hb} 1, 10-11), éstos estarán presentes al servicio del juicio del Señor (cf. \emph{Mt} 13, 41; 25, 31 ; \emph{Lc} 12, 8-9).
	
	\n{530} \emph{La Huida a Egipto} y la matanza de los inocentes (cf. \emph{Mt} 2, 13-18) manifiestan la oposición de las tinieblas a la luz: \textquote{Vino a su Casa, y los suyos no lo recibieron} (\emph{Jn} 1, 11). Toda la vida de Cristo estará bajo el signo de la persecución. Los suyos la comparten con él (cf. \emph{Jn} 15, 20). Su vuelta de Egipto (cf. \emph{Mt} 2, 15) recuerda el éxodo (cf. \emph{Os} 11, 1) y presenta a Jesús como el liberador definitivo.
	
	%María Madre
	Jesucristo, verdadero Dios y verdadero hombre
	
	CEC 464-469:
	
	\n{464} El acontecimiento único y totalmente singular de la Encarnación del Hijo de Dios no significa que Jesucristo sea en parte Dios y en parte hombre, ni que sea el resultado de una mezcla confusa entre lo divino y lo humano. Él se hizo verdaderamente hombre sin dejar de ser verdaderamente Dios. Jesucristo es verdadero Dios y verdadero hombre. La Iglesia debió defender y aclarar esta verdad de fe durante los primeros siglos frente a unas herejías que la falseaban.
	
	\n{465} Las primeras herejías negaron menos la divinidad de Jesucristo que su humanidad verdadera (docetismo gnóstico). Desde la época apostólica la fe cristiana insistió en la verdadera encarnación del Hijo de Dios, \textquote{venido en la carne} (cf. \emph{1 Jn} 4, 2-3; \emph{2 Jn} 7). Pero desde el siglo III, la Iglesia tuvo que afirmar frente a Pablo de Samosata, en un Concilio reunido en Antioquía, que Jesucristo es Hijo de Dios por naturaleza y no por adopción. El primer Concilio Ecuménico de Nicea, en el año 325, confesó en su Credo que el Hijo de Dios es \textquote{engendrado, no creado, \textquote{de la misma substancia} {[}en griego \emph{homousion}{]} que el Padre} y condenó a Arrio que afirmaba que \textquote{el Hijo de Dios salió de la nada} (Concilio de Nicea I: DS 130) y que sería \textquote{de una substancia distinta de la del Padre} (\emph{Ibíd}., 126).
	
	\n{466} La herejía nestoriana veía en Cristo una persona humana junto a la persona divina del Hijo de Dios. Frente a ella san Cirilo de Alejandría y el tercer Concilio Ecuménico reunido en Efeso, en el año 431, confesaron que \textquote{el Verbo, al unirse en su persona a una carne animada por un alma racional, se hizo hombre} (Concilio de Efeso: DS, 250). La humanidad de Cristo no tiene más sujeto que la persona divina del Hijo de Dios que la ha asumido y hecho suya desde su concepción. Por eso el concilio de Efeso proclamó en el año 431 que María llegó a ser con toda verdad Madre de Dios mediante la concepción humana del Hijo de Dios en su seno: \textquote{Madre de Dios, no porque el Verbo de Dios haya tomado de ella su naturaleza divina, sino porque es de ella, de quien tiene el cuerpo sagrado dotado de un alma racional [\ldots{}] unido a la persona del Verbo, de quien se dice que el Verbo nació según la carne} (DS 251).
	
	\n{467} Los monofisitas afirmaban que la naturaleza humana había dejado de existir como tal en Cristo al ser asumida por su persona divina de Hijo de Dios. Enfrentado a esta herejía, el cuarto Concilio Ecuménico, en Calcedonia, confesó en el año 451:
	
	\textquote{Siguiendo, pues, a los Santos Padres, enseñamos unánimemente que hay que confesar a un solo y mismo Hijo y Señor nuestro Jesucristo: perfecto en la divinidad, y perfecto en la humanidad; verdaderamente Dios y verdaderamente hombre compuesto de alma racional y cuerpo; consubstancial con el Padre según la divinidad, y consubstancial con nosotros según la humanidad, \textquote{en todo semejante a nosotros, excepto en el pecado} (\emph{Hb} 4, 15); nacido del Padre antes de todos los siglos según la divinidad; y por nosotros y por nuestra salvación, nacido en los últimos tiempos de la Virgen María, la Madre de Dios, según la humanidad.
		
		Se ha de reconocer a un solo y mismo Cristo Señor, Hijo único en dos naturalezas, sin confusión, sin cambio, sin división, sin separación. La diferencia de naturalezas de ningún modo queda suprimida por su unión, sino que quedan a salvo las propiedades de cada una de las naturalezas y confluyen en un solo sujeto y en una sola persona} (Concilio de Calcedonia; DS, 301-302).
	
	\n{468} Después del Concilio de Calcedonia, algunos concibieron la naturaleza humana de Cristo como una especie de sujeto personal. Contra éstos, el quinto Concilio Ecuménico, en Constantinopla, el año 553 confesó a propósito de Cristo: \textquote{No hay más que una sola hipóstasis {[}o persona{]} [\ldots{}] que es nuestro Señor Jesucristo, \emph{uno de la Trinidad}} (Concilio de Constantinopla II: DS, 424). Por tanto, todo en la humanidad de Jesucristo debe ser atribuido a su persona divina como a su propio sujeto (cf. ya Concilio de Éfeso: DS, 255), no solamente los milagros sino también los sufrimientos (cf. Concilio de Constantinopla II: DS, 424) y la misma muerte: \textquote{El que ha sido crucificado en la carne, nuestro Señor Jesucristo, es verdadero Dios, Señor de la gloria y uno de la Santísima Trinidad} (\emph{ibíd}., 432).
	
	\n{469} La Iglesia confiesa así que Jesús es inseparablemente verdadero Dios y verdadero Hombre. Él es verdaderamente el Hijo de Dios que se ha hecho hombre, nuestro hermano, y eso sin dejar de ser Dios, nuestro Señor:
	
	\emph{Id quod fuit remansit et quod non fuit assumpsit} (\textquote{Sin dejar de ser lo que era ha asumido lo que no era}), canta la liturgia romana (\emph{Solemnidad de la Santísima Virgen María, Madre de Dios}, Antífona al \textquote{Benedictus}; cf. san León Magno, \emph{Sermones} 21, 2-3: PL 54, 192). Y la liturgia de san Juan Crisóstomo proclama y canta: \textquote{¡Oh Hijo unigénito y Verbo de Dios! Tú que eres inmortal, te dignaste, para salvarnos, tomar carne de la santa Madre de Dios y siempre Virgen María. Tú, Cristo Dios, sin sufrir cambio te hiciste hombre y, en la cruz, con tu muerte venciste la muerte. Tú, Uno de la Santísima Trinidad, glorificado con el Padre y el Santo Espíritu, ¡sálvanos! (\emph{Oficio Bizantino de las Horas, Himno O' Monogenés}}).
	
	María es la Madre de Dios
	
	CEC 495, 2677:
	
	\ccesec{La maternidad divina de María}
	
	\n{495} Llamada en los Evangelios \textquote{la Madre de Jesús} (\emph{Jn} 2, 1; 19, 25; cf. \emph{Mt} 13, 55, etc.), María es aclamada bajo el impulso del Espíritu como \textquote{la madre de mi Señor} desde antes del nacimiento de su hijo (cf. \emph{Lc} 1, 43). En efecto, aquél que ella concibió como hombre, por obra del Espíritu Santo, y que se ha hecho verdaderamente su Hijo según la carne, no es otro que el Hijo eterno del Padre, la segunda persona de la Santísima Trinidad. La Iglesia confiesa que María es verdaderamente \emph{Madre de Dios} {[}\emph{Theotokos}{]} (cf. Concilio de Éfeso, año 649: DS, 251).
	
	\n{2677} \emph{\textquote{Santa María, Madre de Dios, ruega por nosotros\ldots{} }} Con Isabel, nos maravillamos y decimos: \textquote{¿De dónde a mí que la madre de mi Señor venga a mí?} (\emph{Lc} 1, 43). Porque nos da a Jesús su hijo, María es madre de Dios y madre nuestra; podemos confiarle todos nuestros cuidados y nuestras peticiones: ora por nosotros como oró por sí misma: \textquote{Hágase en mí según tu palabra} (\emph{Lc} 1, 38). Confiándonos a su oración, nos abandonamos con ella en la voluntad de Dios: \textquote{Hágase tu voluntad}.
	
	\emph{\textquote{Ruega por nosotros, pecadores, ahora y en la hora de nuestra muerte}}. Pidiendo a María que ruegue por nosotros, nos reconocemos pecadores y nos dirigimos a la \textquote{Madre de la Misericordia}, a la Toda Santa. Nos ponemos en sus manos \textquote{ahora}, en el hoy de nuestras vidas. Y nuestra confianza se ensancha para entregarle desde ahora, \textquote{la hora de nuestra muerte}. Que esté presente en esa hora, como estuvo en la muerte en Cruz de su Hijo, y que en la hora de nuestro tránsito nos acoja como madre nuestra (cf. \emph{Jn} 19, 27) para conducirnos a su Hijo Jesús, al Paraíso.
	
	Nuestra adopción como hijos de Dios
	
	CEC 1, 52, 270, 294, 422, 654, 1709, 2009:
	
	\n{1} Dios, infinitamente perfecto y bienaventurado en sí mismo, en un designio de pura bondad ha creado libremente al hombre para hacerle partícipe de su vida bienaventurada. Por eso, en todo tiempo y en todo lugar, se hace cercano del hombre: le llama y le ayuda a buscarle, a conocerle y a amarle con todas sus fuerzas. Convoca a todos los hombres, que el pecado dispersó, a la unidad de su familia, la Iglesia. Para lograrlo, llegada la plenitud de los tiempos, envió a su Hijo como Redentor y Salvador. En Él y por Él, llama a los hombres a ser, en el Espíritu Santo, sus hijos de adopción, y por tanto los herederos de su vida bienaventurada.
	
	\n{52} Dios, que \textquote{habita una luz inaccesible} (\emph{1 Tm} 6,16) quiere comunicar su propia vida divina a los hombres libremente creados por él, para hacer de ellos, en su Hijo único, hijos adoptivos (cf. \emph{Ef} 1,4-5). Al revelarse a sí mismo, Dios quiere hacer a los hombres capaces de responderle, de conocerle y de amarle más allá de lo que ellos serían capaces por sus propias fuerzas.
	
	\ccesec{\textquote{Te compadeces de todos porque lo puedes todo} (\emph{Sb} 11, 23)}
	
	\n{270} Dios es el \emph{Padre} todopoderoso. Su paternidad y su poder se esclarecen mutuamente. Muestra, en efecto, su omnipotencia paternal por la manera como cuida de nuestras necesidades (cf. \emph{Mt} 6,32); por la adopción filial que nos da (\textquote{Yo seré para vosotros padre, y vosotros seréis para mí hijos e hijas, dice el Señor todopoderoso}: \emph{2 Co} 6,18); finalmente, por su misericordia infinita, pues muestra su poder en el más alto grado perdonando libremente los pecados.
	
	\n{294} La gloria de Dios consiste en que se realice esta manifestación y esta comunicación de su bondad para las cuales el mundo ha sido creado. Hacer de nosotros \textquote{hijos adoptivos por medio de Jesucristo, según el beneplácito de su voluntad, \emph{para alabanza de la gloria} de su gracia} (\emph{Ef} 1,5-6): \textquote{Porque la gloria de Dios es que el hombre viva, y la vida del hombre es la visión de Dios: si ya la revelación de Dios por la creación procuró la vida a todos los seres que viven en la tierra, cuánto más la manifestación del Padre por el Verbo procurará la vida a los que ven a Dios} (San Ireneo de Lyon, \emph{Adversus haereses}, 4,20,7). El fin último de la creación es que Dios, \textquote{Creador de todos los seres, sea por fin \textquote{todo en todas las cosas} (\emph{1 Co} 15,28), \emph{procurando al mismo tiempo su gloria y nuestra felicidad}} (AG 2).
	
	\ccesec{La Buena Nueva: Dios ha enviado a su Hijo}
	
	\n{422} \textquote{Pero, al llegar la plenitud de los tiempos, envió Dios a su Hijo, nacido de mujer, nacido bajo la Ley, para rescatar a los que se hallaban bajo la Ley, y para que recibiéramos la filiación adoptiva} (\emph{Ga} 4, 4-5). He aquí \textquote{la Buena Nueva de Jesucristo, Hijo de Dios} (\emph{Mc} 1, 1): Dios ha visitado a su pueblo (cf. \emph{Lc} 1, 68), ha cumplido las promesas hechas a Abraham y a su descendencia (cf. \emph{Lc} 1, 55); lo ha hecho más allá de toda expectativa: Él ha enviado a su \textquote{Hijo amado} (\emph{Mc} 1, 11).
	
	\n{654} Hay un doble aspecto en el misterio pascual: por su muerte nos libera del pecado, por su Resurrección nos abre el acceso a una nueva vida. Esta es, en primer lugar, la \emph{justificación} que nos devuelve a la gracia de Dios (cf. \emph{Rm} 4, 25) \textquote{a fin de que, al igual que Cristo fue resucitado de entre los muertos [\ldots{}] así también nosotros vivamos una nueva vida} (\emph{Rm} 6, 4). Consiste en la victoria sobre la muerte y el pecado y en la nueva participación en la gracia (cf. \emph{Ef} 2, 4-5; \emph{1 P} 1, 3). Realiza la \emph{adopción filial} porque los hombres se convierten en hermanos de Cristo, como Jesús mismo llama a sus discípulos después de su Resurrección: \textquote{Id, avisad a mis hermanos} (\emph{Mt} 28, 10; \emph{Jn} 20, 17). Hermanos no por naturaleza, sino por don de la gracia, porque esta filiación adoptiva confiere una participación real en la vida del Hijo único, la que ha revelado plenamente en su Resurrección.
	
	\n{1709} El que cree en Cristo es hecho hijo de Dios. Esta adopción filial lo transforma dándole la posibilidad de seguir el ejemplo de Cristo. Le hace capaz de obrar rectamente y de practicar el bien. En la unión con su Salvador, el discípulo alcanza la perfección de la caridad, la santidad. La vida moral, madurada en la gracia, culmina en vida eterna, en la gloria del cielo.
	
	\n{2009} La adopción filial, haciéndonos partícipes por la gracia de la naturaleza divina, puede conferirnos, según la justicia gratuita de Dios, \emph{un verdadero mérito}. Se trata de un derecho por gracia, el pleno derecho del amor, que nos hace \textquote{coherederos} de Cristo y dignos de obtener la herencia prometida de la vida eterna (cf. Concilio de Trento: DS 1546). Los méritos de nuestras buenas obras son dones de la bondad divina (cf. Concilio de Trento: DS 1548). \textquote{La gracia ha precedido; ahora se da lo que es debido [\ldots{}] Los méritos son dones de Dios} (San Agustín, \emph{Sermo} 298, 4-5).
	
	Jesús observa la Ley y la perfecciona
	
	CEC 527, 577-582:
	
	\n{527} La \emph{Circuncisión} de Jesús, al octavo día de su nacimiento (cf. \emph{Lc} 2, 21) es señal de su inserción en la descendencia de Abraham, en el pueblo de la Alianza, de su sometimiento a la Ley (cf. \emph{Ga} 4, 4) y de su consagración al culto de Israel en el que participará durante toda su vida. Este signo prefigura \textquote{la circuncisión en Cristo} que es el Bautismo (\emph{Col} 2, 11-13).
	
	\ccesec{Jesús y la Ley}
	
	\n{577} Al comienzo del Sermón de la Montaña, Jesús hace una advertencia solemne presentando la Ley dada por Dios en el Sinaí con ocasión de la Primera Alianza, a la luz de la gracia de la Nueva Alianza:
	
	\textquote{No penséis que he venido a abolir la Ley y los Profetas. No he venido a abolir sino a dar cumplimiento. Sí, os lo aseguro: el cielo y la tierra pasarán antes que pase una \textquote{i} o un ápice de la Ley sin que todo se haya cumplido. Por tanto, el que quebrante uno de estos mandamientos menores, y así lo enseñe a los hombres, será el menor en el Reino de los cielos; en cambio el que los observe y los enseñe, ése será grande en el Reino de los cielos} (\emph{Mt} 5, 17-19).
	
	\n{578} Jesús, el Mesías de Israel, por lo tanto el más grande en el Reino de los cielos, se debía sujetar a la Ley cumpliéndola en su totalidad hasta en sus menores preceptos, según sus propias palabras. Incluso es el único en poderlo hacer perfectamente (cf. \emph{Jn} 8, 46). Los judíos, según su propia confesión, jamás han podido cumplir la Ley en su totalidad, sin violar el menor de sus preceptos (cf. \emph{Jn} 7, 19; \emph{Hch} 13, 38-41; 15, 10). Por eso, en cada fiesta anual de la Expiación, los hijos de Israel piden perdón a Dios por sus transgresiones de la Ley. En efecto, la Ley constituye un todo y, como recuerda Santiago, \textquote{quien observa toda la Ley, pero falta en un solo precepto, se hace reo de todos} (\emph{St} 2, 10; cf. \emph{Ga} 3, 10; 5, 3).
	
	\n{579} Este principio de integridad en la observancia de la Ley, no sólo en su letra sino también en su espíritu, era apreciado por los fariseos. Al subrayarlo para Israel, muchos judíos del tiempo de Jesús fueron conducidos a un celo religioso extremo (cf. \emph{Rm} 10, 2), el cual, si no quería convertirse en una casuística \textquote{hipócrita} (cf. \emph{Mt} 15, 3-7; \emph{Lc} 11, 39-54) no podía más que preparar al pueblo a esta intervención inaudita de Dios que será la ejecución perfecta de la Ley por el único Justo en lugar de todos los pecadores (cf. \emph{Is} 53, 11; \emph{Hb} 9, 15).
	
	\n{580} El cumplimiento perfecto de la Ley no podía ser sino obra del divino Legislador que nació sometido a la Ley en la persona del Hijo (cf. \emph{Ga} 4, 4). En Jesús la Ley ya no aparece grabada en tablas de piedra sino \textquote{en el fondo del corazón} (\emph{Jr} 31, 33) del Siervo, quien, por \textquote{aportar fielmente el derecho} (\emph{Is} 42, 3), se ha convertido en \textquote{la Alianza del pueblo} (\emph{Is} 42, 6). Jesús cumplió la Ley hasta tomar sobre sí mismo \textquote{la maldición de la Ley} (\emph{Ga} 3, 13) en la que habían incurrido los que no \textquote{practican todos los preceptos de la Ley} (\emph{Ga} 3, 10) porque \textquote{ha intervenido su muerte para remisión de las transgresiones de la Primera Alianza} (\emph{Hb} 9, 15).
	
	\n{581} Jesús fue considerado por los judíos y sus jefes espirituales como un \textquote{rabbi} (cf. \emph{Jn} 11, 28; 3, 2; \emph{Mt} 22, 23-24, 34-36). Con frecuencia argumentó en el marco de la interpretación rabínica de la Ley (cf. \emph{Mt} 12, 5; 9, 12; \emph{Mc} 2, 23-27; \emph{Lc} 6, 6-9; \emph{Jn} 7, 22-23). Pero al mismo tiempo, Jesús no podía menos que chocar con los doctores de la Ley porque no se contentaba con proponer su interpretación entre los suyos, sino que \textquote{enseñaba como quien tiene autoridad y no como los escribas} (\emph{Mt} 7, 28-29). La misma Palabra de Dios, que resonó en el Sinaí para dar a Moisés la Ley escrita, es la que en Él se hace oír de nuevo en el Monte de las Bienaventuranzas (cf. \emph{Mt} 5, 1). Esa palabra no revoca la Ley sino que la perfecciona aportando de modo divino su interpretación definitiva: \textquote{Habéis oído también que se dijo a los antepasados [\ldots{}] pero yo os digo} (\emph{Mt} 5, 33-34). Con esta misma autoridad divina, desaprueba ciertas \textquote{tradiciones humanas} (\emph{Mc} 7, 8) de los fariseos que \textquote{anulan la Palabra de Dios} (\emph{Mc} 7, 13).
	
	\n{582} Yendo más lejos, Jesús da plenitud a la Ley sobre la pureza de los alimentos, tan importante en la vida cotidiana judía, manifestando su sentido \textquote{pedagógico} (cf. \emph{Ga} 3, 24) por medio de una interpretación divina: \textquote{Todo lo que de fuera entra en el hombre no puede hacerle impuro [\ldots{}] ---así declaraba puros todos los alimentos---. Lo que sale del hombre, eso es lo que hace impuro al hombre. Porque de dentro, del corazón de los hombres, salen las intenciones malas} (\emph{Mc} 7, 18-21). Jesús, al dar con autoridad divina la interpretación definitiva de la Ley, se vio enfrentado a algunos doctores de la Ley que no aceptaban su interpretación a pesar de estar garantizada por los signos divinos con que la acompañaba (cf. \emph{Jn} 5, 36; 10, 25. 37-38; 12, 37). Esto ocurre, en particular, respecto al problema del sábado: Jesús recuerda, frecuentemente con argumentos rabínicos (cf. \emph{Mt} 2,25-27; \emph{Jn} 7, 22-24), que el descanso del sábado no se quebranta por el servicio de Dios (cf. \emph{Mt} 12, 5; \emph{Nm} 28, 9) o al prójimo (cf. \emph{Lc} 13, 15-16; 14, 3-4) que realizan sus curaciones.
	
	la Ley nueva nos libera de las restricciones de la Ley antigua
	
	CEC 580, 1972:
	
	\n{580} El cumplimiento perfecto de la Ley no podía ser sino obra del divino Legislador que nació sometido a la Ley en la persona del Hijo (cf. \emph{Ga} 4, 4). En Jesús la Ley ya no aparece grabada en tablas de piedra sino \textquote{en el fondo del corazón} (\emph{Jr} 31, 33) del Siervo, quien, por \textquote{aportar fielmente el derecho} (\emph{Is} 42, 3), se ha convertido en \textquote{la Alianza del pueblo} (\emph{Is} 42, 6). Jesús cumplió la Ley hasta tomar sobre sí mismo \textquote{la maldición de la Ley} (\emph{Ga} 3, 13) en la que habían incurrido los que no \textquote{practican todos los preceptos de la Ley} (\emph{Ga} 3, 10) porque \textquote{ha intervenido su muerte para remisión de las transgresiones de la Primera Alianza} (\emph{Hb} 9, 15).
	
	\n{1972} La Ley nueva es llamada \emph{ley de amor}, porque hace obrar por el amor que infunde el Espíritu Santo más que por el temor; \emph{ley de gracia}, porque confiere la fuerza de la gracia para obrar mediante la fe y los sacramentos; \emph{ley de libertad} (cf. \emph{St} 1, 25; 2, 12), porque nos libera de las observancias rituales y jurídicas de la Ley antigua, nos inclina a obrar espontáneamente bajo el impulso de la caridad y nos hace pasar de la condición del siervo \textquote{que ignora lo que hace su señor}, a la de amigo de Cristo, \textquote{porque todo lo que he oído a mi Padre os lo he dado a conocer} (\emph{Jn} 15, 15), o también a la condición de hijo heredero (cf. \emph{Ga} 4, 1-7. 21-31; \emph{Rm} 8, 15).
	
	Por medio del Espíritu Santo podemos llamar a Dios \textquote{Abba}
	
	CEC 683, 689, 1695, 2766, 2777-2778:
	
	\n{683} \textquote{Nadie puede decir: \textquote{¡Jesús es Señor!} sino por influjo del Espíritu Santo} (\emph{1 Co} 12, 3). \textquote{Dios ha enviado a nuestros corazones el Espíritu de su Hijo que clama ¡\emph{Abbá}, Padre!} (\emph{Ga} 4, 6). Este conocimiento de fe no es posible sino en el Espíritu Santo. Para entrar en contacto con Cristo, es necesario primeramente haber sido atraído por el Espíritu Santo. Él es quien nos precede y despierta en nosotros la fe. Mediante el Bautismo, primer sacramento de la fe, la vida, que tiene su fuente en el Padre y se nos ofrece por el Hijo, se nos comunica íntima y personalmente por el Espíritu Santo en la Iglesia:
	
	El Bautismo \textquote{nos da la gracia del nuevo nacimiento en Dios Padre por medio de su Hijo en el Espíritu Santo. Porque los que son portadores del Espíritu de Dios son conducidos al Verbo, es decir al Hijo; pero el Hijo los presenta al Padre, y el Padre les concede la incorruptibilidad. Por tanto, sin el Espíritu no es posible ver al Hijo de Dios, y, sin el Hijo, nadie puede acercarse al Padre, porque el conocimiento del Padre es el Hijo, y el conocimiento del Hijo de Dios se logra por el Espíritu Santo} (San Ireneo de Lyon, \emph{Demonstratio praedicationis apostolicae}, 7: SC 62 41-42).
	
	\ccesec{La misión conjunta del Hijo y del Espíritu Santo}
	
	\n{689} Aquel al que el Padre ha enviado a nuestros corazones, el Espíritu de su Hijo (cf. \emph{Ga} 4, 6) es realmente Dios. Consubstancial con el Padre y el Hijo, es inseparable de ellos, tanto en la vida íntima de la Trinidad como en su don de amor para el mundo. Pero al adorar a la Santísima Trinidad vivificante, consubstancial e indivisible, la fe de la Iglesia profesa también la distinción de las Personas. Cuando el Padre envía su Verbo, envía también su Aliento: misión conjunta en la que el Hijo y el Espíritu Santo son distintos pero inseparables. Sin ninguna duda, Cristo es quien se manifiesta, Imagen visible de Dios invisible, pero es el Espíritu Santo quien lo revela.
	
	\n{1695} \textquote{Justificados [\ldots{}] en el nombre del Señor Jesucristo y en el Espíritu de nuestro Dios} (\emph{1 Co} 6,11.), \textquote{santificados y llamados a ser santos} (\emph{1 Co} 1,2.), los cristianos se convierten en \textquote{el templo [\ldots{}] del Espíritu Santo} (cf. \emph{1 Co} 6,19). Este \textquote{Espíritu del Hijo} les enseña a orar al Padre (\emph{Ga} 4, 6) y, haciéndose vida en ellos, les hace obrar (cf. \emph{Ga} 5, 25) para dar \textquote{los frutos del Espíritu} (\emph{Ga} 5, 22.) por la caridad operante. Sanando las heridas del pecado, el Espíritu Santo nos renueva interiormente mediante una transformación espiritual (cf. \emph{Ef} 4, 23.), nos ilumina y nos fortalece para vivir como \textquote{hijos de la luz} (\emph{Ef} 5, 8.), \textquote{por la bondad, la justicia y la verdad} en todo (\emph{Ef} 5,9.).
	
	\n{2766} Pero Jesús no nos deja una fórmula para repetirla de modo mecánico (cf. \emph{Mt} 6, 7; \emph{1 R} 18, 26-29). Como en toda oración vocal, el Espíritu Santo, a través de la Palabra de Dios, enseña a los hijos de Dios a hablar con su Padre. Jesús no sólo nos enseña las palabras de la oración filial, sino que nos da también el Espíritu por el que estas se hacen en nosotros \textquote{espíritu [\ldots{}] y vida} (\emph{Jn} 6, 63). Más todavía: la prueba y la posibilidad de nuestra oración filial es que el Padre \textquote{ha enviado [\ldots{}] a nuestros corazones el Espíritu de su Hijo que clama: \textquote{¡Abbá, Padre!}} (\emph{Ga} 4, 6). Ya que nuestra oración interpreta nuestros deseos ante Dios, es también \textquote{el que escruta los corazones}, el Padre, quien \textquote{conoce cuál es la aspiración del Espíritu, y que su intercesión en favor de los santos es según Dios} (\emph{Rm} 8, 27). La oración al Padre se inserta en la misión misteriosa del Hijo y del Espíritu.
	
	\ccesec{Acercarse a Él con toda confianza}
	
	\n{2777} En la liturgia romana, se invita a la asamblea eucarística a rezar el Padre Nuestro con una audacia filial; las liturgias orientales usan y desarrollan expresiones análogas: \textquote{Atrevernos con toda confianza}, \textquote{Haznos dignos de}. Ante la zarza ardiendo, se le dijo a Moisés: \textquote{No te acerques aquí. Quita las sandalias de tus pies} (\emph{Ex} 3, 5). Este umbral de la santidad divina, sólo lo podía franquear Jesús, el que \textquote{después de llevar a cabo la purificación de los pecados} (\emph{Hb} 1, 3), nos introduce en presencia del Padre: \textquote{Hénos aquí, a mí y a los hijos que Dios me dio} (\emph{Hb} 2, 13):
	
	\textquote{La conciencia que tenemos de nuestra condición de esclavos nos haría meternos bajo tierra, nuestra condición terrena se desharía en polvo, si la autoridad de nuestro mismo Padre y el Espíritu de su Hijo, no nos empujasen a proferir este grito: \textquote{Abbá, Padre} (\emph{Rm} 8, 15) \ldots{} ¿Cuándo la debilidad de un mortal se atrevería a llamar a Dios Padre suyo, sino solamente cuando lo íntimo del hombre está animado por el Poder de lo alto?} (San Pedro Crisólogo, \emph{Sermón} 71, 3).
	
	\n{2778} Este poder del Espíritu que nos introduce en la Oración del Señor se expresa en las liturgias de Oriente y de Occidente con la bella palabra, típicamente cristiana: \emph{parrhesia}, simplicidad sin desviación, conciencia filial, seguridad alegre, audacia humilde, certeza de ser amado (cf. \emph{Ef} 3, 12; \emph{Hb} 3, 6; 4, 16; 10, 19; \emph{1 Jn} 2,28; 3, 21; 5, 14).
	
	El nombre de Jesús
	
	CEC 430-435, 2666-2668, 2812:
	
	\n{430} \emph{Jesús} quiere decir en hebreo: \textquote{Dios salva}. En el momento de la anunciación, el ángel Gabriel le dio como nombre propio el nombre de Jesús que expresa a la vez su identidad y su misión (cf. \emph{Lc} 1, 31). Ya que \textquote{¿quién puede perdonar pecados, sino sólo Dios?} (\emph{Mc} 2, 7), es Él quien, en Jesús, su Hijo eterno hecho hombre \textquote{salvará a su pueblo de sus pecados} (\emph{Mt} 1, 21). En Jesús, Dios recapitula así toda la historia de la salvación en favor de los hombres.
	
	\n{431} En la historia de la salvación, Dios no se ha contentado con librar a Israel de \textquote{la casa de servidumbre} (\emph{Dt} 5, 6) haciéndole salir de Egipto. Él lo salva además de su pecado. Puesto que el pecado es siempre una ofensa hecha a Dios (cf. \emph{Sal} 51, 6), sólo Él es quien puede absolverlo (cf. \emph{Sal} 51, 12). Por eso es por lo que Israel, tomando cada vez más conciencia de la universalidad del pecado, ya no podrá buscar la salvación más que en la invocación del nombre de Dios Redentor (cf. \emph{Sal} 79, 9).
	
	\n{432} El nombre de Jesús significa que el Nombre mismo de Dios está presente en la Persona de su Hijo (cf. \emph{Hch} 5, 41; \emph{3 Jn} 7) hecho hombre para la Redención universal y definitiva de los pecados. Él es el Nombre divino, el único que trae la salvación (cf. \emph{Jn} 3, 18; \emph{Hch} 2, 21) y de ahora en adelante puede ser invocado por todos porque se ha unido a todos los hombres por la Encarnación (cf. \emph{Rm} 10, 6-13) de tal forma que \textquote{no hay bajo el cielo otro nombre dado a los hombres por el que nosotros debamos salvarnos} (\emph{Hch} 4, 12; cf. \emph{Hch} 9, 14; \emph{St} 2, 7).
	
	\n{433} El Nombre de Dios Salvador era invocado una sola vez al año por el sumo sacerdote para la expiación de los pecados de Israel, cuando había asperjado el propiciatorio del Santo de los Santos con la sangre del sacrificio (cf. \emph{Lv} 16, 15-16; \emph{Si} 50, 20; \emph{Hb} 9, 7). El propiciatorio era el lugar de la presencia de Dios (cf. \emph{Ex} 25, 22; \emph{Lv} 16, 2; \emph{Nm} 7, 89; \emph{Hb} 9, 5). Cuando san Pablo dice de Jesús que \textquote{Dios lo exhibió como instrumento de propiciación por su propia sangre} (\emph{Rm} 3, 25) significa que en su humanidad \textquote{estaba Dios reconciliando al mundo consigo} (\emph{2 Co} 5, 19).
	
	\n{434} La Resurrección de Jesús glorifica el Nombre de Dios \textquote{Salvador} (cf. \emph{Jn} 12, 28) porque de ahora en adelante, el Nombre de Jesús es el que manifiesta en plenitud el poder soberano del \textquote{Nombre que está sobre todo nombre} (\emph{Flp} 2, 9). Los espíritus malignos temen su Nombre (cf. \emph{Hch} 16, 16-18; 19, 13-16) y en su nombre los discípulos de Jesús hacen milagros (cf. \emph{Mc} 16, 17) porque todo lo que piden al Padre en su Nombre, Él se lo concede (\emph{Jn} 15, 16).
	
	\n{435} El Nombre de Jesús está en el corazón de la plegaria cristiana. Todas las oraciones litúrgicas se acaban con la fórmula \emph{Per Dominum nostrum Jesum Christum\ldots{}} (\textquote{Por nuestro Señor Jesucristo\ldots{}}). El \textquote{Avemaría} culmina en \textquote{y bendito es el fruto de tu vientre, Jesús}. La oración del corazón, en uso en Oriente, llamada \textquote{oración a Jesús} dice: \textquote{Señor Jesucristo, Hijo de Dios, ten piedad de mí pecador}. Numerosos cristianos mueren, como santa Juana de Arco, teniendo en sus labios una única palabra: \textquote{Jesús}.
	
	\n{2666} Pero el Nombre que todo lo contiene es aquel que el Hijo de Dios recibe en su encarnación: JESÚS. El nombre divino es inefable para los labios humanos (cf. \emph{Ex} 3, 14; 33, 19-23), pero el Verbo de Dios, al asumir nuestra humanidad, nos lo entrega y nosotros podemos invocarlo: \textquote{Jesús}, \textquote{YHVH salva} (cf. \emph{Mt} 1, 21). El Nombre de Jesús contiene todo: Dios y el hombre y toda la Economía de la creación y de la salvación. Decir \textquote{Jesús} es invocarlo desde nuestro propio corazón. Su Nombre es el único que contiene la presencia que significa. Jesús es el resucitado, y cualquiera que invoque su Nombre acoge al Hijo de Dios que le amó y se entregó por él (cf. \emph{Rm} 10, 13; \emph{Hch} 2, 21; 3, 15-16; \emph{Ga} 2, 20).
	
	\n{2667} Esta invocación de fe bien sencilla ha sido desarrollada en la tradición de la oración bajo formas diversas en Oriente y en Occidente. La formulación más habitual, transmitida por los espirituales del Sinaí, de Siria y del Monte Athos es la invocación: \textquote{Señor Jesucristo, Hijo de Dios, ten piedad de nosotros, pecadores} Conjuga el himno cristológico de \emph{Flp} 2, 6-11 con la petición del publicano y del mendigo ciego (cf. \emph{Lc} 18,13; \emph{Mc} 10, 46-52). Mediante ella, el corazón está acorde con la miseria de los hombres y con la misericordia de su Salvador.
	
	\n{2668} La invocación del santo Nombre de Jesús es el camino más sencillo de la oración continua. Repetida con frecuencia por un corazón humildemente atento, no se dispersa en \textquote{palabrerías} (\emph{Mt} 6, 7), sino que \textquote{conserva la Palabra y fructifica con perseverancia} (cf. \emph{Lc} 8, 15). Es posible \textquote{en todo tiempo} porque no es una ocupación al lado de otra, sino la única ocupación, la de amar a Dios, que anima y transfigura toda acción en Cristo Jesús.
	
	\n{2812} Finalmente, el Nombre de Dios Santo se nos ha revelado y dado, en la carne, en Jesús, como Salvador (cf. \emph{Mt} 1, 21; \emph{Lc} 1, 31): revelado por lo que Él es, por su Palabra y por su Sacrificio (cf. \emph{Jn} 8, 28; 17, 8; 17, 17-19). Esto es el núcleo de su oración sacerdotal: \textquote{Padre santo \ldots{} por ellos me consagro a mí mismo, para que ellos también sean consagrados en la verdad} (\emph{Jn} 17, 19). Jesús nos \textquote{manifiesta} el Nombre del Padre (\emph{Jn} 17, 6) porque \textquote{santifica} Él mismo su Nombre (cf. \emph{Ez} 20, 39; 36, 20-21). Al terminar su Pascua, el Padre le da el Nombre que está sobre todo nombre: Jesús es Señor para gloria de Dios Padre (cf. \emph{Flp} 2, 9-11).
	
	%II NAV
	Prólogo del Evangelio de Juan
	
	CEC 151, 241, 291, 423, 445, 456-463, 504-505, 526, 1216, 2466, 2787:
	
	\ccesec{Creer en Jesucristo, el Hijo de Dios}
	
	\n{151} Para el cristiano, creer en Dios es inseparablemente creer en Aquel que él ha enviado, \textquote{su Hijo amado}, en quien ha puesto toda su complacencia (\emph{Mc} 1,11). Dios nos ha dicho que le escuchemos (cf. \emph{Mc} 9,7). El Señor mismo dice a sus discípulos: \textquote{Creed en Dios, creed también en mí} (\emph{Jn} 14,1). Podemos creer en Jesucristo porque es Dios, el Verbo hecho carne: \textquote{A Dios nadie le ha visto jamás: el Hijo único, que está en el seno del Padre, él lo ha contado} (\emph{Jn} 1,18). Porque \textquote{ha visto al Padre} (\emph{Jn} 6,46), él es único en conocerlo y en poderlo revelar (cf. \emph{Mt} 11,27).
	
	\n{241} Por eso los Apóstoles confiesan a Jesús como \textquote{el Verbo que en el principio estaba junto a Dios y que era Dios} (\emph{Jn} 1,1), como \textquote{la imagen del Dios invisible} (\emph{Col} 1,15), como \textquote{el resplandor de su gloria y la impronta de su esencia} (\emph{Hb} 1,3).
	
	\n{291} \textquote{En el principio existía el Verbo [\ldots{}] y el Verbo era Dios [\ldots{}] Todo fue hecho por él y sin él nada ha sido hecho} (\emph{Jn} 1,1-3). El Nuevo Testamento revela que Dios creó todo por el Verbo Eterno, su Hijo amado. \textquote{En él fueron creadas todas las cosas, en los cielos y en la tierra [\ldots{}] todo fue creado por él y para él, él existe con anterioridad a todo y todo tiene en él su consistencia} (\emph{Col} 1, 16-17). La fe de la Iglesia afirma también la acción creadora del Espíritu Santo: él es el \textquote{dador de vida} (\emph{Símbolo Niceno-Constantinopolitano}), \textquote{el Espíritu Creador} (\emph{Liturgia de las Horas}, Himno \emph{Veni, Creator Spiritus}), la \textquote{Fuente de todo bien} (\emph{Liturgia bizantina}, Tropario de vísperas de Pentecostés).
	
	\n{423} Nosotros creemos y confesamos que Jesús de Nazaret, nacido judío de una hija de Israel, en Belén en el tiempo del rey Herodes el Grande y del emperador César Augusto I; de oficio carpintero, muerto crucificado en Jerusalén, bajo el procurador Poncio Pilato, durante el reinado del emperador Tiberio, es el Hijo eterno de Dios hecho hombre, que ha \textquote{salido de Dios} (\emph{Jn} 13, 3), \textquote{bajó del cielo} (\emph{Jn} 3, 13; 6, 33), \textquote{ha venido en carne} (\emph{1 Jn} 4, 2), porque \textquote{la Palabra se hizo carne, y puso su morada entre nosotros, y hemos visto su gloria, gloria que recibe del Padre como Hijo único, lleno de gracia y de verdad [\ldots{}] Pues de su plenitud hemos recibido todos, y gracia por gracia} (\emph{Jn} 1, 14. 16).
	
	\n{445} Después de su Resurrección, su filiación divina aparece en el poder de su humanidad glorificada: \textquote{Constituido Hijo de Dios con poder, según el Espíritu de santidad, por su Resurrección de entre los muertos} (\emph{Rm} 1, 4; cf. \emph{Hch} 13, 33). Los apóstoles podrán confesar \textquote{Hemos visto su gloria, gloria que recibe del Padre como Hijo único, lleno de gracia y de verdad} (\emph{Jn} 1, 14).
	
	\n{456} Con el Credo Niceno-Constantinopolitano respondemos confesando: \textquote{\emph{Por nosotros los hombres y por nuestra salvación} bajó del cielo, y por obra del Espíritu Santo se encarnó de María la Virgen y se hizo hombre} (DS 150).
	
	\n{457} El Verbo se encarnó \emph{para salvarnos reconciliándonos con Dios}: \textquote{Dios nos amó y nos envió a su Hijo como propiciación por nuestros pecados} (\emph{1 Jn} 4, 10). \textquote{El Padre envió a su Hijo para ser salvador del mundo} (\emph{1 Jn} 4, 14). \textquote{Él se manifestó para quitar los pecados} (\emph{1 Jn} 3, 5):
	
	\textquote{Nuestra naturaleza enferma exigía ser sanada; desgarrada, ser restablecida; muerta, ser resucitada. Habíamos perdido la posesión del bien, era necesario que se nos devolviera. Encerrados en las tinieblas, hacía falta que nos llegara la luz; estando cautivos, esperábamos un salvador; prisioneros, un socorro; esclavos, un libertador. ¿No tenían importancia estos razonamientos? ¿No merecían conmover a Dios hasta el punto de hacerle bajar hasta nuestra naturaleza humana para visitarla, ya que la humanidad se encontraba en un estado tan miserable y tan desgraciado?} (San Gregorio de Nisa, \emph{Oratio catechetica}, 15: PG 45, 48B).
	
	\n{458} El Verbo se encarnó \emph{para que nosotros conociésemos así el amor de Dios}: \textquote{En esto se manifestó el amor que Dios nos tiene: en que Dios envió al mundo a su Hijo único para que vivamos por medio de él} (\emph{1 Jn} 4, 9). \textquote{Porque tanto amó Dios al mundo que dio a su Hijo único, para que todo el que crea en él no perezca, sino que tenga vida eterna} (\emph{Jn} 3, 16).
	
	\n{459} El Verbo se encarnó \emph{para ser nuestro modelo de santidad}: \textquote{Tomad sobre vosotros mi yugo, y aprended de mí \ldots{} } (\emph{Mt} 11, 29). \textquote{Yo soy el Camino, la Verdad y la Vida. Nadie va al Padre sino por mí} (\emph{Jn} 14, 6). Y el Padre, en el monte de la Transfiguración, ordena: \textquote{Escuchadle} (\emph{Mc} 9, 7; cf. \emph{Dt} 6, 4-5). Él es, en efecto, el modelo de las bienaventuranzas y la norma de la Ley nueva: \textquote{Amaos los unos a los otros como yo os he amado} (\emph{Jn} 15, 12). Este amor tiene como consecuencia la ofrenda efectiva de sí mismo (cf. \emph{Mc} 8, 34).
	
	\n{460} El Verbo se encarnó \emph{para hacernos \textquote{partícipes de la naturaleza divina}} (\emph{2 P} 1, 4): \textquote{Porque tal es la razón por la que el Verbo se hizo hombre, y el Hijo de Dios, Hijo del hombre: para que el hombre al entrar en comunión con el Verbo y al recibir así la filiación divina, se convirtiera en hijo de Dios} (San Ireneo de Lyon, \emph{Adversus haereses}, 3, 19, 1). \textquote{Porque el Hijo de Dios se hizo hombre para hacernos Dios} (San Atanasio de Alejandría, \emph{De Incarnatione}, 54, 3: PG 25, 192B). \emph{Unigenitus} [\ldots{}] \emph{Dei Filius, suae divinitatis volens nos esse participes, naturam nostram assumpsit, ut homines deos faceret factus homo} (\textquote{El Hijo Unigénito de Dios, queriendo hacernos partícipes de su divinidad, asumió nuestra naturaleza, para que, habiéndose hecho hombre, hiciera dioses a los hombres}) (Santo Tomás de Aquino, \emph{Oficio de la festividad del Corpus}, Of. de Maitines, primer Nocturno, Lectura I).
	
	\n{461} Volviendo a tomar la frase de san Juan (\textquote{El Verbo se encarnó}: \emph{Jn} 1, 14), la Iglesia llama \textquote{Encarnación} al hecho de que el Hijo de Dios haya asumido una naturaleza humana para llevar a cabo por ella nuestra salvación. En un himno citado por san Pablo, la Iglesia canta el misterio de la Encarnación:
	
	\textquote{Tened entre vosotros los mismos sentimientos que tuvo Cristo: el cual, siendo de condición divina, no retuvo ávidamente el ser igual a Dios, sino que se despojó de sí mismo tomando condición de siervo, haciéndose semejante a los hombres y apareciendo en su porte como hombre; y se humilló a sí mismo, obedeciendo hasta la muerte y muerte de cruz} (\emph{Flp} 2, 5-8; cf. \emph{Liturgia de las Horas, Cántico de las Primeras Vísperas de Domingos}).
	
	\n{462} La carta a los Hebreos habla del mismo misterio:
	
	\textquote{Por eso, al entrar en este mundo, {[}Cristo{]} dice: No quisiste sacrificio y oblación; pero me has formado un cuerpo. Holocaustos y sacrificios por el pecado no te agradaron. Entonces dije: ¡He aquí que vengo [\ldots{}] a hacer, oh Dios, tu voluntad!} (\emph{Hb} 10, 5-7; \emph{Sal} 40, 7-9 {[}LXX{]}).
	
	\n{463} La fe en la verdadera encarnación del Hijo de Dios es el signo distintivo de la fe cristiana: \textquote{Podréis conocer en esto el Espíritu de Dios: todo espíritu que confiesa a Jesucristo, venido en carne, es de Dios} (\emph{1 Jn} 4, 2). Esa es la alegre convicción de la Iglesia desde sus comienzos cuando canta \textquote{el gran misterio de la piedad}: \textquote{Él ha sido manifestado en la carne} (\emph{1 Tm} 3, 16).
	
	\n{504} Jesús fue concebido por obra del Espíritu Santo en el seno de la Virgen María porque él es el \emph{Nuevo Adán} (cf. \emph{1 Co} 15, 45) que inaugura la nueva creación: \textquote{El primer hombre, salido de la tierra, es terreno; el segundo viene del cielo} (\emph{1 Co} 15, 47). La humanidad de Cristo, desde su concepción, está llena del Espíritu Santo porque Dios \textquote{le da el Espíritu sin medida} (\emph{Jn} 3, 34). De \textquote{su plenitud}, cabeza de la humanidad redimida (cf. \emph{Col} 1, 18), \textquote{hemos recibido todos gracia por gracia} (\emph{Jn} 1, 16).
	
	\n{505} Jesús, el nuevo Adán, inaugura por su concepción virginal el nuevo nacimiento de los hijos de adopción en el Espíritu Santo por la fe \textquote{¿Cómo será eso?} (\emph{Lc} 1, 34; cf. \emph{Jn} 3, 9). La participación en la vida divina no nace \textquote{de la sangre, ni de deseo de carne, ni de deseo de hombre, sino de Dios} (\emph{Jn} 1, 13). La acogida de esta vida es virginal porque toda ella es dada al hombre por el Espíritu. El sentido esponsal de la vocación humana con relación a Dios (cf. \emph{2 Co} 11, 2) se lleva a cabo perfectamente en la maternidad virginal de María.
	
	\n{526} \textquote{Hacerse niño} con relación a Dios es la condición para entrar en el Reino (cf. \emph{Mt} 18, 3-4); para eso es necesario abajarse (cf. \emph{Mt} 23, 12), hacerse pequeño; más todavía: es necesario \textquote{nacer de lo alto} (\emph{Jn} 3,7), \textquote{nacer de Dios} (\emph{Jn} 1, 13) para \textquote{hacerse hijos de Dios} (\emph{Jn} 1, 12). El misterio de Navidad se realiza en nosotros cuando Cristo \textquote{toma forma} en nosotros (\emph{Ga} 4, 19). Navidad es el misterio de este \textquote{admirable intercambio}:
	
	\textquote{¡Oh admirable intercambio! El Creador del género humano, tomando cuerpo y alma, nace de la Virgen y, hecho hombre sin concurso de varón, nos da parte en su divinidad} (\emph{Solemnidad de la Santísima Virgen María, Madre de Dios,} Antífona de I y II Vísperas: \emph{Liturgia de las Horas}).
	
	\n{1216} \textquote{Este baño es llamado \emph{iluminación} porque quienes reciben esta enseñanza (catequética) su espíritu es iluminado} (San Justino, \emph{Apología} 1,61). Habiendo recibido en el Bautismo al Verbo, \textquote{la luz verdadera que ilumina a todo hombre} (\emph{Jn} 1,9), el bautizado, \textquote{tras haber sido iluminado} (\emph{Hb} 10,32), se convierte en \textquote{hijo de la luz} (\emph{1 Ts} 5,5), y en \textquote{luz} él mismo (\emph{Ef} 5,8):
	
	El Bautismo \textquote{es el más bello y magnífico de los dones de Dios [\ldots{}] lo llamamos don, gracia, unción, iluminación, vestidura de incorruptibilidad, baño de regeneración, sello y todo lo más precioso que hay. \emph{Don}, porque es conferido a los que no aportan nada; \emph{gracia}, porque es dado incluso a culpables; \emph{bautismo}, porque el pecado es sepultado en el agua; \emph{unción}, porque es sagrado y real (tales son los que son ungidos); \emph{iluminación}, porque es luz resplandeciente; \emph{vestidura}, porque cubre nuestra vergüenza; \emph{baño}, porque lava; \emph{sello}, porque nos guarda y es el signo de la soberanía de Dios} (San Gregorio Nacianceno, \emph{Oratio} 40,3-4).
	
	\n{2466} En Jesucristo la verdad de Dios se manifestó en plenitud. \textquote{Lleno de gracia y de verdad} (\emph{Jn} 1, 14), él es la \textquote{luz del mundo} (\emph{Jn} 8, 12), \emph{la Verdad} (cf. \emph{Jn} 14, 6). El que cree en él, no permanece en las tinieblas (cf. \emph{Jn} 12, 46). El discípulo de Jesús, \textquote{permanece en su palabra}, para conocer \textquote{la verdad que hace libre} (cf. \emph{Jn} 8, 31-32) y que santifica (cf. \emph{Jn} 17, 17). Seguir a Jesús es vivir del \textquote{Espíritu de verdad} (\emph{Jn} 14, 17) que el Padre envía en su nombre (cf. \emph{Jn} 14, 26) y que conduce \textquote{a la verdad completa} (\emph{Jn} 16, 13). Jesús enseña a sus discípulos el amor incondicional de la verdad: \textquote{Sea vuestro lenguaje: \textquote{sí, sí}; \textquote{no, no}} (\emph{Mt} 5, 37).
	
	\n{2787} Cuando decimos Padre \textquote{nuestro}, reconocemos ante todo que todas sus promesas de amor anunciadas por los profetas se han cumplido en la \emph{nueva y eterna Alianza} en Cristo: hemos llegado a ser \textquote{su Pueblo} y Él es desde ahora en adelante \textquote{nuestro Dios}. Esta relación nueva es una pertenencia mutua dada gratuitamente: por amor y fidelidad (cf. \emph{Os} 2, 21-22; 6, 1-6) tenemos que responder a la gracia y a la verdad que nos han sido dadas en Jesucristo (cf. \emph{Jn} 1, 17).
	
	Cristo, Sabiduría de Dios
	
	CEC 272, 295, 299, 474, 721, 1831:
	
	\ccesec{El misterio de la aparente impotencia de Dios}
	
	\n{272} La fe en Dios Padre Todopoderoso puede ser puesta a prueba por la experiencia del mal y del sufrimiento. A veces Dios puede parecer ausente e incapaz de impedir el mal. Ahora bien, Dios Padre ha revelado su omnipotencia de la manera más misteriosa en el anonadamiento voluntario y en la Resurrección de su Hijo, por los cuales ha vencido el mal. Así, Cristo crucificado es \textquote{poder de Dios y sabiduría de Dios. Porque la necedad divina es más sabia que la sabiduría de los hombres, y la debilidad divina, más fuerte que la fuerza de los hombres} (\emph{1 Co} 2, 24-25). En la Resurrección y en la exaltación de Cristo es donde el Padre \textquote{desplegó el vigor de su fuerza} y manifestó \textquote{la soberana grandeza de su poder para con nosotros, los creyentes} (\emph{Ef} 1,19-22).
	
	\ccesec{Dios crea por sabiduría y por amor}
	
	\n{295} Creemos que Dios creó el mundo según su sabiduría (cf. \emph{Sb} 9,9). Este no es producto de una necesidad cualquiera, de un destino ciego o del azar. Creemos que procede de la voluntad libre de Dios que ha querido hacer participar a las criaturas de su ser, de su sabiduría y de su bondad: \textquote{Porque tú has creado todas las cosas; por tu voluntad lo que no existía fue creado} (\emph{Ap} 4,11). \textquote{¡Cuán numerosas son tus obras, Señor! Todas las has hecho con sabiduría} (\emph{Sal} 104,24). \textquote{Bueno es el Señor para con todos, y sus ternuras sobre todas sus obras} (\emph{Sal} 145,9).
	
	\ccesec{Dios crea un mundo ordenado y bueno}
	
	\n{299} Porque Dios crea con sabiduría, la creación está ordenada: \textquote{Tú todo lo dispusiste con medida, número y peso} (\emph{Sb} 11,20). Creada en y por el Verbo eterno, \textquote{imagen del Dios invisible} (\emph{Col} 1,15), la creación está destinada, dirigida al hombre, imagen de Dios (cf. \emph{Gn} 1,26), llamado a una relación personal con Dios. Nuestra inteligencia, participando en la luz del Entendimiento divino, puede entender lo que Dios nos dice por su creación (cf. \emph{Sal} 19,2-5), ciertamente no sin gran esfuerzo y en un espíritu de humildad y de respeto ante el Creador y su obra (cf. \emph{Jb} 42,3). Salida de la bondad divina, la creación participa en esa bondad (\textquote{Y vio Dios que era bueno [\ldots{}] muy bueno}: \emph{Gn} 1,4.10.12.18.21.31). Porque la creación es querida por Dios como un don dirigido al hombre, como una herencia que le es destinada y confiada. La Iglesia ha debido, en repetidas ocasiones, defender la bondad de la creación, comprendida la del mundo material (cf. San León Magno, c. \emph{Quam laudabiliter}, DS, 286; Concilio de Braga I: \emph{ibíd}., 455-463; Concilio de Letrán IV: \emph{ibíd.,} 800; Concilio de Florencia: \emph{ibíd.,}1333; Concilio Vaticano I: \emph{ibíd.,} 3002).
	
	\n{474} Debido a su unión con la Sabiduría divina en la persona del Verbo encarnado, el conocimiento humano de Cristo gozaba en plenitud de la ciencia de los designios eternos que había venido a revelar (cf. \emph{Mc} 8,31; 9,31; 10, 33-34; 14,18-20. 26-30). Lo que reconoce ignorar en este campo (cf. \emph{Mc} 13,32), declara en otro lugar no tener misión de revelarlo (cf. \emph{Hch} 1, 7).
	
	\ccesec{\textquote{Alégrate, llena de gracia}}
	
	\n{721} María, la Santísima Madre de Dios, la siempre Virgen, es la obra maestra de la Misión del Hijo y del Espíritu Santo en la Plenitud de los tiempos. Por primera vez en el designio de Salvación y porque su Espíritu la ha preparado, el Padre encuentra la Morada en donde su Hijo y su Espíritu pueden habitar entre los hombres. Por ello, los más bellos textos sobre la Sabiduría, la Tradición de la Iglesia los ha entendido frecuentemente con relación a María (cf. \emph{Pr} 8, 1-9, 6; \emph{Si} 24): María es cantada y representada en la Liturgia como el \textquote{Trono de la Sabiduría}.
	
	En ella comienzan a manifestarse las \textquote{maravillas de Dios}, que el Espíritu va a realizar en Cristo y en la Iglesia:
	
	\n{1831} Los siete \emph{dones} del Espíritu Santo son: sabiduría, inteligencia, consejo, fortaleza, ciencia, piedad y temor de Dios. Pertenecen en plenitud a Cristo, Hijo de David (cf. \emph{Is} 11, 1-2). Completan y llevan a su perfección las virtudes de quienes los reciben. Hacen a los fieles dóciles para obedecer con prontitud a las inspiraciones divinas.
	
	\textquote{Tu espíritu bueno me guíe por una tierra llana} (\emph{Sal} 143,10).
	
	\textquote{Todos los que son guiados por el Espíritu de Dios son hijos de Dios [\ldots{}] Y, si hijos, también herederos; herederos de Dios y coherederos de Cristo} (\emph{Rm} 8, 14.17).
	
	Dios nos dona la Sabiduría
	
	CEC 158, 283, 1303, 1831, 2500:
	
	\n{158} \textquote{La fe \emph{trata de comprender}} (San Anselmo de Canterbury, \emph{Proslogion}, proemium: PL 153, 225A) es inherente a la fe que el creyente desee conocer mejor a aquel en quien ha puesto su fe, y comprender mejor lo que le ha sido revelado; un conocimiento más penetrante suscitará a su vez una fe mayor, cada vez más encendida de amor. La gracia de la fe abre \textquote{los ojos del corazón} (\emph{Ef} 1,18) para una inteligencia viva de los contenidos de la Revelación, es decir, del conjunto del designio de Dios y de los misterios de la fe, de su conexión entre sí y con Cristo, centro del Misterio revelado. Ahora bien, \textquote{para que la inteligencia de la Revelación sea más profunda, el mismo Espíritu Santo perfecciona constantemente la fe por medio de sus dones} (DV 5). Así, según el adagio de san Agustín (\emph{Sermo} 43,7,9: PL 38, 258), \textquote{creo para comprender y comprendo para creer mejor}.
	
	\n{283} La cuestión sobre los orígenes del mundo y del hombre es objeto de numerosas investigaciones científicas que han enriquecido magníficamente nuestros conocimientos sobre la edad y las dimensiones del cosmos, el devenir de las formas vivientes, la aparición del hombre. Estos descubrimientos nos invitan a admirar más la grandeza del Creador, a darle gracias por todas sus obras y por la inteligencia y la sabiduría que da a los sabios e investigadores. Con Salomón, éstos pueden decir: \textquote{Fue él quien me concedió el conocimiento verdadero de cuanto existe, quien me dio a conocer la estructura del mundo y las propiedades de los elementos [\ldots{}] porque la que todo lo hizo, la Sabiduría, me lo enseñó} (\emph{Sb} 7,17-21).
	
	\n{1303} Por este hecho, la Confirmación confiere crecimiento y profundidad a la gracia bautismal:
	
	--- nos introduce más profundamente en la filiación divina que nos hace decir \textquote{\emph{Abbá}, Padre} (\emph{Rm} 8,15).;
	
	--- nos une más firmemente a Cristo;
	
	--- aumenta en nosotros los dones del Espíritu Santo;
	
	--- hace más perfecto nuestro vínculo con la Iglesia (cf. LG 11);
	
	--- nos concede una fuerza especial del Espíritu Santo para difundir y defender la fe mediante la palabra y las obras como verdaderos testigos de Cristo, para confesar valientemente el nombre de Cristo y para no sentir jamás vergüenza de la cruz (cf. DS 1319; LG 11,12):
	
	\textquote{Recuerda, pues, que has recibido el signo espiritual, el Espíritu de sabiduría e inteligencia, el Espíritu de consejo y de fortaleza, el Espíritu de conocimiento y de piedad, el Espíritu de temor santo, y guarda lo que has recibido. Dios Padre te ha marcado con su signo, Cristo Señor te ha confirmado y ha puesto en tu corazón la prenda del Espíritu} (San Ambrosio, \emph{De mysteriis} 7,42).
	
	\n{1831} Los siete \emph{dones} del Espíritu Santo son: sabiduría, inteligencia, consejo, fortaleza, ciencia, piedad y temor de Dios. Pertenecen en plenitud a Cristo, Hijo de David (cf. \emph{Is} 11, 1-2). Completan y llevan a su perfección las virtudes de quienes los reciben. Hacen a los fieles dóciles para obedecer con prontitud a las inspiraciones divinas.
	
	\textquote{Tu espíritu bueno me guíe por una tierra llana} (\emph{Sal} 143,10).
	
	\textquote{Todos los que son guiados por el Espíritu de Dios son hijos de Dios [\ldots{}] Y, si hijos, también herederos; herederos de Dios y coherederos de Cristo} (\emph{Rm} 8, 14.17)
	
	\ccesec{Verdad, belleza y arte sacro}
	
	\n{2500} La práctica del bien va acompañada de un placer espiritual gratuito y de belleza moral. De igual modo, la verdad entraña el gozo y el esplendor de la belleza espiritual. La verdad es bella por sí misma. La verdad de la palabra, expresión racional del conocimiento de la realidad creada e increada, es necesaria al hombre dotado de inteligencia, pero la verdad puede también encontrar otras formas de expresión humana, complementarias, sobre todo cuando se trata de evocar lo que ella entraña de indecible, las profundidades del corazón humano, las elevaciones del alma, el Misterio de Dios. Antes de revelarse al hombre en palabras de verdad, Dios se revela a él, mediante el lenguaje universal de la Creación, obra de su Palabra, de su Sabiduría: el orden y la armonía del cosmos, que percibe tanto el niño como el hombre de ciencia, \textquote{pues por la grandeza y hermosura de las criaturas se llega, por analogía, a contemplar a su Autor} (\emph{Sb} 13, 5), \textquote{pues fue el Autor mismo de la belleza quien las creó} (\emph{Sb} 13, 3).
	
	\textquote{La sabiduría es un hálito del poder de Dios, una emanación pura de la gloria del Omnipotente, por lo que nada manchado llega a alcanzarla. Es un reflejo de la luz eterna, un espejo sin mancha de la actividad de Dios, una imagen de su bondad} (\emph{Sb} 7, 25-26). \textquote{La sabiduría es, en efecto, más bella que el Sol, supera a todas las constelaciones; comparada con la luz, sale vencedora, porque a la luz sucede la noche, pero contra la sabiduría no prevalece la maldad} (\emph{Sb} 7, 29-30). \textquote{Yo me constituí en el amante de su belleza} (\emph{Sb} 8, 2).
	
	He manchado mi cuerpo,
	
	he ensuciado mi espíritu,
	
	estoy todo lleno de llagas;
	
	pero tú, oh Cristo médico,
	
	cura mi espíritu y cuerpo con la penitencia,
	
	báñame, purifícame, lávame:
	
	déjame más puro que la nieve\ldots{}
	
	Crucificado por todos,
	
	has ofrecido tu cuerpo y tu sangre, oh Verbo:
	
	el cuerpo para re-plasmarme,
	
	la sangre para lavarme;
	
	y has entregado el espíritu
	
	para portarme, oh Cristo, a tu Engendrador.
	
	Has obrado la salvación
	
	en medio de la tierra.
	
	Por tu voluntad
	
	has sido clavado en el árbol de la Cruz
	
	y el Edén que había sido cerrado, se ha abierto\ldots{}
	
	Sea mi fuente bautismal
	
	la sangre de tu costado,
	
	y bebida el agua de remisión que ha brotado\ldots{}
	
	y sea ungido, bebiendo como crisma y bebida,
	
	tu vivificante palabra, oh Verbo.
	
	(Canon de San Andrés de Creta).
	
	%EPH
	
	La Epifanía del Señor
	
	CEC 528, 724:
	
	\n{528} La \emph{Epifanía} es la manifestación de Jesús como Mesías de Israel, Hijo de Dios y Salvador del mundo. Con el bautismo de Jesús en el Jordán y las bodas de Caná (cf. \emph{Solemnidad de la Epifanía del Señor}, Antífona del \textquote{Magnificat} en II Vísperas, LH), la Epifanía celebra la adoración de Jesús por unos \textquote{magos} venidos de Oriente (\emph{Mt} 2, 1) En estos \textquote{magos}, representantes de religiones paganas de pueblos vecinos, el Evangelio ve las primicias de las naciones que acogen, por la Encarnación, la Buena Nueva de la salvación. La llegada de los magos a Jerusalén para \textquote{rendir homenaje al rey de los Judíos} (\emph{Mt} 2, 2) muestra que buscan en Israel, a la luz mesiánica de la estrella de David (cf. \emph{Nm} 24, 17; \emph{Ap} 22, 16) al que será el rey de las naciones (cf. \emph{Nm} 24, 17-19). Su venida significa que los gentiles no pueden descubrir a Jesús y adorarle como Hijo de Dios y Salvador del mundo sino volviéndose hacia los judíos (cf. \emph{Jn} 4, 22) y recibiendo de ellos su promesa mesiánica tal como está contenida en el Antiguo Testamento (cf. \emph{Mt} 2, 4-6). La Epifanía manifiesta que \textquote{la multitud de los gentiles entra en la familia de los patriarcas} (San León Magno, \emph{Sermones}, 23: PL 54, 224B) y adquiere la \emph{israelitica dignitas} (la dignidad israelítica) (Vigilia pascual, Oración después de la tercera lectura: \emph{Misal Romano}).
	
	\n{724} En María, el Espíritu Santo \emph{manifiesta} al Hijo del Padre hecho Hijo de la Virgen. Ella es la zarza ardiente de la teofanía definitiva: llena del Espíritu Santo, presenta al Verbo en la humildad de su carne dándolo a conocer a los pobres (cf. \emph{Lc} 2, 15-19) y a las primicias de las naciones (cf. \emph{Mt} 2, 11).
	
	Cristo, luz de las naciones
	
	CEC 280, 529, 748, 1165, 2466, 2715:
	
	\n{280} La creación es el fundamento de \textquote{todos los designios salvíficos de Dios}, \textquote{el comienzo de la historia de la salvación} (DCG 51), que culmina en Cristo. Inversamente, el Misterio de Cristo es la luz decisiva sobre el Misterio de la creación; revela el fin en vista del cual, \textquote{al principio, Dios creó el cielo y la tierra} (\emph{Gn} 1,1): desde el principio Dios preveía la gloria de la nueva creación en Cristo (cf. \emph{Rm} 8,18-23).
	
	\n{529} \emph{La Presentación de Jesús en el Templo} (cf. \emph{Lc} 2, 22-39) lo muestra como el Primogénito que pertenece al Señor (cf. \emph{Ex} 13,2.12-13). Con Simeón y Ana, toda la expectación de Israel es la que viene al \emph{Encuentro} de su Salvador (la tradición bizantina llama así a este acontecimiento). Jesús es reconocido como el Mesías tan esperado, \textquote{luz de las naciones} y \textquote{gloria de Israel}, pero también \textquote{signo de contradicción}. La espada de dolor predicha a María anuncia otra oblación, perfecta y única, la de la Cruz que dará la salvación que Dios ha preparado \textquote{ante todos los pueblos}.
	
	\n{748} \textquote{Cristo es la luz de los pueblos. Por eso, este sacrosanto Sínodo, reunido en el Espíritu Santo, desea vehementemente iluminar a todos los hombres con la luz de Cristo, que resplandece sobre el rostro de la Iglesia (LG 1), anunciando el Evangelio a todas las criaturas}. Con estas palabras comienza la \textquote{Constitución dogmática sobre la Iglesia} del Concilio Vaticano II. Así, el Concilio muestra que el artículo de la fe sobre la Iglesia depende enteramente de los artículos que se refieren a Cristo Jesús. La Iglesia no tiene otra luz que la de Cristo; ella es, según una imagen predilecta de los Padres de la Iglesia, comparable a la luna cuya luz es reflejo del sol.
	
	\n{1165} Cuando la Iglesia celebra el Misterio de Cristo, hay una palabra que jalona su oración: \emph{¡Hoy!}, como eco de la oración que le enseñó su Señor (\emph{Mt} 6,11) y de la llamada del Espíritu Santo (\emph{Hb} 3,7-4,11; \emph{Sal} 95,7). Este \textquote{hoy} del Dios vivo al que el hombre está llamado a entrar, es la \textquote{Hora} de la Pascua de Jesús, que atraviesa y guía toda la historia humana:
	
	\textquote{La vida se ha extendido sobre todos los seres y todos están llenos de una amplia luz: el Oriente de los orientes invade el universo, y el que existía \textquote{antes del lucero de la mañana} y antes de todos los astros, inmortal e inmenso, el gran Cristo brilla sobre todos los seres más que el sol. Por eso, para nosotros que creemos en él, se instaura un día de luz, largo, eterno, que no se extingue: la Pascua mística} (Pseudo-Hipólito Romano, \emph{In Sanctum Pascha} 1-2).
	
	\n{2466} En Jesucristo la verdad de Dios se manifestó en plenitud. \textquote{Lleno de gracia y de verdad} (\emph{Jn} 1, 14), él es la \textquote{luz del mundo} (\emph{Jn} 8, 12), \emph{la Verdad} (cf. \emph{Jn} 14, 6). El que cree en él, no permanece en las tinieblas (cf. \emph{Jn} 12, 46). El discípulo de Jesús, \textquote{permanece en su palabra}, para conocer \textquote{la verdad que hace libre} (cf. \emph{Jn} 8, 31-32) y que santifica (cf. \emph{Jn} 17, 17). Seguir a Jesús es vivir del \textquote{Espíritu de verdad} (\emph{Jn} 14, 17) que el Padre envía en su nombre (cf. \emph{Jn} 14, 26) y que conduce \textquote{a la verdad completa} (\emph{Jn} 16, 13). Jesús enseña a sus discípulos el amor incondicional de la verdad: \textquote{Sea vuestro lenguaje: \textquote{sí, sí}; \textquote{no, no}} (\emph{Mt} 5, 37).
	
	\n{2715} La oración contemplativa es mirada de fe, fijada en Jesús. \textquote{Yo le miro y él me mira}, decía a su santo cura un campesino de Ars que oraba ante el Sagrario (cf. F. Trochu, \emph{Le Curé d'Ars Saint Jean-Marie Vianney}). Esta atención a Él es renuncia a \textquote{mí}. Su mirada purifica el corazón. La luz de la mirada de Jesús ilumina los ojos de nuestro corazón; nos enseña a ver todo a la luz de su verdad y de su compasión por todos los hombres. La contemplación dirige también su mirada a los misterios de la vida de Cristo. Aprende así el \textquote{conocimiento interno del Señor} para más amarle y seguirle (cf. San Ignacio de Loyola, \emph{Exercitia spiritualia,} 104).
	
	La Iglesia, sacramento de la unidad del género humano
	
	CEC 60, 442, 674, 755, 767, 774-776, 781, 831:
	
	\n{60} El pueblo nacido de Abraham será el depositario de la promesa hecha a los patriarcas, el pueblo de la elección (cf. \emph{Rm} 11,28), llamado a preparar la reunión un día de todos los hijos de Dios en la unidad de la Iglesia (cf. \emph{Jn} 11,52; 10,16); ese pueblo será la raíz en la que serán injertados los paganos hechos creyentes (cf. \emph{Rm} 11,17-18.24).
	
	\n{442} No ocurre así con Pedro cuando confiesa a Jesús como \textquote{el Cristo, el Hijo de Dios vivo} (\emph{Mt} 16, 16) porque Jesús le responde con solemnidad \textquote{\emph{no te ha revelado} esto ni la carne ni la sangre, sino \emph{mi Padre} que está en los cielos} (\emph{Mt} 16, 17). Paralelamente Pablo dirá a propósito de su conversión en el camino de Damasco: \textquote{Cuando Aquel que me separó desde el seno de mi madre y me llamó por su gracia, tuvo a bien revelar en mí a su Hijo para que le anunciase entre los gentiles\ldots{}} (\emph{Ga} 1,15-16). \textquote{Y en seguida se puso a predicar a Jesús en las sinagogas: que él era el Hijo de Dios} (\emph{Hch} 9, 20). Este será, desde el principio (cf. \emph{1 Ts} 1, 10), el centro de la fe apostólica (cf. \emph{Jn} 20, 31) profesada en primer lugar por Pedro como cimiento de la Iglesia (cf. \emph{Mt} 16, 18).
	
	\n{674} La venida del Mesías glorioso, en un momento determinado de la historia (cf. \emph{Rm} 11, 31), se vincula al reconocimiento del Mesías por \textquote{todo Israel} (\emph{Rm} 11, 26; \emph{Mt} 23, 39) del que \textquote{una parte está endurecida} (\emph{Rm} 11, 25) en \textquote{la incredulidad} (\emph{Rm} 11, 20) respecto a Jesús. San Pedro dice a los judíos de Jerusalén después de Pentecostés: \textquote{Arrepentíos, pues, y convertíos para que vuestros pecados sean borrados, a fin de que del Señor venga el tiempo de la consolación y envíe al Cristo que os había sido destinado, a Jesús, a quien debe retener el cielo hasta el tiempo de la restauración universal, de que Dios habló por boca de sus profetas} (\emph{Hch} 3, 19-21). Y san Pablo le hace eco: \textquote{si su reprobación ha sido la reconciliación del mundo ¿qué será su readmisión sino una resurrección de entre los muertos?} (\emph{Rm} 11, 5). La entrada de \textquote{la plenitud de los judíos} (\emph{Rm} 11, 12) en la salvación mesiánica, a continuación de \textquote{la plenitud de los gentiles} (Rm 11, 25; cf. Lc 21, 24), hará al pueblo de Dios \textquote{llegar a la plenitud de Cristo} (\emph{Ef} 4, 13) en la cual \textquote{Dios será todo en nosotros} (\emph{1 Co} 15, 28).
	
	\n{755} \textquote{La Iglesia es \emph{labranza} o campo de Dios (\emph{1 Co} 3, 9). En este campo crece el antiguo olivo cuya raíz santa fueron los patriarcas y en el que tuvo y tendrá lugar la reconciliación de los judíos y de los gentiles (\emph{Rm} 11, 13-26). El labrador del cielo la plantó como viña selecta (\emph{Mt} 21, 33-43 par.; cf. \emph{Is} 5, 1-7). La verdadera vid es Cristo, que da vida y fecundidad a los sarmientos, es decir, a nosotros, que permanecemos en él por medio de la Iglesia y que sin él no podemos hacer nada (\emph{Jn} 15, 1-5)}. (LG 6).
	
	\n{\\ }
	
	\ccesec{La Iglesia, manifestada por el Espíritu Santo}
	
	\n{767} \textquote{Cuando el Hijo terminó la obra que el Padre le encargó realizar en la tierra, fue enviado el Espíritu Santo el día de Pentecostés para que santificara continuamente a la Iglesia} (LG 4). Es entonces cuando \textquote{la Iglesia se manifestó públicamente ante la multitud; se inició la difusión del Evangelio entre los pueblos mediante la predicación} (AG 4). Como ella es \textquote{convocatoria} de salvación para todos los hombres, la Iglesia es, por su misma naturaleza, misionera enviada por Cristo a todas las naciones para hacer de ellas discípulos suyos (cf. \emph{Mt} 28, 19-20; AG 2,5-6).
	
	\ccesec{La Iglesia, sacramento universal de la salvación}
	
	\n{774} La palabra griega \emph{mysterion} ha sido traducida en latín por dos términos: \emph{mysterium} y \emph{sacramentum}. En la interpretación posterior, el término \emph{sacramentum} expresa mejor el signo visible de la realidad oculta de la salvación, indicada por el término \emph{mysterium}. En este sentido, Cristo es Él mismo el Misterio de la salvación: \emph{Non est enim aliud Dei mysterium, nisi Christus} (\textquote{No hay otro misterio de Dios fuera de Cristo}; san Agustín, \emph{Epistula} 187, 11, 34). La obra salvífica de su humanidad santa y santificante es el sacramento de la salvación que se manifiesta y actúa en los sacramentos de la Iglesia (que las Iglesias de Oriente llaman también \textquote{los santos Misterios}). Los siete sacramentos son los signos y los instrumentos mediante los cuales el Espíritu Santo distribuye la gracia de Cristo, que es la Cabeza, en la Iglesia que es su Cuerpo. La Iglesia contiene, por tanto, y comunica la gracia invisible que ella significa. En este sentido analógico ella es llamada \textquote{sacramento}.
	
	\n{775} \textquote{La Iglesia es en Cristo como un sacramento o signo e instrumento de la unión íntima con Dios y de la unidad de todo el género humano} (LG 1): Ser el \emph{sacramento de la unión íntima de los hombres con Dios} es el primer fin de la Iglesia. Como la comunión de los hombres radica en la unión con Dios, la Iglesia es también el sacramento de la \emph{unidad del género humano}. Esta unidad ya está comenzada en ella porque reúne hombres \textquote{de toda nación, raza, pueblo y lengua} (\emph{Ap} 7, 9); al mismo tiempo, la Iglesia es \textquote{signo e instrumento} de la plena realización de esta unidad que aún está por venir.
	
	\n{776} Como sacramento, la Iglesia es instrumento de Cristo. Ella es asumida por Cristo \textquote{como instrumento de redención universal} (LG 9), \textquote{sacramento universal de salvación} (LG 48), por medio del cual Cristo \textquote{manifiesta y realiza al mismo tiempo el misterio del amor de Dios al hombre} (GS 45, 1). Ella \textquote{es el proyecto visible del amor de Dios hacia la humanidad} (Pablo VI, \emph{Discurso a los Padres del Sacro Colegio Cardenalicio}, 22 junio 1973) que quiere \textquote{que todo el género humano forme un único Pueblo de Dios, se una en un único Cuerpo de Cristo, se coedifique en un único templo del Espíritu Santo} (AG 7; cf. LG 17).
	
	\n{781} \textquote{En todo tiempo y lugar ha sido grato a Dios el que le teme y practica la justicia. Sin embargo, quiso santificar y salvar a los hombres no individualmente y aislados, sin conexión entre sí, sino hacer de ellos un pueblo para que le conociera de verdad y le sirviera con una vida santa. Eligió, pues, a Israel para pueblo suyo, hizo una alianza con él y lo fue educando poco a poco. Le fue revelando su persona y su plan a lo largo de su historia y lo fue santificando. Todo esto, sin embargo, sucedió como preparación y figura de su alianza nueva y perfecta que iba a realizar en Cristo [\ldots{}], es decir, el Nuevo Testamento en su sangre, convocando a las gentes de entre los judíos y los gentiles para que se unieran, no según la carne, sino en el Espíritu} (LG 9).
	
	\n{831} {[}La Iglesia{]} es católica porque ha sido enviada por Cristo en misión a la totalidad del género humano (cf. \emph{Mt} 28, 19):
	
	\textquote{Todos los hombres están invitados al Pueblo de Dios. Por eso este pueblo, uno y único, ha de extenderse por todo el mundo a través de todos los siglos, para que así se cumpla el designio de Dios, que en el principio creó una única naturaleza humana y decidió reunir a sus hijos dispersos [\ldots{}] Este carácter de universalidad, que distingue al pueblo de Dios, es un don del mismo Señor. Gracias a este carácter, la Iglesia Católica tiende siempre y eficazmente a reunir a la humanidad entera con todos sus valores bajo Cristo como Cabeza, en la unidad de su Espíritu} (LG 13).
	
	
	%BAPT
	
	El Directorio Homilético no indica temas del Catecismo para esta fiesta. Pero podemos considerar los siguientes:
	
	Juan, precursor, profeta y bautista
	
	CEC 523; 717-720:
	
	\n{523} \emph{San Juan Bautista} es el precursor (cf. \emph{Hch} 13, 24) inmediato del Señor, enviado para prepararle el camino (cf. \emph{Mt} 3, 3). \textquote{Profeta del Altísimo} (\emph{Lc} 1, 76), sobrepasa a todos los profetas (cf. \emph{Lc} 7, 26), de los que es el último (cf. \emph{Mt} 11, 13), e inaugura el Evangelio (cf. \emph{Hch} 1, 22; \emph{Lc} 16,16); desde el seno de su madre (cf. \emph{Lc} 1,41) saluda la venida de Cristo y encuentra su alegría en ser \textquote{el amigo del esposo} (\emph{Jn} 3, 29) a quien señala como \textquote{el Cordero de Dios que quita el pecado del mundo} (\emph{Jn} 1, 29). Precediendo a Jesús \textquote{con el espíritu y el poder de Elías} (\emph{Lc} 1, 17), da testimonio de él mediante su predicación, su bautismo de conversión y finalmente con su martirio (cf. \emph{Mc} 6, 17-29).
	
	\n{717} \textquote{Hubo un hombre, enviado por Dios, que se llamaba Juan}. (\emph{Jn} 1, 6). Juan fue \textquote{lleno del Espíritu Santo ya desde el seno de su madre} (\emph{Lc} 1, 15. 41) por obra del mismo Cristo que la Virgen María acababa de concebir del Espíritu Santo. La \textquote{Visitación} de María a Isabel se convirtió así en \textquote{visita de Dios a su pueblo} (\emph{Lc} 1, 68).
	
	\n{718} Juan es \textquote{Elías que debe venir} (\emph{Mt} 17, 10-13): El fuego del Espíritu lo habita y le hace correr delante {[}como \textquote{precursor}{]} del Señor que viene. En Juan el Precursor, el Espíritu Santo culmina la obra de \textquote{preparar al Señor un pueblo bien dispuesto} (\emph{Lc} 1, 17).
	
	\n{719} Juan es \textquote{más que un profeta} (\emph{Lc} 7, 26). En él, el Espíritu Santo consuma el \textquote{hablar por los profetas}. Juan termina el ciclo de los profetas inaugurado por Elías (cf. \emph{Mt} 11, 13-14). Anuncia la inminencia de la consolación de Israel, es la \textquote{voz} del Consolador que llega (\emph{Jn} 1, 23; cf. \emph{Is} 40, 1-3). Como lo hará el Espíritu de Verdad, \textquote{vino como testigo para dar testimonio de la luz} (\emph{Jn} 1, 7; cf. \emph{Jn} 15, 26; 5, 33). Con respecto a Juan, el Espíritu colma así las \textquote{indagaciones de los profetas} y la ansiedad de los ángeles (\emph{1 P} 1, 10-12): \textquote{Aquél sobre quien veas que baja el Espíritu y se queda sobre él, ése es el que bautiza con el Espíritu Santo. Y yo lo he visto y doy testimonio de que éste es el Hijo de Dios [\ldots{}] He ahí el Cordero de Dios} (\emph{Jn} 1, 33-36).
	
	\n{720} En fin, con Juan Bautista, el Espíritu Santo, inaugura, prefigurándolo, lo que realizará con y en Cristo: volver a dar al hombre la \textquote{semejanza} divina. El bautismo de Juan era para el arrepentimiento, el del agua y del Espíritu será un nuevo nacimiento (cf. \emph{Jn} 3, 5).
	
	El Bautismo de Jesús
	
	CEC 535-537:
	
	\n{535} El comienzo (cf. \emph{Lc} 3, 23) de la vida pública de Jesús es su bautismo por Juan en el Jordán (cf. \emph{Hch} 1, 22). Juan proclamaba \textquote{un bautismo de conversión para el perdón de los pecados} (\emph{Lc} 3, 3). Una multitud de pecadores, publicanos y soldados (cf. \emph{Lc} 3, 10-14), fariseos y saduceos (cf. \emph{Mt} 3, 7) y prostitutas (cf. \emph{Mt} 21, 32) viene a hacerse bautizar por él. \textquote{Entonces aparece Jesús}. El Bautista duda. Jesús insiste y recibe el bautismo. Entonces el Espíritu Santo, en forma de paloma, viene sobre Jesús, y la voz del cielo proclama que él es \textquote{mi Hijo amado} (\emph{Mt} 3, 13-17). Es la manifestación (\textquote{Epifanía}) de Jesús como Mesías de Israel e Hijo de Dios.
	
	\n{536} El bautismo de Jesús es, por su parte, la aceptación y la inauguración de su misión de Siervo doliente. Se deja contar entre los pecadores (cf. \emph{Is} 53, 12); es ya \textquote{el Cordero de Dios que quita el pecado del mundo} (\emph{Jn} 1, 29); anticipa ya el \textquote{bautismo} de su muerte sangrienta (cf. \emph{Mc} 10, 38; \emph{Lc} 12, 50). Viene ya a \textquote{cumplir toda justicia} (\emph{Mt} 3, 15), es decir, se somete enteramente a la voluntad de su Padre: por amor acepta el bautismo de muerte para la remisión de nuestros pecados (cf. \emph{Mt} 26, 39). A esta aceptación responde la voz del Padre que pone toda su complacencia en su Hijo (cf. \emph{Lc} 3, 22; \emph{Is} 42, 1). El Espíritu que Jesús posee en plenitud desde su concepción viene a \textquote{posarse} sobre él (\emph{Jn} 1, 32-33; cf. \emph{Is} 11, 2). De él manará este Espíritu para toda la humanidad. En su bautismo, \textquote{se abrieron los cielos} (\emph{Mt} 3, 16) que el pecado de Adán había cerrado; y las aguas fueron santificadas por el descenso de Jesús y del Espíritu como preludio de la nueva creación.
	
	\n{537} Por el Bautismo, el cristiano se asimila sacramentalmente a Jesús que anticipa en su bautismo su muerte y su resurrección: debe entrar en este misterio de rebajamiento humilde y de arrepentimiento, descender al agua con Jesús, para subir con él, renacer del agua y del Espíritu para convertirse, en el Hijo, en hijo amado del Padre y \textquote{vivir una vida nueva} (\emph{Rm} 6, 4):
	
	\textquote{Enterrémonos con Cristo por el Bautismo, para resucitar con él; descendamos con él para ser ascendidos con él; ascendamos con él para ser glorificados con él} (San Gregorio Nacianceno, \emph{Oratio} 40, 9: PG 36, 369).
	
	\textquote{Todo lo que aconteció en Cristo nos enseña que después del baño de agua, el Espíritu Santo desciende sobre nosotros desde lo alto del cielo y que, adoptados por la Voz del Padre, llegamos a ser hijos de Dios}. (San Hilario de Poitiers, \emph{In evangelium Matthaei}, 2, 6: PL 9, 927).
	
	El Mesías esperado
	
	CEC 1286:
	
	\n{1286} En el Antiguo Testamento, los profetas anunciaron que el Espíritu del Señor reposaría sobre el Mesías esperado (cf. \emph{Is} 11,2) para realizar su misión salvífica (cf. \emph{Lc} 4,16-22; \emph{Is} 61,1). El descenso del Espíritu Santo sobre Jesús en su Bautismo por Juan fue el signo de que Él era el que debía venir, el Mesías, el Hijo de Dios (\emph{Mt} 3,13-17; \emph{Jn} 1,33- 34). Habiendo sido concedido por obra del Espíritu Santo, toda su vida y toda su misión se realizan en una comunión total con el Espíritu Santo que el Padre le da \textquote{sin medida} (\emph{Jn} 3,34).
	
	El Cordero de Dios que quita el pecado del mundo
	
	CEC 608:
	
	\n{608} Juan Bautista, después de haber aceptado bautizarle en compañía de los pecadores (cf. \emph{Lc} 3, 21; \emph{Mt} 3, 14-15), vio y señaló a Jesús como el \textquote{Cordero de Dios que quita los pecados del mundo} (\emph{Jn} 1, 29; cf. \emph{Jn} 1, 36). Manifestó así que Jesús es a la vez el Siervo doliente que se deja llevar en silencio al matadero (\emph{Is} 53, 7; cf. \emph{Jr} 11, 19) y carga con el pecado de las multitudes (cf. \emph{Is} 53, 12) y el cordero pascual símbolo de la redención de Israel cuando celebró la primera Pascua (\emph{Ex} 12, 3-14; cf. \emph{Jn} 19, 36; \emph{1 Co} 5, 7). Toda la vida de Cristo expresa su misión: \textquote{Servir y dar su vida en rescate por muchos} (\emph{Mc} 10, 45).
	
	El Hijo amado
	
	CEC 444:
	
	\n{444} Los evangelios narran en dos momentos solemnes, el Bautismo y la Transfiguración de Cristo, que la voz del Padre lo designa como su \textquote{Hijo amado} (\emph{Mt} 3, 17; 17, 5). Jesús se designa a sí mismo como \textquote{el Hijo Único de Dios} (\emph{Jn} 3, 16) y afirma mediante este título su preexistencia eterna (cf. \emph{Jn} 10, 36). Pide la fe en \textquote{el Nombre del Hijo Único de Dios} (\emph{Jn} 3, 18). Esta confesión cristiana aparece ya en la exclamación del centurión delante de Jesús en la cruz: \textquote{Verdaderamente este hombre era Hijo de Dios} (\emph{Mc} 15, 39), porque es solamente en el misterio pascual donde el creyente puede alcanzar el sentido pleno del título \textquote{Hijo de Dios}.
	
	Símbolos del Espíritu Santo en el Bautismo: agua, fuego y paloma
	
	CEC 694. 696. 701:
	
	\n{694} \emph{El agua}. El simbolismo del agua es significativo de la acción del Espíritu Santo en el Bautismo, ya que, después de la invocación del Espíritu Santo, ésta se convierte en el signo sacramental eficaz del nuevo nacimiento: del mismo modo que la gestación de nuestro primer nacimiento se hace en el agua, así el agua bautismal significa realmente que nuestro nacimiento a la vida divina se nos da en el Espíritu Santo. Pero \textquote{bautizados [\ldots{}] en un solo Espíritu}, también \textquote{hemos bebido de un solo Espíritu} (\emph{1 Co} 12, 13): el Espíritu es, pues, también personalmente el Agua viva que brota de Cristo crucificado (cf. Jn 19, 34; 1 Jn 5, 8) como de su manantial y que en nosotros brota en vida eterna (cf. \emph{Jn} 4, 10-14; 7, 38; \emph{Ex} 17, 1-6; \emph{Is} 55, 1; \emph{Za} 14, 8; \emph{1 Co} 10, 4; \emph{Ap} 21, 6; 22, 17).
	
	\n{696} \emph{El fuego}. Mientras que el agua significaba el nacimiento y la fecundidad de la vida dada en el Espíritu Santo, el fuego simboliza la energía transformadora de los actos del Espíritu Santo. El profeta Elías que \textquote{surgió [\ldots{}] como el fuego y cuya palabra abrasaba como antorcha} (\emph{Si} 48, 1), con su oración, atrajo el fuego del cielo sobre el sacrificio del monte Carmelo (cf. \emph{1 R} 18, 38-39), figura del fuego del Espíritu Santo que transforma lo que toca. Juan Bautista, \textquote{que precede al Señor con el espíritu y el poder de Elías} (\emph{Lc} 1, 17), anuncia a Cristo como el que \textquote{bautizará en el Espíritu Santo y el fuego} (\emph{Lc} 3, 16), Espíritu del cual Jesús dirá: \textquote{He venido a traer fuego sobre la tierra y ¡cuánto desearía que ya estuviese encendido!} (\emph{Lc} 12, 49). En forma de lenguas \textquote{como de fuego} se posó el Espíritu Santo sobre los discípulos la mañana de Pentecostés y los llenó de él (\emph{Hch} 2, 3-4). La tradición espiritual conservará este simbolismo del fuego como uno de los más expresivos de la acción del Espíritu Santo (cf. San Juan de la Cruz, \emph{Llama de amor viva}). \textquote{No extingáis el Espíritu} (\emph{1 Ts} 5, 19).
	
	\n{701} \emph{La paloma}. Al final del diluvio (cuyo simbolismo se refiere al Bautismo), la paloma soltada por Noé vuelve con una rama tierna de olivo en el pico, signo de que la tierra es habitable de nuevo (cf. \emph{Gn} 8, 8-12). Cuando Cristo sale del agua de su bautismo, el Espíritu Santo, en forma de paloma, baja y se posa sobre él (cf. \emph{Mt} 3, 16 paralelos). El Espíritu desciende y reposa en el corazón purificado de los bautizados. En algunos templos, la Santa Reserva eucarística se conserva en un receptáculo metálico en forma de paloma (el \emph{columbarium}), suspendido por encima del altar. El símbolo de la paloma para sugerir al Espíritu Santo es tradicional en la iconografía cristiana.
	
	El Bautismo de Cristo y el Bautismo de los cristianos
	
	CEC 1223-1225:
	
	\n{1223} Todas las prefiguraciones de la Antigua Alianza culminan en Cristo Jesús. Comienza su vida pública después de hacerse bautizar por san Juan el Bautista en el Jordán (cf. \emph{Mt} 3,13) y, después de su Resurrección, confiere esta misión a sus Apóstoles: \textquote{Id, pues, y haced discípulos a todas las gentes bautizándolas en el nombre del Padre y del Hijo y del Espíritu Santo, y enseñándoles a guardar todo lo que yo os he mandado} (\emph{Mt} 28,19-20; cf. \emph{Mc} 16,15-16).
	
	\n{1224} Nuestro Señor se sometió voluntariamente al Bautismo de san Juan, destinado a los pecadores, para \textquote{cumplir toda justicia} (\emph{Mt} 3,15). Este gesto de Jesús es una manifestación de su \textquote{anonadamiento} (\emph{Flp} 2,7). El Espíritu que se cernía sobre las aguas de la primera creación desciende entonces sobre Cristo, como preludio de la nueva creación, y el Padre manifiesta a Jesús como su \textquote{Hijo amado} (\emph{Mt} 3,16-17).
	
	\n{1225} En su Pascua, Cristo abrió a todos los hombres las fuentes del Bautismo. En efecto, había hablado ya de su pasión que iba a sufrir en Jerusalén como de un \textquote{Bautismo} con que debía ser bautizado (\emph{Mc} 10,38; cf. \emph{Lc} 12,50). La sangre y el agua que brotaron del costado traspasado de Jesús crucificado (cf. \emph{Jn} 19,34) son figuras del Bautismo y de la Eucaristía, sacramentos de la vida nueva (cf. \emph{1 Jn} 5,6-8): desde entonces, es posible \textquote{nacer del agua y del Espíritu} para entrar en el Reino de Dios (\emph{Jn} 3,5).
	
	\textquote{Considera dónde eres bautizado, de dónde viene el Bautismo: de la cruz de Cristo, de la muerte de Cristo. Ahí está todo el misterio: Él padeció por ti. En él eres rescatado, en él eres salvado}. (San Ambrosio, \emph{De sacramentis} 2, 2, 6).
	
	Frutos del Bautismo
	
	CEC 1262-1266
	
	\ccesec{La gracia del Bautismo}
	
	\n{1262} Los distintos efectos del Bautismo son significados por los elementos sensibles del rito sacramental. La inmersión en el agua evoca los simbolismos de la muerte y de la purificación, pero también los de la regeneración y de la renovación. Los dos efectos principales, por tanto, son la purificación de los pecados y el nuevo nacimiento en el Espíritu Santo (cf. \emph{Hch}2,38; \emph{Jn} 3,5).
	
	\n{1263 Para la remisión de los pecados\ldots{}}
	
	Por el Bautismo, \emph{todos los pecados} son perdonados, el pecado original y todos los pecados personales así como todas las penas del pecado (cf. DS 1316). En efecto, en los que han sido regenerados no permanece nada que les impida entrar en el Reino de Dios, ni el pecado de Adán, ni el pecado personal, ni las consecuencias del pecado, la más grave de las cuales es la separación de Dios.
	
	\n{1264} No obstante, en el bautizado permanecen ciertas consecuencias temporales del pecado, como los sufrimientos, la enfermedad, la muerte o las fragilidades inherentes a la vida como las debilidades de carácter, etc., así como una inclinación al pecado que la Tradición llama \emph{concupiscencia}, o metafóricamente \emph{fomes peccati}: \textquote{La concupiscencia, dejada para el combate, no puede dañar a los que no la consienten y la resisten con coraje por la gracia de Jesucristo. Antes bien \textquote{el que legítimamente luchare, será coronado} (\emph{2 Tm} 2,5)} (Concilio de Trento: DS 1515).
	
	\ccesec{\textquote{Una criatura nueva}}
	
	\n{1265} El Bautismo no solamente purifica de todos los pecados, hace también del neófito \textquote{una nueva creatura} (\emph{2 Co} 5,17), un hijo adoptivo de Dios (cf. \emph{Ga} 4,5-7) que ha sido hecho \textquote{partícipe de la naturaleza divina} (\emph{2 P} 1,4), miembro de Cristo (cf. \emph{1 Co} 6,15; 12,27), coheredero con Él (\emph{Rm} 8,17) y templo del Espíritu Santo (cf. \emph{1 Co} 6,19).
	
	\n{1266} La Santísima Trinidad da al bautizado \emph{la gracia santificante, la gracia de la justificación} que :
	
	--- le hace capaz de creer en Dios, de esperar en Él y de amarlo mediante las \emph{virtudes teologales};
	
	--- le concede poder vivir y obrar bajo la moción del Espíritu Santo mediante los \emph{dones del Espíritu Santo};
	
	--- le permite crecer en el bien mediante las \emph{virtudes morales}.
	
	Así todo el organismo de la vida sobrenatural del cristiano tiene su raíz en el santo Bautismo.
	
	Bautismo y Transfiguración
	
	CEC 556:
	
	\n{556} En el umbral de la vida pública se sitúa el Bautismo; en el de la Pascua, la Transfiguración. Por el bautismo de Jesús \textquote{fue manifestado el misterio de la primera regeneración}: nuestro Bautismo; la Transfiguración \textquote{es el sacramento de la segunda regeneración}: nuestra propia resurrección (Santo Tomás de Aquino, \emph{S.Th}., 3, q. 45, a. 4, ad 2). Desde ahora nosotros participamos en la Resurrección del Señor por el Espíritu Santo que actúa en los sacramentos del Cuerpo de Cristo. La Transfiguración nos concede una visión anticipada de la gloriosa venida de Cristo \textquote{el cual transfigurará este miserable cuerpo nuestro en un cuerpo glorioso como el suyo} (\emph{Flp} 3, 21). Pero ella nos recuerda también que \textquote{es necesario que pasemos por muchas tribulaciones para entrar en el Reino de Dios} (\emph{Hch} 14, 22):
	
	\textquote{Pedro no había comprendido eso cuando deseaba vivir con Cristo en la montaña (cf. \emph{Lc} 9, 33). Te ha reservado eso, oh Pedro, para después de la muerte. Pero ahora, él mismo dice: Desciende para penar en la tierra, para servir en la tierra, para ser despreciado y crucificado en la tierra. La Vida desciende para hacerse matar; el Pan desciende para tener hambre; el Camino desciende para fatigarse andando; la Fuente desciende para sentir la sed; y tú, ¿vas a negarte a sufrir?} (San Agustín, \emph{Sermo}, 78, 6: PL 38, 492-493).