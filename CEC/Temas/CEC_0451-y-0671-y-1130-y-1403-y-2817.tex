\cceth{¡Ven, Señor Jesús!}

\cceref{CEC 451, 671, 1130, 1403, 2817}

\begin{ccebody}
	
	\n{451} La oración cristiana está marcada por el título \textquote{Señor}, ya sea en la invitación a la oración \textquote{el Señor esté con vosotros}, o en su conclusión \textquote{por Jesucristo nuestro Señor} o incluso en la exclamación llena de confianza y de esperanza: \emph{Maran atha} (\textquote{¡el Señor viene!}) o \emph{Marana tha} (\textquote{¡Ven, Señor!}) (\emph{1 Co} 16, 22): \textquote{¡Amén! ¡ven, Señor Jesús!} (\emph{Ap} 22, 20).


	\ccesec{\ldots{} esperando que todo le sea sometido}

	\n{671} El Reino de Cristo, presente ya en su Iglesia, sin embargo, no está todavía acabado \textquote{con gran poder y gloria} (\emph{Lc} 21, 27; cf. \emph{Mt} 25, 31) con el advenimiento del Rey a la tierra. Este Reino aún es objeto de los ataques de los poderes del mal (cf. \emph{2 Ts} 2, 7), a pesar de que estos poderes hayan sido vencidos en su raíz por la Pascua de Cristo. Hasta que todo le haya sido sometido (cf. \emph{1 Co} 15, 28), y \textquote{mientras no [\ldots{}] haya nuevos cielos y nueva tierra, en los que habite la justicia, la Iglesia peregrina lleva en sus sacramentos e instituciones, que pertenecen a este tiempo, la imagen de este mundo que pasa. Ella misma vive entre las criaturas que gimen en dolores de parto hasta ahora y que esperan la manifestación de los hijos de Dios} (LG 48). Por esta razón los cristianos piden, sobre todo en la Eucaristía (cf. \emph{1 Co} 11, 26), que se apresure el retorno de Cristo (cf. \emph{2 P} 3, 11-12) cuando suplican: \textquote{Ven, Señor Jesús} (\emph{Ap} 22, 20; cf. \emph{1 Co} 16, 22; \emph{Ap} 22, 17-20).

	\ccesec{Sacramentos de la vida eterna}

	\n{1130} La Iglesia celebra el Misterio de su Señor \textquote{hasta que él venga} y \textquote{Dios sea todo en todos} (\emph{1 Co} 11, 26; 15, 28). Desde la era apostólica, la liturgia es atraída hacia su término por el gemido del Espíritu en la Iglesia: \emph{¡Marana tha!} (\emph{1 Co} 16,22). La liturgia participa así en el deseo de Jesús: \textquote{Con ansia he deseado comer esta Pascua con vosotros [\ldots{}] hasta que halle su cumplimiento en el Reino de Dios} (\emph{Lc} 22,15-16). En los sacramentos de Cristo, la Iglesia recibe ya las arras de su herencia, participa ya en la vida eterna, aunque \textquote{aguardando la feliz esperanza y la manifestación de la gloria del Gran Dios y Salvador nuestro Jesucristo} (\emph{Tt} 2,13). \textquote{El Espíritu y la Esposa dicen: ¡Ven! [\ldots{}] ¡Ven, Señor Jesús!} (\emph{Ap} 22,17.20).

	\begin{quote} 
		
		Santo Tomás resume así las diferentes dimensiones del signo sacramental: \textquote{\emph{Unde sacramentum est signum rememorativum eius quod praecessit, scilicet passionis Christi; et desmonstrativum eius quod in nobis efficitur per Christi passionem, scilicet gratiae; et prognosticum, id est, praenuntiativum futurae gloriae}} (\textquote{Por eso el sacramento es un signo que rememora lo que sucedió, es decir, la pasión de Cristo; es un signo que demuestra lo que se realiza en nosotros en virtud de la pasión de Cristo, es decir, la gracia; y es un signo que anticipa, es decir, que preanuncia la gloria venidera}) (\emph{Summa theologiae} 3, q. 60, a. 3, c.) 
		
	\end{quote}

	\n{1403} En la última Cena, el Señor mismo atrajo la atención de sus discípulos hacia el cumplimiento de la Pascua en el Reino de Dios: \textquote{Y os digo que desde ahora no beberé de este fruto de la vid hasta el día en que lo beba con vosotros, de nuevo, en el Reino de mi Padre} (\emph{Mt} 26,29; cf. \emph{Lc} 22,18; \emph{Mc} 14,25). Cada vez que la Iglesia celebra la Eucaristía recuerda esta promesa y su mirada se dirige hacia \textquote{el que viene} (\emph{Ap} 1,4). En su oración, implora su venida: \emph{Marana tha} (\emph{1 Co} 16,22), \textquote{Ven, Señor Jesús} (\emph{Ap} 22,20), \textquote{que tu gracia venga y que este mundo pase} (\emph{Didaché} 10,6).

	\ccesec{Venga a nosotros tu Reino}

	\n{2817} Esta petición es el \emph{Marana Tha}, el grito del Espíritu y de la Esposa: \textquote{Ven, Señor Jesús}:

	\begin{quote} 
		
		\textquote{Incluso aunque esta oración no nos hubiera mandado pedir el advenimiento del Reino, habríamos tenido que expresar esta petición, dirigiéndonos con premura a la meta de nuestras esperanzas. Las almas de los mártires, bajo el altar, invocan al Señor con grandes gritos: \textquote{¿Hasta cuándo, Dueño santo y veraz, vas a estar sin hacer justicia por nuestra sangre a los habitantes de la tierra?} (\emph{Ap} 6, 10). En efecto, los mártires deben alcanzar la justicia al fin de los tiempos. Señor, ¡apresura, pues, la venida de tu Reino!} (Tertuliano, \emph{De oratione}, 5, 2-4). 
	
	\end{quote}

\end{ccebody}