\cceth{Tribulación final y venida de Cristo en gloria}

\cceref{CEC 668-677, 769}

\begin{ccebody}
	\n{668} \textquote{Cristo murió y volvió a la vida para eso, para ser Señor de muertos y vivos} (\emph{Rm} 14, 9). La Ascensión de Cristo al Cielo significa su participación, en su humanidad, en el poder y en la autoridad de Dios mismo. Jesucristo es Señor: posee todo poder en los cielos y en la tierra. El está \textquote{por encima de todo principado, potestad, virtud, dominación} porque el Padre \textquote{bajo sus pies sometió todas las cosas} (\emph{Ef} 1, 20-22). Cristo es el Señor del cosmos (cf. \emph{Ef} 4, 10; \emph{1 Co} 15, 24. 27-28) y de la historia. En Él, la historia de la humanidad e incluso toda la Creación encuentran su recapitulación (\emph{Ef} 1, 10), su cumplimiento transcendente.
	
	\n{669} Como Señor, Cristo es también la cabeza de la Iglesia que es su Cuerpo (cf. \emph{Ef} 1, 22). Elevado al cielo y glorificado, habiendo cumplido así su misión, permanece en la tierra en su Iglesia. La Redención es la fuente de la autoridad que Cristo, en virtud del Espíritu Santo, ejerce sobre la Iglesia (cf. \emph{Ef} 4, 11-13). \textquote{La Iglesia, o el reino de Cristo presente ya en misterio} (LG 3), \textquote{constituye el germen y el comienzo de este Reino en la tierra} (LG 5).
	
	\n{670} Desde la Ascensión, el designio de Dios ha entrado en su consumación. Estamos ya en la \textquote{última hora} (\emph{1 Jn} 2, 18; cf. \emph{1 P} 4, 7). \textquote{El final de la historia ha llegado ya a nosotros y la renovación del mundo está ya decidida de manera irrevocable e incluso de alguna manera real está ya por anticipado en este mundo. La Iglesia, en efecto, ya en la tierra, se caracteriza por una verdadera santidad, aunque todavía imperfecta} (LG 48). El Reino de Cristo manifiesta ya su presencia por los signos milagrosos (cf. \emph{Mc} 16, 17-18) que acompañan a su anuncio por la Iglesia (cf. \emph{Mc} 16, 20).
	
	\ccesec{\ldots{} esperando que todo le sea sometido}
	
	\n{671} El Reino de Cristo, presente ya en su Iglesia, sin embargo, no está todavía acabado \textquote{con gran poder y gloria} (\emph{Lc} 21, 27; cf. \emph{Mt} 25, 31) con el advenimiento del Rey a la tierra. Este Reino aún es objeto de los ataques de los poderes del mal (cf. \emph{2 Ts} 2, 7), a pesar de que estos poderes hayan sido vencidos en su raíz por la Pascua de Cristo. Hasta que todo le haya sido sometido (cf. \emph{1 Co} 15, 28), y \textquote{mientras no [\ldots{}] haya nuevos cielos y nueva tierra, en los que habite la justicia, la Iglesia peregrina lleva en sus sacramentos e instituciones, que pertenecen a este tiempo, la imagen de este mundo que pasa. Ella misma vive entre las criaturas que gimen en dolores de parto hasta ahora y que esperan la manifestación de los hijos de Dios} (LG 48). Por esta razón los cristianos piden, sobre todo en la Eucaristía (cf. \emph{1 Co} 11, 26), que se apresure el retorno de Cristo (cf. \emph{2 P} 3, 11-12) cuando suplican: \textquote{Ven, Señor Jesús} (\emph{Ap} 22, 20; cf. \emph{1 Co} 16, 22; \emph{Ap} 22, 17-20).
	
	\n{672} Cristo afirmó antes de su Ascensión que aún no era la hora del establecimiento glorioso del Reino mesiánico esperado por Israel (cf. \emph{Hch} 1, 6-7) que, según los profetas (cf. \emph{Is} 11, 1-9), debía traer a todos los hombres el orden definitivo de la justicia, del amor y de la paz. El tiempo presente, según el Señor, es el tiempo del Espíritu y del testimonio (cf. \emph{Hch} 1, 8), pero es también un tiempo marcado todavía por la \textquote{tribulación} (\emph{1 Co} 7, 26) y la prueba del mal (cf. \emph{Ef} 5, 16) que afecta también a la Iglesia (cf. \emph{1 P} 4, 17) e inaugura los combates de los últimos días (\emph{1 Jn} 2, 18; 4, 3; \emph{1 Tm} 4, 1). Es un tiempo de espera y de vigilia (cf. \emph{Mt} 25, 1-13; \emph{Mc} 13, 33-37).
	
	\ccesec{El glorioso advenimiento de Cristo, esperanza de Israel}
	
	\n{673} Desde la Ascensión, el advenimiento de Cristo en la gloria es inminente (cf. \emph{Ap} 22, 20) aun cuando a nosotros no nos \textquote{toca conocer el tiempo y el momento que ha fijado el Padre con su autoridad} (\emph{Hch} 1, 7; cf. \emph{Mc} 13, 32). Este acontecimiento escatológico se puede cumplir en cualquier momento (cf. \emph{Mt} 24, 44: \emph{1 Ts} 5, 2), aunque tal acontecimiento y la prueba final que le ha de preceder estén \textquote{retenidos} en las manos de Dios (cf. \emph{2 Ts} 2, 3-12).
	
	\n{674} La venida del Mesías glorioso, en un momento determinado de la historia (cf. \emph{Rm} 11, 31), se vincula al reconocimiento del Mesías por \textquote{todo Israel} (\emph{Rm} 11, 26; \emph{Mt} 23, 39) del que \textquote{una parte está endurecida} (\emph{Rm} 11, 25) en \textquote{la incredulidad} (\emph{Rm} 11, 20) respecto a Jesús. San Pedro dice a los judíos de Jerusalén después de Pentecostés: \textquote{Arrepentíos, pues, y convertíos para que vuestros pecados sean borrados, a fin de que del Señor venga el tiempo de la consolación y envíe al Cristo que os había sido destinado, a Jesús, a quien debe retener el cielo hasta el tiempo de la restauración universal, de que Dios habló por boca de sus profetas} (\emph{Hch} 3, 19-21). Y san Pablo le hace eco: \textquote{si su reprobación ha sido la reconciliación del mundo ¿qué será su readmisión sino una resurrección de entre los muertos?} (\emph{Rm} 11, 5). La entrada de \textquote{la plenitud de los judíos} (\emph{Rm} 11, 12) en la salvación mesiánica, a continuación de \textquote{la plenitud de los gentiles} (Rm 11, 25; cf. Lc 21, 24), hará al pueblo de Dios \textquote{llegar a la plenitud de Cristo} (\emph{Ef} 4, 13) en la cual \textquote{Dios será todo en nosotros} (\emph{1 Co} 15, 28).
	
	\ccesec{La última prueba de la Iglesia}
	
	\n{675} Antes del advenimiento de Cristo, la Iglesia deberá pasar por una prueba final que sacudirá la fe de numerosos creyentes (cf. \emph{Lc} 18, 8; \emph{Mt} 24, 12). La persecución que acompaña a su peregrinación sobre la tierra (cf. \emph{Lc} 21, 12; \emph{Jn} 15, 19-20) desvelará el \textquote{misterio de iniquidad} bajo la forma de una impostura religiosa que proporcionará a los hombres una solución aparente a sus problemas mediante el precio de la apostasía de la verdad. La impostura religiosa suprema es la del Anticristo, es decir, la de un seudo-mesianismo en que el hombre se glorifica a sí mismo colocándose en el lugar de Dios y de su Mesías venido en la carne (cf. \emph{2 Ts} 2, 4-12; \emph{1Ts} 5, 2-3;2 \emph{Jn} 7; \emph{1 Jn} 2, 18.22).
	
	\n{676} Esta impostura del Anticristo aparece esbozada ya en el mundo cada vez que se pretende llevar a cabo la esperanza mesiánica en la historia, lo cual no puede alcanzarse sino más allá del tiempo histórico a través del juicio escatológico: incluso en su forma mitigada, la Iglesia ha rechazado esta falsificación del Reino futuro con el nombre de milenarismo (cf. DS 3839), sobre todo bajo la forma política de un mesianismo secularizado, \textquote{intrínsecamente perverso} (cf. Pío XI, carta enc. \emph{Divini Redemptoris}, condenando \textquote{los errores presentados bajo un falso sentido místico de esta especie de falseada redención de los más humildes}; GS 20-21).
	
	\n{677} La Iglesia sólo entrará en la gloria del Reino a través de esta última Pascua en la que seguirá a su Señor en su muerte y su Resurrección (cf. \emph{Ap} 19, 1-9). El Reino no se realizará, por tanto, mediante un triunfo histórico de la Iglesia (cf. \emph{Ap} 13, 8) en forma de un proceso creciente, sino por una victoria de Dios sobre el último desencadenamiento del mal (cf. \emph{Ap} 20, 7-10) que hará descender desde el cielo a su Esposa (cf. \emph{Ap} 21, 2-4). El triunfo de Dios sobre la rebelión del mal tomará la forma de Juicio final (cf. \emph{Ap} 20, 12) después de la última sacudida cósmica de este mundo que pasa (cf. \emph{2 P} 3, 12-13).
	
	\ccesec{La Iglesia, consumada en la gloria}
	
	\n{769} La Iglesia \textquote{sólo llegará a su perfección en la gloria del cielo} (LG 48), cuando Cristo vuelva glorioso. Hasta ese día, \textquote{la Iglesia avanza en su peregrinación a través de las persecuciones del mundo y de los consuelos de Dios} (San Agustín, \emph{De civitate Dei} 18, 51; cf. LG 8). Aquí abajo, ella se sabe en exilio, lejos del Señor (cf. \emph{2Co} 5, 6; LG 6), y aspira al advenimiento pleno del Reino, \textquote{y espera y desea con todas sus fuerzas reunirse con su Rey en la gloria} (LG 5). La consumación de la Iglesia en la gloria, y a través de ella la del mundo, no sucederá sin grandes pruebas. Solamente entonces, \textquote{todos los justos descendientes de Adán, \textquote{desde Abel el justo hasta el último de los elegidos} se reunirán con el Padre en la Iglesia universal} (LG 2).
	
\end{ccebody}