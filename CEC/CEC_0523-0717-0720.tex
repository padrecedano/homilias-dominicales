\cceth{La misión de Juan Bautista}

\cceref{CEC 523, 717-720}

\begin{ccebody}
	
	\n{523} \emph{San Juan Bautista} es el precursor (cf. \emph{Hch} 13, 24) inmediato del Señor, enviado para prepararle el camino (cf. \emph{Mt} 3, 3). \textquote{Profeta del Altísimo} (\emph{Lc} 1, 76), sobrepasa a todos los profetas (cf. \emph{Lc} 7, 26), de los que es el último (cf. \emph{Mt} 11, 13), e inaugura el Evangelio (cf. \emph{Hch} 1, 22; \emph{Lc} 16,16); desde el seno de su madre (cf. \emph{Lc} 1,41) saluda la venida de Cristo y encuentra su alegría en ser \textquote{el amigo del esposo} (\emph{Jn} 3, 29) a quien señala como \textquote{el Cordero de Dios que quita el pecado del mundo} (\emph{Jn} 1, 29). Precediendo a Jesús \textquote{con el espíritu y el poder de Elías} (\emph{Lc} 1, 17), da testimonio de él mediante su predicación, su bautismo de conversión y finalmente con su martirio (cf. \emph{Mc} 6, 17-29).
	
	\ccesec{El Espíritu de Cristo en la plenitud de los tiempos. Juan, Precursor, Profeta y Bautista}
	
	\n{717} \textquote{Hubo un hombre, enviado por Dios, que se llamaba Juan} (\emph{Jn} 1, 6). Juan fue \textquote{lleno del Espíritu Santo ya desde el seno de su madre} (\emph{Lc} 1, 15. 41) por obra del mismo Cristo que la Virgen María acababa de concebir del Espíritu Santo. La \textquote{Visitación} de María a Isabel se convirtió así en \textquote{visita de Dios a su pueblo} (\emph{Lc} 1, 68).
	
	\n{718} Juan es \textquote{Elías que debe venir} (\emph{Mt} 17, 10-13): El fuego del Espíritu lo habita y le hace correr delante {[}como \textquote{precursor}{]} del Señor que viene. En Juan el Precursor, el Espíritu Santo culmina la obra de \textquote{preparar al Señor un pueblo bien dispuesto} (\emph{Lc} 1, 17).
	
	\n{719} Juan es \textquote{más que un profeta} (\emph{Lc} 7, 26). En él, el Espíritu Santo consuma el \textquote{hablar por los profetas}. Juan termina el ciclo de los profetas inaugurado por Elías (cf. \emph{Mt} 11, 13-14). Anuncia la inminencia de la consolación de Israel, es la \textquote{voz} del Consolador que llega (\emph{Jn} 1, 23; cf. \emph{Is} 40, 1-3). Como lo hará el Espíritu de Verdad, \textquote{vino como testigo para dar testimonio de la luz} (\emph{Jn} 1, 7; cf. \emph{Jn} 15, 26; 5, 33). Con respecto a Juan, el Espíritu colma así las \textquote{indagaciones de los profetas} y la ansiedad de los ángeles (\emph{1 P} 1, 10-12): \textquote{Aquél sobre quien veas que baja el Espíritu y se queda sobre él, ése es el que bautiza con el Espíritu Santo. Y yo lo he visto y doy testimonio de que éste es el Hijo de Dios [\ldots{}] He ahí el Cordero de Dios} (\emph{Jn} 1, 33-36).
	
	\n{720} En fin, con Juan Bautista, el Espíritu Santo, inaugura, prefigurándolo, lo que realizará con y en Cristo: volver a dar al hombre la \textquote{semejanza} divina. El bautismo de Juan era para el arrepentimiento, el del agua y del Espíritu será un nuevo nacimiento (cf. \emph{Jn} 3, 5).
	
\end{ccebody}