\chapter{Introducción General}

\begin{bodyintro}Durante más de veinte años me he dedicado a recopilar las homilías litúrgicas y comentarios a los Evangelios de los Padres de la Iglesia y de los últimos sucesores de Pedro. Ha sido un trabajo paciente que ha supuesto no solamente recopilar, sino también traducir, ordenar, clasificar, corregir… porque muchas de las fuentes de las que provenía el texto eran antiguas y los errores (tipográficos y ortográficos) varios.\end{bodyintro}

\begin{bodyintro}Esta obra recoge todo ese trabajo y también alguna novedad. El contenido se ha organizado de acuerdo a la siguiente estructura:\end{bodyintro}

\begin{bodyintro}1. Las \textbf{lecturas bíblicas} a partir de los textos de la \textit{Sagrada Biblia, versión oficial de la Conferencia Episcopal Española}\anote{id1};\end{bodyintro}

\begin{bodyintro}2. Uno o varios \textbf{comentarios patrísticos} relativos al Evangelio o a la fiesta del día;\end{bodyintro}

\begin{bodyintro}3. Varias \textbf{homilías }(u otros escritos en su defecto), generalmente de los últimos papas o de los padres de la Iglesia, para la celebración correspondiente;\end{bodyintro}

\begin{bodyintro}4. Los \textbf{temas doctrinales} que sugiere el \textit{Directorio Homilético} para esa celebración, acompañados de los textos del \textit{Catecismo de la Iglesia Católica} indicados por ese mismo \textit{Directorio}.\end{bodyintro}

\begin{bodyintro}Son cuatro pilares sobre los que podremos apoyarnos en la preparación de las homilías, sabiendo que cada contexto es distinto y que cada predicación, también la nuestra, es una obra de arte en la que se mezcla nuestra criatura de barro con la asistencia consoladora del Espíritu Santo.\end{bodyintro}
\newpage
\section {Plan de la Colección}


\begin{bodyintro}Las \textbf{homilías dominicales} estarán agrupadas por ciclo litúrgico (A, B, y C) y dentro de cada ciclo cinco volúmenes (cada uno con unas 400 páginas). La intención es recoger en una obra todas las homilías de los útimos pontífices de una forma exhaustiva (al menos a partir de Juan Pablo II).\end{bodyintro}

\begin{bodyintro}La distribución es la siguiente:\end{bodyintro}

\begin{itemize}
	\item \textbf {Homilías Dominicales (A)}
	\begin{enumerate}
		\renewcommand{\labelenumii}{\arabic{enumii}.}
		\item Adviento-Navidad
		\item Cuaresma-Triduo Pascual
		\item Pascua
		\item Tiempo Ordinario (Semanas II-XVII) *
		\item Tiempo Ordinario (Semanas XVIII-XXXIV)
	\end{enumerate}
\end{itemize}

\begin{itemize}
	\item \textbf {Homilías Dominicales (B)}
	\begin{enumerate}
		\renewcommand{\labelenumii}{\arabic{enumii}.}
		\item Adviento-Navidad
		\item Cuaresma-Triduo Pascual
		\item Pascua
		\item Tiempo Ordinario (Semanas II-XVII) *
		\item Tiempo Ordinario (Semanas XVIII-XXXIV)
	\end{enumerate}
\end{itemize}

\begin{itemize}
	\item \textbf {Homilías Dominicales (C)}
	\begin{enumerate}
		\renewcommand{\labelenumii}{\arabic{enumii}.}
		\item Adviento-Navidad
		\item Cuaresma-Triduo Pascual
		\item Pascua
		\item Tiempo Ordinario (Semanas II-XVII) *
		\item Tiempo Ordinario (Semanas XVIII-XXXIV)
	\end{enumerate}
\end{itemize}


\begin{bodyintro}\small *Incluye también las tres celebraciones del Señor durante el Tiempo Ordinario (Santísima Trinidad, Corpus Christi y Sagrado Corazón de Jesús).\end{bodyintro}
