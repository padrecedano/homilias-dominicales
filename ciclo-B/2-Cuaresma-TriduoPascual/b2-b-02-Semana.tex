\chapter{Domingo II de Cuaresma (B)}

\section{Lecturas}

\rtitle{PRIMERA LECTURA}

\rbook{Del libro del Génesis} \rred{22, 1-2. 9a. 10-13. 15-18}

\rtheme{El sacrificio de Abrahán, nuestro padre en la fe}

\begin{scripture}
En aquellos días, Dios puso a prueba a Abrahán.

Le dijo:

\>{¡Abrahán!}.

Él respondió:

\>{Aquí estoy}.

Dios dijo:

\>{Toma a tu hijo único, al que amas, a Isaac, y vete a la tierra de Moria y ofrécemelo allí en holocausto en uno de los montes que yo te indicaré}.

Cuando llegaron al sitio que le había dicho Dios, Abrahán levantó allí el altar y apiló la leña, luego ató a su hijo Isaac y lo puso sobre el altar, encima de la leña.

Entonces Abrahán alargó la mano y tomó el cuchillo para degollar a su hijo.

Pero el ángel del Señor le gritó desde el cielo:

\>{¡Abrahán, Abrahán!}.

Él contestó:

\>{Aquí estoy}.

El ángel le ordenó:

\>{No alargues la mano contra el muchacho ni le hagas nada. \\Ahora he comprobado que temes a Dios, porque no te has reservado a tu hijo, a tu único hijo}.


Abrahán levantó los ojos y vio un carnero enredado por los cuernos en la maleza. Se acercó, tomó el carnero y lo ofreció en holocausto en lugar de su hijo.

El ángel del Señor llamó a Abrahán por segunda vez desde el cielo y le dijo:

\>{Juro por mí mismo, oráculo del Señor: por haber hecho esto, por no haberte reservado tu hijo, tu hijo único, te colmaré de bendiciones y multiplicaré a tus descendientes como las estrellas del cielo y como la arena de la playa. Tus descendientes conquistarán las puertas de sus enemigos. Todas las naciones de la tierra se bendecirán con tu descendencia, porque has escuchado mi voz}.
\end{scripture}

\rtitle{SALMO RESPONSORIAL}

\rbook{Salmo} \rred{115, 10 y 15. 16-17. 18-19}

\rtheme{Caminaré en presencia del Señor en el país de los vivos}

\begin{psbody}
Tenía fe, aun cuando dije:
\textquote{¡Qué desgraciado soy!}.
Mucho le cuesta al Señor
la muerte de sus fieles. 

Señor, yo soy tu siervo,
siervo tuyo, hijo de tu esclava:
rompiste mis cadenas.

Te ofreceré un sacrificio de alabanza,
invocando tu nombre, Señor. 

Cumpliré al Señor mis votos
en presencia de todo el pueblo,
en el atrio de la casa del Señor,
en medio de ti, Jerusalén. 
\end{psbody}

\rtitle{SEGUNDA LECTURA}

\rbook{De la carta del apóstol san Pablo a los Romanos} \rred{8, 31b-34}

\rtheme{Dios no se reservó a su propio Hijo}

\begin{scripture}
Hermanos:

Si Dios está con nosotros, ¿quién estará contra nosotros?

El que no se reservó a su propio Hijo, sino que lo entregó por todos nosotros, ¿cómo no nos dará todo con él? ¿Quién acusará a los elegidos de Dios? Dios es el que justifica. ¿Quién condenará? ¿Acaso Cristo Jesús, que murió, más todavía, resucitó y está a la derecha de Dios y que además intercede por nosotros?
\end{scripture}

\rtitle{EVANGELIO}

\rbook{Del Santo Evangelio según san Marcos} \rred{9, 2-10}

\rtheme{Este es mi Hijo, el amado}

\begin{scripture}
En aquel tiempo, Jesús tomó consigo a Pedro, a Santiago y a Juan, sube aparte con ellos solos a un monte alto, y se transfiguró delante de ellos. Sus vestidos se volvieron de un blanco deslumbrador, como no puede dejarlos ningún batanero del mundo.

Se les aparecieron Elías y Moisés, conversando con Jesús.

Entonces Pedro tomó la palabra y dijo a Jesús:

\>{Maestro, ¡qué bueno es que estemos aquí! Vamos a hacer tres tiendas, una para ti, otra para Moisés y otra para Elías}.

No sabía qué decir, pues estaban asustados.

Se formó una nube que los cubrió y salió una voz de la nube:

\>{Este es mi Hijo, el amado; escuchadlo}.

De pronto, al mirar alrededor, no vieron a nadie más que a Jesús, solo con ellos.

Cuando bajaban del monte, les ordenó que no contasen a nadie lo que habían visto hasta que el Hijo del hombre resucitara de entre los muertos.

Esto se les quedó grabado y discutían qué quería decir aquello de resucitar de entre los muertos.
\end{scripture}

\begin{patercite}
En la Eucaristía Jesús nos da, bajo las especies del pan y del vino, su carne vivificada por el Espíritu Santo y vivificadora de nuestra carne con el fin de hacernos participar con todo nuestro ser, espíritu y cuerpo, en su resurrección y en su condición de gloria. A este respecto, san Ireneo de Lyon enseña: \textquote{Porque de la misma manera que el pan, que proviene de la tierra, después de recibir la invocación de Dios, ya no es un pan ordinario, sino la Eucaristía, constituida de dos cosas: una celeste, otra terrestre, así nuestros cuerpos, al recibir la Eucaristía ya no son corruptibles, puesto que tienen la esperanza de la resurrección} (\textit{Adversus haereses}, IV, 18, 4-5).


(\ldots) El credo cristiano (...) culmina en la proclamación de la resurrección de los muertos al fin de los tiempos, y en la vida eterna» (Catecismo de la Iglesia católica, n. 988). Con la encarnación el Verbo de Dios asumió la carne humana (cf. Jn 1, 14), haciéndola partícipe, por su muerte y resurrección, de su misma gloria de Unigénito del Padre. Mediante los dones del Espíritu y de la carne de Cristo glorificada en la Eucaristía, Dios Padre infunde en todo el ser del hombre y, en cierto modo, en el cosmos mismo el deseo de ese destino.

\textbf{San Juan Pablo II, papa}, \textit{Catequesis}, Audiencia general, 4 de noviembre de 1998, cf. nn. 4-5. 
\end{patercite}

\newsection
\section{Comentarios Patrísticos}

\subsection{San Cirilo de Alejandría, obispo}

\ptheme{Hablaban de la muerte que Jesús iba a consumar en Jerusalén}

\src{ Homilía 9 en la transfiguración del Señor: \\PG 77, 1011-1014.}

\begin{body}
\ltr{J}{esús} subió a una montaña con sus tres discípulos preferidos. Allí se transfiguró en un resplandor tan extraordinario y divino, que su vestido parecía hecho de luz. Se les aparecieron también Moisés y Elías conversando con Jesús: hablaban de su muerte, que iba a consumar en Jerusalén, o sea, del misterio de aquella salvación que había de operarse mediante su cuerpo, de aquella pasión –repito– que habría de consumarse en la cruz. Pues la verdad es que la ley de Moisés y los vaticinios de los santos profetas preanunciaron el misterio de Cristo: las losas de la ley lo describían como en imagen y veladamente; los profetas, en cambio, lo predicaron en distintas ocasiones y de muchas maneras, diciendo que en el momento oportuno aparecería en forma humana y aceptaría morir en la cruz por la salvación y la vida de todos.

Y el hecho de que estuviesen allí presentes Moisés y Elías conversando con Jesús, quería indicar que la ley y los profetas son como los dos aliados de nuestro Señor Jesucristo, presentado por ellos como Dios a través de las cosas que habían preanunciado y que concordaban entre sí. En efecto, no disuenan de la ley los vaticinios de los profetas: y, a mi modo de ver, de esto hablaban Moisés y Elías, el más grande de los profetas.

Habiéndose aparecido, no se mantuvieron en silencio, sino que hablaban de la gloria que el mismo Jesús iba a consumar en Jerusalén, a saber, de la pasión y de la cruz y, en ellas, vislumbraban también la resurrección. Pensando quizá el bienaventurado Pedro que había llegado el tiempo del reinado de Dios, gustoso se quedaría a vivir en la montaña; de hecho, y sin saber lo que decía, propone la construcción de tres tiendas. Pero aún no había llegado el fin de los tiempos, ni en la presente vida entrarán los santos a participar de la esperanza a ellos prometida. Dice, en efecto, Pablo: \textit{El trasformará nuestra condición humilde, según el modelo de su condición gloriosa}, es decir, de la condición gloriosa de Cristo.

Ahora bien, estando estos planes todavía en sus comienzos, sin haber llegado aún a su culminación, sería una incongruencia que Cristo, que por amor había venido al mundo, abandonase el proyecto de padecer voluntariamente por él. Conservó, pues, aquella naturaleza infraceleste, con la que padeció la muerte según la carne y la borró por su resurrección de entre los muertos.

Por lo demás y al margen de este admirable y arcano espectáculo de la gloria de Cristo, ocurrió además otro hecho útil y necesario para consolidar la fe en Cristo, no sólo de los discípulos, sino también de nosotros mismos. Allí, en lo alto, resonó efectivamente la voz del Padre que decía: \textit{Este es mi Hijo, el amado, mi predilecto. Escuchadlo}.
\end{body}

\begin{patercite}(\ldots) Éste es el misterio, saludable para nosotros, que ahora se ha cumplido en la montaña, ya que ahora nos reúne la muerte y, al mismo tiempo, la festividad de Cristo. Por esto, para que podamos penetrar, junto con los elegidos entre los discípulos inspirados por Dios, el sentido profundo de estos inefables y sagrados misterios, escuchemos la voz divina y sagrada que nos llama con insistencia desde lo alto, desde la cumbre de la montaña. Debemos apresurarnos a ir hacia allí –así me atrevo a decirlo– como Jesús, que allí en el cielo es nuestro guía y precursor, con quien brillaremos con nuestra mirada espiritualizada, renovados en cierta manera en los trazos de nuestra alma, hechos conformes a su imagen, y, como él, transfigurados continuamente y hechos partícipes de la naturaleza divina, y dispuestos para los dones celestiales.

Corramos hacia allí, animosos y alegres, y penetremos en la intimidad de la nube, a imitación de Moisés y Elías, o de Santiago y Juan. Seamos como Pedro, arrebatado por la visión y aparición divina, transfigurado por aquella hermosa transfiguración, desasido del mundo, abstraído de la tierra; despojémonos de lo carnal, dejemos lo creado y volvámonos al Creador, al que Pedro, fuera de sí, dijo: \textit{Señor, ¡qué bien se está aquí!} Ciertamente, Pedro, en verdad qué bien se está aquí con Jesús; aquí nos quedaríamos para siempre. ¿Hay algo más dichoso, más elevado, más importante que estar con Dios, ser hechos conformes con él, vivir en la luz? Cada uno de nosotros, por el hecho de tener a Dios en sí y de ser transfigurado en su imagen divina, tiene derecho a exclamar con alegría: \textit{¡Qué bien se está aquí!} donde todo es resplandeciente, donde está el gozo, la felicidad y la alegría, donde el corazón disfruta de absoluta tranquilidad, serenidad y dulzura, donde vemos a (Cristo) Dios, donde él, junto con el Padre, pone su morada y dice, al entrar: \textit{Hoy ha sido la salvación de esta casa,} donde con Cristo se hallan acumulados los tesoros de los bienes eternos, donde hallamos reproducidas, como en un espejo, las imágenes de las realidades futuras.

\textbf{Anastasio Sinaíta}, \textit{Sermón} en el día de la Transfiguración del Señor, 6-10: \textquote{Mélanges d’archeologie et d’histoire} 67 [1955], 241-244 (Breviario, 6 de agosto).
\end{patercite}

\newsection
\subsection{San Juan Pablo II, papa}

\ptheme{Dios cumple la promesa entregando a su propio Hijo}

\src{Homilía durante las celebraciones en recuerdo de \\Abraham \textquote{Padre de todos los creyentes} \\23 de febrero del 2000.}

\begin{body}
\ltr[1. «] Yo soy el Señor que te saqué de Ur de los caldeos, para darte esta tierra en propiedad. (\ldots) Aquel día firmó el Señor una alianza con Abram, diciendo: “A tu descendencia he dado esta tierra, desde el río de Egipto hasta el gran río, el río Éufrates”» (\textit{Gn} 15, 7. 18).

Antes de que Moisés oyera en el monte Sinaí las conocidas palabras de Yahveh: \textquote{Yo soy el Señor, tu Dios, que te he sacado del país de Egipto, de la situación de esclavitud} (\textit{Ex} 20, 2), el patriarca Abraham ya había escuchado estas otras palabras: \textquote{Yo soy el Señor que te saqué de Ur de los caldeos}. Por consiguiente, debemos dirigirnos con el pensamiento hacia ese lugar tan importante en la historia del pueblo de Dios, para buscar en él \textit{los inicios de la alianza de Dios con el hombre}. Precisamente por ello, en este año del gran jubileo, mientras con el corazón nos remontamos hasta los orígenes de la alianza de Dios con la humanidad, \textit{nuestra mirada se vuelve hacia Abraham}, hacia el lugar donde escuchó la llamada de Dios y respondió a ella con la obediencia de la fe. Juntamente con nosotros, también los judíos y los musulmanes contemplan la figura de Abraham como un modelo de sumisión incondicional a la voluntad de Dios (cf. \textit{Nostra aetate}, 3).

El autor de la carta a los Hebreos escribe: \textquote{Por la fe, Abraham, al ser llamado por Dios, obedeció y salió para el lugar que había de recibir en herencia, y salió sin saber a dónde iba} (\textit{Hb} 11, 8). Abraham, a quien el Apóstol llama \textquote{nuestro Padre en la fe} (cf. \textit{Rm} 4, 11-16), creyó en Dios, \textit{se fió de él}, que lo llamaba. \textit{Creyó en la promesa}. Dios dijo a Abraham: \textquote{Sal de tu tierra, y de tu patria, y de la casa de tu padre, a la tierra que yo te mostraré. De ti haré una nación grande y te bendeciré. Engrandeceré tu nombre; y serás tú una bendición. (\ldots) Por ti serán bendecidos todos los linajes de la tierra} (\textit{Gn} 12, 1-3). ¿Estamos, acaso, hablando de la ruta de una de las múltiples emigraciones típicas de una época en la que la ganadería era una forma fundamental de vida económica? Es probable. Pero, con toda seguridad, \textit{no sólo se trató de esto.} En la historia de Abraham, con el que comenzó la historia de la salvación, ya podemos percibir otro significado de la llamada y de la promesa. La tierra hacia la que se encamina el hombre guiado por la voz de Dios \textit{no pertenece exclusivamente a la geografía de este mundo}. Abraham, el creyente que acoge la invitación de Dios, es el que se pone en camino hacia una tierra prometida que no es de aquí abajo.

2. En la carta a los Hebreos leemos: \textquote{Por la fe, Abraham, sometido a la prueba, presentó a Isaac como ofrenda, y el que había recibido las promesas, ofrecía a su unigénito, respecto del cual se le había dicho: Por Isaac tendrás descendencia} (\textit{Hb} 11, 17-18). \textit{He aquí el culmen de la fe de Abraham}. Fue puesto a prueba por el Dios en quien había depositado su confianza, por el Dios del que había recibido la promesa relativa al futuro lejano: \textquote{Por Isaac tendrás descendencia} (\textit{Hb} 11, 18). Pero es invitado a ofrecer en sacrifico a Dios precisamente a ese Isaac, su único hijo, a quien estaba vinculada toda su esperanza, de acuerdo con la promesa divina. ¿Cómo podrá cumplirse la promesa que Dios le hizo de una descendencia numerosa si Isaac, su único hijo, debe ser ofrecido en sacrificio?

Por la fe, Abraham sale victorioso de esta prueba, una prueba dramática, que comprometía directamente su fe. En efecto, como escribe el autor de la carta a los Hebreos, \textquote{pensaba que Dios era poderoso aun para resucitarlo de entre los muertos} (\textit{Hb} 11, 19). Incluso en el instante, humanamente trágico, en que estaba a punto de infligir el golpe mortal a su hijo, Abraham no dejó de creer. Más aún, su fe en la promesa alcanzó entonces su culmen. Pensaba: \textquote{Dios es poderoso aun para resucitarlo de entre los muertos}. Eso pensaba este padre probado, humanamente hablando, por encima de toda medida. Y su fe, su abandono total en Dios, no lo defraudó. Está escrito: \textquote{Por eso lo recobró} (\textit{Hb} 11, 19). Recobró a Isaac, puesto que creyó en Dios plenamente y de forma incondicional.

El autor de la carta a los Hebreos parece expresar aquí algo más: toda la experiencia de Abraham le resulta \textit{una analogía del evento salvífico de la muerte y la resurrección de Cristo}. Este hombre, que está en el origen de nuestra fe, forma parte del eterno designio divino. Según una tradición, el lugar donde Abraham estuvo a punto de sacrificar a su propio hijo es el mismo sobre el que otro padre, el Padre eterno, aceptaría la ofrenda de su Hijo unigénito, Jesucristo. Así, el sacrificio de Abraham se presenta como anuncio profético del sacrificio de Cristo. \textquote{Porque tanto amó Dios al mundo –escribe san Juan– que le dio a su Hijo unigénito} (\textit{Jn} 3, 16). En cierto sentido, el patriarca Abraham, nuestro padre en la fe, sin saberlo, introduce a todos los creyentes en el plan eterno de Dios, en el que se realiza la redención del mundo.

3. Un día Cristo afirmó: \textquote{En verdad, en verdad os digo: antes de que Abraham existiera, Yo Soy} (\textit{Jn} 8, 58) y estas palabras despertaron el asombro de los oyentes, que objetaron: \textquote{¿Aún no tienes cincuenta años y has visto a Abraham?} (\textit{Jn} 8, 57). Los que reaccionaban así razonaban de modo puramente humano, y por eso no aceptaron lo que Cristo les decía. \textquote{¿Eres tú acaso más grande que nuestro padre Abraham, que murió? También los profetas murieron. ¿Por quién te tienes a ti mismo?} (\textit{Jn} 8, 53). Jesús les replicó: \textquote{Vuestro padre Abraham se regocijó pensando en ver mi día; lo vio y se alegró} (\textit{Jn} 8, 56). La vocación de Abraham se presenta completamente orientada hacia el día del que habla Cristo. Aquí no valen los cálculos humanos; \textit{es preciso aplicar el metro de Dios}. Sólo entonces podemos comprender el significado exacto de la obediencia de Abraham, que \textquote{creyó, esperando contra toda esperanza} (\textit{Rm} 4, 18). Esperó que se iba a convertir en padre de numerosas naciones, y hoy seguramente se alegra con nosotros porque la promesa de Dios se cumple a lo largo de los siglos, de generación en generación.

El hecho de haber creído, esperando contra toda esperanza, \textquote{le fue reputado como justicia} (\textit{Rm} 4, 22), no sólo en consideración a él, sino también a todos nosotros, sus descendientes en la fe. Nosotros \textquote{creemos en aquel que resucitó de entre los muertos a Jesús, Señor nuestro} (\textit{Rm} 4, 24), que murió por nuestros pecados y resucitó para nuestra justificación (cf. \textit{Rm} 4, 25). Esto no lo sabía Abraham; sin embargo, por la obediencia de la fe, se dirigía hacia el cumplimiento de todas las promesas divinas, impulsado por la esperanza de que se realizarían. Y ¿existe promesa más grande que la que se cumplió en el misterio pascual de Cristo? Realmente, en la fe de Abraham Dios todopoderoso selló una alianza eterna con el género humano, y Jesucristo es el cumplimiento definitivo de esa alianza. El Hijo unigénito del Padre, de su misma naturaleza, se hizo hombre para introducirnos, mediante la humillación de la cruz y la gloria de la resurrección, en la tierra de salvación que Dios, rico en misericordia, prometió a la humanidad desde el inicio.

4. El modelo insuperable del pueblo redimido, en camino hacia el cumplimiento de esta promesa universal, es María, \textquote{la que creyó que se cumplirían las cosas que le fueron dichas de parte del Señor} (\textit{Lc} 1, 45).

María, hija de Abraham por la fe, además de serlo por la carne, compartió personalmente su experiencia. También ella, como Abraham, aceptó la inmolación de su Hijo, pero mientras que a Abraham no se le pidió el sacrificio efectivo de Isaac, Cristo bebió el cáliz del sufrimiento hasta la última gota. Y María participó personalmente en la prueba de su Hijo, creyendo y esperando de pie junto a la cruz (cf. \textit{Jn} 19, 25).

Era el epílogo de una larga espera. María, formada en la meditación de las páginas proféticas, presagiaba lo que le esperaba y, al alabar la misericordia de Dios, fiel a su pueblo de generación en generación, expresó su adhesión personal al plan divino de salvación; y, en particular, dio su \textquote{sí} al acontecimiento central de aquel plan, el sacrificio del Niño que llevaba en su seno. Como Abraham, aceptó el sacrificio de su Hijo.

Hoy nosotros unimos nuestra voz a la suya, y con ella, la Virgen Hija de Sión, proclamamos que Dios se acordó de su misericordia, \textquote{como lo había prometido a nuestros padres, en favor de Abraham y su descendencia por siempre} (\textit{Lc} 1, 55).
\end{body}



\newsection
\section{Homilías}

\subsection{San Pablo VI, papa}

\subsubsection{Homilía (1967): ¿Conoces realmente a Jesús?}

\src{19 de febrero de 1967.}

\begin{body}
\homsec{El acontecimiento luminoso del Tabor}

[\ldots]

\ltr{L}{os} heraldos del Evangelio, los obispos y, en primer lugar, el Papa, tienen la obligación de anunciar y difundir la palabra de Dios, de explicarla y comentarla.

Repasemos juntos, con espíritu atento, el pasaje de San Mateo\anote{id8} que nos acaba de presentar la Liturgia. Es la historia de la Transfiguración del Señor. Una página de la historia de Cristo, entre las más bellas, espléndidas y misteriosas.

Jesús, de noche, en un monte, al aire libre, quizás durante la primavera, con tres de sus discípulos: Pedro, Juan y Santiago. Mientras estos, cansados de la subida, se detienen a descansar en la hierba, Jesús se aleja un poco para atender la oración, como hacía siempre durante la noche: \textquote{Erat pernoctans in oratione Dei}, nos recuerda San Lucas.

En la oscuridad profunda, en cierto punto, los tres durmientes son despertados por un deslumbrante destello de luz. Y aquí, asombrados, ven a Jesús –San Marcos da algunos detalles– brillando como el sol, mientras su ropa es blanca como la nieve.

Sol y nieve. Es la fiesta de la luz. En ese triunfo los discípulos ven a dos excelentes figuras del Antiguo Testamento, Moisés y Elías, conversando con Jesús.

San Pedro no puede resistir la alegría y el entusiasmo. Luego de exclamar: \textquote{¡Qué bueno es estar aquí !}, propone levantar tres tiendas para una estadía permanente de los tres Personajes.

Pero, al mismo tiempo, los tres Apóstoles ven una nube blanca que se forma para envolver todo el cuadro beatífico: y desde la nube escuchan una voz poderosa que exclama: \textquote{Este es mi Hijo amado, escúchadlo}.

Pedro, Juan y Santiago están aterrorizados y ya no se atreven a mirar hacia arriba. Unos momentos después se sienten conmovidos. Él es todavía y siempre Jesús, pero desprovisto del prodigioso esplendor de hace un momento; los invita a bajar de la montaña; y les prohíbe contar lo sucedido hasta que –otro motivo de asombro para los Apóstoles– el Hijo del Hombre (era el título que Jesús se dio a sí mismo) haya resucitado de entre los muertos.

\homsec{El pleno conocimiento de Jesús}

Podría escribirse un volumen para ilustrar este rasgo del Evangelio. Pero hoy el Santo Padre quiere proponer sólo algunos temas de importancia más inmediata.

¿Qué problema plantea el episodio de la Transfiguración? Se puede condensar en una pregunta que cada uno querrá hacerse: ¿realmente conoces a Jesús? Es decir, ¿tienes un conocimiento real, positivo y concreto de él?

¿Realmente podrías decir quién es? ¿Lo tienes presente en tu alma?

Existe el peligro, dada la debilidad de la naturaleza humana, de insistir en respuestas y títulos correctos, sí, pero no siempre completos. Un cristiano, sin embargo, debe saber responder más y mejor, yendo más allá de lo que resulta de un interés, de una noticia superficial.

Mientras tanto: esta misma pregunta recorre toda la historia evangélica, de principio a fin.

\homsec{Sabiduría, bodad y amor de Cristo}

¿Quién es Jesús? preguntan sus contemporáneos. Hay varias respuestas: el hijo de María, el hijo del carpintero, un profeta, el Mesías. Esta diversidad de nombres persiste: incluso sobre ella se construye un proceso: la Pasión de Jesús. En la noche terrible, después de ser capturado en Getsemaní, Caifás, el Sumo Sacerdote, pregunta a Cristo si es el Hijo de Dios. Jesús responde: \textquote{Sí lo soy}. Más tarde es Pilato quien le pregunta si es Rey: idéntica respuesta afirmativa. De ahí la condenación, por la cual, en la Cruz, se coloca el signo con la motivación de la sentencia: \textquote{Jesús Nazareno, Rey de los Judíos}.

Después de hechos tan excepcionales y terribles, es lógico que los fieles se pregunten si conocen a Jesús.

Para facilitar la respuesta, pensemos en dos tipos de argumentos. El primero proviene del mismo Jesús. Contemplando cómo se presenta y se revela a sí mismo, debe notarse una especie de gradualidad. El Salvador del mundo se nos aparece en la pobreza, en la humildad, quitando a su alrededor todo aparato, toda pompa y todo signo de su Divinidad. Quería comenzar su vida terrena, en secreto, introduciéndose en la humanidad sin acontecimientos extraordinarios; y vivió durante muchos años como un trabajador pobre. No puede haber una humildad más profunda. Y quien no acepte esta presentación se escandalizará y no comprenderá el resto de la vida y la revelación de Cristo. Parecería, por tanto, que no quiere hacer notar su presencia. Esto explica por qué pasan tantos a su lado y no sienten su llamada.

Ahora bien, esta revelación humana, sensible, caracterizada por la pobreza no está sola. Jesús se presentó ante todos, pero a algunos, a los que se le acercaron y le siguieron, otorgó otras manifestaciones de sí mismo: la sabiduría, su palabra maravillosa. Por ejemplo, quedaron impresionados los enviados de los enemigos del Divino Maestro, que un día quisieron tenderle una trampa. Les da vergüenza oírle hablar. En otra ocasión una mujer, después de haberlo escuchado, alza la voz entre la multitud exclamando; \textquote{Bienaventurada la que te generó, porque nunca nadie ha hablado tan bien como Tú enseñas}.

Junto a la revelación de la sabiduría, la del poder: los milagros. Hay muchos, asombrosos: todos los tenemos presentes. Ciertamente, ningún hombre podría hacer tales maravillas.

En tercer lugar, y en un grado aún mayor: Jesús se revela a sí mismo. Está en la bondad. Quien se acerca tiene la emoción y el encanto de esta bondad incomparable. \textquote{Venid a mí todos los que estáis cansados; y yo os restauraré}. Y el perdón a los pecadores, el amor a los niños, a los pobres, a los que sufren. Todos, ahora y siempre, pueden hacer el experimento de pasar junto a Jesús y captar su luz penetrante, en el perfecto conocimiento de las almas. \textquote{Sciebat quid esset in homine}. Sabía lo que había dentro de los corazones y derramaba en ellos su bondad.

Finalmente, –mientras más se estrecha el cerco de los que conocen la suprema aparición celestial–, Jesús también se revela como realmente es. He aquí la Transfiguración. En él palpita no solo una vida humana, sino la vida divina. \textquote{Este es mi Hijo amado}. Es el Hijo de Dios hecho hombre. Precisamente este aspecto se volverá, al parecer, normal después de la muerte y resurrección del Señor. ¿Habéis conocido alguna vez al Señor así?

\homsec{Abrir el alma a la fe y a la gracia}

Ahora tenemos que examinar un segundo orden de elementos que condicionan nuestro conocimiento de Jesús, el cual depende de nuestra disposición: la de abrir los ojos, el corazón, el alma. Si acudimos a él con el corazón cerrado, con los ojos cerrados, con prejuicios e incredulidad preestablecida, Él no se mostrará. La luz pasará cerca de nosotros y nos quedaremos ciegos, indiferentes.

Por tanto, debemos abrir los ojos. Todo el mundo tiene que hacerlo. El Redentor no vino para una categoría específica, por ejemplo, para los sabios. Se mostró al mundo, a toda la humanidad: y ésta estaría, de por sí, en grado de captar los rayos de su rostro divino. La realidad nos muestra en cambio que, lamentablemente, \textquote{no omnes oboediunt Evangelio}: no todos obedecieron al Evangelio, como dice San Pablo. Algunos miran y no ven: siguen siendo extraños y débiles ante la Revelación.

Por eso es necesario abrir la mente al conocimiento de Jesús, y esta invitación explícita no parece exagerada, ya que nunca poseemos suficiente conocimiento de Jesús. Siempre somos ignorantes, porque lo que podemos aprender de Jesús es tan grande e infinito que nuestras pobres facultades, aunque fuéramos teólogos consumados, deberían ser consideradas mezquinas e insuficientes.

Entonces, ¿qué debemos hacer?

Primero, educarnos a nosotros mismos; atesorar la palabra del Señor difundida en la predicación sagrada, en la catequesis, en libros adecuados.

\homsec{La Transfiguración final}

Jesús no se reveló tanto por la vía de los ojos, sino por la escucha que debemos prestarle. El Evangelio nos lo recuerda: \textquote{Ipsum audite}: debes escucharlo a Él. Y de nuevo \textquote{Fides ex auditu}: la fe, es decir, el conocimiento misterioso de Jesús, lo tendremos solamente si recibimos la gracia de escucharlo.

En consecuencia, no solo debemos ser buenos oyentes, sino deseosos de aprender, porque la palabra de Jesús es el mismo Jesús, es la Palabra de Dios, que viene de manera intencional, misericordiosa, muy amplia, a nuestras almas, para que allí se reciba su palabra y se convierta en la norma de nuestra vida.

Lo segundo es amar a Jesús, quien lo ama lo conocerá de la manera más válida. Él mismo lo afirmó: \textquote{Qui diligit me, diligetur a Patre meo; et ego diligam eum et manifestabo ei meipsum} – Si alguno me ama, me abriré a él, me daré a conocer a él. Son las experiencias espirituales las que a menudo tienen una certeza mucho mayor que los silogismos de nuestro razonamiento. Este regalo es ofrecido a todas las almas; los que realmente quieran estar con Cristo podrán poseerlo.

He aquí el voto del Papa, queridísimos hijos y aquí no estamos tanto en el anuncio como en el deseo: que un día todos veamos a nuestro Salvador en su plenitud de vida, en su humanidad, que es igual a la nuestra, en su Divinidad que le viene del Padre. En Él veremos al Dios vivo. Ese encuentro bendito, esa transfiguración final, será nuestra gloria y felicidad eterna: nuestro Paraíso. ¡Amén!
\end{body}


\newsection
\subsection{San Juan Pablo II, papa}

\subsubsection{Homilía (1979): Dios está con nosotros}

\src{Visita Pastoral a la Parroquia Romana de San Basilio. \\11 de marzo de 1979.}

\begin{body}
 \ltr[\ldots]{E}{ste} es un encuentro en la fe, cuyo contenido nos precisa la Palabra de Dios en la liturgia de hoy. Contenido fuerte, profundo y esencial. Escuchando la \textbf{Carta de San Pablo a los Romanos}, encontramos inmediatamente la realidad-clave de la fe. \textquote{Si Dios está por nosotros, ¿quién contra nosotros? El que no perdonó a su propio Hijo, antes lo entregó por todos nosotros, ¿cómo no nos ha de dar con Él todas las cosas? ¿Quién acusará a los elegidos de Dios? Siendo Dios quien justifica. ¿quién condenará? ¿Acaso Cristo Jesús, el que murió, aún más, el que resucitó? Él, que está a la diestra de Dios, y que intercede por nosotros} (\textit{Rm} 8, 31-34). ¡Dios está con nosotros! ¡Dios con el hombre! Con la humanidad. La prueba única y completa de esto es y permanece siempre ésta: \textquote{no perdonó a su propio Hijo, antes le entregó por todos nosotros} (\textit{Rm} 8, 32).

Para poner más de relieve aún esta verdad, la liturgia hace referencia en el libro del \textbf{Génesis}, al sacrificio de Isaac. Cuando Dios pidió a Abraham esta ofrenda, quería preparar en cierto modo la conciencia del pueblo elegido para el sacrificio que después realizaría su Hijo. Dios perdonó a Isaac y perdonó también el corazón de su padre Abraham. ¡Pero no ha perdonado al propio Hijo! Abraham fue \textquote{padre de nuestra fe}, porque con la disposición al sacrificio de su hijo Isaac, preanunció el sacrificio de Cristo, que constituye un momento cumbre en los caminos de la fe de toda la humanidad. Todos somos conscientes de ello. Esta conciencia vivifica nuestras almas, particularmente durante la Cuaresma. Esta conciencia plasma nuestra vida cristiana desde las raíces más profundas. La plasma desde el principio al fin.

Dios está con nosotros a través de la cruz de su Hijo. Y ésta es también la fuente primera de nuestra fuerza espiritual. Cuando el Apóstol pregunta: \textquote{Si Dios está por nosotros, ¿quién contra nosotros?}, con esta pregunta abraza a todo y a todos los que puedan ser un peligro para nuestro espíritu, para nuestra salvación. \textquote{¿Quién condenará? Cristo Jesús, el que murió, aún más, el que resucitó, el que está sentado a la diestra de Dios, es quien intercede por nosotros} (\textit{Rm} 8, 34). De la fe en Cristo, en su cruz y resurrección, nace la esperanza. ¡Gran confianza! Sea ésta nuestra fuerza, particularmente en los momentos difíciles de la vida.

Mi pensamiento y mi palabra se dirigen de modo especial a todos los que se encuentran en dificultades de diverso género: a quienes sufren en el cuerpo y en el espíritu; a quienes sufren pruebas de carácter social, como experiencias negativas en el trabajo, o malentendidos de familia: a los jóvenes que acaso están pasando un momento de crisis: a quienes afrontan con tesón dificultades de naturaleza pastoral, como la incomprensión o la tibieza ante los valores espirituales y la resistencia al Espíritu Santo en Cristo. Todos tienen derecho a esperar.

En el \textbf{Evangelio} de hoy encontramos una manifestación especial de la esperanza que nace de la fe en Jesucristo. Precisamente en el tiempo de Cuaresma la Iglesia nos lee de nuevo el Evangelio de la Transfiguración del Señor. En efecto, este acontecimiento tuvo lugar a fin de preparar a los Apóstoles a las pruebas difíciles de Getsemaní, de la pasión, de la humillación de la flagelación, de la coronación de espinas, del vía crucis, del Calvario. En esta perspectiva Jesús quería demostrar a sus Apóstoles más íntimos el esplendor de la gloria que refulge en El, la que el Padre le confirma con la voz de lo alto, revelando su filiación divina y su misión: \textquote{Este es mi Hijo amado, en quien tengo mi complacencia: escuchadle} (\textit{Mt} 17, 5).

El esplendor de la gloria de la Transfiguración abraza casi toda la Antigua Alianza y llega a los ojos llenos de estupor de los Apóstoles, que se convertirían en maestros de esa fe que hace nacer la esperanza: de aquellos Apóstoles que deberían anunciar todo el misterio de Cristo. \textquote{Señor, ¡qué bien estamos aquí!} (\textit{Mt} 17, 4), exclaman Pedro, Santiago y Juan, como si quisieran decir: ¡Tú eres la encarnación de la esperanza que anhelan el alma y el cuerpo humanos! ¡Esperanza que es más fuerte que la cruz y que el Calvario! Esperanza que disipa las tinieblas de nuestra existencia, del pecado y de la muerte.

¡Qué bien estamos aquí: contigo!

Sea vuestra parroquia, y cada vez lo sea más, el lugar, la comunidad donde los hombres, profundizando por medio de la fe en el misterio de Cristo, adquieran más confianza, más conciencia del valor y del sentido de la vida, y repitan a Cristo: \textquote{¡Qué bien estamos aquí!}: contigo. Aquí, en este templo. Ante este tabernáculo. Y no sólo aquí, sino acaso en una cama de hospital; acaso en los puestos de trabajo; a la mesa en la comunidad de la familia. En todas partes.

[...]
\end{body}

\newpage 

\subsubsection{Homilía (1982): Tres montañas}

\src{Visita pastoral a la parroquia romana \\de la Inmaculada Concepción alla Cervelletta. \\7 de marzo de 1982.}

\begin{body}
\ltr[1. ]{L}{a} liturgia del segundo domingo de Cuaresma es, en cierto sentido, la liturgia de las tres montañas. En la primera escuchamos, como relata el libro del \textbf{Génesis}, las palabras que Dios dirigió a Abraham: \textquote{Toma a tu hijo, tu único hijo a quien amas, Isaac, ve al territorio de Moria y ofrécelo en holocausto en el monte que Yo te indicaré} (\textit{Gn} 22, 2).

La prueba de Abraham. \textquote{Dios puso a prueba a Abraham} (\textit{Gn} 22, 1). Esta fue la prueba de su fe. En el lugar indicado, Abraham construyó el altar, colocó la madera sobre él y colocó a su hijo Isaac sobre la madera: el hijo unigénito. El hijo de la promesa. El hijo de la esperanza. Abraham estaba listo para ofrecerlo como holocausto a Dios, para derramar su sangre y quemar su cuerpo en la hoguera.

En el momento decisivo recibió la prohibición de Dios: \textquote{¡No extiendas tu mano contra el muchacho y no le hagas daño! ahora sé que temes a Dios porque no te has reservado a tu hijo, tu único hijo} (\textit{Gn} 22, 12). En la zarza cercana, Abraham encontró un carnero y lo ofreció sobre el altar preparado. La prueba de fe se cumplió. La gran prueba. La dura prueba. Adecuada a la gran promesa. Dios renovó su promesa ante Abraham, después de haberlo sometido a la prueba: \textquote{Haré muy numerosa tu descendencia, como las estrellas del cielo y como la arena que está a la orilla del mar} (\textit{Gn} 22, 17). La descendencia no tanto según la carne como según el espíritu. Los descendientes de Abraham en la fe son, en cierto sentido, seguidores de las tres grandes religiones monoteístas del mundo: el judaísmo, el cristianismo, el islam. \textquote{Todas las naciones de la tierra serán bendecidas por tu descendencia, porque obedeciste a mi voz} (\textit{Gn} 22, 18).

Los descendientes de la fe de Abraham creen que Dios tiene el poder de probar al hombre. Tiene derecho a ofrecer de su espíritu.

2. La liturgia del segundo domingo de Cuaresma nos lleva a otra montaña, a Galilea. Más allá de la llanura de Galilea, se eleva majestuoso el \textbf{monte Tabor}: el monte de la Transfiguración, según la tradición cristiana. En este monte, Jesús de Nazaret, que vino entre los descendientes de Abraham como el Mesías enviado por Dios, fue transformado milagrosamente ante los ojos de sus Apóstoles: Pedro, Santiago y Juan. A los ojos de los Apóstoles, se manifestó transfigurado en gloria y, junto a él, Moisés y Elías. Al milagro del oído se añadió el milagro de la visión. Oyeron la voz que salió de la nube: \textquote{Este es mi Hijo amado: escuchadlo} (\textit{Mc} 9, 7); las mismas palabras que ya había oído Juan el Bautista cerca del Jordán, con motivo de la primera venida de Jesucristo, después de su bautismo. La teofanía del monte Tabor tiene un carácter pascual. Anuncia la gloria de Cristo resucitado. Al mismo tiempo, prepara a los Apóstoles para la muerte del Cordero de Dios, para la Teofanía del Gólgota.

3. El apóstol Pablo nos lleva al monte Gólgota, el tercer monte, con las palabras de la \textbf{carta a los Romanos}. La Teofanía del Gólgota se indica con las siguientes palabras: \textquote{Si Dios está por nosotros, ¿quién estará contra nosotros? El que no se reservó ni a su propio Hijo, sino que lo entregó por todos nosotros} (\textit{Rm} 8, 31-32). Sabemos que el Padre entregó a su Hijo en el Gólgota; sabemos que este es el nombre de esa colina fuera de los muros de Jerusalén, en la cual Dios \textquote{no se reservó ni a su propio Hijo} (\textit{Rm} 8, 32). Y a través de esto, demostró \textquote{estar con nosotros hasta el final}. \textquote{¿Cómo no nos dará todo junto con él?}, pregunta el Apóstol (\textit{Rm} 8, 32).

Este Dios, que no permitió que Abraham sacrificara a su hijo Isaac en la muerte, no perdonó a su propio Hijo. ¿Acaso haciendo esto no confirmó nuestra elección hasta el final? \textquote{¿Quién acusará a los elegidos de Dios?}, pregunta el Apóstol (\textit{Rm} 8, 33). Él mismo tomó la causa de la justificación del hombre en sus propias manos\ldots \textquote{Es Dios quien justifica} (\textit{Rm} 8, 33). Y si es así, ¿quién puede condenar al hombre? (cf. \textit{Rm} 8, 34). Una sentencia así solo podía ser dictada por Cristo, quien en el Gólgota conocía el peso de los pecados de los hombres. Pero en el Gólgota Jesucristo sufrió la muerte por nosotros \textquote{en verdad –escribe el Apóstol– \ldots resucitó, está a la diestra de Dios e intercede por nosotros} (\textit{Rm} 8, 34).

4. La liturgia dominical de hoy nos invita a subir a una montaña, lugar de la teofanía de la antigua y la nueva alianza. En estos montes estamos invitados, según el espíritu de la Cuaresma, a meditar sobre las grandes obras de Dios (cf. \textit{Hch} 2, 11): los misterios de nuestra redención, de nuestra justificación en Cristo.

En estos montes nos conviene aprender estos misterios, asimilarlos con el corazón y el alma, moldear nuestro espíritu, transformarlo según el aspecto que Cristo le da. Este domingo de Cuaresma nos enseña que estamos llamados a una gran transformación espiritual. Debemos participar en la Transfiguración de Cristo, así como sus discípulos en el monte Tabor. Tenemos que prepararnos para la Santa Pascua. El maestro de esta actitud, a través de la cual Cristo desciende a nuestro corazón, realizando una transformación y conversión, es Abraham: el Padre de todos los creyentes.

5. De hecho, las palabras del \textbf{salmista} parecen resonar en nuestro corazón: \textquote{Tenía fe aun cuando dije: ¡Qué desgraciado soy!} (\textit{Sal} 115 [116], 10). ¿Acaso no se sentía Abraham igualmente \textquote{desgraciado} cuando fue a la montaña indicada por Dios para sacrificar a su propio hijo? ¿Acaso no fue solo la fe lo que le permitió repetir: \textquote{Mucho le cuesta al Señor la muerte de sus fieles} (\textit{Sal} 115 [116], 15)? De Abraham la familia humana comienza a aprender la fe, que se manifiesta en la actitud interior del espíritu humano: se manifiesta en el sacrificio del corazón Jesucristo es el Maestro definitivo y perfecto de tal actitud: \textquote{consumator fidei nostrae} (cf. \textit{Hb} 12, 2).

6. El fruto de la liturgia del segundo domingo de Cuaresma debe ser la disposición a ofrecer los sacrificios espirituales en los que se manifiesta nuestra fe. Es lo que pedimos en las palabras del \textbf{Salmo}: \textquote{Señor, yo soy tu siervo, siervo tuyo, hijo de tu esclava; rompiste mis cadenas. Te ofreceré un sacrificio de alabanza invocando el nombre del Señor. Cumpliré al Señor mis votos en presencia de todo el pueblo} (\textit{Sal} 115 [116], 16-18). 

7.  \txtsmall{[Con este espíritu, vosotros, parroquianos de la parroquia de la \textquote{Inmaculada Concepción en la Cervelletta} de Tor Sapienza, os habéis reunido hoy con vuestro Obispo\ldots]}

8. Queridos hermanos y hermanas, (\ldots) junto con vosotros hoy hice una visita al monte de la fe de Abraham, al monte de la Transfiguración en Galilea y al monte Gólgota. Siguiendo el espíritu de la liturgia de Cuaresma hemos conocido la grandeza de nuestra Redención y de la Justificación en el Sacrificio de Cristo.

Que nuestra fe madure en el mismo espíritu: a través de las obras de todas las horas, a través de las pruebas de la vida diaria y, a veces, a través de las grandes pruebas y experiencias, en las que el espíritu humano es probado como oro acrisolado al fuego.

Para nosotros, redimidos y justificados en la Sangre de Cristo, ninguna prueba o experiencia cierra la perspectiva de la vida. Al contrario, la revelan aún más profundamente en Dios.

Aprendemos esta perspectiva, ofreciendo los sacrificios espirituales de todo lo que compone nuestra vida.

Que la participación en la Eucaristía nos una –cada vez, y particularmente hoy– en esta comunidad, a la que el Padre revela y entrega a su Hijo: \textquote{Este es mi Hijo amado, escúchadlo} (\textit{Mc} 9, 7).

Amén.
\end{body}

\img{holy_trinity}

\label{b2-03-02-1982H}
\newpage 


\subsubsection{Homilía (1985): Comprender la gravedad del mal}

\src{Celebración eucarística en la parroquia romana de San Tarcisio al IV Miglio. \\3 de marzo de 1985.}

\begin{body}
\textquote{[Dios], que no escatimó ni a su propio Hijo, sino que lo entregó por todos nosotros\ldots} (\textit{Rm} 8, 32).

\ltr[1. ]{E}{l} período de Cuaresma, más que cualquier otro, pone ante los ojos de nuestra fe y de nuestra conciencia esta verdad, esta imagen de Dios, que da a su Hijo en sacrificio por los pecados del hombre. Sobre la cruz. En la muerte. No lo perdona, sino que lo da. Dios, en quien la justicia y la misericordia se encuentran y se compenetran de una manera maravillosa. Es estrictamente justo ante el pecado. Es infinitamente misericordioso ante los pecadores. Por tanto, \textquote{no perdona} al Hijo. Y el Hijo \textquote{no se reserva} a sí mismo. Se entrega en sacrificio como \textquote{víctima divina} de la justicia y la misericordia.

2. La liturgia de la Iglesia en Cuaresma está dirigida a ese Dios, Padre e Hijo, en particular el domingo de hoy. Esto ya lo demuestra la \textbf{primera lectura} del libro del \textbf{Génesis} donde –en el sacrificio de Abraham– encontramos una \textquote{prefiguración}, es decir, una figura y un anticipo, en cierto sentido un signo remoto, de ese misterio inescrutable de la cruz. Este sacrificio de Abraham es sólo una prueba de fe para aquel a quien el apóstol llamó \textquote{padre de nuestra fe} (cf. \textit{Rm} 4, 11). Abraham, por la fe, llegó a tener un descendiente y heredero en Isaac. Y, sin embargo, a través de la fe, basada en una rigurosa obediencia a Dios, estaba dispuesto a ofrecer ese primogénito y único hijo como sacrificio a Dios.

Dentro de estos límites, Abraham, el padre, tiene cierto parecido con Dios-Padre, e Isaac, el hijo, es una imagen de Cristo-Hijo. Sin embargo, solo dentro de estos límites. Como parte de una prueba de obediencia y sinceridad de intención. De hecho, en última instancia, Dios no permite que Abraham sacrifique a Isaac. \textquote{No extiendas tu mano contra el muchacho –dice– ¡y no lo lastimes! Ahora sé que temes a Dios} (temer significa tener fe) \textquote{porque no te has reservado a tu hijo, a tu único hijo} (\textit{Gn} 22, 12). Y Abraham ofrece un cordero como sacrificio en lugar de su hijo.

3. En cambio, a su propio Hijo \textquote{Dios no lo perdonó, sino que lo entregó por todos nosotros}. Así lo anuncia \textbf{san Pablo} escribiendo a los \textbf{Romanos}. Y en este contexto plantea una serie de cuestiones fundamentales. En primer lugar: \textquote{Si Dios está por nosotros, ¿quién estará contra nosotros?} (\textit{Rm} 8, 31). Y, al dar a su propio Hijo, Dios revela que está con nosotros. Nos revela que está dispuesto a perdonarnos todo: si nos ofrece a su Hijo en holocausto, \textquote{¿cómo no nos dará todo junto con él?} (\textit{Rm} 8, 32).

Dios, Padre del Hijo crucificado, es Dios \textquote{rico en misericordia} (\textit{Ef} 2, 4). Y al mismo tiempo un Dios sumamente justo, que se encargó personalmente del problema de la justificación del hombre, del hombre pecador.

Y por eso \textbf{el apóstol} pregunta: \textquote{¿Quién acusará a los elegidos de Dios? Es Dios quien justifica} (\textit{Rm} 8, 33). Sabemos que si Él mismo justifica, significa que no quiere acusar. Quiere salvar. No quiere condenar. \textquote{¿Quién condenará?} pregunta el apóstol. \textquote{¿Acaso Cristo Jesús, que murió, más aún, que resucitó, está a la diestra de Dios e intercede por nosotros?} (\textit{Rm} 8, 34).

4. La liturgia de Cuaresma contiene en sí misma una llamada radical para cada uno de nosotros. ¡Meditemos a fondo el problema del pecado! ¡Reflexionemos a fondo sobre el problema de la culpa del hombre ante Dios! Este problema ha sido algo empañado y degradado en la conciencia contemporánea. La Cuaresma es el momento de una conversión particular. Convertirse significa descubrir la malicia del pecado. Redescubrirla en la propia conciencia. Poner en marcha todos los criterios humanos para este fin. Pero los criterios humanos no son suficientes aquí. La maldad del pecado se revela en su plenitud sólo cuando lo pensamos a la luz del misterio del Padre, \textquote{que no escatimó ni a su propio Hijo}. Sólo entonces comprendemos la profundidad del mal, cuando la necesidad de la justificación del pecado por Dios mismo se vuelve clara para nosotros. ¡Sólo entonces nos acercamos a la cruz de Cristo, para que se demuestre la infinitud del amor misericordioso, que cumple toda medida de justicia y de juicio.

5. La liturgia de Cuaresma contiene esta invitación en sí misma. La encíclica \textit{Dives in misericordia}, que se puede leer y meditar como comentario sobre la liturgia de Cuaresma, corresponde a esta invitación.

Este período nos introduce gradualmente en el corazón mismo del misterio pascual. Por eso también el \textbf{Evangelio} dominical de hoy presenta la Transfiguración de Cristo en el monte Tabor.

El Dios de Abraham no aceptó el sacrificio de la vida de Isaac. El Dios y Padre de Jesucristo, en cambio, acoge el sacrificio de la vida de su Hijo. El Padre y el Hijo, en este sacrificio, comienzan a cumplir la Justificación del hombre.

Para preparar a los apóstoles para la horrible muerte de Cristo en la cruz, Dios les permite saborear, casi como un anticipo, la gloria de su resurrección en la Transfiguración en el monte Tabor. Allí, desde el centro de la nube luminosa, se oye la voz del Padre (como después del Bautismo en el Jordán): \textquote{Este es mi Hijo amado: escuchadlo} (\textit{Mc} 9, 7). La muerte en la cruz será una prueba terrible y el despojo del Hijo de Dios, pero al mismo tiempo se convertirá en el comienzo de una nueva vida. Cristo retornará en la gloria del Padre.


6. \txtsmall{[He aquí los rasgos principales de la liturgia cuaresmal, que quise meditar con vosotros, queridos hermanos y hermanas de la parroquia de San Tarcisio.

A todos los aquí presentes (\ldots) A todos ustedes, queridos fieles, mis pensamientos afectuosos y de bendición. A vuestras familias, jóvenes, ancianos, niños, enfermos, todos aquellos que trabajan con buena voluntad en el territorio parroquial por una convivencia humana y civil más justa y serena, para alejar cualquier forma de peligro u ofensa a la dignidad de las personas y el bien común. ¡Que se consolide cada vez más un diálogo fructífero de resultados concretos en el contexto de estas necesidades esenciales de promoción humana!]}

7. La comunidad parroquial está dotada por Dios de fuerzas sobrenaturales que le permiten actuar como levadura en todo el ámbito de su territorio para una continua elevación de la vida moral del entorno, a pesar de las dificultades.

Sé cómo vuestra comunidad se centra mucho en los valores de la liturgia, la catequesis, la evangelización. ¡Muy bien! Os animo a continuar, con un compromiso renovado. Y saber esperar los frutos con paciencia. El \textbf{Evangelio}, y en particular la liturgia de hoy, nos enseñan que de alguna manera es necesario \textquote{morir} para dar vida. Siguiendo el ejemplo de nuestro Señor, imitemos su sacrificio, con la certeza de que los resultados llegarán.

El sacrificio evangélico por los hermanos no nos empobrece; al contrario, nos hace crecer. No nos quita nuestra dignidad o nuestros intereses genuinos; al contrario, nos despoja del \textquote{hombre viejo} y fortalece al \textquote{hombre nuevo} en nosotros. No tengamos miedo de seguir el ejemplo de Nuestro Señor, de Nuestra Señora y de los santos en esto, y nuestra acción será extraordinariamente fructífera.

 \txtsmall{[Entre los santos tenéis como modelo, de manera especial, a san Tarcisio, a quien está dedicada vuestra parroquia. Dio su vida por la santísima Eucaristía, porque sabía que el pan eucarístico es fuente de vida.]} Que el alimento divino que el Padre nos ofrece en Cristo sustente también vuestra acción al servicio del Señor y por el bien de los hermanos.

8. ¡Queridos hermanos y hermanas!

Deseo ardientemente que  \txtsmall{[esta visita a vuestra parroquia y]} la meditación común sobre el misterio de la muerte salvadora del Hijo de Dios despierten en vosotros una vida de profunda fe.

\textquote{Si Dios está con nosotros, ¿quién estará contra nosotros?}.

Y Dios está con nosotros. De hecho, \textquote{no se reservó ni a su propio Hijo}. Y nosotros, ¿respondemos a eso? ¿Estamos con Dios en lo más profundo de nuestros pensamientos, nuestras obras y nuestras conciencias?

¿Estamos con Dios como él \textquote{pide}? ¿Él, \textquote{que dio a su propio Hijo por todos nosotros}?

¿Estamos con Dios?
\end{body}



\label{b2-03-02-1985H}
\newpage 


\subsubsection{Homilía (1988): La Cruz y la Resurrección}

\src{Visita pastoral a la parroquia romana \\de la Resurrección de Nuestro Señor Jesucristo en Torre Nova. \\28 de febrero de 1988.}

\begin{body}
1. \textquote{Este es mi Hijo amado; ¡escuchadlol!} (\textit{Mc} 9, 7).

\homsec{La teofanía del Monte de la Transfiguración.} 

\ltr{L}{as} palabras que llegan a los oídos de los apóstoles Pedro, Santiago y Juan deslumbrados por la visión, son las palabras del Padre. En ellas se revela a sí mismo y a su Hijo. Así fue también con ocasión del bautismo de Jesús en el río Jordán. Ahora la situación es diferente y el tiempo es diferente.

Entonces Juan el Bautista había indicado \textquote{el cordero de Dios\ldots que quita el pecado del mundo} (\textit{Jn} 1, 29). Ahora los caminos de Jesús se han acercado al tiempo en que los pecados del mundo deben ser realmente quitados por medio de Él. Este será el momento del despojo, el momento de la elevación en el monte Gólgota, el momento que, desde el punto de vista humano, constituye la más profunda humillación: la \textquote{kénosis} de la cruz.

Cuando terminó la Transfiguración, Jesús ordenó a los apóstoles que no se lo mencionaran a nadie, \textquote{hasta que el Hijo del Hombre resucitara de entre los muertos} (\textit{Mc} 9, 9). Sin embargo, se preguntaban \textquote{qué significaba resucitar de entre los muertos} (\textit{Mc} 9, 10). Ni sabían ni previeron que esto tenía que hacerse al precio de la cruz y de la muerte.

2. La liturgia del segundo domingo de Cuaresma nos prepara para el misterio de la cruz de Cristo en el Gólgota, para los acontecimientos pascuales y, en primer lugar, nos conduce al Monte de la Transfiguración, de la misma manera que Cristo preparó a sus apóstoles. Y no solamente Cristo, sino también al Padre Celestial. La liturgia del domingo de hoy se perfila para nosotros en su conjunto como centrada en el misterio del Padre, el mismo Padre que \textquote{tanto amó al mundo que le dio a su Hijo unigénito} (\textit{Jn} 3, 16).

En el sacrificio de Abraham descrito en el libro del \textbf{Génesis}, encontramos la figura que presagia este misterio del Padre y del Hijo. Incluso en ese hecho hay un cerro ubicado en el territorio de Moria, sobre el cual Abraham sube con su hijo Isaac: el único hijo de la promesa.

Dios pidió el sacrificio de este hijo, tan esperado antes, y por tanto tan amado: de este hijo al que se unieron todas las esperanzas de Abraham. Sin embargo, cuando Dios pidió tal sacrificio, Abraham no dudó en hacerlo. Estaba dispuesto a sacrificar a su único hijo.

Cuando Abraham levantó su cuchillo para realizar el gesto de \textquote{sacrificar a su hijo} (\textit{Gn} 22, 10) –Dios a través del ángel– tomó su mano. Aceptó el sacrificio del corazón y no permitió la inmolación de su hijo. \textquote{Ahora sé que temes a Dios y no te reservaste a tu hijo, tu único hijo} (\textit{Gn} 22, 12). Podríamos añadir: \textquote{Amas a Dios más que a tu hijo}.

3. Cuán cerca estamos, en este punto, del misterio del Padre celestial, de este Padre, \textquote{que no se reservó ni a su propio Hijo, sino que lo entregó por todos nosotros}, como escribe el apóstol Pablo en su \textbf{carta a los Romanos} (\textit{Rm} 8, 32). Y añade \textquote{¿cómo no nos va a dar todo junto con él?} (\textit{Rm} 8, 32). El sacrificio de la cruz es el sacrificio de la satisfacción y de la expiación. En él está contenida la redención y la remisión de los pecados. El Apóstol penetra desde muchos puntos de vista en el significado y los frutos de este sacrificio cuando insiste en preguntar: \textquote{¿Quién acusará a los elegidos de Dios? Es Dios quien justifica. ¿Quién condenará? ¿Acaso Cristo Jesús, que murió, o más bien, que resucitó, está a la diestra de Dios e intercede por nosotros?} (\textit{Rm} 8, 33-34).

4. Se puede decir que estas preguntas, tan características de Pablo, nos introducen en toda la perspectiva del misterio pascual. El Padre celestial no perdonó a su Hijo unigénito, no lo salvó de la muerte en la cruz, en la que consistía el sacrificio. Pero ese sacrificio, que el Hijo sufrió voluntariamente, se convirtió en la fuente de nuestra justificación. Fuimos comprados a un alto precio (cf. \textit{1 Cor} 7, 23). Al participar en la redención mediante la cruz de Cristo, también hemos sido llamados a participar en la nueva vida revelada mediante la resurrección.

Se puede decir que aquí se abren verdaderas profundidades, \textquote{profundidades insondables} ante nuestro espíritu humano. La Cuaresma es el tiempo de la valentía espiritual, de adentrarse en estas profundidades de las que emerge la verdad definitiva sobre Dios y el hombre. La verdad que verdaderamente nos libera. Todo cristiano y toda comunidad cristiana está llamada a ello en el período actual. Vuestra parroquia también está llamada a ello \txtsmall{[de una manera particular con motivo de la visita de hoy del Obispo de Roma].}

5. Sin dudarlo, queridos hermanos y hermanas, responded a esta llamada, haciendo sin dudarlo, según el espíritu de Cuaresma, ese éxodo espiritual del pecado y del egoísmo, para progresar en la fe, que es escucha y obediencia, libre asentimiento y abandono confiado. Cuando el cristiano sigue la verdad, alcanza la sabiduría del corazón y, humilde y arrepentido, se abre al Redentor, recibiendo su consoladora bendición junto con el perdón.

[\ldots]

6. [\ldots] Rezo para que la luz de Cristo transfigurado ilumine vuestras mentes y corazones para llevaros a vivir con él el misterio pascual de su muerte y resurrección, os exhorto a ser siempre auténticos testigos del Evangelio, de la plenitud de la ley y del cumplimiento de las Profecías.

Por último, dirijo mi cordial saludo a todos los hermanos y hermanas de la parroquia de la Resurrección, y a cada uno de vosotros, queridos míos, deseo recordaros el deber de comprometerse apostólicamente para que, después de haber aceptado la palabra que es Espíritu y vida, hagáis presente, en los lugares donde el Señor os ha colocado, su plan de salvación. Y esto ocurrirá, si os apoyáis en la absolución de los pecados en el confesionario y os alimentáis del pan eucarístico del altar, para ser cada vez más efectivamente piedras vivas de ese edificio espiritual, que es la casa del Padre y la morada de todo hombre.

\txtsmall{[\ldots]}

7. Finalmente, volvamos a la meditación de las palabras del \textbf{salmista}, escuchadas en la liturgia de hoy: \textquote{Tenía fe aun cuando dije: ¡qué desgraciado soy!} (\textit{Sal} 116 [115], 10). ¿No habla el salmista aquí también de la fe de Abraham, el cual \textquote{creyó esperando contra toda esperanza} (\textit{Rm} 4, 18) y así se convirtió en padre de todos los creyentes (cf. \textit{Rm} 4, 11, 16)?

Abraham dice: \textquote{Cumpliré al Señor mis votos} (\textit{Sal} 116 [115], 14). Cumpliré\ldots \textquote{Señor, yo soy tu siervo} (\textit{Sal} 116 [115], 16). Abraham, el siervo del Dios de la Alianza. Abraham, el amigo de Dios, Abraham, la imagen del Padre, que \textquote{tanto amó al mundo que le entregó a su Hijo unigénito} (\textit{Jn} 3, 16). No tanto a Isaac, sino a Cristo. \textquote{Si Dios está con nosotros, ¿quién estará contra nosotros?} (cf. \textit{Rm} 8, 31). Abraham, figura del Padre. De este Padre que, en el Monte de la Transfiguración se revela en la voz: \textquote{Este es mi Hijo amado, escuchadlo} (\textit{Mc} 9, 7).

Con estas palabras prepara a los Apóstoles para el misterio que estará contenido en los acontecimientos de la Pascua en Jerusalén. \textquote{¡Escuchadlo a él!}. Aquí está la obediencia de la fe de todos los discípulos, –los cuales, por esta virtud conforman la descendencia de Abraham– basada de generación en generación en la elocuencia de la cruz y de la resurrección, en la que el Hijo ha revelado plenamente el amor del Padre.
\end{body}

\img{cross_tau}

\label{b2-03-02-1988H}

\newpage 
\subsubsection{Homilía (1991): Cruz que transforma}

\src{Visita pastoral a la parroquia romana de la Santísima Trinidad. \\Domingo 24 de febrero de 1991.}

\begin{body}

\ltr[1. «]{J}{esús} tomó consigo a Pedro, Santiago y Juan y los llevó a un monte alto, a un lugar apartado\ldots y se transfiguró delante de ellos» (\textit{Mc} 9, 2).

Queridos hermanos y hermanas, en el itinerario penitencial de la Cuaresma, la liturgia de hoy nos invita a hacer una pausa para contemplar la divina Transfiguración de Cristo.

Este es un hecho clave no solo en la experiencia terrena de Jesús, el Siervo obediente y sufriente que va a Jerusalén para cumplir, con su sacrificio pascual, la misión que le ha confiado el Padre, sino también para la experiencia de fe de los discípulos, que caminan con él hacia la misma meta, y de toda la comunidad de creyentes que \textquote{entre las persecuciones del mundo y los consuelos de Dios prosigue su peregrinaje terrenal} (San Agustín, \textit{De Civitate Dei}, XVIII, 51, 2) hacia la eterna Pascua.

2. Jesús, por tanto, va camino de Jerusalén, donde tendrá que \textquote{sufrir mucho, ser reprobado por los ancianos, los sumos sacerdotes y los escribas, luego ser matado y, a los tres días, resucitar} (\textit{Mc} 8, 31). Allí, de hecho, se cumplirán las antiguas profecías que predijeron la venida del Mesías, no como un gobernante poderoso o agitador político, sino como un humilde y manso Siervo de Dios y de los hombres, que tendrá que dar su vida en sacrificio, pasando por el camino de la persecución, el sufrimiento y la muerte.

Jesús tiene ante sí una meta difícil hacia la que la voluntad de Dios lo empuja y su vocación de \textquote{Siervo} lo dirige, y al mismo tiempo predice su trágico y glorioso epílogo. Su humanidad, para pasar la prueba, debe ser \textquote{confirmada} por el amor poderoso del Padre y consolada por la solidaridad de los discípulos que caminan con él. Y así les introduce en la comprensión de lo que está por realizarse, para que se hagan sus \textquote{compañeros} en el camino que tendrá que seguir hasta el final. En efecto, ellos parecen dispuestos a seguirlo con palabras, pero, ante los hechos, se retiran temerosos y escandalizados.

Hay una pausa en este camino hacia la Cruz. Jesús, con sus más fieles discípulos, sube al monte y allí, por un momento, les hace vislumbrar su destino último: la gloriosa Resurrección. Pero también les anticipa que primero es necesario seguirlo por el camino de la Pasión y la Cruz.

3. Queridos hermanos y hermanas (\ldots), sentiros también desafiados por el acontecimiento de la Transfiguración y por la invitación divina que os urge a seguir a Cristo. La \textquote{palabra de la Cruz} debe transformar no sólo a vosotros, sino a toda la Iglesia (\ldots), que vive el tiempo favorable de la Cuaresma, como momento fuerte de ese camino de fe y renovación  \txtsmall{[que quiere ser el Sínodo pastoral diocesano].}

Es muy importante que el itinerario espiritual y pastoral caracterice de forma indeleble la existencia de la fe personal y de toda la comunidad eclesial. Lo que hay que poner en primer lugar y que hay que tener siempre presente, para no perderse y no tomar el camino equivocado, es la escucha de Dios. Sólo pasando \textquote{por la muerte podemos alcanzar el triunfo de la resurrección} (cf. \textit{Prefacio}). Escuchar es lo que define al discípulo y lo convierte en servidor de la verdad y del amor de Dios, manifestado en plenitud en Cristo Jesús: \textquote{Presta atención y ven a mí –te dice por medio del profeta– y vivirás. Estableceré contigo una alianza eterna} (\textit{Is} 55, 3).

Sin duda el camino es difícil; pide disponibilidad, coraje, renuncia, para poder hacer de la propia vida, como Cristo hizo de la suya, un \textquote{don} de amor al Padre y a los hermanos. Sólo así podremos hacernos aptos, por la fuerza del Espíritu, para anunciar el \textquote{Evangelio de la Cruz} y realizar esa \textquote{nueva evangelización} que tiene su centro y eje en Cristo crucificado y resucitado. El anuncio, del que los discípulos son portadores, es exigente, difícil de comprender y sobre todo de acoger y vivir. Pero no están solos: están en comunión unos con otros y con Cristo, que murió y resucitó y ahora está glorioso a la diestra del Padre que intercede por ellos (cf. \textit{Rm} 8, 34b). Esta certeza que se basa en la fe, mientras nos consuela en medio de las dificultades, ¡nos empuja, como hijos de Abraham, a esperar contra toda esperanza!

4. Precisamente para que esta esperanza no disminuya, sino que crezca día a día, en el camino de la escucha y el anuncio de la \textquote{palabra de la Cruz}, es indispensable de vez en cuando, queridos hermanos y hermanas, subir al monte con Jesús y estar con él: es decir, estar más atentos a la voz de Dios y dejarse envolver y transformar por el Espíritu. En una palabra: ¡la experiencia de la contemplación y la oración es necesaria! \textquote{La oración, de hecho, es un bien supremo. Es\ldots una comunión íntima con Dios. Así como los ojos del cuerpo al ver la luz están iluminados por él, así también el alma que se extiende hacia Dios es iluminada por la luz inefable de la oración} (San Juan Crisóstomo, \textit{Homilía} 6, \textquote{\textit{De Oratione}}).

Esto no hemos de hacerlo, ciertamente, para escapar de la dureza de la vida cotidiana y huir de los gravosos compromisos del servicio al hombre, sino para disfrutar de la familiaridad con Dios, para luego reanudar con renovado vigor el fatigoso camino de la Cruz, que conduce a la resurrección.

5. Estos son los principales pensamientos que nos llegan de la liturgia de este segundo domingo de Cuaresma y que quería meditar con vosotros, queridos hermanos y hermanas (\ldots).

A todos los aquí presentes \ldots les dirijo mi cordial saludo y mis mejores deseos para todos\ldots [\ldots] Queridos amigos, os animo a continuar con renovado compromiso en este camino, por una convivencia humana y civil más justa, más serena y más fraterna y por una continua elevación de la vida espiritual y moral del entorno en el que están llamados a vivir y a dar testimonio de vuestra fe. Que el Señor Jesús, que hoy os invita a \textquote{transfiguraros}, os ayude a transformar y mejorar vuestra vida a la luz resplandeciente de su gracia.

\newpage 
6. Sí, queridos hermanos y hermanas, caminad juntos ante el Señor Dios y en fidelidad a Cristo, no solo en este tiempo de Cuaresma, sino a lo largo de vuestra vida.

Así, vuestro barrio y la ciudad de [Roma] se convertirán verdaderamente en \textquote{la tierra de los vivos}; es decir, la tierra en la que Dios habita y continúa revelándose en su Hijo y en la que florecen la verdad, la esperanza, el amor y la paz.

¡Amén!
\end{body}

\label{b2-03-02-1991H}
%\newpage 

\subsubsection{Homilía (1994): Amor que destruye el pecado}

\src{Visita pastoral a la parroquia romana de San Alejandro. \\27 de febrero de 1994.}

\begin{body}
\ltr{H}{emos} escuchado la Palabra de Dios, tratemos de resumir lo que hemos escuchado. Me vienen a la mente las palabras de San Juan: \textquote{Tanto amó Dios al mundo que dio a su Hijo} (\textit{Jn} 3, 16). Esto es lo que quiere decirnos la liturgia de hoy. Especialmente con la \textbf{primera lectura} que habla de Abraham. Abraham estaba dispuesto a dar a su único hijo. Fue una gran profecía.

No conocemos sus palabras, no ha escrito libros, pero este gesto de estar dispuesto a entregar a su único hijo, Isaac, en holocausto a Dios es ya una grandísima profecía que anticipa todo el misterio pascual. ¿Qué significa el evangelio de hoy para nosotros? Dios se prepara para dar a su único Hijo primogénito, Jesús hecho hombre, para darlo como sacrificio por todos los pecados del mundo.

Dios no permitió que Abraham ofreciera a su hijo Isaac, pero no renunció a dar a su único Hijo, Jesús, el cual se prepara para este holocausto de la Semana Santa, del Sagrado Triduo, y también prepara a sus Apóstoles. Para ello los lleva junto con él al monte Tabor, y allí el Padre les muestra cómo Jesús es su amado. \textquote{He aquí mi Hijo, mi Hijo amado} (cf. \textit{Mc} 9, 7). Lo manifiesta a los dos Testamentos: lo hace ante los profetas, lo hace ante Moisés, Elías, y evidentemente lo hace ante estos tres Apóstoles escogidos para ser testigos: Pedro, Santiago y Juan.

Jesús se apareció a sus Apóstoles transfigurado, elevado al cielo en su gloriosa figura. Se dice: \textquote{transfiguración}, se trata de una figura celeste después de la figura terrenal. La figura celestial de Jesús apareció precisamente en el monte Tabor. Los Apóstoles se asombran y dicen: \textquote{¡Qué bueno es que estemos aquí!}. Y Jesús les dice algunas palabras un tanto enigmáticas. Les dice que no le digan a nadie lo que vieron antes de la Resurrección.

Los apóstoles se preguntan qué significa la resurrección, qué significa ser resucitado. ¿Significa estar muerto primero? Jesús, con estas enigmáticas palabras, ya anunciaba la Semana Santa, el Viernes Santo, la Pascua.

Así la Iglesia hoy, con estas estupendas lecturas, nos prepara para la solemnidad pascual. Hace esto todos los años. Cuando era joven, me preguntaba por qué había estas lecturas el segundo domingo de Cuaresma, especialmente el pasaje evangélico de la Transfiguración. Hoy comprendo bien que esto se deba al misterio pascual y a la preparación pascual en Cuaresma.

Entonces, ¿qué dice San Pablo en la \textbf{segunda lectura}? San Pablo nos habla casi todos los domingos. Nos dice: ¿quién nos separará del amor de Dios manifestado en Cristo Jesús?

[\ldots]

Esto le da a San Pablo la oportunidad de preguntarse: ¿quién nos separará del amor de Cristo? \ldots El amor de Cristo es más fuerte. Por eso, queridos amigos, durante la Cuaresma debemos recordar siempre, cada año, renovar la conciencia de que el amor de Cristo es más fuerte que todo. Pablo se pregunta: ¿quién nos separará? ¿El pecado? El pecado no es nada delante de Él. Sí, es una falta, pesa en la conciencia del hombre, pero ante la Resurrección, especialmente la Pasión, la Cruz de Cristo, el amor de Cristo, el pecado es derrotado. Podemos eliminarlo, podemos superarlo, podemos pedir perdón.

Y este es el mensaje continuo de Cuaresma. Se repite todos los años, a todos y cada uno. Se repite con la fuerza de nuestros santos, apóstoles y mártires, testigos de la Transfiguración. Estamos llamados a transfigurarnos durante la Cuaresma, a ser semejantes al glorioso Jesús.

Nosotros también estamos llamados a la gloria, a participar de su gloria. Esto es lo que nos dice la bella liturgia de este domingo.

 \txtsmall{[Estoy muy feliz de estar aquí y traer este mensaje, en este maravilloso entorno, en esta basílica paleocristiana vinculada a la memoria de San Alejandro, y a estas catacumbas, que son todas testigos del heroísmo de la vida cristiana de tantos desconocidos como Alejandro. Heroísmo de la vida cristiana que fue posible para ellos y será posible también para nosotros porque la gracia de Dios es siempre más fuerte\ldots]}

 \txtsmall{[\ldots]}

Se perciben muchas fuerzas vivas trabajando por la Resurrección, por la Transfiguración del mundo, por un mundo mejor. Este trabajo es muy necesario. Hay que cambiar el mundo: no podemos permanecer en nuestra antigua forma, no transfigurarnos. El mundo necesita una transfiguración profunda, la que viene de Jesús.

¡Queridos amigos, estas son las pocas palabras que quería deciros en este encuentro que nos prepara para el Ofertorio y la Santísima Comunión Eucarística con Cristo!

¡Alabado sea Jesucristo!
\end{body}

\label{b2-03-02-1994H}
\newpage 

\subsubsection{Homilía (1997): Amor revelado en la Cruz}

\src{Visita Pastoral a la Parroquia Romana de la Santa Cruz.\\23 de febrero de 1997.}

\begin{body}
\textquote{Este es mi Hijo amado: escuchadlo} (\textit{Mc} 9, 7). 

\ltr{H}{oy,} en el marco de la transfiguración del Señor, volvemos a escuchar estas palabras, que resonaron en el momento del bautismo de Jesús en el Jordán (cf. \textit{Mt} 3, 17). \textquote{Jesús se llevó a Pedro, a Santiago y a Juan (\ldots), y se transfiguró delante de ellos (\ldots). Se les aparecieron Elías y Moisés conversando con Jesús (\ldots). Pedro (\ldots) le dijo a Jesús: \textquote{Maestro. ¡Qué bien se está aquí! Vamos a hacer tres chozas, una para ti, otra para Moisés y otra para Elías}} (\textit{Mc} 9, 2-5). En ese preciso instante se oyó una voz: \textquote{Este es mi Hijo amado; escuchadlo} (\textit{Mc} 9, 7).

No duró mucho esa extraordinaria manifestación de la filiación divina de Jesús. Cuando los Apóstoles alzaron nuevamente su mirada, no vieron más que a Jesús, el cual, \textquote{cuando bajaban de la montaña –prosigue el evangelista– (\ldots), les mandó: \textquote{No contéis a nadie lo que habéis visto hasta que el Hijo del hombre resucite de entre los muertos}} (\textit{Mc} 9, 9).

Así, en este segundo domingo de Cuaresma, escuchamos junto con los Apóstoles el anuncio de la Resurrección. Lo escuchamos mientras nos encaminamos con ellos hacia Jerusalén, donde reviviremos el misterio de la pasión y muerte del Señor. En efecto, el ayuno y la penitencia de este tiempo sagrado se orientan precisamente hacia este acontecimiento-clave de toda la economía salvífica.

La transfiguración del Señor, que según la tradición tuvo lugar en el monte Tabor, sitúa en primer plano la persona y la obra de Dios Padre, presente junto al Hijo de modo invisible pero real. Esto explica el hecho de que, en el trasfondo del evangelio de la Transfiguración, la liturgia de hoy sitúa un importante episodio del Antiguo Testamento, en el que se pone de relieve de modo particular la paternidad.

En efecto, la \textbf{primera lectura}, tomada del libro del \textbf{Génesis}, nos recuerda el sacrificio de Abraham. Éste tenía un hijo, Isaac, que había nacido en su vejez. Era el hijo de la promesa. Pero un día Abraham recibe de Dios la orden de ofrecerlo en sacrificio. El anciano patriarca se encuentra ante la perspectiva de un sacrificio que para él, padre, es seguramente el mayor que se pueda imaginar. A pesar de ello, no duda ni un instante y, después de haber preparado lo necesario, parte con Isaac hacia el lugar establecido. Construye un altar, coloca la leña y, una vez atado el muchacho, toma el cuchillo para inmolarlo. Sólo entonces lo detiene una orden de lo alto: \textquote{No alargues tu mano contra tu hijo, ni le hagas nada, que ahora ya sé que tú eres temeroso de Dios, ya que no me has negado tu hijo, tu único hijo} (\textit{Gn} 22, 12).

Es conmovedor este acontecimiento, en el que la fe y el abandono de un padre en las manos de Dios alcanzan la cima. Con razón san Pablo llama a Abraham \textquote{padre de todos los creyentes} (\textit{Rm} 4, 11. 17). Tanto la religión judía como la cristiana hacen referencia a su fe. El Corán destaca también la figura de Abraham. La fe del padre de los creyentes es un espejo en el que se refleja el misterio de Dios, misterio de amor que une al Padre y al Hijo.

[\ldots]

\textquote{El que no perdonó a su propio Hijo, sino que lo entregó a la muerte por nosotros, ¿cómo no nos dará todo con él?} (\textit{Rm} 8, 32). Estas palabras de \textbf{san Pablo} en la \textbf{carta a los Romanos} nos introducen en el tema fundamental de la liturgia de hoy: el misterio del amor divino revelado en el sacrificio de la cruz.

El sacrificio de Isaac anticipa el de Cristo: el Padre no escatimó a su propio Hijo, sino que lo entregó para la salvación del mundo. Él, que detuvo el brazo de Abraham en el momento en que estaba a punto de inmolar a Isaac, no dudó en sacrificar a su propio Hijo por nuestra redención. De ese modo, el sacrifico de Abraham pone de relieve que nunca y en ningún lugar se deben realizar sacrificios humanos, porque el único sacrificio verdadero y perfecto es el del Hijo unigénito y eterno de Dios vivo. Jesús, que por nosotros y por nuestra salvación nació de María virgen, se inmoló voluntariamente una vez para siempre, como víctima de expiación por nuestros pecados, obteniéndonos así la salvación total y definitiva (cf. \textit{Hb} 10, 5-10). Después del sacrifico del Hijo de Dios, no se necesita ninguna otra expiación humana, puesto que su sacrificio en la cruz abarca y supera todos los demás sacrificios que el hombre podría ofrecer a Dios. Aquí nos encontramos en el centro del misterio pascual.

Desde el Tabor, el monte de la transfiguración, el itinerario cuaresmal nos lleva hasta el Gólgota, el monte del sacrifico supremo del único sacerdote de la alianza nueva y eterna. Dicho sacrificio encierra la mayor fuerza de transformación del hombre y de la historia. Asumiendo en sí mismo todas las consecuencias del mal y del pecado, Jesús resucitará al tercer día y saldrá de esa dramática experiencia como vencedor de la muerte, del infierno y de Satanás. La Cuaresma nos prepara para participar personalmente en este gran misterio de la fe, que celebraremos en el triduo de la pasión, la muerte y la resurrección de Cristo.

Pidamos al Señor la gracia de prepararnos de modo conveniente: \textquote{Jesús, Hijo amado del Padre, haz que te escuchemos y te sigamos hasta el Calvario, hasta la cruz, para poder participar contigo en la gloria de la resurrección}. Amén.
\end{body}


\label{b2-03-02-1997A}
\newpage 


\subsubsection{Ángelus (1997): Subida al Tabor}

\src{23 de febrero de 1997.}

\begin{body}
\ltr{E}{n} este segundo domingo de Cuaresma, la liturgia nos presenta la Transfiguración en el monte Tabor. Es la revelación de la gloria, que precede la prueba suprema de la cruz y anticipa la victoria de la resurrección.

Pedro, Santiago y Juan fueron testigos de este evento extraordinario. El evangelio de hoy relata que Jesús los llamó aparte y los llevó consigo \textquote{a un monte alto} (\textit{Mc} 9, 2).

La subida de los discípulos al Tabor nos impulsa a reflexionar sobre el itinerario penitencial de estos días. También la Cuaresma es un camino de subida. Es una invitación a redescubrir el silencio pacificador y regenerador de la meditación. Se trata de un esfuerzo de purificación del corazón, para liberarlo del pecado que pesa sobre él. Ciertamente se trata de un camino arduo, pero que orienta hacia una meta rica en belleza, esplendor y alegría.

En la Transfiguración se oye la voz del Padre celestial: \textquote{Este es mi Hijo amado, escuchadlo} (\textit{Mc} 9, 7). Estas palabras encierran todo el programa de la Cuaresma: debemos ponernos a la escucha de Jesús. Él nos revela al Padre, porque, como Hijo eterno, es \textquote{imagen de Dios invisible} (\textit{Col} 1, 15). Pero, al mismo tiempo, como verdadero \textquote{Hijo del hombre}, revela lo que sabemos, revela el hombre al hombre (cf. \textit{Gaudium et spes}, 22). Por tanto, ¡no tengamos miedo a Cristo! Al elevarnos a la altura de su vida divina, no nos aleja de nuestra humanidad sino que, por el contrario, nos humaniza, dando sentido pleno a nuestra existencia personal y social. [A este redescubrimiento cada vez más vivo de Jesús también nos impulsa la perspectiva del gran jubileo, que en este primer año de preparación inmediata se centra principalmente en] la contemplación de Cristo: una contemplación que debe alimentarse del Evangelio y de la oración, y que siempre tiene que ir acompañada por una conversión auténtica y por el redescubrimiento constante de la caridad como ley de vida diaria.

Queridos hermanos, contemplemos a María, la Virgen a la escucha, siempre dispuesta a acoger y conservar en su corazón cada una de las palabras de su Hijo divino (cf. \textit{Lc} 2, 51). El evangelio la define \textquote{feliz porque ha creído que se cumplirían las cosas que le fueron dichas de parte del Señor} (\textit{Lc} 1, 45). La Madre celestial de Dios nos ayude a entrar en sintonía profunda con la palabra de Dios, para que Cristo se convierta en luz y guía de toda nuestra vida.
\end{body}

\label{b2-03-02-1997A}
\newpage 

\subsubsection{Homilía (2000): Paternidad por la fe}

\src{Celebración Eucarística en el Jubileo de los Artesanos. \\19 de marzo del 2000.}

\begin{body}
\ltr{D}{ios,} \textquote{que no perdonó a su propio Hijo, sino que lo entregó a la muerte por todos nosotros, ¿cómo no nos dará todo con él?} (\textit{Rm} 8, 32).

El apóstol Pablo, en la \textbf{carta a los Romanos}, formula esta pregunta, en la que destaca con claridad el tema central de la liturgia de este día: el misterio de la paternidad de Dios. En el pasaje evangélico es el mismo Padre eterno quien se presenta a nosotros cuando, desde la nube luminosa que envuelve a Jesús y a los Apóstoles en el monte de la Transfiguración, hace oír su voz, que exhorta: \textquote{Éste es mi Hijo amado, escuchadlo} (\textit{Mc} 9, 7). Pedro, Santiago y Juan intuyen –luego lo comprenderán mejor– que Dios les ha hablado revelándose a sí mismo y el misterio de su realidad más íntima.

Después de la resurrección, ellos, junto con los demás Apóstoles, llevarán al mundo este impresionante anuncio: en su Hijo encarnado Dios se ha acercado a todo hombre como Padre misericordioso. En Cristo todo ser humano es envuelto por el abrazo tierno y fuerte de un Padre. [\ldots] Cristo es el Hijo amado del Padre. Es, sobre todo, la palabra \textquote{amado} la que, respondiendo a nuestros interrogantes, descorre en cierto modo el velo que oculta el misterio de la paternidad divina. En efecto, nos da a conocer el amor infinito del Padre al Hijo y, al mismo tiempo, nos revela su \textquote{pasión} por el hombre, por cuya salvación no duda en entregar a este Hijo tan amado. Todo ser humano puede saber ya que en Jesús, Verbo encarnado, es objeto de un amor ilimitado por parte del Padre celestial.

Una contribución ulterior al conocimiento de este misterio nos la da la \textbf{primera lectura}, tomada del libro del \textbf{Génesis}. Dios pide a Abraham el sacrificio de su hijo: \textquote{Toma a tu hijo único, al que quieres, a Isaac, y vete al país de Moria y ofrécemelo en sacrificio, sobre uno de los montes que yo te indicaré} (\textit{Gn} 22, 2). Con el corazón destrozado, Abraham se dispone a cumplir la orden de Dios. Pero, cuando está a punto de clavar a su hijo el cuchillo del sacrificio, el Señor lo detiene y, por medio de un ángel, le dice: \textquote{No alargues la mano contra tu hijo ni le hagas nada. Ahora sé que temes a Dios, porque no te has reservado a tu hijo, tu único hijo} (\textit{Gn} 22, 12).

A través de las vicisitudes de una paternidad humana sometida a una prueba dramática, se revela otra paternidad, basada en la fe. Precisamente en virtud del extraordinario testimonio de fe dado en aquella circunstancia, Abraham obtiene la promesa de una descendencia numerosa: \textquote{Todos los pueblos del mundo se bendecirán con tu descendencia, porque me has obedecido} (\textit{Gn} 22, 18). Gracias a su fe incondicional en la palabra de Dios, Abraham se convierte en padre de todos los creyentes.

Dios Padre \textquote{no perdonó a su propio Hijo, sino que lo entregó a la muerte por nosotros} (\textit{Rm} 8, 32). Abraham, con su disponibilidad a inmolar a Isaac, anuncia el sacrificio de Cristo por la salvación del mundo. La ejecución efectiva del sacrificio, que le fue ahorrada a Abraham, se realizará con Jesucristo. Él mismo informa a los Apóstoles: al bajar del monte de la Transfiguración, les prohíbe que cuenten lo que han visto antes de que el Hijo del hombre resucite de entre los muertos. El evangelista añade: \textquote{Esto se les quedó grabado y discutían qué querría decir aquello de resucitar de entre los muertos} (\textit{Mc} 9, 10).

Los discípulos intuyen que Jesús es el Mesías y que en él se realiza la salvación. Pero no logran comprender por qué habla de pasión y de muerte: no aceptan que el amor de Dios pueda esconderse detrás de la cruz. Y, sin embargo, donde los hombres verán sólo una muerte, Dios manifestará su gloria, resucitando a su Hijo; donde los hombres pronunciarán palabras de condena, Dios realizará su misterio de salvación y amor al género humano.

Ésta es la lección que cada generación cristiana debe volver a aprender. Cada generación, ¡también la nuestra! Aquí radica la razón de ser de nuestro camino de conversión en este tiempo singular de gracia. [El jubileo ilumina toda la vida y la experiencia de los hombres.] Incluso la fatiga y el cansancio del trabajo diario reciben de la fe en Cristo muerto y resucitado una nueva luz de esperanza. Aparecen como elementos significativos del designio de salvación que el Padre celestial está realizando mediante la cruz de su Hijo.

\txtsmall{[Apoyados en esta certeza, queridos artesanos, podéis fortalecer y concretar los valores que desde siempre caracterizan vuestra actividad: el perfil cualitativo, el espíritu de iniciativa, la promoción de las capacidades artísticas, la libertad y la cooperación, la relación correcta entre tecnología y ambiente, el arraigo familiar y las buenas relaciones de vecindad. La civilización artesana ha sabido crear, en el pasado, grandes ocasiones de encuentro entre los pueblos, y ha transmitido a las épocas sucesivas síntesis admirables de cultura y fe.

El misterio de la vida de Nazaret, del que san José, patrono de la Iglesia y vuestro protector, fue custodio fiel y testigo sabio, es el icono de esta admirable síntesis entre vida de fe y trabajo humano, entre crecimiento personal y compromiso de solidaridad.

Amadísimos artesanos, habéis venido hoy para celebrar vuestro jubileo. Que la luz del Evangelio ilumine cada vez más vuestra experiencia laboral diaria. El jubileo os ofrece la ocasión de encontraros con Jesús, José y María, entrando en su casa y en el humilde taller de Nazaret.

En la singular escuela de la Sagrada Familia se aprenden las realidades esenciales de la vida y se profundiza el significado del seguimiento de Jesús. Nazaret enseña a superar la tensión aparente entre la vida activa y la contemplativa; invita a crecer en el amor a la verdad divina que irradia la humanidad de Cristo y a prestar con valentía el exigente servicio de la tutela de Cristo presente en todo hombre (cf. \textit{Redemptoris custos}, 27).

Crucemos, por tanto, en una peregrinación espiritual, el umbral de la casa de Nazaret, el humilde hogar que tendré la alegría de visitar, Dios mediante, la próxima semana, durante mi peregrinación jubilar a Tierra Santa.

Contemplemos a María, testigo del cumplimiento de la promesa hecha por el Señor \textquote{en favor de Abraham y su descendencia por siempre} (\textit{Lc} 1, 54-55).

Que ella, junto con José, su casto esposo, os ayude, queridos artesanos, a permanecer en constante escucha de Dios, uniendo oración y trabajo. Ellos os sostengan en vuestros propósitos jubilares de renovada fidelidad cristiana y hagan que vuestras manos prolonguen, en cierto modo, la obra creadora y providente de Dios.

La Sagrada Familia, lugar de entendimiento y amor, os ayude a realizar gestos de solidaridad, paz y perdón. Así, seréis heraldos del amor infinito de Dios Padre, rico en misericordia y bondad para con todos.]}

Amén.
\end{body}

\begin{patercite}
La Transfiguración del Señor, (\ldots)  proyecta una luz deslumbrante sobre nuestra vida diaria y nos lleva a dirigir la mente al destino inmortal que este hecho esconde.

En la cima del Tabor, durante unos instantes, Cristo levanta el velo que oculta el resplandor de su divinidad y se manifiesta a los testigos elegidos como es realmente, el Hijo de Dios, \textquote{el esplendor de la gloria del Padre y la imagen de su substancia} (cf. \textit{Heb} 1, 5); pero al mismo tiempo desvela el destino trascendente de nuestra naturaleza humana que Él ha tomado para salvarnos, destinada también ésta (por haber sido redimida por su sacrificio de amor irrevocable) a participar en la plenitud de la vida, en la \textquote{herencia de los santos en la luz} (\textit{Col} 1, 12).

Ese cuerpo que se transfigura ante los ojos atónitos de los Apóstoles es el cuerpo de Cristo nuestro hermano, pero es también nuestro cuerpo destinado a la gloria; la luz que le inunda es y será también nuestra parte de herencia y de esplendor.

Estamos llamados a condividir tan gran gloria, porque somos \textquote{partícipes de la divina naturaleza} (\textit{2 Pe} 1. 4).

Nos espera una suerte incomparable, en el caso de que hayamos hecho honor a nuestra vocación cristiana y hayamos vivido con la lógica
consecuencia de palabras y comportamiento, a que nos obligan los compromisos de nuestro bautismo.

\textbf{San Pablo VI, papa}, \textit{Ángelus}, 6 de agosto de 1978.

\tiny{(*) Esta alocución que el Papa había preparado para dirigirla a los files no pudo ser leída por él debido a que su estado de salud se agravó. Pablo VI descansó en la paz del Señor ese mismo día, domingo 6 de agosto, fiesta de la Transfiguración del Señor a las 21,40 horas.}
\end{patercite}
\label{b2-03-02-2000H}

\newsection
\subsection{Benedicto XVI, papa}

\subsubsection{Ángelus (2006): Escuchar a Dios}

\src{12 de marzo del 2006.}

\begin{body}
\ltr[(\ldots)]{E}{n} el tiempo de Cuaresma estamos llamados de un modo particular a la escucha del Señor, que siempre nos habla, pero espera de nosotros mayor atención \ldots 

Nos lo recuerda también la \textbf{página evangélica} de este domingo, que propone de nuevo la narración de la transfiguración de Cristo en el monte Tabor. Mientras estaban atónitos en presencia del Señor transfigurado, que conversaba con Moisés y Elías, Pedro, Santiago y Juan fueron envueltos repentinamente por una nube, de la que salió una voz que proclamó: \textquote{Este es mi Hijo amado; escuchadlo} (\textit{Mc} 9, 7). 

Cuando se tiene la gracia de vivir una fuerte experiencia de Dios, es como si se viviera algo semejante a lo que les sucedió a los discípulos durante la Transfiguración: por un momento se gusta anticipadamente algo de lo que constituirá la bienaventuranza del paraíso. En general, se trata de breves experiencias que Dios concede a veces, especialmente con vistas a duras pruebas. Pero a nadie se le concede vivir \textquote{en el Tabor} mientras está en esta tierra. En efecto, la existencia humana es un camino de fe y, como tal, transcurre más en la penumbra que a plena luz, con momentos de oscuridad e, incluso, de tinieblas. Mientras estamos aquí, nuestra relación con Dios se realiza más en la escucha que en la visión; y la misma contemplación se realiza, por decirlo así, con los ojos cerrados, gracias a la luz interior encendida en nosotros por la palabra de Dios.

También la Virgen María, aun siendo entre todas las criaturas humanas la más cercana a Dios, caminó día a día como en una peregrinación de la fe (cf. \textit{Lumen gentium}, 58), conservando y meditando constantemente en su corazón las palabras que Dios le dirigía, ya sea a través de las Sagradas Escrituras o bien mediante los acontecimientos de la vida de su Hijo, en los que reconocía y acogía la misteriosa voz del Señor. 

He aquí, pues, el don y el compromiso de cada uno de nosotros durante el tiempo cuaresmal: escuchar a Cristo, como María. Escucharlo en su palabra, custodiada en la Sagrada Escritura. Escucharlo en los acontecimientos mismos de nuestra vida, tratando de leer en ellos los mensajes de la Providencia. Por último, escucharlo en los hermanos, especialmente en los pequeños y en los pobres, para los cuales Jesús mismo pide nuestro amor concreto. Escuchar a Cristo y obedecer su voz: este es el camino real, el único que conduce a la plenitud de la alegría y del amor.
\end{body}


\label{b2-03-02-2006A}

\begin{patercite}
El evento de la Transfiguración del Señor nos ofrece un mensaje de esperanza ---así seremos nosotros, con Él---: nos invita a encontrar a Jesús, para estar al servicio de los hermanos.

La ascensión de los discípulos al monte Tabor nos induce a reflexionar sobre la importancia de separarse de las cosas mundanas, para cumplir un camino hacia lo alto y contemplar a Jesús. Se trata de ponernos a la escucha atenta y orante del Cristo, el Hijo amado del Padre, buscando momentos de oración que permiten la acogida dócil y alegre de la Palabra de Dios. En esta ascensión espiritual, en esta separación de las cosas mundanas, estamos llamados a redescubrir el silencio pacificador y regenerador de la meditación del Evangelio, de la lectura de la Biblia, que conduce hacia una meta rica de belleza, de esplendor y de alegría. Y cuando nosotros nos ponemos así, con la Biblia en la mano, en silencio, comenzamos a escuchar esta belleza interior, esta alegría que genera la Palabra de Dios en nosotros.

Al finalizar la experiencia maravillosa de la Transfiguración, los discípulos bajaron del monte (cf. v. 9) con ojos y corazón transfigurados por el encuentro con el Señor. Es el recorrido que podemos hacer también nosotros. El redescubrimiento cada vez más vivo de Jesús no es fin en sí mismo, pero nos lleva a \textquote{bajar del monte}, cargados con la fuerza del Espíritu divino, para decidir nuevos pasos de conversión y para testimoniar constantemente la caridad, como ley de vida cotidiana. Transformados por la presencia de Cristo y del ardor de su palabra, seremos signo concreto del amor vivificante de Dios para todos nuestros hermanos, especialmente para quien sufre, para los que se encuentran en soledad y abandono, para los enfermos y para la multitud de hombres y de mujeres que, en distintas partes del mundo, son humillados por la injusticia, la prepotencia y la violencia. En la Transfiguración se oye la voz del Padre celeste que dice: \textquote{Este es mi hijo amado, ¡escuchadle!} (v. 5). Miremos a María, la \emph{Virgen de la escucha}, siempre
preparada a acoger y custodiar en el corazón cada palabra del Hijo divino (cf. \emph{Lucas} 1, 51). Quiera nuestra Madre y Madre de Dios
ayudarnos a entrar en sintonía con la Palabra de Dios, para que Cristo se convierta en luz y guía de toda nuestra vida. [\ldots]

\textbf{Francisco, papa}, \textit{Ángelus}, 6 de agosto del 2017, parr. 2-3.
\end{patercite}
\newpage


\subsubsection{Ángelus (2009): Una experiencia de oración}

\src{8 de marzo del 2009.}

\begin{body}
\ltr{J}{esús} [como nos muestra el \textbf{Evangelio} de hoy] llevó a los apóstoles Pedro, Santiago y Juan, solos a un monte alto, en un lugar apartado, y mientras oraba se \textquote{transfiguró}: su rostro y su persona se volvieron luminosos, resplandecientes.

La liturgia vuelve a proponer este célebre episodio precisamente hoy, segundo domingo de Cuaresma (cf. \textit{Mc} 9, 2-10). Jesús quería que sus discípulos, de modo especial los que tendrían la responsabilidad de guiar a la Iglesia naciente, experimentaran directamente su gloria divina, para afrontar el escándalo de la cruz. En efecto, cuando llegue la hora de la traición y Jesús se retire a rezar a Getsemaní, tomará consigo a los mismos Pedro, Santiago y Juan, pidiéndoles que velen y oren con él (cf. \textit{Mt} 26, 38). Ellos no lo lograrán, pero la gracia de Cristo los sostendrá y les ayudará a creer en la resurrección.

Quiero subrayar que la Transfiguración de Jesús fue esencialmente una experiencia de oración (cf. \textit{Lc} 9, 28-29). En efecto, la oración alcanza su culmen, y por tanto se convierte en fuente de luz interior, cuando el espíritu del hombre se adhiere al de Dios y sus voluntades se funden como formando una sola cosa. Cuando Jesús subió al monte, se sumergió en la contemplación del designio de amor del Padre, que lo había mandado al mundo para salvar a la humanidad. Junto a Jesús aparecieron Elías y Moisés, para significar que las Sagradas Escrituras concordaban en anunciar el misterio de su Pascua, es decir, que Cristo debía sufrir y morir para entrar en su gloria (cf. \textit{Lc} 24, 26. 46). En aquel momento Jesús vio perfilarse ante él la cruz, el extremo sacrificio necesario para liberarnos del dominio del pecado y de la muerte. Y en su corazón, una vez más, repitió su \textquote{Amén}. Dijo \textquote{sí}, \textquote{heme aquí}, \textquote{hágase, oh Padre, tu voluntad de amor}. Y, como había sucedido después del bautismo en el Jordán, llegaron del cielo los signos de la complacencia de Dios Padre: la luz, que transfiguró a Cristo, y la voz que lo proclamó \textquote{Hijo amado} (\textit{Mc} 9, 7).

Juntamente con el ayuno y las obras de misericordia, la oración forma la estructura fundamental de nuestra vida espiritual. Queridos hermanos y hermanas, os exhorto a encontrar en este tiempo de Cuaresma momentos prolongados de silencio, posiblemente de retiro, para revisar vuestra vida a la luz del designio de amor del Padre celestial. En esta escucha más intensa de Dios dejaos guiar por la Virgen María, maestra y modelo de oración. Ella, incluso en la densa oscuridad de la pasión de Cristo, no perdió la luz de su Hijo divino, sino que la custodió en su alma. Por eso, la invocamos como Madre de la confianza y de la esperanza.
\end{body}

\label{b2-03-02-2009A}
\newpage

\subsubsection{Homilía (2012): Sin cruz no hay salvación}

\src{Visita Pastoral a la Parroquia Romana \\de San Juan Bautista de la Salle en Torrino. \\4 de marzo del 2012.}

\begin{body}
\ltr{L}{a} liturgia de este día nos prepara sea para el misterio de la Pasión –como escuchamos en la primera lectura– sea para la alegría de la Resurrección.

La \textbf{primera lectura} nos refiere el episodio en el que Dios pone a prueba a Abrahán (cf. \textit{Gn} 22, 1-18). Abrahán tenía un hijo único, Isaac, que le nació en la vejez. Era el hijo de la promesa, el hijo que debería llevar luego la salvación también a los pueblos. Pero un día Abrahán recibe de Dios la orden de ofrecerlo en sacrificio. El anciano patriarca se encuentra ante la perspectiva de un sacrificio que para él, padre, es ciertamente el mayor que se pueda imaginar. Sin embargo, no duda ni siquiera un instante y, después de preparar lo necesario, parte junto con Isaac hacia el lugar establecido. Y podemos imaginar esta caminata hacia la cima del monte, lo que sucedió en su corazón y en el corazón de su hijo. Construye un altar, coloca la leña y, después de atar al muchacho, aferra el cuchillo para inmolarlo. Abrahán se fía de Dios hasta tal punto que está dispuesto incluso a sacrificar a su propio hijo y, juntamente con el hijo, su futuro, porque sin ese hijo la promesa de la tierra no servía para nada, acabaría en la nada. Y sacrificando a su hijo se sacrifica a sí mismo, todo su futuro, toda la promesa. Es realmente un acto de fe radicalísimo. En ese momento lo detiene una orden de lo alto: Dios no quiere la muerte, sino la vida; el verdadero sacrificio no da muerte, sino que es la vida, y la obediencia de Abrahán se convierte en fuente de una inmensa bendición hasta hoy. Dejemos esto, pero podemos meditar este misterio.

En la \textbf{segunda lectura}, san Pablo afirma que Dios mismo realizó un sacrificio: nos dio a su propio Hijo, lo donó en la cruz para vencer el pecado y la muerte, para vencer al maligno y para superar toda la malicia que existe en el mundo. Y esta extraordinaria misericordia de Dios suscita la admiración del Apóstol y una profunda confianza en la fuerza del amor de Dios a nosotros; de hecho, san Pablo afirma: \textquote{[Dios], que no se reservó a su propio Hijo, sino que lo entregó por todos nosotros, ¿cómo no nos dará todo con él?} (\textit{Rm} 8, 32). Si Dios se da a sí mismo en el Hijo, nos da todo. Y san Pablo insiste en la potencia del sacrificio redentor de Cristo contra cualquier otro poder que pueda amenazar nuestra vida. Se pregunta: \textquote{¿Quién acusará a los elegidos de Dios? Dios es el que justifica. ¿Quién condenará? ¿Acaso Cristo Jesús, que murió; más todavía, resucitó y está a la derecha de Dios y que además intercede por nosotros?} (\textit{Rm} 8, 33-34). Nosotros estamos en el corazón de Dios; esta es nuestra gran confianza. Esto crea amor y en el amor vamos hacia Dios. Si Dios ha entregado a su propio Hijo por todos nosotros, nadie podrá acusarnos, nadie podrá condenarnos, nadie podrá separarnos de su inmenso amor. Precisamente el sacrificio supremo de amor en la cruz, que el Hijo de Dios aceptó y eligió voluntariamente, se convierte en fuente de nuestra justificación, de nuestra salvación. Y pensemos que en la Sagrada Eucaristía siempre está presente este acto del Señor, que en su corazón permanece por toda la eternidad, y este acto de su corazón nos atrae, nos une a él.

Por último, el \textbf{Evangelio} nos habla del episodio de la Transfiguración (cf. \textit{Mc} 9, 2-10): Jesús se manifiesta en su gloria antes del sacrificio de la cruz y Dios Padre lo proclama su Hijo predilecto, el amado, e invita a los discípulos a escucharlo. Jesús sube a un monte alto y toma consigo a tres apóstoles –Pedro, Santiago y Juan–, que estarán especialmente cercanos a él en la agonía extrema, en otro monte, el de los Olivos. Poco tiempo antes el Señor había anunciado su pasión y Pedro no había logrado comprender por qué el Señor, el Hijo de Dios, hablaba de sufrimiento, de rechazo, de muerte, de cruz; más aún, se había opuesto decididamente a esta perspectiva. Ahora Jesús toma consigo a los tres discípulos para ayudarlos a comprender que el camino para llegar a la gloria, el camino del amor luminoso que vence las tinieblas, pasa por la entrega total de sí mismo, pasa por el escándalo de la cruz. Y el Señor debe volver a llevarnos siempre con Él, para que empecemos al menos a comprender que ese es el camino necesario. La transfiguración es un momento anticipado de luz que nos ayuda también a nosotros a contemplar la pasión de Jesús con una mirada de fe. La pasión de Jesús es un misterio de sufrimiento, pero también es la \textquote{bienaventurada pasión} porque en su núcleo es un misterio de amor extraordinario de Dios; es el éxodo definitivo que nos abre la puerta hacia la libertad y la novedad de la Resurrección, de la salvación del mal. Tenemos necesidad de ella en nuestro camino diario, a menudo marcado también por la oscuridad del mal.

[\ldots]

Por último, quiero recordaros a todos la importancia y la centralidad de la Eucaristía en la vida personal y comunitaria. La santa misa debe estar en el centro de vuestro Domingo, que es preciso redescubrir y vivir como día de Dios y de la comunidad, día en el cual alabar y celebrar a Aquel que murió y resucitó por nuestra salvación, día en el cual vivir juntos en la alegría de una comunidad abierta y dispuesta a acoger a toda persona sola o en dificultades. Reunidos en torno a la Eucaristía, de hecho, percibimos más fácilmente que la misión de toda comunidad cristiana consiste en llevar el mensaje del amor de Dios a todos los hombres. Precisamente por eso es importante que la Eucaristía esté siempre en el corazón de la vida de los fieles, como lo está hoy.

Queridos hermanos y hermanas, desde el Tabor, el monte de la Transfiguración, el itinerario cuaresmal nos conduce hasta el Gólgota, monte del supremo sacrificio de amor del único Sacerdote de la alianza nueva y eterna. En ese sacrificio se encierra la mayor fuerza de transformación del hombre y de la historia. Asumiendo sobre sí todas las consecuencias del mal y del pecado, Jesús resucitó al tercer día como vencedor de la muerte y del Maligno. La Cuaresma nos prepara para participar personalmente en este gran misterio de la fe, que celebraremos en el Triduo de la pasión, muerte y resurrección de Cristo. Encomendemos a la Virgen María nuestro camino cuaresmal, así como el de toda la Iglesia. Ella, que siguió a su Hijo Jesús hasta la cruz, nos ayude a ser discípulos fieles de Cristo, cristianos maduros, para poder participar juntamente con ella en la plenitud de la alegría pascual. Amén.
\end{body}

\begin{patercite}
(\ldots) La apertura del alma a Dios y a su acción en la fe incluye también el elemento de la oscuridad. La relación del ser humano con Dios no cancela la distancia entre Creador y criatura, no elimina cuanto afirma el apóstol Pablo ante las profundidades de la sabiduría de Dios: «¡Qué insondables sus decisiones y qué irrastreables sus caminos!» (\emph{Rm} 11, 33). Pero precisamente quien ---como María--- está totalmente abierto a Dios, llega a aceptar el querer divino, incluso si es misterioso, también si a menudo no corresponde al propio querer y es una espada que traspasa el alma, como dirá proféticamente el anciano Simeón a María, en el momento de la presentación de Jesús en el Templo (cf. \emph{Lc} 2, 35). El camino de fe de Abrahán comprende el momento de alegría por el don del hijo Isaac, pero también el momento de la oscuridad, cuando debe subir al monte Moria para realizar un gesto paradójico: Dios le pide que sacrifique el hijo que le había dado. En el monte el ángel le ordenó: «No alargues la mano contra el muchacho ni le hagas nada. Ahora he comprobado que temes a Dios, porque no te has reservado a tu hijo, a tu único hijo» (\emph{Gn} 22, 12). La plena confianza de Abrahán en el Dios fiel a las promesas no disminuye incluso cuando su palabra es misteriosa y difícil, casi imposible, de acoger. Así es para María; su fe vive la alegría de la Anunciación, pero pasa también a través de la oscuridad de la crucifixión del Hijo para poder llegar a la luz de la Resurrección.

No es distinto incluso para el camino de fe de cada uno de nosotros:encontramos momentos de luz, pero hallamos también momentos en los que Dios parece ausente, su silencio pesa en nuestro corazón y su voluntad no corresponde a la nuestra, a aquello que nosotros quisiéramos. Pero cuanto más nos abrimos a Dios, acogemos el don de la fe, ponemos totalmente en Él nuestra confianza ---como Abrahán y como María---, tanto más Él nos hace capaces, con su presencia, de vivir cada situación de la vida en la paz y en la certeza de su fidelidad y de su amor. Sin embargo, esto implica salir de uno mismo y de los propios proyectos para que la Palabra de Dios sea la lámpara que guíe nuestros pensamientos y nuestras acciones.
	
	\textbf{Benedicto XVI, papa}, \textit{Catequesis}, Audiencia general,  19 de diciembre de 2012, parr. 6-7.
\end{patercite}

\label{b2-03-02-2012H}
\newpage

\subsubsection{Ángelus (2012): Una luz para superar las pruebas}

\src{4 de marzo del 2012.}

\begin{body}
\ltr{E}{ste} domingo, el segundo de Cuaresma, se caracteriza por ser el domingo de la Transfiguración de Cristo. De hecho, durante la Cuaresma, la liturgia, después de habernos invitado a seguir a Jesús en el desierto, para afrontar y superar con él las tentaciones, nos propone subir con él al \textquote{monte} de la oración, para contemplar en su rostro humano la luz gloriosa de Dios. Los evangelistas Mateo, \textbf{Marcos} y Lucas atestiguan de modo concorde el episodio de la transfiguración de Cristo. Los elementos esenciales son dos: en primer lugar, Jesús sube con sus discípulos Pedro, Santiago y Juan a un monte alto, y allí \textquote{se transfiguró delante de ellos} (\textit{Mc} 9, 2), su rostro y sus vestidos irradiaron una luz brillante, mientras que junto a él aparecieron Moisés y Elías; y, en segundo lugar, una nube envolvió la cumbre del monte y de ella salió una voz que decía: \textquote{Este es mi Hijo amado, escuchadlo} (\textit{Mc} 9, 7). Por lo tanto, la luz y la voz: la luz divina que resplandece en el rostro de Jesús, y la voz del Padre celestial que da testimonio de él y manda escucharlo.

El misterio de la Transfiguración no se debe separar del contexto del camino que Jesús está recorriendo. Ya se ha dirigido decididamente hacia el cumplimiento de su misión, a sabiendas de que, para llegar a la resurrección, tendrá que pasar por la pasión y la muerte de cruz. De esto les ha hablado abiertamente a sus discípulos, los cuales sin embargo no han entendido; más aun, han rechazado esta perspectiva porque no piensan como Dios, sino como los hombres (cf. \textit{Mt} 16, 23). Por eso Jesús lleva consigo a tres de ellos al monte y les revela su gloria divina, esplendor de Verdad y de Amor. Jesús quiere que esta luz ilumine sus corazones cuando pasen por la densa oscuridad de su pasión y muerte, cuando el escándalo de la cruz sea insoportable para ellos. Dios es luz, y Jesús quiere dar a sus amigos más íntimos la experiencia de esta luz, que habita en él. Así, después de este episodio, él será en ellos una luz interior, capaz de protegerlos de los asaltos de las tinieblas. Incluso en la noche más oscura, Jesús es la luz que nunca se apaga. San Agustín resume este misterio con una expresión muy bella. Dice: \textquote{Lo que para los ojos del cuerpo es el sol que vemos, lo es [Cristo] para los ojos del corazón} (\textit{Sermo} 78, 2: PL 38, 490).

Queridos hermanos y hermanas, todos necesitamos luz interior para superar las pruebas de la vida. Esta luz viene de Dios, y nos la da Cristo, en quien habita la plenitud de la divinidad (cf. \textit{Col} 2, 9). Subamos con Jesús al monte de la oración y, contemplando su rostro lleno de amor y de verdad, dejémonos colmar interiormente de su luz. Pidamos a la Virgen María, nuestra guía en el camino de la fe, que nos ayude a vivir esta experiencia en el tiempo de la Cuaresma, encontrando cada día algún momento para orar en silencio y para escuchar la Palabra de Dios.
\end{body}

\newsection
\subsection{Francisco, papa}

\subsubsection{Ángelus (2015): Escuchar al Hijo}

\src{Plaza de San Pedro. \\1 de marzo del 2015.}

\begin{body}
\ltr{E}{l} domingo pasado la liturgia nos presentó a Jesús tentado por Satanás en el desierto, pero victorioso en la tentación. A la luz de este Evangelio, hemos tomado nuevamente conciencia de nuestra condición de pecadores, pero también de la victoria sobre el mal donada a quienes inician el camino de conversión y que, como Jesús, quieren hacer la voluntad del Padre. En este segundo domingo de Cuaresma, la Iglesia nos indica la meta de este itinerario de conversión, es decir, la participación en la gloria de Cristo, que resplandece en el rostro del Siervo obediente, muerto y resucitado por nosotros.

El \textbf{pasaje evangélico} narra el acontecimiento de la Transfiguración, que se sitúa en la cima del ministerio público de Jesús. Él está en camino hacia Jerusalén, donde se cumplirán las profecías del \textquote{Siervo de Dios} y se consumará su sacrificio redentor. La multitud no entendía esto: ante las perspectivas de un Mesías que contrasta con sus expectativas terrenas, lo abandonaron. Pero ellos pensaban que el Mesías sería un liberador del dominio de los romanos, un liberador de la patria, y esta perspectiva de Jesús no les gusta y lo abandonan. Incluso los Apóstoles no entienden las palabras con las que Jesús anuncia el cumplimiento de su misión en la pasión gloriosa, ¡no comprenden! Jesús entonces toma la decisión de mostrar a Pedro, Santiago y Juan una anticipación de su gloria, la que tendrá después de la resurrección, para confirmarlos en la fe y alentarlos a seguirlo por la senda de la prueba, por el camino de la Cruz. Y, así, sobre un monte alto, inmerso en oración, se transfigura delante de ellos: su rostro y toda su persona irradian una luz resplandeciente. Los tres discípulos están asustados, mientras una nube los envuelve y desde lo alto resuena –como en el Bautismo en el Jordán– la voz del Padre: \textquote{Este es mi Hijo amado; escuchadlo} (\textit{Mc} 9, 7). Jesús es el Hijo hecho Siervo, enviado al mundo para realizar a través de la Cruz el proyecto de la salvación, para salvarnos a todos nosotros. Su adhesión plena a la voluntad del Padre hace su humanidad transparente a la gloria de Dios, que es el Amor.

Jesús se revela así como el icono perfecto del Padre, la irradiación de su gloria. Es el cumplimiento de la revelación; por eso junto a Él transfigurado aparecen Moisés y Elías, que representan la Ley y los Profetas, para significar que todo termina y comienza en Jesús, en su pasión y en su gloria.

La consigna para los discípulos y para nosotros es esta: \textquote{¡Escuchadlo!}. Escuchad a Jesús. Él es el Salvador: seguidlo. Escuchar a Cristo, en efecto, lleva a asumir la lógica de su misterio pascual, ponerse en camino con Él para hacer de la propia vida un don de amor para los demás, en dócil obediencia a la voluntad de Dios, con una actitud de desapego de las cosas mundanas y de libertad interior. Es necesario, en otras palabras, estar dispuestos a \textquote{perder la propia vida} (cf. \textit{Mc} 8, 35), entregándola a fin de que todos los hombres se salven: así, nos encontraremos en la felicidad eterna. El camino de Jesús nos lleva siempre a la felicidad, ¡no lo olvidéis! El camino de Jesús nos lleva siempre a la felicidad. Habrá siempre una cruz en medio, pruebas, pero al final nos lleva siempre a la felicidad. Jesús no nos engaña, nos prometió la felicidad y nos la dará si vamos por sus caminos.

Con Pedro, Santiago y Juan subamos también nosotros hoy al monte de la Transfiguración y permanezcamos en contemplación del rostro de Jesús, para acoger su mensaje y traducirlo en nuestra vida; para que también nosotros podamos ser transfigurados por el Amor. En realidad, el amor es capaz de transfigurar todo. ¡El amor transfigura todo! ¿Creéis en esto? Que la Virgen María, que ahora invocamos con la oración del Ángelus, nos sostenga en este camino.
\end{body}

\img{cross_quedlinburg}

\label{b2-03-02-2015A}
\newpage

\subsubsection{Homilía (2018): Nos prepara para las pruebas}

\src{Visita pastoral a la parroquia romana \\de San Gelasio I, papa, en Ponte Mammolo.\\25 de febrero del 2018.}

\begin{body}
\ltr{J}{esús} se deja ver a los Apóstoles como es en el cielo: glorioso, luminoso, triunfante, vencedor. Y esto lo hace para prepararles a soportar la Pasión, el escándalo de la cruz, porque ellos no podían entender que Jesús hubiera muerto como un criminal, no podían entenderlo. Ellos pensaban que Jesús fuera un libertador, pero como son los libertadores terrenales, los que ganan en la batalla, los que son siempre triunfadores. Y el camino de Jesús es otro: Jesús triunfa a través de la humillación, la humillación de la cruz. Pero puesto que esto hubiera sido un escándalo para ellos, Jesús les hace ver lo que viene después, lo que hay después de la cruz, lo que nos espera a todos nosotros. Esta gloria y este cielo.

¡Y eso es muy hermoso! Es muy hermoso porque Jesús –y esto escuchadlo bien– nos prepara siempre para la prueba. En un modo o en otro, pero este es el mensaje: nos prepara siempre. Nos da la fuerza para ir adelante en los momentos de prueba y vencerlos con su fuerza. Jesús no nos deja solos en las pruebas de la vida: siempre nos prepara, nos ayuda, como ha preparado a estos [los discípulos], con la visión de su gloria. Y así ellos después recordaron esto [el momento] para soportar el peso de la humillación.

Esto es lo primero que nos enseña la Iglesia: Jesús nos prepara siempre para las pruebas y en las pruebas está con nosotros, no nos deja solos. Nunca. Lo segundo, podemos tomarlo de las palabras de Dios: \textquote{Este es mi Hijo, el amado. Escuchadle}. Este es el mensaje que el Padre da a los Apóstoles. El mensaje de Jesús es prepararlos, haciéndoles ver su gloria; el mensaje del Padre es: \textquote{Escuchadle}. No hay un momento en la vida que no se pueda vivir plenamente escuchando a Jesús. En los momentos hermosos, deteneos y escuchad a Jesús; en los momentos malos, deteneos y escuchad a Jesús. Este es el camino. Él nos dirá lo que tenemos que hacer. Siempre. Y vamos adelante en esta Cuaresma con estas dos cosas: en las pruebas, recordad la gloria de Jesús, es decir, lo que nos espera; que Jesús está presente siempre, con esa gloria para darnos fuerza.

Y durante toda la vida, escuchad a Jesús, lo que nos dice Jesús: en el Evangelio, en la liturgia, siempre nos habla; o en el corazón.

En la vida cotidiana, tal vez tengamos problemas, o tengamos que resolver muchas cosas. Hagámonos esta pregunta: ¿Qué nos dice Jesús hoy? Y busquemos escuchar la voz de Jesús, la inspiración desde dentro. Y así seguimos el consejo del Padre: \textquote{Este es mi Hijo, el amado. Escuchadle}. Será la Virgen la que te dé el segundo consejo en Caná, en Galilea, cuando se produce el milagro del agua [trasformada] en vino. ¿Qué dice la Virgen? \textquote{Haced lo que Él diga}. Escuchar a Jesús y hacer lo que Él dice: este es el camino seguro. Ir adelante con el recuerdo de la gloria de Jesús, con este consejo: escuchar a Jesús y hacer lo que Él nos dice.
\end{body}

\begin{patercite}
\textbf{Jesucristo, el amor de Dios encarnado}

(\ldots) La verdadera originalidad del Nuevo Testamento no consiste en nuevas ideas, sino en la figura misma de Cristo, que da carne y sangre a los conceptos: un realismo inaudito. Tampoco en el Antiguo Testamento la novedad bíblica consiste simplemente en nociones abstractas, sino en la actuación imprevisible y, en cierto sentido inaudita, de Dios. Este actuar de Dios adquiere ahora su forma dramática, puesto que, en Jesucristo, el propio Dios va tras la \textquote{oveja perdida}, la humanidad doliente y extraviada. Cuando Jesús habla en sus parábolas del pastor que va tras la oveja descarriada, de la mujer que busca el dracma, del padre que sale al encuentro del hijo pródigo y lo abraza, no se trata sólo de meras palabras, sino que es la explicación de su propio ser y actuar. En su muerte en la cruz se realiza ese ponerse Dios contra sí mismo, al entregarse para dar nueva vida al hombre y salvarlo: esto es amor en su forma más radical. Poner la mirada en el costado traspasado de Cristo, del que habla Juan (cf. 19, 37), ayuda a comprender lo que [significa que] \textquote{Dios es amor} (\textit{1 Jn} 4, 8). Es allí, en la cruz, donde puede contemplarse esta verdad. Y a partir de allí se debe definir ahora qué es el amor. Y, desde esa mirada, el cristiano encuentra la orientación de su vivir y de su amar.

Jesús ha perpetuado este acto de entrega mediante la institución de la Eucaristía durante la Última Cena. Ya en aquella hora, Él anticipa su muerte y resurrección, dándose a sí mismo a sus discípulos en el pan y en el vino, su cuerpo y su sangre como nuevo maná (cf. \textit{Jn} 6, 31-33). Si el mundo antiguo había soñado que, en el fondo, el verdadero alimento del hombre ---aquello por lo que el hombre vive--- era el \textit{Logos}, la sabiduría eterna, ahora este \textit{Logos} se ha hecho para nosotros verdadera comida, como amor. La Eucaristía nos adentra en el acto oblativo de Jesús. No recibimos solamente de modo pasivo el \textit{Logos} encarnado, sino que nos implicamos en la dinámica de su entrega. La imagen de las nupcias entre Dios e Israel se hace realidad de un modo antes inconcebible: lo que antes era estar frente a Dios, se transforma ahora en unión por la participación en la entrega de Jesús, en su cuerpo y su sangre. La \textquote{mística} del Sacramento, que se basa en el abajamiento de Dios hacia nosotros, tiene otra dimensión de gran alcance y que lleva mucho más alto de lo que cualquier elevación mística del hombre podría alcanzar.

\textbf{Benedicto XVI, papa}, \textit{Deus Caritas est}, nn. 12-13.
\end{patercite}
\label{b2-03-02-2018H}
\newpage


\subsubsection{Ángelus (2018): Anticipo del cielo}

\src{Plaza de San Pedro. \\25 de febrero de 2018.}

\begin{body}
\ltr{E}{l} \textbf{Evangelio} hoy, segundo domingo de Cuaresma, nos invita a contemplar la transfiguración de Jesús (cf. \textit{Mc} 9, 2-10). Este episodio está ligado a lo que sucedió seis días antes, cuando Jesús había desvelado a sus discípulos que en Jerusalén debería \textquote{sufrir mucho y ser reprobado por los ancianos, los sumos sacerdotes y los escribas, ser matado y resucitado a los tres días} (\textit{Mc} 8, 31). Este anuncio había puesto en crisis a Pedro y a todo el grupo de discípulos, que rechazaban la idea de que Jesús terminara rechazado por los jefes del pueblo y después matado. Ellos, de hecho, esperaban a un Mesías poderoso, fuerte, dominador; en cambio, Jesús se presenta como humilde, como manso, siervo de Dios, siervo de los hombres, que deberá entregar su vida en sacrificio, pasando por el camino de la persecución, del sufrimiento y de la muerte.

Pero, ¿cómo poder seguir a un Maestro y Mesías cuya vivencia terrenal terminaría de ese modo? Así pensaban ellos. Y la respuesta llega precisamente de la transfiguración. ¿Qué es la transfiguración de Jesús? Es una aparición pascual anticipada. Jesús toma consigo a los tres discípulos Pedro, Santiago y Juan y \textquote{los lleva, a ellos solos, a parte, a un monte alto} (\textit{Mc} 9, 2); y allí, por un momento, les muestra su gloria, gloria de Hijo de Dios. Este evento de la transfiguración permite así a los discípulos afrontar la pasión de Jesús de un modo positivo, sin ser arrastrados. Lo vieron como será después de la pasión, glorioso. Y así Jesús les prepara para la prueba. La transfiguración ayuda a los discípulos, y también a nosotros, a entender que la pasión de Cristo es un misterio de sufrimiento, pero es sobre todo un regalo de amor, de amor infinito por parte de Jesús.

El evento de Jesús transfigurándose sobre el monte nos hace entender mejor también su resurrección. Para entender el misterio de la cruz es necesario saber con antelación que el que sufre y que es glorificado no es solamente un hombre, sino el Hijo de Dios, que con su amor fiel hasta la muerte nos ha salvado. El padre renueva así su declaración mesiánica sobre el Hijo, ya hecha en la orilla del Jordán después del bautismo y exhorta: \textquote{Escuchadle} (\textit{Mc} 9, 7).

Los discípulos están llamados a seguir al Maestro con confianza, con esperanza, a pesar de su muerte; la divinidad de Jesús debe manifestarse precisamente en la cruz, precisamente en su morir \textquote{de aquel modo}, tanto que el evangelista Marcos pone en la boca del centurión la profesión de fe: \textquote{Verdaderamente este hombre era el Hijo de Dios} (\textit{Mc} 15, 39). Nos dirigimos ahora en oración a la Virgen María, la criatura humana transfigurada interiormente por la gracia de Cristo. Nos encomendamos confiados a su maternal ayuda para proseguir con fe y generosidad el camino de la Cuaresma.
\end{body}

\newsection
\section{Temas}

\cceth{La Transfiguración} 
\cceref{CEC 554-556, 568}

\begin{ccebody}
\ccesec{Una visión anticipada del Reino: La Transfiguración.}

\n{554} A partir del día en que Pedro confesó que Jesús es el Cristo, el Hijo de Dios vivo, el Maestro \textquote{comenzó a mostrar a sus discípulos que él debía ir a Jerusalén, y sufrir [\ldots] y ser condenado a muerte y resucitar al tercer día} (\textit{Mt} 16, 21): Pedro rechazó este anuncio (cf. \textit{Mt} 16, 22-23), los otros no lo comprendieron mejor (cf. \textit{Mt} 17, 23; \textit{Lc} 9, 45). En este contexto se sitúa el episodio misterioso de la Transfiguración de Jesús (cf. \textit{Mt} 17, 1-8 par.; \textit{2 P} 1, 16-18), sobre una montaña, ante tres testigos elegidos por él: Pedro, Santiago y Juan. El rostro y los vestidos de Jesús se pusieron fulgurantes como la luz, Moisés y Elías aparecieron y le \textquote{hablaban de su partida, que estaba para cumplirse en Jerusalén} (\textit{Lc} 9, 31). Una nube les cubrió y se oyó una voz desde el cielo que decía: \textquote{Este es mi Hijo, mi elegido; escuchadle} (\textit{Lc} 9, 35).

\n{555} Por un instante, Jesús muestra su gloria divina, confirmando así la confesión de Pedro. Muestra también que para \textquote{entrar en su gloria} (\textit{Lc} 24, 26), es necesario pasar por la Cruz en Jerusalén. Moisés y Elías habían visto la gloria de Dios en la Montaña; la Ley y los profetas habían anunciado los sufrimientos del Mesías (cf. \textit{Lc} 24, 27). La Pasión de Jesús es la voluntad por excelencia del Padre: el Hijo actúa como siervo de Dios (cf. \textit{Is} 42, 1). La nube indica la presencia del Espíritu Santo: \textit{Tota Trinitas apparuit: Pater in voce; Filius in homine, Spiritus in nube clara} – \textquote{Apareció toda la Trinidad: el Padre en la voz, el Hijo en el hombre, el Espíritu en la nube luminosa} (Santo Tomás de Aquino, \textit{S. th.} 3, q. 45, a. 4, ad 2):

\ccecite{\textquote{En el monte te transfiguraste, Cristo Dios, y tus discípulos contemplaron tu gloria, en cuanto podían comprenderla. Así, cuando te viesen crucificado, entenderían que padecías libremente, y anunciarían al mundo que tú eres en verdad el resplandor del Padre} (\textit{Liturgia bizantina, Himno Breve de la festividad de la Transfiguración del Señor}).}

\n{556} En el umbral de la vida pública se sitúa el Bautismo; en el de la Pascua, la Transfiguración. Por el bautismo de Jesús \textquote{fue manifestado el misterio de la primera regeneración}: nuestro Bautismo; la Transfiguración \textquote{es el sacramento de la segunda regeneración}: nuestra propia resurrección (Santo Tomás de Aquino, \textit{S.Th.}, 3, q. 45, a. 4, ad 2). Desde ahora nosotros participamos en la Resurrección del Señor por el Espíritu Santo que actúa en los sacramentos del Cuerpo de Cristo. La Transfiguración nos concede una visión anticipada de la gloriosa venida de Cristo \textquote{el cual transfigurará este miserable cuerpo nuestro en un cuerpo glorioso como el suyo} (\textit{Flp} 3, 21). Pero ella nos recuerda también que \textquote{es necesario que pasemos por muchas tribulaciones para entrar en el Reino de Dios} (\textit{Hch} 14, 22):

\ccecite{\textquote{Pedro no había comprendido eso cuando deseaba vivir con Cristo en la montaña (cf. \textit{Lc} 9, 33). Te ha reservado eso, oh Pedro, para después de la muerte. Pero ahora, él mismo dice: Desciende para penar en la tierra, para servir en la tierra, para ser despreciado y crucificado en la tierra. La Vida desciende para hacerse matar; el Pan desciende para tener hambre; el Camino desciende para fatigarse andando; la Fuente desciende para sentir la sed; y tú, ¿vas a negarte a sufrir?} (San Agustín, \textit{Sermo}, 78, 6: PL 38, 492-493).}

\n{568} \textit{La Transfiguración de Cristo tiene por finalidad fortalecer la fe de los apóstoles ante la proximidad de la Pasión: la subida a un \textquote{monte alto} prepara la subida al Calvario. Cristo, Cabeza de la Iglesia, manifiesta lo que su cuerpo contiene e irradia en los sacramentos: \textquote{la esperanza de la gloria}} (\textit{Col} 1, 27). (cf. San León Magno, \textit{Sermo} 51, 3: PL 54, 310C). 
\end{ccebody}

\cceth{La obediencia de Abrahán} 
\cceref{CEC 59, 145-146, 2570-2572}

\begin{ccebody}
\ccesec{Dios elige a Abraham}

\n{59} Para reunir a la humanidad dispersa, Dios elige a Abram llamándolo \textquote{fuera de su tierra, de su patria y de su casa} (\textit{Gn} 12,1), para hacer de él \textquote{Abraham}, es decir, \textquote{el padre de una multitud de naciones} (\textit{Gn} 17,5): \textquote{En ti serán benditas todas las naciones de la tierra} (\textit{Gn} 12,3; cf. \textit{Ga} 3,8).

\ccesec{Abraham, \textquote{padre de todos los creyentes}}

\n{145} La carta a los Hebreos, en el gran elogio de la fe de los antepasados, insiste particularmente en la fe de Abraham: \textquote{Por la fe, Abraham obedeció y salió para el lugar que había de recibir en herencia, y salió sin saber a dónde iba} (\textit{Hb} 11,8; cf. \textit{Gn} 12,1-4). Por la fe, vivió como extranjero y peregrino en la Tierra prometida (cf. \textit{Gn} 23,4). Por la fe, a Sara se le otorgó el concebir al hijo de la promesa. Por la fe, finalmente, Abraham ofreció a su hijo único en sacrificio (cf. \textit{Hb} 11,17).

\n{146} Abraham realiza así la definición de la fe dada por la carta a los Hebreos: \textquote{La fe es garantía de lo que se espera; la prueba de las realidades que no se ven} (\textit{Hb} 11,1). \textquote{Creyó Abraham en Dios y le fue reputado como justicia} (\textit{Rm} 4,3; cf. \textit{Gn} 15,6). Y por eso, fortalecido por su fe, Abraham fue hecho \textquote{padre de todos los creyentes} (\textit{Rm} 4,11.18; cf. \textit{Gn} 15, 5).

\ccesec{La Promesa y la oración de la fe}

\n{2570} Cuando Dios lo llama, Abraham se pone en camino \textquote{como se lo había dicho el Señor} (\textit{Gn} 12, 4): todo su corazón \textquote{se somete a la Palabra} y obedece. La escucha del corazón a Dios que llama es esencial a la oración, las palabras tienen un valor relativo. Por eso, la oración de Abraham se expresa primeramente con hechos: hombre de silencio, en cada etapa construye un altar al Señor. Solamente más tarde aparece su primera oración con palabras: una queja velada recordando a Dios sus promesas que no parecen cumplirse (cf. \textit{Gn} 15, 2-3). De este modo surge desde los comienzos uno de los aspectos de la tensión dramática de la oración: la prueba de la fe en Dios que es fiel.

\n{2571} Habiendo creído en Dios (cf. \textit{Gn} 15, 6), marchando en su presencia y en alianza con él (cf. \textit{Gn} 17, 2), el patriarca está dispuesto a acoger en su tienda al Huésped misterioso: es la admirable hospitalidad de Mambré, preludio a la anunciación del verdadero Hijo de la promesa (cf. \textit{Gn} 18, 1-15; \textit{Lc} 1, 26-38). Desde entonces, habiéndole confiado Dios su plan, el corazón de Abraham está en consonancia con la compasión de su Señor hacia los hombres y se atreve a interceder por ellos con una audaz confianza (cf. \textit{Gn} 18, 16-33).

\n{2572} Como última purificación de su fe, se le pide al \textquote{que había recibido las promesas} (\textit{Hb} 11, 17) que sacrifique al hijo que Dios le ha dado. Su fe no vacila: \textquote{Dios proveerá el cordero para el holocausto} (\textit{Gn} 22, 8), \textquote{pensaba que poderoso era Dios aun para resucitar a los muertos} (\textit{Hb} 11, 19). Así, el padre de los creyentes se hace semejante al Padre que no perdonará a su propio Hijo, sino que lo entregará por todos nosotros (cf. \textit{Rm} 8, 32). La oración restablece al hombre en la semejanza con Dios y le hace participar en la potencia del amor de Dios que salva a la multitud (cf. \textit{Rm} 4, 16-21).
\end{ccebody}

\cceth{Las características de la fe} 
\cceref{CEC 153-159}

\begin{ccebody}
\ccesec{La fe es una gracia}

\n{153} Cuando san Pedro confiesa que Jesús es el Cristo, el Hijo de Dios vivo, Jesús le declara que esta revelación no le ha venido \textquote{de la carne y de la sangre, sino de mi Padre que está en los cielos} (\textit{Mt} 16,17; cf. \textit{Ga} 1,15; \textit{Mt} 11,25). La fe es un don de Dios, una virtud sobrenatural infundida por Él. \textquote{Para dar esta respuesta de la fe es necesaria la gracia de Dios, que se adelanta y nos ayuda, junto con los auxilios interiores del Espíritu Santo, que mueve el corazón, lo dirige a Dios, abre los ojos del espíritu y concede \textquote{a todos gusto en aceptar y creer la verdad}} (DV 5).

\ccesec{La fe es un acto humano}

\n{154} Sólo es posible creer por la gracia y los auxilios interiores del Espíritu Santo. Pero no es menos cierto que creer es un acto auténticamente humano. No es contrario ni a la libertad ni a la inteligencia del hombre depositar la confianza en Dios y adherirse a las verdades por Él reveladas. Ya en las relaciones humanas no es contrario a nuestra propia dignidad creer lo que otras personas nos dicen sobre ellas mismas y sobre sus intenciones, y prestar confianza a sus promesas (como, por ejemplo, cuando un hombre y una mujer se casan), para entrar así en comunión mutua. Por ello, es todavía menos contrario a nuestra dignidad \textquote{presentar por la fe la sumisión plena de nuestra inteligencia y de nuestra voluntad al Dios que revela} (Concilio Vaticano I: DS 3008) y entrar así en comunión íntima con Él.

\n{155} En la fe, la inteligencia y la voluntad humanas cooperan con la gracia divina: \textquote{Creer es un acto del entendimiento que asiente a la verdad divina por imperio de la voluntad movida por Dios mediante la gracia} (Santo Tomás de Aquino, \textit{S. Th.}, 2-2, q. 2 a. 9; cf. Concilio Vaticano I: DS 3010).

\newpage
\ccesec{La fe y la inteligencia}

\n{156} El \textit{motivo} de creer no radica en el hecho de que las verdades reveladas aparezcan como verdaderas e inteligibles a la luz de nuestra razón natural. Creemos \textquote{a causa de la autoridad de Dios mismo que revela y que no puede engañarse ni engañarnos}. \textquote{Sin embargo, para que el homenaje de nuestra fe fuese conforme a la razón, Dios ha querido que los auxilios interiores del Espíritu Santo vayan acompañados de las pruebas exteriores de su revelación} (\textit{ibíd.}, DS 3009). Los milagros de Cristo y de los santos (cf. \textit{Mc} 16,20; \textit{Hch} 2,4), las profecías, la propagación y la santidad de la Iglesia, su fecundidad y su estabilidad \textquote{son signos certísimos de la Revelación divina, adaptados a la inteligencia de todos}, motivos de credibilidad que muestran que \textquote{el asentimiento de la fe no es en modo alguno un movimiento ciego del espíritu} (Concilio Vaticano I: DS 3008-3010).

\n{157} La fe es \textit{cierta}, más cierta que todo conocimiento humano, porque se funda en la Palabra misma de Dios, que no puede mentir. Ciertamente las verdades reveladas pueden parecer oscuras a la razón y a la experiencia humanas, pero \textquote{la certeza que da la luz divina es mayor que la que da la luz de la razón natural} (Santo Tomás de Aquino, \textit{S. Th.}, 2-2, q. 171, a. 5, 3). \textquote{Diez mil dificultades no hacen una sola duda} (J. H. Newman, \textit{Apologia pro vita sua,} c. 5).

\n{158} \textquote{La fe \textit{trata de comprender}} (San Anselmo de Canterbury, \textit{Proslogion}, \hyphenation{proemium}: PL 153, 225A) es inherente a la fe que el creyente desee conocer mejor a aquel en quien ha puesto su fe, y comprender mejor lo que le ha sido revelado; un conocimiento más penetrante suscitará a su vez una fe mayor, cada vez más encendida de amor. La gracia de la fe abre \textquote{los ojos del corazón} (\textit{Ef} 1,18) para una inteligencia viva de los contenidos de la Revelación, es decir, del conjunto del designio de Dios y de los misterios de la fe, de su conexión entre sí y con Cristo, centro del Misterio revelado. Ahora bien, \textquote{para que la inteligencia de la Revelación sea más profunda, el mismo Espíritu Santo perfecciona constantemente la fe por medio de sus dones} (DV 5). Así, según el adagio de san Agustín (\textit{Sermo} 43, 7.9: PL 38, 258), \textquote{creo para comprender y comprendo para creer mejor}.

\n{159} \textit{Fe y ciencia}. \textquote{A pesar de que la fe esté por encima de la razón, jamás puede haber contradicción entre ellas. Puesto que el mismo Dios que revela los misterios e infunde la fe otorga al espíritu humano la luz de la razón, Dios no puede negarse a sí mismo ni lo verdadero contradecir jamás a lo verdadero} (Concilio Vaticano I: DS 3017). \textquote{Por eso, la investigación metódica en todas las disciplinas, si se procede de un modo realmente científico y según las normas morales, nunca estará realmente en oposición con la fe, porque las realidades profanas y las realidades de fe tienen su origen en el mismo Dios. Más aún, quien con espíritu humilde y ánimo constante se esfuerza por escrutar lo escondido de las cosas, aun sin saberlo, está como guiado por la mano de Dios, que, sosteniendo todas las cosas, hace que sean lo que son} (GS 36,2).
\end{ccebody}

\newpage
\cceth{Dios manifiesta su Gloria para revelarnos su voluntad} 
\cceref{CEC 2059}

\begin{ccebody}
\n{2059} Las \textquote{diez palabras} son pronunciadas por Dios dentro de una teofanía (\textquote{el Señor os habló cara a cara en la montaña, en medio del fuego}: \textit{Dt} 5, 4). Pertenecen a la revelación que Dios hace de sí mismo y de su gloria. El don de los mandamientos es don de Dios y de su santa voluntad. Dando a conocer su voluntad, Dios se revela a su pueblo.
\end{ccebody}

\cceth{Cristo es para todos nosotros} 
\cceref{CEC 603, 1373, 2634, 2852}

\begin{ccebody}
\n{603} Jesús no conoció la reprobación como si él mismo hubiese pecado (cf. \textit{Jn} 8, 46). Pero, en el amor redentor que le unía siempre al Padre (cf. \textit{Jn} 8, 29), nos asumió desde el alejamiento con relación a Dios por nuestro pecado hasta el punto de poder decir en nuestro nombre en la cruz: \textquote{Dios mío, Dios mío, ¿por qué me has abandonado?} (\textit{Mc} 15, 34; \textit{Sal} 22,2). Al haberle hecho así solidario con nosotros, pecadores, \textquote{Dios no perdonó ni a su propio Hijo, antes bien le entregó por todos nosotros} (\textit{Rm} 8, 32) para que fuéramos \textquote{reconciliados con Dios por la muerte de su Hijo} (\textit{Rm} 5, 10).

\ccesec{La presencia de Cristo por el poder de su Palabra y del Espíritu Santo}

\n{1373} \textquote{Cristo Jesús que murió, resucitó, que está a la derecha de Dios e intercede por nosotros} (\textit{Rm} 8,34), está presente de múltiples maneras en su Iglesia (cf. LG 48): en su Palabra, en la oración de su Iglesia, \textquote{allí donde dos o tres estén reunidos en mi nombre} (\textit{Mt} 18,20), en los pobres, los enfermos, los presos (\textit{Mt} 25,31-46), en los sacramentos de los que Él es autor, en el sacrificio de la misa y en la persona del ministro. Pero, \textquote{\textit{sobre todo}, (está presente) \textit{bajo las especies eucarísticas}} (SC 7).

\ccesec{La oración de intercesión}

\n{2634} La intercesión es una oración de petición que nos conforma muy de cerca con la oración de Jesús. Él es el único intercesor ante el Padre en favor de todos los hombres, de los pecadores en particular (cf. \textit{Rm} 8, 34; \textit{1 Jn} 2, 1; \textit{1 Tm} 2, 5-8). Es capaz de \textquote{salvar perfectamente a los que por Él se llegan a Dios, ya que está siempre vivo para interceder en su favor} (\textit{Hb} 7, 25). El propio Espíritu Santo \textquote{intercede por nosotros [\ldots] y su intercesión a favor de los santos es según Dios} (\textit{Rm} 8, 26-27).

\n{2852} \textquote{Homicida [\ldots] desde el principio [\ldots] mentiroso y padre de la mentira} (\textit{Jn} 8, 44), \textquote{Satanás, el seductor del mundo entero} (\textit{Ap} 12, 9), es aquél por medio del cual el pecado y la muerte entraron en el mundo y, por cuya definitiva derrota toda la creación entera será \textquote{liberada del pecado y de la muerte} (\textit{Plegaria Eucarística IV}, 123: \textit{Misal Romano}). \textquote{Sabemos que todo el que ha nacido de Dios no peca, sino que el Engendrado de Dios le guarda y el Maligno no llega a tocarle. Sabemos que somos de Dios y que el mundo entero yace en poder del Maligno} (\textit{1 Jn} 5, 18-19):

\newpage 
\ccecite{\textquote{El Señor que ha borrado vuestro pecado y perdonado vuestras faltas también os protege y os guarda contra las astucias del Diablo que os combate para que el enemigo, que tiene la costumbre de engendrar la falta, no os sorprenda. Quien confía en Dios, no tema al demonio. \textquote{Si Dios está con nosotros, ¿quién estará contra nosotros?}(\textit{Rm} 8, 31)} (San Ambrosio, \textit{De sacramentis}, 5, 30).}
\end{ccebody}

\begin{patercite}
[\ldots] Tiempo de cuaresma, oh Señor: No permitáis que acudamos a las cisternas agrietadas (\textit{Jer} 2,13), ni que imitemos al siervo infiel, a la virgen necia; no permitáis que el goce de los bienes de la Tierra haga insensible nuestro corazón al lamento de los pobres, de los enfermos, de los niños huérfanos y de los innumerables hermanos nuestros que todavía hoy carecen del mínimo necesario para comer, para cubrir los desnudos miembros, para reunir la familia bajo un mismo techo.	

Las aguas del Jordán descendieron también sobre Vos, oh Jesús, a la vista de la multitud; pero fueron pocos entonces los que pudieron reconoceros; y este misterio de lenta fe o de indiferencia, prolongándose en los siglos, es siempre un motivo de dolor para los que os aman y han recibido la misión de haceros conocer por el mundo. Conceded a los sucesores de los Apóstoles y de los discípulos y cuantos reciben su nombre de Vos y vuestra cruz que puedan llevar adelante la empresa evangelizadora y sostenerla con la oración, con el sufrimiento, y con la íntima fidelidad a vuestra voluntad.

Y así como Vos, inocente Cordero, os presentasteis a Juan en actitud de pecador, atraednos también a nosotros, oh Jesús, a las aguas del Jordán. A ellas queremos acudir para confesar nuestros pecados y purificar nuestras almas. Y así como los cielos abiertos dejaron oír la voz de Vuestro Padre que en Vos, oh Jesús, se complacía, así también superada victoriosamente la prueba, vivido austeramente el período cuaresmal, en los albores de Vuestra resurrección, podamos volver a oír en lo íntimo de nuestras almas la misma voz del Padre celestial, que en nosotros reconoce a sus hijos.

¡Oh santa cuaresma (\ldots)! Ascienda esta oración, en este atardecer de sereno recogimiento, desde cada una de las casas donde se trabaja, se ama y se sufre. Que los ángeles del cielo recojan el suspiro de tantas almas de pequeños inocentes, de jóvenes generosos, de padres de familia trabajadores y sacrificados, y de cuantos sufren en el cuerpo y en el espíritu, para presentarlos luego a Dios. Desde allí descenderán abundantes los dones de las consolaciones celestiales, de los que quiere ser prenda y reflejo nuestra bendición apostólica.

\textbf{San Juan XXIII, papa}, \textit{Radiomensaje}, a todos los fieles con ocasión del inicio de la Cuaresma,  Miércoles de Ceniza, 27 de febrero de 1963, parr. 16-21.
\end{patercite}