\chapter{Domingo III de Cuaresma (B)}

\section{Lecturas}

\rtitle{PRIMERA LECTURA (forma larga)}

\rbook{Del libro del Éxodo} \rred{20, 1-17}

\rtheme{La Ley se dio por medio de Moisés}

\begin{scripture}
En aquellos días, el Señor pronunció estas palabras:

«Yo soy el Señor, tu Dios, que te saqué de la tierra de Egipto, de la casa de esclavitud.

No tendrás otros dioses frente a mí.

No te fabricarás ídolos, ni figura alguna de lo que hay arriba en el cielo, abajo en la tierra, o en el agua debajo de la tierra.

No te postrarás ante ellos, ni les darás culto; porque yo, el Señor, tu Dios, soy un Dios celoso, que castigo el pecado de los padres en los hijos, hasta la tercera y la cuarta generación de los que me odian.

Pero tengo misericordia por mil generaciones de los que me aman y guardan mis preceptos.

No pronunciarás el nombre del Señor, tu Dios, en falso. Porque no dejará el Señor impune a quien pronuncie su nombre en falso.
        
Recuerda el día del sábado para santificarlo. 

Durante seis días trabajarás y harás todas tus tareas, pero el día séptimo es día de descanso, consagrado al Señor, tu Dios. No harás trabajo alguno, ni tú, ni tu hijo, ni tu hija, ni tu esclavo, ni tu esclava, ni tu ganado, ni el emigrante que reside en tus ciudades. Porque en seis días hizo el Señor el cielo, la tierra, el mar y lo que hay en ellos; y el séptimo día descansó. Por eso bendijo el Señor el sábado y lo santificó.

Honra a tu padre y a tu madre, para que se prolonguen tus días en la tierra, que el Señor, tu Dios, te va a dar.

No matarás.

No cometerás adulterio.

No robarás.

No darás falso testimonio contra tu prójimo.

No codiciarás los bienes de tu prójimo. No codiciarás la mujer de tu prójimo, ni su esclavo, ni su esclava, ni su buey, ni su asno, ni nada que sea de tu prójimo».
\end{scripture}


\rtitle{PRIMERA LECTURA (forma breve)}

\rbook{Del libro del Éxodo} \rred{20, 1-3. 7-8. 12-17}

\rtheme{La ley se dio por medio de Moisés}

\begin{scripture}
	En aquellos días, el Señor pronunció estas palabras:
	
	«Yo soy el Señor, tu Dios, que te saqué de la tierra de Egipto, de la casa de esclavitud.
	
	No tendrás otros dioses frente a mí.
	
	No pronunciarás el nombre del Señor, tu Dios, en falso. Porque no dejará el Señor impune a quien pronuncie su nombre en falso.
	
	Recuerda el día del sábado para santificarlo. 
	
	Honra a tu padre y a tu madre, para que se prolonguen tus días en la tierra, que el Señor, tu Dios, te va a dar.
	
	No matarás.
	
	No cometerás adulterio.
	
	No robarás.
	
	No darás falso testimonio contra tu prójimo.
	
	No codiciarás los bienes de tu prójimo. No codiciarás la mujer de tu prójimo, ni su esclavo, ni su esclava, ni su buey, ni su asno, ni nada que sea de tu prójimo».
\end{scripture}

\img{decalog}

\newpage
\rtitle{SALMO RESPONSORIAL}

\rbook{Salmo} \rred{18, 8. 9. 10. 11}

\rtheme{Señor, tú tienes palabras de vida eterna}

\begin{psbody}
La ley del Señor es perfecta
y es descanso del alma;
el precepto del Señor es fiel
e instruye a los ignorantes. 

Los mandatos del Señor son rectos
y alegran el corazón;
la norma del Señor es límpida
y da luz a los ojos. 

El temor del Señor es puro
y eternamente estable;
los mandamientos del Señor son verdaderos
y enteramente justos. 

Más preciosos que el oro,
más que el oro fino;
más dulces que la miel
de un panal que destila. 
\end{psbody}


\rtitle{SEGUNDA LECTURA}

\rbook{De la primera carta del apóstol san Pablo a los Corintios} \rred{1, 22-25}

\rtheme{Predicamos a Cristo crucificado: escándalo para los hombres;
pero para los llamados es sabiduría de Dios}

\begin{scripture}
Hermanos:

Los judíos exigen signos, los griegos buscan sabiduría; pero nosotros predicamos a Cristo crucificado: escándalo para los judíos, necedad para los gentiles; pero para los llamados –judíos o griegos–, un Cristo que es fuerza de Dios y sabiduría de Dios.

Pues lo necio de Dios es más sabio que los hombres; y lo débil de Dios es más fuerte que los hombres.
\end{scripture}

\newpage
\rtitle{EVANGELIO}

\rbook{Del Evangelio según san Juan} \rred{2, 13-25}

\rtheme{Destruid este templo, y en tres días lo levantaré}

\begin{scripture}
Se acercaba la Pascua de los judíos y Jesús subió a Jerusalén. Y encontró en el templo a los vendedores de bueyes, ovejas y palomas, y a los cambistas sentados; y, haciendo un azote de cordeles, los echó a todos del templo, ovejas y bueyes; y a los cambistas les esparció las monedas y les volcó las mesas; y a los que vendían palomas les dijo:

\>{Quitad esto de aquí: no convirtáis en un mercado la casa de mi Padre}.


Sus discípulos se acordaron de lo que está escrito: \textquote{El celo de tu casa me devora}.

Entonces intervinieron los judíos y le preguntaron:

\>{¿Qué signos nos muestras para obrar así?}.

Jesús contestó:

\>{Destruid este templo, y en tres días lo levantaré}.

Los judíos replicaron:

\>{Cuarenta y seis años ha costado construir este templo, ¿y tú lo vas a levantar en tres días?}.

Pero él hablaba del templo de su cuerpo.

Y cuando resucitó de entre los muertos, los discípulos se acordaron de que lo había dicho, y creyeron a la Escritura y a la Palabra que había dicho Jesús.

Mientras estaba en Jerusalén por las fiestas de Pascua, muchos creyeron en su nombre, viendo los signos que hacía; pero Jesús no se confiaba a ellos, porque los conocía a todos y no necesitaba el testimonio de nadie sobre un hombre, porque él sabía lo que hay dentro de cada hombre.
\end{scripture}

\img{cross_romanesque}

\newsection
\section{Comentario Patrístico}

\subsection{San Agustín, obispo}

\ptheme{Somos las piedras vivas con las que se edifica el templo de Dios}
 
\src{Comentario sobre el salmo 130, nn. 1-3: \\CCL 40, 1198-1200.}

\begin{body}
\ltr{C}{on} frecuencia hemos advertido a vuestra Caridad que no hay que considerar los salmos como la voz aislada de un hombre que canta, sino como la voz de todos aquellos que están en el Cuerpo de Cristo. Y como en el Cuerpo de Cristo están todos, habla como un solo hombre, pues él es a la vez uno y muchos. Son muchos considerados aisladamente; son uno en aquel que es uno. Él es también el templo de Dios, del que dice el Apóstol: \textit{El templo de Dios es santo: ese templo sois vosotros}: todos los que creen en Cristo y creyendo, aman. Pues en esto consiste creer en Cristo: en amar a Cristo; no a la manera de los demonios, que creían, pero no amaban. Por eso, a pesar de creer, decían: ¿Qué tenemos nosotros contigo, Hijo de Dios? Nosotros, en cambio, de tal manera creamos que, creyendo en Él, le amemos y no digamos: ¿Qué tenemos nosotros contigo?, sino digamos más bien: \textquote{Te pertenecemos, tú nos has redimido}.

Efectivamente, todos cuantos creen así, son como las piedras vivas con las que se edifica el templo de Dios, y como la madera incorruptible con que se construyó aquella arca que el diluvio no consiguió sumergir. Este es el templo –esto es, los mismos hombres– en que se ruega a Dios y Dios escucha. Sólo al que ora en el templo de Dios se le concede ser escuchado para la vida eterna. Y ora en el templo de Dios el que ora en la paz de la Iglesia, en la unidad del cuerpo de Cristo. Este Cuerpo de Cristo consta de una multitud de creyentes esparcidos por todo el mundo; y por eso es escuchado el que ora en el templo. Ora, pues, en espíritu y en verdad el que ora en la paz de la Iglesia, no en aquel templo que era sólo una figura.

A nivel de figura, el Señor arrojó del templo a los que en el templo buscaban su propio interés, es decir, los que iban al templo a comprar y vender. Ahora bien, si aquel templo era una figura, es evidente que también en el Cuerpo de Cristo –que es el verdadero templo del que el otro era una imagen– existe una mezcolanza de compradores y vendedores, esto es, gente que busca su interés, no el de Jesucristo.

Y puesto que los hombres son vapuleados por sus propios pecados, el Señor hizo un azote de cordeles y arrojó del templo a todos los que buscaban sus intereses, no los de Jesucristo.

Pues bien, la voz de este templo es la que resuena en el salmo. En este templo –y no en el templo material– se ruega a Dios, como os he dicho, y Dios escucha en espíritu y en verdad. Aquel templo era una sombra, figura de lo que había de venir. Por eso aquel templo se derrumbó ya. ¿Quiere decir esto que se derrumbó nuestra casa de oración? De ningún modo. Pues aquel templo que se derrumbó no pudo ser llamado casa de oración, de la que se dijo: \textit{Mi casa es casa de oración, y así la llamarán todos los pueblos}. Y ya habéis oído lo que dice nuestro Señor Jesucristo: \textit{Escrito está: \textquote{Mi casa es casa de oración para todos los pueblos}; pero vosotros la habéis convertido en una \textquote{cueva de bandidos}}.

¿Acaso los que pretendieron convertir la casa de Dios en una cueva de bandidos, consiguieron destruir el templo? Del mismo modo, los que viven mal en la Iglesia católica, en cuanto de ellos depende, quieren convertir la casa de Dios en una cueva de bandidos; pero no por eso destruyen el templo. Pero llegará el día en que, con el azote trenzado con sus pecados, serán arrojados fuera. Por el contrario, este templo de Dios, este Cuerpo de Cristo, esta asamblea de fieles tiene una sola voz y como un solo hombre canta en el salmo. Esta voz la hemos oído en muchos salmos; oigámosla también en éste. Si queremos, es nuestra voz; si queremos, con el oído oímos al cantor, y con el corazón cantamos también nosotros. Pero si no queremos, seremos en aquel templo como los compradores y vendedores, es decir, como los que buscan sus propios intereses: entramos, sí, en la Iglesia, pero no para hacer lo que agrada a los ojos de Dios.
\end{body}


\newsection
\section{Homilías}

\subsection{San Juan Pablo II, papa}

\subsubsection{Homilía (1979): Tengamos celo por la casa}

\src{Visita Pastoral a la Parroquia Romana de \\San José en el Barrio \textquote{Forte Boccea}. \\18 de marzo de 1979.}

\begin{body}
1. \textquote{La casa de mi Padre}.

\ltr{H}{oy} Cristo pronuncia estas palabras en el umbral del templo de Jerusalén. Se presenta sobre este umbral para \textquote{reivindicar} frente a los hombres la casa de su Padre, para reclamar sus derechos sobre esta casa. Los hombres hicieron de ella una plaza de mercado. Cristo los reprende severamente; se pone decididamente contra tales desviaciones. El celo por la casa de Dios lo devora (cf. \textit{Jn} 2, 17), por esto Él no duda en exponerse a la malevolencia de los ancianos del pueblo judío y de todos los que son responsables de lo que se ha hecho contra la casa de su Padre, contra el templo.

Es memorable este acontecimiento. Memorable la escena. Cristo, con las palabras de su ira santa, ha inscrito profundamente en la tradición de la Iglesia la ley de la santidad de la casa de Dios. Pronunciando esas palabras misteriosas que se referían al templo de su cuerpo: \textquote{Destruid este templo, y en tres días lo levantaré} (\textit{Jn} 2, 19), Jesús ha consagrado de una sola vez todos los templos del Pueblo de Dios. Estas palabras adquieren una riqueza de significado totalmente particular en el tiempo de Cuaresma cuando, meditando la pasión de Cristo y su muerte –destrucción del templo de su cuerpo–, nos preparamos a la solemnidad de la Pascua, esto es, al momento en que Jesús se nos revelará todavía en el templo mismo de su cuerpo, levantado de nuevo por el poder de Dios, que quiere construir en él, de generación en generación, el edificio espiritual de la nueva fe, esperanza y caridad [\ldots].

4. La casa es la morada del hombre. Es una condición necesaria para que el hombre pueda venir al mundo, crecer, desarrollarse, para que pueda trabajar, educar, y educarse, para que los hombres puedan constituir esa unión más profunda y más fundamental que se llama \textquote{familia}.

Se construyen las casas para las familias. Después, las mismas familias se construyen en las casas sobre la verdad y el amor. El fundamento primero de esta construcción es la alianza matrimonial, que se expresa en las palabras del sacramento con las que el esposo y la esposa se prometen recíprocamente la unión, el amor, la fidelidad conyugal. Sobre este fundamento se apoya ese edificio espiritual, cuya construcción no puede cesar nunca. Los cónyuges, como padres, deben aplicar constantemente a la propia vida de constructores sabios, la medida de la unión, del amor, de la honestidad y de la fidelidad matrimonial. Deben renovar cada día esa promesa en sus corazones y a veces recordarla también con las palabras. Hoy, [con ocasión de esta visita pastoral,] yo os invito a hacerlo de modo particular, [porque la visita pastoral] debe servir para la renovación de ese templo que formamos todos en Cristo crucificado y resucitado. \textbf{San Pablo} dice que Cristo es \textquote{poder y sabiduría de Dios} (\textit{1 Cor} 1, 24). Sea Él vuestro poder y vuestra sabiduría, queridos esposos y padres. Lo sea para todas las familias de esta parroquia. ¡No os privéis de este poder y de esta sabiduría! Consolidaos en ellos. Educad en ellos a vuestros hijos y no permitáis que este poder y esta sabiduría, que es Cristo, les sea quitado un día. Por ningún ambiente y por ninguna institución. No permitáis que alguien pueda destruir ese \textquote{templo} que vosotros construís en vuestros hijos. Este es vuestro deber, pero éste es también vuestro sacrosanto derecho. Y es un derecho que nadie puede violar sin cometer una arbitrariedad.

5. La familia está construida sobre la sabiduría y el poder del mismo Cristo, porque se apoya sobre un sacramento. Y está construida también y se construye constantemente sobre la ley divina, que no puede ser sustituida en modo alguno por cualquier otra ley. ¿Acaso puede un legislador humano abolir los mandamientos que nos recuerda hoy la lectura del \textbf{libro del Éxodo}: \textquote{No matar, no cometer adulterio, no robar, no decir falsos testimonios} (\textit{Ex} 20, 13-16)? Todos sabemos de memoria el Decálogo. Los diez mandamientos constituyen la concatenación necesaria de la vida humana personal, familiar, social. Si falta esta concatenación, la vida del hombre se hace inhumana. Por esto el deber fundamental de la familia, y después de la escuela, y de todas las instituciones, es la educación y consolidación de la vida humana sobre el fundamento de esta ley, que a nadie es lícito violar.

Así estamos construyendo con Cristo el templo de la vida humana, en el que habita Dios. Construyamos en nosotros la casa del Padre. Que el celo por la construcción de esta casa constituya el núcleo de la vida de todos nosotros aquí presentes (\ldots).
\end{body}

\newpage
\subsubsection{Homilía (1982): Lo que hay en el hombre}

\src{Visita Pastoral a la Parroquia Romana del Santísimo Crucifijo. \\14 de marzo de 1982.}

\begin{body}
\ltr[1. \ldots «]{N}{osotros} predicamos a Cristo crucificado» (\textit{1 Cor} 1, 23). En estas palabras de la \textbf{carta a los Corintios}, Pablo de Tarso pronuncia su mensaje apostólico. \textquote{Predicamos a Cristo crucificado}, que es \textquote{poder y sabiduría de Dios} (\textit{1 Cor} 1, 24). Este mensaje encuentra oposición: para los judíos, que piden milagros, Cristo crucificado es un \textquote{escándalo}; para los griegos, que buscan sabiduría, es \textquote{necedad}. Pablo de Tarso es consciente de la oposición que encuentra su mensaje a los ojos de sus contemporáneos. Sin embargo, lo anuncia con una fuerza mucho mayor, que es la fe: \textquote{Lo que es la locura de Dios es más sabio que los hombres, y lo que es la debilidad de Dios es más fuerte que los hombres} (\textit{1 Cor} 1, 25).

\txtsmall{[Hoy vengo a visitar la parroquia del \textquote{Santísimo Crucifijo}. Lo hago, como obispo de Roma, por amor a vuestra comunidad y con profunda reverencia a Cristo crucificado. ¿No refleja vuestra parroquia, incluso con el mismo nombre, el mensaje de Pablo a los corintios y, por tanto, a todos los cristianos, a todos los hombres? ¡Parroquia del \textquote{Santísimo Crucifijo}!]}

2. \textquote{¡Predicamos a Cristo \ldots}! Este Cristo que conocía y sabe \textquote{lo que hay en cada hombre}. De hecho, en el \textbf{Evangelio} de hoy leemos lo siguiente: \textquote{Mientras él estaba en Jerusalén para la Pascua, durante la fiesta muchos, viendo las señales que estaba haciendo, creyeron en su nombre. Sin embargo, Jesús no se fiaba de ellos porque conocía a todos y no necesitaba que nadie le diera testimonio de otro, más bien sabía lo que hay en cada hombre} (\textit{Jn} 2, 23-25).

Así fue durante la vida terrenal de Jesús. Desde entonces, muchos otros todavía \textquote{creyeron en su nombre}. Aquí en [Roma] muchos creen en Jesucristo. También hay muchos en esta parroquia. Quizás incluso aquellos que aun sin saberlo, creen de alguna manera; incluso aquellos que piensan que no creen. A veces hacemos preguntas a los hombres sobre su fe, incluso se hacen preguntas especiales. Y obtenemos algunas respuestas, ciertamente sinceras.

Sin embargo, en última instancia, sólo él, Cristo, sabe \textquote{lo que hay en cada hombre}. Él sabe esto con la ciencia que solo le pertenece. Ciencia divina y humana, ciencia del Evangelio y de la Redención. Él lo sabe, porque nos ha redimido a cada uno de nosotros. De hecho, fuimos comprados a un precio muy alto (cf. \textit{1 Cor} 6, 20; 7, 23). Y por eso \textquote{predicamos a Cristo crucificado}. Lo predicamos continuamente, sin descanso. También lo predicamos este domingo de Cuaresma, aquí, en esta parroquia.

Es necesario que el hombre, mirándose profundamente en su interior, piense en lo que hay en él; quizás paz de conciencia o quizás inquietud, el peso de los pecados, el peso de una gran responsabilidad, el remordimiento.

Sin embargo, al mismo tiempo, todos deben mirar el Crucifijo y pensar que para él también existe siempre el \textquote{precio alto}. De hecho, ¡a tal precio somos comprados a través de la Cruz!

3. La palabra de este domingo nos recuerda el \textbf{Decálogo}, la ley de Dios dada a Israel a través de Moisés en el monte Sinaí; y dada a todos los hombres. Conocemos estos mandamientos. Muchos los repiten a diario en sus oraciones. ¡El cielo desearía que todos lo hicieran! Es un muy buen hábito. Repitámoslos ahora, como están escritos en el \textbf{libro del Éxodo}, para reconfirmar y renovar lo que recordamos. Los mandamientos fueron dados durante la salida de Israel de Egipto, por obra de Dios; por tanto, las primeras palabras recuerdan este episodio.

\begin{bodyprose}
«Yo soy el Señor, tu Dios, que te saqué de la tierra de Egipto, de la condición de esclavitud:
   No tendrás otros dioses frente a mí \ldots.
   No tomarás el nombre del Señor tu Dios en vano \ldots.

Acuérdate del día de reposo para santificarlo \ldots, 
   aquí decimos: \textit{Acuérdate de santificar las fiestas}.

Honra a tu padre y a tu madre \ldots.
   No mates.

No cometas adulterio.
   No robes.
   No des falso testimonio contra tu prójimo.
   No codicies la casa de tu vecino. 
   No desees a la mujer de tu prójimo, ni a su esclavo, ni a su esclava, 
   ni a su buey, ni a su asno, ni nada que sea de tu prójimo». 
(\textit{Ex} 20, 2-3. 7-8. 17).
\end{bodyprose}

Pronunciamos el último mandamiento con dos fórmulas. El primero: no desear a la mujer de los demás, y el segundo: no desear las cosas de los demás.

¿Fueron todos estos mandamientos grabados solo en las dos tablas que recibió Moisés, e Israel los guardó como la cosa más sagrada en el Arca de la Alianza? ¡No solo! Estos mandamientos están, al mismo tiempo, inscritos en el corazón, en la conciencia de todo hombre.

¿Por qué Dios nos dio a su Hijo Unigénito, como recuerda la liturgia de hoy en el canto al Evangelio? Para que el grabado de los mandamientos divinos no se borre de la conciencia humana; para que el hombre pueda conocer y practicar estos mandamientos, y así tener \textquote{vida eterna}.

A un joven que le pregunta a Jesús: \textquote{¿Qué debo hacer de bueno para obtener la vida eterna?}, El Maestro responde: \textquote{Guarda los mandamientos}. \textquote{¿Cuales?}. Jesús enumera los mismos que recibió Moisés en el monte Sinaí en la alianza antigua (cf. \textit{Mt} 19, 16-22).

4. Jesucristo sabe \textquote{lo que hay en cada hombre}; sabe que los mandamientos del Padre están inscritos en su corazón.

En el \textbf{evangelio} de hoy, Cristo se muestra severo con los que violan el mandamiento del culto y la adoración debidos a Dios mismo: un mandamiento escrito más en la conciencia que en la simple ley.

De hecho, esos vendedores y cambistas quizás estaban de acuerdo con la ley humana, pero Cristo es el que sabe \textquote{lo que hay en cada hombre} y al mismo tiempo lo devora el celo por la casa de Dios (cf. \textit{Jn} 2, 17).

Conduciendo al hombre por la senda de los mandamientos, le enseña no solo a cumplir la ley de Dios, sino también a comprender cada vez mejor y amar esta ley cada vez más profundamente, como afirma el \textbf{Salmo Responsorial} de la Santa Misa.

En la medida en que el hombre comprende los mandamientos divinos, se da cuenta de la gran ayuda que éstos suponen en la vida personal, familiar y social. Son verdaderamente el camino del hombre; son para el hombre.

\begin{bodyprose}
La ley del Señor es perfecta
   y es descanso del alma;
   el precepto del Señor es fiel
   e instruye a los ignorantes.

Los mandatos del Señor son rectos
   y alegran el corazón;
   la norma del Señor es límpida
   y da luz a los ojos.

El temor del Señor es puro
   y eternamente estable;
   los mandamientos del Señor son verdaderos
   y enteramente justos.

Más preciosos que el oro,
   más que el oro fino;
   más dulces que la miel
   de un panal que destila.
(Sal 18 [19], 8-11).
\end{bodyprose}

Valdría la pena detenerse más en estos versículos del Salmo. Entonces veremos mejor cuál es el camino que conduce al amor a los mandamientos divinos, en particular al mayor mandamiento del Evangelio, a ese poder y ese amor divino que Cristo crucificado se ha hecho por nosotros.

¿Acaso no es la Cruz la conciencia suprema de la humanidad? ¿Acaso no es la Cruz la voz misma de Dios hablando con más fuerza que las propias conciencias humanas? ¿Voz que habla de manera particular cuando las diferentes \textquote{medidas humanas} disminuyen esta conciencia y la sofocan? Por tanto, el \textbf{Apóstol} tiene razón cuando clama: \textquote{Predicamos a Cristo crucificado\ldots poder de Dios y sabiduría de Dios}.

5. Meditando sobre la ley divina, sobre la conciencia humana y sobre la cruz de Cristo, la liturgia cuaresmal de hoy nos prepara para el misterio pascual.

Después de la expulsión de los comerciantes y cambistas, algunos judíos se dirigieron a Jesús con esta pregunta: \textquote{¿Qué signo nos muestras para hacer estas cosas? Jesús les respondió: Destruid este templo y en tres días lo levantaré. Entonces los judíos le dijeron: Este templo tardó cuarenta y seis años en construirse, ¿y en tres días lo levantarás? Pero Él hablaba del templo de su cuerpo. Entonces, cuando resucitó de entre los muertos, sus discípulos se acordaron de lo que había dicho, y creyeron en la Escritura y en la palabra de Jesús} (\textit{Jn} 2, 18-22).

6. ¡Queridos hermanos y hermanas! Acoged esta meditación que pronuncio, siguiendo las palabras de la liturgia de hoy, para venerar a Cristo crucificado en la parroquia [del \textquote{Santísimo Crucifijo}].

7. Desde este altar deseo ahora dirigir mi cordial saludo a todos los fieles presentes y a toda la familia parroquial (\ldots) Recuerdo a todos con cariño y ofrezco mis oraciones por todos. (\ldots) 

(\ldots) [Os invito a] profundizar la fe de manera global y exhaustiva para vivirla con coherencia y valentía en la sociedad moderna. Participad en las actividades parroquiales con espíritu de auténtica dedicación, para ser y sentiros cada vez más cristianos convencidos, felices y fervientes, abiertos a la caridad y la ayuda mutua. En particular, me gustaría recomendar la participación en la Santa Misa dominical. Haced el propósito de no fallar nunca. El cristiano es el hombre de la Santa Misa, porque comprende que en ella Cristo renueva su sacrificio redentor por él.

Termino con el más sincero deseo de que en esta parroquia la gente no deje nunca de anunciar a Cristo Crucificado. Que este anuncio sea para toda la comunidad, para cada uno y para todos, \textquote{poder de Dios y sabiduría de Dios} y dé fruto abundante en las conciencias humanas, a pesar de las diversas oposiciones que encuentra en el mundo contemporáneo. De hecho, las encontró no sólo entre los \textquote{judíos} y los \textquote{griegos}, de los que escribe el \textbf{Apóstol}; sino también en el mundo contemporáneo. Pero esto no nos desanima de cara a nuestra misión de anunciar a Cristo, Cristo crucificado.
\end{body}

\newpage
\subsubsection{Homilía (1985): Conversión y amor}

\src{Celebración Eucarística en la Parroquia de \\Nuestra Señora de Bonaria ad Ostia Lido.\\10 de marzo de 1985.}

\begin{body}

\ltr[1. ]{Y}{o} soy el Señor, tu Dios, que te saqué de la tierra de Egipto, de la casa de esclavitud: 

\begin{bodyprose}
No tomarás el nombre del Señor en vano \ldots
   Acuérdate del día de reposo para santificarlo \ldots
   Honra a tu padre y a tu madre \ldots

No mates.
   No cometas adulterio.
   No robes.

No des falso testimonio contra tu prójimo.
   No codicies la casa de tu vecino.
   No desees a la mujer de tu prójimo \ldots ni nada que sea de tu prójimo
(\textit{Ex} 20, 2-3. 7-8. 12-17).
\end{bodyprose}

2. Hoy, tercer domingo de Cuaresma, la Iglesia todavía proclama en este pasaje del \textbf{libro del Éxodo}, que contiene el \textquote{Decálogo}: los diez mandamientos proclamados e impuestos a los hijos e hijas de Israel al pie del monte Sinaí. Ésta es la ley divina que determina los principios fundamentales del comportamiento humano, las principales reglas de la moral, según las cuales las obras humanas adquieren el carácter de bien o mal moral. La observancia de estas normas, de estos mandamientos, imprime el signo del bien en nuestras obras, hace al hombre bueno. La infracción, la transgresión de ellos imprime en nuestras obras el signo del mal: hace malo al hombre. Este bien y este mal conciernen al hombre en su propia humanidad. A través del bien moral, el hombre, como hombre, se vuelve y es bueno. A través del mal moral, el hombre, como hombre, se vuelve y es malo.

Por tanto, es el problema fundamental desde el punto de vista del propio valor esencial del hombre. La ley moral permanece estrictamente relacionada con este valor. Conectada, se puede decir, con la dignidad del hombre, y unida a la dignidad de toda convivencia de los hombres entre sí. La ley moral tiene un significado tanto personal como social. Es Dios quien proclama esta Ley: el Decálogo manifiesta así la Providencia de Dios, su preocupación paterna por el bien fundamental del hombre. Él es el Dios que sacó a los hijos de Israel de la tierra de Egipto, de la condición de esclavitud.

3. Recordemos el Decálogo durante la Cuaresma, porque es el período en el que Jesucristo desafía de manera particular a la conciencia humana. Desde el comienzo de estas jornadas cuaresmales hemos sentido la llamada a la conversión, a la reconciliación con Dios, llamada que encuentra su fundamento objetivo en el Decálogo. Convertirse significa romper con el mal, romper con el pecado, fortalecerse nuevamente en el bien y consolidar el comportamiento en él. Sabemos que Jesucristo ha reconfirmado plenamente los mandamientos divinos del monte Sinaí. Dio instrucciones a los hombres para que los observaran. Indicó que la observancia de los mandamientos es la condición fundamental de la reconciliación con Dios, la condición fundamental para el logro de la salvación eterna.

Por eso la liturgia de hoy proclama: \textquote{¡Señor, tú tienes palabras de vida eterna!} \textquote{La ley del Señor es perfecta\ldots Los mandatos del Señor son rectos, y alegran el corazón\ldots los mandamientos del Señor son verdaderos y enteramente justos. Más preciosos que el oro, más que el oro fino} (\textit{Sal} 19, 8-11). El período de Cuaresma es el tiempo en el que debemos volver a los mandamientos de Dios. A su luz debemos examinar nuestra conciencia, para que no crezca sobre ella una capa de pecado e iniquidad.

4. Jesucristo reconfirma la ley divina de la antigua Alianza, proclamada en el Sinaí. Pero, al mismo tiempo, la misión con la que se dirige a la humanidad va más allá. \textquote{Dios\ldots amó tanto al mundo hasta entregar a su único Hijo; todo el que cree en Él tiene vida eterna} (cf. \textit{Jn} 3, 16). La fe en Cristo es todavía algo más que la pura obediencia a la Ley, aunque sea dictada por el sincero \textquote{temor del Señor} del que habla el salmo (cf. \textit{Sal} 112). Creer en Dios significa afrontar el mismo amor con el que Dios amó al mundo. El amor de Dios se expresa en el hecho de que dio a su único Hijo.

Por eso todo el orden moral de la nueva alianza alcanza su cúspide y su centro en el mandamiento del amor. Hacemos, sí, la voluntad de Dios, observando todos los mandamientos divinos; pero a Dios que se ha revelado en Cristo como amor, ¡solo podemos responder por medio del amor! Por tanto, nuestro examen de conciencia cuaresmal debe centrarse en las exigencias del amor a Dios y al prójimo. Esta es, al mismo tiempo, la principal vía de conversión que Cristo espera de nosotros. Es una conversión constante y continua. Así como debemos orar continuamente, también debemos convertirnos constantemente.

5. Por tanto, la Iglesia, en el período de Cuaresma, no se limita a proclamar la ley divina: el Decálogo del Monte Sinaí. Ella también proclama para nosotros, junto con \textbf{San Pablo}, a Cristo crucificado en el monte Calvario.

Una vez los judíos pidieron milagros y los griegos buscaron sabiduría (cf. \textit{1 Co} 1, 22). Los contemporáneos se comportan como los judíos y griegos de la época apostólica. De hecho, sus solicitudes van mucho más allá. A veces, la crítica y la oposición a la enseñanza de Dios, a los mandamientos divinos es mucho más dura. Sin embargo, al mismo tiempo, la Iglesia permanece fiel a la indicación del apóstol: \textquote{Predicamos a Cristo crucificado} (\textit{1 Co} 1, 23). ¡En él está la respuesta a todo! Toda crítica, toda oposición a la doctrina divina palidece ante la elocuencia de Cristo crucificado. La cruz del Calvario \textquote{es más sabia que los hombres} y \textquote{es más fuerte que los hombres}, como escribe San Pablo (\textit{1 Co} 1, 25).

6. Hoy, tercer domingo de Cuaresma (\ldots), juntos \textquote{predicamos a Cristo crucificado}, y juntos lo profesamos, tal como la Iglesia  lo ha predicado y profesado durante casi dos mil años, heredera de la fe de los santos apóstoles Pedro y Pablo.

[\ldots]

(\ldots) La misa dominical, queridos hermanos y hermanas: esta es la base de todo, y debo pediros que no la descuidéis, que seáis más asiduos a ella, que sintáis, todos los domingos y fiestas, que el Señor viene a vuestro encuentro, os reúne en torno a la mesa doble de la palabra y del cuerpo de Cristo. Sin esta referencia constante a la mesa de Cristo no es posible construir una vida verdaderamente cristiana. Toda la fuerza misionera de una parroquia y toda su capacidad o esperanza para la formación de los jóvenes proviene de esta fuente constante y rica de la presencia de Cristo en la misa dominical.

\txtsmall{[Deseo alentar todos vuestros trabajos, deseo subrayar los resultados significativos de vuestra presencia cristiana en un área que comenzó a formarse con grandes privaciones y sacrificios. Os acordáis bien del primer cuartel, la iglesia en el sótano, la afluencia de gente en dificultad para trabajar, familias y personas casi expulsadas de las afueras de la gran metrópoli Habéis recorrido un largo camino desde entonces; continuemos ahora con todos nuestros esfuerzos para construir juntos la comunidad espiritual de fe, la parroquia del servicio interior a las conciencias y almas, la comunidad educadora en el Espíritu de Cristo, una comunidad de hermanos que irradian el Evangelio, una familia de creyentes y testigos, guiados e iluminados por la presencia bendita y la protección de la Virgen.]}

8. \textquote{Yo soy el Señor, tu Dios, que te saqué de la tierra de Egipto, de la casa de esclavitud}. Dios, que liberó a Israel de Egipto, libera constantemente al hombre: lo libera del pecado. Hacia esta liberación conduce el camino de los mandamientos divinos: el Decálogo y el mandamiento del amor. En este camino encontramos a Cristo que en el \textbf{Evangelio} de hoy dice: \textquote{Destruid este templo y en tres días lo levantaré} (\textit{Jn} 2, 19). Y lo dice \textquote{del templo de su cuerpo} (\textit{Jn} 2, 21), lo dice de la resurrección.

En el período de Cuaresma escudriñamos nuestra conciencia a la luz de los mandamientos de Dios, para librarnos del pecado. Y renovamos en nosotros la esperanza ligada a la resurrección de Cristo, en la que se encierra el principio de la liberación completa del mal, el pecado y la muerte. La liberación de la tierra de Egipto, de la condición de esclavitud, de hecho no estaba lejos de anunciar la liberación, que es compartida por nosotros en Jesucristo.
\end{body}

\newpage
\subsubsection{Homilía (1988): Ídolos contemporáneos}

\src{Visita a la Parroquia de San Dámaso en Monteverde. \\6 de marzo de 1988.}

\begin{body}
1. \textquote{Jesús\ldots sabía lo que hay dentro de cada hombre} (\textit{Jn} 2, 25).

\ltr{L}{a} liturgia del tercer domingo de Cuaresma nos invita a seguir esta \textquote{sabiduría}. La \textquote{sabiduría} de Dios sobre el corazón humano está profundamente inscrita en los acontecimientos del Sinaí que nos refiere el \textbf{libro del Éxodo}. Aquí, habla a Israel, al pueblo elegido, el mismo Dios que \textquote{lo sacó de la tierra de Egipto} (\textit{Ex} 20, 5). Cuando dice: \textquote{No mates. No cometas adulterio. No robes. No des falso testimonio contra tu prójimo. No desees\ldots} (\textit{Ex} 20, 12-17). Dios sabe que en el corazón del hombre se esconde una \textquote{inclinación}, una predisposición a cada uno de estos pecados, a todas las facetas del mal. Incluso una inclinación al crimen. El Dios de nuestros padres sabe todo esto desde el principio, desde el tiempo del árbol del conocimiento del bien y del mal y desde el tiempo del primer pecado. Desde entonces, el hombre, cediendo a la tentación del Maligno, por primera vez se ha creído que podía ser \textquote{como Dios} (cf. \textit{Gn} 3, 5), y ha descendido por la senda del pecado.

2. Sin embargo, en este hombre permanecía una misteriosa necesidad de buscar \textquote{dioses} fuera del único Dios verdadero. El pueblo que estaba al pie del monte Sinaí, aunque elegido por el Dios verdadero, también mostró esta propensión: \textquote{a tener otros dioses} (cf. \textit{Ex} 20, 3). Durante los días en que Moisés se quedó con Dios en el monte Sinaí, recibiendo de él las tablas de la ley divina –o Decálogo– su pueblo también se hizo a sí mismo un \textquote{dios} en forma de \textquote{becerro fundido} (\textit{Ex} 32, 4). Y en esta forma superficial y falseada dio rienda suelta a la perenne necesidad del corazón humano de volver a Dios, poniendo \textquote{un dios de oro} en el lugar del Dios verdadero. Este hecho debe hacernos meditar mucho, cuánto espacio, de hecho, el \textbf{libro del Éxodo} dedica a este problema: \textquote{No tendrás otros dioses fuera de mí. No te fabricarás ídolos ni imagen alguna\ldots (como ese ‘becerro fundido’, por ejemplo)\ldots No te inclinarás ante ellos ni les servirás\ldots} (\textit{Ex} 20, 3-5).

Ese problema, ¿era solo actual en aquellos tiempos lejanos? ¿O no es siempre actual, aunque sea de otras formas? El hombre contemporáneo ciertamente ya no adora a los \textquote{ídolos} como lo hacían los antiguos paganos. Hoy, sin embargo, el hombre hace otra cosa con esa necesidad muy profunda de su ser humano, con la necesidad de \textquote{trascendencia} (como solemos decir hoy). Y aunque no reemplaza materialmente al Dios verdadero con un \textquote{becerro fundido}, hay algún otro \textquote{ídolo} contemporáneo que se traga las energías más profundas de su alma.

A menudo estos \textquote{ídolos} contemporáneos son de naturaleza sutil, conectados con el progreso del pensamiento, con el refinamiento de las propensiones humanas, con el estilo de civilización que exalta un programa de vida que prescinde de Dios: como si él no existiera.

3. Dios, que habla en el \textbf{libro del Éxodo} se llama a sí mismo: \textquote{Dios celoso} (\textit{Ex} 20, 5). ¡Sí! Dios es \textquote{celoso}, de un \textquote{celo} divino por el hombre. Celoso de esta criatura, en la que imprimió su imagen y semejanza desde el principio, y en cuya forma corporal inspiró el alma inmortal. ¡Sí! Dios es \textquote{celoso} de lo que existe de él en el hombre, y que no puede satisfacerse de otra manera que sólo en él y para él. \textquote{No tendrás otros dioses frente a mí\ldots Amarás al Señor tu Dios con todo tu corazón, con toda tu alma y con todas tus fuerzas\ldots}. De lo contrario, tú, hombre, ¡no te encontrarás a ti mismo! ¡Te perderás! ¡Sí! Dios es \textquote{celoso} del hombre así como Cristo estaba \textquote{celoso} de la santidad de la casa de Dios en Jerusalén. El \textbf{evangelio} de hoy nos recuerda esto: \textquote{No hagáis de la casa de mi Padre un mercado} (\textit{Jn} 2, 16). Entonces \textquote{recordaron los discípulos lo que está escrito: \textquote{el celo por tu casa me devora}} (\textit{Jn} 2, 17). Cristo expulsó a los comerciantes del templo, así como Moisés, al pie del Sinaí, había \textquote{disipado} a los idólatras.

4. El mensaje central de este domingo de Cuaresma nos manda a seguir esta \textquote{sabiduría} de Dios sobre el hombre, que se ha revelado plenamente en Cristo.

¿Quién es el Dios \textquote{celoso}? ¿Celoso de los \textquote{celos} divinos? Es ese Dios que amó al mundo. Con amor eterno amó al hombre en el mundo. Y sabiendo \textquote{lo que hay en cada hombre} y de lo que es capaz su corazón dividido en el conocimiento del bien y del mal, este Dios \textquote{ha dado a su Hijo unigénito}. El don del Hijo de la misma sustancia que el Padre es la vara de medir del amor de Dios por el mundo: ¡por el hombre que está en el mundo! Sólo en este Hijo, sólo por él, puede el hombre alcanzar la vida eterna. Y tenerla. Sólo este Dios y nadie más ha inscrito en las profundidades del alma humana la inmortalidad, llamándola a la existencia.

Dios da a la humanidad a su Hijo consustancial , como redentor del mundo, porque conoce plenamente \textquote{lo que hay en cada hombre}. Él solo. Porque solo Él es el creador del hombre. Y Él es un amante del hombre (\textquote{filo-anthropos}).

5. El apóstol Pablo es plenamente consciente de esta \textquote{sabiduría} de Dios, y de este misterio divino que se reveló plenamente en Cristo: crucificado y resucitado. Cristo crucificado: el que se puso en el lugar del templo de Jerusalén cuando dijo \textquote{destruid este templo y en tres días lo levantaré} (\textit{Jn} 2, 19). Habló de su muerte y resurrección al tercer día. Pablo, aún siendo un enemigo acérrimo, se encontró con el Resucitado cerca de Damasco y, a la luz de la resurrección, creyó en el poder de su cruz. De hecho, escribió a los \textbf{Corintios}: \textquote{Predicamos a Cristo crucificado \ldots poder de Dios y sabiduría de Dios} (\textit{1 Cor} 1, 23-24). ¡Sí! Es poder de Dios. He aquí que: \textquote{Él hace volver a levantar el templo de su cuerpo torturado\ldots}.

Es sabiduría. Sí, es la sabiduría de Dios: él sabe –hasta el fondo– \textquote{lo que hay en cada hombre}. El hombre no se conoce a sí mismo si no participa de esta \textquote{sabiduría} de la cruz y la resurrección. Es al mismo tiempo la \textquote{sabiduría} del \textquote{amor con que tanto amó Dios al mundo} (\textit{Jn} 3, 16). Esta \textquote{sabiduría} es poder. Sólo ella es el poder del hombre. Sólo ella es capaz de transformar profundamente el corazón humano.

[\ldots]

7. \txtsmall{(\ldots) A todos los presentes, y a sus seres queridos, les dirijo mis pensamientos afectuosos y mis mejores deseos. (\ldots)} Como es sabido, la participación en la liturgia y los sacramentos ocupan un lugar central en la vida parroquial. De esta fuente de vida sobrenatural nace la comunidad; de ella brota la linfa vital que sostiene la fe y el fervor de cada creyente. En este sentido, quisiera llamar la atención sobre la importancia de la práctica del sacramento de la Reconciliación, especialmente en este tiempo de Cuaresma. La fuerza espiritual que se desprende de este sacramento para la vida cristiana es inconmensurable: de hecho nos acerca a la santidad de Dios; nos permite encontrar la paz interior cuando estamos atribulados por el pecado, y recuperar la alegría perdida, haciéndonos sentir acogidos íntimamente por el abrazo misericordioso de Dios.

\txtsmall{[Deseo expresar mi satisfacción por las iniciativas que la parroquia promueve a favor de los jóvenes, especialmente de aquellos que se encuentran en situaciones difíciles, debido al uso de drogas y la consiguiente marginación; animo a las personas a continuar las actividades culturales a favor de aquellos que deseen ser educados en la doctrina de la fe, que lleguen a perfeccionar y completar las lecciones catequéticas impartidas a los jóvenes en preparación para la Primera Comunión y la Confirmación. Agradezco también a todos los que colaboran en la pastoral parroquial\ldots Por último, mi aplauso a quienes de manera encomiable dedican su tiempo libre y su energía a ayudar y cuidar a los ancianos o enfermos en casa. ¡Que el Señor recompense su generosa entrega y su solidaridad evangélica!]}

8. El \textbf{salmista} proclama: \textquote{El temor del Señor es puro, y eternamente estable; los mandamientos del Señor son verdaderos y enteramente justos. Más preciosos que el oro, más que el oro fino} (\textit{Sal} 19 [18], 10-11). ¡Oremos para tener temor de Dios! A veces falta en el hombre de nuestra época. ¡Sí! Oremos para que nos sea concedido este temor, que es \textquote{el principio de la sabiduría}. Aprendamos esta sabiduría, la sabiduría más profunda y definitiva, que se manifiesta en la cruz de Cristo, a través de su resurrección. Para que no nos sorprenda el \textquote{juicio divino}, que siempre es \textquote{justo}. Dios sabe lo que hay en cada hombre. No necesita el testimonio de nadie. Acojamos sólo este único testimonio: es el testimonio de la cruz y resurrección de Cristo.

Amén.
\end{body}

\newpage
\subsubsection{Homilía (1991): Verdadero templo}

\src{Visita Pastoral a la Parroquia Romana de Todos los Santos.\\3 de marzo de 1991.}

\begin{body}
\textquote{Destruid este templo y en tres días lo levantaré} (\textit{Jn} 2, 19).

\ltr[1. ]{Q}{ueridos} hermanos y hermanas, el \textbf{pasaje evangélico} de la liturgia de hoy nos proyecta hacia la Pascua del Señor. Nos revela su significado, especialmente en lo que respecta a las relaciones de fidelidad y servicio que Dios pide a quienes, muertos y resucitados con Cristo, forman el pueblo de la Nueva Alianza.

Jesús sube a Jerusalén y, como todo israelita piadoso, va al templo a orar, pero lo encuentra transformado en un \textquote{mercado}. Sobre todo, se da cuenta de que el culto que allí se celebra, como ya habían denunciado los profetas, ya no se inspira en la fidelidad a la \textquote{ley de la Alianza}, sino que ha degenerado en actos formalistas y externos, desligados de la vida.

Con un gesto profético, que escandaliza a los judíos allí reunidos para la fiesta de Pascua, expulsa a los comerciantes y reafirma con fuerza el destino original del templo, como \textquote{casa de Dios}.

Jesús, \textquote{Hijo} de Dios que vino a ocuparse de las cosas que conciernen al Padre (cf. \textit{Lc} 2, 49), se siente herido por cómo se ha deshonrado el culto exigido a los verdaderos adoradores. Pero los judíos, cegados por su incredulidad, piden una señal autorizada de confirmación de sus palabras y del hecho realizado. Y Jesús lo da con un anuncio a primera vista incomprensible y muy diferente a sus expectativas, pero que luego quedará claro para los discípulos: \textquote{Destruid este templo y en tres días lo levantaré}. De hecho, como señala el evangelista, no hablaba del templo de piedra, sino del tempo de su Cuerpo. De este modo se nos introduce en la comprensión de otra gran verdad que recuerda la resurrección de Cristo y su significado salvador en la vida de quienes creen en él.

2. En esta perspectiva, la humanidad de Cristo, glorificada con la resurrección, se convierte en el verdadero templo de Dios, \textquote{en el que habita la plenitud de la divinidad} (\textit{Col} 2, 9); se convierte en el único \textquote{lugar} que Dios ha elegido para hacerse presente y revelarse al hombre, con su santidad y su misericordia.

El edificio en el que nos reunimos para darle a Dios el culto que le agrada es importante, pero sigue siendo secundario. Lo esencial es la experiencia de Cristo resucitado, posible gracias a la acción del Espíritu que emana de su Pascua. Es Él la \textquote{nueva ley}, escrita en el corazón de los creyentes, que los guía al conocimiento de toda la verdad, los capacita para realizar el culto espiritual que involucra toda su existencia y los empuja a testificar y servir al hombre.

Lo mismo debe suceder también para vuestra comunidad [y para la Iglesia de Roma, que vive este tiempo propicio de Cuaresma en un clima de renovación espiritual al que la llama el Sínodo pastoral diocesano].

La Cuaresma, de hecho, es un tiempo de \textquote{iluminación}: la palabra divina conduce gradualmente al pueblo de Dios a descubrir y recordar la riqueza del misterio pascual. Será el mismo Espíritu, fuente de sabiduría y santidad, dado a los fieles a través de los sacramentos pascuales del bautismo y la confirmación, que les permitirá \textquote{recordar} el pleno significado de las palabras de Cristo y entrar en una comunión más intensa con él, especialmente en la Eucaristía.

3. Queridos hermanos y hermanas [de la parroquia de Todos los Santos], las palabras que se acaban de proclamar [en este templo restaurado y renovado] están dirigidas a cada uno de vosotros. Os invitan a experimentar los frutos de la \textquote{nueva ley} ya consagrarse al servicio del Reino.

Haced de Cristo el centro de vuestras vidas y el corazón de vuestro apostolado: esta es la llamada misionera que os anima; \txtsmall{[este es el programa apostólico que guió a Don Orione y que aún hoy conserva toda su actualidad. Han pasado más de 80 años desde que Pío X envió al apóstol de la caridad de Porta San Giovanni en 1908. El Pontífice lo envió como misionero a la \textquote{Patagonia romana}. Desde entonces vuestra parroquia ha crecido mucho y se han multiplicado sus actividades pastorales y caritativas. Siguiendo los pasos del Fundador y de sus hijos espirituales que trabajaron aquí y siguen trabajando,]} vosotros queréis ser los apóstoles de la hora presente, amando a Dios y amando a vuestros hermanos: es más, amando a Dios sin reservas para poder servir al prójimo sin cesar.

(\ldots) El anuncio del Evangelio debe llevarse a todos, para que el anuncio de la muerte y resurrección del Señor resuene en todos los rincones y en todas las casas del barrio. Sin embargo, este anuncio sólo puede ser proclamado de manera creíble por una comunidad viva y unida, humilde y valiente, fiel al designio divino y al servicio de los más pobres.

4. \txtsmall{[Precisamente para animaros a continuar en este esfuerzo de renovación espiritual y evangelización he querido visitar vuestra parroquia, ferviente y llena de iniciativas.]} (\ldots) Os saludo a todos vosotros, queridos fieles\ldots y a vuestros familiares, especialmente a quienes están enfermos, a los ancianos y a los que sufren\ldots

[\ldots]

Finalmente, expreso mi exhortación a todos a perseverar en el compromiso de conversión personal y atención a los hermanos. Así, la comunidad cristiana, de la que cada uno es parte viva, será un centro de animación de la paz y la alegría que emana del Redentor.

Que la ayuda maternal de María, Madre de la Divina Providencia, y la intercesión [del Beato Luis Orione] os sostengan en esta misión.

Que Cristo sea en vosotros fuente de vida nueva. Que Cristo sea vuestra vida y vuestro gozo.

Amén.
\end{body}


%1994
\subsubsection{Homilía (1994): El cimiento de la Verdad}

\src{Visita Pastoral a la Parroquia Romana de San Bernardo de Claraval.\\6 de marzo de 1994.}

\begin{body}
Hoy la Palabra de Dios en la liturgia nos habla del templo. 

\txtsmall{[Tenéis la alegría de tener un nuevo templo, una nueva iglesia parroquial construida a costa de muchos esfuerzos y sacrificios. Los niños de la Primera Comunión cantaron sobre este tema, ilustrando con sus manos y voces cómo se ha colocado ladrillo a ladrillo para construir esta iglesia. Me alegro con vosotros, con los diseñadores, con los constructores, con toda vuestra comunidad, con vuestro párroco y con vuestro clero. Es una gran alegría para mí también. Esta nueva iglesia pertenece a la Iglesia de Roma y así, a través de ella, se consolida la Iglesia de Roma. No habiendo podido venir en noviembre, mi alegría por esta visita de hoy es aún mayor.]}

\ltr{L}{a} liturgia nos habla de un templo metafórico. No es solo el templo construido con piedras, sino también el templo construido por personas. Por eso centramos nuestra reflexión en la familia, porque la antigua tradición de la Iglesia de los Padres llamaba a la familia iglesia doméstica. Está formada por padres, madres, hijos e hijas. Dios habita entre ellos y quiere encontrar su hogar en este templo viviente. Por tanto, la familia es una Iglesia doméstica, pero para serlo debe estar basada en un fundamento sólido que es la Verdad, como recordaba en la Encíclica \textit{Veritatis Splendor}.

El poder de la verdad nos permite construir un templo dentro de nosotros mismos y entre nosotros, construirlo en la familia, en la sociedad, en toda la humanidad.

Hay quienes quisieran construirlo sin la Verdad y contra la Verdad. Sobre todo, se ataca el templo, la familia, iglesia doméstica. Se quiere quitar el fundamento de la Verdad, aprovechando las debilidades humanas, afirmando la legitimidad de los divorcios, las separaciones y todo lo que esté en contra de la vida de los no nacidos y los ancianos. Se intenta afirmar esto, contra el fundamento y contra el precepto que claramente está a favor de la vida. La vida es sagrada.

Lo mismo ocurre con el intento de legitimar a las familias falsas formadas por dos hombres o dos mujeres. Respetamos a todos los hombres y a todas las mujeres, pero construir una familia sobre esta base es incorrecto y peligroso.

[Durante este Año de la Familia,] toda la Iglesia debe ser muy prudente y muy valiente en la defensa de la verdadera familia. Debemos tener una gran comprensión de todas las debilidades humanas, como lo hizo Cristo, pero para la familia, entendida como el principio de construcción de la sociedad, debemos ser intrépidos e intransigentes. También intento serlo, aunque el Papa es por naturaleza un hombre dulce, no severo y rígido. Pero es necesario ser estricto con los principios. La construcción se basa en la verdad y por tanto en los preceptos. La Iglesia nos recuerda hoy el \textbf{Decálogo}, los Diez Mandamientos, que son las piedras inmóviles. Ninguno de ellos puede ser abolido, todas estas piedras deben mantenerse firmes. Así se construye la Iglesia.

Junto al pasaje del Antiguo Testamento que nos recuerda los Diez Mandamientos, está la Lectura del \textbf{Evangelio} en la que Cristo, en el templo de Jerusalén, dice que el edificio será destruido, previendo la catástrofe del año setenta. El templo construido de piedras se puede destruir, dice Jesús, también podéis destruir el templo de mi cuerpo, pero en tres días lo reconstruiré. Se trata de la promesa de la Resurrección.

Cristo, muerto y resucitado, viene a ser para nosotros la primera piedra, la piedra angular de toda la construcción de la Iglesia doméstica y de toda la humanidad. El ministerio de la Iglesia católica es precisamente construir la vida humana sobre la piedra que es Jesús. \textbf{San Pablo} nos dice que Jesús crucificado es nuestra sabiduría y nuestra fuerza. Es una paradoja, pero de este Crucifijo viene toda nuestra fuerza, la fuerza de los que sufren y de todos aquellos que no quieren equivocarse en la vida, que quieren seguir el camino recto, que quieren construir y no destruir.

En esta Cuaresma, con gran energía, presentamos a Cristo crucificado como nuestra sabiduría, como nuestra fuerza ante todo el mundo. Miradlo a Él, no lo dejéis de lado. Si queréis excluir el nombre de Dios, toda vuestra construcción será en vano. Se destruirá a sí misma.

Hay un cierto drama en la Liturgia de la Palabra durante el tiempo de Cuaresma, más que en otros períodos: el Decálogo, el discurso de Jesús en el templo, las palabras de Pablo.

A través de estas palabras debemos encontrar siempre la serenidad: Dios es más fuerte que las debilidades humanas y que las desviaciones humanas. Dios es siempre más fuerte, siempre tendrá la última palabra. Debemos tener el espíritu de San Bernardo, que fue un doctor mariano y en María, incluso en la turbulenta época medieval, siempre supo encontrar la serenidad mariana.

Sigamos el ejemplo del Patrón de vuestra parroquia y busquemos la serenidad en María para resolver los conflictos de nuestro tiempo.
\end{body}

\img{two_testaments}

\newpage
\subsubsection{Ángelus (1997): Vendedores del templo de nuestra época}

\src{2 de marzo de 1997.}

\begin{body}
\ltr[1. ]{E}{n} el \textbf{evangelio} de este tercer domingo de Cuaresma, \textbf{san Juan} relata que Jesús, al encontrar en el templo de Jerusalén a vendedores y cambistas, hizo un azote de cordeles y los arrojó con palabras encendidas: \textquote{¡Quitad esto de aquí: no convirtáis en un mercado la casa de mi Padre!} (\textit{Jn} 2, 16). La actitud \textquote{severa} del Señor parecería estar en contraste con la mansedumbre habitual con la que se acerca a los pecadores, cura a los enfermos, acoge a los pequeños y a los débiles. Sin embargo, observando con atención, la mansedumbre y la severidad son expresiones del mismo amor, que sabe ser, según la necesidad, tierno y exigente. El amor auténtico va acompañado siempre por la verdad.

Ciertamente, el celo y el amor de Jesús a la casa del Padre no se limitan a un templo de piedra. El mundo entero pertenece a Dios, y no se ha de profanar. Con el gesto profético que nos refiere el texto evangélico de hoy, Cristo nos pone en guardia contra la tentación de \textquote{comerciar} incluso con la religión, supeditándola a intereses mundanos o, de cualquier modo, ajenos a ella.

Cristo alza su voz también contra los \textquote{vendedores del templo} de nuestra época, es decir, contra cuantos convierten el mercado en su \textquote{religión} hasta ofender, en nombre del \textquote{dios-poder y del dios-dinero}, la dignidad de la persona humana con abusos de todo tipo. Pensemos, por ejemplo, en la falta de respeto a la vida, hecha objeto a veces de peligrosos experimentos; pensemos en la contaminación ecológica, la comercialización del sexo, el tráfico de drogas y la explotación de los pobres y los niños.

2. La \textbf{página evangélica} también tiene un significado más específico, que remite al misterio de Cristo y anuncia la alegría de la Pascua. Respondiendo a quienes le pedían que confirmara con un \textquote{signo} su profecía, Jesús lanza una especie de desafío: \textquote{Destruid este templo, y en tres días lo levantaré} (\textit{Jn} 2, 19). El mismo evangelista advierte que hablaba de su cuerpo, aludiendo a su futura resurrección. Así, la humanidad de Cristo se presenta como el verdadero \textquote{templo}, la casa viva de Dios. Será \textquote{destruida} en el Gólgota, pero inmediatamente volverá a ser \textquote{reconstruida} en la gloria, para transformarse en morada espiritual de cuantos acogen el mensaje evangélico y se dejan plasmar por el Espíritu de Dios.

3. Que la Virgen nos ayude a acoger las palabras de su Hijo divino. La misión de María consiste, precisamente, en llevarnos a él, repitiéndonos la invitación que hizo a los sirvientes en Caná: \textquote{Haced lo que él os diga} (\textit{Jn} 2, 5). Escuchemos su voz materna. María sabe bien que las exigencias del Evangelio, incluso cuando son pesadas y duras, constituyen el secreto de la verdadera libertad y de nuestra felicidad auténtica.
\end{body}


%1997
\subsubsection{Homilía (1997): La Resurrección}

\src{Visita Pastoral a la Parroquia Romana de San Juliano Mártir. \\2 de marzo de 1997.}

\begin{body}
1. \textquote{Señor, tú tienes palabras de vida eterna} (cf. \textit{Jn} 6, 68).

\ltr{E}{l} \textbf{Salmo responsorial} que acabamos de proclamar nos lleva al corazón del mensaje de la liturgia de hoy. El poder de la Palabra divina se manifestó por primera vez en la creación del mundo, cuando Dios dijo: \textquote{\textit{Fiat}} (cf. \textit{Gn} 1, 3), llamando a la existencia a todas las criaturas. Pero las lecturas bíblicas de este tercer domingo de Cuaresma destacan otra dimensión del poder de la Palabra de Dios: la que se refiere al orden moral.

Dios entregó al pueblo elegido el Decálogo en el monte Sinaí, montaña que reviste singular valor simbólico en la historia de la salvación. \txtsmall{[Precisamente por esto, con ocasión del gran jubileo de año 2000, se ha propuesto un encuentro en ese monte (cf. \textit{Tertio millennio adveniente}, 53).]} La \textbf{primera lectura} de hoy, tomada del \textbf{libro del Éxodo}, desarrolla de modo particular los primeros tres mandamientos dados a Israel, esto es, los de la que se suele llamar \textquote{primera tabla}: \textquote{Yo soy el Señor, tu Dios (\ldots). No tendrás otros dioses frente a mí (\ldots). No pronunciarás el nombre del Señor, tu Dios, en falso (\ldots). Guardarás el sábado para santificarlo} (\textit{Ex} 20, 2. 7-8).

2. Es fundamental el primer mandamiento, en el que se afirma solemnemente la unicidad de Dios: no hay otras divinidades, además de Él. En la ley dada a Moisés, se manifiesta el Dios invisible, que ninguna imagen realizada por las manos del hombre puede representar dignamente. Con la encarnación del Verbo, Dios se hizo hombre, y así el Dios invisible se hizo visible y, desde ese momento, la humanidad puede contemplar su gloria. La cuestión de la representación artística de Dios fue examinada detenidamente en el segundo concilio de Nicea, y se aclaró entonces que, dado que el Dios invisible se había hecho hombre en la Encarnación, su reproducción artística era legítima para los cristianos.

Al primer mandamiento está muy unido el segundo, que no sólo quiere condenar el abuso del nombre de Dios, sino que también tiene como finalidad advertir que no se siga la idolatría difundida en las religiones paganas. De la misma forma, por lo que concierne al tercer mandamiento: \textquote{Guarda el sábado para santificarlo} (\textit{Ex} 20, 8), la normativa es detallada y se remonta al modelo originario del descanso, del que dio ejemplo Dios al término de la creación. En cambio, se describen de manera sintética los mandamientos de la que se suele llamar \textquote{segunda tabla}.

3. \textquote{Señor, tú tienes palabras de vida eterna}. Las palabras que Dios pronuncia en el Antiguo Testamento encuentran pleno cumplimiento en Cristo, Palabra de Dios encarnada. En la antigua alianza, el poder creador de Dios en el ámbito moral se expresó en el Decálogo; en la nueva alianza, en cambio, Cristo es la actuación plena de ese poder; por tanto, no es una ley escrita, sino la persona misma del Salvador.

Se trata de una verdad que san Pablo expresa con eficacia al escribir a los Gálatas y a los Romanos: a la justificación mediante la observancia de la ley contrapone la Justificación mediante la fe en Cristo. Hoy, en cambio, en la \textbf{segunda lectura}, tomada de la \textbf{primera carta a los Corintios}, leemos estas palabras: \textquote{Nosotros predicamos a Cristo crucificado: escándalo para los judíos, necedad para los griegos; pero para los llamados en Cristo –judíos o griegos– fuerza de Dios y sabiduría de Dios} (\textit{1 Co} 1, 22-24).

El poder y la sabiduría que Dios manifestó al crear el mundo y al hombre, hecho \textquote{a su imagen y semejanza} (cf. \textit{Gn} 1, 26), se expresan plenamente en el orden moral. Por tanto, está al servicio del bien del hombre y de la sociedad humana. Esto lo confirma el Nuevo Testamento que determina con claridad el papel de la moral al servicio de la salvación eterna del hombre.

Precisamente por esto, en la aclamación antes del Evangelio acabamos de proclamar las palabras que Jesús pronunció en el diálogo con Nicodemo: \textquote{Tanto amó Dios al mundo, que entregó a su Hijo único. Todo el que cree en él, tiene vida eterna} (\textit{Jn} 3, 16). No sólo los mandamientos, sino sobre todo el Verbo eterno, que se hizo hombre, es la fuente de la vida eterna. [\ldots]

5. \textquote{Él hablaba del templo de su cuerpo} (\textit{Jn} 2, 21).

En el \textbf{evangelio} hemos releído el episodio de la expulsión de los vendedores del templo. La descripción de san Juan es viva y elocuente: por una parte está Jesús que, \textquote{haciendo un azote de cordeles, los echó a todos del templo, con sus ovejas y bueyes} (\textit{Jn} 2, 14-15), y por otra están los judíos, en particular los fariseos. El contraste es fuerte, hasta el punto de que algunos de los presentes preguntan a Jesús: \textquote{¿Qué signos nos muestras para obrar así?} (\textit{Jn} 2, 18).

\textquote{Destruid este templo y en tres días lo levantaré} (\textit{Jn} 2, 19), responde Cristo. La gente replica: \textquote{Cuarenta y seis años ha costado construir este templo, ¿y tú lo vas a levantar en tres días?} (\textit{Jn} 2, 20). No habían comprendido –anota san Juan– que el Señor estaba hablando del templo vivo de su cuerpo que, durante los acontecimientos pascuales, sería destruido con la muerte en la cruz, pero que resucitaría al tercer día. \textquote{Y cuando resucitó de entre los muertos –escribe el evangelista–, los discípulos se acordaron de que lo había dicho, y dieron fe a la Escritura y a la palabra que había dicho Jesús} (\textit{Jn} 2, 22).

El acontecimiento pascual da significado auténtico a todos los elementos presentes en las lecturas de hoy. En la Pascua se revela plenamente el poder del Verbo encarnado, poder del Hijo eterno de Dios, que se hizo hombre por nosotros y por nuestra salvación.

\textquote{Señor, tú tienes palabras de vida eterna}. Creemos que tú eres verdaderamente el Hijo de Dios. Y te damos gracias por habernos hecho partícipes de tu misma vida divina. Amén.
\end{body}

\subsubsection{Homilía (2000): La Resurrección, signo de fidelidad}

\src{Misa en la Basílica del Santo Sepulcro de Jerusalén, nn. 1. 3-5. \\26 de marzo del 2000.}

\begin{body}
1. \txtsmall{[Siguiendo el camino de la historia de la salvación, tal como se narra en el Símbolo de los Apóstoles, mi peregrinación jubilar me ha traído a Tierra Santa. De Nazaret, donde Jesús fue concebido en el seno de la Virgen María por obra del Espíritu Santo, he llegado a Jerusalén, donde \textquote{padeció bajo el poder de Poncio Pilato, fue crucificado, muerto y sepultado}. Aquí, en la basílica del Santo Sepulcro, me arrodillo ante el lugar de su sepultura: \textquote{He aquí el lugar donde lo pusieron} (\textit{Mc} 16, 6).]}

\ltr{L}{a} tumba está vacía. Es un testigo silencioso del acontecimiento central de la historia humana: la resurrección de nuestro Señor Jesucristo. Durante casi dos mil años la tumba vacía ha dado testimonio de la victoria de la Vida sobre la muerte. Con los Apóstoles y los evangelistas, con la Iglesia de todos los tiempos y lugares, también nosotros damos testimonio y proclamamos: \textquote{¡Cristo resucitó! Una vez resucitado de entre los muertos, ya no muere más; la muerte no tiene ya señorío sobre él} (cf. \textit{Rm} 6, 9). \textquote{Mors et vita duello conflixere mirando; dux vitae mortuus, regnat vivus} (\textit{Secuencia pascual latina Victimae paschali}). El Señor de la vida estaba muerto; ahora reina, victorioso sobre la muerte, fuente de vida eterna para todos los creyentes.

3. \textquote{Destruid este templo y en tres días lo levantaré} (\textit{Jn} 2, 19). El evangelista \textbf{san Juan} nos narra que, después de la resurrección de Jesús de entre los muertos, los discípulos recordaron estas palabras y creyeron (cf. \textit{Jn} 2, 22). Jesús las pronunció a fin de que fueran un signo para sus discípulos. Cuando fue al templo con sus discípulos, expulsó a los cambistas y a los vendedores del lugar santo (cf. \textit{Jn} 2, 15). En el momento en que los presentes protestaron, preguntándole: \textquote{¿Qué señal nos muestras para obrar así?}, Jesús les replicó: \textquote{Destruid este templo y en tres días lo levantaré}. El evangelista anota que \textquote{él hablaba del templo de su cuerpo} (\textit{Jn} 2, 18-21).

La profecía encerrada en las palabras de Jesús se cumplió en la Pascua, cuando \textquote{al tercer día resucitó de entre los muertos}. La resurrección de nuestro Señor Jesucristo es el signo de que el Padre eterno es fiel a su promesa y hace nacer nueva vida de la muerte: \textquote{la resurrección del cuerpo y la vida eterna}. \txtsmall{[El misterio se refleja claramente en esta antigua iglesia de la \textit{Anástasis}, que contiene tanto el sepulcro vacío, signo de la Resurrección, como el Gólgota, lugar de la crucifixión.]} 

La buena nueva de la Resurrección no puede separarse nunca del misterio de la cruz. \textbf{San Pablo} nos lo dice en la \textbf{segunda lectura} de hoy: \textquote{Nosotros predicamos a Cristo crucificado} (\textit{1 Co} 1, 23). Cristo, que se ofreció a sí mismo como sacrificio vespertino en el altar de la cruz (cf. \textit{Sal} 141, 2), se revela ahora como \textquote{fuerza de Dios y sabiduría de Dios} (\textit{1 Co} 1, 24). Y en su resurrección, los hijos y las hijas de Adán han sido hechos partícipes de su vida divina, que tenía desde toda la eternidad, con el Padre, en el Espíritu Santo.

4. [\ldots] Mediante el Decálogo y la ley moral inscrita en el corazón del hombre (cf. \textit{Rm} 2, 15), Dios desafía radicalmente la libertad de cada hombre y cada mujer. Responder a la voz de Dios que resuena en lo más profundo de nuestra conciencia y elegir el bien es la opción más sublime de la libertad humana. Equivale, realmente, a elegir entre la vida y la muerte (cf. \textit{Dt} 30, 15). Caminando por la senda de la Alianza con Dios santísimo, el pueblo se convierte en heraldo y testigo de la promesa, la promesa de una auténtica liberación y de la plenitud de vida.

La resurrección de Jesús es el sello definitivo de todas las promesas de Dios, el lugar de nacimiento de una humanidad nueva y resucitada, la prenda de una historia caracterizada por los dones mesiánicos de paz y alegría espiritual. En el alba de un nuevo milenio, los cristianos pueden y deben mirar al futuro con firme confianza en el poder glorioso del Resucitado de renovar todas las cosas (cf. \textit{Ap} 21, 5). Él es el único que libra a toda la creación de la servidumbre de la corrupción (cf. \textit{Rm} 8, 20). Con su resurrección, abre el camino al gran descanso del \textit{sabbath}, el octavo día, cuando la peregrinación de la humanidad llegue a su fin y Dios sea todo en todos (cf. \textit{1 Co} 15, 28).

Aquí, [en el Santo Sepulcro y en el Gólgota,] a la vez que renovamos nuestra profesión de fe en el Señor resucitado, ¿podemos dudar de que con el poder del Espíritu de vida recibiremos la fuerza para superar nuestras divisiones y trabajar juntos a fin de construir un futuro de reconciliación, unidad y paz? Aquí, como en ningún otro lugar de la tierra, oímos una vez más al Señor que dice a sus discípulos: \textquote{¡Ánimo!: yo he vencido al mundo} (\textit{Jn} 16, 33).

5. \textquote{Mors et vita duello conflixere mirando; dux vitae mortuus, regnat vivus}. El Señor resucitado, resplandeciente por la gloria del Espíritu, es la Cabeza de la Iglesia, su Cuerpo místico. Él la sostiene en su misión de proclamar el Evangelio de la salvación a los hombres y mujeres de cada generación, hasta que vuelva en la gloria. \txtsmall{[En este lugar, donde se dio a conocer la Resurrección primero a las mujeres y luego a los Apóstoles,]} invito a todos los miembros de la Iglesia a renovar su obediencia al mandato del Señor de anunciar el Evangelio hasta los confines de la tierra. \txtsmall{[En el alba de un nuevo milenio]} es muy necesario proclamar desde los tejados la buena nueva de que \textquote{tanto amó Dios al mundo, que dio a su Hijo único, para que todo el que crea en él no perezca, sino que tenga vida eterna} (\textit{Jn} 3, 16). \textquote{Señor, (\ldots) tú tienes palabras de vida eterna} (\textit{Jn} 6, 68). \txtsmall{[Hoy, como indigno Sucesor de Pedro, deseo repetir estas palabras mientras celebramos el sacrificio eucarístico en este lugar, el más santo de la tierra.]} Con toda la humanidad redimida, hago mías las palabras que Pedro, el pescador, dirigió a Cristo, Hijo del Dios vivo: \textquote{Señor, ¿a quién iremos? Tú tienes palabras de vida eterna}.

\textit{Christós anésti}! ¡Jesucristo ha resucitado! ¡En verdad, ha resucitado! Amén.
\end{body}

\newsection
\subsection{Benedicto XVI, papa}

\subsubsection{Homilía (2006): Crucificado y Resucitado}

\src{Celebración Eucarística con los Trabajadores en la Fiesta de San José. \\Domingo 19 de marzo del 2006.}

\begin{body}
\ltr{H}{emos} escuchado juntos una famosa y bella página del \textbf{libro del Éxodo}, en la que el autor sagrado narra la entrega del Decálogo a Israel por parte de Dios. Un detalle llama enseguida la atención: la enumeración de los diez mandamientos se introduce con una significativa referencia a la liberación del pueblo de Israel. Dice el texto: \textquote{Yo soy el Señor, tu Dios, que te saqué de Egipto, de la esclavitud} (\textit{Ex} 20, 2). Por tanto, el Decálogo quiere ser una confirmación de la libertad conquistada. En efecto, los mandamientos, si se analizan en profundidad, son el instrumento que el Señor nos da para defender nuestra libertad tanto de los condicionamientos internos de las pasiones como de los abusos externos de los maliciosos. Los \textquote{no} de los mandamientos son otros tantos \textquote{sí} al crecimiento de una libertad auténtica. Conviene subrayar también una segunda dimensión del Decálogo: con la Ley dada por medio de Moisés el Señor revela que quiere establecer con Israel una alianza. Por consiguiente, la Ley, más que una imposición, es un don. Más que mandar lo que el hombre debe hacer, quiere manifestar a todos la elección de Dios: él está de parte del pueblo elegido; lo liberó de la esclavitud y lo rodeó con su bondad misericordiosa. El Decálogo es testimonio de un amor de predilección.

La liturgia de hoy nos ofrece un segundo mensaje: la Ley mosaica se cumplió plenamente en Jesús, que reveló la sabiduría y el amor de Dios mediante el misterio de la cruz, \textquote{escándalo para los judíos, necedad para los griegos –como nos dice san Pablo en la \textbf{segunda lectura}–; pero para los llamados (\ldots), judíos o griegos, fuerza de Dios y sabiduría de Dios} (\textit{1 Co} 1, 23-24). Precisamente a este misterio se refiere la \textbf{página evangélica} que se acaba de proclamar: Jesús expulsa del templo a los vendedores y a los cambistas. El evangelista ofrece la clave de lectura de este significativo episodio en el versículo de un salmo: \textquote{El celo por tu casa me devora} (cf. \textit{Sal} 69, 10). A Jesús lo \textquote{devora} este \textquote{celo} por la \textquote{casa de Dios}, utilizada con un fin diferente de aquel para el que estaba destinada. Ante la petición de los responsables religiosos, que pretenden un signo de su autoridad, en medio del asombro de los presentes, afirma: \textquote{Destruid este templo, y en tres días lo levantaré} (\textit{Jn} 2, 19). Palabras misteriosas, incomprensibles en aquel momento, pero que san Juan vuelve a formular para sus lectores cristianos, observando: \textquote{Él hablaba del templo de su cuerpo} (\textit{Jn} 2, 21).

Sus adversarios destruirán este \textquote{templo}, pero él, al cabo de tres días, lo reconstruirá mediante la resurrección. La muerte dolorosa y \textquote{escandalosa} de Cristo se coronará con el triunfo de su gloriosa resurrección. Mientras en este tiempo cuaresmal nos preparamos para revivir en el triduo pascual este acontecimiento central de nuestra salvación, contemplamos al Crucificado vislumbrando ya en él el resplandor del Resucitado.

\txtsmall{[Queridos hermanos y hermanas, esta celebración eucarística, que a la meditación de los textos litúrgicos del tercer domingo de Cuaresma une el recuerdo de san José, nos ofrece la oportunidad de considerar, a la luz del misterio pascual, otro aspecto importante de la existencia humana. Me refiero a la realidad del trabajo, que hoy está en el centro de cambios rápidos y complejos. En numerosas páginas la Biblia muestra cómo el trabajo pertenece a la condición originaria del hombre. Cuando el Creador plasmó al hombre a su imagen y semejanza, lo invitó a trabajar la tierra (cf. \textit{Gn} 2, 5-6). A causa del pecado de nuestros primeros padres, el trabajo se transformó en fatiga y sudor (cf. \textit{Gn} 3, 6-8), pero el proyecto divino mantiene inalterado su valor. El mismo Hijo de Dios, haciéndose semejante en todo a nosotros, se dedicó durante muchos años a actividades manuales, hasta el punto de que lo conocían como el \textquote{hijo del carpintero} (cf. \textit{Mt} 13, 55). La Iglesia ha mostrado siempre, especialmente durante el último siglo, interés y solicitud por este ámbito de la sociedad, como testimonian las numerosas intervenciones sociales del Magisterio y la acción de múltiples asociaciones de inspiración cristiana, algunas de las cuales han venido hoy aquí a representar a todo el mundo de los trabajadores. Me alegra acogeros, queridos amigos, y os dirijo a cada uno mi cordial saludo (\ldots)}

\txtsmall{El trabajo reviste una importancia primaria para la realización del hombre y el desarrollo de la sociedad, y por eso es preciso que se organice y desarrolle siempre en el pleno respeto de la dignidad humana y al servicio del bien común. Al mismo tiempo, es indispensable que el hombre no se deje dominar por el trabajo, que no lo idolatre, pretendiendo encontrar en él el sentido último y definitivo de la vida. Al respecto, es oportuna la invitación de la primera lectura: \textquote{Fíjate en el sábado para santificarlo. Durante seis días trabaja y haz tus tareas, pero el día séptimo es un día de descanso dedicado al Señor, tu Dios} (\textit{Ex} 20, 8-9). El sábado es día santificado, es decir, consagrado a Dios, en el que el hombre comprende mejor el sentido de su existencia y también de la actividad laboral. Por tanto, se puede afirmar que la enseñanza bíblica sobre el trabajo culmina en el mandamiento del descanso. Al respecto, el \textit{Compendio de la doctrina social de la Iglesia} observa oportunamente: \textquote{El descanso abre al hombre, sujeto a la necesidad del trabajo, la perspectiva de una libertad más plena, la del sábado eterno (cf. \textit{Hb} 4, 9-10). El descanso permite a los hombres recordar y revivir las obras de Dios, desde la creación hasta la Redención, reconocerse a sí mismos como obra suya (cf. \textit{Ef} 2, 10), y dar gracias por su vida y su subsistencia a él, que de ellas es el Autor} (n. 258).}

\txtsmall{La actividad laboral debe contribuir al verdadero bien de la humanidad, permitiendo \textquote{al hombre individual y socialmente cultivar y realizar plenamente su vocación} (\textit{Gaudium et spes}, 35). Para que esto suceda no basta la preparación técnica y profesional, por lo demás necesaria; ni siquiera es suficiente la creación de un orden social justo y atento al bien de todos. Es preciso vivir una espiritualidad que ayude a los creyentes a santificarse a través de su trabajo, imitando a san José, que cada día debió proveer con sus manos a las necesidades de la Sagrada Familia, y por eso la Iglesia lo propone como patrono de los trabajadores. Su testimonio muestra que el hombre es sujeto y protagonista del trabajo. Quisiera encomendarle a él a los jóvenes que con esfuerzo logran insertarse en el mundo del trabajo, a los desempleados y a todos los que sufren las dificultades debidas a la crisis laboral generalizada. Que junto con María, su esposa, san José vele sobre todos los trabajadores y obtenga serenidad y paz para las familias y para toda la humanidad. Que al contemplar a este gran santo, los cristianos aprendan a testimoniar en todos los ámbitos laborales el amor de Cristo, manantial de solidaridad verdadera y de paz estable. Amén.]}
\end{body}
\label{b2-03-03-2009}

\begin{patercite}
El Decálogo nos remite al monte Sinaí, cuando Dios entra de modo particular en la historia del pueblo judío y, a través de este pueblo, en la historia de toda la humanidad, dando las \textquote{Diez Palabras} (\ldots)  Pero ¿qué sentido tienen para nosotros estas Diez Palabras en el actual contexto cultural, en el que se corre el riesgo de que el laicismo y el relativismo se conviertan en los criterios de toda decisión, y en esta sociedad que parece vivir como si Dios no existiese? Nosotros respondemos que Dios nos ha dado los Mandamiento para educarnos en la verdadera libertad y en el amor auténtico, de modo que podamos ser realmente felices. Son un signo del amor de Dios Padre, de su deseo de enseñarnos a distinguir correctamente el bien del mal, lo verdadero de lo falso, lo justo de lo injusto. Todos los pueden comprender y, precisamente porque fijan los valores fundamentales en normas y reglas concretas, al ponerlos en práctica el hombre puede recorrer el camino de la verdadera libertad, que lo consolida en el camino que lleva a la vida y a la felicidad. Al contrario, cuando en su existencia el hombre ignora los Mandamientos, no sólo se aliena de Dios y abandona la alianza con él, sino que también se aleja de la vida y de la felicidad duradera. El hombre abandonado a sí mismo, indiferente hacia Dios, orgulloso de su propia autonomía absoluta, acaba por seguir los ídolos del egoísmo, del poder, del dominio, contaminando las relaciones consigo mismo y con los demás, y recorriendo sendas no de vida, sino de muerte. Las tristes experiencias de la historia, sobre todo del siglo pasado, siguen siendo una advertencia para toda la humanidad.

\textbf{Benedicto XVI, papa}, \textit{Videomensaje},  8 de septiembre del 2012, cf. parr. 2-3.
\end{patercite}

\newpage
\subsubsection{Ángelus (2012): Nuevo culto y nuevo Templo}

\src{11 de marzo del 2012.}

\begin{body}
\ltr{E}{l} \textbf{Evangelio} de este tercer domingo de Cuaresma refiere, en la redacción de \textbf{san Juan}, el célebre episodio en el que Jesús expulsa del templo de Jerusalén a los vendedores de animales y a los cambistas (cf. \textit{Jn} 2, 13-25). El hecho, recogido por todos los evangelistas, tuvo lugar en la proximidad de la fiesta de la Pascua y suscitó gran impresión tanto entre la multitud como entre sus discípulos. ¿Cómo debemos interpretar este gesto de Jesús? En primer lugar, hay que señalar que no provocó ninguna represión de los guardianes del orden público, porque lo vieron como una típica acción profética: de hecho, los profetas, en nombre de Dios, con frecuencia denunciaban los abusos, y a veces lo hacían con gestos simbólicos. El problema, en todo caso, era su autoridad. Por eso los judíos le preguntaron a Jesús: \textquote{¿Qué signos nos muestras para obrar así?} (\textit{Jn} 2, 18); demuéstranos que actúas verdaderamente en nombre de Dios.

La expulsión de los mercaderes del templo también se ha interpretado en sentido político revolucionario, colocando a Jesús en la línea del movimiento de los zelotes. Estos, de hecho, eran \textquote{celosos} de la ley de Dios y estaban dispuestos a usar la violencia para hacer que se cumpliera. En tiempos de Jesús esperaban a un mesías que liberase a Israel del dominio de los romanos. Pero Jesús decepcionó estas expectativas, por lo que algunos discípulos lo abandonaron, y Judas Iscariote incluso lo traicionó. En realidad, es imposible interpretar a Jesús como violento: la violencia es contraria al reino de Dios, es un instrumento del anticristo. La violencia nunca sirve a la humanidad, más aún, la deshumaniza.

Escuchemos entonces las palabras que Jesús dijo al realizar ese gesto: \textquote{Quitad esto de aquí: no convirtáis en un mercado la casa de mi Padre} (\textit{Jn} 2, 16). Sus discípulos se acordaron entonces de lo que está escrito en un Salmo: \textquote{El celo de tu casa me devora} (\textit{Sal} 69, 10). Este Salmo es una invocación de ayuda en una situación de extremo peligro a causa del odio de los enemigos: la situación que Jesús vivirá en su pasión. El celo por el Padre y por su casa lo llevará hasta la cruz: el suyo es el celo del amor que paga en carne propia, no el que querría servir a Dios mediante la violencia. De hecho, el \textquote{signo} que Jesús dará como prueba de su autoridad será precisamente su muerte y resurrección. \textquote{Destruid este templo –dijo–, y en tres días lo levantaré}. Y san Juan observa: \textquote{Él hablaba del templo de su cuerpo} (\textit{Jn} 2, 19. 21). Con la Pascua de Jesús se inicia un nuevo culto, el culto del amor, y un nuevo templo que es él mismo, Cristo resucitado, por el cual cada creyente puede adorar a Dios Padre \textquote{en espíritu y verdad} (\textit{Jn} 4, 23). Queridos amigos, el Espíritu Santo comenzó a construir este nuevo templo en el seno de la Virgen María. Por su intercesión, pidamos que cada cristiano sea piedra viva de este edificio espiritual.
\end{body}

\newsection
\subsection{Francisco, papa}

\subsubsection{Homilía (2015): Invitación a un culto auténtico}

\src{Visita a la Parroquia Romana de Todos los Santos. \\Sábado 7 de marzo del 2015.}

\begin{body}
\ltr{C}{on} ocasión de la fiesta de la Pascua judía, Jesús va a Jerusalén. Al llegar al templo, no encuentra gente que busca a Dios, sino gente que hace sus propios negocios: los mercaderes de animales para la ofrenda de los sacrificios; los cambistas, quienes cambian dinero \textquote{impuro} que lleva la imagen del emperador con monedas aprobadas por la autoridad religiosa para pagar el impuesto anual del templo. ¿Qué encontramos nosotros cuando visitamos, cuando vamos a nuestros templos? Dejo la pregunta. El indigno comercio, fuente de ricas ganancias, provoca la enérgica reacción de Jesús. Él volcó los bancos y esparció el dinero por el piso, echó a los vendedores diciéndoles: \textquote{No convirtáis en un mercado la casa de mi Padre} (\textit{Jn} 2, 16).

Esta expresión no se refiere sólo a los negocios que se realizaban en los patios del templo. Se refiere más bien a un tipo de religiosidad. El gesto de Jesús es un gesto de \textquote{limpieza}, de purificación, y la actitud que Él desautoriza se la puede sacar de los textos proféticos, según los cuales Dios no soporta un culto exterior hecho de sacrificios materiales y basado en el interés personal (cf. \textit{Is} 1, 11-17; \textit{Jer} 7, 2-11). Este gesto es la llamada al culto auténtico, a la correspondencia entre liturgia y vida; una llamada válida para todos los tiempos y también hoy para nosotros. Esa correspondencia entre liturgia y vida. La liturgia no es algo extraño, allá, lejano, y mientras se celebra yo pienso en muchas cosas, o rezo el Rosario. No, no. Hay una correspondencia con la celebración litúrgica que luego llevo a mi vida; y en esto se debe aún ir más adelante, se debe aún recorrer mucho camino.

La constitución conciliar \textit{Sacrosanctum Concilium} define la liturgia como \textquote{la primera y más necesaria fuente en la que los fieles beben el espíritu verdaderamente cristiano} (n. 14). Esto significa reafirmar el vínculo esencial que une la vida del discípulo de Jesús y el culto litúrgico. Esto no es ante todo una doctrina que se debe comprender, o un rito que hay que cumplir; es naturalmente también esto pero de otra forma, es esencialmente distinto: es una fuente de vida y de luz para nuestro camino de fe.

Por lo tanto, la Iglesia nos llama a tener y promover una vida litúrgica auténtica, a fin de que pueda haber sintonía entre lo que la liturgia celebra y lo que nosotros vivimos en nuestra existencia. Se trata de expresar en la vida lo que hemos recibido mediante la fe y lo que hemos celebrado (cf. \textit{Sacrosanctum Concilium}, 10).

El discípulo de Jesús no va a la iglesia sólo para cumplir un precepto, para sentirse bien con un Dios que luego no tiene que \textquote{molestar} demasiado. \textquote{Pero yo, Señor, voy todos los domingos, cumplo\ldots, tú no te metas en mi vida, no me molestes}. Esta es la actitud de muchos católicos, muchos. El discípulo de Jesús va a la iglesia para encontrarse con el Señor y encontrar en su gracia, operante en los sacramentos, la fuerza para pensar y obrar según el Evangelio. Por lo que no podemos ilusionarnos con entrar en la casa del Señor y \textquote{encubrir}, con oraciones y prácticas de devoción, comportamientos contrarios a las exigencias de la justicia, la honradez o la caridad hacia el prójimo. No podemos sustituir con \textquote{honores religiosos} lo que debemos dar al prójimo, postergando una verdadera conversión. El culto, las celebraciones litúrgicas, son el ámbito privilegiado para escuchar la voz del Señor, que guía por el camino de la rectitud y de la perfección cristiana.

Se trata de realizar un itinerario de conversión y de penitencia, para quitar de nuestra vida las escorias del pecado, como hizo Jesús, limpiando el templo de intereses mezquinos. Y la Cuaresma es el tiempo favorable para todo esto, es el tiempo de la renovación interior, de la remisión de los pecados, el tiempo en el que somos llamados a redescubrir el sacramento de la Penitencia y de la Reconciliación, que nos hace pasar de las tinieblas del pecado a la luz de la gracia y de la amistad con Jesús. No hay que olvidar la gran fuerza que tiene este sacramento para la vida cristiana: nos hace crecer en la unión con Dios, nos hace reconquistar la alegría perdida y experimentar el consuelo de sentirnos personalmente acogidos por el abrazo misericordioso de Dios.

\txtsmall{[(\ldots) este templo fue construido gracias al celo apostólico de san Luis Orione. Precisamente aquí, hace cincuenta años, el beato Pablo VI inauguró, en cierto sentido, la reforma litúrgica con la celebración de la misa en la lengua hablada por la gente.]} Os deseo que esta circunstancia reavive en todos vosotros el amor por la casa de Dios. En ella encontráis una gran ayuda espiritual. Aquí podéis experimentar, cada vez que queráis, el poder regenerador de la oración personal y de la oración comunitaria. La escucha de la Palabra de Dios, proclamada en la asamblea litúrgica, os sostiene en el camino de vuestra vida cristiana. Os encontráis entre estos muros no como extraños, sino como hermanos, capaces de darse la mano con gusto, porque os congrega el amor a Cristo, fundamento de la esperanza y del compromiso de cada creyente.

A Él, Jesucristo, Piedra angular, nos estrechamos confiados en esta santa misa, renovando el propósito de comprometernos en favor de la purificación y la limpieza interior de la Iglesia edificio espiritual, del cual cada uno de nosotros es parte viva en virtud del Bautismo. Así sea.
\end{body}


\subsubsection{Homilía (2015): No podemos engañar a Jesús}

\src{Visita Pastoral a la Parroquia Romana de \\Santa María Madre del Redentor en Tor Bella Monaca. \\8 de marzo del 2015.}

\begin{body}
\ltr{E}{n} este \textbf{pasaje del Evangelio} que hemos escuchado, hay dos cosas que me impresionan: una imagen y una palabra. La imagen es la de Jesús con el látigo en la mano que echa fuera a todos los que aprovechaban el Templo para hacer negocios. Estos comerciantes que vendían los animales para los sacrificios, cambiaban las monedas\ldots Estaba lo sagrado –el templo, sagrado– y esto sucio, afuera. Esta es la imagen. Y Jesús toma el látigo y procede, para limpiar un poco el Templo. Y la frase, la palabra, está ahí donde se dice que mucha gente creía en Él, una frase terrible: \textquote{Pero Jesús no se confiaba a ellos, porque los conocía a todos, y no necesitaba el testimonio de nadie sobre un hombre, porque Él sabía lo que hay dentro de cada hombre} (\textit{Jn} 2, 24-25).

Nosotros no podemos engañar a Jesús: Él nos conoce por dentro. No se fiaba. Él, Jesús, no se fiaba. Y esta puede ser una buena pregunta en la mitad de la Cuaresma: ¿Puede fiarse Jesús de mí? ¿Puede fiarse Jesús de mí, o tengo una doble cara? ¿Me presento como católico, como uno cercano a la Iglesia, y luego vivo como un pagano? \textquote{Pero Jesús no lo sabe, nadie va a contárselo}. Él lo sabe. \textquote{Él no tenía necesidad de que alguien diese testimonio; Él, en efecto, conocía lo que había en el hombre}. Jesús conoce todo lo que está dentro de nuestro corazón: no podemos engañar a Jesús. No podemos, ante Él, aparentar ser santos, y cerrar los ojos, actuar así, y luego llevar una vida que no es la que Él quiere. Y Él lo sabe. Y todos sabemos el nombre que Jesús daba a estos con doble cara: hipócritas.

Nos hará bien, hoy, entrar en nuestro corazón y mirar a Jesús. Decirle: \textquote{Señor, mira, hay cosas buenas, pero también hay cosas no buenas. Jesús, ¿te fías de mí? Soy pecador\ldots}. Esto no asusta a Jesús. Si tú le dices: \textquote{Soy un pecador}, no se asusta. Lo que a Él lo aleja es la doble cara: mostrarse justo para cubrir el pecado oculto. \textquote{Pero yo voy a la iglesia, todos los domingos, y yo\ldots}. Sí, podemos decir todo esto. Pero si tu corazón no es justo, si tú no vives la justicia, si tú no amas a los que necesitan amor, si tú no vives según el espíritu de las bienaventuranzas, no eres católico. Eres hipócrita. Primero: ¿Puede Jesús fiarse de mí? En la oración, preguntémosle: \textquote{Señor, ¿Tú te fías de mí?}.

Segundo, el gesto. Cuando entramos en nuestro corazón, encontramos cosas que no funcionan, que no están bien, como Jesús encontró en el Templo esa suciedad del comercio, de los vendedores. También dentro de nosotros hay suciedad, hay pecados de egoísmo, de soberbia, de orgullo, de codicia, de envidia, de celos\ldots ¡tantos pecados! Podemos incluso continuar el diálogo con Jesús: \textquote{Jesús, ¿Tú te fías de mí? Yo quiero que Tú te fíes de mí. Entonces te abro la puerta y tú limpia mi alma}. Y pedir al Señor que así como limpió el Templo, venga a limpiar el alma. E imaginamos que Él viene con un látigo de cuerdas\ldots No, con eso no limpia el alma. ¿Vosotros sabéis cuál es el látigo de Jesús para limpiar nuestra alma? La misericordia. Abrid el corazón a la misericordia de Jesús. Decid: \textquote{Jesús, mira cuánta suciedad. Ven, limpia. Limpia con tu misericordia, con tus palabras dulces; limpia con tus caricias}. Y si abrimos nuestro corazón a la misericordia de Jesús, para que limpie nuestro corazón, nuestra alma, Jesús se fiará de nosotros.
\end{body}

\begin{patercite}
El antiguo Templo estaba edificado por las manos de los hombres: se quería \textquote{dar una casa} a Dios para tener un signo visible de su presencia en medio del pueblo. Con la Encarnación del Hijo de Dios, se cumple la profecía de Natán al rey David (cf. \emph{2 Sam} 7, 1-29): no es el rey, no somos nosotros quienes \textquote{damos una casa a Dios}, sino que es Dios mismo quien \textquote{construye su casa} para venir a habitar entre nosotros, como escribe san Juan en su Evangelio (cf. 1, 14). Cristo es el Templo viviente del Padre, y Cristo mismo edifica su \textquote{casa espiritual}, la Iglesia, hecha no de piedras materiales, sino de \textquote{piedras vivientes}, que somos nosotros. El Apóstol Pablo dice a los cristianos de Éfeso: \textquote{Estáis edificados sobre el cimiento de los apóstoles y profetas, y el mismo Cristo Jesús es la piedra angular. Por Él todo el edificio queda ensamblado, y se va levantado hasta formar un templo consagrado al Señor. Por Él también vosotros entráis con ellos en la construcción, para ser morada de Dios, por el Espíritu} (\emph{Ef} 2, 20-22). ¡Esto es algo bello! Nosotros somos las piedras vivas del edificio de Dios, unidas profundamente a Cristo, que es la piedra de sustentación, y también de sustentación entre nosotros. ¿Qué quiere decir esto? Quiere decir que el templo somos nosotros, nosotros somos la Iglesia viviente, el templo viviente, y cuando estamos juntos entre nosotros está también el Espíritu Santo, que nos ayuda a crecer como Iglesia. Nosotros no estamos aislados, sino que somos pueblo de Dios: ¡ésta es la Iglesia! (\ldots) La Iglesia no es un entramado de cosas y de intereses, sino que es el Templo del Espíritu Santo, el Templo en el que Dios actúa, el Templo en el que cada uno de nosotros, con el don del Bautismo, es piedra viva. Esto nos dice que nadie es inútil en la Iglesia, y si alguien dice a veces a otro: \textquote{Vete a casa, eres inútil}, esto no es verdad, porque nadie es inútil en la Iglesia, ¡todos somos necesarios para construir este Templo! Nadie es secundario. Nadie es el más importante en la Iglesia; todos somos iguales a los ojos de Dios. Alguno de vosotros podría decir: \textquote{Oiga, señor Papa, usted no es igual a nosotros}. Sí: soy como uno de vosotros, todos somos iguales, ¡somos hermanos! Nadie es anónimo: todos formamos y construimos la Iglesia. Esto nos invita también a reflexionar sobre el hecho de que si falta la piedra de nuestra vida cristiana, falta algo a la belleza de la Iglesia. (\ldots) todos debemos llevar a la Iglesia nuestra vida, nuestro corazón, nuestro amor, nuestro pensamiento, nuestro trabajo.

\textbf{Francisco, papa}, \textit{Catequesis}, 26 de junio de 2013, parr. 4-5.
\end{patercite}



\label{b2-03-03-2015}

\newpage
\subsubsection{Ángelus (2015): El templo en el que Dios se da a conocer}

\src{8 de marzo del 2015.}

\begin{body}
\ltr{E}{l} \textbf{Evangelio} de hoy (\textit{Jn} 2, 13-25) nos presenta el episodio de la expulsión de los vendedores del templo. Jesús \textquote{hizo un látigo con cuerdas, los echó a todos del Templo, con ovejas y bueyes} (\textit{Jn} 2, 15), el dinero, todo. Tal gesto suscitó una fuerte impresión en la gente y en los discípulos. Aparece claramente como un gesto profético, tanto que algunos de los presentes le preguntaron a Jesús: \textquote{¿Qué signos nos muestras para obrar así?} (\textit{Jn} 2, 18), ¿quién eres para hacer estas cosas? Muéstranos una señal de que tienes realmente autoridad para hacerlas. Buscaban una señal divina, prodigiosa, que acreditara a Jesús como enviado de Dios. Y Él les respondió: \textquote{Destruid este templo y en tres días lo levantaré} (\textit{Jn} 2, 19). Le replicaron: \textquote{Cuarenta y seis años se ha costado construir este templo, ¿y tú lo vas a levantar en tres días?} (\textit{Jn} 2, 20). No habían comprendido que el Señor se refería al templo vivo de su cuerpo, que sería destruido con la muerte en la cruz, pero que resucitaría al tercer día. Por eso, \textquote{en tres días}. \textquote{Cuando resucitó de entre los muertos –comenta el evangelista–, los discípulos se acordaron de que lo había dicho, y creyeron a la Escritura y a la palabra que había dicho Jesús} (\textit{Jn} 2, 22).

En efecto, este gesto de Jesús y su mensaje profético se comprenden plenamente a la luz de su Pascua. Según el evangelista Juan, este es el primer anuncio de la muerte y resurrección de Cristo: su cuerpo, destruido en la cruz por la violencia del pecado, se convertirá con la Resurrección en lugar de la cita universal entre Dios y los hombres. Cristo resucitado es precisamente el lugar de la cita universal –de todos– entre Dios y los hombres. Por eso su humanidad es el verdadero templo en el que Dios se revela, habla, se lo puede encontrar; y los verdaderos adoradores de Dios no son los custodios del templo material, los detentadores del poder o del saber religioso, sino los que adoran a Dios \textquote{en espíritu y verdad} (\textit{Jn} 4, 23).

En este tiempo de Cuaresma nos estamos preparando para la celebración de la Pascua, en la que renovaremos las promesas de nuestro bautismo. Caminemos en el mundo como Jesús y hagamos de toda nuestra existencia un signo de su amor para nuestros hermanos, especialmente para los más débiles y los más pobres, construyamos para Dios un templo en nuestra vida. Y así lo hacemos \textquote{encontrable} para muchas personas que encontramos en nuestro camino. Si somos testigos de este Cristo vivo, mucha gente encontrará a Jesús en nosotros, en nuestro testimonio. Pero –nos preguntamos, y cada uno de nosotros puede preguntarse–, ¿se siente el Señor verdaderamente como en su casa en mi vida? ¿Le permitimos que haga \textquote{limpieza} en nuestro corazón y expulse a los ídolos, es decir, las actitudes de codicia, celos, mundanidad, envidia, odio, la costumbre de murmurar y \textquote{despellejar} a los demás? ¿Le permito que haga limpieza de todos los comportamientos contra Dios, contra el prójimo y contra nosotros mismos, como hemos escuchado hoy en la \textbf{primera lectura}? Cada uno puede responder a sí mismo, en silencio, en su corazón. \textquote{¿Permito que Jesús haga un poco de limpieza en mi corazón?}. \textquote{Oh padre, tengo miedo de que me reprenda}. Pero Jesús no reprende jamás. Jesús hará limpieza con ternura, con misericordia, con amor. La misericordia es su modo de hacer limpieza. Dejemos –cada uno de nosotros–, dejemos que el Señor entre con su misericordia –no con el látigo, no, sino con su misericordia– para hacer limpieza en nuestros corazones. El látigo de Jesús para nosotros es su misericordia. Abrámosle la puerta, para que haga un poco de limpieza.

Cada Eucaristía que celebramos con fe nos hace crecer como templo vivo del Señor, gracias a la comunión con su Cuerpo crucificado y resucitado. Jesús conoce lo que hay en cada uno de nosotros, y también conoce nuestro deseo más ardiente: el de ser habitados por Él, sólo por Él. Dejémoslo entrar en nuestra vida, en nuestra familia, en nuestro corazón. Que María santísima, morada privilegiada del Hijo de Dios, nos acompañe y nos sostenga en el itinerario cuaresmal, para que redescubramos la belleza del encuentro con Cristo, que nos libera y nos salva.
\end{body}

\begin{patercite}
Cristo Jesús es nuestro sumo sacerdote, y su precioso cuerpo, que inmoló en el ara de la cruz por la salvación de todos los hombres, es nuestro sacrificio. La sangre que se derramó para nuestra redención no fue la de los becerros y los machos cabríos (como en la ley antigua), sino la del inocentísimo Cordero, Cristo Jesús, nuestro salvador. El templo en el que nuestro sumo sacerdote ofrecía el sacrificio no era hecho por manos de hombres, sino que había sido levantado por el solo poder de Dios; pues Cristo derramó su sangre a la vista del mundo: un templo ciertamente edificado por la sola mano de Dios. Y este templo tiene dos partes: una es la tierra, que ahora nosotros habitamos; la otra nos es aún desconocida a nosotros, mortales.

Así, primero, ofreció su sacrificio aquí en la tierra, cuando sufrió la más acerba muerte. Luego, cuando revestido de la nueva vestidura de la inmortalidad entró por su propia sangre en el santuario, o sea, en el cielo, presentó ante el trono del Padre celestial aquella sangre de inmenso valor, que había derramado una vez para siempre en favor de todos los hombres, pecadores.

Este sacrificio resultó tan grato y aceptable a Dios, que así que lo hubo visto, compadecido inmediatamente de nosotros, no pudo menos que otorgar su perdón a todos los verdaderos penitentes.

De este santo y definitivo sacrificio se hacen partícipes todos aquellos que llegaron a tener verdadera contrición y aceptaron la penitencia por sus crímenes, aquellos que con firmeza decidieron no repetir en adelante sus maldades, sino que perseveran con constancia en el inicial propósito de las virtudes. (\ldots)

\textbf{San Juan Fisher}, \textit{Comentario} sobre el Salmo 129, Opera omnia, ed. 1579 p. 1610.
\end{patercite}
\label{b2-03-03-2015A}

\newpage 
\subsubsection{Ángelus (2018): Peligro de instrumentalizar a Dios}

\src{Plaza de San Pedro. \\4 de marzo del 2018.}

\begin{body}
\ltr{E}{l} \textbf{Evangelio} de hoy presenta, en la versión de \textbf{Juan}, el episodio en el que Jesús expulsa a los vendedores del templo de Jerusalén (cf. \textit{Jn} 2, 13-25). Él hizo este gesto ayudándose con un látigo, volcó las mesas y dijo: \textquote{No hagáis de la Casa de mi Padre una casa de mercado} (\textit{Jn} 2, 16). Esta acción decidida, realizada en proximidad de la Pascua, suscitó gran impresión en la multitud y la hostilidad de las autoridades religiosas y de los que se sintieron amenazados en sus intereses económicos. Pero, ¿cómo debemos interpretarla? Ciertamente no era una acción violenta, tanto es verdad que no provocó la intervención de los tutores del orden público: de la policía. ¡No! Sino que fue entendida como una acción típica de los profetas, los cuales a menudo denunciaban, en nombre de Dios, abusos y excesos. La cuestión que se planteaba era la de la autoridad. De hecho los judíos preguntaron a Jesús: \textquote{¿Qué señal nos muestras para obrar así?} (\textit{Jn} 2, 18), es decir ¿qué autoridad tienes para hacer estas cosas? Como pidiendo la demostración de que Él actuaba en nombre de Dios. Para interpretar el gesto de Jesús de purificar la casa de Dios, sus discípulos usaron un texto bíblico tomado del salmo 69: \textquote{El celo por tu casa me devorará} (\textit{Sal} 69, 17); así dice el salmo: \textquote{pues me devora el celo de tu casa}. Este salmo es una invocación de ayuda en una situación de extremo peligro a causa del odio de los enemigos: la situación que Jesús vivirá en su pasión. El celo por el Padre y por su casa lo llevará hasta la cruz: su celo es el del amor que lleva al sacrificio de sí, no el falso que presume de servir a Dios mediante la violencia. De hecho, el \textquote{signo} que Jesús dará como prueba de su autoridad será precisamente su muerte y resurrección: \textquote{Destruid este santuario –dice– y en tres días lo levantaré} (\textit{Jn} 2, 19). Y el evangelista anota: \textquote{Él hablaba del Santuario de su cuerpo} (\textit{Jn} 2, 21). Con la Pascua de Jesús inicia el nuevo culto en el nuevo templo, el culto del amor, y el nuevo templo es Él mismo.

La actitud de Jesús contada en la actual página evangélica, nos exhorta a vivir nuestra vida no en la búsqueda de nuestras ventajas e intereses, sino por la gloria de Dios que es el amor. Somos llamados a tener siempre presentes esas palabras fuertes de Jesús: \textquote{No hagáis de la Casa de mi Padre una casa de mercado} (\textit{Jn} 2, 16). Es muy feo cuando la Iglesia se desliza hacia esta actitud de hacer de la casa de Dios un mercado. Estas palabras nos ayudan a rechazar el peligro de hacer también de nuestra alma, que es la casa de Dios, un lugar de mercado que viva en la continua búsqueda de nuestro interés en vez de en el amor generoso y solidario. Esta enseñanza de Jesús es siempre actual, no solamente para las comunidades eclesiales, sino también para los individuos, para las comunidades civiles y para toda la sociedad. Es común, de hecho, la tentación de aprovechar las buenas actividades, a veces necesarias, para cultivar intereses privados, o incluso ilícitos. Es un peligro grave, especialmente cuando instrumentaliza a Dios mismo y el culto que se le debe a Él, o el servicio al hombre, su imagen. Por eso Jesús esa vez usó \textquote{las maneras fuertes}, para sacudirnos de este peligro mortal. Que la Virgen María nos sostenga en el compromiso de hacer de la Cuaresma una buena ocasión para reconocer a Dios como único Señor de nuestra vida, quitando de nuestro corazón y de nuestras obras todo tipo de idolatría.
\end{body}

\begin{patercite}
Él vino por su benignidad hacia nosotros y se nos hizo visible. Tuvo piedad de nuestra raza y de nuestra debilidad y, compadecido de nuestra corrupción, no soportó que la muerte nos dominase, para que no pereciese lo que había sido creado, con lo que hubiera resultado inútil la obra de su Padre al crear al hombre, y por esto tomó para sí un cuerpo como el nuestro, ya que no se contentó con habitar en un cuerpo ni tampoco en hacerse simplemente visible. (\ldots) 

En el seno de la Virgen, se construyó un templo, es decir, su cuerpo, y lo hizo su propio instrumento, en el que había de darse a conocer y habitar; de este modo, habiendo tomado un cuerpo semejante al de cualquiera de nosotros, ya que todos estaban sujetos a la corrupción de la muerte, lo entregó a la muerte por todos, ofreciéndolo al Padre con un amor sin límites; con ello, al morir en su persona todos los hombres, quedó sin vigor la ley de la corrupción que afectaba a todos, ya que agotó toda la eficacia de la muerte en el cuerpo del Señor; y así ya no le quedó fuerza alguna para ensañarse con los demás hombres, semejantes a él; con ello, también hizo de nuevo incorruptibles a los hombres, que habían caído en la corrupción, y los llamó de muerte a vida, consumiendo totalmente en ellos la muerte, con el cuerpo que había asumido y con el poder de su resurrección, del mismo modo que la paja es consumida por el fuego.

De ahí que el cuerpo que él había tomado, al entregarlo a la muerte como una hostia y víctima limpia de toda mancha, alejó al momento la muerte de todos los hombres, a los que él se había asemejado, ya que se ofreció en lugar de ellos.

De este modo, el Verbo de Dios, superior a todo lo que existe, ofreciendo en sacrificio su cuerpo, templo e instrumento de su divinidad, pagó con su muerte la deuda que habíamos contraído, y, así, el Hijo de Dios, inmune a la corrupción, por la promesa de la resurrección, hizo partícipes de esta misma inmunidad a todos los hombres, con los que se había hecho una misma cosa por su cuerpo semejante al de ellos.

\textbf{San Atanasio, obispo}, \textit{Sermón} sobre la encarnación del Verbo, 8-9: PG 25,110-111 (Liturgia de las Horas, Común de Pastores).
\end{patercite}

\newsection
\section{Temas}

\cceth{Jesús y la Ley} 
\cceref{CEC 459, 577-582}

\begin{ccebody}
\n{459} El Verbo se encarnó \textit{para ser nuestro modelo de santidad}: \textquote{Tomad sobre vosotros mi yugo, y aprended de mí \ldots} (\textit{Mt} 11, 29). \textquote{Yo soy el Camino, la Verdad y la Vida. Nadie va al Padre sino por mí} (\textit{Jn} 14, 6). Y el Padre, en el monte de la Transfiguración, ordena: \textquote{Escuchadle} (\textit{Mc} 9, 7; cf. \textit{Dt} 6, 4-5). Él es, en efecto, el modelo de las bienaventuranzas y la norma de la Ley nueva: \textquote{Amaos los unos a los otros como yo os he amado} (\textit{Jn} 15, 12). Este amor tiene como consecuencia la ofrenda efectiva de sí mismo (cf. \textit{Mc} 8, 34).

\ccesec{Jesús y la Ley}

\n{577} Al comienzo del Sermón de la Montaña, Jesús hace una advertencia solemne presentando la Ley dada por Dios en el Sinaí con ocasión de la Primera Alianza, a la luz de la gracia de la Nueva Alianza:

\ccecite{\textquote{No penséis que he venido a abolir la Ley y los Profetas. No he venido a abolir sino a dar cumplimiento. Sí, os lo aseguro: el cielo y la tierra pasarán antes que pase una ‘i’ o un ápice de la Ley sin que todo se haya cumplido. Por tanto, el que quebrante uno de estos mandamientos menores, y así lo enseñe a los hombres, será el menor en el Reino de los cielos; en cambio el que los observe y los enseñe, ése será grande en el Reino de los cielos} (\textit{Mt} 5, 17-19).}

\n{578} Jesús, el Mesías de Israel, por lo tanto el más grande en el Reino de los cielos, se debía sujetar a la Ley cumpliéndola en su totalidad hasta en sus menores preceptos, según sus propias palabras. Incluso es el único en poderlo hacer perfectamente (cf. \textit{Jn} 8, 46). Los judíos, según su propia confesión, jamás han podido cumplir la Ley en su totalidad, sin violar el menor de sus preceptos (cf. \textit{Jn} 7, 19; \textit{Hch} 13, 38-41; 15, 10). Por eso, en cada fiesta anual de la Expiación, los hijos de Israel piden perdón a Dios por sus transgresiones de la Ley. En efecto, la Ley constituye un todo y, como recuerda Santiago, \textquote{quien observa toda la Ley, pero falta en un solo precepto, se hace reo de todos} (\textit{St} 2, 10; cf. \textit{Ga} 3, 10; 5, 3).

\n{579} Este principio de integridad en la observancia de la Ley, no sólo en su letra sino también en su espíritu, era apreciado por los fariseos. Al subrayarlo para Israel, muchos judíos del tiempo de Jesús fueron conducidos a un celo religioso extremo (cf. \textit{Rm} 10, 2), el cual, si no quería convertirse en una casuística \textquote{hipócrita} (cf. \textit{Mt} 15, 3-7; \textit{Lc} 11, 39-54) no podía más que preparar al pueblo a esta intervención inaudita de Dios que será la ejecución perfecta de la Ley por el único Justo en lugar de todos los pecadores (cf. \textit{Is} 53, 11; \textit{Hb} 9, 15).

\n{580} El cumplimiento perfecto de la Ley no podía ser sino obra del divino Legislador que nació sometido a la Ley en la persona del Hijo (cf. \textit{Ga} 4, 4). En Jesús la Ley ya no aparece grabada en tablas de piedra sino \textquote{en el fondo del corazón} (\textit{Jr} 31, 33) del Siervo, quien, por \textquote{aportar fielmente el derecho} (\textit{Is} 42, 3), se ha convertido en \textquote{la Alianza del pueblo} (\textit{Is} 42, 6). Jesús cumplió la Ley hasta tomar sobre sí mismo \textquote{la maldición de la Ley} (\textit{Ga} 3, 13) en la que habían incurrido los que no \textquote{practican todos los preceptos de la Ley} (\textit{Ga} 3, 10) porque \textquote{ha intervenido su muerte para remisión de las transgresiones de la Primera Alianza} (\textit{Hb} 9, 15).

\n{581} Jesús fue considerado por los judíos y sus jefes espirituales como un \textquote{rabbi} (cf. \textit{Jn} 11, 28; 3, 2; \textit{Mt} 22, 23-24. 34-36). Con frecuencia argumentó en el marco de la interpretación rabínica de la Ley (cf. \textit{Mt} 12, 5; 9, 12; \textit{Mc} 2, 23-27; \textit{Lc} 6, 6-9; \textit{Jn} 7, 22-23). Pero al mismo tiempo, Jesús no podía menos que chocar con los doctores de la Ley porque no se contentaba con proponer su interpretación entre los suyos, sino que \textquote{enseñaba como quien tiene autoridad y no como los escribas} (\textit{Mt} 7, 28-29). La misma Palabra de Dios, que resonó en el Sinaí para dar a Moisés la Ley escrita, es la que en Él se hace oír de nuevo en el Monte de las Bienaventuranzas (cf. \textit{Mt} 5, 1). Esa palabra no revoca la Ley sino que la perfecciona aportando de modo divino su interpretación definitiva: \textquote{Habéis oído también que se dijo a los antepasados [\ldots] pero yo os digo} (\textit{Mt} 5, 33-34). Con esta misma autoridad divina, desaprueba ciertas \textquote{tradiciones humanas} (\textit{Mc} 7, 8) de los fariseos que \textquote{anulan la Palabra de Dios} (\textit{Mc} 7, 13).

\n{582} Yendo más lejos, Jesús da plenitud a la Ley sobre la pureza de los alimentos, tan importante en la vida cotidiana judía, manifestando su sentido \textquote{pedagógico} (cf. \textit{Ga} 3, 24) por medio de una interpretación divina: \textquote{Todo lo que de fuera entra en el hombre no puede hacerle impuro [\ldots] –así declaraba puros todos los alimentos–. Lo que sale del hombre, eso es lo que hace impuro al hombre. Porque de dentro, del corazón de los hombres, salen las intenciones malas} (\textit{Mc} 7, 18-21). Jesús, al dar con autoridad divina la interpretación definitiva de la Ley, se vio enfrentado a algunos doctores de la Ley que no aceptaban su interpretación a pesar de estar garantizada por los signos divinos con que la acompañaba (cf. \textit{Jn} 5, 36; 10, 25. 37-38; 12, 37). Esto ocurre, en particular, respecto al problema del sábado: Jesús recuerda, frecuentemente con argumentos rabínicos (cf. \textit{Mt} 2,25-27; \textit{Jn} 7, 22-24), que el descanso del sábado no se quebranta por el servicio de Dios (cf. \textit{Mt} 12, 5; \textit{Nm} 28, 9) o al prójimo (cf. \textit{Lc} 13, 15-16; 14, 3-4) que realizan sus curaciones.
\end{ccebody}

\cceth{El Templo prefigura a Cristo; Él es el Templo} 
\cceref{CEC 593, 583-586}

\begin{ccebody}
\n{593} \textit{Jesús veneró el Templo subiendo a él en peregrinación en las fiestas judías y amó con gran celo esa morada de Dios entre los hombres. El Templo prefigura su Misterio. Anunciando la destrucción del Templo anuncia su propia muerte y la entrada en una nueva edad de la historia de la salvación, donde su cuerpo será el Templo definitivo}.

\ccesec{Jesús y el Templo}

\n{583} Como los profetas anteriores a Él, Jesús profesó el más profundo respeto al Templo de Jerusalén. Fue presentado en él por José y María cuarenta días después de su nacimiento (\textit{Lc} 2, 22-39). A la edad de doce años, decidió quedarse en el Templo para recordar a sus padres que se debía a los asuntos de su Padre (cf. \textit{Lc} 2, 46-49). Durante su vida oculta, subió allí todos los años al menos con ocasión de la Pascua (cf. \textit{Lc} 2, 41); su ministerio público estuvo jalonado por sus peregrinaciones a Jerusalén con motivo de las grandes fiestas judías (cf. \textit{Jn} 2, 13-14; 5, 1. 14; 7, 1. 10. 14; 8, 2; 10, 22-23).

\n{584} Jesús subió al Templo como al lugar privilegiado para el encuentro con Dios. El Templo era para Él la casa de su Padre, una casa de oración, y se indigna porque el atrio exterior se haya convertido en un mercado (\textit{Mt} 21, 13). Si expulsa a los mercaderes del Templo es por celo hacia las cosas de su Padre: \textquote{No hagáis de la Casa de mi Padre una casa de mercado. Sus discípulos se acordaron de que estaba escrito: \textit{El celo por tu Casa me devorará} (\textit{Sal} 69, 10)} (\textit{Jn} 2, 16-17). Después de su Resurrección, los Apóstoles mantuvieron un respeto religioso hacia el Templo (cf. \textit{Hch} 2, 46; 3, 1; 5, 20. 21).

\n{585} Jesús anunció, no obstante, en el umbral de su Pasión, la ruina de ese espléndido edificio del cual no quedará piedra sobre piedra (cf. \textit{Mt} 24, 1-2). Hay aquí un anuncio de una señal de los últimos tiempos que se van a abrir con su propia Pascua (cf. \textit{Mt} 24, 3; \textit{Lc} 13, 35). Pero esta profecía pudo ser deformada por falsos testigos en su interrogatorio en casa del sumo sacerdote (cf. \textit{Mc} 14, 57-58) y serle reprochada como injuriosa cuando estaba clavado en la cruz (cf. \textit{Mt} 27, 39-40).

\n{586} Lejos de haber sido hostil al Templo (cf. \textit{Mt} 8, 4; 23, 21; \textit{Lc} 17, 14; \textit{Jn} 4, 22) donde expuso lo esencial de su enseñanza (cf. \textit{Jn} 18, 20), Jesús quiso pagar el impuesto del Templo asociándose con Pedro (cf. \textit{Mt} 17, 24-27), a quien acababa de poner como fundamento de su futura Iglesia (cf. \textit{Mt} 16, 18). Aún más, se identificó con el Templo presentándose como la morada definitiva de Dios entre los hombres (cf. \textit{Jn} 2, 21; \textit{Mt} 12, 6). Por eso su muerte corporal (cf. \textit{Jn} 2, 18-22) anuncia la destrucción del Templo que señalará la entrada en una nueva edad de la historia de la salvación: \textquote{Llega la hora en que, ni en este monte, ni en Jerusalén adoraréis al Padre} (\textit{Jn} 4, 21; cf. \textit{Jn} 4, 23-24; \textit{Mt} 27, 51; \textit{Hb} 9, 11; \textit{Ap} 21, 22).
\end{ccebody}

\cceth{La nueva Ley completa la antigua} 
\cceref{CEC 1967-1968}

\begin{ccebody}
\n{1967} La Ley evangélica \textquote{da cumplimiento} (cf. \textit{Mt} 5, 17-19), purifica, supera, y lleva a su perfección la Ley antigua. En las \textquote{Bienaventuranzas} \textit{da cumplimiento a las promesas} divinas elevándolas y ordenándolas al \textquote{Reino de los cielos}. Se dirige a los que están dispuestos a acoger con fe esta esperanza nueva: los pobres, los humildes, los afligidos, los limpios de corazón, los perseguidos a causa de Cristo, trazando así los caminos sorprendentes del Reino.

\n{1968} La Ley evangélica \textit{lleva a plenitud los mandamientos} de la Ley. El Sermón del monte, lejos de abolir o devaluar las prescripciones morales de la Ley antigua, extrae de ella sus virtualidades ocultas y hace surgir de ella nuevas exigencias: revela toda su verdad divina y humana. No añade preceptos exteriores nuevos, pero llega a reformar la raíz de los actos, el corazón, donde el hombre elige entre lo puro y lo impuro (cf. \textit{Mt} 15, 18-19), donde se forman la fe, la esperanza y la caridad, y con ellas las otras virtudes. El Evangelio conduce así la Ley a su plenitud mediante la imitación de la perfección del Padre celestial (cf. \textit{Mt} 5, 48), mediante el perdón de los enemigos y la oración por los perseguidores, según el modelo de la generosidad divina (cf. \textit{Mt} 5, 44).
\end{ccebody}

\cceth{La potencia de Cristo revelada en la cruz} 
\cceref{CEC 272, 550, 853}

\begin{ccebody}
\ccesec{El misterio de la aparente impotencia de Dios}

\n{272} La fe en Dios Padre Todopoderoso puede ser puesta a prueba por la experiencia del mal y del sufrimiento. A veces Dios puede parecer ausente e incapaz de impedir el mal. Ahora bien, Dios Padre ha revelado su omnipotencia de la manera más misteriosa en el anonadamiento voluntario y en la Resurrección de su Hijo, por los cuales ha vencido el mal. Así, Cristo crucificado es \textquote{poder de Dios y sabiduría de Dios. Porque la necedad divina es más sabia que la sabiduría de los hombres, y la debilidad divina, más fuerte que la fuerza de los hombres} (\textit{1 Co} 2, 24-25). En la Resurrección y en la exaltación de Cristo es donde el Padre \textquote{desplegó el vigor de su fuerza} y manifestó \textquote{la soberana grandeza de su poder para con nosotros, los creyentes} (\textit{Ef} 1, 19-22).

\n{550} La venida del Reino de Dios es la derrota del reino de Satanás (cf. \textit{Mt} 12, 26): \textquote{Pero si por el Espíritu de Dios expulso yo los demonios, es que ha llegado a vosotros el Reino de Dios} (\textit{Mt} 12, 28). Los \textit{exorcismos} de Jesús liberan a los hombres del dominio de los demonios (cf. \textit{Lc} 8, 26-39). Anticipan la gran victoria de Jesús sobre \textquote{el príncipe de este mundo} (\textit{Jn} 12, 31). Por la Cruz de Cristo será definitivamente establecido el Reino de Dios: \textit{Regnavit a ligno Deus} – \textquote{Dios reinó desde el madero de la Cruz}, (Venancio Fortunato, \textit{Hymnus \textquote{Vexilla Regis}}: MGH 1/4/1, 34: PL 88, 96).

\n{853} Pero en su peregrinación, la Iglesia experimenta también \textquote{hasta qué punto distan entre sí el mensaje que ella proclama y la debilidad humana de aquellos a quienes se confía el Evangelio} (GS 43, 6). Sólo avanzando por el camino \textquote{de la conversión y la renovación} (LG 8; cf. ibíd.,15) y \textquote{por el estrecho sendero de la cruz} (AG 1) es como el Pueblo de Dios puede extender el reino de Cristo (cf. RM 12-20). En efecto, \textquote{como Cristo realizó la obra de la redención en la pobreza y en la persecución, también la Iglesia está llamada a seguir el mismo camino para comunicar a los hombres los frutos de la salvación} (LG 8).
\end{ccebody}

\img{cross_panyvino}