\chapter{Domingo IV de Cuaresma (B)}

\section{Lecturas}

\rtitle{PRIMERA LECTURA}

\rbook{Del segundo libro de las Crónicas} \rred{36, 14-16. 19-23}

\rtheme{La ira y la misericordia del Señor serán manifestadas en el exilio y en la liberación del pueblo}

\begin{scripture}
En aquellos días, todos los jefes, los sacerdotes y el pueblo multiplicaron sus infidelidades, imitando las aberraciones de los pueblos y profanando el templo del Señor, que él había consagrado en Jerusalén.

El Señor, Dios de sus padres, les enviaba mensajeros a diario porque sentía lástima de su pueblo y de su morada; pero ellos escarnecían a los mensajeros de Dios, se reían de sus palabras y se burlaban de sus profetas, hasta que la ira del Señor se encendió irremediablemente contra su pueblo.

Incendiaron el templo de Dios, derribaron la muralla de Jerusalén, incendiaron todos sus palacios y destrozaron todos los objetos valiosos. Deportó a Babilonia a todos los que habían escapado de la espada. Fueron esclavos suyos y de sus hijos hasta el advenimiento del reino persa. Así se cumplió lo que había dicho Dios por medio de Jeremías:

\>{Hasta que la tierra pague los sábados, descansará todos los días de la desolación, hasta cumplirse setenta años}.

En el año primero de Ciro, rey de Persia, para cumplir lo que había dicho Dios por medio de Jeremías, el Señor movió a Ciro, rey de Persia, a promulgar de palabra y por escrito en todo su reino:

\>{Así dice Ciro, rey de Persia: El Señor, Dios del cielo, me ha entregado todos los reinos de la tierra. Él me ha encargado construirle un templo en Jerusalén de Judá. Quien de entre vosotros pertenezca a ese pueblo, puede volver. ¡Que el Señor, su Dios, esté con él!}.
\end{scripture}

\rtitle{SALMO RESPONSORIAL}

\rbook{Salmo} \rred{136, 1-2. 3. 4-5. 6}

\rtheme{Que se me pegue la lengua al paladar si no me acuerdo de ti}

\begin{psbody}
Junto a los canales de Babilonia
nos sentamos a llorar
con nostalgia de Sion;
en los sauces de sus orillas
colgábamos nuestras cítaras. 

Allí los que nos deportaron
nos invitaban a cantar;
nuestros opresores, a divertirlos:
\textquote{Cantadnos un cantar de Sión}. 

¡Cómo cantar un cántico del Señor
en tierra extranjera!

Si me olvido de ti, Jerusalén,
que se me paralice la mano derecha. 

Que se me pegue la lengua al paladar
si no me acuerdo de ti,
si no pongo a Jerusalén
en la cumbre de mis alegrías. 
\end{psbody}

\rtitle{SEGUNDA LECTURA}

\rbook{De la carta del apóstol san Pablo a los Efesios} \rred{2, 4-10}

\rtheme{Muertos por los pecados, estáis salvados por pura gracia}

\begin{scripture}
Hermanos:

Dios, rico en misericordia, por el gran amor con que nos amó, estando nosotros muertos por los pecados, nos ha hecho revivir con Cristo –estáis salvados por pura gracia–; nos ha resucitado con Cristo Jesús, nos ha sentado en el cielo con él, para revelar en los tiempos venideros la inmensa riqueza de su gracia, mediante su bondad para con nosotros en Cristo Jesús.

En efecto, por gracia estáis salvados, mediante la fe. Y esto no viene de vosotros: es don de Dios. Tampoco viene de las obras, para que nadie pueda presumir.

Somos, pues, obra suya. Dios nos ha creado en Cristo Jesús, para que nos dediquemos a las buenas obras, que de antemano dispuso él que practicásemos.
\end{scripture}

\rtitle{EVANGELIO}

\rbook{Del Evangelio según san Juan} \rred{3, 14-21}

\rtheme{Dios envió a su Hijo para que el mundo se salve por él}

\begin{scripture}
En aquel tiempo, dijo Jesús a Nicodemo:

\>{Lo mismo que Moisés elevó la serpiente en el desierto, así tiene que ser elevado el Hijo del hombre, para que todo el que cree en él tenga vida eterna. \\Porque tanto amó Dios al mundo, que entregó a su Unigénito, para que todo el que cree en él no perezca, sino que tenga vida eterna. \\Porque Dios no envió a su Hijo al mundo para juzgar al mundo, sino para que el mundo se salve por él. \\El que cree en él no será juzgado; el que no cree ya está juzgado, porque no ha creído en el nombre del Unigénito de Dios. \\Este es el juicio: que la luz vino al mundo, y los hombres prefirieron la tiniebla a la luz, porque sus obras eran malas. Pues todo el que obra el mal detesta la luz, y no se acerca a la luz, para no verse acusado por sus obras. \\En cambio, el que obra la verdad se acerca a la luz, para que se vea que sus obras están hechas según Dios}.
\end{scripture}

%\img{cross_of_jerusalem}

\begin{patercite}
Entregado el Señor a la voluntad de sus enemigos, se le obligó a llevar	el instrumento de su suplicio para burlarse de su dignidad real. Así se	cumplió lo predicho por el profeta Isaías, cuando dice: \textit{Un niño nos ha nacido, un hijo se nos ha dado: lleva a hombros el principado}. Pues	que el Señor saliera llevando el leño de la cruz ---ese leño que había de convertirse en cetro de su soberanía---, era a los ojos de los impíos ciertamente objeto de enorme humillación, pero que aparecía a los ojos de los fieles como un gran misterio. Pues este gloriosísimo vencedor del diablo y potentísimo debelador de los poderes adversos, llevaba muy	significativamente el trofeo de su triunfo, y cargaba sobre los hombros	de su invicta paciencia el símbolo de la salvación, digno de ser adorado por todos los reinos; se diría que en aquel momento, con el espectáculo	de su comportamiento, quería confirmar y decir a todos sus imitadores: \textit{El que no toma su cruz y me sigue, no es digno de mí}.

[\ldots]

\textbf{San León Magno, papa}, \textit{Tratado}, sobre la Pasión del Señor, 59, 4-6: CCL 138A, 354-359.
\end{patercite}

\newsection
\section{Comentario Patrístico}

\subsection{San Juan Crisóstomo, obispo}

\ptheme{Dios no perdonó a su propio Hijo, sino que lo entregó a la muerte por nosotros}

\src{Tratado sobre la Providencia, 17, 1-8: PG 52, 516-518.}

\begin{body}
\ltr{H}{onrando} como honramos por tan diversos motivos a nuestro común Señor, ¿no debemos, sobre todo, honrarlo, glorificarlo y admirarlo por la cruz, por aquella muerte tan ignominiosa? ¿O es que Pablo no aduce una y otra vez la muerte de Cristo como prueba de su amor por nosotros? Y morir, ¿por quiénes? Silenciando todo lo que Cristo ha hecho para nuestra utilidad y solaz, vuelve casi obsesivamente al tema de la cruz, diciendo: \textit{La prueba de que Dios nos ama es que Cristo, siendo nosotros pecadores, murió por nosotros}. De este hecho, san Pablo intenta elevarnos a las más halagüeñas esperanzas, diciendo: \textit{Si cuando éramos enemigos, fuimos reconciliados con Dios por la muerte de su Hijo, ¡con cuánta más razón, estando ya reconciliados, seremos salvos por su vida!} El mismo Pablo tiene esto por motivo de gozo y de orgullo, y salta de alegría escribiendo a los Gálatas: \textit{Dios me libre de gloriarme si no es en la cruz de nuestro Señor Jesucristo}.

Y ¿por qué te admiras de que esto haga saltar, brincar y alegrarse a Pablo? El mismo que padeció tales sufrimientos llama al suplicio su \textit{gloria}: \textit{Padre} –dice–, \textit{ha llegado la hora; glorifica a tu Hijo}.

Y el discípulo que escribió estas cosas, decía: \textit{Todavía no se había dado el Espíritu, porque Jesús no había sido glorificado}, llamando \textit{gloria} a la cruz. Y cuando quiso poner en evidencia la caridad de Cristo, ¿de qué echó mano Juan? ¿De sus milagros?, ¿de las maravillas que realizó?, ¿de los prodigios que obró? Nada de eso: saca a colación la cruz, diciendo: \textit{Tanto amó Dios al mundo que entregó a su Hijo único, para que no perezca ninguno de los que creen en él, sino que tengan vida eterna}. Y nuevamente Pablo: \textit{El que no perdonó a su propio Hijo, sino que lo entregó a la muerte por nosotros, ¿cómo no nos dará todo con él?}

\newpage 
Y cuando desea incitarnos a la humildad, de ahí toma pie su exhortación y se expresa así: \textit{Tened entre vosotros los sentimientos de una vida en Cristo Jesús. Él, a pesar de su condición divina, no hizo alarde de su categoría de Dios; al contrario, se despojó de su rango y tomó la condición de esclavo, pasando por uno de tantos. Y así, actuando como un hombre cualquiera, se rebajó hasta someterse incluso a la muerte, y una muerte de cruz}.

En otra ocasión, dando consejos acerca de la caridad, vuelve sobre el mismo tema, diciendo: \textit{Vivid en el amor como Cristo os amó y se entregó por nosotros como oblación y víctima de suave olor}.

Y, finalmente, el mismo Cristo, para demostrar cómo la cruz era su principal preocupación y cómo su pasión primaba en Él, escucha qué es lo que le dijo al príncipe de los apóstoles, al fundamento de la Iglesia, al corifeo del coro de los apóstoles, cuando, desde su ignorancia, le decía: ¡No lo permita Dios, Señor! Eso no puede pasarte: Quítate –le dijo– \textit{de mi vista, Satanás, que me haces tropezar}. Con lo exagerado del reproche y de la reprimenda, quiso dejar bien sentado la gran importancia que a sus ojos tenía la cruz.

¿Por qué te maravillas, pues, de que en esta vida sea la cruz tan célebre como para que Cristo la llame su \textquote{gloria} y Pablo en ella se gloríe?
\end{body}

\begin{patercite}
[\ldots]
	
Amadísimos: habiendo sido levantado Cristo en la cruz, no debe nuestra alma contemplar tan sólo aquella imagen que impresionó vivamente la vista de los impíos a quienes se dirigía Moisés con estas palabras: \textit{Tu vida estará ante ti como pendiente de un hilo, temblarás día y noche, y ni de tu vida te sentirás seguro}. Estos hombres no fueron capaces de ver en el Señor crucificado otra cosa que su acción culpable, llenos de temor, y no del temor con que se justifica la fe verdadera, sino el temor que atormenta la mala conciencia.

Que nuestra alma, iluminada por el Espíritu de verdad; reciba con puro y libre corazón la gloria de la cruz, que irradia por cielo y tierra, y trate de penetrar interiormente lo que el Señor quiso significar cuando, hablando de la pasión cercana, dijo: \textit{Ha llegado la hora de que sea glorificado el Hijo del hombre}. Y más adelante: \textit{Ahora mi alma está agitada, y ¿qué diré? Padre, líbrame de esta hora. Pero si por esto he venido, para esta hora, Padre, glorifica a tu Hijo}. Y como oyera la voz del Padre, que decía desde el cielo: \textit{Le he glorificado y volveré a glorificarlo}, dijo Jesús a los que lo rodeaban: \textit{Esta voz no ha venido por mí, sino por vosotros. Ahora va a ser juzgado el mundo; ahora el Príncipe de este mundo va a ser echado fuera. Y cuando yo sea elevado sobre la tierra atraeré a todos hacia mí}.
	
\textbf{San León Magno, papa}, \textit{Tratado}, sobre la Pasión del Señor, 59, 4-6: CCL 138A, 354-359.
\end{patercite}

\newsection
\section{Homilías}

\subsection{San Pablo VI, papa}

\subsubsection{Homilía (1970): Dios nos ama}

\src{8 de marzo de 1970.}

\begin{body}
\ltr{H}{emos} escuchado (\ldots) estas palabras \textquote{profundas y hermosas} de Jesús contenidas en el \textbf{Evangelio de san Juan} donde se describe el encuentro con Nicodemo, el fariseo creyente y dudoso. La entrevista se desarrolla de noche. Hay una pequeña luz, y aparece Jesús, la Luz del mundo, que habla a este hombre en busca de luz. Le dice tantas cosas. Revela la razón de su venida al mundo. Cristo se presenta a sí mismo como el Hijo de Dios, como el Hijo del hombre, como el único, como el Mesías. Y le dice al mundo la palabra muy simple, pero impactante y exultante: Dios\ldots

Los hijos de nuestro tiempo conocemos la dificultad, el misterio que se espesa sobre esta palabra. Conocemos toda la negación que quiere anular el nombre de Dios de las conciencias y de la profesión pública. Y escuchamos las mil voces que dicen de Dios tantas y tantas cosas que no siempre asimilamos. Tenemos una idea de Su existencia, sentimos algo de Su grandeza. Nuestra experiencia, que se detiene en las cosas que se ven y se tocan, sin hablarnos de Dios, deja que algo brille. Los que estudian ciencia se encuentran en una posición ambigua y dicen: \textquote{no he encontrado a Dios estudiando las cosas}. Los que han cruzado las fronteras del cielo dicen: \textquote{no encontré a Dios viajando por el espacio}. Sin embargo, mirando hacia atrás, hay que decir: Todo esto es tan hermoso que hay algo dentro: un diseño, una palabra impresa en las cosas. Quienes estudian deben sentir que hay una presencia del Señor.

No es cierto, por tanto, que la ciencia se aleje de Dios, la ciencia nos permite vislumbrar una realidad inmensa. Los signos que encontramos a lo largo de la creación nos dicen que hay una ley, un pensamiento, una personalidad infinita que domina la existencia del universo. [En este punto el Santo Padre recuerda la oportunidad que tuvo hace unos días de ver de cerca algunos fragmentos de rocas lunares.] Mirándolos, parecían piedras de tierra ordinarias. Pero incluso si fueran diferentes, la investigación científica siempre concluye que las mismas leyes que gobiernan la naturaleza terrestre dominan el universo. El universo entero está penetrado por un diseño que será misterioso, pero que dice una verdad: es un diseño, es un pensamiento. Y luego nos quedamos con un gemido en el alma: ¿por qué se esconde el Señor? Atormenta a las grandes almas, y también puede atormentar la nuestra, el sentido de ser, de existir, de vivir. ¿Qué pensará Dios de mí? ¿En qué relación estoy con él? ¿Será el Dios terrible, que no me conoce y dejará que yo sea aplastado por las leyes del mundo que ha creado? ¿Soy un ser que no tiene importancia ante Él? O por el contrario\ldots

Aquí está la noticia, aquí viene el \textbf{Evangelio}, para decirnos por qué vino Jesús: en el misterio del ser, en la gran curiosidad del hombre, abrió una brecha, abrió una ventana y surgió una maravillosa ola de luz. Dios nos ama. Esta es la revelación. Somos amados, somos pensados, somos queridos por Dios. Dios cuida de nosotros más de lo que una madre cuida a su hijo. Y cuando quisimos darle un nombre a este Ser ilimitado, infinito y tremendamente misterioso, Jesús nos enseñó a invocarlo con plena confianza, con un amor perfecto: \textquote{llamadlo Padre}.

Dios es nuestro padre. En el mundo, en la humanidad, en la historia, el Papa repite el eco de esta verdad evangélica. Dios nos ama. Dios piensa en nosotros, tiene los ojos siempre abiertos sobre nosotros y escudriña nuestra respuesta. Dios nos ama, se compadece de nosotros, nos perdona, nos consuela y no deja que caigan nuestras palabras, nuestros gemidos, nuestras invocaciones, nuestras lágrimas, nuestras buenas obras. Quiere que nuestra vida se resuma en un acto de amor. Y el misterioso contacto entre Dios y el hombre tiene lugar solo a través de Cristo. Necesitábamos un puente entre nosotros y Dios, un intermediario que nos llevara a la plenitud a la que tiende nuestra vida, nuestro destino eterno. Es el misterio de la alegría y de la salvación que es la Redención, que tendrá su fiesta más solemne en la Santa Pascua.

Cuando en el silencio de nuestra alma o en la confusión de nuestra existencia nos preguntamos por qué estamos en el mundo, recordemos que Dios nos ama. Amó tanto al mundo que dio a Cristo, su único Hijo, para la salvación de los hombres. Tenemos la suerte de llamarnos hijos de Dios y de vincular nuestra miserable vida a su existencia infinita, como pequeñas chispas que deben terminar en el sol, a la luz del Señor. ¡Dios nos ama! Recordemos esta verdad y seremos felices, bendecidos y salvados para siempre.
\end{body}


\newsection
\subsection{San Juan Pablo II, papa}

\subsubsection{Homilía (1979): El signo de Dios}

\src{Visita Pastoral a la Parroquia Romana de la Santa Cruz de Jerusalén. \\25 de marzo de 1979.}

\begin{body}
1. (\ldots) La liturgia dominical de hoy comienza con la palabra: \textit{Laetare}: \textquote{¡Alégrate!}, es decir, con la invitación a la alegría espiritual. [Yo me alegro porque también en este domingo, se me ha concedido encontrarme en un lugar santificado por la tradición de tantas generaciones; en el santuario de la Santa Cruz, que hoy es estación cuaresmal y, al mismo tiempo, vuestra iglesia parroquial.]

2. [Vengo aquí para adorar en espíritu, junto con vosotros,] el misterio de la cruz del Señor. Hacia este misterio nos orienta el coloquio de Cristo con Nicodemo, que volvemos a leer hoy en el \textbf{Evangelio}. Jesús tiene ante sí a un escriba, un perito en la Escritura, un miembro del Sanedrín y, al mismo tiempo, un hombre de buena voluntad. Por esto decide encaminarlo al misterio de la cruz. Recuerda, pues, en primer lugar, que Moisés levantó en el desierto la serpiente de bronce durante el camino de cuarenta años de Israel desde Egipto a la Tierra Prometida. Cuando alguno a quien había mordido la serpiente en el desierto, miraba aquel signo, quedaba con vida (cf. \textit{Núm} 21, 4-9). Este signo, que era la serpiente de bronce, preanunciaba otra Elevación: \textquote{Es preciso –dice, desde luego, Jesús– que sea levantado el Hijo del hombre} –y aquí habla de la elevación sobre la cruz– \textquote{para que todo el que creyere en Él tenga la vida eterna} (\textit{Jn} 3, 14-15). ¡La cruz: ya no sólo la figura que preanuncia, sino la Realidad misma de la salvación! Y he aquí que Cristo explica hasta el fondo a su interlocutor, estupefacto pero al mismo tiempo pronto a escuchar y a continuar el coloquio, el significado de la cruz: \textquote{Porque tanto amó Dios al mundo, que le dio su Hijo Unigénito, para que todo el que crea en Él no perezca, sino que tenga la vida eterna} (\textit{Jn} 3, 16).

La cruz es una nueva revelación de Dios. Es la revelación definitiva. En el camino del pensamiento humano, en el camino del conocimiento de Dios, se realiza un vuelco radical. Nicodemo, el hombre noble y honesto, y al mismo tiempo discípulo y conocedor del Antiguo Testamento, debió sentir una sacudida interior. Para todo Israel Dios era sobre todo Majestad y Justicia. Era considerado como Juez que recompensa o castiga. Dios, de quien habla Jesús, es Dios que envía a su propio Hijo no \textquote{para que juzgue al mundo, sino para que el mundo sea salvo por Él} (\textit{Jn} 3, 17). Es Dios del amor, el Padre que no retrocede ante el sacrificio del Hijo para salvar al hombre.

3. \textbf{San Pablo}, con la mirada fija en la misma revelación de Dios, repite hoy por dos veces en la \textbf{Carta a los Efesios}: \textquote{Por pura gracia habéis sido salvados} (\textit{Ef} 2, 5). \textquote{Por gracia habéis sido salvados mediante la fe} (\textit{Ef} 2, 8). Sin embargo, este Pablo, así como también Nicodemo, hasta su conversión fue el hombre de la Ley Antigua. En el camino de Damasco se le reveló Cristo y desde ese momento Pablo entendió de Dios lo que proclama hoy: \textquote{\ldots Dios, que es rico en misericordia, por el gran amor con que nos amó, y estando nosotros muertos por nuestros delitos, nos dio vida por Cristo –por gracia habéis sido salvados–} (\textit{Ef} 2, 4-5).

¿Qué es la gracia? \textquote{Es un don de Dios}. El don que se explica con su amor. El don está allí donde está el amor. Y el amor se revela mediante la cruz. Así dijo Jesús a Nicodemo. El amor, que se revela mediante la cruz, es precisamente la gracia. En ella se desvela el más profundo rostro de Dios. Él no es sólo el juez. Es Dios de infinita majestad y de extrema justicia. Es Padre, que quiere que el mundo se salve; que entienda el significado de la cruz. 

Esta es la elocuencia más fuerte del significado de la Ley y de la pena. Es la palabra que habla de modo diverso a las conciencias humanas. Es la palabra que obliga de modo diverso a las palabras de la Ley y a la amenaza de la pena. Para entender esta palabra es preciso ser un hombre transformado. Ser un hombre de la Gracia y de la Verdad. ¡La Gracia es un don que compromete! ¡El don de Dios vivo, que compromete al hombre para la vida nueva! Y precisamente en esto consiste ese juicio del que habla también Cristo a Nicodemo: la cruz salva y, al mismo tiempo, juzga. Juzga diversamente. Juzga más profundamente. \textquote{Porque todo el que obra el mal, aborrece la luz} \ldots –¡precisamente esta luz estupenda que emana de la cruz!– \textquote{Pero el que obra la verdad viene a la luz} (\textit{Jn} 3, 20-21). Viene a la cruz. Se somete a las exigencias de la gracia. Quiere que lo comprometa ese inefable don de Dios. Que forje toda su vida. Este hombre oye en la cruz la voz de Dios, que dirige la palabra a los hijos de esta tierra nuestra, del mismo modo que habló una vez a los desterrados de Israel mediante Ciro, rey de Persia, con la invocación de esperanza. La cruz es invocación de esperanza.

4. Es preciso que nosotros reunidos en esta estación cuaresmal de la cruz de Cristo, nos hagamos estas preguntas fundamentales, que fluyen de la cruz hacia nosotros. ¿Qué hemos hecho y qué hacemos para conocer mejor a Dios? Este Dios que nos ha revelado Cristo. ¿Quién es Él para nosotros? ¿Qué lugar ocupa en nuestra conciencia, en nuestra vida? Preguntémonos por este lugar, porque tantos factores y tantas circunstancias quitan a Dios este puesto en nosotros. ¿No ha venido a ser Dios para nosotros ya sólo algo marginal? ¿No está cubierto su nombre en nuestra alma con un montón de otras palabras? ¿No ha sido pisoteado como aquella semilla caída \textquote{junto al camino} (\textit{Mc} 4, 4)? ¿No hemos renunciado interiormente a la redención mediante la cruz de Cristo, poniendo en su lugar otros programas puramente temporales, parciales, superficiales?

5.(\ldots) Confesemos con humildad nuestras culpas, nuestras negligencias nuestra indiferencia en relación con este Amor que se ha revelado en la cruz. Y a la vez renovémonos en el espíritu con gran deseo de la vida, de la vida de gracia, que eleva continuamente al hombre, lo fortifica, lo compromete. Esa gracia que da la plena dimensión a nuestra existencia sobre la tierra.

Así sea.
\end{body}

\label{b2-03-04-1979H}

\begin{patercite}
¡Oh don preciosísimo de la cruz! ¡Qué aspecto tiene más esplendoroso! No contiene, como el árbol del paraíso, el bien y el mal entremezclados, sino que en él todo es hermoso y atractivo, tanto para la vista como para el paladar.

Es un árbol que engendra la vida, sin ocasionar la muerte; que ilumina sin producir sombras; que introduce en el paraíso, sin expulsar a nadie de él; es el madero al que Cristo subió, como rey que monta en su cuadriga, para derrotar al diablo que detentaba el poder de la muerte, y librar al género humano de la esclavitud a que la tenía sometido el diablo.

Este madero, en el que el Señor, cual valiente luchador en el combate,fue herido en sus divinas manos, pies y costado, curó las huellas del pecado y las heridas que el pernicioso dragón había infligido a nuestra naturaleza.

Si al principio un madero nos trajo la muerte, ahora otro madero nos da la vida: entonces fuimos seducidos por el árbol; ahora por el árbol ahuyentamos la antigua serpiente. Nuevos e inesperados cambios: en lugar de la muerte alcanzamos la vida; en lugar de la corrupción, la incorrupción; en lugar del deshonor, la gloria.

No le faltaba, pues, razón al Apóstol para exclamar: \textit{Dios me libre de gloriarme si no es en la cruz de nuestro Señor Jesucristo, en la cual el mundo está crucificado para mí, y yo para el mundo}. Pues aquella suprema sabiduría, que, por así decir, floreció en la cruz, puso de	manifiesto la jactancia y la arrogante estupidez de la sabiduría mundana. El conjunto maravilloso de bienes que provienen de la cruz	acabaron con los gérmenes de la malicia y del pecado.

[\ldots] Con la cruz sucumbió la muerte, y Adán se vio restituido a la vida. En la cruz se gloriaron todos los apóstoles, en ella se coronaron los mártires y se santificaron los santos. Con la cruz nos revestimos de Cristo y nos despojamos del hombre viejo; fue la cruz la que nos reunió en un solo rebaño, como ovejas de Cristo, y es la cruz la que nos lleva al aprisco celestial.

\textbf{San Teodoro de Studion}, \textit{Sermón}, sobre la adoración de la cruz: PG 99, 691-694. 695.
\end{patercite}

\newpage

\subsubsection{Homilía (1985): Gracia y pecado}

\src{Celebración Eucarística en la Parroquia Romana \\del Sagrado Corazón de Jesús y María en Tor Fiorenza. \\17 de marzo de 1985.}

\begin{body}
1. \textquote{Por gracia \ldots habéis sido salvados} (\textit{Ef} 2, 5).

\ltr{E}{stas} palabras de la \textbf{Carta a los Efesios} están casi en el centro de las lecturas y meditaciones de la Iglesia en este cuarto domingo de Cuaresma.

[Estas palabras parecen tener una elocuencia particular en esta visita a la comunidad parroquial, dedicada al divino Corazón de Jesús y al Inmaculado Corazón de su Madre: estos dos Corazones unidos por un santísimo vínculo de amor, de ese amor que tiene en Dios uno y trino su fuente eterna e inagotable. De hecho,] Dios es amor, y lo que el apóstol llama \textquote{gracia} tiene su comienzo en ese amor, cuando dice: \textquote{Por gracia fuisteis salvados}.

Sí. La gracia es ese don inefable, mediante el cual Dios quiere salvar al hombre permitiéndole participar de su divinidad: en su naturaleza divina, en la vida inescrutable del Padre, del Hijo y del Espíritu Santo.

El más alto y pleno testimonio de esta voluntad salvífica de la Santísima Trinidad es el Corazón del Redentor y, unido a él, el Corazón de la Sierva del Señor, en el que la gracia ha alcanzado la inefable plenitud de la divinidad materna.

2. El período de Cuaresma requiere que meditemos sobre el misterio de la gracia divina. No solo el misterio del pecado, sino también el de la gracia. En la economía revelada de la salvación, uno no puede separarse del otro. No podemos separar la gracia del pecado y el pecado de la gracia. Están en una relación recíproca, estrictamente complementaria.

El clima cuaresmal nos obliga a penetrar más profundamente en ambos misterios: de la gracia y el pecado, del pecado y de la gracia. En su oposición radical a la condición humana. Y, al mismo tiempo, en su admirable complementariedad por parte de Dios.

Precisamente esta maravillosa complementariedad recuerda la misericordia de Dios. La \textbf{Carta a los Efesios} lo dice en la liturgia de hoy: \textquote{Pero Dios, rico en misericordia, por el gran amor con que nos amó, estando nosotros muertos por los pecados, nos ha hecho revivir con Cristo} (\textit{Ef} 2, 4-5).

¡\textit{Dives en misericordia} (rico en misericordia)! [en cierto sentido es la encíclica programática de mi ministerio en la Sede de San Pedro.]

Misericordia significa precisamente este amor que no se aparta del pecado, que no se desprende de él, sino que se acerca a él. Este es precisamente el amor de Dios, revelado en Jesucristo.

3. San Pablo habla de ello, y también del \textbf{Evangelio de San Juan} de hoy: \textquote{Dios no envió a su Hijo al mundo para juzgar al mundo, sino para que el mundo se salve por él} (\textit{Jn} 3, 17).

Según la lógica de la justicia \textquote{pura}, el pecado merece condenación. El Hijo de Dios, al venir al mundo, tomó sobre sí la herencia del pecado; él \textquote{murió por los pecados} (cf. \textit{Ef} 2, 5), mientras que debería haberse presentado ante la humanidad como un juez que castiga.

¡Pero no!

Vino \textquote{para que el mundo se salve por él}. Por esto fue elevado sobre el leño de la cruz. El Evangelio compara esta elevación de Cristo con el gesto que hizo Moisés en el desierto, cuando los israelitas murieron por la mordedura de serpientes venenosas: él levantó sobre el leño \textquote{una serpiente de bronce} (\textit{Núm} 21, 9) y quien la miraba después de ser mordido, era salvado de la muerte.

4. Así pues, hoy la liturgia de la Cuaresma nos recuerda y nos invita a profundizar en este maravilloso misterio del amor misericordioso en el cual reside la fuente de la vida.

En efecto, Dios, rico en misericordia, junto con Cristo, –¡y precisamente con Cristo crucificado!– \textquote{nos hizo revivir}, estando nosotros \textquote{muertos por los pecados} (cf. \textit{Ef} 2, 5).

Y junto con Cristo resucitado, \textquote{él también nos resucitó} (\textit{Ef} 2, 6). ¿A qué vida? ¡A la de Cristo mismo! A la vida de Dios en el hombre. A la Vida divina, es decir, sobrenatural: por encima de las exigencias y leyes de la propia naturaleza humana. A la Vida divina, es decir, eterna. Aquí en la tierra experimentamos su comienzo en nuestras almas, mientras que su cumplimiento pertenece a la eternidad: \textquote{para revelar en los tiempos venideros la inmensa riqueza de su gracia mediante su bondad para con nosotros en Cristo Jesús} (\textit{Ef} 2, 7).

La gracia –don maravilloso de Dios– hace que en cierto sentido seamos creados de nuevo: \textquote{creados en Cristo Jesús para que nos dediquemos a las buenas obras que de antemano Dios dispuso que practicásemos} (\textit{Ef} 2, 10).

5. El período de Cuaresma debe conducirnos a esta \textquote{nueva criatura} en Jesucristo. ¡Este es el momento en que debemos eliminar el pecado de nosotros mismos y recibir el don de Dios!

La liturgia nos recuerda el maravilloso \textbf{Salmo 136}: el cántico que cantan los hijos de Israel deportados de su tierra natal en la esclavitud de Babilonia: \textquote{¡Cómo cantar un cántico del Señor en tierra extranjera! Si me olvido de ti, Jerusalén, que se me paralice la mano derecha} (\textit{Sal} 137, 4-5).

Por eso, queridos hermanos y hermanas, es necesario \textquote{acercarse a la luz} (cf. \textit{Jn} 3, 21). Es necesario, a través de esta luz, que es Cristo, ver toda nuestra vida terrenal en sus proporciones correctas, reconociendo que esta no es la vida definitiva: \textquote{Aquí abajo no tenemos ciudad permanente, sino que vamos en busca de la futura} (\textit{Hb} 13, 14). La gracia injertada en nuestras almas por Jesucristo dirige toda nuestra existencia hacia la \textquote{Jerusalén eterna}, abriendo nuestros corazones hacia ella, así como los corazones de los hijos de Israel hacia los ríos de Babilonia.

6. \txtsmall{[¡Queridos hermanos y hermanas! Vuestra parroquia está dedicada a los Corazones más sagrados de Jesús y María. Hoy se me permite visitar esta parroquia para meditar con vosotros los misterios de la gracia de Dios según la liturgia del cuarto domingo de Cuaresma.]}

\txtsmall{[\ldots]}

7. \txtsmall{[Quiero ahora saludar cordialmente a todos los presentes (\ldots) las personas aquí reunidas: las familias, los jóvenes, los ancianos, los niños; y les pido que lleven un saludo especial a quienes, a pesar de su deseo, no pudieron estar presentes por diversos motivos: especialmente los enfermos y los que están lejos de casa. Mi saludo va también a toda la población residente en el territorio de esta parroquia: a ellos va mi respetuoso pensamiento sobre la base de los valores comunes de la humanidad y la promoción de la persona humana, a los que todos estamos llamados a prestar la máxima atención. [\ldots] Sé que en sus treinta y cinco años de vida, esta parroquia ha dado a la Iglesia un cierto número de vocaciones sacerdotales y religiosas: este es un signo indudable de auténtica vida cristiana y comunión eclesial. Y también sé que esta sensibilidad al problema de las vocaciones sigue viva y efectiva. Con tantas fuerzas en acción y tantas posibilidades, está claro que vuestra parroquia está llamada, quizás más que otras, a ser una \textquote{parroquia abierta}, es decir, con una actitud de disponibilidad generosa también hacia otras parroquias y toda la diócesis. En cualquier caso, os invito a acentuar este compromiso.}

\txtsmall{Solo puedo felicitaros por todo el bien que hacéis en el nombre del Señor, bajo la guía de vuestros pastores. Nunca pongáis límite a los resultados obtenidos, más bien inspirad siempre vuestra conducta en el Corazón desbordante de amor de Jesús y en el Corazón inmaculado de su Santísima Madre. Tomemos un ejemplo de la generosidad de la misericordia del Padre. La parroquia debe ser una comunidad de misericordia. A tal fin, será bueno subrayar cada vez más, en vosotros y en los demás, la promoción de esas condiciones espirituales interiores de conversión y penitencia, que nos atraen de manera especial los dones de la misericordia divina, y que se expresan de una manera eminente en una ferviente práctica del sacramento de la Reconciliación. Por ello, sólo puedo alabar el plan pastoral, previsto para el próximo año, destinado precisamente a tener un mayor aprecio por el sacramento del perdón divino, en el que es posible tener una experiencia muy preciosa de la gracia que nos salva.]}

8. \textquote{Por gracia \ldots hemos sido salvados} (\textit{Ef} 2, 8). Que nuestro encuentro de hoy, en este período especial de gracia salvadora que es la Cuaresma, renueve en vosotros el deseo de vivir en la gracia divina.

Una vez más me refiero al Corazón de nuestro Redentor, que es \textquote{fuente de vida y santidad}, ¡me refiero a él a través del Corazón inmaculado de María, Madre de Dios! Que Dios, rico en misericordia (\textquote{\textit{Dives in misericordia}}), se manifieste siempre en la maravillosa unión de estos Corazones que protegen vuestra parroquia. ¡Ojalá obre en vosotros con el poder de su gracia!
\end{body}

\newpage
\subsubsection{Ángelus (1985): Caridad concreta}

\src{17 de marzo del 1985.}

\textquote{Tanto amó Dios al mundo, que le dio su Hijo unigénito} (\textit{Jn} 3, 16).

\begin{body}
	\ltr[1. ] {L}{a} liturgia del IV domingo de Cuaresma invita a perseverar en la práctica de la penitencia como preparación a la Pascua, en el sublime contexto del amor de Dios. Dios, que es amor en la intimidad de su ser, envió por amor su Hijo unigénito al mundo, para que sufriese, muriese y resucitase por nosotros.
	
	La respuesta del hombre a este inefable proyecto que tiene a Dios como protagonista, está grabada en el axioma sobre el que se apoya la perfección de toda la ley: \textquote{Ama al Señor tu Dios; ama al prójimo como a ti mismo} (\textit{Mt} 22, 37-39). El cristianismo es la religión del amor. El cristianismo es la religión de la \textquote{sociabilidad}, de esa sociabilidad que encuentra en la parábola del samaritano su paradigma programático y vital, su explicitud existencial más concreta e imperativa: \textquote{Anda, haz tú lo mismo} (\textit{Lc} 10, 37).
	
	2. La Cuaresma, por su conexión íntima con la vicisitud pascual del Hombre-Dios, es un tiempo privilegiado para el ejercicio del amor al prójimo. Tiempo de genuina caridad.
	
	En la Exhortación Apostólica \textit{Reconciliatio et paenitentia}, [a la que quiero referirme en los encuentros dominicales de esta Cuaresma,] he subrayado que la penitencia tiene una dimensión social. La Iglesia, entre las varias formas penitenciales, ha recomendado siempre la limosna, y la recomienda aún como \textquote{medio para hacer concreta la caridad, compartiendo lo que se tiene con quien sufre las consecuencias de la pobreza} (n. 26).
	
	No es raro encontrar en la mentalidad contemporánea, marcadamente sensible a los cánones de la justicia, varias contraindicaciones para con la caridad menuda. Y sin embargo, Jesús asegura que ni un vaso de agua, dado en su nombre, será olvidado en el balance de la vida (cf. \textit{Mc} 9, 41). Basta la palabra del Maestro para precaver contra las diversas insinuaciones del egoísmo, que querría inducir al cristiano a cerrar la mano y volver la espalda a quien le pide algo (cf. \textit{Mt} 5, 42).
	
	Las privaciones penitenciales, realizadas tanto por obediencia a la norma eclesial, como por impulso de creatividad personal, encuentran un campo casi ilimitado de aplicación. El drama del hambre, que existe en más de una región de nuestro planeta, interpela apremiantemente a las conciencias.
	
	Cada hermano que muere de hambre, pesa sobre la conciencia de todos. Para estimularnos en este grave deber de solidaridad, contribuye la Virgen María con las palabras amonestadoras del Magníficat: \textquote{A los hambrientos los colma de bienes, y a los ricos los despide vacíos} (\textit{Lc} 1, 53).
\end{body}

\newpage
\subsubsection{Homilía (1988): Dios amó al mundo}

\src{Visita a la Parroquia de San Pedro Damián.\par 13 de marzo de 1988.}

\begin{body}
1. \textquote{Tanto amó Dios al mundo\ldots} (\textit{Jn} 3,16).

\ltr{L}{a} Iglesia nos prepara la mesa de la Palabra de Dios, lo hace todos los días. Pero los domingos de Cuaresma tienen un significado particular, al igual que todo este período, que nos introduce en el corazón mismo del misterio pascual.

Se podría decir que la liturgia de la Palabra es una polifonía \textquote{\textit{sui generis}} en la que se desarrollan diversos temas. Cada lectura constituye, en cierto sentido, un tema distinto en esta polifonía litúrgica de la Palabra de Dios. Y aunque estos temas aparentemente difieren, de hecho, de libros lejanos en el tiempo, sin embargo todos se encuentran en torno a un hecho, que se constituye como la clave dominante de toda la polifonía. En esta majestuosa composición que es la liturgia, la Iglesia, como un gran artista, quiere hablarnos de una manera particular. Quiere que la gran verdad de la revelación divina sea injertada y revitalizada de una manera particular en nuestra conciencia.

La verdad, que constituye propiamente el tema central de la liturgia de hoy, está contenida en estas palabras: \textquote{Dios amó al mundo}.

2. \textquote{Tanto amó Dios al mundo que le dio a su Hijo unigénito: todo el que cree en él tiene vida eterna} (cf. \textit{Jn} 3, 16).

La liturgia repite esta frase en el cántico del Evangelio, y toda la asamblea responde: \textquote{¡Gloria y alabanza a ti, oh Cristo!}; o: \textquote{¡Gloria y alabanza a ti, oh rey de los siglos!}.

Las palabras citadas fueron escritas en el cuarto evangelio. Según \textbf{San Juan}, Jesús las pronunció en el transcurso de la conversación nocturna con Nicodemo. Con estas palabras el Hijo da testimonio del Padre, de modo que constituyen la verdad central de la autorrevelación de Dios: Dios que ama, Dios que es amor.

3. El pasaje de la \textbf{carta a los Efesios} contiene casi un comentario particular sobre esta verdad. Allí leemos: \textquote{Dios, rico en misericordia, por el gran amor con que nos amó, estando nosotros muertos por los pecados, nos ha hecho revivir con Cristo} (\textit{Ef} 2, 4-5).

El amor de Dios se remonta al principio mismo, a la creación misma del mundo; de hecho, precede a la creación: \textquote{Somos pues obra suya –leemos en la carta a los Efesios–, creados en Cristo Jesús, para que nos dediquemos a las buenas obras, que de antemano dispuso Dios que practicásemos} (\textit{Ef} 2, 10).

Esta frase debe recordarnos el último día de la creación cuando Dios \textquote{vio lo que había hecho, y he aquí que era muy bueno} (\textit{Gn} 1, 31). ¡El hombre fue creado por Dios para siempre!

Ésta es la verdad de nuestro principio.

Pero la verdad de la historia del hombre está, lamentablemente, ligada al pecado. ¿Acaso esto podría destruir ese amor eterno con el que el mundo –y el hombre en el mundo– fue amado en el Verbo eterno, en el Hijo consustancial, en Cristo? ¡No! El pecado ha revelado y confirmado este amor de una manera nueva. Dios, de quien Jesús de Nazaret habla a Nicodemo, es \textquote{rico en misericordia}. Su amor es \textquote{misericordioso}, se acerca a nosotros, \textquote{muertos por los pecados}, para hacernos \textquote{vivir de nuevo con Cristo}.

4. En la conversación nocturna con el escriba Nicodemo, Jesús habla precisamente de este amor. Habla del Padre y habla de sí mismo, el Hijo. \textquote{Porque tanto amó Dios al mundo que dio a su Hijo unigénito} (\textit{Jn} 3, 16). Y lo dio \textquote{para que todo el que crea en él no se pierda, sino que tenga vida eterna} (\textit{Jn} 3, 16).

Por tanto: \textquote{Dios, rico en misericordia} (\textit{Dives in misericordia}). Dios, Padre e Hijo. Como justicia pura, se suponía que este Dios juzgaría y condenaría al mundo y al hombre a causa del pecado. Pero \textquote{Dios no envió a su Hijo al mundo para juzgar al mundo, sino para que el mundo se salve por él} (\textit{Jn} 3, 17). El amor es más grande que el pecado. Y Dios es este amor.

5. ¿Qué significa que este amor es indulgente?

Si entendiéramos la autorrevelación de Dios en Jesucristo como un amor indulgente al mal, estaríamos en un gran error. Al contrario: habríamos falsificado la verdad sobre Dios y la verdad sobre el amor. Dios, rico en misericordia, es ese Padre que no es indulgente con el mal. Va al encuentro del pecador, el hijo pródigo de la parábola, pero no es indulgente con el mal.

De hecho, ¿qué significa que \textquote{entregó a su único Hijo}? Significa (como escribe el propio apóstol Pablo), que \textquote{lo entregó por todos nosotros} (\textit{Rm} 8, 32).

Lo entregó para ser \textquote{exaltado}, levantado en la cruz. Así como Moisés una vez, por orden de Yahvé, levantó la serpiente de bronce en el desierto para que los hijos e hijas de Israel, mordidos por serpientes venenosas, pudieran liberarse del veneno y la infección al volver la mirada hacia aquel signo.

Por tanto, este \textquote{amor misericordioso} del Padre conlleva un precio enorme: la medida de la justicia divina; la muerte en la cruz del Hijo unigénito. ¡Su sacrificio redentor!

6. ¿Significa esto que la sentencia ya se ha cumplido? ¿Que la condena y la responsabilidad por el pecado han sido ya suprimidas? Encontramos la respuesta a esas interrogantes también en la conversación con Nicodemo. Cristo dice: \textquote{El juicio es este: la luz ha venido al mundo, pero\ldots Todo el que obra el mal detesta la luz, y no se acerca a la luz, para no verse acusado por sus obras. En cambio, el que obra la verdad se acerca a la luz, para que se vea claramente que sus obras están hechas según Dios} (\textit{Jn} 3, 19. 21).

Nos encontramos así en el punto central de esta verdad, que la Iglesia quiere renovar en la conciencia de todos durante la Cuaresma.

He aquí que la Iglesia nos dice a cada uno de nosotros: ¡mira tu vida, mira tus obras a la luz de Cristo \textquote{exaltado}, levantado en la cruz! Lee la verdad del amor divino en su totalidad, pronunciada en esta cruz. Y ahora, a la luz de esta verdad, ¡examina tu conciencia! A su luz juzga tus pensamientos, palabras, obras, toda tu vida.

7. La conversación de Cristo con Nicodemo se lleva a cabo en un lugar y tiempo específicos.

Sin embargo, la \textquote{estructura polifónica} de la liturgia de la Palabra nos permite ver que toda la historia de Israel, del pueblo de la antigua alianza, aunque desde diferentes puntos, se acerca a este lugar, a esta conversación. El \textbf{libro de las Crónicas} lo demuestra en la \textbf{primera lectura}. El \textbf{salmo}, el cántico impresionante de los desterrados en los ríos de Babilonia, también habla de ello. Y si, por orden de Ciro, los exiliados deben regresar y reconstruir el templo de Jerusalén para Dios, entonces ¿este templo no significará también esperar al que Dios mismo entregará para que este mundo \textquote{sea salvado por él} (\textit{Jn} 3, 17)?

Y no sólo los hombres de ese linaje, los herederos de la antigua alianza con Dios, sino toda la humanidad, todos los hombres, incluso por caminos diferentes, están orientados hacia esta verdad central del Dios \textquote{rico en misericordia}, del Dios que realiza la salvación definitiva del hombre.

8. También [vuestra parroquia, queridos fieles de San Pedro Damiani,] es una voz en la compleja historia del nuevo Pueblo de Dios, llamada a converger hacia el lugar de encuentro y conversación con el Señor. También a vosotros hoy se os dirige la invitación a levantar los ojos a Cristo exaltado en la cruz, a encontrar en él la salvación, la esperanza, la fuerza para caminar en el árido y traicionero desierto de las dificultades de la vida. Dios, rico en misericordia, también se vuelve constantemente hacia vosotros para realizar la salvación entre vosotros y para vosotros. Este es también el objetivo del trabajo asiduo y diario de vuestra comunidad parroquial \ldots

\txtsmall{[\ldots]

[(\ldots) Deseo dirigir un pensamiento particular, con profunda satisfacción, a todos los jóvenes novios y matrimonios que, con interés y con un vivo deseo de descubrir cada vez mejor el plan de Dios para su familia, asisten a las reuniones de preparación al Sacramento del matrimonio, o continúan, en grupos familiares, reflexionando sobre el sentido cristiano de la vocación conyugal, dando vida así a un proceso de confraternización del que espera mucho la vida de la parroquia.

Finalmente, un pensamiento de agradecimiento va dirigido a todos los sacerdotes y laicos que, con celo y espíritu de amistad, se prestan a dar vida a las iniciativas apostólicas de este territorio, encomendado a la parroquia de San Pedro Damiani.]}

9. La liturgia de hoy es también una voz de esperanza para las dificultades que toda comunidad cristiana encuentra en su camino. Dios, rico en misericordia, se dirige, de hecho, a todo hombre, incluso al más lejano, para invitarle a acoger la salvación que de él procede. Se dirige a los jóvenes para iluminar su camino, invitándolos a descubrir en la mirada de Cristo y en el conocimiento de su amor la verdad sobre los valores fundamentales de la vida. Cristo invita a los jóvenes a descubrir su misión en el avance del verdadero bien para la sociedad y los insta a un compromiso vigoroso con el servicio social.

Cristo os invita, queridos fieles, a incluir la experiencia del amor, la fraternidad y la participación en la estructura de vuestra comunidad social y parroquial. Sólo el camino del afecto y la aceptación de los principios de la fe puede desviar al hombre de esas formas de egoísmo que muchas veces se vuelven contra el hombre mismo y lo destruyen haciéndole olvidar las necesidades y sufrimientos de sus hermanos.

Queridos fieles [de San Pedro Damiani], sed generosos al poner a disposición vuestros recursos y vuestro tiempo para trabajar juntos y construir juntos la comunidad de la que formáis parte. Tenemos que salir de toda tentación del aislamiento y trabajar con las manos, con el corazón, con la inteligencia, con la fe, para construir el nuevo rostro de la sociedad: un rostro desde el que se pueda captar el deseo, el anuncio y la esperanza de una atmósfera de mayor seguridad y confianza mutua. Sabéis bien que el futuro de la humanidad no se construye sobre el odio, la violencia o la opresión, sea cual sea; tampoco se construye sobre el vacío de ideas o en la ausencia de fe. La humanidad no se construye sobre el egoísmo individual o colectivo, ni sobre una falsa concepción de la libertad. Seguid el camino de Cristo, enviado por el Padre para que el mundo se salve por él, con el don de su gracia.

10. \textquote{En efecto, por gracia estáis salvados, mediante la fe. Y esto no viene de vosotros, sino que es un don de Dios}, leemos hoy en la \textbf{carta a los Efesios} (\textit{Ef} 2, 8).

Es un don, un regalo de Dios. Sí. Completamente, \textquote{gratuitamente} entregado. Este don no tiene otra explicación, ninguna otra \textquote{tapadera}, sino solo amor. El \textquote{precio} de este don sólo lo conoce Él, el Hijo enviado por el Padre, Cristo crucificado. Él solo, único, lo \textquote{conoce} a fondo. Él \textquote{trajo} este amor al mundo porque \textquote{cargó} con su precio, sobreabundantemente.

Meditemos sobre este misterio inescrutable especialmente ahora: en el período de Cuaresma. En torno a él, maduremos nuestras mentes, nuestros corazones, nuestras conciencias.
\end{body}

\newpage

\subsubsection{Homilía (1994): Dios rico en misericordia}

\src{Visita Pastoral a la Parroquia Romana de San Francisco de Sales.\par 13 de marzo de 1994.}

\begin{body}
\ltr[1. ]{L}{a} palabra clave en la Liturgia de la Palabra de hoy es esta: \textquote{\textit{Dives in misericordia}} – Dios rico en misericordia. [Ha sido una palabra clave para mí desde el comienzo de mi pontificado, de mi ministerio aquí en Roma.]

Esta inspiración llegó a [mi tierra natal,] Cracovia, a través de una sencilla monja llamada Faustina, probablemente [también conocida en Roma,] conocida en todo el mundo, aunque vivió muy escondida en Cristo. Vivió entre las dos guerras mundiales. Es una de las más grandes místicas de la historia de la Iglesia. Tenía una maravillosa cercanía con Jesús, y Jesús se le reveló misericordioso. Hay cuadros, imágenes de Jesús misericordioso con el corazón traspasado.

Viniendo de ese entorno, traje aquí una inspiración, casi un deber: no puedes dejar de escribir sobre la misericordia. Así nació la segunda encíclica de mi pontificado: \textit{Dives in misericordia} – Dios rico en misericordia.

2. ¿Quién es este Dios \textquote{rico en misericordia}? Jesús nos explica esto en la conversación nocturna con Nicodemo que acabamos de escuchar en el \textbf{Evangelio}. Dice así: \textquote{Tanto amó Dios al mundo que le entregó a su Hijo unigénito, para que todo el que crea en él no se pierda, sino que tenga vida eterna. Dios no envió a su Hijo al mundo para juzgar al mundo, sino para que el mundo se salve por él} (\textit{Jn} 3, 16-17). Estas palabras tan profundas, que casi constituyen el núcleo de la Revelación cristiana, se ilustran luego con un ejemplo, una parábola particular de la vida de Moisés.

Sabemos bien que Moisés fue un gran líder, un gran profeta de su pueblo. Guió a los judíos de la esclavitud de Egipto a la Tierra Prometida a través del desierto durante cuarenta años. En este viaje, en este itinerario hubo una etapa muy difícil: las serpientes atacaron a los judíos y empezaron a morir. Entonces, en ese momento, Moisés recibe de Dios, de Yavhé, una orden: \textquote{Hazte una serpiente y ponla en un mástil; el que la mire después de ser mordido, vivirá} (\textit{Núm} 21, 8).

Es un símbolo, es más que un símbolo: es la profecía de la Cruz. Quien mire a Jesús Crucificado será sanado físicamente pero, sobre todo, espiritualmente salvado de sus pecados.

Este es el mensaje central y constitutivo del Evangelio.

3. Después de estas palabras, Jesús, en el \textbf{Evangelio de San Juan}, añade otras palabras muy significativas, muy cuaresmales. Jesús habla de caminar en tinieblas o caminar en luz. Dice: debemos caminar en la luz. El que camina en la luz, el que se acerca a la luz, hace buenas obras. Si sus obras son buenas, quiere estar en la luz. Si sus obras no son buenas, son malas, vergonzosas, entonces no quiere aparecer, no quiere estar en la luz, quiere permanecer en la oscuridad.

Estas palabras son casi una indicación clásica de lo que debemos hacer durante la Cuaresma. Debemos acercarnos a la luz de nuestra conciencia, a la luz del Sacramento de la Reconciliación, de la Penitencia. Tenemos que acercarnos para caminar en la luz.

4. Este es el tercer elemento de la Palabra de Dios hoy. Queda un elemento, el cuarto, que me gustaría enfatizar para una circunstancia particular.

\txtsmall{[Pasado mañana, 15 de marzo, debemos comenzar lo que ya he invitado tantas veces a Italia, la Iglesia que está en este país: una gran oración por Italia. Debemos comenzar esta oración en la Basílica de San Pedro, cerca de la tumba de San Pedro, en sus reliquias, donde comenzó la historia de este país, de la Iglesia en este país: las dos cosas van juntas. Debemos comenzar esta oración allí con los obispos italianos, con nuestro cardenal, con todos los obispos. Debemos comenzar en el nombre de la Alianza que Dios hizo una vez con su pueblo.]}

Hoy la \textbf{primera lectura} nos recuerda la antigua Alianza que Dios estableció con su pueblo Israel. Dios fue fiel a esta Alianza. Cuando Israel fue infiel, castigó a Israel. Después de sesenta años en Babilonia, Dios le dio la orden a Ciro, rey de Persia, de dejar a los israelitas libres para reconstruir el templo, porque Dios es libertador, Dios es fiel a sus promesas, a su Alianza.

Estoy convencido –y todos los obispos, todos los cristianos en Italia están convencidos de ello– que Dios ha sellado esta Alianza con toda la humanidad, con todos los pueblos, con todas las naciones, también con este pueblo, [con la nación italiana, durante dos mil años]. Ahora debemos invocar esta fidelidad de Dios, debemos ser fieles a Él, [como italianos,] como Pueblo que ha sido objeto de una gran elección, [que ha recibido a los más grandes Apóstoles, Pedro y Pablo. Habrá varias etapas de esta oración a lo largo del año, desde el 15 de marzo hasta la fiesta de Nuestra Señora de Loreto, el 10 de diciembre.]

5. \txtsmall{[Estas son las cosas que quería deciros. Saludo una vez más a todos los presentes, como hice antes de la misa. Saludo también a todos los miembros de vuestra comunidad. Todavía no teneís una iglesia, un templo construido con piedras. Pero vosotros sois \textquote{piedras vivas}. Y tenemos la esperanza, junto con vuestro pastor, de que algún día también tendréis la iglesia.]}

¡Alabado sea Jesucristo!
\end{body}

\newpage

\subsubsection{Ángelus (1997): Esperanza, no consuelos baratos}

\src{9 de marzo de 1997.}

\begin{body}
\ltr[1. ]{A}{} mitad de nuestro camino cuaresmal, en este cuarto domingo de Cuaresma, se nos invita a meditar sobre un tema que está en el centro del anuncio cristiano, es decir, el gran amor que Dios siente por la humanidad. En el \textbf{evangelio} de hoy leemos: \textquote{Tanto amó Dios al mundo, que entregó a su Hijo único, para que no perezca ninguno de los que creen en él, sino que tengan vida eterna} (\textit{Jn} 3, 16).

El hombre de nuestro tiempo, ¿siente la necesidad de este anuncio? A primera vista parecería que no, ya que, sobre todo en las manifestaciones públicas y en cierta cultura dominante, emerge la imagen de una humanidad segura de sí misma, que prescinde tranquilamente de Dios, reivindicando también una libertad absoluta contra la ley moral.

2. Pero cuando miramos de cerca la realidad de cada persona, obligada a confrontarse con su fragilidad y su soledad, nos damos cuenta de que, mucho más de lo que se cree, los ánimos están dominados por la angustia, por el ansia ante el futuro, y por el miedo a la enfermedad y a la muerte. Esto explica por qué tantas personas, buscando un camino de salida, toman a veces atajos aberrantes, como por ejemplo el túnel de la droga o el de supersticiones y ritos mágicos desconcertantes.

El cristianismo no ofrece consuelos baratos, porque es exigente cuando pide una fe auténtica y una vida moral rigurosa. Pero nos da motivo de esperanza, al indicarnos a Dios como Padre rico en misericordia, que nos ha entregado a su Hijo, mostrándonos así su inmenso amor.

3. Que María, Madre de misericordia, ponga en nuestro corazón la certidumbre de que Dios nos ama. Que ella esté cercana a nosotros en los momentos en que nos sentimos solos, cuando tenemos la tentación de rendirnos ante las dificultades de la vida, y nos inculque los sentimientos de su Hijo divino, para que nuestro itinerario cuaresmal se convierta en experiencia de perdón, de acogida y de caridad.
\end{body}

\newpage

\subsubsection{Ángelus (2003): El amor de Dios}

\src{23 de marzo del 2003.}

\begin{body}
\ltr[1. ]{H}{oy,} IV domingo de Cuaresma, el \textbf{evangelio} nos recuerda que Dios \textquote{amó tanto al mundo, que entregó a su Hijo único, para que no perezca ninguno de los que creen en él, sino que tengan vida eterna} (\textit{Jn} 3, 16).

[Escuchamos este anuncio consolador en un momento en el que dolorosos enfrentamientos armados amenazan la esperanza de la humanidad en un futuro mejor.] Dios \textquote{amó tanto al mundo\ldots}, afirma Jesús. Por tanto, el amor del Padre llega a todo ser humano que vive en el mundo. ¿Cómo no ver el compromiso que brota de esa iniciativa de Dios? El ser humano, consciente de un amor tan grande, no puede menos de abrirse a una actitud de acogida fraterna con respecto a sus semejantes.

2. Dios \textquote{amó tanto al mundo, que entregó a su Hijo único\ldots}. Eso es lo que sucedió en el sacrificio del Calvario: Cristo murió y resucitó por nosotros, sellando con su sangre la nueva y definitiva alianza con la humanidad. El sacramento de la Eucaristía es el memorial perenne de este supremo testimonio de amor. En él Jesús, Pan de vida y verdadero \textquote{maná}, sostiene a los creyentes en el camino a través del \textquote{desierto} de la historia hacia la \textquote{tierra prometida} del cielo (cf. \textit{Jn} 6, 32-35).

3. \txtsmall{[Precisamente al tema de la Eucaristía he dedicado la encíclica que, con ocasión del próximo Jueves santo, Dios mediante, firmaré durante la misa in \textit{cena Domini}. La entregaré simbólicamente a los sacerdotes en lugar de la carta que para esa circunstancia suelo dirigirles, y, a través de ellos, a todo el pueblo de Dios. Encomiendo desde ahora a María este importante documento, que recuerda el valor intrínseco y la importancia que tiene para la Iglesia el sacramento que nos dejó Jesús como memorial vivo de su muerte y resurrección, y de nuestra redención.]}

Nos dirigimos también a María, pidiéndole una vez más por las víctimas de los conflictos actuales. Invocamos con ferviente y confiada insistencia su intercesión por la paz [en Irak] y en todas las demás regiones del mundo.
\end{body}



\newsection
\subsection{Benedicto XVI, papa}

\subsubsection{Homilía (2006): Laetare, ¿por qué hemos de alegrarnos?}

\src{Visita Pastoral a la Parroquia Romana de Dios, Padre Misericordioso.\\26 de marzo del 2006.}

\begin{body}
\ltr{E}{ste} IV domingo de Cuaresma, tradicionalmente designado como \textquote{domingo \textit{Laetare}}, está impregnado de una alegría que, en cierta medida, atenúa el clima penitencial de este tiempo santo: \textquote{Alégrate Jerusalén –dice la Iglesia en la antífona de entrada–, (\ldots) gozad y alegraos vosotros, que por ella estabais tristes}. De esta invitación se hace eco el estribillo del \textbf{salmo responsorial}: \textquote{El recuerdo de ti, Señor, es nuestra alegría}. Pensar en Dios da alegría. Surge espontáneamente la pregunta: pero ¿cuál es el motivo por el que debemos alegrarnos? Desde luego, un motivo es la cercanía de la Pascua, cuya previsión nos hace gustar anticipadamente la alegría del encuentro con Cristo resucitado. Pero la razón más profunda está en el mensaje de las lecturas bíblicas que la liturgia nos propone hoy y que acabamos de escuchar. Nos recuerdan que, a pesar de nuestra indignidad, somos los destinatarios de la misericordia infinita de Dios. Dios nos ama de un modo que podríamos llamar \textquote{obstinado}, y nos envuelve con su inagotable ternura.

Esto es lo que resalta ya en la \textbf{primera lectura}, tomada del libro de las Crónicas del Antiguo Testamento (cf. \textit{2 Cr} 36, 14-16. 19-23): el autor sagrado propone una interpretación sintética y significativa de la historia del pueblo elegido, que experimenta el castigo de Dios como consecuencia de su comportamiento rebelde: el templo es destruido y el pueblo, en el exilio, ya no tiene una tierra; realmente parece que Dios se ha olvidado de él. Pero luego ve que a través de los castigos Dios tiene un plan de misericordia. Como hemos dicho, la destrucción de la ciudad santa y del templo, y el exilio, tocarán el corazón del pueblo y harán que vuelva a su Dios para conocerlo más a fondo. Y entonces el Señor, demostrando el primado absoluto de su iniciativa sobre cualquier esfuerzo puramente humano, se servirá de un pagano, Ciro, rey de Persia, para liberar a Israel. En el texto que hemos escuchado, la ira y la misericordia del Señor se confrontan en una secuencia dramática, pero al final triunfa el amor, porque Dios es amor. ¿Cómo no recoger, del recuerdo de aquellos hechos lejanos, el mensaje válido para todos los tiempos, incluido el nuestro? Pensando en los siglos pasados podemos ver cómo Dios sigue amándonos incluso a través de los castigos. Los designios de Dios, también cuando pasan por la prueba y el castigo, se orientan siempre a un final de misericordia y de perdón.

Eso mismo nos lo ha confirmado, en la \textbf{segunda lectura}, el apóstol san Pablo, recordándonos que \textquote{Dios, rico en misericordia, por el gran amor con que nos amó, estando nosotros muertos por los pecados, nos ha hecho vivir con Cristo} (\textit{Ef} 2, 4-5). Para expresar esta realidad de salvación, el Apóstol, además del término \textquote{misericordia}, \textit{eleos}, utiliza también la palabra \textquote{amor}, \textit{agape}, recogida y amplificada ulteriormente en la bellísima afirmación que hemos escuchado en la \textbf{página evangélica}: \textquote{Tanto amó Dios al mundo, que entregó a su Hijo único, para que no perezca ninguno de los que creen en él, sino que tengan vida eterna} (\textit{Jn} 3, 16).

Sabemos que esa \textquote{entrega} por parte del Padre tuvo un desenlace dramático: llegó hasta el sacrificio de su Hijo en la cruz. Si toda la misión histórica de Jesús es signo elocuente del amor de Dios, lo es de modo muy singular su muerte, en la que se manifestó plenamente la ternura redentora de Dios. Por consiguiente, siempre, pero especialmente en este tiempo cuaresmal, la cruz debe estar en el centro de nuestra meditación; en ella contemplamos la gloria del Señor que resplandece en el cuerpo martirizado de Jesús. Precisamente en esta entrega total de sí se manifiesta la grandeza de Dios, que es amor.

Todo cristiano está llamado a comprender, vivir y testimoniar con su existencia la gloria del Crucificado. La cruz –la entrega de sí mismo del Hijo de Dios– es, en definitiva, el \textquote{signo} por excelencia que se nos ha dado para comprender la verdad del hombre y la verdad de Dios: todos hemos sido creados y redimidos por un Dios que por amor inmoló a su Hijo único. Por eso, como escribí en la encíclica \textit{Deus caritas est}, en la cruz \textquote{se realiza ese ponerse Dios contra sí mismo, al entregarse para dar nueva vida al hombre y salvarlo: esto es amor en su forma más radical} (n. 12). ¿Cómo responder a este amor radical del Señor? El \textbf{evangelio} nos presenta a un personaje de nombre Nicodemo, miembro del Sanedrín de Jerusalén, que de noche va a buscar a Jesús. Se trata de un hombre de bien, atraído por las palabras y el ejemplo del Señor, pero que tiene miedo de los demás, duda en dar el salto de la fe. Siente la fascinación de este Rabbí, tan diferente de los demás, pero no logra superar los condicionamientos del ambiente contrario a Jesús y titubea en el umbral de la fe.

¡Cuántos, también en nuestro tiempo, buscan a Dios, buscan a Jesús y a su Iglesia, buscan la misericordia divina, y esperan un \textquote{signo} que toque su mente y su corazón! Hoy, como entonces, el evangelista nos recuerda que el único \textquote{signo} es Jesús elevado en la cruz: Jesús muerto y resucitado es el signo absolutamente suficiente. En él podemos comprender la verdad de la vida y obtener la salvación. Este es el anuncio central de la Iglesia, que no cambia a lo largo de los siglos. Por tanto, la fe cristiana no es ideología, sino encuentro personal con Cristo crucificado y resucitado. De esta experiencia, que es individual y comunitaria, surge un nuevo modo de pensar y de actuar: como testimonian los santos, nace una existencia marcada por el amor. [\ldots]

Dirigiendo la mirada a María, \textquote{Madre de la santa alegría}, pidámosle que nos ayude a profundizar las razones de nuestra fe, para que, como nos exhorta la liturgia hoy, renovados en el espíritu y con corazón alegre correspondamos al amor eterno e infinito de Dios. Amén.
\end{body}

\label{b2-03-04-2006H}

\begin{patercite}
	\textbf{Junto a los canales de Babilonia}
	
	(\ldots) El
	salmo 136 se ha hecho célebre en la versión latina de su inicio, \emph{Super flumina Babylonis}. El texto evoca la tragedia que vivió el
	pueblo judío durante la destrucción de Jerusalén, acaecida en el año 586 a.C., y el sucesivo y consiguiente destierro en Babilonia. Se trata de un canto nacional de dolor, marcado por una profunda nostalgia por lo que se había perdido.
	
	Esta apremiante invocación al Señor para que libre a sus fieles de la esclavitud babilónica expresa también los sentimientos de esperanza y espera de la salvación (\ldots).
	
	La primera parte del Salmo (cf. vv. 1-4) tiene como telón de fondo la tierra del destierro, con sus ríos y canales, que regaban la llanura de Babilonia, sede de los judíos deportados. Es casi la anticipación simbólica de los campos de concentración, en los que el pueblo judío ---en el siglo que acaba de concluir--- sufrió una operación infame de muerte, que ha quedado como una vergüenza indeleble en la historia de la humanidad.
	
	La  segunda  parte  del Salmo (cf. vv. 5-6), por el contrario, está impregnada del recuerdo amoroso de Sión, la ciudad perdida pero viva en el corazón de los desterrados.
	
	En sus palabras, el salmista se refiere a la mano, la lengua, el paladar, la voz y las lágrimas. La mano es indispensable para el músico que toca la cítara, pero está paralizada (cf. v. 5) por el dolor, entre otras causas porque las cítaras están colgadas de los sauces.
	
	La lengua es necesaria para el cantor, pero está pegada al paladar (cf. v. 6). En vano los verdugos babilonios \textquote{los invitan a cantar, para divertirlos} (cf. v. 3). Los \textquote{cantos de Sión} son \textquote{cantos del Señor} (vv. 3-4); no son canciones folclóricas, para espectáculo. Sólo pueden elevarse al cielo en la liturgia y en la libertad de un pueblo.
	
	Dios, que es el árbitro último de la historia, sabrá comprender y acoger según su justicia también el grito de las víctimas, por encima de los graves acentos que a veces asume.
	
	[\ldots]
	
	\textbf{Benedicto XVI, papa}, \textit{Catequesis} sobre el Salmo 136, 30 de noviembre de 2005, nn. 1-2.
\end{patercite}

\newpage

\subsubsection{Homilía (2009): Reconstruir}

\src{Viaje Apostólico a Camerún y Angola. \\Celebración Eucarística con los Obispos de la IMBISA \\(Asamblea Interregional de Obispos de África del Sur). \\Explanada de Cimangola, Luanda. 22 de marzo de 2009.}

\begin{body}
\textquote{Tanto amó Dios al mundo que entregó a su Hijo único, para que no perezca ninguno de los que creen en él, sino que tengan vida eterna} (\textit{Jn} 3, 16). Estas palabras nos colman de gozo y esperanza, pues anhelamos el cumplimiento de las promesas de Dios. 

\txtsmall{[Para mí es hoy un motivo de alegría celebrar como Sucesor del Apóstol Pedro esta Misa con vosotros, mis hermanos y hermanas en Cristo, que venís de diversas regiones (\ldots)]}

\ltr{L}{a} \textbf{primera lectura} de hoy tiene una resonancia particular para el Pueblo de Dios [en Angola]. Es un mensaje de esperanza para el Pueblo elegido en la lejanía de su destierro, una invitación a volver a Jerusalén para reconstruir el Templo del Señor. [La descripción vibrante de la destrucción y la ruina causada por la guerra refleja la experiencia personal de muchos en este País durante las terribles devastaciones de la guerra civil.] Qué verdad es el que la guerra puede destruir \textquote{todo lo que tiene valor} (cf. \textit{2 Cr} 36, 19): familias, comunidades enteras, el fruto de la fatiga de los hombres, las esperanzas que guían y alientan sus vidas y su trabajo. [Esta experiencia es demasiado familiar en el conjunto de África: el poder destructivo de la guerra civil, el caer en el torbellino del odio y la venganza, el despilfarro de los esfuerzos de generaciones de gente de bien.] Cuando se descuida la Palabra del Señor –una Palabra que tiende a la edificación de las personas, de las comunidades y de toda la familia humana–, y la Ley de Dios es objeto de \textquote{burla, desprecio y escarnio} (cf. ibíd., v. 16), el resultado sólo puede ser destrucción e injusticia, deshonra de nuestra común humanidad y traición de nuestra vocación a ser hijos e hijas del Padre misericordioso, hermanos y hermanas de su Hijo predilecto.

Nos confortan, pues, las palabras consoladoras que hemos escuchado en la \textbf{primera lectura}. La llamada a volver y a reconstruir el Templo de Dios tiene un significado particular para todos nosotros. San Pablo, [de cuyo nacimiento celebramos este año el bimilenario,] nos dice que \textquote{somos santuario del Dios vivo} (2 \textit{Co} 6, 16). Como sabemos, Dios habita en el corazón de los que ponen su confianza en Cristo, han renacido en el Bautismo y se han convertido en templo del Espíritu Santo. También ahora, en la unidad del Cuerpo de Cristo, que es la Iglesia, Dios nos llama a reconocer en nosotros la fuerza de su presencia, a acoger de nuevo el don de su amor y su perdón, y a convertirnos en mensajeros de este amor misericordioso en nuestras familias y comunidades, en la escuela, el trabajo y en cada sector de la vida social y política.

[Aquí en Angola, este domingo ha sido declarado como día de oración y sacrificio por la reconciliación nacional.] El \textbf{Evangelio} nos enseña que la reconciliación –una verdadera reconciliación– sólo puede ser fruto de una conversión, de una transformación del corazón, de un nuevo modo de pensar. Nos enseña que sólo la fuerza del amor de Dios puede cambiar nuestros corazones y hacernos triunfar sobre el poder del pecado y la división. Cuando estábamos \textquote{muertos por nuestros pecados} (cf. \textit{Ef} 2, 5), su amor y su misericordia nos han ofrecido la reconciliación y la vida nueva en Cristo. Éste es el núcleo de la enseñanza del apóstol Pablo, y es importante para nosotros volver a traer a la memoria que sólo la gracia de Dios puede crear en nosotros un corazón nuevo. Sólo su amor puede cambiar nuestro \textquote{corazón de piedra} (\textit{Ez} 11, 19) y hacernos capaces de construir, en lugar de demoler. Sólo Dios puede hacer nuevas todas las cosas.

\txtsmall{[He venido a África precisamente para predicar este mensaje de perdón, de esperanza y de una vida nueva en Cristo. Hace tres días, en Yaundé, he tenido la alegría de hacer público el textit{Instrumentum laboris} de la Segunda Asamblea Especial para África del Sínodo de los Obispos, que estará dedicada al tema: \textit{La Iglesia en África al servicio de la reconciliación, la justicia y la paz}. Hoy os pido que recéis, junto con nuestros hermanos y hermanas de toda África, por esta intención: que todo cristiano en este gran Continente sienta el toque saludable del amor misericordioso de Dios, y que la Iglesia en África sea \textquote{gracias al testimonio ofrecido por sus hijos e hijas, lugar de auténtica reconciliación} (\textit{Ecclesia in Africa}, 79).]}

\txtsmall{[Queridos amigos, éste es el mensaje que el Papa os dirige a vosotros y a vuestros hijos.]} Habéis recibido del Espíritu Santo la fuerza de ser los constructores de un porvenir mejor para vuestro querido País. En el Bautismo se os ha dado el Espíritu para ser heraldos del Reino de Dios, reino de la verdad y la vida, de la santidad y la gracia, de la justicia, el amor y la paz (cf. \textit{Misal Romano}, Jesucristo, Rey del universo, Prefacio). El día de vuestro Bautismo habéis recibido la luz de Cristo. Sed fieles a este don, con la certeza de que el Evangelio puede confirmar, purificar y ennoblecer los profundos valores humanos que hay en vuestra cultura nativa y en vuestras tradiciones: familias unidas, profundo sentido religioso, alegre celebración del don de la vida, estima por la sabiduría de los ancianos y por las aspiraciones de los jóvenes. Y agradeced también la luz de Cristo. Mostrad vuestro reconocimiento a quienes os la han traído: \txtsmall{[generaciones y generaciones de misioneros que tanto han contribuido y siguen contribuyendo al desarrollo humano y espiritual de este País.]} Agradeced el testimonio de tantos padres y maestros cristianos, catequistas, sacerdotes, religiosas y religiosos, que han sacrificado su propia vida para transmitiros este precioso tesoro. Asumid el reto que representa este gran patrimonio. Tened presente que la Iglesia [en Angola y en toda África,] tiene la tarea de ser ante el mundo un signo de esa unidad a la que, a través de la fe en Cristo redentor, está llamada toda la familia humana.

\newpage 
En el \textbf{Evangelio} de hoy hay palabras de Jesús que suscitan una cierta impresión: Él nos dice que ya se ha dictado la sentencia de Dios sobre el mundo (cf. \textit{Jn} 3, 19 ss). La luz ha venido al mundo. Pero los hombres han preferido las tinieblas a la luz, porque sus obras eran malas. Cuántas tinieblas hay en tantas partes del mundo. Las nubes del mal han oscurecido trágicamente [también África, incluida esta amada Nación de Angola. Pensemos en el drama de la guerra, en las feroces consecuencias del tribalismo y las rivalidades étnicas, en la codicia que corrompe el corazón del hombre, esclaviza a los pobres y priva a las generaciones futuras de los recursos que necesitan para crear una sociedad más solidaria y más justa, una sociedad real y auténticamente africana en su genio y en sus valores.] Y ¿qué decir de ese insidioso espíritu de egoísmo que encierra a las personas en sí mismas, divide las familias y, suplantando los grandes ideales de generosidad y abnegación, lleva inevitablemente al hedonismo, a la evasión en falsas utopías mediante el uso de la droga, a la irresponsabilidad sexual, al debilitamiento de la unión matrimonial, a la destrucción de las familias y la eliminación de vidas humanas inocentes por el aborto?

Sin embargo, la palabra de Dios es una palabra de esperanza sin límites. En efecto, \textquote{tanto amó Dios al mundo que entregó a su Hijo único\ldots para que el mundo se salve por él} (\textit{Jn} 3, 16-17). Dios nunca nos considera desahuciados. Él sigue invitándonos a levantar los ojos hacia un futuro de esperanza y nos promete la fuerza para conseguirlo. Como dice San Pablo en la \textbf{segunda lectura} de hoy, Dios nos ha creado en Cristo Jesús para vivir una vida justa, una vida en que hagamos buenas obras según su voluntad (cf. \textit{Ef} 2, 10). Nos ha dado sus mandamientos, no como una rémora, sino como un manantial de libertad: libertad para ser hombres y mujeres llenos de sabiduría, maestros de justicia y paz, gente que tiene confianza en los otros y busca su auténtico bien. Dios nos ha creado para vivir en la luz y para ser luz del mundo que nos rodea. Esto es lo que Jesús nos dice en el \textbf{Evangelio de hoy}: \textquote{El que realiza la verdad, se acerca a la luz, para que se vea que sus obras están hechas según Dios} (\textit{Jn} 3, 21).

\textquote{Vivid, pues, conforme a la verdad}. Irradiad la luz de la fe, la esperanza y el amor en vuestras familias y comunidades. Sed testigos de la santa verdad que hace libres a los hombres y las mujeres. Sabéis por una amarga experiencia que, tras la repentina furia destructora del mal, el trabajo de reconstrucción es penosamente lento y duro. Requiere tiempo, esfuerzo y perseverancia: debe comenzar en nuestros corazones, en los pequeños sacrificios cotidianos necesarios para ser fieles a la ley de Dios, en los pequeños gestos mediante los cuales demostramos amar a nuestros prójimos –todos ellos, sin distinción de raza, etnia o lengua– con la disponibilidad de colaborar con ellos para construir juntos sobre fundamentos duraderos. Haced que vuestras parroquias se conviertan en comunidades donde la luz de la verdad de Dios y el poder del amor reconciliador de Cristo no solamente se celebren, sino que también se manifiesten en obras concretas de caridad. No tengáis miedo. Aunque esto signifique ser un \textquote{signo de contradicción} (\textit{Lc} 2, 34) frente a actitudes duras y una mentalidad que considera a los otros como instrumentos para usar, en vez de como hermanos y hermanas a los que amar, respetar y ayudar a lo largo del camino de la libertad, la vida y la esperanza.

Permitidme concluir con una palabra dirigida particularmente a los jóvenes [de Angola y a todos los jóvenes de África]. Queridos jóvenes amigos, vosotros sois la esperanza del futuro de vuestro País, la promesa de un mañana mejor. Comenzad a crecer desde hoy en vuestra amistad con Jesús, que es \textquote{el camino, y la verdad, y la vida} (\textit{Jn} 14, 6): una amistad alimentada y profundizada por la oración humilde y perseverante. Buscad su voluntad sobre vosotros, escuchando cotidianamente su palabra y dejando que su ley modele vuestra vida y vuestras relaciones. De este modo os convertiréis en profetas sabios y generosos del amor salvador de Dios; llegaréis a ser evangelizadores de vuestros propios compañeros, llevándolos con vuestro ejemplo personal a que aprecien la belleza y la verdad del Evangelio, y a encaminarse por la esperanza de un futuro plasmado por los valores del Reino de Dios. La Iglesia necesita vuestro testimonio. No tengáis miedo de responder generosamente a la llamada de Dios para servirlo, bien como sacerdotes, religiosas o religiosos, bien como padres cristianos o en tantas otras formas de servicio que la Iglesia os propone.

Queridos hermanos y hermanas, al final de la \textbf{primera lectura} de hoy, Ciro, rey de Persia, inspirado por Dios, ordena al Pueblo elegido que vuelva a su querida Patria y reconstruya el Templo del Señor. Que estas palabras del Señor sean una llamada [para todo el Pueblo de Dios en Angola y en toda África del Sur]: Levantaos, poneos en camino (cf. \textit{2 Cr} 36, 23). Mirad al futuro con esperanza, confiad en las promesas de Dios y vivid en su verdad. De este modo construiréis algo destinado a permanecer, y dejaréis a las generaciones futuras una herencia duradera de reconciliación, de justicia y de paz. Amén.
\end{body}

\label{b2-03-04-2012A}
\newpage

\subsubsection{Ángelus (2012): Jesús será levantado en la cruz}

\src{18 de marzo del 2012.}

\begin{body}
\ltr{E}{n} nuestro itinerario hacia la Pascua, hemos llegado al cuarto domingo de Cuaresma. Es un camino con Jesús a través del \textquote{desierto}, es decir, un tiempo para escuchar más la voz de Dios y también para desenmascarar las tentaciones que hablan dentro de nosotros. En el horizonte de este desierto se vislumbra la cruz. Jesús sabe que la cruz es el culmen de su misión: en efecto, la cruz de Cristo es la cumbre del amor, que nos da la salvación. Lo dice él mismo en el \textbf{Evangelio} de hoy: \textquote{Lo mismo que Moisés elevó la serpiente en el desierto, así tiene que ser elevado el Hijo del hombre, para que todo el que cree en él tenga vida eterna} (\textit{Jn} 3, 14-15). Se hace referencia al episodio en el que, durante el éxodo de Egipto, los judíos fueron atacados por serpientes venenosas y muchos murieron; entonces Dios ordenó a Moisés que hiciera una serpiente de bronce y la pusiera sobre un estandarte: si alguien era mordido por las serpientes, al mirar a la serpiente de bronce, quedaba curado (cf. \textit{Núm} 21, 4-9). También Jesús será levantado sobre la cruz, para que todo el que se encuentre en peligro de muerte a causa del pecado, dirigiéndose con fe a él, que murió por nosotros, sea salvado. \textquote{Porque Dios –escribe san Juan– no envió a su Hijo al mundo para juzgar al mundo, sino para que el mundo se salve por él} (\textit{Jn} 3, 17).

San Agustín comenta: \textquote{El médico, en lo que depende de él, viene a curar al enfermo. Si uno no sigue las prescripciones del médico, se perjudica a sí mismo. El Salvador vino al mundo\ldots Si tú no quieres que te salve, te juzgarás a ti mismo} (\textit{Sobre el Evangelio de Juan}, 12, 12: PL 35, 1190). Así pues, si es infinito el amor misericordioso de Dios, que llegó al punto de dar a su Hijo único como rescate de nuestra vida, también es grande nuestra responsabilidad: cada uno, por tanto, para poder ser curado, debe reconocer que está enfermo; cada uno debe confesar su propio pecado, para que el perdón de Dios, ya dado en la cruz, pueda tener efecto en su corazón y en su vida. Escribe también san Agustín: \textquote{Dios condena tus pecados; y si también tú los condenas, te unes a Dios\ldots Cuando comienzas a detestar lo que has hecho, entonces comienzan tus buenas obras, porque condenas tus malas obras. Las buenas obras comienzan con el reconocimiento de las malas obras} (\textit{ib}., 13: PL 35, 1191). A veces el hombre ama más las tinieblas que la luz, porque está apegado a sus pecados. Sin embargo, la verdadera paz y la verdadera alegría sólo se encuentran abriéndose a la luz y confesando con sinceridad las propias culpas a Dios. Es importante, por tanto, acercarse con frecuencia al sacramento de la Penitencia, especialmente en Cuaresma, para recibir el perdón del Señor e intensificar nuestro camino de conversión.
\end{body}

\newsection
\subsection{Francisco, papa}

\subsubsection{Ángelus (2015): Dios nos ama, nos creó a su imagen}

\src{15 de marzo del 2015.}

\begin{body}
\ltr{E}{l} \textbf{Evangelio} de hoy nos vuelve a proponer las palabras que Jesús dirigió a Nicodemo: \textquote{Tanto amó Dios al mundo, que entregó a su Unigénito} (\textit{Jn} 3, 16). Al escuchar estas palabras, dirijamos la mirada de nuestro corazón a Jesús Crucificado y sintamos dentro de nosotros que Dios nos ama, nos ama de verdad, y nos ama en gran medida. Esta es la expresión más sencilla que resume todo el Evangelio, toda la fe, toda la teología: Dios nos ama con amor gratuito y sin medida.

Así nos ama Dios y este amor Dios lo demuestra ante todo en la creación, como proclama la liturgia, en la Plegaria eucarística IV: \textquote{A imagen tuya creaste al hombre y le encomendaste el universo entero, para que, sirviéndote sólo a ti, su Creador, dominara todo lo creado}. En el origen del mundo está sólo el amor libre y gratuito del Padre. San Ireneo un santo de los primeros siglos escribe: \textquote{Dios no creó a Adán porque tenía necesidad del hombre, sino para tener a alguien a quien donar sus beneficios} (\textit{Adversus haereses}, IV, 14, 1). Es así, el amor de Dios es así.

Continúa así la Plegaria eucarística IV: \textquote{Y cuando por desobediencia perdió tu amistad, no lo abandonaste al poder de la muerte, sino que, compadecido, tendiste la mano a todos}. Vino con su misericordia. Como en la creación, también en las etapas sucesivas de la historia de la salvación destaca la gratuidad del amor de Dios: el Señor elige a su pueblo no porque se lo merezca, sino porque es el más pequeño entre todos los pueblos, como dice Él. Y cuando llega \textquote{la plenitud de los tiempos}, a pesar de que los hombres en más de una ocasión quebrantaron la alianza, Dios, en lugar de abandonarlos, estrechó con ellos un vínculo nuevo, en la sangre de Jesús –el vínculo de la nueva y eterna alianza–, un vínculo que jamás nada lo podrá romper.

San Pablo nos recuerda: \textquote{Dios, rico en misericordia, –nunca olvidarlo, es rico en misericordia– por el gran amor con que nos amó, estando nosotros muertos por los pecados, nos ha hecho revivir con Cristo} (\textit{Ef} 2, 4-5). La Cruz de Cristo es la prueba suprema de la misericordia y del amor de Dios por nosotros: Jesús nos amó \textquote{hasta el extremo} (\textit{Jn} 13, 1), es decir, no sólo hasta el último instante de su vida terrena, sino hasta el límite extremo del amor. Si en la creación el Padre nos dio la prueba de su inmenso amor dándonos la vida, en la pasión y en la muerte de su Hijo nos dio la prueba de las pruebas: vino a sufrir y morir por nosotros. Así de grande es la misericordia de Dios: Él nos ama, nos perdona; Dios perdona todo y Dios perdona siempre.

Que María, que es Madre de misericordia, nos ponga en el corazón la certeza de que somos amados por Dios; nos sea cercana en los momentos de dificultad y nos done los sentimientos de su Hijo, para que nuestro itinerario cuaresmal sea experiencia de perdón, acogida y caridad.
\end{body}

\label{b2-03-04-2015A}

\begin{patercite}
	¿Te das cuenta de cómo el diablo es vencido en aquello mismo en que antes había triunfado? En un árbol el diablo hizo caer a Adán; en un árbol derrotó Cristo al diablo. Aquel árbol hacía descender a la región de los muertos; éste, en cambio, hace volver de este lugar a los que a él habían descendido. Otro árbol ocultó la desnudez del hombre, después de su caída; éste, en cambio, mostró a todos, elevado en alto, al vencedor, también desnudo. Aquella primera muerte condenó a todos los que habían de nacer después de ella; esta segunda muerte resucitó incluso a los nacidos anteriormente a ella. ¿Quién podrá contar las hazañas de Dios? Una muerte se ha convertido en causa de nuestra inmortalidad: éstas son las obras esclarecidas de la cruz.
	
	¿Has entendido el modo y significado de esta victoria? Entérate ahora cómo esta victoria fue lograda sin esfuerzo ni sudor por nuestra parte. Nosotros no tuvimos que ensangrentar nuestras armas, ni resistir en la batalla, recibir heridas, ni tan siquiera vimos la batalla, y, con todo, obtuvimos la victoria; fue el Señor quien luchó y nosotros quienes hemos sido coronados. Por tanto, ya que la victoria es nuestra, imitando a los soldados, cantemos hoy, llenos de alegría, las alabanzas de esta victoria, alabemos al Señor, diciendo: \textquote{La muerte ha sido absorbida en la victoria. ¿Dónde está, muerte, tu victoria? ¿Dónde está, muerte, tu aguijón?}
	
	Éstos son los admirables beneficios de la cruz en favor nuestro: la cruz es el trofeo erigido contra los demonios, la espada contra el pecado, la espada con la que Cristo atravesó a la serpiente; la cruz es la voluntad del Padre, la gloria de su Hijo único, el júbilo del Espíritu Santo, el ornato de los ángeles, la seguridad de la Iglesia, el motivo de gloriarse de Pablo, la protección de los santos, la luz de todo el orbe.
	
	\textbf{San Juan Crisóstomo, obispo}, \textit{Homilía}, sobre el cementerio y la cruz, 2: PG 49, 396.
\end{patercite}

\newpage

\subsubsection{Ángelus (2018): Dios está cerca}

\src{11 de marzo del 2018.}

\begin{body}
\ltr{E}{n} este cuarto domingo de Cuaresma llamado domingo \textit{laetare}, o sea \textquote{alégrate}, la antífona de entrada de la liturgia eucarística nos invita a la alegría: \textquote{Alégrate Jerusalén, alegraos y regocijaos los que estáis tristes}. Así comienza la misa. ¿Cuál es el motivo de esta alegría? Es el gran amor de Dios por la humanidad, como nos lo indica el \textbf{Evangelio} de hoy: \textquote{Porque tanto amó Dios al mundo que envió a su Hijo único para que todo el que crea en él, no perezca sino que tenga vida eterna} (\textit{Jn} 3, 16). Estas palabras, pronunciadas por Jesús durante su diálogo con Nicodemo, sintetizan un tema que es el centro del anuncio cristiano: incluso cuando la situación parece desesperada, Dios interviene, ofreciendo al hombre la salvación y la alegría.

Dios en efecto, no se quedará apartado, sino más bien entra en la historia de la humanidad para animarla con su gracia y salvarla.

Estamos llamados a escuchar este anuncio, rechazando la tentación de estar seguros de nosotros mismos, de querer prescindir de Dios, de reclamar la libertad absoluta de Él y su Palabra. Cuando encontramos el coraje de reconocernos tal como somos, nos damos cuenta que estamos llamados a lidiar con nuestra fragilidad y nuestros límites y es necesario tener mucho coraje.

Entonces puede pasar que nos agobie la angustia, la ansiedad por el mañana, el miedo a la enfermedad y a la muerte. Esto explica por qué muchas personas, en busca de una salida, a veces toman atajos peligrosos como el túnel de las drogas o de supersticiones o de rituales ruinosos de magia. Es bueno conocer los propios límites, las propias fragilidades, no para desesperar, sino para ofrecerlas al Señor; Él nos ayuda en el camino de la curación y nos lleva de la mano, nunca nos deja solos y por esto nos alegramos hoy, porque Dios está con nosotros.

Tenemos la verdadera y gran esperanza en Dios Padre rico en misericordia, que nos ha dado a su Hijo para salvarnos, y esa es nuestra alegría. También tenemos muchas tristezas, pero cuando somos verdaderos cristianos, existe esta esperanza que es una pequeña alegría que crece y te da seguridad. No debemos desanimarnos cuando vemos nuestros límites, nuestros pecados, nuestras debilidades: Dios está allí, próximo, cercano, Jesús está en la cruz para curarnos. Es el amor de Dios. Mira el crucifijo y di: \textquote{Dios me ama}. Es cierto, que existen estos límites, estas debilidades, estos pecados, pero Él es mayor que los límites, que las debilidades y los pecados. No olvidéis esto: Dios es mayor que nuestras debilidades, que nuestras infidelidades, que nuestros pecados. Y tomemos al Señor de la mano, miremos al Crucifijo y avancemos.

Que María Madre de la Misericordia nos ponga en el corazón la certeza de que somos amados por Dios. Que ella esté cerca de nosotros en los momentos en los cuales nos sentimos solos, cuando estamos tentados de capitular ante las dificultades de la vida. Que ella nos comunique los sentimientos de su Hijo Jesús, para que nuestro camino de cuaresma se convierta en una experiencia de perdón, de acogida y de caridad.
\end{body}

\label{b2-03-04-2018A}

\begin{patercite}
	(\ldots) Ningún tema de predicación es más apropiado ---según creo--- que Jesucristo, y éste crucificado. (\ldots) ¿Puede predicarse algo más conforme con la fe? ¿Hay algo más saludable para el auditorio o más apto para sanear las costumbres? ¿Hay algo tan eficaz como el recuerdo del Crucificado para destruir el pecado, crucificar los vicios, nutrir y robustecer la virtud?
	
	Hable, pues, Pablo entre los perfectos \textit{una sabiduría misteriosa,	escondida}; hábleme a mí, cuya imperfección perciben hasta los hombres, hábleme de Cristo crucificado, \textit{necedad} ciertamente \textit{para los que están en	vías de perdición}, pero para mí y para los que están en vías de	salvación es \textit{fuerza de Dios y sabiduría de Dios}; para mí es una	filosofía altísima y nobilísima, gracias a la cual me río yo de la infatuada sabiduría tanto del mundo como de la carne.
	
	¡Cuán perfecto me consideraría, cuán aprovechado en la sabiduría si llegase a ser por lo menos un idóneo oyente del crucificado, a quien Dios ha hecho para nosotros no sólo \textit{sabiduría}, sino también \textit{justicia,	santificación y redención}! Si realmente estás crucificado con Cristo,	eres sabio, eres justo, eres santo, eres libre. ¿O no es sabio quien,	elevado con Cristo sobre la tierra, saborea y busca los bienes de allá	arriba? ¿Acaso no es justo aquel en quien ha quedado destruida su	personalidad de pecador y él libre de la esclavitud al pecado? ¿Por	ventura no es santo el que a sí mismo se presenta como \textit{hostia viva,	santa, agradable a Dios}? ¿O no es verdaderamente libre aquel a quien	el Hijo liberó, quien, desde la libertad de la conciencia, confía hacer	suya aquella libre afirmación del Hijo: \textit{Se acerca el Príncipe de este	mundo; no es que él tenga poder sobre mí}? Realmente del Crucificado \textit{viene la misericordia, la redención copiosa}, que de tal modo \textit{redimió a Israel de todos sus delitos}, que mereció salir libre de	las calumnias del Príncipe de este mundo.
	
	\textit{Que lo confiesen, pues, los redimidos por el Señor, los que él rescató	de la mano del enemigo, los que reunió de todos los países}; que lo confiesen, repito, con la voz y el espíritu de su Maestro; \textit{Dios me libre de gloriarme si no es en la cruz de nuestro Señor Jesucristo}.
	
	\textbf{Beato Guerrico de Igny}, \textit{Sermón} 2, 1 en el Domingo de Ramos: PL 185, 130-131.
\end{patercite}


\newsection
\section{Temas}

\cceth{Cristo, el Salvador} 
\cceref{CEC 389, 457-458, 846, 1019, 1507}

\begin{ccebody}
\n{389} La doctrina del pecado original es, por así decirlo, \textquote{el reverso} de la Buena Nueva de que Jesús es el Salvador de todos los hombres, que todos necesitan salvación y que la salvación es ofrecida a todos gracias a Cristo. La Iglesia, que tiene el sentido de Cristo (cf. \textit{1 Cor} 2, 16) sabe bien que no se puede lesionar la revelación del pecado original sin atentar contra el Misterio de Cristo.

\n{457} El Verbo se encarnó \textit{para salvarnos reconciliándonos con Dios}: \textquote{Dios nos amó y nos envió a su Hijo como propiciación por nuestros pecados} (\textit{1 Jn} 4, 10). \textquote{El Padre envió a su Hijo para ser salvador del mundo} (\textit{1 Jn} 4, 14). \textquote{Él se manifestó para quitar los pecados} (\textit{1 Jn} 3, 5):

\ccecite{\textquote{Nuestra naturaleza enferma exigía ser sanada; desgarrada, ser restablecida; muerta, ser resucitada. Habíamos perdido la posesión del bien, era necesario que se nos devolviera. Encerrados en las tinieblas, hacía falta que nos llegara la luz; estando cautivos, esperábamos un salvador; prisioneros, un socorro; esclavos, un libertador. ¿No tenían importancia estos razonamientos? ¿No merecían conmover a Dios hasta el punto de hacerle bajar hasta nuestra naturaleza humana para visitarla, ya que la humanidad se encontraba en un estado tan miserable y tan desgraciado?} (San Gregorio de Nisa, \textit{Oratio catechetica}, 15: PG 45, 48B).}

\n{458} El Verbo se encarnó \textit{para que nosotros conociésemos así el amor de Dios}: \textquote{En esto se manifestó el amor que Dios nos tiene: en que Dios envió al mundo a su Hijo único para que vivamos por medio de él} (\textit{1 Jn} 4, 9). \textquote{Porque tanto amó Dios al mundo que dio a su Hijo único, para que todo el que crea en él no perezca, sino que tenga vida eterna} (\textit{Jn} 3, 16).

\ccesec{\textquote{Fuera de la Iglesia no hay salvación}}

\n{846} ¿Cómo entender esta afirmación tantas veces repetida por los Padres de la Iglesia? Formulada de modo positivo significa que toda salvación viene de Cristo-Cabeza por la Iglesia que es su Cuerpo:

\ccecite{\textquote{El santo Sínodo [\ldots] basado en la sagrada Escritura y en la Tradición, enseña que esta Iglesia peregrina es necesaria para la salvación. Cristo, en efecto, es el único Mediador y camino de salvación que se nos hace presente en su Cuerpo, en la Iglesia. Él, al inculcar con palabras, bien explícitas, la necesidad de la fe y del bautismo, confirmó al mismo tiempo la necesidad de la Iglesia, en la que entran los hombres por el Bautismo como por una puerta. Por eso, no podrían salvarse los que sabiendo que Dios fundó, por medio de Jesucristo, la Iglesia católica como necesaria para la salvación, sin embargo, no hubiesen querido entrar o perseverar en ella} (LG 14).}

\n{1019} \textit{Jesús, el Hijo de Dios, sufrió libremente la muerte por nosotros en una sumisión total y libre a la voluntad de Dios, su Padre. Por su muerte venció a la muerte, abriendo así a todos los hombres la posibilidad de la salvación}.

\n{1507} El Señor resucitado renueva este envío (\textquote{En mi nombre [\ldots] impondrán las manos sobre los enfermos y se pondrán bien}, (\textit{Mc} 16, 17-18) y lo confirma con los signos que la Iglesia realiza invocando su nombre (cf. \textit{Hch} 9, 34; 14, 3). Estos signos manifiestan de una manera especial que Jesús es verdaderamente \textquote{Dios que salva} (cf. \textit{Mt} 1, 21; \textit{Hch} 4, 12).
\end{ccebody}

\cceth{Cristo es el Señor de la vida eterna} 
\cceref{CEC 679}

\begin{ccebody}
\n{679} Cristo es Señor de la vida eterna. El pleno derecho de juzgar definitivamente las obras y los corazones de los hombres pertenece a Cristo como Redentor del mundo. \textquote{Adquirió} este derecho por su Cruz. El Padre también ha entregado \textquote{todo juicio al Hijo} (\textit{Jn} 5, 22; cf. \textit{Jn} 5, 27; \textit{Mt} 25, 31; \textit{Hch} 10, 42; 17, 31; \textit{2 Tm} 4, 1). Pues bien, el Hijo no ha venido para juzgar sino para salvar (cf. \textit{Jn} 3, 17) y para dar la vida que hay en él (cf. \textit{Jn} 5, 26). Es por el rechazo de la gracia en esta vida por lo que cada uno se juzga ya a sí mismo (cf. \textit{Jn} 3, 18; 12, 48); es retribuido según sus obras (cf. \textit{1 Co} 3, 12- 15) y puede incluso condenarse eternamente al rechazar el Espíritu de amor (cf. \textit{Mt} 12, 32; \textit{Hb} 6, 4-6; 10, 26-31).
\end{ccebody}

\cceth{Dios quiere dar a los hombres la vida eterna} 
\cceref{CEC 55}

\begin{ccebody}
\n{55} Esta revelación no fue interrumpida por el pecado de nuestros primeros padres. Dios, en efecto, \textquote{después de su caída [\ldots] alentó en ellos la esperanza de la salvación con la promesa de la redención, y tuvo incesante cuidado del género humano, para dar la vida eterna a todos los que buscan la salvación con la perseverancia en las buenas obras} (DV 3).
\end{ccebody}

\cceth{El exilio de Israel presagio de la Pasión} 
\cceref{CEC 710}

\begin{ccebody}
\n{710} El olvido de la Ley y la infidelidad a la Alianza llevan a la muerte: el Exilio, aparente fracaso de las Promesas, es en realidad fidelidad misteriosa del Dios Salvador y comienzo de una restauración prometida, pero según el Espíritu. Era necesario que el Pueblo de Dios sufriese esta purificación (cf. \textit{Lc} 24, 26); el Exilio lleva ya la sombra de la Cruz en el Designio de Dios, y el Resto de pobres que vuelven del Exilio es una de la figuras más transparentes de la Iglesia.
\end{ccebody}