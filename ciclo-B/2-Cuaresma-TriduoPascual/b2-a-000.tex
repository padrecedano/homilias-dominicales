			\chapter{Domingo I de Cuaresma (B)}

\section{Lecturas}
\begin{readtitle}PRIMERA LECTURA\end{readtitle}
\begin{readbook}
	Del libro del Génesis Del libro del GénesisDel libro del Génesis \rred{9, 8-15}
\end{readbook}

\begin{readtheme}Pacto de Dios con Noé liberado del diluvio de las aguas\end{readtheme}

\begin{readbody}Dios dijo a Noé y a sus hijos:\end{readbody}

\begin{readbody}\textquote {Yo establezco mi alianza con vosotros y con vuestros descendientes, con todos los animales que os acompañan, aves, ganados y fieras, con todos los que salieron del arca y ahora viven en la tierra. Establezco, pues, mi alianza con vosotros: el diluvio no volverá a destruir criatura alguna ni habrá otro diluvio que devaste la tierra}.\end{readbody}

\begin{readbody}Y Dios añadió:\end{readbody}

\begin{readbody}\textquote {Esta es la señal de la alianza que establezco con vosotros y con todo lo que vive con vosotros, para todas las generaciones: pondré mi arco en el cielo, como señal de mi alianza con la tierra. Cuando traiga nubes sobre la tierra, aparecerá en las nubes el arco y recordaré mi alianza con vosotros y con todos los animales, y el diluvio no volverá a destruir a los vivientes}.\end{readbody}

\begin{readtitle}SALMO RESPONSORIAL\end{readtitle}

\begin{readbook}Salmo \rred{24, 4-5a. 6-7cd. 8-9}\end{readbook}

\begin{readtheme}Tus sendas, Señor, son misericordia y lealtad para los que guardan tu alianza\end{readtheme}

\begin{readbody} 
	\sv V. 	\h Señor, enséñame tus caminos*,\par
			\hl instrúyeme en tus sendas:\par
			\hl haz que camine con lealtad;\par
			\hl enséñame, porque tú eres mi Dios y Salvador. \sr R.\par\par
	\sv V. 	\h Recuerda, Señor, que tu ternura\par
			\hl y tu misericordia son eternas;\par
			\hl acuérdate de mí con misericordia,\par
			\hl por tu bondad, Señor. \sr{R.}\par
	\sv V. 	\h El Señor es bueno y es recto,\par
			\hl enseña el camino a los pecadores;\par
			\hl hace caminar a los humilles con rectitud,\par
			\hl enseña su camino a los humildes. \sr R.
\end{readbody}

\begin{readtitle}SEGUNDA LECTURA\end{readtitle}

\begin{readbook}De la primera carta del apóstol san Pedro \rred{3, 18-22}\end{readbook}

\begin{readtheme}El bautismo que actualmente os está salvando\end{readtheme}

\begin{readbody}Queridos hermanos:\end{readbody}

\begin{readbody}Cristo sufrió su pasión, de una vez para siempre, por los pecados, el justo por los injustos, para conduciros a Dios.\end{readbody}

\begin{readbody}Muerto en la carne pero vivificado en el Espíritu; en el espíritu fue a predicar incluso a los espíritus en prisión, a los desobedientes en otro tiempo, cuando la paciencia de Dios aguardaba, en los días de Noé, a que se construyera el arca, para que unos pocos, es decir, ocho personas, se salvaran por medio del agua.\end{readbody}

\begin{readbody}Aquello era también un símbolo del bautismo que actualmente os está salvando, que no es purificación de una mancha física, sino petición a Dios de una buena conciencia, por la resurrección de Jesucristo, el cual fue al cielo, está sentado a la derecha de Dios y tiene a su disposición ángeles, potestades y poderes.\end{readbody}

\begin{readtitle}EVANGELIO\end{readtitle}

\begin{readbook}Del Santo Evangelio según san Marcos \rred{1, 12-15}\end{readbook}

\begin{readtheme}Era tentado por Satanás, y los ángeles lo servían\end{readtheme}

\begin{readbody}En aquel tiempo, el Espíritu empujó a Jesús al desierto.\end{readbody}

\begin{readbody}Se quedó en el desierto cuarenta días, siendo tentado por Satanás; vivía con las fieras y los ángeles lo servían.\end{readbody}

\begin{readbody}Después de que Juan fue entregado, Jesús se marchó a Galilea a proclamar el Evangelio de Dios; decía:\end{readbody}

\begin{readtalk} "Se ha cumplido el tiempo y está cerca el reino de Dios. Convertíos y creed en el Evangelio".\end{readtalk}


\begin{ccetheme}Alianza y sacramentos (especialmente el Bautismo) \end{ccetheme}

\begin{ccereference}\end{ccereference}CEC 1116, 1129, 1222:</p>

\begin{ccebody}\begin{ccenumber}1116 \end{ccenumber}Los sacramentos, como "fuerzas que brotan" del Cuerpo de Cristo (cf. \textit{Lc} 5,17; 6,19; 8,46) siempre vivo y vivificante, y como acciones del Espíritu Santo que actúa en su Cuerpo que es la Iglesia, son "las obras maestras de Dios" en la nueva y eterna Alianza.\end{ccebody}

\begin{ccebody}\redbf{1129} La Iglesia afirma que para los creyentes los sacramentos de la Nueva Alianza son \textit{necesarios para la salvación} (cf. Concilio de Trento: DS 1604). La "gracia sacramental" es la gracia del Espíritu Santo dada por Cristo y propia de cada sacramento. El Espíritu cura y transforma a los que lo reciben conformándolos con el Hijo de Dios. El fruto de la vida sacramental consiste en que el Espíritu de adopción deifica (cf. \textit{2 Pe} 1,4) a los fieles uniéndolos vitalmente al Hijo único, el Salvador.\end{ccebody}

\begin{ccebody}\begin{ccenumber}1222 \end{ccenumber}Finalmente, el Bautismo es prefigurado en el paso del Jordán, por el que el pueblo de Dios recibe el don de la tierra prometida a la descendencia de Abraham, imagen de la vida eterna. La promesa de esta herencia bienaventurada se cumple en la nueva Alianza.\end{ccebody}

\section{Comentario Patrístico}

\subsection{San Agustín, obispo}

\begin{patertheme}La pasión de Cristo como premio de la piedad\end{patertheme}

\begin{patersource} Carta 140, 13-15, Libro sobre la gracia del nuevo Testamento: CSEL 44, 164-166.\end{patersource}

\begin{body}De nada sirve reconocer a nuestro Señor como hijo de la bienaventurada Virgen María y como hombre verdadero y perfecto, si no se le cree descendiente de aquella estirpe que en el Evangelio se le atribuye.\end{body}

\begin{body}Para que por medio de Cristo se revelara la gracia del nuevo Testamento, que no dice relación con la vida temporal, sino con la eterna, no convenía que el hombre Cristo fuera propuesto como ejemplo de felicidad eterna. Esto explica la sujeción, la pasión, la flagelación, los salivazos, los desprecios, la cruz, las heridas y la misma muerte como a un vencido y derrotado, para que sus fieles supieran cuál era el premio que por la piedad cabía pedir y esperar de Aquel cuyos hijos han llegado a ser; no fuera que sus servidores se consagraran al servicio de Dios con la intención de conseguir –como una gran cosa–, la felicidad eterna, desdeñando y conculcando su fe, considerándola de escasísimo valor.\end{body}

\begin{readbody}Por esta razón, el hombre-Cristo que es al mismo tiempo el Dios-Cristo, por cuya misericordiosísima humanidad y en cuya condición de siervo deberemos aprender lo que hemos de desdeñar en esta vida y lo que hemos de esperar en la otra, durante su pasión –en la que sus enemigos se consideraban los grandes vencedores–, hizo suya la voz de nuestra debilidad, en la que era también crucificado nuestro hombre viejo, y dijo: \textit{Dios mío, Dios mío, ¿por qué me has abandonado?}\end{readbody}

\begin{readbody}Por la voz, pues, de nuestra debilidad, que en sí transfiguró nuestra cabeza, se dice en el salmo veintiuno: \textit{Dios mío, Dios mío, mírame, ¿por qué me has abandonado?}, pues el que ora, si no es escuchado, se siente abandonado. Esta es la voz que Jesús transfiguró en sí mismo, a saber, la voz de su cuerpo, es decir, de su Iglesia, que iba a ser transformada de hombre viejo en hombre nuevo; a saber, la voz de su debilidad humana, a la que fue preciso negarle los bienes del antiguo Testamento, para que aprendiera de una vez a desear y a esperar los bienes del nuevo Testamento.\end{readbody}


\begin{readtheme}Pacto de Dios con Noé liberado del diluvio de las aguas\end{readtheme}

\begin{readbody}Dios dijo a Noé y a sus hijos:

\begin{verse}"Yo establezco mi alianza con vosotros y con vuestros descendientes, con todos los animales que os acompañan, aves, ganados y fieras, con todos los que salieron del arca y ahora viven en la tierra. Establezco, pues, mi alianza con vosotros: el diluvio no volverá a destruir criatura alguna ni habrá otro diluvio que devaste la tierra".\end{verse}

Y Dios añadió:

\begin{verse}"Esta es la señal de la alianza que establezco con vosotros y con todo lo que vive con vosotros, para todas las generaciones: pondré mi arco en el cielo, como señal de mi alianza con la tierra. Cuando traiga nubes sobre la tierra, aparecerá en las nubes el arco y recordaré mi alianza con vosotros y con todos los animales, y el diluvio no volverá a destruir a los vivientes".\end{verse}
\end{readbody}
\begin{readtitle}SALMO RESPONSORIAL\end{readtitle}

\begin{readbook}Salmo \rightline{24, 4-5a. 6-7cd. 8-9}\end{readbook}

\begin{readtheme}Tus sendas, Señor, son misericordia y lealtad para los que guardan tu alianza\end{readtheme}

\begin{verse} 
	\s{V.} Señor, enséñame tus caminos,
	instrúyeme en tus sendas:
	haz que camine con lealtad;
 	enséñame, porque tú eres mi Dios y Salvador.\s{R.}
 	
	\s{V.} Recuerda, Señor, que tu ternura
 	y tu misericordia son eternas;
 	acuérdate de mí con misericordia,
 	por tu bondad, Señor. \s{R.}
 	
	\s{V.} El Señor es bueno y es recto,
 	enseña el camino a los pecadores;
 	hace caminar a los humilles con rectitud,
 	enseña su camino a los humildes. \s{R.} 
\end{verse}

\begin{readtitle}SEGUNDA LECTURA\end{readtitle}

\begin{readbook}De la primera carta del apóstol san Pedro \rightline{3, 18-22}\end{readbook}

\begin{readtheme}El bautismo que actualmente os está salvando\end{readtheme}

\begin{readbody}Queridos hermanos:\end{readbody}

\begin{readbody}Cristo sufrió su pasión, de una vez para siempre, por los pecados, el justo por los injustos, para conduciros a Dios.\end{readbody}

\begin{readbody}Muerto en la carne pero vivificado en el Espíritu; en el espíritu fue a predicar incluso a los espíritus en prisión, a los desobedientes en otro tiempo, cuando la paciencia de Dios aguardaba, en los días de Noé, a que se construyera el arca, para que unos pocos, es decir, ocho personas, se salvaran por medio del agua.\end{readbody}

\begin{readbody}Aquello era también un símbolo del bautismo que actualmente os está salvando, que no es purificación de una mancha física, sino petición a Dios de una buena conciencia, por la resurrección de Jesucristo, el cual fue al cielo, está sentado a la derecha de Dios y tiene a su disposición ángeles, potestades y poderes.\end{readbody}

\begin{redtitle}EVANGELIO \textbf{EVANGELIO}\end{redtitle}




\begin{readbook}Del Santo Evangelio según san Marcos \rightline{1, 12-15}\end{readbook}

\begin{readtheme}Era tentado por Satanás, y los ángeles lo servían\end{readtheme}

\begin{readbody}En aquel tiempo, el Espíritu empujó a Jesús al desierto.\end{readbody}

\begin{readbody}Se quedó en el desierto cuarenta días, siendo tentado por Satanás; vivía con las fieras y los ángeles lo servían.\end{readbody}

\begin{readbody}Después de que Juan fue entregado, Jesús se marchó a Galilea a proclamar el Evangelio de Dios; decía:\end{readbody}

\begin{readtalk} "Se ha cumplido el tiempo y está cerca el reino de Dios. Convertíos y creed en el Evangelio".\end{readtalk}
