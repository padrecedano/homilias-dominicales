\chapter{Domingo I de Cuaresma (B)}

\section{Lecturas}

\rtitle{PRIMERA LECTURA}

\rbook{Del libro del Génesis} \rred{9, 8-15}

\rtheme{Pacto de Dios con Noé liberado del diluvio de las aguas}

\begin{scripture}
Dios dijo a Noé y a sus hijos:

\>{Yo establezco mi alianza con vosotros y con vuestros descendientes, con todos los animales que os acompañan, aves, ganados y fieras, con todos los que salieron del arca y ahora viven en la tierra. Establezco, pues, mi alianza con vosotros: el diluvio no volverá a destruir criatura alguna ni habrá otro diluvio que devaste la tierra}.

Y Dios añadió:

\>{Esta es la señal de la alianza que establezco con vosotros y con todo lo que vive con vosotros, para todas las generaciones: pondré mi arco en el cielo, como señal de mi alianza con la tierra. Cuando traiga nubes sobre la tierra, aparecerá en las nubes el arco y recordaré mi alianza con vosotros y con todos los animales, y el diluvio no volverá a destruir a los vivientes}.
\end{scripture}

\newpage
\rtitle{SALMO RESPONSORIAL}


\rbook{Salmo} \rred{24, 4-5a. 6-7cd. 8-9}

\rtheme{Tus sendas, Señor, son misericordia y lealtad para los que guardan tu alianza}

\begin{psbody}
Señor, enséñame tus caminos,
instrúyeme en tus sendas:
haz que camine con lealtad;
enséñame, porque tú eres mi Dios y Salvador. 

Recuerda, Señor, que tu ternura
y tu misericordia son eternas;
acuérdate de mí con misericordia,
por tu bondad, Señor. 

El Señor es bueno y es recto,
enseña el camino a los pecadores;
hace caminar a los humilles con rectitud,
enseña su camino a los humildes. 
\end{psbody}

\rtitle{SEGUNDA LECTURA}

\rbook{De la primera carta del apóstol san Pedro} \rred{3, 18-22}

\rtheme{El bautismo que actualmente os está salvando}

\begin{scripture}
Queridos hermanos:

Cristo sufrió su pasión, de una vez para siempre, por los pecados, el justo por los injustos, para conduciros a Dios.

Muerto en la carne pero vivificado en el Espíritu; en el espíritu fue a predicar incluso a los espíritus en prisión, a los desobedientes en otro tiempo, cuando la paciencia de Dios aguardaba, en los días de Noé, a que se construyera el arca, para que unos pocos, es decir, ocho personas, se salvaran por medio del agua.

Aquello era también un símbolo del bautismo que actualmente os está salvando, que no es purificación de una mancha física, sino petición a Dios de una buena conciencia, por la resurrección de Jesucristo, el cual fue al cielo, está sentado a la derecha de Dios y tiene a su disposición ángeles, potestades y poderes.
\end{scripture}

\newpage
\rtitle{EVANGELIO}

\rbook{Del Santo Evangelio según san Marcos} \rred{1, 12-15}

\rtheme{Era tentado por Satanás, y los ángeles lo servían}

\begin{scripture}
En aquel tiempo, el Espíritu empujó a Jesús al desierto.

Se quedó en el desierto cuarenta días, siendo tentado por Satanás; vivía con las fieras y los ángeles lo servían.

Después de que Juan fue entregado, Jesús se marchó a Galilea a proclamar el Evangelio de Dios; decía:

\>{Se ha cumplido el tiempo y está cerca el reino de Dios. Convertíos y creed en el Evangelio}.
\end{scripture}

\begin{patercite}
	El primer domingo del itinerario cuaresmal subraya nuestra condición de hombre en esta tierra. La batalla victoriosa contra las tentaciones, que da inicio a la misión de Jesús, es una invitación a tomar conciencia de la propia fragilidad para acoger la Gracia que libera del pecado e infunde nueva fuerza en Cristo, camino, verdad y vida (cf. \textit{Ordo Initiationis Christianae Adultorum}, n. 25). Es una llamada decidida a	recordar que la fe cristiana implica, siguiendo el ejemplo de Jesús y en unión con él, una lucha \textquote{contra los Dominadores de este mundo tenebroso}	(\textit{Ef} 6, 12), en el cual el diablo actúa y no se cansa, tampoco hoy, de tentar al hombre que quiere acercarse al Señor: Cristo sale victorioso, para abrir también nuestro corazón a la esperanza y guiarnos a vencer las seducciones del mal.
	
	En síntesis, el itinerario cuaresmal, en el cual se nos invita a contemplar el Misterio de la cruz, es \textquote{hacerme semejante a él en su muerte} (\textit{Flp} 3, 10), para llevar a cabo una \textit{conversión} profunda de nuestra vida: dejarnos transformar por la acción del Espíritu Santo, como san Pablo en el camino de Damasco; orientar con decisión nuestra existencia según la voluntad de Dios; liberarnos de nuestro egoísmo, superando el instinto de dominio sobre los demás y abriéndonos a la caridad de Cristo. El período cuaresmal es el momento favorable para reconocer nuestra debilidad, acoger, con una sincera revisión de vida, la Gracia renovadora del Sacramento de la Penitencia y caminar con decisión hacia Cristo.
		
	\textbf{Benedicto XVI, papa}, \textit{Mensaje} para la Cuaresma del 2011, n. 2, parr. 2 y n. 3, parr. 4.
\end{patercite}

\newsection
\section{Comentario Patrístico}

\subsection{San Agustín, obispo}

\ptheme{En Cristo fuimos tentados, en él vencimos al diablo}

\src{Comentario sobre el salmo 60, 2-3: \\CCL 39, 766.}

\begin{body}
\ltr{D}{\textit{ios}} \textit{mío, escucha mi clamor, atiende a mi súplica}. ¿Quién es el que habla? Parece que sea uno solo. Pero veamos si es uno solo: \textit{Te invoco desde los confines de la tierra con el corazón abatido}. Por lo tanto, si invoca desde los confines de la tierra, no es uno solo; y, sin embargo, es uno solo, porque Cristo es uno solo, y todos nosotros somos sus miembros. ¿Y quién es ese único hombre que clama desde los confines de la tierra? Los que invocan desde los confines de la tierra son los llamados a aquella herencia, a propósito de la cual se dijo al mismo Hijo: \textit{Pídemelo: te daré en herencia las naciones, en posesión, los confines de la tierra}. De manera que quien clama desde los confines de la tierra es el cuerpo de Cristo, la heredad de Cristo, la única Iglesia de Cristo, esta unidad que formamos todos nosotros.

Y ¿qué es lo que pide? Lo que he dicho antes: \textit{Dios mío, escucha mi clamor, atiende a mi súplica; te invoco desde los confines de la tierra}. O sea: \textquote{Esto que pido, lo pido desde los confines de la tierra}, es decir, desde todas partes.

Pero, ¿por qué ha invocado así? Porque tenía \textit{el corazón abatido}. Con ello da a entender que el Señor se halla presente en todos los pueblos y en los hombres del orbe entero no con gran gloria, sino con graves tentaciones.

Pues nuestra vida en medio de esta peregrinación no puede estar sin tentaciones, ya que nuestro progreso se realiza precisamente a través de la tentación, y nadie se conoce a sí mismo si no es tentado, ni puede ser coronado si no ha vencido, ni vencer si no ha combatido, ni combatir si carece de enemigo y de tentaciones.

Este que invoca desde los confines de la tierra está angustiado, pero no se encuentra abandonado. Porque a nosotros mismos, esto es, a su cuerpo, quiso prefigurarnos también en aquel cuerpo suyo en el que ya murió, resucitó y ascendió al cielo, a fin de que sus miembros no desesperen de llegar adonde su cabeza los precedió.

De forma que nos incluyó en sí mismo cuando quiso verse tentado por Satanás. Nos acaban de leer que Jesucristo, nuestro Señor, \textit{se dejó tentar por el diablo}. ¡Nada menos que Cristo tentado por el diablo! Pero en Cristo estabas siendo tentado tú, porque Cristo tenía de ti la carne, y de él procedía para ti la salvación; de ti procedía la muerte para él, y de él para ti la vida; de ti para él los ultrajes, y de él para ti los honores; en definitiva, de ti para él la tentación, y de él para ti la victoria.

Si hemos sido tentados en él, también en él vencemos al diablo. ¿Te fijas en que Cristo fue tentado, y no te fijas en que venció? Reconócete a ti mismo tentado en él, y reconócete también vencedor en él. Podía haber evitado al diablo; pero, si no hubiese sido tentado, no te habría aleccionado para la victoria cuando tú fueras tentado.
\end{body}

\begin{patercite}
[\ldots] Tratemos de nuevo de pensar en el desierto. El desierto es \textit{el lugar de lo esencial}. Miremos nuestras vidas: ¡cuántas cosas inútiles nos rodean! Perseguimos mil cosas que parecen necesarias y en realidad no lo son. ¡Qué bien nos haría liberarnos de tantas realidades superfluas, para redescubrir lo que de verdad importa, para encontrar los rostros de quienes están a nuestro lado! También en esto Jesús nos da su ejemplo, ayunando. \textit{Ayunar} es saber renunciar a las cosas vanas, a lo superfluo, para ir a lo esencial. Ayunar no es solamente adelgazar, ayunar es ir precisamente a lo esencial, es buscar la belleza de una vida más sencilla.  

El desierto, finalmente, es el lugar de la soledad. También hoy, cerca de nosotros, hay tantos desiertos. Son las personas solas y abandonadas. Cuantos pobres y ancianos están cerca de nosotros y viven en silencio, sin clamor, marginados y descartados. Hablar de ellos no aumenta las audiencias. Pero el desierto nos lleva a ellos, a cuantos, forzados a callar, piden en silencio nuestra ayuda. Tantas miradas silenciosas que piden nuestra ayuda. El camino en el desierto cuaresmal es un camino de \textit{caridad} hacia quien es más débil.

Oración, ayuno, obras de misericordia: he aquí el camino en el desierto cuaresmal.

(\ldots) con la voz del profeta Isaías, Dios hizo esta promesa: \textquote{Pues bien, he aquí que yo lo renuevo: pongo en el desierto un camino} (\textit{Isaías} 43, 19). En el desierto se abre el camino que nos lleva de la muerte a la vida. Entremos en el desierto con Jesús, saldremos saboreando la Pascua, la potencia del amor de Dios que nos renueva la vida. Sucederá a nosotros como a esos desiertos que en primavera florecen, haciendo germinar de repente \textquote{de la nada} gemas y plantas. Ánimo, entremos en este desierto de la Cuaresma. Sigamos a Jesús en el desierto: con Él nuestros desiertos florecerán.

\textbf{Francisco, papa}, \textit{Catequesis}, Audiencia general, 26 de febrero de 2020, parr. 4-7.
\end{patercite}

\newsection 
\section{Homilías}

\subsection{San Juan Pablo II, papa}

\subsubsection{Homilía (1982): Despertar las conciencias}

\src{Visita pastoral a la parroquia Romana de Sant’Andrea delle Fratte.\par28 de febrero de 1982.}

\begin{body}
\ltr{C}{on} palabras muy concisas, el \textbf{evangelista Marcos} alude a ese ayuno de Jesús de Nazaret, que duró cuarenta días, y que cada año encuentra su reflejo en la liturgia de Cuaresma: \textquote{El Espíritu lo empujó al desierto y permaneció allí durante cuarenta días, siendo tentado por satanás; estaba entre los animales del campo y los ángeles le servían} (\textit{Mc} 1, 12). Luego, después del encarcelamiento de Juan el Bautista, Jesús fue a Galilea y comenzó a enseñar. Decía: \textquote{El tiempo se ha cumplido y el reino de Dios está cerca; convertíos y creed en el Evangelio} (\textit{Mc} 1, 15). El ayuno de cuarenta días de Jesús de Nazaret fue una introducción al anuncio del Evangelio del Reino de Dios. Trazó en las almas de los hombres el camino de la fe, sin el cual el Evangelio del Reino permanece como grano arrojado en tierra estéril.

2. La liturgia de hoy compara este comienzo del Evangelio del Reino, que llega a la Iglesia a través del ayuno de cuarenta días, con el arco iris, que fue un signo de la \textbf{alianza de Dios con los descendientes de Noé} después del diluvio. La Iglesia también se compara con el Arca de Noé en la \textbf{primera carta del Apóstol San Pedro}, en la que Cristo, después de haber ganado la victoria sobre la muerte y el pecado, realiza continuamente la obra de la redención. Sin embargo, el Arca de Noé era un espacio cerrado. La obra de Cristo es ilimitada en tiempo y espacio. La Iglesia sirve a esta obra como signo e instrumento.

He aquí Cristo: Aquel que murió de una vez por todas por los pecados; el Justo por el injusto, para llevarnos de regreso a Dios.

He aquí Cristo: muerto, es cierto, en el cuerpo, pero llamado a la vida en el Espíritu.

He aquí Cristo: sentado a la diestra de Dios, porque ascendió al cielo, donde los ángeles, los poderes y las dominaciones le fueron sometidos.

El mismo Cristo que, en el Espíritu Santo, \textquote{fue a anunciar la salvación también a los espíritus encarcelados; que en otro tiempo se negaron a creer} (\textit{1 Pe} 3, 19), como en los días de Noé. Cristo mismo en el bautismo nos salva, es decir, nos redime \textquote{no quitando la suciedad del cuerpo, sino mediante la invocación de la salvación dirigida a Dios por la buena conciencia} (cf. \textit{1 Pe} 3, 21): nos salva y nos redime gracias a su resurrección .

3. Así pues, la liturgia dominical de hoy abre el ayuno de Cuaresma, refiriéndose primero al ejemplo de Cristo, y luego al poder redentor de Cristo, que opera en la Iglesia y en toda la creación: a su poder redentor y santificador. La Cuaresma es el camino que se abre ante nosotros. Toda la Iglesia desea caminar así durante estos cuarenta días.

4. Y por eso ora hoy: \textquote{Señor, hazme conocer tus caminos, enséñame tus sendas. Guíame en tu verdad y enséñame, porque tú eres el Dios de mi salvación, en ti siempre he esperado} (\textit{Sal} 25 [24], 4-5). La Cuaresma es el camino de la verdad. El hombre debe encontrarse en toda su verdad ante Dios. También debe releer la verdad de las enseñanzas divinas, de los mandamientos divinos, de la voluntad divina, y debe confrontar su conciencia con ellos. Por aquí pasa el camino de la salvación. Es el camino de la esperanza.

5. Así pues, la Iglesia todavía reza: \textquote{Acuérdate, Señor, de tu amor, y de tu fidelidad que son eternos. Acuérdate de mí en tu misericordia, por tu bondad, Señor} (\textit{Sal} 25 [24], 6-7). La Cuaresma es el camino de la verdad, es el momento del despertar de las conciencias. Pero sobre todo, es el camino del Amor y la Misericordia. Sólo a través del Amor la verdad despierta al hombre a la vida. Sólo el Amor, que es Misericordia, enciende la esperanza. El ayuno durante la Cuaresma es un gran grito de amor. Un grito desgarrador. Un llanto definitivo. Es el gran momento de la misericordia. ¡Que todos reconozcan este camino!

6. Por tanto, la Iglesia sigue rezando en la liturgia de hoy: \textquote{Bueno y recto es el Señor, muestra el camino recto a los pecadores; conduce a los humildes según la justicia, enseña a los pobres su sendero} (\textit{Sal} 25 [24], 8-9). La Iglesia reza por la humildad del corazón humano. Ora para que el hombre, a través de la humildad, se encuentre a sí mismo en la verdad, para que pueda encontrarse en la verdad interior, para que así pueda encontrarse con el Amor, que es más fuerte que el pecado y la muerte, más fuerte que todo mal, y porque se deje guiar por el Verbo Divino: \textquote{No sólo de pan vive el hombre, sino de toda palabra que sale de la boca de Dios} (\textit{Mt} 4, 4).

7. He aquí el programa del Primer Domingo de Cuaresma. [\ldots] [En este tiempo] descubramos de nuevo la belleza de ser cristianos y ofrezcamos un testimonio consecuente y luminoso de ello.

8. En este primer domingo de Cuaresma deseo repetir las palabras de \textbf{San Pedro Apóstol}, primer obispo de la Iglesia de Roma: Queridos amigos, \textquote{Cristo murió una sola vez por los pecados, el justo por los injustos, para llevarnos de regreso a Dios} (\textit{1 Pe} 3, 18).

Amén.
\end{body}

\subsubsection{Homilía (1985): Guíame en tu verdad}

\src{Celebración eucarística en la Parroquia Romana de San Lorenzo in Damaso. \\24 de febrero de 1985.}

\begin{body}
\ltr[1. «]{E}{l} Espíritu llevó a Jesús al desierto, y permaneció allí cuarenta días, siendo tentado por Satanás» (cf. \textit{Mc} 1, 12-13). Cada año, en este primer domingo de Cuaresma, recordamos el ayuno de cuarenta días de Jesús y las tentaciones de satanás. El texto del \textbf{Evangelio según Marcos}, que leemos este año, es muy conciso.

Jesús de Nazaret comienza su misión mesiánica con su bautismo en el Jordán. Recibió de manos de Juan el Bautista el bautismo de penitencia, asemejándose a todos aquellos a quienes Juan lo administraba. Y, después, Jesús va al desierto, donde ayuna durante cuarenta días. Este ayuno se refiere a los cuarenta años de peregrinaje de Israel desde la esclavitud de Egipto a la Tierra Prometida.

El ayuno de cuarenta días de Jesús en el desierto es un modelo para la Cuaresma de la Iglesia. Debe llevarnos de la esclavitud del pecado a la victoria y la libertad en la resurrección de Cristo. Con un ayuno de cuarenta días nos preparamos para la Pascua.

2. Durante este período, Jesús predica el Evangelio de Dios de una manera particularmente intensa. Dice: \textquote{El tiempo se ha cumplido y el reino de Dios está cerca; convertíos y creed en el Evangelio} (\textit{Mc} 1, 15).

La Iglesia desea imitar a su Maestro. Con especial intensidad predica el Evangelio. La Cuaresma se define en la liturgia como un \textquote{tiempo fuerte}.

El Evangelio es un mensaje de conversión: \textquote{conviértete}. Al anunciar la conversión, Jesús da a conocer al hombre el estado de amenaza de múltiples males. Por lo tanto, incluso con respecto a sí mismo, admite la tentación de Satanás y la vence. Este es un doble momento en el que el camino mesiánico de Jesús pasa por los caminos del hombre, prisionero del pecado. El primer aspecto es el bautismo del Jordán, bautismo de penitencia; el segundo es la tentación.

La Iglesia recuerda esta tentación de Jesús el primer domingo de Cuaresma. De esta manera, quiere llegar a todos los caminos del hombre amenazado por múltiples tentaciones; quiere llegar a los caminos en que están los hombres envueltos en pecado. En estos caminos Cristo está verdaderamente presente con su poder salvador.

3. El estado de tentación –de amenaza de múltiples males, con pecado leve o grave– es un estado ordinario del hombre. Por eso Cristo nos recomendó orar al Padre: \textquote{Y no nos dejes caer en tentación, mas líbranos del mal} (\textit{Mt} 6, 13).

Por eso, en la liturgia de hoy, el \textbf{salmista} implora con fervor: \textquote{Señor, hazme conocer tus caminos, enséñame tus caminos, guíame en tu verdad e instrúyeme \ldots} (\textit{Sal} 25, 4-5). \textquote{Guíame en tu verdad} significa exactamente: ¡no permitas que caiga en la tentación! De hecho, la tentación siempre está relacionada con la pérdida de la verdad en el comportamiento humano. El corazón y la voluntad son \textquote{seducidos} de tal manera que, al obrar, se desprenden del verdadero bien y siguen un aparente bien. La tentación es siempre mentira y tiene su origen definitivo en aquel a quien la Escritura llama \textquote{el padre de la mentira} (\textit{Jn} 8, 44).

Si el hombre es tentado por el \textquote{mundo}, si la fuente de las tentaciones se encuentra en la concupiscencia de los ojos, de la carne y en la soberbia de la vida (es decir, \textquote{dentro del hombre}), entonces el inicio de toda tentación se origina de aquel por quien Cristo mismo permitió ser tentado durante el ayuno de cuarenta días en el desierto, el que es \textquote{el padre de la mentira}.

4. La Iglesia, por tanto, entiende su Cuaresma como un desafío particular en la lucha contra el mal, incluso hasta sus raíces. La tentación no es solo una ocasión para pecar, sino que también es la raíz del pecado. El hombre no solo se siente atraído por el mal, sino que a veces también está rodeado por él.

Todo esto Cristo lo hace presente al hombre desde el comienzo mismo de ese camino que es la Cuaresma. Al mismo tiempo, hace presente a cada uno de nosotros la fuerza salvífica del Evangelio. El Evangelio no es solo la palabra de Dios, es \textquote{poder de salvación} (\textit{Rm} 1, 16). Y en este sentido es una buena noticia. Tiene sus raíces en el pacto de Dios con la creación. La liturgia de hoy recuerda la antigua \textbf{alianza} que se estableció con \textbf{Noé} y sus hijos después del diluvio. En primer lugar, el Evangelio se expresa con la alianza nueva y eterna. Es la alianza estipulada en la cruz de Cristo, –por medio de su cuerpo y de su sangre–, y reconfirmada con la resurrección.

Cada año, ayunando durante cuarenta días, la Iglesia se prepara para la renovación singular de esta alianza, la cual contiene también la fuerza definitiva del poder salvador, que es capaz de conducir al hombre a través del estado de amenaza de múltiples males. Solo es necesario que el hombre se arraigue en esta alianza, resistiendo todo lo que viene del \textquote{padre de la mentira}.

5. Meditamos este importante contenido litúrgico del primer domingo de Cuaresma [en esta parroquia de San Lorenzo en Dámaso.]

[\ldots]

6. [¡Aceptad, queridos hermanos y hermanas, la visita de hoy del Obispo de Roma!] ¡Aceptad el mensaje del primer domingo de Cuaresma! Entre los caminos por los que os lleva la vida diaria, no dejéis de orar con las palabras del \textbf{salmista}: \textquote{Guíame, Señor, en tu verdad e instrúyeme} \ldots, y con las palabras del Padrenuestro: \textquote{No nos dejes caer en tentación, mas líbranos del mal}.

Que estas palabras marquen el tiempo de la Cuaresma en vuestras almas. En vuestra parroquia, durante este tiempo fuerte.

Amén.
\end{body}


\subsubsection{Homilía (1988): Alianza nueva y eterna}

\src{Visita pastoral a la Parroquia Romana de Santa Prisca.\\21 de febrero de 1988.}

\begin{body}
1. \textquote{He aquí que yo establezco mi alianza contigo} (\textit{Gn} 9, 9).

\ltr{H}{oy}, primer domingo de Cuaresma, la Iglesia nos recuerda en la liturgia la alianza concertada por Dios con el patriarca \textbf{Noé} después del diluvio. Y esta es una de las alianzas que forman la historia de la salvación en el Antiguo Testamento: \textquote{Muchas veces y de diferentes maneras Dios habló a los padres por medio de los profetas}, leemos en la carta a los Hebreos (cf. \textit{Hb} 1, 1); \textquote{Muchas veces haz ofrecido tu alianza a los hombres} proclama la \textit{Plegaria Eucarística IV}.

\textquote{He aquí que establezco mi alianza contigo y con tu descendencia después de ti; con cada ser vivo\ldots aves, vacas y fieras, con todos los animales que han salido del arca} (\textit{Gn} 9, 9-10). En estas palabras del Libro del Génesis escuchamos un claro eco del primer capítulo del mismo libro, en el que Dios somete toda la creación al dominio del hombre. En la historia bíblica, la obra de la creación y la alianza van de la mano.

2. La alianza con el patriarca Noé se caracteriza por el hecho de que se estableció después del diluvio. Esto fue causado por los pecados cometidos por los hombres de la época. La alianza fue, por tanto, un signo de perdón y gracia de parte de Dios.

\textquote{Yo establezco mi alianza contigo: ningún ser vivo será destruido por las aguas del diluvio, ni el diluvio devastará la tierra} (\textit{Gn} 9, 11).

Del \textbf{libro del Génesis} se puede deducir que el diluvio bíblico, que devastó la tierra y todo lo que en ella existía, excepto los seres salvados en el arca de Noé, fue el castigo por otro diluvio, el del pecado (cf. \textit{Gn} 6), en el que pronto se hicieron evidentes los efectos de la corrupción causada por el pecado original en el corazón y la conciencia de la humanidad. A causa de la primera transgresión, el hombre se encuentra bajo la influencia del \textquote{padre de la mentira} (cf. \textit{Jn} 8, 44), a quien en la Sagrada Escritura también se le llama \textquote{el príncipe de este mundo} (\textit{Jn} 12, 31; \textit{Jn} 14, 30; 16, 11).

3. Si la Iglesia nos recuerda todo esto el primer domingo de Cuaresma, lo hace para introducirnos en el misterio y al mismo tiempo en la realidad de la Nueva y Eterna Alianza, que el Padre Eterno concluyó con los hombres en Cristo: en su cruz y en su sangre. En su muerte y resurrección.

He aquí que Cristo ya está presente en el mundo. El pasaje del \textbf{Evangelio de Marcos} nos dice que vino a Galilea para proclamar el Evangelio de Dios, y que, incluso antes, sufrió una tentación en el desierto por obra del mismo \textquote{padre de la mentira} y \textquote{príncipe de este mundo}.

\textquote{El Espíritu lo empujó al desierto y permaneció allí cuarenta días, siendo tentado por Satanás} (\textit{Mc} 1, 12).

La historia es concisa. Los otros evangelistas sinópticos dan detalles más extensos. El hecho de la tentación de Jesús en el desierto debe leerse en el contexto de la lógica de la encarnación. Desde que el Hijo de Dios se hizo hombre, desde que vino \textquote{al mundo} y quiso mostrar que acoge a este mundo y al hombre como realmente son, quiso también someter su verdadera humanidad a la influencia tentadora del \textquote{príncipe de las tinieblas}. Sólo en ese contexto es posible comprender plenamente las palabras: \textquote{Yo soy la luz del mundo} (\textit{Jn} 8, 12), o la expresión de Simeón: \textquote{Luz para iluminar a las naciones} (\textit{Lc} 2, 32).

4. La Iglesia recuerda todo esto al comienzo del período que, como el ayuno de Cristo en el desierto, debe durar cuarenta días. De ahí esta referencia en la liturgia de hoy.

Sin embargo, el pensamiento de esta celebración no se detiene solo en este evento. Va hacia la alianza de Dios con el hombre, que debe cumplirse definitivamente en la muerte de Cristo.

Aquí están las palabras del \textbf{apóstol Pedro}: \textquote{Cristo murió una vez por todas por los pecados, el justo por los injustos, para llevarlos de regreso a Dios; muerto en la carne, pero vivificado en el espíritu} (\textit{1 Pe} 3, 18). Aquí San Pedro se refiere a Noé y su arca, para decir que la muerte de Cristo anuncia la salvación a los que murieron entonces (cf. \textit{1 Pe} 3, 19-20). Pero luego el apóstol explica el significado del bautismo, en el que ve una analogía con la experiencia bíblica del diluvio y el arca, cuando los hombres se salvaron por medio del agua (cf. \textit{1 Pe} 3, 20). En el bautismo, el poder salvador del sacramento no deriva del agua misma, que es sólo un símbolo expresivo, sino del poder de la resurrección de Cristo.

Es la misma verdad que proclama san Pablo en la carta a los Romanos, escribiendo sobre el bautismo que recibimos en la muerte de Cristo, para luego participar de su vida, revelada por la resurrección (cf. \textit{Rm} 6, 1 ss).

5. La liturgia de Cuaresma –como vemos– nos prepara desde el principio para los acontecimientos pascuales. Este es su significado y propósito fundamental. Con tal espíritu, cada uno de nosotros debe meditar en las palabras del \textbf{salmista} de la liturgia de hoy y orar con él: \textquote{Hazme conocer, Señor, tus caminos, enséñame tus senderos. Guíame en tu verdad e instrúyeme} (\textit{Sal} 25 [24], 4-5).

Se trata aquí de las enseñanzas más importantes para la vida de la Iglesia, las verdades decisivas y definitivas, que en el tiempo pascual, e incluso antes de la Cuaresma, están particularmente condensadas.

A la luz de estas verdades y enseñanzas, podemos reconocer plenamente que: \textquote{Bueno y recto es el Señor, indica a los pecadores el justo camino; juzga a los humildes según la justicia, enseña sus caminos a los pobres} (\textit{Sal} 25 [24], 8-9).

Precisamente este es el sentido de la alianza, que Dios ha ofrecido muchas veces a los hombres, en la historia de la salvación, para preparar la alianza última y definitiva en la sangre de Cristo, en su cruz y en su resurrección.

Si el \textbf{salmista} reza: \textquote{Señor, recuerda tu amor y tu fidelidad que son eternas} (\textit{Sal} 25 [24], 6), el misterio de la redención de Cristo constituye la realización de estas palabras.

Una vez, después del diluvio, la señal de la alianza fue un arco iris en el horizonte. Ahora, este arco iris de paz es, en última instancia, la cruz del Gólgota: la cruz que se extiende sobre el mundo entero.

[\ldots]

7. [\ldots] Quisiera deciros a todos: tened los ojos abiertos y vigilantes, tened un corazón generoso, tanto para notar las situaciones que os desafían, considerándolas con el alma iluminada por la fe, como para responder con generosidad a los necesitados, aunque sean personas que vivan lejos de vosotros. 

\txtsmall{[Por eso quiero animar fuertemente todas las iniciativas del grupo \textquote{caritas} y de los grupos de voluntariado organizados por la parroquia. \ldots Un pensamiento a los grupos de oración y animación litúrgica, con especial atención a los ancianos que, como \textquote{lámparas vivientes}, haciendo sus adoraciones y súplicas ascienden permanentemente al Señor en esta iglesia parroquial.]}

8. \textquote{No sólo de pan vivirá el hombre, sino de toda palabra que sale de la boca de Dios} (\textit{Mt} 4, 4). La liturgia nos recuerda estas palabras de Cristo dirigidas al tentador. ¡Cuán importantes son al comienzo de la Cuaresma!

\textquote{No solo de pan}\ldots este es el sentido del ayuno, se trata de practicar la templanza al comer y al usar los bienes materiales. Sino: \textquote{de toda palabra que sale de la boca de Dios}. Así que dediquemos más tiempo y espacio a este alimento que nutre la mente y el corazón, que nutre el alma.

En el espíritu de estas palabras de Cristo, repetimos a menudo en el período actual: \textquote{Gloria a ti, oh Cristo, Palabra de Dios} (\textit{Cantus ad Evangelium}).

Sí. Gloria a ti, Verbo, que te hiciste carne. Gloria a ti, Cristo, nuestro Redentor. \textquote{Tú tienes palabras de vida eterna} (\textit{Jn} 6, 68).
\end{body}


\label{b-03-01-1988H}
\newpage

\subsubsection{Homilía (1991): Dejar que el Espíritu nos empuje}

\src{Visita pastoral a la Parroquia Romana de Santa Dorotea. \\17 de febrero de 1991.}

\begin{body}
\ltr[1. «]{E}{l} Espíritu empujó a Jesús al desierto y allí permaneció cuarenta días siendo tentado por Satanás» (\textit{Mc} 1, 12). Con estas pocas palabras el \textbf{evangelista Marcos} describe la prueba que sufrió Jesús antes de comenzar su misión. Es una prueba de la que el Señor sale victorioso y que lo capacita para anunciar el Evangelio del Reino, llamando a todos a acogerlo en la fe, en actitud de conversión, para hacerse sus discípulos. (\ldots) En los cuarenta días de este tiempo de Cuaresma, que acaba de comenzar, también vuestra comunidad cristiana es impulsada por el Espíritu a adentrarse en el desierto\ldots Es decir, en ese clima espiritual donde, a través de la escucha asidua de la Palabra de Dios, la oración y la Caridad activa es concedido a los bautizados entrar en un diálogo más intenso con el Padre, que está en los Cielos, a abandonar los \textquote{ídolos} de este mundo, a madurar opciones valientes orientadas hacia una auténtica fidelidad a las necesidades evangélicas y a redescubrir una fuerte solidaridad con los hermanos. Entremos, por tanto, en este \textquote{tiempo propicio} de purificación e iluminación. Entremos y salgamos todos renovados, siguiendo a Cristo nuestro guía, nuestro maestro y modelo. Caminemos este \textquote{camino espiritual}, dejándonos llevar por el Espíritu, que quiere hacer de cada uno de nosotros una \textquote{nueva criatura}, capaz de anunciar y dar testimonio del Evangelio a todos los hombres.

2. El camino en el \textquote{desierto}, que la Iglesia está urgida a realizar en los cuarenta días de Cuaresma, adquiere un significado rico, una profundidad salvífica e implica elecciones de renovación exigentes, que conviene destacar. El itinerario de Cuaresma parte de una fuerte convicción de fe. Implica ser conscientes de que en el camino uno no está solo y abandonado a sí mismo, sino que es guiado por el Espíritu. Ese mismo Espíritu dado a Cristo y que ha sido comunicado también a los creyentes en el bautismo, el primero de los sacramentos de la nueva Alianza.

La \textbf{primera lectura} de esta celebración eucarística en clave profética de figura y anuncio, y la \textbf{segunda lectura}, que declara su cumplimiento en Cristo y en la Iglesia, ofrecen la correcta comprensión del acontecimiento que tuvo lugar con el \textquote{paso} de la muerte a la vida, realizado en el bautismo. A través de las aguas del diluvio, Dios destruyó el pecado de rebelión de los hombres y dio lugar a una nueva humanidad reconciliada con Él. Esto lo confirma la Alianza hecha con Noé, simbolizada en la señal del arco iris, casi un arco de caza que estaba colgado en las nubes, para indicar una pacificación universal y cósmica que ya no debe ser perturbada. Estamos ante un anuncio de la victoria de Dios y de su misericordia sobre un mundo que siempre está tentado a rebelarse contra Dios y prescindir de él.

La Cuaresma, por tanto, debe llevarnos a ser cada vez más conscientes de lo que el Espíritu, invocado en el bautismo, ha obrado en nosotros, para que podamos renovar con mayor conciencia, en la vigilia pascual, la alianza bautismal y los compromisos que de ella brotan. Como en el antiguo Israel vagando por el desierto, Dios ofrece al pueblo de la nueva Alianza, que camina hacia la Pascua eterna, los signos de la benevolencia y la gracia que libera y salva, con la condición, sin embargo, de que el hombre diga el \textquote{sí} de su fidelidad y obediencia a la divina propuesta de salvación.

3. Queridos hermanos y hermanas\ldots la liturgia de hoy constituye para vuestra comunidad\ldots un \textquote{programa} de vida estimulante y exigente. [Este tiempo puede ser para nosotros] un verdadero itinerario de purificación e iluminación de cara a una comunión más intensa con Dios y con los hermanos que se realiza a través de la oración, la penitencia y un servicio evangélico más fiel y generoso [a la ciudad]. Esta realmente puede considerarse como una especie de \textquote{desierto}, es decir, un lugar en el que se destacan por un lado los signos de la presencia del Tentador, que induce a los hombres a apartarse de Dios, a ceder a muchas formas de idolatría y pecado, provocando desintegración y división; y, por otro, el espacio en el que el Señor sigue dando el Espíritu a través de múltiples signos de misericordia, gracia y caridad.

\txtsmall{[Vuestro barrio también está marcado por esta \textquote{ambivalencia}. Lo caracterizan la indiferencia religiosa, el pluralismo ideológico (\ldots) múltiples formas de marginación social y pobreza. Todo esto exige una renovación en la fe, un compromiso misionero más fuerte, una pastoral más orgánica y armoniosa para servir al hombre inspirándose en el Evangelio. Hay mucha gente pobre que ve a esta comunidad parroquial como un punto de referencia y un centro de acogida, ayuda y apoyo, tanto material como espiritual. Siguen vivas en vuestra comunidad las huellas de los muchos santos que han vivido aquí, dejando vestigios de su testimonio y su amor por los pobres, los marginados, los jóvenes: San José de Calasanz, San Gaetano da Thiene, Santa Paola Frassinetti, y San Maximiliano Kolbe, el padre conventual polaco, que se ha detenido aquí varias veces en oración y que honra a la Iglesia y a vuestra Orden con su fidelidad a Cristo y al hombre, hasta el holocausto de su vida. Seguid sus pasos para dar respuesta a las viejas y nuevas miserias que hoy es posible encontrar en las calles de vuestro barrio. Hacedlo sobre todo \textquote{juntos}, sintiéndoos y trabajando en comunidad, evitando la tentación del absentismo o la delegación a pocas personas. Cread cada vez más un clima de fraternidad y de corresponsabilidad entre vosotros (\ldots)] [\ldots]}

5. \textquote{Los caminos del Señor son verdad y gracia} (Salmo Responsorial). Queridos amigos, el tiempo de Cuaresma os abre un camino y traza las vías para recorrerlo. Ante todo, la vía de la verdad, de una adhesión más consciente a la verdad evangélica, para haceros testigos y heraldos de ella para una \textquote{nueva evangelización}. Y luego la vía de la gracia, es decir, de una participación más activa y fecunda en los sacramentos, para vivir como hombres nuevos, capaces de renovar el mundo. Dios os guíe y el Espíritu os sostenga. ¡En esta Cuaresma y siempre! ¡Amén!
\end{body}

\subsubsection{Homilía (1997): Hacer alianza con Dios}

\src{Visita pastoral a la Parroquia Romana de San Andrés Avellino. \\16 de febrero de 1997.}

\begin{body}
\ltr[1. «]{Y}{o} establezco mi alianza con vosotros» (\textit{Gn} 9, 8). La liturgia de la Palabra de este primer domingo de Cuaresma nos presenta la alianza que Dios establece con los hombres y con la creación, después del diluvio, a través de \textbf{Noé}. Hemos vuelto a escuchar las solemnes palabras que pronunció Dios: \textquote{Yo hago una alianza con vosotros y con vuestros descendientes, con todos los animales que os acompañaron (\ldots). Hago una alianza con vosotros: el diluvio no volverá a destruir la vida, ni habrá otro diluvio que devaste la tierra} (\textit{Gn} 9, 9-11). Esta alianza tiene su valor típico en el Antiguo Testamento. Dios, creador del hombre y de todos los seres vivos, en cierto sentido había aniquilado con el diluvio cuanto él mismo había creado. Ese castigo tuvo como causa el pecado, difundido en el mundo después de la caída de nuestros primeros padres. Sin embargo, las aguas no exterminaron a Noé y a su familia, y tampoco a los animales que había recogido en el arca. De ese modo, se salvaron el hombre y los demás seres vivos que, habiendo sobrevivido al castigo del Creador, constituyeron después del diluvio el comienzo de una nueva alianza entre Dios y la creación. Esa alianza tuvo su signo tangible en el arco iris: \textquote{Pondré mi arco en el cielo –dice Dios–, como señal de mi alianza con la tierra. Cuando traiga nubes sobre la tierra, aparecerá en las nubes el arco, y recordaré mi alianza con vosotros} (\textit{Gn} 9, 13-15).

2. Las lecturas de hoy nos permiten, por tanto, mirar de un modo nuevo al hombre y al mundo en el que vivimos. En efecto, el mundo y el hombre no sólo representan la realidad de la existencia en cuanto expresión de la obra creadora de Dios; también son la imagen de la alianza. Toda la creación habla de esta alianza. A lo largo de las diversas épocas de la historia los hombres han seguido cometiendo pecados, tal vez incluso mayores que los descritos antes del diluvio. Sin embargo, las palabras de la alianza que Dios estableció con Noé nos permiten comprender que ya ningún pecado podrá llevar a Dios a aniquilar el mundo que él mismo creó. La liturgia de hoy abre ante nuestros ojos una visión nueva del mundo. Nos ayuda a tomar conciencia del valor que el mundo tiene a los ojos de Dios, quien incluyó toda la obra de la creación en la alianza que selló con Noé, y se comprometió a salvarla de la destrucción.

3. El miércoles pasado, con la imposición de la ceniza, comenzó la Cuaresma, y hoy es el primer domingo de este tiempo fuerte, que hace referencia al ayuno de cuarenta días que Jesús empezó después de su bautismo en el Jordán. A este propósito, \textbf{san Marcos}, que nos acompaña este año en la liturgia dominical, escribe: \textquote{El Espíritu impulsó a Jesús al desierto. Se quedó en el desierto cuarenta días, dejándose tentar por Satanás; vivía entre alimañas, y los ángeles le servían} (\textit{Mc} 1, 12-13). San Mateo, en el pasaje paralelo, anota sólo la respuesta que el Señor dio al tentador que lo provocaba para que transformara las piedras en panes: \textquote{Si eres Hijo de Dios, di que estas piedras se conviertan en panes} (\textit{Mt} 4, 3). Jesús respondió: \textquote{No sólo de pan vive el hombre, sino de toda palabra que sale de la boca de Dios} (\textit{Mt} 4, 4; cf. \textit{Aleluya}). Esta es una de las tres respuestas de Cristo a Satanás, que trataba de engañarlo y vencerlo, haciendo referencia a las tres concupiscencias de la naturaleza humana caída. En el umbral de la Cuaresma, la victoria de Cristo contra el diablo constituye, en cierta manera, una invitación a vencer el mal con el esfuerzo ascético, una de cuyas manifestaciones es el ayuno, a fin de vivir este período con autenticidad. \txtsmall{[\ldots]}

5. \textquote{Se ha cumplido el plazo, está cerca el reino de Dios: convertíos y creed en el Evangelio} (\textit{Mc} 1, 15). Estas palabras del \textbf{evangelista Marcos} resuenan en nuestro corazón. El evangelio comienza con la misión de Jesús, misión que se cumplirá con los acontecimientos pascuales. La Iglesia prosigue en el tiempo esta misión, a la que cada uno de nosotros está llamado a dar su propia aportación personal, anunciando y testimoniando a Cristo, muerto y resucitado por la salvación del mundo. \txtsmall{[\ldots]}

6. Escribe \textbf{san Pedro} en su \textbf{primera carta}: \textquote{Cristo murió por los pecados una vez para siempre: el inocente por los culpables (\ldots). Con este espíritu, fue a proclamar su mensaje a los espíritus encarcelados que en un tiempo habían sido rebeldes, cuando la paciencia de Dios aguardaba en tiempos de Noé, mientras se construía el arca, en la que unos pocos –ocho personas– se salvaron cruzando las aguas} (\textit{1 Pe} 3, 18-20). Estas palabras de Pedro hacen referencia a la alianza de Noé, de la que nos ha hablado la \textbf{primera lectura}. Esa alianza representa un modelo, un símbolo, una figura de la nueva alianza que Dios concluyó con toda la humanidad en Jesucristo, por medio de su muerte en la cruz y de su resurrección. Si la antigua alianza tenía que ver, ante todo, con la creación, la nueva, fundada en el misterio pascual de Cristo, es la alianza de la Redención.

En el texto que hemos escuchado, el \textbf{apóstol Pedro} alude al sacramento del bautismo. Las aguas destructoras del diluvio son sustituidas por las aguas bautismales, que santifican. El bautismo es el sacramento fundamental en el que se hace realidad la alianza de la redención del hombre. Ya desde el origen de la tradición cristiana, la Cuaresma era prácticamente una preparación para el bautismo, que se administraba a los catecúmenos en la solemne Vigilia de Pascua.

Amadísimos hermanos y hermanas, renovemos en nosotros mismos, especialmente durante este período cuaresmal, la conciencia de nuestra alianza con Dios. Dios estableció una alianza con Noé y la inscribió en la obra de la creación. Cristo, Redentor del hombre y de todo el hombre, llevó a plenitud la obra del Creador con su muerte y su resurrección.

Hemos sido redimidos por la sangre de Cristo. Cristo murió por los pecados una vez para siempre: el inocente por los culpables. Amén.
\end{body}

\subsubsection{Homilía (2000): Pedir perdón y perdonar}

\src{Santa Misa de la Jornada del Perdón del Año Santo 2000. \\12 de marzo del 2000.}

\rbr{Esta es una celebración especial, en el contexto del Gran Jubileo del año 2000, por ese motivo, algunas de las lecturas referidas en la homilía no son las habituales de este domingo. No obstante, la reflexiones del Papa pueden aplicar tanto para este domingo, como para una celebración penitencial durante el tiempo de Cuaresma.}


\begin{body}
\ltr[1. «]{E}{n} nombre de Cristo os suplicamos: ¡reconciliaos con Dios! A quien no conoció pecado, le hizo pecado por nosotros, para que viniésemos a ser justicia de Dios en él» (\textit{2 Co} 5, 20-21). La Iglesia relee estas palabras de san Pablo cada año, el miércoles de Ceniza, al comienzo de la Cuaresma. Durante el tiempo cuaresmal, la Iglesia desea unirse de modo particular a Cristo, que, impulsado interiormente por el Espíritu Santo, inició su misión mesiánica dirigiéndose al desierto, donde ayunó durante cuarenta días y cuarenta noches (cf. \textit{Mc} 1, 12-13). Al término de ese ayuno fue tentado por Satanás, como narra sintéticamente, en la liturgia de hoy, el \textbf{evangelista san Marcos} (cf. \textit{Mc} 1, 13). San Mateo y san Lucas, en cambio, tratan con mayor amplitud ese combate de Cristo en el desierto y su victoria definitiva sobre el tentador: \textquote{Apártate, Satanás, porque está escrito: \textquote{Al Señor tu Dios adorarás, y sólo a él darás culto}} (\textit{Mt} 4, 10). Quien habla así es aquel \textquote{que no conoció pecado} (\textit{2 Co} 5, 21), Jesús, \textquote{el Santo de Dios} (\textit{Mc} 1, 24).

2. \textquote{A quien no conoció pecado, le hizo pecado por nosotros} (\textit{2 Co} 5, 21). Acabamos de escuchar en la segunda lectura esta afirmación sorprendente del Apóstol. ¿Qué significan estas palabras? Parecen una paradoja y, efectivamente, lo son. ¿Cómo pudo Dios, que es la santidad misma, \textquote{hacer pecado} a su Hijo unigénito, enviado al mundo? Sin embargo, esto es precisamente lo que leemos en el pasaje de la segunda carta de san Pablo a los Corintios. Nos encontramos ante un misterio: misterio que, a primera vista, resulta desconcertante, pero que se inscribe claramente en la Revelación divina. Ya en el Antiguo Testamento, el libro de Isaías habla de ello con inspiración profética en el cuarto canto del Siervo de Yahveh: \textquote{Todos nosotros como ovejas erramos, cada uno marchó por su camino, y el Señor descargó sobre él la culpa de todos nosotros} (\textit{Is} 53, 6). Cristo, el Santo, a pesar de estar absolutamente sin pecado, acepta tomar sobre sí nuestros pecados. Acepta para redimirnos cargar con nuestros pecados para cumplir la misión recibida del Padre, que, como escribe el evangelista san Juan, \textquote{tanto amó al mundo que dio a su Hijo único, para que todo el que crea en él (\ldots) tenga vida eterna} (\textit{Jn} 3, 16).

3. Ante Cristo que, por amor, cargó con nuestras iniquidades, todos estamos invitados a un profundo examen de conciencia. Uno de los elementos característicos del gran jubileo es el que he calificado como \textquote{purificación de la memoria} (\textit{Incarnationis mysterium}, 11). Como Sucesor de Pedro, he pedido que \textquote{en este año de misericordia la Iglesia, persuadida de la santidad que recibe de su Señor, se postre ante Dios e implore perdón por los pecados pasados y presentes de sus hijos} (\textit{ib}.). Este primer domingo de Cuaresma me ha parecido la ocasión propicia para que la Iglesia, reunida espiritualmente en torno al Sucesor de Pedro, implore el perdón divino por las culpas de todos los creyentes. ¡Perdonemos y pidamos perdón!

Esta exhortación ha suscitado en la comunidad eclesial una profunda y provechosa reflexión, que ha llevado a la publicación, en días pasados, de un documento de la Comisión teológica internacional, titulado: \textquote{\textit{Memoria y reconciliación: la Iglesia y las culpas del pasado}}. Doy las gracias a todos los que han contribuido a la elaboración de este texto. Es muy útil para una comprensión y aplicación correctas de la auténtica petición de perdón, fundada en la responsabilidad objetiva que une a los cristianos, en cuanto miembros del Cuerpo místico, y que impulsa a los fieles de hoy a reconocer, además de sus culpas propias, las de los cristianos de ayer, a la luz de un cuidadoso discernimiento histórico y teológico. En efecto, \textquote{por el vínculo que une a unos y otros en el Cuerpo místico, y aun sin tener responsabilidad personal ni eludir el juicio de Dios, el único que conoce los corazones, somos portadores del peso de los errores y de las culpas de quienes nos han precedido} (\textit{Incarnationis mysterium}, 11). Reconocer las desviaciones del pasado sirve para despertar nuestra conciencia ante los compromisos del presente, abriendo a cada uno el camino de la conversión.

4. ¡Perdonemos y pidamos perdón! A la vez que alabamos a Dios, que, en su amor misericordioso, ha suscitado en la Iglesia una cosecha maravillosa de santidad, de celo misionero y de entrega total a Cristo y al prójimo, no podemos menos de reconocer las infidelidades al Evangelio que han cometido algunos de nuestros hermanos, especialmente durante el segundo milenio. Pidamos perdón por las divisiones que han surgido entre los cristianos, por el uso de la violencia que algunos de ellos hicieron al servicio de la verdad, y por las actitudes de desconfianza y hostilidad adoptadas a veces con respecto a los seguidores de otras religiones.

Confesemos, con mayor razón, nuestras responsabilidades de cristianos por los males actuales. Frente al ateísmo, a la indiferencia religiosa, al secularismo, al relativismo ético, a las violaciones del derecho a la vida, al desinterés por la pobreza de numerosos países, no podemos menos de preguntarnos cuáles son nuestras responsabilidades. Por la parte que cada uno de nosotros, con sus comportamientos, ha tenido en estos males, contribuyendo a desfigurar el rostro de la Iglesia, pidamos humildemente perdón. Al mismo tiempo que confesamos nuestras culpas, perdonemos las culpas cometidas por los demás contra nosotros. En el curso de la historia los cristianos han sufrido muchas veces atropellos, prepotencias y persecuciones a causa de su fe. Al igual que perdonaron las víctimas de dichos abusos, así también perdonemos nosotros. La Iglesia de hoy y de siempre se siente comprometida a purificar la memoria de esos tristes hechos de todo sentimiento de rencor o venganza. De este modo, el jubileo se transforma para todos en ocasión propicia de profunda conversión al Evangelio. De la acogida del perdón divino brota el compromiso de perdonar a los hermanos y de reconciliación recíproca.

5. Pero ¿qué significa para nosotros el término \textquote{reconciliación}? Para captar su sentido y su valor exactos, es necesario ante todo darse cuenta de la posibilidad de la división, de la separación. Sí, el hombre es la única criatura en la tierra que puede establecer una relación de comunión con su Creador, pero también es la única que puede separarse de él. De hecho, por desgracia, con frecuencia se aleja de Dios.

Afortunadamente, muchos, como el hijo pródigo, del que habla el evangelio de san Lucas (cf. \textit{Lc} 15, 13), después de abandonar la casa paterna y disipar la herencia recibida, al tocar fondo, se dan cuenta de todo lo que han perdido (cf. \textit{Lc} 15, 13-17). Entonces, emprenden el camino de vuelta: \textquote{Me levantaré, iré a mi padre y le diré: \textquote{Padre, pequé\ldots}} (\textit{Lc} 15, 18). Dios, bien representado por el padre de la parábola, acoge a todo hijo pródigo que vuelve a él. Lo acoge por medio de Cristo, en quien el pecador puede volver a ser \textquote{justo} con la justicia de Dios. Lo acoge, porque hizo pecado por nosotros a su Hijo eterno. Sí, sólo por medio de Cristo podemos llegar a ser justicia de Dios (cf. \textit{2 Co} 5, 21).

6. \textquote{Dios tanto amó al mundo que dio a su Hijo único}. ¡Éste es en síntesis, el significado, del misterio de la redención del mundo! Hay que darse cuenta plenamente del valor del gran don que el Padre nos ha hecho en Jesús. Es necesario que ante la mirada de nuestra alma se presente Cristo, el Cristo de Getsemaní, el Cristo flagelado, coronado de espinas, con la cruz a cuestas y, por último, crucificado. Cristo tomó sobre sí el peso de los pecados de todos los hombres, el peso de nuestros pecados, para que, en virtud de su sacrificio salvífico, pudiéramos reconciliarnos con Dios.

Saulo de Tarso, convertido en san Pablo, se presenta hoy ante nosotros como testigo: él experimentó, de modo singular, la fuerza de la cruz en el camino de Damasco. El Resucitado se le manifestó con todo el esplendor de su poder: \textquote{Saulo, Saulo, ¿por qué me persigues? (\ldots) ¿Quién eres, Señor? (\ldots) Yo soy Jesús, a quien tú persigues} (\textit{Hch} 9, 4-5). San Pablo, que experimentó con tanta fuerza el poder de la cruz de Cristo, se dirige hoy a nosotros con una ardiente súplica: \textquote{Os exhortamos a que no recibáis en vano la gracia de Dios}. San Pablo insiste en que esta gracia nos la ofrece Dios mismo, que nos dice hoy a nosotros: \textquote{En el tiempo favorable te escuché y en el día de salvación te ayudé} (\textit{2 Co} 6, 2).

María, Madre del perdón, ayúdanos a acoger la gracia del perdón que el jubileo nos ofrece abundantemente. Haz que la Cuaresma de este extraordinario Año santo sea para todos los creyentes, y para cada hombre que busca a Dios, el momento favorable, el tiempo de la reconciliación, el tiempo de la salvación.
\end{body}


\subsubsection{Ángelus (2003): Purificar la conciencia}

\src{9 de marzo de 2003.}

\begin{body}
\ltr[1. ]{E}{l} miércoles pasado, con el rito de la ceniza, entramos en la Cuaresma, itinerario penitencial de preparación para la Pascua, ocasión para que todos los bautizados renueven su espíritu de fe y afiancen su compromiso de coherencia evangélica. Como sugiere el \textbf{evangelio} de hoy (\textit{Mc} 1, 12-15), durante los cuarenta días de la Cuaresma los creyentes están llamados a seguir a Cristo al \textquote{desierto}, para afrontar y vencer con él al espíritu del mal. Se trata de una lucha interior, de la que depende el planteamiento concreto de la vida. En efecto, del corazón del hombre brotan sus intenciones y sus acciones (cf. \textit{Mc} 7, 21); por tanto, sólo purificando la conciencia se prepara el camino de la justicia y de la paz, tanto en el plano personal como en el ámbito social.

2. En el actual contexto internacional, se siente con más fuerza la exigencia de purificar la conciencia y convertir el corazón a la paz verdadera. Al respecto, es muy elocuente el ejemplo de Cristo que desenmascara y vence las mentiras de Satanás con la fuerza de la verdad, contenida en la palabra de Dios. En lo más íntimo de cada persona resuenan la voz de Dios y la insidiosa voz del maligno. Esta última trata de engañar al hombre, seduciéndolo con la perspectiva de falsos bienes, para alejarlo del verdadero bien, que consiste precisamente en cumplir la voluntad divina. Pero la oración humilde y confiada, fortalecida por el ayuno, permite superar también las pruebas más duras, e infunde la valentía necesaria para combatir el mal con el bien. La Cuaresma se convierte así en un tiempo de provechoso entrenamiento del espíritu.

3. Amadísimos hermanos y hermanas, invoquemos a la Virgen santísima para que nos guíe a todos a recorrer con generosidad este exigente camino cuaresmal. [A vuestras oraciones quisiera encomendar, de modo especial, los ejercicios espirituales que, a partir de esta tarde, como todos los años, tendré la oportunidad de hacer juntamente con mis más íntimos colaboradores de la Curia romana. Durante esta semana de silencio y oración tendré presentes las necesidades de la Iglesia y las preocupaciones de toda la humanidad, sobre todo por lo que concierne a la paz en Irak y en Tierra Santa.]
\end{body}

\newsection
\subsection{Benedicto XVI, papa}

\subsubsection{Ángelus (2006): Vencer la tentación para ser libres}

\src{5 de marzo de 2006.}

\begin{body}
\ltr{E}{l} miércoles pasado iniciamos la Cuaresma, y hoy celebramos el primer domingo de este tiempo litúrgico, que estimula a los cristianos a comprometerse en un camino de preparación para la Pascua. Hoy el \textbf{evangelio} nos recuerda que Jesús, después de haber sido bautizado en el río Jordán, impulsado por el Espíritu Santo, que se había posado sobre él revelándolo como el Cristo, se retiró durante cuarenta días al desierto de Judá, donde superó las tentaciones de Satanás (cf. \textit{Mc} 1, 12-13). Siguiendo a su Maestro y Señor, también los cristianos entran espiritualmente en el desierto cuaresmal para afrontar junto con él \textquote{el combate contra el espíritu del mal}. La imagen del desierto es una metáfora muy elocuente de la condición humana. El libro del Éxodo narra la experiencia del pueblo de Israel que, habiendo salido de Egipto, peregrinó por el desierto del Sinaí durante cuarenta años antes de llegar a la tierra prometida. A lo largo de aquel largo viaje, los judíos experimentaron toda la fuerza y la insistencia del tentador, que los inducía a perder la confianza en el Señor y a volver atrás; pero, al mismo tiempo, gracias a la mediación de Moisés, aprendieron a escuchar la voz de Dios, que los invitaba a convertirse en su pueblo santo.

Al meditar en esta página bíblica, comprendemos que, para realizar plenamente la vida en la libertad, es preciso superar la prueba que la misma libertad implica, es decir, la tentación. Sólo liberada de la esclavitud de la mentira y del pecado, la persona humana, gracias a la obediencia de la fe, que la abre a la verdad, encuentra el sentido pleno de su existencia y alcanza la paz, el amor y la alegría. Precisamente por eso, la Cuaresma constituye un tiempo favorable para una atenta revisión de vida en el recogimiento, la oración y la penitencia.

[(\ldots) Invoquemos la intercesión materna de la Virgen María a fin de que la Cuaresma sea para todos los cristianos una ocasión de conversión y de impulso aún más valiente hacia la santidad.]
\end{body}


\subsubsection{Ángelus (2009): La kénosis de Cristo}

\src{1 de marzo de 2009.}

\begin{body}
\ltr{H}{oy} es el primer domingo de Cuaresma, y el Evangelio, con el estilo sobrio y conciso de \textbf{san Marcos}, nos introduce en el clima de este tiempo litúrgico: \textquote{El Espíritu impulsó a Jesús al desierto y permaneció en el desierto cuarenta días, siendo tentado por Satanás} (\textit{Mc} 1, 12-13). En Tierra Santa, al oeste del río Jordán y del oasis de Jericó, se encuentra el desierto de Judea, que, por valles pedregosos, superando un desnivel de cerca de mil metros, sube hasta Jerusalén. Después de recibir el bautismo de Juan, Jesús se adentró en aquella soledad conducido por el mismo Espíritu Santo que se había posado sobre él consagrándolo y revelándolo como Hijo de Dios. En el desierto, lugar de la prueba, como muestra la experiencia del pueblo de Israel, aparece con intenso dramatismo la realidad de la kénosis, del vaciamiento de Cristo, que se despojó de la forma de Dios (cf. \textit{Flp} 2, 6-7). Él, que no ha pecado y no puede pecar, se somete a la prueba y por eso puede compadecerse de nuestras flaquezas (cf. \textit{Hb} 4, 15). Se deja tentar por Satanás, el adversario, que desde el principio se opuso al designio salvífico de Dios en favor de los hombres. Casi de pasada, en la brevedad del relato, ante esta figura oscura y tenebrosa que tiene la osadía de tentar al Señor, aparecen los ángeles, figuras luminosas y misteriosas. Los ángeles, dice el evangelio, \textquote{servían} a Jesús (\textit{Mc} 1, 13); son el contrapunto de Satanás. \textquote{Ángel} quiere decir \textquote{enviado}. 

En todo el Antiguo Testamento encontramos estas figuras que, en nombre de Dios, ayudan y guían a los hombres. Basta recordar el libro de Tobías, en el que aparece la figura del ángel Rafael, que ayuda al protagonista en numerosas vicisitudes. La presencia tranquilizadora del ángel del Señor acompaña al pueblo de Israel en todas las circunstancias, tanto en las buenas como en las malas. En el umbral del Nuevo Testamento, Gabriel es enviado a anunciar a Zacarías y a María los acontecimientos felices que constituyen el inicio de nuestra salvación; y un ángel, cuyo nombre no se dice, advierte a José, orientándolo en aquel momento de incertidumbre. Un coro de ángeles lleva a los pastores la buena nueva del nacimiento del Salvador; y, del mismo modo, son también los ángeles quienes anuncian a las mujeres la feliz noticia de su resurrección. Al final de los tiempos, los ángeles acompañarán a Jesús en su venida en la gloria (cf. \textit{Mt} 25, 31). Los ángeles sirven a Jesús, que es ciertamente superior a ellos, y su dignidad se proclama aquí, en el evangelio, de modo claro aunque discreto. En efecto, incluso en la situación de extrema pobreza y humildad, cuando es tentado por Satanás, sigue siendo el Hijo de Dios, el Mesías, el Señor. (\ldots) Quitaríamos una parte notable del Evangelio, si dejáramos de lado a estos seres enviados por Dios, que anuncian su presencia en medio de nosotros y son un signo de ella. Invoquémoslos a menudo, para que nos sostengan en el compromiso de seguir a Jesús hasta identificarnos con él. [\ldots] María, Reina de los ángeles, ruega por nosotros.
\end{body}


\label{b2-03-012009A}
\newpage

\subsubsection{Ángelus (2012): Paciencia y humildad}

\src{Plaza de San Pedro, 26 de febrero de 2012.}

\begin{body}
\ltr{E}{n} este primer domingo de Cuaresma encontramos a Jesús, quien, tras haber recibido el bautismo en el río Jordán por Juan el Bautista (cf. \textit{Mc} 1, 9), sufre la tentación en el desierto (cf. \textit{Mc} 1, 12-13). La narración de \textbf{san Marcos} es concisa, carente de los detalles que leemos en los otros dos evangelios de Mateo y de Lucas. El desierto del que se habla tiene varios significados. Puede indicar el estado de abandono y de soledad, el \textquote{lugar} de la debilidad del hombre donde no existen apoyos ni seguridades, donde la tentación se hace más fuerte. Pero puede también indicar un lugar de refugio y de amparo –como lo fue para el pueblo de Israel en fuga de la esclavitud egipcia– en el que se puede experimentar de modo particular la presencia de Dios. Jesús \textquote{se quedó en el desierto cuarenta días, siendo tentado por Satanás} (\textit{Mc} 1, 13). San León Magno comenta que \textquote{el Señor quiso sufrir el ataque del tentador para defendernos con su ayuda y para instruirnos con su ejemplo} (\textit{Tractatus} XXXIX, 3 De ieiunio quadragesimae: CCL 138/a, Turnholti 1973, 214-215).

¿Qué puede enseñarnos este episodio? Como leemos en el libro de la \textit{Imitación de Cristo}, \textquote{el hombre jamás está del todo exento de las tentaciones mientras vive\ldots pero es con la paciencia y con la verdadera humildad como nos haremos más fuertes que cualquier enemigo} (\textit{Liber} I, c. XIII, Ciudad del Vaticano 1982, 37); con la paciencia y la humildad de seguir cada día al Señor, aprendemos a construir nuestra vida no fuera de Él y como si no existiera, sino en Él y con Él, porque es la fuente de la vida verdadera. La tentación de suprimir a Dios, de poner orden solos en uno mismo y en el mundo contando exclusivamente con las propias capacidades, está siempre presente en la historia del hombre.

Jesús proclama que \textquote{se ha cumplido el tiempo y está cerca el reino de Dios} (\textit{Mc} 1, 15), anuncia que en Él sucede algo nuevo: Dios se dirige al hombre de forma insospechada, con una cercanía única y concreta, llena de amor; Dios se encarna y entra en el mundo del hombre para cargar con el pecado, para vencer el mal y volver a llevar al hombre al mundo de Dios. Pero este anuncio se acompaña de la petición de corresponder a un don tan grande. Jesús, en efecto, añade: \textquote{convertíos y creed en el Evangelio} (\textit{Mc} 1, 15); es la invitación a tener fe en Dios y a convertir cada día nuestra vida a su voluntad, orientando hacia el bien cada una de nuestras acciones y pensamientos. El tiempo de Cuaresma es el momento propicio para renovar y fortalecer nuestra relación con Dios a través de la oración diaria, los gestos de penitencia, las obras de caridad fraterna.

Supliquemos con fervor a María santísima que acompañe nuestro camino cuaresmal con su protección y nos ayude a imprimir en nuestro corazón y en nuestra vida las palabras de Jesucristo para convertirnos a Él.
\end{body}

\newsection
\subsection{Francisco, papa}

\subsubsection{Ángelus (2015): Tiempo de combate espiritual}

\src{Plaza de San Pedro, 22 de febrero de 2015.}

\begin{body}
\ltr{E}{l} miércoles pasado, con el rito de la Ceniza, inició la Cuaresma, y hoy es el primer domingo de este tiempo litúrgico que hace referencia a los cuarenta días que Jesús pasó en el desierto, después del bautismo en el río Jordán. Escribe \textbf{san Marcos} en el Evangelio de hoy: \textquote{El Espíritu lo empujó al desierto. Se quedó en el desierto cuarenta días, siendo tentado por Satanás; vivía con las fieras y los ángeles lo servían} (\textit{Mc} 1, 12-13). Con estas escuetas palabras el evangelista describe la prueba que Jesús afrontó voluntariamente, antes de iniciar su misión mesiánica. Es una prueba de la que el Señor sale victorioso y que lo prepara para anunciar el Evangelio del Reino de Dios. Él, en esos cuarenta días de soledad, se enfrentó a Satanás \textquote{cuerpo a cuerpo}, desenmascaró sus tentaciones y lo venció. Y en Él hemos vencido todos, pero a nosotros nos toca proteger esta victoria en nuestra vida diaria.

La Iglesia nos hace recordar ese misterio al inicio de la Cuaresma, porque nos da la perspectiva y el sentido de este tiempo, que es un tiempo de combate –en Cuaresma se debe combatir–, un tiempo de combate espiritual contra el espíritu del mal (cf. \textit{Oración colecta del Miércoles de Ceniza}). Y mientras atravesamos el \textquote{desierto} cuaresmal, mantengamos la mirada dirigida a la Pascua, que es la victoria definitiva de Jesús contra el Maligno, contra el pecado y contra la muerte. He aquí entonces el significado de este primer domingo de Cuaresma: volver a situarnos decididamente en la senda de Jesús, la senda que conduce a la vida. Mirar a Jesús, lo que hizo Jesús, e ir con Él.

Y este camino de Jesús pasa a través del desierto. El desierto es el lugar donde se puede escuchar la voz de Dios y la voz del tentador. En el rumor, en la confusión esto no se puede hacer; se oyen sólo las voces superficiales. En cambio, en el desierto podemos bajar en profundidad, donde se juega verdaderamente nuestro destino, la vida o la muerte. ¿Y cómo escuchamos la voz de Dios? La escuchamos en su Palabra. Por eso es importante conocer las Escrituras, porque de otro modo no sabremos responder a las asechanzas del maligno. Y aquí quisiera volver a mi consejo de leer cada día el Evangelio: cada día leer el Evangelio, meditarlo, un poco, diez minutos; y llevarlo incluso siempre con nosotros: en el bolsillo, en la cartera\ldots Pero tener el Evangelio al alcance de la mano. El desierto cuaresmal nos ayuda a decir no a la mundanidad, a los \textquote{ídolos}, nos ayuda a hacer elecciones valientes conformes al Evangelio y a reforzar la solidaridad con los hermanos.

Entonces entramos en el desierto sin miedo, porque no estamos solos: estamos con Jesús, con el Padre y con el Espíritu Santo. Es más, como lo fue para Jesús, es precisamente el Espíritu Santo quien nos guía por el camino cuaresmal, el mismo Espíritu que descendió sobre Jesús y que recibimos en el Bautismo. La Cuaresma, por ello, es un tiempo propicio que debe conducirnos a tomar cada vez más conciencia de cuánto el Espíritu Santo, recibido en el Bautismo, obró y puede obrar en nosotros. Y al final del itinerario cuaresmal, en la Vigilia pascual, podremos renovar con mayor consciencia la alianza bautismal y los compromisos que de ella derivan.

Que la Virgen santa, modelo de docilidad al Espíritu, nos ayude a dejarnos conducir por Él, que quiere hacer de cada uno de nosotros una \textquote{nueva creatura}.

\txtsmall{[A Ella encomiendo, en especial, esta semana de ejercicios espirituales, que iniciará hoy por la tarde, y en la que participaré juntamente con mis colaboradores de la Curia romana. Rezad para que en este \textquote{desierto} que son los ejercicios espirituales podamos escuchar la voz de Jesús y también corregir tantos defectos que todos nosotros tenemos, y hacer frente a las tentaciones que cada día nos atacan. Os pido, por lo tanto, que nos acompañéis con vuestra oración.]}

\end{body}

\label{b2-03-01-2015A}
\newpage


\subsubsection{Ángelus (2018): Conviértete y cree en el Evangelio}

\src{Plaza de San Pedro, 18 de febrero de 2018.}

\begin{body}
\ltr{E}{n} este primer domingo de Cuaresma, el Evangelio menciona los temas de la tentación, la conversión y la Buena Noticia. Escribe el \textbf{evangelista Marcos}: \textquote{El Espíritu le empuja al desierto, y permaneció en el desierto cuarenta días, siendo tentando por Satanás} (\textit{Mc} 1, 12-13). Jesús va al desierto a prepararse para su misión en el mundo. Él no necesita conversión, pero, en cuanto hombre, debe pasar a través de esta prueba, ya sea por sí mismo, para obedecer a la voluntad del Padre, como por nosotros, para darnos la gracia de vencer las tentaciones. Esta preparación consiste en la lucha contra el espíritu del mal, es decir, contra el diablo. También para nosotros la Cuaresma es un tiempo de \textquote{agonismo} espiritual, de lucha espiritual: estamos llamados a afrontar al maligno mediante la oración para ser capaces, con la ayuda de Dios, de vencerlo en nuestra vida cotidiana. Nosotros lo sabemos, el mal está lamentablemente funcionando en nuestra existencia y entorno a nosotros, donde se manifiestan violencias, rechazo del otro, clausuras, guerras, injusticias. Todas estas son obra del maligno, del mal.

Inmediatamente después de las tentaciones en el desierto, Jesús empieza a predicar el Evangelio, es decir, la Buena Noticia, la segunda palabra. La primera era \textquote{tentación}; la segunda, \textquote{Buena Noticia}. Y esta Buena Noticia exige del hombre conversión –tercera palabra– y fe. Él anuncia: \textquote{El tiempo se ha cumplido y el Reino de Dios está cerca}; después dirige la exhortación: \textquote{convertíos y creed en la Buena Nueva} (\textit{Mc} 1, 15), es decir creed en esta Buena Noticia que el Reino de Dios está cerca. En nuestra vida siempre necesitamos conversión –¡todos los días!–, y la Iglesia nos hace rezar por esto. De hecho, no estamos nunca suficientemente orientados hacia Dios y debemos continuamente dirigir nuestra mente y nuestro corazón a Él. Para hacer esto es necesario tener la valentía de rechazar todo lo que nos lleva fuera del camino, los falsos valores que nos engañan atrayendo nuestro egoísmo de forma sutil. Sin embargo, debemos fiarnos del Señor, de su bondad y de su proyecto de amor para cada uno de nosotros.

La Cuaresma es un tiempo de penitencia, sí, ¡pero no es un tiempo triste! Es un tiempo de penitencia, pero no es un tiempo triste, de luto. Es un compromiso alegre y serio para despojarnos de nuestro egoísmo, de nuestro hombre viejo, y renovarnos según la gracia de nuestro bautismo. Solamente Dios nos puede donar la verdadera felicidad: es inútil que perdamos nuestro tiempo buscándola en otro lugar, en las riquezas, en los placeres, en el poder, en la carrera\ldots El Reino de Dios es la realización de todas nuestras aspiraciones, porque es, al mismo tiempo, salvación del hombre y gloria de Dios.

En este primer domingo de Cuaresma, estamos invitados a escuchar con atención y recoger este llamamiento de Jesús a convertirnos y a creer en el Evangelio. Somos exhortados a iniciar con compromiso el camino hacia la Pascua, para acoger cada vez más la gracia de Dios, que quiere transformar el mundo en un reino de justicia, de paz, de fraternidad.

Que María Santísima nos ayude a vivir esta Cuaresma con fidelidad a la Palabra de Dios y con una oración incesante, como hizo Jesús en el desierto.

¡No es imposible! Se trata de vivir las jornadas con el deseo de acoger el amor que viene de Dios y que quiere transformar nuestra vida y el mundo entero.
\end{body}

\begin{patercite}
\textquote{Convertíos y creed en el Evangelio} (\ldots) La conversión, una palabra que hay que considerar en su extraordinaria seriedad, dándonos cuenta de la sorprendente novedad que implica. En efecto, la llamada a la conversión revela y denuncia la fácil superficialidad que con frecuencia caracteriza nuestra vida. Convertirse significa cambiar de dirección en el camino de la vida: pero no con un pequeño ajuste, sino con un verdadero cambio de sentido. Conversión es ir contracorriente, donde la \textquote{corriente} es el estilo de vida superficial, incoherente e ilusorio que a menudo nos arrastra, nos domina y nos hace esclavos del mal, o en cualquier caso prisioneros de la mediocridad moral. Con la conversión, en cambio, aspiramos a la medida alta de la vida cristiana, nos adherimos al Evangelio vivo y personal, que es Jesucristo. La meta final y el sentido profundo de la conversión es su persona, él es la senda por la que todos están llamados a caminar en la vida, dejándose iluminar por su luz y sostener por su fuerza que mueve nuestros pasos. De este modo la conversión manifiesta su rostro más espléndido y fascinante: no es una simple decisión moral, que rectifica nuestra conducta de vida, sino una elección de fe, que nos implica totalmente en la comunión íntima con la persona viva y concreta de Jesús. Convertirse y creer en el Evangelio no son dos cosas distintas o de alguna manera sólo conectadas entre sí, sino que expresan la misma realidad. La conversión es el \textquote{sí} total de quien entrega su existencia al Evangelio, respondiendo libremente a Cristo, que antes se ha ofrecido al hombre como camino, verdad y vida, como el único que lo libera y lo salva. Este es precisamente el sentido de las primeras palabras con las que, según el evangelista san Marcos, Jesús inicia la predicación del \textquote{Evangelio de Dios}: \textquote{El tiempo se ha cumplido y el reino de Dios está cerca; convertíos y creed en el Evangelio} (\textit{Mc} 1, 15).

\textbf{Benedicto XVI, papa}, \textit{Catequesis}, Audiencia general, 17 de febrero de 2010, parr. 3.
\end{patercite}

\newsection
\section{Temas}

\cceth{La tentación de Jesús}
 
\cceref{CEC 394, 538-540, 2119}

\begin{ccebody}
\n{394} La Escritura atestigua la influencia nefasta de aquel a quien Jesús llama \textquote{homicida desde el principio} (\textit{Jn} 8,44) y que incluso intentó apartarlo de la misión recibida del Padre (cf. \textit{Mt} 4,1-11). \textquote{El Hijo de Dios se manifestó para deshacer las obras del diablo} (\textit{1 Jn} 3,8). La más grave en consecuencias de estas obras ha sido la seducción mentirosa que ha inducido al hombre a desobedecer a Dios.

\ccesec{Las tentaciones de Jesús}

\n{538} Los evangelios hablan de un tiempo de soledad de Jesús en el desierto inmediatamente después de su bautismo por Juan: \textquote{Impulsado por el Espíritu} al desierto, Jesús permanece allí sin comer durante cuarenta días; vive entre los animales y los ángeles le servían (cf. \textit{Mc} 1, 12-13). Al final de este tiempo, Satanás le tienta tres veces tratando de poner a prueba su actitud filial hacia Dios. Jesús rechaza estos ataques que recapitulan las tentaciones de Adán en el Paraíso y las de Israel en el desierto, y el diablo se aleja de él \textquote{hasta el tiempo determinado} (\textit{Lc} 4, 13).

\n{539} Los evangelistas indican el sentido salvífico de este acontecimiento misterioso. Jesús es el nuevo Adán que permaneció fiel allí donde el primero sucumbió a la tentación. Jesús cumplió perfectamente la vocación de Israel: al contrario de los que anteriormente provocaron a Dios durante cuarenta años por el desierto (cf. \textit{Sal} 95, 10), Cristo se revela como el Siervo de Dios totalmente obediente a la voluntad divina. En esto Jesús es vencedor del diablo; él ha \textquote{atado al hombre fuerte} para despojarle de lo que se había apropiado (\textit{Mc} 3, 27). La victoria de Jesús en el desierto sobre el Tentador es un anticipo de la victoria de la Pasión, suprema obediencia de su amor filial al Padre.

\n{540} La tentación de Jesús manifiesta la manera que tiene de ser Mesías el Hijo de Dios, en oposición a la que le propone Satanás y a la que los hombres (cf. \textit{Mt} 16, 21-23) le quieren atribuir. Por eso Cristo ha vencido al Tentador \textit{en beneficio nuestro}: \textquote{Pues no tenemos un Sumo Sacerdote que no pueda compadecerse de nuestras flaquezas, sino probado en todo igual que nosotros, excepto en el pecado} (\textit{Hb} 4, 15). La Iglesia se une todos los años, durante los cuarenta días de \textit{la Gran Cuaresma}, al Misterio de Jesús en el desierto.

\n{2119} La acción de \textit{tentar a Dios} consiste en poner a prueba, de palabra o de obra, su bondad y su omnipotencia. Así es como Satán quería conseguir de Jesús que se arrojara del templo y obligase a Dios, mediante este gesto, a actuar (cf. \textit{Lc} 4, 9). Jesús le opone las palabras de Dios: \textquote{No tentaréis al Señor, tu Dios} (\textit{Dt} 6, 16). El reto que contiene este tentar a Dios lesiona el respeto y la confianza que debemos a nuestro Creador y Señor. Incluye siempre una duda respecto a su amor, su providencia y su poder (cf. \textit{1 Co} 10, 9; \textit{Ex} 17, 2-7; \textit{Sal} 95, 9).
\end{ccebody}

\cceth{\textquote{No nos dejes caer en la tentación}} 

\cceref{CEC 2846-2849}

\begin{ccebody}
\n{2846} Esta petición llega a la raíz de la anterior, porque nuestros pecados son los frutos del consentimiento a la tentación. Pedimos a nuestro Padre que no nos \textquote{deje caer} en ella. Traducir en una sola palabra el texto griego es difícil: significa \textquote{no permitas entrar en} (cf. \textit{Mt} 26, 41), \textquote{no nos dejes sucumbir a la tentación}. \textquote{Dios ni es tentado por el mal ni tienta a nadie} (\textit{St} 1, 13), al contrario, quiere librarnos del mal. Le pedimos que no nos deje tomar el camino que conduce al pecado, pues estamos empeñados en el combate \textquote{entre la carne y el Espíritu}. Esta petición implora el Espíritu de discernimiento y de fuerza.

\n{2847} El Espíritu Santo nos hace \textit{discernir} entre la prueba, necesaria para el crecimiento del hombre interior (cf. \textit{Lc} 8, 13-15; \textit{Hch} 14, 22; \textit{2 Tm} 3, 12) en orden a una \textquote{virtud probada} (\textit{Rm} 5, 3-5), y la tentación que conduce al pecado y a la muerte (cf. \textit{St} 1, 14-15). También debemos distinguir entre \textquote{ser tentado} y \textquote{consentir} en la tentación. Por último, el discernimiento desenmascara la mentira de la tentación: aparentemente su objeto es \textquote{bueno, seductor a la vista, deseable} (\textit{Gn} 3, 6), mientras que, en realidad, su fruto es la muerte.

\ccecite{\textquote{Dios no quiere imponer el bien, quiere seres libres [\ldots] En algo la tentación es buena. Todos, menos Dios, ignoran lo que nuestra alma ha recibido de Dios, incluso nosotros. Pero la tentación lo manifiesta para enseñarnos a conocernos, y así, descubrirnos nuestra miseria, y obligarnos a dar gracias por los bienes que la tentación nos ha manifestado} (Orígenes, \textit{De oratione}, 29, 15 y 17).}

\n{2848} \textquote{No entrar en la tentación} implica una \textit{decisión del corazón:} \textquote{Porque donde esté tu tesoro, allí también estará tu corazón [\ldots] Nadie puede servir a dos señores} (\textit{Mt} 6, 21-24). \textquote{Si vivimos según el Espíritu, obremos también según el Espíritu} (\textit{Ga} 5, 25). El Padre nos da la fuerza para este \textquote{dejarnos conducir} por el Espíritu Santo. \textquote{No habéis sufrido tentación superior a la medida humana. Y fiel es Dios que no permitirá que seáis tentados sobre vuestras fuerzas. Antes bien, con la tentación os dará modo de poderla resistir con éxito} (\textit{1 Co} 10, 13).

\n{2849} Pues bien, este combate y esta victoria sólo son posibles con la oración. Por medio de su oración, Jesús es vencedor del Tentador, desde el principio (cf. \textit{Mt} 4, 11) y en el último combate de su agonía (cf. \textit{Mt} 26, 36-44). En esta petición a nuestro Padre, Cristo nos une a su combate y a su agonía. La vigilancia del corazón es recordada con insistencia en comunión con la suya (cf. \textit{Mc} 13, 9. 23. 33-37; 14, 38; \textit{Lc} 12, 35-40). La vigilancia es \textquote{guarda del corazón}, y Jesús pide al Padre que \textquote{nos guarde en su Nombre} (\textit{Jn} 17, 11). El Espíritu Santo trata de despertarnos continuamente a esta vigilancia (cf. \textit{1 Co} 16, 13; \textit{Col} 4, 2; \textit{1 Ts} 5, 6; \textit{1 Pe} 5, 8). Esta petición adquiere todo su sentido dramático referida a la tentación final de nuestro combate en la tierra; pide la \textit{perseverancia final}. \textquote{Mira que vengo como ladrón. Dichoso el que esté en vela} (\textit{Ap} 16, 15).
\end{ccebody}

\newpage

\cceth{La Alianza con Noé} 
\cceref{CEC 56-58, 71}

\begin{ccebody}
\n{56} Una vez rota la unidad del género humano por el pecado, Dios decide desde el comienzo salvar a la humanidad a través de una serie de etapas. La alianza con Noé después del diluvio (cf. \textit{Gn} 9,9) expresa el principio de la Economía divina con las \textquote{naciones}, es decir con los hombres agrupados \textquote{según sus países, cada uno según su lengua, y según sus clanes} (\textit{Gn} 10,5; cf. \textit{Gn} 10,20-31).

\n{57} Este orden a la vez cósmico, social y religioso de la pluralidad de las naciones (cf. \textit{Hch} 17,26-27), está destinado a limitar el orgullo de una humanidad caída que, unánime en su perversidad (cf. \textit{Sb} 10,5), quisiera hacer por sí misma su unidad a la manera de Babel (cf. \textit{Gn} 11,4-6). Pero, a causa del pecado (cf. \textit{Rm} 1,18-25), el politeísmo, así como la idolatría de la nación y de su jefe, son una amenaza constante de vuelta al paganismo para esta economía aún no definitiva.

\n{58} La alianza con Noé permanece en vigor mientras dura el tiempo de las naciones (cf. \textit{Lc} 21,24), hasta la proclamación universal del Evangelio. La Biblia venera algunas grandes figuras de las \textquote{naciones}, como \textquote{Abel el justo}, el rey-sacerdote Melquisedec (cf. \textit{Gn} 14,18), figura de Cristo (cf. \textit{Hb} 7,3), o los justos \textquote{Noé, Daniel y Job} (\textit{Ez} 14,14). De esta manera, la Escritura expresa qué altura de santidad pueden alcanzar los que viven según la alianza de Noé en la espera de que Cristo \textquote{reúna en uno a todos los hijos de Dios dispersos} (\textit{Jn} 11,52).

\n{71} \textit{Dios selló con Noé una alianza eterna entre Él y todos los seres vivientes (cf.} \textit{Gn} 9,16). Esta alianza durará tanto como dure el mundo.
\end{ccebody}

\cceth{El Arca de Noé prefigura la Iglesia y el Bautismo} 

\cceref{CEC 845, 1094, 1219}

\begin{ccebody}
\n{845} El Padre quiso convocar a toda la humanidad en la Iglesia de su Hijo para reunir de nuevo a todos sus hijos que el pecado había dispersado y extraviado. La Iglesia es el lugar donde la humanidad debe volver a encontrar su unidad y su salvación. Ella es el \textquote{mundo reconciliado} (San Agustín, \textit{Sermo} 96, 7-9). Es, además, este barco que \textit{pleno dominicae crucis velo Sancti Spiritus flatu in hoc bene navigat mundo} – \textquote{con su velamen que es la cruz de Cristo, empujado por el Espíritu Santo, navega bien en este mundo} (San Ambrosio, \textit{De virginitate} 18, 119); según otra imagen estimada por los Padres de la Iglesia, está prefigurada por el Arca de Noé que es la única que salva del diluvio (cf. \textit{1 Pe} 3, 20-21).

\n{1094} Sobre esta armonía de los dos Testamentos (cf. DV 14-16) se articula la catequesis pascual del Señor (cf. \textit{Lc} 24,13-49), y luego la de los Apóstoles y de los Padres de la Iglesia. Esta catequesis pone de manifiesto lo que permanecía oculto bajo la letra del Antiguo Testamento: el misterio de Cristo. Es llamada catequesis \textquote{tipológica}, porque revela la novedad de Cristo a partir de \textquote{figuras} (tipos) que lo anunciaban en los hechos, las palabras y los símbolos de la primera Alianza. Por esta relectura en el Espíritu de Verdad a partir de Cristo, las figuras son explicadas (cf. \textit{2 Co} 3, 14-16). Así, el diluvio y el arca de Noé prefiguraban la salvación por el Bautismo (cf. \textit{1 Pe} 3, 21), y lo mismo la nube, y el paso del mar Rojo; el agua de la roca era la figura de los dones espirituales de Cristo (cf. \textit{1 Co} 10,1-6); el maná del desierto prefiguraba la Eucaristía \textquote{el verdadero Pan del Cielo} (\textit{Jn} 6,32).

\n{1219} La Iglesia ha visto en el arca de Noé una prefiguración de la salvación por el bautismo. En efecto, por medio de ella \textquote{unos pocos, es decir, ocho personas, fueron salvados a través del agua} (\textit{1 Pe} 3,20):

\ccecite{\textquote{¡Oh Dios!, que incluso en las aguas torrenciales del diluvio prefiguraste el nacimiento de la nueva humanidad, de modo que una misma agua pusiera fin al pecado y diera origen a la santidad} (\textit{Vigilia Pascual, Bendición del agua: Misal Romano}).}
\end{ccebody}

\cceth{Alianza y sacramentos (especialmente el Bautismo)} 

\cceref{CEC 1116, 1129, 1222}

\begin{ccebody}
\n{1116} Los sacramentos, como \textquote{fuerzas que brotan} del Cuerpo de Cristo (cf. \textit{Lc} 5,17; 6,19; 8,46) siempre vivo y vivificante, y como acciones del Espíritu Santo que actúa en su Cuerpo que es la Iglesia, son \textquote{las obras maestras de Dios} en la nueva y eterna Alianza.

\n{1129} La Iglesia afirma que para los creyentes los sacramentos de la Nueva Alianza son \textit{necesarios para la salvación} (cf. Concilio de Trento: DS 1604). La \textquote{gracia sacramental} es la gracia del Espíritu Santo dada por Cristo y propia de cada sacramento. El Espíritu cura y transforma a los que lo reciben conformándolos con el Hijo de Dios. El fruto de la vida sacramental consiste en que el Espíritu de adopción deifica (cf. \textit{2 Pe} 1,4) a los fieles uniéndolos vitalmente al Hijo único, el Salvador.

\n{1222} Finalmente, el Bautismo es prefigurado en el paso del Jordán, por el que el pueblo de Dios recibe el don de la tierra prometida a la descendencia de Abraham, imagen de la vida eterna. La promesa de esta herencia bienaventurada se cumple en la nueva Alianza.
\end{ccebody}

\cceth{Dios nos salva por medio del Bautismo} 

\cceref{CEC 1257, 1811}

\begin{ccebody}
\ccesec{La necesidad del Bautismo}

\n{1257} El Señor mismo afirma que el Bautismo es necesario para la salvación (cf. \textit{Jn} 3,5). Por ello mandó a sus discípulos a anunciar el Evangelio y bautizar a todas las naciones (cf. \textit{Mt} 28, 19-20; cf. DS 1618; LG 14; AG 5). El Bautismo es necesario para la salvación en aquellos a los que el Evangelio ha sido anunciado y han tenido la posibilidad de pedir este sacramento (cf. \textit{Mc} 16,16). La Iglesia no conoce otro medio que el Bautismo para asegurar la entrada en la bienaventuranza eterna; por eso está obligada a no descuidar la misión que ha recibido del Señor de hacer \textquote{renacer del agua y del Espíritu} a todos los que pueden ser bautizados. \textit{Dios ha vinculado la salvación al sacramento del Bautismo, sin embargo, Él no queda sometido a sus sacramentos}.

\n{1811} Para el hombre herido por el pecado no es fácil guardar el equilibrio moral. El don de la salvación por Cristo nos otorga la gracia necesaria para perseverar en la búsqueda de las virtudes. Cada cual debe pedir siempre esta gracia de luz y de fortaleza, recurrir a los sacramentos, cooperar con el Espíritu Santo, seguir sus invitaciones a amar el bien y guardarse del mal.
\end{ccebody}