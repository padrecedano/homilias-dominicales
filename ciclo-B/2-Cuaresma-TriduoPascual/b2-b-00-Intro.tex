\part{Tiempo de Cuaresma}
\chapter{Introducción al Tiempo~de~Cuaresma}
\begin{introstyle}
\section{Normativa litúrgica}
El tiempo de Cuaresma\anote{id2} está ordenado a la preparación de la celebración de Pascua. En efecto, la liturgia cuaresmal dispone a la celebración del Misterio Pascual, tanto a los catecúmenos, haciéndolos pasar por los diversos grados de la iniciación cristiana, como a los fieles, que recuerdan el bautismo y hacen penitencia\anote{id3}. Este tiempo va desde el Miércoles de Ceniza hasta la Misa de la Cena del Señor, exclusive. Desde el comienzo de Cuaresma hasta la Vigilia Pascual no se dice \textit{Aleluya}.

El miércoles que comienza la Cuaresma, que es en todas partes día de ayuno\anote{id4}, se imponen las cenizas.

Los domingos de este tiempo se llaman: primer, segundo, tercer, cuarto, y quinto, domingo de Cuaresma. El sexto domingo, con el que comienza la Semana Santa, se llama \textquote{Domingo de Ramos de la Pasión del Señor}.

La Semana Santa está destinada a conmemorar la Pasión de Cristo desde su entrada mesiánica en Jerusalén. 

Durante la mañana del Jueves Santo\anote{id5}, el Obispo, que concelebra la Misa con su presbiterio, bendice los óleos sagrados y consagra el santo crisma.

\newpage
\section{Lecturas del Leccionario}

Las lecturas\anote{id6} del Evangelio están distribuidas de la siguiente manera: en los domingos primero y segundo se conservan las narraciones de las tentaciones y de la transfiguración del Señor, según el Evangelio de Marcos. 

En los tres domingos siguientes se leen unos textos de san Juan sobre la futura glorificación de Cristo por su cruz y resurrección. Los Evangelios de la samaritana, del ciego de nacimiento y de la resurrección de Lázaro, que se leen en el año A, pueden leerse también en los años B y C, sobre todo cuando hay catecúmenos, debido a su gran importancia en relación con la iniciación cristiana.

El domingo de Ramos en la Pasión del Señor: para la procesión, se han escogido los textos que se refieren a la solemne entrada del Señor en Jerusalén. En el año B se puede optar por el texto de Marcos o de Juan, sobre el mismo tema; en la misa, se lee el relato de la pasión del Señor según san Marcos.

Las lecturas del Antiguo Testamento se refieren a la historia de la salvación, que es uno de los temas propios de la catequesis cuaresmal. Cada año hay una serie de textos que presentan los principales elementos de esta historia, desde el principio hasta la promesa de la nueva alianza. 

Las lecturas del Apóstol se han escogido de manera que tengan relación con las lecturas del Evangelio y del Antiguo Testamento y haya, en lo posible, una adecuada conexión entre las mismas.

Para el año B concretamente en el primer domingo leemos como primera lectura un pasaje sobre la alianza de Dios con Noé después del diluvio, el cual el apóstol Pedro en su primera carta relaciona con el Bautismo cristiano. 

El segundo domingo la primera lectura narra el sacrificio de Isaac, en quien el apóstol Pablo en la carta a los Romanos ve una imagen del sacrificio de Cristo.

El tercer domingo la primera lectura trae el tema del Decálogo, cuyo cumplimiento se realiza en la cruz de Cristo como relata san Pablo en la primera carta a los Corintios.

El cuatro domingo la primera lectura habla sobre la destrucción del tempo y el exilio del pueblo elegido a causa de su infidelidad. La respuesta de Dios es mostrada por san Pablo en la carta a los Efesios: Dios envía de nuevo su salvación por medio de Jesucristo.


\newpage 
\section{Indicaciones pastorales y de piedad sobre la Cuaresma}

La Cuaresma\anote{id7} es un tiempo de escucha de la Palabra de Dios y de conversión, de preparación y de memoria del Bautismo, de reconciliación con Dios y con los hermanos, de recurso más frecuente a las \textquote{armas de la penitencia cristiana}: la oración, el ayuno y la limosna (cfr. Mt 6,1-6.16-18).

En el ámbito de la piedad popular no se percibe fácilmente el sentido mistérico de la Cuaresma y no se han asimilado algunos de los grandes valores y temas, como la relación entre el \textquote{sacramento de los cuarenta días} y los sacramentos de la iniciación cristiana, o el misterio del \textquote{éxodo}, presente a lo largo de todo el itinerario cuaresmal. Según una constante de la piedad popular, que tiende a centrarse en los misterios de la humanidad de Cristo, en la Cuaresma los fieles concentran su atención en la Pasión y Muerte del Señor.

El comienzo de los cuarenta días de penitencia, en el Rito romano, se caracteriza por el austero símbolo de las Cenizas, que distingue la Liturgia del Miércoles de Ceniza. Propio de los antiguos ritos con los que los pecadores convertidos se sometían a la penitencia canónica, el gesto de cubrirse con ceniza tiene el sentido de reconocer la propia fragilidad y mortalidad, que necesita ser redimida por la misericordia de Dios. Lejos de ser un gesto puramente exterior, la Iglesia lo ha conservado como signo de la actitud del corazón penitente que cada bautizado está llamado a asumir en el itinerario cuaresmal. Se debe ayudar a los fieles, que acuden en gran número a recibir la Ceniza, a que capten el significado interior que tiene este gesto, que abre a la conversión y al esfuerzo de la renovación pascual.

A pesar de la secularización de la sociedad contemporánea, el pueblo cristiano advierte claramente que durante la Cuaresma hay que dirigir el espíritu hacia las realidades que son verdaderamente importantes; que hace falta un esfuerzo evangélico y una coherencia de vida, traducida en buenas obras, en formas de renuncia a lo superfluo y suntuoso, en expresiones de solidaridad con los que sufren y con los necesitados.

También los fieles que frecuentan poco los sacramentos de la Penitencia y de la Eucaristía saben, por una larga tradición eclesial, que el tiempo de Cuaresma-Pascua está en relación con el precepto de la Iglesia de confesar los propios pecados graves, al menos una vez al año, preferentemente en el tiempo pascual.

La divergencia existente entre la concepción litúrgica y la visión popular de la Cuaresma, no impide que el tiempo de los \textquote{Cuarenta días} sea un espacio propicio para una interacción fecunda entre Liturgia y piedad popular.

Un ejemplo de esta interacción lo tenemos en el hecho de que la piedad popular favorece algunos días, algunos ejercicios de piedad y algunas actividades apostólicas y caritativas, que la misma Liturgia cuaresmal prevé y recomienda. La práctica del ayuno, tan característica desde la antigüedad en este tiempo litúrgico, es un \textquote{ejercicio} que libera voluntariamente de las necesidades de la vida terrena para redescubrir la necesidad de la vida que viene del cielo: \textquote{No sólo de pan vive el hombre, sino de toda palabra que sale de la boca de Dios} (Mt 4,4; cfr. Dt 8,3; Lc 4,4; antífona de comunión del I Domingo de Cuaresma)

\subsubsection{La veneración de Cristo crucificado}

El camino cuaresmal termina con el comienzo del Triduo pascual, es decir, con la celebración de la Misa In Cena Domini. En el Triduo pascual, el Viernes Santo, dedicado a celebrar la Pasión del Señor, es el día por excelencia para la \textquote{Adoración de la santa Cruz}.

Sin embargo, la piedad popular desea anticipar la veneración cultual de la Cruz. De hecho, a lo largo de todo el tiempo cuaresmal, el viernes, que por una antiquísima tradición cristiana es el día conmemorativo de la Pasión de Cristo, los fieles dirigen con gusto su piedad hacia el misterio de la Cruz.

Contemplando al Salvador crucificado captan más fácilmente el significado del dolor inmenso e injusto que Jesús, el Santo, el Inocente, padeció por la salvación del hombre, y comprenden también el valor de su amor solidario y la eficacia de su sacrificio redentor.

Las expresiones de devoción a Cristo crucificado, numerosas y variadas, adquieren un particular relieve en las iglesias dedicadas al misterio de la Cruz o en las que se veneran reliquias, consideradas auténticas, del lignum Crucis. La \textquote{invención de la Cruz}, acaecida según la tradición durante la primera mitad del siglo IV, con la consiguiente difusión por todo el mundo de fragmentos de la misma, objeto de grandísima veneración, determinó un aumento notable del culto a la Cruz.

En las manifestaciones de devoción a Cristo crucificado, los elementos acostumbrados de la piedad popular como cantos y oraciones, gestos como la ostensión y el beso de la cruz, la procesión y la bendición con la cruz, se combinan de diversas maneras, dando lugar a ejercicios de piedad que a veces resultan preciosos por su contenido y por su forma.

No obstante, la piedad respecto a la Cruz, con frecuencia, tiene necesidad de ser iluminada. Se debe mostrar a los fieles la referencia esencial de la Cruz al acontecimiento de la Resurrección: la Cruz y el sepulcro vacío, la Muerte y la Resurrección de Cristo, son inseparables en la narración evangélica y en el designio salvífico de Dios. En la fe cristiana, la Cruz es expresión del triunfo sobre el poder de las tinieblas, y por esto se la presenta adornada con gemas y convertida en signo de bendición, tanto cuando se traza sobre uno mismo, como cuando se traza sobre otras personas y objetos.

El texto evangélico, particularmente detallado en la narración de los diversos episodios de la Pasión, y la tendencia a especificar y a diferenciar, propia de la piedad popular, ha hecho que los fieles dirijan su atención, también, a aspectos particulares de la Pasión de Cristo y hayan hecho de ellos objeto de diferentes devociones: el \textquote{Ecce homo}, el Cristo vilipendiado, \textquote{con la corona de espinas y el manto de púrpura} (Jn 19,5), que Pilato muestra al pueblo; las llagas del Señor, sobre todo la herida del costado y la sangre vivificadora que brota de allí (cfr. Jn 19,34); los instrumentos de la Pasión, como la columna de la flagelación, la escalera del pretorio, la corona de espinas, los clavos, la lanza de la transfixión; la sábana santa o lienzo de la deposición.

Estas expresiones de piedad, promovidas en ocasiones por personas de santidad eminente, son legítimas. Sin embargo, para evitar una división excesiva en la contemplación del misterio de la Cruz, será conveniente subrayar la consideración de conjunto de todo el acontecimiento de la Pasión, conforme a la tradición bíblica y patrística.

\subsubsection{La lectura de la Pasión del Señor}

La Iglesia exhorta a los fieles a la lectura frecuente, de manera individual o comunitaria, de la Palabra de Dios. Ahora bien, no hay duda de que entre las páginas de la Biblia, la narración de la Pasión del Señor tiene un valor pastoral especial, por lo que, por ejemplo, el Ordo unctionis infirmorum eorumque pastoralis curae sugiere la lectura, en el momento de la agonía del cristiano, de la narración de la Pasión del Señor o de algún pasaje de la misma.

Durante el tiempo de Cuaresma, el amor a Cristo crucificado deberá llevar a la comunidad cristiana a preferir el miércoles y el viernes, sobre todo, para la lectura de la Pasión del Señor.

Esta lectura, de gran sentido doctrinal, atrae la atención de los fieles tanto por el contenido como por la estructura narrativa, y suscita en ellos sentimientos de auténtica piedad: arrepentimiento de las culpas cometidas, porque los fieles perciben que la Muerte de Cristo ha sucedido para remisión de los pecados de todo el género humano y también de los propios; compasión y solidaridad con el Inocente injustamente perseguido; gratitud por el amor infinito que Jesús, el Hermano primogénito, ha demostrado en su Pasión para con todos los hombres, sus hermanos; decisión de seguir los ejemplos de mansedumbre, paciencia, misericordia, perdón de las ofensas y abandono confiado en las manos del Padre, que Jesús dio de modo abundante y eficaz durante su Pasión.

Fuera de la celebración litúrgica, la lectura de la Pasión se puede \textquote{dramatizar} si es oportuno, confiando a lectores distintos los textos correspondientes a los diversos personajes; asimismo, se pueden intercalar cantos o momentos de silencio meditativo.

\subsubsection{El \textquote{Vía Crucis}}

Entre los ejercicios de piedad con los que los fieles veneran la Pasión del Señor, hay pocos que sean tan estimados como el Vía Crucis. A través de este ejercicio de piedad los fieles recorren, participando con su afecto, el último tramo del camino recorrido por Jesús durante su vida terrena: del Monte de los Olivos, donde en el \textquote{huerto llamado Getsemaní} (Mc 14,32) el Señor fue \textquote{presa de la angustia} (Lc 22,44), hasta el Monte Calvario, donde fue crucificado entre dos malhechores (cfr. Lc 23,33), al jardín donde fue sepultado en un sepulcro nuevo, excavado en la roca (cfr. Jn 19,40-42).

Un testimonio del amor del pueblo cristiano por este ejercicio de piedad son los innumerables Vía Crucis erigidos en las iglesias, en los santuarios, en los claustros e incluso al aire libre, en el campo, o en la subida a una colina, a la cual las diversas estaciones le confieren una fisonomía sugestiva.

El Vía Crucis es la síntesis de varias devociones surgidas desde la alta Edad Media: la peregrinación a Tierra Santa, durante la cual los fieles visitan devotamente los lugares de la Pasión del Señor; la devoción a las \textquote{caídas de Cristo} bajo el peso de la Cruz; la devoción a los \textquote{caminos dolorosos de Cristo}, que consiste en ir en procesión de una iglesia a otra en memoria de los recorridos de Cristo durante su Pasión; la devoción a las \textquote{estaciones de Cristo}, esto es, a los momentos en los que Jesús se detiene durante su camino al Calvario, o porque le obligan sus verdugos o porque está agotado por la fatiga, o porque, movido por el amor, trata de entablar un diálogo con los hombres y mujeres que asisten a su Pasión.

En su forma actual, que está ya atestiguada en la primera mitad del siglo XVII, el Vía Crucis, difundido sobre todo por San Leonardo de Porto Mauricio (+1751), ha sido aprobado por la Sede Apostólica, dotado de indulgencias y consta de catorce estaciones.

El Vía Crucis es un camino trazado por el Espíritu Santo, fuego divino que ardía en el pecho de Cristo (cfr. Lc 12,49-50) y lo impulsó hasta el Calvario; es un camino amado por la Iglesia, que ha conservado la memoria viva de las palabras y de los acontecimientos de los último días de su Esposo y Señor.

En el ejercicio de piedad del Vía Crucis confluyen también diversas expresiones características de la espiritualidad cristiana: la comprensión de la vida como camino o peregrinación; como paso, a través del misterio de la Cruz, del exilio terreno a la patria celeste; el deseo de conformarse profundamente con la Pasión de Cristo; las exigencias de la sequela Christi, según la cual el discípulo debe caminar detrás del Maestro, llevando cada día su propia cruz (cfr. Lc 9,23)

Por todo esto el Vía Crucis es un ejercicio de piedad especialmente adecuado al tiempo de Cuaresma.

Para realizar con fruto el Vía Crucis pueden ser útiles las siguientes indicaciones:

– la forma tradicional, con sus catorce estaciones, se debe considerar como la forma típica de este ejercicio de piedad; sin embargo, en algunas ocasiones, no se debe excluir la sustitución de una u otra \textquote{estación} por otras que reflejen episodios evangélicos del camino doloroso de Cristo, y que no se consideran en la forma tradicional;

– en todo caso, existen formas alternativas del Vía Crucis aprobadas por la Sede Apostólica o usadas públicamente por el Romano Pontífice: estas se deben considerar formas auténticas del mismo, que se pueden emplear según sea oportuno;

– el Vía Crucis es un ejercicio de piedad que se refiere a la Pasión de Cristo; sin embargo es oportuno que concluya de manera que los fieles se abran a la expectativa, llena de fe y de esperanza, de la Resurrección; tomando como modelo la estación de la Anastasis al final del Vía Crucis de Jerusalén, se puede concluir el ejercicio de piedad con la memoria de la Resurrección del Señor.

Los textos para el Vía Crucis son innumerables. Han sido compuestos por pastores movidos por una sincera estima a este ejercicio de piedad y convencidos de su eficacia espiritual; otras veces tienen por autores a fieles laicos, eminentes por la santidad de vida, doctrina o talento literario.

La selección del texto, teniendo presente las eventuales indicaciones del Obispo, se deberá hacer considerando sobre todo las características de los que participan en el ejercicio de piedad y el principio pastoral de combinar sabiamente la continuidad y la innovación. En todo caso, serán preferibles los textos en los que resuenen, correctamente aplicadas, las palabras de la Biblia, y que estén escritos con un estilo digno y sencillo.

Un desarrollo inteligente del Vía Crucis, en el que se alternan de manera equilibrada: palabra, silencio, canto, movimiento procesional y parada meditativa, contribuye a que se obtengan los frutos espirituales de este ejercicio de piedad.

\newpage
\subsubsection{El \textquote{Vía Matris}}

Así como en el plan salvífico de Dios (cfr. Lc 2,34-35) están asociados Cristo crucificado y la Virgen dolorosa, también los están en la Liturgia y en la piedad popular.

Como Cristo es el \textquote{hombre de dolores} (Is 53,3), por medio del cual se ha complacido Dios en \textquote{reconciliar consigo todos los seres: los del cielo y los de la tierra, haciendo la paz por la sangre de su cruz} (Col 1,20), así María es la \textquote{mujer del dolor}, que Dios ha querido asociar a su Hijo, como madre y partícipe de su Pasión (socia Passionis).

Desde los días de la infancia de Cristo, toda la vida de la Virgen, participando del rechazo de que era objeto su Hijo, transcurrió bajo el signo de la espada (cfr. Lc 2,35). Sin embargo, la piedad del pueblo cristiano ha señalado siete episodios principales en la vida dolorosa de la Madre y los ha considerado como los \textquote{siete dolores} de Santa María Virgen.

Así, según el modelo del Vía Crucis, ha nacido el ejercicio de piedad del Vía Matris dolorosae, o simplemente Vía Matris, aprobado también por la Sede Apostólica. Desde el siglo XVI hay ya formas incipientes del Vía Matris, pero en su forma actual no es anterior al siglo XIX. La intuición fundamental es considerar toda la vida de la Virgen, desde el anuncio profético de Simeón (cfr. Lc 2,34-35) hasta la muerte y sepultura del Hijo, como un camino de fe y de dolor: camino articulado en siete \textquote{estaciones}, que corresponden a los \textquote{siete dolores} de la Madre del Señor.

El ejercicio de piedad del Vía Matris se armoniza bien con algunos temas propios del itinerario cuaresmal. Como el dolor de la Virgen tiene su causa en el rechazo que Cristo ha sufrido por parte de los hombres, el Vía Matris remite constante y necesariamente al misterio de Cristo, siervo sufriente del Señor (cfr. Is 52,13–53,12), rechazado por su propio pueblo (cfr. Jn 1,11; Lc 2,1-7; 2,34-35; 4,28-29; Mt 26,47-56; Hech 12,1-5). Y remite también al misterio de la Iglesia: las estaciones del Vía Matris son etapas del camino de fe y dolor en el que la Virgen ha precedido a la Iglesia y que esta deberá recorrer hasta el final de los tiempos.

El Vía Matris tiene como máxima expresión la \textquote{Piedad}, tema inagotable del arte cristiano desde la Edad Media.

\newpage
\section{La Semana Santa}

\textquote{Durante la Semana Santa la Iglesia celebra los misterios de la salvación actuados por Cristo en los últimos días de su vida, comenzando por su entrada mesiánica en Jerusalén}.

Es muy intensa la participación del pueblo en los ritos de la Semana Santa. Algunos muestran todavía señales de su origen en el ámbito de la piedad popular. Sin embargo ha sucedido que, a lo largo de los siglos, se ha producido en los ritos de la Semana Santa una especie de paralelismo celebrativo, por lo cual se dan prácticamente dos ciclos con planteamiento diverso: uno rigurosamente litúrgico, otro caracterizado por ejercicios de piedad específicos, sobre todo las procesiones.

Esta diferencia se debería reconducir a una correcta armonización entre las celebraciones litúrgicas y los ejercicios de piedad. En relación con la Semana Santa, el amor y el cuidado de las manifestaciones de piedad tradicionalmente estimadas por el pueblo debe llevar necesariamente a valorar las acciones litúrgicas, sostenidas ciertamente por los actos de piedad popular.

\section{Domingo de Ramos}

\textit{Las palmas y los ramos de olivo o de otros árboles}

La Semana Santa comienza con el Domingo de Ramos \textquote{de la Pasión del Señor}, que comprende a la vez el triunfo real de Cristo y el anuncio de la Pasión.

La procesión que conmemora la entrada mesiánica de Jesús en Jerusalén tiene un carácter festivo y popular. A los fieles les gusta conservar en sus hogares, y a veces en el lugar de trabajo, los ramos de olivo o de otros árboles, que han sido bendecidos y llevados en la procesión.

Sin embargo es preciso instruir a los fieles sobre el significado de la celebración, para que entiendan su sentido. Será oportuno, por ejemplo, insistir en que lo verdaderamente importante es participar en la procesión y no simplemente procurarse una palma o ramo de olivo; que estos no se conserven como si fueran amuletos, con un fin curativo o para mantener alejados a los malos espíritus y evitar así, en las casas y los campos, los daños que causan, lo cual podría ser una forma de superstición.

La palma y el ramo de olivo se conservan, ante todo, como un testimonio de la fe en Cristo, rey mesiánico, y en su victoria pascual.
\end{introstyle}