\chapter{Domingo de Ramos en~la~Pasión~del~Señor~(B)}

\section{Procesión}

\rtitle{EVANGELIO (opción 1)}

\rbook{Del Santo Evangelio según san Marcos} \rred{11, 1-10}

\rtheme{Bendito el que viene en nombre del Señor}

\begin{scripture}
Cuando se acercaban a Jerusalén, por Betfagé y Betania, junto al monte de los Olivos, mandó a dos de sus discípulos, diciéndoles:

\>{Id a la aldea de enfrente y, en cuanto entréis, encontraréis un pollino atado, que nadie ha montado todavía. Desatadlo y traedlo. Y si alguien os pregunta por qué lo hacéis, contestadle: \textquote{El Señor lo necesita, y lo devolverá pronto}}.

Fueron y encontraron el pollino en la calle atado a una puerta; y lo soltaron. Algunos de los presentes les preguntaron:

\>{¿Qué hacéis desatando el pollino?}.

Ellos les contestaron como había dicho Jesús; y se lo permitieron.

Llevaron el pollino, le echaron encima los mantos, y Jesús se montó. Muchos alfombraron el camino con sus mantos, otros con ramas cortadas en el campo. Los que iban delante y detrás, gritaban:

\>{¡Hosanna! ¡Bendito el que viene en nombre del Señor! ¡Bendito el reino que llega, el de nuestro padre David! ¡Hosanna en las alturas!}.
\end{scripture}

\newpage
\rtitle{EVANGELIO (opción 2)}

\rbook{Del Evangelio según san Juan} \rred{12, 12-16}

\rtheme{Bendito el que viene en nombre del Señor}

\begin{scripture}
En aquel tiempo, la gran multitud de gente que había venido a la fiesta, al oír que Jesús venía a Jerusalén, tomaron ramos de palmeras y salieron a su encuentro gritando:

\>{¡Hosanna! ¡Bendito el que viene en nombre del Señor, el Rey de Israel}.

Encontrando Jesús un pollino montó sobre él, como está escrito:

\>{No temas, hija de Sión; he aquí que viene tu Rey, sentado sobre un pollino de asna}.

Estas cosas no las comprendieron sus discípulos al principio, pero cuando Jesús fue glorificado, entonces se acordaron de que esto estaba escrito acerca de él y que así lo habían hecho para con él.
\end{scripture}

\img{corona_espinas}

\newsection
\section{Lecturas}

\rtitle{PRIMERA LECTURA}

\rbook{Del libro del profeta Isaías} \rred{50, 4-7}

\rtheme{No escondí el rostro ante ultrajes, sabiendo que no quedaría defraudado}


\begin{readprose}
El Señor Dios me ha dado una lengua de discípulo; 
   para saber decir al abatido una palabra de aliento. 

Cada mañana me espabila el oído, 
   para que escuche como los discípulos.
   
El Señor Dios me abrió el oído; 
   yo no resistí ni me eché atrás.

Ofrecí la espalda a los que me golpeaban, 
   las mejillas a los que mesaban mi barba; 
   no escondí el rostro ante ultrajes y salivazos.

El Señor Dios me ayuda, 
   por eso no sentía los ultrajes; 
   por eso endurecí el rostro como pedernal, 
   sabiendo que no quedaría defraudado.\end{readprose}

\newpage
\rtitle{SALMO RESPONSORIAL}

\rbook{Salmo} \rred{21, 8-9. 17-18a. 19-20. 23-24}

\rtheme{Dios mío, Dios mío, ¿por qué me has abandonado?}

\begin{psbody}
Al verme, se burlan de mí,
hacen visajes, menean la cabeza:
\textquote{Acudió al Señor, que lo ponga a salvo;
que lo libre si tanto lo quiere}. 

Me acorrala una jauría de mastines,
me cerca una banda de malhechores;
me taladran las manos y los pies,
puedo contar mis huesos. 

Se reparten mi ropa,
echan a suerte mi túnica.
Pero tú, Señor, no te quedes lejos;
fuerza mía, ven corriendo a ayudarme. 

Contaré tu fama a mis hermanos,
en medio de la asamblea te alabaré.
\textquote{Los que teméis al Señor, alabadlo;
linaje de Jacob, glorificadlo;
temedlo, linaje de Israel}. 
\end{psbody}

\newpage
\rtitle{SEGUNDA LECTURA}

\rbook{De la carta del apóstol san Pablo a los Filipenses} \rred{2, 6-11}

\rtheme{Se humilló a sí mismo; por eso Dios lo exaltó sobre todo}

\begin{readprose}

Cristo Jesús, siendo de condición divina,
   no retuvo ávidamente el ser igual a Dios;
   al contrario, se despojó de sí mismo 
   tomando la condición de esclavo,
   hecho semejante a los hombres.

Y así, reconocido como hombre por su presencia,
   se humilló a sí mismo,
   hecho obediente hasta la muerte, 
   y una muerte de cruz.

Por eso Dios lo exaltó sobre todo
   y le concedió el Nombre-sobre-todo-nombre;
   de modo que al nombre de Jesús
   toda rodilla se doble
   en el cielo, en la tierra, en el abismo,
   y toda lengua proclame:

   Jesucristo es Señor,
   para gloria de Dios Padre.
\end{readprose}

\rtitle{EVANGELIO}

\rbook{Pasión de nuestro Señor Jesucristo según san Marcos} \rred{Mc 14, 1 – 15, 47}

\rbr{Omitimos el texto del Evangelio debido a su gran extensión.}

\img{cruz_copta}

\newsection
\section{Comentarios Patrísticos}

\subsection{San Cirilo de Alejandría, obispo}

\ptheme{Mirad al Rey justo}

\src{Comentario sobre el libro del profeta Isaías, \\Lib. 4, orat 2: PG 70, 967-970.}

\begin{body}
\ltr{M}{\textit{irad:}} \textit{un rey reinará con justicia y sus jefes gobernarán según derecho}. El Verbo, Unigénito de Dios, era el Rey universal juntamente con Dios Padre y, al venir, se sometió toda criatura visible e invisible. Y si bien el hombre terreno, alejándose y desvinculándose de su reino, hizo poco caso de sus mandatos hasta el punto de dejarse enredar por la tiránica mano del diablo con los lazos del pecado, él, administrador y dispensador de toda justicia, nuevamente volvió a someterle a su yugo. Los caminos del Señor son efectivamente rectos.

Llamamos caminos de Cristo a los oráculos evangélicos, por medio de los cuales, atentos a todo tipo de virtud y ornando nuestras cabezas con las insignias de la piedad, conseguimos el premio de nuestra vocación celestial. Rectos son realmente estos caminos, sin curva o perversidad alguna: los llamaríamos rectos y transitables. Está efectivamente escrito: \textit{La senda del justo es recta, tú allanas el sendero del justo}. Pues la senda de la ley es áspera, serpentea entre símbolos y figuras y es de una intolerable dificultad. En cambio, el camino de los oráculos evangélicos es llano, sin absolutamente nada de áspero o escabroso.

Así, pues, los caminos de Cristo son rectos. Él ha edificado la ciudad santa, esto es, la Iglesia, en la que él mismo ha establecido su morada. Él, en efecto, habita en los santos y nosotros nos hemos convertido en templos de Dios vivo, pues, por la participación del Espíritu Santo, tenemos a Cristo dentro de nosotros. Fundó, pues, la Iglesia y él es el cimiento sobre el que también nosotros, como piedras suntuosas y preciosas, nos vamos integrando en la construcción del templo santo, para ser morada de Dios, por el Espíritu.

Absolutamente inconmovible es la Iglesia que tiene a Cristo por fundamento y base inamovible. \textit{Mirad }–dice–, \textit{yo coloco en Sión una piedra probada, angular, preciosa, de cimiento: \textquote{quien en ella se apoya no vacila}}. Así que, una vez fundada la Iglesia, él mismo cambió la suerte de su pueblo. Y a nosotros, derribado por tierra el tirano, nos salvó y liberó del pecado y nos sometió a su yugo, y no precisamente pagándole un precio o a base de regalos. Claramente lo dice uno de sus discípulos: \textit{Nos rescataron de ese proceder inútil recibido de nuestros padres: no con bienes efímeros, con oro y plata, sino a precio de la sangre de Cristo, el Cordero sin defecto ni mancha}. Cristo dio por nosotros su propia sangre: por tanto no nos pertenecemos, sino que somos del que nos compró y nos salvó.

Con razón, pues, todos cuantos conculcan la recta norma de la verdadera fe, se ven acusados por boca de los santos como negadores del Dios que los ha redimido.
\end{body}

\begin{patercite}
	\textit{Os exhorto, por la misericordia de Dios}, nos dice san Pablo. Él nos	exhorta, o mejor dicho, Dios nos exhorta por medio de él. El Señor se	presenta como quien ruega porque prefiere ser amado que temido, y le agrada más mostrarse como Padre que aparecer como Señor. Dios, pues, 	suplica por misericordia para no tener que castigar con rigor.
	
	Escucha cómo suplica el Señor: «Mirad y contemplad en mí vuestro mismo cuerpo, vuestros miembros, vuestras entrañas, vuestros huesos, vuestra sangre. Y si ante lo que es propio de Dios teméis, ¿por qué no amáis al contemplar lo que es de vuestra misma naturaleza? Si teméis a Dios como Señor, por qué no acudís a él como Padre? Pero quizá sea la inmensidad de mi pasión, cuyos responsables fuisteis vosotros, lo que os confunde. No temáis. Esta cruz no es mi aguijón, sino el aguijón de la muerte. Estos clavos no me infligen dolor, lo que hacen es acrecentar en mí el amor por vosotros. Estas llagas no provocan mis gemidos, lo que hacen es introduciros más en mis entrañas. Mi cuerpo al ser extendido en la cruz os acoge con un seno más dilatado pero no aumenta mi sufrimiento. Mi sangre no es para mí una pérdida, sino el pago de vuestro precio. Venid, pues, retornad, y comprobaréis que soy un padre, que devuelvo bien por mal, amor por injurias, inmensa caridad como paga de las muchas heridas».
	
	[\ldots]
	
	Hombre, procura, pues, ser tú mismo el sacrificio y el sacerdote de Dios. No desprecies lo que el poder de Dios te ha dado y concedido. Revístete con la túnica de la santidad, que la castidad sea tu ceñidor, que Cristo sea el casco de tu cabeza, que la cruz defienda tu frente, que en tu pecho more el conocimiento de los misterios de Dios, que tu oración arda continuamente, como perfume de incienso: toma en tus manos la espada del Espíritu: haz de tu corazón un altar, y así, afianzado en Dios, presenta tu cuerpo al Señor como sacrificio. Dios te pide fe, no desea tu muerte; tiene sed de tu entrega, no de tu sangre; se aplaca, no con tu muerte, sino con tu buena voluntad.
	
	\textbf{San Pedro Crisólogo}, \textit{Sermón} 108: cf. PL 52, 499-500.
\end{patercite}


\newpage

\subsection{Benedicto XVI, papa}

\ptheme{Pasión que conduce a la Resurrección}

\src{Catequesis sobrel el Salmo 22[21], 14 de septiembre del 2011.}

\begin{body} 
\ltr{E}{l} \emph{Salmo} \emph{22}, según la tradición judía, 21 según la tradición greco-latina, es un Salmo con fuertes implicaciones cristológicas, que continuamente aparece en los relatos de la pasión de Jesús, con su doble dimensión de humillación y de gloria, de muerte y de vida. Es una oración triste y conmovedora, de una profundidad humana y una riqueza teológica que hacen que sea uno de los Salmos más rezados y estudiados de todo el Salterio. Se trata de una larga composición poética, y nosotros nos detendremos en particular en la primera parte, centrada en el lamento, para profundizar algunas dimensiones significativas de la oración de súplica a Dios. 

Este Salmo presenta la figura de un inocente perseguido y circundado por los adversarios que quieren su muerte; y él recurre a Dios en un lamento doloroso que, en la certeza de la fe, se abre misteriosamente a la alabanza. En su oración se alternan la realidad angustiosa del presente y la memoria consoladora del pasado, en una sufrida toma de conciencia de la propia situación desesperada que, sin embargo, no quiere renunciar a la esperanza. Su grito inicial es un llamamiento dirigido a un Dios que parece lejano, que no responde y parece haberlo abandonado: 

\textquote{\emph{Dios mío, Dios mío, ¿por qué me has abandonado? A pesar de mis gritos, mi oración no te alcanza. Dios mío, de día te grito, y no me respondes; de noche, y no me haces caso}} (vv. 2-3). 

Dios calla, y este silencio lacera el ánimo del orante, que llama incesantemente, pero sin encontrar respuesta. Los días y las noches se suceden en una búsqueda incansable de una palabra, de una ayuda que no llega; Dios parece tan distante, olvidadizo, tan ausente. La oración pide escucha y respuesta, solicita un contacto, busca una relación que pueda dar consuelo y salvación. Pero si Dios no responde, el grito de ayuda se pierde en el vacío y la soledad llega a ser insostenible. Sin embargo, el orante de nuestro Salmo tres veces, en su grito, llama al Señor \textquote{mi} Dios, en un extremo acto de confianza y de fe. No obstante toda apariencia, el salmista no puede creer que el vínculo con el Señor se haya interrumpido totalmente; y mientras pregunta el por qué de un supuesto abandono incomprensible, afirma que \textquote{su} Dios no lo puede abandonar. 

Como es sabido, el grito inicial del Salmo, \textquote{Dios mío, Dios mío, ¿por qué me has abandonado?}, es citado por los evangelios de san Mateo y de san Marcos como el grito lanzado por Jesús moribundo en la cruz (cf. \emph{Mt} 27, 46; \emph{Mc} 15, 34). Ello expresa toda la desolación del Mesías, Hijo de Dios, que está afrontando el drama de la muerte, una realidad totalmente contrapuesta al Señor de la vida. Abandonado por casi todos los suyos, traicionado y negado por los discípulos, circundado por quien lo insulta, Jesús está bajo el peso aplastante de una misión que debe pasar por la humillación y la aniquilación. Por ello grita al Padre, y su sufrimiento asume las sufridas palabras del Salmo. Pero su grito no es un grito desesperado, como no lo era el grito del salmista, en cuya súplica recorre un camino atormentado, desembocando al final en una perspectiva de alabanza, en la confianza de la victoria divina. Puesto que en la costumbre judía citar el comienzo de un Salmo implicaba una referencia a todo el poema, la oración desgarradora de Jesús, incluso manteniendo su tono de sufrimiento indecible, se abre a la certeza de la gloria. \textquote{¿No era necesario que el Mesías padeciera esto y entrara así en su gloria?}, dirá el Resucitado a los discípulos de Emaús (\emph{Lc} 24, 26). En su Pasión, en obediencia al Padre, el Señor Jesús pasa por el abandono y la muerte para alcanzar la vida y donarla a todos los creyentes. 

A este grito inicial de súplica, en nuestro \emph{Salmo 22}, responde, en doloroso contraste, el recuerdo del pasado: 

\textquote{\emph{En ti confiaban nuestros padres, confiaban, y los ponías a salvo; a ti gritaban, y quedaban libres, en ti confiaban, y no los defraudaste}} (vv. 5-6). 

Aquel Dios que al salmista parece hoy tan lejano, es, sin embargo, el Señor misericordioso que Israel siempre experimentó en su historia. El pueblo al cual pertenece el orante fue objeto del amor de Dios y puede testimoniar su fidelidad. Comenzando por los patriarcas, luego en Egipto y en la larga peregrinación por el desierto, en la permanencia en la tierra prometida en contacto con poblaciones agresivas y enemigas, hasta la oscuridad del exilio, toda la historia bíblica fue una historia de clamores de ayuda por parte del pueblo y de respuestas salvíficas por parte de Dios. Y el salmista hace referencia a la fe inquebrantable de sus padres, que \textquote{confiaron} ---por tres veces se repite esta palabra--- sin quedar nunca decepcionados. Ahora, sin embargo, parece que esta cadena de invocaciones confiadas y respuestas divinas se haya interrumpido; la situación del salmista parece desmentir toda la historia de la salvación, haciendo todavía más dolorosa la realidad presente. 

Pero Dios no se puede retractar, y es entonces que la oración vuelve a describir la triste situación del orante, para inducir al Señor a tener piedad e intervenir, come siempre había hecho en el pasado. El salmista se define \textquote{gusano, no un hombre, vergüenza de la gente, desprecio del pueblo} (v. 7), se burlan, se mofan de él (cf. v. 8), y herido precisamente en la fe: \textquote{Acudió al Señor, que lo ponga a salvo; que lo libre si tanto lo quiere} (v. 9), dicen. Bajo los golpes socarrones de la ironía y del desprecio, parece que el perseguido casi pierde los propios rasgos humanos, como el siervo sufriente esbozado en el \emph{Libro de Isaías} (cf. \emph{Is} 52, 14; 53, 2b-3). Y como el justo oprimido del \emph{Libro de la Sabiduría} (cf. 2, 12-20), como Jesús en el Calvario (cf. \emph{Mt} 27, 39-43), el salmista ve puesta en tela de juicio la relación con su Señor, con relieve cruel y sarcástico de aquello que lo está haciendo sufrir: el silencio de Dios, su ausencia aparente. Sin embargo, Dios ha estado presente en la existencia del orante con una cercanía y una ternura incuestionables. El salmista recuerda al Señor: \textquote{Tú eres quien me sacó del vientre, me tenías confiado en los pechos de mi madre; desde el seno pasé a tus manos} (vv. 10-11a). El Señor es el Dios de la vida, que hace nacer y acoge al neonato, y lo cuida con afecto de padre. Y si antes se había hecho memoria de la fidelidad de Dios en la historia del pueblo, ahora el orante evoca de nuevo la propia historia personal de relación con el Señor, remontándose al momento particularmente significativo del comienzo de su vida. Y ahí, no obstante la desolación del presente, el salmista reconoce una cercanía y un amor divinos tan radicales que puede ahora exclamar, en una confesión llena de fe y generadora de esperanza: \textquote{desde el vientre materno tú eres mi Dios} (v. 11b). El lamento se convierte ahora en súplica afligida: \textquote{No te quedes lejos, que el peligro está cerca y nadie me socorre} (v. 12). La única cercanía que percibe el salmista y que le asusta es la de los enemigos. Por lo tanto, es necesario que Dios se haga cercano y lo socorra, porque los enemigos circundan al orante, lo acorralan, y son como toros poderosos, como leones que abren de par en par la boca para rugir y devorar (cf. vv. 13-14). La angustia altera la percepción del peligro, agrandándolo. Los adversarios se presentan invencibles, se han convertido en animales feroces y peligrosísimos, mientras que el salmista es como un pequeño gusano, impotente, sin defensa alguna. Pero estas imágenes usadas en el Salmo sirven también para decir que cuando el hombre se hace brutal y agrede al hermano, algo de animalesco toma la delantera en él, parece perder toda apariencia humana; la violencia siempre tiene en sí algo de bestial y sólo la intervención salvífica de Dios puede restituir al hombre su humanidad. Ahora, para el salmista, objeto de una agresión tan feroz, parece que ya no hay salvación, y la muerte empieza a posesionarse de él: \textquote{Estoy como agua derramada, tengo los huesos descoyuntados [\ldots] mi garganta está seca como una teja, la lengua se me pega al paladar [\ldots] se reparten mi ropa, echan a suerte mi túnica} (vv. 15.16.19). Con imágenes dramáticas, que volvemos a encontrar en los relatos de la pasión de Cristo, se describe el desmoronamiento del cuerpo del condenado, la aridez insoportable que atormenta al moribundo y que encuentra eco en la petición de Jesús \textquote{Tengo sed} (cf. \emph{Jn} 19, 28), para llegar al gesto definitivo de los verdugos que, como los soldados al pie de la cruz, se repartían las vestiduras de la víctima, considerada ya muerta (cf. \emph{Mt} 27, 35; \emph{Mc} 15, 24; \emph{Lc} 23, 34; \emph{Jn} 19, 23-24). 

He aquí entonces, imperiosa, de nuevo la petición de ayuda: \textquote{Pero tú, Señor, no te quedes lejos; fuerza mía, ven corriendo a ayudarme [\ldots] Sálvame} (vv. 20.22a). Este es un grito que abre los cielos, porque proclama una fe, una certeza que va más allá de toda duda, de toda oscuridad y de toda desolación. Y el lamento se transforma, deja lugar a la alabanza en la acogida de la salvación: \textquote{Tú me has dado respuesta. Contaré tu fama a mis hermanos, en medio de la asamblea te alabaré} (vv. 22c-23). De esta forma, el Salmo se abre a la acción de gracias, al gran himno final que implica a todo el pueblo, los fieles del Señor, la asamblea litúrgica, las generaciones futuras (cf. vv. 24-32). El Señor acudió en su ayuda, salvó al pobre y le mostró su rostro de misericordia. Muerte y vida se entrecruzaron en un misterio inseparable, y la vida ha triunfado, el Dios de la salvación se mostró Señor invencible, que todos los confines de la tierra celebrarán y ante el cual se postrarán todas las familias de los pueblos. Es la victoria de la fe, que puede transformar la muerte en don de la vida, el abismo del dolor en fuente de esperanza. 

Hermanos y hermanas queridísimos, este Salmo nos ha llevado al Gólgota, a los pies de la cruz de Jesús, para revivir su pasión y compartir la alegría fecunda de la resurrección. Dejémonos, por tanto, invadir por la luz del misterio pascual incluso en la aparente ausencia de Dios, también en el silencio de Dios, y, como los discípulos de Emaús, aprendamos a discernir la realidad verdadera más allá de las apariencias, reconociendo el camino de la exaltación precisamente en la humillación, y la manifestación plena de la vida en la muerte, en la cruz. De este modo, volviendo a poner toda nuestra confianza y nuestra esperanza en Dios Padre, en el momento de la angustia también nosotros le podremos rezar con fe, y nuestro grito de ayuda se transformará en canto de alabanza. (\ldots)
	
\end{body}

\img{cross_romanesquebell}

\newsection
\section{Homilías}

\subsection{San Pablo VI, papa}

\subsubsection{Homilía (1967): Viene como Mesías hoy para nosotros}

\src{19 de marzo de 1967.}

\begin{body}
\ltr[¡]{H}{ermanos} e hijos queridos! ¡Y vosotros, jóvenes amigos, que habéis aceptado nuestra invitación a participar en este rito extraordinariamente significativo! ¿Sabéis lo que estamos haciendo?

Queremos renovar la memoria y, en ciertos aspectos, la escena, de hecho, más que la escena, el hecho popular y modesto, pero sensacional y sumamente importante y decisivo, de la entrada mesiánica de Jesús en Jerusalén, la ciudad santa, abarrotada en esos días debido a la afluencia de personas de toda Palestina, debido a la celebración anual de la Pascua judía. Esta era la fiesta histórica de los judíos en la que recordaban el pasado: la liberación del pueblo elegido de la esclavitud egipcia; renovaban la conciencia de su destino teocrático, y confirmaban la esperanza profética de acontecimientos futuros y gloriosos, los inherentes a la promesa divina que ese pueblo custodiaba con la antigua fe de Abraham.

\homsec{El advenimiento y la victoria del Mesías}

Siempre surgía una tensión espiritual de esa celebración; pero ese año esta tensión pascual pareció alcanzar un grado muy alto de intensidad: la predicación de Jesús, que sucedió a la de Juan el Precursor, había puesto a los espíritus en expectativa; las polémicas cada vez más amargas entre Jesús y los judíos, y cada vez más encaminadas a dar una respuesta decisiva sobre la Persona de Jesús y sobre su misión, el asombroso milagro de la resurrección de Lázaro, realizado en aquellos días no lejos de Jerusalén, todo concurrió para producir una singular emoción, tanto en el grupo que se reunió en torno a Jesús, como entre la gente, que había escuchado hablar de su cercanía a la ciudad santa. Fue entonces cuando tuvo lugar el gran acontecimiento: Jesús, que hasta ese momento se había mostrado reacio a permitir manifestaciones solemnes de las personas que lo rodeaban, fue él mismo quien ese día (el domingo anterior a la tragedia del Calvario) lo quiso y lo predispuso; vosotros recordáis cómo tuvo lugar el humilde y glorioso viaje de Jesús desde Betania, desde Betfagé a Jerusalén. La aparición de Jesús en la cresta del monte de los Olivos, encima del burro, fue como una chispa que provocó un fuego de entusiasmo, de alegría, de vítores, de hosannas; e inmediatamente el triunfo popular improvisado adquirió un sentido sagrado, religioso, extraordinario; el significado del advenimiento del Mesías: ese era el Mesías, esperado durante siglos; ese era el Mesías, era el Cristo, el enviado y consagrado por Dios, aquel en quien se resumía toda la historia pasada del pueblo judío, a la espera de Cristo, Aquel en quien se cumplían las expectativas y las promesas, Aquel que finalmente inauguró el nuevo reino de David, en verdad el maravilloso reino de Dios. Jesús, en esa hora decisiva, fue reconocido, fue proclamado, el que había de venir, el Cristo.

Cristo: ¿entendemos el valor ilimitado de este título? Lo usamos con tanta frecuencia, y quizás no medimos la importancia que tiene, por su significado desbordante: Cristo significa Rey consagrado, lleno del Espíritu Santo, lugarteniente de Dios en el mundo; un significado universal y central para toda la humanidad, un significado que no se limita a los confines de la historia judía, sino que se desborda y se extiende al mundo, a todos los tiempos y a todos los hombres; viene a nosotros. Hoy estamos invitados a reconocer en Cristo el centro de nuestros destinos, nuestro Maestro, nuestro Salvador, el Dios hecho hombre, Aquel que es principio y fin de nuestra historia temporal y espiritual, Aquel que está Presente y que afortunadamente para nosotros y para nuestra alegría podemos reconocer, como se decía a sí mismo: el camino, la verdad, la vida.

\homsec{Cristo centro de nuestros destinos}

Más que profundizar, en este breve momento, el inmenso significado de la exaltadora palabra \textquote{Cristo}, queremos detenernos en el hecho de que hoy, como entonces, estamos invitados a reconocer a Cristo en Jesús de Nazaret; estamos invitados a una profesión de fe, que irradia en dos direcciones: hacia él, Jesús, a quien rendimos, siguiendo el ejemplo de Pedro, el homenaje exultante de nuestro descubrimiento, de nuestra adhesión, de nuestra alegría: \textquote{¡Tú eres el Cristo, Hijo del Dios viviente!}; y hacia nosotros, de nuestra vida, que legítimamente puede y debe llamarse sinceramente cristiana. Lo que hacemos es una gran elección: incluso hoy queremos decirnos a nosotros mismos, contarle a la sociedad que nos rodea, decirle al mundo cercano y lejano, que creemos en Jesucristo y que queremos seguirlo, y que siguiéndolo no caminamos a ciegas, en tinieblas, sino a la luz de su palabra, de sus ejemplos, de su gracia (cf. \textit{Jn} 8, 12).

Este debe ser el sentimiento y el propósito de este día para nosotros: el anuncio mesiánico de Jesús se renueva para nosotros.

Quisiera señalar tres circunstancias en relación con esto; y quiero hacerlo especialmente para vosotros, jóvenes que Nos escuchan.

\homsec{Jesús el verdadero gozo de nuestra vida}

La primera circunstancia es la de la alegría que, entonces y ahora, acompaña al anuncio de Jesús como Cristo, revelador y realizador de nuestra fortuna humana y sobrehumana. Recordad esto, jóvenes: Cristo es la alegría del mundo; es nuestra alegría. ¿Verás a Cristo en la Cruz en unos días, verás la vida cristiana marcada por la austeridad y la penitencia? Verás el dolor humano, el tuyo y el de los demás, entrar en la esencia de la fidelidad y de la humanidad cristiana. No seremos nosotros los que ocultemos esta dramática realidad de nuestra fe y nuestro seguimiento de Jesús, pero recordemos igualmente que Jesús es el gozo, el verdadero gozo de nuestra vida. No explicamos las razones ahora, pero anunciamos la realidad. Recuerda que la vida cristiana no es triste, no es infeliz. Es una vida feliz y serena. Viviendo cristianamente sabremos verdaderamente disfrutar de los bienes honestos y las buenas horas de esta vida, y sabremos, en todas las condiciones de la existencia humana, encontrar razones y formas de felicidad secreta e inagotable. Si eres fiel en seguir a Jesús, tendrás la garantía de esta certeza. Te lo deseamos, sí, con alegría pascual.

\homsec{El Rey de la paz}

Segunda circunstancia. Jesús fue proclamado Mesías, pero no como lo esperaba el imaginario político y el \textquote{triunfalismo} de gran parte del pueblo de la época; Rey, sí, pero sin armas, sin riqueza, sin poder económico y temporal; Rey, pero cuyo Reino no es de este mundo, ni en competencia ni en antagonismo con los poderes civiles, Rey de corazones humanos, Rey en el orden de la Redención, Rey manso, Rey de paz. Este aspecto del reino establecido por Jesucristo también requeriría un sinfín de explicaciones y comentarios. Pero todo lo dice el símbolo, que estás sosteniendo, la palma, el olivo. Contentémonos ahora con este lenguaje simbólico: Jesús es nuestra paz (\textit{Ef} 2, 14). Si la paz es el orden, establecido en la justicia y la sabiduría, si la paz es el resultado comunitario, no de la opresión, la venganza, el terror, la violencia, sino de los sentimientos colectivos que obran para un bien común; si la paz es fruto de la libertad, el perdón, la hermandad, el amor; si la paz es el esfuerzo generoso y continuo por generar un bien razonable, fuerte, accesible a todos; si la paz entre los hombres es el reflejo de la paz de conciencia con Dios, recordad también esto, jóvenes: sólo de Cristo, de sus enseñanzas y de ese misterioso fluir de verdadera energía espiritual, que emana de él y que llamamos gracia, podemos tener paz; una paz, que es verdadera y en continua fase de componerse y recomponerse, y capaz de nutrir, apoyar y sublimar los esfuerzos que los hombres están haciendo para darse la paz, la propia paz, a menudo efímera y frágil, cuando no hipócrita y opresiva. Una verdadera paz, digamos, que educa a los hombres a respetarse unos a otros; a colaborar fraternalmente, a no basar sus esperanzas en la hegemonía y la carrera armamentista; una paz que cree en el amor y que hace brotar del corazón cerrado y rebelde de los hombres fuentes insospechadas de bondad. No lo olvides, Cristo es nuestra paz y Él puede realizar este milagro. Sacude tu palma y tus ramas de olivo, y cuéntaselo al mundo.

\homsec{A los jóvenes: ¡Proclamad la presencia y la misión de Cristo en nuestros días!}

¡Díselo al mundo! ¿Y quién mejor que vosotros, jóvenes, puede decirlo? Esta es la tercera circunstancia a la que, para terminar, nos referimos. Se dice en la liturgia y se aclara en el relato del Evangelio (\textit{Mt} 21, 15) que entre la multitud que aclamaba al Mesías reconocido, los más fervientes eran los jóvenes, los muchachos. Este es un detalle muy hermoso y natural; nadie iguala a los jóvenes, a los muchachos en entusiasmo y vivacidad; nadie los detiene y silencia cuando están juntos y se dejan llevar por una fantasía que los posee y exalta. Pero en este caso el episodio de la juventud alabando a Cristo asciende a un sentido particular, que revela una capacidad, una vocación propia de los adolescentes, la de convertirse en promotores valientes y clamorosos de un ideal, que se destella como grande y vivo ante sus espíritus; la historia contemporánea nos ofrece ejemplos impresionantes y no siempre edificantes. Pero, ¿y si este ideal fuera Cristo? ¿Cristo con su palabra de verdad, amor y paz? ¿No podría repetirse la escena evangélica del triunfo mesiánico de Cristo con la obra de un joven inteligente y atrevido, que comprendió Quién es Él?

¡Jóvenes amigos! ¡Sí, esa escena puede repetirse; puede convertirse en historia de nuestro tiempo! ¡Depende de los jóvenes, de vosotros, proclamar la presencia y misión de Cristo en nuestros días! Depende de vosotros, de vuestra fascinación instintiva por la libertad y el coraje, liberar este período histórico incierto y cansado del escepticismo de las generaciones pasadas, y asumir la posición de hijos de luz y testigos de la verdad cristiana; depende de ti atreverte a reconstruir el mundo moderno sobre la base de la fe; a ti te toca demostrar, si no sabes hacerlo con discursos difíciles, con el argumento maravilloso y más elocuente de tu vida consciente y recta; que las expresiones seductoras y equívocas de la decadencia intelectual y moral en muchos ambientes modernos pueden ser contrarrestadas y reemplazadas por un estilo juvenil, lleno de fuerza, belleza, alegría y, si es necesario, heroísmo y sacrificio; un estilo cristiano. Y finalmente os corresponde a vosotros, queridos jóvenes, anunciar la paz de Cristo en el mundo: sin los jóvenes y sin Cristo no se puede establecer una paz eficaz en la sociedad civil y en las relaciones internacionales. Ningún ejército feroz ni ninguna diplomacia hábil pueden fundar una paz sincera y duradera sin la contribución de la juventud y sin principios cristianos. Lo que significa que podéis ser los heraldos de la paz más convencidos y dinámicos. Por eso os hemos invitado a esta celebración; y para que seáis dignos y estéis orgullosos de ser portadores del olivo de Cristo, os bendecimos a todos de corazón.
\end{body}


\subsubsection{Homilía (1970): ¿Quién es Cristo para ti?}

\src{22 de marzo de 1970.}

\begin{body}
\ltr{D}{os} páginas del Evangelio se abren ante nosotros en la liturgia de hoy, Domingo de Ramos, la de la sensacional entrada de Jesús en Jerusalén y la de su Pasión. Se trata del evangelista Marcos, probablemente testigo presencial de los hechos narrados y confiados a él, en la estela de la primera catequesis de la naciente comunidad cristiana, por el apóstol Pedro, quizás aquí en Roma. De las dos páginas, escojamos ahora la primera, que es característica de este domingo, el llamado Evangelio de las Palmas, para Nuestra breve meditación.

Acabamos de escuchar su lectura. Repensad la escena descrita. Es singular en la historia evangélica, porque es un escenario público, festivo, intencional. Hemos visto otras veces, leyendo el Evangelio, a Jesús rodeado de multitudes atraídas por su palabra, por sus milagros, por su figura; pero siempre hemos notado que Jesús era ajeno a provocar aclamaciones por sí mismo; por el contrario, era tímido para despertar la popularidad en torno a su persona. Esta vez no: Jesús desea ser reconocido y aclamado, tanto que cuando \textquote{unos fariseos en medio del pueblo –hipócritamente solícitos del orden público, pero en realidad molestos de que todo el pueblo lo siguiera– (\textit{Jn} 12, 19), le dijeron: \textquote{Maestro, reprende a tus discípulos}, Él les respondió: \textquote{Os digo que si éstos callan, las piedras clamarán} (\textit{Lc} 19, 39-40)}. ¿Por qué esta nueva actitud en el Señor? Jesús quiere entrar en Jerusalén, en aquellos días rebosantes quizás también de gente que venía para la inminente celebración de la Pascua judía, en una nueva forma, digamos, oficial. Sabe lo que le espera, se lo confió a sus discípulos: \textquote{He aquí estamos subiendo a Jerusalén, y el Hijo del Hombre –es decir, él mismo, Jesús– será entregado en manos de los príncipes de los sacerdotes y escribas, y lo condenarán a muerte, y lo entregarán a los gentiles para que se burlen de él, lo azoten y lo crucifiquen} (\textit{Mt} 20, 18-19). Así comienza su pasión, y quiere resaltar no sólo su aspecto libre y voluntario (cf. \textit{Is} 53, 7; \textit{Hb} 9, 14; \textit{Ef} 5, 2), sino también el aspecto mesiánico; antes de consumir su sacrificio, porque tal es su muerte, su inmolación, quiere revelar finalmente y abiertamente quién es y cuál es su misión; Él es el Mesías y, como tal, quiere ser reconocido libre y clamorosamente por su pueblo.

\homsec{Aspecto mesiánico de la Pasión}

Aquí deberíamos tener una idea del significado fecundo de esta palabra \textquote{Mesías}, que significa Cristo, el hombre elegido y consagrado, en quien se concentraron las expectativas seculares y proféticas de Israel, todas las esperanzas de la nación privilegiada y predestinada a ser, a través del Mesías, piedra angular de los destinos del mundo. El Mesías fue considerado como el Hijo de David, el Rey de la historia guiado por los designios de Dios, el Salvador prodigioso, que traería el remedio a todas las dolencias de la humanidad (cf. \textit{Mt} 11, 3ss). Jesús dará un significado más profundo, dramático y sobrenatural a este maravilloso título, y lo reclamará para sí mismo, se lo atribuirá a sí mismo, querrá que se lo reconozca claramente. Y recordamos hoy el fatídico momento en el que se celebra a Jesús como el Mesías, como Cristo. Es su momento. El epílogo de su vida temporal deberá consumarse con esta calificación de Mesías. El episodio de la entrada de Jesús en Jerusalén asume una importancia decisiva con respecto a las preguntas que se habían acumulado en torno a la misteriosa personalidad de Jesús: ¿Quién era Jesús? ¿\textquote{El hijo del carpintero}? (\textit{Mt} 13, 55) ¿Una figura singular: \textquote{El Hijo del Hombre}, como se llamó él mismo? ¿un profeta? (\textit{Mt} 16, 14; 21, 11; etc.) ¿era realmente el Mesías? (\textit{Jn} 1, 41) ¿qué está por venir? (\textit{Mt} 11, 3, 5) ¿Podría ser el Hijo de David (\textit{Mt} 20, 30-31) para ser proclamado Rey? (\textit{Jn} 6, 15) ¿o alguien aún más grande y misterioso, el Hijo de Dios? (\textit{Mt} 16, 16; Jn 1, 49; 8, passim) la duda crece a medida que Jesús se permite revelar el misterio de su filiación divina, hasta la cuestión apremiante, en el proceso del Sanedrín, durante la última noche: \textquote{¿Eres tú el Cristo, el Hijo del Bendito, el Hijo de Dios?} (\textit{Mc} 14, 61). La identificación de la verdadera personalidad de Jesús es la pregunta que atraviesa todo el Evangelio, y que al final lo vuelve dramático y trágico.

Jesús había dado muchas definiciones de sí mismo, que forman el objeto y el deleite de nuestra fe; será bueno recordarlas; como: \textquote{Yo soy el pan de vida} (\textit{Jn} 6, 48); \textquote{Yo soy el buen Pastor} (\textit{Jn} 10, 11); \textquote{Yo soy la luz del mundo} (\textit{Jn} 12, 46); etc; ¿recordáis, en la Última Cena: \textquote{Yo soy el camino, la verdad, la vida}? (\textit{Jn} 14, 6). Pero en la escena sobre la que estamos meditando, Jesús se define no con palabras sino con un acto: el Mesías. No es un acto triunfalista, sino una humilde, aunque pública y estudiada presentación del Yo, cuya grandeza no queremos considerar en su aspecto modesto y popular, sino en la explosión festiva de la multitud, en la certeza ahora adquirida del pueblo, en la profesión que los jóvenes hacen especialmente de su fe y de su alegría por el reconocimiento irreversible del carácter mesiánico de Jesús: es Él, es el esperado durante siglos, es el esperado por esta generación, es la llave de toda la historia pasada y futura. La curiosidad, la duda, la vacilación, la fascinación, la admiración que hasta entonces habían rodeado a Jesús, estallan ahora en la certeza de los vítores entusiastas: es Él, es Él, el Hijo de David, el Cristo, el Señor.

Ahora prestad atención. En la liturgia que estamos celebrando se repite este encuentro. La Iglesia relata esa escena, ese momento decisivo ante nuestras almas. Jesús se presenta ante nosotros, humilde y formidable, revelándose. Habla, casi por sí mismo, y, cosa impresionante en tanta fiesta que lo rodea, llora. Llora mirando la ciudad cercana y diciendo, casi en diálogo con ella, previendo, a pesar de esa hora de gloria: \textquote{¡Oh! si tú también supieras, y en este mismo día, ¡conocieras lo que te conduce a la paz! En cambio, ahora tus ojos han permanecido cerrados\ldots}; y, predicando la futura ruina de la ciudad santa, pero infiel, agrega: \textquote{Porque no reconociste el momento en que fuiste visitada} (\textit{Lc} 19, 42-44).

\homsec{Jesús, nuestra elección}

El significado de este Evangelio de las palmas, que ahora hemos releído, es una cuestión inevitable. Propone una elección, que concierne al destino de nuestra vida. Sí o no: ¿reconocemos a Jesús por quien es, el Cristo? Es decir, ¿el Mesías, el enviado de Dios, que descendió al mundo para dar salvación a la humanidad? ¿Venido también entre nosotros como \textquote{signo de la contradicción} (\textit{Lc} 2, 34), la aguja de cambio entre los dos caminos definitivos, de salud o de perdición, de vida o de muerte? Tenemos la feliz intuición, la frescura, la alegría, la audacia de proclamar aún hoy que Jesús es Él, nuestra elección, Él es nuestro Redentor, necesario, suficiente; Él, que vino por todos, vino por cada uno de nosotros; ¿Él, el Maestro, Él el Amigo, Él \textquote{la resurrección y la vida}? (\textit{Jn} 11, 25) ¿Él, sí, el camino, Él, la verdad, Él, la vida de nuestras existencias individuales y de toda la comunidad de quienes creen en Él, confían en Él, se sienten amados por Él y le ofrecen a Él su pobre y gran amor?

Jesús, el Cristo, atraviesa todavía, siempre y en todas partes, los caminos de la humanidad, y se plantea como la gran cuestión, como la elección suprema y decisiva, que todo hombre, que todo pueblo está llamado a hacer. Jesús es la gran responsabilidad en la historia de toda existencia humana, Jesús está en el grado supremo de tensión de la libertad de la vida consciente. Jesús está en el último y primer nudo, donde se definen nuestros destinos. Jesús es la invitación más íntima y personal dirigida a nuestra conciencia lúcida y operante.

\homsec{Llamamiento a los jóvenes}

Este discurso elemental y esencial, que resume el \textquote{kerigma}, el anuncio, la proclamación del Evangelio, es para todos; pero está dirigido especialmente a vosotros, jóvenes\ldots Nos atrevemos a hablaros directamente, porque vosotros, como en el Evangelio de las palmas, sois los protagonistas del encuentro siempre dramático de Jesús, el Cristo de los siglos, con la humanidad. Muchos hoy hablan \textit{de} los jóvenes; pero no muchos, nos parece, hablan \textit{con} los jóvenes. Quizás no sepan, quizás no confíen. Nosotros te hablamos, porque un deber ineludible nos obliga a hacerlo. Y lo hacemos como alguien que te ama; como tus Padres, como tus Maestros; y nos atrevemos a decir, con una palabra aún mayor, más profunda que la de ellos, porque nuestra palabra, en verdad, no es nuestra, sino de Cristo mismo, de quien no somos más que el eco humilde y fiel.

Nos gustaría hacernos entender. ¿Queréis escucharnos? Si es así, escúchate a ti mismo primero. ¿Qué voces surgen de tu espíritu? Intenta permitirte unos momentos de silencio interior: ¿qué sientes? Creemos que escuchas muchas voces confusas, a veces hasta el punto del clamor. ¿Qué rumores? Son las voces del mundo que te rodea, y que escuchas resonando dentro de ti: voces de la conversación doméstica, voces de tu colegio, voces de tus compañeros, voces que comienzan a agobiar a los demás; son las voces de nuestro tiempo, de nuestro mundo; palabras grandes y difíciles, música agradable y frívola, gritos humanos, que comienzan a ser impresionantes, y que generan dentro de ti otras voces, estas son las tuyas: son las voces de los primeros juicios, voces de las primeras vivencias, incluso voces inquietantes y atractivas: curiosidad , fantasías, tentaciones que llaman; comienzan a despertar en ti las voces que se volverán imperiosas, las voces de los deseos, las voces que quieren dar a la vida – ¡fíjate! – su sentido, su valor, su destino. Son las voces personales.

¿Las has escuchado alguna vez? ¿Qué te dicen? Te dicen algo ideal, muy bonito y muy difícil; tan difícil que a veces te vuelves impaciente, a veces iluso, a veces triste. Son las voces que suenan: libertad, verdad, amor. Es decir: grandeza, heroísmo, felicidad. Son las voces de la vida. ¿Son sinceras o mentirosas? ¿Podemos llenarlas de realidad, o permanecen vacías y nos quitan la fe en la vida? ¿Nos hacen buenos o malos? ¿Nos dan la alegría de la acción y la esperanza de algo que no muere, o nos vuelven rebeldes y con ganas de protestar y destruir? ¿Nos alejan de nosotros mismos y de nuestra sociedad, o nos hacen anticipar, e incluso saborear en cierta medida, la autenticidad de nuestra conquista y la de las correctas relaciones con los demás?

No queremos continuar esta introspección, este psicoanálisis moral y social ahora. Te decimos simplemente, pero con la fe y el amor de los que somos capaces, que hay una respuesta suprema a todas estas maravillosas y tormentosas preguntas. Hay Uno, que es Él mismo la respuesta. Una palabra, que es una persona. Una Persona, que se llama luz: \textquote{Yo soy la luz del mundo}, dice (\textit{Jn} 8, 12). Una Persona que se hace pasar por guía: \textquote{El que me sigue no anda en tinieblas} (\textit{Jn} 8, 12). Una Persona, piensa, que dice de sí mismo: \textquote{Yo soy el Pan de vida} (\textit{Jn} 6, 48). Podríamos continuar; pero habéis entendido: ese Verbo, esa Persona es Jesús, es el Cristo, \textquote{el cual fue hecho para nosotros sabiduría, justicia, santidad, redención} (\textit{1 Co} 1, 30). Él es quien da a nuestra existencia su amor verdadero, su dignidad intangible, su libertad responsable, su valor auténtico, su amor pleno. Él es nuestro Salvador; Él es la cabeza de nuestro inmenso cuerpo en formación, que es la humanidad creyente y redimida, la Iglesia; Él es el que nos perdona y nos hace capaces de cosas más grandes que nosotros, es el defensor de los pobres, es el consolador del sufrimiento, es, en una palabra, nuestro Mesías, es Cristo, Cristo Jesús.

¿Lo conoces? ¿Lo reconoces? ¿También lo aclamas hoy con la respuesta de alabanza de tu fe y de tu ideal? \textquote{Sabiendo esto, bienaventurados seréis si lo cumplís} (\textit{Jn} 13, 17).
\end{body}

\newpage 
\subsubsection{Homilía (1973): Vivir para amar}

\src{8 de abril de 1973.}

\begin{body}
[\ldots]

\homsec{Predicar a Cristo entre el pueblo}

\ltr{E}{l} Evangelio de este día presenta un tema inmenso y maravilloso\ldots Jesús entra en Jerusalén. Ha estado allí muchas veces, pero esta vez se adentra en esta ciudad de una manera inusual, es decir, montando un burro. Y este iba a ser su triunfo, su reconocimiento oficial frente al pueblo judío.

Eran días especiales. ¡Toda Jerusalén estaba llena de gente, porque las fiestas de la Pascua habían convocado gente desde todas las regiones de Palestina. Multitudes de fieles que acampaban aquí y allá. Y había una gran vivacidad, porque todos tenían la impresión de que iba a pasar algo extraordinario, es decir, la revelación de lo que durante siglos habían esperado. Tenía que venir el Mesías, el enviado de Dios, Jesús se presenta como el Mesías y el pueblo, casi tocado por una chispa que enciende el fuego, se entusiasma. \textquote{¡Es él, es él, el hijo de David está aquí!} gritaron. Los muchachos fueron a arrancar ramas de olivo y de palma de los árboles, gritando: \textquote{¡Viva, viva, hosanna al hijo de David!}

Esta es una de las páginas del Evangelio más ricas en detalles, que son casi fotográficos. Hay, por ejemplo, griegos, extranjeros que llegaron a Jerusalén, ciudad que acogió a tanta gente de paso que venía por motivos comerciales o para transitar hacia países más internos de Asia. Estos griegos miran hacia fuera y, como todos los curiosos, repiten: \textquote{Nos gustaría ver a Jesús}. Es una frase que aparece dos o tres veces en el Evangelio para indicar la curiosidad de verlo con los ojos, de poder conocerlo, de leer su fisonomía. Pero siempre hay mucha gente alrededor de Jesús, los griegos son incapaces de acercarse. Y luego uno de ellos se acerca a Felipe, uno de los discípulos. El nombre de Felipe, nombre griego, nos lleva a creer que en él habían encontrado uno que hablaba su idioma. Y Felipe, que era uno de los apóstoles, pero no el primero, se vuelve hacia Andrés, hermano de Pedro, que era el cabeza reconocido de la pequeña comunidad por el mismo Cristo, y le dice: \textquote{Quieren ver a Jesús}. Ambos se acercan a Jesús y le dicen: \textquote{Hay griegos a los que les gustaría verte}. No sabemos cómo terminó, porque en este punto Jesús comienza su discurso, el discurso revelador de su psicología, de lo que sentía. De hecho, es una de las páginas del Evangelio para leer con particular inteligencia, ya que nos introduce en la psicología interior de Cristo, nos la abre. Jesús no habla a los que están cerca de él, sino a sí mismo, a la historia, al mundo. Los muros de Jerusalén se alzaban gigantescos y fuertes ante ellos. Otro evangelista, Lucas, nos dice que Jesús, en ese momento, comenzó a llorar. También en otras partes del Evangelio leemos que Jesús lloró.

\homsec{¿Cuál es la gloria de Cristo?}

Por ejemplo, cuando le anunciaron la muerte de Lázaro. Esta vez también llora. Llora por el destino de esta ciudad que ya ve destruida. Estos poderosos muros los ve derrumbarse y caer. Jesús tiene dos imágenes ante sus ojos: la futura caída de Jerusalén y su propio destino: \textquote{Para eso he venido, para esta hora}\ldots \textquote{Y estalla en dolor; siente que este triunfo suyo, que lo declara pública y oficialmente Mesías, significa su muerte. Y se entrega a esta pasión, que en menos de siete días lo llevará a la Cruz. Siente que ha llegado su hora: \textquote{Padre, glorifica tu nombre}}.

Entonces se produce otro evento, uno de los tres hechos milagrosos e inexplicables que encontramos registrados en el Evangelio, cuando una voz del cielo responde. Encontramos esta voz en la Transfiguración, la encontramos en el Bautismo de Jesús y la encontramos ahora. Dice: \textquote{Yo le glorificaré}. Y Jesús, entonces, piensa en su gloria. ¿Pero qué gloria? La Cruz, que es la vergüenza, el deshonor, la agonía, el dolor y la muerte que debe sufrir porque entró en el designio de Dios y se declaró enviado de Dios. La gloria de Cristo es su sacrificio, es su crucifixión, es su muerte. Y aquí la palabra se extiende desde su destino al nuestro, al de los que quieren ser seguidores de Cristo, como dice con acento poético el pasaje del Evangelio de hoy. \textquote{Si el grano de trigo no cae en tierra y muere, queda estéril; si en cambio se disuelve en la tierra y parece morir, entonces se vuelve fecundo, da fruto}. Este es el diseño del cristianismo, este es el diseño de quienes lo seguirán. Es nuestra gran ley de morir para vivir, de morir por amor para vivir en gloria. Es la piedra angular del Evangelio y de la vida cristiana.

[Esta palabra nos] recuerda a todos la necesidad de sacrificarse para ser verdaderos cristianos. Hay dos actitudes características de los hombres hacia la vida de este mundo. Hay quienes conciben la vida como un goce. Debemos –dicen– ser felices, tener todo lo necesario, alcanzar la plenitud de los bienes de este mundo. Muchos conciben la vida de forma hedonista, es decir, hecha de placeres, hecha para la felicidad y los bienes de la tierra. No es que estos bienes de la tierra nos estén prohibidos, especialmente cuando son necesarios para la vida. Vemos que el pan, la dignidad, todos los derechos humanos están efectivamente protegidos por el Evangelio, e incluso convertidos en objeto de oración, de conversación entre nosotros y Dios: \textquote{Danos hoy nuestro pan de cada día}. Pero aquellos que sólo piensan en garantizarse estos bienes, traicionan el plan de Dios que, en cambio, quiere establecerse sobre la base del amor.

\newpage 
\homsec{El sacrificio, fuente de vida}

Amor es una palabra ambigua. Hay amor por uno mismo, que se llama egoísmo. Hay amor por los demás, que se llama sacrificio, y eso es lo que el Señor nos muestra con su ejemplo como fuente de vida. El Hijo de Dios que vino al mundo da su vida de una manera tan generosa, tan compasiva, dramática, trágica. Él muere por nosotros en medio de su ignominiosa tortura en la Cruz. Muere para salvarnos. El sacrificio del Señor nos dice que es necesario concebir nuestra vida como un deber. Cada uno de nosotros es traído al mundo para hacer algo, no solo por nosotros mismos, sino por los demás, por amor, por un amor gratuito, desinteresado y generoso, que incluso cuesta la propia existencia. Estamos llamados imitar a Cristo que muere por nosotros. También nosotros estamos llamados a ser como el grano de trigo que se entrega para encontrar en sí mismo las virtudes superiores, la fecundidad, la riqueza que el Señor ha destinado a toda existencia humana.

Es una palabra difícil, pero la madre de familia que da la vida por sus hijos y por su casa, o el trabajador que trabaja y suda para ganarse el pan de su casa, o el servidor público que trabaja, piensa y dispone por el bien de los demás, pueden entender bien esta palabra. Cada uno de nosotros está llamado a dar su vida por los demás y no a encerrarse en sí mismo, contentándose con su salvación y su felicidad. Debemos proporcionar la felicidad y el bienestar de los demás incluso a costa del don de nosotros mismos. El Señor nos enseña la gran ley del amor verdadero, la ley de morir para vivir.

Estamos llamados a vivir para amar. El que ejerce su profesión no solo por su propio bien, sino por el bien de los demás, por el bien de la sociedad en la que vivimos en este momento histórico tan convulso, tan ansioso por disfrutar, acoge y vive la palabra de Cristo; para hacer hombres buenos, educados, libres, contemporáneos y hermanos. El Papa trae este anuncio dramático porque es portador de la Palabra del Evangelio. Y el Evangelio nos dice que debemos ser imitadores de Cristo. Jesús anuncia que en unos días estará con los brazos extendidos, mutilado, con las manos atravesadas por clavos, todo vestido con su propia sangre y su propia angustia: \textquote{Cuando sea elevado –y quiso decir en lo alto de la Cruz– atraeré a todos hacia mí}: las multitudes, los fieles, los que lo siguen, que lo imitan, que recogen la virtud misteriosa de la Cruz que nos hace buenos, valientes y capaces de amar.

Este es el deseo que os traigo en esta Misa previa a la Pascua. Procura amar a Cristo crucificado y haz de él el libro de tu existencia, el código de tu imitación, el signo de tu felicidad y de tu esperanza inmortal.
\end{body}

\newpage 
\subsubsection{Homilía (1976): ¡Nosotros seremos cristianos!}

\src{11 de abril de 1976.}

\begin{body}
\ltr[¿]{Q}{ué} os recuerda la rama de olivo o la palma que lleváis en la mano? Todos lo sabemos: recuerda un hecho singular del Evangelio, el de la entrada de Jesús en Jerusalén, cinco días antes de que fuera condenado a muerte y crucificado. Una entrada inusual, porque se distingue por un signo, bastante modesto, pero intencionalmente festivo, solemne por la gran multitud presente y alegre, que la rodeaba. Estamos en Betania, a pocos kilómetros de Jerusalén, una aldea en la ladera oriental del Monte de los Olivos, donde estaba el hospitalario hogar de las hermanas Marta y María, y de su hermano Lázaro, recientemente resucitado por Jesús, y donde la curiosidad crecía, en medio del asombro y la emoción: había amigos, discípulos admirados por lo que había ocurrido con Lázaro\ldots Jesús estaba adquiriendo mucha popularidad y decidieron matar tanto a Jesús como a Lázaro, para poner fin al éxito creciente del Maestro (\textit{Jn} 12, 10). 

En este ambiente, cargado de entusiasmo explosivo por un lado y de odio radical y secreto por el otro, se formó una procesión que partía de Betania, y para gran alegría de los seguidores de Jesús los discípulos acogieron su inusual orden, la de procurarle un pollino para continuar con alegría hacia Jerusalén. De hecho, en Betfagé, por orden de Jesús, se tomó prestado un burro, nunca antes montado por nadie, y se hizo sentar al Maestro en él; e inmediatamente la escena se convirtió en una manifestación popular, solemne en su pobre sencillez por dos circunstancias: la muchedumbre acampada alrededor de Jerusalén para la Pascua hebrea, y, viniendo de la ciudad rebosante de gente y forasteros, se apresuró hacia la llegada de la comitiva; y, segunda circunstancia, los vítores espontáneos y gozosos de todas aquellas personas que aplaudieron con gritos muy significativos, y al mismo tiempo molestos para los enemigos de Jesús: \textquote{¡Hosanna! Bendito el que viene en nombre del Señor}.

¿Qué significaba esta bienvenida, tan alegre y tan sensacional?

Es importante tenerlo en cuenta. El momento se vuelve dramático y adquiere su significado, que es decisivo para la historia y para la comprensión del Evangelio; el significado consiste en el reconocimiento y proclamación del carácter mesiánico de Jesús, el que ha de venir. Él está aquí, después de esperar durante siglos, por generaciones; ¡Es el hijo de David! ¡Él es el Cristo! Jesús es el Cristo, el enviado de Dios, el Salvador, el Mesías, es el centro de la historia, es el Rey de los judíos (¿recuerdan la inscripción de la sentencia de muerte, escrita por Pilato y pegada a la Cruz de Jesús: \textquote{\textit{Jesús Nazareno, Rey de los judíos}}?). \textquote{Este es el punto en el que se encontraron\ldots el mesianismo de la plebe y el de Jesús} (G. Ricciotti, \textit{Vita di Gesù Cristo}, 505). Este no fue solo un momento excepcional; era un destino, que resumía la vida privilegiada y turbulenta del Pueblo Elegido, que concentraba en sí el cumplimiento de las profecías y que abría los horizontes del futuro, que celebraba un acontecimiento de salvación inagotable, la Redención, y que comprometía a toda la humanidad a una elección suprema, a esa nueva alianza entre el mundo y Dios, la del cristianismo sí, o no. Esto se comprendió luego del cumplimiento de los hechos a los que ese acontecimiento dio origen, qué suerte se jugó en torno a ese nombre, Jesús; en torno a ese Maestro, Jesús; alrededor de ese Mesías, Jesús; alrededor de ese Cordero de Dios, esa víctima para la salvación de los hombres, Jesús. En esa misma ocasión, en su lenguaje revelador y misterioso, Él predijo: \textquote{Cuando Yo sea levantado sobre la tierra –es decir, en la cruz–, atraeré a todos hacia mí} (\textit{Jn} 12, 32). El espectáculo entonces, a los ojos del espíritu, se vuelve tan grande como el mundo. El drama se desborda hasta extenderse por toda la humanidad. Y este pasaje evangélico, si lo pensáis bien, se vuelve tremendamente interesante, tanto que no deja indiferente a nadie; nos concierne personalmente; cada uno de nosotros es partícipe de él.

Hermanos, especialmente los jóvenes, reflexionad bien sobre lo que os digo: esta celebración, que se refiere al anuncio de Jesús el Mesías, de Jesús el Cristo, de Jesús nuestro Salvador, también concierne a nuestro destino, nuestra primera elección. Pensad en el episodio decisivo, que estamos celebrando: Jesús reconocido por el Pueblo, y al mismo tiempo, Jesús hostigado y luego asesinado por los propios líderes del Pueblo, que no quisieron acogerlo ni creer en Él, ni siquiera después de la resurrección de Lázaro, ni siquiera después de su entrada triunfal y humilde como Mesías en Jerusalén. ¿Recordáis las palabras proféticas pronunciadas por el piadoso y anciano Simeón, cuando el niño Jesús fue presentado en el templo: será un \textquote{signo de contradicción}? (\textit{Lc} 2, 34). Sí, signo de contradicción: habrá una lucha a su alrededor; los hombres estarán divididos y opuestos entre sí. Esta lucha continuará durante siglos. ¡Oh! Éste es uno de los misterios más difíciles y dolorosos de la historia humana: la unidad en torno a Cristo, centro, polo, salvador de la humanidad, no será ni espontánea ni fácil; será blanco de una oposición feroz y severa por un lado; sin embargo, será un punto de fidelísima convergencia con el otro.

Ahora observad: ¿quién en ese día decisivo tuvo la intuición de que Jesús de Nazaret, el Maestro, peregrino y predicador extremadamente sabio, milagroso y misericordioso en Palestina, era Él el Mesías, era Él el hijo de David, era Él el Salvador esperado y prometido? El Pueblo, y entre el Pueblo los más entusiastas y activos eran los Jóvenes. Eran los heraldos del Mesías. Ellos tuvieron la intuición.

Se expusieron, con signos de audacia, de alegría y de gozo. Comprendieron que esa era la hora de Dios, la ansiada y bendita hora de la llegada del Mesías; y fue entonces cuando agitando ramas de árboles, ramas de olivo y palmeras, reconocieron a Jesús como el Maestro, el Mesías, el Cristo, el Príncipe de paz (cf. \textit{Is} 9, 6), era su primer triunfo, popular e incontenible (cf. \textit{Lc} 19, 39-40). Jesús fue visto llorando en ese momento, y ese llanto presagiaba su pasión y su cruz, así como la ruina futura de la ciudad que se había resistido a su suprema llamada mesiánica. Pero una voz atronadora del cielo anunció un epílogo de gloria (\textit{Jn} 12, 28), y los gritos de los niños que vitoreaban prevalecieron sobre el estruendo de la multitud y la ira de los jerarcas, y acompañaron a Jesús al templo, alabando sin cesar al nuevo hijo de David (\textit{Mt} 21, 15).

Ahora observad con atención: la escena se repite, la misma escena se perpetúa y se renueva en la liturgia de la Iglesia. A lo largo de los siglos, cada año, cuando llega la Pascua, esta ceremonia que estamos celebrando proclama a Jesús como Cristo, como Mesías, como árbitro de los destinos de la humanidad, verdadero Salvador del mundo. 

¿Cuáles son las voces más cualificadas para el anuncio de este bendito mensaje al mundo? son las voces de los miembros del Pueblo de Dios, son las vuestras, jóvenes reunidos en este rito maravilloso y misterioso. Hoy os toca a vosotros, hijos de esta generación histórica, ser eco de las aclamaciones a Jesús, reconocido como Cristo, Salvador y Señor. Gracias a una maduración afortunada y secreta de los tiempos, hoy son los Jóvenes, grupos privilegiados de Jóvenes, quienes intuyen y comprenden que ese Jesús del Evangelio es quien inaugura y abre legítimamente el Reino de la salvación. Es Él, el Cristo, quien, colocándose en el camino torrencial de la civilización, la divide en dos corrientes diferentes y muchas veces opuestas: por un lado, la suya, la de Jesucristo, corriente de paz y fraternidad universal entre sus seguidores; por el otro, la corriente de violencia, división y lucha, de la guerra en definitiva; por un lado la corriente de los \textquote{pobres de espíritu}, de los buscadores del reino de Dios, de los creyentes en la vida eterna, por otro lado la corriente de los egoístas y de los buscadores del reino de la tierra, de los hombres que sólo en este siglo tienen su confianza; por un lado, la corriente que hace del amor a Dios y al prójimo la ley suprema de la vida individual y social; por el otro, la corriente que hace de la fuerza y de la revolución agresiva y arrolladora la razón ciega de los destinos de los pueblos; por un lado la corriente de fe y verdad y por tanto de libertad (cf. \textit{Jn} 8, 32); por el otro, la corriente de mil y desenfrenadas opiniones, que desde el exterior se imponen violando los derechos de conciencia\ldots Dos concepciones del mundo, de la verdad, de la vida: ¿cuál eliges?

Oh, bienaventurados vosotros, queridos hijos, que ya habéis elegido, y elegido según la sabiduría y la dicha, desde el día de vuestro bautismo, entregando la vida a esta profesión plena y feliz: ¡nosotros seremos cristianos, seremos de Cristo, estaremos con Cristo, en esta vida y en la futura! Y hoy, agitando las palmas, con la conciencia renovada, con más energía, confirmas tu elección, tu promesa: ¡sí, seremos cristianos!

Entonces, dos sentimientos llenan vuestros corazones: ¡valor y alegría!

Con nuestra bendición apostólica.
\end{body}



\newsection
\subsection{San Juan Pablo II, papa}

\subsubsection{Homilía (1979): Del Hosanna a la Cruz}

\src{Plaza de San Pedro, 8 de abril de 1979.}

\begin{body}
\ltr[1. ]{D}{urante} la próxima semana, la liturgia quiere ser estrictamente obediente a la sucesión de los acontecimientos. Precisamente los acontecimientos, que se desarrollaron en Jerusalén [hace poco menos de dos mil años], deciden que ésta sea la Semana Santa, la Semana de la Pasión del Señor.

El domingo de hoy permanece estrechamente unido con el acontecimiento que tuvo lugar cuando Jesús se acercó a Jerusalén para cumplir allí todo lo que había sido anunciado por los Profetas. Precisamente en este día los discípulos, por orden del Maestro, le llevaron un borriquillo, después de haber solicitado poderlo tomar prestado por cierto tiempo. Y Jesús se sentó sobre él para que se cumpliese también aquel detalle de los escritos proféticos. En efecto, así dice el Profeta Zacarías: \textquote{Alégrate sobremanera, hija de Sión, grita exultante, hija de Jerusalén. He aquí que viene a ti tu Rey, justo y victorioso, humilde, montado en un asno, en un pollino de asna} (\textit{Zac }9, 9).

Entonces, también la gente que se trasladaba a Jerusalén con motivo de las fiestas –la gente que veía los hechos que Jesús realizaba y escuchaba sus palabras– manifestando la fe mesiánica que Él había despertado, gritaba: \textquote{¡Hosanna! ¡Bendito el que viene en el nombre del Señor! ¡Bendito el reino que viene de David, nuestro Padre! ¡Hosanna en las alturas!} (\textit{Mc} 11, 9-10). Nosotros repetimos estas palabras en cada Misa cuando se acerca el momento de la transubstanciación.

2. Así, pues, en el camino hacia la Ciudad Santa, cerca de la entrada de Jerusalén, surge ante nosotros la escena del triunfo entusiasmante: \textquote{Muchos extendían sus mantos sobre el camino, otros cortaban follaje de los campos} (\textit{Mc} 11, 8).

El pueblo de Israel mira a Jesús con los ojos de la propia historia; ésta es la historia que llevaba al pueblo elegido, a través de todos los caminos de su espiritualidad, de su tradición, de su culto, precisamente hacia el Mesías. Al mismo tiempo, esta historia es difícil. El reino de David representa el punto culminante de la prosperidad y de la gloria terrestre del pueblo, que desde los tiempos de Abraham, varias veces, había encontrado su alianza con Dios-Yavé, pero también más de una vez la había roto.

Y ahora, ¿cerrará esta alianza de manera definitiva? ¿O acaso perderá de nuevo este hilo de la vocación, que ha marcado desde el comienzo el sentido de su historia?

Jesús entra en Jerusalén sobre un borriquillo que le habían prestado. La multitud parece estar más cercana al cumplimiento de la promesa de la que habían dependido tantas generaciones. Los gritos: \textquote{¡Hosanna!} \textquote{¡Bendito el que viene en el nombre del Señor!}, parecían ser expresión del encuentro ahora ya cercano de los corazones humanos con la eterna Elección. En medio de esta alegría que precede a las solemnidades pascuales, Jesús está recogido y silencioso. Es plenamente consciente de que el encuentro de los corazones humanos con la eterna Elección no sucederá mediante los \textquote{hosannas}, sino mediante la cruz.

Antes de que viniese a Jerusalén, acompañado por la multitud de sus paisanos, peregrinos para las fiestas de Pascua, otro lo había dado a conocer y había definido su puesto en medio de Israel. Fue precisamente Juan Bautista en el Jordán. Pero Juan, cuando vio a Jesús, al que esperaba, no gritó \textquote{hosanna}, sino señalándolo con el dedo, dijo: \textquote{He aquí el Cordero de Dios, que quita el pecado del mundo} (\textit{Jn} 1, 29). Jesús siente el grito de la multitud el día de su entrada en Jerusalén, pero su pensamiento está fijo en las palabras de Juan junto al Jordán: \textquote{He aquí el que quita el pecado del mundo} (\textit{Jn} 1, 29).

3. Hoy leemos la narración de la Pasión del Señor, según Marcos. En ella está la descripción completa de los acontecimientos que se irán sucediendo en el curso de esta semana. Y en cierto sentido, constituyen su programa.

Nos detenemos con recogimiento ante esta narración. Es difícil conocer estos sucesos de otro modo. Aunque los sepamos de memoria, siempre volvemos a escucharlos con el mismo recogimiento. \txtsmall{[Recuerdo con qué atención escuchaban los niños cuando siendo yo todavía joven sacerdote les contaba la Pasión del Señor. Era siempre una catequesis completamente distinta de las otras.]} La Iglesia, pues, no cesa de leer nuevamente la narración de la Pasión de Cristo, y desea que esta descripción permanezca en nuestra conciencia y en nuestro corazón. En esta semana estamos llamados a una solidaridad particular con Jesucristo: \textquote{Varón de dolores} (\textit{Is} 53, 3).

4. Así, pues, junto a la figura de este Mesías, que el Israel de la Antigua Alianza esperaba y, más aún, que parecía haber alcanzado ya con la propia fe en el momento de la entrada en Jerusalén, la liturgia de hoy nos presenta al mismo tiempo otra figura. La descrita por los Profetas, de modo particular por Isaías: \textquote{He dado mis espaldas a los que me herían\ldots sabiendo que no sería confundido} (\textit{Is} 50, 6-7).

Cristo viene a Jerusalén para que se cumplan en Él estas palabras, para realizar la figura del \textquote{Siervo de Yavé}, mediante la cual el Profeta, ocho siglos antes, había revelado la intención de Dios. El \textquote{Siervo de Yavé}: el Mesías, el descendiente de David, pero en quien se cumple el \textquote{hosanna} del pueblo, pero el que es sometido a la más terrible prueba: \textquote{Búrlanse de mí cuantos me ven\ldots, líbrele, sálvele, pues dice que le es grato} (\textit{Sal} 21, 8-9).

En cambio, no mediante la \textquote{liberación} del oprobio, sino precisamente mediante la obediencia hasta la muerte, mediante la cruz, debía realizarse el designio eterno del amor. Y he aquí que habla ahora no ya el Profeta, sino el Apóstol, habla \textbf{Pablo}, en quien \textquote{la palabra de la cruz} ha encontrado un camino particular. Pablo, consciente del misterio de la redención, da testimonio de quien \textquote{existiendo en forma de Dios\ldots se anonadó, tomando la forma de siervo\ldots, se humilló, hecho obediente hasta la muerte, y muerte de cruz} (\textit{Flp} 2, 6-8). He aquí la verdadera figura del Mesías, del Ungido, del hijo de Dios, del Siervo de Yavé. Jesús con esta figura entraba en Jerusalén, cuando los peregrinos, que lo acompañaban por el camino, cantaban: \textquote{Hosanna}. Y extendían sus mantos y los ramos de los árboles en el camino por el que pasaba.

5. Y nosotros hoy llevamos en nuestras manos los ramos de olivo. Sabemos que después estos ramos se secarán. Con su ceniza cubriremos nuestras cabezas el próximo año, para recordar que el Hijo de Dios, hecho hombre, aceptó la muerte humana para merecernos la Vida.
\end{body}

\label{b2-04-01-1979H}

\begin{patercite}
[\ldots] Y escucha ahora cuál será el fin: \textit{Está sentado a la derecha del trono de Dios}. ¿Ves cuál es el premio de la competición?

También san Pablo escribe sobre el tema y dice: \textit{Por eso Dios lo	levantó sobre todo, y le concedió el \textquote{Nombre-sobre-todo-nombre}, de modo que al nombre de Jesús toda rodilla se doble}. Se refiere a Cristo en su condición de hombre. Y aun cuando no se nos hubiera prometido ningún	premio por la competición, bastaría ---y con creces--- un ejemplo tal para persuadirnos a soportar espontáneamente todos los contratiempos; pero es que además se nos prometen premios, y no unos premios	cualquiera, sino magníficos e inefables premios.

Por lo cual, cuando también nosotros hayamos padecido algo semejante, pensemos en Cristo antes que en los apóstoles. ¿Y eso? Pues porque toda su vida estuvo llena de ultrajes; oía continuamente hablar mal de él, hasta el punto de llamársele loco, seductor, impostor. Y esto se lo echaban en cara, mientras él les colmaba de beneficios, hacía milagros y les mostraba las obras de Dios.

\textbf{San Juan Crisóstomo}, obispo, \textit{Homilía} 28 sobre la carta a los	Hebreos, cf.n. 2: PG 63, 195.
\end{patercite}

\newpage

\subsubsection{Homilía (1982): ¿Dónde encontrar a Cristo?}

\src{4 de abril de 1982.}

\begin{body}
1. \textquote{¡Hosanna! ¡Bendito el que viene en nombre del Señor! ¡Bendito el reino venidero de nuestro padre David! ¡Hosanna en lo alto del cielo!} (\textit{Mc} 11, 9s).

\ltr{E}{l} día de la exaltación de Jesús de Nazaret. Hubo un día en el que Jesús de Nazaret fue exaltado ante los ojos del pueblo. Y permitió esto. En efecto, en cierto sentido, él mismo creó las condiciones para que esto sucediera, al entrar en Jerusalén en un burro, rodeado de sus discípulos, justo cuando una multitud innumerable se dirigía hacia allí desde varios puntos de Tierra Santa.

Cuando los fariseos dijeron: \textquote{Maestro, reprende a tus discípulos}, les respondió: \textquote{Os digo que si estos callan, las piedras clamarán} (\textit{Lc} 19, 39s). Hubo un día en que Jesús de Nazaret, cumpliendo la voluntad del Padre, permitió que la gloria terrenal del Mesías se manifestara en él: que se manifestara en Jerusalén y que fuera proclamada por los labios de sus compatriotas. De esta manera, de hecho, se cumpliría la Escritura, que expresa la gloria del Mesías de manera regia: como exaltación de la descendencia de David. Así pues, hoy celebramos el día de la exaltación de Jesús de Nazaret ante los ojos de los hombres. También hoy, entrando en la liturgia de la Semana Santa, comenzamos a meditar sobre el misterio de la exaltación del Mesías ante Dios.

2. La liturgia del Domingo de Ramos es tan maravillosa como los acontecimientos del día a que se refiere. Sobre el entusiasmo del mesiánico \textquote{Hosanna} se cierne una profunda sombra. Ésta es la sombra de la pasión que se acerca. Cuán significativas son incluso estas palabras del profeta que se cumplen en este día: \textquote{¡No temas, hija de Sion! ¡He aquí que viene tu rey, sentado sobre un pollino de asno!} (\textit{Jn} 12, 15; cf. \textit{Zc} 9, 9). ¿Podría la hija de Sión tener motivos para temer en el día del entusiasmo general del pueblo por la venida del Mesías? Sin embargo, sí. Se acerca el momento en que se cumplirán las palabras del \textbf{salmista} en labios de Jesús de Nazaret: \textquote{Dios mío, Dios mío, ¿por qué me has abandonado?} (\textit{Sal} 21 [22], 2). Él mismo pronunciará estas palabras desde lo alto de la Cruz.

Entonces, en lugar del entusiasmo del pueblo que canta \textquote{Hosanna}, asistiremos a la burla en el patio de Pilatos, en el Gólgota, como proclama el salmista: \textquote{Al verme, se burlan de mí, hacen visajes, menean la cabeza: \textquote{Acudió al Señor, que lo ponga a salvo; que lo libre si tanto lo quiere}} (\textit{Sal} 21 [22], 8s).

3. La liturgia de hoy –la liturgia del Domingo de Ramos–, que nos permite detenernos en la entrada triunfal de Cristo en Jerusalén, nos conduce simultáneamente al final de la Pasión. \textquote{Me taladran las manos y los pies, puedo contar todos mis huesos\ldots}. Y también: \textquote{Se reparten mi ropa, echan a suerte mi túnica} (\textit{Sal} 21 [22], 17-19). Como si el salmista ya viera con sus propios ojos el desarrollo del Viernes Santo. Verdaderamente en ese día, ahora cercano, Cristo se hará obediente hasta la muerte, y muerte de cruz (cf. \textit{Flp} 2, 8).

4. Y aquí mismo, al final de la Pasión, comienza el misterio de la exaltación del Mesías. Esta exaltación es diferente de la exaltación \textquote{histórica} ante los hombres en el día del alegre \textquote{hosanna}. Ésta es la exaltación en Dios mismo. La humillación de Cristo y su abnegación definitiva a través de la Cruz se convirtió en una introducción inmediata a esta exaltación en Dios. \textquote{Cristo Jesús, siendo de condición divina, no retuvo ávidamente el ser igual a Dios; al contrario, se despojó de sí mismo tomando la condición de esclavo\ldots} (\textit{Flp} 2, 6s). Estas palabras de la \textbf{carta a los Filipenses} se refieren no solo a la Pasión. Constituyen, en cierto sentido, la síntesis de toda la vida de Cristo. Son el indicador de todo el misterio de la Encarnación.

En efecto, de estas palabras se desprende que \textquote{se despojó de sí mismo} por el hecho mismo de que, \textquote{a pesar de ser de naturaleza divina}, aceptó la condición humana, la naturaleza humana: asumió la \textquote{condición de siervo}. Pudiendo en cada paso \textquote{aprovechar la oportunidad de ser igual a Dios}, eligió conscientemente todo lo que lo colocaba \textquote{a la par} del hombre: \textquote{reconocido externamente como hombre}. Y he aquí, nos acercamos al final de esta nivelación. Lo alcanzaremos entonces, cuando Cristo \textquote{se humille haciéndose obediente hasta la muerte y la muerte de cruz}.

5. Pero precisamente este término significa el comienzo de la exaltación.

La exaltación de Cristo está contenida en su abajamiento. La gloria tiene su comienzo y su fuente en la Cruz. En su \textbf{carta a los Filipenses}, San Pablo lo subraya claramente cuando comienza la siguiente frase de su magnífico texto con la palabra \textquote{por eso}. \textquote{Por eso Dios lo exaltó y le dio el Nombre-sobre-todo-Nombre} (\textit{Flp} 2, 9). El Apóstol ve esta exaltación en la medida del mundo visible e invisible. Por lo tanto, escribe: \textquote{\ldots Y le dio el Nombre-sobre-todo-Nombre; para que en el nombre de Jesús toda rodilla se doble en los cielos, en la tierra y en los abismos; y toda lengua proclame que Jesucristo es el Señor, para gloria de Dios Padre} (\textit{Flp} 2, 9-11). Esta es la medida de la exaltación de Cristo en Dios, de ese Cristo, que el Domingo de Ramos permitió su \textquote{exaltación} ante los ojos de Jerusalén, cuando faltaban pocos días para la crucifixión.

Con el domingo de hoy, la Iglesia está en el umbral de la Semana Santa. Esta es la Semana Santa. Contiene el misterio del despojamiento de Cristo y su exaltación: de la exaltación mediante la abnegación. Con mucha humildad, con fe y con amor vamos al encuentro de este Misterio.
\end{body}

\label{b2-04-01-1982H}
\newpage

\subsubsection{Homilía (1985): Asumió la causa del hombre}

\src{31 de marzo de 1985.}

\begin{body}
1. \textquote{¡Hosanna! ¡Bendito el que viene en nombre del Señor!} (\textit{Mt} 21, 9).

\ltr{H}{emos} venido a este lugar \txtsmall{[–a esta Plaza de San Pedro– vosotros, jóvenes de diferentes naciones y el Obispo de Roma;]} hemos venido repitiendo el grito que sonó, hace [casi dos mil años], en las calles de Jerusalén. Este grito se refería entonces, y también hoy, a Jesús de Nazaret. ¡Él es el que viene en nombre del Señor! ¡Es a él a quien cantan \textquote{Hosanna}! Él es el bendito: ¡es el Mesías!

El día de su entrada en Jerusalén, los labios de los habitantes de la ciudad santa y de los numerosos peregrinos proclaman esta alegre noticia. En primer lugar lo proclaman los jóvenes: \textquote{\textit{Pueri Hebraeorum}}. Queremos que se escuche el grito de aquellos jóvenes, repetido hoy de nuevo por los jóvenes. \txtsmall{[Que se sienta especialmente en este año, proclamado Año de la Juventud en todo el mundo.]}

2. Aquí, hemos venido aquí para celebrar la liturgia del Domingo de Ramos. Según Marcos, la descripción de la pasión de nuestro Señor Jesucristo nos introduce de inmediato a los eventos de la semana, que comienza hoy. ¡Es la semana de la pasión del Señor! Jesús de Nazaret, a quien hoy cantamos \textquote{Hosanna} y a quien reconocemos como \textquote{bendito} será condenado a muerte. El Viernes Santo otros labios, también en la misma Jerusalén, gritarán \textquote{¡Crucifícalo!\ldots ¡Crucifícalo!} (\textit{Mc} 15, 13-14). Gritarán: \textquote{Que su sangre sea sobre nosotros y sobre nuestros hijos} (\textit{Mt} 27, 25). ¿Por qué? Buscamos la respuesta en la descripción de los evangelistas: Mateo, Marcos, Lucas y Juan.

Es como sigue: Jesús escuchó ante el Sanedrín la pregunta: \textquote{Te conjuro por el Dios vivo que nos digas si eres el Cristo, el Hijo de Dios} (\textit{Mt} 26, 63). Él respondió: \textquote{Tú lo has dicho} (\textit{Mt} 26, 64). Entonces el Sumo Sacerdote se rasgó las vestiduras y el tribunal dictó la sentencia: \textquote{¡Es culpable de muerte!} (\textit{Mt} 26, 66). El motivo de la sentencia fue religioso. Jesús fue condenado por blasfemo. Ante Pilato, el procurador romano, Jesús no es acusado de lo mismo. No es acusado de llamarse a sí mismo el Hijo de Dios (cf. \textit{Jn} 19, 7), sino de afirmar ser Cristo Rey (cf. \textit{Lc} 23, 2). El motivo de la sentencia es político. Jesús es condenado por usurpador.

3. Sin embargo, entre una frase y la otra tuvo lugar una entrevista, en la que el imputado respondió a Pilato de la siguiente manera: \textquote{Tú lo dices; soy rey. Para esto nací y para esto vine al mundo: para dar testimonio de la verdad. Todo el que es de la verdad escucha mi voz} (\textit{Jn} 18, 37). ¿Entendió Pilato? Parece que no. Sin embargo, cuando volvió a conducir a Jesús con la corona de espinas en la cabeza ante los acusadores, lo señaló y dijo: \textquote{He aquí el hombre} (\textit{Jn} 19, 5). Entonces Jesús de Nazaret, condenado por el Sanedrín como blasfemo, es condenado por Pilato como usurpador: es condenado ante todo como hombre. Es condenado porque asumió la causa del hombre: la causa eterna y última: \textquote{Para esto nací y para esto vine al mundo} (\textit{Jn} 18, 37).

4. El \textbf{apóstol Pablo} habla en la liturgia de hoy de la causa del hombre, que Jesucristo tomó sobre sí y cargó en la cruz. El texto de la \textbf{Carta a los Filipenses} es conciso y al mismo tiempo maravillosamente rico, profundo. El apóstol escribe: \textquote{Cristo Jesus, siendo de condición divina, no retuvo ávidamente el ser igual a Dios; al contrario, se despojó de sí mismo tomando la condición de esclavo, hecho semejante a los hombres} (\textit{Flp} 2, 6-7).

¿Quién es Jesucristo? Está \textquote{en igualdad con Dios} (cf. \textit{Flp} 2, 6). Él es de la misma sustancia que el Padre. Él es Dios de Dios, luz de luz. Él es el hijo de Dios, es el verdadero Dios. Al mismo tiempo, este Hijo de Dios, de la misma sustancia que el Padre, \textquote{se hizo hombre}. Él es un hombre real.

¿Quien es el hombre? Es una criatura. \textquote{Dios creó al hombre a su imagen; varón y hembra los creó} (\textit{Gen} 1, 27). El hombre es criatura de Dios y, al mismo tiempo, es imagen y semejanza de Dios. El problema del hombre, el problema eterno y definitivo está contenido aquí: el hombre es, entre todas las criaturas del mundo visible, \textquote{semejante a Dios} y al mismo tiempo es criatura.

Cristo tomó sobre sí la causa del hombre, haciéndose hombre. Dios-Hijo, de la misma sustancia que el Padre, ya que el hombre tomó su lugar en el orden de las criaturas. En cierto sentido, se \textquote{despojó} de la divinidad, permaneciendo Dios Hijo. Como hombre-criatura se convirtió en un sirviente: el sirviente de su Creador. El Siervo de Yahvé.

5. Sí, Cristo ha venido a ser el Siervo de Yahvé según la \textbf{profecía de Isaías}, cuyo pasaje leemos en la liturgia de hoy: \textquote{Ofrecí la espalda a los que me golpeaban, las mejillas a los que mesaban mi barba; no escondí el rostro ante ultrajes y salivazos} (\textit{Is} 50, 6). Siervo de Yahvé: el Siervo sufriente de Yahvé, la imagen profética más completa del Mesías torturado. La misma imagen se presenta en el \textbf{salmo responsorial} de la liturgia de hoy, que es el \textbf{Salmo 21}.

Dios-Hijo, de la misma sustancia que el Padre, \textquote{en igualdad con el Padre}, como criatura, como hombre, ¡se convierte en servidor de Dios y de los hombres! Proclama el programa de este servicio en el Evangelio, sobre todo cuando, mediante la cena pascual, se prepara para la pasión: cuando lava los pies de sus discípulos. \textquote{El Hijo del Hombre\ldots no vino para ser servido, sino para servir} (\textit{Mt} 20, 28). ¡Para servir\ldots a los hombres!

De esta manera asume sobre sí la causa del hombre, para llevarla a cabo.

De hecho, el hombre es imagen y semejanza de Dios. Esta imagen y semejanza llega a su cúspide cuando el Hijo de Dios, el mismo Verbo eterno, se hace hombre. Al mismo tiempo, el hombre es una criatura. No puede olvidar el hecho de que es la imagen de Dios, no puede olvidar el hecho de que es una criatura de Dios, y que como criatura es el servidor de su Creador. El uno y el otro determinan fundamentalmente el ser mismo del hombre y su lugar en el cosmos. Ser hombre significa mantener la justa proporción entre la criatura y la imagen de Dios. Mantener el equilibrio.

El hombre ha perdido este equilibrio. Se lo ha dejado quitar. Consciente y voluntariamente siguió la voz del tentador que decía a ambos, mujer y hombre, llegaréis a ser \textquote{como Dios, conocedores del bien y del mal} (\textit{Gen} 3, 5). El hombre rechazó la voluntad de Dios en ese momento, destruyó la proporción entre la imagen de Dios y la criatura de Dios.

6. Jesucristo vino al mundo para restaurar, por así decirlo, esta proporción en su raíz: el equilibrio perdido. ¡Por lo tanto, él es el nuevo comienzo! El nuevo comienzo de la historia del hombre en Dios. Precisamente por eso él, el Hijo, \textquote{en igualdad con Dios}, de la misma sustancia que el Padre, como hombre, asume \textquote{la condición de siervo}. Más aún: como hombre \textquote{que apareció en forma humana, se humilló haciéndose obediente hasta la muerte y la muerte de cruz} (\textit{Flp} 2, 7-8). De esta manera llegó al comienzo mismo del equilibrio perdido. En la escala de la desobediencia original, colocó su obediencia hasta la muerte y la muerte de cruz.

Cuando agonizaba en la cruz con los brazos clavados en el madero, hasta el punto de no poder \textquote{esconder su rostro de insultos y salivazos\ldots}. Entonces se cumplieron las palabras que Cristo le dijo a Pilato: \textquote{Para esto nací y para esto vine al mundo: para dar testimonio de la verdad}. Da testimonio de la verdad: de la verdad sobre Dios y el hombre; de esta verdad que, al comienzo de la historia del hombre en la tierra, fue falsificada. Aquel a quien la Escritura llama \textquote{el padre de la mentira} (\textit{Jn} 8, 44) la ha falsificado.

Simplemente dijo al hombre: llegarás a ser \textquote{como Dios}. Si bien el hombre es criatura y al mismo tiempo es imagen y semejanza de Dios, no por rebelión y oposición, sino por gracia y amor, debe llegar a ser –en Cristo-Hijo–, un hijo de Dios. He aquí el Hijo del hombre, muriendo en el Gólgota. Así, el Verbo que se hizo carne da a los hombres el \textquote{poder de convertirse en hijos de Dios} (\textit{Jn} 1, 12). Este poder se opone a la mentira de la tentación eterna.

7. De este modo, Jesús de Nazaret tomó sobre sí la causa del hombre, la causa eterna y última, la causa del hombre: ¡ayer y hoy y hasta el fin!

\txtsmall{[Vosotros, jóvenes, que os habéis reunido aquí, –viniendo de Roma, Italia, de diferentes países, naciones y continentes–, habéis venido en el Domingo de Ramos para repetir: \textquote{¡Hosanna! ¡Bendito el que viene en nombre del Señor}.]} Habéis venido a oír, el próximo Viernes Santo: \textquote{Para esto nací y para esto vine al mundo: para dar testimonio de la verdad} (\textit{Jn} 18, 37).

¿No viene cada uno de nosotros, cada hombre, al mundo en primer lugar para dar testimonio de la verdad? ¡Que este testimonio de la verdad que Jesús de Nazaret da, penetre profundamente en vosotros! En él está contenida la causa del hombre: ¡la causa eterna y al mismo tiempo última! Jesucristo es: ayer, hoy y siempre. Y la causa del hombre está en él: ayer, hoy y siempre.

8. {[Vosotros, jóvenes, conocéis la causa del hombre de hoy, en este final del segundo milenio después de Cristo.]}

Hoy el hombre está orgulloso de sus logros. ¡Nunca hemos sido testigos de avances tan gigantescos en ciencia y tecnología! ¿No encuentran aquí su confirmación las palabras \textquote{llegarás a ser como Dios}?

Y, al mismo tiempo, el hombre de hoy se siente amenazado\ldots amenazado de varias formas. Nunca antes el hombre se había sentido tan amenazado como hoy\ldots ¿No encuentran aquí las palabras \textquote{llegarás a ser como Dios} su negación más radical?

Los jóvenes se preguntan: ¿cuál será nuestro futuro en este \textquote{mundo nuevo y magnífico}?

¿Cuál será el futuro del hombre en este mundo de la electrónica y descubrimientos maravillosos, espléndidos y al mismo tiempo amenazantes? ¿El futuro de la persona?

En este mundo, donde algunos hombres parecen dominar tan ampliamente, mientras que otros hombres, hay millones y niños indefensos entre ellos, ¡se mueren de hambre! Se encuentran en campos de refugiados. También son perseguidos por su fe, por la voz de su conciencia.

Si todos los jóvenes de todos los confines y rincones de la tierra pudieran reunirse aquí, la pregunta sobre la causa del hombre hoy crecería con muchas otras preguntas. Y en estas preguntas habría muchos miedos y preocupaciones. Muchas quejas y acusaciones.

¿No sentimos que en este mundo falta cada vez más el equilibrio entre la imagen humana de Dios y la criatura humana?

¿No sentimos que en este mundo se ha manipulado y falseado la causa del hombre: la eterna y definitiva? ¿No sentimos que los choques cósmicos y apocalípticos de la desobediencia original se sienten continuamente en este mundo?

9. Por tanto, en este mundo, en el mundo del segundo milenio [que se acerca a su fin], el que se ha hecho obediente hasta la muerte y la muerte de cruz es cada vez más necesario. Cristo es indispensable para el mundo.

\txtsmall{[Y esto, jóvenes, queréis profesar hoy junto conmigo, obispo de Roma y sucesor de Pedro, con los cardenales, obispos, sacerdotes, consagrados aquí presentes. Y queréis decirlo en voz alta a todos los hombres y especialmente a todos vuestros coetáneos en el contexto del Año Internacional de la Juventud.]}

\textquote{Bendito el que viene en nombre del Señor. ¡Hosanna al hijo de David!}.

10. Dios eterno: el Padre, el Hijo y el Espíritu Santo asumieron la causa del hombre –la causa eterna y última del hombre– en Cristo, que dio testimonio de la verdad; en Cristo, condenado como blasfemo y usurpador; en Cristo, azotado y coronado de espinas; en Cristo crucificado Dios tomó la causa del hombre: ayer, hoy y siempre. Y en este mismo Cristo le da un \textquote{nuevo comienzo}.

De esta manera la pregunta sobre la causa del hombre está llena de esperanza. Está llena de confianza\ldots

Aquí están las palabras del mismo \textbf{apóstol Pablo}: \textquote{Por eso Dios lo exaltó y le dio el nombre que está por encima de cualquier otro nombre; para que en el nombre de Jesús toda rodilla se doble en los cielos, en la tierra y debajo de la tierra; y toda lengua proclama que Jesucristo es el Señor, para gloria de Dios Padre} (\textit{Flp} 2, 9-11).

Sí. ¡Jesucristo es el Señor!

Él es el Señor del siglo venidero. En él la causa del hombre está llena de esperanza. Nuestra \textquote{esperanza en él está llena de inmortalidad} (\textit{Sab} 3, 4).

¡Bendito, bendito el que viene en nombre del Señor! ¡Hosanna, Hosanna! ¡Amén!
\end{body}

\label{b2-04-01-1985H}

\begin{patercite}
	(\ldots) La cruz impuesta a nuestro Salvador y que los hombres tenían por un simple leño, a los ojos del Padre común de la humanidad era considerada como un grandioso y excelso altar, erigido para la salvación del mundo e impregnado del incienso de una víctima santa y purísima.
	
	Por eso Cristo, mientras su cuerpo era flagelado y al mismo tiempo	escupido por los atrevidísimos judíos, decía, por el profeta Isaías,	estas palabras: \textit{Ofrecí la espalda a los que me golpeaban, la mejilla a	los que mesaban mi barba}. Pues el Padre es un solo Dios, y Jesucristo,	un solo Señor: ¡bendito él por siempre! El cual, desdeñando la ignominia	por nuestra salvación, y hecho obediente al Padre, \textit{se rebajó hasta	someterse incluso a la muerte}, para que habiendo el Salvador dado su	vida por nosotros y en nuestro lugar, pudiera a su vez resucitarnos de	entre los muertos, vivificados por el Espíritu Santo; situarnos en el	domicilio celestial, abiertas de par en par las puertas del cielo y	colocar en la presencia del Padre y ante sus ojos, aquella naturaleza	humana, que desde tiempo inmemorial se le había sustraído huyendo de él	por el pecado.
	
	Amados hermanos, que por estas egregias hazañas de nuestro Salvador,	prorrumpan las bocas de todos en alabanza, y que todas las lenguas se	afanen en componer cantos de alabanza en su honor, haciendo suyo aquel	dulcísimo cántico: \textit{Dios asciende entre aclamaciones, el Señor, al son	de trompetas}. Asciende una vez consumada la obra de la salvación	humana. Y no sólo sube, sino que: \textit{subiste a la cumbre llevando	cautivos, te dieron tributo de hombres}.
	
	\textbf{San Cirilo de Alejandría}, obispo, \textit{Homilías Pascuales} cf. 5, 7: PG 77, 495-498.
\end{patercite}

\newpage

\subsubsection{Homilía (1988): Palabras de vida eterna}

\src{III Jornada Mundial de la Juventud. \\Plaza de San Pedro. 27 de marzo de 1988.}

\begin{body}
1. \textquote{Señor, ¿a quién iremos? Tú tienes palabras de vida eterna} (\textit{Jn} 6, 68).

\ltr{C}{elebramos} la liturgia del Domingo de Ramos en la plaza de San Pedro. Esta es también la Jornada internacional de la Juventud. [El Domingo de Ramos reúne todos los años en esta plaza a muchos jóvenes, que se sienten como llamados por el acontecimiento que se conmemora este día. Efectivamente,] durante la entrada de Jesús en Jerusalén, entre los que gritaban \textquote{Hosana al Hijo de David}, no faltaron los jóvenes. El himno litúrgico canta: \textquote{\textit{Pueri hebraeorum portantes ramos olivarum obviaverunt Domino}}.

\textit{Pueri}: es decir, los jóvenes hebreos. \textit{Obviaverunt}: es decir, fueron al encuentro de Cristo. Cantaron \textquote{Bendito el que viene en nombre del Señor} (\textit{Mt} 21, 9). Cada año, el Domingo de Ramos sucede lo mismo: Los jóvenes van al encuentro de Cristo, enarbolan las palmas, cantan el himno mesiánico, para saludar a Aquel que viene en el nombre del Señor. \txtsmall{[Así sucede aquí en Roma, como en otros lugares del mundo. El año pasado fue así en Buenos Aires, donde pude celebrar la Jornada de la Juventud especialmente con los jóvenes de América Latina.]}

Todos vosotros, jóvenes, allí donde estéis y cualquier día que os reunáis para celebrar vuestra fiesta, sentiréis la necesidad de repetir las palabras de Pedro: \textquote{Señor, ¿a quién iremos? Tú tienes palabras de vida eterna}. Sólo Tú.

2. Las \textquote{palabras de vida eterna} nos describen hoy la pasión y la muerte de Cristo según el \textbf{Evangelio de San Marcos}.

Hemos escuchado esta descripción. Hemos escuchado también las palabras del \textbf{Profeta Isaías}, que desde las profundidades de los siglos preanuncia al Mesías, como varón de dolores: \textquote{Ofrecí la espalda a los que me golpeaban, la mejilla a los que mesaban mi barba. No oculté el rostro a insultos y salivazos} (\textit{Is} 50, 6).

De hecho fue precisamente así, como había previsto el Profeta.

Y, fue también así, como había proclamado el \textbf{Salmista} –también él desde la profundidad de los siglos–: \textquote{Me taladran las manos y los pies, puedo contar todos mis huesos\ldots Se reparten mi ropa, echan a suerte mi túnica} (\textit{Sal} 21[22], 17-19).

Así fue. Y aún más. Las palabras con que el Profeta (David) comienza su Salmo estuvieron en los labios de Cristo durante la agonía en Getsemaní: \textquote{Dios mío; Dios mío, ¿por qué me has abandonado?} (¿Elí, Elí, lamá sabactaní?) (\textit{Mt} 27, 46; \textit{Sal} 21[22], 2).

La pasión y la muerte de Cristo emergen de los textos del Antiguo Testamento para confirmarse como la realidad decisiva de la Nueva y Eterna Alianza de Dios con la humanidad.

3. Hemos escuchado finalmente las palabras impresionantes del \textbf{Apóstol Pablo} en la \textbf{Carta a los Filipenses}. Son una síntesis del misterio pascual. El texto es conciso, pero al mismo tiempo tiene un contenido insondable, como lo es el misterio. San Pablo nos lleva al límite mismo de lo que en la historia de la creación comenzó a suceder entre Dios y el hombre, y que encontró su culmen y su plenitud en Jesucristo. En definitiva, en la cruz y resurrección.

Jesucristo \textquote{a pesar de su condición divina, no hizo alarde de su categoría de Dios; al contrario, se despojó de su rango, y tomó la condición de esclavo, pasando por uno de tantos. Y así, actuando como un hombre cualquiera, se rebajó hasta someterse incluso a la muerte, y una muerte de cruz. Por eso Dios lo levantó sobre todo\ldots} (\textit{Flp} 2, 6-9).

Así \textquote{las palabras de vida eterna} fueron pronunciadas por medio de la cruz y de la muerte. No eran sólo teoría. Fueron una realidad tremenda entre Aquel que \textquote{Es} \textit{ab aeterno}, que no pasa, y aquel que pasa, para el que está establecido que debe morir una sola vez. Al mismo tiempo el hombre, como ser creado a imagen y semejanza de Dios, espera las palabras de vida eterna. Las encuentra en el Evangelio de Cristo. Se confirman de forma definitiva en su muerte y resurrección.

¿A quién iremos?

Cristo es Aquel que \textquote{en la misma revelación del misterio del Padre y de su amor}, no cesa de manifestar \textquote{plenamente el hombre al propio hombre y le descubre la sublimidad de su vocación, revelando el misterio del Padre y de su amor}. Esto dice el Concilio Vaticano II en la Constitución pastoral \textit{Gaudium et spes} (n. 22).

4. \txtsmall{[¿Por qué, pues, precisamente este día, Domingo de Ramos, se ha convertido en la Iglesia desde hace algunos años en la \textquote{fiesta de los jóvenes}?: Jornada de los jóvenes. Es cierto que esta jornada de la juventud se celebra en cada país y en ambientes y períodos diversos, pero el Domingo de Ramos queda siempre para ella como un punto central de referencia.

¿Por qué? Parece que los mismos jóvenes dan a esta pregunta una respuesta espontánea. Una respuesta así la dais todos vosotros, que desde hace años peregrináis a Roma precisamente para celebrar este día (y esto se realizó especialmente el Año de la Redención y el Año dedicado a la juventud). Con este hecho, ¿acaso no queréis hacer ver vosotros mismos que buscáis a Cristo en el centro de su misterio?]}

Buscáis a Cristo en la plenitud de esa verdad que es Él mismo en la historia del hombre: \textquote{Para esto he nacido y para esto he venido al mundo: para dar testimonio de la verdad} (\textit{Jn} 18, 37). Vosotros buscáis a Cristo en la palabra definitiva del Evangelio, como lo hizo el Apóstol Pablo: En la cruz, que es \textquote{fuerza de Dios y sabiduría de Dios} (\textit{1 Co} 1, 24), como confirmó la resurrección.

En Cristo –crucificado y resucitado– buscáis precisamente esa fuerza y esa sabiduría.


\newpage

5. Cristo revela plenamente el propio hombre al hombre –cada uno de nosotros–. ¿Podría revelarlo \textquote{plenamente} si no hubiera pasado también este sufrimiento, y este despojo sin límites? ¿Si no hubiera finalmente gritado en la cruz: \textquote{Por qué me has abandonado?} (cf. \textit{Mt} 27, 46).

El campo de la experiencia del hombre es limitado. Inefable es también el cúmulo de sus sufrimientos. El que tiene \textquote{palabras de vida eterna}, no dudó en fijar esta palabra en todas las dimensiones de la temporalidad humana\ldots

\textquote{Por eso Dios lo levantó}. Por eso, \textquote{Jesucristo es el Señor, para gloria de Dios Padre} (cf. \textit{Flp} 2, 9. 11). Y de este modo da testimonio de \textquote{la sublimidad de su vocación} (cf. \textit{Gaudium et spes}, 22): ninguna dificultad, ningún sufrimiento o despojo, pueden separarnos del amor de Dios (cf. \textit{Rm} 8, 35): De ese amor que está en Jesucristo.

6. Así, pues, esta \textquote{Jornada para los jóvenes} queda en la Iglesia como un momento elocuente de vuestra \textquote{peregrinación a través de la fe}.

Este año dirigimos nuestra mirada a la Madre de Dios presente en el misterio de Cristo y de la Iglesia, presente también en la agonía del Gólgota. Allí precisamente se encuentra el punto culminante de la peregrinación de María, de la que el Concilio, siguiendo las iniciativas de la Tradición, nos enseña que nos precede a todos en el camino: Va delante en la peregrinación \textquote{de la fe, de la caridad y de la unión perfecta con Cristo} (cf. \textit{Lumen gentium}, 63).

\txtsmall{[En este Año mariano deseo a todos los jóvenes que, mirando a María como \textquote{modelo}, descubran todas las profundidades escondidas en el misterio de Cristo. Ya que Cristo dice siempre de nuevo a los jóvenes, como dijo en el Evangelio: \textquote{Sígueme} (\textit{Lc} 18, 22). El análisis de esta llamada se encuentra en la Carta enviada a todos lo jóvenes del mundo en el año 1985. Es necesario que sintáis esta llamada. Y es necesario que la maduréis constantemente para darle vuestra respuesta.]}

\textquote{Señor, ¿a quién iremos? Tú tienes palabras de vida eterna}.
\end{body}

\label{b2-04-01-1988H}



\newpage

\subsubsection{Homilía (1991): Los caminos de Cristo}

\src{VI Jornada Mundial de la Juventud. \\Plaza de San Pedro. 24 de marzo de 1991.}

\begin{body}
\textquote{Llevaron el pollino, le echaron encima los mantos, y Jesús se montó} (\textit{Mc} 11, 7).

\ltr[1. ]{A}{sí} Cristo inició el camino que lo llevó a Jerusalén para celebrar la Pascua, después de haber atravesado con sus propios pies muchos caminos, es decir, toda la tierra de Palestina. Pero solo así es como caminaba sobre el lomo del burro. Y así se cumplieron las palabras del profeta: \textquote{¡No temas, hija de Sion! He aquí que viene tu Rey, sentado sobre un pollino de asna} (\textit{Jn} 12, 15; cf. \textit{Zc} 9, 9).

¡Viene el Rey! Incluso los peregrinos que acompañaron a Jesús en ese camino lo reconocieron como tal, aclamando: \textquote{Bendito el que viene, el Rey, en el nombre del Señor} (\textit{Lc} 19, 38); \textquote{¡Hosanna al hijo de David!\ldots ¡Hosanna en las alturas!} (\textit{Mt} 21, 9).

¡Viene el Rey! Unos días después quedaría claro cuál era su Reino. Pero en ese momento los anuncios de los profetas coincidieron, de manera sorprendente, con este hecho. La entrada del Mesías en Jerusalén estaba prevista como la entrada de un Rey.

2. Sólo él sabía adónde le llevarían los caminos de Galilea, Samaría, Judea, que recorrió durante los años de su vida. ¡Él también sabe adónde conduce este camino hoy!

Lleva consigo toda la verdad del Evangelio que proclamó. Sabe que \textquote{si el grano\ldots caído en la tierra no muere, queda solo} (\textit{Jn} 12, 22). Él es el grano que debe producir el fruto, muriendo. Él es el grano que cayó en esa tierra antigua, que constituye una pequeña parte de toda la tierra, de todo el planeta destinado por el Creador como morada de los hombres.

Él, Cristo, es también la Encarnación viviente de las Ocho Bienaventuranzas. Él conoce toda la verdad. Así como también conoce profundamente la verdad de esas palabras aparentemente paradójicas: \textquote{\ldots El que quiera salvar su vida, la perderá; pero el que pierda su vida por mí (por el Evangelio), la encontrará} (\textit{Mt} 16, 25).

He aquí, él es el primero de aquellos a quienes se refieren estas palabras. Él es el que entra en Jerusalén para \textquote{perder la vida}, \textquote{para\ldots dar su vida en rescate por muchos} (\textit{Mt} 20, 28), para \textquote{darse a sí mismo} (cf. \textit{1 Tm} 2, 6). En estos próximos días, durante la Pascua en Jerusalén, la paradoja de la cruz nos hará descubrir toda la profundidad de la verdad que contiene.

Después de todo, ¿no dijo Jesús de Nazaret: \textquote{si alguien quiere venir en pos de mí\ldots Tome su cruz todos los días y sígame}? (\textit{Lc} 9, 23). En unos días el camino que hoy conduce a la entrada triunfal a Jerusalén se transformará en el camino del Condenado, que lleva la cruz para entregar su alma al Padre.

3. Una vez más os habéis reunido en torno a Jesucristo, \txtsmall{[queridos jóvenes de Roma y de diferentes países. Habéis aceptado que el Domingo de Ramos es el Día de la Juventud en la Iglesia. Sabéis que, además de este día, os espera en el próximo mes de agosto –tras los encuentros en Buenos Aires, Argentina, y Santiago de Compostela, España– el de Czestochowa, Polonia.]
	
[A veces surgen voces que intentan frustrar el sentido de esta peregrinación. Pero su significado es demasiado obvio. Esta peregrinación permite que Cristo hable al hombre. Al hombre de nuestra época. En particular a los jóvenes, cuyas perspectivas van más allá de las fronteras del segundo milenio. Peregrinamos en pos de Cristo para escuchar, más allá de las palabras e imágenes de las que nos nutre nuestra civilización, su palabra en toda su sencillez y austeridad evangélica. \textquote{Tú tienes palabras de vida eterna} (\textit{Jn} 6, 68).]

[Hacemos un peregrinaje en pos de Cristo para conocer la verdad sobre nosotros mismos, la verdad sobre el hombre. Esta verdad no puede separarse de sus raíces eternas. Nos ayuda a no desapegarnos de la verdad del Dios Viviente. Cristo en realidad \textquote{revela\ldots el hombre al hombre} (\textit{Gaudium et spes}, 22) y revela su altísima vocación de modo que sin él, sin el Evangelio, sin el Domingo de Ramos y el misterio pascual el hombre no puede conocer completamente la verdad sobre sí mismo.]}

4. ¿Quién es el hombre? El último Concilio responde: \textquote{en la tierra es la única criatura que Dios quiso para sí}. Por lo tanto, no puede \textquote{encontrarse plenamente a sí mismo excepto a través de un sincero don de sí mismo} (\textit{Gaudium et spes}, 24). Esta respuesta es la síntesis de la verdad contenida en el Evangelio, de la verdad profundizada y verificada a través de las generaciones de quienes han seguido a Cristo a lo largo de los siglos.

Cristo mismo lo expresó de la manera más completa. Lo expresó a través de sí mismo. ¿Qué significa, de hecho, \textquote{convertirse en un don desinteresado para los demás}, si no \textquote{dar el alma}, \textquote{perder el alma}? ¿No fue Cristo quien nos aseguró que cuando el hombre \textquote{se encuentra} entonces \textquote{da cien veces más fruto} (cf. \textit{Mt} 13, 23; Lc 8, 8)? No considera su existencia como una \textquote{pasión inútil}, sino que la llena con la certeza del Sentido Último.

5. Mientras Jesús entraba en Jerusalén, también escuchó estas palabras: \textquote{Maestro, reprende a tus discípulos} (\textit{Lc} 19, 39). ¡Reprende! ¡Manda que se callen, que dejen de cantar, que no hagan peregrinaciones! ¡El mundo también ha llegado lejos en muchas otras direcciones! Jesús respondió: \textquote{Os digo que si estos callan, las piedras clamarán} (\textit{Lc} 19, 40). Y así, después de dos mil años, los hombres continúan clamando su venida al mundo y su Evangelio de salvación.

\textquote{Bendito el que viene en nombre del Señor} (\textit{Jn} 12, 13).

¡Amén!
\end{body}

\label{b2-04-01-1991H}


\newpage

\subsubsection{Homilía (1994): Participar de la misión de Cristo}

\src{IX Jornada Mundial de la Juventud. \\27 de marzo de 1994.}

\begin{body}
1. \textquote{Gritarán las piedras} (\textit{Lc} 19, 40).

\ltr{V}{osotros,} los jóvenes, sabéis que las piedras gritan. Son mudas, pero tienen una elocuencia particular, su grito. Cualquiera que se encuentre en las cumbres de los montes, por ejemplo en las de los Alpes o el Himalaya, lo percibe. La elocuencia, el grito de esos imponentes macizos es emocionante y hace que el hombre caiga de rodillas, lo impulsa a volver a entrar en sí mismo y a dirigirse al Creador invisible. Esas piedras mudas hablan. Vosotros, los jóvenes, lo sabéis mejor que los demás, porque exploráis su misteriosa elocuencia realizando excursiones a las montañas más altas, a fin de realizar un esfuerzo que os sirva para emplear vuestras energías jóvenes.

Vosotros lo sabéis y por eso Cristo dice de vosotros: \textquote{Si éstos callan, gritarán las piedras} (\textit{Lc} 19, 40). Lo dice en el momento de su entrada mesiánica en Jerusalén, mientras algunos fariseos trataban de hacer que callara a esos jóvenes que gritaban: \textquote{¡Hosanna! ¡Bendito el que viene en nombre del Señor!} (\textit{Mc} 11, 9). Cristo respondió: \textquote{Si éstos callan, gritarán las piedras}. Con esas palabras, amadísimos jóvenes, Jesús os ha lanzado un desafío. Y vosotros lo habéis aceptado. Se trata de un desafío que se renueva, desde hace diez años, con ocasión del domingo de Ramos, en el que vosotros, los jóvenes, os reunís en la plaza de San Pedro para repetir: \textquote{¡Hosanna! ¡Bendito el que viene en nombre del Señor!}.

\txtsmall{[Nuestro encuentro de 1984, en esta misma plaza, suscitó la idea de la Jornada mundial de la juventud. Hoy, por décima vez, esa idea se hace realidad. Este año habéis llegado aquí también vosotros, amigos americanos, desde Denver, para traer la cruz peregrina y entregarla a vuestros coetáneos de Filipinas, donde, Dios mediante, en enero del año próximo, se celebrará el nuevo encuentro mundial de los jóvenes: Manila 1995.]}

2. \textquote{Gritarán las piedras}. La piedra encierra una gran energía. En ella se manifiestan las fuerzas de la naturaleza, que elevan la corteza terrestre, formando cadenas de altas montañas. La piedra puede constituir una fuerza amenazadora. Pero, además de las rocas de las montañas, en las que se revela el misterio de la creación, hay también piedras que sirven al hombre para las obras de su talento. Basta pensar en todos los templos del mundo, en las catedrales góticas, en las obras del Renacimiento, \txtsmall{[como esta basílica de San Pedro,]} o en ciertos edificios sagrados del lejano Oriente.

Hoy, sin embargo, os invito a visitar espiritualmente un templo específico: el templo del Dios de la alianza en Jerusalén. De él sólo ha quedado un pequeño fragmento, llamado Muro de las Lamentaciones, porque junto a sus piedras se reúnen los hijos de Israel, recordando la grandeza del antiguo santuario, en el que Dios habitó y que fue objeto de un sano orgullo por parte de todo Israel. Fue arrasado en el año 70 después de Cristo. Por eso, hoy, ese Muro de las Lamentaciones es tan elocuente para los hijos de Israel, y también para nosotros, porque sabemos que en ese templo Dios estableció realmente su morada, y el espacio vacío del Santo de los santos guardaba en su interior las tablas del Decálogo, que el Señor confió a Moisés en el Sinaí. Ese lugar santísimo estaba separado del resto del templo por un velo, que en el momento de la muerte de Cristo se rasgó de arriba abajo: signo conmovedor de la presencia del Dios de la alianza en medio de su pueblo.

Así pues, subamos a Jerusalén, donde el Hijo del hombre será entregado a la muerte y crucificado, para resucitar al tercer día. La fiesta de hoy, domingo de Ramos, nos recuerda y hace presente la entrada de Jesús en Jerusalén, cuando los hijos e hijas de Israel proclamaron la gloria de Dios, saludando \textquote{al que viene en nombre del Señor}: \textquote{¡Hosanna al Hijo de David!}.

3. \textquote{Si éstos callan, gritarán las piedras}. En realidad, los jóvenes no callan. Contemplamos con asombro cómo gritan. No dejan que hablen sólo las piedras; no permiten que los templos del Dios vivo se conviertan en frías piezas de museo. Hablan a voz en grito. Hablan en los diversos lugares de la tierra, y su voz se ha de oír. Así sucede que, gracias a su testimonio, los jóvenes discípulos de Jesús son para muchos una sorpresa.

\txtsmall{[Eso aconteció precisamente el año pasado en Denver, Colorado, donde, con ocasión de una reunión tan numerosa de jóvenes de todo el mundo, se preveían excesos juveniles, o incluso casos de violencia y atropello, con lo que se hubiera dado más bien un antitestimonio. Se calculaba que eso iba a suceder, y por eso se tomaron las debidas precauciones. Para vosotros, queridos amigos, fue un desafío. Y lo aceptasteis y respondisteis con vuestro testimonio. Un testimonio vivo, con el que habéis destruido los tópicos según los cuales se os quería ver y juzgar. Habéis manifestado lo que de verdad sois y deseáis. Y vuestra voz ha resonado en la metrópoli americana que está al pie de las Montañas Rocosas, de forma que tanto las cumbres de esas montañas como las gigantescas construcciones modernas debieron de asombrarse al oíros y veros como sois de verdad.]}

4. {[Por eso, amadísimos jóvenes no os sorprenda que, después de las experiencias de Buenos Aires, Santiago de Compostela, Jasna Góra y Denver,]} hoy quiera hablaros con el mensaje que Cristo dejó a los Apóstoles en su misterio pascual. Estamos entrando en la Semana Santa. Iremos a Jerusalén, al cenáculo del Jueves Santo; subiremos al Gólgota; nos detendremos ante el sepulcro, en el silencio de la Vigilia pascual; y luego volveremos de nuevo al cenáculo para encontrarnos con el Resucitado, que nos repetirá lo que dijo a los Apóstoles, alegres por su presencia: \textquote{Como el Padre me envió, también yo os envío} (\textit{Jn} 20, 21).

\textquote{Los discípulos se alegraron de ver al Señor} (\textit{Jn} 20, 20), escribe el evangelista Juan. También vosotros os alegraréis viéndolo entre vosotros vivo, vencedor de la muerte, que no pudo triunfar sobre él. Os alegraréis oyendo las palabras que os dirigirá. Os alegraréis porque se fía de vosotros, porque tiene tanta confianza en vosotros que os dice, por medio de vuestros pastores: \textquote{Como el Padre me envió, también yo os envío}. Vosotros esperáis que os envíe, que os confíe su Evangelio, que os encomiende la salvación del mundo. Vuestros corazones jóvenes esperan oír del Redentor precisamente esas palabras.

El hombre debe tener la conciencia de ser enviado. \txtsmall{[Así lo dije el jueves pasado a los jóvenes de Roma.]} Sin esa conciencia, la vida humana se hace gris y polvorienta. Ser enviado quiere decir tener una tarea por desempeñar, una tarea comprometedora. Ser enviado quiere decir abrir los caminos a un bien grande, esperado por todos. Ser enviado quiere decir estar al servicio de una causa suprema.

Vosotros, [los jóvenes,] esperáis precisamente eso. Cristo desea encontrarse con vosotros y comprometeros en la gran misión que el Padre le confió. Es una misión que sigue viva y actual en el mundo, pero aún incompleta, siempre por realizar hasta el último día.

\textquote{Ven conmigo a salvar al mundo, ya estamos en el siglo veinte}: así cantaban en Polonia los jóvenes, en los tiempos tan difíciles de la lucha por la verdad y la vida, que es Cristo, y por el camino que él señala (cf. \textit{Jn} 14, 6). \txtsmall{[Hoy, mientras este siglo veinte se acerca a su fin, debemos pensar en el futuro, en el siglo veintiuno, en el tercer milenio. Este futuro os pertenece a vosotros. El futuro os pertenece. Sois los hombres y las mujeres del mañana.]} Cristo es \textquote{el mismo ayer, hoy y siempre} (\textit{Hb} 13, 8). Decid a todos vuestros coetáneos que él los espera y que únicamente él tiene palabras de vida eterna (cf. \textit{Jn} 6, 68). Decidlo a todos vuestros coetáneos.

Amén.
\end{body}

\label{b2-04-01-1994H}
\newpage

\subsubsection{Homilía (1997): Rey manso y humilde}

\src{XII Jornada Mundial de la Juventud. \\23 marzo de 1997.}

\begin{body}
1. \textquote{¡Bendito el que viene en nombre del Señor! (\ldots). ¡Hosanna en el cielo!} (\textit{Mc} 11, 9-10). 

\ltr{E}{stas} aclamaciones de la multitud, reunida para la fiesta de Pascua en Jerusalén, acompañan la entrada de Cristo y de los Apóstoles en la ciudad santa. Jesús entra en Jerusalén montado en un borrico, según las palabras del profeta: \textquote{Decid a la hija de Sión: Mira a tu rey que viene a ti, humilde, montado en un asno, en un pollino, hijo de animal de carga} (\textit{Mt} 21, 5).

El animal elegido indica que no se trata de una entrada triunfal, sino de la de un rey manso y humilde de corazón. Sin embargo, las multitudes reunidas en Jerusalén, casi sin notar esta expresión de humildad, o quizá reconociendo en ella un signo mesiánico, saludan a Cristo con palabras llenas de emoción: \textquote{¡Hosanna al Hijo de David! ¡Bendito el que viene en nombre del Señor! ¡Hosanna en las alturas!} (\textit{Mt} 21, 9). Y cuando Jesús entra en Jerusalén, toda la ciudad esta alborotada. La gente se pregunta: \textquote{¿Quién es éste?}. Y algunos responden: \textquote{Es Jesús, el profeta de Nazaret de Galilea} (\textit{Mt} 21, 10-11).

No era la primera vez que la gente reconocía en Cristo al rey esperado. Ya había sucedido después de la multiplicación milagrosa del pan, cuando la multitud quería aclamarlo triunfalmente. Pero Jesús sabía que su reino no era de este mundo; por eso se había alejado de ese entusiasmo. Ahora se encamina hacia Jerusalén para afrontar la prueba que le espera. Es consciente de que va allí por última vez, para una semana \textquote{santa}, al final de la cual afrontará la pasión, la cruz y la muerte. Sale al encuentro de todo esto con plena disponibilidad, sabiendo que así se cumple en él el designio eterno del Padre. Desde ese día, la Iglesia extendida por toda la tierra repite las palabras de la multitud de Jerusalén: \textquote{¡Bendito el que viene en nombre del Señor!}. Las repite cada día al celebrar la Eucaristía, poco antes de la consagración. Las repite con particular énfasis hoy, domingo de Ramos.

2. Las \textbf{lecturas litúrgicas} nos presentan al Mesías que sufre. Se refieren, ante todo, a sus padecimientos y a su humillación. La Iglesia proclama el \textbf{evangelio} de la pasión del Señor según uno de los sinópticos: el apóstol Pablo, en cambio, en la \textbf{carta a los Filipenses} nos ofrece una síntesis admirable del misterio de Cristo, quien, \textquote{a pesar de su condición divina, no hizo alarde de su categoría de Dios; al contrario, se despojo de su rango, y tomó la condición de esclavo (\ldots). Por eso Dios lo levanto sobre todo y le concedió el nombre que está sobre todo nombre; de modo que, al nombre de Jesús (\ldots), toda lengua proclame: ¡Jesucristo es Señor!, para gloria de Dios Padre} (\textit{Flp} 2, 6-11). Este himno de inestimable valor teológico presenta una síntesis completa de la Semana santa, desde el domingo de Ramos, pasando por el Viernes santo, hasta el domingo de Resurrección. Las palabras de la carta a los \textbf{Filipenses}, citadas de modo progresivo en un antiguo responsorio, nos acompañaran durante todo el Triduo sacro.

El \textbf{texto paulino} encierra en sí el anuncio de la resurrección y de la gloria, pero la liturgia de la Palabra del domingo de Ramos se concentra ante todo en la pasión. Tanto la \textbf{primera lectura} como el \textbf{Salmo responsorial} hablan de ella. En el texto, que forma parte de los llamados \textquote{cantos del Siervo de Yahveh}, se esboza el momento de la flagelación y la coronación de espinas; en el Salmo se describe, con impresionante realismo, la dolorosa agonía de Cristo en la cruz: \textquote{Dios mío, Dios mío, ¿por qué me has abandonado?} (\textit{Sal} 21, 2).

Estas palabras, las más conmovedoras, las más emotivas, qué pronuncio Jesús desde la cruz en la hora de la agonía, resuenan hoy como una antítesis evidente, expresada en voz alta de aquel \textquote{Hosanna}, que también resuena durante la procesión de los ramos.

3. \txtsmall{[Desde hace algunos años, el domingo de Ramos se ha convertido en la gran Jornada mundial de la juventud. Fueron los jóvenes mismos los que abrieron ese camino: desde el comienzo de mi ministerio en la Iglesia de Roma, en este día miles y miles de jóvenes se reunían en la plaza de San Pedro. A partir de ese hecho, a lo largo de los años se han desarrollado las Jornadas mundiales de la juventud, que se celebran en toda la Iglesia, en las parroquias y diócesis y, cada dos años, en un lugar elegido para todo el mundo. Desde 1984, los encuentros mundiales han tenido lugar sucesivamente, cada dos años: en Roma, en Buenos Aires (Argentina), en Santiago de Compostela (España), en Czestochowa-Jasna Góra (Polonia), en Denver (Estados Unidos), y en Manila (Filipinas). El próximo mes de agosto la cita es en París (Francia). Por esta razón, el año pasado, durante la celebración del domingo de Ramos, los representantes de los jóvenes de Filipinas entregaron a sus coetáneos franceses la cruz peregrinante de la \textquote{Jornada mundial de la juventud}. Este gesto tiene una elocuencia particular: es casi un redescubrimiento del significado del domingo de Ramos por parte de los jóvenes que son, efectivamente, sus protagonistas. La liturgia recuerda que \textquote{\textit{pueri hebraeorum, portantes ramos olivarum} \ldots}, \textquote{los niños hebreos, llevando ramos de olivo, salieron al encuentro del Señor, aclamando: ¡Hosanna al Hijo de David!} (Antífona).]}

\txtsmall{[Se puede decir que la primera \textquote{Jornada mundial de la juventud} fue precisamente la de Jerusalén, cuando Cristo entró en la ciudad santa; año tras año recordamos ese acontecimiento. El lugar de los \textquote{\textit{pueri hebraeorum}} ha sido ocupado por jóvenes de diversas lenguas y razas. Todos, como sus predecesores en Tierra Santa, desean acompañar a Cristo y participar en su semana de pasión, en su Triduo sacro, en su cruz y en su resurrección. Saben que él es el \textquote{bendito} que \textquote{viene en nombre del Señor}, trayendo la paz a la tierra y la gloria en las alturas. Lo que cantaron los ángeles la noche de Navidad sobre la cueva de Belén, resuena hoy con un gran eco en el umbral de la Semana santa, en la que Jesús se dispone a cumplir su misión mesiánica, realizando la redención del mundo mediante la cruz y la resurrección.]}

¡Gloria a ti, oh Cristo, Redentor del mundo! ¡Hosanna!
\end{body}

\subsubsection{Homilía (2000): Humillación y exaltación}

\src{XV Jornada Mundial de la Juventud. 16 de abril del 2000.}

\begin{body}
\ltr[1. «]{B}{\textit{enedictus,}} \textit{qui venit in nomine Domini} –Bendito el que viene en nombre del Señor» (\textit{Mt} 21, 9; cf. \textit{Sal} 118, 26).  Al escuchar estas palabras, llega hasta nosotros el eco del entusiasmo con el que los habitantes de Jerusalén acogieron a Jesús para la fiesta de la Pascua. Las volvemos a escuchar cada vez que durante la misa cantamos el Sanctus. Después de decir: \textquote{\textit{Pleni sunt coeli et terra gloria tua}}, añadimos: \textquote{\textit{Benedictus qui venit in nomine Domini. Hosanna in excelsis}}. En este himno, cuya primera parte está tomada del profeta Isaías (cf. \textit{Is} 6, 3), se exalta a Dios \textquote{tres veces santo}. Se prosigue, luego, en la segunda, expresando la alegría y la acción de gracias de la asamblea por el cumplimiento de las promesas mesiánicas: \textquote{Bendito el que viene en nombre del Señor. ¡Hosanna en el cielo!}.

Nuestro pensamiento va, naturalmente, al pueblo de la Alianza, que, durante siglos y generaciones, vivió a la espera del Mesías. Algunos creyeron ver en Juan Bautista a aquel en quien se cumplían las promesas. Pero, como sabemos, a la pregunta explícita sobre su posible identidad mesiánica, el Precursor respondió con una clara negación, remitiendo a Jesús a cuantos le preguntaban. El convencimiento de que los tiempos mesiánicos ya habían llegado fue creciendo en el pueblo, primero por el testimonio del Bautista y después gracias a las palabras y a los signos realizados por Jesús y, de modo especial, a causa de la resurrección de Lázaro, que se produjo algunos días antes de la entrada en Jerusalén, de la que habla el \textbf{evangelio de hoy}. Por eso la muchedumbre, cuando Jesús llega a la ciudad montado en un asno, lo acoge con una explosión de alegría: \textquote{Bendito el que viene en nombre del Señor. ¡Hosanna en el cielo!} (\textit{Mt} 21, 9).

2. Los ritos del domingo de Ramos reflejan el júbilo del pueblo que espera al Mesías, pero, al mismo tiempo, se caracterizan como liturgia \textquote{de pasión} en sentido pleno. En efecto, nos abren la perspectiva del drama ya inminente, que acabamos de revivir en la narración del \textbf{evangelista san Marcos}. También las otras lecturas nos introducen en el misterio de la pasión y muerte del Señor. Las palabras del \textbf{profeta Isaías}, a quien algunos consideran casi como un evangelista de la antigua Alianza, nos presentan la imagen de un condenado flagelado y abofeteado (cf. \textit{Is} 50, 6). 

El estribillo del \textbf{Salmo responsorial}: \textquote{Dios mío, Dios mío, ¿por qué me has abandonado?}, nos permite contemplar la agonía de Jesús en la cruz (cf. \textit{Mc} 15, 34). Sin embargo, el apóstol \textbf{san Pablo}, en la \textbf{segunda lectura}, nos introduce en el análisis más profundo del misterio pascual: Jesús, \textquote{a pesar de su condición divina, no hizo alarde de su categoría de Dios; al contrario, se despojó de su rango, y tomó la condición de esclavo, pasando por uno de tantos. Y así, actuando como un hombre cualquiera, se rebajó hasta someterse incluso a la muerte, y una muerte de cruz} (\textit{Flp} 2, 6-8). En la austera liturgia del Viernes santo volveremos a escuchar estas palabras, que prosiguen así: \textquote{Por eso Dios lo exaltó sobre todo, y le concedió el nombre que está sobre todo nombre; de modo que al nombre de Jesús toda rodilla se doble en el cielo, en la tierra y en el abismo, y toda lengua proclame: ¡Jesucristo es Señor!, para gloria de Dios Padre} (\textit{Flp} 2, 9-11). La humillación y la exaltación: esta es la clave para comprender el misterio pascual; ésta es la clave para penetrar en la admirable economía de Dios, que se realiza en los acontecimientos de la Pascua.

3. \txtsmall{[¿Por qué, como todos los años, están presentes numerosos jóvenes en esta solemne liturgia? En efecto, desde hace algunos años, el domingo de Ramos se ha convertido en la fiesta anual de los jóvenes. Aquí, en 1984 (\ldots) comenzó la peregrinación de las Jornadas mundiales de la juventud (\ldots) Así pues, ¿por qué tantos jóvenes se dan cita para el domingo de Ramos aquí en Roma y en todas las diócesis? Ciertamente, son muchas las razones y las circunstancias que pueden explicar este hecho. Sin embargo, al parecer, la motivación más profunda, que subyace en todas las otras, se puede identificar en lo que nos revela la liturgia de hoy: el misterioso plan de salvación del Padre celestial, que se realiza en la humillación y en la exaltación de su Hijo unigénito, Jesucristo. Esta es la respuesta a los interrogantes y a las inquietudes fundamentales de todo hombre y de toda mujer y, especialmente, de los jóvenes.]}

\textquote{Por nosotros Cristo se hizo obediente hasta la muerte, y una muerte de cruz. Por eso Dios lo exaltó}. ¡Qué cercanas a nuestra existencia están estas palabras! Vosotros, \txtsmall{[queridos jóvenes, comenzáis a experimentar el carácter dramático de la vida.]} Y os interrogáis sobre el sentido de la existencia, sobre vuestra relación con vosotros mismos, con los demás y con Dios. A vuestro corazón sediento de verdad y paz, a vuestros numerosos interrogantes y problemas, a veces incluso llenos de angustia, Cristo, Siervo sufriente y humillado, que se abajó hasta la muerte de cruz y fue exaltado en la gloria a la diestra del Padre, se ofrece a sí mismo como única respuesta válida. De hecho, no existe ninguna otra respuesta tan sencilla, completa y convincente.

4. \txtsmall{(\ldots)} Cristo, con su entrada en Jerusalén, comienza el camino de amor y de dolor de la cruz. Contempladlo con renovado impulso de fe. ¡Seguidlo! Él no promete una felicidad ilusoria; al contrario, para que logréis la auténtica madurez humana y espiritual, os invita a seguir su ejemplo exigente, haciendo vuestras sus comprometedoras elecciones. María, la fiel discípula del Señor, os acompañe en este itinerario de conversión y progresiva intimidad con su Hijo divino, quien, \txtsmall{[como recuerda el tema de la próxima Jornada mundial de la juventud,]} \textquote{se hizo carne y habitó entre nosotros} (\textit{Jn} 1, 14). Jesús se hizo pobre para enriquecernos con su pobreza, y cargó con nuestras culpas para redimirnos con su sangre derramada en la cruz. Sí, por nosotros Cristo se hizo obediente hasta la muerte, y una muerte de cruz.

\textquote{¡Gloria y alabanza a ti, oh Cristo!}.
\end{body}


\subsubsection{Homilía (2003): Rey de verdad y de justicia}

\src{XVIII Jornada mundial de la juventud. \\13 de abril de 2003.}

\begin{body}
\ltr[1. «]{B}{endito} el que viene en nombre del Señor» (\textit{Mc} 11, 9). La liturgia del domingo de Ramos es casi un solemne pórtico de ingreso en la Semana santa. Asocia dos momentos opuestos entre sí: la acogida de Jesús en Jerusalén y el drama de la Pasión; el \textquote{Hosanna} festivo y el grito repetido muchas veces: \textquote{¡Crucifícalo!}; la entrada triunfal y la aparente derrota de la muerte en la cruz. Así, anticipa la \textquote{hora} en la que el Mesías deberá sufrir mucho, lo matarán y resucitará al tercer día (cf. \textit{Mt} 16, 21), y nos prepara para vivir con plenitud el misterio pascual.

2. \textquote{Alégrate, hija de Sión; (\ldots) mira a tu rey que viene a ti} (\textit{Zc} 9, 9). Al acoger a Jesús, se alegra la ciudad en la que se conserva el recuerdo de David; la ciudad de los profetas, muchos de los cuales sufrieron allí el martirio por la verdad; la ciudad de la paz, que a lo largo de los siglos ha conocido violencia, guerra y deportación.

En cierto modo, Jerusalén puede considerarse la ciudad símbolo de la humanidad, \txtsmall{[especialmente en el dramático inicio del tercer milenio que estamos viviendo.]} Por eso, los ritos del domingo de Ramos cobran una elocuencia particular. Resuenan consoladoras las palabras del profeta Zacarías: \textquote{Alégrate, hija de Sión; canta, hija de Jerusalén; mira a tu rey que viene a ti justo y victorioso, modesto y cabalgando en un asno. (\ldots) Romperá los arcos guerreros, dictará la paz a las naciones} (\textit{Zc} 9, 9-10). Hoy estamos de fiesta, porque entra en Jerusalén Jesús, el Rey de la paz.

3. En aquel tiempo, a lo largo de la bajada del monte de los Olivos, fueron al encuentro de Cristo los niños y los jóvenes de Jerusalén, aclamando y agitando con júbilo ramos de olivo y de palmas. 

\txtsmall{[Hoy lo acogen los jóvenes del mundo entero, que en cada comunidad diocesana celebran la XVIII Jornada mundial de la juventud. Os saludo con gran afecto, queridos jóvenes (\ldots) ¿Y cómo no expresar solidaridad fraterna a vuestros coetáneos probados por la guerra y la violencia en Irak, en Tierra Santa y en muchas otras regiones del mundo?]}

Hoy acogemos con fe y con júbilo a Cristo, que es nuestro \textquote{rey}: rey de verdad, de libertad, de justicia y de amor. Estos son los cuatro \textquote{pilares} sobre los que es posible construir el edificio de la verdadera paz, como escribió hace cuarenta años en la encíclica \textit{Pacem in terris} el beato Papa Juan XXIII. \txtsmall{[A vosotros, jóvenes del mundo entero, os entrego idealmente este histórico documento, plenamente actual: leedlo, meditadlo y esforzaos por ponerlo en práctica. Así seréis \textquote{bienaventurados}, por ser auténticos hijos del Dios de la paz (cf. \textit{Mt} 5, 9).]}

\newpage
4. La paz es don de Cristo, que nos lo obtuvo con el sacrificio de la cruz. Para conseguirla eficazmente, es necesario subir con el divino Maestro hasta el Calvario. Y en esta subida, ¿quién puede guiarnos mejor que María, que precisamente al pie de la cruz nos fue dada como madre en el apóstol fiel, san Juan? \txtsmall{[Para ayudar a los jóvenes a descubrir esta maravillosa realidad espiritual, elegí como tema del \textit{Mensaje} para la Jornada mundial de la juventud de este año]} las palabras de Cristo moribundo: \textquote{He ahí a tu Madre} (\textit{Jn} 19, 27). Aceptando este testamento de amor, Juan acogió a María en su casa (cf. \textit{Jn} 19, 27), es decir, la acogió en su vida, compartiendo con ella una cercanía espiritual completamente nueva. El vínculo íntimo con la Madre del Señor llevará al \textquote{discípulo amado} a convertirse en el apóstol del Amor que él había tomado del Corazón de Cristo a través del Corazón inmaculado de María.

5. \textquote{He ahí a tu Madre}. Jesús os dirige estas palabras a cada uno de vosotros, queridos amigos. También a vosotros os pide que acojáis a María como madre \textquote{en vuestra casa}, que la recibáis \textquote{entre vuestros bienes}, porque \textquote{ella, desempeñando su ministerio materno, os educa y os modela hasta que Cristo sea formado plenamente en vosotros} (\textit{Mensaje}, 3). María os lleve a responder generosamente a la llamada del Señor y a perseverar con alegría y fidelidad en la misión cristiana.

\txtsmall{[A lo largo de los siglos, ¡cuántos jóvenes han aceptado esta invitación y cuántos siguen haciéndolo también en nuestro tiempo! Jóvenes del tercer milenio, ¡no tengáis miedo de ofrecer vuestra vida como respuesta total a Cristo! Él, sólo él cambia la vida y la historia del mundo.]}

6. \textquote{Realmente, este hombre era el Hijo de Dios} (\textit{Mc} 15, 39). Hemos vuelto a escuchar la clara profesión de fe del centurión, \textquote{al ver cómo había expirado} (\textit{Mc} 15, 39). De cuanto vio brota el sorprendente testimonio del soldado romano, el primero en proclamar que ese hombre \textquote{era el Hijo de Dios}.


\newpage
\begin{bodyprose}
Señor Jesús,
   también nosotros hemos \textquote{visto}
   cómo has padecido
   y cómo has muerto por nosotros.

Fiel hasta el extremo,
   nos has arrancado de la muerte
   con tu muerte.

Con tu cruz nos ha redimido.

Tú, María, Madre dolorosa,
   eres testigo silenciosa
   de aquellos instantes decisivos
   para la historia de la salvación.

Danos tus ojos para reconocer
   en el rostro del Crucificado,
   desfigurado por el dolor,
   la imagen del Resucitado glorioso.

Ayúdanos a abrazarlo
   y a confiar en él,
   para que seamos dignos de sus promesas.

Ayúdanos a serle fieles hoy
   y durante toda nuestra vida. 

Amén.
\end{bodyprose}
\end{body}


\newsection
\subsection{Benedicto XVI, papa}

\subsubsection{Homilía (2006): Montado en un asno prestado}

\src{XXI Jornada Mundial de la Juventud. 9 de abril del 2006.}

\begin{body}
\ltr[\ldots ]{P}{ara} entender lo que sucedió el domingo de Ramos y saber qué significa, no sólo para aquella hora, sino para toda época, es importante un detalle, que también para sus discípulos se transformó en la clave para la comprensión del acontecimiento, cuando, después de la Pascua, repasaron con una mirada nueva aquellas jornadas agitadas. Jesús entra en la ciudad santa montado en un asno, es decir, en el animal de la gente sencilla y común del campo, y además un asno que no le pertenece, sino que pide prestado para esta ocasión. No llega en una suntuosa carroza real, ni a caballo, como los grandes del mundo, sino en un asno prestado. San Juan nos relata que, en un primer momento, los discípulos no lo entendieron. Sólo después de la Pascua cayeron en la cuenta de que Jesús, al actuar así, cumplía los anuncios de los profetas, que su actuación derivaba de la palabra de Dios y la realizaba. Recordaron –dice san Juan– que en el profeta Zacarías se lee: \textquote{No temas, hija de Sión; mira que viene tu Rey montado en un pollino de asna} (\textit{Jn} 12, 15; cf. \textit{Za} 9, 9). Para comprender el significado de la profecía y, en consecuencia, de la misma actuación de Jesús, debemos escuchar todo el texto de Zacarías, que prosigue así: \textquote{El destruirá los carros de Efraím y los caballos de Jerusalén; romperá el arco de combate, y él proclamará la paz a las naciones. Su dominio irá de mar a mar y desde el río hasta los confines de la tierra} (\textit{Za} 9, 10). Así afirma el profeta tres cosas sobre el futuro rey.

En primer lugar, dice que será rey de los pobres, pobre entre los pobres y para los pobres. La pobreza, en este caso, se entiende en el sentido de los \textit{anawin} de Israel, de las almas creyentes y humildes que encontramos en torno a Jesús, en la perspectiva de la primera bienaventuranza del Sermón de la montaña. Uno puede ser materialmente pobre, pero tener el corazón lleno de afán de riqueza material y del poder que deriva de la riqueza. Precisamente el hecho de que vive en la envidia y en la codicia demuestra que, en su corazón, pertenece a los ricos. Desea cambiar la repartición de los bienes, pero para llegar a estar él mismo en la situación de los ricos de antes. La pobreza, en el sentido que le da Jesús –el sentido de los profetas–, presupone sobre todo estar libres interiormente de la avidez de posesión y del afán de poder. Se trata de una realidad mayor que una simple repartición diferente de los bienes, que se limitaría al campo material y más bien endurecería los corazones. Ante todo, se trata de la purificación del corazón, gracias a la cual se reconoce la posesión como responsabilidad, como tarea con respecto a los demás, poniéndose bajo la mirada de Dios y dejándose guiar por Cristo que, siendo rico, se hizo pobre por nosotros (cf. \textit{2 Co} 8, 9). La libertad interior es el presupuesto para superar la corrupción y la avidez que arruinan al mundo; esta libertad sólo puede hallarse si Dios llega a ser nuestra riqueza; sólo puede hallarse en la paciencia de las renuncias diarias, en las que se desarrolla como libertad verdadera. Al rey que nos indica el camino hacia esta meta –Jesús– lo aclamamos el domingo de Ramos; le pedimos que nos lleve consigo por su camino.

En segundo lugar, el profeta nos muestra que este rey será un rey de paz; hará desaparecer los carros de guerra y los caballos de batalla, romperá los arcos y anunciará la paz. En la figura de Jesús esto se hace realidad mediante el signo de la cruz. Es el arco roto, en cierto modo, el nuevo y verdadero arco iris de Dios, que une el cielo y la tierra y tiende un puente entre los continentes sobre los abismos. La nueva arma, que Jesús pone en nuestras manos, es la cruz, signo de reconciliación, de perdón, signo del amor que es más fuerte que la muerte. Cada vez que hacemos la señal de la cruz debemos acordarnos de no responder a la injusticia con otra injusticia, a la violencia con otra violencia; debemos recordar que sólo podemos vencer al mal con el bien, y jamás devolviendo mal por mal.

La tercera afirmación del profeta es el anuncio de la universalidad. Zacarías dice que el reino del rey de la paz se extiende \textquote{de mar a mar (\ldots) hasta los confines de la tierra}. La antigua promesa de la tierra, hecha a Abraham y a los Padres, se sustituye aquí con una nueva visión: el espacio del rey mesiánico ya no es un país determinado, que luego se separaría de los demás y, por tanto, se pondría inevitablemente contra los otros países. Su país es la tierra, el mundo entero. Superando toda delimitación, él crea unidad en la multiplicidad de las culturas. Atravesando con la mirada las nubes de la historia que separaban al profeta de Jesús, vemos cómo desde lejos emerge en esta profecía la red de las comunidades eucarísticas que abraza a la tierra, a todo el mundo, una red de comunidades que constituyen el \textquote{reino de la paz} de Jesús de mar a mar hasta los confines de la tierra.

Él llega a todas las culturas y a todas las partes del mundo, adondequiera, a las chozas miserables y a los campos pobres, así como al esplendor de las catedrales. Por doquier él es el mismo, el Único, y así todos los orantes reunidos, en comunión con él, están también unidos entre sí en un único cuerpo. Cristo domina convirtiéndose él mismo en nuestro pan y entregándose a nosotros. De este modo construye su reino.

Este nexo resulta totalmente claro en la otra frase del Antiguo Testamento que caracteriza y explica la liturgia del domingo de Ramos y su clima particular. La multitud aclama a Jesús: \textquote{Hosanna, bendito el que viene en nombre del Señor} (\textit{Mc} 11, 9; \textit{Sal} 118, 25). Estas palabras forman parte del rito de la fiesta de las tiendas, durante el cual los fieles dan vueltas en torno al altar llevando en las manos ramos de palma, mirto y sauce.

Ahora la gente grita eso mismo, con palmas en las manos, delante de Jesús, en quien ve a Aquel que viene en nombre del Señor. En efecto, la expresión \textquote{el que viene en nombre del Señor} se había convertido desde hacía tiempo en la manera de designar al Mesías. En Jesús reconocen a Aquel que verdaderamente viene en nombre del Señor y les trae la presencia de Dios. Este grito de esperanza de Israel, esta aclamación a Jesús durante su entrada en Jerusalén, ha llegado a ser con razón en la Iglesia la aclamación a Aquel que, en la Eucaristía, viene a nuestro encuentro de un modo nuevo. Con el grito \textquote{Hosanna} saludamos a Aquel que, en carne y sangre, trajo la gloria de Dios a la tierra. Saludamos a Aquel que vino y, sin embargo, sigue siendo siempre Aquel que debe venir. Saludamos a Aquel que en la Eucaristía viene siempre de nuevo a nosotros en nombre del Señor, uniendo así en la paz de Dios los confines de la tierra. Esta experiencia de la universalidad forma parte esencial de la Eucaristía. Dado que el Señor viene, nosotros salimos de nuestros particularismos exclusivos y entramos en la gran comunidad de todos los que celebran este santo sacramento. Entramos en su reino de paz y, en cierto modo, saludamos en él también a todos nuestros hermanos y hermanas a quienes él viene, para llegar a ser verdaderamente un reino de paz en este mundo desgarrado.

Las tres características anunciadas por el profeta –pobreza, paz y universalidad– se resumen en el signo de la cruz. Por eso, con razón, la cruz se ha convertido en el centro de las Jornadas mundiales de la juventud. Hubo un período –que aún no se ha superado del todo– en el que se rechazaba el cristianismo precisamente a causa de la cruz. La cruz habla de sacrificio –se decía–; la cruz es signo de negación de la vida. En cambio, nosotros queremos la vida entera, sin restricciones y sin renuncias. Queremos vivir, sólo vivir. No nos dejamos limitar por mandamientos y prohibiciones; queremos riqueza y plenitud; así se decía y se sigue diciendo todavía.

Todo esto parece convincente y atractivo; es el lenguaje de la serpiente, que nos dice: \textquote{¡No tengáis miedo! ¡Comed tranquilamente de todos los árboles del jardín!}. Sin embargo, el domingo de Ramos nos dice que el auténtico gran \textquote{sí} es precisamente la cruz; que precisamente la cruz es el verdadero árbol de la vida. No hallamos la vida apropiándonos de ella, sino donándola. El amor es entregarse a sí mismo, y por eso es el camino de la verdadera vida, simbolizada por la cruz. \txtsmall{[Hoy la cruz, que estuvo en el centro de la última Jornada mundial de la juventud, en Colonia, se entrega a una delegación para que comience su camino hacia Sydney, donde, en 2008, la juventud del mundo quiere reunirse nuevamente en torno a Cristo para construir con él el reino de paz. Desde Colonia hasta Sydney, un camino a través de los continentes y las culturas, un camino a través de un mundo desgarrado y atormentado por la violencia. Simbólicamente es el camino indicado por el profeta, de mar a mar, desde el río hasta los confines de la tierra. Es el camino de Aquel que, con el signo de la cruz, nos da la paz y nos transforma en portadores de la reconciliación y de su paz. Doy las gracias a los jóvenes que ahora llevarán por los caminos del mundo esta cruz, en la que casi podemos tocar el misterio de Jesús. Pidámosle que, al mismo tiempo, nos toque a nosotros y abra nuestro corazón, a fin de que siguiendo su cruz lleguemos a ser mensajeros de su amor y de su paz. Amén.]}
\end{body}

\label{b2-04-01-2006H}

\begin{patercite}  
	(\ldots) Este himno a Cristo parte de su ser \textquote{\emph{en morphe tou Theou}}, dice el texto griego, es decir, de su ser \textquote{en la forma de Dios}, o mejor, en la condición de Dios. Jesús, verdadero Dios y verdadero hombre, no vive su \textquote{ser como Dios} para triunfar o para imponer su supremacía; no lo considera una posesión, un privilegio, un tesoro que guardar celosamente. Más aún, \textquote{se despojó de sí mismo}, se vació de sí mismo asumiendo, dice el texto griego, la \textquote{\emph{morphe doulou}}, la \textquote{forma de esclavo}, la realidad humana marcada por el sufrimiento, por la pobreza, por la muerte; se hizo plenamente semejante a los hombres, excepto en el pecado, para actuar como siervo completamente entregado al servicio de los demás. Al respecto, Eusebio de Cesarea, en el siglo IV, afirma: \textquote{Tomó sobre sí mismo las pruebas de los miembros que sufren. Hizo suyas nuestras humildes enfermedades. Sufrió y padeció por nuestra causa y lo hizo por su gran amor a la humanidad} (\emph{La demostración evangélica,} 10, 1, 22). San Pablo prosigue delineando el cuadro \textquote{histórico} en el que se realizó este abajamiento de Jesús: \textquote{Se humilló a sí mismo, hecho obediente hasta la muerte} (\emph{Flp} 2, 8). El Hijo de Dios se hizo verdaderamente hombre y recorrió un camino en la completa obediencia y fidelidad a la voluntad del Padre hasta el sacrificio supremo de su vida. El Apóstol especifica más aún: \textquote{hasta la muerte, y una muerte de cruz}. En la cruz Jesucristo alcanzó el máximo grado de la humillación, porque la crucifixión era el castigo reservado a los esclavos y no a las personas libres: \textquote{\emph{mors turpissima crucis}}, escribe Cicerón (cf. \emph{In Verrem}, v, 64, 165).  
	
	En la cruz de Cristo el hombre es redimido, y se invierte la experiencia de Adán: Adán, creado a imagen y semejanza de Dios, pretendió ser como Dios con sus propias fuerzas, ocupar el lugar de Dios, y así perdió la dignidad originaria que se le había dado. Jesús, en cambio, era \textquote{de condición divina}, pero se humilló, se sumergió en la condición humana, en la fidelidad total al Padre, para redimir al Adán que hay en nosotros y devolver al hombre la dignidad que había perdido. Los Padres subrayan que se hizo obediente, restituyendo a la naturaleza humana, a través de su humanidad y su obediencia, lo que se había perdido por la desobediencia de Adán.  
	
	\textbf{Benedicto XVI}, papa, \textit{Catequesis} sobre el Cántico de Filipenses, 27 de junio del 2012, parr. 4-5. 
\end{patercite}

\newpage

\subsubsection{Homilía (2009): ¿Comprendemos su Reino?}

\src{XXIV Jornada Mundial de la Juventud. \\Plaza de San Pedro. 5 de abril de 2009.}

\begin{body}
\ltr{J}{unto} con una creciente muchedumbre de peregrinos, Jesús había subido a Jerusalén para la Pascua. En la última etapa del camino, cerca de Jericó, había curado al ciego Bartimeo, que lo había invocado como Hijo de David y suplicado piedad. Ahora que ya podía ver, se había sumado con gratitud al grupo de los peregrinos. Cuando a las puertas de Jerusalén Jesús montó en un borrico, que simbolizaba el reinado de David, entre los peregrinos explotó espontáneamente la alegre certeza: Es él, el Hijo de David. Y saludan a Jesús con la aclamación mesiánica: \textquote{¡Bendito el que viene en nombre del Señor!}; y añaden: \textquote{¡Bendito el reino que llega, el de nuestro padre David! ¡Hosanna en el cielo!} (\textit{Mc} 11, 9s). No sabemos cómo se imaginaban exactamente los peregrinos entusiastas el reino de David que llega. Pero nosotros, ¿hemos entendido realmente el mensaje de Jesús, Hijo de David? ¿Hemos entendido lo que es el Reino del que habló al ser interrogado por Pilato? ¿Comprendemos lo que quiere decir que su Reino no es de este mundo? ¿O acaso quisiéramos más bien que fuera de este mundo?

San Juan, en su Evangelio, después de narrar la entrada en Jerusalén, añade una serie de dichos de Jesús, en los que Él explica lo esencial de este nuevo género de reino. A simple vista podemos distinguir en estos textos tres imágenes diversas del reino en las que, aunque de modo diferente, se refleja el mismo misterio. Ante todo, Juan relata que, entre los peregrinos que querían \textquote{adorar a Dios} durante la fiesta, había también algunos griegos (cf. \textit{Jn }12, 20). Fijémonos en que el verdadero objetivo de estos peregrinos era adorar a Dios. Esto concuerda perfectamente con lo que Jesús dice en la purificación del Templo: \textquote{Mi casa será llamada casa de oración para todos los pueblos} (\textit{Mc} 11, 17). La verdadera meta de la peregrinación ha de ser encontrar a Dios, adorarlo, y así poner en el justo orden la relación de fondo de nuestra vida. Los griegos están en busca de Dios, con su vida están en camino hacia Dios. Ahora, mediante dos Apóstoles de lengua griega, Felipe y Andrés, hacen llegar al Señor esta petición: \textquote{Quisiéramos ver a Jesús} (\textit{Jn} 12, 21). Son palabras mayores. Queridos amigos, por eso nos hemos reunido aquí: Queremos ver a Jesús. \txtsmall{[Para eso han ido a Sydney el año pasado miles de jóvenes. Ciertamente, habrán puesto muchas ilusiones en esta peregrinación. Pero el objetivo esencial era éste: Queremos ver a Jesús.]}

¿Qué dijo, qué hizo Jesús en aquel momento ante esta petición? En el Evangelio no aparece claramente que hubiera un encuentro entre aquellos griegos y Jesús. La vista de Jesús va mucho más allá. El núcleo de su respuesta a la solicitud de aquellas personas es: \textquote{Si el grano de trigo no cae en tierra y muere, queda infecundo; pero si muere, da mucho fruto} (\textit{Jn} 12, 24). Y esto quiere decir: ahora no tiene importancia un coloquio más o menos breve con algunas personas, que después vuelven a casa. Vendré al encuentro del mundo de los griegos como grano de trigo muerto y resucitado, de manera totalmente nueva y por encima de los límites del momento. Por su resurrección, Jesús supera los límites del espacio y del tiempo. Como Resucitado, recorre la inmensidad del mundo y de la historia. Sí, como Resucitado, va a los griegos y habla con ellos, se les manifiesta, de modo que ellos, los lejanos, se convierten en cercanos y, precisamente en su lengua, en su cultura, la palabra de Jesús irá avanzando y será entendida de un modo nuevo: así viene su Reino. 

Por tanto, podemos reconocer dos características esenciales de este Reino. La primera es que este Reino pasa por la cruz. Puesto que Jesús se entrega totalmente, como Resucitado puede pertenecer a todos y hacerse presente a todos. En la sagrada Eucaristía recibimos el fruto del grano de trigo que muere, la multiplicación de los panes que continúa hasta el fin del mundo y en todos los tiempos. La segunda característica dice: su Reino es universal. Se cumple la antigua esperanza de Israel: esta realeza de David ya no conoce fronteras. Se extiende \textquote{de mar a mar}, como dice el profeta Zacarías (9, 10), es decir, abarca todo el mundo. Pero esto es posible sólo porque no es la soberanía de un poder político, sino que se basa únicamente en la libre adhesión del amor; un amor que responde al amor de Jesucristo, que se ha entregado por todos. Pienso que siempre hemos de aprender de nuevo ambas cosas. Ante todo, la universalidad, la catolicidad. Ésta significa que nadie puede considerarse a sí mismo, a su cultura a su tiempo y su mundo como absoluto. Y eso requiere que todos nos acojamos recíprocamente, renunciando a algo nuestro. La universalidad incluye el misterio de la cruz, la superación de sí mismos, la obediencia a la palabra de Jesucristo, que es común, en la común Iglesia. La universalidad es siempre una superación de sí mismos, renunciar a algo personal. La universalidad y la cruz van juntas. Sólo así se crea la paz.

La palabra sobre el grano de trigo que muere sigue formando parte de la respuesta de Jesús a los griegos, es su respuesta. Pero, a continuación, Él formula una vez más la ley fundamental de la existencia humana: \textquote{El que se ama a sí mismo, se pierde, y el que se aborrece a sí mismo en este mundo, se guardará para la vida eterna} (\textit{Jn} 12, 25). Es decir, quien quiere tener su vida para sí, vivir sólo para él mismo, tener todo en puño y explotar todas sus posibilidades, éste es precisamente quien pierde la vida. Ésta se vuelve tediosa y vacía. Solamente en el abandono de sí mismo, en la entrega desinteresada del yo en favor del tú, en el \textquote{sí} a la vida más grande, la vida de Dios, nuestra vida se ensancha y engrandece. Así, este principio fundamental que el Señor establece es, en último término, simplemente idéntico al principio del amor. En efecto, el amor significa dejarse a sí mismo, entregarse, no querer poseerse a sí mismo, sino liberarse de sí: no replegarse sobre sí mismo –¡qué será de mí!– sino mirar adelante, hacia el otro, hacia Dios y hacia los hombres que Él pone a mi lado. Y este principio del amor, que define el camino del hombre, es una vez más idéntico al misterio de la cruz, al misterio de muerte y resurrección que encontramos en Cristo. Queridos amigos, tal vez sea relativamente fácil aceptar esto como gran visión fundamental de la vida. Pero, en la realidad concreta, no se trata simplemente de reconocer un principio, sino de vivir su verdad, la verdad de la cruz y la resurrección. Y por ello, una vez más, no basta una única gran decisión. Indudablemente, es importante, esencial, lanzarse a la gran decisión fundamental, al gran \textquote{sí} que el Señor nos pide en un determinado momento de nuestra vida. Pero el gran \textquote{sí} del momento decisivo en nuestra vida –el \textquote{sí} a la verdad que el Señor nos pone delante– ha de ser después reconquistado cotidianamente en las situaciones de todos los días en las que, una y otra vez, hemos de abandonar nuestro yo, ponernos a disposición, aun cuando en el fondo quisiéramos más bien aferrarnos a nuestro yo. También el sacrificio, la renuncia, son parte de una vida recta. Quien promete una vida sin este continuo y renovado don de sí mismo, engaña a la gente. Sin sacrificio, no existe una vida lograda. Si echo una mirada retrospectiva sobre mi vida personal, tengo que decir que precisamente los momentos en que he dicho \textquote{sí} a una renuncia han sido los momentos grandes e importantes de mi vida.

Finalmente, san Juan ha recogido también en su relato de los dichos del Señor para el \textquote{Domingo de Ramos} una forma modificada de la oración de Jesús en el Huerto de los Olivos. Ante todo una afirmación: \textquote{Mi alma está agitada} (\textit{12}, 27). Aquí aparece el pavor de Jesús, ampliamente descrito por los otros tres evangelistas: su terror ante el poder de la muerte, ante todo el abismo de mal que ve, y al cual debe bajar. El Señor sufre nuestras angustias junto con nosotros, nos acompaña a través de la última angustia hasta la luz. En Juan, siguen después dos súplicas de Jesús. La primera formulada sólo de manera condicional: \textquote{¿Qué diré? Padre, líbrame de esta hora} (12, 27). Como ser humano, también Jesús se siente impulsado a rogar que se le libre del terror de la pasión. También nosotros podemos orar de este modo. También nosotros podemos lamentarnos ante el Señor, como Job, presentarle todas las peticiones que surgen en nosotros frente a la injusticia en el mundo y las trabas de nuestro propio yo. Ante Él, no hemos de refugiarnos en frases piadosas, en un mundo ficticio. Orar siempre significa luchar también con Dios y, como Jacob, podemos decirle: \textquote{no te soltaré hasta que me bendigas} (\textit{Gn} 32, 27). Pero luego viene la segunda petición de Jesús: \textquote{Glorifica tu nombre} (\textit{Jn} 12, 28). En los sinópticos, este ruego se expresa así: \textquote{No se haga mi voluntad, sino la tuya} (\textit{Lc} 22, 42). Al final, la gloria de Dios, su señorío, su voluntad, es siempre más importante y más verdadera que mi pensamiento y mi voluntad. Y esto es lo esencial en nuestra oración y en nuestra vida: aprender este orden justo de la realidad, aceptarlo íntimamente; confiar en Dios y creer que Él está haciendo lo que es justo; que su voluntad es la verdad y el amor; que mi vida se hace buena si aprendo a ajustarme a este orden. Vida, muerte y resurrección de Jesús, son para nosotros la garantía de que verdaderamente podemos fiarnos de Dios. De este modo se realiza su Reino.

\txtsmall{[Queridos amigos. Al término de esta liturgia, los jóvenes de Australia entregarán la Cruz de la Jornada Mundial de la Juventud a sus coetáneos de España.]} La Cruz está en camino de una a otra parte del mundo, de mar a mar. Y nosotros la acompañamos. Avancemos con ella por su camino y así encontraremos nuestro camino. Cuando tocamos la Cruz, más aún, cuando la llevamos, tocamos el misterio de Dios, el misterio de Jesucristo: el misterio de que Dios ha tanto amado al mundo, a nosotros, que entregó a su Hijo único por nosotros (cf. \textit{Jn} 3, 16). Toquemos el misterio maravilloso del amor de Dios, la única verdad realmente redentora. Pero hagamos nuestra también la ley fundamental, la norma constitutiva de nuestra vida, es decir, el hecho que sin el \textquote{sí} a la Cruz, sin caminar día tras día en comunión con Cristo, no se puede lograr la vida. Cuanto más renunciemos a algo por amor de la gran verdad y el gran amor –por amor de la verdad y el amor de Dios–, tanto más grande y rica se hace la vida. Quien quiere guardar su vida para sí mismo, la pierde. Quien da su vida –cotidianamente, en los pequeños gestos que forman parte de la gran decisión–, la encuentra. Esta es la verdad exigente, pero también profundamente bella y liberadora, en la que queremos entrar paso a paso durante el camino de la Cruz por los continentes. Que el Señor bendiga este camino. Amén.
\end{body}

\label{b2-04-01-2009H}
\newpage

\subsubsection{Homilía (2012): Mirar con la mirada de Cristo}

\src{XXVII Jornada Mundial de la Juventud. Plaza de San Pedro. \\1 de abril de 2012.}

\begin{body}
\ltr{E}{l} Domingo de Ramos es el gran pórtico que nos lleva a la Semana Santa, la semana en la que el Señor Jesús se dirige hacia la culminación de su vida terrena. Él va a Jerusalén para cumplir las Escrituras y para ser colgado en la cruz, el trono desde el cual reinará por los siglos, atrayendo a sí a la humanidad de todos los tiempos y ofrecer a todos el don de la redención. Sabemos por los evangelios que Jesús se había encaminado hacia Jerusalén con los doce, y que poco a poco se había ido sumando a ellos una multitud creciente de peregrinos. \textbf{San Marcos} nos dice que ya al salir de Jericó había una \textquote{gran muchedumbre} que seguía a Jesús (cf. \textit{Mc} 10, 46).

En la última parte del trayecto se produce un acontecimiento particular, que aumenta la expectativa sobre lo que está por suceder y hace que la atención se centre todavía más en Jesús. A lo largo del camino, al salir de Jericó, está sentado un mendigo ciego, llamado Bartimeo. Apenas oye decir que Jesús de Nazaret está llegando, comienza a gritar: \textquote{¡Hijo de David, Jesús, ten compasión de mí} (\textit{Mc} 10, 47). Tratan de acallarlo, pero en vano, hasta que Jesús lo manda llamar y le invita a acercarse. \textquote{¿Qué quieres que te haga?}, le pregunta. Y él contesta: \textquote{Rabbuní, que vea} (\textit{Mc} 10, 51). Jesús le dice: \textquote{Anda, tu fe te ha salvado}. Bartimeo recobró la vista y se puso a seguir a Jesús en el camino (\textit{Mc} 10, 52). Y he aquí que, tras este signo prodigioso, acompañado por aquella invocación: \textquote{Hijo de David}, un estremecimiento de esperanza atraviesa la multitud, suscitando en muchos una pregunta: ¿Este Jesús que marchaba delante de ellos a Jerusalén, no sería quizás el Mesías, el nuevo David? Y, con su ya inminente entrada en la ciudad santa, ¿no habría llegado tal vez el momento en el que Dios restauraría finalmente el reino de David?

También la preparación del ingreso de Jesús con sus discípulos contribuye a aumentar esta esperanza. Como hemos escuchado en el \textbf{Evangelio de hoy} (cf. \textit{Mc} 11, 1-10), Jesús llegó a Jerusalén desde Betfagé y el monte de los Olivos, es decir, la vía por la que había de venir el Mesías. Desde allí, envía por delante a dos discípulos, mandándoles que le trajeran un pollino de asna que encontrarían a lo largo del camino. Encuentran efectivamente el pollino, lo desatan y lo llevan a Jesús. A este punto, el ánimo de los discípulos y los otros peregrinos se deja ganar por el entusiasmo: toman sus mantos y los echan encima del pollino; otros alfombran con ellos el camino de Jesús a medida que avanza a grupas del asno. Después cortan ramas de los árboles y comienzan a gritar las palabras del Salmo 118, las antiguas palabras de bendición de los peregrinos que, en este contexto, se convierten en una proclamación mesiánica: \textquote{¡Hosanna!, bendito el que viene en el nombre del Señor. ¡Bendito el reino que llega, el de nuestro padre David! ¡Hosanna en las alturas!} (\textit{Mc} 11, 9-10). Esta alegría festiva, transmitida por los cuatro evangelistas, es un grito de bendición, un himno de júbilo: expresa la convicción unánime de que, en Jesús, Dios ha visitado su pueblo y ha llegado por fin el Mesías deseado. Y todo el mundo está allí, con creciente expectación por lo que Cristo hará una vez que entre en su ciudad.

Pero, ¿cuál es el contenido, la resonancia más profunda de este grito de júbilo? La respuesta está en toda la Escritura, que nos recuerda cómo el Mesías lleva a cumplimiento la promesa de la bendición de Dios, la promesa originaria que Dios había hecho a Abraham, el padre de todos los creyentes: \textquote{Haré de ti una gran nación, te bendeciré\ldots y en ti serán benditas todas las familias de la tierra} (\textit{Gn} 12, 2-3). Es la promesa que Israel siempre había tenido presente en la oración, especialmente en la oración de los Salmos. Por eso, el que es aclamado por la muchedumbre como bendito es al mismo tiempo aquel en el cual será bendecida toda la humanidad. Así, a la luz de Cristo, la humanidad se reconoce profundamente unida y cubierta por el manto de la bendición divina, una bendición que todo lo penetra, todo lo sostiene, lo redime, lo santifica.

Podemos descubrir aquí un primer gran mensaje que nos trae la festividad de hoy: la invitación a mirar de manera justa a la humanidad entera, a cuantos conforman el mundo, a sus diversas culturas y civilizaciones. La mirada que el creyente recibe de Cristo es una mirada de bendición: una mirada sabia y amorosa, capaz de acoger la belleza del mundo y de compartir su fragilidad. En esta mirada se transparenta la mirada misma de Dios sobre los hombres que él ama y sobre la creación, obra de sus manos. En el Libro de la Sabiduría, leemos: \textquote{Te compadeces de todos, porque todo lo puedes, cierras los ojos a los pecados de los hombres, para que se arrepientan. Amas a todos los seres y no aborreces nada de lo que hiciste\ldots Tú eres indulgente con todas las cosas, porque son tuyas, Señor, amigo de la vida} (\textit{Sb} 11, 23-24. 26).

Volvamos al texto del \textbf{Evangelio de hoy} y preguntémonos: ¿Qué late realmente en el corazón de los que aclaman a Cristo como Rey de Israel? Ciertamente tenían su idea del Mesías, una idea de cómo debía actuar el Rey prometido por los profetas y esperado por tanto tiempo. No es de extrañar que, pocos días después, la muchedumbre de Jerusalén, en vez de aclamar a Jesús, gritaran a Pilato: \textquote{¡Crucifícalo!}. Y que los mismos discípulos, como también otros que le habían visto y oído, permanecieran mudos y desconcertados. En efecto, la mayor parte estaban desilusionados por el modo en que Jesús había decidido presentarse como Mesías y Rey de Israel. Este es precisamente el núcleo de la fiesta de hoy también para nosotros. ¿Quién es para nosotros Jesús de Nazaret? ¿Qué idea tenemos del Mesías, qué idea tenemos de Dios? Esta es una cuestión crucial que no podemos eludir, sobre todo en esta semana en la que estamos llamados a seguir a nuestro Rey, que elige como trono la cruz; estamos llamados a seguir a un Mesías que no nos asegura una felicidad terrena fácil, sino la felicidad del cielo, la eterna bienaventuranza de Dios. Ahora, hemos de preguntarnos: ¿Cuáles son nuestras verdaderas expectativas? ¿Cuáles son los deseos más profundos que nos han traído hoy aquí para celebrar el Domingo de Ramos e iniciar la Semana Santa?

\txtsmall{[Queridos jóvenes que os habéis reunido aquí. Esta es de modo particular vuestra Jornada en todo lugar del mundo donde la Iglesia está presente. Por eso os saludo con gran afecto.]} Que el Domingo de Ramos sea para vosotros el día de la decisión, la decisión de acoger al Señor y de seguirlo hasta el final, la decisión de hacer de su Pascua de muerte y resurrección el sentido mismo de vuestra vida de cristianos. \txtsmall{[Como he querido recordar en el \textit{Mensaje a los jóvenes} para esta Jornada – \textit{alegraos siempre en el Señor} (\textit{Flp} 4, 4) –, esta es la decisión que conduce a la verdadera alegría, como sucedió con santa Clara de Asís que, hace ochocientos años, fascinada por el ejemplo de san Francisco y de sus primeros compañeros, dejó la casa paterna precisamente el Domingo de Ramos para consagrarse totalmente al Señor: tenía 18 años, y tuvo el valor de la fe y del amor de optar por Cristo, encontrando en él la alegría y la paz.]}

Queridos hermanos y hermanas, que reinen particularmente en este día dos sentimientos: la alabanza, como hicieron aquellos que acogieron a Jesús en Jerusalén con su \textquote{hosanna}; y el agradecimiento, porque en esta Semana Santa el Señor Jesús renovará el don más grande que se puede imaginar, nos entregará su vida, su cuerpo y su sangre, su amor. Pero a un don tan grande debemos corresponder de modo adecuado, o sea, con el don de nosotros mismos, de nuestro tiempo, de nuestra oración, de nuestro estar en comunión profunda de amor con Cristo que sufre, muere y resucita por nosotros. Los antiguos Padres de la Iglesia han visto un símbolo de todo esto en el gesto de la gente que seguía a Jesús en su ingreso a Jerusalén, el gesto de tender los mantos delante del Señor. Ante Cristo –decían los Padres–, debemos deponer nuestra vida, nuestra persona, en actitud de gratitud y adoración. En conclusión, escuchemos de nuevo la voz de uno de estos antiguos Padres, la de san Andrés, obispo de Creta: \textquote{Así es como nosotros deberíamos prosternarnos a los pies de Cristo, no poniendo bajo sus pies nuestras túnicas o unas ramas inertes, que muy pronto perderían su verdor, su fruto y su aspecto agradable, sino revistiéndonos de su gracia, es decir, de él mismo\ldots Así debemos ponernos a sus pies como si fuéramos unas túnicas\ldots Ofrezcamos ahora al vencedor de la muerte no ya ramas de palma, sino trofeos de victoria. Repitamos cada día aquella sagrada exclamación que los niños cantaban, mientras agitamos los ramos espirituales del alma: \textquote{Bendito el que viene, como rey, en nombre del Señor}} (\textit{PG} 97, 994). Amén.
\end{body}


\newsection
\subsection{Francisco, papa}

\subsubsection{Homilía (2015): El estilo de Dios}

\src{XXX Jornada Mundial de la Juventud.\\Plaza de San Pedro. 29 de marzo de 2015.}

\begin{body}
\ltr{E}{n} el centro de esta celebración, que se presenta tan festiva, está la palabra que hemos escuchado en el himno de la \textbf{Carta a los Filipenses}: \textquote{Se humilló a sí mismo} (\textit{Flp} 2, 8). La humillación de Jesús.

Esta palabra nos desvela el estilo de Dios y, en consecuencia, aquel que debe ser el del cristiano: la humildad. Un estilo que nunca dejará de sorprendernos y ponernos en crisis: nunca nos acostumbraremos a un Dios humilde.

Humillarse es ante todo el estilo de Dios: Dios se humilla para caminar con su pueblo, para soportar sus infidelidades. Esto se aprecia bien leyendo la historia del Éxodo: ¡Qué humillación para el Señor oír todas aquellas murmuraciones, aquellas quejas! Estaban dirigidas contra Moisés, pero, en el fondo, iban contra él, contra su Padre, que los había sacado de la esclavitud y los guiaba en el camino por el desierto hasta la tierra de la libertad.

En esta semana, la Semana Santa, que nos conduce a la Pascua, seguiremos este camino de la humillación de Jesús. Y sólo así será \textquote{santa} también para nosotros.

Veremos el desprecio de los jefes del pueblo y sus engaños para acabar con él. Asistiremos a la traición de Judas, uno de los Doce, que lo venderá por treinta monedas. Veremos al Señor apresado y tratado como un malhechor; abandonado por sus discípulos; llevado ante el Sanedrín, condenado a muerte, azotado y ultrajado. Escucharemos cómo Pedro, la \textquote{roca} de los discípulos, lo negará tres veces. Oiremos los gritos de la muchedumbre, soliviantada por los jefes, pidiendo que Barrabás quede libre y que a él lo crucifiquen. Veremos cómo los soldados se burlarán de él, vestido con un manto color púrpura y coronado de espinas. Y después, a lo largo de la vía dolorosa y a los pies de la cruz, sentiremos los insultos de la gente y de los jefes, que se ríen de su condición de Rey e Hijo de Dios.

Esta es la vía de Dios, el camino de la humildad. Es el camino de Jesús, no hay otro. Y no hay humildad sin humillación.

Al recorrer hasta el final este camino, el Hijo de Dios tomó la \textquote{condición de siervo} (\textit{Flp} 2, 7). En efecto, la humildad quiere decir también servicio, significa dejar espacio a Dios negándose a uno mismo, \textquote{despojándose}, como dice la Escritura (v. 7). Este \textquote{despojarse} es la humillación más grande.

Hay otra vía, contraria al camino de Cristo: la mundanidad. La mundanidad nos ofrece el camino de la vanidad, del orgullo, del éxito\ldots Es la otra vía. El maligno se la propuso también a Jesús durante cuarenta días en el desierto. Pero Jesús la rechazó sin dudarlo. Y, con él, solamente con su gracia y con su ayuda, también nosotros podemos vencer esta tentación de la vanidad, de la mundanidad, no sólo en las grandes ocasiones, sino también en las circunstancias ordinarias de la vida.

En esto, nos ayuda y nos conforta el ejemplo de muchos hombres y mujeres que, en silencio y sin hacerse ver, renuncian cada día a sí mismos para servir a los demás: un familiar enfermo, un anciano solo, una persona con discapacidad, una persona sin techo\ldots

Pensemos también en la humillación de los que, por mantenerse fieles al Evangelio, son discriminados y sufren las consecuencias en su propia carne. Y pensemos en nuestros hermanos y hermanas perseguidos por ser cristianos, los mártires de hoy –que son muchos–: no reniegan de Jesús y soportan con dignidad insultos y ultrajes. Lo siguen por su camino. Podemos hablar, verdaderamente, de \textquote{una nube de testigos}: los mártires de hoy (cf. \textit{Hb} 12, 1).

Durante esta semana, emprendamos también nosotros con decisión este camino de la humildad, movidos por el amor a nuestro Señor y Salvador. El amor nos guiará y nos dará fuerza. Y, donde está él, estaremos también nosotros (cf. \textit{Jn} 12, 26).
\end{body}

\label{b2-04-01-2015H}
\newpage

\subsubsection{Homilía (2018): ¡Crucifícalo!}

\src{XXXIII Jornada Mundial de la Juventud. Plaza de San Pedro. \\25 de marzo de 2018.}

\begin{body}
\ltr{J}{esús} entra en Jerusalén. La liturgia nos invitó a hacernos partícipes y tomar parte de la alegría y fiesta del pueblo que es capaz de gritar y alabar a su Señor; alegría que se empaña y deja un sabor amargo y doloroso al terminar de escuchar el relato de la Pasión. Pareciera que en esta celebración se entrecruzan historias de alegría y sufrimiento, de errores y aciertos que forman parte de nuestro vivir cotidiano como discípulos, ya que logra desnudar los sentimientos contradictorios que también hoy, hombres y mujeres de este tiempo, solemos tener: capaces de amar mucho\ldots y también de odiar ―y mucho―; capaces de entregas valerosas y también de saber \textquote{lavarnos las manos} en el momento oportuno; capaces de fidelidades pero también de grandes abandonos y traiciones. Y se ve claro en todo el relato evangélico que la alegría que Jesús despierta es motivo de enojo e irritación en manos de algunos.

Jesús entra en la ciudad rodeado de su pueblo, rodeado por cantos y gritos de algarabía. Podemos imaginar que es la voz del hijo perdonado, la del leproso sanado o el balar de la oveja perdida, que resuenan a la vez con fuerza en ese ingreso. Es el canto del publicano y del impuro; es el grito del que vivía en los márgenes de la ciudad. Es el grito de hombres y mujeres que lo han seguido porque experimentaron su compasión ante su dolor y su miseria\ldots Es el canto y la alegría espontánea de tantos postergados que tocados por Jesús pueden gritar: \textquote{Bendito el que llega en nombre del Señor}. ¿Cómo no alabar a Aquel que les había devuelto la dignidad y la esperanza? Es la alegría de tantos pecadores perdonados que volvieron a confiar y a esperar. Y estos gritan. Se alegran. Es la alegría.

Esta alegría y alabanza resulta incómoda y se transforma en sinrazón escandalosa para aquellos que se consideran a sí mismos justos y \textquote{fieles} a la ley y a los preceptos rituales (Cf. R. Guardini, \textit{El Señor}, 383). Alegría insoportable para quienes han bloqueado la sensibilidad ante el dolor, el sufrimiento y la miseria. Muchos de estos piensan: \textquote{¡Mira que pueblo más maleducado!}. Alegría intolerable para quienes perdieron la memoria y se olvidaron de tantas oportunidades recibidas. ¡Qué difícil es comprender la alegría y la fiesta de la misericordia de Dios para quien quiere justificarse a sí mismo y acomodarse! ¡Qué difícil es poder compartir esta alegría para quienes solo confían en sus propias fuerzas y se sienten superiores a otros! (Cf. Exhort. ap. \textit{Evangelii gaudium}, 94). Y así nace el grito del que no le tiembla la voz para gritar: \textquote{¡Crucifícalo!}. No es un grito espontáneo, sino el grito armado, producido, que se forma con el desprestigio, la calumnia, cuando se levanta falso testimonio. Es el grito que nace cuando se pasa del hecho a lo que se cuenta, nace de lo que se cuenta. Es la voz de quien manipula la realidad y crea un relato a su conveniencia y no tiene problema en \textquote{manchar} a otros para salirse con la suya. Esto es un falso relato. El grito del que no tiene problema en buscar los medios para hacerse más fuerte y silenciar las voces disonantes. Es el grito que nace de \textquote{trucar} la realidad y pintarla de manera tal que termina desfigurando el rostro de Jesús y lo convierte en un \textquote{malhechor}. Es la voz del que quiere defender la propia posición desacreditando especialmente a quien no puede defenderse. Es el grito fabricado por la \textquote{tramoya} de la autosuficiencia, el orgullo y la soberbia que afirma sin problemas: \textquote{Crucifícalo, crucifícalo}.

Y así se termina silenciando la fiesta del pueblo, derribando la esperanza, matando los sueños, suprimiendo la alegría; así se termina blindando el corazón, enfriando la caridad. Es el grito del \textquote{sálvate a ti mismo} que quiere adormecer la solidaridad, apagar los ideales, insensibilizar la mirada \ldots el grito que quiere borrar la compasión, ese \textquote{padecer con}, la compasión, que es la debilidad de Dios. Frente a todos estos titulares, el mejor antídoto es mirar la cruz de Cristo y dejarnos interpelar por su último grito. Cristo murió gritando su amor por cada uno de nosotros; por jóvenes y mayores, santos y pecadores, amor a los de su tiempo y a los de nuestro tiempo. En su cruz hemos sido salvados para que nadie apague la alegría del evangelio; para que nadie, en la situación que se encuentre, quede lejos de la mirada misericordiosa del Padre. Mirar la cruz es dejarse interpelar en nuestras prioridades, opciones y acciones. Es dejar cuestionar nuestra sensibilidad ante el que está pasando o viviendo un momento de dificultad. Hermanos y hermanas: ¿Qué mira nuestro corazón? ¿Jesucristo sigue siendo motivo de alegría y alabanza en nuestro corazón o nos avergüenzan sus prioridades hacia los pecadores, los últimos, los olvidados?

Y a ustedes, queridos jóvenes, la alegría que Jesús despierta en ustedes es para algunos motivo de enojo y también de irritación, ya que un joven alegre es difícil de manipular. ¡Un joven alegre es difícil de manipular! Pero existe en este día la posibilidad de un tercer grito: \textquote{Algunos fariseos de entre la gente le dijeron: Maestro, reprende a tus discípulos} y él responde: \textquote{Yo les digo que, si éstos callan, gritarán las piedras} (\textit{Lc} 19, 39-40).

Hacer callar a los jóvenes es una tentación que siempre ha existido. Los mismos fariseos increpan a Jesús y le piden que los calme y silencie. Hay muchas formas de silenciar y de volver invisibles a los jóvenes. Muchas formas de anestesiarlos y adormecerlos para que no hagan \textquote{ruido}, para que no se pregunten y cuestionen. \textquote{¡Estad callados!}. Hay muchas formas de tranquilizarlos para que no se involucren y sus sueños pierdan vuelo y se vuelvan ensoñaciones rastreras, pequeñas, tristes.


En este Domingo de ramos, festejando la Jornada Mundial de la Juventud, nos hace bien escuchar la respuesta de Jesús a los fariseos de ayer y de todos los tiempos, también a los de hoy: \textquote{Si ellos callan, gritarán las piedras} (\textit{Lc} 19, 40). Queridos jóvenes: Está en ustedes la decisión de gritar, está en ustedes decidirse por el Hosanna del domingo para no caer en el \textquote{crucifícalo} del viernes\ldots Y está en ustedes no quedarse callados. Si los demás callan, si nosotros los mayores y responsables ―tantas veces corruptos― callamos, si el mundo calla y pierde alegría, les pregunto: ¿Ustedes gritarán? Por favor, decídanse antes de que griten las piedras.
\end{body}


\begin{patercite}
(\ldots) \textquote{Aquí tenéis al hombre}. Esta expresión \emph{encierra} en cierto sentido \emph{toda la verdad sobre Cristo verdadero hombre}: sobre Aquél que se ha hecho \textquote{en todo semejante a nosotros excepto en el pecado}; sobre Aquél que \textquote{se ha unido en cierto modo con todo hombre} (cf. \emph{Gaudium et spes}, 22). Lo llamaron \textquote{amigo de publicanos y pecadores}. Y justamente \emph{como víctima por el pecado} se hace \emph{solidario con todos}, incluso con los \textquote{pecadores}, hasta la muerte de cruz. Pero precisamente en esta condición de víctima, a la que Jesús está reducido, resalta un último aspecto de su humanidad, que debe ser aceptado y meditado profundamente a la luz del misterio de su \textquote{despojamiento} (Kenosis). Según San Pablo, Él, \textquote{siendo de condición divina, no retuvo ávidamente el ser igual a Dios. Sino que \emph{se despojó de sí mismo tomando condición de siervo, haciéndose semejante a los hombres} y apareciendo en su porte como hombre; y \emph{se humilló a sí mismo, obedeciendo hasta la muerte y} muerte \emph{de cruz}} (\emph{Fil} 2, 6-8).

El texto paulino de la Carta a los Filipenses nos introduce en el misterio de la \textquote{Kenosis} de Cristo. Para expresar esto misterio, el Apóstol \emph{utiliza} primero la palabra \textquote{se despojó}, y ésta \emph{se refiere} sobre todo \emph{a la realidad de la Encarnación}: \textquote{la Palabra se hizo carne} (\emph{Jn} 1, 14). ¡Dios-Hijo asumió la naturaleza humana, la humanidad, \emph{se hizo} verdadero \emph{hombre, permaneciendo Dios}! La verdad sobre Cristo-hombre debe considerarse siempre en relación a Dios-Hijo. Precisamente esta referencia permanente la señala el texto de Pablo. \textquote{Se despojó de sí mismo} no significa en ningún modo que \emph{cesó de ser Dios}: ¡sería un absurdo! Por el contrario significa, como se expresa de modo perspicaz el Apóstol, que \textquote{no retuvo ávidamente el ser igual a Dios}, sino que \textquote{\emph{siendo de condición divina}} (\textquote{in forma Dei}) ---como verdadero Dios-Hijo---, Él asumió una naturaleza humana privada de gloria, sometida al sufrimiento y a la muerte, en la cual poder vivir la obediencia al Padre hasta el extremo sacrificio.

En este contexto, el \emph{hacerse semejante a los hombres} comportó \emph{una renuncia voluntaria}, que se extendió incluso a los \textquote{privilegios} que Él habría podido gozar como hombre. Efectivamente, asumió \textquote{la condición de siervo}. No \emph{quiso pertenecer a las categorías de los poderosos}, quiso ser como el que sirve: pues, \textquote{el Hijo del hombre no ha venido a ser servido, sino a servir} (\emph{Mc} 10, 45).

\textbf{San Juan Pablo II}, papa, \textit{Catequesis}, en la Audiencia general del 17 de febrero de 1988, cf. n. 1-3.
\end{patercite}

\newsection
\section{Temas}

\cceth {La entrada de Jesús en Jerusalén} 
\cceref{CEC 557-560}

\begin{ccebody}
\ccesec{La subida de Jesús a Jerusalén}

\n{557} \textquote{Como se iban cumpliendo los días de su asunción, él se afirmó en su voluntad de ir a Jerusalén} (\textit{Lc} 9, 51; cf. \textit{Jn} 13, 1). Por esta decisión, manifestaba que subía a Jerusalén dispuesto a morir. En tres ocasiones había repetido el anuncio de su Pasión y de su Resurrección (cf. \textit{Mc} 8, 31-33; 9, 31-32; 10, 32-34). Al dirigirse a Jerusalén dice: \textquote{No cabe que un profeta perezca fuera de Jerusalén} (\textit{Lc} 13, 33).

\n{558} Jesús recuerda el martirio de los profetas que habían sido muertos en Jerusalén (cf. \textit{Mt} 23, 37a). Sin embargo, persiste en llamar a Jerusalén a reunirse en torno a él: \textquote{¡Cuántas veces he querido reunir a tus hijos, como una gallina reúne a sus pollos bajo las alas y no habéis querido!} (\textit{Mt} 23, 37b). Cuando está a la vista de Jerusalén, llora sobre ella (cf. \textit{Lc} 19, 41) y expresa una vez más el deseo de su corazón: \textquote{¡Si también tú conocieras en este día el mensaje de paz! pero ahora está oculto a tus ojos} (\textit{Lc} 19, 41-42).

\ccesec{La entrada mesiánica de Jesús en Jerusalén}

\n{559} ¿Cómo va a acoger Jerusalén a su Mesías? Jesús rehuyó siempre las tentativas populares de hacerle rey (cf. \textit{Jn} 6, 15), pero elige el momento y prepara los detalles de su entrada mesiánica en la ciudad de \textquote{David, su padre} (\textit{Lc} 1, 32; cf. \textit{Mt} 21, 1-11). Es aclamado como hijo de David, el que trae la salvación (\textquote{Hosanna} quiere decir \textquote{¡sálvanos!}, \textquote{Danos la salvación!}). Pues bien, el \textquote{Rey de la Gloria} (\textit{Sal} 24, 7-10) entra en su ciudad \textquote{montado en un asno} (\textit{Za} 9, 9): no conquista a la hija de Sión, figura de su Iglesia, ni por la astucia ni por la violencia, sino por la humildad que da testimonio de la Verdad (cf. \textit{Jn} 18, 37). Por eso los súbditos de su Reino, aquel día fueron los niños (cf. \textit{Mt} 21, 15-16; \textit{Sal} 8, 3) y los \textquote{pobres de Dios}, que le aclamaban como los ángeles lo anunciaron a los pastores (cf. \textit{Lc} 19, 38; 2, 14). Su aclamación \textquote{Bendito el que viene en el nombre del Señor} (\textit{Sal} 118, 26), ha sido recogida por la Iglesia en el \textit{Sanctus} de la liturgia eucarística para introducir al memorial de la Pascua del Señor.

\n{560} La \textit{entrada de Jesús en Jerusalén} manifiesta la venida del Reino que el Rey-Mesías llevará a cabo mediante la Pascua de su Muerte y de su Resurrección. Con su celebración, el domingo de Ramos, la liturgia de la Iglesia abre la gran Semana Santa.
\end{ccebody}

\cceth {La Pasión de Cristo} 
\cceref{CEC 602-618, 1992}

\begin{ccebody}
\ccesec{\textquote{Dios le hizo pecado por nosotros}}

\n{602} En consecuencia, san Pedro pudo formular así la fe apostólica en el designio divino de salvación: \textquote{Habéis sido rescatados de la conducta necia heredada de vuestros padres, no con algo caduco, oro o plata, sino con una sangre preciosa, como de cordero sin tacha y sin mancilla, Cristo, predestinado antes de la creación del mundo y manifestado en los últimos tiempos a causa de vosotros} (\textit{1 P} 1, 18-20). Los pecados de los hombres, consecuencia del pecado original, están sancionados con la muerte (cf. \textit{Rm} 5, 12; \textit{1 Co} 15, 56). Al enviar a su propio Hijo en la condición de esclavo (cf. \textit{Flp} 2, 7), la de una humanidad caída y destinada a la muerte a causa del pecado (cf. \textit{Rm} 8, 3), \textquote{a quien no conoció pecado, Dios le hizo pecado por nosotros, para que viniésemos a ser justicia de Dios en él} (\textit{2 Co} 5, 21).

\n{603} Jesús no conoció la reprobación como si él mismo hubiese pecado (cf. \textit{Jn} 8, 46). Pero, en el amor redentor que le unía siempre al Padre (cf. \textit{Jn} 8, 29), nos asumió desde el alejamiento con relación a Dios por nuestro pecado hasta el punto de poder decir en nuestro nombre en la cruz: \textquote{Dios mío, Dios mío, ¿por qué me has abandonado?} (\textit{Mc} 15, 34; \textit{Sal} 22, 2). Al haberle hecho así solidario con nosotros, pecadores, \textquote{Dios no perdonó ni a su propio Hijo, antes bien le entregó por todos nosotros} (\textit{Rm} 8, 32) para que fuéramos \textquote{reconciliados con Dios por la muerte de su Hijo} (\textit{Rm} 5, 10).

\ccesec{Dios tiene la iniciativa del amor redentor universal}

\n{604} Al entregar a su Hijo por nuestros pecados, Dios manifiesta que su designio sobre nosotros es un designio de amor benevolente que precede a todo mérito por nuestra parte: \textquote{En esto consiste el amor: no en que nosotros hayamos amado a Dios, sino en que él nos amó y nos envió a su Hijo como propiciación por nuestros pecados} (\textit{1 Jn} 4, 10; cf. \textit{Jn} 4, 19). \textquote{La prueba de que Dios nos ama es que Cristo, siendo nosotros todavía pecadores, murió por nosotros} (\textit{Rm} 5, 8).

\n{605} Jesús ha recordado al final de la parábola de la oveja perdida que este amor es sin excepción: \textquote{De la misma manera, no es voluntad de vuestro Padre celestial que se pierda uno de estos pequeños} (\textit{Mt} 18, 14). Afirma \textquote{dar su vida en rescate \textit{por muchos}} (\textit{Mt} 20, 28); este último término no es restrictivo: opone el conjunto de la humanidad a la única persona del Redentor que se entrega para salvarla (cf. \textit{Rm} 5, 18-19). La Iglesia, siguiendo a los Apóstoles (cf. \textit{2 Co} 5, 15; \textit{1 Jn} 2, 2), enseña que Cristo ha muerto por todos los hombres sin excepción: \textquote{no hay, ni hubo ni habrá hombre alguno por quien no haya padecido Cristo} (Concilio de Quiercy, año 853: DS, 624).

\newpage
\ccesec{Cristo se ofreció a su Padre por nuestros pecados \\Toda la vida de Cristo es oblación al Padre}


\n{606} El Hijo de Dios \textquote{bajado del cielo no para hacer su voluntad sino la del Padre que le ha enviado} (\textit{Jn} 6, 38), \textquote{al entrar en este mundo, dice: [\ldots] He aquí que vengo [\ldots] para hacer, oh Dios, tu voluntad [\ldots] En virtud de esta voluntad somos santificados, merced a la oblación de una vez para siempre del cuerpo de Jesucristo} (\textit{Hb} 10, 5-10). Desde el primer instante de su Encarnación el Hijo acepta el designio divino de salvación en su misión redentora: \textquote{Mi alimento es hacer la voluntad del que me ha enviado y llevar a cabo su obra} (\textit{Jn} 4, 34). El sacrificio de Jesús \textquote{por los pecados del mundo entero} (\textit{1 Jn} 2, 2), es la expresión de su comunión de amor con el Padre: \textquote{El Padre me ama porque doy mi vida} (\textit{Jn} 10, 17). \textquote{El mundo ha de saber que amo al Padre y que obro según el Padre me ha ordenado} (\textit{Jn} 14, 31).

\n{607} Este deseo de aceptar el designio de amor redentor de su Padre anima toda la vida de Jesús (cf. \textit{Lc} 12, 50; 22, 15; \textit{Mt} 16, 21-23) porque su Pasión redentora es la razón de ser de su Encarnación: \textquote{¡Padre líbrame de esta hora! Pero ¡si he llegado a esta hora para esto!} (\textit{Jn} 12, 27). \textquote{El cáliz que me ha dado el Padre ¿no lo voy a beber?} (\textit{Jn} 18, 11). Y todavía en la cruz antes de que \textquote{todo esté cumplido} (\textit{Jn} 19, 30), dice: \textquote{Tengo sed} (\textit{Jn} 19, 28).

\ccesec{\textquote{El cordero que quita el pecado del mundo}}

\n{608} Juan Bautista, después de haber aceptado bautizarle en compañía de los pecadores (cf. \textit{Lc} 3, 21; \textit{Mt} 3, 14-15), vio y señaló a Jesús como el \textquote{Cordero de Dios que quita los pecados del mundo} (\textit{Jn} 1, 29; cf. \textit{Jn} 1, 36). Manifestó así que Jesús es a la vez el Siervo doliente que se deja llevar en silencio al matadero (\textit{Is} 53, 7; cf. \textit{Jr} 11, 19) y carga con el pecado de las multitudes (cf. \textit{Is} 53, 12) y el cordero pascual símbolo de la redención de Israel cuando celebró la primera Pascua (\textit{Ex} 12, 3-14; cf. \textit{Jn} 19, 36; \textit{1 Co} 5, 7). Toda la vida de Cristo expresa su misión: \textquote{Servir y dar su vida en rescate por muchos} (\textit{Mc} 10, 45).

\ccesec{Jesús acepta libremente el amor redentor del Padre}

\n{609} Jesús, al aceptar en su corazón humano el amor del Padre hacia los hombres, \textquote{los amó hasta el extremo} (\textit{Jn} 13, 1) porque \textquote{nadie tiene mayor amor que el que da su vida por sus amigos} (\textit{Jn} 15, 13). Tanto en el sufrimiento como en la muerte, su humanidad se hizo el instrumento libre y perfecto de su amor divino que quiere la salvación de los hombres (cf. \textit{Hb }2, 10. 17-18; 4, 15; 5, 7-9). En efecto, aceptó libremente su pasión y su muerte por amor a su Padre y a los hombres que el Padre quiere salvar: \textquote{Nadie me quita [la vida]; yo la doy voluntariamente} (\textit{Jn} 10, 18). De aquí la soberana libertad del Hijo de Dios cuando Él mismo se encamina hacia la muerte (cf. \textit{Jn} 18, 4-6; \textit{Mt} 26, 53).

\ccesec{Jesús anticipó en la cena la ofrenda libre de su vida}

\n{610} Jesús expresó de forma suprema la ofrenda libre de sí mismo en la cena tomada con los doce Apóstoles (cf. \textit{Mt} 26, 20), en \textquote{la noche en que fue entregado} (\textit{1 Co} 11, 23). En la víspera de su Pasión, estando todavía libre, Jesús hizo de esta última Cena con sus Apóstoles el memorial de su ofrenda voluntaria al Padre (cf. \textit{1 Co} 5, 7), por la salvación de los hombres: \textquote{Este es mi Cuerpo que va a \textit{ser entregado} por vosotros} (\textit{Lc} 22, 19). \textquote{Esta es mi sangre de la Alianza que va a \textit{ser derramada} por muchos para remisión de los pecados} (\textit{Mt} 26, 28).

\n{611} La Eucaristía que instituyó en este momento será el \textquote{memorial} (\textit{1 Co} 11, 25) de su sacrificio. Jesús incluye a los Apóstoles en su propia ofrenda y les manda perpetuarla (cf. \textit{Lc} 22, 19). Así Jesús instituye a sus apóstoles sacerdotes de la Nueva Alianza: \textquote{Por ellos me consagro a mí mismo para que ellos sean también consagrados en la verdad} (\textit{Jn} 17, 19; cf. Concilio de Trento: DS, 1752; 1764).

\ccesec{La agonía de Getsemaní}

\n{612} El cáliz de la Nueva Alianza que Jesús anticipó en la Cena al ofrecerse a sí mismo (cf. \textit{Lc} 22, 20), lo acepta a continuación de manos del Padre en su agonía de Getsemaní (cf. \textit{Mt} 26, 42) haciéndose \textquote{obediente hasta la muerte} (\textit{Flp} 2, 8; cf. \textit{Hb} 5, 7-8). Jesús ora: \textquote{Padre mío, si es posible, que pase de mí este cáliz\ldots} (\textit{Mt} 26, 39). Expresa así el horror que representa la muerte para su naturaleza humana. Esta, en efecto, como la nuestra, está destinada a la vida eterna; además, a diferencia de la nuestra, está perfectamente exenta de pecado (cf. \textit{Hb} 4, 15) que es la causa de la muerte (cf. \textit{Rm} 5, 12); pero sobre todo está asumida por la persona divina del \textquote{Príncipe de la Vida} (\textit{Hch} 3, 15), de \textquote{el que vive}, \textit{Viventis assumpta} (\textit{Ap} 1, 18; cf. \textit{Jn} 1, 4; 5, 26). Al aceptar en su voluntad humana que se haga la voluntad del Padre (cf. \textit{Mt} 26, 42), acepta su muerte como redentora para \textquote{llevar nuestras faltas en su cuerpo sobre el madero} (\textit{1 P} 2, 24).

\ccesec{La muerte de Cristo es el sacrificio único y definitivo}

\n{613} La muerte de Cristo es a la vez el \textit{sacrificio} pascual que lleva a cabo la redención definitiva de los hombres (cf. \textit{1 Co} 5, 7; \textit{Jn} 8, 34-36) por medio del \textquote{Cordero que quita el pecado del mundo} (\textit{Jn} 1, 29; cf. \textit{1 P} 1, 19) y el \textit{sacrificio de la Nueva Alianza} (cf. \textit{1 Co} 11, 25) que devuelve al hombre a la comunión con Dios (cf. \textit{Ex} 24, 8) reconciliándole con Él por \textquote{la sangre derramada por muchos para remisión de los pecados} (\textit{Mt} 26, 28; cf. \textit{Lv} 16, 15-16).

\n{614} Este sacrificio de Cristo es único, da plenitud y sobrepasa a todos los sacrificios (cf. \textit{Hb} 10, 10). Ante todo es un don del mismo Dios Padre: es el Padre quien entrega al Hijo para reconciliarnos consigo (cf. \textit{1 Jn} 4, 10). Al mismo tiempo es ofrenda del Hijo de Dios hecho hombre que, libremente y por amor (cf. \textit{Jn} 15, 13), ofrece su vida (cf. \textit{Jn} 10, 17-18) a su Padre por medio del Espíritu Santo (cf. \textit{Hb} 9, 14), para reparar nuestra desobediencia.

\ccesec{Jesús reemplaza nuestra desobediencia por su obediencia}

\n{615} \textquote{Como [\ldots] por la desobediencia de un solo hombre, todos fueron constituidos pecadores, así también por la obediencia de uno solo todos serán constituidos justos} (\textit{Rm} 5, 19). Por su obediencia hasta la muerte, Jesús llevó a cabo la sustitución del Siervo doliente que \textquote{se dio a sí mismo en \textit{expiación}}, \textquote{cuando llevó el pecado de muchos}, a quienes \textquote{justificará y cuyas culpas soportará} (\textit{Is} 53, 10-12). Jesús repara por nuestras faltas y satisface al Padre por nuestros pecados (cf. Concilio de Trento: DS, 1529).

\newpage
\ccesec{En la cruz, Jesús consuma su sacrificio}

\n{616} El \textquote{amor hasta el extremo} (\textit{Jn} 13, 1) es el que confiere su valor de redención y de reparación, de expiación y de satisfacción al sacrificio de Cristo. Nos ha conocido y amado a todos en la ofrenda de su vida (cf. \textit{Ga} 2, 20; \textit{Ef} 5, 2. 25). \textquote{El amor [\ldots] de Cristo nos apremia al pensar que, si uno murió por todos, todos por tanto murieron} (\textit{2 Co} 5, 14). Ningún hombre aunque fuese el más santo estaba en condiciones de tomar sobre sí los pecados de todos los hombres y ofrecerse en sacrificio por todos. La existencia en Cristo de la persona divina del Hijo, que al mismo tiempo sobrepasa y abraza a todas las personas humanas, y que le constituye Cabeza de toda la humanidad, hace posible su sacrificio redentor por todos.

\n{617} \textit{Sua sanctissima passione in ligno crucis nobis justificationem meruit} – Por su sacratísima pasión en el madero de la cruz nos mereció la justificación, enseña el Concilio de Trento (DS, 1529) subrayando el carácter único del sacrificio de Cristo como \textquote{causa de salvación eterna} (\textit{Hb} 5, 9). Y la Iglesia venera la Cruz cantando: \textit{O crux, ave, spes unica} – \textquote{Salve, oh cruz, única esperanza} (Añadidura litúrgica al himno \textquote{Vexilla Regis}: \textit{Liturgia de las Horas}).

\ccesec{Nuestra participación en el sacrificio de Cristo}

\n{618} La Cruz es el único sacrificio de Cristo \textquote{único mediador entre Dios y los hombres} (\textit{1 Tm} 2, 5). Pero, porque en su Persona divina encarnada, \textquote{se ha unido en cierto modo con todo hombre} (GS 22, 2) Él \textquote{ofrece a todos la posibilidad de que, en la forma de Dios sólo conocida [\ldots] se asocien a este misterio pascual} (GS 22, 5). Él llama a sus discípulos a \textquote{tomar su cruz y a seguirle} (\textit{Mt} 16, 24) porque Él \textquote{sufrió por nosotros dejándonos ejemplo para que sigamos sus huellas} (\textit{1 P} 2, 21). Él quiere, en efecto, asociar a su sacrificio redentor a aquellos mismos que son sus primeros beneficiarios (cf. \textit{Mc} 10, 39; \textit{Jn} 21, 18-19; \textit{Col} 1, 24). Eso lo realiza en forma excelsa en su Madre, asociada más íntimamente que nadie al misterio de su sufrimiento redentor (cf. \textit{Lc} 2, 35):

\ccecite{\textquote{Esta es la única verdadera escala del paraíso, fuera de la Cruz no hay otra por donde subir al cielo} (Santa Rosa de Lima, cf. P. Hansen, \textit{Vita mirabilis}, Lovaina, 1668).}

\n{1992} La justificación nos fue \textit{merecida por la pasión de Cristo}, que se ofreció en la cruz como hostia viva, santa y agradable a Dios y cuya sangre vino a ser instrumento de propiciación por los pecados de todos los hombres. La justificación es concedida por el Bautismo, sacramento de la fe. Nos asemeja a la justicia de Dios que nos hace interiormente justos por el poder de su misericordia. Tiene por fin la gloria de Dios y de Cristo, y el don de la vida eterna (cf. Concilio de Trento: DS 1529).

\ccecite{\textquote{Pero ahora, independientemente de la ley, la justicia de Dios se ha manifestado, atestiguada por la ley y los profetas, justicia de Dios por la fe en Jesucristo, para todos los que creen –pues no hay diferencia alguna; todos pecaron y están privados de la gloria de Dios– y son justificados por el don de su gracia, en virtud de la redención realizada en Cristo Jesús, a quien Dios exhibió como instrumento de propiciación por su propia sangre, mediante la fe, para mostrar su justicia, pasando por alto los pecados cometidos anteriormente, en el tiempo de la paciencia de Dios; en orden a mostrar su justicia en el tiempo presente, para ser él justo y justificador del que cree en Jesús} (\textit{Rm} 3, 21-26).}
\end{ccebody}

\cceth{El Señorío de Cristo obtenido por medio de su Muerte y Resurrección} 
\cceref{CEC 2816}

\begin{ccebody}
\ccesec{Venga a nosotros tu Reino}

\n{2816} En el Nuevo Testamento, la palabra \textit{basileia} se puede traducir por realeza (nombre abstracto), reino (nombre concreto) o reinado (de reinar, nombre de acción). El Reino de Dios es para nosotros lo más importante. Se aproxima en el Verbo encarnado, se anuncia a través de todo el Evangelio, llega en la muerte y la Resurrección de Cristo. El Reino de Dios adviene en la Última Cena y por la Eucaristía está entre nosotros. El Reino de Dios llegará en la gloria cuando Jesucristo lo devuelva a su Padre:

\ccecite{\textquote{Incluso [\ldots] puede ser que el Reino de Dios signifique Cristo en persona, al cual llamamos con nuestras voces todos los días y de quien queremos apresurar su advenimiento por nuestra espera. Como es nuestra Resurrección porque resucitamos en él, puede ser también el Reino de Dios porque en él reinaremos} (San Cipriano de Cartago, \textit{De dominica Oratione}, 13).}
\end{ccebody}

\cceth{El Misterio Pascual y la Liturgia}
\cceref{CEC 654, 1067-1068, 1085, 1362}

\begin{ccebody}

\n{654} Hay un doble aspecto en el misterio pascual: por su muerte nos libera del pecado, por su Resurrección nos abre el acceso a una nueva vida. Esta es, en primer lugar, la \textit{justificación} que nos devuelve a la gracia de Dios (cf. \textit{Rm} 4, 25) \textquote{a fin de que, al igual que Cristo fue resucitado de entre los muertos [\ldots] así también nosotros vivamos una nueva vida} (\textit{Rm} 6, 4). Consiste en la victoria sobre la muerte y el pecado y en la nueva participación en la gracia (cf. \textit{Ef} 2, 4-5; \textit{1 P} 1, 3). Realiza la \textit{adopción filial} porque los hombres se convierten en hermanos de Cristo, como Jesús mismo llama a sus discípulos después de su Resurrección: \textquote{Id, avisad a mis hermanos} (\textit{Mt} 28, 10; \textit{Jn} 20, 17). Hermanos no por naturaleza, sino por don de la gracia, porque esta filiación adoptiva confiere una participación real en la vida del Hijo único, la que ha revelado plenamente en su Resurrección.

\n{1067} \textquote{Cristo el Señor realizó esta obra de la redención humana y de la perfecta glorificación de Dios, preparada por las maravillas que Dios hizo en el pueblo de la Antigua Alianza, principalmente por el misterio pascual de su bienaventurada pasión, de su resurrección de entre los muertos y de su gloriosa ascensión. Por este misterio, \textquote{con su muerte destruyó nuestra muerte y con su resurrección restauró nuestra vida}. Pues del costado de Cristo dormido en la cruz nació el sacramento admirable de toda la Iglesia} (SC 5). Por eso, en la liturgia, la Iglesia celebra principalmente el misterio pascual por el que Cristo realizó la obra de nuestra salvación.

\n{1068} Es el Misterio de Cristo lo que la Iglesia anuncia y celebra en su liturgia a fin de que los fieles vivan de él y den testimonio del mismo en el mundo:

\ccecite{\textquote{En efecto, la liturgia, por medio de la cual \textquote{se ejerce la obra de nuestra redención}, sobre todo en el divino sacrificio de la Eucaristía, contribuye mucho a que los fieles, en su vida, expresen y manifiesten a los demás el misterio de Cristo y la naturaleza genuina de la verdadera Iglesia} (SC 2).}

\newpage
\ccesec{Significación de la palabra \textquote{Liturgia}}

\n{1085} En la liturgia de la Iglesia, Cristo significa y realiza principalmente su misterio pascual. Durante su vida terrestre Jesús anunciaba con su enseñanza y anticipaba con sus actos el misterio pascual. Cuando llegó su hora (cf. \textit{Jn} 13, 1; 17, 1), vivió el único acontecimiento de la historia que no pasa: Jesús muere, es sepultado, resucita de entre los muertos y se sienta a la derecha del Padre \textquote{una vez por todas} (\textit{Rm} 6, 10; \textit{Hb} 7, 27; 9, 12). Es un acontecimiento real, sucedido en nuestra historia, pero absolutamente singular: todos los demás acontecimientos suceden una vez, y luego pasan y son absorbidos por el pasado. El misterio pascual de Cristo, por el contrario, no puede permanecer solamente en el pasado, pues por su muerte destruyó a la muerte, y todo lo que Cristo es y todo lo que hizo y padeció por los hombres participa de la eternidad divina y domina así todos los tiempos y en ellos se mantiene permanentemente presente. El acontecimiento de la Cruz y de la Resurrección \textit{permanece} y atrae todo hacia la Vida.

\ccesec{El memorial sacrificial de Cristo y de su Cuerpo, que es la Iglesia}

\n{1362} La Eucaristía es el memorial de la Pascua de Cristo, la actualización y la ofrenda sacramental de su único sacrificio, en la liturgia de la Iglesia que es su Cuerpo. En todas las plegarias eucarísticas encontramos, tras las palabras de la institución, una oración llamada \textit{anámnesis} o memorial.
\end{ccebody}

\img{monograma_ihs}
