			\part{Triduo Pascual}
			
			\chapter{Introducción}
			
			\begin{introsection}Normativa litúrgica\footnote{9}\end{introsection}
			
			\begin{bodyintro}La Iglesia, a lo largo del año, conmemora todo el Misterio de Cristo desde la Encarnación hasta Pentecostés, y la espera de la Venida del Señor. La obra de la redención humana y de la perfecta glorificación de Dios, fue realizada por Cristo principalmente por el Misterio Pascual, mediante el cual con su muerte destruyó nuestra muerte y con su Resurrección restauró nuestra vida. Por esta razón el santo Triduo pascual de la Pasión y Resurrección del Señor es el centro del año litúrgico. Así como el domingo constituye el núcleo de la semana, también la solemnidad de Pascua constituye el núcleo del año litúrgico.\end{bodyintro}
			
			\begin{bodyintro}El Triduo de la Pasión y Resurrección del Señor comienza con la Misa vespertina de la Cena del Señor; tiene su centro en la Vigilia Pascual y concluye con las vísperas del domingo de Resurrección. El Viernes Santo de la Pasión del Señor y –según las posibilidades– también el Sábado Santo hasta la Vigilia Pascual se guarda en todas partes el sagrado ayuno pascual. \end{bodyintro}
			
			\begin{bodyintro}La Vigilia Pascual, en la noche santa de la Resurrección del Señor, es considerada como “la madre de todas las santas vigilias”, en ella, la Iglesia espera en vela la Resurrección de Cristo y la celebra en los sacramentos. Por consiguiente, la celebración de esta santa Vigilia debe hacerse totalmente de noche, es decir, empezar después del comienzo de la noche y terminar antes del alba del domingo. \end{bodyintro}
			
			\begin{introsection}Indicaciones pastorales y de piedad sobre el Triduo\footnote{10}\end{introsection}
			
			\begin{bodyintro}Todos los años en el “sacratísimo triduo del crucificado, del sepultado y del resucitado” o Triduo pascual la Iglesia celebra, “en íntima comunión con Cristo su Esposo”, los grandes misterios de la redención humana.\end{bodyintro}
			
			\begin{bodyintro}\begin{introsubsection}Jueves Santo\end{introsubsection} \end{bodyintro}
			
			\begin{bodyintro}\textbf{La Misa de la Cena del Señor}\footnote{11}\end{bodyintro}
			
			\begin{bodyintro}Con esta misa, que se celebra en la tarde del jueves de la Semana Santa, la Iglesia comienza el santo Triduo Pascual, y desea conmemorar aquella última cena en la que el Señor Jesús, la noche en que iba a ser entregado, amando hasta el extremo a los suyos que estaban en el mundo, ofreció a Dios Padre su Cuerpo y su Sangre bajo las especies de pan y vino, y lo entregó a los apóstoles para que lo tomaran, ordenándoles a ellos y a sus sucesores en el sacerdocio que lo ofrecieran.\end{bodyintro}
			
			\begin{bodyintro}Con esta misa, en efecto, se hace el memorial tanto de la institución de la eucaristía –es decir, el memorial de la Pascua del Señor, con el que se perpetúa en nosotros el sacrificio de la nueva ley, bajo los signos del sacramento– como también de la institución del sacerdocio, mediante el cual se perpetúan en el mundo la misión y el sacrificio de Cristo; además, es memorial de la caridad con que Cristo nos amó hasta la muerte. \end{bodyintro}
			
			\begin{bodyintro}\textit{La visita al lugar de la reserva}\end{bodyintro}
			
			\begin{bodyintro}La piedad popular es especialmente sensible a la adoración del santísimo Sacramento, que sigue a la celebración de la Misa \textit{en la cena del Señor}. A causa de un proceso histórico, que todavía no está del todo claro en algunas de sus fases, el lugar de la reserva se ha considerado como “santo sepulcro”; los fieles acudían para venerar a Jesús que después del descendimiento de la Cruz fue sepultado en la tumba, donde permaneció unas Cuarenta horas.\end{bodyintro}
			
			\begin{bodyintro}Es preciso iluminar a los fieles sobre el sentido de la reserva: realizada con austera solemnidad y ordenada esencialmente a la conservación del Cuerpo del Señor, para la comunión de los fieles en la Celebración litúrgica del Viernes Santo y para el Viático de los enfermos, es una invitación a la adoración, silenciosa y prolongada, del Sacramento admirable, instituido en este día.\end{bodyintro}
			
			\begin{bodyintro}Por lo tanto, para el lugar de la reserva hay que evitar el término “sepulcro” (“monumento”), y en su disposición no se le debe dar la forma de una sepultura; el sagrario no puede tener la forma de un sepulcro o urna funeraria: el Sacramento hay que conservarlo en un sagrario cerrado, sin hacer la exposición con la custodia.\end{bodyintro}
			
			\begin{bodyintro}Después de la media noche del Jueves Santo, la adoración se realiza sin solemnidad, pues ya ha comenzado el día de la Pasión del Señor.\end{bodyintro}
			
			\begin{bodyintro} \footnote{12}\end{bodyintro}
			
			\begin{bodyintro}En este día en que “ha sido inmolada nuestra víctima pascual: Cristo” (\textit{1 Cor} 5, 7), tuvo manifiesto y cumplido efecto todo aquello que desde antiguo había sido misteriosamente prefigurado: sustituyó el verdadero Cordero al cordero simbólico, y con un único sacrificio se llevó a cumplimiento los de las diferentes víctimas precedentes.\end{bodyintro}
			
			\begin{bodyintro}“Cristo el Señor realizó esta obra de redención humana y de glorificación perfecta de Dios, preparada por las maravillas que Dios hizo en el pueblo de la Antigua Alianza, principalmente por el misterio pascual de su bienaventurada Pasión, de su Resurrección de entre los muertos y de su gloriosa Ascensión. ‘Por este misterio, con su muerte destruyó nuestra muerte y con su Resurrección restauró nuestra vida’. Pues del costado de Cristo dormido en la cruz nació el sacramento admirable de toda la Iglesia”\footnote{13}.\end{bodyintro}
			
			\begin{bodyintro}La Iglesia, contemplando la cruz de su Señor y Esposo, conmemora su propio nacimiento y su misión de extender a todos los pueblos los maravillosos efectos de la pasión de Cristo, que hoy celebra, dado gracias por un don tan sublime.\end{bodyintro}
			
			\begin{bodyintro}\textit{La procesión del Viernes Santo}\end{bodyintro}
			
			\begin{bodyintro}El Viernes Santo la Iglesia celebra la Muerte salvadora de Cristo. En el Acto litúrgico de la tarde, medita en la Pasión de su Señor, intercede por la salvación del mundo, adora la Cruz y conmemora su propio nacimiento del costado abierto del Salvador (cfr. \textit{Jn} 19, 34).\end{bodyintro}
			
			\begin{bodyintro}Entre las manifestaciones de piedad popular del Viernes Santo, además del \textit{Vía Crucis}, destaca la procesión del “Cristo muerto”. Esta destaca, según las formas expresivas de la piedad popular, el pequeño grupo de amigos y discípulos que, después de haber bajado de la Cruz el Cuerpo de Jesús, lo llevaron al lugar en el cual había una “tumba excavada en la roca, en la cual todavía no se había dado sepultura a nadie” (\textit{Lc} 23, 53).\end{bodyintro}
			
			\begin{bodyintro}La procesión del “Cristo muerto” se desarrolla, por lo general, en un clima de austeridad, de silencio y de oración, con la participación de numerosos fieles, que perciben no pocos sentidos del misterio de la sepultura de Jesús.\end{bodyintro}
			
			\begin{bodyintro}Sin embargo, es necesario que estas manifestaciones de la piedad popular nunca aparezcan ante los fieles, ni por la hora ni por el modo de convocatoria, como sucedáneo de las celebraciones litúrgicas del Viernes Santo.\end{bodyintro}
			
			\begin{bodyintro}Por lo tanto, al planificar pastoralmente el Viernes Santo se deberá conceder el primer lugar y el máximo relieve a la Celebración litúrgica, y se deberá explicar a los fieles que ningún ejercicio de piedad debe sustituir a esta celebración, en su valor objetivo.\end{bodyintro}
			
			\begin{bodyintro}Finalmente, hay que evitar introducir la procesión de “Cristo muerto” en el ámbito de la solemne Celebración litúrgica del Viernes Santo, porque esto constituiría una mezcla híbrida de celebraciones.\end{bodyintro}
			
			\begin{bodyintro}\textit{Representación de la Pasión de Cristo}\end{bodyintro}
			
			\begin{bodyintro}En muchas regiones, durante la Semana Santa, sobre todo el Viernes, tienen lugar representaciones de la Pasión de Cristo. Se trata, frecuentemente, de verdaderas “representaciones sagradas”, que con razón se pueden considerar un ejercicio de piedad. Las representaciones sagradas hunden sus raíces en la Liturgia. Algunas de ellas, nacidas casi en el coro de los monjes, mediante un proceso de dramatización progresiva, han pasado al atrio de la iglesia.\end{bodyintro}
			
			\begin{bodyintro}En muchos lugares, la preparación y ejecución de la representación de la Pasión de Cristo está encomendada a cofradías, cuyos miembros han asumido determinados compromisos de vida cristiana. En estas representaciones, actores y espectadores son introducidos en un movimiento de fe y de auténtica piedad. Es muy deseable que las representaciones sagradas de la Pasión del Señor no se alejen de este estilo de expresión sincera y gratuita de piedad, para convertirse en manifestaciones folclóricas, que atraen no tanto el espíritu religioso cuanto el interés de los turistas.\end{bodyintro}
			
			\begin{bodyintro}Respecto a las representaciones sagradas hay que explicar a los fieles la profunda diferencia que hay entre una “representación” que es mímesis, y la “acción litúrgica”, que es anámnesis, presencia mistérica del acontecimiento salvífico de la Pasión.\end{bodyintro}
			
			\begin{bodyintro}Hay que rechazar las prácticas penitenciales que consisten en hacerse crucificar con clavos.\end{bodyintro}
			
			\begin{bodyintro}\textit{El recuerdo de la Virgen de los Dolores}\end{bodyintro}
			
			\begin{bodyintro}Dada su importancia doctrinal y pastoral, se recomienda no descuidar el “recuerdo de los dolores de la Santísima Virgen María”. La piedad popular, siguiendo el relato evangélico, ha destacado la asociación de la Madre a la Pasión salvadora del Hijo (cfr. \textit{Jn} 19, 25-27; \textit{Lc} 2, 34ss) y ha dado lugar a diversos ejercicios de piedad entre los que se deben recordar:\end{bodyintro}
			
			\begin{bodyintro}- el \textit{Planctus Mariae}, expresión intensa de dolor, que con frecuencia contiene elementos de gran valor literario y musical, en el que la Virgen llora no sólo la muerte del Hijo, inocente y santo, su bien sumo, sino también la pérdida de su pueblo y el pecado de la humanidad.\end{bodyintro}
			
			\begin{bodyintro}- la “Hora de la Dolorosa”, en la que los fieles, con expresiones de conmovedora devoción, “hacen compañía” a la Madre del Señor, que se ha quedado sola y sumergida en un profundo dolor, después de la muerte de su único Hijo; al contemplar a la Virgen con el Hijo entre sus brazos –la Piedad– comprenden que en María se concentra el dolor del universo por la muerte de Cristo; en ella ven la personificación de todas las madres que, a lo largo de la historia, han llorado la muerte de un hijo. Este ejercicio de piedad, que en algunos lugares de América Latina se denomina “El \textit{pésame”}, no se debe limitar a expresar el sentimiento humano ante una madre desolada, sino que, desde la fe en la Resurrección, debe ayudar a comprender la grandeza del amor redentor de Cristo y la participación en el mismo de su Madre.\end{bodyintro}
			
			\begin{bodyintro} \begin{introsubsection}Sábado Santo\end{introsubsection} \end{bodyintro}
			
			\begin{bodyintro}Durante el Sábado Santo la Iglesia permanece junto al sepulcro del Señor, meditando su Pasión y Muerte, su descenso a los infiernos y esperando en la oración y el ayuno su Resurrección.\end{bodyintro}
			
			\begin{bodyintro}La piedad popular no puede permanecer ajena al carácter particular del Sábado Santo; así pues, las costumbres y las tradiciones festivas vinculadas a este día, en el que durante una época se anticipaba la celebración pascual, se deben reservar para la noche y el día de Pascua.\end{bodyintro}
			
			\begin{bodyintro}\textit{La “Hora de la Madre”}\end{bodyintro}
			
			\begin{bodyintro}En María, conforme a la enseñanza de la tradición, está como concentrado todo el cuerpo de la Iglesia: ella es la “credentium collectio universa”. Por esto la Virgen María, que permanece junto al sepulcro de su Hijo, tal como la representa la tradición eclesial, es imagen de la Iglesia Virgen que vela junto a la tumba de su Esposo, en espera de celebrar su Resurrección.\end{bodyintro}
			
			\begin{bodyintro}En esta intuición de la relación entre María y la Iglesia se inspira el ejercicio de piedad de la \textit{Hora de la Madre}: mientras el cuerpo del Hijo reposa en el sepulcro y su alma desciende a los infiernos para anunciar a sus antepasados la inminente liberación de la región de las tinieblas, la Virgen, anticipando y representando a la Iglesia, espera llena de fe la victoria del Hijo sobre la muerte.\end{bodyintro}
			
			\begin{bodyintro} \end{bodyintro}
			
			\begin{bodyintro}\begin{introsubsection}Domingo de Pascua de la Resurrección del Señor\end{introsubsection} \end{bodyintro}
			
			\begin{bodyintro}\textbf{Vigilia pascual en la noche santa}\footnote{14}\end{bodyintro}
			
			\begin{bodyintro}Según una antiquísima tradición, esta es la noche en que veló el Señor (cf. \textit{Ex} 12, 42). Los fieles, tal como lo recomienda el Evangelio (\textit{Lc }12, 35-37), deben asemejarse a los criados que, con las lámparas encendidas en sus manos, esperan el retorno de su Señor, para que cuando llegue les encuentre en vela y los invite a sentarse a su mesa.\end{bodyintro}
			
			\begin{bodyintro}La Vigilia de esta noche, en la que el Señor ha resucitado, se considera “la madre de todas las vigilias sagradas”\footnote{15}, la mayor y más noble de todas las solemnidades. Pues en ella, la Iglesia espera vigilante la Resurrección del Señor y en ella celebra los sacramentos de la iniciación cristiana.\end{bodyintro}
			
			\begin{bodyintro}Toda la celebración de la Vigilia pascual se realiza durante la noche, de modo que no debe comenzar antes de anochecer y debe concluir antes de que apunte la luz del domingo.\end{bodyintro}
			
			\begin{bodyintro}\textbf{Domingo de Pascua}\end{bodyintro}
			
			\begin{bodyintro}También en el Domingo de Pascua, máxima solemnidad del año litúrgico, tienen lugar no pocas manifestaciones de la piedad popular: son, todas, expresiones cultuales que exaltan la nueva condición y la gloria de Cristo resucitado, así como su poder divino que brota de su victoria sobre el pecado y sobre la muerte.\end{bodyintro}
			
			\begin{bodyintro}\textit{El encuentro del Resucitado con la Madre}\end{bodyintro}
			
			\begin{bodyintro}La piedad popular ha intuido que la asociación del Hijo con la Madre es permanente: en la hora del dolor y de la muerte, en la hora de la alegría y de la Resurrección.\end{bodyintro}
			
			\begin{bodyintro}La afirmación litúrgica de que Dios ha colmado de alegría a la Virgen en la Resurrección del Hijo, ha sido, por decirlo de algún modo, traducida y representada por la piedad popular en el \textit{Encuentro de la Madre con el Hijo resucitado}: la mañana de Pascua dos procesiones, una con la imagen de la Madre dolorosa, otra con la de Cristo resucitado, se encuentran para significar que la Virgen fue la primera que participó, y plenamente, del misterio de la Resurrección del Hijo.\end{bodyintro}
			
			\begin{bodyintro}Para este ejercicio de piedad es válida la observación que se hizo respecto a la procesión del “Cristo muerto”: su realización no debe dar a entender que sea más importante que las celebraciones litúrgicas del domingo de Pascua, ni dar lugar a mezclas rituales inadecuadas.\end{bodyintro}
			
			\begin{bodyintro}\textit{Bendición de la mesa familiar}\end{bodyintro}
			
			\begin{bodyintro}Toda la Liturgia pascual está penetrada de un sentido de novedad: es nueva la naturaleza, porque en el hemisferio norte la pascua coincide con el despertar primaveral; son nuevos el fuego y el agua; son nuevos los corazones de los cristianos, renovados por el sacramento de la Penitencia y, a ser posible, por los mismos sacramentos de la Iniciación cristiana; es nueva, por decirlo de alguna manera, la Eucaristía: son signos y realidades-signo de la nueva condición de vida inaugurada por Cristo con su Resurrección.\end{bodyintro}
			
			\begin{bodyintro}Entre los ejercicios de piedad que se relacionan con la Pascua se cuentan las tradicionales bendiciones de huevos, símbolos de vida, y la bendición de la mesa familiar; esta última, que es además una costumbre diaria de las familias cristianas, que se debe alentar, adquiere un significado particular en el día de Pascua: con el agua bendecida en la Vigilia Pascual, que los fieles llevan a sus hogares, según una loable costumbre, el cabeza de familia u otro miembro de la comunidad doméstica bendice la mesa pascual.\end{bodyintro}
			
			\begin{bodyintro}\textit{El saludo pascual a la Madre del Resucitado}\end{bodyintro}
			
			\begin{bodyintro}En algunos lugares, al final de la Vigilia pascual o después de las II Vísperas del Domingo de Pascua, se realiza un breve ejercicio de piedad: se bendicen flores, que se distribuyen a los fieles como signo de la alegría pascual, y se rinde homenaje a la imagen de la Dolorosa, que a veces se corona, mientras se canta el \textit{Regina caeli}. Los fieles, que se habían asociado al dolor de la Virgen por la Pasión del Hijo, quieren así alegrarse con ella por el acontecimiento de la Resurrección.\end{bodyintro}
			
			\begin{bodyintro}Este ejercicio de piedad, que no se debe mezclar con el acto litúrgico, es conforme a los contenidos del Misterio pascual y constituye una prueba ulterior de cómo la piedad popular percibe la asociación de la Madre a la obra salvadora del Hijo.\end{bodyintro}
			
			\chapter{Jueves Santo en la Cena del Señor}
			
			\section{Lecturas}
			
			\begin{readtitle}PRIMERA LECTURA\end{readtitle}
			
			\begin{readbook}Del libro del Éxodo \rightline{12, 1-8. 11-14}\end{readbook}
			
			\begin{readtheme}Prescripciones sobre la cena pascual\end{readtheme}
			
			\begin{readbody}En aquellos días, dijo el Señor a Moisés y a Aarón en tierra de Egipto: \end{readbody}
			
			\begin{readtalk}“Este mes será para vosotros el principal de los meses; será para vosotros el primer mes del año. Decid a toda la asamblea de los hijos de Israel: ‘El diez de este mes cada uno procurará un animal para su familia, uno por casa. Si la familia es demasiado pequeña para comérselo, que se junte con el vecino más próximo a su casa, hasta completar el número de personas; y cada uno comerá su parte hasta terminarlo. \end{readtalk}
			
			\begin{readtalk}Será un animal sin defecto, macho, de un año; lo escogeréis entre los corderos o los cabritos. \end{readtalk}
			
			\begin{readtalk}Lo guardaréis hasta el día catorce del mes y toda la asamblea de los hijos de Israel lo matará al atardecer’. Tomaréis la sangre y rociaréis las dos jambas y el dintel de la casa donde lo comáis. Esa noche comeréis la carne, asada a fuego, y comeréis panes sin fermentar y hierbas amargas. \end{readtalk}
			
			\begin{readtalk}Y lo comeréis así: la cintura ceñida, las sandalias en los pies, un bastón en la mano; y os lo comeréis a toda prisa, porque es la Pascua, el Paso del Señor. \end{readtalk}
			
			\begin{readtalk}Yo pasaré esta noche por la tierra de Egipto y heriré a todos los primogénitos de la tierra de Egipto, desde los hombres hasta los ganados, y me tomaré justicia de todos los dioses de Egipto. Yo, el Señor. \end{readtalk}
			
			\begin{readtalk}La sangre será vuestra señal en las casas donde habitáis. Cuando yo vea la sangre, pasaré de largo ante vosotros, y no habrá entre vosotros plaga exterminadora, cuando yo hiera a la tierra de Egipto. \end{readtalk}
			
			\begin{readtalk}Este será un día memorable para vosotros; en él celebraréis fiesta en honor del Señor. De generación en generación, como ley perpetua lo festejaréis”.\end{readtalk}
			
			\begin{readtitle}SALMO RESPONSORIAL\end{readtitle}
			
			\begin{readbook}Salmo \rightline{115, 12-13. 15-16. 17-18}\end{readbook}
			
			\begin{readtheme}El cáliz de la bendición es comunión de la sangre de Cristo\end{readtheme}
			
			\begin{readps}\begin{readred}℣.\end{readred} ¿Cómo pagaré al Señor \end{readps}
			
			\begin{readtabbed}todo el bien que me ha hecho? \end{readtabbed}
			
			\begin{readtabbed}Alzaré la copa de la salvación, \end{readtabbed}
			
			\begin{readtabbed}invocando el nombre del Señor. \begin{readred}℟.\end{readred}\end{readtabbed}
			
			\begin{readps}\begin{readred}℣.\end{readred} Mucho le cuesta al Señor \end{readps}
			
			\begin{readtabbed}la muerte de sus fieles. \end{readtabbed}
			
			\begin{readtabbed}Señor, yo soy tu siervo, \end{readtabbed}
			
			\begin{readtabbed}hijo de tu esclava: \end{readtabbed}
			
			\begin{readtabbed}rompiste mis cadenas. \begin{readred}℟.\end{readred}\end{readtabbed}
			
			\begin{readps}\begin{readred}℣.\end{readred} Te ofreceré un sacrificio de alabanza, \end{readps}
			
			\begin{readtabbed}invocando el nombre del Señor. \end{readtabbed}
			
			\begin{readtabbed}Cumpliré al Señor mis votos \end{readtabbed}
			
			\begin{readtabbed}en presencia de todo el pueblo. \begin{readred}℟.\end{readred}\end{readtabbed}
			
			\begin{readtitle}SEGUNDA LECTURA\end{readtitle}
			
			\begin{readbook}De la primera carta del apóstol san Pablo a los Corintios \rightline{11, 23-26}\end{readbook}
			
			\begin{readtheme}Cada vez que coméis de este pan y bebéis del cáliz, proclamáis la muerte del Señor\end{readtheme}
			
			\begin{readbody}Hermanos: \end{readbody}
			
			\begin{readbody}Yo he recibido una tradición, que procede del Señor y que a mi vez os he transmitido: Que el Señor Jesús, en la noche en que iba a ser entregado, tomó pan y, pronunciando la Acción de Gracias, lo partió y dijo: \end{readbody}
			
			\begin{readtalk}“Esto es mi cuerpo, que se entrega por vosotros. Haced esto en memoria mía”. \end{readtalk}
			
			\begin{readbody}Lo mismo hizo con el cáliz, después de cenar, diciendo: \end{readbody}
			
			\begin{readtalk}“Este cáliz es la nueva alianza en mi sangre; haced esto cada vez que lo bebáis, en memoria mía”. \end{readtalk}
			
			\begin{readbody}Por eso, cada vez que coméis de este pan y bebéis del cáliz, proclamáis la muerte del Señor, hasta que vuelva.\end{readbody}
			
			\begin{readtitle}EVANGELIO\end{readtitle}
			
			\begin{readbook}Del Evangelio según san Juan \rightline{13, 1-15}\end{readbook}
			
			\begin{readtheme}Los amó hasta el extremo\end{readtheme}
			
			\begin{readbody}Antes de la fiesta de la Pascua, sabiendo Jesús que había llegado su hora de pasar de este mundo al Padre, habiendo amado a los suyos que estaban en el mundo, los amó hasta el extremo. \end{readbody}
			
			\begin{readbody}Estaban cenando; ya el diablo había suscitado en el corazón de Judas, hijo de Simón Iscariote, la intención de entregarlo; y Jesús, sabiendo que el Padre había puesto todo en sus manos, que venía de Dios y a Dios volvía, se levanta de la cena, se quita el manto y, tomando una toalla, se la ciñe; luego echa agua en la jofaina y se pone a lavarles los pies a los discípulos, secándoselos con la toalla que se había ceñido. \end{readbody}
			
			\begin{readbody}Llegó a Simón Pedro y este le dice: \end{readbody}
			
			\begin{readtalk}“Señor, ¿lavarme los pies tú a mí?”. \end{readtalk}
			
			\begin{readbody}Jesús le replicó: \end{readbody}
			
			\begin{readtalk}“Lo que yo hago, tú no lo entiendes ahora, pero lo comprenderás más tarde”. \end{readtalk}
			
			\begin{readbody}Pedro le dice: \end{readbody}
			
			\begin{readtalk}“No me lavarás los pies jamás”. \end{readtalk}
			
			\begin{readbody}Jesús le contestó: \end{readbody}
			
			\begin{readtalk}“Si no te lavo, no tienes parte conmigo”. \end{readtalk}
			
			\begin{readbody}Simón Pedro le dice: \end{readbody}
			
			\begin{readtalk}“Señor, no solo los pies, sino también las manos y la cabeza”. \end{readtalk}
			
			\begin{readbody}Jesús le dice: \end{readbody}
			
			\begin{readtalk}“Uno que se ha bañado no necesita lavarse más que los pies, porque todo él está limpio. También vosotros estáis limpios, aunque no todos”. \end{readtalk}
			
			\begin{readbody}Porque sabía quién lo iba a entregar, por eso dijo: “No todos estáis limpios”. \end{readbody}
			
			\begin{readbody}Cuando acabó de lavarles los pies, tomó el manto, se lo puso otra vez y les dijo: \end{readbody}
			
			\begin{readtalk}“¿Comprendéis lo que he hecho con vosotros? Vosotros me llamáis ‘el Maestro’ y ‘el Señor’, y decís bien, porque lo soy. Pues si yo, el Maestro y el Señor, os he lavado los pies, también vosotros debéis lavaros los pies unos a otros: os he dado ejemplo para que lo que yo he hecho con vosotros, vosotros también lo hagáis”.\end{readtalk}
			
			\section{Comentario Patrístico}
			
			\subsection{<a id="_idTextAnchor002"></a>Francisco, papa}
			
			\begin{patertheme}Amor concreto\end{patertheme}
			
			\begin{patersource}Audiencia jubilar, 12 de marzo de 2016.\end{patersource}
			
			\begin{body}Nos estamos acercando a la fiesta de Pascua, misterio central de nuestra fe. El evangelio de Juan –como hemos escuchado– narra que antes de morir y resucitar por nosotros, Jesús realizó un gesto que quedó esculpido en la memoria de los discípulos: el lavatorio de los pies. Un gesto inesperado y sorprendente, al punto que Pedro no quería aceptarlo. Quisiera detenerme en las palabras finales de Jesús: “¿Comprendéis lo que he hecho con vosotros? […] Pues si yo, el Señor y el Maestro os he lavado los pies, vosotros también deberéis lavaros los pies unos a los otros” (\textit{Jn }13, 12. 14). De este modo Jesús le indica a sus discípulos\textit{ el servicio} como el camino que es necesario recorrer para vivir la fe en Él y dar testimonio de su amor. El mismo Jesús ha aplicado a sí la imagen del “Siervo de Dios” utilizada por el profeta Isaías. ¡Él que es el Señor, se hace siervo!\end{body}
			
			\begin{body}Lavando los pies a los apóstoles, Jesús quiso revelar el modo de actuar de Dios en relación a nosotros, y dar el ejemplo de su “mandamiento nuevo” \textit{(Jn}\textbf{ }13, 34) de amarnos los unos a los otros como Él nos ha amado, o sea dando la vida por nosotros. El mismo Juan lo escribe en su Primera Carta: “En esto hemos conocido lo que es el amor: en que él dio su vida por nosotros. También nosotros debemos dar la vida por los hermanos […] Hijos míos, no amemos de palabras ni de boca, sino con obras y según la verdad” (3, 16.18).\end{body}
			
			\begin{body}El amor, por lo tanto, es el \textit{servicio concreto} que nos damos los unos a los otros. El amor no son palabras, son obras y servicio; un servicio \textit{humilde}, hecho en el\textit{ silencio} y \textit{escondido}, como Jesús mismo dijo: “Que no sepa tu mano izquierda lo que hace tu derecha” (\textit{Mt} 6, 3). Esto comporta poner a disposición los dones que el Espíritu Santo nos ha dado, para que la comunidad pueda crecer (cf.\textit{ 1 Cor} 12, 4-11). Además se expresa en el \textit{compartir} los bienes materiales, para que nadie tenga necesidad. Este gesto de\textit{ compartir }y de dedicarse a los necesitados es un estilo de vida que Dios sugiere también a muchos no cristianos, como un camino de auténtica humanidad.\end{body}
			
			\begin{body}Por último, no nos olvidemos que lavando los pies a los discípulos y pidiéndoles que hagan lo mismo, Jesús también nos ha invitado a confesarnos mutuamente nuestras faltas y a rezar los unos por los otros, para saber perdonarnos de corazón. En este sentido, recordamos las palabras del santo obispo Agustín cuando escribía: “No desdeñe el cristiano hacer lo que hizo Cristo. Porque cuando el cuerpo se inclina hasta los pies del hermano, también el corazón se enciende, o si ya estaba se alimenta el sentimiento de humildad […] Perdonémonos mutuamente nuestros errores y recemos mutuamente por nuestras culpas y así de algún modo nos lavaremos los pies mutuamente” (\textit{In Ion }58, 4-5). El amor, la caridad es el servicio, ayudar a los demás, servir a los demás. Hay mucha gente que pasa la vida así, sirviendo a los otros. La semana pasada recibí una carta de una persona que me agradecía por el Año de la Misericordia; me pedía rezar por ella, para que pudiera estar más cerca del Señor. La vida de esta persona es cuidar a la mamá y al hermano: la mamá en cama, anciana, lúcida pero no se puede mover y el hermano es discapacitado, en una silla de ruedas. Esta persona, su vida es servir, ayudar. ¡Y esto es amor! ¡Cuando te olvidas de ti mismo y piensas en los demás, esto es amor! Y con el lavatorio de los pies el Señor nos enseña a ser servidores, más aún: siervos, como Él ha sido siervo para nosotros, para cada uno de nosotros.\end{body}
			
			\begin{body}Por lo tanto, queridos hermanos y hermanas, \textit{ser misericordiosos como el Padre, significa seguir a Jesús en el camino del servicio}.\end{body}
			
			\section{Homilías}
			
			\subsection{<a id="_idTextAnchor003"></a>San Pablo VI, papa}
			
			\subsubsection{Homilía (1964): Unidad mística y humana}
			
			\begin{referencia}Basílica de San Juan de Letrán. Jueves Santo 26 de marzo de 1964.\end{referencia}
			
			\begin{body} Nos mismo hemos querido celebrar este rito “in coena Domini” porque hemos sido solicitados por la invitación, por el impulso de la reciente Constitución del Concilio Ecuménico sobre la Sagrada Liturgia, decididamente dirigida a aproximar las estructuras jerárquicas y comunitarias de la Iglesia lo más posible al ejercicio del culto, a la celebración, a la comprensión, al gozo de los sagrados misterios, contenidos en la oración sacramental y oficial de la Iglesia misma. Si todo sacerdote, como cabeza de una comunidad de fieles, si todo obispo, consciente de ser el centro operante y santificador de una Iglesia, desea, pudiéndolo, celebrar personalmente la santa misa del Jueves Santo, día memorable en que la santa misa fue celebrada por primera vez e instituida por el mismo Cristo para que lo fuese luego por los elegidos para ejercer su sacerdocio, ¿no debería el Papa, dichoso de tener esta oportunidad, realizar él mismo el rito en la conmemoración anual, que evoca su origen, medita su típica institución, exalta con sencillez pero con toda la posible e inefable interioridad su santísimo significado, y adora la oculta pero cierta presencia de Cristo santificador mismo para nuestra salvación?\end{body}
			
			\begin{body}Si quisiéramos justificar con otros motivos este propósito nuestro no tendríamos dificultades en encontrar muchos y excelentes; dos, por ejemplo, que pueden contribuir a hacer más piadosa y gozosa nuestra presente celebración; Nos sugiere el primero el múltiple movimiento, que fermenta en tantas formas diversas en el seno de nuestra sociedad contemporánea, y la estimula, aún a su pesar, a expresiones uniformes primero y unitarias después; el pensamiento humano, la cultura, la acción, la política, la vida social, la económica también –de por sí particular y que tiende al interés que distingue y opone a cada uno de los interesados–, están encaminados a una convergencia unificadora; el progreso lo exige y la paz se encuentra allí y de todo aquello tiene necesidad.\end{body}
			
			\begin{body}Pero el misterio que nosotros celebramos esta tarde es un misterio de unificación, de unidad mística y humana; bien lo sabemos; y aunque se realiza en una esfera distinta de la temporal, no prescinde, no ignora, no descuida la socialidad humana en el acto mismo que la supone, la cultiva, la conforta, la sublima, cuando este misterio, que justamente llamamos comunión, nos pone en inefable sociedad con Cristo, y mediante Él en sociedad con Dios y en sociedad con los hermanos mediante relaciones diversas, según sean o no partícipes con nosotros de la mesa que juntamente nos une, de la fe que unifica nuestros espíritus, de la caridad que nos compagina en un solo cuerpo, el Cuerpo místico de Cristo.\end{body}
			
			\begin{body}El segundo motivo, que sí hace referencia, como decíamos, a todo sacerdote, a todo obispo, respecta principalmente a Nos, a nuestra persona y a nuestra misión que Cristo quiere poner en el corazón de la unidad de toda la Iglesia católica, y ennoblecerla con un título, impuesto por un Padre, desde los albores de la historia eclesiástica, de “presidente de la caridad”. Creemos nos incumbe el grande y grave oficio de recapitular aquí la historia humana anudada como a su luz y salvación, al sacrificio de Cristo, sacrificio que aquí se refleja y de modo incruento se renueva; nos toca atender una mesa a la que están invitados místicamente todos los obispos, todos los sacerdotes, todos los fieles de la tierra; aquí se celebra la hermandad de todos los hijos de la Iglesia católica; aquí está la fuente de la socialidad cristiana, convocada a sus principios constitutivos trascendentes y socorridas por energías alimentadas, no por intereses terrenos, que son siempre de funcionamiento ambiguo, ni por cálculos políticos, siempre de efímera consistencia, ni por ambiciones imperialistas o dictámenes coercitivos, ni siquiera por el sueño noble e ideal de la concordia universal, que puede, a lo más, intentar el hombre, pero que no sabe realizar ni conservar; por energías, decimos, potenciadas por una corriente superior, divina, por la corriente, por la urgencia de la caridad, que Cristo nos ha conseguido de Dios y hace circular en nosotros, para ayudarnos a “ser una sola cosa” como lo es Él con el Padre.\end{body}
			
			\begin{body}Hermanos e hijos míos; ni las palabras ni el tiempo son suficientes para decirnos a nosotros mismos la plenitud de este momento; esta es la celebración de uno y de muchos, la escuela del amor superior de los unos para con los otros, la profesión de la estima mutua, la alianza de la colaboración recíproca, el empeño del servicio gratuito, la razón de la sabia tolerancia, el precepto del perdón mutuo, la fuente del gozo por la fortuna de los demás, y del dolor por la desventura ajena, el estímulo para preferir el dar dones a recibirlos, la fuente de la verdadera amistad, el arte de gobernar sirviendo y de obedecer queriendo, la formación en las relaciones corteses y sinceras con los hombres, la defensa del respeto y veneración a la personalidad, la armonía de los espíritus libres y dóciles, la comunión de las almas, la caridad. Leíamos, estos días, unas tristes palabras de un escrito contemporáneo, profeta del mundo sin amor y del egoísmo proclamado libertador: “Yo no quiero comunión de almas...”. El cristianismo no es así, está en los antípodas. Nos quisiéramos construir, bajo los auspicios de Cristo, una comunión de almas, la comunión más grande posible.\end{body}
			
			\begin{body}Digamos, por tanto a nuestros sacerdotes, ante todo, las palabras sacrosantas del Jueves Santo: “Amémonos los unos a los otros como Cristo nos ha amado”. ¿Puede haber un programa más grande, más sencillo, más innovador de nuestra vida eclesiástica?\end{body}
			
			\begin{body}Dirijimos a vosotros, fieles, que formáis un coro en torno a este altar, y a todos vosotros distribuidos en el inmenso círculo de la santa Iglesia de Dios, otras palabras igualmente pronunciadas por Cristo el Jueves Santo; recordad que éste ha de ser el signo distintivo a los ojos del mundo de vuestra cualidad de discípulos de Cristo, el amor mutuo. “En esto todos conocerán...”.\end{body}
			
			\begin{body}Diremos a cuantos pueda llegar el eco de nuestra celebración de la cena pascual, en la fe de Cristo y en su caridad, las palabras del Apóstol Pedro: “Complaceos en ser hermanos” (\textit{1 P} 2, 17). Por este motivo confirmamos aquí también nuestro propósito al Señor, de conducir a buen término el Concilio Ecuménico, como un gran acontecimiento de caridad en la Iglesia, dando a la colegialidad episcopal el significado y el valor que Cristo pretendió conferir a sus apóstoles en la comunión y en el obsequio al primero de ellos, Pedro, promoviendo todos los propósitos encaminados a aumentar en la Iglesia de Dios la caridad, la colaboración, la confianza.\end{body}
			
			\begin{body}\begin{bodysmall}[También con este sentimiento de caridad en el corazón saludamos desde esta Basílica, cabeza y madre de todas las Iglesias, a todos los hermanos cristianos, por desgracia aún separados de nosotros, pero pretendiendo buscar la unidad querida por Cristo para su única Iglesia. Enviamos nuestro saludo pascual, el primero quizás en ocasión tan sagrada como ésta, a las Iglesias orientales ahora separadas de Nos, pero a Nos muy ligadas en la fe; salud y paz pascual para el patriarca ecuménico Atenágoras, por Nos abrazado en Jerusalén en la fiesta latina de la Epifanía; paz y salud a los demás patriarcas saludados por Nos, en la misma ocasión; paz y salud a los demás jerarcas de aquellas antiguas y venerables Iglesias, que han mandado sus representantes al Concilio Ecuménico Vaticano; paz y salud también a todos cuantos esperamos encontrar confiados un día en el abrazo de Cristo.\end{bodysmall}\end{body}
			
			\begin{body}\begin{bodysmall}Salud y paz a toda la Iglesia anglicana, mientras que con sincera caridad y con la misma esperanza deseamos poder un día verla unida honrosamente en el único y universal redil de Cristo. Salud y paz a todas las demás comunidades cristianas procedentes de la reforma del siglo XVI, que las separó de nosotros. Que la virtud de la Pascua de Cristo indique el justo y quizás largo camino para renovarnos en la perfecta comunión, mientras que ya buscamos con mutua estima y respeto cómo abreviar las distancias y cómo practicar la caridad, que esperamos un día verdaderamente victoriosa.\end{bodysmall}\end{body}
			
			\begin{body}\begin{bodysmall}Mandamos también un saludo cordial, de reconocimiento, a los creyentes en Dios de una y otra confesión religiosa no cristiana, que acogieron con festiva reverencia nuestra peregrinación a los Santos Lugares. También pensamos en estos momentos en toda la humanidad, estimulados por la caridad de Aquél que amó de tal forma al mundo que por él dio su vida. Nuestro corazón adquiere las dimensiones del mundo; ojalá adquiriera las infinitas del corazón de Cristo. Y vosotros, hermanos e hijos y fieles, estáis aquí presentes, ciertamente para celebrar con Nos el Jueves Santo, el día de la caridad consumada y perpetuada de Cristo por nuestra salvación.]\end{bodysmall}\end{body}
			
			\subsubsection{Homilía (1967): Eucaristía, síntesis de nuestra fe}
			
			\begin{referencia}23 de marzo de 1967. \end{referencia}
			
			\begin{bodycenter}\textbf{Hoy celebramos con más vivo fervor el }“Mysterium Fide”\end{bodycenter}
			
			\begin{body}¡Venerables hermanos y queridos hijos!\end{body}
			
			\begin{body}Si hay un momento de nuestra vida espiritual, de nuestra profesión cristiana, de nuestra pertenencia a la Iglesia, en el que debe estar comprometida nuestra atención, nuestra conciencia, nuestro fervor, es este momento. Un momento tremendamente hermoso y significativo, pero igualmente intenso y difícil, contrario a nuestra distracción habitual. Es un momento de atracción hacia una Realidad presente y misteriosa, que compromete nuestras facultades espirituales a una concentración singular. Entramos en el misterio. Es necesario que seamos iniciados. Simplemente decimos: es necesario que seamos creyentes. Nos acercamos, de hecho celebramos el “\textit{mysterium fidei}”. Necesitamos ese complemento cognitivo, esa virtud intelectual, sostenida por la buena voluntad e iluminada por el Espíritu Santo, que se llama fe, para entrar en el secreto de la Realidad, que hoy está preparada para nosotros y recibir algún goce vital de ella. ¿Por qué hoy, y no siempre, cuando celebramos los misterios divinos? Siempre, respondemos sin cuestionar; pero hoy con mayor intensidad, porque el sacrificio divino de la Misa, que celebramos otros días, deriva y se refiere a esto. Aquí está el misterio pascual, tal cual nos es dado para recordarlo y revivirlo; y cada vez que renovamos la oblación litúrgica celebramos este mismo misterio pascual.\end{body}
			
			\begin{bodycenter}\textbf{Cristo Jesús mediador en las relaciones entre Dios y el hombre }\end{bodycenter}
			
			\begin{body}Y habiendo entrado así en el cenáculo de las supremas comunicaciones divinas, debemos quedarnos callados y extáticos, como quien ve demasiado y solo entiende algo; y con temor deberíamos sentir al menos esto: que en la Cena del Señor, como nudo central, convergen los hilos de la historia antigua de la Salvación, porque la Pascua judía deposita allí sus símbolos proféticos, que aquí disuelven sus secretos y se transfunden en la nueva forma, también simbólica y profética, pero sustanciada por una Realidad muy diferente, a través de la cual se forma el perenne memorial de nuestra redención cumplida con el Sacrificio de la Cruz y la Resurrección gloriosa, y se nos da para participar de su virtud y tener su promesa; para que de la misma cena del Señor parta otro manojo de nuevos hilos, que invaden el mundo y la historia, y por cada ser vivo se ramifican y llegan, si queremos, a cada uno de nosotros. El lenguaje bíblico es más claro que cualquiera de nuestros discursos: el Antiguo Testamento y el Nuevo Testamento se tocan allí, y el uno al otro cede las intenciones divinas, de hecho las intervenciones divinas en el diseño sublime y formidable de las relaciones entre Dios y el hombre, siendo mediador, aquí plenamente Cristo Jesús. Océanos de verdad, y por tanto de doctrina, se abren ante nosotros: la Eucaristía, sabéis, hermanos e hijos aquí presentes, es la síntesis de nuestra fe; y por eso, después de haber hecho un esfuerzo de conciencia religiosa por abstraer nuestro espíritu de todo interés circundante y diferente para fijar la mente y el corazón en el punto focal, al que se dirige esta celebración tan especial, nos sentimos obligados a revisar, a la nueva luz de este mismo punto focal, todo: el mundo, la historia, la vida, nosotros mismos. Demasiado, demasiado, quisiéramos exclamar, y con la voz de los santos más comprensivos a nosotros también nos gustaría tartamudear: \textit{satis, Domine}, basta, Señor, basta. Esto requiere que nos contentemos ahora con un solo pensamiento entre los muchos posibles, y que mantengamos nuestra atención por un momento en uno de los aspectos esenciales del misterio del Jueves Santo, aquel sobre el cual queremos unir el pensamiento y la oración de esta santa asamblea. \end{body}
			
			\begin{bodycenter}\textbf{La sublime realidad, más allá de todo obstáculo del orden natural }\end{bodycenter}
			
			\begin{body}¿Qué aspecto? El intencional, el final, el de la “comunión”. Como quien es experto en ciertas prodigiosas técnicas modernas y sabe utilizar ciertos instrumentos mágicos, victorioso en el tiempo y el espacio, y sabe relacionarse con sensatez con escenas y palabras muy lejanas y esquivas de nuestra percepción inmediata, así nosotros, entrando con la fe y con el amor en el sistema sacramental concebido por Cristo e instituido por él, esto es, puesto en práctica por él en la misma noche en que fue traicionado, “\textit{in qua nocte tradebatur”} (\textit{1 Cor} 11, 23), podemos ponernos en contacto con él. Cristo, sobrevolando, en virtud de su Palabra, leyes y obstáculos de orden natural, insuperables en sí mismos, y “comulgando”, como solemos decir; hacer Pascua. La Eucaristía es el sacramento de la permanencia de Cristo, que vive ahora en la gloria eterna del Padre, en nuestro tiempo, en nuestra historia, en nuestra peregrinación terrena. “\textit{Vobiscum sum}”, estoy con vosotros, dirá Jesús cerrando la escena del Evangelio, y cumplirá su promesa. La Eucaristía es el sacramento de su presencia viva, real y sustancial, en todas partes; en todas partes está su ministro que hace lo que Él ha hecho, en su memoria. “Haced esto –dijo Jesús aquella tarde, instituyendo junto con la Eucaristía el sacramento del Orden Sagrado, instrumento humano autorizado, para renovar su misterio y difundirlo por toda la tierra– haced esto en memoria mía” (\textit{Lc} 22, 19). La Eucaristía es el sacramento que multiplica, que universaliza la presencia y la acción de Jesús: como una misma palabra puede ser escuchada por muchos y adquirir lógica eficacia en quien la oye y comprende, así el Señor, a través de la Eucaristía, se vuelve accesible a todos los que le acogen bajo este signo. La Eucaristía es Cristo para cada uno de nosotros, revestido precisamente de las apariencias del pan para decir que está dispuesto a saciar nuestra hambre, a ser deseado, abordado, asumido, asimilado en sí mismo. La Eucaristía es la figura de Cristo sacrificado por nosotros, para que nos sea posible y urgente recordar para siempre su Pasión, participar de su drama sacrificial y obtener su eficacia redentora. Lo decimos para que nos quede clara la intención global de Cristo: unirse a nosotros, admitirnos en su comunión. No es posible hacerse una idea de esto sin admitir un amor excesivo, un amor infinito que se proyecta en cada uno de nosotros y que no nos da paz hasta que de nuestro árido corazón también brota algún entendimiento, alguna correspondencia. La Eucaristía es una escuela de amor; y, para poner nuestras almas en sintonía con la corriente ardiente y abrumadora de su caridad, hay que decir al menos con el Apóstol, que en esa bendita y trágica tarde del Jueves Santo puso su oído en el pecho de Cristo y escuchó los latidos de su corazón: sí, “hemos creído en la caridad” (\textit{1 Jn} 4, 16). Y aquí se perfecciona la nueva vida espiritual, interior, de todo aquel que así ha entrado en comunión con Cristo. \end{body}
			
			\begin{body}Pero esto no es todo. La gracia que nos ofrece la Eucaristía no es sólo en relación con la comunión con Cristo; otra comunión resulta de este sacramento; y es comunión con todos aquellos hermanos en la fe y en la caridad que están sentados a la misma mesa. Las palabras de \textbf{San Pablo} son muy conocidas, pero siempre memorables: “Hablo a personas inteligentes; juzga por ti mismo lo que digo. La copa de bendición que bendecimos, ¿no es una comunión de la sangre de Cristo? ¿No es el pan que partimos comunión con el cuerpo de Cristo? Porque el pan es uno, formamos un solo cuerpo, aunque somos muchos cuando todos compartimos ese único pan” (\textit{1 Cor} 10, 15-17).\end{body}
			
			\begin{bodycenter}\textbf{Comunión con los hermanos, }\end{bodycenter}
			
			\begin{bodycenter}\textbf{en especial con los que sufren por el Señor} \end{bodycenter}
			
			\begin{body}Y he aquí, amados hermanos e hijos, que la realidad profunda y sobrenatural del misterio pascual nos devuelve a la realidad, mística sí, pero también visible y experimental, de la sociedad nacida de Cristo, su cuerpo místico, la Iglesia (cf. \textit{S. Th.} III, 73, 3), que nos gustaría que sea inundada, precisamente en virtud de este Jueves Santo, por la gracia propia de este día bendito, la gracia de la comunión, la gracia de la unidad, con Cristo y consigo misma; y para ello os pedimos a todos el aporte de vuestras oraciones, de vuestra colaboración espiritual. \end{body}
			
			\begin{body}[...] La unidad tiene diferentes grados: puede ser superficial y formal, sufrida y no amada, habitual e inoperante; y puede ser profunda y cordial, convencida y operante, toda impregnada de caridad mutua y santificante: esta unidad, viviendo en la fe y el amor a Cristo y en la fraternidad sincera, queremos sea infundida en Nuestra Ciudad, (...) queremos que Cristo sea su maestro, su salvador, su ciudadano. \end{body}
			
			\begin{body}Y agregamos un voto similar para toda la Iglesia Católica. Pensamos en este momento en toda nuestra gran fraternidad que en esta tarde, esparcida por toda la tierra, realiza el mismo rito pascual con igual sentimiento; pensemos en aquellas comunidades, impedidas o mortificadas, donde continúa la Pasión del Señor; pensemos en las Iglesias jóvenes de países en territorios de misión; y a toda esta inmensa y amada comunión enviamos nuestros saludos de bendición: ave, una, santa, católica y apostólica Iglesia; ave, Iglesia viva de Cristo: todos en él hoy somos uno. \end{body}
			
			\begin{bodycenter}\textbf{Saludos y deseos a los asistentes }\end{bodycenter}
			
			\begin{body}Y no nos olvidemos de las muchas Iglesias y comunidades cristianas, a las que nos une el mismo bautismo y tantos lazos de fe y de amor al único Cristo Señor, y con las que aún no podemos gozar de una perfecta comunión. Esto es lo que deseamos, esperamos e invocamos, mientras enviamos a todas y cada una desde Nuestra Catedral, llena del fiel recuerdo y presencia mística de Cristo Salvador, Nuestro mensaje de caridad pascual. \end{body}
			
			\begin{body}[...]\end{body}
			
			\subsubsection{Homilía (1970): Fuente del amor, hasta la muerte}
			
			\begin{referencia}Archibasílica Laterana, 26 de marzo de 1970.\end{referencia}
			
			\begin{body}Obligados por nuestro ministerio a abrir los labios en este lugar sagrado, “magnum stratum”, amplio y ornamentado, cenáculo por excelencia de la Iglesia católica y romana, y en este momento, entre todos, int<a id="_idTextAnchor006"></a>enso de sentimientos y pensamientos religiosos y humanos, si bien nos agradaría escuchar en silencio interior las grandes voces que surgen de la sublime liturgia que estamos celebrando, ofreceremos a vuestra benevolente atención algunas indicaciones elementales, que servirán para estimular nuestra reflexión sobre los aspectos obvios y fundamentales de este rito y poner nuestras almas en armonía en un coro espiritual común. \end{body}
			
			\begin{bodycenter}\textbf{Plenitud de la comunión eclesial }\end{bodycenter}
			
			\begin{body}Y el primer indicio es precisamente este relativo a la comunión eclesial, que nos reúne aquí y que ahora adquiere una plenitud singular, un sentido propio. Este es un momento particular de comunión entre nosotros, entre aquellos que han aceptado nuestra invitación y nos han regalado su presencia. Si alguna vez se nos ofrece una ocasión feliz para realizar las palabras del Señor: “Dondequiera que dos o tres personas se reúnan en mi nombre, yo estoy entre ellos” (\textit{Mt 18, 20}), esto es para nosotros, mientras que precisamente este su nombre, y sólo su nombre, polariza nuestra asistencia, y emerge entre nosotros, como si estuviera aquí ahora y pronto lo estará sacramentalmente, y de ahora en adelante llena nuestras almas de sí mismo, y las une en la fe, en la armonía, en la paz, en la alegría de conocernos y de sentirnos “iglesia”, es decir, unión, su único redil, su cuerpo místico. Que cada distancia entre nosotros, cada desconfianza, cada descuido, cada extrañeza caiga en este momento; que caiga todo rencor, toda rivalidad; y que cada uno trate de experimentar “qué hermoso y qué gozo es que los hermanos estén juntos” (\textit{Sal} 132, 1); y que cada uno sienta en sí mismo que tener la suerte de ser, como la primera comunidad de creyentes, “un solo corazón y una sola alma” (\textit{Hch }4, 32) significa realizar nuestra exigente calificación de cristianos católicos. Caridad \textit{dentro de} la Iglesia, caridad que la reúne y la compone, caridad que especifica su “cuerpo místico” y hace hermanos a todos los que aceptan su socialidad organizada (\textit{Mt }23, 8; \textit{Lc} 10, 16), la caridad humilde, amigable y solidaria entre los fieles y seguidores y ministros de Cristo es el primer requisito exigente para sentarse a la mesa el Jueves Santo (Cf. \textit{Lc} 22, 24 ss.). \end{body}
			
			\begin{body}Juntos, por tanto, más que nunca, vivimos esta hora fugaz. Pero, ¿cuál es el propósito, cuál es la intención? ¿Por qué estamos reunidos aquí? He aquí, pues, una segunda indicación Nuestra, también muy conocida. Estamos aquí para una conmemoración. Este es un rito de la memoria. Una Misa es siempre así, pero en este día queremos resaltar su carácter conmemorativo. Celebramos la memoria del Señor, obedeciendo a sus palabras, que podemos decir testamentarias: “Haced esto en memoria mía” (\textit{Lc} 22, 19; 1 \textit{Cor} 11, 25). Todo nuestro espíritu está ahora lleno del recuerdo de él, de Jesús: nos gustaría poder traerlo a nuestra imaginación, cómo era, cómo era su figura, su rostro, cómo era el sonido de su voz, la luz de sus ojos, los gestos de sus manos... No nos ha llegado ninguna imagen sensible de él; pensamos con asombro en aquella imagen tan impresionante y profunda de la Sábana Santa; pensamos en la elección de nuestro genio por las piadosas efigies de los grandes artistas favoritos, por las descripciones de los sabios y los santos; pero siempre con el descontento propio de nosotros los modernos, incluso demasiado favorecido por la civilización de la imagen, porque la suya no se muestra a nuestra mirada, sino sólo a nuestro deseo escatológico: “¡Ven, Señor Jesús!” (\textit{Ap} 22, 20). ¡Nuestra memoria debe contentarse con otra presencia suya, la de su palabra! Entonces todo el Evangelio pasa ante nuestra mente, que sin embargo se detiene en esa palabra que Cristo pronunció en la cena de aquella noche y que recomendó a nuestra memoria. ¿Qué palabra? Oh, bien lo sabemos: “Tomad y comed: este es mi Cuerpo; tomad y bebed: esta es la copa de mi Sangre”. \end{body}
			
			\begin{bodycenter}\textbf{Presencia viva y real del Señor }\end{bodycenter}
			
			\begin{body}El banquete pascual, porque tal era aquella Cena ritual (Cfr. \textit{Lc }22, 7 y ss.), debía ser objeto de recuerdos inolvidables, pero desde un punto de vista nuevo, no ya del sacrificio y la comida del cordero, signo y prenda de la antigua alianza, sino del pan y del vino, transformados en el Cuerpo y la Sangre de Jesús. El ágape, en este punto, se convierte en misterio. La presencia del Señor se vuelve viva y real. Las apariencias sensibles siguen siendo lo que fueron, pan y vino; pero su sustancia, su realidad ha cambiado íntimamente; el pan y el vino permanecen sólo para significar lo que la palabra omnipotente, porque divina, de Jesús ha dicho que son: cuerpo y sangre. Ante esto quedamos asombrados. También porque este prodigio es precisamente lo que el Señor nos dijo que recordemos y que renovemos. Él dijo a los Apóstoles “haced esto”, es decir, les transmitió la virtud de repetir su acto consecratorio, y no sólo de repensarlo, sino de volverlo a hacer; el sacramento del Orden Sagrado, como custodia, como fuente del sacramento de la Eucaristía, fue instituido junto con este, en aquella noche. Quedamos asombrados e inmediatamente tentados: pero ¿es cierto? ¿Realmente es verdad ? ¿Cómo se explican esas sacrosantas sílabas de Cristo: este es mi cuerpo, esta es mi sangre? ¿Es posible encontrar una interpretación que no viole nuestra mentalidad elemental? ¿A nuestra reflexión metafísica habitual? También llega a nuestros labios el repulsivo comentario de los oyentes de Cafarnaum: “Este lenguaje es duro; y ¿quién podrá escucharlo?” (\textit{Jn }6, 61). Pero el Señor no admite dudas o exégesis elusivas de la auténtica realidad de sus palabras textuales; lo convierte en una cuestión de confianza; dejaría que el amado grupo de sus discípulos se dispersara, en lugar de eximirlos de adherirse a sus paradójicas pero verdaderas palabras, proponiéndoles en un lenguaje no menos duro: “¿Vosotros también queréis marcharos?” (\textit{Ibíd}. 68). \end{body}
			
			\begin{bodycenter}\textbf{La hora de la fe }\end{bodycenter}
			
			\begin{body}Entonces esta es una hora decisiva, la hora de la fe, la hora que acepta en su integridad, aunque sea incomprensible, la palabra de Jesús; la hora en la que celebramos el “misterio de la fe”, la hora en la que repetimos incluso con ciego y sabio abandono la respuesta de Simón Pedro: “Señor, ¿a quién iremos? Sólo tú tienes palabras de vida eterna. Hemos creído y conocido que eres el Cristo, el Hijo de Dios” (\textit{Jn} 6, 69-70). Sí, hermanos e hijos, esta es la hora de la fe, que absorbe y consume la oscura e inmensa nube de objeciones, que nuestra ignorancia por un lado, y la refinada dialéctica del pensamiento profano, por otro, acumulan sobre nuestro espíritu, que humilde y gozosamente se deja impresionar por la luminosa palabra del Maestro y le dice temblando como el implorador evangélico: “Creo, Señor; pero tú, ayuda mi incredulidad” (\textit{Mc }9, 24). \end{body}
			
			\begin{body}Y entonces la fe vuelve a preguntar: ¿pero qué significa esta forma de recordar al Señor? ¿Cuál es el significado, cuál es el valor de este memorial? ¿De este sacramento de la presencia, de este misterio de fe? ¿Cuál es la intención dominante del Señor, que quiso grabar en la memoria de sus seguidores en ese último encuentro cordial? \end{body}
			
			\begin{body}Hay quienes no hacen esta pregunta, como si no quisieran descubrir una verdad nueva y sorprendente. Pero no podemos detenernos sin recoger el último tesoro del testamento de Jesús, todo nos obliga a hacerlo, porque todo en esa última noche de su vida temporal es sumamente intencional y dramático: bastaría la observación de este aspecto de la Última Cena para nunca poner fin a nuestra meditación extática. La tensión espiritual casi te deja sin aliento. \end{body}
			
			\begin{body}La apariencia, la palabra, los gestos, los discursos del Maestro son exuberantes con la sensibilidad y profundidad de quien está cerca de la muerte; lo siente, lo ve, lo expresa. Dos notas resuenan por encima de las demás en esta atmósfera de asombro silenciada por los actos y presagios del Maestro: el amor y la muerte. El lavatorio de los pies, impresionante ejemplo de amor humilde, el mandato, el último y nuevo mandato: amaos los unos a los otros como yo os he amado; y esa angustia por la traición inminente, esa tristeza que se desprende de las palabras y de la actitud del Maestro, y esa efusión mística y encantadora de los discursos finales, casi soliloquios de Cristo desbordados de un corazón que se abre a confidencias extremas, todo está concentrado en la acción sacramental, recién mencionada: ¡cuerpo y sangre! Sí, allí se representan el amor y la muerte; una sola palabra los expresa: sacrificio. Allí se quiere decir muerte, muerte sangrienta, la muerte que habría separado su sangre del cuerpo de Cristo; una inmolación, una víctima. Es una víctima voluntaria, una víctima consciente, una víctima por amor. Entregada por nosotros. Para ser recordada como anunciadora de la muerte de Jesús, de su sacrificio para siempre, hasta que Él vuelva al fin del mundo (1 \textit{Co }11, 26). Cristo ha sellado en un rito, renovable por sus discípulos, hechos Apóstoles y Sacerdotes, la ofrenda de sí mismo al Padre, como víctima por nuestra salvación, por amor a nosotros. Esto es la Misa. Es el ejemplo, es la fuente del amor que se da hasta la muerte. \end{body}
			
			\begin{body}Es Jueves Santo, es lo que estamos recordando y celebrando. Es el corazón y el paradigma de la vida cristiana. Es el mandato, es la memoria, es la pasión, es la caridad de Cristo, que ha transmitido a su Iglesia; a nosotros, para que vivamos de Él, por Él y para Él (\textit{Jn }6, 57), y nos ofrezcamos en sacrificio por nuestros hermanos, por la salud del mundo (cf. \textit{Jn }12, 24 ss.), y un día resucitar en Él (cf. \textit{Jn }6, 54-58).\end{body}
			
			\subsubsection{Homilía (1973): Amor total}
			
			\begin{referencia}19 de abril de 1973.\end{referencia}
			
			\begin{body} \textit{Hermanos, }\end{body}
			
			\begin{body}Bienvenidos a esta ceremonia de Jueves Santo, a la que todos sentimos que debemos asistir con total adhesión. El hecho mismo de que lo celebremos en esta basílica, corazón de la Iglesia católica, y que estemos deliberadamente juntos, todos penetrados por el sentido interior de la solemnidad del rito, y deseosos de sumarnos a la participación en la comprensión de lo que hacemos, nos empuja a la búsqueda, casa ansiosa, ciertamente ferviente, de su signif<a id="_idTextAnchor007"></a>icado. \end{body}
			
			\begin{body}Lo diremos muy brevemente, centrando nuestra atención en unas palabras de Jesús, el invitado protagonista de aquella última cena. Él mismo dijo que para él era la última (\textit{Lc }22, 15-16), y lo hizo entender a lo largo de todos los discursos de ese íntimo y triste encuentro de convivencia, motivado por la celebración de la Pascua ritual judía (Cf. \textit{Jn 16, }5-7; etc.), que culminó, como sabemos, en las misteriosas palabras de la institución de la santísima Eucaristía, concluidas con aquellas palabras preceptivas e instituyentes de otro sacramento, el de las Sagradas Órdenes, generador ministerial de la Eucaristía misma: “Haced esto en memoria mía” (\textit{Lc} 22, 19; \textit{1 Cor} 11, 24-25), dijo Cristo. Es en virtud de estas palabras que nos reunimos aquí esta noche. Son palabras testamentarias. Serán verdaderas y efectivas hasta su última venida, al final del presente orden temporal, al final de los siglos: \textit{donec veniat}, hasta que Él, Jesús, vuelva, declara \textbf{san Pablo}. Es, por tanto, el acto memorial por excelencia que recordamos y repetimos en este momento, cumpliendo el precepto que lo hace perenne en el desarrollo de la historia; es la presencia del Señor que acompaña el camino de su Iglesia en el tiempo, en el “misterio de la fe”, que presupone la presencia real de Jesús en la envoltura sacramental, y requiere una comprensión obediente, una acogida en la fe de nuestra parte, el homenaje amoroso de nuestra calificada memoria. \end{body}
			
			\begin{body}Este esfuerzo de recuerdo es fundamental para nuestra celebración. La prodigiosa facultad de la memoria se ejerce como estímulo para nuestra capacidad receptiva a la Eucaristía. Ésta afecta a quienes la reciben por virtud propia \textit{ex opere operato}, pero su acción está orientada al ejercicio de nuestro recuerdo, es decir, a la acogida de Cristo recibido y meditado dentro de nosotros, a su permanencia personal, viva y real en nosotros, pero también conceptual y reflejado en nuestra mente, en nuestra psicología, en nuestro corazón, según nuestra actitud de asimilarlo, aceptarlo, amarlo, coincidir, por así decirlo, con él: \textit{donec formetur Christus in vobis}, hasta que Cristo sea formado en vosotros, dice San Pablo (\textit{Gal} 4, 19). Una intención fundamental de permanencia domina el misterio de la Eucaristía; es decir, de la permanencia de Jesús entre nosotros más allá del abismal límite de su pasión y muerte, de la verdadera permanencia, pero bajo el signo sacramental, que al quitarnos la alegría de su visión sensible, nos ofrece la seguridad de su presencia efectiva, y al mismo tiempo la otra inestimable ventaja de su indefinida y unívoca multiplicabilidad, en tiempos y lugares, lo que se necesita para saciar el hambre de quienes permanecerán en su fe y en su amor. Permanecer es la intención sacramental de la Eucaristía, en lo que respecta a Jesús; permanecer es la intención moral, en lo que respecta a nosotros, de quienes Jesús quiere ser el viático, el compañero, el sustento a lo largo de nuestra peregrinación en el tiempo: debemos, pues, permanecer en su amor. Veréis, en apoyo de esta afirmación, cuántas veces se repite la palabra “permanecer” en los discursos de Jesús en esa última cena (cf. especialmente \textit{Jn} 15).\end{body}
			
			\begin{body}Por eso, hermanos, hay algo que debemos reavivar en nuestras almas: “recordar” a Jesús, como él quiso ser; y he aquí, de este particular memorial nuestro, brota con ímpetu, es decir, con amorosa abundancia, nuestro culto eucarístico, al que la Iglesia nos invita y exhorta con incansable preocupación. \end{body}
			
			\begin{body}Entonces, limitando siempre nuestra búsqueda al sentido esencial de ese banquete pascual, con el que Cristo quiso despedirse de sus discípulos, no podremos pretender el paso de la figura del cordero a la realidad de la verdadera víctima de nuestra Pascua, que es el mismo Cristo, inmolado (Cfr. \textit{1 Co} 5, 7), paso operado por la institución de la Eucaristía, que en la figura del pan y del vino, representa y renueva de forma incruenta el sacrificio redentor de Jesús. ¿Cómo hablar en este breve momento de una teología tan elevada y dramática? Bienaventurados seamos si la deficiencia de nuestro discurso y aún más de nuestro pensamiento, después del acto de fe que hemos mencionado, son compensadas por el amor.\end{body}
			
			\begin{body}La Eucaristía es el punto privilegiado del encuentro del amor de Cristo por nosotros; un amor que se pone a disposición de cada uno de nosotros, un amor que se convierte en cordero de sacrificio y alimento de nuestra hambre de vida, un amor que se expresa en la forma y medida de su autenticidad específica, más alta y exclusiva, es decir, un amor que se da totalmente: \textit{dilexit me} –dice el Apóstol– \textit{et tradidit semetipsum} \textit{pro me}, él me amó y se sacrificó por mí (\textit{Gál }2, 20; \textit{Ef }5, 2; 5, 25); y del encuentro de nuestro amor pobre y vacilante por él, que en tanta caridad apremiante encuentra finalmente el atrevimiento de superar toda timidez, toda debilidad y responder con Pedro: “Señor... ¡Tú sabes que te quiero!” (\textit{Jn} 21, 15-17). El amor tendrá la suerte de penetrar algunas de sus intuiciones místicas y con algo de su anticipada plenitud (cf. \textit{Ef }3, 17, 19) en el misterio de la caridad, que supera todo entendimiento, el misterio del sacrificio eucarístico, y hundirse en él, participando en ese rito humilde e inconmensurable, que es nuestra santa Misa. \end{body}
			
			\begin{body}Hermanos, no os contamos más. Pero no concluiremos estas balbuceantes palabras sin confiaros que tenemos otra en nuestro corazón, también tomada de aquellas palabras inolvidables de la Cena del Señor, y es esta: “Os doy el mandamiento nuevo: amaos los unos a los otros, como yo os he amado a vosotros” (\textit{Jn} 13, 34; 15, 12). Ese “Yo” es Jesús, el Cristo, nuestro Señor; ese “vosotros” son los Apóstoles, todos los fieles que creyeron en él, “según su palabra” (\textit{Ibid}. 17, 20); somos nosotros, la Iglesia romana y la Iglesia católica, nosotros, hijos de la tierra y del siglo, los que hoy, Jueves Santo, debemos sentirnos todos impresionados por el amor crucificado y eucarístico de Cristo; y aún nos queda mucho por aprender a amarnos, según su ejemplo y precepto.\end{body}
			
			\subsubsection{Homilía (1976): Comunión}
			
			\begin{referencia}Jueves Santo, 15 de abril de 1976.\end{referencia}
			
			\begin{body} Comunión es la\textbf{ }palabra que sale de los labios, si se quiere romper el silencio de los corazones sobrecogidos por los misterios que estamos celebrando. Rememoramos, más aún revivimos la hora de la Última Cena de Jesús con sus discípulos; una hora ya seria por su significado conmemorativo, como para formar la conciencia religiosa e histórica del pueblo judío, que recordó, con el sacrificio del cordero, la experiencia del éxodo de la esclavitud a una patria a reconquistar y a poseer en fidelidad a su destino religioso, durante siglos. \end{body}
			
			\begin{body}La comunión fue la nueva atmósfera en la que se c<a id="_idTextAnchor008"></a>elebró aquella cena pascual: un ambiente afectivo e intenso cargado de esos sentimientos que van más allá del estilo habitual de conversación, aunque el lenguaje del Maestro siempre tuvo como objetivo llevar el entendimiento de sus discípulos más allá de los márgenes de la experiencia sensible, invitándoles a respirar en un área superior del misterio y a un descubrimiento trascendente de la verdad oculta y de la realidad divina. Pero esa noche, el nivel sentimental y espiritual era tan alto que a los discípulos de la cena les resulta más difícil que nunca hablar de ello. Mientras tanto, escuchemos los acentos sumamente cordiales, que son la clave para abrir el derroche discursivo del Maestro. “Cuando llegó el momento, –escribe el evangelista San Lucas–, tomó su lugar a la mesa y los apóstoles con él, y dijo: He deseado comer esta Pascua con vosotros, antes de mi pasión, ya que os digo: no la comeré más, hasta que se cumpla en el reino de Dios” (\textit{Lc} 22, 15). La Cena adquiere un carácter testamentario: el propio Gesto la define como el epílogo de su vida terrena; le da al banquete un carácter concluyente. El evangelista Juan, el amado iniciado en los secretos del corazón del Señor, escribe: “Antes de la fiesta de la Pascua, Jesús, sabiendo que había llegado su hora de pasar de este mundo al Padre, después de haber amado a los suyos que estaban en el mundo, los amó hasta el extremo” (\textit{Jn} 13, 1). San Agustín comenta: “El amor lo llevó a la muerte” (\textit{In Io. Tract}. 55, 2: \textit{PL} 35, 1786); y también la exégesis moderna: “Jesús, que siempre ha amado a los suyos, ahora demuestra su amor hasta el final, no sólo cronológicamente hasta el final de su vida, sino mucho más intensamente hasta el final alcanzable, hasta el límite extremo posible del amor mismo “(G. Ricciotti, \textit{Vida de Jesucristo}, 541).\end{body}
			
			\begin{body}El grado de intensidad emocional que producen las palabras y los actos de Jesús en ese banquete ritual, ya en sí mismo capaz de despertar en el espíritu una emoción fuerte y comunicativa, crece durante la vigilia convivial en escala ascendente: del anuncio de la próxima muerte sangrienta del Maestro (cf. \textit{Jn }11, 16; 12, 24; etc.) que llenó de temor a los discípulos, y ahora afirmada abiertamente, hasta la inesperada y vergonzosa escena del lavatorio de los pies, realizado por Jesús después de la primera parte de la cena (\textit{Jn }13, 2-17), y luego el patético y ahora abierto indicio de traición inminente; y luego, habiendo abandonado la mesa el presunto traidor (\textit{Ibid}. 13, 26 ss.), un momento de suprema despedida: “Hijos –¡así llama a los discípulos!–, todavía estoy un rato con vosotros... Os doy un mandamiento nuevo: que os améis unos a otros, como –como: fijaros en la comparación, fijaros en la medida–, como yo os he amado, así también amaros entre vosotros. De este modo todos sabrán que sois mis discípulos si os amáis los unos a los otros” (\textit{Ibid}. 13, 33-35). También aquí permanece una relación, una comunión, en el rasgo constitutivo de una sociedad compuesta de amor. Llegamos así al momento de la suprema y misteriosa sorpresa. Escuchemos las reveladoras palabras: “Mientras cenaban, Jesús tomó el pan y, habiendo dicho la bendición, lo partió y se lo dio a los discípulos diciendo: tomad y comed, esto es mi cuerpo. Luego tomó la copa y, después de dar gracias, se la dio, diciendo: bebed todos, porque esto es mi sangre de la alianza, derramada por muchos, para remisión de los pecados” (\textit{Mt} 26, 26-28). \end{body}
			
			\begin{body}¡Milagro! ¡Misterio de la fe! ¡Creemos en el milagro logrado! Creemos, como dice el Concilio de Trento, que Él, Cristo, “habiendo celebrado la antigua Pascua... instituyó una nueva Pascua, inmolándose, confiriendo poder a la Iglesia a través de los Sacerdotes, bajo signos visibles, en memoria de su paso de este mundo al Padre” (Denz-Schön, 1741). \end{body}
			
			\begin{body}Si esto es así, y así es, el misterio irradia ante nosotros, mientras podamos contemplarlo, una epifanía de comunión.\end{body}
			
			\begin{body}Comunión con Cristo, Sacerdote y víctima de un Sacrificio consumido de manera sangrienta en la cruz, incruenta en la Misa, cumbre de nuestra vida religiosa, donde él, a través de su palabra sacramental, redujo el pan y el vino a simples signos sensibles para convertir su sustancia en su carne y sangre, ofreciéndose a sí mismo, Cordero sacrificado en holocausto, restableciendo una comunión de gracia entre vivos y muertos, con Dios Padre todopoderoso y misericordioso (Cfr. Denz-Schön, 1743; 3847). Comunión ontológica, teológica, vital. \end{body}
			
			\begin{body}Comunión nuevamente con Cristo, personal, mística, interior; comunión bipolar de nuestra humilde y fugaz vida humana y mortal con la Vida misma de Cristo, que es la Vida por definición (\textit{Jn} 14, 6), y que dijo de sí mismo: “Yo soy el Pan de Vida” (\textit{Ibíd}. 6, 35-49. 51), para que resuenen en nuestra conciencia profunda las palabras de la comunión más íntima y coexistente: “Ya no vivo yo, sino que es Cristo quien vive en mí” (\textit{Gal} 2, 20). ¿Quién podrá medir la fecundidad de esta comunión interior, que tiene Cristo maestro, camino, verdad y vida (\textit{Jn} 14, 6), como la savia de un árbol hasta sus brotes florecientes y fructíferos? (\textit{Ibíd}. 15, 1 ss.)\end{body}
			
			\begin{body}Comunión también de inefable eficacia social, principio que es válido para cimentar en la unidad sobrenatural pero también eclesial y comunitaria del Cuerpo Místico de Cristo a quienes se nutren del pan eucarístico. San Pablo lo enseña de nuevo: “La copa de bendición que consagramos, ¿no es comunión con la sangre de Cristo? Y el pan que partimos, ¿no es comunión con el cuerpo de Cristo? Puesto que hay un solo pan, nosotros, aunque muchos, somos un solo cuerpo; de hecho, todos participamos de un solo pan” (1 \textit{Cor} 10, 16-17). \end{body}
			
			\begin{body}Comunión, pues, en el espacio de la tierra y en la dimensión de la humanidad creyente y de la participación en el banquete divino, dondequiera que se celebre regularmente: todos son invitados por el mismo Señor: \textit{compelle intrare,} ¡oblígalos a entrar! nos enseña la parábola del Evangelio (\textit{Lc} 14, 23). El hecho mismo de que Cristo hizo posible, a través del ministerio de los sacerdotes, multiplicar este bendito pan eucarístico, que es Él mismo, el Emmanuel, el Dios-con-nosotros que acompaña a los hombres en todos sus caminos y llama a todos con Voz pentecostal a su única Iglesia, ¿no hace evidente su divina intención de comunión universal a la más simple observación? \textit{¡Ut omnes unum sint}, para que todos sean uno! así Cristo oró en esa noche profética, después de la Última Cena.\end{body}
			
			\begin{body}Y ¿acaso a esta no se le suma otra comunión, aquella que se da en el tiempo, la de la permanencia de Jesucristo con nosotros, la de la tradición viva a lo largo de los siglos, comunión coherente, fiel, victoriosa del tiempo que pasa devorando, porque este milagro eucarístico está destinado, como escribe San Pablo, hasta el último \textit{donec veniat}, hasta que Él, Cristo, vuelva (1 \textit{Cor }11, 26), el último día de la Parusía? Y así lo declaró el mismo Cristo, como nos dicen las últimas palabras de su Evangelio: “He aquí que yo estoy con vosotros todos los días hasta el fin de los tiempos” (\textit{Mt} 28, 20). \end{body}
			
			\begin{body}En este punto nuestra meditación, que investiga la comunión polivalente resultante del misterio eucarístico, se vuelve curiosa por los cálculos y la estadística. Si Cristo es el centro, en el sacramento de su sacrificio, que atrae a todos hacia sí (cf. \textit{Jn} 12, 32), surge la pregunta: ¿están todos realmente fascinados y atraídos por esta comunión con él? ¿Cuántos estamos unidos en esa unidad que Cristo nos dejó como su aspiración testamentaria? (\textit{Ibid. }17). Y, ¿estamos verdaderamente en aquella unidad de fe, de amor y de vida que constituían el deseo soberano y misericordioso de Jesús? ¿Estamos dispuestos a hacer de la unidad al interior de la Iglesia y en la Iglesia nuestra aspiración constitutiva, nuestro programa de vida eclesial? ¿Es real y siempre un soplo del Espíritu Santo eso que a menudo frena y a veces rompe los lazos de nuestra bendita comunión en el cuerpo visible y místico de Cristo con empuje centrífugo y ambición individualista? ¿No es este el día, el momento de abandonar todas las reservas egoístas por la reconciliación fraterna, el perdón mutuo, la unidad del amor humilde? ¿Podemos hacer llegar a los hijos lejanos un afectuoso recuerdo de su regreso a la mesa espiritual común? ¡Qué fervor misionero surge en nosotros de la celebración de este Jueves Santo! ¡Qué espíritu fraterno, qué celo pastoral, qué propósito apostólico! ¡Qué esperanza de comunión cristiana!\end{body}
			
			\begin{body}¿Y no tendremos en esta bendita velada un pensamiento, un saludo, una oración ecuménica por tantos hermanos cristianos todavía separados de nosotros? \end{body}
			
			\begin{body}Y a todos aquellos que sufren o tienen hambre de verdad, justicia y paz, pero con los ojos nublados en su búsqueda insatisfecha, ¿no podremos recordarles, al menos en nuestra oración interior, la invitación que siempre les dirige Aquel que es el único que puede colmar sus deseos: “Venid a mí todos los que estáis cansados y agobiado, y yo os aliviaré”? (\textit{Mt }11, 28) ¡La Iglesia es una comunión!\end{body}
			
			\begin{body}Que así sea, que así sea, con nuestra cordial Bendición.\end{body}
			
			\subsection{San Juan Pablo II, papa}
			
			\subsubsection{Homilía (1979): ¿Qué significa amor hasta el fin?}
			
			\begin{referencia}12 de abril de 1979.\end{referencia}
			
			\begin{body}1. Ha llegado la “hora” de Jesús. Hora de su paso de este mundo al Padre. Comienza el triduo sacro. El misterio pascual, como cada año, se reviste de su aspecto litúrgico, comenzando por esta Misa, única durante el año, que lleva el nombre de “Cena del Señor”. Después de haber amado a los suyos que estaban en el mundo, \textit{“los amó hasta el fin}” (\textit{Jn} 13, 1). La última Cena es precisamente testimonio del amor con que Cristo, Cordero de Dios, nos ha amado hasta el fin.\end{body}
			
			\begin{body}En esta tarde los hijos de Israel comían el cordero, según la prescripción antigua dada por Moisés en la víspera de la salida de la esclavitud de Egipto. Jesús hace lo mismo con los discípulos, fiel a la tradición, que era sólo la “sombra de los bienes futuros” (\textit{Heb} 10, 1), sólo la “figura” de la Nueva Alianza, de la nueva Ley.\end{body}
			
			\begin{body}2. ¿Qué significa “los amó hasta el fin”? Significa: hasta el cumplimiento que debía realizarse mañana, Viernes Santo. En este día se debía manifestar cuánto amó Dios al mundo, y cómo, en el amor, se ha llegado al límite extremo de la donación, esto es, al punto de “dar a su unigénito Hijo” (\textit{Jn} 3, 16). En ese día Cristo ha mostrado que no hay “amor mayor que éste: de dar uno la vida por sus amigos” (\textit{Jn} 15, 13). El amor del Padre se reveló en la donación del Hijo. En la donación mediante la muerte.\end{body}
			
			\begin{body}El Jueves Santo, el día de la última Cena, es, en cierto sentido, el prólogo de esta donación; es la preparación última. Y en cierto modo lo que se cumplía en este día va ya más allá de tal donación. Precisamente el Jueves Santo, durante la última Cena, se manifestaba lo que quiere decir: “Amó hasta el fin”. En efecto, pensamos justamente que amar hasta el fin signifique \textit{hasta la muerte}, hasta el último aliento. Sin embargo, la última Cena nos muestra que, para Jesús, “hasta el fin” significa más allá del último aliento. \textit{Más allá de la muerte}.\end{body}
			
			\begin{body}3. Este es precisamente el significado de la Eucaristía. La muerte no es su fin, sino su comienzo. La Eucaristía comienza en la muerte, como enseña \textbf{San Pablo}: “Cuantas veces comáis este pan y bebáis este cáliz, anunciáis la muerte del Señor hasta que Él venga” (\textit{1 Cor }11, 26).\end{body}
			
			\begin{body}La Eucaristía es fruto de esta muerte. La recuerda constantemente. La renueva de continuo. La significa siempre. La proclama. La muerte, que ha venido a ser principio de la nueva venida: de la resurrección a la parusía, “hasta que Él venga”. La muerte, que \textit{es “sustrato” de una nueva vida}. Amar “hasta el fin” significa, pues, para Cristo, amar mediante la muerte y más allá de la barrera de la muerte: ¡\textit{Amar hasta los extremos de la Eucaristía!}\end{body}
			
			\begin{body}4. Precisamente Jesús ha amado así en esta última Cena. Ha amado a los “suyos” –a los que entonces estaban con,Él– y a todos los que debían heredar de ellos el misterio:\end{body}
			
			\begin{body}– las palabras que ha pronunciado sobre el pan,\end{body}
			
			\begin{body}– las palabras que ha pronunciado sobre el cáliz, lleno de vino,\end{body}
			
			\begin{body}– las palabras que nosotros repetimos hoy con particular emoción y que repetimos siempre cuando celebramos la Eucaristía, ¡son precisamente la revelación del amor a través del cual, de una vez para siempre, para todos los tiempos y hasta el fin de los siglos, se ha repartido a Sí mismo!\end{body}
			
			\begin{body}Antes aún de \textit{darse a Sí mismo} en la cruz, como “Cordero que quita los pecados del mundo”, \textit{se ha repartido a Sí mismo} como comida y bebida: pan y vino para que “tengamos vida y la tengamos en abundancia” (\textit{Jn} 10, 10). Así Él “amó hasta el fin”.\end{body}
			
			\begin{body}5. Por lo tanto, Jesús no dudó en arrodillarse delante de los Apóstoles para lavar sus pies. Cuando Simón Pedro se opone a ello, Él le convenció para que le dejara hacer. Efectivamente, era una exigencia particular de la grandeza del momento. Era necesario este lavatorio de los pies, esta purificación en orden a la comunión de la que habrían de participar desde aquel momento.\end{body}
			
			\begin{body}Era necesario. Cristo mismo sintió la necesidad de humillarse a los pies de sus discípulos: una humillación que nos dice tanto de Él en ese momento. De ahora en adelante, distribuyéndose a Sí mismo en la comunión eucarística, ¿no se abajará continuamente al nivel de tantos corazones humanos? ¿No los servirá siempre de este modo?\end{body}
			
			\begin{body}“Eucaristía” significa “agradecimiento”. “Eucaristía” significa también “servicio”, el tenderse hacia el hombre: \textit{el servir a tantos corazones humanos}. “Porque yo os he dado el ejemplo, para que vosotros hagáis también como yo he hecho” (\textit{Jn }13, 15). ¡No podemos ser dispensadores de la Eucaristía, sino sirviendo!\end{body}
			
			\begin{body}6. Así, pues, es la última Cena. Cristo se prepara a \textit{irse a través de la muerte}, y a través de la misma muerte se prepara a permanecer. De esta forma la muerte se ha convertido en el fruto maduro del amor: nos amó “hasta el fin”.\end{body}
			
			\begin{body}¿No bastaría aun sólo el contexto de la última Cena para dar a Jesús el “derecho” de decirnos a todos: “Este es mi precepto: que os améis unos a otros como yo os he amado” (\textit{Jn} 15, 12)?\end{body}
			
			\subsubsection{Homilía (1982): El deseo de Cristo}
			
			\begin{referencia}Basílica de San Juan de Letrán. 8 de abril de 1982.\end{referencia}
			
			\begin{body} 1. “El Padre había puesto todo en sus manos” ( \textit{Jn} 13, 3). \end{body}
			
			\begin{body}Antes de la Cena pascual Cristo tiene conciencia clara de que el Padre le ha puesto todo en las manos. Es libre con toda la plenitud de la libertad que goza el Hijo del hombre, el Verbo encarnado. \textit{Es libre} con una libertad tal que no es propia de ningún otro hombre. La última Cena: todo lo que en ella se cumplirá tiene su origen \textit{en la perfecta libertad del Hijo respecto del Padre.}\end{body}
			
			\begin{body}Dentro de poco llevará esta libertad suya a Getsemaní y dirá: “Padre, si quieres, aparta de mí este cáliz; pero no se haga mi voluntad sino la tuya” ( \textit{Lc} 22, 42). Entonces aceptará el sufrimiento que caerá sobre Él y que es al mismo tiempo objeto de una opción; un sufrimiento de dimensiones inconcebibles para nosotros. \end{body}
			
			\begin{body}Pero durante la última Cena la opción <a id="_idTextAnchor009"></a>e<a id="_idTextAnchor010"></a>staba ya hecha. Cristo a<a id="_idTextAnchor011"></a>ctúa con plena conciencia de la opción ya realizada. Sólo tal conciencia explica el hecho de que Él, “tomando el pan, dio gracias, lo partió y se lo dio diciendo: Este es mi cuerpo que es entregado por vosotros” (\textit{Lc} 22, 19). Y después de cenar tomó el cáliz diciendo: “Este cáliz es la nueva alianza sellada con mi sangre”, como refiere San Pablo \textit{(1 Cor} 11, 25), mientras los Evangelistas puntualizan: “en mi sangre que es derramada por vosotros” (\textit{Lc} 22, 20), o “mi sangre de la Nueva Alianza que será derramada por muchos” ( \textit{Mt} 26, 28; \textit{Mc} 14, 24). \end{body}
			
			\begin{body}Al pronunciar estas palabras en el Cenáculo, Cristo ya ha hecho la opción. \end{body}
			
			\begin{body}La ha hecho hace tiempo. Ahora la realiza de nuevo. Y en Getsemaní la cumplirá una vez más al aceptar en el dolor toda la inmensidad del sufrimiento vinculado a esta opción. “El Padre había puesto todo en sus manos”. Todo, el designio completo de la salvación, el Padre lo ha entregado a su libertad perfecta. \textit{Y a su amor perfecto.}\end{body}
			
			\begin{body}2. Mediante la opción de Cristo, mediante su libertad perfecta y su amor perfecto, en la cena pascual \textit{la figura del cordero pascual} llegó al culmen de su significado. \end{body}
			
			\begin{body}De su institución habla la lectura de hoy del \textbf{Éxodo}: “Lo comeréis así: la cintura ceñida, las sandalias en los pies, un bastón en la mano; y os lo comeréis a toda prisa, porque \textit{es la Pascua} del Señor”. “Será un animal sin defecto... lo guardaréis hasta el día catorce del mes y toda la asamblea de Israel lo matará al atardecer. Tomaréis la sangre y rociaréis las dos jambas y el dintel de la casa donde lo hayáis comido...”, “Yo pasaré esta noche por la tierra de Egipto... Cuando yo vea la sangre pasaré de largo ante vosotros, y no habrá entre vosotros plaga exterminadora cuando yo hiera al país de Egipto” (\textit{Ex} 12, 11. 5-7. 12-13). \end{body}
			
			\begin{body}\textit{Esta es la Pascua de la Antigua Alianza.}\end{body}
			
			\begin{body}El recuerdo del \textit{Paso} por Egipto de la mano purificadora del Señor. El recuerdo de la \textit{salvación} mediante la sangre del cordero inocente. El recuerdo de la liberación de la esclavitud. El día catorce de Nisán de cada año, celebra todavía Israel la Pascua. Por su parte Cristo celebra con los Apóstoles la última Cena. Meditan sobre la \textit{liberación} de la esclavitud \textit{mediante la sangre del cordero inocente} . \end{body}
			
			\begin{body}Y Cristo dice sobre el pan: “Tomad y comed; éste es mi cuerpo, \textit{que ha sido entregado por vosotros}”. Después dice sobre el vino: “Tomad y bebed, éste es el cáliz de mi sangre \textit{que será derramada por vosotros”}. Por vosotros y por todos (cfr. \textit{Mt} 26, 26-28; \textit{Lc} 22, 19-20). En el marco de estas palabras aparece ya el \textit{cumplimiento de la figura del cordero} de la Antigua Alianza. Y de pronto, en la historia de la humanidad, en la historia de la salvación, entra el Cordero de la Nueva Alianza, el Cordero más inocente, el \textit{Cordero de Dios} . \end{body}
			
			\begin{body}Entra mediante su Cuerpo y Sangre; mediante el Cuerpo que será entregado, mediante la Sangre que será derramada. Entra a través de la muerte que libera de la esclavitud de la muerte del pecado. \textit{Entra a través de la muerte que da la vida}. El sacramento de la última Cena es el signo visible de esta vida. Es alimento de vida eterna. \end{body}
			
			\begin{body}3. Sucedió “antes de la fiesta de Pascua”. Aquella fue la \textit{hora de Cristo}, la hora “de pasar de este mundo al Padre”. \end{body}
			
			\begin{body}En aquella hora, “habiendo amado a los suyos que estaban en el mundo, los amó hasta el extremo” (\textit{Jn} 13, 1). “Los suyos que estaban en el mundo”, ¿acaso solamente los que estaban con Él en la hora de la última Cena? No sólo ellos. Amó a todos “los suyos”, a todos los que iba a redimir. A todos desde el principio del mundo hasta el fin. A todos \textit{y en todos los sitios}. \end{body}
			
			\begin{body}Y entonces les lavó los pies; a los que estaban en el Cenáculo A Pedro, el primero. Entonces, en el momento de su primera Eucaristía, les deseó pureza, una pureza mayor de cuanto ellos mismos habían pensado; de cuanto había pensado Pedro. Y desea esta pureza a todos. El amor le obliga a desear pureza a todos y en todos los sitios. “Si no te lavo, no tienes nada que ver conmigo” (\textit{Jn} 13, 8). \end{body}
			
			\begin{body}En la Eucaristía Cristo desea compartir su vida conmigo, \textit{desea la comunión. En la perspectiva de esta comunión con el hombre, desea la pureza} de su alma.\end{body}
			
			\begin{body}Esta es, pues, la hora de la última Cena. La hora de Cristo. \textit{La hora} del grande e ilimitado deseo de su corazón; Él desea la comunión con el hombre y desea la pureza del alma humana. \end{body}
			
			\begin{body}¿Acaso podemos rehuir este deseo?\end{body}
			
			\subsubsection{Homilía (1985): Sacramento del Siervo}
			
			\begin{referencia}Basílica de San Juan de Letrán. 4 de abril de 1985.\end{referencia}
			
			\begin{body} 1. \textit{“¡No me lavarás los pies jamás!”} (\textit{Jn }13, 8). \end{body}
			
			\begin{body}Hoy, reunidos (...) para la liturgia de la Cena del Señor, escuchamos estas palabras de rechazo de Pedro. \end{body}
			
			\begin{body}Sin embargo, Jesús convence al apóstol. El lavatorio de los pies es ciertamente una función de servicio, pero también es expresión y signo de \textit{participación en toda la obra mesiánica de Cristo}. Pedro aún no lo ve. “Si no te lavo, no tendrás parte conmigo” (\textit{Jn} 13, 8). \end{body}
			
			\begin{body}Pedro todavía no comprende; pero \textit{su corazón ya está completamente volcado a la obra mesiánica de Cristo}: a lo que Cristo quiere. Por eso dice: “¡No solo los pies, sino también las manos y la cabeza!” (\textit{Jn} 13, 9). \end{body}
			
			\begin{body}2. Cristo lava los pies de Pedro y de todos los apóstoles. Dentro de poco, en memoria e imitación de ese gesto del Señor, lavaré los pies a doce [sacerdotes que concelebran conmigo en la liturgia eucarística de esta noche]. El lavatorio de los pies a los apóstoles por Je<a id="_idTextAnchor012"></a>s<a id="_idTextAnchor013"></a>ús fue \textit{una introducción a la Cena de Pascua}. Esta función de servicio debe confirmar una vez más que Jesús no vino al mundo para ser servido, sino para ser él mismo un servidor: Él, el Maestro y el Señor. \end{body}
			
			\begin{body}Los apóstoles deben pensar y actuar de la misma manera: “Yo os he dado... ejemplo” (\textit{Jn }13, 15). \end{body}
			
			\begin{body}La función de servicio al comienzo de esta noche pascual \textit{manifiesta la presencia del “siervo”}. Es el \textit{“siervo de Yahvé”} de la profecía de Isaías. Jesús quiere indicar de esta manera que la Cena de Pascua da inicio al cumplimiento de las palabras de Isaías. Más aún, la misma \textit{Cena se convertirá en el sacramento del siervo}: “Yo soy tu siervo, hijo de tu sierva” (\textit{Sal} 116, 16). \end{body}
			
			\begin{body}3. Durante la Cena de Pascua, todos los participantes dirigen su recuerdo \textit{hacia el cordero pascual} , cuya sangre en los dinteles de las casas salvó de la muerte a los primogénitos de Israel y abrió el camino al éxodo de Egipto. \end{body}
			
			\begin{body}\textit{También Jesús} dirige la mirada de su alma hacia el cordero pascual, recuerda la liberación de la esclavitud en Egipto. \end{body}
			
			\begin{body}Y al mismo tiempo tiene en sus oídos \textit{la voz de Juan el Bautista} , que a orillas del Jordán le señaló y proclamó: “He aquí el Cordero de Dios, he aquí el que quita el pecado del mundo” (\textit{Jn }1, 29). Y, he aquí la Última Cena. Jesús sabe que ha llegado el momento del cumplimiento de las palabras de Juan cerca del Jordán. \textit{La sangre del cordero debe quitar los pecados del mundo. }\end{body}
			
			\begin{body}4. De esta manera la Cena de Pascua alcanza su cenit. Jesús toma primero el pan, lo parte y, habiendo pronunciado la oración de acción de gracias, lo da a los apóstoles para que lo coman: “\textit{Esto es mi cuerpo que es entregado por vosotros} ; haced esto en memoria mía” (\textit{Lc }22, 19). \end{body}
			
			\begin{body}Luego toma el cáliz lleno de vino. Y dice (según el texto de Pablo): “Esta copa \textit{es la nueva alianza en mi sangre}; haced esto cada vez que lo bebáis en memoria de mí” (\textit{1 Co }11, 25). \end{body}
			
			\begin{body}El profeta Isaías compara al siervo que sufre con un cordero. Juan el Bautista dice expresamente: el Cordero de Dios. \end{body}
			
			\begin{body}Jesús, teniendo que cumplir las palabras del profeta y de Juan, instituye en la Eucaristía \textit{la alianza nueva y eterna en su sangre}. \end{body}
			
			\begin{body}En la Eucaristía \textit{ya está contenido }todo lo que pronto comenzará a suceder místicamente. La Escritura dice: “En efecto, cada vez que coméis de este pan y bebéis de esta copa, proclamáis la muerte del Señor hasta que Él venga” (\textit{1 Co} 11, 26). \end{body}
			
			\begin{body}De esta manera, la Eucaristía de la Última Cena \textit{anticipa la realidad} de la que es signo. \end{body}
			
			\begin{body}Y al mismo tiempo, a través de la Eucaristía, también se anuncia otra realidad: la redención del mundo, la nueva alianza en la sangre del Cordero de Dios, una realidad que continúa. \end{body}
			
			\begin{body}A través de la Eucaristía, esta realidad \textit{se hace presente constantemente y se renueva de manera sacramental}: “Anuncia la muerte del Señor hasta que venga”. \end{body}
			
			\begin{body}5. Esta realidad \textit{se explica por el amor}: la cruz y la muerte del Cordero de Dios se explican por el amor. La redención del mundo se explica por el amor. La alianza nueva y eterna en la sangre de Cristo se explica por el amor. \end{body}
			
			\begin{body}\textit{“Porque tanto amó Dios al mundo...” }(\textit{Jn }3, 16). \end{body}
			
			\begin{body}Y Jesús, “sabiendo que había llegado la hora de pasar de este mundo al Padre, después de haber amado a los suyos que estaban en el mundo, \textit{los amó hasta el extremo}” (\textit{Jn} 13, 1). \end{body}
			
			\begin{body}Esto es precisamente la Eucaristía. La Eucaristía se explica a través del amor. \textit{La Eucaristía nace del amor y da origen al amor. }En ella \textit{está inscrito} y arraigado definitivamente \textit{el mandamiento del amor}. “Os doy un mandamiento nuevo: amaos los unos a los otros \textit{como yo os he amado}” (\textit{Jn} 13, 34).\end{body}
			
			\begin{body}He aquí la Última Cena: el misterio de la Pascua. A partir de ahora \textit{, el amor y la muerte} caminarán juntos por la historia del hombre, hasta que venga de nuevo Aquel que los unió con un vínculo inquebrantable y nos los dejó en la Eucaristía, para que nosotros hagamos lo mismo en memoria suya.\end{body}
			
			\subsubsection{Homilía (1988): }
			
			\begin{referencia}Basílica de San Juan de Letrán. 31 de marzo de 1988.\end{referencia}
			
			\begin{body} 1. “No me lavarás los pies jamás” (\textit{Jn} 13, 8). \end{body}
			
			\begin{body}Así dice Simón Pedro en el Cenáculo, cuando Cristo, antes de la cena pascual, decide lavar los pies de sus apóstoles. \end{body}
			
			\begin{body}Cristo sabía que “había llegado su hora” (\textit{Jn} 13, 1). Su Pascua. \end{body}
			
			\begin{body}Pero Simón Pedro todavía no lo sabía. \end{body}
			
			\begin{body}Cerca de Cesarea de Filipo fue el primero en confesar: “Tú eres el Cristo, el Hijo del Dios vivo” (\textit{Mt }16, 16). \end{body}
			
			\begin{body}Sin embargo, no sabía que el significado de “siervo”, el siervo de Yahvé, también estaba oculto en esta definición de “Cristo-Mesías”. ¡No sabía! En cierto sentido, no quería tomar conciencia de la verdad que, según la interpretación del Maestro, tenía “su hora”: esto es “la hora del paso de este mundo al Padre” (cf. \textit{Jn} 13, 1). \end{body}
			
			\begin{body}No aceptaba que Cristo tuviera que ser un siervo, como lo había visto muchos siglos antes el profeta Isaías: el siervo de Yavé, el siervo sufriente de Dios. \end{body}
			
			\begin{body}2. Sin embargo, en el horizonte de la historia el papel de la sangre se ha definido ahora de forma definitiva: la sangre del cordero pascual debía encontrar su cumplimiento en la sangre de Cristo qu<a id="_idTextAnchor014"></a>e selló la nueva y eterna alianza. \end{body}
			
			\begin{body}Para aquellos que se estaban preparando para la cena pascual en el Cenáculo de Jerusalén, la sangre del cordero estaba relacionada con el recuerdo del Éxodo. Recordaba la liberación de la esclavitud de Egipto, que había iniciado el pacto de Yahweh con Israel, en tiempos de Moisés. \end{body}
			
			\begin{body}“Tomaréis un poco de su sangre, la colocaréis en las dos jambas de las puertas y en el dintel de las casas, donde lo comáis… ¡Es la Pascua del Señor!... Veré la sangre y pasaré, no habrá azote de exterminio para vosotros cuando golpee la tierra de Egipto” (\textit{Ex} 12, 7. 11. 13). \end{body}
			
			\begin{body}La sangre del cordero constituyó un umbral frente al cual se detuvo la ira castigadora de Yahvé. Los que se preparaban para la cena pascual, en el Cenáculo, guardaban en su memoria la liberación de Israel a través de esta sangre. \end{body}
			
			\begin{body}3. Todos ellos, y Simón Pedro junto con ellos, no eran plenamente conscientes de que esa liberación por la sangre del cordero pascual era al mismo tiempo un anticipo. Era una “figura” que esperaba su realización en Cristo. \end{body}
			
			\begin{body}Cuando los apóstoles se reunieron en el Cenáculo para la Última Cena, esto estaba próximo a cumplirse. Cristo sabe que “ha llegado su hora”, la hora en que él mismo cumplirá lo que estaba ya anunciado y revelará plenamente la realidad que durante siglos ha sido indicada por la “figura” del cordero pascual: la liberación por su sangre. \end{body}
			
			\begin{body}Cristo va al encuentro de esta “plenitud”, entra en esta realidad. Él es consciente de lo que traerán consigo la noche que se avecina y el día siguiente. \end{body}
			
			\begin{body}4. Y he aquí, toma el pan en sus manos y, dando gracias, dice: “Esto es mi cuerpo, entregado por vosotros” (\textit{1 Co} 11, 24). Y al final de la cena (como leemos en la primera carta a los Corintios) toma el cáliz y dice: “Este cáliz es la nueva alianza en mi sangre” (\textit{1 Co} 11, 25). \end{body}
			
			\begin{body}Cuando el cuerpo de Cristo sea ofrecido en la cruz, entonces esta sangre, derramada en la pasión, será el comienzo de la nueva alianza de Dios con la humanidad. \end{body}
			
			\begin{body}La alianza antigua, en la sangre del cordero pascual, la sangre de la liberación de la esclavitud de Egipto. \end{body}
			
			\begin{body}La alianza nueva y eterna, en la sangre de Cristo. \end{body}
			
			\begin{body}Cristo va al sacrificio, que tiene el poder redentor: el poder de liberar al hombre de la esclavitud del pecado y la muerte. El poder de arrancar al hombre del abismo de la muerte espiritual y la condenación. \end{body}
			
			\begin{body}Jesús pasa a los discípulos la copa de la salvación, la sangre de la nueva alianza y dice: “Haced esto cada vez que lo bebáis en memoria de mí” (\textit{1 Co} 11, 25). \end{body}
			
			\begin{body}5. “Habiendo amado a los suyos que están en el mundo, los amó hasta el extremo” (\textit{Jn} 13, 1). \end{body}
			
			\begin{body}Aquí está la verdad más profunda de la Última Cena. El cuerpo y la sangre, la pasión y muerte en la cruz significan precisamente esto: “los amó hasta el extremo”. \end{body}
			
			\begin{body}La sangre del cordero en el dintel de las casas en Egipto no tenía, en sí misma, un poder liberador. El poder vino de Dios y durante mucho tiempo no nos atrevimos a llamar a este poder por su nombre. \end{body}
			
			\begin{body}Cristo lo llamó por su nombre. El cuerpo y la sangre, la pasión y la muerte, el sacrificio, son el amor \end{body}
			
			\begin{body}****que vuelve a los límites de su poder salvador. \end{body}
			
			\begin{body}Cristo lo llamó por su nombre. Cristo hizo que sucediera. Cristo nos dejó este poder en la Eucaristía. He aquí “su hora” en la que: \end{body}
			
			\begin{body}- pasa de este mundo al Padre por la sangre de la nueva alianza; \end{body}
			
			\begin{body}- pasa de este mundo al Padre por el amor, \end{body}
			
			\begin{body}****que vuelve a los límites de su poder salvífico. *****\end{body}
			
			\begin{body}6. Él dice: “Haced esto en memoria mía”. \end{body}
			
			\begin{body}E incluso antes de eso, dice: “Os he dado ejemplo, para que como yo lo hice, también vosotros podáis hacerlo” (\textit{Jn }13, 15). \end{body}
			
			\begin{body} “Amaos los unos a los otros como yo os he amado” (\textit{Jn} 13, 34). Última Cena. El comienzo de la nueva alianza en la sangre de Cristo. \end{body}
			
			\begin{body}¡Revivámosla con un corazón lleno de fe y amor!\end{body}
			
			\subsubsection{Homilía (1991): Entregó todo en sus manos}
			
			\begin{referencia}Basílica de San Juan de Letrán. 28 de marzo de 1991.\end{referencia}
			
			\begin{body} “\textit{El Padre había puesto todo en sus manos}” (\textit{Jn} 13, 3).\end{body}
			
			\begin{body}1. Cuando Cristo se reunió con los Apóstoles en el Cenáculo para comer la Pascua con ellos, ya lo sabía. Todo esto lo supo durante su vida terrena, durante los años de su misión mesiánica, pero ahora lo sabe de manera particular, de manera definitiva: “El Padre había puesto todo en sus manos”. \end{body}
			
			\begin{body}\textit{Con esta conciencia, irá a Getsemaní}, será juzgado y condenado a muerte \textit{en la Cruz}. Esta conciencia, esta certeza, se convertirá en su sufrimiento; un sufrimiento humano, aunque humanamente inexpresable. Se convertirá en su sacrificio redentor. \end{body}
			
			\begin{body}“El Padre había puesto todo en sus manos”. \end{body}
			
			\begin{body}\textit{Todo significa la creación entera}, incluida en el plan divino y eterno. Todo significa \textit{cada hombre y toda la humanidad}. Nadie más podía entender este signo, además de él, el Hijo consustancial al Padre, el Verbo Eterno, el Primogénito de todas las criaturas. ¡Sólo él! \end{body}
			
			\begin{body}El Padre puso en sus manos todo el futuro del Reino de Dios, la escatología de la historia humana. Finalmente, \textit{sólo Él puede devolverlo todo al Padre}, “para que Dios sea todo en todos” (\textit{1 Co }15, 28). \end{body}
			
			\begin{body}2. Esta conciencia del Hijo significa, al mismo tiempo, u<a id="_idTextAnchor015"></a>na particular plenitud de amor. Cuando “llegó su tiempo de pasar de este mundo al Padre, \textit{habiendo amado a los suyos que estaban en el mundo, los amó hasta el extremo”} (\textit{Jn} 13, 1). \end{body}
			
			\begin{body}¡Hasta el extremo! \end{body}
			
			\begin{body}De este amor suyo “hasta el extremo” nace la Eucaristía. De este amor nacen Getsemaní y el Gólgota; nace la obediencia hasta la muerte y muerte de cruz (cf. \textit{Fil} 2, 8); ¡nace la Eucaristía! \end{body}
			
			\begin{body}Cristo, volviendo al Padre, sabe que no puede dejarnos. Debe quedarse porque el Padre “le había entregado todo en sus manos”. No puede pasar como “todo” pasa por el universo creado. No puede simplemente pasar a la historia. \textit{Debe permanecer por encima de la historia y dentro de la historia}, para que Dios sea “todo en todos” (cf. \textit{1 Co} 15, 28). \end{body}
			
			\begin{body}3. \textit{La Eucaristía: ¡Acontecimiento y Sacramento! }Hoy la vivimos de una manera particular. Más que en cualquier otro momento, la liturgia del Jueves Santo, “in Cena Domini”, es “la memoria” de este Acontecimiento. Y, al mismo tiempo, es el Sacramento que perdura y se hace presente \textit{en su profundidad y potencia originales} cada vez que “comemos este pan y bebemos este cáliz”; siempre que “anunciamos la muerte del Señor hasta que venga”; cada vez que expresamos “la Nueva Alianza” en la sangre de Cristo (cf. \textit{1 Co} 11, 25-26): ¡la Nueva y Eterna Alianza! \end{body}
			
			\begin{body}4. Cristo, a quien el Padre “había entregado todo en sus manos”, entra en esta Hora culminante de la historia \textit{como Siervo}. La imagen del Siervo de Dios, tomada del profeta Isaías (cf. \textit{Is} 42, 1-2) se realiza plenamente en él. \end{body}
			
			\begin{body}“Vosotros me llamáis Maestro y Señor y decís bien, porque lo soy” (\textit{Jn }13, 13). Y he aquí que yo, Maestro y Señor, a quien el Padre ha puesto todo en sus manos, os \textit{lavo los pies }(cf. \textit{Jn} 13, 13). \end{body}
			
			\begin{body}\textit{También se hace Siervo}. \end{body}
			
			\begin{body}Y así lo hizo Cristo, y así ha venido a ser para siempre: como la luz de nuestra conciencia, como el servidor de la redención del hombre. \textit{El mayor servicio del} Cordero de Dios \textit{es el sacrificio redentor en la Cruz}. En la Eucaristía, el Hijo, glorificado a la diestra del Padre, permanece como servidor de nuestra redención. \end{body}
			
			\begin{body}“Os he dado... el ejemplo, para que... vosotros también hagáis (así)” (\textit{Jn }13, 15). \end{body}
			
			\begin{body}El Jueves Santo, “in Cena Domini”, redescubrimos cada vez mejor el significado de este “sacerdocio ministerial”. \end{body}
			
			\begin{body}¡Gloria a Ti, Rey de los siglos! (cf. 1 \textit{Tm }1,17).\end{body}
			
			\subsubsection{Homilía (1994): Servicio de una humilde caridad}
			
			\begin{referencia}31 de marzo de 1994.\end{referencia}
			
			\begin{body} 1. “\textit{Y comenzó a lavar los pies de los discípulos}...” (\textit{Jn} 13, 5) \end{body}
			
			\begin{body}Esta es la tarde en la que la Iglesia revive el gesto y el sentido del lavatorio de los pies, que debía introducir a los Apóstoles, reunidos en el Cenáculo, a la institución de la Eucaristía. \end{body}
			
			\begin{body}\textit{¿Por qué quiso Cristo comenzar con el lavatorio de los pies?} Hizo esto para presentarse ante ellos en la condición de un sirviente. Él mismo lo explica cuando dice: “Si yo, el Señor y el Maestro, os he lavado los pies, vosotros también debéis lavaros los pies unos a otros” (\textit{Jn }13, 14). El lavatorio de los pies expresa el servicio de una humilde caridad. Durante la Última Cena, Cristo \textit{quiere revelarse como el que sirve}: “Yo estoy entre vosotros como el que sirve” (\textit{Lc} 22, 27). Un verdadero discípulo de Cristo es sólo el que tiene “parte” con el Maestro, dispuesto a servir como Él. El servicio, es decir, el cuidado de las necesidades de los demás, constituye la esencia de todo poder. Servir significa reinar. \end{body}
			
			\begin{body}2. En la hora en que se prepara para cumplir el misterio pascual, Cristo se manifiesta entre nosotros como el que sirve. En efecto, la verdadera razón última de su venida al mundo aparece a los ojos de los discípulos: el ministerio de la redención del hombre y la salvación del mundo. \end{body}
			
			\begin{body}En este ministerio \textit{se ofrece a sí mismo}: se entr<a id="_idTextAnchor016"></a>ega a la muerte en la cruz \textit{para darse a sí mismo}. Por eso anticipa la crucifixión a través de la institución de la Eucaristía. En ella, Cristo se ofrece a sí mismo como don a los Apóstoles en el Cenáculo; luego, diciéndoles: “Haced esto en memoria de mí” (\textit{Lc }22, 19), les da la misión de a\textit{ entregarlo a los demás hasta el fin del mundo}. \end{body}
			
			\begin{body}Cristo, que vive totalmente para el Padre, desea que también nosotros vivamos por él; por eso se ofrece a nosotros bajo las apariencias de pan y vino. El pan es el alimento diario del hombre, sin el cual es difícil vivir; el vino es la bebida beneficiosa para la salud del organismo. \end{body}
			
			\begin{body}Se da a sí mismo –su Cuerpo y su Sangre– como un don hasta el fin del mundo, porque \textit{esta es la lógica de su amor}: “nos amó hasta el extremo” (cf. \textit{Jn} 13, 1). \end{body}
			
			\begin{body}3. La esencia de su ministerio es precisamente esta: es el ministerio de salvación que aún hoy ejerce y que llevará a cabo hasta el fin de los tiempos, a través de la Iglesia. Por eso \textit{es necesario que la Iglesia, Esposa de Cristo, lleve a cabo fielmente el ministerio que le ha sido encomendado, realizando el misterio de la redención y la Eucaristía}. En la realización de este “servicio”, Cristo se escondió bajo las especies del pan y del vino y, en tan misteriosas apariencias, alimenta y guía a su pueblo a través de los siglos. Él es el único sacerdote, rey y profeta, y nos hacemos partícipes de él a través de los sacramentos. \end{body}
			
			\begin{body}4. La liturgia de la Última Cena destaca el vínculo misterioso que existe entre la liberación de Israel de la esclavitud de Egipto y la institución de la Eucaristía. Este segundo tema encontrará plena expresión durante la Vigilia Pascual, cuando se conmemorará el sacramento del Bautismo. Hoy encuentra su expresión en relación con la Eucaristía. \end{body}
			
			\begin{body}Aquí está el anuncio: \textit{Cristo es el Cordero pascual}, que libera a su pueblo de la esclavitud por medio de la sangre derramada en la Cruz. \end{body}
			
			\begin{body}En la noche del éxodo de Egipto, la sangre del cordero en las jambas de las puertas de las casas en las que vivían los hijos de Israel era la señal de su salvación. Se puede decir que fue precisamente esta sangre la que sacó a los israelitas de la condición de esclavitud y les mostró el camino a la Tierra Prometida. Durante la Última Cena, Jesús dice: “Este cáliz es la Nueva Alianza en mi sangre; haced esto cada vez que lo bebáis en memoria de mí” (\textit{1 Co }11, 25). El salmista pregunta: “¿Cómo pagaré al Señor, todo el bien que me ha hecho?” (\textit{Sal} 115 [116], 12). Y nosotros, con toda la Iglesia, nos hacemos la misma pregunta esta tarde de Jueves Santo: “¿Cómo pagaré al Señor?”.\end{body}
			
			\subsubsection{Homilía (1997): Memoria viva de su amor}
			
			\begin{referencia}Basílica de San Juan de Letrán. 27 de marzo de 1997.\end{referencia}
			
			\begin{body} 1. Cada año esta Basílica de san Juan de Letrán acoge a la asamblea reunida para el solemne Memorial de la Última Cena.\end{body}
			
			\begin{body}Acuden fieles de Roma y de todo el mundo para renovar el recuerdo de aquel acontecimiento que se realizó un jueves de hace muchos años en el Cenáculo, y que la liturgia conmemora como siempre actual en el día de hoy. Lo prolonga como Sacramento del Altar, Sacramento del Cuerpo y de la Sangre de Cristo. Lo prolonga como Eucaristía.\end{body}
			
			\begin{body}Estamos convocados para repetir ante todo el gesto que Cristo hizo al comienzo de la Última Cena, esto es, el lavatorio de los pies. El Evangelio de Juan presenta a nuestra consideración la resistencia de Pedro ante la humillación del Maestro y la enseñanza con la que Jesús ha comentado su propio gesto: “Vosotros me llamáis ‘el Maestro’ y ‘el Señor’, y decís bien, porque lo soy. Pues si yo, el Maestro y el Señor, os he lavado los pies, también vosotros debéis lavaros los pies unos a otros: os he dado ejemplo para que lo que yo he hecho con vosotros, vosotros también lo hagáis” (\textit{Jn} 13, 13-15).\end{body}
			
			\begin{body}\textit{En la hora del banquete eucarístico, Cristo afirma la necesidad del servicio}. “El Hijo del hombre no ha venido para que le sirvan, sino para servir y dar su vida en rescate por todos” (\textit{Mc} 10, 45).\end{body}
			
			\begin{body}Estamos, pues, convocados para expresar de nuevo la memoria viva del mayor de los mandamientos, el mandamiento del amor: <a id="_idTextAnchor017"></a>“<a id="_idTextAnchor018"></a>Nadie tiene amor más grande que el que da la vida por sus amigos” (\textit{Jn} 15, 13). El gesto de Cristo lo representa en vivo ante la mirada de los Apóstoles: “Había llegado la hora de pasar de este mundo al Padre”; la hora del sumo amor: “Habiendo amado a los suyos que estaban en el mundo, los amó hasta el extremo” (\textit{Jn} 13, 1).\end{body}
			
			\begin{body}2. Todo esto culmina en la Última Cena, en el Cenáculo de Jerusalén. Estamos convocados \textit{para revivir este acontecimiento, la institución del Sacramento admirable, del que la Iglesia vive incesantemente}, del Sacramento que constituye la Iglesia en su realidad más auténtica y profunda. No hay Eucaristía sin Iglesia, pero, antes aún, no hay Iglesia sin Eucaristía.\end{body}
			
			\begin{body}Eucaristía quiere decir \textit{acción de gracias}. Por esto hemos rezado con el \textbf{salmo responsorial}: “¿Cómo pagaré al Señor todo el bien que me ha hecho? (\textit{Sal} 115, 12). Presentamos sobre el altar las ofrendas del pan y del vino, como incesante acción de gracias por todos los bienes que recibimos de Dios, por los bienes de la creación y de la redención. La redención se ha realizado mediante el Sacrificio de Cristo. La Iglesia, que anuncia la redención y vive de la redención, ha de continuar haciendo presente sacramentalmente este Sacrificio, del cual debe sacar fuerza para ser ella misma.\end{body}
			
			\begin{body}3. La celebración eucarística \textit{in Coena Domini} nos lo recuerda con singular elocuencia. La \textbf{primera lectura}, del libro del \textbf{Éxodo}, evoca el momento de la historia del pueblo de la Antigua Alianza en el que con más fuerza ha estado prefigurado el misterio de la Eucaristía: se trata de la institución de la Pascua. El pueblo debía ser liberado de la esclavitud de Egipto, debía dejar la tierra de esclavitud y el precio de este rescate era la sangre del cordero.\end{body}
			
			\begin{body}Aquel cordero de la Antigua Alianza ha encontrado plenitud de significado en la Nueva Alianza. Esto se ha realizado también mediante el ministerio profético de Juan Bautista, quien, al ver a Jesús de Nazaret que venía al río Jordán para recibir el bautismo, exclamó: “Este es el Cordero de Dios, que quita el pecado del mundo” (\textit{Jn} 1, 29).\end{body}
			
			\begin{body}No es casual que estas palabras se hayan colocado en el centro de la liturgia eucarística. Nos lo recuerdan las lecturas de la santa Misa de la Cena del Señor para indicar que \textit{con este vivo Memorial entramos en la hora de la Pasión de Cristo}. Precisamente en esta hora será desvelado \textit{el misterio del Cordero de Dios}. Las palabras pronunciadas por el Bautista junto al Jordán se cumplirán así claramente. Cristo será crucificado. Como Hijo de Dios aceptará la muerte para liberar al mundo del pecado.\end{body}
			
			\begin{body}Abramos nuestros corazones, participemos con fe en este gran misterio y aclamemos junto con toda la Iglesia convocada en asamblea eucarística: “Anunciamos tu muerte, proclamamos tu resurrección. ¡Ven, Señor Jesús!”\end{body}
			
			\subsubsection{Homilía (2000): Testigos del amor del Crucificado}
			
			\begin{referencia}20 de abril del 2000.\end{referencia}
			
			\begin{body}1. “Con ansia he deseado comer esta Pascua con vosotros, antes de padecer” (\textit{Lc} 22, 15).\end{body}
			
			\begin{body}Cristo da a conocer, con estas palabras, el significado profético de la cena pascual, que está a punto de celebrar con los discípulos en el Cenáculo de Jerusalén.\end{body}
			
			\begin{body}Con la \textbf{primera lectura}, tomada del \textbf{libro del Éxodo}, la liturgia ha puesto de relieve cómo la Pascua de Jesús se inscribe en el contexto de la Pascua de la antigua Alianza. Con ella, los israelitas conmemoraban la cena consumada por sus padres en el momento del éxodo de Egipto, de la liberación de la esclavitud. El texto sagrado prescribía que se untara con un poco de sangre del cordero las dos jambas y el dintel de las casas. Y añadía cómo había que comer el cordero: “Ceñidas vuestras cinturas, calzados vuestros pies, y el bastón en vuestra mano; (...) de prisa. (...) Yo pasaré esa noche por la tierra de Egipto y heriré a todos los primogénitos. (...) La sangre será vuestra señal en las casas donde moráis. Cuando yo vea la sangre pasaré de largo ante vosotros, y no habrá entre vosotros plaga exterminadora” (\textit{Ex} 12, 11-13).\end{body}
			
			\begin{body}Con la sangre del cordero los hijos e hijas de Israel obtienen la liberación de la esclavitud de Egipto, bajo la guía de Moisés. El recuerdo de un acontecimiento tan extraordinario se convirtió en una ocasión de fiesta para el pueblo, agradecido al Señor por la libertad recuperada, don divino y compromiso humano siempre actual. “Este será un día memorable para vosotros, y lo celebraréis como fiesta en honor del Señor” (\textit{Ex} 12, 14). ¡Es la Pascua del Señor! ¡La Pascua de la antigua Alianza!\end{body}
			
			\begin{body}2. “Con ansia he deseado comer esta Pascua con vosotros, antes de padecer” (\textit{Lc} 22, 15). En el Cenáculo, Cristo, cumpliendo las prescripciones de la antigua Alianza, celebra la cena pascual con los Apóstoles, pero da a este rito un contenido nuevo. Hemos escuchado lo que dice de él \textbf{san Pablo} en la \textbf{segunda lectura}, tomada de la primera carta a los Corintios. En este texto, que se suele considerar como la más antigua descripción de la cena del Señor, se recuerda que Jesús, “la noche en que iban a entregarle, tomó pan y, pronunciando la acción de gracias, lo partió y dijo: ‘Esto es mi cuerpo, que se entrega por vosotros. Haced esto en memoria mía’. Lo mismo hizo con el cáliz, después de cenar, diciendo: ‘Este cáliz es la nueva Alianza sellada con mi sangre; haced esto cada vez que bebáis, en memoria mía’. Por eso, cada que vez que coméis de este pan y bebéis del cáliz, proclamáis la muerte del Señor, hasta que vuelva” (\textit{1 Co} 11, 23-26).\end{body}
			
			\begin{body}Con estas palabras solemnes se entrega, para todos los siglos, la memoria de la institución de la Eucaristía. Cada año, en este día, las recordamos volviendo espiritualmente al Cenáculo. \end{body}
			
			\begin{body}\begin{bodysmall}[Esta tarde las revivo con emoción particular, porque conservo en mis ojos y en mi corazón las imágenes del Cenáculo, donde tuve la alegría de celebrar la Eucaristía, con ocasión de mi reciente peregrinación jubilar a Tierra Santa. La emoción es más fuerte aún porque este es el año del jubileo bimilenario de la Encarnación. Desde esta perspectiva, la celebración que estamos viviendo adquiere una profundidad especial (...)]\end{bodysmall}\end{body}
			
			\begin{body}\begin{bodysmall}(...) E\end{bodysmall}n el Cenáculo Jesús infundió un nuevo contenido a las antiguas tradiciones y anticipó los acontecimientos del día siguiente, cuando su cuerpo, cuerpo inmaculado del Cordero de Dios, sería inmolado y su sangre sería derramada para la redención del mundo. La Encarnación se había realizado precisamente con vistas a este acontecimiento: ¡la Pascua de Cristo, la Pascua de la nueva Alianza!\end{body}
			
			\begin{body}3. “Cada vez que coméis de este pan y bebéis del cáliz, proclamáis la muerte del Señor, hasta que vuelva” (\textit{1 Co} 11, 26). El \textbf{Apóstol} nos exhorta a hacer constantemente memoria de este misterio. Al mismo tiempo, nos invita a vivir diariamente nuestra misión de testigos y heraldos del amor del Crucificado, en espera de su vuelta gloriosa.\end{body}
			
			\begin{body}Pero ¿cómo hacer memoria de este acontecimiento salvífico? ¿Cómo vivir en espera de que Cristo vuelva? Antes de instituir el sacramento de su Cuerpo y su Sangre, Cristo, inclinado y arrodillado, como un esclavo, lava en el Cenáculo los pies a sus discípulos. Lo vemos de nuevo mientras realiza este gesto, que en la cultura judía es propio de los siervos y de las personas más humildes de la familia. Pedro, al inicio, se opone, pero el Maestro lo convence, y al final también él se deja lavar los pies, como los demás discípulos. Pero, inmediatamente después, vestido y sentado nuevamente a la mesa, Jesús explica el sentido de su gesto: “Vosotros me llamáis ‘el Maestro’ y ‘el Señor’, y decís bien, porque lo soy. Pues si yo, el Maestro y el Señor, os he lavado los pies, también vosotros debéis lavaros los pies unos a otros” (\textit{Jn} 13, 12-14). Estas palabras, que unen el misterio eucarístico al servicio del amor, pueden considerarse propedéuticas de la institución del sacerdocio ministerial.\end{body}
			
			\begin{body}Con la institución de la Eucaristía, Jesús comunica a los Apóstoles la participación ministerial en su sacerdocio, el sacerdocio de la Alianza nueva y eterna, en virtud de la cual él, y sólo él, es siempre y por doquier artífice y ministro de la Eucaristía. Los Apóstoles, a su vez, se convierten en ministros de este excelso misterio de la fe, destinado a perpetuarse hasta el fin del mundo. Se convierten, al mismo tiempo, en servidores de todos los que van a participar de este don y misterio tan grandes.\end{body}
			
			\begin{body}La Eucaristía, el supremo sacramento de la Iglesia, está unida al sacerdocio ministerial, que nació también en el Cenáculo, como don del gran amor de Jesús, que “sabiendo que había llegado la hora de pasar de este mundo al Padre, habiendo amado a los suyos que estaban en el mundo, los amó hasta el extremo” (\textit{Jn} 13, 1). La Eucaristía, el sacerdocio y el mandamiento nuevo del amor. ¡Este es el memorial vivo que contemplamos en el Jueves santo!\end{body}
			
			\begin{body}“Haced esto en memoria mía”: ¡esta es la Pascua de la Iglesia, nuestra Pascua!\end{body}
			
			\subsubsection{Homilía (2003): Dos manifestaciones, un mismo misterio}
			
			\begin{referencia}17 de abril del 2003.\end{referencia}
			
			\begin{body}\textit{“Los amó hasta el extremo”} (\textit{Jn} 13, 1).\end{body}
			
			\begin{body}1. En la víspera de su pasión y muerte, el Señor Jesús quiso reunir en torno a sí, una vez más, a sus Apóstoles para dejarles las últimas consignas y darles el testimonio supremo de su amor.\end{body}
			
			\begin{body}Entremos también nosotros en la “sala grande en el piso de arriba, arreglada con divanes” (\textit{Mc }14, 15) y dispongámonos a escuchar los pensamientos más íntimos que quiere comunicarnos; dispongámonos, en particular, a acoger el \textit{gesto} y el \textit{don} que ha preparado para esta última cita.\end{body}
			
			\begin{body}2. Mientras están cenando, Jesús se levanta de la mesa y comienza a \textit{lavar los pies a los discípulos. }Pedro, al principio, se resiste; luego, comprende y acepta. También a nosotros se nos invita a comprender: \textit{lo primero} que el discípulo debe hacer es ponerse a la escucha de su Señor, abriendo el corazón para \textit{acoger la iniciativa de su amor. }Sólo después será invitado a reproducir a su vez lo que ha hecho el Maestro. También él deberá “lavar los pies” a sus hermanos, traduciendo en gestos de servicio mutuo ese amor, que constituye la síntesis de todo el Evangelio (cf. \textit{Jn} 13, 1-20).\end{body}
			
			\begin{body}También durante la Cena, sabiendo que ya había llegado su “hora”, Jesús bendice y \textit{parte el pan}, luego lo distribuye a los Apóstoles, diciendo: “Esto es mi cuerpo”; lo mismo hace con \textit{el cáliz}: “Esta es mi sangre”. Y les manda: “Haced esto en conmemoración mía” (\textit{1 Co} 11, 24-25). Realmente aquí se manifiesta el testimonio de un amor llevado “hasta el extremo” (\textit{Jn} 13, 1). Jesús se da como alimento a los discípulos para llegar a ser uno con ellos. Una vez más se pone de relieve la “lección” que debemos aprender: \textit{lo primero} que hemos de hacer es \textit{abrir el corazón a la acogida del amor de Cristo}. La iniciativa es suya: su amor es lo que nos hace capaces de amar también nosotros a nuestros hermanos.\end{body}
			
			\begin{body}Así pues, el lavatorio de los pies y el sacramento de la Eucaristía son \textit{dos manifestaciones de un mismo misterio de amor} confiado a los discípulos “para que –dice Jesús– lo que yo he hecho con vosotros, vosotros también lo hagáis” (\textit{Jn} 13, 15).\end{body}
			
			\begin{body}3. “Haced esto en conmemoración mía” (\textit{1 Co} 11, 24). La “memoria” que el Señor nos dejó aquella noche se refiere al \textit{momento culminante de su existencia terrena}, es decir, el momento de su ofrenda sacrificial al Padre por amor a la humanidad. Y es una “memoria” que se sitúa en el marco de una cena, la cena pascual, en la que Jesús se da a sus Apóstoles bajo las especies del pan y del vino, como su alimento en el camino hacia la patria del cielo.\end{body}
			
			\begin{body}\textit{Mysterium fidei!} Así proclama el celebrante después de pronunciar las palabras de la consagración. Y la asamblea litúrgica responde expresando con alegría su fe y su adhesión, llena de esperanza. ¡Misterio realmente grande es la Eucaristía! Misterio “incomprensible” para la razón humana, pero sumamente luminoso para los ojos de la fe. La mesa del Señor en la sencillez de los símbolos eucarísticos –el pan y el vino compartidos– es también \textit{la mesa de la fraternidad concreta}. El mensaje que brota de ella es demasiado claro como para ignorarlo: todos los que participan en la celebración eucarística \textit{no pueden quedar insensibles} ante las expectativas de los pobres y los necesitados.\end{body}
			
			\begin{body}4. \begin{bodysmall}[Precisamente desde esta perspectiva deseo que \end{bodysmall}\textit{los donativos}\begin{bodysmall} que se recojan durante esta celebración sirvan para aliviar \end{bodysmall}\textit{las urgentes necesidades de los que sufren en Irak}\begin{bodysmall} por las consecuencias de la guerra.]\end{bodysmall} Un corazón que ha experimentado el amor del Señor se abre espontáneamente a la caridad hacia sus hermanos.\end{body}
			
			\begin{body}“\textit{O sacrum convivium, in quo Christus sumitur}”.\end{body}
			
			\begin{body}Hoy estamos todos invitados a celebrar y adorar, hasta muy entrada la noche, al Señor que se hizo alimento para nosotros, peregrinos en el tiempo, dándonos su carne y su sangre.\end{body}
			
			\begin{body}La Eucaristía \textit{es un gran don para la Iglesia y para el mundo}. \begin{bodysmall}[Precisamente para que se preste una atención cada vez más profunda al sacramento de la Eucaristía, he querido entregar a toda la comunidad de los creyentes \end{bodysmall}\textit{una encíclica}\begin{bodysmall}, cuyo tema central es el misterio eucarístico:\end{bodysmall}\begin{bodysmall} \end{bodysmall}\textit{Ecclesia de Eucharistia}\begin{bodysmall}. Dentro de poco tendré la alegría de firmarla durante esta celebración, que evoca la última Cena, cuando Jesús nos dejó a sí mismo como supremo testamento de amor. La encomiendo desde ahora, en primer lugar, a los sacerdotes, para que ellos, a su vez, la difundan para bien de todo el pueblo cristiano.]\end{bodysmall}\end{body}
			
			\begin{body}5.\textit{ Adoro te devote, latens Deitas!} Te adoramos, oh admirable sacramento de la presencia de Aquel que amó a los suyos “hasta el extremo”. Te damos gracias, Señor, que en la Eucaristía edificas, congregas y vivificas a la Iglesia.\end{body}
			
			\begin{body}¡Oh divina Eucaristía, llama del amor de Cristo, que ardes en el altar del mundo, haz que la Iglesia, confortada por ti, sea cada vez más solícita para enjugar las lágrimas de los que sufren y sostener los esfuerzos de los que anhelan la justicia y la paz!\end{body}
			
			\begin{body}Y tú, María, mujer “eucarística”, que ofreciste tu seno virginal para la encarnación del Verbo de Dios, \textit{ayúdanos a vivir el misterio eucarístico con el espíritu del Magníficat}. Que nuestra vida sea una alabanza sin fin al Todopoderoso, que se ocultó bajo la humildad de los signos eucarísticos.\end{body}
			
			\begin{body}\textit{Adoro te devote, latens Deitas...}\end{body}
			
			\begin{body}\textit{Adoro te..., adiuva me!}\end{body}
			
			\subsection{Benedicto XVI, papa}
			
			\subsubsection{Homilía (2006): El gesto de Jesús y su significado}
			
			\begin{referencia}13 de abril del 2006.\end{referencia}
			
			\begin{body}“Habiendo amado a los suyos que estaban en el mundo, los amó hasta el extremo” (\textit{Jn} 13, 1). Dios ama a su criatura, el hombre; lo ama también en su caída y no lo abandona a sí mismo. Él ama hasta el fin. Lleva su amor hasta el final, hasta el extremo: baja de su gloria divina. Se desprende de las vestiduras de su gloria divina y se viste con ropa de esclavo. Baja hasta la extrema miseria de nuestra caída. Se arrodilla ante nosotros y desempeña el servicio del esclavo; lava nuestros pies sucios, para que podamos ser admitidos a la mesa de Dios, para hacernos dignos de sentarnos a su mesa, algo que por nosotros mismos no podríamos ni deberíamos hacer jamás.\end{body}
			
			\begin{body}Dios no es un Dios lejano, demasiado distante y demasiado grande como para ocuparse de nuestras bagatelas. Dado que es grande, puede interesarse también de las cosas pequeñas. Dado que es grande, el alma del hombre, el hombre mismo, creado por el amor eterno, no es algo pequeño, sino que es grande y digno de su amor. La santidad de Dios no es sólo un poder incandescente, ante el cual debemos alejarnos aterrorizados; es poder de amor y, por esto, es poder purificador y sanador.\end{body}
			
			\begin{body}Dios desciende y se hace esclavo; nos lava los pies para que podamos sentarnos a su mesa. Así se revela todo el misterio de Jesucristo. Así resulta manifiesto lo que significa redención. El baño con que nos lava es su amor dispuesto a afrontar la muerte. Sólo el amor tiene la fuerza purificadora que nos limpia de nuestra impureza y nos eleva a la altura de Dios. El baño que nos purifica es él mismo, que se entrega totalmente a nosotros, desde lo más profundo de su sufrimiento y de su muerte.\end{body}
			
			\begin{body}Él es continuamente este amor que nos lava. En los sacramentos de la purificación –el Bautismo y la Penitencia– él está continuamente arrodillado ante nuestros pies y nos presta el servicio de esclavo, el servicio de la purificación; nos hace capaces de Dios. Su amor es inagotable; llega realmente hasta el extremo.\end{body}
			
			\begin{body}“Vosotros estáis limpios, pero no todos”, dice el Señor (\textit{Jn} 13, 10). En esta frase se revela el gran don de la purificación que él nos hace, porque desea estar a la mesa juntamente con nosotros, de convertirse en nuestro alimento. “Pero no todos”: existe el misterio oscuro del rechazo, que con la historia de Judas se hace presente y debe hacernos reflexionar precisamente en el Jueves santo, el día en que Jesús nos hace el don de sí mismo. El amor del Señor no tiene límites, pero el hombre puede ponerle un límite.\end{body}
			
			\begin{body}“Vosotros estáis limpios, pero no todos”: ¿Qué es lo que hace impuro al hombre? Es el rechazo del amor, el no querer ser amado, el no amar. Es la soberbia que cree que no necesita purificación, que se cierra a la bondad salvadora de Dios. Es la soberbia que no quiere confesar y reconocer que necesitamos purificación.\end{body}
			
			\begin{body}En Judas vemos con mayor claridad aún la naturaleza de este rechazo. Juzga a Jesús según las categorías del poder y del éxito: para él sólo cuentan el poder y el éxito; el amor no cuenta. Y es avaro: para él el dinero es más importante que la comunión con Jesús, más importante que Dios y su amor. Así se transforma también en un mentiroso, que hace doble juego y rompe con la verdad; uno que vive en la mentira y así pierde el sentido de la verdad suprema, de Dios. De este modo se endurece, se hace incapaz de conversión, del confiado retorno del hijo pródigo, y arruina su vida.\end{body}
			
			\begin{body}“Vosotros estáis limpios, pero no todos”. El Señor hoy nos pone en guardia frente a la autosuficiencia, que pone un límite a su amor ilimitado. Nos invita a imitar su humildad, a tratar de vivirla, a dejarnos “contagiar” por ella. Nos invita –por más perdidos que podamos sentirnos– a volver a casa y a permitir a su bondad purificadora que nos levante y nos haga entrar en la comunión de la mesa con él, con Dios mismo.\end{body}
			
			\begin{body}Reflexionemos sobre otra frase de este inagotable pasaje evangélico: “Os he dado ejemplo...” (\textit{Jn} 13, 15); “También vosotros debéis lavaros los pies unos a otros” (\textit{Jn} 13, 14). ¿En qué consiste el “lavarnos los pies unos a otros”? ¿Qué significa en concreto? Cada obra buena hecha en favor del prójimo, especialmente en favor de los que sufren y los que son poco apreciados, es un servicio como lavar los pies. El Señor nos invita a bajar, a aprender la humildad y la valentía de la bondad; y también a estar dispuestos a aceptar el rechazo, actuando a pesar de ello con bondad y perseverando en ella.\end{body}
			
			\begin{body}Pero hay una dimensión aún más profunda. El Señor limpia nuestra impureza con la fuerza purificadora de su bondad. Lavarnos los pies unos a otros significa sobre todo perdonarnos continuamente unos a otros, volver a comenzar juntos siempre de nuevo, aunque pueda parecer inútil. Significa purificarnos unos a otros soportándonos mutuamente y aceptando ser soportados por los demás; purificarnos unos a otros dándonos recíprocamente la fuerza santificante de la palabra de Dios e introduciéndonos en el Sacramento del amor divino.\end{body}
			
			\begin{body}El Señor nos purifica; por esto nos atrevemos a acercarnos a su mesa. Pidámosle que nos conceda a todos la gracia de poder ser un día, para siempre, huéspedes del banquete nupcial eterno. Amén.\end{body}
			
			\subsubsection{Homilía (2009): Entender lo que ocurrió en aquella Cena}
			
			\begin{referencia}9 de abril del 2009.\end{referencia}
			
			\begin{body}\textit{Qui, pridie quam pro nostra omniumque salute pateretur, hoc est hodie, accepit panem. }Así diremos hoy en el Canon de la Santa Misa. \textit{“Hoc est hodie”}. La Liturgia del Jueves Santo incluye la palabra “hoy” en el texto de la plegaria, subrayando con ello la dignidad particular de este día. Ha sido “hoy” cuando Él lo ha hecho: se nos ha entregado para siempre en el Sacramento de su Cuerpo y de su Sangre. Este “hoy” es sobre todo el memorial de la Pascua de entonces. Pero es más aún. Con el Canon entramos en este “hoy”. Nuestro hoy se encuentra con su hoy. Él hace esto ahora. Con la palabra “hoy”, la Liturgia de la Iglesia quiere inducirnos a que prestemos gran atención interior al misterio de este día, a las palabras con que se expresa. Tratemos, pues, de escuchar de modo nuevo el relato de la institución, tal y como la Iglesia lo ha formulado basándose en la Escritura y contemplando al Señor mismo.\end{body}
			
			\begin{body}Lo primero que nos sorprende es que el relato de la institución no es una frase suelta, sino que empieza con un pronombre relativo: \textit{qui pridie}. Este “\textit{qui}”\textit{ }enlaza todo el relato con la palabra precedente de la oración, “...de manera que sea para nosotros Cuerpo y Sangre de tu Hijo amado, Jesucristo, nuestro Señor”. De este modo, el relato está unido a la oración anterior, a todo el Canon, y se hace él mismo oración. En efecto, en modo alguno se trata de un relato sencillamente insertado aquí; tampoco se trata de palabras aisladas de autoridad, que quizás interrumpirían la oración. Es oración. Y solamente en la oración se cumple el acto sacerdotal de la consagración que se convierte en transformación, transustanciación de nuestros dones de pan y vino en el Cuerpo y la Sangre de Cristo. Rezando en este momento central, la Iglesia concuerda totalmente con el acontecimiento del Cenáculo, ya que el actuar de Jesús se describe con las palabras:\textit{ “gratias agens benedixit}” – “te dio gracias con la plegaria de bendición”. Con esta expresión, la Liturgia romana ha dividido en dos palabras, lo que en hebreo es una sola, \textit{berakha}, que en griego, en cambio, aparece en los dos términos de \textit{eucharistía} y \textit{eulogía}. El Señor agradece. Al agradecer, reconocemos que una cosa determinada es un don de otro. El Señor agradece, y de este modo restituye a Dios el pan, “fruto de la tierra y del trabajo del hombre”, para poder recibirlo nuevamente de Él. Agradecer se transforma en bendecir. Lo que ha sido puesto en las manos de Dios, vuelve de Él bendecido y transformado. Por tanto, la Liturgia romana tiene razón al interpretar nuestro orar en este momento sagrado con las palabras: “ofrecemos”, “pedimos”, “acepta”, “bendice esta ofrenda”. Todo esto se oculta en la palabra \textit{eucharistia.}\end{body}
			
			\begin{body}Hay otra particularidad en el relato de la institución del Canon Romano que queremos meditar en esta hora. La Iglesia orante se fija en las manos y los ojos del Señor. Quiere casi observarlo, desea percibir el gesto de su orar y actuar en aquella hora singular, encontrar la figura de Jesús, por decirlo así, también a través de los sentidos. “Tomó pan en sus santas y venerables manos”. Nos fijamos en las manos con las que Él ha curado a los hombres; en las manos con las que ha bendecido a los niños; en las manos que ha impuesto sobre los hombres; en las manos clavadas en la Cruz y que llevarán siempre los estigmas como signos de su amor dispuesto a morir. Ahora tenemos el encargo de hacer lo que Él ha hecho: tomar en las manos el pan para que sea convertido mediante la plegaria eucarística. En la Ordenación sacerdotal, nuestras manos fueron ungidas, para que fuesen manos de bendición. Pidamos al Señor ahora que nuestras manos sirvan cada vez más para llevar la salvación, para llevar la bendición, para hacer presente su bondad.\end{body}
			
			\begin{body}De la introducción a la Oración sacerdotal de Jesús (cf. \textit{Jn} 17, 1), el Canon usa luego las palabras: “elevando los ojos al cielo, hacia ti, Dios, Padre suyo todopoderoso”. El Señor nos enseña a levantar los ojos y sobre todo el corazón. A levantar la mirada, apartándola de las cosas del mundo, a orientarnos hacia Dios en la oración y así elevar nuestro ánimo. En un himno de la Liturgia de las Horas pedimos al Señor que custodie nuestros ojos, para que no acojan ni dejen que en nosotros entren las “\textit{vanitates}”, las vanidades, la banalidad, lo que sólo es apariencia. Pidamos que a través de los ojos no entre el mal en nosotros, falsificando y ensuciando así nuestro ser. Pero queremos pedir sobre todo que tengamos ojos que vean todo lo que es verdadero, luminoso y bueno, para que seamos capaces de ver la presencia de Dios en el mundo. Pidamos, para que miremos el mundo con ojos de amor, con los ojos de Jesús, reconociendo así a los hermanos y las hermanas que nos necesitan, que están esperando nuestra palabra y nuestra acción.\end{body}
			
			\begin{body}Después de bendecir, el Señor parte el pan y lo da a los discípulos. Partir el pan es el gesto del padre de familia que se preocupa de los suyos y les da lo que necesitan para la vida. Pero es también el gesto de la hospitalidad con que se acoge al extranjero, al huésped, y se le permite participar en la propia vida. Dividir, com-partir, es unir. A través del compartir se crea comunión. En el pan partido, el Señor se reparte a sí mismo. El gesto del partir alude misteriosamente también a su muerte, al amor hasta la muerte. Él se da a sí mismo, que es el verdadero “pan para la vida del mundo” (cf. \textit{Jn} 6, 51). El alimento que el hombre necesita en lo más hondo es la comunión con Dios mismo. Al agradecer y bendecir, Jesús transforma el pan, y ya no es pan terrenal lo que da, sino la comunión consigo mismo. Esta transformación, sin embargo, quiere ser el comienzo de la transformación del mundo. Para que llegue a ser un mundo de resurrección, un mundo de Dios. Sí, se trata de transformación. Del hombre nuevo y del mundo nuevo que comienzan en el pan consagrado, transformado, transustanciado.\end{body}
			
			\begin{body}Hemos dicho que partir el pan es un gesto de comunión, de unir mediante el compartir. Así, en el gesto mismo se alude ya a la naturaleza íntima de la Eucaristía: ésta es \textit{agape}, es amor hecho corpóreo. En la palabra “\textit{agape}”,\textit{ }se compenetran los significados de Eucaristía y amor. En el gesto de Jesús que parte el pan, el amor que se comparte ha alcanzado su extrema radicalidad: Jesús se deja partir como pan vivo. En el pan distribuido reconocemos el misterio del grano de trigo que muere y así da fruto. Reconocemos la nueva multiplicación de los panes, que deriva del morir del grano de trigo y continuará hasta el fin del mundo. Al mismo tiempo vemos que la Eucaristía nunca puede ser sólo una acción litúrgica. Sólo es completa, si el \textit{agape} litúrgico se convierte en amor cotidiano. En el culto cristiano, las dos cosas se transforman en una, el ser agraciados por el Señor en el acto cultual y el cultivo del amor respecto al prójimo. Pidamos en esta hora al Señor la gracia de aprender a vivir cada vez mejor el misterio de la Eucaristía, de manera que comience así la transformación del mundo.\end{body}
			
			\begin{body}Después del pan, Jesús toma el cáliz de vino. El Canon Romano designa el cáliz que el Señor da a los discípulos, como \textit{“praeclarus calix”}, cáliz glorioso, aludiendo con ello al Salmo 23 [22], el Salmo que habla de Dios como del Pastor poderoso y bueno. En él se lee: “preparas una mesa ante mí, enfrente de mis enemigos; ...y mi copa rebosa” (v. 5), \textit{calix praeclarus}. El Canon Romano interpreta esta palabra del Salmo como una profecía que se cumple en la Eucaristía. Sí, el Señor nos prepara la mesa en medio de las amenazas de este mundo, y nos da el cáliz glorioso, el cáliz de la gran alegría, de la fiesta verdadera que todos anhelamos, el cáliz rebosante del vino de su amor. El cáliz significa la boda: ahora ha llegado “la hora” a la que en las bodas de Caná se aludía de forma misteriosa. Sí, la Eucaristía es más que un banquete, es una fiesta de boda. Y esta boda se funda en la autodonación de Dios hasta la muerte. En las palabras de la última Cena de Jesús y en el Canon de la Iglesia, el misterio solemne de la boda se esconde bajo la expresión \textit{“novum Testamentum}”. Este cáliz es el nuevo Testamento, “la nueva Alianza sellada con mi sangre”, según la palabra de Jesús sobre el cáliz, que Pablo transmite en la \textbf{segunda lectura} de hoy (cf. \textit{1 Co} 11, 25). El Canon Romano añade: “de la alianza nueva y eterna”, para expresar la indisolubilidad del vínculo nupcial de Dios con la humanidad. El motivo por el cual las traducciones antiguas de la Biblia no hablan de Alianza, sino de Testamento, es que no se trata de dos contrayentes iguales quienes la establecen, sino que entra en juego la infinita distancia entre Dios y el hombre. Lo que nosotros llamamos nueva y antigua Alianza no es un acuerdo entre dos partes iguales, sino un mero don de Dios, que nos deja como herencia su amor, a sí mismo. Y ciertamente, a través de este don de su amor Él, superando cualquier distancia, nos convierte verdaderamente en \textit{partner} y se realiza el misterio nupcial del amor.\end{body}
			
			\begin{body}Para poder comprender lo que allí ocurre en profundidad, hemos de escuchar más cuidadosamente aún las palabras de la Biblia y su sentido originario. Los estudiosos nos dicen que, en los tiempos remotos de que hablan las historias de los Patriarcas de Israel, “ratificar una alianza” significaba “entrar con otros en una unión fundada en la sangre, o bien acoger a alguien en la propia federación y entrar así en una comunión de derechos recíprocos”. De este modo se crea una consanguinidad real, aunque no material. Los aliados se convierten en cierto modo en “hermanos de la misma carne y la misma sangre”. La alianza realiza un conjunto que significa paz (cf. ThWNT II 105-137). ¿Podemos ahora hacernos al menos una idea de lo que ocurrió en la hora de la última Cena y que, desde entonces, se renueva cada vez que celebramos la Eucaristía? Dios, el Dios vivo establece con nosotros una comunión de paz, más aún, Él crea una “consanguinidad” entre Él y nosotros. Por la encarnación de Jesús, por su sangre derramada, hemos sido injertados en una consanguinidad muy real con Jesús y, por tanto, con Dios mismo. La sangre de Jesús es su amor, en el que la vida divina y la humana se han hecho una cosa sola. Pidamos al Señor que comprendamos cada vez más la grandeza de este misterio. Que Él despliegue su fuerza trasformadora en nuestro interior, de modo que lleguemos a ser realmente consanguíneos de Jesús, llenos de su paz y, así, también en comunión unos con otros.\end{body}
			
			\begin{body}Sin embargo, ahora surge aún otra pregunta. En el Cenáculo, Cristo entrega a los discípulos su Cuerpo y su Sangre, es decir, Él mismo en la totalidad de su persona. Pero, ¿puede hacerlo? Todavía está físicamente presente entre ellos, está ante ellos. La respuesta es que, en aquella hora, Jesús cumple lo que previamente había anunciado en el discurso sobre el Buen Pastor: “Nadie me quita la vida, sino que yo la entrego libremente. Tengo poder para entregarla y tengo poder para recuperarla” (cf. \textit{Jn} 10, 18). Nadie puede quitarle la vida: la da por libre decisión. En aquella hora anticipa la crucifixión y la resurrección. Lo que, por decirlo así, se cumplirá físicamente en Él, Él ya lo lleva a cabo anticipadamente en la libertad de su amor. Él entrega su vida y la recupera en la resurrección para poderla compartir para siempre.\end{body}
			
			\begin{body}Señor, Tú nos entregas hoy tu vida, Tú mismo te nos das. Llénanos de tu amor. Haznos vivir en tu “hoy”. Haznos instrumentos de tu paz. Amén.\end{body}
			
			\subsubsection{Homilía (2012): Obediencia y libertad}
			
			\begin{referencia}Basílica de San Juan de Letrán. 5 de abril de 2012.\end{referencia}
			
			\begin{body}El Jueves Santo no es sólo el día de la Institución de la Santa Eucaristía, cuyo esplendor ciertamente se irradia sobre todo lo demás y, por así decir, lo atrae dentro de sí. También forma parte del Jueves Santo la noche oscura del Monte de los Olivos, hacia la cual Jesús se dirige con sus discípulos; forma parte también la soledad y el abandono de Jesús que, orando, va al encuentro de la oscuridad de la muerte; forma parte de este Jueves Santo la traición de Judas y el arresto de Jesús, así como también la negación de Pedro, la acusación ante el Sanedrín y la entrega a los paganos, a Pilato. En esta hora, tratemos de comprender con más profundidad estos eventos, porque en ellos se lleva a cabo el misterio de nuestra Redención.\end{body}
			
			\begin{body}Jesús sale en la noche. La noche significa falta de comunicación, una situación en la que uno no ve al otro. Es un símbolo de la incomprensión, del ofuscamiento de la verdad. Es el espacio en el que el mal, que debe esconderse ante la luz, puede prosperar. Jesús mismo es la luz y la verdad, la comunicación, la pureza y la bondad. Él entra en la noche. La noche, en definitiva, es símbolo de la muerte, de la pérdida definitiva de comunión y de vida. Jesús entra en la noche para superarla e inaugurar el nuevo día de Dios en la historia de la humanidad.\end{body}
			
			\begin{body}Durante este camino, él ha cantado con sus Apóstoles los Salmos de la liberación y de la redención de Israel, que recuerdan la primera Pascua en Egipto, la noche de la liberación. Como él hacía con frecuencia, ahora se va a orar solo y hablar como Hijo con el Padre. Pero, a diferencia de lo acostumbrado, quiere cerciorarse de que estén cerca tres discípulos: Pedro, Santiago y Juan. Son los tres que habían tenido la experiencia de su Transfiguración –la manifestación luminosa de la gloria de Dios a través de su figura humana– y que lo habían visto en el centro, entre la Ley y los Profetas, entre Moisés y Elías. Habían escuchado cómo hablaba con ellos de su “éxodo” en Jerusalén. El éxodo de Jesús en Jerusalén, ¡qué palabra misteriosa!; el éxodo de Israel de Egipto había sido el episodio de la fuga y la liberación del pueblo de Dios. ¿Qué aspecto tendría el éxodo de Jesús, en el cual debía cumplirse definitivamente el sentido de aquel drama histórico?; ahora, los discípulos son testigos del primer tramo de este éxodo, de la extrema humillación que, sin embargo, era el paso esencial para salir hacia la libertad y la vida nueva, hacia la que tiende el éxodo. Los discípulos, cuya cercanía quiso Jesús en esta hora de extrema tribulación, como elemento de apoyo humano, pronto se durmieron. No obstante, escucharon algunos fragmentos de las palabras de la oración de Jesús y observaron su actitud. Ambas cosas se grabaron profundamente en sus almas, y ellos las transmitieron a los cristianos para siempre. Jesús llama a Dios “Abbá”. Y esto significa –como ellos añaden– “Padre”. Pero no de la manera en que se usa habitualmente la palabra “padre”, sino como expresión del lenguaje de los niños, una palabra afectuosa con la cual no se osaba dirigirse a Dios. Es el lenguaje de quien es verdaderamente “niño”, Hijo del Padre, de aquel que se encuentra en comunión con Dios, en la más profunda unidad con él.\end{body}
			
			\begin{body}Si nos preguntamos cuál es el elemento más característico de la imagen de Jesús en los evangelios, debemos decir: su relación con Dios. Él está siempre en comunión con Dios. El ser con el Padre es el núcleo de su personalidad. A través de Cristo, conocemos verdaderamente a Dios. “A Dios nadie lo ha visto jamás”, dice san Juan. Aquel “que está en el seno del Padre… lo ha dado a conocer” (1, 18). Ahora conocemos a Dios tal como es verdaderamente. Él es Padre, bondad absoluta a la que podemos encomendarnos. El evangelista Marcos, que ha conservado los recuerdos de Pedro, nos dice que Jesús, al apelativo “Abbá”, añadió aún: Todo es posible para ti, tú lo puedes todo (cf. 14, 36). Él, que es la bondad, es al mismo tiempo poder, es omnipotente. El poder es bondad y la bondad es poder. Esta confianza la podemos aprender de la oración de Jesús en el Monte de los Olivos.\end{body}
			
			\begin{body}Antes de reflexionar sobre el contenido de la petición de Jesús, debemos prestar atención a lo que los evangelistas nos relatan sobre la actitud de Jesús durante su oración. Mateo y Marcos dicen que “cayó rostro en tierra” (\textit{Mt} 26, 39; cf. \textit{Mc} 14, 35); asume por consiguiente la actitud de total sumisión, que ha sido conservada en la liturgia romana del Viernes Santo. Lucas, en cambio, afirma que Jesús oraba arrodillado. En los Hechos de los Apóstoles, habla de los santos, que oraban de rodillas: Esteban durante su lapidación, Pedro en el contexto de la resurrección de un muerto, Pablo en el camino hacia el martirio. Así, Lucas ha trazado una pequeña historia del orar arrodillados de la Iglesia naciente. Los cristianos, al arrodillarse, se ponen en comunión con la oración de Jesús en el Monte de los Olivos. En la amenaza del poder del mal, ellos, en cuanto arrodillados, están de pie ante el mundo, pero, en cuanto hijos, están de rodillas ante el Padre. Ante la gloria de Dios, los cristianos nos arrodillamos y reconocemos su divinidad, pero expresando también en este gesto nuestra confianza en que él triunfe.\end{body}
			
			\begin{body}Jesús forcejea con el Padre. Combate consigo mismo. Y combate por nosotros. Experimenta la angustia ante el poder de la muerte. Esto es ante todo la turbación propia del hombre, más aún, de toda creatura viviente ante la presencia de la muerte. En Jesús, sin embargo, se trata de algo más. En las noches del mal, él ensancha su mirada. Ve la marea sucia de toda la mentira y de toda la infamia que le sobreviene en aquel cáliz que debe beber. Es el estremecimiento del totalmente puro y santo frente a todo el caudal del mal de este mundo, que recae sobre él. Él también me ve, y ora también por mí. Así, este momento de angustia mortal de Jesús es un elemento esencial en el proceso de la Redención. Por eso, la Carta a los Hebreos ha definido el combate de Jesús en el Monte de los Olivos como un acto sacerdotal. En esta oración de Jesús, impregnada de una angustia mortal, el Señor ejerce el oficio del sacerdote: toma sobre sí el pecado de la humanidad, a todos nosotros, y nos conduce al Padre.\end{body}
			
			\begin{body}Finalmente, debemos prestar atención aún al contenido de la oración de Jesús en el Monte de los Olivos. Jesús dice: “Padre: tú lo puedes todo, aparta de mí ese cáliz. Pero no sea como yo quiero, sino como tú quieres” (\textit{Mc} 14, 36). La voluntad natural del hombre Jesús retrocede asustada ante algo tan ingente. Pide que se le evite eso. Sin embargo, en cuanto Hijo, abandona esta voluntad humana en la voluntad del Padre: no yo, sino tú. Con esto ha transformado la actitud de Adán, el pecado primordial del hombre, salvando de este modo al hombre. La actitud de Adán había sido: No lo que tú has querido, Dios; quiero ser dios yo mismo. Esta soberbia es la verdadera esencia del pecado. Pensamos ser libres y verdaderamente nosotros mismos sólo si seguimos exclusivamente nuestra voluntad. Dios aparece como el antagonista de nuestra libertad. Debemos liberarnos de él, pensamos nosotros; sólo así seremos libres. Esta es la rebelión fundamental que atraviesa la historia, y la mentira de fondo que desnaturaliza la vida. Cuando el hombre se pone contra Dios, se pone contra la propia verdad y, por tanto, no llega a ser libre, sino alienado de sí mismo. Únicamente somos libres si estamos en nuestra verdad, si estamos unidos a Dios. Entonces nos hacemos verdaderamente “como Dios”, no oponiéndonos a Dios, no desentendiéndonos de él o negándolo. En el forcejeo de la oración en el Monte de los Olivos, Jesús ha deshecho la falsa contradicción entre obediencia y libertad, y abierto el camino hacia la libertad. Oremos al Señor para que nos adentre en este “sí” a la voluntad de Dios, haciéndonos verdaderamente libres. Amén.\end{body}
			
			\subsection{Francisco, papa}
			
			\subsubsection{Homilía (2015): Se hizo esclavo por nosotros}
			
			\begin{referencia}Iglesia “Padre Nuestro”, Nuevo Complejo Penitenciario de Rebibbia, Roma.\end{referencia}
			
			\begin{referencia}2 de abril de 2015.\end{referencia}
			
			\begin{body}Este jueves, Jesús estaba en la mesa con los discípulos, celebrando la fiesta de la Pascua. Y el pasaje del \textbf{Evangelio} que hemos escuchado contiene una frase que es precisamente el centro de lo que hizo Jesús por todos nosotros: “Habiendo amado a los suyos que estaban en el mundo, los amó hasta el extremo” (\textit{Jn} 13, 1). Jesús nos amó. Jesús nos ama. Sin límites, siempre, hasta el extremo. El amor de Jesús por nosotros no tiene límites: cada vez más, cada vez más. No se cansa de amar. A ninguno. Nos ama a todos nosotros, hasta el punto de dar la vida por nosotros. Sí, dar la vida por nosotros; sí, dar la vida por todos nosotros, dar la vida por cada uno de nosotros. Y cada uno puede decir: “Dio la vida por mí”. Por cada uno. Ha dado la vida por ti, por ti, por ti, por mí, por él… por cada uno, con nombre y apellido. Su amor es así: personal. El amor de Jesús nunca defrauda, porque Él no se cansa de amar, como no se cansa de perdonar, no se cansa de abrazarnos. Esta es la primera cosa que quería deciros: Jesús nos amó, a cada uno de nosotros, hasta el extremo.\end{body}
			
			\begin{body}Y luego, hizo lo que los discípulos no comprendieron: lavar los pies. En ese tiempo era habitual, era una costumbre, porque cuando la gente llegaba a una casa tenía los pies sucios por el polvo del camino; no existían los adoquines en ese tiempo… Había polvo por el camino. Y en el ingreso de la casa se lavaban los pies. Pero esto no lo hacía el dueño de casa, lo hacían los esclavos. Era un trabajo de esclavos. Y Jesús lava como esclavo nuestros pies, los pies de los discípulos, y por eso dice: “Lo que yo hago, tú no lo entiendes ahora –dice a Pedro–, pero lo comprenderás más tarde” (\textit{Jn} 13, 7). Es tan grande el amor de Jesús que se hizo esclavo para servirnos, para curarnos, para limpiarnos.\end{body}
			
			\begin{body}Y hoy, en esta misa, la Iglesia quiere que el sacerdote lave los pies de doce personas, en memoria de los doce apóstoles. Pero en nuestro corazón debemos tener la certeza, debemos estar seguros de que el Señor, cuando nos lava los pies, nos lava todo, nos purifica, nos hace sentir de nuevo su amor. En la Biblia hay una frase, del profeta Isaías, muy bella, que dice: “¿Puede una madre olvidar a su hijo? Aunque ella se olvidara de su hijo, yo nunca me olvidaré de ti” (cf. 49, 15). Así es el amor de Dios por nosotros.\end{body}
			
			\begin{body}Y yo lavaré hoy los pies de doce de vosotros, pero en estos hermanos y hermanas estáis todos vosotros, todos, todos. Todos los que viven aquí. Vosotros los representáis a ellos. Y también yo necesito ser lavado por el Señor, y por eso rezad durante esta misa para que el Señor lave también mis suciedades, para que yo llegue a ser un mejor siervo vuestro, un mejor siervo al servicio de la gente, como lo fue Jesús.\end{body}
			
			\begin{body}Ahora comenzaremos esta parte de la celebración.\end{body}
			
			\subsubsection{Homilía (2018): Jesús arriesga por nosotros}
			
			\begin{referencia}Cárcel de “Regina Coeli”, Roma.\end{referencia}
			
			\begin{referencia}29 de marzo de 2018.\end{referencia}
			
			\begin{body}Jesús termina su discurso diciendo: “Porque os he dado ejemplo, para que también vosotros hagáis como yo he hecho con vosotros” (\textit{Jn} 13, 15). Lavar los pies. Los pies, en esa época, eran lavados por los esclavos. La gente recorría el camino, no había asfalto, no había “sanpietrini”; en aquel tiempo había polvo en el camino y la gente se manchaba los pies. Y en la entrada de la casa estaban los esclavos que lavaban los pies. Era un trabajo para esclavos. Pero era un servicio: un servicio hecho por esclavos. Y Jesús quiere hacer este servicio, para darnos un ejemplo de cómo nosotros debemos servirnos los unos a los otros. \end{body}
			
			\begin{body}Una vez, cuando estaban en camino, dos de los discípulos que querían hacer carrera, habían pedido a Jesús ocupar puestos importantes, uno a la derecha y otro a la izquierda (cf. \textit{Mc} 10, 35-45). Y Jesús los miró con amor –Jesús miraba siempre con amor– y dijo: “No sabéis lo que pedís” (v. 38). Los jefes de las naciones –dice Jesús– mandan, se hacen servir, y ellos están bien (cf. v. 42). Pensemos en esa época de los reyes, de los emperadores tan crueles, que se hacían servir por los esclavos... Pero entre vosotros –dice Jesús– no debe ser lo mismo: quien manda debe servir. Vuestro jefe debe ser vuestro servidor (cf. v. 43). Jesús da la vuelta a la costumbre histórica, cultural de esa época –también esta de hoy– aquel que manda, para ser un buen jefe, sea donde sea, debe servir. Yo pienso muchas veces –no en este tiempo porque cada uno todavía está vivo y tiene la oportunidad de cambiar de vida y no podemos juzgar, pero pensemos en la historia– si muchos reyes, emperadores, jefes de Estado hubieran entendido esta enseñanza de Jesús y en vez de mandar, ser crueles, matar gente, hubieran hecho esto, ¡cuántas guerras no se hubieran hecho! El servicio: realmente hay gente que no facilita esta actitud, gente soberbia, gente odiosa, gente que quizá nos desea el mal; pero nosotros estamos llamados a servirles más. Y también hay gente que sufre, que está descartada por la sociedad, al menos por un periodo, y Jesús va ahí a decirles: Tú eres importante para mí. Jesús viene a servirnos, y la señal que Jesús nos sirve hoy aquí, [en la cárcel de Regina Coeli,] es que ha querido elegir a doce de vosotros, como los doce apóstoles, para lavar los pies. Jesús arriesga sobre cada uno de nosotros. Sabed esto: Jesús se llama Jesús, no se llama Poncio Pilato. Jesús no sabe lavarse las manos: ¡solamente sabe arriesgar! Mirad esta imagen tan bonita: Jesús arrodillado entre las espinas, arriesgando herirse para tomar la oveja perdida. \end{body}
			
			\begin{body}Hoy yo, que soy pecador como vosotros, pero represento a Jesús, soy embajador de Jesús. Hoy, cuando yo me arrodillo delante de cada uno de vosotros, pensad: “Jesús ha arriesgado en este hombre, un pecador, para venir a mí y decirme que me ama”. Este es el servicio, este es Jesús: no nos abandona nunca, no se cansa nunca de perdonarnos. Nos ama mucho. ¡Mirad cómo arriesga Jesús! Y así, con estos sentimientos, vamos adelante con esta ceremonia que es simbólica. Antes de darnos su cuerpo y su sangre, Jesús arriesga por cada uno de nosotros, y arriesga en el servicio porque nos ama mucho. \end{body}
			
			\section{Temas}
			
			\begin{ccetheme}Institución de la Eucaristía \end{ccetheme}
			
			\begin{ccereference}\end{ccereference}CEC 1337-1344:</p>
			
			\begin{ccebody}\textbf{La institución de la Eucaristía}\end{ccebody}
			
			\begin{ccebody}\begin{ccenumber}1337\end{ccenumber} El Señor, habiendo amado a los suyos, los amó hasta el fin. Sabiendo que había llegado la hora de partir de este mundo para retornar a su Padre, en el transcurso de una cena, les lavó los pies y les dio el mandamiento del amor (\textit{Jn} 13,1-17). Para dejarles una prenda de este amor, para no alejarse nunca de los suyos y hacerles partícipes de su Pascua, instituyó la Eucaristía como memorial de su muerte y de su resurrección y ordenó a sus apóstoles celebrarlo hasta su retorno, “constituyéndoles entonces sacerdotes del Nuevo Testamento” (Concilio de Trento: DS 1740).\end{ccebody}
			
			\begin{ccebody}\begin{ccenumber}1338\end{ccenumber} Los tres evangelios sinópticos y san Pablo nos han transmitido el relato de la institución de la Eucaristía; por su parte, san Juan relata las palabras de Jesús en la sinagoga de Cafarnaúm, palabras que preparan la institución de la Eucaristía: Cristo se designa a sí mismo como el pan de vida, bajado del cielo (cf. \textit{Jn} 6).\end{ccebody}
			
			\begin{ccebody}\begin{ccenumber}1339 \end{ccenumber}Jesús escogió el tiempo de la Pascua para realizar lo que había anunciado en Cafarnaúm: dar a sus discípulos su Cuerpo y su Sangre:\end{ccebody}
			
			\begin{ccecite}“Llegó el día de los Ázimos, en el que se había de inmolar el cordero de Pascua; [Jesús] envió a Pedro y a Juan, diciendo: ‘Id y preparadnos la Pascua para que la comamos’ [...] fueron [...] y prepararon la Pascua. Llegada la hora, se puso a la mesa con los Apóstoles; y les dijo: ‘Con ansia he deseado comer esta Pascua con vosotros antes de padecer; porque os digo que ya no la comeré más hasta que halle su cumplimiento en el Reino de Dios’ [...] Y tomó pan, dio gracias, lo partió y se lo dio diciendo: ‘Esto es mi cuerpo que va a ser entregado por vosotros; haced esto en recuerdo mío’. De igual modo, después de cenar, tomó el cáliz, diciendo: ‘Este cáliz es la Nueva Alianza en mi sangre, que va a ser derramada por vosotros’” (\textit{Lc} 22,7-20; cf. \textit{Mt} 26,17-29; \textit{Mc} 14,12-25; \textit{1 Co} 11,23-26).\end{ccecite}
			
			\begin{ccebody}\begin{ccenumber}1340\end{ccenumber} Al celebrar la última Cena con sus Apóstoles en el transcurso del banquete pascual, Jesús dio su sentido definitivo a la pascua judía. En efecto, el paso de Jesús a su Padre por su muerte y su resurrección, la Pascua nueva, es anticipada en la Cena y celebrada en la Eucaristía que da cumplimiento a la pascua judía y anticipa la pascua final de la Iglesia en la gloria del Reino.\end{ccebody}
			
			\begin{ccebody}\textbf{“Haced esto en memoria mía”}\end{ccebody}
			
			\begin{ccebody}\begin{ccenumber}1341\end{ccenumber} El mandamiento de Jesús de repetir sus gestos y sus palabras “hasta que venga” (\textit{1 Co} 11,26), no exige solamente acordarse de Jesús y de lo que hizo. Requiere la celebración litúrgica por los Apóstoles y sus sucesores del \textit{memorial} de Cristo, de su vida, de su muerte, de su resurrección y de su intercesión junto al Padre.\end{ccebody}
			
			\begin{ccebody}\begin{ccenumber}1342\end{ccenumber} Desde el comienzo la Iglesia fue fiel a la orden del Señor. De la Iglesia de Jerusalén se dice:\end{ccebody}
			
			\begin{ccecite}“Acudían asiduamente a la enseñanza de los apóstoles, fieles a la comunión fraterna, a la fracción del pan y a las oraciones [...] Acudían al Templo todos los días con perseverancia y con un mismo espíritu, partían el pan por las casas y tomaban el alimento con alegría y con sencillez de corazón” (\textit{Hch} 2,42.46).\end{ccecite}
			
			\begin{ccebody}\begin{ccenumber}1343\end{ccenumber} Era sobre todo “el primer día de la semana”, es decir, el domingo, el día de la resurrección de Jesús, cuando los cristianos se reunían para “partir el pan” (\textit{Hch} 20,7). Desde entonces hasta nuestros días, la celebración de la Eucaristía se ha perpetuado, de suerte que hoy la encontramos por todas partes en la Iglesia, con la misma estructura fundamental. Sigue siendo el centro de la vida de la Iglesia.\end{ccebody}
			
			\begin{ccebody}\begin{ccenumber}1344\end{ccenumber} Así, de celebración en celebración, anunciando el misterio pascual de Jesús “hasta que venga” (\textit{1 Co} 11,26), el pueblo de Dios peregrinante “camina por la senda estrecha de la cruz” (AG 1) hacia el banquete celestial, donde todos los elegidos se sentarán a la mesa del Reino.\end{ccebody}
			
			\begin{ccetheme}La Eucaristía como acción de gracias \end{ccetheme}
			
			\begin{ccereference}\end{ccereference}CEC 1359-1361: </p>
			
			\begin{ccebody}\textbf{La acción de gracias y la alabanza al Padre}\end{ccebody}
			
			\begin{ccebody}\begin{ccenumber}1359\end{ccenumber} La Eucaristía, sacramento de nuestra salvación realizada por Cristo en la cruz, es también un sacrificio de alabanza en acción de gracias por la obra de la creación. En el Sacrificio Eucarístico, toda la creación amada por Dios es presentada al Padre a través de la muerte y resurrección de Cristo. Por Cristo, la Iglesia puede ofrecer el sacrificio de alabanza en acción de gracias por todo lo que Dios ha hecho de bueno, de bello y de justo en la creación y en la humanidad.\end{ccebody}
			
			\begin{ccebody}\begin{ccenumber}1360\end{ccenumber} La Eucaristía es un sacrificio de acción de gracias al Padre, una bendición por la cual la Iglesia expresa su reconocimiento a Dios por todos sus beneficios, por todo lo que ha realizado mediante la creación, la redención y la santificación. “Eucaristía” significa, ante todo, acción de gracias.\end{ccebody}
			
			\begin{ccebody}\begin{ccenumber}1361\end{ccenumber} La Eucaristía es también el sacrificio de alabanza por medio del cual la Iglesia canta la gloria de Dios en nombre de toda la creación. Este sacrificio de alabanza sólo es posible a través de Cristo: Él une los fieles a su persona, a su alabanza y a su intercesión, de manera que el sacrificio de alabanza al Padre es ofrecido \textit{por} Cristo y \textit{con} Cristo para ser aceptado \textit{en }él.\end{ccebody}
			
			\begin{ccetheme}La Eucaristía como sacrificio \end{ccetheme}
			
			\begin{ccereference}\end{ccereference}CEC 610, 1362-1372, 1382, 1436: </p>
			
			\begin{ccebody}\textbf{Jesús anticipó en la cena la ofrenda libre de su vida}\end{ccebody}
			
			\begin{ccebody}\begin{ccenumber}610\end{ccenumber} Jesús expresó de forma suprema la ofrenda libre de sí mismo en la cena tomada con los doce Apóstoles (cf. \textit{Mt} 26, 20), en “la noche en que fue entregado” (\textit{1 Co} 11, 23). En la víspera de su Pasión, estando todavía libre, Jesús hizo de esta última Cena con sus Apóstoles el memorial de su ofrenda voluntaria al Padre (cf. \textit{1 Co} 5, 7), por la salvación de los hombres: “Este es mi Cuerpo que va a \textit{ser entregado} por vosotros” (\textit{Lc} 22, 19). “Esta es mi sangre de la Alianza que va a \textit{ser derramada} por muchos para remisión de los pecados” (\textit{Mt} 26, 28).\end{ccebody}
			
			\begin{ccebody}\textbf{El memorial sacrificial de Cristo y de su Cuerpo, que es la Iglesia}\end{ccebody}
			
			\begin{ccebody}\begin{ccenumber}1362\end{ccenumber} La Eucaristía es el memorial de la Pascua de Cristo, la actualización y la ofrenda sacramental de su único sacrificio, en la liturgia de la Iglesia que es su Cuerpo. En todas las plegarias eucarísticas encontramos, tras las palabras de la institución, una oración llamada \textit{anámnesis} o memorial.\end{ccebody}
			
			\begin{ccebody}\begin{ccenumber}1363\end{ccenumber} En el sentido empleado por la Sagrada Escritura, el \textit{memorial} no es solamente el recuerdo de los acontecimientos del pasado, sino la proclamación de las maravillas que Dios ha realizado en favor de los hombres (cf. \textit{Ex} 13,3). En la celebración litúrgica, estos acontecimientos se hacen, en cierta forma, presentes y actuales. De esta manera Israel entiende su liberación de Egipto: cada vez que es celebrada la pascua, los acontecimientos del Éxodo se hacen presentes a la memoria de los creyentes a fin de que conformen su vida a estos acontecimientos.\end{ccebody}
			
			\begin{ccebody}\begin{ccenumber}1364\end{ccenumber} El memorial recibe un sentido nuevo en el Nuevo Testamento. Cuando la Iglesia celebra la Eucaristía, hace memoria de la Pascua de Cristo y ésta se hace presente: el sacrificio que Cristo ofreció de una vez para siempre en la cruz, permanece siempre actual (cf. \textit{Hb} 7,25-27): “Cuantas veces se renueva en el altar el sacrificio de la cruz, en el que ‘Cristo, nuestra Pascua, fue inmolado’ (\textit{1Co} 5, 7), se realiza la obra de nuestra redención” (LG 3).\end{ccebody}
			
			\begin{ccebody}\begin{ccenumber}1365\end{ccenumber} Por ser memorial de la Pascua de Cristo, \textit{la Eucaristía es también un sacrificio}. El carácter sacrificial de la Eucaristía se manifiesta en las palabras mismas de la institución: “Esto es mi Cuerpo que será entregado por vosotros” y “Esta copa es la nueva Alianza en mi sangre, que será derramada por vosotros” (\textit{Lc} 22,19-20). En la Eucaristía, Cristo da el mismo cuerpo que por nosotros entregó en la cruz, y la sangre misma que “derramó por muchos [...] para remisión de los pecados” (\textit{Mt} 26,28).\end{ccebody}
			
			\begin{ccebody}\begin{ccenumber}1366\end{ccenumber} La Eucaristía es, pues, un sacrificio porque \textit{representa} (= hace presente) el sacrificio de la cruz, porque es su \textit{memorial} y \textit{aplica} su fruto:\end{ccebody}
			
			\begin{ccecite}“(Cristo), nuestro Dios y Señor [...] se ofreció a Dios Padre [...] una vez por todas, muriendo como intercesor sobre el altar de la cruz, a fin de realizar para ellos (los hombres) la redención eterna. Sin embargo, como su muerte no debía poner fin a su sacerdocio (\textit{Hb} 7,24.27), en la última Cena, ‘la noche en que fue entregado’ (\textit{1 Co}11,23), quiso dejar a la Iglesia, su esposa amada, un sacrificio visible (como lo reclama la naturaleza humana) [...] donde se representara el sacrificio sangriento que iba a realizarse una única vez en la cruz, cuya memoria se perpetuara hasta el fin de los siglos (\textit{1 Co} 11,23) y cuya virtud saludable se aplicara a la remisión de los pecados que cometemos cada día” (Concilio de Trento: DS 1740).\end{ccecite}
			
			\begin{ccebody}\begin{ccenumber}1367\end{ccenumber} El sacrificio de Cristo y el sacrificio de la Eucaristía son, pues, \textit{un} \textit{único sacrificio}: “La víctima es una y la misma. El mismo el que se ofrece ahora por el ministerio de los sacerdotes, el que se ofreció a sí mismo en la cruz, y solo es diferente el modo de ofrecer” (Concilio de Trento: DS 1743). “Y puesto que en este divino sacrificio que se realiza en la misa, se contiene e inmola incruentamente el mismo Cristo que en el altar de la cruz “se ofreció a sí mismo una vez de modo cruento”; […] este sacrificio [es] verdaderamente propiciatorio” (\textit{Ibíd}).\end{ccebody}
			
			\begin{ccebody}\begin{ccenumber}1368\end{ccenumber} \textit{La Eucaristía es igualmente el sacrificio de la Iglesia}. La Iglesia, que es el Cuerpo de Cristo, participa en la ofrenda de su Cabeza. Con Él, ella se ofrece totalmente. Se une a su intercesión ante el Padre por todos los hombres. En la Eucaristía, el sacrificio de Cristo se hace también el sacrificio de los miembros de su Cuerpo. La vida de los fieles, su alabanza, su sufrimiento, su oración y su trabajo se unen a los de Cristo y a su total ofrenda, y adquieren así un valor nuevo. El sacrificio de Cristo presente sobre el altar da a todas las generaciones de cristianos la posibilidad de unirse a su ofrenda.\end{ccebody}
			
			\begin{ccebody}En las catacumbas, la Iglesia es con frecuencia representada como una mujer en oración, los brazos extendidos en actitud de orante. Como Cristo que extendió los brazos sobre la cruz, por él, con él y en él, la Iglesia se ofrece e intercede por todos los hombres.\end{ccebody}
			
			\begin{ccebody}\begin{ccenumber}1369\end{ccenumber} \textit{Toda la Iglesia se une a la ofrenda y a la intercesión de Cristo}. Encargado del ministerio de Pedro en la Iglesia, \textit{el Papa} es asociado a toda celebración de la Eucaristía en la que es nombrado como signo y servidor de la unidad de la Iglesia universal. \textit{El obispo} del lugar es siempre responsable de la Eucaristía, incluso cuando es presidida por un \textit{presbítero}; el nombre del obispo se pronuncia en ella para significar su presidencia de la Iglesia particular en medio del presbiterio y con la asistencia de los \textit{diáconos}. La comunidad intercede también por todos los ministros que, por ella y con ella, ofrecen el Sacrificio Eucarístico:\end{ccebody}
			
			\begin{ccecite}“Que sólo sea considerada como legítima la Eucaristía que se hace bajo la presidencia del obispo o de quien él ha señalado para ello” (San Ignacio de Antioquía, \textit{Epistula ad Smyrnaeos} 8,1).\end{ccecite}
			
			\begin{ccecite}“Por medio del ministerio de los presbíteros, se realiza a la perfección el sacrificio espiritual de los fieles en unión con el sacrificio de Cristo, único Mediador. Este, en nombre de toda la Iglesia, por manos de los presbíteros, se ofrece incruenta y sacramentalmente en la Eucaristía, hasta que el Señor venga” (PO 2).\end{ccecite}
			
			\begin{ccebody}\begin{ccenumber}1370\end{ccenumber} A la ofrenda de Cristo se unen no sólo los miembros que están todavía aquí abajo, sino también los que están ya \textit{en la gloria del cielo}: La Iglesia ofrece el Sacrificio Eucarístico en comunión con la santísima Virgen María y haciendo memoria de ella, así como de todos los santos y santas. En la Eucaristía, la Iglesia, con María, está como al pie de la cruz, unida a la ofrenda y a la intercesión de Cristo.\end{ccebody}
			
			\begin{ccebody}\begin{ccenumber}1371\end{ccenumber} El Sacrificio Eucarístico es también ofrecido \textit{por los fieles difuntos} “que han muerto en Cristo y todavía no están plenamente purificados” (Concilio de Trento: DS 1743), para que puedan entrar en la luz y la paz de Cristo:\end{ccebody}
			
			\begin{ccecite}“Enterrad […] este cuerpo en cualquier parte; no os preocupe más su cuidado; solamente os ruego que, dondequiera que os hallareis, os acordéis de mí ante el altar del Señor” (San Agustín, \textit{Confessiones}, 9, 11, 27; palabras de santa Mónica, antes de su muerte, dirigidas a san Agustín y a su hermano).\end{ccecite}
			
			\begin{ccecite}“A continuación oramos (en la anáfora) por los santos padres y obispos difuntos, y en general por todos los que han muerto antes que nosotros, creyendo que será de gran provecho para las almas, en favor de las cuales es ofrecida la súplica, mientras se halla presente la santa y adorable víctima […] Presentando a Dios nuestras súplicas por los que han muerto, aunque fuesen pecadores […], presentamos a Cristo inmolado por nuestros pecados, haciendo propicio para ellos y para nosotros al Dios amigo de los hombres” (San Cirilo de Jerusalén, \textit{Catecheses mistagogicae} 5, 9.10).\end{ccecite}
			
			\begin{ccebody}\begin{ccenumber}1372\end{ccenumber} San Agustín ha resumido admirablemente esta doctrina que nos impulsa a una participación cada vez más completa en el sacrificio de nuestro Redentor que celebramos en la Eucaristía:\end{ccebody}
			
			\begin{ccecite}“Esta ciudad plenamente rescatada, es decir, la asamblea y la sociedad de los santos, es ofrecida a Dios como un sacrificio universal […] por el Sumo Sacerdote que, bajo la forma de esclavo, llegó a ofrecerse por nosotros en su pasión, para hacer de nosotros el cuerpo de una tan gran Cabeza […] Tal es el sacrificio de los cristianos: ‘siendo muchos, no formamos más que un sólo cuerpo en Cristo’ (\textit{Rm} 12,5). Y este sacrificio, la Iglesia no cesa de reproducirlo en el Sacramento del altar bien conocido de los fieles, donde se muestra que en lo que ella ofrece se ofrece a sí misma” (San Agustín, \textit{De civitate Dei} 10, 6).\end{ccecite}
			
			\begin{ccebody}\textbf{El banquete pascual}\end{ccebody}
			
			\begin{ccebody}\begin{ccenumber}1382\end{ccenumber} La misa es, a la vez e inseparablemente, el memorial sacrificial en que se perpetúa el sacrificio de la cruz, y el banquete sagrado de la comunión en el Cuerpo y la Sangre del Señor. Pero la celebración del sacrificio eucarístico está totalmente orientada hacia la unión íntima de los fieles con Cristo por medio de la comunión. Comulgar es recibir a Cristo mismo que se ofrece por nosotros.\end{ccebody}
			
			\begin{ccebody}\begin{ccenumber}1436\end{ccenumber} \textit{Eucaristía y Penitencia}. La conversión y la penitencia diarias encuentran su fuente y su alimento en la Eucaristía, pues en ella se hace presente el sacrificio de Cristo que nos reconcilió con Dios; por ella son alimentados y fortificados los que viven de la vida de Cristo; “es el antídoto que nos libera de nuestras faltas cotidianas y nos preserva de pecados mortales” (Concilio de Trento: DS 1638).\end{ccebody}
			
			\begin{ccetheme}La presencia real de Cristo en la Eucaristía \end{ccetheme}
			
			\begin{ccereference}\end{ccereference}CEC 1373-1381: </p>
			
			\begin{ccebody}\textbf{La presencia de Cristo por el poder de su Palabra y del Espíritu Santo}\end{ccebody}
			
			\begin{ccebody}\begin{ccenumber}1373\end{ccenumber} “Cristo Jesús que murió, resucitó, que está a la derecha de Dios e intercede por nosotros” (\textit{Rm} 8,34), está presente de múltiples maneras en su Iglesia (cf. LG 48): en su Palabra, en la oración de su Iglesia, “allí donde dos o tres estén reunidos en mi nombre” (\textit{Mt} 18,20), en los pobres, los enfermos, los presos (\textit{Mt} 25,31-46), en los sacramentos de los que Él es autor, en el sacrificio de la misa y en la persona del ministro. Pero, \textit{“sobre todo,} (está presente) \textit{bajo las especies eucarísticas”} (SC 7).\end{ccebody}
			
			\begin{ccebody}\begin{ccenumber}1374\end{ccenumber} El modo de presencia de Cristo bajo las especies eucarísticas es singular. Eleva la Eucaristía por encima de todos los sacramentos y hace de ella “como la perfección de la vida espiritual y el fin al que tienden todos los sacramentos” (Santo Tomás de Aquino, \textit{Summa theologiae} 3, q. 73, a. 3). En el Santísimo Sacramento de la Eucaristía están “contenidos \textit{verdadera, real y substancialmente} el Cuerpo y la Sangre junto con el alma y la divinidad de nuestro Señor Jesucristo, y, por consiguiente, \textit{Cristo entero”} (Concilio de Trento: DS 1651). “Esta presencia se denomina ‘real’, no a título exclusivo, como si las otras presencias no fuesen ‘reales’, sino por excelencia, porque es \textit{substancial}, y por ella Cristo, Dios y hombre, se hace totalmente presente” (MF 39).\end{ccebody}
			
			\begin{ccebody}\begin{ccenumber}1375\end{ccenumber} Mediante la \textit{conversión} del pan y del vino en su Cuerpo y Sangre, Cristo se hace presente en este sacramento. Los Padres de la Iglesia afirmaron con fuerza la fe de la Iglesia en la eficacia de la Palabra de Cristo y de la acción del Espíritu Santo para obrar esta conversión. Así, san Juan Crisóstomo declara que:\end{ccebody}
			
			\begin{ccecite}“No es el hombre quien hace que las cosas ofrecidas se conviertan en Cuerpo y Sangre de Cristo, sino Cristo mismo que fue crucificado por nosotros. El sacerdote, figura de Cristo, pronuncia estas palabras, pero su eficacia y su gracia provienen de Dios. \textit{Esto es mi Cuerpo}, dice. Esta palabra transforma las cosas ofrecidas” (\textit{De proditione Iudae homilia} 1, 6).\end{ccecite}
			
			\begin{ccebody}Y san Ambrosio dice respecto a esta conversión:\end{ccebody}
			
			\begin{ccecite}“Estemos bien persuadidos de que esto no es lo que la naturaleza ha producido, sino lo que la bendición ha consagrado, y de que la fuerza de la bendición supera a la de la naturaleza, porque por la bendición la naturaleza misma resulta cambiada” (\textit{De mysteriis} 9, 50). “La palabra de Cristo, que pudo hacer de la nada lo que no existía, ¿no podría cambiar las cosas existentes en lo que no eran todavía? Porque no es menos dar a las cosas su naturaleza primera que cambiársela” (\textit{Ibíd.}, 9,50.52).\end{ccecite}
			
			\begin{ccebody}\begin{ccenumber}1376\end{ccenumber} El Concilio de Trento resume la fe católica cuando afirma: “Porque Cristo, nuestro Redentor, dijo que lo que ofrecía bajo la especie de pan era verdaderamente su Cuerpo, se ha mantenido siempre en la Iglesia esta convicción, que declara de nuevo el Santo Concilio: por la consagración del pan y del vino se opera la conversión de toda la substancia del pan en la substancia del Cuerpo de Cristo nuestro Señor y de toda la substancia del vino en la substancia de su Sangre; la Iglesia católica ha llamado justa y apropiadamente a este cambio \textit{transubstanciación”} (DS 1642).\end{ccebody}
			
			\begin{ccebody}\begin{ccenumber}1377\end{ccenumber} La presencia eucarística de Cristo comienza en el momento de la consagración y dura todo el tiempo que subsistan las especies eucarísticas. Cristo está todo entero presente en cada una de las especies y todo entero en cada una de sus partes, de modo que la fracción del pan no divide a Cristo (cf. Concilio de Trento: DS 1641).\end{ccebody}
			
			\begin{ccebody}\begin{ccenumber}1378\end{ccenumber} \textit{El culto de la Eucaristía}. En la liturgia de la misa expresamos nuestra fe en la presencia real de Cristo bajo las especies de pan y de vino, entre otras maneras, arrodillándonos o inclinándonos profundamente en señal de adoración al Señor. “La Iglesia católica ha dado y continua dando este culto de adoración que se debe al sacramento de la Eucaristía no solamente durante la misa, sino también fuera de su celebración: conservando con el mayor cuidado las hostias consagradas, presentándolas a los fieles para que las veneren con solemnidad, llevándolas en procesión en medio de la alegría del pueblo” (MF 56).\end{ccebody}
			
			\begin{ccebody}\begin{ccenumber}1379\end{ccenumber} El sagrario (tabernáculo) estaba primeramente destinado a guardar dignamente la Eucaristía para que pudiera ser llevada a los enfermos y ausentes fuera de la misa. Por la profundización de la fe en la presencia real de Cristo en su Eucaristía, la Iglesia tomó conciencia del sentido de la adoración silenciosa del Señor presente bajo las especies eucarísticas. Por eso, el sagrario debe estar colocado en un lugar particularmente digno de la iglesia; debe estar construido de tal forma que subraye y manifieste la verdad de la presencia real de Cristo en el santísimo sacramento.\end{ccebody}
			
			\begin{ccebody}\begin{ccenumber}1380\end{ccenumber} Es grandemente admirable que Cristo haya querido hacerse presente en su Iglesia de esta singular manera. Puesto que Cristo iba a dejar a los suyos bajo su forma visible, quiso darnos su presencia sacramental; puesto que iba a ofrecerse en la cruz por muestra salvación, quiso que tuviéramos el memorial del amor con que nos había amado “hasta el fin” (\textit{Jn} 13,1), hasta el don de su vida. En efecto, en su presencia eucarística permanece misteriosamente en medio de nosotros como quien nos amó y se entregó por nosotros (cf. \textit{Ga} 2,20), y se queda bajo los signos que expresan y comunican este amor:\end{ccebody}
			
			\begin{ccecite}“La Iglesia y el mundo tienen una gran necesidad del culto eucarístico. Jesús nos espera en este sacramento del amor. No escatimemos tiempo para ir a encontrarlo en la adoración, en la contemplación llena de fe y abierta a reparar las faltas graves y delitos del mundo. No cese nunca nuestra adoración” (Juan Pablo II, Carta \textit{Dominicae Cenae}, 3).\end{ccecite}
			
			\begin{ccebody}\begin{ccenumber}1381\end{ccenumber} “La presencia del verdadero Cuerpo de Cristo y de la verdadera Sangre de Cristo en este sacramento, ‘no se conoce por los sentidos, dice santo Tomás, sino sólo \textit{por la fe} , la cual se apoya en la autoridad de Dios’. Por ello, comentando el texto de san Lucas 22, 19: \textit{‘Esto es mi Cuerpo que será entregado por vosotros’}, san Cirilo declara: ‘No te preguntes si esto es verdad, sino acoge más bien con fe las palabras del Salvador, porque Él, que es la Verdad, no miente’” (MF 18; cf. Santo Tomás de Aquino, Summa theologiae 3, q. 75, a. 1; San Cirilo de Alejandría, \textit{Commentarius in Lucam} 22, 19):\end{ccebody}
			
			\begin{ccecite}Adoro Te devote, latens Deitas,<br />Quae sub his figuris vere latitas:<br />Tibi se cor meum totum subjicit,<br />Quia Te contemplans totum deficit.<br /><br />Visus, gustus, tactus in te fallitur,<br />Sed auditu solo tuto creditur:<br />Credo quidquid dixit Dei Filius:<br />Nil hoc Veritatis verbo verius.<br /><br />Adórote devotamente, oculta Deidad,<br />que bajo estas sagradas especies te ocultas verdaderamente:<br />A ti mi corazón totalmente se somete,<br />pues al contemplarte, se siente desfallecer por completo.<br /><br />La vista, el tacto, el gusto, son aquí falaces;<br />sólo con el oído se llega a tener fe segura.<br />Creo todo lo que ha dicho el Hijo de Dios,<br />nada más verdadero que esta palabra de Verdad. \end{ccecite}
			
			\begin{ccecite}[AHMA 50, 589]\end{ccecite}
			
			\begin{ccetheme}La Comunión \end{ccetheme}
			
			\begin{ccereference}\end{ccereference}CEC 1384-1401, 2837: </p>
			
			\begin{ccebody}\begin{ccenumber}1384\end{ccenumber} El Señor nos dirige una invitación urgente a recibirle en el sacramento de la Eucaristía: “En verdad, en verdad os digo: si no coméis la carne del Hijo del hombre, y no bebéis su sangre, no tendréis vida en vosotros” (\textit{Jn} 6,53).\end{ccebody}
			
			\begin{ccebody}\begin{ccenumber}1385\end{ccenumber} Para responder a esta invitación, debemos \textit{prepararnos} para este momento tan grande y santo. San Pablo exhorta a un examen de conciencia: “Quien coma el pan o beba el cáliz del Señor indignamente, será reo del Cuerpo y de la Sangre del Señor. Examínese, pues, cada cual, y coma entonces del pan y beba del cáliz. Pues quien come y bebe sin discernir el Cuerpo, come y bebe su propio castigo” (\textit{1 Co} 11,27-29). Quien tiene conciencia de estar en pecado grave debe recibir el sacramento de la Reconciliación antes de acercarse a comulgar.\end{ccebody}
			
			\begin{ccebody}\begin{ccenumber}1386\end{ccenumber} Ante la grandeza de este sacramento, el fiel sólo puede repetir humildemente y con fe ardiente las palabras del Centurión (cf. \textit{Mt} 8,8): \textit{“Señor, no soy digno de que entres en mi casa, pero una palabra tuya bastará para sanarme”}. En la Liturgia de san Juan Crisóstomo, los fieles oran con el mismo espíritu:\end{ccebody}
			
			\begin{ccecite}“A tomar parte en tu cena sacramental invítame hoy, Hijo de Dios: no revelaré a tus enemigos el misterio, no te te daré el beso de Judas; antes como el ladrón te reconozco y te suplico: ¡Acuérdate de mí, Señor, en tu reino!” (Liturgia Bizantina. \textit{Anaphora Iohannis Chrysostomi}, Oración antes de la Comunión).\end{ccecite}
			
			\begin{ccebody}\begin{ccenumber}1387\end{ccenumber} Para prepararse convenientemente a recibir este sacramento, los fieles deben observar el ayuno prescrito por la Iglesia (cf. CIC can. 919). Por la actitud corporal (gestos, vestido) se manifiesta el respeto, la solemnidad, el gozo de ese momento en que Cristo se hace nuestro huésped.\end{ccebody}
			
			\begin{ccebody}\begin{ccenumber}1388\end{ccenumber} Es conforme al sentido mismo de la Eucaristía que los fieles, con las debidas disposiciones (cf. CIC, cans. 916-917), comulguen cuando participan en la misa [Los fieles pueden recibir la Sagrada Eucaristía solamente dos veces el mismo día. Pontificia Comisión para la auténtica interpretación del Código de Derecho Canónico, \textit{Responsa ad proposita dubia} 1]. “Se recomienda especialmente la participación más perfecta en la misa, recibiendo los fieles, después de la comunión del sacerdote, del mismo sacrificio, el cuerpo del Señor” (SC 55).\end{ccebody}
			
			\begin{ccebody}\begin{ccenumber}1389\end{ccenumber} La Iglesia obliga a los fieles “a participar los domingos y días de fiesta en la divina liturgia” (cf. OE 15) y a recibir al menos una vez al año la Eucaristía, si es posible en tiempo pascual (cf. CIC can. 920), preparados por el sacramento de la Reconciliación. Pero la Iglesia recomienda vivamente a los fieles recibir la santa Eucaristía los domingos y los días de fiesta, o con más frecuencia aún, incluso todos los días.\end{ccebody}
			
			\begin{ccebody}\begin{ccenumber}1390\end{ccenumber} Gracias a la presencia sacramental de Cristo bajo cada una de las especies, la comunión bajo la sola especie de pan ya hace que se reciba todo el fruto de gracia propio de la Eucaristía. Por razones pastorales, esta manera de comulgar se ha establecido legítimamente como la más habitual en el rito latino. “La comunión tiene una expresión más plena por razón del signo cuando se hace bajo las dos especies. Ya que en esa forma es donde más perfectamente se manifiesta el signo del banquete eucarístico” (\textit{Institución general del Misal Romano}, 240). Es la forma habitual de comulgar en los ritos orientales.\end{ccebody}
			
			\begin{ccebody}\begin{ccenumber}1391\end{ccenumber} \textit{La comunión acrecienta nuestra unión con Cristo}. Recibir la Eucaristía en la comunión da como fruto principal la unión íntima con Cristo Jesús. En efecto, el Señor dice: “Quien come mi Carne y bebe mi Sangre habita en mí y yo en él” (\textit{Jn} 6,56). La vida en Cristo encuentra su fundamento en el banquete eucarístico: “Lo mismo que me ha enviado el Padre, que vive, y yo vivo por el Padre, también el que me coma vivirá por mí” (\textit{Jn} 6,57):\end{ccebody}
			
			\begin{ccecite}“Cuando en las fiestas [del Señor] los fieles reciben el Cuerpo del Hijo, proclaman unos a otros la Buena Nueva, se nos han dado las arras de la vida, como cuando el ángel dijo a María [de Magdala]: ‘¡Cristo ha resucitado!’ He aquí que ahora también la vida y la resurrección son comunicadas a quien recibe a Cristo” (\textit{Fanqîth, Breviarium iuxta ritum Ecclesiae Antiochenae Syrorum}, v. 1).\end{ccecite}
			
			\begin{ccebody}\begin{ccenumber}1392\end{ccenumber} Lo que el alimento material produce en nuestra vida corporal, la comunión lo realiza de manera admirable en nuestra vida espiritual. La comunión con la Carne de Cristo resucitado, “vivificada por el Espíritu Santo y vivificante” (PO 5), conserva, acrecienta y renueva la vida de gracia recibida en el Bautismo. Este crecimiento de la vida cristiana necesita ser alimentado por la comunión eucarística, pan de nuestra peregrinación, hasta el momento de la muerte, cuando nos sea dada como viático.\end{ccebody}
			
			\begin{ccebody}\begin{ccenumber}1393\end{ccenumber} \textit{La comunión nos separa del pecado}. El Cuerpo de Cristo que recibimos en la comunión es “entregado por nosotros”, y la Sangre que bebemos es “derramada por muchos para el perdón de los pecados”. Por eso la Eucaristía no puede unirnos a Cristo sin purificarnos al mismo tiempo de los pecados cometidos y preservarnos de futuros pecados:\end{ccebody}
			
			\begin{ccecite}“Cada vez que lo recibimos, anunciamos la muerte del Señor (cf. \textit{1 Co }11,26). Si anunciamos la muerte del Señor, anunciamos también el perdón de los pecados. Si cada vez que su Sangre es derramada, lo es para el perdón de los pecados, debo recibirle siempre, para que siempre me perdone los pecados. Yo que peco siempre, debo tener siempre un remedio” (San Ambrosio, \textit{De sacramentis} 4, 28).\end{ccecite}
			
			\begin{ccebody}\begin{ccenumber}1394\end{ccenumber} Como el alimento corporal sirve para restaurar la pérdida de fuerzas, la Eucaristía fortalece la caridad que, en la vida cotidiana, tiende a debilitarse; y esta caridad vivificada \textit{borra los pecados veniales} (cf. Concilio de Trento: DS 1638). Dándose a nosotros, Cristo reaviva nuestro amor y nos hace capaces de romper los lazos desordenados con las criaturas y de arraigarnos en Él:\end{ccebody}
			
			\begin{ccecite}“Porque Cristo murió por nuestro amor, cuando hacemos conmemoración de su muerte en nuestro sacrificio, pedimos que venga el Espíritu Santo y nos comunique el amor; suplicamos fervorosamente que aquel mismo amor que impulsó a Cristo a dejarse crucificar por nosotros sea infundido por el Espíritu Santo en nuestro propios corazones, con objeto de que consideremos al mundo como crucificado para nosotros, y sepamos vivir crucificados para el mundo [...] y, llenos de caridad, muertos para el pecado vivamos para Dios” (San Fulgencio de Ruspe, \textit{Contra gesta Fabiani} 28, 17-19).\end{ccecite}
			
			\begin{ccebody}\begin{ccenumber}1395\end{ccenumber} Por la misma caridad que enciende en nosotros, la Eucaristía nos \textit{preserva de futuros pecados mortales}. Cuanto más participamos en la vida de Cristo y más progresamos en su amistad, tanto más difícil se nos hará romper con Él por el pecado mortal. La Eucaristía no está ordenada al perdón de los pecados mortales. Esto es propio del sacramento de la Reconciliación. Lo propio de la Eucaristía es ser el sacramento de los que están en plena comunión con la Iglesia.\end{ccebody}
			
			\begin{ccebody}\begin{ccenumber}1396\end{ccenumber} \textit{La unidad del Cuerpo místico: La Eucaristía hace la Iglesia}. Los que reciben la Eucaristía se unen más estrechamente a Cristo. Por ello mismo, Cristo los une a todos los fieles en un solo cuerpo: la Iglesia. La comunión renueva, fortifica, profundiza esta incorporación a la Iglesia realizada ya por el Bautismo. En el Bautismo fuimos llamados a no formar más que un solo cuerpo (cf. \textit{1 Co} 12,13). La Eucaristía realiza esta llamada: “El cáliz de bendición que bendecimos ¿no es acaso comunión con la sangre de Cristo? y el pan que partimos ¿no es comunión con el Cuerpo de Cristo? Porque aun siendo muchos, un solo pan y un solo cuerpo somos, pues todos participamos de un solo pan” (\textit{1 Co} 10,16-17):\end{ccebody}
			
			\begin{ccecite}“Si vosotros mismos sois Cuerpo y miembros de Cristo, sois el sacramento que es puesto sobre la mesa del Señor, y recibís este sacramento vuestro. Respondéis ‘Amén’ [es decir, ‘sí’, ‘es verdad’] a lo que recibís, con lo que, respondiendo, lo reafirmáis. Oyes decir ‘el Cuerpo de Cristo’, y respondes ‘amén’. Por lo tanto, sé tú verdadero miembro de Cristo para que tu ‘amén’ sea también verdadero” (San Agustín, \textit{Sermo} 272).\end{ccecite}
			
			\begin{ccebody}\begin{ccenumber}1397\end{ccenumber} \textit{La Eucaristía entraña un compromiso en favor de los pobres:} Para recibir en la verdad el Cuerpo y la Sangre de Cristo entregados por nosotros debemos reconocer a Cristo en los más pobres, sus hermanos (cf. \textit{Mt} 25,40):\end{ccebody}
			
			\begin{ccecite}“Has gustado la sangre del Señor y no reconoces a tu hermano. [...] Deshonras esta mesa, no juzgando digno de compartir tu alimento al que ha sido juzgado digno [...] de participar en esta mesa. Dios te ha liberado de todos los pecados y te ha invitado a ella. Y tú, aún así, no te has hecho más misericordioso” (San Juan Crisóstomo, hom. in 1 Co 27,4).\end{ccecite}
			
			\begin{ccebody}\begin{ccenumber}1398\end{ccenumber} \textit{La Eucaristía y la unidad de los cristianos}. Ante la grandeza de esta misterio, san Agustín exclama: \textit{O sacramentum pietatis! O signum unitatis! O vinculum caritatis!} – “¡Oh sacramento de piedad, oh signo de unidad, oh vínculo de caridad!” (\textit{In Iohannis evangelium tractatus} 26,13; cf. SC 47). Cuanto más dolorosamente se hacen sentir las divisiones de la Iglesia que rompen la participación común en la mesa del Señor, tanto más apremiantes son las oraciones al Señor para que lleguen los días de la unidad completa de todos los que creen en Él.\end{ccebody}
			
			\begin{ccebody}\begin{ccenumber}1399\end{ccenumber} Las Iglesias orientales que no están en plena comunión con la Iglesia católica celebran la Eucaristía con gran amor. “Estas Iglesias, aunque separadas, [tienen] verdaderos sacramentos [...] y sobre todo, en virtud de la sucesión apostólica, el sacerdocio y la Eucaristía, con los que se unen aún más con nosotros con vínculo estrechísimo” (UR 15). Una cierta comunión \textit{in sacris}, por tanto, en la Eucaristía, “no solamente es posible, sino que se aconseja... en circunstancias oportunas y aprobándolo la autoridad eclesiástica” (UR 15, cf. CIC can. 844, § 3).\end{ccebody}
			
			\begin{ccebody}\begin{ccenumber}1400\end{ccenumber} Las comunidades eclesiales nacidas de la Reforma, separadas de la Iglesia católica, “sobre todo por defecto del sacramento del orden, no han conservado la sustancia genuina e íntegra del misterio eucarístico” (UR 22). Por esto, para la Iglesia católica, la intercomunión eucarística con estas comunidades no es posible. Sin embargo, estas comunidades eclesiales “al conmemorar en la Santa Cena la muerte y la resurrección del Señor, profesan que en la comunión de Cristo se significa la vida, y esperan su venida gloriosa” (UR 22).\end{ccebody}
			
			\begin{ccebody}\begin{ccenumber}1401\end{ccenumber} Si, a juicio del Ordinario, se presenta una necesidad grave, los ministros católicos pueden administrar los sacramentos (Eucaristía, Penitencia, Unción de los enfermos) a cristianos que no están en plena comunión con la Iglesia católica, pero que piden estos sacramentos con deseo y rectitud: en tal caso se precisa que profesen la fe católica respecto a estos sacramentos y estén bien dispuestos (cf. CIC, can. 844, § 4).\end{ccebody}
			
			\begin{ccebody}\begin{ccenumber}2837\end{ccenumber} [Nuestro Pan] \textit{“De cada día”}. La palabra griega, \textit{epiousion}, no tiene otro sentido en el Nuevo Testamento. Tomada en un sentido temporal, es una repetición pedagógica de “hoy” (cf. \textit{Ex} 16, 19-21) para confirmarnos en una confianza “sin reserva”. Tomada en un sentido cualitativo, significa lo necesario a la vida, y más ampliamente cualquier bien suficiente para la subsistencia (cf. \textit{1 Tm} 6, 8). Tomada al pie de la letra (\textit{epiousion}: “lo más esencial”), designa directamente el Pan de Vida, el Cuerpo de Cristo, “remedio de inmortalidad” (San Ignacio de Antioquía, \textit{Epistula ad Ephesios,} 20, 2) sin el cual no tenemos la Vida en nosotros (cf. \textit{Jn} 6, 53-56) Finalmente, ligado a lo que precede, el sentido celestial es claro: este “día” es el del Señor, el del Festín del Reino, anticipado en la Eucaristía, en que pregustamos el Reino venidero. Por eso conviene que la liturgia eucarística se celebre “cada día”.\end{ccebody}
			
			\begin{ccecite}“La Eucaristía es nuestro pan cotidiano [...] La virtud propia de este divino alimento es una fuerza de unión: nos une al Cuerpo del Salvador y hace de nosotros sus miembros para que vengamos a ser lo que recibimos [...] Este pan cotidiano se encuentra, además, en las lecturas que oís cada día en la Iglesia, en los himnos que se cantan y que vosotros cantáis. Todo eso es necesario en nuestra peregrinación” (San Agustín, \textit{Sermo} 57, 7, 7).\end{ccecite}
			
			\begin{ccebody}El Padre del cielo nos exhorta a pedir como hijos del cielo el Pan del cielo (cf. \textit{Jn} 6, 51). Cristo “mismo es el pan que, sembrado en la Virgen, florecido en la Carne, amasado en la Pasión, cocido en el Horno del sepulcro, reservado en la iglesia, llevado a los altares, suministra cada día a los fieles un alimento celestial” (San Pedro Crisólogo, \textit{Sermo} 67, 7).\end{ccebody}
			
			\begin{ccetheme}La Eucaristía “prenda de la gloria futura” \end{ccetheme}
			
			\begin{ccereference}\end{ccereference}CEC 1402-1405: </p>
			
			\begin{ccebody}\textbf{La Eucaristía, “Pignus futurae gloriae”}\end{ccebody}
			
			\begin{ccebody}\begin{ccenumber}1402\end{ccenumber} En una antigua oración, la Iglesia aclama el misterio de la Eucaristía: \textit{O sacrum convivium in quo Christus sumitur. Recolitur memoria passionis Eius; mens impletur gratia et futurae gloriae nobis pignus datur} – “¡Oh sagrado banquete, en que Cristo es nuestra comida; se celebra el memorial de su pasión; el alma se llena de gracia, y se nos da la prenda de la gloria futura!” (\textit{Solemnidad del Santísimo Cuerpo y Sangre de Cristo}, Antífona del “Magnificat” para las II Vísperas: \textit{Liturgia de las Horas}). Si la Eucaristía es el memorial de la Pascua del Señor y si por nuestra comunión en el altar somos colmados “de gracia y bendición” (\textit{Plegaria Eucarística I o Canon Romano} 96: \textit{Misal Romano}), la Eucaristía es también la anticipación de la gloria celestial.\end{ccebody}
			
			\begin{ccebody}\begin{ccenumber}1403\end{ccenumber} En la última Cena, el Señor mismo atrajo la atención de sus discípulos hacia el cumplimiento de la Pascua en el Reino de Dios: “Y os digo que desde ahora no beberé de este fruto de la vid hasta el día en que lo beba con vosotros, de nuevo, en el Reino de mi Padre” (\textit{Mt} 26,29; cf. \textit{Lc} 22,18; \textit{Mc} 14,25). Cada vez que la Iglesia celebra la Eucaristía recuerda esta promesa y su mirada se dirige hacia “el que viene” (\textit{Ap} 1,4). En su oración, implora su venida: \textit{Marana tha} (\textit{1 Co} 16,22), “Ven, Señor Jesús” (\textit{Ap} 22,20), “que tu gracia venga y que este mundo pase” (\textit{Didaché} 10,6).\end{ccebody}
			
			\begin{ccebody}\begin{ccenumber}1404\end{ccenumber} La Iglesia sabe que, ya ahora, el Señor viene en su Eucaristía y que está ahí en medio de nosotros. Sin embargo, esta presencia está velada. Por eso celebramos la Eucaristía \textit{expectantes beatam spem et adventum Salvatoris nostri Jesu Christi} – “Mientras esperamos la gloriosa venida de Nuestro Salvador Jesucristo” (\textit{Ritual de la Comunión}, 126 [Embolismo después del «Padrenuestro»]: \textit{Misal Romano}; cf. \textit{Tit} 2,13), pidiendo entrar “[en tu Reino], donde esperamos gozar todos juntos de la plenitud eterna de tu gloria; allí enjugarás las lágrimas de nuestros ojos, porque, al contemplarte como Tú eres, Dios nuestro, seremos para siempre semejantes a ti y cantaremos eternamente tus alabanzas, por Cristo, Señor Nuestro” (\textit{Plegaria Eucarística III}, 116: \textit{Misal Romano}).\end{ccebody}
			
			\begin{ccebody}\begin{ccenumber}1405\end{ccenumber} De esta gran esperanza, la de los cielos nuevos y la tierra nueva en los que habitará la justicia (cf. \textit{2 P} 3,13), no tenemos prenda más segura, signo más manifiesto que la Eucaristía. En efecto, cada vez que se celebra este misterio, “se realiza la obra de nuestra redención” (LG 3) y “partimos un mismo pan [...] que es remedio de inmortalidad, antídoto para no morir, sino para vivir en Jesucristo para siempre” (San Ignacio de Antioquía, \textit{Epistula ad Ephesios}, 20, 2).\end{ccebody}
			
			\begin{ccetheme}La institución del sacerdocio en la Última Cena \end{ccetheme}
			
			\begin{ccereference}\end{ccereference}CEC 611, 1366: </p>
			
			\begin{ccebody}\begin{ccenumber}611\end{ccenumber} La Eucaristía que instituyó en este momento será el “memorial” (\textit{1 Co} 11, 25) de su sacrificio. Jesús incluye a los Apóstoles en su propia ofrenda y les manda perpetuarla (cf. \textit{Lc} 22, 19). Así Jesús instituye a sus apóstoles sacerdotes de la Nueva Alianza: “Por ellos me consagro a mí mismo para que ellos sean también consagrados en la verdad” (\textit{Jn} 17, 19; cf. Concilio de Trento: DS, 1752; 1764).\end{ccebody}
			
			\begin{ccebody}\begin{ccenumber}1366\end{ccenumber} La Eucaristía es, pues, un sacrificio porque \textit{representa} (= hace presente) el sacrificio de la cruz, porque es su \textit{memorial} y \textit{aplica} su fruto:\end{ccebody}
			
			\begin{ccecite}“(Cristo), nuestro Dios y Señor [...] se ofreció a Dios Padre [...] una vez por todas, muriendo como intercesor sobre el altar de la cruz, a fin de realizar para ellos (los hombres) la redención eterna. Sin embargo, como su muerte no debía poner fin a su sacerdocio (\textit{Hb} 7,24.27), en la última Cena, ‘la noche en que fue entregado’ (\textit{1 Co}11,23), quiso dejar a la Iglesia, su esposa amada, un sacrificio visible (como lo reclama la naturaleza humana) [...] donde se representara el sacrificio sangriento que iba a realizarse una única vez en la cruz, cuya memoria se perpetuara hasta el fin de los siglos (\textit{1 Co} 11,23) y cuya virtud saludable se aplicara a la remisión de los pecados que cometemos cada día” (Concilio de Trento: DS 1740).\end{ccecite}
			
			\chapter{Viernes Santo en la Pasión del Señor}
			
			\section{Lecturas}
			
			\begin{readtitle}PRIMERA LECTURA\end{readtitle}
			
			\begin{readbook}Del libro del profeta Isaías \rightline{52, 13–53, 12}\end{readbook}
			
			\begin{readtheme}Él fue traspasado por nuestras rebeliones\end{readtheme}
			
			\begin{readtalk}Mirad, mi siervo tendrá éxito, <br />subirá y crecerá mucho. \end{readtalk}
			
			\begin{readtalk}Como muchos se espantaron de él <br />porque desfigurado no parecía hombre, <br />ni tenía aspecto humano, <br />así asombrará a muchos pueblos, <br />ante él los reyes cerrarán la boca, <br />al ver algo inenarrable <br />y comprender algo inaudito. \end{readtalk}
			
			\begin{readtalk}¿Quién creyó nuestro anuncio?; <br />¿a quién se reveló el brazo del Señor? \end{readtalk}
			
			\begin{readtalk}Creció en su presencia como brote, <br />como raíz en tierra árida, <br />sin figura, sin belleza. \end{readtalk}
			
			\begin{readtalk}Lo vimos sin aspecto atrayente, <br />despreciado y evitado de los hombres, <br />como un hombre de dolores, <br />acostumbrado a sufrimientos, <br />ante el cual se ocultaban los rostros, <br />despreciado y desestimado. \end{readtalk}
			
			\begin{readtalk}Él soportó nuestros sufrimientos <br />y aguantó nuestros dolores; <br />nosotros lo estimamos leproso, <br />herido de Dios y humillado; <br />pero él fue traspasado por nuestras rebeliones, <br />triturado por nuestros crímenes. \end{readtalk}
			
			\begin{readtalk}Nuestro castigo saludable cayó sobre él, <br />sus cicatrices nos curaron. \end{readtalk}
			
			\begin{readtalk}Todos errábamos como ovejas, <br />cada uno siguiendo su camino; <br />y el Señor cargó sobre él <br />todos nuestros crímenes. \end{readtalk}
			
			\begin{readtalk}Maltratado, voluntariamente se humillaba <br />y no abría la boca: <br />como cordero llevado al matadero, <br />como oveja ante el esquilador, <br />enmudecía y no abría la boca. \end{readtalk}
			
			\begin{readtalk}Sin defensa, sin justicia, se lo llevaron, <br />¿quién se preocupará de su estirpe? \end{readtalk}
			
			\begin{readtalk}Lo arrancaron de la tierra de los vivos, <br />por los pecados de mi pueblo lo hirieron. \end{readtalk}
			
			\begin{readtalk}Le dieron sepultura con los malvados <br />y una tumba con los malhechores, <br />aunque no había cometido crímenes <br />ni hubo engaño en su boca. \end{readtalk}
			
			\begin{readtalk}El Señor quiso triturarlo con el sufrimiento, <br />y entregar su vida como expiación: <br />verá su descendencia, <br />prolongará sus años, <br />lo que el Señor quiere prosperará por su mano. \end{readtalk}
			
			\begin{readtalk}Por los trabajos de su alma verá la luz, <br />el justo se saciará de conocimiento. \end{readtalk}
			
			\begin{readtalk}Mi siervo justificará a muchos, <br />porque cargó con los crímenes de ellos. \end{readtalk}
			
			\begin{readtalk}Le daré una multitud como parte, <br />y tendrá como despojo una muchedumbre. \end{readtalk}
			
			\begin{readtalk}Porque expuso su vida a la muerte <br />y fue contado entre los pecadores, <br />él tomó el pecado de muchos <br />e intercedió por los pecadores.\end{readtalk}
			
			\begin{readtitle}SALMO RESPONSORIAL\end{readtitle}
			
			\begin{readbook}Salmo \rightline{30, 2 y 6. 12-13. 15-16. 17 y 25}\end{readbook}
			
			\begin{readtheme}Padre, a tus manos encomiendo mi espíritu\end{readtheme}
			
			\begin{readps}\begin{readred}℣.\end{readred} A ti , Señor, me acojo: \end{readps}
			
			\begin{readtabbed}no quede yo nunca defraudado; \end{readtabbed}
			
			\begin{readtabbed}tú, que eres justo, ponme a salvo. \end{readtabbed}
			
			\begin{readtabbed}A tus manos encomiendo mi espíritu: \end{readtabbed}
			
			\begin{readtabbed}tú, el Dios leal, me librarás. \begin{readred}℟.\end{readred}\end{readtabbed}
			
			\begin{readps}\begin{readred}℣.\end{readred} Soy la burla de todos mis enemigos, \end{readps}
			
			\begin{readtabbed}la irrisión de mis vecinos, \end{readtabbed}
			
			\begin{readtabbed}el espanto de mis conocidos: \end{readtabbed}
			
			\begin{readtabbed}me ven por la calle, y escapan de mí. \end{readtabbed}
			
			\begin{readtabbed}Me han olvidado como a un muerto, \end{readtabbed}
			
			\begin{readtabbed}me han desechado como a un cacharro inútil. \begin{readred}℟.\end{readred}\end{readtabbed}
			
			\begin{readps}\begin{readred}℣.\end{readred} Pero yo confío en ti, Señor; \end{readps}
			
			\begin{readtabbed}te digo: “Tú eres mi Dios”. \end{readtabbed}
			
			\begin{readtabbed}En tu mano están mis azares: \end{readtabbed}
			
			\begin{readtabbed}líbrame de los enemigos que me persiguen. \begin{readred}℟.\end{readred}\end{readtabbed}
			
			\begin{readps}\begin{readred}℣.\end{readred} Haz brillar tu rostro sobre tu siervo, \end{readps}
			
			\begin{readtabbed}sálvame por tu misericordia. \end{readtabbed}
			
			\begin{readtabbed}Sed fuertes y valientes de corazón, \end{readtabbed}
			
			\begin{readtabbed}los que esperáis en el Señor. \begin{readred}℟.\end{readred}\end{readtabbed}
			
			\begin{readtitle}SEGUNDA LECTURA\end{readtitle}
			
			\begin{readbook}De la carta a los Hebreos \rightline{4, 14-16; 5, 7-9}\end{readbook}
			
			\begin{readtheme}Aprendió a obedecer; y se convirtió, para todos los que lo obedecen, \end{readtheme}
			
			\begin{readtheme}en autor de salvación\end{readtheme}
			
			\begin{readbody}Hermanos: \end{readbody}
			
			\begin{readbody}Ya que tenemos un sumo sacerdote grande que ha atravesado el cielo, Jesús, Hijo de Dios, mantengamos firme la confesión de fe. \end{readbody}
			
			\begin{readbody}No tenemos un sumo sacerdote incapaz de compadecerse de nuestras debilidades, sino que ha sido probado en todo, como nosotros, menos en el pecado. Por eso, comparezcamos confiados ante el trono de la gracia, para alcanzar misericordia y encontrar gracia para un auxilio oportuno. \end{readbody}
			
			\begin{readbody}Cristo, en los días de su vida mortal, a gritos y con lágrimas, presentó oraciones y súplicas al que podía salvarlo de la muerte, siendo escuchado por su piedad filial. Y, aun siendo Hijo, aprendió, sufriendo, a obedecer. Y, llevado a la consumación, se convirtió, para todos los que lo obedecen, en autor de salvación eterna.\end{readbody}
			
			\begin{readtitle}EVANGELIO\end{readtitle}
			
			
			
			\begin{readbook}Del Evangelio según san Juan \rightline{18, 1–19, 42}\end{readbook}
			
			
			
			\begin{readtheme}Pasión de nuestro Señor Jesucristo\end{readtheme}
			
			
			
			\begin{rubrica}Omitimos el texto del Evangelio debido a su gran extensión.\end{rubrica}
			
			
			
			\section{Comentarios Patrísticos}
			
			
			
			
			
			\subsection{San Juan Crisóstomo, obispo}
			
			
			
			\begin{patertheme}El valor de la sangre de Cristo\end{patertheme}
			
			\begin{patersource}Catequesis 3, 13-19: SC 50, 174-177.\end{patersource}
			
			\begin{body}¿Deseas conocer el valor de la sangre de Cristo? Remontémonos a las figuras que la profetizaron y recordemos los antiguos relatos de Egipto.\end{body}
			
			\begin{body}Inmolad –dice Moisés– un cordero de un año; tomad su sangre y rociad las dos jambas y el dintel de la casa. “¿Qué dices, Moisés? La sangre de un cordero irracional ¿puede salvar a los hombres dotados de razón?” “Sin duda –responde Moisés–: no porque se trate de sangre, sino porque en esta sangre se contiene una profecía de la sangre del Señor”.\end{body}
			
			\begin{body}Si hoy, pues, el enemigo, en lugar de ver las puertas rociadas con sangre simbólica, ve brillar en los labios de los fieles, puertas de los templos de Cristo, la sangre del verdadero Cordero, huirá todavía más lejos.\end{body}
			
			\begin{body}¿Deseas descubri<a id="_idTextAnchor019"></a>r aún por otro medio el valor de esta sangre? Mira de dónde brotó y cuál sea su fuente. Empezó a brotar de la misma cruz y su fuente fue el costado del Señor. Pues muerto ya el Señor, dice el Evangelio, \textit{uno de los soldados se acercó con la lanza, le traspasó el costado, y al punto salió agua y sangre}: \textit{agua}, como símbolo del bautismo; \textit{sangre}, como figura de la eucaristía. El soldado le traspasó el costado, abrió una brecha en el muro del templo santo, y yo encuentro el tesoro escondido y me alegro con la riqueza hallada. Esto fue lo que ocurrió con el cordero: los judíos sacrificaron el cordero, y yo recibo el fruto del sacrificio.\end{body}
			
			\begin{body}Del costado salió \textit{sangre} y \textit{agua}. No quiero, amado oyente, que pases con indiferencia ante tan gran misterio, pues me falta explicarte aún otra interpretación mística. He dicho que esta \textit{agua} y esta \textit{sangre} eran símbolos del bautismo y de la eucaristía. Pues bien, con estos dos sacramentos se edifica la Iglesia: con el \textit{agua} de la regeneración y con la renovación del Espíritu Santo, es decir, con el bautismo y la eucaristía, que han brotado, ambos, del costado. Del costado de Jesús se formó, pues, la Iglesia, como del costado de Adán fue formada Eva.\end{body}
			
			\begin{body}Por esta misma razón, afirma san Pablo: \textit{Somos miembros de su cuerpo, formados de sus huesos}, aludiendo con ello al costado de Cristo. Pues del mismo modo que Dios formó a la mujer del costado de Adán, de igual manera Jesucristo nos dio el \textit{agua} y la \textit{sangre }salidas de su costado, para edificar la Iglesia. Y de la misma manera que entonces Dios tomó la costilla de Adán, mientras éste dormía, así también nos dio el \textit{agua} y la \textit{sangre} después que Cristo hubo muerto.\end{body}
			
			\begin{body}Mirad de qué manera Cristo se ha unido a su esposa, considerad con qué alimento la nutre. Con un mismo alimento hemos nacido y nos alimentamos. De la misma manera que la mujer se siente impulsada por su misma naturaleza a alimentar con su propia sangre y con su leche a aquel a quien ha dado a luz, así también Cristo alimenta siempre con su \textit{sangre} a aquellos a quienes él mismo ha hecho renacer.\end{body}
			
			\subsection{San Cirilo de Alejandría, obispo}
			
			\begin{patertheme}Cristo entregó su alma en manos del Padre, \end{patertheme}
			
			\begin{patertheme}abriéndonos a nuevas y luminosas esperanzas\end{patertheme}
			
			\begin{patersource}Comentario sobre el evangelio de san Juan, Lib. 12: PG 74, 667-670.\end{patersource}
			
			\begin{body}Jesús, cuando tomó el vinagre, dijo: “\textit{Está cumplido”}. E, inclinando la cabeza, entregó el espíritu.\end{body}
			
			\begin{body}Con razón dijo: \textit{“Está cumplido”. }Ha sonado ya la hora de llevar el mensaje de salvación a los espíritus que se encuentran en los abismos. Él vino efectivamente para establecer su señorío sobre vivos y muertos. Por nosotros soportó la misma muerte en la carne asunta, enteramente igual a la nuestra, él que por naturaleza, Dios como es, es la vida misma. Todo esto, lo ha querido él expresamente para destronar a los poderes abismales y preparar de este modo el retorno de la naturaleza humana a la vida verdadera, \textit{él primicia de todos los que han muerto y primogénito de toda criatura.}\end{body}
			
			\begin{body}\textit{Inclinando la cabeza: }es el gesto característico del que acaba de morir, cuando, al faltar el espíritu que mantiene unido a todo el cuerpo, los músculos y los nervios se relajan. Por eso, la expresión del evangelista no es del todo apropiada, aunque inmediatamente introduzca otra frase comúnmente utilizada, también ella, para indicar que uno ha muerto: \textit{entregó el espíritu.}\end{body}
			
			\begin{body}Parece como si impulsado por una particular inspiración, el evangelista no haya dicho simplemente \textit{murió, }sino \textit{entregó el espíritu. }Es decir, entregó su espíritu en manos de Dios Padre, de acuerdo con lo que él mismo había dicho, si bien a través de la profética voz del salmista: \textit{Padre, a tus manos encomiendo mi espíritu. }Y mientras tanto, la fuerza y el sentido de estas palabras constituían para nosotros el comienzo y el fundamento de una dichosa esperanza.\end{body}
			
			\begin{body}Debemos efectivamente creer que las almas de los santos, al salir del cuerpo, no sólo se confían a las manos del Padre amadísimo, Dios de bondad y de misericordia, sino que en la mayoría de los casos se apresuran al encuentro del Padre común y de nuestro Salvador Jesucristo, que nos despejó el camino. Ni es correcto pensar –como hacen los paganos–, que estas almas estén revoloteando en torno a sus tumbas, en espera de los sacrificios ofrecidos por los muertos, o bien que sean arrojadas, como las almas de los pecadores, en el lugar del inmenso suplicio, esto es, en el infierno.\end{body}
			
			\begin{body}Cristo entregó su alma en las manos del Padre, para que en ella y por ella logremos nosotros el comienzo de la luminosa esperanza, sintiendo y creyendo firmemente que, después de haber soportado la muerte de la carne, estaremos en las manos de Dios, en un estado de vida infinitamente mejor que el que teníamos mientras vivíamos en la carne. Por eso el Doctor de los gentiles escribe que es mucho mejor partir de este cuerpo para estar con Cristo.\end{body}
			
			\subsection{San Cirilo de Alejandría, obispo}
			
			\begin{patertheme}Con su muerte corporal, Cristo redimió la vida de todos\end{patertheme}
			
			\begin{patersource}Comentario sobre el evangelio de san Juan, Lib. 12: PG 74, 679-682.\end{patersource}
			
			\begin{body}\textit{Tomaron el cuerpo de Jesús y lo vendaron todo, con los aromas, según se acostumbra a enterrar entre los judíos. Había un huerto en el sitio donde lo crucificaron, y en el huerto un sepulcro nuevo donde nadie había sido enterrado todavía.}\end{body}
			
			\begin{body}Fue contado entre los muertos el que por nosotros murió según la carne; huelga decir que él tiene la vida en sí mismo y en el Padre, pues ésta es la realidad. Mas para cumplir todo lo que Dios quiere, es decir, para compartir todas las exigencias inherentes a la condición humana, sometió el templo de su cuerpo no sólo a la muerte voluntariamente aceptada, sino asimismo a aquella serie de situaciones que son secuelas de la muerte: la sepultura y la colocación en una tumba.\end{body}
			
			\begin{body}El evangelista precisa que en el huerto había un sepulcro y que este sepulcro era nuevo. Lo cual, a nivel de símbolo, significa que con la muerte de Cristo se nos preparaba y concedía el retorno al paraíso. Y allí, en efecto, entró Cristo como precursor nuestro.\end{body}
			
			\begin{body}La precisión de que el sepulcro era nuevo indica el nuevo e inaudito retorno de Jesús de la muerte a la vida, y la restauración por él operada como alternativa a la corrupción. Efectivamente, en lo sucesivo nuestra muerte se ha transformado, en virtud de la muerte de Cristo, en una especie de sueño o de descanso. Vivimos, en efecto, como aquellos que –según la Escritura–, \textit{viven para el Señor. }Por esta razón, el apóstol san Pablo, para designar a los que han muerto en Cristo, usa casi siempre la expresión “los que se durmieron”.\end{body}
			
			\begin{body}Es verdad que en el pasado prevaleció la fuerza de la muerte contra nuestra naturaleza. \textit{La muerte reinó desde Adán hasta Moisés, incluso sobre los que no habían pecado con un delito como el de Adán,} y, como él, llevamos la imagen del hombre terreno, soportando la muerte que nos amenazaba por la maldición de Dios. Pero cuando apareció entre nosotros el segundo Adán, divino y celestial que, combatiendo por la vida de todos, con su muerte corporal redimió la vida de todos y, resucitando, destruyó el reino de la muerte, entonces fuimos transformados a su imagen y nos enfrentamos a una muerte, en cierto sentido, nueva. De hecho esta muerte no nos disuelve en una corrupción sempiterna, sino que nos infunde un sueño lleno de consoladora esperanza, a semejanza del que para nosotros inauguró esta vía, es decir, de Cristo.\end{body}
			
			\section{Homilías}
			
			\begin{rubrica}Tradicionalmente el Santo Padre no preside la celebración litúrgica del Viernes Santo. Recogemos aquí las Alocuciones en el \textit{Vía Crucis}, que aplican también perfectamente para la celebración de la Pasión del Señor.\end{rubrica}
			
			\subsection{San Pablo VI, papa }
			
			\subsubsection{Alocución (1964): Sufrimiento que da fruto}
			
			\begin{referencia}27 de marzo de 1964.\end{referencia}
			
			\begin{body} Acabamos de contemplar la Pasión del Señor en el Señor. Queremos creer que todos vosotros habréis intuido su profundidad y riqueza. Ahora extenderemos una mirada a la irradiación de esta Pasión, única y típica, centro de los destinos humanos, sobre la humanidad misma. Es el faro que ilumina al mundo. \textit{Crux lux}.\end{body}
			
			\begin{body}Uno de estos aspectos es el sufrimiento humano. Está iluminado de un modo bien conocido, pero siempre singular, a la luz de la cruz el dolor (podríamos señalar todas las miserias, toda la pobreza, todas las enfermedades y hasta todas las debilidades, es decir, todas las condiciones que hacen una vida deficiente y necesitada de atenciones), el dolor aparece extrañamente asimilable a la Pasión de Cristo, como llamado a integrarse con ella, como constituyendo una condición de favor respecto a la redención obrada por la Cruz del Señor. El dolor se hace sagrado. Antes –y todavía, para quien se olvida que es cristiano– el sufrimiento parecía pura desgracia, pura inferioridad, más digna de desprecio y repugnancia que merecedora de comprensión, de compasión, de amor. Quien ha dado al dolor del hombre su carácter sobrehumano, objeto de respeto, de cuidados y de culto, es Cristo doliente, el gran hermano de todos los pobres, de todos los afligidos. Hay más, Cristo no demuestra solamente la dignidad del dolor; Cristo lanza un llamamiento al dolor. Esta voz, hijos y hermanos, es la más misteriosa y la más benéfica que ha atravesado la escena de la vida humana. Cristo invita al dolor a salir de su desesperada inutilidad, a ser, unido al suyo, fuente positiva de bien, fuente no sólo de las más sublimes virtudes –desde la paciencia hasta el heroísmo y la sabiduría–, sino también de capacidad expiadora, redentora, beatificante, propia de la Cruz de Cristo. El poder salvífico de la Pasión de Cristo puede hacerse universal e inmanente en nuestros sufrimientos, si –he ahí la condición– se acepta y soporta en comunión con sus sufrimientos. La “com-pasión”, de pasiva se hace activa; idealiza y santifica el dolor humano, lo complementa con el del Redentor (Cfr. \textit{Col} 1, 24). Todos, debemos recordar esta inefable posibilidad. Nuestros sufrimientos (siempre dignos de cuidados y remedios), se hacen buenos, preciosos. En el cristiano se inicia un arte extraño y estupendo, de saber sufrir, hacer que el propio dolor sirva para la redención propia y ajena.\end{body}
			
			\begin{body}Esta providencialidad del sufrimiento nos hace pensar en las condiciones, siempre tristes y ofensivas para los ideales humanos, en que la civilización moderna quisiera inspirarse, en las cuales todavía se encuentra en gran parte a la Iglesia católica. El cuerpo de Cristo está crucificado moralmente, pero con saña, todavía hoy, en muchas regiones del mundo; la Iglesia del silencio es todavía la Iglesia doliente, la Iglesia paciente, y en ciertos lugares, la Iglesia amordazada. Cristo podría preguntar, hoy también, a los modernos y hábiles perseguidores: “…¿por qué me persigues?” (\textit{Hch} 9, 4). Es triste para quien es objeto de tales tratos; es indigno para quienes los practican, aunque se enmascaren de hipocresías legales. Pero estamos seguros que estas prolongadas pasiones están fortificadas por la asistencia divina y consoladas por nuestra com-pasión y la de toda la fraternidad universal cristiana, y esperamos que sean precisamente, en virtud de la cruz de Cristo a la que se ofrecen y por la que sufren, fuente de gracia para cuantos las padecen, para toda la Iglesia y para todo el mundo.\end{body}
			
			\begin{body}Y otro aspecto, reflejo de la cruz de Cristo, sobre la faz de la tierra, es la paz. La paz, que es el bien supremo del orden humano, esa paz que es tanto más deseable, cuanto más se inclina el mundo a formas de vida interdependientes y comunitarias, de forma que una infracción de la paz en un punto determinado repercute sobre todo el sistema organizativo de las naciones; esa paz que se hace, por tanto, cada vez más necesaria y obligada; esa paz, que los esfuerzos humanos, aunque muy nobles y dignos de aplauso y de solidaridad, difícilmente consiguen tutelar en su integridad y sostener con otros medios que no sean el temor y el interés temporal. La paz de Cristo llueve de lo alto, es decir, proyecta sobre la tierra y entre los hombres motivos y sentimientos originales y prodigiosos; lo sabemos, y viene precisamente de Aquel, como escribe San Pablo, que “por divina complacencia debía recapitular en sí todas las cosas habiéndolas pacificado con su sangre desde su cruz” (Cfr. \textit{Col} 1, 20), de forma que los hombres, divididos y enemigos entre sí fueran “reconciliados en un cuerpo único por medio de la cruz” (Cfr. \textit{Ef} 2, 16). Cristo Redentor nos ha enseñado cómo y por qué los hombres debemos y podemos vivir en la verdadera paz, y nos la ha conseguido si de verdad queremos.\end{body}
			
			\begin{body}Terminaremos esta conmovida y pública oración del Viernes Santo pidiendo a Cristo “nuestra paz” (\textit{Ef} 2, 14.) la paz para el mundo. En este momento están presentes a nuestro espíritu, los puntos geográficos y políticos, donde está herida la paz, donde está amenazada. Enviamos nuestro pensamiento lleno de buenos augurios a los hombres que se esfuerzan rectamente por salvar la paz, y para que los hombres sepan mantenerse hermanos en Cristo enviamos al mundo –y a vosotros aquí presentes que oráis y esperáis–, nuestra bendición apostólica.\end{body}
			
			\subsubsection{Alocución (1967): Sello de autenticidad del discípulo}
			
			\begin{referencia}24 de marzo de 1967. \end{referencia}
			
			\begin{body}En este lugar, que nos habla del testimonio de fe, fortaleza y sangre de tantos Mártires por el Nombre de Cristo, \begin{bodysmall}[en este día, en el que el doloroso recuerdo de las víctimas de la Fosse Ardeatine revive en Roma] (...)\end{bodysmall} En esta hora de la historia contemporánea (...) meditamos sobre la Pasión del Señor. Esta forma de meditación, casi guionizada, y alternada con cánticos y oraciones, nos ayuda no solo a recordar los sufrimientos de Cristo, sino a descubrir, en cierta medida, la profundidad, el drama, el misterio sumamente complejo, donde el dolor humano en su grado más alto, el pecado humano en su más trágica repercusión, el amor en su expresión más generosa y heroica, la muerte en su más cruel victoria y su derrota definitiva... adquieren la evidencia más impresionante. \end{body}
			
			\begin{centerbold}Teniendo un concepto exacto de Cristo y del cristianismo \end{centerbold}
			
			\begin{body}Haremos bien en grabar esta dolorosa pero sabia meditación en nuestras almas; recordarlo, repetirlo. Por dos motivos. \end{body}
			
			\begin{body}El primero, tener un concepto exacto de Cristo y del cristianismo. La Pasión de Cristo ocupa un lugar esencial en el Evangelio. Existe una tendencia generalizada a mantener cerradas las páginas del Evangelio, que documentan el trágico epílogo de la corta vida temporal de Jesús; son páginas inquietantes. Quisiéramos un Evangelio más sereno, más fácil, más cómodo, más acorde con nuestro muy fuerte instinto y nuestro muy hábil interés de quitar el dolor de la vida, y ante todo el dolor voluntario, es decir, el sacrificio. ¿Qué sería un Evangelio, es decir, un cristianismo, sin la cruz, sin dolor, sin el sacrificio de Jesús? Sería un evangelio, un cristianismo sin la redención, sin la salvación, de la que –aquí debemos reconocerlo con despiadada sinceridad– necesitamos absolutamente. El Señor nos salvó con la Cruz; nos devolvió la esperanza, el derecho a la vida con su muerte. No podemos honrar a Cristo si no lo reconocemos como nuestro Salvador; y no podemos reconocerlo como nuestro Salvador si no honramos el misterio de su Cruz. \end{body}
			
			\begin{centerbold}Llevando nuestra cruz: Jesús estará con nosotros \end{centerbold}
			
			\begin{body}Y luego debemos repetir la invocación con la que varias veces, en cada estación del \textit{Vía Crucis}, solemos dirigirnos a Nuestra Señora, la más afligida Madre de Cristo: ¡eh! ¡dejad que las llagas del Señor se graben en mi corazón! \end{body}
			
			\begin{body}¿Por qué esta impresión? ¿No es suficiente que hayamos contemplado las llagas en el mismo Cristo? ¿No ha satisfecho Él todo por nosotros? Sí, Él nos ha salvado y cargó su Cruz por nosotros, ¿por qué deberíamos llevarla nosotros también? Esta es la segunda enseñanza del \textit{Vía Crucis}: el Señor hizo del dolor un medio de redención; con su dolor, sí, nos ha redimido, siempre que no nos neguemos a unir nuestro dolor al suyo, y hacer de él un medio para nuestra redención. En otras palabras: también nosotros debemos llevar, de alguna manera y en cierta medida, nuestra cruz, validada para la salvación por la Cruz de Cristo. \end{body}
			
			\begin{body}¡Llevar la cruz! ¡Una cosa grande, una cosa grande, queridos hijos! Significa afrontar la vida con valentía, sin pusilanimidad y sin cobardía; significa transformar las inevitables dificultades de nuestra existencia en energía moral; ¡significa saber comprender el dolor humano y finalmente saber cómo amar de verdad! Significa aceptar el sello de autenticidad de los discípulos y seguidores de Cristo y establecer con él una comunión incomparable.\end{body}
			
			\subsubsection{Alocución (1970): Soy culpable de su sangre}
			
			\begin{referencia}27 de marzo de 1970. \end{referencia}
			
			\begin{body}Esta oración peregrinante por el camino de la Cruz nos deja al final muy pensativos. Sentimos que nosotros mismos hemos entrado en el plan profético de este doloroso drama; Jesús mismo lo había predicho: “cuando yo sea levantado de la tierra, atraeré a todos hacia mí” (\textit{Jn }12, 32). Nos sentimos descritos por el texto bíblico, con el que el evangelista Juan concluye su relato de la crucifixión del Señor: “mirarán a aquel a quien traspasaron” (\textit{Jn }19, 37). \end{body}
			
			\begin{body}Sí, estamos mirando. Por atroz que sea la imagen de Jesús crucificado, nos sentimos atraídos por este Hombre de dolor; y la espantosa repugnancia que suele despertar la visión del cadáver de un ejecutado todo lleno de llagas y ensangrentado, se ve superado por una singular fascinación, que fija no solo nuestra mirada, sino más aún el alma en esa figura “sin ninguna belleza ni esplendor” (cfr. \textit{Is} 53, 2). Inmediatamente nos convencemos de que nos encontramos ante una revelación que va más allá de la imagen sensible; la revelación intencional de un símbolo, de un prototipo, de una personificación extrema del sufrimiento humano. Jesús, el Cristo, quería ser presentado así. \end{body}
			
			\begin{body}¿Por qué así? ¡Oh! ¡Qué exploración se ofrece a nuestra piedad, a nuestra ciencia del hombre, a nuestra teología! Ciertamente no podemos consumarlo aquí, pero solo, en algunos puntos, decirlo. ¡Aquí el dolor parece consciente! La terrible pasión era prevista. ¡La tortura y el deshonor de la Cruz eran conocidos! y fueron queridos en su cruel totalidad hasta el final, sin los habituales narcóticos, que mitigan nuestro sufrimiento: desconocimiento de sí, de cuándo, de cómo vendrá; o más bien el misericordioso y sabio alivio del arte médico. Jesús es el “que conoce la enfermedad” en toda su extensión, en toda su profundidad, en toda su intensidad, en todo su horror , tanto como para exprimir la sangre de sus venas en la agonía espiritual de Getsemaní. Y eso es suficiente para hacerlo hermano de todo hombre que llora y sufre; hermano mayor, nuestro hermano. Tiene una primacía que centra en él la simpatía, la solidaridad, la comunión de todo hombre sufriente. \end{body}
			
			\begin{body}Y luego: vemos en este sublime protagonista del dolor humano otra nota, también brillando en él más que en cualquier otro afectado por nuestros dolores: la inocencia. Cuando nos encontramos con un niño que está sufriendo, cuando observamos a alguien que suma al sufrimiento físico o moral la agonía de una pregunta ciega, que parece quedar sin respuesta: ¿por qué? ¿Por qué este desorden, por qué este atropello inexplicable al derecho fundamental de la existencia, a vivir bien, cuando la experiencia del mal arrecia sin razón aparente? Misterio, sí, el dolor inocente es un misterio para nosotros; pero el encuentro que hacemos de este misterio en el divino Crucifijo, en Él, el Supremo, el verdaderamente inocente (cf. \textit{Lc} 23, 41) al menos detiene la blasfemia que vendría a nuestros labios. Jesús también era inocente, era un cordero, era el cordero de Dios, que humilde y débil se dejó llevar al matadero (\textit{Is} 53, 7). Si es así, la pregunta vuelve a surgir, pero ya no desesperada y rebelde, sino ansiosa por una respuesta anticipada y prodigiosa.\end{body}
			
			\begin{body}Y es esta: Jesús murió inocente, porque Él lo quiso (\textit{Ibid}.: \textit{Jn} 10, 17-18). Pero, ¿por qué lo quería? Aquí está la clave de toda esta tragedia: porque quería asumir toda la expiación de la humanidad (\textit{Is} 53, 6; \textit{Jn} 11, 51; \textit{2 Cor} 5, 21); Él se ofreció como víctima para reemplazarnos; sí, Él es “el Cordero de Dios que quita el pecado del mundo” (\textit{Jn} 1, 29); Él se sacrificó por nosotros; Él se entregó a sí mismo por nosotros; ¡así Él nos redimió! ¡Él es así nuestra salvación! \end{body}
			
			\begin{body}Y por eso el Crucifijo encadena nuestra atención casi alucinada: si Cristo ha asumido sobre sí la deuda de la justicia de Dios por mis faltas, yo soy corresponsable, ¡soy culpable de su sangre! y entonces ese descubrimiento se convierte en alegría, que estalla en gratitud y amor: “Me amó y se sacrificó por mí” (\textit{Gá} 2, 20). \end{body}
			
			\begin{body}Y todo desemboca en la verdadera ciencia del amor, que traeremos a nuestra vida desde este Viernes Santo: es el dolor consciente, inocente, sufrido por el amor que redime y salva; como Cristo, debemos entregarnos voluntaria, libremente y hasta dolorosamente, por el bien de los demás, por la redención de la humanidad, por la salvación y la paz del mundo. ¡Así volvemos afligidos, pensativos, valientes, después del Vía Crucis!\end{body}
			
			\subsubsection{Alocución (1973): Cruz que irradia la esperanza}
			
			\begin{referencia}20 de abril de 1973. \end{referencia}
			
			\begin{body}Este doloroso camino de la Cruz nos ha llevado a la última estación, a la del sepulcro, a la<a id="_idTextAnchor020"></a> de la piedad, donde María, la Madre del divino Hijo, tiene en su seno el fruto del dolor, el odio y la muerte: el cadáver de Cristo crucificado. \end{body}
			
			\begin{body}Hermanos ¡viajeros de esta común peregrinación! ¿Sabéis por qué todos nos sentimos atraídos, casi a pesar de nosotros mismos, por este espectáculo tan triste? ¿Y por qué no podemos separar nuestro corazón de esta contemplación terminal y trágica? ¿Por qué nuestra mirada no se horroriza al ver un tormento humano tan cruel y tan horrendo, infligido al más bello y al mejor de los hijos de la humanidad, a nuestro hermano más cercano y solidario, al amigo, al maestro, al pastor de todos nosotros, al Verbo de Dios hecho hombre como nosotros? \end{body}
			
			\begin{body}¿No sentimos asco y miedo? ¿No sentimos remordimiento? ¿No nos asalta un extraño e instintivo sentido de corresponsabilidad? ¿No nos sentimos todos nosotros cómplices de esta muerte, la más humillante, la más injusta entre las que han ensangrentado la tierra? ¿No se reflejan todos nuestros pecados en el Cristo inmolado, como en la víctima más sensible y central del mundo entero? Sí, cada uno de nosotros puede acusarse ante la pasión y muerte de Jesús: ¡también es culpa mía! \end{body}
			
			\begin{body}Sin embargo, hacia Él, hacia el Cristo crucificado, nos sentimos atraídos, en esta hora de tinieblas, pero atravesados por los destellos de una nueva conciencia; atraídos, decimos: Él lo había predicho: “Cuando yo sea levantado de la tierra (quiso decir: cuando sea exaltado en la cruz), atraeré a todos hacia mí” (\textit{Jn} 3, 14; 12, 32).\end{body}
			
			\begin{body}Hermanos, dejemos que este encanto misterioso nos domine con su doble sentimiento, de reproche y de esperanza. \end{body}
			
			\begin{body}De reproche: ¿acaso no reflejan crudamente las heridas todavía sangrantes de Cristo toda la violencia, la tortura, las masacres, la barbarie, el odio, la maldad, la soberbia, la insensibilidad de las que es capaz el hombre moderno? Sí, el hombre ha alcanzado grandes avances en la civilización, pero sigue siendo miope sobre cómo usarlos sabiamente. Y ahora nos decimos a nosotros mismos: ¡que cesen los atropellos contra la vida y la dignidad de los hombres! ¡Que acabe la impasible inhumanidad, que ataca la vida inocente e indefensa que nace en el útero! ¡Que cese el crimen, que hoy se profesionaliza y organiza! ¡Que termine la estrategia, que se basa en la competencia por el poder mortal de las armas científicas! ¡Que acabe la degradante licencia del vicioso placer, erigido como ideal de libertad y felicidad ciega y egoísta! Esta invectiva podría extenderse hasta donde llega la degradación humana, ¡muy lejos! \end{body}
			
			\begin{body}Pero escuchemos más bien las efusiones de esperanza que irradia la Cruz de Cristo. La primera esperanza es la misericordia, el perdón, la reconciliación de Dios con nosotros. Así como el pecado, recordemos bien, es nuestra primera y más grave desgracia, porque corta nuestra relación con la verdadera Vida, que es Dios, así la liberación del pecado es nuestra primera e indispensable fortuna. Y qué suerte para nosotros saber que Cristo, con Su Sangre, ha pagado por nosotros, ha expiado por nosotros, ha reparado nuestra irremediable maldición; y nos hizo levantarnos a una nueva existencia, y tener la esperanza de la felicidad eterna.\end{body}
			
			\begin{body}Y nos devolvió, con su muerte por amor, el amor a nuestros hermanos, nos enseñó a perdonar, a sentir las necesidades de los demás, a servir a los más débiles, a sacrificarnos por los demás; es decir, llevar humanidad a los hombres, bondad y justicia al mundo. Y por tanto paz. Y luego también: nos enseñó el valor del sufrimiento y la fecundidad del dolor, la dignidad en la desgracia. Y nos ha concedido, porque así lo ha querido, que su cruz, profecía y garantía de resurrección, fuera plantada en cada tumba. \end{body}
			
			\begin{body}¡Hermanos! No pongamos fin a este camino hacia la Cruz sin el secreto propósito personal de continuarlo. Cristianos somos y debemos ser, y por eso respondemos con el corazón y con nuestra forma de vivir a la invitación de Cristo: “Venid a mí todos los que estáis cansados y oprimidos, y yo os consolaré” (\textit{Mt} 11, 28).\end{body}
			
			\begin{body}Que nuestra Bendición Apostólica conforte ahora en cada uno de nosotros estos sentimientos y resoluciones.\end{body}
			
			\subsubsection{Alocución (1976): Una muerte que nos concierne}
			
			\begin{referencia}16 de abril de 1976. \end{referencia}
			
			\begin{body}Hemos completado\textbf{ }el “Vía Crucis”, el camino de la Cruz. Seguimos este triste y trágico itinerario, recordando paso a paso la ejecución bárbara y cruel del condenado Jesús, el Maestro, el predicador del rei<a id="_idTextAnchor021"></a>no de Dios, el buen Pastor “manso y humilde de corazón” (\textit{Mt} 11, 29), quien había pasado “haciendo el bien y sanando a todos” (\textit{Hch }10, 38), y se había llamado a sí mismo el Hijo del hombre y luego el Hijo de Dios, el Mesías y por lo tanto, el “Rey de los judíos”, desatando contra sí mismo la furia de los líderes del pueblo y la tácita condena del Procurador Romano Poncio Pilato: un drama complicado de pretextos políticos (\textit{Jn} 11, 48; 19, 12), y aún más de razones religiosas (\textit{Mt} 26, 63-64; \textit{Jn} 11, 51; 19, 7); una muerte desgarradora e injusta; un episodio violento y doloroso, ciertamente, como el de quien hace de la muerte un testigo, un martirio; que terminó pronto, a la hora novena de ese día santo cercano a la Pascua Oficial, concluido súbitamente con un entierro apresurado. “\textit{Consummatum est }– Todo está cumplido” \textit{(Jn }19, 30), exclamó Jesús agonizante.\end{body}
			
			\begin{body}¡Hijos y hermanos! en nuestros ojos, en nuestras almas, se ha reproducido la desgarradora historia de Jesús; nos cautiva y quizás incluso nos conmueve, como ocurre con las escenas sangrientas y los casos dramáticos y singulares. Pero queda una duda, una cuestión por resolver; que ahora nos concierne; nosotros personalmente: ¿estamos involucrados en esta historia? ¿cómo hemos asistido? ¿como meros y extraños espectadores? ¿como curiosos y eruditos de la muerte de un hombre sabio y justo, como lo fue, por ejemplo, la muerte de Sócrates? No, hermanos e hijos; ¡No asistimos como observadores curiosos e impasibles! No, todos prestamos atención a la conclusión de esta historia, que nos involucra a todos. Lo queramos o no, somos corresponsables de la muerte de Jesús, esta es la primera conclusión que este piadoso ejercicio del “Vía Crucis” debe suscitar en nuestras conciencias. Sabemos bien que la afirmación de nuestra culpa en la crucifixión de Cristo requeriría pruebas formidables, que nuestros tribunales no podrían reconocer como legales; pero la realidad de la historia humana, como nos recuerda la más sabia teología, hace de toda la humanidad la causa de la muerte de la víctima divina. Una solidaridad universal hace a todos los hijos de Adán culpables y a todos deudores ante Dios, con esta doble conclusión: la primera, que todo hombre pesa en la balanza de la redención, en la necesidad de expiación, de la que Cristo es víctima, “el Cordero de Dios que quita el pecado del mundo” (\textit{Jn} 1, 29. 36 ); los santos, estos conocedores de la conciencia humana profunda y real, sintieron esta experiencia moral, es decir, cómo cada uno de nosotros fue verdugo en la crucifixión del Señor (cf. \textit{Heb }6, 6), porque todo pecado humano contribuye a la necesidad de una reparación, que sólo la Palabra de Dios Salvador, que vino al mundo por nuestra salud, podía ofrecer a la justicia y misericordia de Dios. Y segunda conclusión: de los crucificadores nos hemos convertido en beneficiarios, en salvados por la víctima misma sacrificada por nosotros, en nuestro nombre, por nuestra salvación. Cuando hablamos de redención, de sacrificio divino, nos referimos a este drama, donde el culpable puede ser recompensado por el arrepentimiento de su fechoría.\end{body}
			
			\begin{body}Tal es el misterio detrás del “Vía Crucis”. El misterio de la redención, el misterio de nuestra salvación, el misterio de la virtud redentora de nuestro dolor si se une al de la pasión de Cristo (cf. \textit{Col }1, 24), el misterio del amor inmolado de Cristo, que él hizo de su muerte fuente de nuestra vida eterna (cf. \textit{Heb} 5, 9).\end{body}
			
			\begin{body}Así, disolviendo esta asamblea orante y ansiosa con nuestros deseos pascuales y con nuestra bendición apostólica, cada uno de nosotros puede hacer suyo el testimonio amargo, pero renovador y muy feliz del Centurión en el momento de la muerte de Cristo: “Verdaderamente este era el Hijo de Dios” (\textit{Mt }27, 54).\end{body}
			
			\subsection{San Juan Pablo II, papa}
			
			\subsubsection{Alocución (1979): Quédate con nosotros}
			
			\begin{referencia}13 de abril de 1979.\end{referencia}
			
			\begin{body} 1. (...) [En este día] estamos siempre presentes en e<a id="_idTextAnchor022"></a>spíritu allí donde \textit{este camino }[de la cruz] \textit{tuvo su lugar “históricamente”:} allí donde se desarrolló, a lo largo de las calles de Jerusalén, desde el Pretorio de Pilato hasta la cima del Gólgota, es decir, del Calvario, fuera de las murallas. Así, pues, también hoy hemos estado en espíritu allí, en la ciudad del “gran Rey”, que como signo de su realeza ha escogido la corona de espinas en vez de la corona real, y la cruz en lugar del trono.\end{body}
			
			\begin{body}¿No tenía razón Pilato cuando, presentándolo al pueblo que esperaba su condenación ante el Pretorio “por no contaminarse, para poder comer la Pascua” (\textit{Jn} 18, 28), en vez de decir “He aquí al rey”, dijo “Ahí tenéis al hombre” (\textit{Jn} 19, 5)? Y así reveló el programa de su reino que quiere verse libre de los atributos del poder terreno para descubrir los pensamientos de muchos corazones (cf. \textit{Lc} 2, 35) y para acercarlos a la verdad y al amor que proviene de Dios.\end{body}
			
			\begin{body}“Mi reino no es de este mundo... Yo para esto he nacido y para esto he venido al mundo, para dar testimonio de la verdad” (\textit{Jn }18, 36-37). Este testimonio ha permanecido en las esquinas de las calles de Jerusalén, en los recodos del \textit{Vía Crucis}, allí por donde caminaba, donde cayó por tres veces, donde aceptó la ayuda de Simón Cirineo y el velo de la Verónica, allí donde habló a algunas mujeres que se apiadaban de Él. Hoy día seguimos aún deseosos de este testimonio. Queremos conocer todos los detalles. Seguimos las huellas del \textit{Vía Crucis} en Jerusalén y a la vez en tantos otros lugares de nuestra tierra; y cada vez nos parece repetir a este Condenado, a este Hombre de dolores: “Señor, ¿a quién iríamos? Tú tienes palabras de vida eterna” (\textit{Jn} 6, 68).\end{body}
			
			\begin{body}2. [Hoy] (...) estamos también sobre las huellas de Cristo, cuya cruz encontró sitio en los corazones de sus mártires y confesores. Ellos anunciaban a Cristo crucificado como “poder y sabiduría de Dios” (\textit{1 Cor} 1, 24). Tomaban cada día la cruz en unión con Cristo (cf. \textit{Lc} 9, 23), y cuando era necesario morían como Él en la cruz, o morían sobre la arena de la Roma antigua, devorados por las fieras, quemados vivos o torturados. El poder de Dios y la sabiduría de Dios revelados en la cruz, se manifestaban así más poderosamente en las debilidades humanas. Ellos no sólo aceptaban los sufrimientos y la muerte por Cristo, sino que se decidían como Él por el amor a los perseguidores y a los enemigos: “Padre, perdónalos, porque no saben lo que hacen” (\textit{Lc} 23, 34).\end{body}
			
			\begin{body}\begin{bodysmall}[Por esto, \end{bodysmall}\textit{sobre las ruinas del Coliseo se levanta la cruz.}\begin{bodysmall}]\end{bodysmall}\end{body}
			
			\begin{body}Mirando hacia esta cruz, la cruz de los comienzos de la Iglesia (...) y la cruz de su historia, debemos sentir y expresar una \textit{solidaridad} particularmente profunda con todos nuestros hermanos en la fe, que también en nuestra época s\textit{on objeto de persecuciones y de discriminaciones} en diversos lugares de la tierra. Pensemos ante todo en aquellos que están condenados, en cierto sentido, a la “muerte civil” por la denegación del derecho a vivir según la propia fe, el propio rito, según las propias condiciones religiosas. Mirando hacia la cruz (...), pedimos a Cristo que no les falte –al igual que a aquellos que en otro tiempo sufrieron (...) el martirio– la fuerza del Espíritu, de que tienen necesidad los confesores y los mártires de nuestro tiempo.\end{body}
			
			\begin{body}Mirando a la cruz (...) sentimos una unión aún más profunda con ellos, una solidaridad aún más fuerte. Al igual que Cristo tiene \textit{un lugar especial en nuestros corazones} por su pasión, así también ellos. Tenemos el deber de hablar de esta pasión de sus confesores contemporáneos, y darles testimonio ante la conciencia de la humanidad entera, que proclama la causa del hombre, como finalidad principal de todo progreso. ¿Cómo conciliar estas afirmaciones con la lesión causada a tantos hombres, que –mirando a la cruz de Cristo– confiesan a Dios y anuncian su amor?\end{body}
			
			\begin{body}3. ¡Cristo Jesús! Estamos para terminar este santo día del Viernes Santo a los pies de tu cruz. Así como en otro tiempo, en Jerusalén, a los pies de tu cruz se encontraban tu Madre, Juan, Magdalena y otras mujeres, así también estamos aquí nosotros. Estamos profundamente emocionados por la importancia del momento. Nos faltan las palabras para expresar todo lo que sienten nuestros corazones. \end{body}
			
			\begin{body}Ahora, en esta noche –cuando después de haberte bajado de la cruz, te han colocado en un sepulcro en la ladera del Calvario–, queremos suplicarte \end{body}
			
			\begin{bodyprose}\textit{que permanezcas con nosotros mediante tu cruz}: \end{bodyprose}
			
			\begin{bodyprose}Tú que por la cruz te has separado de nosotros. \end{bodyprose}
			
			\begin{bodyprose}Te suplicamos que permanezcas con la Iglesia; \end{bodyprose}
			
			\begin{bodyprose}que permanezcas con la humanidad; \end{bodyprose}
			
			\begin{bodyprose}que no te asustes si muchos pasan \end{bodyprose}
			
			\begin{bodyprose}tal vez indiferentes al lado de tu cruz, \end{bodyprose}
			
			\begin{bodyprose}si algunos se alejan y otros no se llegan a ella.\end{bodyprose}
			
			\begin{bodyprose}No obstante, tal vez hoy más que nunca \end{bodyprose}
			
			\begin{bodyprose}el hombre tiene necesidad de esta fuerza \end{bodyprose}
			
			\begin{bodyprose}y de esta sabiduría que eres Tú mismo, \end{bodyprose}
			
			\begin{bodyprose}¡Tú solo: mediante tu cruz!\end{bodyprose}
			
			\begin{bodyprose}Quédate, pues, con nosotros \end{bodyprose}
			
			\begin{bodyprose}en este penetrante \textit{mysterium} de tu muerte, \end{bodyprose}
			
			\begin{bodyprose}con la que has revelado cuánto “ha amado Dios al mundo” (cf. \textit{Jn} 3, 16). \end{bodyprose}
			
			\begin{bodyprose}Quédate con nosotros y atráenos hacia Ti (cf. \textit{Jn} 12, 32), \end{bodyprose}
			
			\begin{bodyprose}Tú que caíste bajo el peso de esta cruz. \end{bodyprose}
			
			\begin{bodyprose}Quédate con nosotros a través de tu Madre, \end{bodyprose}
			
			\begin{bodyprose}a la que desde la cruz has encomendado \end{bodyprose}
			
			\begin{bodyprose}particularmente a cada hombre (cf. \textit{Jn} 19, 27).\end{bodyprose}
			
			\begin{bodyprose}¡Quédate con nosotros!\end{bodyprose}
			
			\begin{body}\textit{Stat crux, dum volvitur orbis! }\end{body}
			
			\begin{body}Sí, “¡la cruz está alzada sobre el mundo que avanza!”\end{body}
			
			\subsubsection{Alocución (1982): La Cruz nos abre a Dios}
			
			\begin{referencia}9 de abril de 1982. \end{referencia}
			
			\begin{body}1. \textit{“Crucem tuam adoramus”}. \end{body}
			
			\begin{body}Este es el día en que adoramos especialmente la Cruz. La Cruz de Cristo. Este signo, infame instrumento de muerte, ha surgido desde el alba, ante nosotros y penetra las horas del Viernes Santo, durante el cual nos apresuramos solícitos, con nuestros pensami<a id="_idTextAnchor023"></a>entos y corazones, detrás de la pasión del Señor: el camino desde el Pretorio de Pilato al Calvario; la agonía del Calvario. La muerte. Estas horas, llenas de silencio religioso, se hicieron oír más tarde, en la elocuencia de la liturgia de la tarde: \textit{la adoración de la Cruz. }\begin{bodysmall}[Y ahora, a última hora de la tarde, llegamos al Coliseo para abrazar una vez más el conjunto: \end{bodysmall}\textit{El “Vía Crucis”}\begin{bodysmall}: crucifixión - muerte - entierro.]\end{bodysmall}\end{body}
			
			\begin{body}2. \begin{bodysmall}[En el Coliseo,]\end{bodysmall} la cruz, \begin{bodysmall}[plantada entre ruinas monumentales,]\end{bodysmall} nos recuerda efectivamente a todos aquellos que \begin{bodysmall}[en las primeras generaciones de la Iglesia]\end{bodysmall} fueron condenados a la cruz, arrojados a las bestias, torturados de otras formas, martirizados hasta la muerte. Cayeron al suelo \textit{como una semilla} que debe morir para dar fruto y, mirando la Cruz de Cristo, repiten quizás sin palabras: “Crucem tuam adoramus”. La Cruz se ha convertido para ellos en el \textit{signo de la Vida} que nace del sufrimiento y de la muerte: “et sanctam resurrectionem tuam laudamus et glorificamus”. \end{body}
			
			\begin{body}3. ¿Por cuántos lugares de la tierra ha pasado esta Cruz? ¿Por cuántas \textit{generaciones}? ¿Para cuántos discípulos de Cristo se ha convertido en el principal referente en la peregrinación terrena? ¿A cuántos ha preparado para el sufrimiento y la muerte? ¿A cuántos al martirio por Cristo, al testimonio sangriento o incruento? ¿Y a cuántos prepara continuamente para todo esto? \textit{La historia de la Iglesia}, en los distintos continentes y en los distintos países, sólo puede registrar una parte de este “martirologio”. Los altares de las iglesias no han podido acoger en su gloria a los que han dado testimonio de Cristo a través de la cruz. Bastaría pensar en los que vivieron en nuestro siglo. \end{body}
			
			\begin{body}4. “Crucem tuam adoramus, Domine”. Sí. En la Cruz Cristo \textit{demostró ser Señor}: aceptó la muerte y dio la vida. No está simplemente “muerto”, sino que “\textit{ha dado la vida}”. “Nadie tiene mayor amor que el que da la vida por sus amigos” (\textit{Jn} 15, 13). ¡Él ha dado su vida! Acogió la muerte y dio la vida. \end{body}
			
			\begin{body}Sus últimas palabras en la Cruz: “¡Padre, en tus manos encomiendo... encomiendo mi espíritu” (cf. \textit{Lc} 23, 46). Dio su vida por nosotros. Por \textit{todos }los hombres. “Nosotros” somos sólo una pequeña parte de todos aquellos por quienes Cristo dio su vida. No hay hombre, \textit{desde el principio hasta el fin del mundo}, por quien no haya dado su vida. Él dio su vida por todos. Ha redimido a todos. La Cruz es signo de \textit{redención universal}: “excepción enim propter lignum venit gaudium en universo mundo”. \end{body}
			
			\begin{body}5. “Venit gaudium... “. La Cruz es la puerta por la que Dios entró definitivamente en la historia de la humanidad. Y permanece en ella. La Cruz es la puerta por la que Dios entra incesantemente en nuestra vida. Precisamente por eso \textit{nos signamos con la señal de la Cruz}, y al mismo tiempo decimos “en el nombre del Padre y del Hijo y del Espíritu Santo”. Y mientras trazamos el signo de la Cruz en la frente, entre los hombros y en el corazón, también pronunciamos esas palabras. Estas palabras son \textit{una invitación a que venga Dios}. Y nos unimos al signo de la Cruz, para que Dios entre en el corazón del hombre a través de la Cruz. Y así entra en cada obra, pensamiento y palabra: \textit{en toda la vida del hombre y del mundo}. La Cruz nos abre a Dios. La Cruz abre el mundo a Dios. \end{body}
			
			\begin{body}6. Y la bendición también se da con el signo de la Cruz. También los obispos y sacerdotes. Los padres también cuando bendicen a sus hijos. Por la Cruz de Cristo esperamos el bien definitivo de Dios mismo y todos los bienes que nos acercan a él. Todo esto se expresa \textit{en cada bendición}. También de la que impartiré en breve. \end{body}
			
			\begin{body}“Stat crux, dum volvitur orbis”. Todo pasa; la Cruz permanece entre el mundo y Dios. Por la Cruz Dios permanece en el mundo. “Crucem tuam adoramus, Domine”. \end{body}
			
			\begin{body}7. ¡Queridos hermanos y hermanas! Que este día de Viernes Santo, dedicado al misterio de la Cruz, en el que hoy hemos meditado, nos acerque cada vez más al Dios vivo: Padre, Hijo y Espíritu Santo. Que el signo de la muerte de Cristo \textit{vivifique} su presencia y su fuerza en nosotros. \end{body}
			
			\begin{body}\textit{Amén.}\end{body}
			
			\subsubsection{Alocución (1985): Muerte y amor se encuentran}
			
			\begin{referencia} 5 de abril de 1985. \end{referencia}
			
			\begin{body}1. \textit{“Padre, perdónalos, porque no saben lo que hacen”} (\textit{Lc }23, 34). \end{body}
			
			\begin{body}Al final de este día, de este Viernes Santo, volvemos una vez más bajo la cruz, al Calvario. Aunque Cristo ya haya sido bajado de la cruz y aunque haya sido colocado apresuradamente en un sepulcro, “con motivo de la fiesta de la Pascua”, \textit{escuchemos una vez más las palabras }que pronunci<a id="_idTextAnchor024"></a>ó en el momento de su agonía en la cruz. ¿No es acaso una \textit{revelación del amor} \textit{que perdona}, Aquel que, ante todo, pide al Padre que perdone a los responsables de su tormento: “Perdónalos”? ¡Eso es lo que significa: “No saben lo que hacen”! \end{body}
			
			\begin{body}2. \textit{“En verdad te digo que hoy estarás conmigo en el paraíso”} (\textit{Lc} 23, 43). ¿Quién es ese Crucificado que le habla así a otro condenado? ¿No es el mismo Jesús de Nazaret que al inicio de su misión mesiánica dijo: “El reino de Dios está cerca; convertíos y creed en el Evangelio” (\textit{Mc} 1, 14-15)? He aquí que, al final de su ministerio mesiánico, también recoge el fruto en el alma del criminal convertido: “Jesús, acuérdate de mí cuando entres en tu reino” (\textit{Lc} 23, 42). Las palabras de Jesús son una respuesta a esta misma pregunta. Hermano, el reino de Dios está cerca, el reino de Dios está en tu alma. \end{body}
			
			\begin{body}3. Y he aquí, desde lo alto de la cruz, \textit{una nueva palabra del Hijo del Hombre}. ¡Cuán importante es esta palabra que en cierto sentido completa todo el Evangelio! ¡Cuán profundamente nace del corazón del Evangelio! \end{body}
			
			\begin{body}“Mujer, he ahí a tu hijo... Ahí tienes a tu madre” (\textit{Jn }19, 27). La Madre pierde al hijo y, al mismo tiempo, \textit{recibe un hijo}; recibe muchos hijos e hijas. Todos y cada uno de aquellos a quienes el Hijo ha dado el poder “de ser hijos de Dios” (\textit{Jn }1, 12);\textit{ hijos en el Hijo}. El discípulo recibe a la Madre. La Iglesia recibe a la Madre. \textit{La humanidad recibe a la Madre}. Es maravillosa la riqueza \textit{con la cual nos enriquece}, quien por nosotros se ha hecho pobre. \end{body}
			
			\begin{body}4. “Tengo sed” (\textit{Jn }19, 28). Sí. Los labios secos tienen sed, el paladar y la lengua quemados por la fiebre de la agonía. El alma de Cristo tiene aún más sed. Sed infinita que lo abraza todo. Sed que, desde el principio, va hasta el fin y más allá del fin: hasta “que el Hijo le someta todas las cosas, para que Dios sea todo en todos” (cf. \textit{1 Co }15, 28). \end{body}
			
			\begin{body}5. “Dios mío, Dios mío, ¿por qué me has abandonado”? (\textit{Sal} 22, 2). El Salmo 22 comienza con estas palabras, y el Hijo del hombre que muere en la cruz, el siervo de Yahvé de la profecía de Isaías, \textit{comienza a orar} con las palabras de este \textit{salmo}. Pero, ¿hasta dónde llegan estas palabras? ¿Cómo puede Dios ser abandonado por Dios? ¿El Hijo abandonado por el Padre? Desde el punto de vista humano, esto parece imposible e inconcebible. \end{body}
			
			\begin{body}Sin embargo en Dios... \end{body}
			
			\begin{body}\textit{Cuando el Hijo es abandonado por el Padre en el Espíritu Santo} ese abandono contiene \textit{la plenitud definitiva de su amor salvador}: la plenitud de la unidad del Hijo con el Padre en el Espíritu Santo. \end{body}
			
			\begin{body}6. “¡Todo está cumplido!” (\textit{Jn }19, 30). Una vez salieron del salmo estas palabras: “He aquí que vengo, oh Dios, para hacer tu voluntad” (cf. \textit{Sal} 40, 8-9). Estas palabras pasaron por la agonía en Getsemaní: “Padre, si quieres, ¡que pase de mí esta copa! \textit{Pero no se haga mi voluntad, sino la tuya}”(\textit{Lc} 22, 42). Y ahora las mismas palabras vuelven al final, para asentarse como el sello sobre la víctima de la redención: “He aquí que vengo”... “\textit{Todo está cumplido}”. \end{body}
			
			\begin{body}7. “\textit{Padre, en tus manos encomiendo mi espíritu}” (\textit{Lc} 23, 46). Jesucristo, Hijo del hombre, acepta la muerte humana, herencia del primer Adán. Esta muerte del cuerpo es la “\textit{entrega del espíritu}” a Dios. Toda muerte humana encuentra su modelo en la muerte de Cristo: es la entrega del espíritu al que creó al hombre \textit{para la inmortalidad}. El Hijo de Dios, que como verdadero hombre entrega el espíritu, se une a través de esto \textit{con el Padre en el Espíritu Santo}, que es el amor mutuo del Padre y del Hijo: él es su aliento eterno. \end{body}
			
			\begin{body}La muerte y el amor se encuentran al final del sacrificio de la cruz. La victoria de la cruz tiene su comienzo en este hecho. El amor gana con la muerte. Al final del Viernes Santo volvemos una vez más bajo la cruz de Cristo en el Calvario. Reflexionamos sobre las palabras que Cristo pronunció desde lo alto de la cruz. \end{body}
			
			\begin{body}Al salir de este lugar, llevemoslas con nosotros como testimonio de nuestra redención. “Te adoramos a ti oh Cristo y te bendecimos, porque con tu santa cruz has redimido al mundo”. Amén.\end{body}
			
			\subsubsection{Alocución (1988): Espada que traspasa el alma}
			
			\begin{referencia}1 de abril de 1988.\end{referencia}
			
			\begin{body}1. “Estaban junto a la cruz de Jesús su madre, la hermana de su madre, María la de Cleofás y María de Magdala. Entonces Jesús, viendo a su madre y junto a ella al discípulo a quien amaba, le dijo a su madre: ¡Mujer, aquí tienes a tu hijo! Luego dijo al discípulo: ¡Ahí tienes a tu madre! Y desde ese momento el discípulo la acogió en su casa” (\textit{Jn} 19, 25-27). \end{body}
			
			\begin{body}2. “Stabat Mater...”. \begin{bodysmall}[Acabamos de recorrer el “Vía Crucis”, a lo largo del cual meditamo<a id="_idTextAnchor025"></a>s sobre el encuentro de la Madre con el Hijo, en la cuarta estación.]\end{bodysmall} El Concilio enseña: “La Santísima Virgen avanzó en la peregrinación de la fe, y mantuvo fielmente su unión con el Hijo hasta la cruz, junto a la cual, no sin designio divino, se mantuvo erguida” (\textit{Lumen Gentium}, 58). \end{body}
			
			\begin{body}3. Este plan divino fue revelado a María ya cuarenta días después del nacimiento de Jesús cuando, durante la Presentación en el templo de Jerusalén, se escucharon las palabras proféticas del viejo Simeón: “Él está aquí para la ruina y resurrección de muchos en Israel, y será signo de contradicción” (\textit{Lc} 2, 34) esto, en cuanto al Hijo. Y luego a la Madre: “Y a ti, una espada te traspasará el alma, para que se revelen los pensamientos de muchos corazones” (cf. \textit{Lc }2, 35). \end{body}
			
			\begin{body}4. Entonces, “no sin un plan divino”, María estaba bajo la cruz en el Gólgota. La espada le atravesó el corazón y le provocó un dolor indecible: el mayor sufrimiento preparado para María en este camino de fe, en el que seguía a Cristo. \end{body}
			
			\begin{body}Sufrimiento - consuelo. \end{body}
			
			\begin{body}El Concilio enseña que María correspondía al designio divino: “Sufrir profundamente con el unigénito y asociarse con su alma maternal” (\textit{Lumen gentium}, 58). La consolación une a la Madre con el Hijo, ya que solo la Madre Inmaculada pudo unirse con el Hijo de Dios en la cruz. “La espada del dolor” atravesó su alma hasta el punto de esta unión. \end{body}
			
			\begin{body}5. El Concilio enseña además: María estuvo bajo la cruz “sufriendo profundamente... consintiendo amorosamente la inmolación de la víctima que ella misma había engendrado” (\textit{Lumen Gentium}, 58). Aquí cada palabra tiene un peso particular. En la Anunciación, María había exclamado: “Que me suceda lo que has dicho” (\textit{Lc} 1, 38). Ahora renueva la misma disponibilidad en el momento del mayor dolor: “Consintiendo amorosamente” porque el que fue concebido por obra del Espíritu Santo, el “Santo de Dios”, su Hijo unigénito, sufrió como víctima el despojo en la cruz. \end{body}
			
			\begin{body}6. Una mujer entre la multitud pronunció en voz alta una bendición ante Jesús por su Madre: “¡Bendito el vientre que te dio a luz y el pecho que te amamantó!” (\textit{Lc} 11, 27). Y Jesús respondió a estas palabras de manera maravillosa: “Bienaventurados los que escuchan la palabra de Dios y la guardan” (\textit{Lc }11, 27). Ciertamente, en aquel momento, parecería que Cristo no acoge la bienaventuranza dirigida por aquella mujer a su Madre. Bajo la cruz se entiende que Cristo dirigió la bienaventuranza entonces manifestada hacia el futuro. ¿Quién es su Madre en este momento? He aquí, es ella la que está junto a la cruz, la que escucha con heroica obediencia de fe la palabra de Dios, que con todo el sufrimiento materno de su corazón “cumple”, junto con el Hijo, “la voluntad del Padre”. \end{body}
			
			\begin{body}7. Y he aquí, así, en la agonía de la cruz de Cristo, ¡tu Madre, oh Juan, te es entregada! Así hemos recibido todos, queridos hermanos y hermanas, a María como Madre. Así recibiste tú, Iglesia del Pueblo de Dios, tu “figura” y tu modelo. \end{body}
			
			\begin{body}“Stabat mater...”. \end{body}
			
			\begin{body}Desde ese momento María “colaboró con su amor maternal en el nacimiento y la educación” de todos nosotros. De hecho, el Padre Eterno ha establecido que Cristo, Hijo de María, “sea el primogénito entre muchos hermanos” (\textit{Rom} 8, 29). \end{body}
			
			\begin{body} “La maternidad de María en la economía de la gracia continúa sin cesar desde el momento del consentimiento dado en la Anunciación y mantenido sin vacilación bajo la cruz, hasta la coronación perpetua de todos los elegidos” (\textit{Lumen Gentium}, 62). \end{body}
			
			\begin{body}8. Queridos fieles (...). Que a la meditación sobre la pasión del Redentor se agregue esta palabra sobre la “espada del dolor”, que traspasó el corazón inmaculado de la Madre al pie de la cruz en el Gólgota. ¡Que a través de su sufrir-con el Hijo, “se revelen los pensamientos de muchos corazones” (cf. \textit{Lc }2, 35)! ¡Que nuestros corazones estén unidos en el misterio de la redención del mundo, y que por la Madre de Dios permanezcan en unión con Cristo en el camino de la fe, la esperanza y la caridad! \end{body}
			
			\begin{body}Amén.\end{body}
			
			\subsubsection{Alocución (1991): Amor redentor}
			
			\begin{body}\begin{referencia}29 de marzo de 1991.\end{referencia}\end{body}
			
			\begin{body} De la carta \textit{a los Hebreos}: “La sangre de Cristo, que con un Espíritu eterno se ofreció a sí mismo sin tacha a Dios, limpiará nuestra conciencia de las obras muertas, para que sirvamos al Dios vivo” (\textit{Heb }9, 14). \end{body}
			
			\begin{body}1. De la encíclica \textit{sobre el Espíritu Santo} \textit{Dominum et vivificantem}: “Las palabras de la Carta a los Hebreos ahora nos explican cómo Cristo ‘se ofreció sin tacha a Dios’, y cómo lo hizo ‘con un Espíritu eterno’. \textit{En el sacrificio del Hijo del Hombre está presente el Espíritu Santo y actúa} como actuó en su concepción, en su venida al mundo, en su vida oculta<a id="_idTextAnchor026"></a> y en su ministerio público” (\textit{Dominum et vivificantem}, 40). Las mismas palabras también muestran “cómo la \textit{humanidad, sometida al pecado} en la descendencia del primer Adán, en Jesucristo se sometió perfectamente a Dios y se unió a él y, al mismo tiempo, se llenó de misericordia hacia los hombres” (\textit{Ibid.}). El mismo Cristo en su humanidad se abrió sin límite a la acción del Paráclito que hace brotar del sufrimiento el amor que salva. \end{body}
			
			\begin{body}2. “Jesucristo, el Hijo de Dios, como hombre, en la oración ardiente de su pasión, permitió al Espíritu Santo, que ya había penetrado hasta el fondo su humanidad, transformarla en sacrificio perfecto mediante el acto de su muerte, como víctima del amor en la Cruz. \textit{Solo Él hizo esta oblación}... En su humanidad fue digno de convertirse en tal sacrificio, ya que solo él estaba “sin defecto”. \textit{Pero lo ofreció “con un Espíritu eterno”}... El Espíritu Santo actuó de manera especial en esta entrega absoluta del Hijo del Hombre, para transformar el sufrimiento en amor redentor” (\textit{Ibid}). \end{body}
			
			\begin{body}“Por analogía se puede decir que el Espíritu Santo es ‘el fuego del cielo’, \textit{que obra en lo más profundo del misterio de la Cruz}... desciende, en cierto sentido, al corazón mismo del sacrificio que se ofrece en la Cruz... \textit{consume este sacrificio con el fuego del amor}, que une al Hijo con el Padre en comunión trinitaria” (\textit{Ibíd}. 41). \end{body}
			
			\begin{body}3. Queridos hermanos y hermanas, peregrinos del Viernes Santo \begin{bodysmall}[que participan del “Vía Crucis” en el Coliseo Romano, fieles de Roma y del mundo, presentes en este lugar o unidos a nosotros gracias a la radio y la televisión.] \end{bodysmall} Aquí termina la celebración del Día de la muerte de Cristo; día en el que la redención del mundo se hace presente de manera muy particular. En la Cruz el poder del mal es derrotado y la esperanza renace en todo hombre que sufre, perseguido, cansado y descorazonado. Silencioso y abandonado, el Crucificado consume en el amor su sacrificio de salvación por nosotros. La vida brota de su sangre; en el misterio de la Pasión triunfa la misericordia del Altísimo. \end{body}
			
			\begin{body}¡Cruz de nuestra salvación, llevas colgado al Señor del mundo!\end{body}
			
			\begin{body}“\textit{Bendito sea Dios, Padre de nuestro Señor Jesucristo}... en quien tenemos la redención por su sangre, el perdón de los pecados” (\textit{Ef} 1, 3-7). Después de la Resurrección, cuando se han cumplido los días de Pascua, el Resucitado entra al Cenáculo a puerta cerrada y muestra a los Apóstoles los signos de la crucifixión; sopla sobre ellos y dice: “\textit{Recibid el Espíritu Santo; a quienes perdonéis los pecados, les serán perdonados}” (\textit{Jn} 20, 22-23). \end{body}
			
			\begin{body}“La sangre de Cristo, que con Espíritu eterno se ofreció a sí mismo sin tacha a Dios” (\textit{Hb }9, 14), purificará nuestra conciencia en la fuerza de este Espíritu hasta el fin del mundo. \end{body}
			
			\begin{body}¡Gloria a Ti, Palabra de Dios! ¡Gloria a Ti, Cristo, inmolado por nosotros! ¡Tu amor ha redimido al mundo y siempre lo salva! ¡Amén!\end{body}
			
			\subsubsection{Alocución (1994): Ne evacuetur Crux!}
			
			\begin{referencia}1 de abril de 1994.\end{referencia}
			
			\begin{body}1. Hermanos y hermanas, hoy estamos aquí para contemplar el misterio de la Cruz, que adoramos en la liturgia del Viernes Santo: “Ecce lignum crucis, venite adoremus”. Adorémosle ahora, aquí, \begin{bodysmall}[en el Coliseo. Aquí donde nuestros antepasados en la fe dieron testimonio, a través de su martirio hasta la muerte, del amor con que Cristo nos amó. Aquí, en este punto del globo, en la antigua Roma, pienso especialmente en la “Montaña de las Cruces” que se encuentra en Lituania, donde fui en una visita pastoral en septiembre pasado. Me conmovió ese otro Coliseo no de la época romana antigua, sino el Coliseo de nuestra época, del siglo pasado. \end{bodysmall}\end{body}
			
			\begin{body}\begin{bodysmall}Antes de ir a Lituania<a id="_idTextAnchor027"></a>, a los países bálticos, recé por estos dos caminos de evangelización: uno que caminaba desde Roma hacia el norte, este, oeste; el otro andando desde Constantinopla, desde la Iglesia de Oriente. Estas dos carreteras se encuentran precisamente allí, en los países bálticos, entre Lituania y Rusia.]\end{bodysmall}\end{body}
			
			\begin{body}2. \begin{bodysmall}[Hoy la sabiduría de la Tradición Oriental nos guió en nuestra meditación sobre el Vía Crucis a través de las palabras de nuestro amado Hermano Bartolomé de Constantinopla, Patriarca Ecuménico. Le agradecemos de todo corazón. Pensé en estos otros Coliseos, muy numerosos, en estas otras “Montañas de las Cruces” que están del otro lado, por la Rusia europea, por Siberia, muchas “Montañas de las Cruces”, muchos Coliseos de los nuevos tiempos.\end{bodysmall} \begin{bodysmall}Hoy quisiera decir a este hermano mío de Constantinopla, a todos estos hermanos nuestros de Oriente: queridos amigos, estamos unidos en estos mártires entre Roma, la “Montaña de las Cruces” y las Islas Soloviesky y muchos otros campos de exterminio.]\end{bodysmall}\end{body}
			
			\begin{body}Estamos unidos en el contexto de los mártires, no podemos dejar de estar unidos. No podemos dejar de decir la misma verdad sobre la Cruz y ¿por qué no podemos no decirla? Porque el mundo de hoy trata de vaciar la Cruz. \end{body}
			
			\begin{body}Esta es la tradición anticristiana que se ha extendido desde hace varios siglos y quiere vaciar la Cruz y quiere decirnos que el hombre no tiene sus raíces en la Cruz, ni siquiera tiene la perspectiva y la esperanza en la Cruz. El hombre es solo humano, debe existir como si Dios no existiera. \end{body}
			
			\begin{body}3. Queridos amigos, tenemos esta tarea común, debemos decir juntos entre Oriente y Occidente: “Ne evacuetur Crux!” Que la Cruz de Cristo no se vacíe, porque si la Cruz de Cristo se vacía, el hombre ya no tiene raíces, ya no tiene perspectivas: ¡está destruido! \end{body}
			
			\begin{body}Este es el grito de finales del siglo XX. Es el grito de Roma, el grito de Moscú, el grito de Constantinopla. Es el grito de todo el cristianismo: de América, de África, de Asia, de todos. Es el grito de la nueva evangelización. \end{body}
			
			\begin{body}Jesús nos dice: ellos me persiguieron, ellos también te perseguirán a ti; me escucharon, recibieron mi Palabra, también recibirán la tuya. Recibirán, no tienen otra solución. Nadie tiene palabras de vida eterna, solo Él, solo Jesús, solo su Cruz. \end{body}
			
			\begin{body}\begin{bodysmall}[Y así, al final de este Vía Crucis en nuestro antiguo Coliseo de Roma, pensamos en todos los demás Coliseos y los saludamos con amor, con fe, con una esperanza común.]\end{bodysmall}\end{body}
			
			\begin{body}4. Nos encomendamos, toda la Iglesia y toda la humanidad, a esta Madre que está bajo la Cruz y que nos abraza a todos como niños. En su amor nosotros, como Juan, sentimos la fuerza de esta unidad, de esta comunión, de la Iglesia y del cristianismo y damos gracias al Padre, al Hijo y al Espíritu Santo por la Cruz de Cristo. \end{body}
			
			\begin{body}¡Alabado sea Jesucristo! \end{body}
			
			\begin{body}¡Feliz Pascua!\end{body}
			
			\subsubsection{Alocución (1997): Tomar parte en su Cruz}
			
			\begin{referencia}28 de marzo de 1997.\end{referencia}
			
			\begin{body} \textit{“Cristus factus est pro nobis oboediens usque ad mortem, mortem autem crucis”} (\textit{Flp} 2, 8).\end{body}
			
			\begin{body}1. “Cristo por nosotros se sometió incluso a la muerte, y una muerte de cruz” (\textit{Flp} 2, <a id="_idTextAnchor028"></a>8-9). Estas palabras de San Pablo resumen el mensaje que el Viernes Santo nos quiere comunicar. La Iglesia no celebra este día la Eucaristía, casi como queriendo subrayar que no es posible, en el día en que se ha consumado el sacrificio \textit{cruento} de Cristo en la cruz, hacerlo presente de manera \textit{incruenta} en el Sacramento. La Liturgia eucarística se sustituye hoy por el sugestivo rito de la \textit{adoración de la Cruz}... Quienes han tomado parte en ella conservan aún viva la emoción experimentada al escuchar los textos litúrgicos sobre la Pasión del Señor.\end{body}
			
			\begin{body}¿Cómo no sentirse conmovidos por la descripción detallada que hace Isaías del “varón de dolores”, despreciado y rechazado de los hombres, que ha tomado sobre sí el peso de nuestros sufrimientos, herido de Dios por nuestros pecados? (cf. \textit{Is} 53, 3ss). Y, ¿cómo permanecer insensibles ante “el poderoso clamor y lágrimas” de Cristo, evocadas por el autor de la Carta a los Hebreos? (cf. \textit{Hb} 5, 7).\end{body}
			
			\begin{body}2. (...) siguiendo las estaciones del \textit{Vía crucis}, contemplamos las dramáticos etapas de la Pasión: Cristo que lleva la Cruz, que cae bajo su peso y agoniza en ella, que en el momento de la agonía ora con aquellas palabras: “Padre, en tus manos encomiendo mi espíritu” (\textit{Lc} 23, 46), manifestando su total y confiado abandono.\end{body}
			
			\begin{body}Hoy se concentra en la Cruz toda nuestra atención. Meditamos sobre el misterio de la Cruz, que se perpetúa a través de los siglos en el sacrificio de tantos creyentes, de tantos hombres y mujeres asociados a la muerte de Jesús con el martirio. Contemplamos el misterio de la agonía y de la muerte del Señor, que perdura también en nuestros día en el dolor y el sufrimiento de los pueblos e individuos afectados por la guerra y la violencia. Allí donde el hombre es golpeado y abatido, se ofende y crucifica a Cristo mismo. ¡Misterio de dolor, misterio de amor sin límites! \end{body}
			
			\begin{body}Quedemos en un recogimiento silencioso ante este misterio insondable.\end{body}
			
			\begin{body}3.\textit{ “Ecce Lignum crucis...”, “Mirad el árbol de la Cruz, donde estuvo clavada la salvación del mundo. ¡Venid a adorarlo!”}\end{body}
			
			\begin{body}La Cruz brilla esta tarde con extraordinario fulgor (...) Revivimos, año tras año, la pasión y la muerte de Cristo. ¡”Ecce lignum Crucis”! ¡Cuántos hermanos y hermanas en la fe participaron de la Cruz de Cristo en [los períodos de persecución de la historia de la Iglesia]!\end{body}
			
			\begin{body}(...) ¡Muchos hermanos y hermanas (...) han tomado parte en la Cruz de Cristo con el sacrificio de sus vidas! Hoy, en unión con ellos y con todos cuantos, en cualquier rincón de la tierra, en cada continente y en los diversos países del orbe, participan en la Cruz de Cristo con sus sufrimientos y con la muerte, queremos repetir: “Ecce lignum Crucis...”, “Mirad el árbol de la Cruz, donde estuvo clavada la salvación del mundo. ¡Venid a adorarlo!”\end{body}
			
			\begin{body}4. ¡Mientras se ciernen las sombras de la noche, imagen elocuente del misterio que envuelve nuestra existencia, nosotros gritamos a Ti, Cruz de nuestra salvación, nuestra fe! Señor, de tu Cruz se desprende un rayo de luz. En tu muerte ha sido vencida nuestra muerte y se nos ha ofrecido la esperanza de la resurrección. ¡Asidos a tu Cruz, quedamos en la espera confiada de tu vuelta, Señor Jesús, Redentor nuestro! \end{body}
			
			\begin{body}\textit{“Anunciamos tu muerte, proclamamos tu resurrección. ¡Ven, Señor Jesús!”.}\end{body}
			
			\begin{body}¡Amén!\end{body}
			
			\subsubsection{Alocución (2000): Muerte necesaria}
			
			\begin{referencia}21 de abril del 2000.\end{referencia}
			
			\begin{body} 1. “¿No era necesario que el Cristo padeciera eso y entrara así en su gloria?” (\textit{Lc} 24, 26). Estas palabras de Jesús a dos discípulos camino de Emaús, resuenan en nuestro espíritu esta tarde (...). También ellos, como nosotros, habían oído hablar de los aconteci<a id="_idTextAnchor029"></a>mientos concernientes a la pasión y la crucifixión de Jesús. De vuelta a su pueblo, Cristo se les acerca como un peregrino desconocido y ellos se apresuran a contarle “lo de Jesús el Nazareno, que fue un profeta poderoso en obras y palabras delante de Dios y de todo el pueblo” (\textit{Lc} 24, 19), y “cómo nuestros sumos sacerdotes y magistrados le condenaron a muerte y le crucificaron” (\textit{Lc} 24, 20). Con tristeza, terminan diciendo: “Nosotros esperábamos que sería él el que iba a librar a Israel; pero, con todas estas cosas, llevamos ya tres días desde que esto pasó” (\textit{Lc }24, 21). “Nosotros esperábamos...”. Los discípulos están desanimados y abatidos. También para nosotros es difícil entender por qué la vía de la salvación deba pasar por el sufrimiento y la muerte.\end{body}
			
			\begin{body}2. “¿No era necesario que el Cristo padeciera eso y entrara así en su gloria?” (\textit{Lc} 24, 26). Nos hacemos la misma pregunta [en este Viernes Santo] (...). Dentro de poco, dejaremos este lugar (...) y nos dispersaremos en diversas direcciones. Volveremos a nuestras casas, reflexionando sobre los mismos acontecimientos de los que hablaban los discípulos de Emaús. ¡Que Jesús se acerque a cada uno de nosotros y se haga también compañero nuestro de viaje! Mientras nos acompaña, Él nos explicará que ha subido al Calvario por nosotros, ha muerto por nosotros, cumpliendo las Escrituras. De este modo, la dolorosa escena de la crucifixión que acabamos de contemplar se convertirá para cada uno en una elocuente enseñanza.\end{body}
			
			\begin{body}Queridos Hermanos y Hermanas. El hombre contemporáneo tiene necesidad de encontrar a Jesús crucificado y resucitado. ¿Quién, si no es el divino Condenado, puede comprender plenamente la pena de quien sufre condenas injustas? ¿Quién, si no es el Rey ultrajado y humillado, puede satisfacer las expectativas de tantos hombres y mujeres sin esperanza y sin dignidad? ¿Quién, si no es el Hijo de Dios crucificado, puede entender el dolor de la soledad de tantas vidas truncadas y sin futuro?\end{body}
			
			\begin{body}El poeta francés Paul Claudel escribía que el Hijo de Dios “nos ha enseñando la vía de salida del dolor y la posibilidad de su transformación” (\textit{Positions et propositions}). Abramos el corazón a Cristo: será Él mismo quien responda a nuestra más profundas expectativas. Él mismo nos desvelará los misterios de su pasión y muerte en la cruz.\end{body}
			
			\begin{body}3. “Entonces se les abrieron los ojos y le reconocieron” (\textit{Lc} 24, 31). Con sus palabras, el corazón de los dos viandantes desconsolados adquirió serenidad y comenzó a henchirse de alegría. Reconocen a su Maestro al partir el pan. Que los hombres de hoy, como ellos, reconozcan en la Eucaristía la presencia de su Salvador. Que lo encuentren en el sacramento de su Pascua y lo acojan como compañero de su camino. Él sabrá escucharles y consolarles. Sabrá ser su guía para conducirles por los senderos de la vida hacia la casa del Padre.\end{body}
			
			\begin{body}\textit{“Adoramus te, Christe, et benedicimus tibi, quian per sanctam Crucem tuam redemisti mundum!”}\end{body}
			
			\subsubsection{Alocución (2003): Mirad el árbol de la Cruz}
			
			\begin{referencia}18 de abril de 2003.\end{referencia}
			
			\begin{body}\textit{“Ecce lignum crucis, in quo salus mundi pependit... Venite adoremus”.}\end{body}
			
			\begin{body}Hemos escuchado estas palabras en la liturgia de hoy: “Mirad el árbol de la cruz”. Son las palabras clave del Viernes santo. Ayer, en el primer día del Triduo sacro, el Jueves santo, escuchamos:\textit{ }“\textit{Hoc est corpus meum, quod pro vobis tradetur. }Esto es mi cuerpo, que será entregado por vosotros”. Hoy vemos cómo se han rea<a id="_idTextAnchor030"></a>lizado esas palabras de ayer, Jueves santo: he aquí el Gólgota, he aquí el cuerpo de Cristo en la cruz. \textit{“Ecce lignum crucis, in quo salus mundi pependit”.}\end{body}
			
			\begin{body}¡Misterio de la fe! El hombre no podía imaginar este misterio, esta realidad. Sólo Dios la podía revelar. El hombre no tiene la posibilidad de dar la vida después de la muerte. La muerte de la muerte. En el orden humano, la muerte es la última palabra. La palabra que viene después, la palabra de la Resurrección, es una palabra exclusiva de Dios y por eso celebramos con gran fervor este Triduo sacro.\end{body}
			
			\begin{body}Hoy oramos a Cristo bajado de la cruz y sepultado. Se ha sellado su sepulcro. Y mañana, en todo el mundo, en todo el cosmos, en todos nosotros, reinará un profundo silencio. Silencio de espera. \textit{“Ecce lignum crucis, in quo salus mundi pependit”.} – Este árbol de la muerte, el árbol en el que murió el Hijo de Dios, abre el camino al día siguiente: jueves, viernes, sábado, domingo. El domingo será Pascua. Y escucharemos las palabras de la liturgia. Hoy hemos escuchado: \textit{“Ecce lignum crucis, in quo salus mundi pependit”. Salus mundi! }¡En la cruz! Y pasado mañana cantaremos: \textit{“Surrexit de sepulcro... qui pro nobis pependit in ligno”. }He aquí la profundidad, la sencillez divina, de este Triduo pascual.\end{body}
			
			\begin{body}Ojalá que todos vivamos este Triduo lo más profundamente posible. Como cada año, nos encontramos aquí (...).\end{body}
			
			\begin{body}\textit{“Ecce lignum crucis, in quo salus mundi pependit”. Salus mundi!}\end{body}
			
			\begin{body}A todos vosotros, amadísimos hermanos y hermanas, os deseo que viváis este Triduo sacro –Jueves, Viernes, Sábado santo, Vigilia pascual, y luego la Pascua– cada vez con más profundidad, y también que lo testimoniéis.\end{body}
			
			\begin{body}¡Alabado sea Jesucristo!\end{body}
			
			\subsection{Benedicto XVI, papa}
			
			\subsubsection{Alocución (2006): Nuestro lugar ante la Cruz}
			
			\begin{referencia}14 de abril de 2006.\end{referencia}
			
			\begin{body}\begin{bodysmall}[Hemos acompañado a Jesús en el vía crucis. Lo hemos acompañado aquí, por el camino de los mártires, en el Coliseo, donde tantos han sufrido por Cristo, han dado la vida por el Señor; donde el Señor mismo ha sufrido de nuevo en tantos.]\end{bodysmall}\end{body}
			
			\begin{body}Así hemos comprendido que el vía crucis no es algo del pasado y de un lugar determinado de la tierra. La cruz del Señor abraza al <a id="_idTextAnchor031"></a>mundo entero; su vía crucis atraviesa los continentes y los tiempos. En el vía crucis no podemos limitarnos a ser espectadores. Estamos implicados también nosotros; por eso, debemos buscar nuestro lugar. ¿Dónde estamos nosotros? \end{body}
			
			\begin{body}En el vía crucis no se puede ser neutral. Pilatos, el intelectual escéptico, trató de ser neutral, de quedar al margen; pero, precisamente así, se puso contra la justicia, por el conformismo de su carrera. \end{body}
			
			\begin{body}Debemos buscar nuestro lugar.\end{body}
			
			\begin{body} En el espejo de la cruz hemos visto todos los sufrimientos de la humanidad de hoy. En la cruz de Cristo hoy hemos visto el sufrimiento de los niños abandonados, de los niños víctimas de abusos; las amenazas contra la familia; la división del mundo en la soberbia de los ricos que no ven a Lázaro a su puerta y la miseria de tantos que sufren hambre y sed. \end{body}
			
			\begin{body}Pero también hemos visto “estaciones” de consuelo. Hemos visto a la Madre, cuya bondad permanece fiel hasta la muerte y más allá de la muerte. Hemos visto a la mujer valiente que se acerca al Señor y no tiene miedo de manifestar solidaridad con este Varón de dolores. Hemos visto a Simón, el Cirineo, un africano, que lleva la cruz juntamente con Jesús. Y mediante estas “estaciones” de consuelo hemos visto, por último, que, del mismo modo que no acaban los sufrimientos, tampoco acaban los consuelos. \end{body}
			
			\begin{body}Hemos visto cómo san Pablo encontró en el “camino de la cruz” el celo de su fe y encendió la luz del amor. Hemos visto cómo san Agustín halló su camino. Lo mismo san Francisco de Asís, san Vicente de Paúl, san Maximiliano Kolbe, la madre Teresa de Calcuta... Del mismo modo también nosotros estamos invitados a encontrar nuestro lugar, a encontrar, como estos grandes y valientes santos, el camino con Jesús y por Jesús: el camino de la bondad, de la verdad; la valentía del amor.\end{body}
			
			\begin{body} Hemos comprendido que el vía crucis no es simplemente una colección de las cosas oscuras y tristes del mundo. Tampoco es un moralismo que, al final, resulta insuficiente. No es un grito de protesta que no cambia nada. El vía crucis es el camino de la misericordia, y de la misericordia que pone el límite al mal: eso lo hemos aprendido del Papa Juan Pablo II. Es el camino de la misericordia y, así, el camino de la salvación. De este modo estamos invitados a tomar el camino de la misericordia y a poner, juntamente con Jesús, el límite al mal. \end{body}
			
			\begin{body}Pidamos al Señor que nos ayude, que nos ayude a ser “contagiados” por su misericordia. Pidamos a la santa Madre de Jesús, la Madre de la misericordia, que también nosotros seamos hombres y mujeres de la misericordia, para contribuir así a la salvación del mundo, a la salvación de las criaturas, para ser hombres y mujeres de Dios. Amén.\end{body}
			
			\subsubsection{Alocución (2009): Un hombre que cambió la historia}
			
			\begin{referencia}Colina del Palatino. 10 de abril de 2009. \end{referencia}
			
			\begin{body} Al terminar el relato dramático de la Pasión, anota el evangelista San Marcos: “El centurión que estaba enfrente, al ver cómo había expirado, dijo: ‘Realmente este hombre era Hijo de Dios’” (\textit{Mc} 15, 39). No deja de sorprendernos la profesión de fe de este soldado romano, que había asistido al desarrollo de las diferentes fases de la crucifixión. Cuando la oscuridad de la noche estaba por caer sobre aquel Viernes único de la historia, cuando el sacrificio de la cruz <a id="_idTextAnchor032"></a>ya se había consumado y los que estaban allí se apresuraban para poder celebrar la Pascua judía a tenor de lo prescrito, las breves palabras oídas de labios de un comandante anónimo de la tropa romana resuenan en el silencio ante aquella muerte tan singular. Este oficial de la tropa romana, que había asistido a la ejecución de uno de tantos condenados a la pena capital, supo reconocer en aquel Hombre crucificado al Hijo de Dios, que expiraba en el más humillante abandono. Su fin ignominioso habría debido marcar el triunfo definitivo del odio y de la muerte sobre el amor y la vida. Pero no fue así. En el Gólgota se erguía la Cruz, de la que colgaba un hombre ya muerto, pero aquel Hombre era el “Hijo de Dios”, como confesó el centurión “al ver cómo había expirado”, en palabras del evangelista.\end{body}
			
			\begin{body}La profesión de fe de este soldado se repite cada vez que volvemos a escuchar el relato de la pasión según san Marcos. También nosotros esta noche, como él, nos detenemos a contemplar el rostro exánime del Crucificado (...). Hemos revivido el episodio trágico de un Hombre único en la historia de todos los tiempos, que ha cambiado el mundo no abatiendo a otros, sino dejando que lo mataran clavado en una cruz. Este Hombre, uno de nosotros, que mientras lo están asesinando perdona a sus verdugos, es el “Hijo de Dios” que, como nos recuerda el apóstol Pablo, “no hizo alarde de su categoría de Dios; al contrario, se despojó de su rango, y tomó la condición de esclavo… se rebajó hasta someterse incluso a la muerte, y una muerte de cruz” (\textit{Flp }2, 6-8).\end{body}
			
			\begin{body}La pasión dolorosa del Señor Jesús suscita necesariamente piedad hasta en los corazones más duros, ya que es el culmen de la revelación del amor de Dios por cada uno de nosotros. Observa san Juan: “Tanto amó Dios al mundo, que entregó a su Hijo único, para que no perezca ninguno de los que creen en Él, sino que tengan vida eterna” (\textit{Jn} 3, 16). Cristo murió en la cruz por amor. A lo largo de los milenios, muchedumbres de hombres y mujeres han quedado seducidos por este misterio y le han seguido, haciendo al mismo tiempo de su vida un don a los hermanos, como Él y gracias a su ayuda. Son los santos y los mártires, muchos de los cuales nos son desconocidos. También en nuestro tiempo, cuántas personas, en el silencio de su existencia cotidiana, unen sus padecimientos a los del Crucificado y se convierten en apóstoles de una auténtica renovación espiritual y social. ¿Qué sería del hombre sin Cristo? San Agustín señala: “Una inacabable miseria se hubiera apoderado de ti, si no se hubiera llevado a cabo esta misericordia. Nunca hubieras vuelto a la vida, si Él no hubiera venido al encuentro de tu muerte. Te hubieras derrumbado, si Él no te hubiera ayudado. Hubieras perecido, si Él no hubiera venido” (\textit{Sermón} 185, 1). Entonces, ¿por qué no acogerlo en nuestra vida?\end{body}
			
			\begin{body}Detengámonos esta noche contemplando su rostro desfigurado: es el rostro del Varón de dolores, que ha cargado sobre sí todas nuestras angustias mortales. Su rostro se refleja en el de cada persona humillada y ofendida, enferma o que sufre, sola, abandonada y despreciada. Al derramar su sangre, Él nos ha rescatado de la esclavitud de la muerte, roto la soledad de nuestras lágrimas, y entrado en todas nuestras penas y en todas nuestras inquietudes.\end{body}
			
			\begin{body}Hermanos y hermanas, mientras se yergue la Cruz sobre el Gólgota, la mirada de nuestra fe se proyecta hacia el amanecer del Día nuevo y gustamos ya el gozo y el fulgor de la Pascua. “Si hemos muerto con Cristo –escribe san Pablo–, creemos que también viviremos con Él” (\textit{Rm} 6, 8). Con esta certeza, continuamos nuestro camino. Mañana, Sábado Santo, velaremos en oración. Pero ya ahora oremos con María, la Virgen Dolorosa, oremos con todos los adolorados, \begin{bodysmall}[oremos sobre todo con los afectados por el terremoto de L’Aquila:]\end{bodysmall} oremos para que también brille para ellos en esta noche oscura la estrella de la esperanza, la luz del Señor resucitado.\end{body}
			
			\begin{body}Desde ahora, deseo a todos una feliz Pascua en la luz del Señor Resucitado.\end{body}
			
			\subsubsection{Alocución (2012): Triunfo definitivo}
			
			\begin{referencia}Palatino. 6 de abril de 2012. \end{referencia}
			
			\begin{body}Hemos recordado en la meditación, la oración y el canto, el camino de Jesús en la vía de la cruz: una vía que parecía sin salida y que, sin embargo, ha cambiado la vida y la historia del hombre, ha abierto el paso hacia los “cielos nuevos y la tierra nueva” (cf.\textit{ Ap} 21, 1). Especialmente en este día del Viernes Santo, la Iglesia celebra con íntima devoción espiritual la memoria de la muerte en cruz del Hijo de Dios y, en su cruz, ve el árbol de la vida, fecundo de una nueva esperanza. La expe<a id="_idTextAnchor033"></a>riencia del sufrimiento y de la cruz marca la humanidad, marca incluso la familia; cuántas veces el camino se hace fatigoso y difícil. Incomprensiones, divisiones, preocupaciones por el futuro de los hijos, enfermedades, dificultades de diverso tipo. En nuestro tiempo, además, la situación de muchas familias se ve agravada \begin{bodysmall}[por la precariedad del trabajo y por otros efectos negativos de la crisis económica. El camino del \end{bodysmall}\textit{Via Crucis}\begin{bodysmall}, que hemos recorrido esta noche espiritualmente,]\end{bodysmall} es una invitación para todos nosotros, y especialmente para las familias, a contemplar a Cristo crucificado para tener la fuerza de ir más allá de las dificultades. La cruz de Jesús es el signo supremo del amor de Dios para cada hombre, la respuesta sobreabundante a la necesidad que tiene toda persona de ser amada. Cuando nos encontramos en la prueba, cuando nuestras familias deben afrontar el dolor, la tribulación, miremos a la cruz de Cristo: allí encontramos el valor y la fuerza para seguir caminando; allí podemos repetir con firme esperanza las palabras de san Pablo: “¿Quién nos separará del amor de Cristo?: ¿la tribulación?, ¿la angustia?, ¿la persecución?, ¿el hambre?, ¿la desnudez?, ¿el peligro?, ¿la espada?... Pero en todo esto vencemos de sobra gracias a aquel que nos ha amado” (\textit{Rm} 8, 35. 37).\end{body}
			
			\begin{body}En la aflicción y la dificultad, no estamos solos; la familia no está sola: Jesús está presente con su amor, la sostiene con su gracia y le da la fuerza para seguir adelante, para afrontar los sacrificios y superar todo obstáculo. Y es a este amor de Cristo al que debemos acudir cuando las vicisitudes humanas y las dificultades amenazan con herir la unidad de nuestra vida y de la familia. El misterio de la pasión, muerte y resurrección de Cristo alienta a seguir adelante con esperanza: la estación del dolor y de la prueba, si la vivimos con Cristo, con fe en él, encierra ya la luz de la resurrección, la vida nueva del mundo resucitado, la pascua de cada hombre que cree en su Palabra. En aquel hombre crucificado, que es el Hijo de Dios, incluso la muerte misma adquiere un nuevo significado y orientación, es rescatada y vencida, es el paso hacia la nueva vida: “si el grano de trigo no cae en tierra y muere, queda infecundo; pero si muere, da mucho fruto” (\textit{Jn }12, 24). \end{body}
			
			\begin{body}Encomendémonos a la Madre de Cristo. A ella, que ha acompañado a su Hijo por la vía dolorosa. Que ella, que estaba junto a la cruz en la hora de su muerte, que ha alentado a la Iglesia desde su nacimiento para que viva la presencia del Señor, dirija nuestros corazones, los corazones de todas las familias a través del inmenso \textit{mysterium passionis }hacia el \textit{mysterium paschale, }hacia aquella luz que prorrumpe de la Resurrección de Cristo y muestra el triunfo definitivo del amor, de la alegría, de la vida, sobre el mal, el sufrimiento, la muerte. Amén.\end{body}
			
			\subsection{Francisco, papa}
			
			\subsubsection{Alocución (2015): Infinita misericordia}
			
			\begin{referencia}Palatino. 3 de abril de 2015.\end{referencia}
			
			\begin{body}Oh Cristo crucificado y victorioso, tu \textit{Vía Crucis} es la síntesis de tu vida; es el icono de tu obediencia a la voluntad del Padre; es la realización de tu infinito amor por nosotros pecadores; es la prueba de tu misión; es la realización definitiva de la revelación y la historia de la salvación. El peso de tu cruz nos libera de todas nuestras cargas. \end{body}
			
			\begin{body}En tu obediencia a la voluntad del Padre, caemos en la cuenta de nuestra rebelión y desobediencia. En ti vendido, traicionado y crucificado por tu gente y por tus seres queridos, vemos nuestras traiciones cotidianas y nuestras usuales infidelidades. En tu inocencia, Cordero inmaculado, vemos nuest<a id="_idTextAnchor034"></a>ra culpa. En tu rostro azotado, escupido y desfigurado, vemos toda la brutalidad de nuestros pecados. En la crueldad de tu Pasión, vemos la crueldad de nuestro corazón y de nuestras acciones. En tu sentirte “abandonado”, vemos a todos los abandonados por los familiares, la sociedad, la atención y la solidaridad. En tu cuerpo destrozado, desgarrado y lacerado, vemos los cuerpos de nuestros hermanos abandonados a lo largo de las calles, desfigurados por nuestra negligencia y nuestra indiferencia. En tu sed, Señor, vemos la sed de Tu Padre misericordioso que en Ti quiso abrazar, perdonar y salvar a toda la humanidad. En Ti, divino amor, vemos también hoy a nuestros hermanos perseguidos, decapitados y crucificados por su fe en Ti, ante nuestros ojos o a menudo con nuestro silencio cómplice. \end{body}
			
			\begin{body}Imprime en nuestro corazón, Señor, sentimientos de fe, esperanza, caridad, de dolor por nuestros pecados y condúcenos a arrepentirnos de nuestros pecados que te han crucificado. Llévanos a transformar nuestra conversión hecha de palabras, en conversión de vida y de obras. Llévanos a custodiar en nosotros un recuerdo vivo de tu Rostro desfigurado, para no olvidar nunca el gran precio que has pagado para liberarnos. Jesús crucificado, refuerza en nosotros la fe para que no decaiga ante las tentaciones; reaviva en nosotros la esperanza, que no pierda el camino siguiendo las seducciones del mundo; custodia en nosotros la caridad para que no se deje engañar por la corrupción y la mundanidad. Enséñanos que la Cruz es el camino hacia la Resurrección. Enséñanos que el Viernes santo es camino hacia la Pascua de la luz; enséñanos que Dios nunca olvida a ninguno de sus hijos y nunca se cansa de perdonarnos y abrazarnos con su infinita misericordia. Pero enséñanos también a no cansarnos nunca de pedir perdón y creer en la misericordia sin límites del Padre. \end{body}
			
			\begin{bodyprose}\textit{Alma de Cristo, santifícanos.}\end{bodyprose}
			
			\begin{bodyprose}\textit{Cuerpo de Cristo, sálvanos.}\end{bodyprose}
			
			\begin{bodyprose}\textit{Sangre de Cristo, embriáganos.}\end{bodyprose}
			
			\begin{bodyprose}\textit{Agua del costado de Cristo, lávanos.}\end{bodyprose}
			
			\begin{bodyprose}\textit{Pasión de Cristo, confórtanos.}\end{bodyprose}
			
			\begin{bodyprose}\textit{Oh buen Jesús, óyenos.}\end{bodyprose}
			
			\begin{bodyprose}\textit{Dentro de tus llagas, escóndenos.}\end{bodyprose}
			
			\begin{bodyprose}\textit{No permitas que nos separemos de ti.}\end{bodyprose}
			
			\begin{bodyprose}\textit{Del maligno enemigo defiéndenos.}\end{bodyprose}
			
			\begin{bodyprose}\textit{En la hora de nuestra muerte llámanos.}\end{bodyprose}
			
			\begin{bodyprose}\textit{Y manda que vengamos a Ti }\end{bodyprose}
			
			\begin{bodyprose}\textit{para que te alabemos con tus santos,}\end{bodyprose}
			
			\begin{bodyprose}\textit{por los siglos de los siglos. Amén.}\end{bodyprose}
			
			\subsubsection{Oración (2018): Vergüenza, arrepentimiento, esperanza}
			
			\begin{referencia}Palatino. 30 de marzo de 2018.\end{referencia}
			
			\begin{body}Señor Jesús, nuestra mirada está dirigida a ti, llena de vergüenza, de arrepentimiento y de esperanza. Que ante tu amor supremo nos impregne la vergüenza para haberte dejado solo y sufrir por nuestros pecados: \end{body}
			
			\begin{body}la vergüenza por haber escapado ante la prueba incluso habiendo dicho miles de veces: “aunque todos te dejen, yo no te dejaré”; \end{body}
			
			\begin{body}la vergüenza de haber elegido a Barrabás y no a ti, el poder y no a ti, la apariencia y no a ti, el dios dinero y no a ti, la mundanidad y no la eternidad; \end{body}
			
			\begin{body}la vergüenza por haberte tentando con la boca y con el corazón, cada vez que nos hemos encontrado ante una prueba, diciéndote: “¡si tú eres el mesías, sálvate y nosotro<a id="_idTextAnchor035"></a>s creeremos!”;\end{body}
			
			\begin{body}la vergüenza porque muchas personas, e incluso algunos ministros tuyos, se han dejado engañar por la ambición y la vana gloria perdiendo su dignidad y su primer amor; \end{body}
			
			\begin{body}la vergüenza porque nuestras generaciones están dejando a los jóvenes un mundo fracturado por las divisiones y las guerras; un mundo devorado por el egoísmo donde los jóvenes, los pequeños, los enfermos, los ancianos son marginados; \end{body}
			
			\begin{body}la vergüenza de haber perdido la vergüenza;\end{body}
			
			\begin{body}Señor Jesús, ¡danos siempre la gracia de la santa vergüenza!\end{body}
			
			\begin{body}Nuestra mirada está llena también de un arrepentimiento que ante su silencio elocuente suplica tu misericordia:\end{body}
			
			\begin{body}el arrepentimiento que brota de la certeza de que solo tú puedes salvarnos del mal, solo tú puedes sanarnos de nuestra lepra de odio, de egoísmo, de soberbia, de codicia, de venganza, de avaricia, de idolatría, solo tú puedes abrazarnos y darnos de nuevo la dignidad filial y alegrarnos por nuestra vuelta a casa, a la vida; \end{body}
			
			\begin{body}el arrepentimiento que florece del sentir nuestra pequeñez, nuestra nada, nuestra vanidad y que se deja acariciar por tu invitación suave y poderosa a la conversión; \end{body}
			
			\begin{body}el arrepentimiento de David que del abismo de su miseria reencuentra en ti su única fuerza; \end{body}
			
			\begin{body}el arrepentimiento que nace de nuestra vergüenza, que nace de la certeza de que nuestro corazón permanecerá siempre inquieto hasta que no te encuentre a ti y en ti su única fuente de plenitud y de quietud; \end{body}
			
			\begin{body}el arrepentimiento de Pedro que encontrando tu mirada lloró amargamente por haberte negado delante de los hombres. \end{body}
			
			\begin{body}Señor Jesús, ¡danos siempre la gracia del santo arrepentimiento! \end{body}
			
			\begin{body}Delante de tu suprema majestad se enciende, en la tenebrosidad de nuestra desesperación, la chispa de la esperanza porque sabemos que tu única medida para amarnos es la de amarnos sin medida; \end{body}
			
			\begin{body}la esperanza porque tu mensaje continúa inspirando, todavía hoy, a muchas personas y pueblos a los que solo el bien puede derrotar el mal y la maldad, solo el perdón puede derrumbar el rencor y la venganza, solo el abrazo fraterno puede dispersar la hostilidad y el miedo al otro; \end{body}
			
			\begin{body}la esperanza porque tu sacrificio continúe, todavía hoy, emanando el perfume del amor divino que acaricia los corazones de tantos jóvenes que continúan consagrándote sus vidas convirtiéndose en ejemplos vivos de caridad y de gratuidad en este nuestro mundo devorado por la lógica del beneficio y de la fácil ganancia; \end{body}
			
			\begin{body}la esperanza porque tantos misioneros y misioneras continúan, todavía hoy, desafiando la adormecida conciencia de la humanidad arriesgando la vida para servirte en los pobres, en los descartados, en los inmigrantes, en los invisibles, en los explotados, en los hambrientos y en los presos; \end{body}
			
			\begin{body}la esperanza porque tu Iglesia, santa y hecha de pecadores, continúa, todavía hoy, no obstante todos los intentos de desacreditarla, siendo una luz que ilumina, alienta, levanta y testimonia tu amor ilimitado por la humanidad, un modelo de altruismo, un arca de salvación y una fuente de certeza y de verdad; \end{body}
			
			\begin{body}la esperanza porque de tu cruz, fruto de la avaricia y la cobardía de tantos doctores de la Ley e hipócritas, ha surgido la Resurrección transformando las tinieblas de la tumba en el brillo del alba del Domingo sin puesta de sol, enseñándonos que tu amor es nuestra esperanza. \end{body}
			
			\begin{body}¡Señor Jesús, danos siempre la gracia de la santa esperanza!\end{body}
			
			\begin{body}Ayúdanos, Hijo del hombre, a despojarnos de la arrogancia del ladrón puesto a tu izquierda y de los miopes y de los corruptos, que han visto en ti una oportunidad para explotar, un condenado para criticar, un derrotado del que burlarse, otra ocasión para cargar sobre otros, e incluso sobre Dios, las propias culpas. \end{body}
			
			\begin{body}Sin embargo te pido, Hijo de Dios, que nos identifiquemos con el ladrón bueno que te ha mirado con ojos llenos de vergüenza, de arrepentimiento y de esperanza; que, con los ojos de la fe, ha visto en tu aparente derrota la divina victoria y así se ha arrodillado delante de tu misericordia y con honestidad ha robado el paraíso! ¡Amén!\end{body}
			
			\section{Temas}
			
			\begin{ccetheme}La Pasión de Cristo \end{ccetheme}
			
			\begin{ccereference}\end{ccereference}CEC 602-618, 1992: </p>
			
			\begin{ccebody}\textbf{“Dios le hizo pecado por nosotros”}\end{ccebody}
			
			\begin{ccebody}\begin{ccenumber}602\end{ccenumber} [...] san Pedro pudo formular así la fe apostólica en el designio divino de salvación: “Habéis sido rescatados de la conducta necia heredada de vuestros padres, no con algo caduco, oro o plata, sino con una sangre preciosa, como de cordero sin tacha y sin mancilla, Cristo, predestinado antes de la creación del mundo y manifestado en los últimos tiempos a causa de vosotros” (\textit{1 P} 1, 18-20). Los pecados de los hombres, consecuencia del pecado original, están sancionados con la muerte (cf. \textit{Rm} 5, 12; \textit{1 Co} 15, 56). Al enviar a su propio Hijo en la condición de esclavo (cf. \textit{Flp} 2, 7), la de una humanidad caída y destinada a la muerte a causa del pecado (cf. \textit{Rm} 8, 3), “a quien no conoció pecado, Dios le hizo pecado por nosotros, para que viniésemos a ser justicia de Dios en él” (\textit{2 Co} 5, 21).\end{ccebody}
			
			\begin{ccebody}\begin{ccenumber}603\end{ccenumber} Jesús no conoció la reprobación como si él mismo hubiese pecado (cf. \textit{Jn} 8, 46). Pero, en el amor redentor que le unía siempre al Padre (cf. \textit{Jn} 8, 29), nos asumió desde el alejamiento con relación a Dios por nuestro pecado hasta el punto de poder decir en nuestro nombre en la cruz: “Dios mío, Dios mío, ¿por qué me has abandonado?” (\textit{Mc} 15, 34; \textit{Sal} 22,2). Al haberle hecho así solidario con nosotros, pecadores, “Dios no perdonó ni a su propio Hijo, antes bien le entregó por todos nosotros” (\textit{Rm} 8, 32) para que fuéramos “reconciliados con Dios por la muerte de su Hijo” (\textit{Rm} 5, 10).\end{ccebody}
			
			\begin{ccebody}\textbf{Dios tiene la iniciativa del amor redentor universal}\end{ccebody}
			
			\begin{ccebody}\begin{ccenumber}604\end{ccenumber} Al entregar a su Hijo por nuestros pecados, Dios manifiesta que su designio sobre nosotros es un designio de amor benevolente que precede a todo mérito por nuestra parte: “En esto consiste el amor: no en que nosotros hayamos amado a Dios, sino en que él nos amó y nos envió a su Hijo como propiciación por nuestros pecados” (\textit{1 Jn} 4, 10; cf. \textit{Jn} 4, 19). “La prueba de que Dios nos ama es que Cristo, siendo nosotros todavía pecadores, murió por nosotros” (\textit{Rm} 5, 8).\end{ccebody}
			
			\begin{ccebody}\begin{ccenumber}605\end{ccenumber} Jesús ha recordado al final de la parábola de la oveja perdida que este amor es sin excepción: “De la misma manera, no es voluntad de vuestro Padre celestial que se pierda uno de estos pequeños” (\textit{Mt} 18, 14). Afirma “dar su vida en rescate \textit{por muchos}” (\textit{Mt} 20, 28); este último término no es restrictivo: opone el conjunto de la humanidad a la única persona del Redentor que se entrega para salvarla (cf. \textit{Rm} 5, 18-19). La Iglesia, siguiendo a los Apóstoles (cf. \textit{2 Co} 5, 15; \textit{1 Jn} 2, 2), enseña que Cristo ha muerto por todos los hombres sin excepción: “no hay, ni hubo ni habrá hombre alguno por quien no haya padecido Cristo” (Concilio de Quiercy, año 853: DS, 624).\end{ccebody}
			
			\begin{ccebody}\textbf{Cristo se ofreció a su Padre por nuestros pecados}\end{ccebody}
			
			\begin{ccebody}\textbf{Toda la vida de Cristo es oblación al Padre}\end{ccebody}
			
			\begin{ccebody}\begin{ccenumber}606\end{ccenumber} El Hijo de Dios “bajado del cielo no para hacer su voluntad sino la del Padre que le ha enviado” (\textit{Jn} 6, 38), “al entrar en este mundo, dice: [...] He aquí que vengo [...] para hacer, oh Dios, tu voluntad [...] En virtud de esta voluntad somos santificados, merced a la oblación de una vez para siempre del cuerpo de Jesucristo” (\textit{Hb} 10, 5-10). Desde el primer instante de su Encarnación el Hijo acepta el designio divino de salvación en su misión redentora: “Mi alimento es hacer la voluntad del que me ha enviado y llevar a cabo su obra” (\textit{Jn} 4, 34). El sacrificio de Jesús “por los pecados del mundo entero” (\textit{1 Jn} 2, 2), es la expresión de su comunión de amor con el Padre: “El Padre me ama porque doy mi vida” (\textit{Jn} 10, 17). “El mundo ha de saber que amo al Padre y que obro según el Padre me ha ordenado” (\textit{Jn} 14, 31).\end{ccebody}
			
			\begin{ccebody}\begin{ccenumber}607\end{ccenumber} Este deseo de aceptar el designio de amor redentor de su Padre anima toda la vida de Jesús (cf. \textit{Lc} 12, 50; 22, 15; \textit{Mt} 16, 21-23) porque su Pasión redentora es la razón de ser de su Encarnación: “¡Padre líbrame de esta hora! Pero ¡si he llegado a esta hora para esto!” (\textit{Jn} 12, 27). “El cáliz que me ha dado el Padre ¿no lo voy a beber?” (\textit{Jn} 18, 11). Y todavía en la cruz antes de que “todo esté cumplido” (\textit{Jn} 19, 30), dice: “Tengo sed” (\textit{Jn} 19, 28).\end{ccebody}
			
			\begin{ccebody}\textbf{“El cordero que quita el pecado del mundo”}\end{ccebody}
			
			\begin{ccebody}\begin{ccenumber}608\end{ccenumber} Juan Bautista, después de haber aceptado bautizarle en compañía de los pecadores (cf. \textit{Lc} 3, 21; \textit{Mt} 3, 14-15), vio y señaló a Jesús como el “Cordero de Dios que quita los pecados del mundo” (\textit{Jn} 1, 29; cf. \textit{Jn} 1, 36). Manifestó así que Jesús es a la vez el Siervo doliente que se deja llevar en silencio al matadero (\textit{Is} 53, 7; cf. \textit{Jr} 11, 19) y carga con el pecado de las multitudes (cf. \textit{Is} 53, 12) y el cordero pascual símbolo de la redención de Israel cuando celebró la primera Pascua (\textit{Ex} 12, 3-14; cf. \textit{Jn} 19, 36; \textit{1 Co} 5, 7). Toda la vida de Cristo expresa su misión: “Servir y dar su vida en rescate por muchos” (\textit{Mc} 10, 45).\end{ccebody}
			
			\begin{ccebody}\textbf{Jesús acepta libremente el amor redentor del Padre}\end{ccebody}
			
			\begin{ccebody}\begin{ccenumber}609\end{ccenumber} Jesús, al aceptar en su corazón humano el amor del Padre hacia los hombres, “los amó hasta el extremo” (\textit{Jn} 13, 1) porque “nadie tiene mayor amor que el que da su vida por sus amigos” (\textit{Jn} 15, 13). Tanto en el sufrimiento como en la muerte, su humanidad se hizo el instrumento libre y perfecto de su amor divino que quiere la salvación de los hombres (cf. \textit{Hb }2, 10. 17-18; 4, 15; 5, 7-9). En efecto, aceptó libremente su pasión y su muerte por amor a su Padre y a los hombres que el Padre quiere salvar: “Nadie me quita [la vida]; yo la doy voluntariamente” (\textit{Jn} 10, 18). De aquí la soberana libertad del Hijo de Dios cuando Él mismo se encamina hacia la muerte (cf. \textit{Jn} 18, 4-6; \textit{Mt} 26, 53).\end{ccebody}
			
			\begin{ccebody}\textbf{Jesús anticipó en la cena la ofrenda libre de su vida}\end{ccebody}
			
			\begin{ccebody}\begin{ccenumber}610\end{ccenumber} Jesús expresó de forma suprema la ofrenda libre de sí mismo en la cena tomada con los doce Apóstoles (cf. \textit{Mt} 26, 20), en “la noche en que fue entregado” (\textit{1 Co} 11, 23). En la víspera de su Pasión, estando todavía libre, Jesús hizo de esta última Cena con sus Apóstoles el memorial de su ofrenda voluntaria al Padre (cf. \textit{1 Co} 5, 7), por la salvación de los hombres: “Este es mi Cuerpo que va a \textit{ser entregado} por vosotros” (\textit{Lc} 22, 19). “Esta es mi sangre de la Alianza que va a \textit{ser derramada} por muchos para remisión de los pecados” (\textit{Mt} 26, 28).\end{ccebody}
			
			\begin{ccebody}\begin{ccenumber}611\end{ccenumber} La Eucaristía que instituyó en este momento será el “memorial” (\textit{1 Co} 11, 25) de su sacrificio. Jesús incluye a los Apóstoles en su propia ofrenda y les manda perpetuarla (cf. \textit{Lc} 22, 19). Así Jesús instituye a sus apóstoles sacerdotes de la Nueva Alianza: “Por ellos me consagro a mí mismo para que ellos sean también consagrados en la verdad” (\textit{Jn} 17, 19; cf. Concilio de Trento: DS, 1752; 1764).\end{ccebody}
			
			\begin{ccebody}\textbf{La agonía de Getsemaní}\end{ccebody}
			
			\begin{ccebody}\begin{ccenumber}612\end{ccenumber} El cáliz de la Nueva Alianza que Jesús anticipó en la Cena al ofrecerse a sí mismo (cf. \textit{Lc} 22, 20), lo acepta a continuación de manos del Padre en su agonía de Getsemaní (cf. \textit{Mt} 26, 42) haciéndose “obediente hasta la muerte” (\textit{Flp} 2, 8; cf. \textit{Hb} 5, 7-8). Jesús ora: “Padre mío, si es posible, que pase de mí este cáliz...” (\textit{Mt} 26, 39). Expresa así el horror que representa la muerte para su naturaleza humana. Esta, en efecto, como la nuestra, está destinada a la vida eterna; además, a diferencia de la nuestra, está perfectamente exenta de pecado (cf. \textit{Hb} 4, 15) que es la causa de la muerte (cf. \textit{Rm} 5, 12); pero sobre todo está asumida por la persona divina del “Príncipe de la Vida” (\textit{Hch} 3, 15), de “el que vive”, \textit{Viventis assumpta} (\textit{Ap} 1, 18; cf. \textit{Jn} 1, 4; 5, 26). Al aceptar en su voluntad humana que se haga la voluntad del Padre (cf. \textit{Mt} 26, 42), acepta su muerte como redentora para “llevar nuestras faltas en su cuerpo sobre el madero” (\textit{1 P} 2, 24).\end{ccebody}
			
			\begin{ccebody}\textbf{La muerte de Cristo es el sacrificio único y definitivo}\end{ccebody}
			
			\begin{ccebody}\begin{ccenumber}613\end{ccenumber} La muerte de Cristo es a la vez el \textit{sacrificio} pascual que lleva a cabo la redención definitiva de los hombres (cf. \textit{1 Co} 5, 7; \textit{Jn} 8, 34-36) por medio del “Cordero que quita el pecado del mundo” (\textit{Jn} 1, 29; cf. \textit{1 P} 1, 19) y el \textit{sacrificio de la Nueva Alianza} (cf. \textit{1 Co} 11, 25) que devuelve al hombre a la comunión con Dios (cf. \textit{Ex} 24, 8) reconciliándole con Él por “la sangre derramada por muchos para remisión de los pecados” (\textit{Mt} 26, 28; cf. \textit{Lv} 16, 15-16).\end{ccebody}
			
			\begin{ccebody}\begin{ccenumber}614\end{ccenumber} Este sacrificio de Cristo es único, da plenitud y sobrepasa a todos los sacrificios (cf. \textit{Hb} 10, 10). Ante todo es un don del mismo Dios Padre: es el Padre quien entrega al Hijo para reconciliarnos consigo (cf. \textit{1 Jn} 4, 10). Al mismo tiempo es ofrenda del Hijo de Dios hecho hombre que, libremente y por amor (cf. \textit{Jn} 15, 13), ofrece su vida (cf. \textit{Jn} 10, 17-18) a su Padre por medio del Espíritu Santo (cf. \textit{Hb} 9, 14), para reparar nuestra desobediencia.\end{ccebody}
			
			\begin{ccebody}\textbf{Jesús reemplaza nuestra desobediencia por su obediencia}\end{ccebody}
			
			\begin{ccebody}\begin{ccenumber}615\end{ccenumber} “Como [...] por la desobediencia de un solo hombre, todos fueron constituidos pecadores, así también por la obediencia de uno solo todos serán constituidos justos” (\textit{Rm} 5, 19). Por su obediencia hasta la muerte, Jesús llevó a cabo la sustitución del Siervo doliente que “se dio a sí mismo en \textit{expiación}”, “cuando llevó el pecado de muchos”, a quienes “justificará y cuyas culpas soportará” (\textit{Is} 53, 10-12). Jesús repara por nuestras faltas y satisface al Padre por nuestros pecados (cf. Concilio de Trento: DS, 1529).\end{ccebody}
			
			\begin{ccebody}\textbf{En la cruz, Jesús consuma su sacrificio}\end{ccebody}
			
			\begin{ccebody}\begin{ccenumber}616\end{ccenumber} El “amor hasta el extremo” (\textit{Jn} 13, 1) es el que confiere su valor de redención y de reparación, de expiación y de satisfacción al sacrificio de Cristo. Nos ha conocido y amado a todos en la ofrenda de su vida (cf. \textit{Ga} 2, 20; \textit{Ef} 5, 2. 25). “El amor [...] de Cristo nos apremia al pensar que, si uno murió por todos, todos por tanto murieron” (\textit{2 Co} 5, 14). Ningún hombre aunque fuese el más santo estaba en condiciones de tomar sobre sí los pecados de todos los hombres y ofrecerse en sacrificio por todos. La existencia en Cristo de la persona divina del Hijo, que al mismo tiempo sobrepasa y abraza a todas las personas humanas, y que le constituye Cabeza de toda la humanidad, hace posible su sacrificio redentor por todos.\end{ccebody}
			
			\begin{ccebody}\begin{ccenumber}617\end{ccenumber} \textit{Sua sanctissima passione in ligno crucis nobis justificationem meruit} – “Por su sacratísima pasión en el madero de la cruz nos mereció la justificación”, enseña el Concilio de Trento (DS, 1529) subrayando el carácter único del sacrificio de Cristo como “causa de salvación eterna” (\textit{Hb} 5, 9). Y la Iglesia venera la Cruz cantando: \textit{O crux, ave, spes unica} – “Salve, oh cruz, única esperanza” (Añadidura litúrgica al himno “Vexilla Regis”: \textit{Liturgia de las Horas}).\end{ccebody}
			
			\begin{ccebody}\textbf{Nuestra participación en el sacrificio de Cristo}\end{ccebody}
			
			\begin{ccebody}\begin{ccenumber}618\end{ccenumber} La Cruz es el único sacrificio de Cristo “único mediador entre Dios y los hombres” (\textit{1 Tm} 2, 5). Pero, porque en su Persona divina encarnada, “se ha unido en cierto modo con todo hombre” (GS 22, 2) Él “ofrece a todos la posibilidad de que, en la forma de Dios sólo conocida [...] se asocien a este misterio pascual” (GS 22, 5). Él llama a sus discípulos a “tomar su cruz y a seguirle” (\textit{Mt} 16, 24) porque Él “sufrió por nosotros dejándonos ejemplo para que sigamos sus huellas” (\textit{1 P} 2, 21). Él quiere, en efecto, asociar a su sacrificio redentor a aquellos mismos que son sus primeros beneficiarios (cf. \textit{Mc} 10, 39; \textit{Jn} 21, 18-19; \textit{Col} 1, 24). Eso lo realiza en forma excelsa en su Madre, asociada más íntimamente que nadie al misterio de su sufrimiento redentor (cf. \textit{Lc} 2, 35):\end{ccebody}
			
			\begin{ccecite}“Esta es la única verdadera escala del paraíso, fuera de la Cruz no hay otra por donde subir al cielo” (Santa Rosa de Lima, cf. P. Hansen, \textit{Vita mirabilis}, Lovaina, 1668)\end{ccecite}
			
			\begin{ccebody}\begin{ccenumber}1992\end{ccenumber} La justificación nos fue \textit{merecida por la pasión de Cristo}, que se ofreció en la cruz como hostia viva, santa y agradable a Dios y cuya sangre vino a ser instrumento de propiciación por los pecados de todos los hombres. La justificación es concedida por el Bautismo, sacramento de la fe. Nos asemeja a la justicia de Dios que nos hace interiormente justos por el poder de su misericordia. Tiene por fin la gloria de Dios y de Cristo, y el don de la vida eterna (cf. Concilio de Trento: DS 1529)\end{ccebody}
			
			\begin{ccecite}“Pero ahora, independientemente de la ley, la justicia de Dios se ha manifestado, atestiguada por la ley y los profetas, justicia de Dios por la fe en Jesucristo, para todos los que creen –pues no hay diferencia alguna; todos pecaron y están privados de la gloria de Dios– y son justificados por el don de su gracia, en virtud de la redención realizada en Cristo Jesús, a quien Dios exhibió como instrumento de propiciación por su propia sangre, mediante la fe, para mostrar su justicia, pasando por alto los pecados cometidos anteriormente, en el tiempo de la paciencia de Dios; en orden a mostrar su justicia en el tiempo presente, para ser él justo y justificador del que cree en Jesús” (\textit{Rm} 3, 21-26).\end{ccecite}
			
			\begin{ccetheme}La oración de Jesús \end{ccetheme}
			
			\begin{ccereference}\end{ccereference}CEC 612, 2606, 2741: </p>
			
			\begin{ccebody}\textbf{La agonía de Getsemaní}\end{ccebody}
			
			\begin{ccebody}\begin{ccenumber}612\end{ccenumber} El cáliz de la Nueva Alianza que Jesús anticipó en la Cena al ofrecerse a sí mismo (cf. \textit{Lc} 22, 20), lo acepta a continuación de manos del Padre en su agonía de Getsemaní (cf. \textit{Mt} 26, 42) haciéndose “obediente hasta la muerte” (\textit{Flp} 2, 8; cf. \textit{Hb} 5, 7-8). Jesús ora: “Padre mío, si es posible, que pase de mí este cáliz...” (\textit{Mt} 26, 39). Expresa así el horror que representa la muerte para su naturaleza humana. Esta, en efecto, como la nuestra, está destinada a la vida eterna; además, a diferencia de la nuestra, está perfectamente exenta de pecado (cf. \textit{Hb} 4, 15) que es la causa de la muerte (cf. \textit{Rm} 5, 12); pero sobre todo está asumida por la persona divina del “Príncipe de la Vida” (\textit{Hch} 3, 15), de “el que vive”, \textit{Viventis assumpta} (\textit{Ap} 1, 18; cf. \textit{Jn} 1, 4; 5, 26). Al aceptar en su voluntad humana que se haga la voluntad del Padre (cf. \textit{Mt} 26, 42), acepta su muerte como redentora para “llevar nuestras faltas en su cuerpo sobre el madero” (\textit{1 P} 2, 24).\end{ccebody}
			
			\begin{ccebody}\begin{ccenumber}2606\end{ccenumber} Todos las angustias de la humanidad de todos los tiempos, esclava del pecado y de la muerte, todas las súplicas y las intercesiones de la historia de la salvación están recogidas en este grito del Verbo encarnado. He aquí que el Padre las acoge y, por encima de toda esperanza, las escucha al resucitar a su Hijo. Así se realiza y se consuma el drama de la oración en la Economía de la creación y de la salvación. El Salterio nos da la clave para la comprensión de este drama por medio de Cristo. Es en el “hoy” de la Resurrección cuando dice el Padre: “Tú eres mi Hijo; yo te he engendrado hoy. \textit{Pídeme}, y te \textit{daré} en herencia las naciones, en propiedad los confines de la tierra” (\textit{Sal} 2, 7-8; cf. \textit{Hch} 13, 33).\end{ccebody}
			
			\begin{ccebody}La carta a los Hebreos expresa en términos dramáticos cómo actúa la plegaria de Jesús en la victoria de la salvación: “El cual, habiendo ofrecido en los días de su vida mortal ruegos y súplicas con poderoso clamor y lágrimas al que podía salvarle de la muerte, fue escuchado por su actitud reverente, y aun siendo Hijo, con lo que padeció experimentó la obediencia; y llegado a la perfección, se convirtió en causa de salvación eterna para todos los que le obedecen” (Hb 5, 7-9).\end{ccebody}
			
			\begin{ccebody}\begin{ccenumber}2741\end{ccenumber} Jesús ora también por nosotros, en nuestro lugar y en favor nuestro. Todas nuestras peticiones han sido recogidas una vez por todas en sus palabras en la Cruz; y escuchadas por su Padre en la Resurrección: por eso no deja de interceder por nosotros ante el Padre (cf. \textit{Hb} 5, 7; 7, 25; 9, 24). Si nuestra oración está resueltamente unida a la de Jesús, en la confianza y la audacia filial, obtenemos todo lo que pidamos en su Nombre, y aún más de lo que pedimos: recibimos al Espíritu Santo, que contiene todos los dones.\end{ccebody}
			
			\begin{ccetheme}Cristo el sumo sacerdote \end{ccetheme}
			
			\begin{ccereference}\end{ccereference}CEC 467, 540, 1137: </p>
			
			\begin{ccebody}\begin{ccenumber}467\end{ccenumber} Los monofisitas afirmaban que la naturaleza humana había dejado de existir como tal en Cristo al ser asumida por su persona divina de Hijo de Dios. Enfrentado a esta herejía, el cuarto Concilio Ecuménico, en Calcedonia, confesó en el año 451:\end{ccebody}
			
			\begin{ccecite}“Siguiendo, pues, a los Santos Padres, enseñamos unánimemente que hay que confesar a un solo y mismo Hijo y Señor nuestro Jesucristo: perfecto en la divinidad, y perfecto en la humanidad; verdaderamente Dios y verdaderamente hombre compuesto de alma racional y cuerpo; consubstancial con el Padre según la divinidad, y consubstancial con nosotros según la humanidad, ‘en todo semejante a nosotros, excepto en el pecado’ (\textit{Hb} 4, 15); nacido del Padre antes de todos los siglos según la divinidad; y por nosotros y por nuestra salvación, nacido en los últimos tiempos de la Virgen María, la Madre de Dios, según la humanidad.\end{ccecite}
			
			\begin{ccecite}Se ha de reconocer a un solo y mismo Cristo Señor, Hijo único en dos naturalezas, sin confusión, sin cambio, sin división, sin separación. La diferencia de naturalezas de ningún modo queda suprimida por su unión, sino que quedan a salvo las propiedades de cada una de las naturalezas y confluyen en un solo sujeto y en una sola persona” (Concilio de Calcedonia; DS, 301-302).\end{ccecite}
			
			\begin{ccebody}\begin{ccenumber}540\end{ccenumber} La tentación de Jesús manifiesta la manera que tiene de ser Mesías el Hijo de Dios, en oposición a la que le propone Satanás y a la que los hombres (cf. \textit{Mt} 16, 21-23) le quieren atribuir. Por eso Cristo ha vencido al Tentador \textit{en beneficio nuestro}: “Pues no tenemos un Sumo Sacerdote que no pueda compadecerse de nuestras flaquezas, sino probado en todo igual que nosotros, excepto en el pecado” (\textit{Hb} 4, 15). La Iglesia se une todos los años, durante los cuarenta días de \textit{la Gran Cuaresma}, al Misterio de Jesús en el desierto.\end{ccebody}
			
			\begin{ccebody}\textbf{Los celebrantes de la liturgia celestial}\end{ccebody}
			
			\begin{ccebody}\begin{ccenumber}1137\end{ccenumber} El Apocalipsis de san Juan, leído en la liturgia de la Iglesia, nos revela primeramente que “un trono estaba erigido en el cielo y Uno sentado en el trono” (\textit{Ap} 4, 2): “el Señor Dios” (\textit{Is }6, 1; cf. \textit{Ez} 1, 26-28). Luego revela al Cordero, “inmolado y de pie” (\textit{Ap} 5, 6; cf. \textit{Jn} 1, 29): Cristo crucificado y resucitado, el único Sumo Sacerdote del santuario verdadero (cf. \textit{Hb} 4, 14-15; 10, 19-21; etc), el mismo “que ofrece y que es ofrecido, que da y que es dado” (\textit{Liturgia Bizantina. Anaphora Iohannis Chrysostomi}). Y por último, revela “el río de agua de vida [...] que brota del trono de Dios y del Cordero” (\textit{Ap} 22, 1), uno de los más bellos símbolos del Espíritu Santo (cf. \textit{Jn} 4, 10-14; \textit{Ap} 21, 6).\end{ccebody}
			
			\begin{ccetheme}\textbf{La obediencia de Cristo y la nuestra }\end{ccetheme}
			
			\begin{ccereference}\end{ccereference}CEC 2825: </p>
			
			\begin{ccebody}\begin{ccenumber}2825\end{ccenumber} Jesús, “aun siendo Hijo, con lo que padeció, experimentó la obediencia” (\textit{Hb} 5, 8). ¡Con cuánta más razón la deberemos experimentar nosotros, criaturas y pecadores, que hemos llegado a ser hijos de adopción en Él! Pedimos a nuestro Padre que una nuestra voluntad a la de su Hijo para cumplir su voluntad, su designio de salvación para la vida del mundo. Nosotros somos radicalmente impotentes para ello, pero unidos a Jesús y con el poder de su Espíritu Santo, podemos poner en sus manos nuestra voluntad y decidir escoger lo que su Hijo siempre ha escogido: hacer lo que agrada al Padre (cf. \textit{Jn} 8, 29):\end{ccebody}
			
			\begin{ccecite}“Adheridos a Cristo, podemos llegar a ser un solo espíritu con Él, y así cumplir su voluntad: de esta forma ésta se hará tanto en la tierra como en el cielo” (Orígenes, \textit{De oratione}, 26, 3).\end{ccecite}
			
			\begin{ccecite}“Considerad cómo [Jesucristo] nos enseña a ser humildes, haciéndonos ver que nuestra virtud no depende sólo de nuestro esfuerzo sino de la gracia de Dios. Él ordena a cada fiel que ora, que lo haga universalmente por toda la tierra. Porque no dice ‘Que tu voluntad se haga’ en mí o en vosotros ‘sino en toda la tierra’: para que el error sea desterrado de ella, que la verdad reine en ella, que el vicio sea destruido en ella, que la virtud vuelva a florecer en ella y que la tierra ya no sea diferente del cielo” (San Juan Crisóstomo, \textit{In Matthaeum} homilia 19, 5).\end{ccecite}
			
			\chapter{Vigilia Pascual en la Noche Santa}
			
			\section{Lecturas}
			
			\begin{readtitle}PRIMERA LECTURA\end{readtitle}
			
			\begin{readbook}Del libro del Génesis \rightline{1, 1–2, 2}\end{readbook}
			
			\begin{readtheme}Vio Dios todo lo que había hecho, y era muy bueno\end{readtheme}
			
			\begin{readbody}Al principio creó Dios el cielo y la tierra. La tierra estaba informe y vacía; la tiniebla cubría la superficie del abismo, mientras el espíritu de Dios se cernía sobre la faz de las aguas. \end{readbody}
			
			\begin{readbody}Dijo Dios: \end{readbody}
			
			\begin{readtalk}“Exista la luz”. \end{readtalk}
			
			\begin{readbody}Y la luz existió. Vio Dios que la luz era buena. Y separó Dios la luz de la tiniebla. Llamó Dios a la luz “día” y a la tiniebla llamó “noche”. Pasó una tarde, pasó una mañana: el día primero. \end{readbody}
			
			\begin{readbody}Y dijo Dios: \end{readbody}
			
			\begin{readtalk}“Exista un firmamento entre las aguas, que separe aguas de aguas”. \end{readtalk}
			
			\begin{readbody}E hizo Dios el firmamento y separó las aguas de debajo del firmamento de las aguas de encima del firmamento. Y así fue. Llamó Dios al firmamento “cielo”. Pasó una tarde, pasó una mañana: el día segundo. \end{readbody}
			
			\begin{readbody}Dijo Dios: \end{readbody}
			
			\begin{readtalk}“Júntense las aguas de debajo del cielo en un solo sitio, y que aparezca lo seco”. \end{readtalk}
			
			\begin{readbody}Y así fue. Llamó Dios a lo seco “tierra”, y a la masa de las aguas llamó “mar”. Y vio Dios que era bueno. \end{readbody}
			
			\begin{readbody}Dijo Dios: \end{readbody}
			
			\begin{readtalk}“Cúbrase la tierra de verdor, de hierba verde que engendre semilla, y de árboles frutales que den fruto según su especie y que lleven semilla sobre la tierra”. \end{readtalk}
			
			\begin{readbody}Y así fue. La tierra brotó hierba verde que engendraba semilla según su especie, y árboles que daban fruto y llevaban semilla según su especie. Y vio Dios que era bueno. Pasó una tarde, pasó una mañana: el día tercero. \end{readbody}
			
			\begin{readbody}Dijo Dios: \end{readbody}
			
			\begin{readtalk}“Existan lumbreras en el firmamento del cielo, para separar el día de la noche, para señalar las fiestas, los días y los años, y sirvan de lumbreras en el firmamento del cielo, para iluminar sobre la tierra”. \end{readtalk}
			
			\begin{readbody}Y así fue. E hizo Dios dos lumbreras grandes: la lumbrera mayor para regir el día, la lumbrera menor para regir la noche; y las estrellas. Dios las puso en el firmamento del cielo para iluminar la tierra, para regir el día y la noche y para separar la luz de la tiniebla. Y vio Dios que era bueno. Pasó una tarde, pasó una mañana: el día cuarto. \end{readbody}
			
			\begin{readbody}Dijo Dios: \end{readbody}
			
			\begin{readtalk}“Bullan las aguas de seres vivientes, y vuelen los pájaros sobre la tierra frente al firmamento del cielo”. \end{readtalk}
			
			\begin{readbody}Y creó Dios los grandes cetáceos y los seres vivientes que se deslizan y que las aguas fueron produciendo según sus especies, y las aves aladas según sus especies. Y vio Dios que era bueno. \end{readbody}
			
			\begin{readbody}Luego los bendijo Dios, diciendo: \end{readbody}
			
			\begin{readtalk}“Sed fecundos y multiplicaos, llenad las aguas del mar; y que las aves se multipliquen en la tierra”. \end{readtalk}
			
			\begin{readbody}Pasó una tarde, pasó una mañana: el día quinto. \end{readbody}
			
			\begin{readbody}Dijo Dios: \end{readbody}
			
			\begin{readtalk}“Produzca la tierra seres vivientes según sus especies: ganados, reptiles y fieras según sus especies”. \end{readtalk}
			
			\begin{readbody}Y así fue. E hizo Dios las fieras según sus especies, los ganados según sus especies y los reptiles según sus especies. Y vio Dios que era bueno. \end{readbody}
			
			\begin{readbody}Dijo Dios: \end{readbody}
			
			\begin{readtalk}“Hagamos al hombre a nuestra imagen y semejanza; que domine los peces del mar, las aves del cielo, los ganados y los reptiles de la tierra”. \end{readtalk}
			
			\begin{readbody}Y creó Dios al hombre a su imagen, a imagen de Dios lo creó, varón y mujer los creó. \end{readbody}
			
			\begin{readbody}Dios los bendijo; y les dijo Dios: \end{readbody}
			
			\begin{readtalk}“Sed fecundos y multiplicaos, llenad la tierra y sometedla; dominad los peces del mar, las aves del cielo y todos los animales que se mueven sobre la tierra”. \end{readtalk}
			
			\begin{readbody}Y dijo Dios: \end{readbody}
			
			\begin{readtalk}“Mirad, os entrego todas las hierbas que engendran semilla sobre la superficie de la tierra y todos los árboles frutales que engendran semilla: os servirán de alimento. Y la hierba verde servirá de alimento a todas las fieras de la tierra, a todas las aves del cielo, a todos los reptiles de la tierra y a todo ser que respira”. \end{readtalk}
			
			\begin{readbody}Y así fue. Vio Dios todo lo que había hecho, y era muy bueno. Pasó una tarde, pasó una mañana: el día sexto. \end{readbody}
			
			\begin{readbody}Así quedaron concluidos el cielo, la tierra y todo el universo. Y habiendo concluido el día séptimo la obra que había hecho, descansó el día séptimo de toda la obra que había hecho.\end{readbody}
			
			\begin{readtitle}Salmo responsorial a la primera lectura (opción 1)\end{readtitle}
			
			\begin{readbook}Salmo \rightline{103, 1-2a. 5-6. 10 y 12. 13-14. 24 y 35c}\end{readbook}
			
			\begin{readtheme}Envía tu espíritu, Señor, y repuebla la faz de la tierra.\end{readtheme}
			
			\begin{readbody}\begin{readred}℣.\end{readred} Bendice, alma mía, al Señor: \end{readbody}
			
			\begin{readtabbed}¡Dios mío, qué grande eres! \end{readtabbed}
			
			\begin{readtabbed}Te vistes de belleza y majestad, \end{readtabbed}
			
			\begin{readtabbed}a luz te envuelve como un manto. \begin{readred}℟.\end{readred}\end{readtabbed}
			
			\begin{readbody}\begin{readred}℣.\end{readred} Asentaste la tierra sobre sus cimientos, \end{readbody}
			
			\begin{readtabbed}y no vacilará jamás; \end{readtabbed}
			
			\begin{readtabbed}la cubriste con el manto del océano, \end{readtabbed}
			
			\begin{readtabbed}y las aguas se posaron sobre las montañas. \begin{readred}℟.\end{readred}\end{readtabbed}
			
			\begin{readbody}\begin{readred}℣.\end{readred} De los manantiales sacas los ríos, \end{readbody}
			
			\begin{readtabbed}para que fluyan entre los montes; \end{readtabbed}
			
			\begin{readtabbed}junto a ellos habitan las aves del cielo, \end{readtabbed}
			
			\begin{readtabbed}y entre las frondas se oye su canto. \begin{readred}℟.\end{readred}\end{readtabbed}
			
			\begin{readbody}\begin{readred}℣.\end{readred} Desde tu morada riegas los montes, \end{readbody}
			
			\begin{readtabbed}y la tierra se sacia de tu acción fecunda; \end{readtabbed}
			
			\begin{readtabbed}haces brotar hierba para los ganados, \end{readtabbed}
			
			\begin{readtabbed}y forraje para los que sirven al hombre. \end{readtabbed}
			
			\begin{readtabbed}Él saca pan de los campos. \begin{readred}℟.\end{readred}\end{readtabbed}
			
			\begin{readps}\begin{readred}℣.\end{readred} Cuántas son tus obras, Señor, \end{readps}
			
			\begin{readtabbed}y todas las hiciste con sabiduría; \end{readtabbed}
			
			\begin{readtabbed}la tierra está llena de tus criaturas. \end{readtabbed}
			
			\begin{readtabbed}¡Bendice, alma mía, al Señor! \begin{readred}℟.\end{readred}\end{readtabbed}
			
			\begin{readtitle}Salmo responsorial a la primera lectura (opción 2)\end{readtitle}
			
			\begin{readbook}Salmo \rightline{32, 4-5. 6-7. 12-13. 20 y 22}\end{readbook}
			
			\begin{readtheme}La misericordia del Señor llena la tierra.\end{readtheme}
			
			\begin{readbody}\begin{readred}℣.\end{readred} La palabra del Señor es sincera, \end{readbody}
			
			\begin{readtabbed}y todas sus acciones son leales; \end{readtabbed}
			
			\begin{readtabbed}él ama la justicia y el derecho, \end{readtabbed}
			
			\begin{readtabbed}y su misericordia llena la tierra. \begin{readred}℟.\end{readred}\end{readtabbed}
			
			\begin{readbody}\begin{readred}℣.\end{readred} La palabra del Señor hizo el cielo; \end{readbody}
			
			\begin{readtabbed}el aliento de su boca, sus ejércitos; \end{readtabbed}
			
			\begin{readtabbed}encierra en un odre las aguas marinas, \end{readtabbed}
			
			\begin{readtabbed}mete en un depósito el océano.\begin{readred}℟.\end{readred}\end{readtabbed}
			
			\begin{readbody}\begin{readred}℣.\end{readred} Dichosa la nación cuyo Dios es el Señor, \end{readbody}
			
			\begin{readtabbed}el pueblo que él se escogió como heredad. \end{readtabbed}
			
			\begin{readtabbed}El Señor mira desde el cielo, \end{readtabbed}
			
			\begin{readtabbed}se fija en todos los hombres. \begin{readred}℟.\end{readred}\end{readtabbed}
			
			\begin{readbody}\begin{readred}℣.\end{readred} Nosotros aguardamos al Señor: \end{readbody}
			
			\begin{readtabbed}él es nuestro auxilio y escudo. \end{readtabbed}
			
			\begin{readtabbed}Que tu misericordia, Señor, venga sobre nosotros, \end{readtabbed}
			
			\begin{readtabbed}como lo esperamos de ti. \begin{readred}℟.\end{readred}\end{readtabbed}
			
			\begin{readtitle}SEGUNDA LECTURA\end{readtitle}
			
			\begin{readbook}Del libro del Génesis \rightline{22, 1-18}\end{readbook}
			
			\begin{readtheme}El sacrificio de Abrahán, nuestro padre en la fe\end{readtheme}
			
			\begin{readbody}En aquellos días, Dios puso a prueba a Abrahán. Le dijo: \end{readbody}
			
			\begin{readtalk}“¡Abrahán!”. \end{readtalk}
			
			\begin{readbody}Él respondió: \end{readbody}
			
			\begin{readtalk}“Aquí estoy”. \end{readtalk}
			
			\begin{readbody}Dios dijo: \end{readbody}
			
			\begin{readtalk}“Toma a tu hijo único, al que amas, a Isaac, y vete a la tierra de Moria y ofrécemelo allí en holocausto en uno de los montes que yo te indicaré”. \end{readtalk}
			
			\begin{readbody}Abrahán madrugó, aparejó el asno y se llevó consigo a dos criados y a su hijo Isaac; cortó leña para el holocausto y se encaminó al lugar que le había indicado Dios. \end{readbody}
			
			\begin{readbody}Al tercer día levantó Abrahán los ojos y divisó el sitio desde lejos. Abrahán dijo a sus criados: \end{readbody}
			
			\begin{readtalk}“Quedaos aquí con el asno; yo con el muchacho iré hasta allá para adorar, y después volveremos con vosotros”. \end{readtalk}
			
			\begin{readbody}Abrahán tomó la leña para el holocausto, se la cargó a su hijo Isaac, y él llevaba el fuego y el cuchillo. Los dos caminaban juntos. \end{readbody}
			
			\begin{readbody}Isaac dijo a Abrahán, su padre: \end{readbody}
			
			\begin{readtalk}“Padre”. \end{readtalk}
			
			\begin{readbody}Él respondió: \end{readbody}
			
			\begin{readtalk}“Aquí estoy, hijo mío”. \end{readtalk}
			
			\begin{readbody}El muchacho dijo: \end{readbody}
			
			\begin{readtalk}“Tenemos fuego y leña, pero ¿dónde está el cordero para el holocausto?”. \end{readtalk}
			
			\begin{readbody}Abrahán contestó: \end{readbody}
			
			\begin{readtalk}“Dios proveerá el cordero para el holocausto, hijo mío”. \end{readtalk}
			
			\begin{readbody}Y siguieron caminando juntos. \end{readbody}
			
			\begin{readbody}Cuando llegaron al sitio que le había dicho Dios, Abrahán levantó allí el altar y apiló la leña, luego ató a su hijo Isaac y lo puso sobre el altar, encima de la leña. Entonces Abrahán alargó la mano y tomó el cuchillo para degollar a su hijo. \end{readbody}
			
			\begin{readbody}Pero el ángel del Señor le gritó desde el cielo: \end{readbody}
			
			\begin{readtalk}“¡Abrahán, Abrahán!”. \end{readtalk}
			
			\begin{readbody}Él contestó: \end{readbody}
			
			\begin{readtalk}“Aquí estoy”. \end{readtalk}
			
			\begin{readbody}El ángel le ordenó: \end{readbody}
			
			\begin{readtalk}“No alargues la mano contra el muchacho ni le hagas nada. Ahora he comprobado que temes a Dios, porque no te has reservado a tu hijo, a tu único hijo”. \end{readtalk}
			
			\begin{readbody}Abrahán levantó los ojos y vio un carnero enredado por los cuernos en la maleza. Se acercó, tomó el carnero y lo ofreció en holocausto en lugar de su hijo. \end{readbody}
			
			\begin{readbody}Abrahán llamó aquel sitio “El Señor ve”, por lo que se dice aún hoy “En el monte el Señor es visto”. \end{readbody}
			
			\begin{readbody}El ángel del Señor llamó a Abrahán por segunda vez desde el cielo y le dijo: \end{readbody}
			
			\begin{readtalk}“Juro por mí mismo, oráculo del Señor: por haber hecho esto, por no haberte reservado tu hijo, tu hijo único, te colmaré de bendiciones y multiplicaré a tus descendientes como las estrellas del cielo y como la arena de la playa. Tus descendientes conquistarán las puertas de sus enemigos. Todas las naciones de la tierra se bendecirán con tu descendencia, porque has escuchado mi voz”.\end{readtalk}
			
			\begin{readtitle}Salmo responsorial a la segunda lectura\end{readtitle}
			
			\begin{readbook}Salmo \rightline{15, 5 y 8. 9-10. 11}\end{readbook}
			
			\begin{readtheme}Protégeme, Dios mío, que me refugio en ti\end{readtheme}
			
			\begin{readbody}\begin{readred}℣.\end{readred} El Señor es el lote de mi heredad y mi copa; \end{readbody}
			
			\begin{readtabbed}mi suerte está en tu mano. \end{readtabbed}
			
			\begin{readtabbed}Tengo siempre presente al Señor, \end{readtabbed}
			
			\begin{readtabbed}con él a mi derecha no vacilaré. \begin{readred}℟.\end{readred}\end{readtabbed}
			
			\begin{readbody}\begin{readred}℣.\end{readred} Por eso se me alegra el corazón, \end{readbody}
			
			\begin{readtabbed}se gozan mis entrañas, \end{readtabbed}
			
			\begin{readtabbed}y mi carne descansa esperanzada. \end{readtabbed}
			
			\begin{readtabbed}Porque no me abandonarás en la región de los muertos \end{readtabbed}
			
			\begin{readtabbed}ni dejarás a tu fiel conocer la corrupción. \begin{readred}℟.\end{readred}\end{readtabbed}
			
			\begin{readbody}\begin{readred}℣.\end{readred} Me enseñarás el sendero de la vida, \end{readbody}
			
			\begin{readtabbed}me saciarás de gozo en tu presencia, \end{readtabbed}
			
			\begin{readtabbed}de alegría perpetua a tu derecha. \begin{readred}℟.\end{readred}\end{readtabbed}
			
			\begin{readtitle}TERCERA LECTURA\end{readtitle}
			
			\begin{readbook}Del libro del Éxodo \rightline{14, 15–15, 1}\end{readbook}
			
			\begin{readtheme}Los hijos de Israel entraron en medio del mar, por lo seco\end{readtheme}
			
			\begin{readbody}En aquellos días, el Señor dijo a Moisés: \end{readbody}
			
			\begin{readtalk}“¿Por qué sigues clamando a mí? Di a los hijos de Israel que se pongan en marcha. Y tú, alza tu cayado, extiende tu mano sobre el mar y divídelo, para que los hijos de Israel pasen por medio del mar, por lo seco. Yo haré que los egipcios se obstinen y entren detrás de vosotros, y me cubriré de gloria a costa del faraón y de todo su ejército, de sus carros y de sus jinetes. Así sabrán los egipcios que yo soy el Señor, cuando me haya cubierto de gloria a costa del faraón, de sus carros y de sus jinetes”. \end{readtalk}
			
			\begin{readbody}Se puso en marcha el ángel del Señor, que iba al frente del ejército de Israel, y pasó a retaguardia. También la columna de nube, que iba delante de ellos, se desplazó y se colocó detrás, poniéndose entre el campamento de los egipcios y el campamento de Israel. La nube era tenebrosa y transcurrió toda la noche sin que los ejércitos pudieran aproximarse el uno al otro. Moisés extendió su mano sobre el mar y el Señor hizo retirarse el mar con un fuerte viento del Este que sopló toda la noche; el mar se secó y se dividieron las aguas. Los hijos de Israel entraron en medio del mar, en lo seco, y las aguas les hacían de muralla a derecha e izquierda. Los egipcios los persiguieron y entraron tras ellos, en medio del mar: todos los caballos del faraón, sus carros y sus jinetes. \end{readbody}
			
			\begin{readbody}Era ya la vigilia matutina cuando el Señor miró desde la columna de fuego y humo hacia el ejército de los egipcios y sembró el pánico en el ejército egipcio. Trabó las ruedas de sus carros, haciéndolos avanzar pesadamente. \end{readbody}
			
			\begin{readbody}Los egipcios dijeron: \end{readbody}
			
			\begin{readtalk}“Huyamos ante Israel, porque el Señor lucha por él contra Egipto”. \end{readtalk}
			
			\begin{readbody}Luego dijo el Señor a Moisés: \end{readbody}
			
			\begin{readtalk}“Extiende tu mano sobre el mar, y vuelvan las aguas sobre los egipcios, sus carros y sus jinetes”. \end{readtalk}
			
			\begin{readbody}Moisés extendió su mano sobre el mar; y al despuntar el día el mar recobró su estado natural, de modo que los egipcios, en su huida, toparon con las aguas. Así precipitó el Señor a los egipcios en medio del mar. \end{readbody}
			
			\begin{readbody}Las aguas volvieron y cubrieron los carros, los jinetes y todo el ejército del faraón, que había entrado en el mar. Ni uno solo se salvó. \end{readbody}
			
			\begin{readbody}Mas los hijos de Israel pasaron en seco por medio del mar, mientras las aguas hacían de muralla a derecha e izquierda. \end{readbody}
			
			\begin{readbody}Aquel día salvó el Señor a Israel del poder de Egipto, e Israel vio a los egipcios muertos, en la orilla del mar. Vio, pues, Israel la mano potente que el Señor había desplegado contra los egipcios, y temió el pueblo al Señor, y creyó en el Señor y en Moisés, su siervo. \end{readbody}
			
			\begin{readbody}Entonces Moisés y los hijos de Israel entonaron este canto al Señor:\end{readbody}
			
			\begin{readtitle}Salmo responsorial a la tercera lectura\end{readtitle}
			
			\begin{readbook}Del libro del Éxodo \rightline{15, 1-2. 3-4. 5-6. 17-18}\end{readbook}
			
			\begin{readtheme}Cantaré al Señor, gloriosa es su victoria\end{readtheme}
			
			\begin{readbody}\begin{readred}℣.\end{readred} Cantaré al Señor, gloriosa es su victoria, \end{readbody}
			
			\begin{readtabbed}caballos y carros ha arrojado en el mar. \end{readtabbed}
			
			\begin{readtabbed}Mi fuerza y mi poder es el Señor, \end{readtabbed}
			
			\begin{readtabbed}El fue mi salvación. \end{readtabbed}
			
			\begin{readtabbed}Él es mi Dios: yo lo alabaré; \end{readtabbed}
			
			\begin{readtabbed}el Dios de mis padres: yo lo ensalzaré. \begin{readred}℟.\end{readred}\end{readtabbed}
			
			\begin{readbody}\begin{readred}℣.\end{readred} El Señor es un guerrero, \end{readbody}
			
			\begin{readtabbed}su nombre es “El Señor”. \end{readtabbed}
			
			\begin{readtabbed}Los carros del faraón los lanzó al mar, \end{readtabbed}
			
			\begin{readtabbed}ahogó en el mar Rojo a sus mejores capitanes. \begin{readred}℟.\end{readred}\end{readtabbed}
			
			\begin{readbody}\begin{readred}℣.\end{readred} Las olas los cubrieron, \end{readbody}
			
			\begin{readtabbed}bajaron hasta el fondo como piedras. \end{readtabbed}
			
			\begin{readtabbed}Tu diestra, Señor, es magnífica en poder, \end{readtabbed}
			
			\begin{readtabbed}tu diestra, Señor, tritura al enemigo. \begin{readred}℟.\end{readred}\end{readtabbed}
			
			\begin{readbody}\begin{readred}℣.\end{readred} Lo introduces y lo plantas en el monte de tu heredad, \end{readbody}
			
			\begin{readtabbed}lugar del que hiciste tu trono, Señor; \end{readtabbed}
			
			\begin{readtabbed}santuario, Señor, que fundaron tus manos. \end{readtabbed}
			
			\begin{readtabbed}El Señor reina por siempre jamás. \begin{readred}℟.\end{readred}\end{readtabbed}
			
			\begin{readtitle}CUARTA LECTURA\end{readtitle}
			
			\begin{readbook}Del libro del profeta Isaías \rightline{54, 5-14}\end{readbook}
			
			\begin{readtheme}Con amor eterno te quiere el Señor, tu libertador\end{readtheme}
			
			\begin{readtalk}Quien te desposa es tu Hacedor: <br />su nombre es Señor todopoderoso. \end{readtalk}
			
			\begin{readtalk}Tu libertador es el Santo de Israel: <br />se llama “Dios de toda la tierra”. \end{readtalk}
			
			\begin{readtalk}Como a mujer abandonada y abatida <br />te llama el Señor; <br />como a esposa de juventud, repudiada <br />–dice tu Dios–. \end{readtalk}
			
			\begin{readtalk}Por un instante te abandoné, <br />pero con gran cariño te reuniré. \end{readtalk}
			
			\begin{readtalk}En un arrebato de ira, <br />por un instante te escondí mi rostro, <br />pero con amor eterno te quiero <br />–dice el Señor, tu libertador–. \end{readtalk}
			
			\begin{readtalk}Me sucede como en los días de Noé: <br />juré que las aguas de Noé <br />no volverían a cubrir la tierra; <br />así juro no irritarme contra ti <br />ni amenazarte. \end{readtalk}
			
			\begin{readtalk}Aunque los montes cambiasen <br />y vacilaran las colinas, <br />no cambiaría mi amor, <br />ni vacilaría mi alianza de paz <br />–dice el Señor que te quiere–. \end{readtalk}
			
			\begin{readtalk}¡Ciudad afligida, azotada por el viento, <br />a quien nadie consuela! \end{readtalk}
			
			\begin{readtalk}Mira, yo mismo asiento tus piedras sobre azabaches, <br />tus cimientos sobre zafiros; <br />haré tus almenas de rubí, <br />tus puertas de esmeralda, <br />y de piedras preciosas tus bastiones. \end{readtalk}
			
			\begin{readtalk}Tus hijos serán discípulos del Señor, <br />gozarán de gran prosperidad tus constructores. \end{readtalk}
			
			\begin{readtalk}Tendrás tu fundamento en la justicia: <br />lejos de la opresión, no tendrás que temer; <br />lejos del terror, que no se acercará.\end{readtalk}
			
			\begin{readtitle}Salmo responsorial a la cuarta lectura\end{readtitle}
			
			\begin{readbook}Salmo \rightline{29, 2 y 4. 5-6. 11-12a y 13b}\end{readbook}
			
			\begin{readtheme}Te ensalzaré, Señor, porque me has librado\end{readtheme}
			
			\begin{readbody}\begin{readred}℣.\end{readred} Te ensalzaré, Señor, porque me has librado \end{readbody}
			
			\begin{readtabbed}y no has dejado que mis enemigos se rían de mí. \end{readtabbed}
			
			\begin{readtabbed}Señor, sacaste mi vida del abismo, \end{readtabbed}
			
			\begin{readtabbed}me hiciste revivir cuando bajaba a la fosa. \begin{readred}℟.\end{readred}\end{readtabbed}
			
			\begin{readbody}\begin{readred}℣.\end{readred} Tañed para el Señor, fieles suyos, \end{readbody}
			
			\begin{readtabbed}celebrad el recuerdo de su nombre santo; \end{readtabbed}
			
			\begin{readtabbed}su cólera dura un instante; \end{readtabbed}
			
			\begin{readtabbed}su bondad, de por vida; \end{readtabbed}
			
			\begin{readtabbed}al atardecer nos visita el llanto; \end{readtabbed}
			
			\begin{readtabbed}por la mañana, el júbilo. \begin{readred}℟.\end{readred}\end{readtabbed}
			
			\begin{readbody}\begin{readred}℣.\end{readred} Escucha, Señor, y ten piedad de mí; \end{readbody}
			
			\begin{readtabbed}Señor, socórreme. \end{readtabbed}
			
			\begin{readtabbed}Cambiaste mi luto en danzas. \end{readtabbed}
			
			\begin{readtabbed}Señor, Dios mío, te daré gracias por siempre. \begin{readred}℟.\end{readred}\end{readtabbed}
			
			\begin{readtitle}QUINTA LECTURA\end{readtitle}
			
			\begin{readbook}Del libro del profeta Isaías \rightline{55, 1-11}\end{readbook}
			
			\begin{readtheme}Venid a mí y viviréis. Sellaré con vosotros una alianza perpetua\end{readtheme}
			
			\begin{readbody}Esto dice el Señor: \end{readbody}
			
			\begin{readtalk}“Oíd, sedientos todos, acudid por agua; <br />venid, también los que no tenéis dinero: <br />comprad trigo y comed, venid y comprad, <br />sin dinero y de balde, vino y leche. \end{readtalk}
			
			\begin{readtalk}¿Por qué gastar dinero en lo que no alimenta <br />y el salario en lo que no da hartura? \end{readtalk}
			
			\begin{readtalk}Escuchadme atentos y comeréis bien, <br />saborearéis platos sustanciosos. \end{readtalk}
			
			\begin{readtalk}Inclinad vuestro oído, venid a mí: <br />escuchadme y viviréis. \end{readtalk}
			
			\begin{readtalk}Sellaré con vosotros una alianza perpetua, <br />las misericordias firmes hechas a David: <br />lo hice mi testigo para los pueblos, <br />guía y soberano de naciones. \end{readtalk}
			
			\begin{readtalk}Tú llamarás a un pueblo desconocido, <br />un pueblo que no te conocía correrá hacia ti; <br />porque el Señor tu Dios, <br />el Santo de Israel te glorifica. \end{readtalk}
			
			\begin{readtalk}Buscad al Señor mientras se deja encontrar, <br />invocadlo mientras está cerca. \end{readtalk}
			
			\begin{readtalk}Que el malvado abandone su camino, <br />y el malhechor sus planes; <br />que se convierta al Señor, y él tendrá piedad, <br />a nuestro Dios, que es rico en perdón. \end{readtalk}
			
			\begin{readtalk}Porque mis planes no son vuestros planes, <br />vuestros caminos no son mis caminos <br />–oráculo del Señor–. \end{readtalk}
			
			\begin{readtalk}Cuanto dista el cielo de la tierra, <br />así distan mis caminos de los vuestros, <br />y mis planes de vuestros planes. \end{readtalk}
			
			\begin{readtalk}Como bajan la lluvia y la nieve desde el cielo, <br />y no vuelven allá, sino después de empapar la tierra, <br />de fecundarla y hacerla germinar, <br />para que dé semilla al sembrador <br />y pan al que come, <br />así será mi palabra que sale de mi boca: <br />no volverá a mí vacía, <br />sino que cumplirá mi deseo <br />y llevará a cabo mi encargo”.\end{readtalk}
			
			\begin{readtitle}Salmo responsorial a la quinta lectura\end{readtitle}
			
			\begin{readbook}Del libro del profeta Isaías \rightline{12, 2-3. 4bcd. 5-6}\end{readbook}
			
			\begin{readtheme}Sacaréis aguas con gozo de las fuentes de la salvación\end{readtheme}
			
			\begin{readbody}\begin{readred}℣.\end{readred} “El Señor es mi Dios y salvador:\end{readbody}
			
			\begin{readtabbed}confiaré y no temeré,\end{readtabbed}
			
			\begin{readtabbed}porque mi fuerza y mi poder es el Señor,\end{readtabbed}
			
			\begin{readtabbed}él fue mi salvación”.\end{readtabbed}
			
			\begin{readtabbed}Y sacaréis aguas con gozo\end{readtabbed}
			
			\begin{readtabbed}de las fuentes de la salvación. \begin{readred}℟.\end{readred}\end{readtabbed}
			
			\begin{readbody}\begin{readred}℣.\end{readred} “Dad gracias al Señor,\end{readbody}
			
			\begin{readtabbed}invocad su nombre,\end{readtabbed}
			
			\begin{readtabbed}contad a los pueblos sus hazañas,\end{readtabbed}
			
			\begin{readtabbed}proclamad que su nombre es excelso”. \begin{readred}℟.\end{readred}\end{readtabbed}
			
			\begin{readbody}\begin{readred}℣.\end{readred} Tañed para el Señor, que hizo proezas,\end{readbody}
			
			\begin{readtabbed}anunciadlas a toda la tierra;\end{readtabbed}
			
			\begin{readtabbed}gritad jubilosos, habitantes de Sión:\end{readtabbed}
			
			\begin{readtabbed}porque es grande medio de ti el Santo de Israel. \begin{readred}℟.\end{readred}\end{readtabbed}
			
			\begin{readtitle}SEXTA LECTURA\end{readtitle}
			
			\begin{readbook}Del libro del profeta Baruc \rightline{3, 9-15. 32–4, 4}\end{readbook}
			
			\begin{readtheme}Camina al resplandor del Señor\end{readtheme}
			
			\begin{readtalk}Escucha, Israel, mandatos de vida; <br />presta oído y aprende prudencia. \end{readtalk}
			
			\begin{readtalk}¿Cuál es la razón, Israel, <br />de que sigas en país enemigo, <br />envejeciendo en tierra extranjera; <br />de que te crean un ser contaminado, <br />un muerto habitante del Abismo? \end{readtalk}
			
			\begin{readtalk}¡Abandonaste la fuente de la sabiduría! \end{readtalk}
			
			\begin{readtalk}Si hubieras seguido el camino de Dios, <br />habitarías en paz para siempre. \end{readtalk}
			
			\begin{readtalk}Aprende dónde está la prudencia, <br />dónde el valor y la inteligencia, <br />dónde una larga vida, <br />la luz de los ojos y la paz. \end{readtalk}
			
			\begin{readtalk}¿Quién encontró su lugar <br />o tuvo acceso a sus tesoros? \end{readtalk}
			
			\begin{readtalk}El que todo lo sabe la conoce, <br />la ha examinado y la penetra; <br />el que creó la tierra para siempre <br />y la llenó de animales cuadrúpedos; <br />el que envía la luz y le obedece, <br />la llama y acude temblorosa; <br />a los astros que velan gozosos <br />arriba en sus puestos de guardia, <br />los llama, y responden: “Presentes”, <br />y brillan gozosos para su Creador. \end{readtalk}
			
			\begin{readtalk}Este es nuestro Dios, <br />y no hay quien se le pueda comparar; <br />rastreó el camino de la inteligencia <br />y se lo enseñó a su hijo, Jacob, <br />se lo mostró a su amado, Israel. \end{readtalk}
			
			\begin{readtalk}Después apareció en el mundo <br />y vivió en medio de los hombres. \end{readtalk}
			
			\begin{readtalk}Es el libro de los mandatos de Dios, <br />la ley de validez eterna: <br />los que la guarden vivirán; <br />los que la abandonen morirán. \end{readtalk}
			
			\begin{readtalk}Vuélvete, Jacob, a recibirla, <br />camina al resplandor de su luz; <br />no entregues a otros tu gloria, <br />ni tu dignidad a un pueblo extranjero. \end{readtalk}
			
			\begin{readtalk}¡Dichosos nosotros, Israel, <br />que conocemos lo que agrada al Señor!\end{readtalk}
			
			\begin{readtitle}Salmo responsorial a la sexta lectura\end{readtitle}
			
			\begin{readbook}Salmo \rightline{18, 8. 9. 10. 11}\end{readbook}
			
			\begin{readtheme}Señor, tú tienes palabras de vida eterna\end{readtheme}
			
			\begin{readbody}\begin{readred}℣.\end{readred} La ley del Señor es perfecta \end{readbody}
			
			\begin{readtabbed}y es descanso del alma; \end{readtabbed}
			
			\begin{readtabbed}el precepto del Señor es fiel \end{readtabbed}
			
			\begin{readtabbed}e instruye a los ignorantes. \begin{readred}℟.\end{readred}\end{readtabbed}
			
			\begin{readbody}\begin{readred}℣.\end{readred} Los mandatos del Señor son rectos \end{readbody}
			
			\begin{readtabbed}y alegran el corazón; \end{readtabbed}
			
			\begin{readtabbed}la norma del Señor es límpida \end{readtabbed}
			
			\begin{readtabbed}y da luz a los ojos. \begin{readred}℟.\end{readred}\end{readtabbed}
			
			\begin{readbody}\begin{readred}℣.\end{readred} El temor del Señor es puro \end{readbody}
			
			\begin{readtabbed}y eternamente estable; \end{readtabbed}
			
			\begin{readtabbed}los mandamientos del Señor son verdaderos \end{readtabbed}
			
			\begin{readtabbed}y enteramente justos. \begin{readred}℟.\end{readred}\end{readtabbed}
			
			\begin{readbody}\begin{readred}℣.\end{readred} Más preciosos que el oro, \end{readbody}
			
			\begin{readtabbed}más que el oro fino; \end{readtabbed}
			
			\begin{readtabbed}más dulces que la miel \end{readtabbed}
			
			\begin{readtabbed}de un panal que destila. \begin{readred}℟.\end{readred}\end{readtabbed}
			
			\begin{readtitle}SÉPTIMA LECTURA\end{readtitle}
			
			\begin{readbook}Del libro del profeta Ezequiel \rightline{36, 16-28}\end{readbook}
			
			\begin{readtheme}Derramaré sobre vosotros un agua pura, y os daré un corazón nuevo\end{readtheme}
			
			\begin{readbody}Me vino esta palabra del Señor: \end{readbody}
			
			\begin{readtalk}“Hijo de hombre, la casa de Israel profanó con su conducta <br />y sus acciones la tierra en que habitaba. \end{readtalk}
			
			\begin{readtalk}Me enfurecí contra ellos, <br />por la sangre que habían derramado en el país, <br />y por haberlo profanado con sus ídolos. \end{readtalk}
			
			\begin{readtalk}Los dispersé por las naciones, <br />y anduvieron dispersos por diversos países. \end{readtalk}
			
			\begin{readtalk}Los he juzgado según su conducta y sus acciones. \end{readtalk}
			
			\begin{readtalk}Al llegar a las diversas naciones, <br />profanaron mi santo nombre, <br />ya que de ellos se decía: <br />“Estos son el pueblo del Señor <br />y han debido abandonar su tierra”. \end{readtalk}
			
			\begin{readtalk}Así que tuve que defender mi santo nombre, <br />profanado por la casa de Israel <br />entre las naciones adonde había ido. \end{readtalk}
			
			\begin{readtalk}Por eso, di a la casa de Israel: <br />“Esto dice el Señor Dios: <br />No hago esto por vosotros, casa de Israel, <br />sino por mi santo nombre, profanado por vosotros <br />en las naciones a las que fuisteis. \end{readtalk}
			
			\begin{readtalk}Manifestaré la santidad de mi gran nombre, <br />profanado entre los gentiles, <br />porque vosotros lo habéis profanado en medio de ellos. \end{readtalk}
			
			\begin{readtalk}Reconocerán las naciones que yo soy el Señor <br />–oráculo del Señor Dios–, <br />cuando por medio de vosotros les haga ver mi santidad. \end{readtalk}
			
			\begin{readtalk}Os recogeré de entre las naciones, <br />os reuniré de todos los países <br />y os llevaré a vuestra tierra. \end{readtalk}
			
			\begin{readtalk}Derramaré sobre vosotros un agua pura <br />que os purificará: <br />de todas vuestras inmundicias e idolatrías <br />os he de purificar; <br />y os daré un corazón nuevo, <br />y os infundiré un espíritu nuevo; <br />arrancaré de vuestra carne el corazón de piedra, <br />y os daré un corazón de carne. \end{readtalk}
			
			\begin{readtalk}Os infundiré mi espíritu, <br />y haré que caminéis según mis preceptos, <br />y que guardéis y cumpláis mis mandatos. \end{readtalk}
			
			\begin{readtalk}Y habitaréis en la tierra que di a vuestros padres. <br />Vosotros seréis mi pueblo, <br />y yo seré vuestro Dios”».\end{readtalk}
			
			\begin{readbook}Salmo responsorial a la séptima lectura<br />(cuando no hay Bautismo)\end{readbook}
			
			\begin{readbook}Salmo \rightline{41, 3. 5bcd; 42, 3. 4}\end{readbook}
			
			\begin{readtheme}Como busca la cierva corrientes de agua, así mi alma te busca a ti, Dios mío\end{readtheme}
			
			\begin{readbody}\begin{readred}℣.\end{readred} Mi alma tiene sed de Dios, del Dios vivo: \end{readbody}
			
			\begin{readtabbed}cuándo entraré a ver el rostro de Dios? \begin{readred}℟.\end{readred}\end{readtabbed}
			
			\begin{readbody}\begin{readred}℣.\end{readred} Cómo entraba en el recinto santo, \end{readbody}
			
			\begin{readtabbed}cómo avanzaba hacia la casa de Dios, \end{readtabbed}
			
			\begin{readtabbed}entre cantos de júbilo y alabanza, \end{readtabbed}
			
			\begin{readtabbed}en el bullicio de la fiesta. \begin{readred}℟.\end{readred}\end{readtabbed}
			
			\begin{readbody}\begin{readred}℣.\end{readred} Envía tu luz y tu verdad: \end{readbody}
			
			\begin{readtabbed}que ellas me guíen \end{readtabbed}
			
			\begin{readtabbed}y me conduzcan hasta tu monte santo, \end{readtabbed}
			
			\begin{readtabbed}hasta tu morada. \begin{readred}℟.\end{readred}\end{readtabbed}
			
			\begin{readbody}\begin{readred}℣.\end{readred} Me acercaré al altar de Dios, \end{readbody}
			
			\begin{readtabbed}al Dios de mi alegría; \end{readtabbed}
			
			\begin{readtabbed}y te daré gracias al son de la cítara, \end{readtabbed}
			
			\begin{readtabbed}Dios, Dios mío. \begin{readred}℟.\end{readred}\end{readtabbed}
			
			\begin{readbook}Salmo responsorial a la séptima lectura<br />(cuando hay Bautismo, opción 1)\end{readbook}
			
			\begin{readbook}Isaías \rightline{12, 2-3. 4bcde. 5-6} \end{readbook}
			
			\begin{readtheme}Sacaréis aguas con gozo de las fuentes de la salvación.\end{readtheme}
			
			\begin{readbody}\begin{readred}℣.\end{readred} “Él es mi Dios y Salvador:\end{readbody}
			
			\begin{readtabbed}confiaré y no temeré,\end{readtabbed}
			
			\begin{readtabbed}porque mi fuerza y mi poder es el Señor,\end{readtabbed}
			
			\begin{readtabbed}él fue mi salvación”.\end{readtabbed}
			
			\begin{readtabbed}Y sacaréis aguas con gozo\end{readtabbed}
			
			\begin{readtabbed}de las fuentes de la salvación. \begin{readred}℟.\end{readred}\end{readtabbed}
			
			\begin{readbody}\begin{readred}℣.\end{readred} “Dad gracias al Señor,\end{readbody}
			
			\begin{readtabbed}invocad su nombre,\end{readtabbed}
			
			\begin{readtabbed}contad a los pueblos sus hazañas,\end{readtabbed}
			
			\begin{readtabbed}proclamad que su nombre es excelso”. \begin{readred}℟.\end{readred}\end{readtabbed}
			
			\begin{readbody}\begin{readred}℣.\end{readred} Tañed para el Señor, que hizo proezas,\end{readbody}
			
			\begin{readtabbed}anunciadlas a toda la tierra;\end{readtabbed}
			
			\begin{readtabbed}gritad jubilosos, habitantes de Sion,\end{readtabbed}
			
			\begin{readtabbed}porque es grande en medio de ti el Santo de Israel. \begin{readred}℟.\end{readred}\end{readtabbed}
			
			\begin{readbook}Salmo responsorial a la séptima lectura<br />(cuando hay Bautismo, opción 2)\end{readbook}
			
			\begin{readbook}Salmo \rightline{50, 12-13. 14-15. 18-19}\end{readbook}
			
			\begin{readtheme}Oh, Dios, crea en mí un corazón puro\end{readtheme}
			
			\begin{readbody}\begin{readred}℣.\end{readred} Oh, Dios, crea en mí un corazón puro, \end{readbody}
			
			\begin{readtabbed}renuévame por dentro con espíritu firme. \end{readtabbed}
			
			\begin{readtabbed}No me arrojes lejos de tu rostro, \end{readtabbed}
			
			\begin{readtabbed}no me quites tu santo espíritu. \begin{readred}℟.\end{readred}\end{readtabbed}
			
			\begin{readbody}\begin{readred}℣.\end{readred} Devuélveme la alegría de tu salvación, \end{readbody}
			
			\begin{readtabbed}afiánzame con espíritu generoso. \end{readtabbed}
			
			\begin{readtabbed}Enseñaré a los malvados tus caminos, \end{readtabbed}
			
			\begin{readtabbed}los pecadores volverán a ti. \begin{readred}℟.\end{readred}\end{readtabbed}
			
			\begin{readbody}\begin{readred}℣.\end{readred} Los sacrificios no te satisfacen: \end{readbody}
			
			\begin{readtabbed}si te ofreciera un holocausto, no lo querrías. \end{readtabbed}
			
			\begin{readtabbed}El sacrificio agradable a Dios \end{readtabbed}
			
			\begin{readtabbed}es un espíritu quebrantado; \end{readtabbed}
			
			\begin{readtabbed}un corazón quebrantado y humillado, \end{readtabbed}
			
			\begin{readtabbed}tú, oh, Dios, tú no lo desprecias. \begin{readred}℟.\end{readred}\end{readtabbed}
			
			\begin{readtitle}EPÍSTOLA\end{readtitle}
			
			\begin{readbook}De la carta del apóstol san Pablo a los Romanos \rightline{6, 3-11}\end{readbook}
			
			\begin{readtheme}Cristo, una vez resucitado de entre los muertos, ya no muere más\end{readtheme}
			
			\begin{readbody}Hermanos: \end{readbody}
			
			\begin{readbody}Cuantos fuimos bautizados en Cristo Jesús fuimos bautizados en su muerte. \end{readbody}
			
			\begin{readbody}Por el bautismo fuimos sepultados con él en la muerte, para que, lo mismo que Cristo resucitó de entre los muertos por la gloria del Padre, así también nosotros andemos en una vida nueva. \end{readbody}
			
			\begin{readbody}Pues si hemos sido incorporados a él en una muerte como la suya, lo seremos también en una resurrección como la suya; sabiendo que nuestro hombre viejo fue crucificado con Cristo, para que fuera destruido el cuerpo de pecado, y, de este modo, nosotros dejáramos de servir al pecado; porque quien muere ha quedado libre del pecado. \end{readbody}
			
			\begin{readbody}Si hemos muerto con Cristo, creemos que también viviremos con él; pues sabemos que Cristo, una vez resucitado de entre los muertos, ya no muere más; la muerte ya no tiene dominio sobre él. Porque quien ha muerto, ha muerto al pecado de una vez para siempre; y quien vive, vive para Dios. \end{readbody}
			
			\begin{readbody}Lo mismo vosotros, consideraos muertos al pecado y vivos para Dios en Cristo Jesús.\end{readbody}
			
			\begin{readtitle}SALMO RESPONSORIAL\end{readtitle}
			
			\begin{readbook}Salmo \rightline{117, 1-2. 16ab-17. 22-23}\end{readbook}
			
			\begin{readtheme}Aleluya, aleluya, aleluya\end{readtheme}
			
			\begin{readbody}\begin{readred}℣.\end{readred} Dad gracias al Señor porque es bueno, \end{readbody}
			
			\begin{readtabbed}porque es eterna su misericordia. \end{readtabbed}
			
			\begin{readtabbed}Diga la casa de Israel: \end{readtabbed}
			
			\begin{readtabbed}eterna es su misericordia. \begin{readred}℟.\end{readred}\end{readtabbed}
			
			\begin{readbody}\begin{readred}℣.\end{readred} “La diestra del Señor es poderosa, \end{readbody}
			
			\begin{readtabbed}la diestra del Señor es excelsa”. \end{readtabbed}
			
			\begin{readtabbed}No he de morir, viviré \end{readtabbed}
			
			\begin{readtabbed}para contar las hazañas del Señor. \begin{readred}℟.\end{readred}\end{readtabbed}
			
			\begin{readbody}\begin{readred}℣.\end{readred} La piedra que desecharon los arquitectos \end{readbody}
			
			\begin{readtabbed}es ahora la piedra angular. \end{readtabbed}
			
			\begin{readtabbed}Es el Señor quien lo ha hecho, \end{readtabbed}
			
			\begin{readtabbed}ha sido un milagro patente. \begin{readred}℟.\end{readred}\end{readtabbed}
			
			\begin{readtitle}EVANGELIO\end{readtitle}
			
			\begin{readbook}Del Santo Evangelio según san Marcos \rightline{16, 1-7}\end{readbook}
			
			\begin{readtheme}Jesús el Nazareno, el crucificado, ha resucitado\end{readtheme}
			
			\begin{readbody}Pasado el sábado, María Magdalena, María la de Santiago y Salomé compraron aromas para ir a embalsamar a Jesús. Y muy temprano, el primer día de la semana, al salir el sol, fueron al sepulcro. Y se decían unas a otras: \end{readbody}
			
			\begin{readtalk}“¿Quién nos correrá la piedra de la entrada del sepulcro?”. \end{readtalk}
			
			\begin{readbody}Al mirar, vieron que la piedra estaba corrida y eso que era muy grande. Entraron en el sepulcro y vieron a un joven sentado a la derecha, vestido de blanco. Y quedaron aterradas. Él les dijo: \end{readbody}
			
			\begin{readtalk}“No tengáis miedo. ¿Buscáis a Jesús el Nazareno, el crucificado? Ha resucitado. No está aquí. Mirad el sitio donde lo pusieron. \end{readtalk}
			
			\begin{readtalk}Pero id a decir a sus discípulos y a Pedro: ‘Él va por delante de vosotros a Galilea. Allí lo veréis, como os dijo’”.\end{readtalk}
			
			\section{Comentario Patrístico}
			
			\subsection{San Juan Pablo II, papa}
			
			\begin{patertheme}El primer signo de la Resurrección\end{patertheme}
			
			\begin{patersource}\textit{Catequesis, }nn. 5-9: Audiencia General, 1 de febrero de 1989.\end{patersource}
			
			\begin{body} “Entraron en el sepulcro y vieron...” (\textit{Mc} 16, 5)\end{body}
			
			\begin{body}\textit{En el ámbito de los acontecimientos pascuales}, el primer elemento ante el que nos encontramos es \textit{el “sepulcro vacío”. }Sin duda no es por sí mismo una prueba directa. La ausencia del cuerpo de Cristo en el sepulcro en el que había sido depositado podría \textit{explicarse de otra forma}, como de hecho pensó por un momento María Magdalena cuando, viendo el sepulcro vacío, supuso que alguno habría sustraído el cuerpo de Jesús (cf. \textit{Jn} 20, 13).\end{body}
			
			\begin{body}Más aún el Sanedrín trató de hacer correr la voz de que, mientras dormían los soldados, el cuerpo habría sido robado por los discípulos. “Y se corrió esa versión entre los judíos, ―anota Mateo― hasta el día de hoy” (\textit{Mt }28, 12-15).\end{body}
			
			\begin{body}A pesar de esto el \textit{“sepulcro vacío” }ha constituido para todos, amigos y enemigos, un signo impresionante. Para las personas de buena voluntad su descubrimiento fue \textit{el primer paso hacia el reconocimiento del “hecho” de la resurrección como una verdad que no podía ser refutada}.\end{body}
			
			\begin{body}Así fue ante todo \textit{para las mujeres}, que muy de mañana se habían acercado al sepulcro para ungir el cuerpo de Cristo. Fueron las primeras en acoger el anuncio: “Ha resucitado, no está aquí... Pero id a decir a sus discípulos y a Pedro…” (\textit{Mc} 16, 6-7). “Recordad cómo os habló cuando estaba todavía en Galilea, diciendo: ‘Es necesario que el Hijo del hombre sea entregado en manos de los pecadores y sea crucificado, y al tercer día resucite’. Y ellas recordaron sus palabras” (\textit{Lc} 24, 6-8).\end{body}
			
			\begin{body}Ciertamente las mujeres estaban sorprendidas y asustadas (cf. \textit{Mc }16, 8; \textit{Lc }24, 5). Ni siquiera ellas estaban dispuestas a rendirse demasiado fácilmente a un hecho que, aún predicho por Jesús, estaba efectivamente por encima de toda posibilidad de imaginación y de invención. Pero en su sensibilidad y finura intuitiva ellas, y especialmente María Magdalena, se aferraron a la realidad y corrieron a donde estaban los Apóstoles para darles la alegre noticia.\end{body}
			
			\begin{body}El Evangelio de Mateo (28, 8-10) nos informa que a lo largo del camino Jesús mismo les salió al encuentro, las saludó y les renovó el mandato de llevar el anuncio a los hermanos (\textit{Mt }28, 10). De esta forma las mujeres fueron las primeras mensajeras de la resurrección de Cristo, y lo fueron para los mismos Apóstoles (\textit{Lc }24, 10). ¡Hecho elocuente sobre la importancia de la mujer ya en los días del acontecimiento pascual!\end{body}
			
			\begin{body}Entre los que recibieron el anuncio de María Magdalena estaban \textit{Pedro} y \textit{Juan} (cf. \textit{Jn} 20, 3-8). Ellos se acercaron al sepulcro no sin titubeos, tanto más cuanto que Marta les había hablado de una sustracción del cuerpo de Jesús del sepulcro (cf. \textit{Jn }20, 2). Llegados al sepulcro, también ellos lo encontraron vacío. Terminaron creyendo, tras haber dudado no poco, porque, como dice Juan, “hasta entonces no habían comprendido que según la Escritura Jesús debía resucitar de entre los muertos” (\textit{Jn }20, 9).\end{body}
			
			\begin{body}Digamos la verdad: el hecho era asombroso para aquellos hombres que se encontraban ante cosas demasiado superiores a ellos. La misma dificultad, que muestran las tradiciones del acontecimiento, al dar una relación de ello plenamente coherente, confirma su carácter extraordinario y el impacto desconcertante que tuvo en el ánimo de los afortunados testigos. La referencia\textit{ “a la Escritura” }es la prueba de la oscura percepción que tuvieron al encontrarse ante un misterio sobre el que sólo la Revelación podía dar luz.\end{body}
			
			\begin{body}Sin embargo, he aquí otro dato que se debe considerar bien: si el \textit{“sepulcro vacío”} dejaba estupefactos a primera vista y podía incluso generar una cierta sospecha, el gradual conocimiento de este hecho inicial, como lo anotan los Evangelios, terminó llevando al descubrimiento de la verdad de la resurrección.\end{body}
			
			\begin{body}En efecto, se nos dice que las mujeres, y sucesivamente los Apóstoles, se encontraron \textit{ante un “signo” particular: el signo de la victoria sobre la muerte}. Si el sepulcro mismo cerrado por una pesada losa, testimoniaba la muerte, el sepulcro vacío y la piedra removida daban el primer anuncio de que allí había sido derrotada la muerte.\end{body}
			
			\begin{body}No puede dejar de impresionar la consideración del estado de ánimo de las tres mujeres, que dirigiéndose al sepulcro al alba se decían entre sí: \textit{“¿Quién nos retirará la piedra de la puerta del sepulcro?”} (\textit{Mc }16, 3), y que después, cuando llegaron al sepulcro, con gran maravilla constataron que “la piedra estaba corrida aunque era muy grande” (\textit{Mc} 16, 4). Según el Evangelio de Marcos encontraron en el sepulcro a alguno que les dio el anuncio de la resurrección (cf. \textit{Mc }16, 5): pero ellas tuvieron miedo y, a pesar de las afirmaciones del joven vestido de blanco, “salieron huyendo del sepulcro, pues un gran temblor y espanto se había apoderado de ellas” (\textit{Mc} 16, 8). ¿Cómo no comprenderlas? Y sin embargo la comparación con los textos paralelos de los demás Evangelistas permite afirmar que, aunque temerosas, las mujeres llevaron el anuncio de la resurrección, de la que el “sepulcro vacío” con la piedra corrida fue el primer signo.\end{body}
			
			\begin{body}Para las mujeres y para los Apóstoles el camino abierto por “el signo” se concluye \textit{mediante el encuentro con el Resucitado}: entonces la percepción aún tímida e incierta se convierte en \textit{convicción} y, más aún, en fe en Aquel que “ha resucitado verdaderamente”. Así sucedió a las mujeres que al ver a Jesús en su camino y escuchar su saludo, se arrojaron a sus pies y lo adoraron (cf. \textit{Mt }28, 9). Así le pasó especialmente a María Magdalena, que al escuchar que Jesús le llamaba por su nombre, le dirigió antes que nada el apelativo habitual: \textit{Rabbuní}, ¡Maestro! (\textit{Jn }20, 16) y cuando Él la iluminó sobre el misterio pascual corrió radiante a llevar el anuncio a los discípulos: “¡He visto al Señor!” (\textit{Jn }20, 18). Lo mismo ocurrió a los discípulos reunidos en el Cenáculo que la tarde de aquel “primer día después del sábado”, cuando vieron finalmente entre ellos a Jesús, se sintieron felices por la nueva certeza que había entrado en su corazón: “Se alegraron al ver al Señor” (cf. \textit{Jn }20, 19-20).\end{body}
			
			\begin{body}¡El contacto directo con Cristo desencadena la chispa que hace saltar la fe!\end{body}
			
			\section{Homilías}
			
			\subsection{San Juan Pablo II, papa}
			
			\subsubsection{<a id="_idTextAnchor036"></a>Homilía (1979): Noche de la gran espera}
			
			\begin{referencia}Basílica de San Pedro. 14 de abril de 1979.\end{referencia}
			
			\begin{body}1. La palabra “muerte” se pronuncia con un nudo en la garganta. Aunque la humanidad, durante tantas generaciones, se haya acostumbrado de algún modo a la realidad inevitable de la muerte, sin embargo resulta siempre desconcertante. La muerte de Cristo había penetrado profundamente en los corazones de sus más allegados, en la conciencia de toda Jerusalén. El silencio que surgió después de ella llenó la tarde del viernes y todo el día siguiente del sábado. En este día, según las prescripciones de los judíos, nadie se había trasladado al lugar de la sepultura. Las tres mujeres, de las que habla el \textbf{Evangelio} de hoy, recuerdan muy bien la pesada piedra con que habían cerrado la entrada del sepulcro. \textit{Esta piedra}, en la que pensaban y de la que hablarían al día siguiente yendo al sepulcro, simboliza también el peso \textit{que había aplastado sus corazones}. La piedra que había separado al Muerto de los vivos, la piedra límite de la vida, el peso de la muerte. Las mujeres, que al amanecer del día después del sábado van al sepulcro, no hablarán de la muerte, sino de la piedra. Al llegar al sitio, comprobarán que la piedra no cierra ya la entrada del sepulcro. Ha sido derribada. No encontrarán a Jesús en el sepulcro. ¡Lo han buscado en vano! “No está aquí; ha resucitado, según lo había dicho” (\textit{Mt} 28, 6). Deben volver a la ciudad y anunciar a los discípulos que Él ha resucitado y que lo verán en Galilea. Las mujeres no son capaces de pronunciar una palabra. La noticia de la muerte se pronuncia en voz baja. Las palabras de la resurrección eran para ellas, desde luego, difíciles de comprender. \textit{Difíciles de repetir}, tanto ha influido la realidad de la muerte en el pensamiento y en el corazón del hombre.\end{body}
			
			\begin{body}2. Desde aquella noche y más aún desde la mañana siguiente, los discípulos de Cristo han aprendido a pronunciar la palabra “resurrección”. Y ha venido a ser la palabra más importante en su lenguaje, la palabra central, la palabra fundamental. Todo toma nuevamente origen de ella. Todo se confirma y se construye de nuevo: “La piedra que desecharon los arquitectos es \textit{ahora la piedra angular}. Es el Señor quien lo ha hecho, ha sido un milagro patente. Este es el día en que actuó el Señor. ¡Sea nuestra alegría y nuestro gozo!” (\textit{Sal} 117 [118], 22-24). Precisamente por esto la vigilia pascual –el día siguiente al Viernes Santo– no es ya sólo el día en que se pronuncia en voz baja la palabra “muerte”, en el que se recuerdan los últimos momentos de la vida del Muerto: \textit{es el día de una gran espera}. Es la Vigilia Pascual: el día y la noche de la espera del día que hizo el Señor.\end{body}
			
			\begin{body}El contenido litúrgico de la Vigilia se expresa mediante las distintas horas del breviario, para concentrarse después con toda su riqueza en esta liturgia de la noche, que alcanza su cumbre, después del período de Cuaresma, en el primer “Alleluia”. ¡\textit{Alleluia}: es el grito que expresa la alegría pascual! La exclamación que resuena todavía en la mitad de la noche de la espera y lleva ya consigo la alegría de la mañana. Lleva consigo la certeza de la resurrección. Lo que, en un primer momento, no han tenido la valentía de pronunciar ante el sepulcro los labios de las mujeres, o la boca de los Apóstoles, ahora la Iglesia, gracias a su testimonio, lo expresa con su Aleluya. Este canto de alegría, cantado casi a media noche, nos anuncia el Día Grande. (En algunas lenguas eslavas, la Pascua se llama la “Noche Grande”, después de la Noche Grande, llega el Día Grande: “Día hecho por el Señor”).\end{body}
			
			\begin{body}3. Y he aquí que estamos para ir al encuentro de este Día Grande con el fuego pascual encendido; en este fuego hemos encendido el cirio –luz de Cristo– y junto a él hemos proclamado la gloria de su resurrección en el canto del \textit{Exultet}. A continuación, hemos penetrado, mediante una serie de lecturas, en el gran proceso de la creación, del mundo, del hombre, del Pueblo de Dios; hemos penetrado en la preparación del conjunto de lo creado en este Día Grande, en el día de la victoria del bien sobre el mal, de la Vida sobre la muerte. ¡No se puede captar el misterio de la resurrección sino volviendo a los orígenes y siguiendo, después, todo el desarrollo de la historia de la economía salvífica hasta ese momento! El momento en que las tres mujeres de Jerusalén, que se detuvieron en el umbral del sepulcro vacío, oyeron el mensaje de un joven vestido de blanco: “No os asustéis. Buscáis a Jesús Nazareno, el crucificado; ha resucitado, no está aquí” (\textit{Mc} 16, 5-6).\end{body}
			
			\begin{body}4. Ese gran momento no nos consiente permanecer fuera de nosotros mismos; nos obliga a entrar en nuestra propia humanidad. Cristo no sólo nos ha revelado la victoria de la vida sobre la muerte, sino que nos ha traído con su resurrección la nueva vida. Nos ha dado esta nueva vida. He aquí cómo se expresa San Pablo: “¿O ignoráis que cuantos hemos sido bautizados en Cristo Jesús fuimos bautizados para participar en su muerte? Con Él hemos sido sepultados por el bautismo para participar en su muerte, para que como Él resucitó de entre los muertos por la gloria del Padre, así también nosotros vivamos una nueva vida” (\textit{Rom} 6, 3-4).\end{body}
			
			\begin{body}Las palabras “hemos sido bautizados en su muerte” dicen mucho. La muerte es el agua en la que se reconquista la vida: el agua “que salta hasta la vida eterna” (\textit{Jn} 4, 14). ¡Es necesario “sumergirse” en este agua; en esta muerte, \textit{para surgir después de ella como hombre nuevo}, como nueva criatura, como ser nuevo, esto es, \textit{vivificado por la potencia de la resurrección de Cristo!}\end{body}
			
			\begin{body}Este es el misterio del agua que esta noche bendecimos, que hacemos penetrar con la “luz de Cristo”, que hacemos penetrar con la nueva vida: ¡es el símbolo de la potencia de la resurrección! Este agua, en el sacramento del bautismo, se convierte en el signo \textit{de la victoria} sobre Satanás, sobre el pecado; el signo de la victoria que Cristo ha traído mediante la cruz, mediante la muerte y \textit{que nos trae} después \textit{a cada uno}: “Nuestro hombre viejo ha sido crucificado para que fuera destruido el cuerpo del pecado y ya no sirvamos al pecado” (\textit{Rom} 6, 6).\end{body}
			
			\begin{body}5. Es pues la noche de la gran espera. Esperemos en la fe, esperemos con todo nuestro ser humano a Aquel que al despuntar el alba ha roto la tiranía de la muerte, y ha revelado la potencia divina de la Vida: Él es nuestra esperanza.\end{body}
			
			\subsubsection{Homilía (<a id="_idTextAnchor037"></a>1982): Tumba vacía}
			
			\begin{referencia}Basílica de San Pedro, 10 de abril de 1982.\end{referencia}
			
			\begin{body} 1. En el centro del día, que acaba de terminar, hay una tumba. El sepulcro de Cristo. Este fue el día del Sábado Santo. Víspera de Pascua. En el centro del Viernes Santo está la Cruz de Cristo. En el centro del Sábado Santo la tumba de Cristo. Esta tumba tiene a las tres mujeres ante sus ojos: María de Magdala, María madre de Santiago y Salomé, cuando al amanecer del día siguiente, “el día después del sábado”, van al lugar del entierro de Cristo incluso antes de levantarse el sol. Su principal preocupación se expresa en estas palabras: “¿Quién nos removerá la piedra de la entrada del sepulcro?” (\textit{Mc} 16, 3). \end{body}
			
			\begin{body}El sepulcro: el lugar donde está enterrado Cristo, aquel cuyo cuerpo quieren embalsamar para protegerlo prontamente de la acción destructiva de la muerte. Y he aquí, el sepulcro está vacío. Las mujeres ven que la piedra ha sido quitada y entran en el sepulcro... Al amanecer del día después del sábado, cambia radicalmente el horizonte de los pensamientos y sentimientos de todos los que vieron la cruz de Cristo, su muerte y su entierro. De los que vieron el sepulcro con la piedra removida al frente. \end{body}
			
			\begin{body}La tumba vacía se coloca en medio de la noche siguiente y del amanecer del día siguiente al sábado. María de Magdala, María madre de Santiago y Salomé se asustaron al principio: “... estaban llenas de temblor y espanto” (\textit{Mc }16, 8). Estaban llenas de temblor y espanto, a pesar de lo que habían escuchado de labios del joven que habían encontrado en el sepulcro, vestido con una túnica blanca. A pesar, o quizás a causa de esto. El joven les había dicho: “Ha resucitado, no está aquí... irá delante de vosotros a Galilea” (\textit{Mc} 16, 6s). Sin embargo, no pudieron repetir esta noticia. “Y no decían nada a nadie, porque tenían miedo” (\textit{Mc }16, 8).\end{body}
			
			\begin{body}Esta es la primera imagen que la liturgia de la Vigilia Pascual, en su parte final, traza ante nosotros. \end{body}
			
			\begin{body}2. La segunda imagen proviene de San Pablo. A partir del día siguiente, el día después del sábado, los discípulos de Cristo se familiarizaron con esta nueva realidad: el sepulcro vacío. Comenzaron a llamarla por su nombre. Poco a poco también comprendieron que en la resurrección del Señor se cumplía todo lo que Él había hecho y lo que había enseñado. Por lo tanto el apóstol Pablo en la carta a los Romanos, hacia el año 57, es decir, 25 años después del acontecimiento de la Pascua, escribe: “... fuimos bautizados en Cristo Jesús, fuimos bautizados en su muerte... por lo tanto fuimos sepultados junto con él en la muerte, para que como Cristo resucitó de entre los muertos por la gloria del Padre, así también nosotros podamos caminar en una vida nueva” (\textit{Rom }6, 4). \end{body}
			
			\begin{body}Por tanto, para ellos: para la primera generación apostólica de los confesores de Cristo, y también para nosotros en el centro de la vigilia pascual está primero el “hombre viejo”, el hombre de pecado, que debe morir junto con Cristo, debe ser junto con él sepultado –para que el pecado muera en la muerte redentora de Cristo– y para que al amanecer del Domingo de Resurrección nazca el “hombre nuevo”. El hombre que vuelve a la vida por Cristo. \end{body}
			
			\begin{body}He aquí la analogía apostólica de “la tumba vacía”. “La tumba vacía” significa no solo la resurrección de Cristo. Significa una nueva vida, una vida en la Gracia. Significa “el hombre nuevo”. \end{body}
			
			\begin{body}Entonces, primero tenemos la Cruz en el centro del Viernes Santo. Y Pablo escribe: “Nuestro hombre viejo fue crucificado en Cristo para que... ya no fuéramos esclavos del pecado. De hecho, el que ha muerto ya está libre de pecado” (\textit{Rom }6, 6s). Posteriormente, en el centro del Sábado Santo, se coloca el sepulcro. Y Pablo escribe: “Hemos sido completamente unidos a Él en una muerte como la suya” (\textit{Rom }6, 5). El Sábado Santo es la víspera del Domingo de Resurrección. Al amanecer del domingo, las mujeres encuentran la tumba vacía. El Apóstol escribe (y estas palabras son como un clamor rotundo de fe y esperanza): “Cristo resucitado de entre los muertos ya no muere; la muerte ya no tiene poder sobre él” (\textit{Rom} 6, 9). Así, también vosotros, consideraos “muertos al pecado, pero vivos para Dios en Cristo Jesús” (\textit{Rom} 6, 11). Esta es la segunda imagen de la liturgia de la Vigilia. \end{body}
			
			\begin{body}3. Acojamos el silencio de las mujeres asustadas al ver el sepulcro vacío, al amanecer del día siguiente al sábado. Acojamos este grito del Apóstol de la carta a los Romanos. [Acojedlo especialmente vosotros, queridos hermanos y hermanas, que durante esta noche de Vigilia recibís de Cristo la nueva vida en el sacramento del Bautismo.] Acojámoslo cada uno de nosotros, que hemos recibido esta nueva vida. Que lo acojan todos aquellos que se han renovado mediante el sacramento de la Penitencia. Cristo se ha convertido en la piedra angular del nuevo edificio en todos nosotros. \end{body}
			
			\begin{body}4. Entonces, mientras todo está aún velado por la noche de Pascua, elevemos nuestro corazón hacia la Vida Nueva: “Es el Señor quien lo ha hecho, ha sido un milagro patente” (\textit{Sal }117 [118], 23). Y junto al salmista damos gracias: “Dad gracias al Señor porque es bueno, porque es eterna su misericordia. Diga la casa de Israel: eterna es su misericordia… La diestra del Señor es poderosa, la diestra del Señor es excelsa” (\textit{Sal} 117 [118], 1-2. 16). \end{body}
			
			\begin{body}Esta noche de la Vigilia proclama el cumplimiento del misterio pascual: al centro del Viernes Santo está la Cruz, al centro del Sábado Santo el sepulcro de Cristo y al amanecer de la noche de la Vigilia se revela el poder de la diestra del Señor. El sepulcro vacío da testimonio de la resurrección de Cristo: estaremos injertados en Él “... también en la resurrección” (\textit{Rm} 6, 5). \end{body}
			
			\begin{body}Vosotros, queridos neófitos, todos nosotros, queridos hermanos y hermanas, participando en esta Eucaristía, renovamos esta certeza de fe, confesada por los labios del salmista: “No he de morir, viviré, para contar las hazañas del Señor” (\textit{Sal} 117 [118], 17).\end{body}
			
			\subsubsection{Homilía (1985): Poder creador y poder salvador}
			
			\begin{referencia}6 de abril de 1985.\end{referencia}
			
			\begin{body} 1. \textit{“O vere beata nox!” }– ¡Oh noche verdaderamente di<a id="_idTextAnchor038"></a>chosa! Así canta la Iglesia durante la vigilia pascual, velando junto al sepulcro de Cristo. En esta tumba fue puesto su cuerpo torturado, bajado de la cruz con rapidez con “motivo de la fiesta de la Pascua”. Todavía era la Pascua del antiguo pacto. \end{body}
			
			\begin{body}2. \textit{“O vere beata nox!”} – ¡Oh noche verdaderamente dichosa! Así canta la Iglesia durante esta vigilia, que precede a la Pascua de la nueva alianza. En toda la superficie de la tierra la Iglesia está reunida en vigilia para adorar \textit{el poder del Altísimo}: “La diestra del Señor es poderosa, la diestra del Señor es excelsa” (\textit{Sal} 118, 16). Es el mismo \textit{poder} que se reveló al principio \textit{en la creación} del mundo. Dios dijo: “¡Hágase!”; y así “creó los cielos y la tierra” (\textit{Gen} 1, 1). Es el mismo poder que se manifestó en la liberación de Israel de Egipto, el poder que condujo al Pueblo Elegido a través del Mar Rojo, salvándolo de las manos del Faraón. \end{body}
			
			\begin{body}La Iglesia, reunida en vigilia junto al sepulcro de Cristo, medita sobre los acontecimientos en los que se manifestó el poder de Dios: \end{body}
			
			\begin{body}- el poder creador, \end{body}
			
			\begin{body}- el poder salvador. \end{body}
			
			\begin{body}3. \textit{“O vere beata nox!”. }Es verdaderamente dichosa esta noche en la que brilla de nuevo la luz de Cristo y en la que la vida vencerá a la muerte. Por eso, desde el fondo del mismo salmo nos habla el que “estaba muerto”: “No he de morir, viviré, para contar las hazañas del Señor... La diestra del Señor es poderosa, la diestra del Señor es excelsa”. \end{body}
			
			\begin{body}4. Estamos reunidos aquí, en vigilia, \textit{para experimentar este poder divino a través de la fe y el amor. }Estamos aquí para recibir su revelación, al igual que \textit{aquellas tres mujeres} que, temprano en la mañana, cuando aún estaba oscuro, fueron las primeras en ir al sepulcro y lo encontraron vacío. También como \textit{los apóstoles}, quienes luego fueron corriendo al sepulcro. \end{body}
			
			\begin{body}En cuanto a las mujeres, el Evangelio dice que estaban “inseguras”, “asustadas” (cf. \textit{Lc} 24, 3-4), que “tenían miedo” (\textit{Mc} 16, 5). Encontraron un misterio que \textit{sobrepasa al hombre y lo asusta: “mysterium tremendum”. }\end{body}
			
			\begin{body}5. Del miedo pasamos al \textit{asombro} (\textit{“mysterium fascinosum”}), a la admiración adoradora. Y la Iglesia reunida junto al sepulcro de Cristo, en el corazón de esta noche bendita, ve incluso el pecado con una nueva luz, porque se atreve a cantar: “O felix culpa, quae talem ac tantum meruit habere \textit{Redemptorem}” – “¡Oh feliz culpa, que mereció tan grande Redentor! ”. Verdaderamente lo que sucedió esa noche es “obra del Señor: un milagro patente” (\textit{Sal} 118, 23). \end{body}
			
			\begin{body}6. \begin{bodysmall}[Invitamos de manera especial a acoger esta revelación del poder divino –del poder creador y salvador–, a \end{bodysmall}\textit{vosotros}\begin{bodysmall} , queridos hermanos y hermanas, que durante esta liturgia de la Vigilia Pascual \end{bodysmall}\textit{recibiréis el Bautismo}.\begin{bodysmall} La Iglesia os recibe con gran alegría, a vosotros que queréis de modo sacramental sumergiros con Cristo \end{bodysmall}\textit{en su muerte}\begin{bodysmall} para resucitar junto \end{bodysmall}\textit{con él a una nueva vida}\begin{bodysmall}. El obispo de Roma os saluda cordialmente invitándoos a las fuentes de la salvación. Sois un grupo de veinticuatro personas, todos jóvenes, de una docena de naciones y de varios continentes. Sois una señal de cómo Cristo llama a sus discípulos de todos los pueblos y naciones de la tierra. Mostráis la universalidad del mensaje del Evangelio. A través de vosotros, nuestra vigilia pascual se \end{bodysmall}\textit{convierte en un signo particularmente }\begin{bodysmall}elocuente.] \end{bodysmall}\end{body}
			
			\begin{body}El misterio pascual de nuestro Señor Jesucristo \textit{está siempre presente }en el sacramento de la Iglesia. El poder de la muerte y la resurrección no deja de actuar en las almas de los hombres. \end{body}
			
			\begin{body}7. Así, por obra del poder divino mismo:\end{body}
			
			\begin{body}- poder creador, \end{body}
			
			\begin{body}- poder salvador, \end{body}
			
			\begin{body}\textit{la Iglesia nace a la vida} en la resurrección del Señor crucificado: “La piedra desechada por los constructores se \textit{ha convertido en la piedra angular}” (\textit{Sal} 118, 22). De ella nacemos todos: como piedras vivas, penetradas por el aliento vivificante de esta noche de Pascua; del aliento de la resurrección de Cristo. “Estamos muertos al pecado, pero vivimos para Dios en Cristo Jesús” (cf. \textit{Rm} 6, 11; \textit{Col} 2, 13).\end{body}
			
			\subsubsection{Homilía (1988): Luz que vence las tinieblas}
			
			\begin{referencia}3 de abril de 1988.\end{referencia}
			
			\begin{body} 1. “Lumen Christi!” En la oscuridad que se extiende por todo el espacio de esta [Basílica de San Pedro], las pala<a id="_idTextAnchor039"></a>bras [del diácono] resuenan tres veces como un anuncio profético de la vigilia pascual: “Lumen Christi!” Poco a poco se iluminó el espacio exterior para expresar lo que ha traído esta noche después del sábado, de cara al amanecer del día. Todos entramos en esta noche, todavía trastornados por los acontecimientos de ayer, por la muerte de Jesús de Nazaret y por su sepultura no lejos de la cruz del Calvario. Caminamos, como aquellos dos discípulos en el camino de Jerusalén a Emaús (cf. \textit{Lc} 24, 13ss). \end{body}
			
			\begin{body}2. Y he aquí que la Iglesia se nos acerca –como el desconocido que se acerca a los discípulos, caminando con ellos hacia Emaús– y despliega ante nosotros, en una serie de lecturas, su inspirada “pedagogía”. Muestra el designio eterno de Dios que se desarrolla a lo largo de la historia del hombre, a partir de la creación, a través de la vocación de Abraham y, más tarde, del pueblo que descendió de él. Los patriarcas y los profetas hablan, los acontecimientos hablan, todos juntos conducen finalmente al acontecimiento de esta noche de Pascua: “Lumen Christi!” \end{body}
			
			\begin{body}3. Esta es la luz que ilumina todo el pasado, revela el sentido profundo de todos los libros del Antiguo Testamento y de todas las lecturas de esta liturgia. La luz de Cristo caminaba delante del hombre desde el comienzo de su historia terrena. ¡Desde la creación, desde el “árbol del conocimiento del bien y del mal”, desde la tentación y el pecado... caminaba esta luz delante de él! \end{body}
			
			\begin{body}Su poder es tan grande que la Iglesia no duda en la liturgia de esta noche de vigilia para exclamar: “O felix culpa, quae talem ac tantum meruit habere Redemptorem”. ¡Oh feliz culpa! Oímos estas palabras en el anuncio pascual del “Exsultet”, cantado por el diácono. De hecho, esta noche de vigilia nos invita a la mayor alegría, a la alegría de la Pascua de Cristo: “Lumen Christi!”. Tal es el poder de esta luz, que es capaz de transformar la oscuridad, tanto exterior como interior, en día: “haec est Dies” – ¡El Día hecho por el Señor! Con el poder de la Pascua de Cristo, donde “abundó el pecado”, es decir, la muerte; la gracia, es decir, la vida, pueden sobreabundar más (cf. \textit{Rm }5, 20). \end{body}
			
			\begin{body}4. Entonces, antes de que las tres mujeres, de las que habla el Evangelio de esta vigilia pascual, encuentren la piedra removida en el lugar del sepulcro de Cristo, la Iglesia desciende con nosotros a las profundidades de esta muerte, que ha provocado tal sobreabundancia de vida. \end{body}
			
			\begin{body} “O Mors, ero mors tua!” – “¡Oh Muerte, yo seré tu muerte!”. Siguiendo las palabras del Apóstol en la carta a los Romanos, descendemos a la historia del pecado humano hasta su primer comienzo. \end{body}
			
			\begin{body}Con el pecado entró la muerte en el mundo (cf. \textit{Rm }5, 12). Y por eso durante esta noche de vigilia somos bautizados en la muerte de Cristo. Junto con él somos sepultados en la muerte, para que podamos caminar en una vida nueva, como Cristo (cf. \textit{Rm} 6, 4). De hecho, Cristo ha resucitado. “Muerto al pecado... vive para Dios” (cf. \textit{Rm }6, 10). Nuestro hombre viejo, el hombre de pecado, debe “ser crucificado con Él”, con Cristo, para que nosotros, participando en su muerte, en su muerte redentora, seamos liberados del pecado. \end{body}
			
			\begin{body}5. \begin{bodysmall}[Queridos hermanos y hermanas, que durante esta vigilia pascual recibiréis el Bautismo que os sumerge en la muerte de Cristo, toda la Iglesia y el Pueblo de Dios que llenan esta venerada Basílica de San Pedro os saludan, ya que estáis a punto de recibir nueva vida en Cristo. En ustedes deseo dirigir mi respetuoso saludo a sus respectivos países de donde proceden: Corea, Alemania, Japón, India, Indonesia, Islas de Cabo Verde, Italia, Perú, Estados Unidos de América, Hungría y Vietnam. Viniendo de diferentes partes del mundo, reflejáis la universalidad de la Iglesia, el ámbito universal de la redención. Vuestro nacimiento a través del Bautismo a una nueva vida en Cristo es para todos nosotros una fuente particular de alegría pascual.] \end{bodysmall}\end{body}
			
			\begin{body}“Dad gracias al Señor porque es bueno; porque es eterna su misericordia” (\textit{Sal} 118 [117], 1). \end{body}
			
			\begin{body}6. Todos llevaremos en la mano, [junto a vosotros,] un cirio de Pascua encendido. Es un testimonio de nuestro bautismo, de nuestra fe, esperanza y caridad. Es testigo de esta noche de vigilia, en la que la Iglesia no duda en cantar “O felix culpa, quae talem ac tantum meruit habere Redemptorem”. \end{body}
			
			\begin{body}He aquí que avanzan las horas de esta vigilia nocturna. Pronto llegará el amanecer. A las tres mujeres, habiendo encontrado la tumba de Cristo vacía, y la piedra removida, se les dirá: “Ha resucitado, no está aquí… id, anunciadlo a sus discípulos y a Pedro” (\textit{Mc} 16, 6-7). \end{body}
			
			\begin{body}“Haec est Dies, quam fecit Dominus” – “Este es el Día que hizo el Señor”. \end{body}
			
			\begin{body}Todos entraremos en este día de la Pascua de Cristo, y las velas encendidas en la noche de vigilia darán testimonio del final de nuestros días terrenales: \end{body}
			
			\begin{body}“Lumen Christi”! \end{body}
			
			\begin{body}¡Sí, Cristo es la luz! Amén.\end{body}
			
			\subsubsection{Homilía (1991): Cielo nuevo y tierra nueva}
			
			\begin{referencia}30 de marzo de 1991.\end{referencia}
			
			\begin{body}“\textit{En el principio, Dios creó los cielos y la tierra}. La tierra estaba informe y vacía y las tinieblas cubrían el abismo y el espíritu de Dios se cernía sobre las aguas”. “\textit{El espíritu de Dios se cernía sobre las aguas}” (\textit{Gen} 1, 1-2). \end{body}
			
			\begin{body}1. En esta noche, durante la Vigilia Pas<a id="_idTextAnchor040"></a>cual, el Sagrado Triduo nos hace volver al origen de la obra divina de la creación. La oscuridad se extendió hasta la basílica antes de que el grito del salmo resonara en las lecturas a medida que se desarrollaba la liturgia: “\textit{Envías tu espíritu, son creados y renuevas la faz de la tierra}” (\textit{Sal} 104, 30). \end{body}
			
			\begin{body}\textit{Envía tu Espíritu, Señor, para renovar la faz de la tierra}. Retornamos al principio de la creación y al mismo tiempo experimentamos profundamente la \textit{noche }que cayó sobre Jerusalén, \textit{después de que Cristo}, bajado de la Cruz, fuera sepultado en la tumba. La muerte del Dios-Hombre inició la nueva creación. Cristo acogió la muerte para renovar el mundo. \textit{En su muerte, el grito por el advenimiento del Espíritu que da vida} adquirió una fuerza definitiva y eficaz. \end{body}
			
			\begin{body}2. \begin{bodysmall}[Ayer, después del “Vía Crucis”, se proclamaron las palabras de la \end{bodysmall}\textit{Carta a los Hebreos}\begin{bodysmall}: “¿Cuánto más la sangre de Cristo, que con un Espíritu eterno se ofreció sin tacha a Dios, purificará nuestra conciencia de las obras muertas para servir al Dios vivo?” (\end{bodysmall}\textit{Heb}\begin{bodysmall} 9, 14).] \end{bodysmall}\end{body}
			
			\begin{body}\textit{El sacrificio de la Cruz} es obra mesiánica de Cristo, es el cumplimiento total de la Redención. Cristo lo realiza no sólo “con” un Espíritu eterno, sino también en este sacrificio en el que “\textit{recibe} “ \textit{el Espíritu Santo para “dárselo”} a los Apóstoles, a la Iglesia, a la humanidad. Una vez resucitado, Jesús se presentará a los Apóstoles reunidos en el Cenáculo y, soplando sobre ellos les dirá: “Recibid el Espíritu Santo; a quienes perdonéis los pecados, les serán perdonados” (\textit{Jn} 20, 22-23). \end{body}
			
			\begin{body}\textit{El Espíritu Santo renueva la faz de la tierra, recurriendo al poder de la Cruz de Cristo}, recurriendo a los recursos infinitos de la Redención del mundo. Renueva todo en el hombre, en su corazón y en su conciencia. \textit{Todo se renueva a través del Amor}, que, precisamente en esta noche de Pascua, se revela más poderoso que la muerte y el pecado, que es la muerte del alma. \end{body}
			
			\begin{body}3. Por eso, \textit{la Vigilia Pascual}, desde los primeros tiempos del cristianismo, ha sido para los catecúmenos \textit{la gran hora del Bautismo}. \end{body}
			
			\begin{body}\begin{bodysmall}[Ahora también somos testigos y participantes de ello en esta Basílica de San Pedro, donde el Obispo de Roma saluda con alegría a \end{bodysmall}\textit{nuestros nuevos hermanos y hermanas en la fe}\begin{bodysmall} que están a punto de recibir el Santo Bautismo. Vosotros venís de Japón, Corea, Vietnam, China, Tailandia, Indonesia, Estados Unidos, Chile, Inglaterra e Italia.]\end{bodysmall}\end{body}
			
			\begin{body}Esta es la fe en Cristo que “resucitado de entre los muertos no muere más; la muerte ya no tiene poder sobre Él” (\textit{Rom} 6, 9). \end{body}
			
			\begin{body}4. He aquí que el grito del salmista se realiza en vosotros, hermanos y hermanas: “\textit{Envía, Señor, tu Espíritu para renovar la faz de la tierra}”... el Espíritu que desde el principio participó íntimamente en la obra de la creación. El Espíritu que flotaba sobre las aguas \textit{regenera}, en el sacramento del Bautismo, al \textit{hombre “por el agua”} que ha recibido la fuerza del Espíritu vivificante. \end{body}
			
			\begin{body}Regeneración “en el agua y en el Espíritu” (cf. \textit{Jn} 3, 5), primer sacramento de la Pascua de Cristo. Junto a este Sacramento, cuando aparece el Día “hecho por el Señor”, la Iglesia espera para todos vosotros \textit{el cumplimiento de la promesa del profeta Ezequiel}: que el Señor os dé un corazón nuevo y os dé un espíritu nuevo... que os haga vivir según los preceptos de Dios y os haga observar y poner en práctica sus leyes. Que seáis \textit{el pueblo de Dios y el Señor sea vuestro Dios} (cf. \textit{Ez} 36, 26-28). \end{body}
			
			\begin{body}Finalmente, que podáis vivir eternamente en la tierra de los vivos (cf. \textit{Ap} 21, 1). La vigilia de Pascua es un anticipo de esta tierra y de esta morada. En efecto, es el comienzo del cielo nuevo y de la tierra nueva donde Dios será “todo en todos” (cf. \textit{1 Co 15, 28} ).\end{body}
			
			\subsubsection{Homilía (1994): Alegría de la Pascua}
			
			\begin{referencia}Basílica de San Pedro, 2 de abril de 1994.\end{referencia}
			
			\begin{body} 1. “\textit{¡No tengáis miedo!}” (\textit{Mc} 16, 6). \end{body}
			
			\begin{body}María Magdalena, María madre de Santiago y Salomé escuchan estas palabras a la entrada del sepulcro en el que fue depositado el cuerpo de Jesús. Se dan cuenta de que la piedra del sepulcro había sido removida y que el sepulcro estaba vacío. Son invadidas por el miedo y el aso<a id="_idTextAnchor041"></a>mbro. Asombro que crece al escuchar las palabras desde el fondo del sepulcro: “¿Buscáis a Jesús Nazareno, el Crucificado? No está aquí, ha resucitado. Aquí está el lugar donde lo habían puesto. Ahora \textit{id, decid a sus discípulos y a Pedro }que Él va delante de vosotros a Galilea. Allí lo veréis, como os dijo” (\textit{Mc} 16, 6-7). Las mujeres se sorprenden y huyen de la tumba, temerosas de contarle a alguien lo que han visto. \end{body}
			
			\begin{body}2. Este es precisamente el momento del misterio pascual, al que nos acercamos participando en la solemne vigilia de la noche de Pascua. El acontecimiento, descrito por el evangelista Marcos, es simple y, al mismo tiempo, impactante. \end{body}
			
			\begin{body}Por eso \textit{la liturgia de la vigilia pascual se refiere a las fuerzas de la naturaleza}. En esta noche debemos llamarlas de nuevo, porque reaccionaron precisamente en aquel momento. La tierra se movió y tembló cuando Cristo dejó la tumba. Un terremoto sacudió la roca que obstruía el sepulcro (cf. \textit{Mt} 28, 2). \end{body}
			
			\begin{body}En esta noche la liturgia se vuelve \textit{fuego}, que posee un poder misterioso, un poder bendito, pero también un poder que destruye. El fuego consume y devora lo que encuentra a su paso, pero también puede ser una fuerza beneficiosa para los hombres. De hecho, las extremidades del cuerpo humano necesitan fuego para calentarse. También el fuego ilumina, ahuyentando las tinieblas y en esta noche la Iglesia lo enciende para extraer de él la luz que, más tarde, acompaña a la asamblea litúrgica en el templo con el canto: “Lumen Christi”. La luz de la llama se convierte en símbolo de la Resurrección. La liturgia de esta noche reserva el mayor espacio para el\textit{ poder del agua}. El agua también puede ser un signo de muerte. Según San Pablo es un símbolo de la muerte de Cristo (cf. \textit{Rm} 6, 3-4) y para pasar por esta muerte es necesario sumergirse en el agua. Inmersión en la muerte de Cristo que sirve no solo para ser lavado, sino, más aún, para ser vivificado. El agua que brota de la fuente es un alivio para el cuerpo cansado, restaura su fuerza; por eso el \textit{agua se ha convertido en el signo sacramental del renacimiento a través del bautismo}. Con este sacramento la Iglesia participa hoy de la Resurrección de Cristo. \end{body}
			
			\begin{body}\begin{bodysmall}[A través del bautismo vosotros, hermanos y hermanas, que recibiréis este sacramento esta noche, participáis de la Resurrección de Cristo.\end{bodysmall}\textit{ El obispo de Roma os saluda cordialmente }\begin{bodysmall}mientras os preparáis para entrar en la nueva vida. Saluda a las naciones de donde venís: Corea, Filipinas, Japón, Guatemala, Hong Kong, Italia, Perú, Portugal, Eslovaquia, España y Suiza.]\end{bodysmall}\end{body}
			
			\begin{body}3. La nueva vida es siempre fuente de alegría. Sentimos la alegría de la Iglesia en las palabras cantadas por el diácono hace un momento. La primera palabra del anuncio de Pascua es “\textit{Exsultet}”: una llamada a la alegría. \end{body}
			
			\begin{body}El gozo de esta noche es mayor que el temor de las mujeres de Jerusalén: es el gozo de la victoria sobre la muerte y el pecado. La Iglesia no duda en cantar: “Feliz culpa”; feliz porque ha encontrado al Redentor en esta noche; porque en su muerte la ha vencido. Cristo ha resucitado dando vida a todos los descendientes de Adán. \end{body}
			
			\begin{body}4. Entonces la Iglesia, ya ahora, durante esta admirable vigilia pascual, nos invita a todos a la alegría. Alegrémonos porque en Cristo la vida es más fuerte que la muerte y la salvación es más fuerte que el pecado.\end{body}
			
			\begin{body}“Annuntio vobis gaudium magnum, quod est - ¡Aleluya!”. \end{body}
			
			\begin{body}¡Sed testigos en el mundo de hoy de la alegría de la Pascua!\end{body}
			
			\subsubsection{Homilía (1997): Fuego y agua}
			
			\begin{referencia}29 de marzo de 1997.\end{referencia}
			
			\begin{body}1. ¡Que exista la luz! (\textit{Gn} 1,3)\end{body}
			
			\begin{body}Durante la Vigilia pascual, la litu<a id="_idTextAnchor042"></a>rgia proclama estas palabras del \textbf{Libro del Génesis}, las cuales son un elocuente motivo central de esta admirable celebración. Al empezar se bendice el “fuego nuevo”, y con él se enciende el cirio pascual, que es llevado en procesión hacia el altar. El cirio entra y avanza primero en la oscuridad, hasta el momento en que, después de cantar el tercer “\textit{Lumen Christi}”, se ilumina toda la Basílica.\end{body}
			
			\begin{body}De este modo están unidos entre sí \textit{los elementos de las tinieblas y de la luz, de la muerte y de la vida}. Con este fondo resuena la narración bíblica de la creación. Dios dice: “Que exista la luz”. \textit{Se trata, en cierto modo, del primer paso hacia la vid}a. En esta noche debe realizarse el singular paso de la muerte a la vida, y el rito de la luz, acompañado por las palabras del Génesis, ofrece el primer anuncio.\end{body}
			
			\begin{body}2. En el Prólogo de su Evangelio, san Juan dice que el Verbo se hizo carne: “En la Palabra había vida, y\textit{ la vida era la luz de los hombres}” (\textit{Jn} 1, 4). Esta noche santa se convierte pues en una extraordinaria manifestación de aquella vida que es la luz de los hombres. En esta manifestación participa toda la Iglesia y, de modo especial, los \textit{catecúmenos}, que durante esta Vigilia reciben el Bautismo.\end{body}
			
			\begin{body}\begin{bodysmall}[La Basílica de san Pedro en esta solemne celebración os acoge a vosotros, \end{bodysmall}\textit{amadísimos hermanos y hermanas, que dentro de poco seréis bautizados en Cristo nuestra Pascua}\begin{bodysmall}. Dos de vosotros provienen de Albania y dos del Zaire, Países que están viviendo horas dramáticas de su historia. ¡Que el Señor se digne escuchar el grito de los pobres y guiarlos en el camino hacia la paz y la libertad! Otros proceden de Benin, Cabo Verde, China y Taiwán. Ruego por cada uno de vosotros y de vosotras que, en esta asamblea representáis las primicias de la nueva humanidad redimida por Cristo, para que seáis siempre fieles testigos de su Evangelio.]\end{bodysmall}\end{body}
			
			\begin{body}Las lecturas litúrgicas de la Vigilia pascual unen entre sí \textit{los dos elementos del fuego y del agua}. El elemento fuego, que da la luz, y el elemento agua, que es la materia del sacramento del renacer, es decir, del santo Bautismo. “El que no nazca de agua y de Espíritu, no puede entrar en el Reino de Dios” (\textit{Jn} 3, 5). El paso de los Israelitas a través del Mar Rojo, es decir, la liberación de la esclavitud de Egipto, es figura y casi anticipación del Bautismo que libera de la esclavitud del pecado.\end{body}
			
			\begin{body}3. Los múltiples motivos que en esta liturgia de la Vigilia de Pascua encuentran su expresión en las Lecturas bíblicas, convergen y se interrelacionan así en una imagen unitaria. Del modo más completo es el apóstol Pablo quien presenta esta verdad en la \textbf{Carta a los Romanos}, proclamada hace poco: “Los que por el bautismo nos incorporamos a Cristo, fuimos incorporados a su muerte. Por el bautismo fuimos sepultados con él en la muerte, para que, así como Cristo fue despertado de entre los muertos por la gloria del Padre, así también nosotros andemos en una vida nueva” (\textit{Rm} 6, 3-4).\end{body}
			
			\begin{body}Estas palabras nos llevan \textit{al centro mismo de la verdad cristiana}. La muerte de Cristo, la muerte redentora, es el comienzo del paso a la vida, manifestado en la resurrección. “Si hemos muerto con Cristo –prosigue san Pablo–, creemos que también viviremos con él, pues sabemos que Cristo, una vez resucitado de entre los muertos, ya no muere más; la muerte ya no tiene dominio sobre él” (\textit{Rm} 6, 8-9).\end{body}
			
			\begin{body}4. Al llevar en las manos la antorcha de la Palabra de Dios, la Iglesia que celebra la Vigilia pascual se detiene como ante un último umbral. Se detiene en gran espera, durante toda esta noche. Junto al sepulcro esperamos el acontecimiento sucedido hace [dos mil años]. Primeros testigos de este suceso extraordinario fueron las mujeres de Jerusalén. Ellas llegaron al lugar donde Jesús había sido depositado el Viernes Santo y encontraron la tumba vacía. Una voz les sorprendió: “¿Buscáis a Jesús el Nazareno, el crucificado? No está aquí. Ha resucitado. Mirad el sitio donde lo pusieron. Ahora id a decir a sus discípulos y a Pedro: Él va por delante de vosotros a Galilea. Allí lo veréis, como os dijo” (\textit{Mc} 16, 6-7).\end{body}
			
			\begin{body}Nadie vio con sus propios ojos la resurrección de Cristo. Las mujeres, llegadas a la tumba, fueron las primeras en constatar el acontecimiento ya sucedido.\end{body}
			
			\begin{body}La Iglesia, congregada por la Vigilia pascual, escucha nuevamente, en silenciosa espera, este testimonio y manifiesta después su gran alegría. [La hemos escuchado anunciar hace poco por el diácono. \textit{“Annuntio vobis gaudium magnum…}”\textit{, “Os anuncio una gran alegría, ¡Aleluya!}”.]\end{body}
			
			\begin{body}Acojamos con corazón abierto este anuncio y participemos juntos en la gran alegría de la Iglesia.\end{body}
			
			\begin{body}¡Cristo ha resucitado verdaderamente! ¡Aleluya!\end{body}
			
			\subsubsection{Homilía (2000): La propia historia de salvación}
			
			\begin{referencia}22 de abril del 2000.\end{referencia}
			
			\begin{body}1. “\textit{Tenéis guardias. Id, aseguradlo como sabéis}” (\textit{Mt} 27, 65).\end{body}
			
			\begin{body}La tumba de Jesús fue cerrada y sellada<a id="_idTextAnchor043"></a>. Según la petición de los sumos sacerdotes y los fariseos, se pusieron soldados de guardia para que nadie pudiera robarlo (\textit{Mt} 27, 62-64). Este es el acontecimiento del que parte la liturgia de la Vigilia Pascual.\end{body}
			
			\begin{body}Vigilaban junto al sepulcro aquellos que habían querido la muerte de Cristo, considerándolo un “impostor” (\textit{Mt }27, 63). Su deseo era que Él y su mensaje fueran enterrados para siempre.\end{body}
			
			\begin{body}Velan, no muy lejos de allí, María y, con ella, los Apóstoles y algunas mujeres. Tenían aún impresa en el corazón la imagen perturbadora de hechos que acaban de ocurrir.\end{body}
			
			\begin{body}2. Vela la Iglesia, esta noche, en todos los rincones de la tierra, y revive las etapas fundamentales de la historia de la salvación. La solemne liturgia que estamos celebrando es una expresión de este “vigilar” que, en cierto modo, recuerda el mismo de Dios, al que se refiere el \textbf{Libro del Éxodo}: “Noche de guardia fue ésta para Yahveh, para sacarlos de la tierra de Egipto. Esta misma noche será la noche de guardia en honor de Yahveh…, por todas sus generaciones” (\textit{Ex} 12, 42).\end{body}
			
			\begin{body}En su amor providente y fiel, que supera el tiempo y el espacio, Dios vela sobre el mundo. Canta el salmista: “Yahveh es tu guardián, tu sombra, Yahveh, a tu diestra. De día el sol no te hará daño, ni la luna de noche. Te guarda Yahveh de todo mal, él guarda tu alma; … desde ahora y por siempre” (\textit{Sal} 120, 4-5. 8).\end{body}
			
			\begin{body}También el pasaje que estamos viviendo entre el segundo y el tercer milenio está guardado en el misterio del Padre. Él “obra siempre” (\textit{Jn} 5, 7) por la salvación del mundo y, mediante el Hijo hecho hombre, guía a su pueblo de la esclavitud a la libertad. Toda la “obra” \begin{bodysmall}[del Gran Jubileo del año 2000]\end{bodysmall} está, por decirlo así, inscrita en esta noche de Vigilia, que lleva a cumplimiento aquella del Nacimiento del Señor. Belén y el Calvario remiten al mismo misterio de amor de Dios, que tanto amó al mundo “que dio a su Hijo único, para que todo el que crea en él no perezca, sino que tenga vida eterna” (\textit{Jn} 3, 16).\end{body}
			
			\begin{body}3. En esta Noche, la Iglesia, en su velar, se centra sobre los textos de la Escritura, que trazan el designio divino de salvación desde el Génesis al Evangelio y que, gracias también a los ritos del agua y del fuego, confieren a esta singular celebración una dimensión cósmica. Todo el universo creado está llamado a velar en esta noche junto al sepulcro de Cristo. Pasa ante nuestros ojos la historia de la salvación, desde la creación a la redención, desde el éxodo a la Alianza en el Sinaí, de la antigua a la nueva y eterna Alianza. En esta noche santa se cumple el proyecto eterno de Dios que arrolla la historia del hombre y del cosmos.\end{body}
			
			\begin{body}4. En la vigilia pascual, madre de todas las vigilias, cada hombre puede reconocer también la propia historia de salvación, que tiene su punto fundamental en el renacer en Cristo mediante el Bautismo.\end{body}
			
			\begin{body}\begin{bodysmall}[Esta es, de manera muy especial, vuestra experiencia, queridos Hermanos y Hermanas que dentro de poco recibiréis los sacramentos de la iniciación cristiana: el Bautismo, la Confirmación y la Eucaristía. Venís de diversos Países del mundo: Japón, China, Camerún, Albania e Italia. La variedad de vuestras naciones de origen pone de relieve la universalidad de la salvación traída por Cristo. Dentro de poco, queridos, seréis insertos íntimamente en el misterio de amor de Dios, Padre, Hijo y Espíritu Santo. Que vuestra existencia se haga un canto de alabanza a la Santísima Trinidad y un testimonio de amor que no conozca fronteras.]\end{bodysmall}\end{body}
			
			\begin{body}5. “\textit{Ecce lignum Crucis, in quo salus mundi pependit: venite adoremus}!” Esto ha cantado ayer la Iglesia, mostrando el árbol la Cruz, “donde estuvo clavada la salvación del mundo”. “Fue crucificado, muerto y sepultado”, recitamos en el Credo.\end{body}
			
			\begin{body}El sepulcro. El lugar donde lo habían puesto (cf. \textit{Mc} 16, 6). Allí está espiritualmente presente toda la Comunidad eclesial de cada rincón de la tierra. Estamos también nosotros con las tres mujeres que se acercan al sepulcro, antes del alba, para ungir el cuerpo sin vida de Jesús (cf. \textit{Mc} 16, 1). Su diligencia es nuestra diligencia. Con ellas descubrimos que la piedra sepulcral ha sido retirada y el cuerpo ya no está allí. “No está aquí”, anuncia el Ángel, mostrando el sepulcro vacío y las vendas por tierra. La muerte ya no tiene poder sobre Él (cf. \textit{Rm} 6, 9).\end{body}
			
			\begin{body}¡Cristo ha resucitado! Anuncia al final de esta noche de Pascua la Iglesia, que ayer había proclamado la muerte de Cristo en la Cruz. Es un anuncio de verdad y de vida.\end{body}
			
			\begin{body}“\textit{Surrexit Dominus de sepulcro, qui pro nobis pependit in ligno. Alleluia}!”\end{body}
			
			\begin{body}Ha resucitado del sepulcro el Señor, que por nosotros fue colgado a la cruz.\end{body}
			
			\begin{body}Sí, Cristo ha resucitado verdaderamente y nosotros somos testigos de ello.\end{body}
			
			\begin{body}Lo gritamos al mundo, para que la alegría que nos embarga llegue a tantos otros corazones, encendiendo en ellos la luz de la esperanza que no defrauda.\end{body}
			
			\begin{body}¡Cristo ha resucitado, aleluya!\end{body}
			
			\subsubsection{Homilía (2003): Todo vuelve a empezar}
			
			\begin{referencia}19 de abril de 2003.\end{referencia}
			
			\begin{body}1. “\textit{No os asustéis. ¿Buscáis a Jesús el Nazareno, el crucificado? No está aquí. Ha resucitado}” (\textit{Mc} 16, 6). Al alba del primer día después del sábado, como narra el Evangelio, algunas mujeres van al sepulcro para embalsa<a id="_idTextAnchor044"></a>mar el cuerpo de Jesús que, crucificado el viernes, rápidamente había sido envuelto en una sábana y depositado en el sepulcro. Lo buscan, pero no lo encuentran: \textit{ya no está donde había sido sepultado}. De Él \textit{sólo quedan las señales de la sepultura}: la tumba vacía, las vendas, la sábana. Las mujeres, sin embargo, quedan turbadas a la vista de un “\textit{joven vestido con una túnica blanca}”, que les anuncia: “\textit{No está aquí. Ha resucitado}”. Esta desconcertante noticia, destinada a cambiar el rumbo de la historia, desde entonces sigue resonando de generación en generación: anuncio antiguo y siempre nuevo. Ha resonado una vez más en esta Vigilia pascual, madre de todas las vigilias, y se está difundiendo en estas horas por toda la tierra.\end{body}
			
			\begin{body}2. ¡\textit{Oh sublime misterio de esta Noche Santa}! Noche en la cual revivimos ¡\textit{el extraordinario acontecimiento de la Resurrección}! Si Cristo hubiera quedado prisionero del sepulcro, la humanidad y toda la creación, en cierto modo, habrían perdido su sentido. Pero Tú, Cristo, ¡has resucitado verdaderamente! Entonces\textit{ se cumplen las Escrituras} que hace poco hemos escuchado de nuevo en la liturgia de la Palabra, recorriendo las etapas de todo el designio salvífico. Al comienzo de la creación “\textit{Vio Dios todo lo que había hecho: y era muy bueno}” (\textit{Gn} 1, 31). A Abrahán había prometido: “\textit{Todos los pueblos del mundo se bendecirán con tu descendencia}” (\textit{Gn} 22, 18). Se ha repetido uno de los cantos más antiguos de la tradición hebrea, que expresa el significado del antiguo éxodo, cuando “\textit{el Señor salvó a Israel de las manos de Egipto}” (\textit{Ex} 14, 30). Siguen cumpliéndose en nuestros días las promesas de los Profetas: “\textit{Os infundiré mi espíritu, y haré que caminéis…}” (\textit{Ez }36, 27).\end{body}
			
			\begin{body}3. En esta noche de \textit{Resurrección} todo vuelve a empezar desde el “principio”; \textit{la creación} recupera su auténtico significado en el plan de la salvación. Es como un \textit{nuevo comienzo }de la historia y del cosmos, porque “\textit{Cristo ha resucitado, primicia de todos los que han muerto}” (\textit{1 Co} 15, 20). Él, “\textit{el último Adán}”, se ha convertido en “\textit{un espíritu que da vida}” (\textit{1 Co} 15, 45). El mismo pecado de nuestros primeros padres es cantado en el \textit{Pregón pascual} como “\textit{felix culpa}”, “¡feliz culpa que mereció tal Redentor!”. Donde abundó el pecado, ahora sobreabundó la Gracia y “\textit{la piedra que desecharon los arquitectos es ahora la piedra angular}” (\textit{Salmo responsorial)} de un edificio espiritual indestructible. En esta Noche Santa ha nacido el nuevo pueblo con el cual \textit{Dios ha sellado una alianza eterna} con la sangre del Verbo encarnado, crucificado y resucitado.\end{body}
			
			\begin{body}4. Se entra a formar parte del pueblo de los redimidos mediante el Bautismo. “\textit{Por el bautismo} –nos ha recordado el apóstol Pablo en su Carta a los Romanos– \textit{fuimos sepultados con Él en la muerte, para que, así como Cristo fue despertado de entre los muertos por la gloria del Padre, así también nosotros andemos en una vida nueva}” (\textit{Rm} 6, 4).\end{body}
			
			\begin{body}\begin{bodysmall}[Esta exhortación va dirigida especialmente a vosotros, queridos \end{bodysmall}\textit{catecúmenos}\begin{bodysmall}, a quienes dentro de poco la Madre Iglesia comunicará el gran don de la vida divina. De diversas Naciones la divina Providencia os ha traído aquí, junto a la tumba de San Pedro, para recibir los Sacramentos de la \end{bodysmall}\textit{iniciación cristiana}\begin{bodysmall}: el Bautismo, la Confirmación y la Eucaristía. Entráis así en la Casa del Señor, sois consagrados con el óleo de la alegría y podéis alimentaros con el Pan del cielo. Sostenidos por la fuerza del Espíritu Santo, \end{bodysmall}\textit{perseverad en vuestra fidelidad a Cristo}\begin{bodysmall} y proclamad con valentía su Evangelio.]\end{bodysmall}\end{body}
			
			\begin{body}5. Queridos hermanos y hermanas aquí presentes. También nosotros, dentro de unos instantes, [nos uniremos a los catecúmenos para] renovar las promesas de nuestro Bautismo. Volveremos a renunciar a Satanás y a todas sus obras para seguir firmemente a Dios y sus planes de salvación. Expresaremos así \textit{un compromiso más fuerte de vida evangélica}.\end{body}
			
			\begin{body}Que María, testigo gozosa del acontecimiento de la Resurrección, ayude a todos a caminar “\textit{en una vida nueva}”; que haga a cada uno consciente de que, estando nuestro hombre viejo crucificado con Cristo, debemos considerarnos y comportarnos como hombres nuevos, personas que “viven para Dios, en Jesucristo” (cf. \textit{Rm} 6, 4. 11).\end{body}
			
			\begin{body}Amén. ¡Aleluya!\end{body}
			
			\subsection{Benedicto XVI, papa}
			
			\subsubsection{Homilía (2006): ¿La Resurrección te ha alcanzado?}
			
			\begin{referencia}Basílica Vaticana. 15 de abril de 2006.\end{referencia}
			
			\begin{body}“\textit{¿Buscáis a Jesús el Nazareno, el crucificado? No está aquí, ha resucitado}” (\textit{Mc} 16, 6). Así dijo el mensajero de Dios, vestido de blanco, a las mujeres que buscaban el cuerpo de Jesús en el sepulcro. Y lo mismo nos dice también a nosotros el evangelista en esta noche santa: Jesús <a id="_idTextAnchor045"></a>no es un personaje d<a id="_idTextAnchor046"></a>el pasado. Él vive y, como ser viviente, camina delante de nosotros; nos llama a seguirlo a Él, el viviente, y a encontrar así también nosotros el camino de la vida.\end{body}
			
			\begin{body}“\textit{Ha resucitado…, no está aquí}”. Cuando Jesús habló por primera vez a los discípulos sobre la cruz y la resurrección, estos, mientras bajaban del monte de la Transfiguración, se preguntaban qué querría decir eso de “resucitar de entre los muertos” (\textit{Mc} 9, 10). En Pascua nos alegramos porque Cristo no ha quedado en el sepulcro, su cuerpo no ha conocido la corrupción; pertenece al mundo de los vivos, no al de los muertos; nos alegramos porque Él es –como proclamamos en el rito del cirio pascual– Alfa y al mismo tiempo Omega, y existe por tanto, no sólo ayer, sino también hoy y por la eternidad (cf. \textit{Hb} 13, 8). Pero, en cierto modo, vemos la resurrección tan fuera de nuestro horizonte, tan extraña a todas nuestras experiencias, que, entrando en nosotros mismos, continuamos con la discusión de los discípulos: ¿En qué consiste propiamente eso de “resucitar”? ¿Qué significa para nosotros? ¿Y para el mundo y la historia en su conjunto? Un teólogo alemán dijo una vez con ironía que el milagro de un cadáver reanimado –si es que eso hubiera ocurrido verdaderamente, algo en lo que no creía– sería a fin de cuentas irrelevante para nosotros porque, justamente, no nos concierne. En efecto, el que solamente una vez alguien haya sido reanimado, y nada más, ¿de qué modo debería afectarnos? Pero la resurrección de Cristo es precisamente algo más, una cosa distinta. Es –si podemos usar por una vez el lenguaje de la teoría de la evolución– la mayor “mutación”, el salto más decisivo en absoluto hacia una dimensión totalmente nueva, que se haya producido jamás en la larga historia de la vida y de sus desarrollos: un salto de un orden completamente nuevo, que nos afecta y que atañe a toda la historia.\end{body}
			
			\begin{body}Por tanto, la discusión comenzada con los discípulos comprendería las siguientes preguntas: ¿Qué es lo que sucedió allí? ¿Qué significa eso para nosotros, para el mundo en su conjunto y para mí personalmente? Ante todo: ¿Qué sucedió? Jesús ya no está en el sepulcro. Está en una vida nueva del todo. Pero, ¿cómo pudo ocurrir eso? ¿Qué fuerzas han intervenido? Es decisivo que este hombre Jesús no estuviera solo, no fuera un Yo cerrado en sí mismo. Él era uno con el Dios vivo, unido talmente a Él que formaba con Él una sola persona. Se encontraba, por así decir, en un mismo abrazo con Aquél que es la vida misma, un abrazo no solamente emotivo, sino que abarcaba y penetraba su ser. Su propia vida no era solamente suya, era una comunión existencial con Dios y un estar insertado en Dios, y por eso no se le podía quitar realmente. Él pudo dejarse matar por amor, pero justamente así destruyó el carácter definitivo de la muerte, porque en Él estaba presente el carácter definitivo de la vida. Él era una cosa sola con la vida indestructible, de manera que ésta brotó de nuevo a través de la muerte. Expresemos una vez más lo mismo desde otro punto de vista.\end{body}
			
			\begin{body}Su muerte fue un acto de amor. En la última Cena, Él anticipó la muerte y la transformó en el don de sí mismo. Su comunión existencial con Dios era concretamente una comunión existencial con el amor de Dios, y este amor es la verdadera potencia contra la muerte, es más fuerte que la muerte. La resurrección fue como un estallido de luz, una explosión del amor que desató el vínculo hasta entonces indisoluble del “morir y devenir”. Inauguró una nueva dimensión del ser, de la vida, en la que también ha sido integrada la materia, de manera transformada, y a través de la cual surge un mundo nuevo.\end{body}
			
			\begin{body}Está claro que este acontecimiento no es un milagro cualquiera del pasado, cuya realización podría ser en el fondo indiferente para nosotros. Es un salto cualitativo en la historia de la “evolución” y de la vida en general hacia una nueva vida futura, hacia un mundo nuevo que, partiendo de Cristo, entra ya continuamente en este mundo nuestro, lo transforma y lo atrae hacia sí. Pero, ¿cómo ocurre esto? ¿Cómo puede llegar efectivamente este acontecimiento hasta mí y atraer mi vida hacia Él y hacia lo alto? La respuesta, en un primer momento quizás sorprendente pero completamente real, es la siguiente: dicho acontecimiento me llega mediante la fe y el bautismo. Por eso el Bautismo es parte de la Vigilia pascual, como se subraya también en esta celebración [con la administración de los sacramentos de la iniciación cristiana a algunos adultos de diversos países.] El Bautismo significa precisamente que no es un asunto del pasado, sino un salto cualitativo de la historia universal que llega hasta mí, tomándome para atraerme. El Bautismo es algo muy diverso de un acto de socialización eclesial, de un ritual un poco fuera de moda y complicado para acoger a las personas en la Iglesia. También es más que una simple limpieza, una especie de purificación y embellecimiento del alma. Es realmente muerte y resurrección, renacimiento, transformación en una nueva vida.\end{body}
			
			\begin{body}¿Cómo lo podemos entender? Pienso que lo que ocurre en el Bautismo se puede aclarar más fácilmente para nosotros si nos fijamos en la parte final de la pequeña autobiografía espiritual que san Pablo nos ha dejado en su \textit{Carta a los Gálatas}. Concluye con las palabras que contienen también el núcleo de dicha biografía: “\textit{Vivo yo, pero no soy yo, es Cristo quien vive en mí}” (2, 20). Vivo, pero ya no soy yo. El yo mismo, la identidad esencial del hombre –de este hombre, Pablo– ha cambiado. Él todavía existe y ya no existe. Ha atravesado un “no” y sigue encontrándose en este “no”: \textit{Yo, pero “no” más yo}. Con estas palabras, Pablo no describe una experiencia mística cualquiera, que tal vez podía habérsele concedido y, si acaso, podría interesarnos desde el punto de vista histórico. No, esta frase es la expresión de lo que ha ocurrido en el Bautismo. Se me quita el propio yo y es insertado en un nuevo sujeto más grande. Así, pues, está de nuevo mi yo, pero precisamente transformado, bruñido, abierto por la inserción en el otro, en el que adquiere su nuevo espacio de existencia. Pablo nos explica lo mismo una vez más bajo otro aspecto cuando, en el tercer capítulo de la \textit{Carta a los Gálatas}, habla de la “promesa” diciendo que ésta se dio en singular, a uno solo: a Cristo. Sólo él lleva en sí toda la “promesa”.\end{body}
			
			\begin{body}Pero, ¿qué sucede entonces con nosotros? Vosotros habéis llegado a ser uno en Cristo, responde Pablo (cf. \textit{Ga} 3, 28). No sólo una cosa, sino uno, un único, un único sujeto nuevo. Esta liberación de nuestro yo de su aislamiento, este encontrarse en un nuevo sujeto es un encontrarse en la inmensidad de Dios y ser trasladados a una vida que ha salido ahora ya del contexto del “morir y devenir”. El gran estallido de la resurrección nos ha alcanzado en el Bautismo para atraernos.\end{body}
			
			\begin{body}Quedamos así asociados a una nueva dimensión de la vida en la que, en medio de las tribulaciones de nuestro tiempo, estamos ya de algún modo inmersos. Vivir la propia vida como un continuo entrar en este espacio abierto: éste es el sentido del ser bautizado, del ser cristiano. Ésta es la alegría de la Vigilia pascual. La resurrección no ha pasado, la resurrección nos ha alcanzado e impregnado. A ella, es decir al Señor resucitado, nos sujetamos, y sabemos que también Él nos sostiene firmemente cuando nuestras manos se debilitan. Nos agarramos a su mano, y así nos damos la mano unos a otros, nos convertimos en un sujeto único y no solamente en una sola cosa. \textit{Yo, pero no más yo:} ésta es la fórmula de la existencia cristiana fundada en el bautismo, la fórmula de la resurrección en el tiempo. \textit{Yo, pero no más yo: }si vivimos de este modo transformamos el mundo.\textit{ }Es la fórmula de contraste con todas las ideologías de la violencia y el programa que se opone a la corrupción y a las aspiraciones del poder y del poseer.\end{body}
			
			\begin{body}“\textit{Viviréis, porque yo sigo viviendo}”, dice Jesús en el \textit{Evangelio de San Juan} (14, 19) a sus discípulos, es decir, a nosotros. Viviremos mediante la comunión existencial con Él, por estar insertos en Él, que es la vida misma. La vida eterna, la inmortalidad beatífica, no la tenemos por nosotros mismos ni en nosotros mismos, sino por una relación, mediante la comunión existencial con Aquél que es la Verdad y el Amor y, por tanto, es eterno, es Dios mismo. La mera indestructibilidad del alma, por sí sola, no podría dar un sentido a una vida eterna, no podría hacerla una vida verdadera. La vida nos llega del ser amados por Aquél que es la Vida; nos viene del vivir con Él y del amar con Él. \textit{Yo, pero no más yo:} ésta es la vía de la Cruz, la vía que “cruza” una existencia encerrada solamente en el yo, abriendo precisamente así el camino a la alegría verdadera y duradera.\end{body}
			
			\begin{body}De este modo, llenos de gozo, podemos cantar con la Iglesia en el \textit{Exultet}: “Exulten por fin los coros de los ángeles… Goce también la tierra”. La resurrección es un acontecimiento cósmico, que comprende cielo y tierra, y asocia el uno con la otra. Y podemos proclamar también con el \textit{Exultet}: “Cristo, tu hijo resucitado… brilla sereno para el linaje humano, y vive y reina glorioso por los siglos de los siglos”. Amén.\end{body}
			
			\subsubsection{Homilía (2009): La luz, el agua y el aleluya}
			
			\begin{referencia}Basílica de San Pedro. 11 de abril de 2009.\end{referencia}
			
			\begin{body}San Marcos nos relata en su Evangelio que los discípulos, bajando del monte de la Transfiguración, discutían entre ellos sobre lo que quería decir ,”resucitar de entre los muertos” (cf. \textit{Mc} 9, 10). Antes, el Señor les había anunciado su pasión y su resurrección a los tres días. Pedro había protestado ante el anuncio de la muerte. Pero ahora se preguntaban qué podía entenderse con el término “resurrección”. ¿Acaso no nos sucede lo mismo a nosotros? La Navidad, el nacimiento del Niño divino, nos resulta enseguida hasta cierto punto comprensible. Pod<a id="_idTextAnchor047"></a>emos amar al Niño, podemos imaginar la noche de Belén, la alegría de María, de san José y de los pastores, el júbilo de los ángeles. Pero resurrección, ¿qué es? No entra en el ámbito de nuestra experiencia y, así, el mensaje muchas veces nos parece en cierto modo incomprensible, como una cosa del pasado. La Iglesia trata de hacérnoslo comprender traduciendo este acontecimiento misterioso al lenguaje de los símbolos, en los que podemos contemplar de alguna manera este acontecimiento sobrecogedor. En la Vigilia Pascual nos indica el sentido de este día especialmente mediante tres símbolos: la luz, el agua y el canto nuevo, el Aleluya.\end{body}
			
			\begin{body}Primero la luz. La creación de Dios –lo acabamos de escuchar en el relato bíblico– comienza con la expresión: “Que exista la luz” (\textit{Gn} 1, 3). Donde hay luz, nace la vida, el caos puede transformarse en cosmos. En el mensaje bíblico, la luz es la imagen más inmediata de Dios: Él es todo Luminosidad, Vida, Verdad, Luz. En la Vigilia Pascual, la Iglesia lee la \textbf{narración de la creación} como profecía. En la resurrección se realiza del modo más sublime lo que este texto describe como el principio de todas las cosas. Dios dice de nuevo: “Que exista la luz”. La resurrección de Jesús es un estallido de luz. Se supera la muerte, el sepulcro se abre de par en par. El Resucitado mismo es Luz, la luz del mundo. Con la resurrección, el día de Dios entra en la noche de la historia. A partir de la resurrección, la luz de Dios se difunde en el mundo y en la historia. Se hace de día. Sólo esta Luz, Jesucristo, es la luz verdadera, más que el fenómeno físico de luz. Él es la pura Luz: Dios mismo, que hace surgir una nueva creación en aquella antigua, y transforma el caos en cosmos.\end{body}
			
			\begin{body}Tratemos de entender esto aún mejor. ¿Por qué Cristo es Luz? En el Antiguo Testamento, se consideraba a la Torah como la luz que procede de Dios para el mundo y la humanidad. Separa en la creación la luz de las tinieblas, es decir, el bien del mal. Indica al hombre la vía justa para vivir verdaderamente. Le indica el bien, le muestra la verdad y lo lleva hacia el amor, que es su contenido más profundo. Ella es “lámpara para mis pasos” y “luz en el sendero” (cf. \textit{Sal} 119, 105). Además, los cristianos sabían que en Cristo está presente la Torah, que la Palabra de Dios está presente en Él como Persona. La Palabra de Dios es la verdadera Luz que el hombre necesita. Esta Palabra está presente en Él, en el Hijo. El Salmo 19 compara la Torah con el sol que, al surgir, manifiesta visiblemente la gloria de Dios en todo el mundo. Los cristianos entienden: sí, en la resurrección, el Hijo de Dios ha surgido como Luz del mundo. Cristo es la gran Luz de la que proviene toda vida. Él nos hace reconocer la gloria de Dios de un confín al otro de la tierra. Él nos indica la senda. Él es el día de Dios que ahora, avanzando, se difunde por toda la tierra. Ahora, viviendo con Él y por Él, podemos vivir en la luz.\end{body}
			
			\begin{body}En la Vigilia Pascual, la Iglesia representa el misterio de luz de Cristo con el signo del cirio pascual, cuya llama es a la vez luz y calor. El simbolismo de la luz se relaciona con el del fuego: luminosidad y calor, luminosidad y energía transformadora del fuego: verdad y amor van unidos. El cirio pascual arde y, al arder, se consume: cruz y resurrección son inseparables. De la cruz, de la autoentrega del Hijo, nace la luz, viene la verdadera luminosidad al mundo. Todos nosotros encendemos nuestras velas del cirio pascual, sobre todo las de los recién bautizados, a los que, en este Sacramento, se les pone la luz de Cristo en lo más profundo de su corazón. La Iglesia antigua ha calificado el Bautismo como \textit{fotismos}, como Sacramento de la iluminación, como una comunicación de luz, y lo ha relacionado inseparablemente con la resurrección de Cristo. En el Bautismo, Dios dice al bautizando: “Recibe la luz”. El bautizando es introducido en la luz de Cristo. Ahora, Cristo separa la luz de las tinieblas. En Él reconocemos lo verdadero y lo falso, lo que es la luminosidad y lo que es la oscuridad. Con Él surge en nosotros la luz de la verdad y empezamos a entender. Una vez, cuando Cristo vio a la gente que había venido para escucharlo y esperaba de Él una orientación, sintió compasión de ellos, porque andaban como ovejas sin pastor (cf. \textit{Mc} 6, 34). Entre las corrientes contrastantes de su tiempo, no sabían dónde ir. Cuánta compasión debe sentir Cristo también en nuestro tiempo por tantas grandilocuencias, tras las cuales se esconde en realidad una gran desorientación. ¿Dónde hemos de ir? ¿Cuáles son los valores sobre los cuales regularnos? ¿Los valores en que podemos educar a los jóvenes, sin darles normas que tal vez no aguantan o exigirles algo que quizás no se les debe imponer? Él es la Luz. El cirio bautismal es el símbolo de la iluminación que recibimos en el Bautismo. Así, en esta hora, también san Pablo nos habla muy directamente. En la \textit{Carta a los Filipenses}, dice que, en medio de una generación tortuosa y convulsa, los cristianos han de brillar como lumbreras del mundo (cf. 2, 15). Pidamos al Señor que la llamita de la vela, que Él ha encendido en nosotros, la delicada luz de su palabra y su amor, no se apague entre las confusiones de estos tiempos, sino que sea cada vez más grande y luminosa, con el fin de que seamos con Él personas amanecidas, astros para nuestro tiempo.\end{body}
			
			\begin{body}El segundo símbolo de la Vigilia Pascual –la noche del Bautismo– es el agua. Aparece en la Sagrada Escritura y, por tanto, también en la estructura interna del Sacramento del Bautismo en dos sentidos opuestos. Por un lado está el mar, que se manifiesta como el poder antagonista de la vida sobre la tierra, como su amenaza constante, pero al que Dios ha puesto un límite. Por eso, el \textit{Apocalipsis} dice que en el mundo nuevo de Dios ya no habrá mar (cf. 21, 1). Es el elemento de la muerte. Y por eso se convierte en la representación simbólica de la muerte en cruz de Jesús: Cristo ha descendido en el mar, en las aguas de la muerte, como Israel en el Mar Rojo. Resucitado de la muerte, Él nos da la vida. Esto significa que el Bautismo no es sólo un lavacro, sino un nuevo nacimiento: con Cristo es como si descendiéramos en el mar de la muerte, para resurgir como criaturas nuevas.\end{body}
			
			\begin{body}El otro modo en que aparece el agua es como un manantial fresco, que da la vida, o también como el gran río del que proviene la vida. Según el primitivo ordenamiento de la Iglesia, se debía administrar el Bautismo con agua fresca de manantial. Sin agua no hay vida. Impresiona la importancia que tienen los pozos en la Sagrada Escritura. Son lugares de donde brota la vida. Junto al pozo de Jacob, Cristo anuncia a la Samaritana el pozo nuevo, el agua de la vida verdadera. Él se manifiesta como el nuevo Jacob, el definitivo, que abre a la humanidad el pozo que ella espera: ese agua que da la vida y que nunca se agota (cf. \textit{Jn} 4, 5. 15). San Juan nos dice que un soldado golpeó con una lanza el costado de Jesús, y que del costado abierto, del corazón traspasado, salió sangre y agua (cf. \textit{Jn} 19, 34). La Iglesia antigua ha visto aquí un símbolo del Bautismo y la Eucaristía, que provienen del corazón traspasado de Jesús. En la muerte, Jesús se ha convertido Él mismo en el manantial. El profeta Ezequiel percibió en una visión el Templo nuevo del que brota un manantial que se transforma en un gran río que da la vida (cf. 47, 1-12): en una Tierra que siempre sufría la sequía y la falta de agua, ésta era una gran visión de esperanza. El cristianismo de los comienzos entendió que esta visión se ha cumplido en Cristo. Él es el Templo auténtico y vivo de Dios. Y es la fuente de agua viva. De Él brota el gran río que fructifica y renueva el mundo en el Bautismo, el gran río de agua viva, su Evangelio que fecunda la tierra. Pero Jesús ha profetizado en un discurso durante la Fiesta de las Tiendas algo más grande aún. Dice: “El que cree en mí… de sus entrañas manarán torrentes de agua viva” (\textit{Jn} 7, 38). En el Bautismo, el Señor no sólo nos convierte en personas de luz, sino también en fuentes de las que brota agua viva. Todos nosotros conocemos personas de este tipo, que nos dejan en cierto modo sosegados y renovados; personas que son como el agua fresca de un manantial. No hemos de pensar sólo en los grandes personajes, como Agustín, Francisco de Asís, Teresa de Ávila, Madre Teresa de Calcuta, y así sucesivamente; personas por las que han entrado en la historia realmente ríos de agua viva. Gracias a Dios, las encontramos continuamente también en nuestra vida cotidiana: personas que son una fuente. Ciertamente, conocemos también lo opuesto: gente de la que promana un vaho como el de un charco de agua putrefacta, o incluso envenenada. Pidamos al Señor, que nos ha dado la gracia del Bautismo, que seamos siempre fuentes de agua pura, fresca, saltarina del manantial de su verdad y de su amor.\end{body}
			
			\begin{body}El tercer gran símbolo de la Vigilia Pascual es de naturaleza singular, y concierne al hombre mismo. Es el cantar el canto nuevo, el aleluya. Cuando un hombre experimenta una gran alegría, no puede guardársela para sí mismo. Tiene que expresarla, transmitirla. Pero, ¿qué sucede cuando el hombre se ve alcanzado por la luz de la resurrección y, de este modo, entra en contacto con la Vida misma, con la Verdad y con el Amor? Simplemente, que no basta hablar de ello. Hablar no es suficiente. Tiene que cantar. En la Biblia, la primera mención de este cantar se encuentra después de la travesía del Mar Rojo. Israel se ha liberado de la esclavitud. Ha salido de las profundidades amenazadoras del mar. Es como si hubiera renacido. Está vivo y libre. La Biblia describe la reacción del pueblo a este gran acontecimiento de salvación con la expresión: “El pueblo creyó en el Señor y en Moisés, su siervo” (cf. \textit{Ex }14, 31). Sigue a continuación la segunda reacción, que se desprende de la primera como una especie de necesidad interior: “Entonces Moisés y los hijos de Israel cantaron un cántico al Señor”. En la Vigilia Pascual, año tras año, los cristianos entonamos después de la tercera lectura este canto, lo entonamos como nuestro cántico, porque también nosotros, por el poder de Dios, hemos sido rescatados del agua y liberados para la vida verdadera.\end{body}
			
			\begin{body}La historia del canto de Moisés tras la liberación de Israel de Egipto y el paso del Mar Rojo, tiene un paralelismo sorprendente en el \textit{Apocalipsis} de san Juan. Antes del comienzo de las últimas siete plagas a las que fue sometida la tierra, al vidente se le aparece “una especie de mar de vidrio veteado de fuego; en la orilla estaban de pie los que habían vencido a la bestia, a su imagen y al número que es cifra de su nombre: tenían en sus manos las arpas que Dios les había dado. Cantaban el cántico de Moisés, el siervo de Dios, y el cántico del Cordero” (\textit{Ap }15, 2s). Con esta imagen se describe la situación de los discípulos de Jesucristo en todos los tiempos, la situación de la Iglesia en la historia de este mundo. Humanamente hablando, es una situación contradictoria en sí misma. Por un lado, se encuentra en el éxodo, en medio del Mar Rojo. En un mar que, paradójicamente, es a la vez hielo y fuego. Y ¿no debe quizás la Iglesia, por decirlo así, caminar siempre sobre el mar, a través del fuego y del frío? Considerándolo humanamente, debería hundirse. Pero mientras aún camina por este Mar Rojo, canta, entona el canto de alabanza de los justos: el canto de Moisés y del Cordero, en el cual se armonizan la Antigua y la Nueva Alianza. Mientras que a fin de cuentas debería hundirse, la Iglesia entona el canto de acción de gracias de los salvados. Está sobre las aguas de muerte de la historia y, no obstante, ya ha resucitado. Cantando, se agarra a la mano del Señor, que la mantiene sobre las aguas. Y sabe que, con eso, está sujeta, fuera del alcance de la fuerza de gravedad de la muerte y del mal –una fuerza de la cual, de otro modo, no podría escapar–, sostenida y atraída por la nueva fuerza de gravedad de Dios, de la verdad y del amor. Por el momento, la Iglesia y todos nosotros nos encontramos entre los dos campos de gravitación. Pero desde que Cristo ha resucitado, la gravitación del amor es más fuerte que la del odio; la fuerza de gravedad de la vida es más fuerte que la de la muerte. ¿Acaso no es ésta realmente la situación de la Iglesia de todos los tiempos, nuestra propia situación? Siempre se tiene la impresión de que ha de hundirse, y siempre está ya salvada. San Pablo ha descrito así esta situación: “Somos… los moribundos que están bien vivos” (\textit{2 Co} 6, 9). La mano salvadora del Señor nos sujeta, y así podemos cantar ya ahora el canto de los salvados, el canto nuevo de los resucitados: ¡aleluya! Amén.\end{body}
			
			\subsubsection{Homilía (2012): Nueva dimensión para el hombre}
			
			\begin{referencia}8 de abril de 2012. Basílica Vaticana.\end{referencia}
			
			\begin{body}Pascua es la fiesta de la nueva creación. Jesús ha resucitado y no morirá de nuevo. Ha descerrajado la puerta hacia una nueva vida que ya no conoce ni la enfermedad ni la muerte. Ha asumido al hombre en Dios mismo. “Ni la carne ni la sangre pueden heredar el reino de Dios”, dice Pablo en la \textit{Primera Carta a los Corintios} (15, 50). El escritor eclesiástico Tertuliano, en el siglo III, tuvo la audacia de escribir refriéndose a la resurrección de Cristo y a nuestra resurrección: “Carne y sangre, tened confianza, gracias a Cristo habéis adquirido un lugar en el cielo y en el reino de Dios” (\textit{CCL} II, 994). Se ha abierto una nueva dimensión para el hombre. La creación se ha hecho más grande y más espaciosa. La Pascua es el día de una nueva creación, pero precisamente por ello la Iglesia comienza la liturgi<a id="_idTextAnchor048"></a>a con la antigua creación, para que aprendamos a comprender la nueva. Así, en la Vigilia de Pascua, al principio de la Liturgia de la Palabra, se lee el relato de la \textbf{creación del mundo}. En el contexto de la liturgia de este día, hay dos aspectos particularmente importantes. En primer lugar, que se presenta a la creación como una totalidad, de la cual forma parte la dimensión del tiempo. Los siete días son una imagen de un conjunto que se desarrolla en el tiempo. Están ordenados con vistas al séptimo día, el día de la libertad de todas las criaturas para con Dios y de las unas para con las otras. Por tanto, la creación está orientada a la comunión entre Dios y la criatura; existe para que haya un espacio de respuesta a la gran gloria de Dios, un encuentro de amor y libertad. En segundo lugar, que en la Vigilia Pascual, la Iglesia comienza escuchando ante todo la primera frase de la historia de la creación: “Dijo Dios: ‘Que exista la luz’” (\textit{Gn} 1, 3). Como una señal, el relato de la creación inicia con la creación de la luz. El sol y la luna son creados sólo en el cuarto día. La narración de la creación los llama fuentes de luz, que Dios ha puesto en el firmamento del cielo. Con ello, los priva premeditadamente del carácter divino, que las grandes religiones les habían atribuido. No, ellos no son dioses en modo alguno. Son cuerpos luminosos, creados por el Dios único. Pero están precedidos por la luz, por la cual la gloria de Dios se refleja en la naturaleza de las criaturas.\end{body}
			
			\begin{body}¿Qué quiere decir con esto el relato de la creación? La luz hace posible la vida. Hace posible el encuentro. Hace posible la comunicación. Hace posible el conocimiento, el acceso a la realidad, a la verdad. Y, haciendo posible el conocimiento, hace posible la libertad y el progreso. El mal se esconde. Por tanto, la luz es también una expresión del bien, que es luminosidad y crea luminosidad. Es el día en el que podemos actuar. El que Dios haya creado la luz significa que Dios creó el mundo como un espacio de conocimiento y de verdad, espacio para el encuentro y la libertad, espacio del bien y del amor. La materia prima del mundo es buena, el ser es bueno en sí mismo. Y el mal no proviene del ser, que es creado por Dios, sino que existe sólo en virtud de la negación. Es el “no”.\end{body}
			
			\begin{body}En Pascua, en la mañana del primer día de la semana, Dios vuelve a decir: “Que exista la luz”. Antes había venido la noche del Monte de los Olivos, el eclipse solar de la pasión y muerte de Jesús, la noche del sepulcro. Pero ahora vuelve a ser el primer día, comienza la creación totalmente nueva. “Que exista la luz”, dice Dios, “y existió la luz”. Jesús resucita del sepulcro. La vida es más fuerte que la muerte. El bien es más fuerte que el mal. El amor es más fuerte que el odio. La verdad es más fuerte que la mentira. La oscuridad de los días pasados se disipa cuando Jesús resurge de la tumba y se hace él mismo luz pura de Dios. Pero esto no se refiere solamente a él, ni se refiere únicamente a la oscuridad de aquellos días. Con la resurrección de Jesús, la luz misma vuelve a ser creada. Él nos lleva a todos tras él a la vida nueva de la resurrección, y vence toda forma de oscuridad. Él es el nuevo día de Dios, que vale para todos nosotros.\end{body}
			
			\begin{body}Pero, ¿cómo puede suceder esto? ¿Cómo puede llegar todo esto a nosotros sin que se quede sólo en palabras sino que sea una realidad en la que estamos inmersos? Por el sacramento del bautismo y la profesión de la fe, el Señor ha construido un puente para nosotros, a través del cual el nuevo día viene a nosotros. En el bautismo, el Señor dice a aquel que lo recibe: \textit{Fiat lux}, que exista la luz. El nuevo día, el día de la vida indestructible llega también para nosotros. Cristo nos toma de la mano. A partir de ahora él te apoyará y así entrarás en la luz, en la vida verdadera. Por eso, la Iglesia antigua ha llamado al bautismo \textit{photismos}, iluminación.\end{body}
			
			\begin{body}¿Por qué? La oscuridad amenaza verdaderamente al hombre porque, sí, éste puede ver y examinar las cosas tangibles, materiales, pero no a dónde va el mundo y de dónde procede. A dónde va nuestra propia vida. Qué es el bien y qué es el mal. La oscuridad acerca de Dios y sus valores son la verdadera amenaza para nuestra existencia y para el mundo en general. Si Dios y los valores, la diferencia entre el bien y el mal, permanecen en la oscuridad, entonces todas las otras iluminaciones que nos dan un poder tan increíble, no son sólo progreso, sino que son al mismo tiempo también amenazas que nos ponen en peligro, a nosotros y al mundo. Hoy podemos iluminar nuestras ciudades de manera tan deslumbrante que ya no pueden verse las estrellas del cielo. ¿Acaso no es esta una imagen de la problemática de nuestro ser ilustrado? En las cosas materiales, sabemos y podemos tanto, pero lo que va más allá de esto, Dios y el bien, ya no lo conseguimos identificar. Por eso la fe, que nos muestra la luz de Dios, es la verdadera iluminación, es una irrupción de la luz de Dios en nuestro mundo, una apertura de nuestros ojos a la verdadera luz.\end{body}
			
			\begin{body}Queridos amigos, quisiera por último añadir todavía una anotación sobre la luz y la iluminación. En la Vigilia Pascual, la noche de la nueva creación, la Iglesia presenta el misterio de la luz con un símbolo del todo particular y muy humilde: el cirio pascual. Esta es una luz que vive en virtud del sacrificio. La luz de la vela ilumina consumiéndose a sí misma. Da luz dándose a sí misma. Así, representa de manera maravillosa el misterio pascual de Cristo que se entrega a sí mismo, y de este modo da mucha luz. Otro aspecto sobre el cual podemos reflexionar es que la luz de la vela es fuego. El fuego es una fuerza que forja el mundo, un poder que transforma. Y el fuego da calor. También en esto se hace nuevamente visible el misterio de Cristo. Cristo, la luz, es fuego, es llama que destruye el mal, transformando así al mundo y a nosotros mismos. Como reza una palabra de Jesús que nos ha llegado a través de Orígenes, “quien está cerca de mí, está cerca del fuego”. Y este fuego es al mismo tiempo calor, no una luz fría, sino una luz en la que salen a nuestro encuentro el calor y la bondad de Dios.\end{body}
			
			\begin{body}El gran himno del \textit{Exsultet}, que el diácono canta al comienzo de la liturgia de Pascua, nos hace notar, muy calladamente, otro detalle más. Nos recuerda que este objeto, el cirio, se debe principalmente a la labor de las abejas. Así, toda la creación entra en juego. En el cirio, la creación se convierte en portadora de luz. Pero, según los Padres, también hay una referencia implícita a la Iglesia. La cooperación de la comunidad viva de los fieles en la Iglesia es algo parecido al trabajo de las abejas. Construye la comunidad de la luz. Podemos ver así también en el cirio una referencia a nosotros y a nuestra comunión en la comunidad de la Iglesia, que existe para que la luz de Cristo pueda iluminar al mundo.\end{body}
			
			\begin{body}Roguemos al Señor en esta hora que nos haga experimentar la alegría de su luz, y pidámosle que nosotros mismos seamos portadores de su luz, con el fin de que, a través de la Iglesia, el esplendor del rostro de Cristo entre en el mundo (cf. \textit{Lumen gentium}, 1). \end{body}
			
			\begin{body}Amén.\end{body}
			
			\subsection{<a id="_idTextAnchor049"></a>Francisco, papa}
			
			\subsubsection{Homilía (<a id="_idTextAnchor050"></a>2015): Entrar en el misterio}
			
			\begin{referencia}Basílica Vaticana. 4 de abril de 2015.\end{referencia}
			
			\begin{body}Esta noche es noche de vigilia. El Señor no duerme, vela el guardián de su pueblo (cf. \textit{Sal} 121, 4), para sacarlo de la esclavitud y para abrirle el camino de la libertad. El Señor vela y, con la fuerza de su amor, hace pasar al pueblo a través del Mar Rojo; y hace pasar a Jesús a través del abismo de la muerte y de los infiernos. Esta fue una noche de vela para los discípulos y las discípulas de Jesús. Noche de dolor y de temor. Los hombres permanecieron cerrados en el Cenáculo. Las mujeres, sin embargo, al alba del día siguiente al sábado, fueron al sepulcro para ungir el cuerpo de Jesús. Sus corazones estaban llenos de emoción y se preguntaban: “¿Cómo haremos para entrar?, ¿quién nos removerá la piedra de la tumba?...”. Pero he aquí el primer signo del Acontecimiento: la gran piedra \textit{ya había sido removida}, y la tumba estaba abierta.\end{body}
			
			\begin{body}“Entraron en el sepulcro y vieron a un joven sentado a la derecha, vestido de blanco” (\textit{Mc} 16, 5). Las mujeres fueron las primeras que vieron este gran signo: el sepulcro vacío; y fueron las primeras en entrar. “Entraron en el sepulcro”. En esta noche de vigilia, nos viene bien detenernos a reflexionar sobre la experiencia de las discípulas de Jesús, que también nos interpela a nosotros. Efectivamente, para eso estamos aquí: para entrar, para \textit{entrar en el misterio }que Dios ha realizado con su vigilia de amor.\end{body}
			
			\begin{body}No se puede vivir la Pascua sin entrar en el misterio. No es un hecho intelectual, no es sólo conocer, leer... Es más, es mucho más. “Entrar en el misterio” significa capacidad de asombro, de contemplación; capacidad de escuchar el silencio y sentir el susurro de ese hilo de silencio sonoro en el que Dios nos habla (cf. \textit{1 Re} 19, 12). Entrar en el misterio nos exige no tener miedo de la realidad: no cerrarse en sí mismos, no huir ante lo que no entendemos, no cerrar los ojos frente a los problemas, no negarlos, no eliminar los interrogantes... Entrar en el misterio significa ir más allá de las cómodas certezas, más allá de la pereza y la indiferencia que nos frenan, y ponerse en busca de la verdad, la belleza y el amor, buscar un sentido no ya descontado, una respuesta no trivial a las cuestiones que ponen en crisis nuestra fe, nuestra fidelidad y nuestra razón.\end{body}
			
			\begin{body}Para entrar en el misterio se necesita humildad, la humildad de abajarse, de apearse del pedestal de nuestro yo, tan orgulloso, de nuestra presunción; la humildad para redimensionar la propia estima, reconociendo lo que realmente somos: criaturas con virtudes y defectos, pecadores necesitados de perdón. Para entrar en el misterio hace falta este abajamiento, que es impotencia, vaciamiento de las propias idolatrías... adoración. Sin adorar no se puede entrar en el misterio.\end{body}
			
			\begin{body}Todo esto nos enseñan las mujeres discípulas de Jesús. Velaron aquella noche, junto a la Madre. Y ella, la Virgen Madre, les ayudó a no perder la fe y la esperanza. Así, no permanecieron prisioneras del miedo y del dolor, sino que salieron con las primeras luces del alba, llevando en las manos sus ungüentos y con el corazón ungido de amor. Salieron y encontraron la tumba abierta. Y entraron. Velaron, salieron y entraron en el misterio. Aprendamos de ellas a velar con Dios y con María, nuestra Madre, para entrar en el misterio que nos hace pasar de la muerte a la vida.\end{body}
			
			\subsubsection{Homilía (2018): El triunfo de la Vida }
			
			\begin{referencia}Basílica Vaticana. 31 de marzo <a id="_idTextAnchor051"></a>de 2018.\end{referencia}
			
			\begin{body}Esta celebración la hemos comenzado fuera... inmersos en la oscuridad de la noche y en el frío que la acompaña. Sentimos el peso del silencio ante la muerte del Señor, un silencio en el que cada uno de nosotros puede reconocerse y cala hondo en las hendiduras del corazón del discípulo que ante la cruz se queda sin palabras. Son las horas del discípulo enmudecido frente al dolor que genera la muerte de Jesús: ¿Qué decir ante tal situación? El discípulo que se queda sin palabras al tomar conciencia de sus reacciones durante las horas cruciales en la vida del Señor: frente a la injusticia que condenó al Maestro, los discípulos hicieron silencio; frente a las calumnias y al falso testimonio que sufrió el Maestro, los discípulos callaron. Durante las horas difíciles y dolorosas de la Pasión, los discípulos experimentaron de forma dramática su incapacidad de “jugársela” y de hablar en favor del Maestro. Es más, no lo conocían, se escondieron, se escaparon, callaron (cfr. \textit{Jn} 18, 25-27). \end{body}
			
			\begin{body}Es la noche del silencio del discípulo que se encuentra entumecido y paralizado, sin saber hacia dónde ir frente a tantas situaciones dolorosas que lo agobian y rodean. Es el discípulo de hoy, enmudecido ante una realidad que se le impone haciéndole sentir, y lo que es peor, creer que nada puede hacerse para revertir tantas injusticias que viven en su carne nuestros hermanos. \end{body}
			
			\begin{body}Es el discípulo atolondrado por estar inmerso en una rutina aplastante que le roba la memoria, silencia la esperanza y lo habitúa al “siempre se hizo así”. Es el discípulo enmudecido que, abrumado, termina “normalizando” y acostumbrándose a la expresión de Caifás: “¿No les parece preferible que un solo hombre muera por el pueblo y no perezca la nación entera?” (\textit{Jn} 11, 50). \end{body}
			
			\begin{body}Y en medio de nuestros silencios, cuando callamos tan contundentemente, entonces las piedras empiezan a gritar (cf. \textit{Lc} 19,40)\footnote{16} y a dejar espacio para el mayor anuncio que jamás la historia haya podido contener en su seno: “No está aquí ha resucitado” (\textit{Mt} 28,6). La piedra del sepulcro gritó y en su grito anunció para todos un nuevo camino. Fue la creación la primera en hacerse eco del triunfo de la Vida sobre todas las formas que intentaron callar y enmudecer la alegría del evangelio. Fue la piedra del sepulcro la primera en saltar y a su manera entonar un canto de alabanza y admiración, de alegría y de esperanza al que todos somos invitados a tomar parte. \end{body}
			
			\begin{body}Y si ayer, con las mujeres contemplábamos “al que traspasaron” (\textit{Jn} 19, 36; cf. \textit{Za} 12, 10); hoy con ellas somos invitados a contemplar la tumba vacía y a escuchar las palabras del ángel: “no tengan miedo… ha resucitado” (\textit{Mt} 28, 5-6). Palabras que quieren tocar nuestras convicciones y certezas más hondas, nuestras formas de juzgar y enfrentar los acontecimientos que vivimos a diario; especialmente nuestra manera de relacionarnos con los demás. La tumba vacía quiere desafiar, movilizar, cuestionar, pero especialmente quiere animarnos a creer y a confiar que Dios “acontece” en cualquier situación, en cualquier persona, y que su luz puede llegar a los rincones menos esperados y más cerrados de la existencia. Resucitó de la muerte, resucitó del lugar del que nadie esperaba nada y nos espera –al igual que a las mujeres– para hacernos tomar parte de su obra salvadora. Este es el fundamento y la fuerza que tenemos los cristianos para poner nuestra vida y energía, nuestra inteligencia, afectos y voluntad en buscar, y especialmente en generar, caminos de dignidad. ¡No está aquí…ha resucitado! Es el anuncio que sostiene nuestra esperanza y la transforma en gestos concretos de caridad. ¡Cuánto necesitamos dejar que nuestra fragilidad sea ungida por esta experiencia, cuánto necesitamos que nuestra fe sea renovada, cuánto necesitamos que nuestros miopes horizontes se vean cuestionados y renovados por este anuncio! Él resucitó y con él resucita nuestra esperanza y creatividad para enfrentar los problemas presentes, porque sabemos que no vamos solos. \end{body}
			
			\begin{body}Celebrar la Pascua, es volver a creer que Dios irrumpe y no deja de irrumpir en nuestras historias desafiando nuestros “conformantes” y paralizadores determinismos. Celebrar la Pascua es dejar que Jesús venza esa pusilánime actitud que tantas veces nos rodea e intenta sepultar todo tipo de esperanza. La piedra del sepulcro tomó parte, las mujeres del evangelio tomaron parte, ahora la invitación va dirigida una vez más a ustedes y a mí: invitación a romper las rutinas, renovar nuestra vida, nuestras opciones y nuestra existencia. Una invitación que va dirigida allí donde estamos, en lo que hacemos y en lo que somos; con la “cuota de poder” que poseemos. ¿Queremos tomar parte de este anuncio de vida o seguiremos enmudecidos ante los acontecimientos?\end{body}
			
			\begin{body}¡No está aquí ha resucitado! Y te espera en Galilea, te invita a volver al tiempo y al lugar del primer amor y decirte: No tengas miedo, sígueme.\end{body}
			
			\chapter{Día de Pascua}
			
			\section{Lecturas}
			
			\begin{readtitle}PRIMERA LECTURA\end{readtitle}
			
			\begin{readbook}De los Hechos de los Apóstoles \rightline{10, 34a. 37-43}\end{readbook}
			
			\begin{readtheme}Hemos comido y bebido con él después de su resurrección de entre los muertos\end{readtheme}
			
			\begin{readbody}En aquellos días, Pedro tomó la palabra y dijo: \end{readbody}
			
			\begin{readtalk}“Vosotros conocéis lo que sucedió en toda Judea, comenzando por Galilea, después del bautismo que predicó Juan. Me refiero a Jesús de Nazaret, ungido por Dios con la fuerza del Espíritu Santo, que pasó haciendo el bien y curando a todos los oprimidos por el diablo, porque Dios estaba con él. \end{readtalk}
			
			\begin{readtalk}Nosotros somos testigos de todo lo que hizo en la tierra de los judíos y en Jerusalén. A este lo mataron, colgándolo de un madero. Pero Dios lo resucitó al tercer día y le concedió la gracia de manifestarse, no a todo el pueblo, sino a los testigos designados por Dios: a nosotros, que hemos comido y bebido con él después de su resurrección de entre los muertos. \end{readtalk}
			
			\begin{readtalk}Nos encargó predicar al pueblo, dando solemne testimonio de que Dios lo ha constituido juez de vivos y muertos. De él dan testimonio todos los profetas: que todos los que creen en él reciben, por su nombre, el perdón de los pecados”.\end{readtalk}
			
			\begin{readtitle}SALMO RESPONSORIAL\end{readtitle}
			
			\begin{readbook}Salmo \rightline{117, 1-2. 16ab-17. 22-23}\end{readbook}
			
			\begin{readtheme}Este es el día que hizo el Señor: sea nuestra alegría y nuestro gozo\end{readtheme}
			
			\begin{readbody}\begin{readred}℣.\end{readred} Dad gracias al Señor porque es bueno, \end{readbody}
			
			\begin{readtabbed}porque es eterna su misericordia. \end{readtabbed}
			
			\begin{readtabbed}Diga la casa de Israel: \end{readtabbed}
			
			\begin{readtabbed}eterna es su misericordia. \begin{readred}℟.\end{readred}\end{readtabbed}
			
			\begin{readbody}\begin{readred}℣.\end{readred} “La diestra del Señor es poderosa, \end{readbody}
			
			\begin{readtabbed}la diestra del Señor es excelsa”. \end{readtabbed}
			
			\begin{readtabbed}No he de morir, viviré \end{readtabbed}
			
			\begin{readtabbed}para contar las hazañas del Señor. \begin{readred}℟.\end{readred}\end{readtabbed}
			
			\begin{readbody}\begin{readred}℣.\end{readred} La piedra que desecharon los arquitectos \end{readbody}
			
			\begin{readtabbed}es ahora la piedra angular. \end{readtabbed}
			
			\begin{readtabbed}Es el Señor quien lo ha hecho, \end{readtabbed}
			
			\begin{readtabbed}ha sido un milagro patente. \begin{readred}℟.\end{readred}\end{readtabbed}
			
			\begin{readtitle}SEGUNDA LECTURA (opción 1)\end{readtitle}
			
			\begin{readbook}De la carta del apóstol san Pablo a los Colosenses \rightline{3, 1-4}\end{readbook}
			
			\begin{readtheme}Buscad los bienes de allá arriba, donde está Cristo\end{readtheme}
			
			\begin{readbody}Hermanos: \end{readbody}
			
			\begin{readbody}Si habéis resucitado con Cristo, buscad los bienes de allá arriba, donde Cristo está sentado a la derecha de Dios; aspirad a los bienes de arriba, no a los de la tierra. \end{readbody}
			
			\begin{readbody}Porque habéis muerto; y vuestra vida está con Cristo escondida en Dios. Cuando aparezca Cristo, vida vuestra, entonces también vosotros apareceréis gloriosos, juntamente con él.\end{readbody}
			
			\begin{readtitle}SEGUNDA LECTURA (opción 2)\end{readtitle}
			
			\begin{readbook}De la primera carta del apóstol san Pablo a los Corintios \rightline{5, 6b-8}\end{readbook}
			
			\begin{readtheme}Barred la levadura vieja para ser una masa nueva\end{readtheme}
			
			\begin{readbody}Hermanos: \end{readbody}
			
			\begin{readbody}¿No sabéis que un poco de levadura fermenta toda la masa? Barred la levadura vieja para ser una masa nueva, ya que sois panes ácimos. Porque ha sido inmolada nuestra víctima pascual: Cristo. \end{readbody}
			
			\begin{readbody}Así, pues, celebremos la Pascua, no con levadura vieja (levadura de corrupción y de maldad), sino con los panes ácimos de la sinceridad y la verdad.\end{readbody}
			
			\begin{readtitle}EVANGELIO\end{readtitle}
			
			\begin{readbook}Del Evangelio según san Juan \rightline{20, 1-9}\end{readbook}
			
			\begin{readtheme}Él había de resucitar de entre los muertos\end{readtheme}
			
			\begin{readbody}El primer día de la semana, María la Magdalena fue al sepulcro al amanecer, cuando aún estaba oscuro, y vio la losa quitada del sepulcro. Echó a correr y fue donde estaban Simón Pedro y el otro discípulo, a quien Jesús amaba, y les dijo: \end{readbody}
			
			\begin{readtalk}“Se han llevado del sepulcro al Señor y no sabemos dónde lo han puesto”. \end{readtalk}
			
			\begin{readbody}Salieron Pedro y el otro discípulo camino del sepulcro. Los dos corrían juntos, pero el otro discípulo corría más que Pedro; se adelantó y llegó primero al sepulcro; e, inclinándose, vio los lienzos tendidos; pero no entró. \end{readbody}
			
			\begin{readbody}Llegó también Simón Pedro detrás de él y entró en el sepulcro: vio los lienzos tendidos y el sudario con que le habían cubierto la cabeza, no con los lienzos, sino enrollado en un sitio aparte. \end{readbody}
			
			\begin{readbody}Entonces entró también el otro discípulo, el que había llegado primero al sepulcro; vio y creyó. \end{readbody}
			
			\begin{readbody}Pues hasta entonces no habían entendido la Escritura: que él había de resucitar de entre los muertos.\end{readbody}
			
			\section{Comentario Patrístico}
			
			\subsection{ Gregorio de Palamás}
			
			\begin{patertheme}Juan es aquel a quien Cristo amó con amor de predilección\end{patertheme}
			
			\begin{patersource}Homilía 20: PG 151, 266. 271.\end{patersource}
			
			\begin{body}Juan es aquel que permaneció virgen y recibió por gracia singular y como tesoro preciosísimo, a la Virgen Madre, única entre las madres; Juan es aquel a quien Cristo amó con amor de predilección y mereció ser llamado hijo, con preferencia a los otros evangelistas. Por eso hace resonar con fuerza la trompeta al anunciarnos los prodigios de la resurrección del Señor de entre los muertos, y al relatarnos con mayor claridad el modo cómo se manifestó a sus discípulos, según lo hallamos escrito en su evangelio, cuando nos dice: \textit{El primer día de la semana, María Magdalena fue al sepulcro al amanecer, cuando aún estaba oscuro y vio la losa quitada del sepulcro. Echó a correr y fue donde estaba Simón Pedro y el otro discípulo, a quien tanto quería Jesús. }Así es como se presenta a sí mismo.\end{body}
			
			\begin{body}Juan y Pedro, habiendo oído a María, van corriendo al sepulcro, donde vieron que había salido la Vida; y habiendo visto y creído, admirados por las pruebas se volvieron a casa.\end{body}
			
			\begin{body}Consideremos, hermanos, cuánta mayor dignidad que María Magdalena no tenía Pedro, el príncipe de los apóstoles, y el mismo Juan, a quien tanto quería Jesús, y sin embargo ella fue considerada digna de una gracia tan grande, con preferencia a ellos. Porque los apóstoles, corriendo al sepulcro, sólo vieron las vendas y el sudario; María, en cambio, por su firmeza y constancia, perseverando hasta el fin a la entrada del sepulcro, llegó a ver no sólo a los ángeles, sino al mismo Señor de los ángeles en la carne, antes que los apóstoles.\end{body}
			
			\begin{body}Este templo que veis, es un símbolo de aquel sepulcro; y no sólo un símbolo, sino una realidad mucho más sublime. Detrás de esa cortina, en el interior, está el lugar donde se coloca el cuerpo del Señor, y ahí está también la mesa o el altar santo. Así pues, lo mismo que María, todo el que se acerque con presteza a la recepción del misterio divino y persevere hasta el fin, teniendo recogida en Dios su propia alma, no sólo reconocerá las enseñanzas de la Escritura santa, redactada por el Espíritu de Dios, ni sólo a los ángeles que anunciaron el misterio de la divinidad y humanidad del Verbo de Dios, encarnado por nosotros, sino que verá también y sin ningún género de duda al mismo Señor con los ojos del alma, y también con los del cuerpo.\end{body}
			
			\begin{body}Pues aquel que con fe ve la mesa mística y el pan de vida depositado sobre ella ve al mismo Verbo de Dios oculto bajo las especies, hecho carne por nosotros y habitando en nosotros como en un sagrario. Más aún: si es considerado digno de recibirle, no sólo le ve, sino que participa de él, le recibe en sí mismo como huésped, y es enriquecido con el don de la misma gracia divina. Y así como María Magdalena vio lo que antes que nada los apóstoles deseaban ver, así el alma, poseída por la fe, será considerada digna de ver y de gozar de aquello que –según el apóstol– los \textit{ángeles desean penetrar, }divinizándose por completo, tanto por la contemplación como por la participación de estos misterios.\end{body}
			
			\subsection{Francisco, papa}
			
			\begin{patertheme}Mañana que cambió la historia\end{patertheme}
			
			\begin{patersource}A los jóvenes italianos, 11 de agosto de 2018, parr. 3-6. 9-13.\end{patersource}
			
			\begin{body}[...] En el pasaje del Evangelio que hemos escuchado (cf. \textit{Jn} 20, 1-8), Juan nos relata esa mañana inimaginable que cambió para siempre la historia de la humanidad. Imaginémosla esa mañana: a las primeras luces del alba del día después del sábado, alrededor de la tumba de Jesús, todos comienzan a correr. María de Magdala corre para advertir a los discípulos; Pedro y Juan corren hacia la tumba... Todos corren, todos sienten la urgencia de moverse: no hay tiempo que perder, debemos apresurarnos... Como había hecho María, ¿os acordáis? ―apenas concebido Jesús–, para ir a ayudar a Isabel.\end{body}
			
			Tenemos muchas razones para correr, a menudo solo porque hay muchas cosas que hacer y el tiempo nunca es suficiente. A veces nos apresuramos porque nos atrae algo nuevo, bello, interesante. A veces, por el contrario, corremos para escapar de una amenaza, de un peligro...
			
			Los discípulos de Jesús corren porque han recibido la noticia de que el cuerpo de Jesús ha desaparecido de la tumba. Los corazones de María Magdalena, de Simón Pedro, de Juan están llenos de amor y palpitan furiosamente después de la separación que parecía definitiva. ¡Quizás se ha reavivado en ellos la esperanza de volver a ver el rostro del Señor! Como en ese primer día cuando había prometido: “Venid y ved” (\textit{Jn} 1, 39). Juan es el que corre más deprisa, ciertamente porque es más joven, pero también porque no ha dejado de esperar después de ver a Jesús morir en la cruz con sus propios ojos; y también porque estaba cerca de María, y por eso estaba “contagiado” por su fe. Cuando sentimos que la fe nos falta o es tibia, vayamos a ella, a María, y ella nos enseñará, nos entenderá, nos hará sentir la fe.
			
			Desde esa mañana (...) la historia ya no es lo misma. Esa mañana cambió la historia. La hora en que la muerte parecía triunfar, en realidad se revela como el momento de su derrota. Incluso esa pesada roca, colocada ante el sepulcro, no pudo resistir. Y desde ese amanecer del primer día después del sábado, cada lugar donde la vida está oprimida, cada espacio en el que dominan la violencia, la guerra, la miseria, donde el hombre es humillado y pisoteado, en ese lugar todavía se puede reavivar la esperanza de la vida.
			
			\begin{body}El Evangelio dice que Pedro entró el primero en el sepulcro y vio las sábanas por el suelo y el sudario envuelto en un lugar separado. Luego también entró el otro discípulo, quien –dice el Evangelio– “vio y creyó” (\textit{Jn }20, 8). Este par de verbos es muy importante: ver y creer. A lo largo del Evangelio de Juan, se dice que los discípulos, viendo los signos que Jesús hacía, creyeron en Él. Ver y creer. ¿De qué signos se trata? El agua transformada en vino para la boda; de algunos enfermos curados; de un ciego que recobra la vista; de una gran multitud saciada con cinco panes y dos peces; de la resurrección de su amigo Lázaro, muerto desde hacía cuatro días. En todos estos signos, Jesús revela el rostro invisible de Dios.\end{body}
			
			\begin{body}No es la representación de la sublime perfección divina, la que se desprende de los signos de Jesús, sino la historia de la fragilidad humana que se encuentra con la Gracia que eleva. Hay una humanidad herida que es sanada tras el encuentro con Él; hay un hombre caído que encuentra una mano tendida para aferrarse; hay el desamparo de los derrotados que descubren una esperanza de redención. Y Juan, cuando entra en el sepulcro de Jesús, lleva en los ojos y en el corazón los signos hechos por Jesús que se sumerge en el drama humano para levantarlo. Jesucristo (...) no es un héroe inmune a <a id="_idTextAnchor052"></a>la muerte, sino Aquel que la transforma con el don de su vida. Y ese sudario cuidadosamente doblado dice que ya no lo necesitará: la muerte ya no tiene poder sobre él.\end{body}
			
			\begin{body}[...] ¿Es posible encontrar vida en los lugares donde reina la muerte? Sí, es posible. Entrarían ganas de decir que no, que es mejor mantenerse alejado, largarse. Sin embargo, esta es la novedad revolucionaria del Evangelio: el sepulcro vacío de Cristo se convierte en el último signo en el que brilla la victoria definitiva de la Vida. ¡Y entonces no tengamos miedo! No nos alejemos de los lugares de sufrimiento, de derrota, de muerte. Dios nos ha dado un poder mayor que todas las injusticias y las debilidades de la historia, más grande que nuestro pecado: Jesús ha vencido la muerte dando su vida por nosotros. Él nos envía a anunciar a nuestros hermanos que Él es el Resucitado, es el Señor, y nos da su Espíritu para sembrar con Él el Reino de Dios. Aquella mañana del domingo de Pascua cambió la historia: ¡Tengamos valor!\end{body}
			
			\begin{body}¡Cuántos sepulcros ―por así decirlo― esperan hoy nuestra visita! Cuántos heridos, incluso jóvenes, han sellado su sufrimiento “poniendo –como se dice– una piedra encima”. Con el poder del Espíritu y la Palabra de Jesús podemos mover esos pedruscos y dejar que los rayos de luz entren en esos barrancos de tinieblas.\end{body}
			
			\begin{body}El camino para llegar a Roma ha sido hermoso y agotador; pensad, ¡cuánto esfuerzo, pero cuánta belleza! Pero el camino de regreso a vuestros hogares, a vuestros países, a vuestras comunidades será igual de hermoso y desafiante. Seguidlo con la confianza y la energía de Juan el “discípulo amado“. Sí, el secreto está todo allí, en ser y saber que eres “amado”, “amada” por Él, ¡Jesús, el Señor, nos ama! Y que cada uno de nosotros, volviendo a casa, se lo grabe en el corazón y en la mente: Jesús, el Señor, me ama. Soy amado. Soy amada. Sentir la ternura de Jesús que me ama. Recorre con valor y alegría el camino hacia casa, recorredlo con la certeza de ser amados por Jesús. Entonces, con este amor, la vida se convierte en algo bueno, sin ansiedad, sin miedo, esa palabra que nos destruye. Sin ansiedad y sin miedo. Una carrera hacia Jesús y hacia los hermanos, con un corazón lleno de amor, de fe y de alegría. ¡Adelante!\end{body}
			
			\section{Homilías}
			
			\subsection{San Pablo VI, papa}
			
			\subsubsection{Homilía (1970): El mayor acontecimiento}
			
			\begin{referencia}Parroquia de San Giorgio en Casa Palocco.\end{referencia}
			
			\begin{referencia}29 de marzo de 1970.\end{referencia}
			
			\begin{body}[...] Felices Pascuas: sea verdaderamente un gran y hermoso día. \end{body}
			
			\begin{body}¡Ha resucitado!: una frase que se imprimirá en la memoria como lo más grande del mundo, el acontecimiento más extraordinario de la historia. \end{body}
			
			\begin{body}El hombre de hoy está acostumbrado a tener noticias de la conquista del espacio, de los maravillosos descubrimientos de la ciencia, de nuevos inventos. Pero saber que la vida, que nuestra existencia resurge, es algo mucho más asombroso y bello. Lo sabe bien quién ha estado enfermo y ha sido sanado, quién ha conocido las tinieblas de la guerra y ha encontrado la paz. La Pascua es la fiesta de la vida, la fiesta de la Resurrección, de la victoria sobre la muerte. Es el nuevo orden que el Señor quiere instaurar en la humanidad, y no es solo un hecho personal. El Señor ha resucitado por cada uno de nosotros, que estamos todos muriendo por la fragilidad de nuestra naturaleza. \end{body}
			
			\begin{body}La vida de Jesús fue tal que el alma dominaba la materia, mientras que nosotros estamos fuertemente condicionados por la composición de nuestro cuerpo. El cuerpo está destinado a convertirse a su vez en un instrumento del alma, porque así lo ha establecido el Señor. Estamos hechos para vivir para siempre. Cuando una madre da a luz a un hijo, le da al mundo algo nuevo que nunca terminará. La vida es sagrada y debemos protegerla desde el útero. \end{body}
			
			\begin{body}Cristo resucitó y todos los que creen en él resucitarán. Es necesario estar en convencida armonía con él, hacer como transfusión de la vida de Cristo a la nuestra. Si podemos estar en comunicación con esta fuente de vida, somos salvados. Si este hilo de conexión se rompe, estamos condenados. Estar con Cristo: eso es lo que significa ser cristiano. \end{body}
			
			\begin{body}Nuestra Resurrección se realiza a través de tres fases. La primera es el Bautismo, a través del cual se infunde en la criatura como un alma nueva, el Espíritu Santo, la Gracia, una comunicación invisible pero real. Es un regalo que el cuerpo no ve, pero el alma sí. La segunda fase consiste en la coherencia, en la fidelidad. Debemos escuchar la Palabra del Señor, debemos convertirnos en discípulos, seguidores, creyentes. Después de todo, no hay personas felices en el mundo como los cristianos, si es que realmente lo son, porque siempre tienen la alegría de la Pascua en sus corazones. \end{body}
			
			\begin{body}Jesús llamó a todos: al niño, al trabajador, al pobre. Derramó felicidad, gozo, alegría en el mundo. Tened siempre el alma llena de la alegría de Cristo. Después de esta fase, de la vida nueva, de la vida cristiana, estará la tercera: nuestra Resurrección. Es la Parusía, la última aparición después de nuestra muerte. Los cementerios se abrirán, los muertos resucitarán, la vida se reanudará animada por el alma inmortal. \end{body}
			
			\begin{body}En una palabra: valentía, esperanza, alegría, la promesa de ser verdaderamente cristianos y la gratitud al Señor por hacernos vivir la Pascua, preludio de nuestra vida eterna.\end{body}
			
			\subsection{San Juan Pablo II, papa}
			
			\subsubsection{Urbi et Orbi (1980): Cristo ha resucitado}
			
			\begin{referencia}Mensaje Urbi et Orbi, Domingo de Resurrección, 6 de abril de 1980.\end{referencia}
			
			\begin{body}1. “... y vio que la piedra había sido removida” (\textit{Jn} 20, 1).\end{body}
			
			\begin{body}En la anotación de los acontecimientos del día que siguió a aquel sábado, estas palabras tienen un significado clave.\end{body}
			
			\begin{body}Al lugar donde había sido puesto Jesús, la tarde del viernes, llega María Magdalena, llegan las otras mujeres. Jesús había sido colocado \textit{en una tumba} nueva, excavada en la roca, en la cual nadie había sido sepultado anteriormente. La tumba se hallaba a los pies del Gólgota, allí donde Jesús crucificado expiró, después de que el centurión le traspasara el costado con la lanza para constatar con certeza la realidad de su muerte. Jesús había sido envuelto en lienzos por las manos caritativas y afectuosas de las piadosas mujeres que, junto con su madre y con Juan, el discípulo predilecto, habían asistido a su extremo sacrificio. Pero, dado que caía rápidamente la tarde e iniciaba el sábado de pascua, las generosas y amorosas discípu<a id="_idTextAnchor053"></a>las <a id="_idTextAnchor054"></a>se vieron obligadas a dejar la unción del cuerpo santo y martirizado de Cristo para la próxima ocasión, apenas la ley religiosa de Israel lo permitiese.\end{body}
			
			\begin{body}Se dirigen pues al sepulcro, el día siguiente al sábado, temprano, es decir, \textit{al romper el día}, preocupadas de cómo remover la gran piedra que había sido puesta a la entrada del sepulcro, el cual además había sido sellado.\end{body}
			
			\begin{body}Y he aquí que, llegadas al lugar, vieron que la piedra había sido removida del sepulcro.\end{body}
			
			\begin{body}2. Aquella piedra, colocada a la entrada de la tumba, se había convertido primeramente en un mudo \textit{testigo} de la muerte del Hijo del Hombre.\end{body}
			
			\begin{body}Con piedras así se concluía el curso de la vida de tantos hombres de entonces en el cementerio de Jerusalén; más aún, el ciclo de la vida de todos los hombres en los cementerios de la tierra.\end{body}
			
			\begin{body}Bajo el peso de la losa sepulcral, tras su barrera imponente, se cumple en el silencio del sepulcro la obra de la muerte, es decir, el hombre salido del polvo se transforma lentamente en polvo (cf. \textit{Gén} 3, 19).\end{body}
			
			\begin{body}La piedra puesta la tarde del Viernes Santo \textit{sobre la tumba de Jesús}, se ha convertido, como todas las losas sepulcrales, en el testigo mudo de la muerte del Hombre, del Hijo del Hombre.\end{body}
			
			\begin{body}¿Qué testimonia esta losa, el día después del sábado, en las primeras horas del día?\end{body}
			
			\begin{body}¿Qué nos dice? ¿Qué \textit{anuncia la piedra removida del sepulcro?}\end{body}
			
			\begin{body}En el Evangelio no hay una respuesta humana adecuada. No aparece en los labios de María de Magdala. Cuando asustada, por la ausencia del cuerpo de Jesús en la tumba, esta mujer corre a avisar a Simón Pedro y al otro discípulo al que Jesús amaba (cf. \textit{Jn }20, 2), su \textit{lenguaje humano} encuentra solamente estas palabras para expresar lo sucedido: “Han tomado al Señor del monumento y no sabemos dónde lo han puesto” (\textit{Jn} 20, 2).\end{body}
			
			\begin{body}También Simón Pedro y el otro discípulo se dirigieron de prisa al sepulcro; y Pedro, entrando dentro, vio las vendas por tierra, y el sudario que había sido puesto sobre la cabeza de Jesús, al lado (cf. \textit{Jn} 20, 7).\end{body}
			
			\begin{body}Entonces entró también el otro discípulo, vio y creyó; “aún \textit{no se habían dado cuenta} de la Escritura, según la cual \textit{era preciso que Él resucitase de entre los muertos}” (\textit{Jn} 20, 9).\end{body}
			
			\begin{body}Vieron y comprendieron que los hombres no habían logrado derrotar a Jesús con la losa sepulcral, sellándola con la señal de la muerte.\end{body}
			
			\begin{body}3. La iglesia que hoy, como cada año, termina su triduo pascual con el Domingo de Resurrección, canta con alegría las palabras del antiguo: Salmo: \end{body}
			
			\begin{body}‘‘Alabad a Yavé porque es bueno, porque es eterna su misericordia. Diga la Casa de Israel: Porque es eterna. su misericordia... La diestra de Yavé ha sido ensalzada; la diestra de Yavé ha hecho proezas... No moriré, sino que viviré para poder narrar las gestas de Yavé... La piedra que rechazaron los constructores ha sido puesta por cabecera angular... Obra de Yavé es ésta, y es admirable a nuestros ojos” (\textit{Sal} 117 [118], 1-2; 16-17; 22-23).\end{body}
			
			\begin{body}Los artífices de la muerte del Hijo del Hombre, para mayor seguridad, “pusieron guardia al sepulcro después de haber sellado la piedra” (\textit{Mt} 27, 66).\end{body}
			
			\begin{body}Muchas veces los constructores del Mundo, por el cual Cristo quiso morir han tratado de poner una piedra definitiva sobre su tumba.\end{body}
			
			\begin{body}Pero la piedra permanece siempre removida de su sepulcro; la piedra, testigo de la muerte, se ha convertido en testigo de la Resurrección: “la diestra de Yavé ha hecho proezas” (\textit{Sal} 117 [118], 16).\end{body}
			
			\begin{body}4. La Iglesia anuncia siempre y de nuevo la Resurrección de Cristo. La Iglesia repite con alegría a los hombres las palabras de los ángeles y de las mujeres pronunciadas en aquella radiante mañana en la que la muerte fue vencida.\end{body}
			
			\begin{body}La Iglesia anuncia que \textit{está vivo Aquel que se ha convertido en nuestra Pascua}. Aquel que ha muerto en la cruz, revela la plenitud de la \textit{Vida}.\end{body}
			
			\begin{body}Este mundo que por desgracia hoy, de diversas maneras, parece querer la “muerte de Dios”, escuche el mensaje de la Resurrección.\end{body}
			
			\begin{body}Todos vosotros que anunciáis “la muerte de Dios”, que tratáis de expulsar a Dios del mundo humano, deteneos y pensad que “\textit{la muerte de Dios”} puede comportar fatalmente “\textit{la muerte del hombre”}.\end{body}
			
			\begin{body}Cristo ha resucitado para que el hombre encuentre el auténtico significado de la existencia, para que el \textit{hombre viva en plenitud su propia vida}, para que el hombre, que viene \textit{de Dios}, viva \textit{en Dios}.\end{body}
			
			\begin{body}Cristo ha resucitado. El \textit{es la piedra angular}. Ya entonces se quiso rechazarlo y vencerlo con la piedra vigilada y sellada del sepulcro. Pero aquella piedra fue removida. Cristo ha resucitado.\end{body}
			
			\begin{body}No rechacéis a Cristo vosotros, los que construís el mundo humano.\end{body}
			
			\begin{body}No lo rechacéis vosotros, los que, de cualquier manera y en cualquier sector, construís el mundo de \textit{hoy} y de \textit{mañana}: el mundo de la cultura y de la civilización, el mundo de la economía y de la política, el mundo de la ciencia y de la información. \textit{Vosotros que construís el mundo de la paz..., ¿o de la guerra?} Vosotros que construís el mundo del orden..., ¿o del terror? No rechacéis a Cristo: ¡Él es la piedra angular!\end{body}
			
			\begin{body}Que no lo rechace ningún hombre, porque cada uno es responsable de su destino: constructor o destructor de la propia existencia.\end{body}
			
			\begin{body}Cristo resucitó ya antes de que el Ángel removiera la losa sepulcral. Él se reveló después como piedra angular, sobre la cual se construye la historia de la humanidad entera y la de cada uno de nosotros.\end{body}
			
			\begin{body}5. ¡Queridos hermanos y hermanas! Con sincera alegría acojamos este día tan esperado. Con viva alegría compartamos el mensaje pascual todos los que acogemos a Cristo como piedra angular.\end{body}
			
			\begin{body}En virtud de esta piedra angular que une, construyamos nuestra común esperanza con los hermanos en Cristo de Oriente y de Occidente, con quienes no nos une aún la plena comunión y la perfecta unidad.\end{body}
			
			\begin{body}Aceptad, queridos hermanos, nuestro beso pascual de paz y de amor. Cristo resucitado despierte en nosotros un deseo todavía más profundo de esta unidad por la cual oró la víspera de su pasión.\end{body}
			
			\begin{body}No cesemos de orar por ella en unión con Él. Pongamos nuestra confianza en la fuerza de la cruz y de la Resurrección; ¡tal fuerza es más poderosa que la debilidad de toda división humana!\end{body}
			
			\begin{body}Amadísimos hermanos: ¡Os anunció un gran gozo: Aleluya!\end{body}
			
			\begin{body}6. La Iglesia se acerca hoy a cada hombre con el \textit{deseo pascual}: el deseo de construir el mundo sobre Cristo: deseo que hace extensivo a toda la familia humana.\end{body}
			
			\begin{body}Ojalá acojan este deseo los que comparten con nosotros el mensaje de la resurrección y de la alegría pascual, y también los que, por desgracia, no lo comparten. Cristo, “nuestra Pascua”, no cesa de ser peregrino con nosotros en el camino de la historia, y cada uno puede encontrarlo, porque Él no cesa de ser el Hermano del hombre en cada época y en cada momento.\end{body}
			
			\begin{body}En \textit{su} nombre os hablo hoy a todos, y a todos os dirijo mi más ferviente y santa felicitación.\end{body}
			
			\subsection{Benedicto XVI, papa}
			
			\subsubsection{Urbi et Orbi (2006): Promesa cumplida}
			
			\begin{body}\textit{Christus resurrexit!}- ¡Cristo ha resucitado! \end{body}
			
			\begin{body}La gran Vigilia de esta noche nos ha hecho revivir el acontecimiento decisivo y siempre actual de la Resurrección, misterio central de la fe cristiana. En las iglesias se han encendido innumerables cirios pascuales para simbolizar la luz de Cristo que ha iluminado e ilumina a la humanidad, venciendo para siempre las tinieblas del pecado y del mal. Y hoy resuenan con fuerza las palabras que asombraron a las mujeres que habían ido la madrugada del primer día de la semana al sepulcro donde habían puesto el cuerpo de Cristo, bajado apresuradamente de la cruz. Tristes y desconsoladas por la pérdida de su Maestro, encontraron apartada la gran piedra y, al entrar, no hallaron su cuerpo. Mientras estaban allí, perplejas y confusas, dos hombres con vestidos resplandecientes les sorprendieron, diciendo: “¿Por qué buscáis entre los muertos al que vive? No está aquí, ha resucitado” (\textit{Lc }24, 5-6) “\textit{Non est hic, sed resurrexit”} (\textit{Lc} 24, 6). Desde aquella mañana, estas palabras siguen resonando en el universo como anuncio perenne, e impregnado a la vez de infinitos y siempre nuevos ecos, que atraviesa los siglos. \end{body}
			
			\begin{body}“No está aquí... ha resucitado”. Los mensajeros celestes comunican ante todo que Jesús “no está aquí”: el Hijo de Dios no ha quedado en el sepulcro, porque no podía permanecer bajo el dominio de la muerte (cf. \textit{Hch} 2, 24) y la tumba no podía retener “al que vive” (\textit{Ap} 1, 18), al que es la fuente misma de la vida. Porque, del mismo modo que Jonás estuvo en el vientre del cetáceo, también Cristo crucificado quedó sumido en el seno de la tierra (cf. \textit{Mt} 12, 40) hasta terminar un sábado. Aquel sábado fue ciertamente “un día solemne”, como escribe el evangelista Juan (19, 31), el más solemne de la historia, porque, en él, el “Señor del sábado” (\textit{Mt }12, 8) llevó a término la obra de la creación (cf. \textit{Gn} 2, 1-4a), elevando al hombre y a todo el cosmos a la gloriosa libertad de los hijos de Dios (cf. \textit{Rm }8, 21). Cumplida esta obra extraordinaria, el cuerpo exánime ha sido traspasado por el aliento vital de Dios y, rotas las barreras del sepulcro, ha resucitado glorioso. Por esto los ángeles proclaman “no está aquí”: ya no se le puede encontrase en la tumba. Ha peregrinado en la tierra de los hombres, ha terminado su camino en la tumba, como todos, pero ha vencido a la muerte y, de modo absolutamente nuevo, por un puro acto de amor, ha abierto la tierra de par en par hacia el Cielo. \end{body}
			
			\begin{body}Su resurrección, gracias al Bautismo que nos “incorpora” a Él, es nuestra resurrección. Lo había preanunciado el profeta Ezequiel: “Yo mismo abriré vuestros sepulcros, y os haré salir de vuestros sepulcros, pueblo mío, y os traeré a la tierra de Israel” (\textit{Ez} 37, 12). Estas palabras proféticas adquieren un valor singular en el día de Pascua, porque hoy se cumple la promesa del Creador; hoy, también en esta época nuestra marcada por la inquietud y la incertidumbre, revivimos el acontecimiento de la resurrección, que ha cambiado el rostro de nuestra vida, ha cambiado la historia de la humanidad. Cuantos permanecen todavía bajo las cadenas del sufrimiento y la muerte, aguardan, a veces de modo inconsciente, la esperanza de Cristo resucitado. \end{body}
			
			\begin{body}Que el espíritu del Resucitado traiga consuelo y seguridad [...]\footnote{17} \end{body}
			
			\begin{body}Que el Señor Resucitado haga sentir por todas partes su fuerza de vida, de paz y de libertad. Las palabras con las que el ángel confortó los corazones atemorizados de las mujeres en la mañana de Pascua, se dirigen a todos: “¡No tengáis miedo!...No está aquí. Ha resucitado” (\textit{Mt} 28, 5-6). Jesús ha resucitado y nos da la paz; Él mismo es la paz. Por eso la Iglesia repite con firmeza: “Cristo ha resucitado – \textit{Christós anésti”}. Que la humanidad del tercer milenio no tenga miedo de abrirle el corazón. Su Evangelio sacia plenamente el anhelo de paz y de felicidad que habita en todo corazón humano. Cristo ahora está vivo y camina con nosotros. ¡Inmenso misterio de amor! \textit{Christus resurrexit, quia Deus caritas est! Alleluia.}\end{body}
			
			\subsubsection{Homilía (2009): No manda en ti la muerte}
			
			\begin{referencia}Domingo de Pascua, 12 de abril de 2009.\end{referencia}
			
			\begin{body}\textit{“Ha sido inmolado Cristo, nuestra Pascua}” (\textit{1 Co} 5, 7). Resuena en este día la exclamación de san Pablo que hemos escuchado en la \textbf{segunda lectura}, tomada de la primera \textit{Carta a los Corintios}. Un texto que se remonta a veinte años apenas después de la muerte y resurrección de Jesús y que, no obstante, contiene en una síntesis impresionante –como es típico de algunas expresiones paulinas– la plena conciencia de la novedad cristiana. El símbolo central de la historia de la salvación –el cordero pascual– se identifica aquí con Jesús, llamado precisamente “nuestra Pascua”. La Pascua judía, memorial de la liberación de la esclavitud de Egipto, prescribía el rito de la inmolación del cordero, un cordero por familia, según la ley mosaica. En su pasión y muerte, Jesús se revela como el Cordero de Dios “inmolado” en la cruz para quitar los pecados del mundo; fue muerto justamente en la hora en que se acostumbraba a inmolar los corderos en el Templo de Jerusalén. El sentido de este sacrificio suyo, lo había anticipado Él mismo durante la Última Cena, poniéndose en el lugar –bajo las especies del pan y el vino– de los elementos rituales de la cena de la Pascua. Así, podemos decir que Jesús, realmente, ha llevado a cumplimiento la tradición de la antigua Pascua y la ha transformado en \textit{su} Pascua. \end{body}
			
			\begin{body}A partir de este nuevo sentido de la fiesta pascual, se comprende también la interpretación de san Pablo sobre los “ázimos”. El Apóstol se refiere a una antigua costumbre judía, según la cual en la Pascua había que limpiar la casa hasta de las migajas de pan fermentado. Eso formaba parte del recuerdo de lo que había pasado con los antepasados en el momento de su huída de Egipto: teniendo que salir a toda prisa del país, llevaron consigo solamente panes sin levadura. Pero, al mismo tiempo, “los ázimos” eran un símbolo de purificación: eliminar lo viejo para dejar espacio a lo nuevo. Ahora, como explica san Pablo, también esta antigua tradición adquiere un nuevo sentido, precisamente a partir del nuevo “éxodo” que es el paso de Jesús de la muerte a la vida eterna. Y puesto que Cristo, como el verdadero Cordero, se ha sacrificado a sí mismo por nosotros, también nosotros, sus discípulos –gracias a Él y por medio de Él– podemos y debemos ser “masa nueva”, “ázimos”, liberados de todo residuo del viejo fermento del pecado: ya no más malicia y perversidad en nuestro corazón.\end{body}
			
			\begin{body}“Así, pues, celebremos la Pascua... con los panes ázimos de la sinceridad y la verdad”. Esta exhortación de san Pablo con que termina la breve lectura que se ha proclamado hace poco, \begin{bodysmall}[resuena aún más intensamente en el contexto del Año Paulino]\end{bodysmall}. Queridos hermanos y hermanas, acojamos la invitación del Apóstol; abramos el corazón a Cristo muerto y resucitado para que nos renueve, para que nos limpie del veneno del pecado y de la muerte y nos infunda la savia vital del Espíritu Santo: la vida divina y eterna. En la secuencia pascual, como haciendo eco a las palabras del Apóstol, hemos cantado: \textit{“Scimus Christum surrexisse a mortuis vere”} – sabemos que estás resucitado, la muerte en ti no manda. Sí, éste es precisamente el núcleo fundamental de nuestra profesión de fe; éste es hoy el grito de victoria que nos une a todos. Y si Jesús ha resucitado, y por tanto está vivo, ¿quién podrá jamás separarnos de Él? ¿Quién podrá privarnos de su amor que ha vencido al odio y ha derrotado la muerte? Que el anuncio de la Pascua se propague por el mundo con el jubiloso canto del \textit{aleluya}. Cantémoslo con la boca, cantémoslo sobre todo con el corazón y con la vida, con un estilo de vida “ázimo”, simple, humilde, y fecundo de buenas obras. \textit{“Surrexit Christus spes mea: precedet vos in Galileam”} – ¡Resucitó de veras mi esperanza! Venid a Galilea, el Señor allí aguarda. El Resucitado nos precede y nos acompaña por las vías del mundo. Él es nuestra esperanza, Él es la verdadera paz del mundo. Amén.\end{body}
			
			\subsubsection{Regina Caeli (2009): He resucitado y estoy contigo}
			
			\begin{referencia}Palacio pontificio de Castelgandolfo. Lunes de la octava de Pascua, 13 de abril de 2009.\end{referencia}
			
			\begin{body}En estos días pascuales oiremos resonar a menudo las palabras de Jesús: “He resucitado y estoy siempre contigo”. La Iglesia, haciéndose eco de este anuncio, proclama con júbilo: “Era verdad, ha resucitado el Señor, aleluya. A él la gloria y el poder por toda la eternidad”. Toda la Iglesia en fiesta manifiesta sus sentimientos cantando: “Este es el día en que actuó el Señor”. En efecto, al resucitar de entre los muertos, Jesús inauguró su día eterno y también abrió la puerta de nuestra alegría. “No he de morir –dice–, viviré”. El Hijo del hombre crucificado, piedra desechada por los arquitectos, es ahora el sólido cimiento del nuevo edificio espiritual, que es la Iglesia, su Cuerpo místico. El pueblo de Dios, cuya Cabeza invisible es Cristo, está destinado a crecer a lo largo de los siglos, hasta el pleno cumplimiento del plan de la salvación. Entonces toda la humanidad se incorporará a él y toda realidad existente participará en su victoria definitiva. Entonces –escribe san Pablo–, él será “la plenitud de todas las cosas” (\textit{Ef} 1, 23) y “Dios será todo en todos” (\textit{1 Co} 15, 28). \end{body}
			
			\begin{body}Por tanto, la comunidad cristiana se alegra porque la resurrección del Señor nos garantiza que el plan divino de la salvación se cumplirá con seguridad, no obstante toda la oscuridad de la historia. Precisamente por eso su Pascua es en verdad nuestra esperanza. Y nosotros, resucitados con Cristo mediante el Bautismo, debemos seguirlo ahora fielmente con una vida santa, caminando hacia la Pascua eterna, sostenidos por la certeza de que las dificultades, las luchas, las pruebas y los sufrimientos de nuestra existencia, incluida la muerte, ya no podrán separarnos de él y de su amor. Su resurrección ha creado un puente entre el mundo y la vida eterna, por el que todo hombre y toda mujer pueden pasar para llegar a la verdadera meta de nuestra peregrinación terrena. \end{body}
			
			\begin{body}“He resucitado y estoy siempre contigo”. Esta afirmación de Jesús se realiza sobre todo en la Eucaristía; en toda celebración eucarística la Iglesia, y cada uno de sus miembros, experimentan su presencia viva y se benefician de toda la riqueza de su amor. En el sacramento de la Eucaristía está presente el Señor resucitado y, lleno de misericordia, nos purifica de nuestras culpas; nos alimenta espiritualmente y nos infunde vigor para afrontar las duras pruebas de la existencia y para luchar contra el pecado y el mal. Él es el apoyo seguro de nuestra peregrinación hacia la morada eterna del cielo. \end{body}
			
			\begin{body}La Virgen María, que vivió junto a su divino Hijo cada fase de su misión en la tierra, nos ayude a acoger con fe el don de la Pascua y nos convierta en testigos felices, fieles y gozosos del Señor resucitado.\end{body}
			
			\subsubsection{Urbi et Orbi (2009): Nuestra esperanza}
			
			\begin{referencia}12 de abril de 2009. \end{referencia}
			
			\begin{body}\textit{Queridos hermanos y hermanas de Roma y del mundo entero: }\end{body}
			
			\begin{body}A todos vosotros dirijo de corazón la felicitación pascual con las palabras de san Agustín: “\textit{Resurrectio Domini, spes nostra”}, “la resurrección del Señor es nuestra esperanza” (\textit{Sermón} 261,1). Con esta afirmación, el gran Obispo explicaba a sus fieles que Jesús resucitó para que nosotros, aunque destinados a la muerte, no desesperáramos, pensando que con la muerte se acaba totalmente la vida; Cristo ha resucitado para darnos la esperanza (cf. \textit{ibíd}.). \end{body}
			
			\begin{body}En efecto, una de las preguntas que más angustian la existencia del hombre es precisamente ésta: ¿qué hay después de la muerte? Esta solemnidad nos permite responder a este enigma afirmando que la muerte no tiene la última palabra, porque al final es la Vida la que triunfa. Nuestra certeza no se basa en simples razonamientos humanos, sino en un dato histórico de fe: Jesucristo, crucificado y sepultado, ha resucitado con su cuerpo glorioso. Jesús ha resucitado para que también nosotros, creyendo en Él, podamos tener la vida eterna. Este anuncio está en el corazón del mensaje evangélico. San Pablo lo afirma con fuerza: “Si Cristo no ha resucitado, nuestra predicación carece de sentido y vuestra fe lo mismo”. Y añade: “Si nuestra esperanza en Cristo acaba con esta vida, somos los hombres más desgraciados” (\textit{1 Co} 15, 14. 19). Desde la aurora de Pascua una nueva primavera de esperanza llena el mundo; desde aquel día nuestra resurrección ya ha comenzado, porque la Pascua no marca simplemente un momento de la historia, sino el inicio de una condición nueva: Jesús ha resucitado no porque su recuerdo permanezca vivo en el corazón de sus discípulos, sino porque Él mismo vive en nosotros y en Él ya podemos gustar la alegría de la vida eterna.\end{body}
			
			\begin{body}Por tanto, la resurrección no es una teoría, sino una realidad histórica revelada por el Hombre Jesucristo mediante su “pascua”, su “paso”, que ha abierto una “nueva vía” entre la tierra y el Cielo (cf. \textit{Hb} 10, 20). No es un mito ni un sueño, no es una visión ni una utopía, no es una fábula, sino un acontecimiento único e irrepetible: Jesús de Nazaret, hijo de María, que en el crepúsculo del Viernes fue bajado de la cruz y sepultado, ha salido vencedor de la tumba. En efecto, al amanecer del primer día después del sábado, Pedro y Juan hallaron la tumba vacía. Magdalena y las otras mujeres encontraron a Jesús resucitado; lo reconocieron también los dos discípulos de Emaús en la fracción del pan; el Resucitado se apareció a los Apóstoles aquella tarde en el Cenáculo y luego a otros muchos discípulos en Galilea. \end{body}
			
			\begin{body}El anuncio de la resurrección del Señor ilumina las zonas oscuras del mundo en que vivimos. Me refiero particularmente al materialismo y al nihilismo, a esa visión del mundo que no logra transcender lo que es constatable experimentalmente, y se abate desconsolada en un sentimiento de la nada, que sería la meta definitiva de la existencia humana. En efecto, si Cristo no hubiera resucitado, el “vacío” acabaría ganando. Si quitamos a Cristo y su resurrección, no hay salida para el hombre, y toda su esperanza sería ilusoria. Pero, precisamente hoy, irrumpe con fuerza el anuncio de la resurrección del Señor, que responde a la pregunta recurrente de los escépticos, referida también por el libro del Eclesiastés: “¿Acaso hay algo de lo que se pueda decir: ‘Mira, esto es nuevo?’” (\textit{Qo} 1,10). Sí, contestamos: todo se ha renovado en la mañana de Pascua. “Lucharon vida y muerte en singular batalla y, muerto el que es Vida, triunfante se levanta” (Secuencia Pascual). Ésta es la novedad. Una novedad que cambia la existencia de quien la acoge, como sucedió a lo santos. Así, por ejemplo, le ocurrió a san Pablo. \end{body}
			
			\begin{body}[\begin{bodysmall}En el contexto del Año Paulino, hemos tenido ocasión muchas veces de meditar sobre la experiencia del gran Apóstol.]\end{bodysmall} Saulo de Tarso, el perseguidor encarnizado de los cristianos, encontró a Cristo resucitado en el camino de Damasco y fue “conquistado” por Él. El resto lo sabemos. A Pablo le sucedió lo que más tarde él escribirá a los cristianos de Corinto: “El que vive con Cristo, es una criatura nueva; lo viejo ha pasado, ha llegado lo nuevo” (\textit{2 Co} 5, 17). Fijémonos en este gran evangelizador, que con el entusiasmo audaz de su acción apostólica, llevó el Evangelio a muchos pueblos del mundo de entonces. Su enseñanza y su ejemplo nos impulsan a buscar al Señor Jesús. Nos animan a confiar en Él, porque ahora el sentido de la nada, que tiende a intoxicar la humanidad, ha sido vencido por la luz y la esperanza que surgen de la resurrección. Ahora son verdaderas y reales las palabras del Salmo: “Ni la tiniebla es oscura para ti, la noche es clara como el día” (139[138], 12). Ya no es la nada la que envuelve todo, sino la presencia amorosa de Dios. Más aún, hasta el reino mismo de la muerte ha sido liberado, porque también al “abismo” ha llegado el Verbo de la vida, aventado por el soplo del Espíritu (v. 8). \end{body}
			
			\begin{body}Aunque es verdad que la muerte ya no tiene poder sobre el hombre y el mundo, quedan todavía muchos, demasiados signos de su antiguo dominio. Aunque Cristo, por la Pascua, ha extirpado la raíz del mal, necesita hombres y mujeres que lo ayuden siempre y en todo lugar a afianzar su victoria con sus mismas armas: las armas de la justicia y de la verdad, de la misericordia, del perdón y del amor. [...]\footnote{18} En un tiempo de carestía global de alimentos, de desbarajuste financiero, de pobrezas antiguas y nuevas, de cambios climáticos preocupantes, de violencias y miserias que obligan a muchos a abandonar su tierra buscando una supervivencia menos incierta, de terrorismo siempre amenazante, de miedos crecientes ante un porvenir problemático, es urgente descubrir nuevamente perspectivas capaces de devolver la esperanza. Que nadie se arredre en esta batalla pacífica comenzada con la Pascua de Cristo, el cual, lo repito, busca hombres y mujeres que lo ayuden a afianzar su victoria con sus mismas armas, las de la justicia y la verdad, la misericordia, el perdón y el amor.\end{body}
			
			\begin{body}\textit{“Resurrectio Domini, spes nostra”. }La resurrección de Cristo es nuestra esperanza. La Iglesia proclama hoy esto con alegría: anuncia la esperanza, que Dios ha hecho firme e invencible resucitando a Jesucristo de entre los muertos; comunica la esperanza, que lleva en el corazón y quiere compartir con todos, en cualquier lugar, especialmente allí donde los cristianos sufren persecución a causa de su fe y su compromiso por la justicia y la paz; invoca la esperanza capaz de avivar el deseo del bien, también y sobre todo cuando cuesta. Hoy la Iglesia canta “el día en que actuó el Señor” e invita al gozo. Hoy la Iglesia ora, invoca a María, Estrella de la Esperanza, para que conduzca a la humanidad hacia el puerto seguro de la salvación, que es el corazón de Cristo, la Víctima pascual, el Cordero que “ha redimido al mundo”, el Inocente que nos “ha reconciliado a nosotros, pecadores, con el Padre”. A Él, Rey victorioso, a Él, crucificado y resucitado, gritamos con alegría nuestro \textit{Alleluia}.\end{body}
			
			\subsubsection{Regina Caeli (2012): Misterio decisivo de la fe}
			
			\begin{referencia}Castelgandolfo. Lunes del Ángel 9 de abril de 2012\end{referencia}
			
			\begin{body}¡Feliz día a todos vosotros! El lunes después de Pascua en muchos países es un día de vacación, en el que se puede dar un paseo en medio de la naturaleza o ir a visitar a parientes un poco lejanos para una reunión en familia. Pero quisiera que en la mente y en el corazón de los cristianos siempre estuviera presente el motivo de esta vacación, es decir, la resurrección de Jesús, el misterio decisivo de nuestra fe. De hecho, como escribe san Pablo a los Corintios, “si Cristo no ha resucitado, vana es nuestra predicación y vana también vuestra fe” (\textit{1 Co} 15, 14). Por eso, en estos días es importante releer los relatos de la resurrección de Cristo que encontramos en los cuatro Evangelios y leerlos con nuestro corazón. Se trata de relatos que, de modos diversos, presentan los encuentros de los discípulos con Jesús resucitado, y así nos permiten meditar en este acontecimiento estupendo que ha transformado la historia y da sentido a la existencia de todo hombre, de cada uno de nosotros.\end{body}
			
			\begin{body}Los evangelistas no describen el acontecimiento de la resurrección en cuanto tal. Ese acontecimiento permanece misterioso, no en el sentido de menos real, sino de oculto, más allá del alcance de nuestro conocimiento: como una luz tan deslumbrante que no se puede observar con los ojos, pues de lo contrario los cegaría. Los relatos comienzan, en cambio, desde que, al alba del día después del sábado, las mujeres se dirigieron al sepulcro y lo encontraron abierto y vacío. San Mateo habla también de un terremoto y de un ángel deslumbrante que corrió la gran piedra de la tumba y se sentó encima de ella (cf. \textit{Mt} 28, 2). Tras recibir del ángel el anuncio de la resurrección, las mujeres, llenas de miedo y de alegría, corrieron a dar la noticia a los discípulos, y precisamente en aquel momento se encontraron con Jesús, se postraron a sus pies y lo adoraron; y él les dijo: “No temáis; id a comunicar a mis hermanos que vayan a Galilea; allí me verán” (\textit{Mt} 28, 10). En todos los Evangelios las mujeres ocupan gran espacio en los relatos de las apariciones de Jesús resucitado, como también en los de la pasión y muerte de Jesús. En aquellos tiempos, en Israel, el testimonio de las mujeres no podía tener valor oficial, jurídico, pero las mujeres vivieron una experiencia de vínculo especial con el Señor, que es fundamental para la vida concreta de la comunidad cristiana, y esto siempre, en todas las épocas, no sólo al inicio del camino de la Iglesia.\end{body}
			
			\begin{body}Modelo sublime y ejemplar de esta relación con Jesús, de modo especial en su Misterio pascual, es naturalmente María, la Madre del Señor. Precisamente a través de la experiencia transformadora de la Pascua de su Hijo, la Virgen María se convierte también en Madre de la Iglesia, es decir, de cada uno de los creyentes y de toda la comunidad. A ella nos dirigimos ahora invocándola como \textit{Regina caeli}, con la oración que la tradición nos hace rezar en lugar del \textit{Ángelus} durante todo el tiempo pascual. Que María nos obtenga experimentar la presencia viva del Señor resucitado, fuente de esperanza y de paz.\end{body}
			
			\subsubsection{Urbi et Orbi (2012): Cristo, mi esperanza}
			
			\begin{referencia}8 de abril de 2012\end{referencia}
			
			\begin{body}\textit{Queridos hermanos y hermanas de Roma y del mundo entero: }\end{body}
			
			\begin{body}\textit{“Surrexit Christus, spes mea” }– “Resucitó Cristo, mi esperanza” (Secuencia pascual). \end{body}
			
			\begin{body}Llegue a todos vosotros la voz exultante de la Iglesia, con las palabras que el antiguo himno pone en labios de María Magdalena, la primera en encontrar en la mañana de Pascua a Jesús resucitado. Ella corrió hacia los otros discípulos y, con el corazón sobrecogido, les anunció: “He visto al Señor” (\textit{Jn} 20, 18). También nosotros, que hemos atravesado el desierto de la Cuaresma y los días dolorosos de la Pasión, hoy abrimos las puertas al grito de victoria: “¡Ha resucitado! ¡Ha resucitado verdaderamente!”.\end{body}
			
			\begin{body}Todo cristiano revive la experiencia de María Magdalena. Es un encuentro que cambia la vida: el encuentro con un hombre único, que nos hace sentir toda la bondad y la verdad de Dios, que nos libra del mal, no de un modo superficial, momentáneo, sino que nos libra de él radicalmente, nos cura completamente y nos devuelve nuestra dignidad. He aquí por qué la Magdalena llama a Jesús “mi esperanza”: porque ha sido Él quien la ha hecho renacer, le ha dado un futuro nuevo, una existencia buena, libre del mal. “Cristo, mi esperanza”, significa que cada deseo mío de bien encuentra en Él una posibilidad real: con Él puedo esperar que mi vida sea buena y sea plena, eterna, porque es Dios mismo que se ha hecho cercano hasta entrar en nuestra humanidad. \end{body}
			
			\begin{body}Pero María Magdalena, como los otros discípulos, han tenido que ver a Jesús rechazado por los jefes del pueblo, capturado, flagelado, condenado a muerte y crucificado. Debe haber sido insoportable ver la Bondad en persona sometida a la maldad humana, la Verdad escarnecida por la mentira, la Misericordia injuriada por la venganza. Con la muerte de Jesús, parecía fracasar la esperanza de cuantos confiaron en Él. Pero aquella fe nunca dejó de faltar completamente: sobre todo en el corazón de la Virgen María, la madre de Jesús, la llama quedó encendida con viveza también en la oscuridad de la noche. En este mundo, la esperanza no puede dejar de hacer cuentas con la dureza del mal. No es solamente el muro de la muerte lo que la obstaculiza, sino más aún las puntas aguzadas de la envidia y el orgullo, de la mentira y de la violencia. Jesús ha pasado por esta trama mortal, para abrirnos el paso hacia el reino de la vida. Hubo un momento en el que Jesús aparecía derrotado: las tinieblas habían invadido la tierra, el silencio de Dios era total, la esperanza una palabra que ya parecía vana. \end{body}
			
			\begin{body}Y he aquí que, al alba del día después del sábado, se encuentra el sepulcro vacío. Después, Jesús se manifiesta a la Magdalena, a las otras mujeres, a los discípulos. La fe renace más viva y más fuerte que nunca, ya invencible, porque fundada en una experiencia decisiva: \end{body}
			
			\begin{bodyprose}“Lucharon vida y muerte \end{bodyprose}
			
			\begin{bodyprose}en singular batalla, \end{bodyprose}
			
			\begin{bodyprose}y, muerto el que es Vida, triunfante se levanta”. \end{bodyprose}
			
			\begin{body}Las señales de la resurrección testimonian la victoria de la vida sobre la muerte, del amor sobre el odio, de la misericordia sobre la venganza: \end{body}
			
			\begin{bodyprose}“Mi Señor glorioso, \end{bodyprose}
			
			\begin{bodyprose}la tumba abandonada, \end{bodyprose}
			
			\begin{bodyprose}los ángeles testigos, \end{bodyprose}
			
			\begin{bodyprose}sudarios y mortaja”. \end{bodyprose}
			
			\begin{body}Queridos hermanos y hermanas: si Jesús ha resucitado, entonces –y sólo entonces– ha ocurrido algo realmente nuevo, que cambia la condición del hombre y del mundo. Entonces Él, Jesús, es alguien del que podemos fiarnos de modo absoluto, y no solamente confiar en su mensaje, sino precisamente \textit{en} \textit{Él}, porque el resucitado no pertenece al \textit{pasado}, sino que \textit{está presente }hoy, vivo. Cristo es esperanza y consuelo de modo particular para las comunidades cristianas que más pruebas padecen a causa de la fe, por discriminaciones y persecuciones. Y está presente como fuerza de esperanza a través de su Iglesia, cercano a cada situación humana de sufrimiento e injusticia. \end{body}
			
			\begin{body}[...]\footnote{19}\end{body}
			
			\begin{body}Feliz Pascua a todos. \end{body}
			
			\subsection{Francisco, papa}
			
			\subsubsection{Urbi et Orbi (2015): Inclinarse ante el misterio}
			
			\begin{referencia}Balcón central de la Basílica Vaticana. 5 de abril de 2015.\end{referencia}
			
			\begin{body}\textit{Queridos hermanos y hermanas}\end{body}
			
			\begin{body}¡Feliz Pascua! ¡Jesucristo ha resucitado!\end{body}
			
			\begin{body}El amor ha derrotado al odio, la vida ha vencido a la muerte, la luz ha disipado la oscuridad.\end{body}
			
			\begin{body}Jesucristo, por amor a nosotros, se despojó de su gloria divina; se vació de sí mismo, asumió la forma de siervo y se humilló hasta la muerte, y muerte de cruz. Por esto Dios lo ha exaltado y le ha hecho Señor del universo. Jesús es el Señor.\end{body}
			
			\begin{body}Con su muerte y resurrección, Jesús muestra a todos la vía de la vida y la felicidad: esta vía es \textit{la humildad}, que comporta \textit{la humillación}. Este es el camino que conduce a la gloria. Sólo quien se humilla puede ir hacia los “bienes de allá arriba”, a Dios (cf. \textit{Col} 3, 1-4). El orgulloso mira “desde arriba hacia abajo”, el humilde, “desde abajo hacia arriba”.\end{body}
			
			\begin{body}La mañana de Pascua, Pedro y Juan, advertidos por las mujeres, corrieron al sepulcro y lo encontraron abierto y vacío. Entonces, se acercaron y se “inclinaron” para entrar en la tumba. Para entrar en el misterio hay que “inclinarse”, abajarse. Sólo quien se abaja comprende la glorificación de Jesús y puede seguirlo en su camino.\end{body}
			
			\begin{body}El mundo propone imponerse a toda costa, competir, hacerse valer... Pero los cristianos, por la gracia de Cristo muerto y resucitado, \textit{son los brotes de otra humanidad}, en la cual tratamos de vivir al servicio de los demás, de no ser altivos, sino disponibles y respetuosos. Esto \textit{no es debilidad, sino auténtica fuerza}. Quien lleva en sí el poder de Dios, de su amor y su justicia, no necesita usar violencia, sino que habla y actúa con la fuerza de la verdad, de la belleza y del amor.\end{body}
			
			\begin{body}Imploremos hoy al Señor resucitado la gracia de no ceder al orgullo que fomenta la violencia y las guerras, sino de tener el valor humilde del perdón y de la paz. Pedimos a Jesús victorioso que alivie el sufrimiento de tantos hermanos nuestros perseguidos a causa de su nombre, así como de todos los que padecen injustamente las consecuencias de los conflictos y las violencias que se están produciendo, y que son tantas.\end{body}
			
			\begin{body}[...] \footnote{20}\end{body}
			
			\begin{body}Pidamos paz y libertad para tantos hombres y mujeres sometidos a nuevas y antiguas formas de esclavitud por parte de personas y organizaciones criminales. Paz y libertad para las víctimas de los traficantes de droga, muchas veces aliados con los poderes que deberían defender la paz y la armonía en la familia humana. E imploremos la paz para este mundo sometido a los traficantes de armas, que se enriquecen con la sangre de hombres y mujeres.\end{body}
			
			\begin{body}Y que a los marginados, los presos, los pobres y los emigrantes, tan a menudo rechazados, maltratados y desechados; a los enfermos y los que sufren; a los niños, especialmente aquellos sometidos a la violencia; a cuantos hoy están de luto; y a todos los hombres y mujeres de buena voluntad, llegue la voz consoladora y curativa del Señor Jesús: “Paz a vosotros” (\textit{Lc} 24,36). “No temáis, he resucitado y siempre estaré con vosotros” (cf. \textit{Misal Romano}, Antífona de entrada del día de Pascua).\end{body}
			
			\subsubsection{Homilía (2018): Situarse ante la Pascua}
			
			\begin{referencia}Plaza de San Pedro. 1 de abril de 2018.\end{referencia}
			
			\begin{body}Después de la escucha de la Pal<a id="_idTextAnchor055"></a>abra <a id="_idTextAnchor056"></a>de Dios, de este paso del Evangelio, me nace decir tres cosas.\end{body}
			
			\begin{body}Primero: el \textit{anuncio}. Ahí hay un anuncio: el Señor ha resucitado. Este anuncio que desde los primeros tiempos de los cristianos iba de boca en boca; era el saludo: el Señor ha resucitado. Y las mujeres, que fueron a ungir el cuerpo del Señor, se encontraron frente a una sorpresa. La sorpresa... Los anuncios de Dios son siempre sorpresas, porque nuestro Dios es el Dios de las sorpresas. Y así desde el inicio de la historia de la salvación, desde nuestro padre Abraham, Dios te sorprende: “Pero ve, ve, deja, vete de tu tierra”. Y siempre hay una sorpresa detrás de la otra. Dios no sabe hacer un anuncio sin sorprendernos. Y la sorpresa es lo que te conmueve el corazón, lo que te toca precisamente allí, donde tú no lo esperas. Para decirlo un poco con un lenguaje de los jóvenes: la sorpresa es un golpe bajo; tú no te lo esperas. Y Él va y te conmueve. Primero: el anuncio hecho sorpresa.\end{body}
			
			\begin{body}Segundo: la \textit{prisa}. Las mujeres corren, van deprisa a decir: “¡Pero hemos encontrado esto!”. Las sorpresas de Dios nos ponen en camino, inmediatamente, sin esperar. Y así corren para ver. Y Pedro y Juan corren. Los pastores la noche de Navidad corren: “Vamos a Belén a ver lo que nos han dicho los ángeles”. Y la Samaritana, corre para decir a su gente: “Esta es una novedad: he encontrado a un hombre que me ha dicho todo lo que he hecho”. Y la gente sabía las cosas que ella había hecho. Y aquella gente, corre, deja lo que está haciendo, también el ama de casa deja las patatas en la cazuela –las encontrará quemadas– pero lo importante es ir, correr, para ver esa sorpresa, ese anuncio. También hoy sucede.\end{body}
			
			\begin{body}En nuestros barrios, en los pueblos cuando sucede algo extraordinario, la gente corre a ver. Ir deprisa. Andrés no perdió tiempo y fue deprisa donde Pedro a decirle: “Hemos encontrado al Mesías”. Las sorpresas, las buenas noticias, se dan siempre así: deprisa. En el Evangelio hay uno que se toma un poco de tiempo; no quiere arriesgar. Pero el Señor es bueno, lo espera con amor, es Tomás. “Yo creeré cuando vea las llagas”, dice. También el Señor tiene paciencia para aquellos que no van tan deprisa.\end{body}
			
			\begin{body}El anuncio-sorpresa, la respuesta deprisa y lo tercero que yo quisiera decir hoy es una pregunta: “¿Y yo qué? ¿Tengo el corazón abierto a las sorpresas de Dios? ¿Soy capaz de ir deprisa, o siempre con esa cantilena, ‘veré mañana, mañana’? ¿Qué me dice a mí la sorpresa?”. Juan y Pedro fueron deprisa al sepulcro. De Juan el Evangelio nos dice: “Creed”. También Pedro: “Creed”, pero a su modo, con la fe un poco mezclada con el remordimiento de haber negado al Señor. El anuncio causó sorpresa, la carrera/ir deprisa y la pregunta: ¿Y yo hoy en esta Pascua de 2018 qué hago? ¿Tú, qué haces?\end{body}
			
			\subsubsection{Urbi et Orbi (2018): Más allá de la muerte}
			
			\begin{referencia}Balcón central de la Basílica Vaticana. 1 de abril de 2018.\end{referencia}
			
			\begin{body}\textit{Queridos hermanos y hermanas, ¡Feliz Pascua!}\end{body}
			
			\begin{body}Jesús ha resucitado de entre los muertos.\end{body}
			
			\begin{body}Junto con el canto del aleluya, resuena en la Iglesia y en todo el mundo, este mensaje: Jesús es el Señor, el Padre lo ha resucitado y él vive para siempre en medio de nosotros.\end{body}
			
			\begin{body}Jesús mismo había preanunciado su muerte y resurrección con la imagen del \textit{grano de trigo}. Decía: “Si el grano de trigo no cae en tierra y muere, queda infecundo; pero si muere, da mucho fruto” (\textit{Jn} 12, 24). Y esto es lo que ha sucedido: Jesús, el grano de trigo sembrado por Dios en los surcos de la tierra, murió víctima del pecado del mundo, permaneció dos días en el sepulcro; pero en su muerte estaba presente toda la potencia del amor de Dios, que se liberó y se manifestó el tercer día, y que hoy celebramos: la Pascua de Cristo Señor.\end{body}
			
			\begin{body}Nosotros, cristianos, creemos y sabemos que la resurrección de Cristo es la verdadera esperanza del mundo, aquella que no defrauda. Es la fuerza del grano de trigo, del amor que se humilla y se da hasta el final, y que renueva realmente el mundo. También hoy esta fuerza produce fruto en los surcos de nuestra historia, marcada por tantas injusticias y violencias. Trae frutos de esperanza y dignidad donde hay miseria y exclusión, donde hay hambre y falta trabajo, a los prófugos y refugiados –tantas veces rechazados por la cultura actual del descarte–, a las víctimas del narcotráfico, de la trata de personas y de las distintas formas de esclavitud de nuestro tiempo.\end{body}
			
			\begin{body}\begin{bodysmall}[Y, hoy, nosotros pedimos frutos de paz para el mundo entero, comenzando por la amada y martirizada Siria, cuya población está extenuada por una guerra que no tiene fin. Que la luz de Cristo resucitado ilumine en esta Pascua las conciencias de todos los responsables políticos y militares, para que se ponga fin inmediatamente al exterminio que se está llevando a cabo, se respete el derecho humanitario y se proceda a facilitar el acceso a las ayudas que estos hermanos y hermanas nuestros necesitan urgentemente, asegurando al mismo tiempo las condiciones adecuadas para el regreso de los desplazados.\end{bodysmall}\end{body}
			
			\begin{body}\begin{bodysmall}Invocamos frutos de reconciliación para Tierra Santa, que en estos días también está siendo golpeada por conflictos abiertos que no respetan a los indefensos, para Yemen y para todo el Oriente Próximo, para que el diálogo y el respeto mutuo prevalezcan sobre las divisiones y la violencia. Que nuestros hermanos en Cristo, que sufren frecuentemente abusos y persecuciones, puedan ser testigos luminosos del Resucitado y de la victoria del bien sobre el mal.\end{bodysmall}\end{body}
			
			\begin{body}\begin{bodysmall}Suplicamos en este día frutos de esperanza para cuantos anhelan una vida más digna, sobre todo en aquellas regiones del continente africano que sufren por el hambre, por conflictos endémicos y el terrorismo. Que la paz del Resucitado sane las heridas en Sudán del Sur: abra los corazones al diálogo y a la comprensión mutua. No olvidemos a las víctimas de ese conflicto, especialmente a los niños. Que nunca falte la solidaridad para las numerosas personas obligadas a abandonar sus tierras y privadas del mínimo necesario para vivir.\end{bodysmall}\end{body}
			
			\begin{body}\begin{bodysmall}Imploramos frutos de diálogo para la península coreana, para que las conversaciones en curso promuevan la armonía y la pacificación de la región. Que los que tienen responsabilidades directas actúen con sabiduría y discernimiento para promover el bien del pueblo coreano y construir relaciones de confianza en el seno de la comunidad internacional.\end{bodysmall}\end{body}
			
			\begin{body}\begin{bodysmall}Pedimos frutos de paz para Ucrania, para que se fortalezcan los pasos en favor de la concordia y se faciliten las iniciativas humanitarias que necesita la población.\end{bodysmall}\end{body}
			
			\begin{body}\begin{bodysmall}Suplicamos frutos de consolación para el pueblo venezolano, el cual –como han escrito sus Pastores– vive en una especie de “tierra extranjera” en su propio país. Para que, por la fuerza de la resurrección del Señor Jesús, encuentre la vía justa, pacífica y humana para salir cuanto antes de la crisis política y humanitaria que lo oprime, y no falten la acogida y asistencia a cuantos entre sus hijos están obligados a abandonar su patria.]\end{bodysmall}\end{body}
			
			\begin{body}Traiga Cristo Resucitado frutos de vida nueva para los niños que, a causa de las guerras y el hambre, crecen sin esperanza, carentes de educación y de asistencia sanitaria; y también para los ancianos desechados por la cultura egoísta, que descarta a quien no es “productivo”.\end{body}
			
			\begin{body}Invocamos frutos de sabiduría para los que en todo el mundo tienen responsabilidades políticas, para que respeten siempre la dignidad humana, se esfuercen con dedicación al servicio del bien común y garanticen el desarrollo y la seguridad a los propios ciudadanos.\end{body}
			
			\begin{body}Queridos hermanos y hermanas:\end{body}
			
			\begin{body}También a nosotros, como a las mujeres que acudieron al sepulcro, van dirigidas estas palabras: “¿Por qué buscáis entre los muertos al que vive? No está aquí. Ha resucitado” (\textit{Lc} 24, 5-6). La muerte, la soledad y el miedo ya no son la última palabra. Hay una palabra que va más allá y que solo Dios puede pronunciar: es la palabra de la Resurrección (cf. Juan Pablo II, \textit{Palabras al término del Vía Crucis}). Ella, con la fuerza del amor de Dios, “ahuyenta los pecados, lava las culpas, devuelve la inocencia a los caídos, la alegría a los tristes, expulsa el odio, trae la concordia, doblega a los poderosos” (Pregón pascual).\end{body}
			
			\begin{body}¡Feliz Pascua a todos!\end{body}
			
			\section{Temas}
			
			\begin{ccetheme}La Resurrección de Cristo y nuestra resurrección \end{ccetheme}
			
			\begin{ccereference}\end{ccereference}CEC 638-655, 989, 1001-1002: </p>
			
			\begin{ccebody}\textbf{Al tercer día resucitó de entre los muertos}\end{ccebody}
			
			\begin{ccebody}\begin{ccenumber}638\end{ccenumber} “Os anunciamos la Buena Nueva de que la Promesa hecha a los padres Dios la ha cumplido en nosotros, los hijos, al resucitar a Jesús (\textit{Hch} 13, 32-33). La Resurrección de Jesús es la verdad culminante de nuestra fe en Cristo, creída y vivida por la primera comunidad cristiana como verdad central, transmitida como fundamental por la Tradición, establecida en los documentos del Nuevo Testamento, predicada como parte esencial del Misterio Pascual al mismo tiempo que la Cruz:\end{ccebody}
			
			\begin{ccecite}Cristo ha resucitado de los muertos,<br />con su muerte ha vencido a la muerte.<br />Y a los muertos ha dado la vida.\end{ccecite}
			
			\begin{ccecite}(Liturgia bizantina: \textit{Tropario del día de Pascua})\end{ccecite}
			
			\begin{ccebody}\begin{ccenumber}639\end{ccenumber} El misterio de la resurrección de Cristo es un acontecimiento real que tuvo manifestaciones históricamente comprobadas como lo atestigua el Nuevo Testamento. Ya san Pablo, hacia el año 56, puede escribir a los Corintios: “Porque os transmití, en primer lugar, lo que a mi vez recibí: que Cristo murió por nuestros pecados, según las Escrituras; que fue sepultado y que resucitó al tercer día, según las Escrituras; que se apareció a Cefas y luego a los Doce” (\textit{1 Co} 15, 3-4). El apóstol habla aquí de \textit{la tradición viva de la Resurrección} que recibió después de su conversión a las puertas de Damasco (cf. \textit{Hch} 9, 3-18).\end{ccebody}
			
			\begin{ccebody}\textbf{El sepulcro vacío}\end{ccebody}
			
			\begin{ccebody}\begin{ccenumber}640\end{ccenumber} “¿Por qué buscar entre los muertos al que vive? No está aquí, ha resucitado” (\textit{Lc} 24, 5-6). En el marco de los acontecimientos de Pascua, el primer elemento que se encuentra es el sepulcro vacío. No es en sí una prueba directa. La ausencia del cuerpo de Cristo en el sepulcro podría explicarse de otro modo (cf. \textit{Jn} 20,13; \textit{Mt} 28, 11-15). A pesar de eso, el sepulcro vacío ha constituido para todos un signo esencial. Su descubrimiento por los discípulos fue el primer paso para el reconocimiento del hecho de la Resurrección. Es el caso, en primer lugar, de las santas mujeres (cf. \textit{Lc} 24, 3. 22- 23), después de Pedro (cf. \textit{Lc} 24, 12). “El discípulo que Jesús amaba” (\textit{Jn} 20, 2) afirma que, al entrar en el sepulcro vacío y al descubrir “las vendas en el suelo” (\textit{Jn} 20, 6) “vio y creyó” (\textit{Jn} 20, 8). Eso supone que constató en el estado del sepulcro vacío (cf. \textit{Jn} 20, 5-7) que la ausencia del cuerpo de Jesús no había podido ser obra humana y que Jesús no había vuelto simplemente a una vida terrenal como había sido el caso de Lázaro (cf. \textit{Jn} 11, 44).\end{ccebody}
			
			\begin{ccebody}\textbf{Las apariciones del Resucitado}\end{ccebody}
			
			\begin{ccebody}\begin{ccenumber}641\end{ccenumber} María Magdalena y las santas mujeres, que iban a embalsamar el cuerpo de Jesús (cf. \textit{Mc} 16, 1; \textit{Lc} 24, 1) enterrado a prisa en la tarde del Viernes Santo por la llegada del Sábado (cf. \textit{Jn }19, 31. 42) fueron las primeras en encontrar al Resucitado (cf. \textit{Mt} 28, 9-10; \textit{Jn} 20, 11-18). Así las mujeres fueron las primeras mensajeras de la Resurrección de Cristo para los propios Apóstoles (cf. \textit{Lc} 24, 9-10). Jesús se apareció en seguida a ellos, primero a Pedro, después a los Doce (cf. \textit{1 Co} 15, 5). Pedro, llamado a confirmar en la fe a sus hermanos (cf. \textit{Lc} 22, 31-32), ve por tanto al Resucitado antes que los demás y sobre su testimonio es sobre el que la comunidad exclama: “¡Es verdad! ¡El Señor ha resucitado y se ha aparecido a Simón!” (\textit{Lc }24, 34).\end{ccebody}
			
			\begin{ccebody}\begin{ccenumber}642\end{ccenumber} Todo lo que sucedió en estas jornadas pascuales compromete a cada uno de los Apóstoles –y a Pedro en particular– en la construcción de la era nueva que comenzó en la mañana de Pascua. Como testigos del Resucitado, los Apóstoles son las piedras de fundación de su Iglesia. La fe de la primera comunidad de creyentes se funda en el testimonio de hombres concretos, conocidos de los cristianos y de los que la mayor parte aún vivían entre ellos. Estos “testigos de la Resurrección de Cristo” (cf. \textit{Hch} 1, 22) son ante todo Pedro y los Doce, pero no solamente ellos: Pablo habla claramente de más de quinientas personas a las que se apareció Jesús en una sola vez, además de Santiago y de todos los Apóstoles (cf. \textit{1 Co} 15, 4-8).\end{ccebody}
			
			\begin{ccebody}\begin{ccenumber}643\end{ccenumber} Ante estos testimonios es imposible interpretar la Resurrección de Cristo fuera del orden físico, y no reconocerlo como un hecho histórico. Sabemos por los hechos que la fe de los discípulos fue sometida a la prueba radical de la pasión y de la muerte en cruz de su Maestro, anunciada por Él de antemano (cf. \textit{Lc} 22, 31-32). La sacudida provocada por la pasión fue tan grande que los discípulos (por lo menos, algunos de ellos) no creyeron tan pronto en la noticia de la resurrección. Los evangelios, lejos de mostrarnos una comunidad arrobada por una exaltación mística, nos presentan a los discípulos abatidos (“la cara sombría”: \textit{Lc} 24, 17) y asustados (cf. \textit{Jn} 20, 19). Por eso no creyeron a las santas mujeres que regresaban del sepulcro y “sus palabras les parecían como desatinos” (\textit{Lc} 24, 11; cf. \textit{Mc} 16, 11. 13). Cuando Jesús se manifiesta a los once en la tarde de Pascua “les echó en cara su incredulidad y su dureza de cabeza por no haber creído a quienes le habían visto resucitado” (\textit{Mc} 16, 14).\end{ccebody}
			
			\begin{ccebody}\begin{ccenumber}644\end{ccenumber} Tan imposible les parece la cosa que, incluso puestos ante la realidad de Jesús resucitado, los discípulos dudan todavía (cf. \textit{Lc} 24, 38): creen ver un espíritu (cf. \textit{Lc} 24, 39). “No acaban de creerlo a causa de la alegría y estaban asombrados” (\textit{Lc} 24, 41). Tomás conocerá la misma prueba de la duda (cf. \textit{Jn} 20, 24-27) y, en su última aparición en Galilea referida por Mateo, “algunos sin embargo dudaron” (\textit{Mt} 28, 17). Por esto la hipótesis según la cual la resurrección habría sido un “producto” de la fe (o de la credulidad) de los apóstoles no tiene consistencia. Muy al contrario, su fe en la Resurrección nació –bajo la acción de la gracia divina– de la experiencia directa de la realidad de Jesús resucitado.\end{ccebody}
			
			\begin{ccebody}\textbf{El estado de la humanidad resucitada de Cristo}\end{ccebody}
			
			\begin{ccebody}\begin{ccenumber}645\end{ccenumber} Jesús resucitado establece con sus discípulos relaciones directas mediante el tacto (cf. \textit{Lc} 24, 39; \textit{Jn} 20, 27) y el compartir la comida (cf. \textit{Lc} 24, 30. 41-43; \textit{Jn} 21, 9. 13-15). Les invita así a reconocer que él no es un espíritu (cf. \textit{Lc} 24, 39), pero sobre todo a que comprueben que el cuerpo resucitado con el que se presenta ante ellos es el mismo que ha sido martirizado y crucificado, ya que sigue llevando las huellas de su pasión (cf. \textit{Lc} 24, 40; \textit{Jn} 20, 20. 27). Este cuerpo auténtico y real posee sin embargo al mismo tiempo, las propiedades nuevas de un cuerpo glorioso: no está situado en el espacio ni en el tiempo, pero puede hacerse presente a su voluntad donde quiere y cuando quiere (cf. \textit{Mt} 28, 9. 16-17; \textit{Lc} 24, 15. 36; \textit{Jn} 20, 14. 19. 26; 21, 4) porque su humanidad ya no puede ser retenida en la tierra y no pertenece ya más que al dominio divino del Padre (cf. \textit{Jn} 20, 17). Por esta razón también Jesús resucitado es soberanamente libre de aparecer como quiere: bajo la apariencia de un jardinero (cf. \textit{Jn} 20, 14-15) o “bajo otra figura” (\textit{Mc} 16, 12) distinta de la que les era familiar a los discípulos, y eso para suscitar su fe (cf. \textit{Jn} 20, 14. 16; 21, 4. 7).\end{ccebody}
			
			\begin{ccebody}\begin{ccenumber}646\end{ccenumber} La Resurrección de Cristo no fue un retorno a la vida terrena como en el caso de las resurrecciones que él había realizado antes de Pascua: la hija de Jairo, el joven de Naím, Lázaro. Estos hechos eran acontecimientos milagrosos, pero las personas afectadas por el milagro volvían a tener, por el poder de Jesús, una vida terrena “ordinaria”. En cierto momento, volverán a morir. La Resurrección de Cristo es esencialmente diferente. En su cuerpo resucitado, pasa del estado de muerte a otra vida más allá del tiempo y del espacio. En la Resurrección, el cuerpo de Jesús se llena del poder del Espíritu Santo; participa de la vida divina en el estado de su gloria, tanto que san Pablo puede decir de Cristo que es “el hombre celestial” (cf. \textit{1 Co} 15, 35-50).\end{ccebody}
			
			\begin{ccebody}\textbf{La Resurrección como acontecimiento transcendente}}\end{ccebody}
			
			\begin{ccebody}\begin{ccenumber}647\end{ccenumber} “¡Qué noche tan dichosa –canta el \textit{Exultet} de Pascua–, sólo ella conoció el momento en que Cristo resucitó de entre los muertos!”. En efecto, nadie fue testigo ocular del acontecimiento mismo de la Resurrección y ningún evangelista lo describe. Nadie puede decir cómo sucedió físicamente. Menos aún, su esencia más íntima, el paso a otra vida, fue perceptible a los sentidos. Acontecimiento histórico demostrable por la señal del sepulcro vacío y por la realidad de los encuentros de los Apóstoles con Cristo resucitado, no por ello la Resurrección pertenece menos al centro del Misterio de la fe en aquello que transciende y sobrepasa a la historia. Por eso, Cristo resucitado no se manifiesta al mundo (cf. \textit{Jn} 14, 22) sino a sus discípulos, “a los que habían subido con él desde Galilea a Jerusalén y que ahora son testigos suyos ante el pueblo” (\textit{Hch} 13, 31).\end{ccebody}
			
			\begin{ccebody}\textbf{La Resurrección obra de la Santísima Trinidad} \end{ccebody}
			
			\begin{ccebody}\begin{ccenumber}648\end{ccenumber} La Resurrección de Cristo es objeto de fe en cuanto es una intervención transcendente de Dios mismo en la creación y en la historia. En ella, las tres Personas divinas actúan juntas a la vez y manifiestan su propia originalidad. Se realiza por el poder del Padre que “ha resucitado” (\textit{Hch} 2, 24) a Cristo, su Hijo, y de este modo ha introducido de manera perfecta su humanidad –con su cuerpo– en la Trinidad. Jesús se revela definitivamente “Hijo de Dios con poder, según el Espíritu de santidad, por su resurrección de entre los muertos” (\textit{Rm} 1, 3-4). San Pablo insiste en la manifestación del poder de Dios (cf. \textit{Rm} 6, 4; 2 Co 13, 4; \textit{Flp} 3, 10; \textit{Ef} 1, 19-22; \textit{Hb} 7, 16) por la acción del Espíritu que ha vivificado la humanidad muerta de Jesús y la ha llamado al estado glorioso de Señor.\end{ccebody}
			
			\begin{ccebody}\begin{ccenumber}649\end{ccenumber} En cuanto al Hijo, él realiza su propia Resurrección en virtud de su poder divino. Jesús anuncia que el Hijo del hombre deberá sufrir mucho, morir y luego resucitar (sentido activo del término) (cf. \textit{Mc} 8, 31; 9, 9-31; 10, 34). Por otra parte, él afirma explícitamente: “Doy mi vida, para recobrarla de nuevo ... Tengo poder para darla y poder para recobrarla de nuevo” (\textit{Jn }10, 17-18). “Creemos que Jesús murió y resucitó” (\textit{1 Ts} 4, 14).\end{ccebody}
			
			\begin{ccebody}\begin{ccenumber}650\end{ccenumber} Los Padres contemplan la Resurrección a partir de la persona divina de Cristo que permaneció unida a su alma y a su cuerpo separados entre sí por la muerte: “Por la unidad de la naturaleza divina que permanece presente en cada una de las dos partes del hombre, las que antes estaban separadas y segregadas, éstas se unen de nuevo. Así la muerte se produce por la separación del compuesto humano, y la Resurrección por la unión de las dos partes separadas” (San Gregorio de Nisa, \textit{De tridui inter mortem et resurrectionem Domini nostri Iesu Christi spatio}; cf. también DS 325; 359; 369; 539).\end{ccebody}
			
			\begin{ccebody}\textbf{Sentido y alcance salvífico de la Resurrección}\end{ccebody}
			
			\begin{ccebody}\begin{ccenumber}651\end{ccenumber} “Si no resucitó Cristo, vana es nuestra predicación, vana también vuestra fe” (\textit{1 Co} 15, 14). La Resurrección constituye ante todo la confirmación de todo lo que Cristo hizo y enseñó. Todas las verdades, incluso las más inaccesibles al espíritu humano, encuentran su justificación si Cristo, al resucitar, ha dado la prueba definitiva de su autoridad divina según lo había prometido.\end{ccebody}
			
			\begin{ccebody}\begin{ccenumber}652\end{ccenumber} La Resurrección de Cristo \textit{es cumplimiento de las promesas} del Antiguo Testamento (cf. \textit{Lc} 24, 26-27. 44-48) y del mismo Jesús durante su vida terrenal (cf. \textit{Mt} 28, 6; \textit{Mc} 16, 7; \textit{Lc} 24, 6-7). La expresión “según las Escrituras” (cf. \textit{1 Co} 15, 3-4 y el Símbolo Niceno-Constantinopolitano: DS 150) indica que la Resurrección de Cristo cumplió estas predicciones.\end{ccebody}
			
			\begin{ccebody}\begin{ccenumber}653\end{ccenumber} La verdad de \textit{la divinidad de Jesús} es confirmada por su Resurrección. Él había dicho: “Cuando hayáis levantado al Hijo del hombre, entonces sabréis que Yo Soy” (\textit{Jn} 8, 28). La Resurrección del Crucificado demostró que verdaderamente, él era “Yo Soy”, el Hijo de Dios y Dios mismo. San Pablo pudo decir a los judíos: “La Promesa hecha a los padres Dios la ha cumplido en nosotros [...] al resucitar a Jesús, como está escrito en el salmo primero: ‘Hijo mío eres tú; yo te he engendrado hoy’” (\textit{Hch} 13, 32-33; cf. \textit{Sal} 2, 7). La Resurrección de Cristo está estrechamente unida al misterio de la Encarnación del Hijo de Dios: es su plenitud según el designio eterno de Dios.\end{ccebody}
			
			\begin{ccebody}\begin{ccenumber}654\end{ccenumber} Hay un doble aspecto en el misterio pascual: por su muerte nos libera del pecado, por su Resurrección nos abre el acceso a una nueva vida. Esta es, en primer lugar, la \textit{justificación} que nos devuelve a la gracia de Dios (cf. \textit{Rm} 4, 25) “a fin de que, al igual que Cristo fue resucitado de entre los muertos [...] así también nosotros vivamos una nueva vida” (\textit{Rm} 6, 4). Consiste en la victoria sobre la muerte y el pecado y en la nueva participación en la gracia (cf. \textit{Ef} 2, 4-5; \textit{1 P} 1, 3). Realiza la \textit{adopción filial} porque los hombres se convierten en hermanos de Cristo, como Jesús mismo llama a sus discípulos después de su Resurrección: “Id, avisad a mis hermanos” (\textit{Mt} 28, 10; \textit{Jn} 20, 17). Hermanos no por naturaleza, sino por don de la gracia, porque esta filiación adoptiva confiere una participación real en la vida del Hijo único, la que ha revelado plenamente en su Resurrección.\end{ccebody}
			
			\begin{ccebody}\begin{ccenumber}655\end{ccenumber} Por último, la Resurrección de Cristo –y el propio Cristo resucitado– es principio y fuente de \textit{nuestra resurrección futura}: “Cristo resucitó de entre los muertos como primicias de los que durmieron [...] del mismo modo que en Adán mueren todos, así también todos revivirán en Cristo” (\textit{1 Co} 15, 20-22). En la espera de que esto se realice, Cristo resucitado vive en el corazón de sus fieles. En Él los cristianos “saborean [...] los prodigios del mundo futuro” (\textit{Hb} 6,5) y su vida es arrastrada por Cristo al seno de la vida divina (cf. \textit{Col} 3, 1-3) para que ya no vivan para sí los que viven, sino para aquel que murió y resucitó por ellos” (\textit{2 Co} 5, 15).\end{ccebody}
			
			\begin{ccebody}\begin{ccenumber}989\end{ccenumber} Creemos firmemente, y así lo esperamos, que del mismo modo que Cristo ha resucitado verdaderamente de entre los muertos, y que vive para siempre, igualmente los justos después de su muerte vivirán para siempre con Cristo resucitado y que Él los resucitará en el último día (cf. \textit{Jn} 6, 39-40). Como la suya, nuestra resurrección será obra de la Santísima Trinidad:\end{ccebody}
			
			\begin{ccecite}“Si el Espíritu de Aquel que resucitó a Jesús de entre los muertos habita en vosotros, Aquel que resucitó a Jesús de entre los muertos dará también la vida a vuestros cuerpos mortales por su Espíritu que habita en vosotros” (\textit{Rm} 8, 11; cf. \textit{1 Ts} 4, 14; \textit{1 Co} 6, 14; \textit{2 Co} 4, 14; \textit{Flp} 3, 10-11).\end{ccecite}
			
			\begin{ccebody}\begin{ccenumber}1001\end{ccenumber} \textit{¿Cuándo?} Sin duda en el “último día” (\textit{Jn} 6, 39-40. 44. 54; 11, 24); “al fin del mundo” (LG 48). En efecto, la resurrección de los muertos está íntimamente asociada a la Parusía de Cristo:\end{ccebody}
			
			\begin{ccecite}“El Señor mismo, a la orden dada por la voz de un arcángel y por la trompeta de Dios, bajará del cielo, y los que murieron en Cristo resucitarán en primer lugar” (\textit{1 Ts} 4, 16).\end{ccecite}
			
			\begin{ccebody}\textbf{Resucitados con Cristo}\end{ccebody}
			
			\begin{ccebody}\begin{ccenumber}1002\end{ccenumber} Si es verdad que Cristo nos resucitará en “el último día”, también lo es, en cierto modo, que nosotros ya hemos resucitado con Cristo. En efecto, gracias al Espíritu Santo, la vida cristiana en la tierra es, desde ahora, una participación en la muerte y en la Resurrección de Cristo:\end{ccebody}
			
			\begin{ccecite}“Sepultados con él en el Bautismo, con él también habéis resucitado por la fe en la acción de Dios, que le resucitó de entre los muertos [...] Así pues, si habéis resucitado con Cristo, buscad las cosas de arriba, donde está Cristo sentado a la diestra de Dios” (\textit{Col} 2, 12; 3, 1).\end{ccecite}
			
			\begin{ccetheme}\textbf{La Pascua, el Día del Señor }\end{ccetheme}
			
			\begin{ccereference}\end{ccereference}CEC 647, 1167-1170, 1243, 1287: 
</p>
			
			\begin{ccebody}\textbf{La Resurrección como acontecimiento transcendente}\end{ccebody}
			
			\begin{ccebody}\begin{ccenumber}647\end{ccenumber} “¡Qué noche tan dichosa –canta el \textit{Exultet} de Pascua–, sólo ella conoció el momento en que Cristo resucitó de entre los muertos!”. En efecto, nadie fue testigo ocular del acontecimiento mismo de la Resurrección y ningún evangelista lo describe. Nadie puede decir cómo sucedió físicamente. Menos aún, su esencia más íntima, el paso a otra vida, fue perceptible a los sentidos. Acontecimiento histórico demostrable por la señal del sepulcro vacío y por la realidad de los encuentros de los Apóstoles con Cristo resucitado, no por ello la Resurrección pertenece menos al centro del Misterio de la fe en aquello que transciende y sobrepasa a la historia. Por eso, Cristo resucitado no se manifiesta al mundo (cf. \textit{Jn} 14, 22) sino a sus discípulos, “a los que habían subido con él desde Galilea a Jerusalén y que ahora son testigos suyos ante el pueblo” (\textit{Hch} 13, 31).\end{ccebody}
			
			\begin{ccebody}\begin{ccenumber}1167\end{ccenumber} El domingo es el día por excelencia de la asamblea litúrgica, en que los fieles “deben reunirse para, escuchando la Palabra de Dios y participando en la Eucaristía, recordar la pasión, la resurrección y la gloria del Señor Jesús y dar gracias a Dios, que los hizo renacer a la esperanza viva por la resurrección de Jesucristo de entre los muertos” (SC 106):\end{ccebody}
			
			\begin{ccecite}“Cuando meditamos, [oh Cristo], las maravillas que fueron realizadas en este día del domingo de tu santa y gloriosa Resurrección, decimos: Bendito es el día del domingo, porque en él tuvo comienzo la Creación [...] la salvación del mundo [...] la renovación del género humano [...] en él el cielo y la tierra se regocijaron y el universo entero quedó lleno de luz. Bendito es el día del domingo, porque en él fueron abiertas las puertas del paraíso para que Adán y todos los desterrados entren en él sin temor” (\textit{Fanqîth, Breviarium iuxta ritum Ecclesiae Antiochenae Syrorum}, v. 6 [Mossul 1886] p. 193b).\end{ccecite}
			
			\begin{ccebody}\textbf{El año litúrgico}\end{ccebody}
			
			\begin{ccebody}\begin{ccenumber}1168\end{ccenumber} A partir del “Triduo Pascual”, como de su fuente de luz, el tiempo nuevo de la Resurrección llena todo el año litúrgico con su resplandor. El año, gracias a esta fuente, queda progresivamente transfigurado por la liturgia. Es realmente “año de gracia del Señor” (cf. \textit{Lc} 4, 19). La economía de la salvación actúa en el marco del tiempo, pero desde su cumplimiento en la Pascua de Jesús y la efusión del Espíritu Santo, el fin de la historia es anticipado, como pregustado, y el Reino de Dios irrumpe en el tiempo de la humanidad.\end{ccebody}
			
			\begin{ccebody}\begin{ccenumber}1169\end{ccenumber} Por ello, la \textit{Pascua} no es simplemente una fiesta entre otras: es la “Fiesta de las fiestas”, “Solemnidad de las solemnidades”, como la Eucaristía es el Sacramento de los sacramentos (el gran sacramento). San Atanasio la llama “el gran domingo” (\textit{Epistula festivalis} 1 [año 329], 10: PG 26, 1366), así como la Semana Santa es llamada en Oriente “la gran semana”. El Misterio de la Resurrección, en el cual Cristo ha aplastado a la muerte, penetra en nuestro viejo tiempo con su poderosa energía, hasta que todo le esté sometido.\end{ccebody}
			
			\begin{ccebody}\begin{ccenumber}1170\end{ccenumber} En el Concilio de Nicea (año 325) todas las Iglesias se pusieron de acuerdo para que la Pascua cristiana fuese celebrada el domingo que sigue al plenilunio (14 del mes de Nisán) después del equinoccio de primavera. Por causa de los diversos métodos utilizados para calcular el 14 del mes de Nisán, en las Iglesias de Occidente y de Oriente no siempre coincide la fecha de la Pascua. Por eso, dichas Iglesias buscan hoy un acuerdo, para llegar de nuevo a celebrar en una fecha común el día de la Resurrección del Señor.\end{ccebody}
			
			\begin{ccebody}\begin{ccenumber}1243\end{ccenumber} La \textit{vestidura blanca} simboliza que el bautizado se ha “revestido de Cristo” (\textit{Ga} 3,27): ha resucitado con Cristo. El \textit{cirio} que se enciende en el Cirio Pascual, significa que Cristo ha iluminado al neófito. En Cristo, los bautizados son “la luz del mundo” (\textit{Mt} 5,14; cf. \textit{Flp} 2,15).\end{ccebody}
			
			\begin{ccebody}El nuevo bautizado es ahora hijo de Dios en el Hijo Único. Puede ya decir la oración de los hijos de Dios: \textit{el Padre Nuestro}.\end{ccebody}
			
			\begin{ccebody}\begin{ccenumber}1287\end{ccenumber} Ahora bien, esta plenitud del Espíritu no debía permanecer únicamente en el Mesías, sino que debía ser comunicada a \textit{todo el pueblo mesiánico} (cf. \textit{Ez} 36,25-27; \textit{Jl} 3,1-2). En repetidas ocasiones Cristo prometió esta efusión del Espíritu (cf. \textit{Lc} 12,12; \textit{Jn} 3,5-8; 7,37-39; 16,7-15; \textit{Hch} 1,8), promesa que realizó primero el día de Pascua (\textit{Jn} 20,22) y luego, de manera más manifiesta el día de Pentecostés (cf. \textit{Hch} 2,1-4). Llenos del Espíritu Santo, los Apóstoles comienzan a proclamar “las maravillas de Dios” (\textit{Hch} 2,11) y Pedro declara que esta efusión del Espíritu es el signo de los tiempos mesiánicos (cf. \textit{Hch} 2, 17-18). Los que creyeron en la predicación apostólica y se hicieron bautizar, recibieron a su vez el don del Espíritu Santo (cf. \textit{Hch} 2,38).\end{ccebody}
			
			\begin{ccetheme}Los Sacramentos de la Iniciación cristiana \end{ccetheme}
			
			\begin{ccereference}\end{ccereference}CEC 1212: </p>
			
			\begin{ccebody}\begin{ccenumber}1212\end{ccenumber} Mediante los sacramentos de la iniciación cristiana, el Bautismo, la Confirmación y la Eucaristía, se ponen los \textit{fundamentos} de toda vida cristiana. “La participación en la naturaleza divina, que los hombres reciben como don mediante la gracia de Cristo, tiene cierta analogía con el origen, el crecimiento y el sustento de la vida natural. En efecto, los fieles renacidos en el Bautismo se fortalecen con el sacramento de la Confirmación y, finalmente, son alimentados en la Eucaristía con el manjar de la vida eterna, y, así por medio de estos sacramentos de la iniciación cristiana, reciben cada vez con más abundancia los tesoros de la vida divina y avanzan hacia la perfección de la caridad” (Pablo VI, Const. apost. \textit{Divinae consortium naturae}; cf. \textit{Ritual de Iniciación Cristiana de Adultos}, Prenotandos 1-2).\end{ccebody}
			
			\begin{ccetheme}El Bautismo \end{ccetheme}
			
			\begin{ccereference}\end{ccereference}CEC 1214-1222, 1226-1228, 1234-1245, 1254: </p>
			
			\begin{ccebody}\textbf{El nombre de este sacramento}\end{ccebody}
			
			\begin{ccebody}\begin{ccenumber}1214\end{ccenumber} Este sacramento recibe el nombre de \textit{Bautismo} en razón del carácter del rito central mediante el que se celebra: bautizar (\textit{baptizein} en griego) significa “sumergir”, “introducir dentro del agua”; la “inmersión” en el agua simboliza el acto de sepultar al catecúmeno en la muerte de Cristo, de donde sale por la resurrección con Él (cf. \textit{Rm} 6,3-4; \textit{Col }2,12) como “nueva criatura” (\textit{2 Co} 5,17; \textit{Ga} 6,15).\end{ccebody}
			
			\begin{ccebody}\begin{ccenumber}1215\end{ccenumber} Este sacramento es llamado también \textit{“baño de regeneración y de renovación del Espíritu Santo”} (\textit{Tt} 3,5), porque significa y realiza ese nacimiento del agua y del Espíritu sin el cual “nadie puede entrar en el Reino de Dios” (\textit{Jn} 3,5).\end{ccebody}
			
			\begin{ccebody}\begin{ccenumber}1216\end{ccenumber} “Este baño es llamado \textit{iluminación} porque quienes reciben esta enseñanza (catequética) su espíritu es iluminado” (San Justino, \textit{Apología} 1,61). Habiendo recibido en el Bautismo al Verbo, “la luz verdadera que ilumina a todo hombre” (\textit{Jn} 1,9), el bautizado, “tras haber sido iluminado” (\textit{Hb} 10,32), se convierte en “hijo de la luz” (\textit{1 Ts} 5,5), y en “luz” él mismo (\textit{Ef }5,8):\end{ccebody}
			
			\begin{ccecite}El Bautismo “es el más bello y magnífico de los dones de Dios [...] lo llamamos don, gracia, unción, iluminación, vestidura de incorruptibilidad, baño de regeneración, sello y todo lo más precioso que hay. \textit{Don}, porque es conferido a los que no aportan nada; \textit{gracia}, porque es dado incluso a culpables; \textit{bautismo}, porque el pecado es sepultado en el agua; \textit{unción}, porque es sagrado y real (tales son los que son ungidos); \textit{iluminación}, porque es luz resplandeciente; \textit{vestidura}, porque cubre nuestra vergüenza; \textit{baño}, porque lava; \textit{sello}, porque nos guarda y es el signo de la soberanía de Dios” (San Gregorio Nacianceno, \textit{Oratio} 40,3-4).\end{ccecite}
			
			\begin{ccebody}\textbf{Las prefiguraciones del Bautismo en la Antigua Alianza}\end{ccebody}
			
			\begin{ccebody}\begin{ccenumber}1217\end{ccenumber} En la liturgia de la vigilia Pascual, cuando \textit{se bendice el agua bautismal}, la Iglesia hace solemnemente memoria de los grandes acontecimientos de la historia de la salvación que prefiguraban ya el misterio del Bautismo:\end{ccebody}
			
			\begin{ccecite}“¡Oh Dios! [...] que realizas en tus sacramentos obras admirables con tu poder invisible, y de diversos modos te has servido de tu criatura el agua para significar la gracia del bautismo” (\textit{Vigilia Pascual, Bendición del agua: Misal Romano}).\end{ccecite}
			
			\begin{ccebody}\begin{ccenumber}1218\end{ccenumber} Desde el origen del mundo, el agua, criatura humilde y admirable, es la fuente de la vida y de la fecundidad. La Sagrada Escritura dice que el Espíritu de Dios “se cernía” sobre ella (cf. \textit{Gn} 1,2):\end{ccebody}
			
			\begin{ccecite}“¡Oh Dios!, cuyo Espíritu, en los orígenes del mundo, se cernía sobre las aguas, para que ya desde entonces concibieran el poder de santificar” (\textit{Vigilia Pascual, Bendición del agua: Misal Romano}).\end{ccecite}
			
			\begin{ccebody}\begin{ccenumber}1219\end{ccenumber} La Iglesia ha visto en el arca de Noé una prefiguración de la salvación por el bautismo. En efecto, por medio de ella “unos pocos, es decir, ocho personas, fueron salvados a través del agua” (\textit{1 P} 3,20):\end{ccebody}
			
			\begin{ccecite}“¡Oh Dios!, que incluso en las aguas torrenciales del diluvio prefiguraste el nacimiento de la nueva humanidad, de modo que una misma agua pusiera fin al pecado y diera origen a la santidad” (\textit{Vigilia Pascual, Bendición del agua: Misal Romano}).\end{ccecite}
			
			\begin{ccebody}\begin{ccenumber}1220\end{ccenumber} Si el agua de manantial simboliza la vida, el agua del mar es un símbolo de la muerte. Por lo cual, pudo ser símbolo del misterio de la Cruz. Por este simbolismo el bautismo significa la comunión con la muerte de Cristo.\end{ccebody}
			
			\begin{ccebody}\begin{ccenumber}1221\end{ccenumber} Sobre todo el paso del mar Rojo, verdadera liberación de Israel de la esclavitud de Egipto, es el que anuncia la liberación obrada por el bautismo:\end{ccebody}
			
			\begin{ccecite}“Oh Dios!, que hiciste pasar a pie enjuto por el mar Rojo a los hijos de Abraham, para que el pueblo liberado de la esclavitud del faraón fuera imagen de la familia de los bautizados” (\textit{Vigilia Pascual, Bendición del agua: Misal Romano}).\end{ccecite}
			
			\begin{ccebody}\begin{ccenumber}1222\end{ccenumber} Finalmente, el Bautismo es prefigurado en el paso del Jordán, por el que el pueblo de Dios recibe el don de la tierra prometida a la descendencia de Abraham, imagen de la vida eterna. La promesa de esta herencia bienaventurada se cumple en la nueva Alianza.\end{ccebody}
			
			\begin{ccebody}\textbf{El Bautismo en la Iglesia}\end{ccebody}
			
			\begin{ccebody}\begin{ccenumber}1226\end{ccenumber} Desde el día de Pentecostés la Iglesia ha celebrado y administrado el santo Bautismo. En efecto, san Pedro declara a la multitud conmovida por su predicación: “Convertíos [...] y que cada uno de vosotros se haga bautizar en el nombre de Jesucristo, para remisión de vuestros pecados; y recibiréis el don del Espíritu Santo” (\textit{Hch} 2,38). Los Apóstoles y sus colaboradores ofrecen el bautismo a quien crea en Jesús: judíos, hombres temerosos de Dios, paganos (\textit{Hch} 2,41; 8,12-13; 10,48; 16,15). El Bautismo aparece siempre ligado a la fe: “Ten fe en el Señor Jesús y te salvarás tú y tu casa”, declara san Pablo a su carcelero en Filipos. El relato continúa: “el carcelero inmediatamente recibió el bautismo, él y todos los suyos” (\textit{Hch} 16,31-33).\end{ccebody}
			
			\begin{ccebody}\begin{ccenumber}1227\end{ccenumber} Según el apóstol san Pablo, por el Bautismo el creyente participa en la muerte de Cristo; es sepultado y resucita con Él:\end{ccebody}
			
			\begin{ccecite}“¿O es que ignoráis que cuantos fuimos bautizados en Cristo Jesús, fuimos bautizados en su muerte? Fuimos, pues, con él sepultados por el bautismo en la muerte, a fin de que, al igual que Cristo fue resucitado de entre los muertos por medio de la gloria del Padre, así también nosotros vivamos una vida nueva” (\textit{Rm} 6,3-4; cf. \textit{Col} 2,12).\end{ccecite}
			
			\begin{ccebody}Los bautizados se han “revestido de Cristo” (\textit{Ga} 3,27). Por el Espíritu Santo, el Bautismo es un baño que purifica, santifica y justifica (cf. \textit{1 Co} 6,11; 12,13).\end{ccebody}
			
			\begin{ccebody}\begin{ccenumber}1228\end{ccenumber} El Bautismo es, pues, un baño de agua en el que la “semilla incorruptible” de la Palabra de Dios produce su efecto vivificador (cf. \textit{1 P} 1,23; \textit{Ef} 5,26). San Agustín dirá del Bautismo: \textit{Accedit verbum ad elementum, et fit sacramentum} – “Se une la palabra a la materia, y se hace el sacramento” (\textit{In Iohannis evangelium tractatus} 80, 3).\end{ccebody}
			
			\begin{ccebody}\textbf{La mistagogia de la celebración}\end{ccebody}
			
			\begin{ccebody}\begin{ccenumber}1234\end{ccenumber} El sentido y la gracia del sacramento del Bautismo aparece claramente en los ritos de su celebración. Cuando se participa atentamente en los gestos y las palabras de esta celebración, los fieles se inician en las riquezas que este sacramento significa y realiza en cada nuevo bautizado.\end{ccebody}
			
			\begin{ccebody}\begin{ccenumber}1235\end{ccenumber} \textit{La señal de la cruz}, al comienzo de la celebración, señala la impronta de Cristo sobre el que le va a pertenecer y significa la gracia de la redención que Cristo nos ha adquirido por su cruz.\end{ccebody}
			
			\begin{ccebody}\begin{ccenumber}1236\end{ccenumber} \textit{El anuncio de la Palabra de Dios} ilumina con la verdad revelada a los candidatos y a la asamblea y suscita la respuesta de la fe, inseparable del Bautismo. En efecto, el Bautismo es de un modo particular “el sacramento de la fe” por ser la entrada sacramental en la vida de fe.\end{ccebody}
			
			\begin{ccebody}\begin{ccenumber}1237\end{ccenumber} Puesto que el Bautismo significa la liberación del pecado y de su instigador, el diablo, se pronuncian uno o varios \textit{exorcismos} sobre el candidato. Este es ungido con el óleo de los catecúmenos o bien el celebrante le impone la mano y el candidato renuncia explícitamente a Satanás. Así preparado, puede \textit{confesar la fe de la Iglesia}, a la cual será “confiado” por el Bautismo (cf. \textit{Rm} 6,17).\end{ccebody}
			
			\begin{ccebody}\begin{ccenumber}1238\end{ccenumber} El \textit{agua bautismal} es entonces consagrada mediante una oración de epíclesis (en el momento mismo o en la noche pascual). La Iglesia pide a Dios que, por medio de su Hijo, el poder del Espíritu Santo descienda sobre esta agua, a fin de que los que sean bautizados con ella “nazcan del agua y del Espíritu” (Jn 3,5).\end{ccebody}
			
			\begin{ccebody}\begin{ccenumber}1239\end{ccenumber} Sigue entonces \textit{el rito esencial} del sacramento: \textit{el Bautismo} propiamente dicho, que significa y realiza la muerte al pecado y la entrada en la vida de la Santísima Trinidad a través de la configuración con el misterio pascual de Cristo. El Bautismo es realizado de la manera más significativa mediante la triple inmersión en el agua bautismal. Pero desde la antigüedad puede ser también conferido derramando tres veces agua sobre la cabeza del candidato.\end{ccebody}
			
			\begin{ccebody}\begin{ccenumber}1240\end{ccenumber} En la Iglesia latina, esta triple infusión va acompañada de las palabras del ministro: “N., yo te bautizo en el nombre del Padre, y del Hijo y del Espíritu Santo”. En las liturgias orientales, estando el catecúmeno vuelto hacia el Oriente, el sacerdote dice: “El siervo de Dios, N., es bautizado en el nombre del Padre, y del Hijo y del Espíritu Santo”. Y mientras invoca a cada persona de la Santísima Trinidad, lo sumerge en el agua y lo saca de ella.\end{ccebody}
			
			\begin{ccebody}\begin{ccenumber}1241\end{ccenumber} \textit{La unción con el santo crisma}, óleo perfumado y consagrado por el obispo, significa el don del Espíritu Santo al nuevo bautizado. Ha llegado a ser un cristiano, es decir, “ungido” por el Espíritu Santo, incorporado a Cristo, que es ungido sacerdote, profeta y rey (cf. \textit{Ritual del Bautismo de niños}, 62).\end{ccebody}
			
			\begin{ccebody}\begin{ccenumber}1242\end{ccenumber} En la liturgia de las Iglesias de Oriente, la unción postbautismal es el sacramento de la Crismación (Confirmación). En la liturgia romana, dicha unción anuncia una segunda unción del santo crisma que dará el obispo: el sacramento de la Confirmación que, por así decirlo, “confirma” y da plenitud a la unción bautismal.\end{ccebody}
			
			\begin{ccebody}\begin{ccenumber}1243\end{ccenumber} La \textit{vestidura blanca} simboliza que el bautizado se ha “revestido de Cristo” (\textit{Ga} 3,27): ha resucitado con Cristo. El \textit{cirio} que se enciende en el Cirio Pascual, significa que Cristo ha iluminado al neófito. En Cristo, los bautizados son “la luz del mundo” (\textit{Mt} 5,14; cf. \textit{Flp} 2,15).\end{ccebody}
			
			\begin{ccebody}El nuevo bautizado es ahora hijo de Dios en el Hijo Único. Puede ya decir la oración de los hijos de Dios: \textit{el Padre Nuestro}.\end{ccebody}
			
			\begin{ccebody}\begin{ccenumber}1244\end{ccenumber} La \textit{primera comunión eucarística}. Hecho hijo de Dios, revestido de la túnica nupcial, el neófito es admitido “al festín de las bodas del Cordero” y recibe el alimento de la vida nueva, el Cuerpo y la Sangre de Cristo. Las Iglesias orientales conservan una conciencia viva de la unidad de la iniciación cristiana, por lo que dan la sagrada comunión a todos los nuevos bautizados y confirmados, incluso a los niños pequeños, recordando las palabras del Señor: “Dejad que los niños vengan a mí, no se lo impidáis” (\textit{Mc} 10,14). La Iglesia latina, que reserva el acceso a la Sagrada Comunión a los que han alcanzado el uso de razón, expresa cómo el Bautismo introduce a la Eucaristía acercando al altar al niño recién bautizado para la oración del Padre Nuestro.\end{ccebody}
			
			\begin{ccebody}\begin{ccenumber}1245\end{ccenumber} La \textit{bendición solemne} cierra la celebración del Bautismo. En el Bautismo de recién nacidos, la bendición de la madre ocupa un lugar especial.\end{ccebody}
			
			\begin{ccebody}\begin{ccenumber}1254\end{ccenumber} En todos los bautizados, niños o adultos, la fe debe crecer \textit{después} del Bautismo. Por eso, la Iglesia celebra cada año en la vigilia pascual la renovación de las promesas del Bautismo. La preparación al Bautismo sólo conduce al umbral de la vida nueva. El Bautismo es la fuente de la vida nueva en Cristo, de la cual brota toda la vida cristiana.\end{ccebody}
			
			\begin{ccetheme}La Confirmación \end{ccetheme}
			
			\begin{ccereference}\end{ccereference}CEC 1286-1289:</p>
			
			\begin{ccebody}\begin{ccenumber}1286\end{ccenumber} En el Antiguo Testamento, los profetas anunciaron que el Espíritu del Señor reposaría sobre el Mesías esperado (cf. \textit{Is} 11,2) para realizar su misión salvífica (cf. \textit{Lc} 4,16-22; \textit{Is} 61,1). El descenso del Espíritu Santo sobre Jesús en su Bautismo por Juan fue el signo de que Él era el que debía venir, el Mesías, el Hijo de Dios (\textit{Mt} 3,13-17; \textit{Jn} 1,33- 34). Habiendo sido concedido por obra del Espíritu Santo, toda su vida y toda su misión se realizan en una comunión total con el Espíritu Santo que el Padre le da “sin medida” (\textit{Jn} 3,34).\end{ccebody}
			
			\begin{ccebody}\begin{ccenumber}1287\end{ccenumber} Ahora bien, esta plenitud del Espíritu no debía permanecer únicamente en el Mesías, sino que debía ser comunicada a \textit{todo el pueblo mesiánico} (cf. \textit{Ez} 36,25-27; \textit{Jl} 3,1-2). En repetidas ocasiones Cristo prometió esta efusión del Espíritu (cf. \textit{Lc} 12,12; \textit{Jn} 3,5-8; 7,37-39; 16,7-15; \textit{Hch} 1,8), promesa que realizó primero el día de Pascua (\textit{Jn} 20,22) y luego, de manera más manifiesta el día de Pentecostés (cf. \textit{Hch} 2,1-4). Llenos del Espíritu Santo, los Apóstoles comienzan a proclamar “las maravillas de Dios” (\textit{Hch} 2,11) y Pedro declara que esta efusión del Espíritu es el signo de los tiempos mesiánicos (cf. \textit{Hch} 2, 17-18). Los que creyeron en la predicación apostólica y se hicieron bautizar, recibieron a su vez el don del Espíritu Santo (cf. \textit{Hch} 2,38).\end{ccebody}
			
			\begin{ccebody}\begin{ccenumber}1288\end{ccenumber} “Desde [...] aquel tiempo, los Apóstoles, en cumplimiento de la voluntad de Cristo, comunicaban a los neófitos, mediante la imposición de las manos, el don del Espíritu Santo, destinado a completar la gracia del Bautismo (cf. \textit{Hch} 8,15-17; 19,5-6). Esto explica por qué en la carta a los Hebreos se recuerda, entre los primeros elementos de la formación cristiana, la doctrina del Bautismo y de la imposición de las manos (cf. \textit{Hb} 6,2). Es esta imposición de las manos la que ha sido con toda razón considerada por la tradición católica como el primitivo origen del sacramento de la Confirmación, el cual perpetúa, en cierto modo, en la Iglesia, la gracia de Pentecostés” (Pablo VI, Const. apost. \textit{Divinae consortium naturae}).\end{ccebody}
			
			\begin{ccebody}\begin{ccenumber}1289 \end{ccenumber}Muy pronto, para mejor significar el don del Espíritu Santo, se añadió a la imposición de las manos una unción con óleo perfumado (crisma). Esta unción ilustra el nombre de “cristiano” que significa “ungido” y que tiene su origen en el nombre de Cristo, al que “Dios ungió con el Espíritu Santo” (\textit{Hch} 10,38). Y este rito de la unción existe hasta nuestros días tanto en Oriente como en Occidente. Por eso, en Oriente se llama a este sacramento crismación, unción con el crisma, o \textit{myron}, que significa “crisma”. En Occidente el nombre de \textit{Confirmación} sugiere que este sacramento al mismo tiempo confirma el Bautismo y robustece la gracia bautismal.\end{ccebody}
			
			\begin{ccetheme}La Eucaristía \end{ccetheme}
			
			\begin{ccereference}\end{ccereference}CEC 1322-1323: </p>
			
			\begin{ccebody}\begin{ccenumber}1322\end{ccenumber} La Sagrada Eucaristía culmina la iniciación cristiana. Los que han sido elevados a la dignidad del sacerdocio real por el Bautismo y configurados más profundamente con Cristo por la Confirmación, participan por medio de la Eucaristía con toda la comunidad en el sacrificio mismo del Señor.\end{ccebody}
			
			\begin{ccebody}\begin{ccenumber}1323\end{ccenumber} “Nuestro Salvador, en la última Cena, la noche en que fue entregado, instituyó el Sacrificio Eucarístico de su cuerpo y su sangre para perpetuar por los siglos, hasta su vuelta, el sacrificio de la cruz y confiar así a su Esposa amada, la Iglesia, el memorial de su muerte y resurrección, sacramento de piedad, signo de unidad, vínculo de amor, banquete pascual en el que se recibe a Cristo, el alma se llena de gracia y se nos da una prenda de la gloria futura” (SC 47).\end{ccebody}
			
			\chapter{Nota final}
			
			\begin{bodyintro}Esta obra nació en un momento de oscuridad: la epidemia del Coronavirus vino de improviso y con ella un confinamiento de muchos días. ¿Qué hacer con tantas horas que antes ocupaba en la misión de manera presencial? Me decidí por preparar y publicar esta colección, esperando que pueda servir también a otros hermanos sacerdotes y diáconos.\end{bodyintro}
			
			\begin{bodyintro}El precio de los libros se ha establecido con el menor margen de ganancia posible, porque los mismos no se publican con fines de lucro, sino sobre todo con una finalidad pastoral (quienes no puedan comprarlos podrán descargarlos en formato electrónico sin coste alguno). Las ganancias que me revengan serán destinadas a la ayuda pastoral de mi Parroquia San Pedro Claver en la República Dominicana. Es el lugar donde he recibido la fe y en el que la he vivido durante todos estos años unido a una comunidad de hermanos con rostros y vidas concretos. La tierra buena, donde Dios quiso plantarme.\end{bodyintro}
			
			\begin{bodyintro}Mi deseo sincero es que esta obra, nacida en un momento difícil para millones de personas, sea una luz a través del ministerio de la predicación.\end{bodyintro}
			
			\begin{referencia}Sevilla (España)\end{referencia}
			
			\begin{referencia}31 de mayo del 2020\end{referencia}
			
			\begin{referencia}Solemnidad de Pentecostés\end{referencia}
			
			\begin{referencia}y Visitación de la Bienaventurada Virgen María\end{referencia}
			
			\begin{patercite}\textbf{El Espíritu Santo en la experiencia del desierto}\end{patercite}
			
			\begin{patercite}Al “comienzo” de la misión mesiánica de Jesús vemos otro hecho interesante y sugestivo, narrado por los evangelistas, que lo hacen depender de la acción del Espíritu Santo: se trata de \textit{la experiencia del desierto}. Leemos en el evangelio según san Marcos: “A continuación (del bautismo), \textit{el Espíritu le empuja al desierto}” (\textit{Mc} 1, 12). Además, Mateo (4, 1) y Lucas (4, 1) afirman que Jesús “fue conducido por el Espíritu al desierto”. Estos textos ofrecen puntos de reflexión que nos llevan a una ulterior investigación sobre el misterio de la íntima unión de Jesús-Mesías con el Espíritu Santo, ya desde el inicio de la obra de la redención.\end{patercite}
			
			\begin{patercite}En primer lugar, una observación de carácter lingüístico: los verbos usados por los evangelistas (“fue conducido” por Mateo y Lucas; “le empuja”, por Marcos) expresan \textit{una iniciativa especialmente enérgica} por parte del Espíritu Santo, iniciativa que se inserta en la lógica de la vida espiritual y en la misma psicología de Jesús: acaba de recibir de Juan un “bautismo de penitencia”, y por ello siente la necesidad de un período de reflexión y de austeridad, aunque personalmente no tenía necesidad de penitencia, dado que estaba “lleno de gracia” y era “santo” desde el momento de su concepción (cf. \textit{Jn} 1, 14; \textit{Lc} 1, 35): como preparación para su ministerio mesiánico. Su misión le exige también vivir en medio de los hombres-pecadores, a quienes ha sido enviado a evangelizar y salvar (cf. santo Tomás, \textit{Summa Theol.,} III, q. 40, a. 1), en lucha contra el poder del demonio. De aquí la conveniencia de esta pausa en el desierto “\textit{para ser tentado por el diablo}”. Por lo tanto, Jesús sigue el impulso interior y se dirige adonde le sugiere el Espíritu Santo.\end{patercite}
			
			\begin{patercite}\textit{El desierto}, además de ser lugar de encuentro con Dios, es también lugar de tentación y de lucha espiritual. Durante la peregrinación a través del desierto, que se prolongó durante cuarenta años, el pueblo de Israel había sufrido muchas tentaciones y había cedido (cf. \textit{Ex} 32, 1-6; \textit{Nm} 14, 1-4; 21, 4-5; 25, 1-3; \textit{Sal }78, 17; \textit{1 Co }10, 7-10). Jesús va al desierto, casi remitiéndose a la experiencia histórica de su pueblo. Pero, a diferencia del comportamiento de Israel, en el momento de inaugurar su actividad mesiánica, es sobre todo \textit{dócil a la acción del Espíritu Santo, que le pide desde el interior aquella definitiva preparación para el cumplimiento de su misión}. Es un período de soledad y de prueba espiritual, que supera con la ayuda de la palabra de Dios y con la oración.\end{patercite}
			
			\begin{patercite}En el espíritu de la tradición bíblica, y en la línea con la psicología israelita, aquel número de “cuarenta días” podía relacionarse fácilmente con otros acontecimientos históricos, llenos de significado para la historia de la salvación: los cuarenta días del diluvio (cf. \textit{Gn} 7, 4. 17); los cuarenta días de permanencia de Moisés en el monte (cf. \textit{Ex} 24, 18); los cuarenta días de camino de Elías, alimentado con el pan prodigioso que le había dado nueva fuerza (cf. \textit{1 R} 19, 8). Según los evangelistas, Jesús, bajo la moción del Espíritu Santo, se acomoda, en lo que se refiere a la permanencia en el desierto, a este número tradicional y casi sagrado (cf. \textit{Mt} 4, 1; Lc 4, 1). Lo mismo hará también en el período de las apariciones a los Apóstoles tras la resurrección y la ascensión al cielo (cf. \textit{Hch} 1, 3). [+]\end{patercite}
			
			\begin{patercite}\textbf{San Juan Pablo II, papa}. \textit{Catequesis,} audiencia general, 21 de julio 1990, nn. 1-2.\end{patercite}
			
			\begin{patercite}[+] Jesús, por tanto, es conducido al desierto con el fin de afrontar \textit{las tentaciones de Satanás} y para que pueda tener, a la vez, un contacto más libre e íntimo con el Padre. Aquí conviene tener presente que los evangelistas suelen presentarnos \textit{el desierto} como \textit{el lugar donde reside Satanás}: baste recordar el pasaje de Lucas sobre el “espíritu inmundo” que “cuando sale del hombre, anda vagando por lugares áridos, en busca de reposo...” (\textit{Lc} 11, 24); y en el pasaje que nos narra el episodio del endemoniado de Gerasa que “era empujado por el demonio al desierto” (\textit{Lc} 8, 29).\end{patercite}
			
			\begin{patercite}En el caso de las tentaciones de Jesús, el ir al desierto es obra del Espíritu Santo, y ante todo significa el inicio de una demostración –se podría decir, incluso, de una nueva toma de conciencia– de la lucha que deberá mantener hasta el final de su vida contra Satanás, artífice del pecado. Venciendo sus tentaciones, manifiesta su propio poder salvífico sobre el pecado y la llegada del reino de Dios, como dirá un día: “Si por el Espíritu de Dios expulso yo los demonios, es que ha llegado a vosotros el reino de Dios” (\textit{Mt} 12, 28).\end{patercite}
			
			\begin{patercite}\textbf{San Juan Pablo II, papa}. \textit{Catequesis,} audiencia general, 21 de julio 1990, n. 3.\end{patercite}
			
			\begin{patercite}[+] Si observamos bien, en las tentaciones sufridas y vencidas por Jesús durante la “experiencia del desierto” se nota la oposición de Satanás contra la llegada del reino de Dios al mundo humano, directa o indirectamente expresada en los textos de los evangelistas. Las respuestas que da Jesús al tentador desenmascaran las intenciones esenciales del “padre de la mentira” (\textit{Jn} 8, 44), que trata de servirse, de modo perverso, de las palabras de la Escritura para alcanzar sus objetivos. Pero Jesús lo refuta apoyándose en la misma palabra de Dios, aplicada correctamente. La narración de los evangelistas incluye, tal vez, alguna reminiscencia y establece un paralelismo tanto con las análogas tentaciones del pueblo de Israel en los cuarenta años de peregrinación por el desierto (la búsqueda de alimento: cf. \textit{Dt} 8, 3; \textit{Ex} 16; la pretensión de la protección divina para satisfacerse a sí mismos: cf. \textit{Dt} 6, 16; \textit{Ex} 17, 1-7; la idolatría: cf. \textit{Dt} 6, 13; \textit{Ex }32, 1-6), como con diversos momentos de la vida de Moisés. Pero se podría decir que el episodio entra específicamente en la historia de Jesús por su lógica biográfica y teológica. Aún estando libre de pecado, Jesús pudo conocer las seducciones externas del mal (cf. \textit{Mt} 16, 23): y era conveniente que fuese tentado para llegar a ser el Nuevo Adán, nuestro guía, nuestro redentor clemente (cf. \textit{Mt} 26, 36-46; \textit{Hb} 2, 10. 17-18; 4, 15; 5, 2. 7-9).\end{patercite}
			
			\begin{patercite}En el fondo de todas las tentaciones estaba la perspectiva de \textit{un mesianismo político y} \textit{glorioso}, como se había difundido y había penetrado en el alma del pueblo de Israel. El diablo trata de inducir a Jesús a acoger esta falsa perspectiva, \end{patercite}
			
			\begin{patercite}porque es el enemigo del plan de Dios, de su ley, de su economía de salvación, y por tanto de Cristo, como aparece claro por el evangelio y los demás escritos del Nuevo Testamento (cf. \textit{Mt} 13, 39; \textit{Jn} 8, 44; 13, 2; \textit{Hch} 10, 38; \textit{Ef} 6, 11; \textit{1 Jn} 3, 8, etc.). Si también Cristo cayese, el imperio de Satanás, que se gloría de ser el amo del mundo (\textit{Lc} 4, 5-6), obtendría la victoria definitiva en la historia. Aquel momento de la lucha en el desierto es, por consiguiente, decisivo.\end{patercite}
			
			\begin{patercite}Jesús es consciente de ser enviado por el Padre para hacer presente el reino de Dios entre los hombres. Con ese fin acepta la tentación, tomando su lugar entre los pecadores, como había hecho ya en el Jordán, para servirles a todos de ejemplo (cf. San Agustín, \textit{De Trinitate}, 4, 13). Pero, por otra parte, en virtud de la “unción” del Espíritu Santo, llega a las mismas raíces del pecado y derrota al “padre de la mentira” (\textit{Jn} 8, 44). Por eso, va voluntariamente al encuentro de la tentación desde el comienzo de su ministerio, siguiendo el impulso del Espíritu Santo (cf. San Agustín, \textit{De Trinitate}, 13, 13).\end{patercite}
			
			\begin{patercite}Un día, dando cumplimiento a su obra, podrá proclamar: “Ahora es el juicio de este mundo; ahora el príncipe de este mundo será echado fuera” (\textit{Jn} 12, 31). Y la víspera de su pasión repetirá una vez más: “Llega el príncipe de este mundo. En mí no tiene ningún poder” (\textit{Jn} 14, 30); es más, “el príncipe de este mundo está (ya) juzgado” (\textit{Jn} 16, 11); “¡Ánimo!, yo he vencido al mundo” (\textit{Jn} 16, 33). La lucha contra el “padre de la mentira”, que es el “principe de este mundo”, iniciada en el desierto, alcanzará su culmen en el Gólgota: la victoria se alcanzará por medio de la cruz del Redentor.\end{patercite}
			
			\begin{patercite}Estamos, por tanto, llamados a reconocer el valor integral del desierto como lugar de una particular experiencia de Dios, como sucedió con Moisés (cf. \textit{Ex }24, 18), con Elías (\textit{1 R} 19, 8), y sobre todo con Jesús que, “conducido” por el Espíritu Santo, acepta realizar la misma experiencia: \textit{el contacto con Dios Padre} (cf. \textit{Os} 2, 16) \textit{en lucha contra las potencias} \textit{opuestas a Dios}. Su experiencia es ejemplar, y nos puede servir también como lección sobre la necesidad de la penitencia, no para Jesús que estaba libre de pecado, sino para todos nosotros. Jesús mismo un día alertará a sus discípulos sobre la necesidad \textit{de la oración y del ayuno }para echar a los “espíritus inmundos” (cf. \textit{Mc} 9, 29) y, en la tensión de la solitaria oración de Getsemaní, recomendará a los Apóstoles presentes: “\textit{Velad y orad,} \textit{para que no caigáis en tentación}; que el espíritu está pronto, pero la carne es débil” (\textit{Mc} 14, 38). Seamos conscientes de que, amoldándonos a Cristo victorioso en la experiencia del desierto, también nosotros tendremos un divino confortador: el Espíritu Santo Paráclito, pues el mismo Cristo ha prometido que “recibirá de lo suyo” y nos lo dará (cf. \textit{Jn} 16, 14): Él, que condujo al Mesías al desierto no sólo “para ser tentado” sino también para que diera la primera demostración de su poderosa victoria sobre el diablo y sobre su reino, tomará de la victoria de Cristo sobre el pecado y sobre Satanás, su primer artífice, para hacer partícipe de ella a todo el que sea tentado.\end{patercite}
			
			\begin{patercite}\textbf{San Juan Pablo II, papa}. \textit{Catequesis,} audiencia general, 21 de julio 1990, nn. 4-6.\end{patercite}
			
			\chapter{Notas}
			
				
			
					\footnote{1 A partir de esa nueva traducción de la Biblia se publicaron los nuevos Leccionarios de la Misa, que son los leccionarios oficiales desde el mes de septiembre del año 2016.}
			
					\footnote{2 Cf. NUALC nn. 27-31. }
			
					\footnote{3 Cf. Concilio Vaticano II, Const. \textit{Sacrosanctum Concilium} sobre la sagrada Liturgia, n. 109.}
			
					\footnote{4 Cf. Pablo VI, Const. Apost. \textit{P}œ\textit{nitemini}, del 17 de febrero de 1966, II, párr. 3: A.A.S. 58 (1966) p. 184.}
			
					\footnote{5 Por motivos pastorales, esta celebración podría tener lugar otro día durante la Semana Santa.}
			
					\footnote{6 \textit{Prenotandos} del Leccionario de la Misa, n. 97.}
			
					\footnote{7 Cf. Congregación para el Culto Divino y la Disciplina de los Sacramentos, \textit{Directorio sobre la piedad popular y la liturgia}, Ciduad del Vaticano (2002), nn. 124-139.}
			
					\footnote{8 El Papa hace referencia al pasaje de la Transfiguración según san Mateo, porque aún no se había reformado el Leccionario de la Misa, ocurrido dos años después, en 1969.}
			
					\footnote{9 Cf. NUALC nn. 18-21.}
			
					\footnote{10 Cf. Congregación para el Culto Divino y la Disciplina de los Sacramentos, \textit{Directorio sobre la piedad popular y la liturgia}, Ciduad del Vaticano (2002), nn. 140-151.}
			
					\footnote{11 Cf. Ceremonial de los Obispos, n. 297.}
			
					\footnote{12 Cf. Ceremonial de los Obispos, n. 312.}
			
					\footnote{13 Concilio Vaticano II, Constitución sobre la sagrada liturgia, \textit{Sacrosanctum Concilium}, n. 5.}
			
					\footnote{14 Cf. Ceremonial de los Obispos, nn. 332-335. }
			
					\footnote{15 San Agustín, \textit{Sermo} 219: PL 38, 108.}
			
					\footnote{16 Les aseguro que si ellos callan, gritarán las piedras.}
			
					\footnote{17 El Papa evoca los principales lugares del mundo donde actualmente hay conflictos.}
			
					\footnote{18 El Papa recuerda los principales conflictos en el mundo.}
			
					\footnote{19 El Papa recuerda los principales conflictos en el mundo, pidiendo por la paz.}
			
					\footnote{20 El Papa evoca los principales conflictos en el mundo.}
			