			
\part{Triduo Pascual}

\chapter{Introducción al Triduo Pascual}
\begin{introstyle}
\section{Normativa litúrgica}

La Iglesia, a lo largo del año, conmemora todo el Misterio de Cristo desde la Encarnación hasta Pentecostés, y la espera de la Venida del Señor. La obra de la redención humana y de la perfecta glorificación de Dios, fue realizada por Cristo principalmente por el Misterio Pascual, mediante el cual con su muerte destruyó nuestra muerte y con su Resurrección restauró nuestra vida. Por esta razón el santo Triduo pascual de la Pasión y Resurrección del Señor es el centro del año litúrgico. Así como el domingo constituye el núcleo de la semana, también la solemnidad de Pascua constituye el núcleo del año litúrgico\anote{id9}.

El Triduo de la Pasión y Resurrección del Señor comienza con la Misa vespertina de la Cena del Señor; tiene su centro en la Vigilia Pascual y concluye con las vísperas del domingo de Resurrección. El Viernes Santo de la Pasión del Señor y –según las posibilidades– también el Sábado Santo hasta la Vigilia Pascual se guarda en todas partes el sagrado ayuno pascual. 

La Vigilia Pascual, en la noche santa de la Resurrección del Señor, es considerada como \textquote{la madre de todas las santas vigilias}, en ella, la Iglesia espera en vela la Resurrección de Cristo y la celebra en los sacramentos. Por consiguiente, la celebración de esta santa Vigilia debe hacerse totalmente de noche, es decir, empezar después del comienzo de la noche y terminar antes del alba del domingo. 

\section{Indicaciones pastorales y de piedad sobre el Triduo}

Todos los años en el \textquote{sacratísimo triduo del crucificado, del sepultado y del resucitado} o Triduo pascual la Iglesia celebra, \textquote{en íntima comunión con Cristo su Esposo}, los grandes misterios de la redención humana\anote{id10}.

\subsection{Jueves Santo} 

\subsubsection{La Misa de la Cena del Señor}

Con esta misa, que se celebra en la tarde del jueves de la Semana Santa, la Iglesia comienza el santo Triduo Pascual, y desea conmemorar aquella última cena en la que el Señor Jesús, la noche en que iba a ser entregado, amando hasta el extremo a los suyos que estaban en el mundo, ofreció a Dios Padre su Cuerpo y su Sangre bajo las especies de pan y vino, y lo entregó a los apóstoles para que lo tomaran, ordenándoles a ellos y a sus sucesores en el sacerdocio que lo ofrecieran\anote{id11}.

Con esta misa, en efecto, se hace el memorial tanto de la institución de la eucaristía –es decir, el memorial de la Pascua del Señor, con el que se perpetúa en nosotros el sacrificio de la nueva ley, bajo los signos del sacramento– como también de la institución del sacerdocio, mediante el cual se perpetúan en el mundo la misión y el sacrificio de Cristo; además, es memorial de la caridad con que Cristo nos amó hasta la muerte. 

\subsubsection{La visita al lugar de la reserva}

La piedad popular es especialmente sensible a la adoración del santísimo Sacramento, que sigue a la celebración de la Misa \textit{en la cena del Señor}. A causa de un proceso histórico, que todavía no está del todo claro en algunas de sus fases, el lugar de la reserva se ha considerado como \textquote{santo sepulcro}; los fieles acudían para venerar a Jesús que después del descendimiento de la Cruz fue sepultado en la tumba, donde permaneció unas Cuarenta horas.

Es preciso iluminar a los fieles sobre el sentido de la reserva: realizada con austera solemnidad y ordenada esencialmente a la conservación del Cuerpo del Señor, para la comunión de los fieles en la Celebración litúrgica del Viernes Santo y para el Viático de los enfermos, es una invitación a la adoración, silenciosa y prolongada, del Sacramento admirable, instituido en este día.

Por lo tanto, para el lugar de la reserva hay que evitar el término \textquote{sepulcro} (\textquote{monumento}), y en su disposición no se le debe dar la forma de una sepultura; el sagrario no puede tener la forma de un sepulcro o urna funeraria: el Sacramento hay que conservarlo en un sagrario cerrado, sin hacer la exposición con la custodia.

Después de la media noche del Jueves Santo, la adoración se realiza sin solemnidad, pues ya ha comenzado el día de la Pasión del Señor.\anote{id12}

En este día en que \textquote{ha sido inmolada nuestra víctima pascual: Cristo} (\textit{1 Cor} 5, 7), tuvo manifiesto y cumplido efecto todo aquello que desde antiguo había sido misteriosamente prefigurado: sustituyó el verdadero Cordero al cordero simbólico, y con un único sacrificio se llevó a cumplimiento los de las diferentes víctimas precedentes.

\textquote{Cristo el Señor realizó esta obra de redención humana y de glorificación perfecta de Dios, preparada por las maravillas que Dios hizo en el pueblo de la Antigua Alianza, principalmente por el misterio pascual de su bienaventurada Pasión, de su Resurrección de entre los muertos y de su gloriosa Ascensión. ‘Por este misterio, con su muerte destruyó nuestra muerte y con su Resurrección restauró nuestra vida’. Pues del costado de Cristo dormido en la cruz nació el sacramento admirable de toda la Iglesia}\anote{id13}.

La Iglesia, contemplando la cruz de su Señor y Esposo, conmemora su propio nacimiento y su misión de extender a todos los pueblos los maravillosos efectos de la pasión de Cristo, que hoy celebra, dado gracias por un don tan sublime.

\subsection{Viernes Santo} 

\subsubsection{La procesión del Viernes Santo}

El Viernes Santo la Iglesia celebra la Muerte salvadora de Cristo. En el Acto litúrgico de la tarde, medita en la Pasión de su Señor, intercede por la salvación del mundo, adora la Cruz y conmemora su propio nacimiento del costado abierto del Salvador (cfr. \textit{Jn} 19, 34).

Entre las manifestaciones de piedad popular del Viernes Santo, además del \textit{Vía Crucis}, destaca la procesión del \textquote{Cristo muerto}. Esta destaca, según las formas expresivas de la piedad popular, el pequeño grupo de amigos y discípulos que, después de haber bajado de la Cruz el Cuerpo de Jesús, lo llevaron al lugar en el cual había una \textquote{tumba excavada en la roca, en la cual todavía no se había dado sepultura a nadie} (\textit{Lc} 23, 53).

La procesión del \textquote{Cristo muerto} se desarrolla, por lo general, en un clima de austeridad, de silencio y de oración, con la participación de numerosos fieles, que perciben no pocos sentidos del misterio de la sepultura de Jesús.

Sin embargo, es necesario que estas manifestaciones de la piedad popular nunca aparezcan ante los fieles, ni por la hora ni por el modo de convocatoria, como sucedáneo de las celebraciones litúrgicas del Viernes Santo.

Por lo tanto, al planificar pastoralmente el Viernes Santo se deberá conceder el primer lugar y el máximo relieve a la Celebración litúrgica, y se deberá explicar a los fieles que ningún ejercicio de piedad debe sustituir a esta celebración, en su valor objetivo.

Finalmente, hay que evitar introducir la procesión de \textquote{Cristo muerto} en el ámbito de la solemne Celebración litúrgica del Viernes Santo, porque esto constituiría una mezcla híbrida de celebraciones.

\subsubsection{Representación de la Pasión de Cristo}

En muchas regiones, durante la Semana Santa, sobre todo el Viernes, tienen lugar representaciones de la Pasión de Cristo. Se trata, frecuentemente, de verdaderas \textquote{representaciones sagradas}, que con razón se pueden considerar un ejercicio de piedad. Las representaciones sagradas hunden sus raíces en la Liturgia. Algunas de ellas, nacidas casi en el coro de los monjes, mediante un proceso de dramatización progresiva, han pasado al atrio de la iglesia.

En muchos lugares, la preparación y ejecución de la representación de la Pasión de Cristo está encomendada a cofradías, cuyos miembros han asumido determinados compromisos de vida cristiana. En estas representaciones, actores y espectadores son introducidos en un movimiento de fe y de auténtica piedad. Es muy deseable que las representaciones sagradas de la Pasión del Señor no se alejen de este estilo de expresión sincera y gratuita de piedad, para convertirse en manifestaciones folclóricas, que atraen no tanto el espíritu religioso cuanto el interés de los turistas.

Respecto a las representaciones sagradas hay que explicar a los fieles la profunda diferencia que hay entre una \textquote{representación} que es mímesis, y la \textquote{acción litúrgica}, que es anámnesis, presencia mistérica del acontecimiento salvífico de la Pasión.

Hay que rechazar las prácticas penitenciales que consisten en hacerse crucificar con clavos.

\subsubsection{El recuerdo de la Virgen de los Dolores}

Dada su importancia doctrinal y pastoral, se recomienda no descuidar el \textquote{recuerdo de los dolores de la Santísima Virgen María}. La piedad popular, siguiendo el relato evangélico, ha destacado la asociación de la Madre a la Pasión salvadora del Hijo (cfr. \textit{Jn} 19, 25-27; \textit{Lc} 2, 34ss) y ha dado lugar a diversos ejercicios de piedad entre los que se deben recordar:

- el \textit{Planctus Mariae}, expresión intensa de dolor, que con frecuencia contiene elementos de gran valor literario y musical, en el que la Virgen llora no sólo la muerte del Hijo, inocente y santo, su bien sumo, sino también la pérdida de su pueblo y el pecado de la humanidad.

- la \textquote{Hora de la Dolorosa}, en la que los fieles, con expresiones de conmovedora devoción, \textquote{hacen compañía} a la Madre del Señor, que se ha quedado sola y sumergida en un profundo dolor, después de la muerte de su único Hijo; al contemplar a la Virgen con el Hijo entre sus brazos –la Piedad– comprenden que en María se concentra el dolor del universo por la muerte de Cristo; en ella ven la personificación de todas las madres que, a lo largo de la historia, han llorado la muerte de un hijo. Este ejercicio de piedad, que en algunos lugares de América Latina se denomina \textquote{El \textit{pésame}}, no se debe limitar a expresar el sentimiento humano ante una madre desolada, sino que, desde la fe en la Resurrección, debe ayudar a comprender la grandeza del amor redentor de Cristo y la participación en el mismo de su Madre.

\subsection{Sábado Santo} 

Durante el Sábado Santo la Iglesia permanece junto al sepulcro del Señor, meditando su Pasión y Muerte, su descenso a los infiernos y esperando en la oración y el ayuno su Resurrección.

La piedad popular no puede permanecer ajena al carácter particular del Sábado Santo; así pues, las costumbres y las tradiciones festivas vinculadas a este día, en el que durante una época se anticipaba la celebración pascual, se deben reservar para la noche y el día de Pascua.

\subsubsection{La \textquote{Hora de la Madre}}

En María, conforme a la enseñanza de la tradición, está como concentrado todo el cuerpo de la Iglesia: ella es la \textquote{credentium collectio universa}. Por esto la Virgen María, que permanece junto al sepulcro de su Hijo, tal como la representa la tradición eclesial, es imagen de la Iglesia Virgen que vela junto a la tumba de su Esposo, en espera de celebrar su Resurrección.

En esta intuición de la relación entre María y la Iglesia se inspira el ejercicio de piedad de la \textit{Hora de la Madre}: mientras el cuerpo del Hijo reposa en el sepulcro y su alma desciende a los infiernos para anunciar a sus antepasados la inminente liberación de la región de las tinieblas, la Virgen, anticipando y representando a la Iglesia, espera llena de fe la victoria del Hijo sobre la muerte.


\section{Domingo de Pascua de la Resurrección del Señor} 

\subsection{Vigilia pascual en la noche santa}

Según una antiquísima tradición, esta es la noche en que veló el Señor (cf. \textit{Ex} 12, 42). Los fieles, tal como lo recomienda el Evangelio (\textit{Lc} 12, 35-37), deben asemejarse a los criados que, con las lámparas encendidas en sus manos, esperan el retorno de su Señor, para que cuando llegue les encuentre en vela y los invite a sentarse a su mesa\anote{id14}.

La Vigilia de esta noche, en la que el Señor ha resucitado, se considera \textquote{la madre de todas las vigilias sagradas}\anote{id15}, la mayor y más noble de todas las solemnidades. Pues en ella, la Iglesia espera vigilante la Resurrección del Señor y en ella celebra los sacramentos de la iniciación cristiana.

Toda la celebración de la Vigilia pascual se realiza durante la noche, de modo que no debe comenzar antes de anochecer y debe concluir antes de que apunte la luz del domingo.

\subsection{Domingo de Pascua}

También en el Domingo de Pascua, máxima solemnidad del año litúrgico, tienen lugar no pocas manifestaciones de la piedad popular: son, todas, expresiones cultuales que exaltan la nueva condición y la gloria de Cristo resucitado, así como su poder divino que brota de su victoria sobre el pecado y sobre la muerte.

\subsubsection{El encuentro del Resucitado con la Madre}

La piedad popular ha intuido que la asociación del Hijo con la Madre es permanente: en la hora del dolor y de la muerte, en la hora de la alegría y de la Resurrección.

La afirmación litúrgica de que Dios ha colmado de alegría a la Virgen en la Resurrección del Hijo, ha sido, por decirlo de algún modo, traducida y representada por la piedad popular en el \textit{Encuentro de la Madre con el Hijo resucitado}: la mañana de Pascua dos procesiones, una con la imagen de la Madre dolorosa, otra con la de Cristo resucitado, se encuentran para significar que la Virgen fue la primera que participó, y plenamente, del misterio de la Resurrección del Hijo.

Para este ejercicio de piedad es válida la observación que se hizo respecto a la procesión del \textquote{Cristo muerto}: su realización no debe dar a entender que sea más importante que las celebraciones litúrgicas del domingo de Pascua, ni dar lugar a mezclas rituales inadecuadas.

\subsubsection{Bendición de la mesa familiar}

Toda la Liturgia pascual está penetrada de un sentido de novedad: es nueva la naturaleza, porque en el hemisferio norte la pascua coincide con el despertar primaveral; son nuevos el fuego y el agua; son nuevos los corazones de los cristianos, renovados por el sacramento de la Penitencia y, a ser posible, por los mismos sacramentos de la Iniciación cristiana; es nueva, por decirlo de alguna manera, la Eucaristía: son signos y realidades-signo de la nueva condición de vida inaugurada por Cristo con su Resurrección.

Entre los ejercicios de piedad que se relacionan con la Pascua se cuentan las tradicionales bendiciones de huevos, símbolos de vida, y la bendición de la mesa familiar; esta última, que es además una costumbre diaria de las familias cristianas, que se debe alentar, adquiere un significado particular en el día de Pascua: con el agua bendecida en la Vigilia Pascual, que los fieles llevan a sus hogares, según una loable costumbre, el cabeza de familia u otro miembro de la comunidad doméstica bendice la mesa pascual.

\subsubsection{El saludo pascual a la Madre del Resucitado}

En algunos lugares, al final de la Vigilia pascual o después de las II Vísperas del Domingo de Pascua, se realiza un breve ejercicio de piedad: se bendicen flores, que se distribuyen a los fieles como signo de la alegría pascual, y se rinde homenaje a la imagen de la Dolorosa, que a veces se corona, mientras se canta el \textit{Regina caeli}. Los fieles, que se habían asociado al dolor de la Virgen por la Pasión del Hijo, quieren así alegrarse con ella por el acontecimiento de la Resurrección.

Este ejercicio de piedad, que no se debe mezclar con el acto litúrgico, es conforme a los contenidos del Misterio pascual y constituye una prueba ulterior de cómo la piedad popular percibe la asociación de la Madre a la obra salvadora del Hijo.
\end{introstyle}