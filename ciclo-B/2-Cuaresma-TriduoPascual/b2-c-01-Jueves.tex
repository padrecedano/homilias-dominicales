\chapter{Jueves Santo en la Cena del Señor}

\section{Lecturas}

\rtitle{PRIMERA LECTURA}

\rbook{Del libro del Éxodo} \rred{12, 1-8. 11-14}

\rtheme{Prescripciones sobre la cena pascual}

\begin{scripture}
En aquellos días, dijo el Señor a Moisés y a Aarón en tierra de Egipto: 	

«Este mes será para vosotros el principal de los meses; será para vosotros el primer mes del año. Decid a toda la asamblea de los hijos de Israel: “El diez de este mes cada uno procurará un animal para su familia, uno por casa. Si la familia es demasiado pequeña para comérselo, que se junte con el vecino más próximo a su casa, hasta completar el número de personas; y cada uno comerá su parte hasta terminarlo. 

Será un animal sin defecto, macho, de un año; lo escogeréis entre los corderos o los cabritos. 

Lo guardaréis hasta el día catorce del mes y toda la asamblea de los hijos de Israel lo matará al atardecer”. Tomaréis la sangre y rociaréis las dos jambas y el dintel de la casa donde lo comáis. Esa noche comeréis la carne, asada a fuego, y comeréis panes sin fermentar y hierbas amargas.

Y lo comeréis así: la cintura ceñida, las sandalias en los pies, un bastón en la mano; y os lo comeréis a toda prisa, porque es la Pascua, el Paso del Señor. 

Yo pasaré esta noche por la tierra de Egipto y heriré a todos los primogénitos de la tierra de Egipto, desde los hombres hasta los ganados, y me tomaré justicia de todos los dioses de Egipto. Yo, el Señor. 

La sangre será vuestra señal en las casas donde habitáis. Cuando yo vea la sangre, pasaré de largo ante vosotros, y no habrá entre vosotros plaga exterminadora, cuando yo hiera a la tierra de Egipto. 

Este será un día memorable para vosotros; en él celebraréis fiesta en honor del Señor. De generación en generación, como ley perpetua lo festejaréis».

\end{scripture}

\rtitle{SALMO RESPONSORIAL}

\rbook{Salmo} \rred{115, 12-13. 15-16. 17-18}

\rtheme{El cáliz de la bendición es comunión de la sangre de Cristo}

\begin{psbody}
¿Cómo pagaré al Señor 
todo el bien que me ha hecho? 
Alzaré la copa de la salvación, 
invocando el nombre del Señor. 

Mucho le cuesta al Señor 
la muerte de sus fieles. 
Señor, yo soy tu siervo, 
hijo de tu esclava: 
rompiste mis cadenas. 

Te ofreceré un sacrificio de alabanza, 
invocando el nombre del Señor. 
Cumpliré al Señor mis votos 
en presencia de todo el pueblo. 
\end{psbody}

\rtitle{SEGUNDA LECTURA}

\rbook{De la primera carta del apóstol san Pablo a los Corintios} \rred{11, 23-26}

\rtheme{Cada vez que coméis de este pan y bebéis del cáliz, proclamáis la muerte del Señor}

\begin{scripture}
Hermanos: 

Yo he recibido una tradición, que procede del Señor y que a mi vez os he transmitido: Que el Señor Jesús, en la noche en que iba a ser entregado, tomó pan y, pronunciando la Acción de Gracias, lo partió y dijo: 

	
\>{Esto es mi cuerpo, que se entrega por vosotros. Haced esto en memoria mía}. 

Lo mismo hizo con el cáliz, después de cenar, diciendo: 

\>{Este cáliz es la nueva alianza en mi sangre; haced esto cada vez que lo bebáis, en memoria mía}. 
	
Por eso, cada vez que coméis de este pan y bebéis del cáliz, proclamáis la muerte del Señor, hasta que vuelva.
\end{scripture}


\newpage 
\rtitle{EVANGELIO}

\rbook{Del Evangelio según san Juan} \rred{13, 1-15}

\rtheme{Los amó hasta el extremo}

\begin{scripture}
Antes de la fiesta de la Pascua, sabiendo Jesús que había llegado su hora de pasar de este mundo al Padre, habiendo amado a los suyos que estaban en el mundo, los amó hasta el extremo. 

Estaban cenando; ya el diablo había suscitado en el corazón de Judas, hijo de Simón Iscariote, la intención de entregarlo; y Jesús, sabiendo que el Padre había puesto todo en sus manos, que venía de Dios y a Dios volvía, se levanta de la cena, se quita el manto y, tomando una toalla, se la ciñe; luego echa agua en la jofaina y se pone a lavarles los pies a los discípulos, secándoselos con la toalla que se había ceñido. 

Llegó a Simón Pedro y este le dice: 

\>{Señor, ¿lavarme los pies tú a mí?}.

Jesús le replicó: 

\>{Lo que yo hago, tú no lo entiendes ahora, pero lo comprenderás más tarde}.

Pedro le dice: 

\>{No me lavarás los pies jamás}.

Jesús le contestó: 

\>{Si no te lavo, no tienes parte conmigo}.

Simón Pedro le dice: 

\>{Señor, no solo los pies, sino también las manos y la cabeza}.

Jesús le dice: 

\>{Uno que se ha bañado no necesita lavarse más que los pies, porque todo él está limpio. También vosotros estáis limpios, aunque no todos}.

Porque sabía quién lo iba a entregar, por eso dijo: \textquote{No todos estáis limpios}. 

Cuando acabó de lavarles los pies, tomó el manto, se lo puso otra vez y les dijo: 

\>{¿Comprendéis lo que he hecho con vosotros? Vosotros me llamáis \textquote{el Maestro} y \textquote{el Señor}, y decís bien, porque lo soy. Pues si yo, el Maestro y el Señor, os he lavado los pies, también vosotros debéis lavaros los pies unos a otros: os he dado ejemplo para que lo que yo he hecho con vosotros, vosotros también lo hagáis}.
	
\end{scripture}




\newsection
\section{Comentario Patrístico}

\subsection{Francisco, papa}

\ptheme{Amor concreto}

\src{Audiencia jubilar, 12 de marzo de 2016.}

\begin{body}
\ltr{N}{os} estamos acercando a la fiesta de Pascua, misterio central de nuestra fe. El evangelio de Juan –como hemos escuchado– narra que antes de morir y resucitar por nosotros, Jesús realizó un gesto que quedó esculpido en la memoria de los discípulos: el lavatorio de los pies. Un gesto inesperado y sorprendente, al punto que Pedro no quería aceptarlo. Quisiera detenerme en las palabras finales de Jesús: \textquote{¿Comprendéis lo que he hecho con vosotros? [\ldots] Pues si yo, el Señor y el Maestro os he lavado los pies, vosotros también deberéis lavaros los pies unos a los otros} (\textit{Jn}13, 12. 14). De este modo Jesús le indica a sus discípulos \textit{el servicio} como el camino que es necesario recorrer para vivir la fe en Él y dar testimonio de su amor. El mismo Jesús ha aplicado a sí la imagen del \textquote{Siervo de Dios} utilizada por el profeta Isaías. ¡Él que es el Señor, se hace siervo!

Lavando los pies a los apóstoles, Jesús quiso revelar el modo de actuar de Dios en relación a nosotros, y dar el ejemplo de su \textquote{mandamiento nuevo} \textit{(Jn} 13, 34) de amarnos los unos a los otros como Él nos ha amado, o sea dando la vida por nosotros. El mismo Juan lo escribe en su Primera Carta: \textquote{En esto hemos conocido lo que es el amor: en que él dio su vida por nosotros. También nosotros debemos dar la vida por los hermanos [\ldots] Hijos míos, no amemos de palabras ni de boca, sino con obras y según la verdad} (3, 16.18).

El amor, por lo tanto, es el \textit{servicio concreto} que nos damos los unos a los otros. El amor no son palabras, son obras y servicio; un servicio \textit{humilde}, hecho en el \textit{silencio} y \textit{escondido}, como Jesús mismo dijo: \textquote{Que no sepa tu mano izquierda lo que hace tu derecha} (\textit{Mt} 6, 3). Esto comporta poner a disposición los dones que el Espíritu Santo nos ha dado, para que la comunidad pueda crecer (cf. \textit{1 Cor} 12, 4-11). Además se expresa en el \textit{compartir} los bienes materiales, para que nadie tenga necesidad. Este gesto de \textit{compartir} y de dedicarse a los necesitados es un estilo de vida que Dios sugiere también a muchos no cristianos, como un camino de auténtica humanidad.

Por último, no nos olvidemos que lavando los pies a los discípulos y pidiéndoles que hagan lo mismo, Jesús también nos ha invitado a confesarnos mutuamente nuestras faltas y a rezar los unos por los otros, para saber perdonarnos de corazón. En este sentido, recordamos las palabras del santo obispo Agustín cuando escribía: \textquote{No desdeñe el cristiano hacer lo que hizo Cristo. Porque cuando el cuerpo se inclina hasta los pies del hermano, también el corazón se enciende, o si ya estaba se alimenta el sentimiento de humildad [\ldots] Perdonémonos mutuamente nuestros errores y recemos mutuamente por nuestras culpas y así de algún modo nos lavaremos los pies mutuamente} (\textit{In Ion} 58, 4-5). El amor, la caridad es el servicio, ayudar a los demás, servir a los demás. Hay mucha gente que pasa la vida así, sirviendo a los otros. La semana pasada recibí una carta de una persona que me agradecía por el Año de la Misericordia; me pedía rezar por ella, para que pudiera estar más cerca del Señor. La vida de esta persona es cuidar a la mamá y al hermano: la mamá en cama, anciana, lúcida pero no se puede mover y el hermano es discapacitado, en una silla de ruedas. Esta persona, su vida es servir, ayudar. ¡Y esto es amor! ¡Cuando te olvidas de ti mismo y piensas en los demás, esto es amor! Y con el lavatorio de los pies el Señor nos enseña a ser servidores, más aún: siervos, como Él ha sido siervo para nosotros, para cada uno de nosotros.

Por lo tanto, queridos hermanos y hermanas, \textit{ser misericordiosos como el Padre, significa seguir a Jesús en el camino del servicio}.
\end{body}

\begin{patercite}
(\ldots)  De un estado de muerte y de pecado hemos sido trasladados a un estado de liberación y de gracia. Para comprender a fondo el don de la salvación, hay que comprender antes el mal inmenso que es el pecado, \textit{quanti ponderis sit peccatum} (San Anselmo).  La muerte de Cristo nos libra de nuestros pecados, ya que la redención es esencialmente la destrucción del pecado.

Ahora podemos comprender mejor el \textit{vocabulario de la redención}, es decir, los términos con los cuales ha hablado de ella el Nuevo Testamento, testimoniando la fe de los Apóstoles y de la primera comunidad cristiana. Una de las expresiones más comunes es la de la \textit{redención}, \textquote{apolytrosis}. Cuando decimos que Jesús nos ha \textquote{redimido} usamos una imagen que significa liberación de la esclavitud, de la prisión, entiéndase, del pecado. Como Dios liberó a su pueblo de la servidumbre de Egipto, de la misma forma que se libera a un prisionero pagando el rescate, como se recupera una cosa estimada que ha pasado a ser posesión de otro, así Dios nos ha rescatado mediante la sangre de Cristo. Escribe San Pedro: \textquote{Considerando que habéis sido rescatados de vuestro vano vivir según la tradición de vuestros padres, no con plata y oro corruptibles, sino con la sangre preciosa de Cristo, como cordero sin defecto ni mancha} (\textit{1 Pe} 1, 18-19).

\textbf{San Juan Pablo II}, papa, \textit{Catequesis}, Audiencia general, 28 de septiembre de 1983, cf. nn. 2-3.
\end{patercite}

\newsection
\section{Homilías}

\subsection{San Pablo VI, papa}

\subsubsection{Homilía (1964): Unidad mística y humana}

\src{Basílica de San Juan de Letrán. \\Jueves Santo 26 de marzo de 1964.}

\begin{body}
\ltr{N}{os} mismo hemos querido celebrar este rito \textquote{in coena Domini} porque hemos sido solicitados por la invitación, por el impulso de la reciente Constitución del Concilio Ecuménico sobre la Sagrada Liturgia, decididamente dirigida a aproximar las estructuras jerárquicas y comunitarias de la Iglesia lo más posible al ejercicio del culto, a la celebración, a la comprensión, al gozo de los sagrados misterios, contenidos en la oración sacramental y oficial de la Iglesia misma. Si todo sacerdote, como cabeza de una comunidad de fieles, si todo obispo, consciente de ser el centro operante y santificador de una Iglesia, desea, pudiéndolo, celebrar personalmente la santa misa delNos mismo hemos querido celebrar este rito \textquote{in coena Domini} porque hemos sido solicitados por la invitación, por el impulso de la reciente Constitución del Concilio Ecuménico sobre la Sagrada Liturgia, decididamente dirigida a aproximar las estructuras jerárquicas y comunitarias de la Iglesia lo más posible al ejercicio del culto, a la celebración, a la comprensión, al gozo de los sagrados misterios, contenidos en la oración sacramental y oficial de la Iglesia misma. Si todo sacerdote, como cabeza de una comunidad de fieles, si todo obispo, consciente de ser el centro operante y santificador de una Iglesia, desea, pudiéndolo, celebrar personalmente la santa misa del Jueves Santo, día memorable en que la santa misa fue celebrada por primera vez e instituida por el mismo Cristo para que lo fuese luego por los elegidos para ejercer su sacerdocio, ¿no debería el Papa, dichoso de tener esta oportunidad, realizar él mismo el rito en la conmemoración anual, que evoca su origen, medita su típica institución, exalta con sencillez pero con toda la posible e inefable interioridad su santísimo significado, y adora la oculta pero cierta presencia de Cristo santificador mismo para nuestra salvación?

Si quisiéramos justificar con otros motivos este propósito nuestro no tendríamos dificultades en encontrar muchos y excelentes; dos, por ejemplo, que pueden contribuir a hacer más piadosa y gozosa nuestra presente celebración; Nos sugiere el primero el múltiple movimiento, que fermenta en tantas formas diversas en el seno de nuestra sociedad contemporánea, y la estimula, aún a su pesar, a expresiones uniformes primero y unitarias después; el pensamiento humano, la cultura, la acción, la política, la vida social, la económica también –de por sí particular y que tiende al interés que distingue y opone a cada uno de los interesados–, están encaminados a una convergencia unificadora; el progreso lo exige y la paz se encuentra allí y de todo aquello tiene necesidad.

Pero el misterio que nosotros celebramos esta tarde es un misterio de unificación, de unidad mística y humana; bien lo sabemos; y aunque se realiza en una esfera distinta de la temporal, no prescinde, no ignora, no descuida la socialidad humana en el acto mismo que la supone, la cultiva, la conforta, la sublima, cuando este misterio, que justamente llamamos comunión, nos pone en inefable sociedad con Cristo, y mediante Él en sociedad con Dios y en sociedad con los hermanos mediante relaciones diversas, según sean o no partícipes con nosotros de la mesa que juntamente nos une, de la fe que unifica nuestros espíritus, de la caridad que nos compagina en un solo cuerpo, el Cuerpo místico de Cristo.

El segundo motivo, que sí hace referencia, como decíamos, a todo sacerdote, a todo obispo, respecta principalmente a Nos, a nuestra persona y a nuestra misión que Cristo quiere poner en el corazón de la unidad de toda la Iglesia católica, y ennoblecerla con un título, impuesto por un Padre, desde los albores de la historia eclesiástica, de \textquote{presidente de la caridad}. Creemos nos incumbe el grande y grave oficio de recapitular aquí la historia humana anudada como a su luz y salvación, al sacrificio de Cristo, sacrificio que aquí se refleja y de modo incruento se renueva; nos toca atender una mesa a la que están invitados místicamente todos los obispos, todos los sacerdotes, todos los fieles de la tierra; aquí se celebra la hermandad de todos los hijos de la Iglesia católica; aquí está la fuente de la socialidad cristiana, convocada a sus principios constitutivos trascendentes y socorridas por energías alimentadas, no por intereses terrenos, que son siempre de funcionamiento ambiguo, ni por cálculos políticos, siempre de efímera consistencia, ni por ambiciones imperialistas o dictámenes coercitivos, ni siquiera por el sueño noble e ideal de la concordia universal, que puede, a lo más, intentar el hombre, pero que no sabe realizar ni conservar; por energías, decimos, potenciadas por una corriente superior, divina, por la corriente, por la urgencia de la caridad, que Cristo nos ha conseguido de Dios y hace circular en nosotros, para ayudarnos a \textquote{ser una sola cosa} como lo es Él con el Padre.

Hermanos e hijos míos; ni las palabras ni el tiempo son suficientes para decirnos a nosotros mismos la plenitud de este momento; esta es la celebración de uno y de muchos, la escuela del amor superior de los unos para con los otros, la profesión de la estima mutua, la alianza de la colaboración recíproca, el empeño del servicio gratuito, la razón de la sabia tolerancia, el precepto del perdón mutuo, la fuente del gozo por la fortuna de los demás, y del dolor por la desventura ajena, el estímulo para preferir el dar dones a recibirlos, la fuente de la verdadera amistad, el arte de gobernar sirviendo y de obedecer queriendo, la formación en las relaciones corteses y sinceras con los hombres, la defensa del respeto y veneración a la personalidad, la armonía de los espíritus libres y dóciles, la comunión de las almas, la caridad. Leíamos, estos días, unas tristes palabras de un escrito contemporáneo, profeta del mundo sin amor y del egoísmo proclamado libertador: \textquote{Yo no quiero comunión de almas\ldots}. El cristianismo no es así, está en los antípodas. Nos quisiéramos construir, bajo los auspicios de Cristo, una comunión de almas, la comunión más grande posible.

Digamos, por tanto a nuestros sacerdotes, ante todo, las palabras sacrosantas del Jueves Santo: \textquote{Amémonos los unos a los otros como Cristo nos ha amado}. ¿Puede haber un programa más grande, más sencillo, más innovador de nuestra vida eclesiástica?

Dirijimos a vosotros, fieles, que formáis un coro en torno a este altar, y a todos vosotros distribuidos en el inmenso círculo de la santa Iglesia de Dios, otras palabras igualmente pronunciadas por Cristo el Jueves Santo; recordad que éste ha de ser el signo distintivo a los ojos del mundo de vuestra cualidad de discípulos de Cristo, el amor mutuo. \textquote{En esto todos conocerán\ldots}.

Diremos a cuantos pueda llegar el eco de nuestra celebración de la cena pascual, en la fe de Cristo y en su caridad, las palabras del Apóstol Pedro: \textquote{Complaceos en ser hermanos} (\textit{1 P} 2, 17). Por este motivo confirmamos aquí también nuestro propósito al Señor, de conducir a buen término el Concilio Ecuménico, como un gran acontecimiento de caridad en la Iglesia, dando a la colegialidad episcopal el significado y el valor que Cristo pretendió conferir a sus apóstoles en la comunión y en el obsequio al primero de ellos, Pedro, promoviendo todos los propósitos encaminados a aumentar en la Iglesia de Dios la caridad, la colaboración, la confianza.

\txtsmall{[También con este sentimiento de caridad en el corazón saludamos desde esta Basílica, cabeza y madre de todas las Iglesias, a todos los hermanos cristianos, por desgracia aún separados de nosotros, pero pretendiendo buscar la unidad querida por Cristo para su única Iglesia. (\ldots) ]}

También pensamos en estos momentos en toda la humanidad, estimulados por la caridad de Aquél que amó de tal forma al mundo que por él dio su vida. Nuestro corazón adquiere las dimensiones del mundo; ojalá adquiriera las infinitas del corazón de Cristo. Y vosotros, hermanos e hijos y fieles, estáis aquí presentes, ciertamente para celebrar con Nos el Jueves Santo, el día de la caridad consumada y perpetuada de Cristo por nuestra salvación.
\end{body}

\label{b-05-01-1964H}

\newpage

\subsubsection{Homilía (1967): Eucaristía, síntesis de nuestra fe}

\src{23 de marzo de 1967.}

\homsec{Hoy celebramos con más vivo fervor el \textquote{Mysterium Fide}}

\begin{body}
¡Venerables hermanos y queridos hijos!

\ltr{S}{i} hay un momento de nuestra vida espiritual, de nuestra profesión cristiana, de nuestra pertenencia a la Iglesia, en el que debe estar comprometida nuestra atención, nuestra conciencia, nuestro fervor, es este momento. Un momento tremendamente hermoso y significativo, pero igualmente intenso y difícil, contrario a nuestra distracción habitual. Es un momento de atracción hacia una Realidad presente y misteriosa, que compromete nuestras facultades espirituales a una concentración singular. Entramos en el misterio. Es necesario que seamos iniciados. Simplemente decimos: es necesario que seamos creyentes. Nos acercamos, de hecho celebramos el \textquote{\textit{mysterium fidei}}. Necesitamos ese complemento cognitivo, esa virtud intelectual, sostenida por la buena voluntad e iluminada por el Espíritu Santo, que se llama fe, para entrar en el secreto de la Realidad, que hoy está preparada para nosotros y recibir algún goce vital de ella. ¿Por qué hoy, y no siempre, cuando celebramos los misterios divinos? Siempre, respondemos sin cuestionar; pero hoy con mayor intensidad, porque el sacrificio divino de la Misa, que celebramos otros días, deriva y se refiere a esto. Aquí está el misterio pascual, tal cual nos es dado para recordarlo y revivirlo; y cada vez que renovamos la oblación litúrgica celebramos este mismo misterio pascual.
\end{body}

\homsec{Cristo Jesús mediador en las relaciones entre Dios y el hombre}

\begin{body}
Y habiendo entrado así en el cenáculo de las supremas comunicaciones divinas, debemos quedarnos callados y extáticos, como quien ve demasiado y solo entiende algo; y con temor deberíamos sentir al menos esto: que en la Cena del Señor, como nudo central, convergen los hilos de la historia antigua de la Salvación, porque la Pascua judía deposita allí sus símbolos proféticos, que aquí disuelven sus secretos y se transfunden en la nueva forma, también simbólica y profética, pero sustanciada por una Realidad muy diferente, a través de la cual se forma el perenne memorial de nuestra redención cumplida con el Sacrificio de la Cruz y la Resurrección gloriosa, y se nos da para participar de su virtud y tener su promesa; para que de la misma cena del Señor parta otro manojo de nuevos hilos, que invaden el mundo y la historia, y por cada ser vivo se ramifican y llegan, si queremos, a cada uno de nosotros. El lenguaje bíblico es más claro que cualquiera de nuestros discursos: el Antiguo Testamento y el Nuevo Testamento se tocan allí, y el uno al otro cede las intenciones divinas, de hecho las intervenciones divinas en el diseño sublime y formidable de las relaciones entre Dios y el hombre, siendo mediador, aquí plenamente Cristo Jesús. Océanos de verdad, y por tanto de doctrina, se abren ante nosotros: la Eucaristía, sabéis, hermanos e hijos aquí presentes, es la síntesis de nuestra fe; y por eso, después de haber hecho un esfuerzo de conciencia religiosa por abstraer nuestro espíritu de todo interés circundante y diferente para fijar la mente y el corazón en el punto focal, al que se dirige esta celebración tan especial, nos sentimos obligados a revisar, a la nueva luz de este mismo punto focal, todo: el mundo, la historia, la vida, nosotros mismos. Demasiado, demasiado, quisiéramos exclamar, y con la voz de los santos más comprensivos a nosotros también nos gustaría tartamudear: \textit{satis, Domine}, basta, Señor, basta. Esto requiere que nos contentemos ahora con un solo pensamiento entre los muchos posibles, y que mantengamos nuestra atención por un momento en uno de los aspectos esenciales del misterio del Jueves Santo, aquel sobre el cual queremos unir el pensamiento y la oración de esta santa asamblea. 
\end{body}

\homsec{La sublime realidad, más allá de todo obstáculo del orden natural}

\begin{body}
¿Qué aspecto? El intencional, el final, el de la \textquote{comunión}. Como quien es experto en ciertas prodigiosas técnicas modernas y sabe utilizar ciertos instrumentos mágicos, victorioso en el tiempo y el espacio, y sabe relacionarse con sensatez con escenas y palabras muy lejanas y esquivas de nuestra percepción inmediata, así nosotros, entrando con la fe y con el amor en el sistema sacramental concebido por Cristo e instituido por él, esto es, puesto en práctica por él en la misma noche en que fue traicionado, \textquote{\textit{in qua nocte tradebatur}} (\textit{1 Cor} 11, 23), podemos ponernos en contacto con él. Cristo, sobrevolando, en virtud de su Palabra, leyes y obstáculos de orden natural, insuperables en sí mismos, y \textquote{comulgando}, como solemos decir; hacer Pascua. La Eucaristía es el sacramento de la permanencia de Cristo, que vive ahora en la gloria eterna del Padre, en nuestro tiempo, en nuestra historia, en nuestra peregrinación terrena. \textquote{\textit{Vobiscum sum}}, estoy con vosotros, dirá Jesús cerrando la escena del Evangelio, y cumplirá su promesa. La Eucaristía es el sacramento de su presencia viva, real y sustancial, en todas partes; en todas partes está su ministro que hace lo que Él ha hecho, en su memoria. \textquote{Haced esto –dijo Jesús aquella tarde, instituyendo junto con la Eucaristía el sacramento del Orden Sagrado, instrumento humano autorizado, para renovar su misterio y difundirlo por toda la tierra– haced esto en memoria mía} (\textit{Lc} 22, 19). La Eucaristía es el sacramento que multiplica, que universaliza la presencia y la acción de Jesús: como una misma palabra puede ser escuchada por muchos y adquirir lógica eficacia en quien la oye y comprende, así el Señor, a través de la Eucaristía, se vuelve accesible a todos los que le acogen bajo este signo. La Eucaristía es Cristo para cada uno de nosotros, revestido precisamente de las apariencias del pan para decir que está dispuesto a saciar nuestra hambre, a ser deseado, abordado, asumido, asimilado en sí mismo. La Eucaristía es la figura de Cristo sacrificado por nosotros, para que nos sea posible y urgente recordar para siempre su Pasión, participar de su drama sacrificial y obtener su eficacia redentora. Lo decimos para que nos quede clara la intención global de Cristo: unirse a nosotros, admitirnos en su comunión. No es posible hacerse una idea de esto sin admitir un amor excesivo, un amor infinito que se proyecta en cada uno de nosotros y que no nos da paz hasta que de nuestro árido corazón también brota algún entendimiento, alguna correspondencia. La Eucaristía es una escuela de amor; y, para poner nuestras almas en sintonía con la corriente ardiente y abrumadora de su caridad, hay que decir al menos con el Apóstol, que en esa bendita y trágica tarde del Jueves Santo puso su oído en el pecho de Cristo y escuchó los latidos de su corazón: sí, \textquote{hemos creído en la caridad} (\textit{1 Jn} 4, 16). Y aquí se perfecciona la nueva vida espiritual, interior, de todo aquel que así ha entrado en comunión con Cristo. 

Pero esto no es todo. La gracia que nos ofrece la Eucaristía no es sólo en relación con la comunión con Cristo; otra comunión resulta de este sacramento; y es comunión con todos aquellos hermanos en la fe y en la caridad que están sentados a la misma mesa. Las palabras de \textbf{San Pablo} son muy conocidas, pero siempre memorables: \textquote{Hablo a personas inteligentes; juzga por ti mismo lo que digo. La copa de bendición que bendecimos, ¿no es una comunión de la sangre de Cristo? ¿No es el pan que partimos comunión con el cuerpo de Cristo? Porque el pan es uno, formamos un solo cuerpo, aunque somos muchos cuando todos compartimos ese único pan} (\textit{1 Cor} 10, 15-17).
\end{body}

\homsec{Comunión con los hermanos, en especial con los que sufren por el Señor}

\begin{body}
Y he aquí, amados hermanos e hijos, que la realidad profunda y sobrenatural del misterio pascual nos devuelve a la realidad, mística sí, pero también visible y experimental, de la sociedad nacida de Cristo, su cuerpo místico, la Iglesia (cf. \textit{S. Th.} III, 73, 3), que nos gustaría que sea inundada, precisamente en virtud de este Jueves Santo, por la gracia propia de este día bendito, la gracia de la comunión, la gracia de la unidad, con Cristo y consigo misma; y para ello os pedimos a todos el aporte de vuestras oraciones, de vuestra colaboración espiritual. [\ldots] La unidad tiene diferentes grados: puede ser superficial y formal, sufrida y no amada, habitual e inoperante; y puede ser profunda y cordial, convencida y operante, toda impregnada de caridad mutua y santificante: esta unidad, viviendo en la fe y el amor a Cristo y en la fraternidad sincera, queremos sea infundida en Nuestra Ciudad, (\ldots) queremos que Cristo sea su maestro, su salvador, su ciudadano. Y agregamos un voto similar para toda la Iglesia Católica. Pensamos en este momento en toda nuestra gran fraternidad que en esta tarde, esparcida por toda la tierra, realiza el mismo rito pascual con igual sentimiento; pensemos en aquellas comunidades, impedidas o mortificadas, donde continúa la Pasión del Señor; pensemos en las Iglesias jóvenes de países en territorios de misión; y a toda esta inmensa y amada comunión enviamos nuestros saludos de bendición: ave, una, santa, católica y apostólica Iglesia; ave, Iglesia viva de Cristo: todos en él hoy somos uno. 
\end{body}

\homsec{Saludos y deseos a los asistentes}

\begin{body}
Y no nos olvidemos de las muchas Iglesias y comunidades cristianas, a las que nos une el mismo bautismo y tantos lazos de fe y de amor al único Cristo Señor, y con las que aún no podemos gozar de una perfecta comunión. Esto es lo que deseamos, esperamos e invocamos, mientras enviamos a todas y cada una desde [Nuestra Catedral], llena del fiel recuerdo y presencia mística de Cristo Salvador, Nuestro mensaje de caridad pascual. [\ldots]

\end{body}

\label{b-05-01-1967H}

\begin{patercite}
El trasfondo temporal y emocional del convite en el que Jesús se despide de sus amigos es la inminencia de su muerte, que él siente ya cercana. Jesús había comenzado a hablar de su Pasión ya desde hacía tiempo, tratando incluso de implicar cada vez más a sus discípulos en esta perspectiva. El Evangelio según san Marcos relata que desde el comienzo del viaje hacia Jerusalén, en los poblados de la lejana Cesarea de Filipo, Jesús había comenzado \textquote{a instruirlos: \textquote{el Hijo del hombre tiene que padecer mucho, ser reprobado por los ancianos, sumos sacerdotes y escribas, ser ejecutado y resucitar a los tres días}} (\emph{Mc} 8, 31). Además, precisamente en los días en que se preparaba para despedirse de sus discípulos, la vida del pueblo estaba marcada por la cercanía de la Pascua, o sea, del memorial de la liberación de Israel de Egipto. Esta liberación, experimentada en el pasado y esperada de nuevo en el presente y para el futuro, se revivía en las celebraciones familiares de la Pascua. La última Cena se inserta en este contexto, pero con una novedad de fondo. Jesús mira a su pasión, muerte y resurrección, siendo plenamente consciente de ello. Él quiere vivir esta Cena con sus discípulos con un carácter totalmente especial y distinto de los demás convites; es su Cena, en la que dona Algo totalmente nuevo: se dona a sí mismo. De este modo, Jesús celebra su Pascua, anticipa su cruz y su resurrección.

Esta novedad la pone de relieve la cronología de la última Cena en el Evangelio de san Juan, el cual no la describe como la cena pascual, precisamente porque Jesús quiere inaugurar algo nuevo, celebrar su Pascua, vinculada ciertamente a los acontecimientos del Éxodo. Para san Juan, Jesús murió en la cruz precisamente en el momento en que, en el templo de Jerusalén, se inmolaban los corderos pascuales. 

\textbf{Benedicto XVI}, papa, \textit{Catequesis}, Audiencia general, 11 de enero de 2012, parr. 2-3.
\end{patercite}

\newpage

\subsubsection{Homilía (1970): Fuente del amor, hasta la muerte}

\src{Archibasílica Laterana, 26 de marzo de 1970.}

\begin{body}
\ltr{O}{bligados} por nuestro ministerio a abrir los labios en este lugar sagrado, \textquote{magnum stratum}, amplio y ornamentado, cenáculo por excelencia de la Iglesia católica y romana, y en este momento, entre todos, intenso de sentimientos y pensamientos religiosos y humanos, si bien nos agradaría escuchar en silencio interior las grandes voces que surgen de la sublime liturgia que estamos celebrando, ofreceremos a vuestra benevolente atención algunas indicaciones elementales, que servirán para estimular nuestra reflexión sobre los aspectos obvios y fundamentales de este rito y poner nuestras almas en armonía en un coro espiritual común. 
\end{body}

\homsec{Plenitud de la comunión eclesial}

\begin{body}
Y el primer indicio es precisamente este relativo a la comunión eclesial, que nos reúne aquí y que ahora adquiere una plenitud singular, un sentido propio. Este es un momento particular de comunión entre nosotros, entre aquellos que han aceptado nuestra invitación y nos han regalado su presencia. Si alguna vez se nos ofrece una ocasión feliz pY el primer indicio es precisamente este relativo a la comunión eclesial, que nos reúne aquí y que ahora adquiere una plenitud singular, un sentido propio. Este es un momento particular de comunión entre nosotros, entre aquellos que han aceptado nuestra invitación y nos han regalado su presencia. Si alguna vez se nos ofrece una ocasión feliz para realizar las palabras del Señor: \textquote{Dondequiera que dos o tres personas se reúnan en mi nombre, yo estoy entre ellos} (\textit{Mt 18, 20}), esto es para nosotros, mientras que precisamente este su nombre, y sólo su nombre, polariza nuestra asistencia, y emerge entre nosotros, como si estuviera aquí ahora y pronto lo estará sacramentalmente, y de ahora en adelante llena nuestras almas de sí mismo, y las une en la fe, en la armonía, en la paz, en la alegría de conocernos y de sentirnos \textquote{iglesia}, es decir, unión, su único redil, su cuerpo místico. Que cada distancia entre nosotros, cada desconfianza, cada descuido, cada extrañeza caiga en este momento; que caiga todo rencor, toda rivalidad; y que cada uno trate de experimentar \textquote{qué hermoso y qué gozo es que los hermanos estén juntos} (\textit{Sal} 132, 1); y que cada uno sienta en sí mismo que tener la suerte de ser, como la primera comunidad de creyentes, \textquote{un solo corazón y una sola alma} (\textit{Hch} 4, 32) significa realizar nuestra exigente calificación de cristianos católicos. Caridad \textit{dentro de} la Iglesia, caridad que la reúne y la compone, caridad que especifica su \textquote{cuerpo místico} y hace hermanos a todos los que aceptan su socialidad organizada (\textit{Mt} 23, 8; \textit{Lc} 10, 16), la caridad humilde, amigable y solidaria entre los fieles y seguidores y ministros de Cristo es el primer requisito exigente para sentarse a la mesa el Jueves Santo (Cf. \textit{Lc} 22, 24 ss.). 

Juntos, por tanto, más que nunca, vivimos esta hora fugaz. Pero, ¿cuál es el propósito, cuál es la intención? ¿Por qué estamos reunidos aquí? He aquí, pues, una segunda indicación Nuestra, también muy conocida. Estamos aquí para una conmemoración. Este es un rito de la memoria. Una Misa es siempre así, pero en este día queremos resaltar su carácter conmemorativo. Celebramos la memoria del Señor, obedeciendo a sus palabras, que podemos decir testamentarias: \textquote{Haced esto en memoria mía} (\textit{Lc} 22, 19; 1 \textit{Cor} 11, 25). Todo nuestro espíritu está ahora lleno del recuerdo de él, de Jesús: nos gustaría poder traerlo a nuestra imaginación, cómo era, cómo era su figura, su rostro, cómo era el sonido de su voz, la luz de sus ojos, los gestos de sus manos\ldots No nos ha llegado ninguna imagen sensible de él; pensamos con asombro en aquella imagen tan impresionante y profunda de la Sábana Santa; pensamos en la elección de nuestro genio por las piadosas efigies de los grandes artistas favoritos, por las descripciones de los sabios y los santos; pero siempre con el descontento propio de nosotros los modernos, incluso demasiado favorecido por la civilización de la imagen, porque la suya no se muestra a nuestra mirada, sino sólo a nuestro deseo escatológico: \textquote{¡Ven, Señor Jesús!} (\textit{Ap} 22, 20). ¡Nuestra memoria debe contentarse con otra presencia suya, la de su palabra! Entonces todo el Evangelio pasa ante nuestra mente, que sin embargo se detiene en esa palabra que Cristo pronunció en la cena de aquella noche y que recomendó a nuestra memoria. ¿Qué palabra? Oh, bien lo sabemos: \textquote{Tomad y comed: este es mi Cuerpo; tomad y bebed: esta es la copa de mi Sangre}. ara realizar las palabras del Señor: \textquote{Dondequiera que dos o tres personas se reúnan en mi nombre, yo estoy entre ellos} (\textit{Mt 18, 20}), esto es para nosotros, mientras que precisamente este su nombre, y sólo su nombre, polariza nuestra asistencia, y emerge entre nosotros, como si estuviera aquí ahora y pronto lo estará sacramentalmente, y de ahora en adelante llena nuestras almas de sí mismo, y las une en la fe, en la armonía, en la paz, en la alegría de conocernos y de sentirnos \textquote{iglesia}, es decir, unión, su único redil, su cuerpo místico. Que cada distancia entre nosotros, cada desconfianza, cada descuido, cada extrañeza caiga en este momento; que caiga todo rencor, toda rivalidad; y que cada uno trate de experimentar \textquote{qué hermoso y qué gozo es que los hermanos estén juntos} (\textit{Sal} 132, 1); y que cada uno sienta en sí mismo que tener la suerte de ser, como la primera comunidad de creyentes, \textquote{un solo corazón y una sola alma} (\textit{Hch} 4, 32) significa realizar nuestra exigente calificación de cristianos católicos. Caridad \textit{dentro de} la Iglesia, caridad que la reúne y la compone, caridad que especifica su \textquote{cuerpo místico} y hace hermanos a todos los que aceptan su socialidad organizada (\textit{Mt} 23, 8; \textit{Lc} 10, 16), la caridad humilde, amigable y solidaria entre los fieles y seguidores y ministros de Cristo es el primer requisito exigente para sentarse a la mesa el Jueves Santo (Cf. \textit{Lc} 22, 24 ss.). 

Juntos, por tanto, más que nunca, vivimos esta hora fugaz. Pero, ¿cuál es el propósito, cuál es la intención? ¿Por qué estamos reunidos aquí? He aquí, pues, una segunda indicación Nuestra, también muy conocida. Estamos aquí para una conmemoración. Este es un rito de la memoria. Una Misa es siempre así, pero en este día queremos resaltar su carácter conmemorativo. Celebramos la memoria del Señor, obedeciendo a sus palabras, que podemos decir testamentarias: \textquote{Haced esto en memoria mía} (\textit{Lc} 22, 19; 1 \textit{Cor} 11, 25). Todo nuestro espíritu está ahora lleno del recuerdo de él, de Jesús: nos gustaría poder traerlo a nuestra imaginación, cómo era, cómo era su figura, su rostro, cómo era el sonido de su voz, la luz de sus ojos, los gestos de sus manos\ldots No nos ha llegado ninguna imagen sensible de él; pensamos con asombro en aquella imagen tan impresionante y profunda de la Sábana Santa; pensamos en la elección de nuestro genio por las piadosas efigies de los grandes artistas favoritos, por las descripciones de los sabios y los santos; pero siempre con el descontento propio de nosotros los modernos, incluso demasiado favorecido por la civilización de la imagen, porque la suya no se muestra a nuestra mirada, sino sólo a nuestro deseo escatológico: \textquote{¡Ven, Señor Jesús!} (\textit{Ap} 22, 20). ¡Nuestra memoria debe contentarse con otra presencia suya, la de su palabra! Entonces todo el Evangelio pasa ante nuestra mente, que sin embargo se detiene en esa palabra que Cristo pronunció en la cena de aquella noche y que recomendó a nuestra memoria. ¿Qué palabra? Oh, bien lo sabemos: \textquote{Tomad y comed: este es mi Cuerpo; tomad y bebed: esta es la copa de mi Sangre}. 
\end{body}

\homsec{Presencia viva y real del Señor}

\begin{body}
El banquete pascual, porque tal era aquella Cena ritual (Cfr. \textit{Lc} 22, 7 y ss.), debía ser objeto de recuerdos inolvidables, pero desde un punto de vista nuevo, no ya del sacrificio y la comida del cordero, signo y prenda de la antigua alianza, sino del pan y del vino, transformados en el Cuerpo y la Sangre de Jesús. El ágape, en este punto, se convierte en misterio. La presencia del Señor se vuelve viva y real. Las apariencias sensibles siguen siendo lo que fueron, pan y vino; pero su sustancia, su realidad ha cambiado íntimamente; el pan y el vino permanecen sólo para significar lo que la palabra omnipotente, porque divina, de Jesús ha dicho que son: cuerpo y sangre. Ante esto quedamos asombrados. También porque este prodigio es precisamente lo que el Señor nos dijo que recordemos y que renovemos. Él dijo a los Apóstoles \textquote{haced esto}, es decir, les transmitió la virtud de repetir su acto consecratorio, y no sólo de repensarlo, sino de volverlo a hacer; el sacramento del Orden Sagrado, como custodia, como fuente del sacramento de la Eucaristía, fue instituido junto con este, en aquella noche. Quedamos asombrados e inmediatamente tentados: pero ¿es cierto? ¿Realmente es verdad ? ¿Cómo se explican esas sacrosantas sílabas de Cristo: este es mi cuerpo, esta es mi sangre? ¿Es posible encontrar una interpretación que no viole nuestra mentalidad elemental? ¿A nuestra reflexión metafísica habitual? También llega a nuestros labios el repulsivo comentario de los oyentes de Cafarnaum: \textquote{Este lenguaje es duro; y ¿quién podrá escucharlo?} (\textit{Jn} 6, 61). Pero el Señor no admite dudas o exégesis elusivas de la auténtica realidad de sus palabras textuales; lo convierte en una cuestión de confianza; dejaría que el amado grupo de sus discípulos se dispersara, en lugar de eximirlos de adherirse a sus paradójicas pero verdaderas palabras, proponiéndoles en un lenguaje no menos duro: \textquote{¿Vosotros también queréis marcharos?} (\textit{Ibíd}. 68). 
\end{body}

\homsec{La hora de la fe}

\begin{body}
Entonces esta es una hora decisiva, la hora de la fe, la hora que acepta en su integridad, aunque sea incomprensible, la palabra de Jesús; la hora en la que celebramos el \textquote{misterio de la fe}, la hora en la que repetimos incluso con ciego y sabio abandono la respuesta de Simón Pedro: \textquote{Señor, ¿a quién iremos? Sólo tú tienes palabras de vida eterna. Hemos creído y conocido que eres el Cristo, el Hijo de Dios} (\textit{Jn} 6, 69-70). Sí, hermanos e hijos, esta es la hora de la fe, que absorbe y consume la oscura e inmensa nube de objeciones, que nuestra ignorancia por un lado, y la refinada dialéctica del pensamiento profano, por otro, acumulan sobre nuestro espíritu, que humilde y gozosamente se deja impresionar por la luminosa palabra del Maestro y le dice temblando como el implorador evangélico: \textquote{Creo, Señor; pero tú, ayuda mi incredulidad} (\textit{Mc} 9, 24). 

Y entonces la fe vuelve a preguntar: ¿pero qué significa esta forma de recordar al Señor? ¿Cuál es el significado, cuál es el valor de este memorial? ¿De este sacramento de la presencia, de este misterio de fe? ¿Cuál es la intención dominante del Señor, que quiso grabar en la memoria de sus seguidores en ese último encuentro cordial? 

Hay quienes no hacen esta pregunta, como si no quisieran descubrir una verdad nueva y sorprendente. Pero no podemos detenernos sin recoger el último tesoro del testamento de Jesús, todo nos obliga a hacerlo, porque todo en esa última noche de su vida temporal es sumamente intencional y dramático: bastaría la observación de este aspecto de la Última Cena para nunca poner fin a nuestra meditación extática. La tensión espiritual casi te deja sin aliento. 

La apariencia, la palabra, los gestos, los discursos del Maestro son exuberantes con la sensibilidad y profundidad de quien está cerca de la muerte; lo siente, lo ve, lo expresa. Dos notas resuenan por encima de las demás en esta atmósfera de asombro silenciada por los actos y presagios del Maestro: el amor y la muerte. El lavatorio de los pies, impresionante ejemplo de amor humilde, el mandato, el último y nuevo mandato: amaos los unos a los otros como yo os he amado; y esa angustia por la traición inminente, esa tristeza que se desprende de las palabras y de la actitud del Maestro, y esa efusión mística y encantadora de los discursos finales, casi soliloquios de Cristo desbordados de un corazón que se abre a confidencias extremas, todo está concentrado en la acción sacramental, recién mencionada: ¡cuerpo y sangre! Sí, allí se representan el amor y la muerte; una sola palabra los expresa: sacrificio. Allí se quiere decir muerte, muerte sangrienta, la muerte que habría separado su sangre del cuerpo de Cristo; una inmolación, una víctima. Es una víctima voluntaria, una víctima consciente, una víctima por amor. Entregada por nosotros. Para ser recordada como anunciadora de la muerte de Jesús, de su sacrificio para siempre, hasta que Él vuelva al fin del mundo (1 \textit{Co} 11, 26). Cristo ha sellado en un rito, renovable por sus discípulos, hechos Apóstoles y Sacerdotes, la ofrenda de sí mismo al Padre, como víctima por nuestra salvación, por amor a nosotros. Esto es la Misa. Es el ejemplo, es la fuente del amor que se da hasta la muerte. 

Es Jueves Santo, es lo que estamos recordando y celebrando. Es el corazón y el paradigma de la vida cristiana. Es el mandato, es la memoria, es la pasión, es la caridad de Cristo, que ha transmitido a su Iglesia; a nosotros, para que vivamos de Él, por Él y para Él (\textit{Jn} 6, 57), y nos ofrezcamos en sacrificio por nuestros hermanos, por la salud del mundo (cf. \textit{Jn} 12, 24 ss.), y un día resucitar en Él (cf. \textit{Jn} 6, 54-58).
\end{body}

\label{b-05-01-1970H}

\begin{patercite}
(\ldots) Las tradiciones neotestamentarias de la institución de la Eucaristía (cf. \emph{1 Co} 11, 23-25; \emph{Lc} 22, 14-20; \emph{Mc} 14, 22-25; \emph{Mt} 26, 26-29), al indicar la oración que introduce los gestos y las palabras de Jesús sobre el pan y sobre el vino, usan dos verbos paralelos y complementarios. San Pablo y san Lucas hablan de \emph{eucaristía}/acción de gracias: \textquote{tomando pan, después de pronunciar la \emph{acción de gracias}, lo partió y se lo dio} (\emph{Lc} 22, 19). San Marcos y san Mateo, en cambio, ponen de relieve el aspecto de \emph{eulogia}/bendición: \textquote{tomó pan y, \emph{pronunciando la bendición}, lo partió y se lo dio} (\emph{Mc} 14, 22). Ambos términos griegos \emph{eucaristeín} y \emph{eulogeín} remiten a la \emph{berakha} judía, es decir, a la gran oración de acción de gracias y de bendición de la tradición de Israel con la que comenzaban los grandes convites. Las dos palabras griegas indican las dos direcciones intrínsecas y complementarias de esta oración. La \emph{berakha}, en efecto, es ante todo acción de gracias y alabanza que sube a Dios por el don recibido: en la última Cena de Jesús, se trata del pan ---elaborado con el trigo que Dios hace germinar y crecer de la tierra--- y del vino, elaborado con el fruto madurado en los viñedos. Esta oración de alabanza y de acción de gracias, que se eleva hacia Dios, vuelve como bendición, que baja desde Dios sobre el don y lo enriquece. Al dar gracias, la alabanza a Dios se convierte en bendición, y el don ofrecido a Dios vuelve al hombre bendecido por el Todopoderoso. Las palabras de la institución de la Eucaristía se sitúan en este contexto de oración; en ellas la alabanza y la bendición de la \emph{berakha} se transforman en bendición y conversión del pan y del vino en el Cuerpo y en la Sangre de Jesús.

\textbf{Benedicto XVI}, papa, \textit{Catequesis}, Audiencia general, 11 de enero de 2012, parr. 5.
\end{patercite}

\newpage

\subsubsection{Homilía (1973): Amor total}

\src{19 de abril de 1973.}

\begin{body}
\ltr{H}{ermanos,} bienvenidos a esta ceremonia de Jueves Santo, a la que todos sentimos que debemos asistir con total adhesión. El hecho mismo de que lo celebremos en esta basílica, corazón de la Iglesia católica, y que estemos deliberadamente juntos, todos penetrados por el sentido interior de la solemnidad del rito, y deseosos de sumarnos a la participación en la comprensión de lo que hacemos, nos empuja a la búsqueda, casa ansiosa, ciertamente ferviente, de su significado. 

Lo diremos muy brevemente, centrando nuestra atención en unas palabras de Jesús, el invitado protagonista de aquella última cena. Él mismo dijo que para él era la última (\textit{Lc} 22, 15-16), y lo hizo entender a lo largo de todos los discursos de ese íntimo y triste encuentro de convivencia, motivado por la celebración de la Pascua ritual judía (Cf. \textit{Jn} 16, 5-7; etc.), que culminó, como sabemos, en las misteriosas palabras de la institución de la santísima Eucaristía, concluidas con aquellas palabras preceptivas e instituyentes de otro sacramento, el de las Sagradas Órdenes, generador ministerial de la Eucaristía misma: \textquote{Haced esto en memoria mía} (\textit{Lc} 22, 19; \textit{1 Cor} 11, 24-25), dijo Cristo. Es en virtud de estas palabras que nos reunimos aquí esta noche. Son palabras testamentarias. Serán verdaderas y efectivas hasta su última venida, al final del presente orden temporal, al final de los siglos: \textit{donec veniat}, hasta que Él, Jesús, vuelva, declara \textbf{san Pablo}. Es, por tanto, el acto memorial por excelencia que recordamos y repetimos en este momento, cumpliendo el precepto que lo hace perenne en el desarrollo de la historia; es la presencia del Señor que acompaña el camino de su Iglesia en el tiempo, en el \textquote{misterio de la fe}, que presupone la presencia real de Jesús en la envoltura sacramental, y requiere una comprensión obediente, una acogida en la fe de nuestra parte, el homenaje amoroso de nuestra calificada memoria. 

Este esfuerzo de recuerdo es fundamental para nuestra celebración. La prodigiosa facultad de la memoria se ejerce como estímulo para nuestra capacidad receptiva a la Eucaristía. Ésta afecta a quienes la reciben por virtud propia \textit{ex opere operato}, pero su acción está orientada al ejercicio de nuestro recuerdo, es decir, a la acogida de Cristo recibido y meditado dentro de nosotros, a su permanencia personal, viva y real en nosotros, pero también conceptual y reflejado en nuestra mente, en nuestra psicología, en nuestro corazón, según nuestra actitud de asimilarlo, aceptarlo, amarlo, coincidir, por así decirlo, con él: \textit{donec formetur Christus in vobis}, hasta que Cristo sea formado en vosotros, dice San Pablo (\textit{Gal} 4, 19). Una intención fundamental de permanencia domina el misterio de la Eucaristía; es decir, de la permanencia de Jesús entre nosotros más allá del abismal límite de su pasión y muerte, de la verdadera permanencia, pero bajo el signo sacramental, que al quitarnos la alegría de su visión sensible, nos ofrece la seguridad de su presencia efectiva, y al mismo tiempo la otra inestimable ventaja de su indefinida y unívoca multiplicabilidad, en tiempos y lugares, lo que se necesita para saciar el hambre de quienes permanecerán en su fe y en su amor. Permanecer es la intención sacramental de la Eucaristía, en lo que respecta a Jesús; permanecer es la intención moral, en lo que respecta a nosotros, de quienes Jesús quiere ser el viático, el compañero, el sustento a lo largo de nuestra peregrinación en el tiempo: debemos, pues, permanecer en su amor. Veréis, en apoyo de esta afirmación, cuántas veces se repite la palabra \textquote{permanecer} en los discursos de Jesús en esa última cena (cf. especialmente \textit{Jn} 15).

Por eso, hermanos, hay algo que debemos reavivar en nuestras almas: \textquote{recordar} a Jesús, como él quiso ser; y he aquí, de este particular memorial nuestro, brota con ímpetu, es decir, con amorosa abundancia, nuestro culto eucarístico, al que la Iglesia nos invita y exhorta con incansable preocupación. 

Después, siempre limitando nuestra búsqueda al sentido esencial de aquel convite pascual, con el que Cristo quiso despedirse de sus discípulos, no podremos omitir el paso de la figura del cordero a la realidad de la víctima verdadera de nuestra Pascua, que es el mismo Cristo inmolado (Cfr. \textit{1 Co} 5, 7), paso obrado con la institución de la Eucaristía, que en la figura del pan y del vino, representa y renueva de forma incruenta el sacrificio redentor de Jesús. ¿De qué manera hablar en tan corto tiempo de tan alta y dramática teología? Bienaventurados seremos si, después del acto de fe que hemos señalado, el amor supliese a la deficiencia de nuestro discurso y más aún de nuestro pensamiento. 

La Eucaristía es el punto privilegiado del encuentro del amor de Cristo por nosotros; un amor que se pone a disposición de cada uno de nosotros, un amor que se convierte en cordero de sacrificio y alimento de nuestra hambre de vida, un amor que se expresa en la forma y medida de su autenticidad específica, más alta y exclusiva, es decir, un amor que se da totalmente: \textit{dilexit me} –dice el Apóstol– \textit{et tradidit semetipsum} \textit{pro me}, él me amó y se sacrificó por mí (\textit{Gál} 2, 20; \textit{Ef} 5, 2; 5, 25); y del encuentro de nuestro amor pobre y vacilante por él, que en tanta caridad apremiante encuentra finalmente el atrevimiento de superar toda timidez, toda debilidad y responder con Pedro: \textquote{Señor\ldots ¡Tú sabes que te quiero!} (\textit{Jn} 21, 15-17). El amor tendrá la suerte de penetrar algunas de sus intuiciones místicas y con algo de su anticipada plenitud (cf. \textit{Ef} 3, 17, 19) en el misterio de la caridad, que supera todo entendimiento, el misterio del sacrificio eucarístico, y hundirse en él, participando en ese rito humilde e inconmensurable, que es nuestra santa Misa. 

Hermanos, no os contamos más. Pero no concluiremos estas balbuceantes palabras sin confiaros que tenemos otra en nuestro corazón, también tomada de aquellas palabras inolvidables de la Cena del Señor, y es esta: \textquote{Os doy el mandamiento nuevo: amaos los unos a los otros, como yo os he amado a vosotros} (\textit{Jn} 13, 34; 15, 12). Ese \textquote{Yo} es Jesús, el Cristo, nuestro Señor; ese \textquote{vosotros} son los Apóstoles, todos los fieles que creyeron en él, \textquote{según su palabra} (\textit{Ibid}. 17, 20); somos nosotros, la Iglesia romana y la Iglesia católica, nosotros, hijos de la tierra y del siglo, los que hoy, Jueves Santo, debemos sentirnos todos impresionados por el amor crucificado y eucarístico de Cristo; y aún nos queda mucho por aprender a amarnos, según su ejemplo y precepto.
\end{body}

\label{b-05-01-1973H}

\begin{patercite}
Si la carne no se salva, entonces el Señor no nos ha redimido con su sangre, ni el cáliz de la eucaristía es participación de su sangre, ni el pan que partimos es participación de su cuerpo. Porque la sangre procede de las venas y de la carne y de toda la substancia humana, de aquella substancia que asumió el Verbo de Dios en toda su realidad y por la que nos pudo redimir con su sangre, como dice el Apóstol: \textit{Por su sangre hemos recibido la redención, el perdón de los pecados}. (\ldots) Porque nos quiere miembros suyos, aseguró el Señor que el cáliz, que proviene de la creación material, es su sangre derramada, con la que enriquece nuestra sangre, y que el pan, que también proviene de esta creación, es su cuerpo, que enriquece nuestro cuerpo.
	
Cuando la copa de vino mezclado con agua y el pan preparado por el hombre reciben la Palabra de Dios, se convierten en la eucaristía de la sangre y del cuerpo de Cristo y con ella se sostiene y se vigoriza la substancia de nuestra carne, ¿cómo pueden, pues, pretender los herejes que la carne es incapaz de recibir el don de Dios, que consiste en la vida eterna, si esta carne se nutre con la sangre y el cuerpo del Señor y llega a ser parte de este mismo cuerpo?
	
Por ello bien dice el Apóstol en su carta a los Efesios: \textit{Somos miembros de su cuerpo, hueso de sus huesos y carne de su carne}. Y esto	lo afirma no de un hombre invisible y mero espíritu ---pues un espíritu	no tiene carne y huesos---, sino de un organismo auténticamente humano,	hecho de carne, nervios y huesos; pues es este organismo el que se nutre 	con la copa, que es la sangre de Cristo, y se fortalece con el pan, que 	es su cuerpo.
	
Del mismo modo que el esqueje de la vid, depositado en tierra, fructifica a su tiempo, y el grano de trigo, que cae en tierra y muere, se multiplica pujante por la eficacia del Espíritu de Dios que sostiene todas las cosas, y así estas criaturas trabajadas con destreza se ponen al servicio del hombre, y después, cuando sobre ellas se pronuncia la Palabra de Dios, se convierten en la eucaristía, es decir, en el cuerpo y la sangre de Cristo; de la misma forma nuestros cuerpos, nutridos con esta eucaristía y depositados en tierra, y desintegrados en ella, resucitarán a su tiempo, cuando la Palabra de Dios les otorgue de nuevo la vida para la gloria de Dios Padre. Él es, pues, quien envuelve a los mortales con su inmortalidad y otorga gratuitamente la incorrupción a lo corruptible, porque la fuerza de Dios se realiza en la debilidad.
	
	\textbf{San Ireneo}, obispo, \textit{Tratado contra las herejías}, cf. Lib. 5, 2, 2-3: SC 153, 30-38.
\end{patercite}

\newpage

\subsubsection{Homilía (1976): Comunión}

\src{Jueves Santo, 15 de abril de 1976.}

\begin{body}
\ltr{C}{omunión} es la palabra que sale de los labios, si se quiere romper el silencio de los corazones sobrecogidos por los misterios que estamos celebrando. Rememoramos, más aún revivimos la hora de la Última Cena de Jesús con sus discípulos; una hora ya seria por su significado conmemorativo, como para formar la conciencia religiosa e histórica del pueblo judío, que recordó, con el sacrificio del cordero, la experiencia del éxodo de la esclavitud a una patria a reconquistar y a poseer en fidelidad a su destino religioso, durante siglos. 

La comunión fue la nueva atmósfera en la que se celebró aquella cena pascual: un ambiente afectivo e intenso cargado de esos sentimientos que van más allá del estilo habitual de conversación, aunque el lenguaje del Maestro siempre tuvo como objetivo llevar el entendimiento de sus discípulos más allá de los márgenes de la experiencia sensible, invitándoles a respirar en un área superior del misterio y a un descubrimiento trascendente de la verdad oculta y de la realidad divina. Pero esa noche, el nivel sentimental y espiritual era tan alto que a los discípulos de la cena les resulta más difícil que nunca hablar de ello. Mientras tanto, escuchemos los acentos sumamente cordiales, que son la clave para abrir el derroche discursivo del Maestro. \textquote{Cuando llegó el momento, –escribe el evangelista San Lucas–, tomó su lugar a la mesa y los apóstoles con él, y dijo: He deseado comer esta Pascua con vosotros, antes de mi pasión, ya que os digo: no la comeré más, hasta que se cumpla en el reino de Dios} (\textit{Lc} 22, 15). La Cena adquiere un carácter testamentario: el propio Gesto la define como el epílogo de su vida terrena; le da al banquete un carácter concluyente. El evangelista Juan, el amado iniciado en los secretos del corazón del Señor, escribe: \textquote{Antes de la fiesta de la Pascua, Jesús, sabiendo que había llegado su hora de pasar de este mundo al Padre, después de haber amado a los suyos que estaban en el mundo, los amó hasta el extremo} (\textit{Jn} 13, 1). San Agustín comenta: \textquote{El amor lo llevó a la muerte} (\textit{In Io. Tract}. 55, 2: \textit{PL} 35, 1786); y también la exégesis moderna: \textquote{Jesús, que siempre ha amado a los suyos, ahora demuestra su amor hasta el final, no sólo cronológicamente hasta el final de su vida, sino mucho más intensamente hasta el final alcanzable, hasta el límite extremo posible del amor mismo} (G. Ricciotti, \textit{Vida de Jesucristo}, 541).

El grado de intensidad emocional que producen las palabras y los actos de Jesús en ese banquete ritual, ya en sí mismo capaz de despertar en el espíritu una emoción fuerte y comunicativa, crece durante la vigilia convivial en escala ascendente: del anuncio de la próxima muerte sangrienta del Maestro (cf. \textit{Jn} 11, 16; 12, 24; etc.) que llenó de temor a los discípulos, y ahora afirmada abiertamente, hasta la inesperada y vergonzosa escena del lavatorio de los pies, realizado por Jesús después de la primera parte de la cena (\textit{Jn} 13, 2-17), y luego el patético y ahora abierto indicio de traición inminente; y luego, habiendo abandonado la mesa el presunto traidor (\textit{Ibid}. 13, 26 ss.), un momento de suprema despedida: \textquote{Hijos –¡así llama a los discípulos!–, todavía estoy un rato con vosotros\ldots Os doy un mandamiento nuevo: que os améis unos a otros, como –como: fijaros en la comparación, fijaros en la medida–, como yo os he amado, así también amaros entre vosotros. De este modo todos sabrán que sois mis discípulos si os amáis los unos a los otros} (\textit{Ibid}. 13, 33-35). También aquí permanece una relación, una comunión, en el rasgo constitutivo de una sociedad compuesta de amor. Llegamos así al momento de la suprema y misteriosa sorpresa. Escuchemos las reveladoras palabras: \textquote{Mientras cenaban, Jesús tomó el pan y, habiendo dicho la bendición, lo partió y se lo dio a los discípulos diciendo: tomad y comed, esto es mi cuerpo. Luego tomó la copa y, después de dar gracias, se la dio, diciendo: bebed todos, porque esto es mi sangre de la alianza, derramada por muchos, para remisión de los pecados} (\textit{Mt} 26, 26-28). 

¡Milagro! ¡Misterio de la fe! ¡Creemos en el milagro logrado! Creemos, como dice el Concilio de Trento, que Él, Cristo, \textquote{habiendo celebrado la antigua Pascua\ldots instituyó una nueva Pascua, inmolándose, confiriendo poder a la Iglesia a través de los Sacerdotes, bajo signos visibles, en memoria de su paso de este mundo al Padre} (Denz-Schön, 1741). 

Si esto es así, y así es, el misterio irradia ante nosotros, mientras podamos contemplarlo, una epifanía de comunión.

Comunión con Cristo, Sacerdote y víctima de un Sacrificio consumido de manera sangrienta en la cruz, incruenta en la Misa, cumbre de nuestra vida religiosa, donde él, a través de su palabra sacramental, redujo el pan y el vino a simples signos sensibles para convertir su sustancia en su carne y sangre, ofreciéndose a sí mismo, Cordero sacrificado en holocausto, restableciendo una comunión de gracia entre vivos y muertos, con Dios Padre todopoderoso y misericordioso (Cfr. Denz-Schön, 1743; 3847). Comunión ontológica, teológica, vital. 

Comunión nuevamente con Cristo, personal, mística, interior; comunión bipolar de nuestra humilde y fugaz vida humana y mortal con la Vida misma de Cristo, que es la Vida por definición (\textit{Jn} 14, 6), y que dijo de sí mismo: \textquote{Yo soy el Pan de Vida} (\textit{Ibíd}. 6, 35-49. 51), para que resuenen en nuestra conciencia profunda las palabras de la comunión más íntima y coexistente: \textquote{Ya no vivo yo, sino que es Cristo quien vive en mí} (\textit{Gal} 2, 20). ¿Quién podrá medir la fecundidad de esta comunión interior, que tiene Cristo maestro, camino, verdad y vida (\textit{Jn} 14, 6), como la savia de un árbol hasta sus brotes florecientes y fructíferos? (\textit{Ibíd}. 15, 1 ss.)

Comunión también de inefable eficacia social, principio que es válido para cimentar en la unidad sobrenatural pero también eclesial y comunitaria del Cuerpo Místico de Cristo a quienes se nutren del pan eucarístico. San Pablo lo enseña de nuevo: \textquote{La copa de bendición que consagramos, ¿no es comunión con la sangre de Cristo? Y el pan que partimos, ¿no es comunión con el cuerpo de Cristo? Puesto que hay un solo pan, nosotros, aunque muchos, somos un solo cuerpo; de hecho, todos participamos de un solo pan} (1 \textit{Cor} 10, 16-17). 

Comunión, pues, en el espacio de la tierra y en la dimensión de la humanidad creyente y de la participación en el banquete divino, dondequiera que se celebre regularmente: todos son invitados por el mismo Señor: \textit{compelle intrare,} ¡oblígalos a entrar! nos enseña la parábola del Evangelio (\textit{Lc} 14, 23). El hecho mismo de que Cristo hizo posible, a través del ministerio de los sacerdotes, multiplicar este bendito pan eucarístico, que es Él mismo, el Emmanuel, el Dios-con-nosotros que acompaña a los hombres en todos sus caminos y llama a todos con Voz pentecostal a su única Iglesia, ¿no hace evidente su divina intención de comunión universal a la más simple observación? \textit{¡Ut omnes unum sint}, para que todos sean uno! así Cristo oró en esa noche profética, después de la Última Cena.

Y ¿acaso a esta no se le suma otra comunión, aquella que se da en el tiempo, la de la permanencia de Jesucristo con nosotros, la de la tradición viva a lo largo de los siglos, comunión coherente, fiel, victoriosa del tiempo que pasa devorando, porque este milagro eucarístico está destinado, como escribe San Pablo, hasta el último \textit{donec veniat}, hasta que Él, Cristo, vuelva (1 \textit{Cor} 11, 26), el último día de la Parusía? Y así lo declaró el mismo Cristo, como nos dicen las últimas palabras de su Evangelio: \textquote{He aquí que yo estoy con vosotros todos los días hasta el fin de los tiempos} (\textit{Mt} 28, 20). 

En este punto nuestra meditación, que investiga la comunión polivalente resultante del misterio eucarístico, se vuelve curiosa por los cálculos y la estadística. Si Cristo es el centro, en el sacramento de su sacrificio, que atrae a todos hacia sí (cf. \textit{Jn} 12, 32), surge la pregunta: ¿están todos realmente fascinados y atraídos por esta comunión con él? ¿Cuántos estamos unidos en esa unidad que Cristo nos dejó como su aspiración testamentaria? (\textit{Ibid}. 17). Y, ¿estamos verdaderamente en aquella unidad de fe, de amor y de vida que constituían el deseo soberano y misericordioso de Jesús? ¿Estamos dispuestos a hacer de la unidad al interior de la Iglesia y en la Iglesia nuestra aspiración constitutiva, nuestro programa de vida eclesial? ¿Es real y siempre un soplo del Espíritu Santo eso que a menudo frena y a veces rompe los lazos de nuestra bendita comunión en el cuerpo visible y místico de Cristo con empuje centrífugo y ambición individualista? ¿No es este el día, el momento de abandonar todas las reservas egoístas por la reconciliación fraterna, el perdón mutuo, la unidad del amor humilde? ¿Podemos hacer llegar a los hijos lejanos un afectuoso recuerdo de su regreso a la mesa espiritual común? ¡Qué fervor misionero surge en nosotros de la celebración de este Jueves Santo! ¡Qué espíritu fraterno, qué celo pastoral, qué propósito apostólico! ¡Qué esperanza de comunión cristiana!

¿Y no tendremos en esta bendita velada un pensamiento, un saludo, una oración ecuménica por tantos hermanos cristianos todavía separados de nosotros? 

Y a todos aquellos que sufren o tienen hambre de verdad, justicia y paz, pero con los ojos nublados en su búsqueda insatisfecha, ¿no podremos recordarles, al menos en nuestra oración interior, la invitación que siempre les dirige Aquel que es el único que puede colmar sus deseos: \textquote{Venid a mí todos los que estáis cansados y agobiado, y yo os aliviaré}? (\textit{Mt} 11, 28) ¡La Iglesia es una comunión!

Que así sea, que así sea, con nuestra cordial Bendición.
\end{body}

\begin{patercite}
	Nuestro Señor Jesucristo, en la noche en que iban a entregarlo, tomó	pan y, pronunciando la acción de gracias, lo partió y lo dio a sus discípulos, diciendo: \textquote{Tomad, comed; esto es mi cuerpo}. Y, después de	tomar el cáliz y pronunciar la acción de gracias, dijo: \textquote{Tomad, bebed; ésta es mi sangre}. Si fue él mismo quien dijo sobre el pan: \textquote{Esto es	mi cuerpo}, ¿quién se atreverá en adelante a dudar? Y si él fue quien	aseguró y dijo:\textquote{Ésta es mi sangre}, ¿quién podrá nunca dudar y decir	que no es su sangre?
	
	Por lo cual estamos firmemente persuadidos de que recibimos como alimento el cuerpo y la sangre de Cristo. Pues bajo la figura del pan se te da el cuerpo, y bajo la figura del vino, la sangre; para que, al	tomar el cuerpo y la sangre de Cristo, llegues a ser un solo cuerpo y	una sola sangre con él. Así, al pasar su cuerpo y su sangre a nuestros	miembros, nos convertimos en portadores de Cristo. Y como dice el	bienaventurado Pedro, nos hacemos \textit{partícipes de la naturaleza divina}.
	
	En la antigua alianza existían también los panes de la proposición: pero	se acabaron precisamente por pertenecer a la antigua alianza. En cambio,	en la nueva alianza, tenemos un pan celestial y una bebida de salvación,	que santifican alma y cuerpo. Porque del mismo modo que el pan es	conveniente para la vida del cuerpo, así el Verbo lo es para la vida del alma.
	
	No pienses, por tanto, que el pan y el vino eucarísticos son elementos simples y comunes: son nada menos que el cuerpo y la sangre de Cristo, de acuerdo con la afirmación categórica del Señor (\ldots) {La fe que has aprendido te da, pues, esta certeza: lo que parece pan no	es pan, aunque tenga gusto de pan, sino el cuerpo de Cristo; y lo que	parece vino no es vino, aun cuando así lo parezca al paladar, sino la	sangre de Cristo; por eso, ya en la antigüedad, decía David en los salmos: \textit{El pan da fuerzas al corazón del hombre y el aceite da brillo	a su rostro}; fortalece, pues, tu corazón comiendo ese pan espiritual,	y da brillo al rostro de tu alma.}
	
	Y que con el rostro descubierto y con el alma limpia, contemplando la	gloria del Señor como en un espejo, vayamos de gloria en gloria, en	Cristo Jesús, nuestro Señor, a quien sea dado el honor, el poder y la	gloria por los siglos de los siglos. Amén.
	
	De las \textbf{Catequesis de Jerusalén}, Catequesis Mistagógica cf. 22, 4, 1. 3-6. 9: PG 33, 1098-1106.
\end{patercite}

\newsection
\subsection{San Juan Pablo II, papa}

\subsubsection{Homilía (1979): ¿Qué significa amor hasta el fin?}

\src{12 de abril de 1979.}

\begin{body}
\ltr[1. ]{H}{a} llegado la \textquote{hora} de Jesús. Hora de su paso de este mundo al Padre. Comienza el triduo sacro. El misterio pascual, como cada año, se reviste de su aspecto litúrgico, comenzando por esta Misa, única durante el año, que lleva el nombre de \textquote{Cena del Señor}. Después de haber amado a los suyos que estaban en el mundo, \textquote{\textit{los amó hasta el fin}} (\textit{Jn} 13, 1). La última Cena es precisamente testimonio del amor con que Cristo, Cordero de Dios, nos ha amado hasta el fin.

En esta tarde los hijos de Israel comían el cordero, según la prescripción antigua dada por Moisés en la víspera de la salida de la esclavitud de Egipto. Jesús hace lo mismo con los discípulos, fiel a la tradición, que era sólo la \textquote{sombra de los bienes futuros} (\textit{Heb} 10, 1), sólo la \textquote{figura} de la Nueva Alianza, de la nueva Ley.

2. ¿Qué significa \textquote{los amó hasta el fin}? Significa: hasta el cumplimiento que debía realizarse mañana, Viernes Santo. En este día se debía manifestar cuánto amó Dios al mundo, y cómo, en el amor, se ha llegado al límite extremo de la donación, esto es, al punto de \textquote{dar a su unigénito Hijo} (\textit{Jn} 3, 16). En ese día Cristo ha mostrado que no hay \textquote{amor mayor que éste: de dar uno la vida por sus amigos} (\textit{Jn} 15, 13). El amor del Padre se reveló en la donación del Hijo. En la donación mediante la muerte.

El Jueves Santo, el día de la última Cena, es, en cierto sentido, el prólogo de esta donación; es la preparación última. Y en cierto modo lo que se cumplía en este día va ya más allá de tal donación. Precisamente el Jueves Santo, durante la última Cena, se manifestaba lo que quiere decir: \textquote{Amó hasta el fin}. En efecto, pensamos justamente que amar hasta el fin signifique \textit{hasta la muerte}, hasta el último aliento. Sin embargo, la última Cena nos muestra que, para Jesús, \textquote{hasta el fin} significa más allá del último aliento. \textit{Más allá de la muerte}.

3. Este es precisamente el significado de la Eucaristía. La muerte no es su fin, sino su comienzo. La Eucaristía comienza en la muerte, como enseña \textbf{San Pablo}: \textquote{Cuantas veces comáis este pan y bebáis este cáliz, anunciáis la muerte del Señor hasta que Él venga} (\textit{1 Cor} 11, 26).

La Eucaristía es fruto de esta muerte. La recuerda constantemente. La renueva de continuo. La significa siempre. La proclama. La muerte, que ha venido a ser principio de la nueva venida: de la resurrección a la parusía, \textquote{hasta que Él venga}. La muerte, que \textit{es \textquote{sustrato} de una nueva vida}. Amar \textquote{hasta el fin} significa, pues, para Cristo, amar mediante la muerte y más allá de la barrera de la muerte: ¡\textit{Amar hasta los extremos de la Eucaristía!}

4. Precisamente Jesús ha amado así en esta última Cena. Ha amado a los \textquote{suyos} –a los que entonces estaban con,Él– y a todos los que debían heredar de ellos el misterio:

– las palabras que ha pronunciado sobre el pan,

– las palabras que ha pronunciado sobre el cáliz, lleno de vino,

– las palabras que nosotros repetimos hoy con particular emoción y que repetimos siempre cuando celebramos la Eucaristía, ¡son precisamente la revelación del amor a través del cual, de una vez para siempre, para todos los tiempos y hasta el fin de los siglos, se ha repartido a Sí mismo!

Antes aún de \textit{darse a Sí mismo} en la cruz, como \textquote{Cordero que quita los pecados del mundo}, \textit{se ha repartido a Sí mismo} como comida y bebida: pan y vino para que \textquote{tengamos vida y la tengamos en abundancia} (\textit{Jn} 10, 10). Así Él \textquote{amó hasta el fin}.

5. Por lo tanto, Jesús no dudó en arrodillarse delante de los Apóstoles para lavar sus pies. Cuando Simón Pedro se opone a ello, Él le convenció para que le dejara hacer. Efectivamente, era una exigencia particular de la grandeza del momento. Era necesario este lavatorio de los pies, esta purificación en orden a la comunión de la que habrían de participar desde aquel momento.

Era necesario. Cristo mismo sintió la necesidad de humillarse a los pies de sus discípulos: una humillación que nos dice tanto de Él en ese momento. De ahora en adelante, distribuyéndose a Sí mismo en la comunión eucarística, ¿no se abajará continuamente al nivel de tantos corazones humanos? ¿No los servirá siempre de este modo?

\textquote{Eucaristía} significa \textquote{agradecimiento}. \textquote{Eucaristía} significa también \textquote{servicio}, el tenderse hacia el hombre: \textit{el servir a tantos corazones humanos}. \textquote{Porque yo os he dado el ejemplo, para que vosotros hagáis también como yo he hecho} (\textit{Jn} 13, 15). ¡No podemos ser dispensadores de la Eucaristía, sino sirviendo!

6. Así, pues, es la última Cena. Cristo se prepara a \textit{irse a través de la muerte}, y a través de la misma muerte se prepara a permanecer. De esta forma la muerte se ha convertido en el fruto maduro del amor: nos amó \textquote{hasta el fin}.

¿No bastaría aun sólo el contexto de la última Cena para dar a Jesús el \textquote{derecho} de decirnos a todos: \textquote{Este es mi precepto: que os améis unos a otros como yo os he amado} (\textit{Jn} 15, 12)?
\end{body}


\subsubsection{Homilía (1982): El deseo de Cristo}

\src{Basílica de San Juan de Letrán. 8 de abril de 1982.}

\begin{body}
1. \textquote{El Padre había puesto todo en sus manos} (\textit{Jn} 13, 3). 

\ltr{A}{ntes} de la Cena pascual Cristo tiene conciencia clara de que el Padre le ha puesto todo en las manos. Es libre con toda la plenitud de la libertad que goza el Hijo del hombre, el Verbo encarnado. \textit{Es libre} con una libertad tal que no es propia de ningún otro hombre. La última Cena: todo lo que en ella se cumplirá tiene su origen \textit{en la perfecta libertad del Hijo respecto del Padre.}

Dentro de poco llevará esta libertad suya a Getsemaní y dirá: \textquote{Padre, si quieres, aparta de mí este cáliz; pero no se haga mi voluntad sino la tuya} ( \textit{Lc} 22, 42). Entonces aceptará el sufrimiento que caerá sobre Él y que es al mismo tiempo objeto de una opción; un sufrimiento de dimensiones inconcebibles para nosotros. 

Pero durante la última Cena la opción estaba ya hecha. Cristo actúa con plena conciencia de la opción ya realizada. Sólo tal conciencia explica el hecho de que Él, \textquote{tomando el pan, dio gracias, lo partió y se lo dio diciendo: Este es mi cuerpo que es entregado por vosotros} (\textit{Lc} 22, 19). Y después de cenar tomó el cáliz diciendo: \textquote{Este cáliz es la nueva alianza sellada con mi sangre}, como refiere San Pablo \textit{(1 Cor} 11, 25), mientras los Evangelistas puntualizan: \textquote{en mi sangre que es derramada por vosotros} (\textit{Lc} 22, 20), o \textquote{mi sangre de la Nueva Alianza que será derramada por muchos} ( \textit{Mt} 26, 28; \textit{Mc} 14, 24). 

Al pronunciar estas palabras en el Cenáculo, Cristo ya ha hecho la opción. 

La ha hecho hace tiempo. Ahora la realiza de nuevo. Y en Getsemaní la cumplirá una vez más al aceptar en el dolor toda la inmensidad del sufrimiento vinculado a esta opción. \textquote{El Padre había puesto todo en sus manos}. Todo, el designio completo de la salvación, el Padre lo ha entregado a su libertad perfecta. \textit{Y a su amor perfecto}.

2. Mediante la opción de Cristo, mediante su libertad perfecta y su amor perfecto, en la cena pascual \textit{la figura del cordero pascual} llegó al culmen de su significado. 

De su institución habla la lectura de hoy del \textbf{Éxodo}: \textquote{Lo comeréis así: la cintura ceñida, las sandalias en los pies, un bastón en la mano; y os lo comeréis a toda prisa, porque \textit{es la Pascua} del Señor}. \textquote{Será un animal sin defecto\ldots lo guardaréis hasta el día catorce del mes y toda la asamblea de Israel lo matará al atardecer. Tomaréis la sangre y rociaréis las dos jambas y el dintel de la casa donde lo hayáis comido\ldots}, \textquote{Yo pasaré esta noche por la tierra de Egipto\ldots Cuando yo vea la sangre pasaré de largo ante vosotros, y no habrá entre vosotros plaga exterminadora cuando yo hiera al país de Egipto} (\textit{Ex} 12, 11. 5-7. 12-13). 

\textit{Esta es la Pascua de la Antigua Alianza.}

El recuerdo del \textit{Paso} por Egipto de la mano purificadora del Señor. El recuerdo de la \textit{salvación} mediante la sangre del cordero inocente. El recuerdo de la liberación de la esclavitud. El día catorce de Nisán de cada año, celebra todavía Israel la Pascua. Por su parte Cristo celebra con los Apóstoles la última Cena. Meditan sobre la \textit{liberación} de la esclavitud \textit{mediante la sangre del cordero inocente} . 

Y Cristo dice sobre el pan: \textquote{Tomad y comed; éste es mi cuerpo, \textit{que ha sido entregado por vosotros}}. Después dice sobre el vino: \textquote{Tomad y bebed, éste es el cáliz de mi sangre \textit{que será derramada por vosotros}}. Por vosotros y por todos (cfr. \textit{Mt} 26, 26-28; \textit{Lc} 22, 19-20). En el marco de estas palabras aparece ya el \textit{cumplimiento de la figura del cordero} de la Antigua Alianza. Y de pronto, en la historia de la humanidad, en la historia de la salvación, entra el Cordero de la Nueva Alianza, el Cordero más inocente, el \textit{Cordero de Dios} . 

Entra mediante su Cuerpo y Sangre; mediante el Cuerpo que será entregado, mediante la Sangre que será derramada. Entra a través de la muerte que libera de la esclavitud de la muerte del pecado. \textit{Entra a través de la muerte que da la vida}. El sacramento de la última Cena es el signo visible de esta vida. Es alimento de vida eterna. 

3. Sucedió \textquote{antes de la fiesta de Pascua}. Aquella fue la \textit{hora de Cristo}, la hora \textquote{de pasar de este mundo al Padre}. 

En aquella hora, \textquote{habiendo amado a los suyos que estaban en el mundo, los amó hasta el extremo} (\textit{Jn} 13, 1). \textquote{Los suyos que estaban en el mundo}, ¿acaso solamente los que estaban con Él en la hora de la última Cena? No sólo ellos. Amó a todos \textquote{los suyos}, a todos los que iba a redimir. A todos desde el principio del mundo hasta el fin. A todos \textit{y en todos los sitios}. 

Y entonces les lavó los pies; a los que estaban en el Cenáculo A Pedro, el primero. Entonces, en el momento de su primera Eucaristía, les deseó pureza, una pureza mayor de cuanto ellos mismos habían pensado; de cuanto había pensado Pedro. Y desea esta pureza a todos. El amor le obliga a desear pureza a todos y en todos los sitios. \textquote{Si no te lavo, no tienes nada que ver conmigo} (\textit{Jn} 13, 8). 

En la Eucaristía Cristo desea compartir su vida conmigo, \textit{desea la comunión. En la perspectiva de esta comunión con el hombre, desea la pureza} de su alma.

Esta es, pues, la hora de la última Cena. La hora de Cristo. \textit{La hora} del grande e ilimitado deseo de su corazón; Él desea la comunión con el hombre y desea la pureza del alma humana. 

¿Acaso podemos rehuir este deseo?
\end{body}

\label{b-05-01-1982H}
\newpage

\subsubsection{Homilía (1985): Sacramento del Siervo}

\src{Basílica de San Juan de Letrán. 4 de abril de 1985.}

\begin{body}
1. \textquote{\textit{¡No me lavarás los pies jamás!}} (\textit{Jn} 13, 8). 

\ltr{H}{oy,} reunidos (\ldots) para la liturgia de la Cena del Señor, escuchamos estas palabras de rechazo de Pedro. 

Sin embargo, Jesús convence al apóstol. El lavatorio de los pies es ciertamente una función de servicio, pero también es expresión y signo de \textit{participación en toda la obra mesiánica de Cristo}. Pedro aún no lo ve. \textquote{Si no te lavo, no tendrás parte conmigo} (\textit{Jn} 13, 8). 

Pedro todavía no comprende; pero \textit{su corazón ya está completamente volcado a la obra mesiánica de Cristo}: a lo que Cristo quiere. Por eso dice: \textquote{¡No solo los pies, sino también las manos y la cabeza!} (\textit{Jn} 13, 9). 

2. Cristo lava los pies de Pedro y de todos los apóstoles. Dentro de poco, en memoria e imitación de ese gesto del Señor, lavaré los pies a doce [sacerdotes que concelebran conmigo en la liturgia eucarística de esta noche]. El lavatorio de los pies a los apóstoles por Jesús fue \textit{una introducción a la Cena de Pascua}. Esta función de servicio debe confirmar una vez más que Jesús no vino al mundo para ser servido, sino para ser él mismo un servidor: Él, el Maestro y el Señor. 

Los apóstoles deben pensar y actuar de la misma manera: \textquote{Yo os he dado\ldots ejemplo} (\textit{Jn} 13, 15). 

La función de servicio al comienzo de esta noche pascual \textit{manifiesta la presencia del \textquote{siervo}}. Es el \textquote{\textit{siervo de Yahvé}} de la profecía de Isaías. Jesús quiere indicar de esta manera que la Cena de Pascua da inicio al cumplimiento de las palabras de Isaías. Más aún, la misma \textit{Cena se convertirá en el sacramento del siervo}: \textquote{Yo soy tu siervo, hijo de tu sierva} (\textit{Sal} 116, 16). 

3. Durante la Cena de Pascua, todos los participantes dirigen su recuerdo \textit{hacia el cordero pascual}, cuya sangre en los dinteles de las casas salvó de la muerte a los primogénitos de Israel y abrió el camino al éxodo de Egipto. 

\textit{También Jesús} dirige la mirada de su alma hacia el cordero pascual, recuerda la liberación de la esclavitud en Egipto. 

Y al mismo tiempo tiene en sus oídos \textit{la voz de Juan el Bautista}, que a orillas del Jordán le señaló y proclamó: \textquote{He aquí el Cordero de Dios, he aquí el que quita el pecado del mundo} (\textit{Jn} 1, 29). Y, he aquí la Última Cena. Jesús sabe que ha llegado el momento del cumplimiento de las palabras de Juan cerca del Jordán. \textit{La sangre del cordero debe quitar los pecados del mundo}. 

4. De esta manera la Cena de Pascua alcanza su cenit. Jesús toma primero el pan, lo parte y, habiendo pronunciado la oración de acción de gracias, lo da a los apóstoles para que lo coman: \textquote{\textit{Esto es mi cuerpo que es entregado por vosotros}; haced esto en memoria mía} (\textit{Lc} 22, 19). 

Luego toma el cáliz lleno de vino. Y dice (según el texto de Pablo): \textquote{Esta copa \textit{es la nueva alianza en mi sangre}; haced esto cada vez que lo bebáis en memoria de mí} (\textit{1 Co} 11, 25). 

El profeta Isaías compara al siervo que sufre con un cordero. Juan el Bautista dice expresamente: el Cordero de Dios. 

Jesús, teniendo que cumplir las palabras del profeta y de Juan, instituye en la Eucaristía \textit{la alianza nueva y eterna en su sangre}. 

En la Eucaristía \textit{ya está contenido} todo lo que pronto comenzará a suceder místicamente. La Escritura dice: \textquote{En efecto, cada vez que coméis de este pan y bebéis de esta copa, proclamáis la muerte del Señor hasta que Él venga} (\textit{1 Co} 11, 26). 

De esta manera, la Eucaristía de la Última Cena \textit{anticipa la realidad} de la que es signo. 

Y al mismo tiempo, a través de la Eucaristía, también se anuncia otra realidad: la redención del mundo, la nueva alianza en la sangre del Cordero de Dios, una realidad que continúa. 

A través de la Eucaristía, esta realidad \textit{se hace presente constantemente y se renueva de manera sacramental}: \textquote{Anuncia la muerte del Señor hasta que venga}. 

5. Esta realidad \textit{se explica por el amor}: la cruz y la muerte del Cordero de Dios se explican por el amor. La redención del mundo se explica por el amor. La alianza nueva y eterna en la sangre de Cristo se explica por el amor. 

\textquote{\textit{Porque tanto amó Dios al mundo\ldots}} (\textit{Jn} 3, 16). 

Y Jesús, \textquote{sabiendo que había llegado la hora de pasar de este mundo al Padre, después de haber amado a los suyos que estaban en el mundo, \textit{los amó hasta el extremo}} (\textit{Jn} 13, 1). 

Esto es precisamente la Eucaristía. La Eucaristía se explica a través del amor. \textit{La Eucaristía nace del amor y da origen al amor}. En ella \textit{está inscrito} y arraigado definitivamente \textit{el mandamiento del amor}. \textquote{Os doy un mandamiento nuevo: amaos los unos a los otros \textit{como yo os he amado}} (\textit{Jn} 13, 34).

He aquí la Última Cena: el misterio de la Pascua. A partir de ahora, \textit{el amor y la muerte} caminarán juntos por la historia del hombre, hasta que venga de nuevo Aquel que los unió con un vínculo inquebrantable y nos los dejó en la Eucaristía, para que nosotros hagamos lo mismo en memoria suya.
\end{body}

\label{b-05-01-1985H}
\newpage

\subsubsection{Homilía (1988): La hora del Siervo} 

\src{Basílica de San Juan de Letrán. 31 de marzo de 1988.}

\begin{body}
1. \textquote{No me lavarás los pies jamás} (\textit{Jn} 13, 8). 

\ltr{A}{sí} dice Simón Pedro en el Cenáculo, cuando Cristo, antes de la cena pascual, decide lavar los pies de sus apóstoles. Cristo sabía que \textquote{había llegado su hora} (\textit{Jn} 13, 1). Su Pascua. Pero Simón Pedro todavía no lo sabía. Cerca de Cesarea de Filipo fue el primero en confesar: \textquote{Tú eres el Cristo, el Hijo del Dios vivo} (\textit{Mt} 16, 16). 

Sin embargo, no sabía que el significado de \textquote{siervo}, el siervo de Yahvé, también estaba oculto en esta definición de \textquote{Cristo-Mesías}. ¡No sabía! En cierto sentido, no quería tomar conciencia de la verdad que, según la interpretación del Maestro, tenía \textquote{su hora}: esto es \textquote{la hora del paso de este mundo al Padre} (cf. \textit{Jn} 13, 1). 

No aceptaba que Cristo tuviera que ser un siervo, como lo había visto muchos siglos antes el profeta Isaías: el siervo de Yavé, el siervo sufriente de Dios. 

2. Sin embargo, en el horizonte de la historia el papel de la sangre se ha definido ahora de forma definitiva: la sangre del cordero pascual debía encontrar su cumplimiento en la sangre de Cristo que selló la nueva y eterna alianza. 

Para aquellos que se estaban preparando para la cena pascual en el Cenáculo de Jerusalén, la sangre del cordero estaba relacionada con el recuerdo del Éxodo. Recordaba la liberación de la esclavitud de Egipto, que había iniciado el pacto de Yahweh con Israel, en tiempos de Moisés. 

\textquote{Tomaréis un poco de su sangre, la colocaréis en las dos jambas de las puertas y en el dintel de las casas, donde lo comáis\ldots ¡Es la Pascua del Señor!\ldots Veré la sangre y pasaré, no habrá azote de exterminio para vosotros cuando golpee la tierra de Egipto} (\textit{Ex} 12, 7. 11. 13). 

La sangre del cordero constituyó un umbral frente al cual se detuvo la ira castigadora de Yahvé. Los que se preparaban para la cena pascual, en el Cenáculo, guardaban en su memoria la liberación de Israel a través de esta sangre. 

3. Todos ellos, y Simón Pedro junto con ellos, no eran plenamente conscientes de que esa liberación por la sangre del cordero pascual era al mismo tiempo un anticipo. Era una \textquote{figura} que esperaba su realización en Cristo. 

Cuando los apóstoles se reunieron en el Cenáculo para la Última Cena, esto estaba próximo a cumplirse. Cristo sabe que \textquote{ha llegado su hora}, la hora en que él mismo cumplirá lo que estaba ya anunciado y revelará plenamente la realidad que durante siglos ha sido indicada por la \textquote{figura} del cordero pascual: la liberación por su sangre. 

Cristo va al encuentro de esta \textquote{plenitud}, entra en esta realidad. Él es consciente de lo que traerán consigo la noche que se avecina y el día siguiente. 

4. Y he aquí, toma el pan en sus manos y, dando gracias, dice: \textquote{Esto es mi cuerpo, entregado por vosotros} (\textit{1 Co} 11, 24). Y al final de la cena (como leemos en la primera carta a los Corintios) toma el cáliz y dice: \textquote{Este cáliz es la nueva alianza en mi sangre} (\textit{1 Co} 11, 25). 

Cuando el cuerpo de Cristo sea ofrecido en la cruz, entonces esta sangre, derramada en la pasión, será el comienzo de la nueva alianza de Dios con la humanidad. 

La alianza antigua, en la sangre del cordero pascual, la sangre de la liberación de la esclavitud de Egipto. 

La alianza nueva y eterna, en la sangre de Cristo. 

Cristo va al sacrificio, que tiene el poder redentor: el poder de liberar al hombre de la esclavitud del pecado y la muerte. El poder de arrancar al hombre del abismo de la muerte espiritual y la condenación. 

Jesús pasa a los discípulos la copa de la salvación, la sangre de la nueva alianza y dice: \textquote{Haced esto cada vez que lo bebáis en memoria de mí} (\textit{1 Co} 11, 25). 

5. \textquote{Habiendo amado a los suyos que están en el mundo, los amó hasta el extremo} (\textit{Jn} 13, 1). 

Aquí está la verdad más profunda de la Última Cena. El cuerpo y la sangre, la pasión y muerte en la cruz significan precisamente esto: \textquote{los amó hasta el extremo}. 

La sangre del cordero en el dintel de las casas en Egipto no tenía, en sí misma, un poder liberador. El poder vino de Dios y durante mucho tiempo no nos atrevimos a llamar a este poder por su nombre. 

Cristo lo llamó por su nombre. El cuerpo y la sangre, la pasión y la muerte, el sacrificio, son el amor que se remonta hasta los confines de su potencia salvífica.

Cristo lo llamó por su nombre. Cristo hizo que sucediera. Cristo nos dejó este poder en la Eucaristía. He aquí \textquote{su hora} en la que: 

-- pasa de este mundo al Padre por la sangre de la nueva alianza; 

-- pasa de este mundo al Padre por el amor, que se remonta hasta los confines de su potencia salvífica.

6. Él dice: \textquote{Haced esto en memoria mía}. 

E incluso antes de eso, dice: \textquote{Os he dado ejemplo, para que como yo lo hice, también vosotros podáis hacerlo} (\textit{Jn} 13, 15). 

\textquote{Amaos los unos a los otros como yo os he amado} (\textit{Jn} 13, 34). Última Cena. El comienzo de la nueva alianza en la sangre de Cristo. 

¡Revivámosla con un corazón lleno de fe y amor!
\end{body}

\label{b-05-01-1988H}
\newpage

\subsubsection{Homilía (1991): Entregó todo en sus manos}

\src{Basílica de San Juan de Letrán. 28 de marzo de 1991.}

\begin{body}
\textquote{\textit{El Padre había puesto todo en sus manos}} (\textit{Jn} 13, 3).

\ltr[1. ]{C}{uando} Cristo se reunió con los Apóstoles en el Cenáculo para comer la Pascua con ellos, ya lo sabía. Todo esto lo supo durante su vida terrena, durante los años de su misión mesiánica, pero ahora lo sabe de manera particular, de manera definitiva: \textquote{El Padre había puesto todo en sus manos}. 

\textit{Con esta conciencia, irá a Getsemaní}, será juzgado y condenado a muerte \textit{en la Cruz}. Esta conciencia, esta certeza, se convertirá en su sufrimiento; un sufrimiento humano, aunque humanamente inexpresable. Se convertirá en su sacrificio redentor. \textquote{El Padre había puesto todo en sus manos}. \textit{Todo significa la creación entera}, incluida en el plan divino y eterno. Todo significa \textit{cada hombre y toda la humanidad}. Nadie más podía entender este signo, además de él, el Hijo consustancial al Padre, el Verbo Eterno, el Primogénito de todas las criaturas. ¡Sólo él! El Padre puso en sus manos todo el futuro del Reino de Dios, la escatología de la historia humana. Finalmente, \textit{sólo Él puede devolverlo todo al Padre}, \textquote{para que Dios sea todo en todos} (\textit{1 Co} 15, 28). 

2. Esta conciencia del Hijo significa, al mismo tiempo, una particular plenitud de amor. Cuando \textquote{llegó su tiempo de pasar de este mundo al Padre, \textit{habiendo amado a los suyos que estaban en el mundo, los amó hasta el extremo}} (\textit{Jn} 13, 1). ¡Hasta el extremo! De este amor suyo \textquote{hasta el extremo} nace la Eucaristía. De este amor nacen Getsemaní y el Gólgota; nace la obediencia hasta la muerte y muerte de cruz (cf. \textit{Fil} 2, 8); ¡nace la Eucaristía! 

Cristo, volviendo al Padre, sabe que no puede dejarnos. Debe quedarse porque el Padre \textquote{le había entregado todo en sus manos}. No puede pasar como \textquote{todo} pasa por el universo creado. No puede simplemente pasar a la historia. \textit{Debe permanecer por encima de la historia y dentro de la historia}, para que Dios sea \textquote{todo en todos} (cf. \textit{1 Co} 15, 28). 

3. \textit{La Eucaristía: ¡Acontecimiento y Sacramento!} Hoy la vivimos de una manera particular. Más que en cualquier otro momento, la liturgia del Jueves Santo, \textquote{in Cena Domini}, es \textquote{la memoria} de este Acontecimiento. Y, al mismo tiempo, es el Sacramento que perdura y se hace presente \textit{en su profundidad y potencia originales} cada vez que \textquote{comemos este pan y bebemos este cáliz}; siempre que \textquote{anunciamos la muerte del Señor hasta que venga}; cada vez que expresamos \textquote{la Nueva Alianza} en la sangre de Cristo (cf. \textit{1 Co} 11, 25-26): ¡la Nueva y Eterna Alianza! 

4. Cristo, a quien el Padre \textquote{había entregado todo en sus manos}, entra en esta Hora culminante de la historia \textit{como Siervo}. La imagen del Siervo de Dios, tomada del profeta Isaías (cf. \textit{Is} 42, 1-2) se realiza plenamente en él. 

\textquote{Vosotros me llamáis Maestro y Señor y decís bien, porque lo soy} (\textit{Jn} 13, 13). Y he aquí que yo, Maestro y Señor, a quien el Padre ha puesto todo en sus manos, os \textit{lavo los pies} (cf. \textit{Jn} 13, 13). 

\textit{También se hace Siervo}. 

Y así lo hizo Cristo, y así ha venido a ser para siempre: como la luz de nuestra conciencia, como el servidor de la redención del hombre. \textit{El mayor servicio del} Cordero de Dios \textit{es el sacrificio redentor en la Cruz}. En la Eucaristía, el Hijo, glorificado a la diestra del Padre, permanece como servidor de nuestra redención. 

\textquote{Os he dado\ldots el ejemplo, para que\ldots vosotros también hagáis (así)} (\textit{Jn} 13, 15). 

El Jueves Santo, \textquote{in Cena Domini}, redescubrimos cada vez mejor el significado de este \textquote{sacerdocio ministerial}. 

¡Gloria a Ti, Rey de los siglos! (cf. 1 \textit{Tm} 1, 17).
\end{body}

\label{b-05-01-1991H}

\begin{patercite}
(\ldots) ¿Cuál es entonces el núcleo de esta Cena? Son los gestos de partir el pan, de distribuirlo a los suyos y de compartir el cáliz del vino con las palabras que los acompañan y en el contexto de oración en el que se colocan: es la institución de la Eucaristía, es la gran oración de Jesús y de la Iglesia. (\ldots)

Antes de las palabras de la institución se realizan los gestos: el de partir el pan y el de ofrecer el vino. Quien parte el pan y pasa el cáliz es ante todo el jefe de familia, que acoge en su mesa a los familiares; pero estos gestos son también gestos de hospitalidad, de acogida del extranjero, que no forma parte de la casa, en la comunión convival. En la cena con la que Jesús se despide de los suyos, estos mismos gestos adquieren una profundidad totalmente nueva: él da un signo visible de acogida en la mesa en la que Dios se dona. Jesús se ofrece y se comunica él mismo en el pan y en el vino.

¿Pero cómo puede realizarse todo esto? ¿Cómo puede Jesús darse, en ese momento, él mismo? Jesús sabe que están por quitarle la vida a través del suplicio de la cruz, la pena capital de los hombres no libres, la que Cicerón definía la \emph{mors turpissima crucis}. Con el don del pan y del vino que ofrece en la última Cena Jesús anticipa su muerte y su resurrección realizando lo que había dicho en el discurso del Buen Pastor: \textquote{Yo entrego mi vida para poder recuperarla. Nadie me la quita, sino que yo la entrego libremente. Tengo poder para entregarla y tengo poder para recuperarla: este mandato he recibido de mi Padre} (\emph{Jn} 10, 17-18). Él, por lo tanto, ofrece por anticipado la vida que se le quitará, y, de este modo, transforma su muerte violenta en un acto libre de donación de sí mismo por los demás y a los demás. La violencia sufrida se transforma en un sacrificio activo, libre y redentor.

\textbf{Benedicto XVI}, papa, \textit{Catequesis}, Audiencia general, 11 de enero de 2012, parr. 4. 6-7.

\end{patercite}
\newpage

\subsubsection{Homilía (1994): Servicio de una humilde caridad}

\src{31 de marzo de 1994.}

\begin{body}
1. \textquote{\textit{Y comenzó a lavar los pies de los discípulos}\ldots} (\textit{Jn} 13, 5) 

\ltr{E}{sta} es la tarde en la que la Iglesia revive el gesto y el sentido del lavatorio de los pies, que debía introducir a los Apóstoles, reunidos en el Cenáculo, a la institución de la Eucaristía. 

\textit{¿Por qué quiso Cristo comenzar con el lavatorio de los pies?} Hizo esto para presentarse ante ellos en la condición de un sirviente. Él mismo lo explica cuando dice: \textquote{Si yo, el Señor y el Maestro, os he lavado los pies, vosotros también debéis lavaros los pies unos a otros} (\textit{Jn} 13, 14). El lavatorio de los pies expresa el servicio de una humilde caridad. Durante la Última Cena, Cristo \textit{quiere revelarse como el que sirve}: \textquote{Yo estoy entre vosotros como el que sirve} (\textit{Lc} 22, 27). Un verdadero discípulo de Cristo es sólo el que tiene \textquote{parte} con el Maestro, dispuesto a servir como Él. El servicio, es decir, el cuidado de las necesidades de los demás, constituye la esencia de todo poder. Servir significa reinar. 

2. En la hora en que se prepara para cumplir el misterio pascual, Cristo se manifiesta entre nosotros como el que sirve. En efecto, la verdadera razón última de su venida al mundo aparece a los ojos de los discípulos: el ministerio de la redención del hombre y la salvación del mundo. 

En este ministerio \textit{se ofrece a sí mismo}: se entrega a la muerte en la cruz \textit{para darse a sí mismo}. Por eso anticipa la crucifixión a través de la institución de la Eucaristía. En ella, Cristo se ofrece a sí mismo como don a los Apóstoles en el Cenáculo; luego, diciéndoles: \textquote{Haced esto en memoria de mí} (\textit{Lc} 22, 19), les da la misión de a \textit{entregarlo a los demás hasta el fin del mundo}. 

Cristo, que vive totalmente para el Padre, desea que también nosotros vivamos por él; por eso se ofrece a nosotros bajo las apariencias de pan y vino. El pan es el alimento diario del hombre, sin el cual es difícil vivir; el vino es la bebida beneficiosa para la salud del organismo. 

Se da a sí mismo –su Cuerpo y su Sangre– como un don hasta el fin del mundo, porque \textit{esta es la lógica de su amor}: \textquote{nos amó hasta el extremo} (cf. \textit{Jn} 13, 1). 

3. La esencia de su ministerio es precisamente esta: es el ministerio de salvación que aún hoy ejerce y que llevará a cabo hasta el fin de los tiempos, a través de la Iglesia. Por eso \textit{es necesario que la Iglesia, Esposa de Cristo, lleve a cabo fielmente el ministerio que le ha sido encomendado, realizando el misterio de la redención y la Eucaristía}. En la realización de este \textquote{servicio}, Cristo se escondió bajo las especies del pan y del vino y, en tan misteriosas apariencias, alimenta y guía a su pueblo a través de los siglos. Él es el único sacerdote, rey y profeta, y nos hacemos partícipes de él a través de los sacramentos. 

4. La liturgia de la Última Cena destaca el vínculo misterioso que existe entre la liberación de Israel de la esclavitud de Egipto y la institución de la Eucaristía. Este segundo tema encontrará plena expresión durante la Vigilia Pascual, cuando se conmemorará el sacramento del Bautismo. Hoy encuentra su expresión en relación con la Eucaristía. 

Aquí está el anuncio: \textit{Cristo es el Cordero pascual}, que libera a su pueblo de la esclavitud por medio de la sangre derramada en la Cruz. 

En la noche del éxodo de Egipto, la sangre del cordero en las jambas de las puertas de las casas en las que vivían los hijos de Israel era la señal de su salvación. Se puede decir que fue precisamente esta sangre la que sacó a los israelitas de la condición de esclavitud y les mostró el camino a la Tierra Prometida. Durante la Última Cena, Jesús dice: \textquote{Este cáliz es la Nueva Alianza en mi sangre; haced esto cada vez que lo bebáis en memoria de mí} (\textit{1 Co} 11, 25). El salmista pregunta: \textquote{¿Cómo pagaré al Señor, todo el bien que me ha hecho?} (\textit{Sal} 115 [116], 12). Y nosotros, con toda la Iglesia, nos hacemos la misma pregunta esta tarde de Jueves Santo: \textquote{¿Cómo pagaré al Señor?}.
\end{body}

\begin{patercite}
En la oración, iniciada según las formas rituales de la tradición bíblica, Jesús muestra una vez más su identidad y la decisión de cumplir hasta el fondo su misión de amor total, de entrega en obediencia a la voluntad del Padre. La profunda originalidad de la donación de sí a los suyos, a través del memorial eucarístico, es la cumbre de la oración que caracteriza la cena de despedida con los suyos. Contemplando los gestos y las palabras de Jesús de aquella noche, vemos claramente que la relación íntima y constante con el Padre es el ámbito donde él realiza el gesto de dejar a los suyos, y a cada uno de nosotros, el Sacramento del amor, el \textquote{\emph{Sacramentum caritatis}}. Por dos veces en el cenáculo resuenan las palabras: \textquote{Haced esto en memoria mía} (\emph{1 Co} 11, 24.25). Él celebra su Pascua con la donación de sí, convirtiéndose en el verdadero Cordero que lleva a cumplimiento todo el culto antiguo. Por ello, san Pablo, hablando a los cristianos de Corinto, afirma: \textquote{Cristo, nuestra Pascua [nuestro Cordero pascual], ha sido inmolado. Así pues, celebremos\ldots con los panes ácimos de la sinceridad y la verdad} (\emph{1 Co} 5, 7-8).

\textbf{Benedicto XVI}, papa, \textit{Catequesis}, Audiencia general, 11 de enero de 2012, parr. 8.
\end{patercite}

\label{b-05-01-1994H}
\newpage
\subsubsection{Homilía (1997): Memoria viva de su amor}

\src{Basílica de San Juan de Letrán. 27 de marzo de 1997.}

\begin{body}
\ltr[1. ]{C}{ada} año esta Basílica de san Juan de Letrán acoge a la asamblea reunida para el solemne Memorial de la Última Cena. Acuden fieles de Roma y de todo el mundo para renovar el recuerdo de aquel acontecimiento que se realizó un jueves de hace muchos años en el Cenáculo, y que la liturgia conmemora como siempre actual en el día de hoy. Lo prolonga como Sacramento del Altar, Sacramento del Cuerpo y de la Sangre de Cristo. Lo prolonga como Eucaristía.

Estamos convocados para repetir ante todo el gesto que Cristo hizo al comienzo de la Última Cena, esto es, el lavatorio de los pies. El Evangelio de Juan presenta a nuestra consideración la resistencia de Pedro ante la humillación del Maestro y la enseñanza con la que Jesús ha comentado su propio gesto: \textquote{Vosotros me llamáis \textquote{el Maestro} y \textquote{el Señor}, y decís bien, porque lo soy. Pues si yo, el Maestro y el Señor, os he lavado los pies, también vosotros debéis lavaros los pies unos a otros: os he dado ejemplo para que lo que yo he hecho con vosotros, vosotros también lo hagáis} (\textit{Jn} 13, 13-15).

\textit{En la hora del banquete eucarístico, Cristo afirma la necesidad del servicio}. \textquote{El Hijo del hombre no ha venido para que le sirvan, sino para servir y dar su vida en rescate por todos} (\textit{Mc} 10, 45). Estamos, pues, convocados para expresar de nuevo la memoria viva del mayor de los mandamientos, el mandamiento del amor: \textquote{Nadie tiene amor más grande que el que da la vida por sus amigos} (\textit{Jn} 15, 13). El gesto de Cristo lo representa en vivo ante la mirada de los Apóstoles: \textquote{Había llegado la hora de pasar de este mundo al Padre}; la hora del sumo amor: \textquote{Habiendo amado a los suyos que estaban en el mundo, los amó hasta el extremo} (\textit{Jn} 13, 1).

2. Todo esto culmina en la Última Cena, en el Cenáculo de Jerusalén. Estamos convocados \textit{para revivir este acontecimiento, la institución del Sacramento admirable, del que la Iglesia vive incesantemente}, del Sacramento que constituye la Iglesia en su realidad más auténtica y profunda. No hay Eucaristía sin Iglesia, pero, antes aún, no hay Iglesia sin Eucaristía.

Eucaristía quiere decir \textit{acción de gracias}. Por esto hemos rezado con el \textbf{salmo responsorial}: \textquote{¿Cómo pagaré al Señor todo el bien que me ha hecho?} (\textit{Sal} 115, 12). Presentamos sobre el altar las ofrendas del pan y del vino, como incesante acción de gracias por todos los bienes que recibimos de Dios, por los bienes de la creación y de la redención. La redención se ha realizado mediante el Sacrificio de Cristo. La Iglesia, que anuncia la redención y vive de la redención, ha de continuar haciendo presente sacramentalmente este Sacrificio, del cual debe sacar fuerza para ser ella misma.

3. La celebración eucarística \textit{in Coena Domini} nos lo recuerda con singular elocuencia. La \textbf{primera lectura}, del libro del \textbf{Éxodo}, evoca el momento de la historia del pueblo de la Antigua Alianza en el que con más fuerza ha estado prefigurado el misterio de la Eucaristía: se trata de la institución de la Pascua. El pueblo debía ser liberado de la esclavitud de Egipto, debía dejar la tierra de esclavitud y el precio de este rescate era la sangre del cordero.

Aquel cordero de la Antigua Alianza ha encontrado plenitud de significado en la Nueva Alianza. Esto se ha realizado también mediante el ministerio profético de Juan Bautista, quien, al ver a Jesús de Nazaret que venía al río Jordán para recibir el bautismo, exclamó: \textquote{Este es el Cordero de Dios, que quita el pecado del mundo} (\textit{Jn} 1, 29). No es casual que estas palabras se hayan colocado en el centro de la liturgia eucarística. Nos lo recuerdan las lecturas de la santa Misa de la Cena del Señor para indicar que \textit{con este vivo Memorial entramos en la hora de la Pasión de Cristo}. Precisamente en esta hora será desvelado \textit{el misterio del Cordero de Dios}. Las palabras pronunciadas por el Bautista junto al Jordán se cumplirán así claramente. Cristo será crucificado. Como Hijo de Dios aceptará la muerte para liberar al mundo del pecado.

Abramos nuestros corazones, participemos con fe en este gran misterio y aclamemos junto con toda la Iglesia convocada en asamblea eucarística: \textquote{Anunciamos tu muerte, proclamamos tu resurrección. ¡Ven, Señor Jesús!}
\end{body}

\label{b-05-01-1997H}

\begin{patercite}
El evangelista san Lucas ha conservado otro elemento valioso de los acontecimientos de la última Cena, que nos permite ver la profundidad conmovedora de la oración de Jesús por los suyos en aquella noche: la atención por cada uno. Partiendo de la oración de acción de gracias y de bendición, Jesús llega al don eucarístico, al don de sí mismo, y, mientras dona la realidad sacramental decisiva, se dirige a Pedro. Ya para terminar la cena, le dice: \textquote{Simón, Simón, mira que Satanás os ha reclamado para cribaros como trigo. Pero yo he pedido por ti, para que tu fe no se apague. Y tú, cuando te hayas convertido, confirma a tus hermanos} (\emph{Lc} 22, 31-32). La oración de Jesús, cuando se acerca la prueba también para sus discípulos, sostiene su debilidad, su dificultad para comprender que el camino de Dios pasa a través del Misterio pascual de muerte y resurrección, anticipado en el ofrecimiento del pan y del vino. La Eucaristía es alimento de los peregrinos que se convierte en fuerza incluso para quien está cansado, extenuado y desorientado. Y la oración es especialmente por Pedro, para que, una vez convertido, confirme a sus hermanos en la fe. El evangelista san Lucas recuerda que fue precisamente la mirada de Jesús la que buscó el rostro de Pedro en el momento en que acababa de realizar su triple negación, para darle la fuerza de retomar el camino detrás de él: \textquote{Y enseguida, estando todavía él hablando, cantó un gallo. El Señor, volviéndose, le echó una mirada a Pedro, y Pedro se acordó de la palabra que el Señor le había dicho} (\emph{Lc} 22, 60-61).

\textbf{Benedicto XVI}, papa, \textit{Catequesis}, Audiencia general, 11 de enero de 2012, parr. 9.
\end{patercite}

\newpage
\subsubsection{Homilía (2000): Testigos del amor del Crucificado}

\src{20 de abril del 2000.}

\begin{body}
\ltr[1. «]{C}{on} ansia he deseado comer esta Pascua con vosotros, antes de padecer» (\textit{Lc} 22, 15). Cristo da a conocer, con estas palabras, el significado profético de la cena pascual, que está a punto de celebrar con los discípulos en el Cenáculo de Jerusalén.

Con la \textbf{primera lectura}, tomada del \textbf{libro del Éxodo}, la liturgia ha puesto de relieve cómo la Pascua de Jesús se inscribe en el contexto de la Pascua de la antigua Alianza. Con ella, los israelitas conmemoraban la cena consumada por sus padres en el momento del éxodo de Egipto, de la liberación de la esclavitud. El texto sagrado prescribía que se untara con un poco de sangre del cordero las dos jambas y el dintel de las casas. Y añadía cómo había que comer el cordero: \textquote{Ceñidas vuestras cinturas, calzados vuestros pies, y el bastón en vuestra mano; (\ldots) de prisa. (\ldots) Yo pasaré esa noche por la tierra de Egipto y heriré a todos los primogénitos. (\ldots) La sangre será vuestra señal en las casas donde moráis. Cuando yo vea la sangre pasaré de largo ante vosotros, y no habrá entre vosotros plaga exterminadora} (\textit{Ex} 12, 11-13).

Con la sangre del cordero los hijos e hijas de Israel obtienen la liberación de la esclavitud de Egipto, bajo la guía de Moisés. El recuerdo de un acontecimiento tan extraordinario se convirtió en una ocasión de fiesta para el pueblo, agradecido al Señor por la libertad recuperada, don divino y compromiso humano siempre actual. \textquote{Este será un día memorable para vosotros, y lo celebraréis como fiesta en honor del Señor} (\textit{Ex} 12, 14). ¡Es la Pascua del Señor! ¡La Pascua de la antigua Alianza!

2. \textquote{Con ansia he deseado comer esta Pascua con vosotros, antes de padecer} (\textit{Lc} 22, 15). En el Cenáculo, Cristo, cumpliendo las prescripciones de la antigua Alianza, celebra la cena pascual con los Apóstoles, pero da a este rito un contenido nuevo. Hemos escuchado lo que dice de él \textbf{san Pablo} en la \textbf{segunda lectura}, tomada de la primera carta a los Corintios. En este texto, que se suele considerar como la más antigua descripción de la cena del Señor, se recuerda que Jesús, \textquote{la noche en que iban a entregarle, tomó pan y, pronunciando la acción de gracias, lo partió y dijo: \textquote{Esto es mi cuerpo, que se entrega por vosotros. Haced esto en memoria mía}. Lo mismo hizo con el cáliz, después de cenar, diciendo: \textquote{Este cáliz es la nueva Alianza sellada con mi sangre; haced esto cada vez que bebáis, en memoria mía}. Por eso, cada que vez que coméis de este pan y bebéis del cáliz, proclamáis la muerte del Señor, hasta que vuelva} (\textit{1 Co} 11, 23-26).

Con estas palabras solemnes se entrega, para todos los siglos, la memoria de la institución de la Eucaristía. Cada año, en este día, las recordamos volviendo espiritualmente al Cenáculo. 

\txtsmall{[Esta tarde las revivo con emoción particular, porque conservo en mis ojos y en mi corazón las imágenes del Cenáculo, donde tuve la alegría de celebrar la Eucaristía, con ocasión de mi reciente peregrinación jubilar a Tierra Santa. La emoción es más fuerte aún porque este es el año del jubileo bimilenario de la Encarnación. Desde esta perspectiva, la celebración que estamos viviendo adquiere una profundidad especial (\ldots)]} [\ldots]

En el Cenáculo Jesús infundió un nuevo contenido a las antiguas tradiciones y anticipó los acontecimientos del día siguiente, cuando su cuerpo, cuerpo inmaculado del Cordero de Dios, sería inmolado y su sangre sería derramada para la redención del mundo. La Encarnación se había realizado precisamente con vistas a este acontecimiento: ¡la Pascua de Cristo, la Pascua de la nueva Alianza!

3. \textquote{Cada vez que coméis de este pan y bebéis del cáliz, proclamáis la muerte del Señor, hasta que vuelva} (\textit{1 Co} 11, 26). El \textbf{Apóstol} nos exhorta a hacer constantemente memoria de este misterio. Al mismo tiempo, nos invita a vivir diariamente nuestra misión de testigos y heraldos del amor del Crucificado, en espera de su vuelta gloriosa.

Pero ¿cómo hacer memoria de este acontecimiento salvífico? ¿Cómo vivir en espera de que Cristo vuelva? Antes de instituir el sacramento de su Cuerpo y su Sangre, Cristo, inclinado y arrodillado, como un esclavo, lava en el Cenáculo los pies a sus discípulos. Lo vemos de nuevo mientras realiza este gesto, que en la cultura judía es propio de los siervos y de las personas más humildes de la familia. Pedro, al inicio, se opone, pero el Maestro lo convence, y al final también él se deja lavar los pies, como los demás discípulos. Pero, inmediatamente después, vestido y sentado nuevamente a la mesa, Jesús explica el sentido de su gesto: \textquote{Vosotros me llamáis \textquote{el Maestro} y \textquote{el Señor}, y decís bien, porque lo soy. Pues si yo, el Maestro y el Señor, os he lavado los pies, también vosotros debéis lavaros los pies unos a otros} (\textit{Jn} 13, 12-14). Estas palabras, que unen el misterio eucarístico al servicio del amor, pueden considerarse propedéuticas de la institución del sacerdocio ministerial.

Con la institución de la Eucaristía, Jesús comunica a los Apóstoles la participación ministerial en su sacerdocio, el sacerdocio de la Alianza nueva y eterna, en virtud de la cual él, y sólo él, es siempre y por doquier artífice y ministro de la Eucaristía. Los Apóstoles, a su vez, se convierten en ministros de este excelso misterio de la fe, destinado a perpetuarse hasta el fin del mundo. Se convierten, al mismo tiempo, en servidores de todos los que van a participar de este don y misterio tan grandes. La Eucaristía, el supremo sacramento de la Iglesia, está unida al sacerdocio ministerial, que nació también en el Cenáculo, como don del gran amor de Jesús, que \textquote{sabiendo que había llegado la hora de pasar de este mundo al Padre, habiendo amado a los suyos que estaban en el mundo, los amó hasta el extremo} (\textit{Jn} 13, 1). La Eucaristía, el sacerdocio y el mandamiento nuevo del amor. ¡Este es el memorial vivo que contemplamos en el Jueves santo! \textquote{Haced esto en memoria mía}: ¡esta es la Pascua de la Iglesia, nuestra Pascua!
\end{body}


\subsubsection{Homilía (2003): Dos manifestaciones, un mismo misterio}

\src{17 de abril del 2003.}

\textquote{\textit{Los amó hasta el extremo}} (\textit{Jn} 13, 1).

\begin{body}
\ltr[1. ]{E}{n} la víspera de su pasión y muerte, el Señor Jesús quiso reunir en torno a sí, una vez más, a sus Apóstoles para dejarles las últimas consignas y darles el testimonio supremo de su amor. Entremos también nosotros en la \textquote{sala grande en el piso de arriba, arreglada con divanes} (\textit{Mc} 14, 15) y dispongámonos a escuchar los pensamientos más íntimos que quiere comunicarnos; dispongámonos, en particular, a acoger el \textit{gesto} y el \textit{don} que ha preparado para esta última cita.

2. Mientras están cenando, Jesús se levanta de la mesa y comienza a \textit{lavar los pies a los discípulos}. Pedro, al principio, se resiste; luego, comprende y acepta. También a nosotros se nos invita a comprender: \textit{lo primero} que el discípulo debe hacer es ponerse a la escucha de su Señor, abriendo el corazón para \textit{acoger la iniciativa de su amor}. Sólo después será invitado a reproducir a su vez lo que ha hecho el Maestro. También él deberá \textquote{lavar los pies} a sus hermanos, traduciendo en gestos de servicio mutuo ese amor, que constituye la síntesis de todo el Evangelio (cf. \textit{Jn} 13, 1-20).

También durante la Cena, sabiendo que ya había llegado su \textquote{hora}, Jesús bendice y \textit{parte el pan}, luego lo distribuye a los Apóstoles, diciendo: \textquote{Esto es mi cuerpo}; lo mismo hace con \textit{el cáliz}: \textquote{Esta es mi sangre}. Y les manda: \textquote{Haced esto en conmemoración mía} (\textit{1 Co} 11, 24-25). Realmente aquí se manifiesta el testimonio de un amor llevado \textquote{hasta el extremo} (\textit{Jn} 13, 1). Jesús se da como alimento a los discípulos para llegar a ser uno con ellos. Una vez más se pone de relieve la \textquote{lección} que debemos aprender: \textit{lo primero} que hemos de hacer es \textit{abrir el corazón a la acogida del amor de Cristo}. La iniciativa es suya: su amor es lo que nos hace capaces de amar también nosotros a nuestros hermanos.

Así pues, el lavatorio de los pies y el sacramento de la Eucaristía son \textit{dos manifestaciones de un mismo misterio de amor} confiado a los discípulos \textquote{para que –dice Jesús– lo que yo he hecho con vosotros, vosotros también lo hagáis} (\textit{Jn} 13, 15).

3. \textquote{Haced esto en conmemoración mía} (\textit{1 Co} 11, 24). La \textquote{memoria} que el Señor nos dejó aquella noche se refiere al \textit{momento culminante de su existencia terrena}, es decir, el momento de su ofrenda sacrificial al Padre por amor a la humanidad. Y es una \textquote{memoria} que se sitúa en el marco de una cena, la cena pascual, en la que Jesús se da a sus Apóstoles bajo las especies del pan y del vino, como su alimento en el camino hacia la patria del cielo.


\newpage 
\textit{Mysterium fidei!} Así proclama el celebrante después de pronunciar las palabras de la consagración. Y la asamblea litúrgica responde expresando con alegría su fe y su adhesión, llena de esperanza. ¡Misterio realmente grande es la Eucaristía! Misterio \textquote{incomprensible} para la razón humana, pero sumamente luminoso para los ojos de la fe. La mesa del Señor en la sencillez de los símbolos eucarísticos –el pan y el vino compartidos– es también \textit{la mesa de la fraternidad concreta}. El mensaje que brota de ella es demasiado claro como para ignorarlo: todos los que participan en la celebración eucarística \textit{no pueden quedar insensibles} ante las expectativas de los pobres y los necesitados.

4. \txtsmall{[Precisamente desde esta perspectiva deseo que \textit{los donativos} que se recojan durante esta celebración sirvan para aliviar \textit{las urgentes necesidades de los que sufren en Irak} por las consecuencias de la guerra.]} Un corazón que ha experimentado el amor del Señor se abre espontáneamente a la caridad hacia sus hermanos.

\textquote{\textit{O sacrum convivium, in quo Christus sumitur}}.

Hoy estamos todos invitados a celebrar y adorar, hasta muy entrada la noche, al Señor que se hizo alimento para nosotros, peregrinos en el tiempo, dándonos su carne y su sangre.

La Eucaristía \textit{es un gran don para la Iglesia y para el mundo}. {[Precisamente para que se preste una atención cada vez más profunda al sacramento de la Eucaristía, he querido entregar a toda la comunidad de los creyentes \textit{una encíclica}, cuyo tema central es el misterio eucarístico: \textit{Ecclesia de Eucharistia}. Dentro de poco tendré la alegría de firmarla durante esta celebración, que evoca la última Cena, cuando Jesús nos dejó a sí mismo como supremo testamento de amor. La encomiendo desde ahora, en primer lugar, a los sacerdotes, para que ellos, a su vez, la difundan para bien de todo el pueblo cristiano.]}

5. \textit{Adoro te devote, latens Deitas!} Te adoramos, oh admirable sacramento de la presencia de Aquel que amó a los suyos \textquote{hasta el extremo}. Te damos gracias, Señor, que en la Eucaristía edificas, congregas y vivificas a la Iglesia.

¡Oh divina Eucaristía, llama del amor de Cristo, que ardes en el altar del mundo, haz que la Iglesia, confortada por ti, sea cada vez más solícita para enjugar las lágrimas de los que sufren y sostener los esfuerzos de los que anhelan la justicia y la paz!

Y tú, María, mujer \textquote{eucarística}, que ofreciste tu seno virginal para la encarnación del Verbo de Dios, \textit{ayúdanos a vivir el misterio eucarístico con el espíritu del Magníficat}. Que nuestra vida sea una alabanza sin fin al Todopoderoso, que se ocultó bajo la humildad de los signos eucarísticos.

\textit{Adoro te devote, latens Deitas\ldots}

\textit{Adoro te\ldots, adiuva me!}
\end{body}

\newsection
\subsection{Benedicto XVI, papa}

\subsubsection{Homilía (2006): El gesto de Jesús y su significado}

\src{13 de abril del 2006.}

\begin{body}
\ltr[«]{H}{abiendo} amado a los suyos que estaban en el mundo, los amó hasta el extremo» (\textit{Jn} 13, 1). Dios ama a su criatura, el hombre; lo ama también en su caída y no lo abandona a sí mismo. Él ama hasta el fin. Lleva su amor hasta el final, hasta el extremo: baja de su gloria divina. Se desprende de las vestiduras de su gloria divina y se viste con ropa de esclavo. Baja hasta la extrema miseria de nuestra caída. Se arrodilla ante nosotros y desempeña el servicio del esclavo; lava nuestros pies sucios, para que podamos ser admitidos a la mesa de Dios, para hacernos dignos de sentarnos a su mesa, algo que por nosotros mismos no podríamos ni deberíamos hacer jamás. Dios no es un Dios lejano, demasiado distante y demasiado grande como para ocuparse de nuestras bagatelas. Dado que es grande, puede interesarse también de las cosas pequeñas. Dado que es grande, el alma del hombre, el hombre mismo, creado por el amor eterno, no es algo pequeño, sino que es grande y digno de su amor. La santidad de Dios no es sólo un poder incandescente, ante el cual debemos alejarnos aterrorizados; es poder de amor y, por esto, es poder purificador y sanador.

Dios desciende y se hace esclavo; nos lava los pies para que podamos sentarnos a su mesa. Así se revela todo el misterio de Jesucristo. Así resulta manifiesto lo que significa redención. El baño con que nos lava es su amor dispuesto a afrontar la muerte. Sólo el amor tiene la fuerza purificadora que nos limpia de nuestra impureza y nos eleva a la altura de Dios. El baño que nos purifica es él mismo, que se entrega totalmente a nosotros, desde lo más profundo de su sufrimiento y de su muerte. Él es continuamente este amor que nos lava. En los sacramentos de la purificación –el Bautismo y la Penitencia– él está continuamente arrodillado ante nuestros pies y nos presta el servicio de esclavo, el servicio de la purificación; nos hace capaces de Dios. Su amor es inagotable; llega realmente hasta el extremo.

\textquote{Vosotros estáis limpios, pero no todos}, dice el Señor (\textit{Jn} 13, 10). En esta frase se revela el gran don de la purificación que él nos hace, porque desea estar a la mesa juntamente con nosotros, de convertirse en nuestro alimento. \textquote{Pero no todos}: existe el misterio oscuro del rechazo, que con la historia de Judas se hace presente y debe hacernos reflexionar precisamente en el Jueves santo, el día en que Jesús nos hace el don de sí mismo. El amor del Señor no tiene límites, pero el hombre puede ponerle un límite.

\textquote{Vosotros estáis limpios, pero no todos}: ¿Qué es lo que hace impuro al hombre? Es el rechazo del amor, el no querer ser amado, el no amar. Es la soberbia que cree que no necesita purificación, que se cierra a la bondad salvadora de Dios. Es la soberbia que no quiere confesar y reconocer que necesitamos purificación.

En Judas vemos con mayor claridad aún la naturaleza de este rechazo. Juzga a Jesús según las categorías del poder y del éxito: para él sólo cuentan el poder y el éxito; el amor no cuenta. Y es avaro: para él el dinero es más importante que la comunión con Jesús, más importante que Dios y su amor. Así se transforma también en un mentiroso, que hace doble juego y rompe con la verdad; uno que vive en la mentira y así pierde el sentido de la verdad suprema, de Dios. De este modo se endurece, se hace incapaz de conversión, del confiado retorno del hijo pródigo, y arruina su vida.

\textquote{Vosotros estáis limpios, pero no todos}. El Señor hoy nos pone en guardia frente a la autosuficiencia, que pone un límite a su amor ilimitado. Nos invita a imitar su humildad, a tratar de vivirla, a dejarnos \textquote{contagiar} por ella. Nos invita –por más perdidos que podamos sentirnos– a volver a casa y a permitir a su bondad purificadora que nos levante y nos haga entrar en la comunión de la mesa con él, con Dios mismo.

Reflexionemos sobre otra frase de este inagotable pasaje evangélico: \textquote{Os he dado ejemplo\ldots} (\textit{Jn} 13, 15); \textquote{También vosotros debéis lavaros los pies unos a otros} (\textit{Jn} 13, 14). ¿En qué consiste el \textquote{lavarnos los pies unos a otros}? ¿Qué significa en concreto? Cada obra buena hecha en favor del prójimo, especialmente en favor de los que sufren y los que son poco apreciados, es un servicio como lavar los pies. El Señor nos invita a bajar, a aprender la humildad y la valentía de la bondad; y también a estar dispuestos a aceptar el rechazo, actuando a pesar de ello con bondad y perseverando en ella.

Pero hay una dimensión aún más profunda. El Señor limpia nuestra impureza con la fuerza purificadora de su bondad. Lavarnos los pies unos a otros significa sobre todo perdonarnos continuamente unos a otros, volver a comenzar juntos siempre de nuevo, aunque pueda parecer inútil. Significa purificarnos unos a otros soportándonos mutuamente y aceptando ser soportados por los demás; purificarnos unos a otros dándonos recíprocamente la fuerza santificante de la palabra de Dios e introduciéndonos en el Sacramento del amor divino.

El Señor nos purifica; por esto nos atrevemos a acercarnos a su mesa. Pidámosle que nos conceda a todos la gracia de poder ser un día, para siempre, huéspedes del banquete nupcial eterno. Amén.
\end{body}


\subsubsection{Homilía (2009): Entender lo que ocurrió en aquella Cena}

\src{9 de abril del 2009.}

\begin{body}
\textit{Qui, pridie quam pro nostra omniumque salute pateretur, hoc est hodie, accepit panem}. 

\ltr{A}{sí} diremos hoy en el Canon de la Santa Misa. \textquote{\textit{Hoc est hodie}}. La Liturgia del Jueves Santo incluye la palabra \textquote{hoy} en el texto de la plegaria, subrayando con ello la dignidad particular de este día. Ha sido \textquote{hoy} cuando Él lo ha hecho: se nos ha entregado para siempre en el Sacramento de su Cuerpo y de su Sangre. Este \textquote{hoy} es sobre todo el memorial de la Pascua de entonces. Pero es más aún. Con el Canon entramos en este \textquote{hoy}. Nuestro hoy se encuentra con su hoy. Él hace esto ahora. Con la palabra \textquote{hoy}, la Liturgia de la Iglesia quiere inducirnos a que prestemos gran atención interior al misterio de este día, a las palabras con que se expresa. Tratemos, pues, de escuchar de modo nuevo el relato de la institución, tal y como la Iglesia lo ha formulado basándose en la Escritura y contemplando al Señor mismo.

Lo primero que nos sorprende es que el relato de la institución no es una frase suelta, sino que empieza con un pronombre relativo: \textit{qui pridie}. Este \textquote{\textit{qui}} enlaza todo el relato con la palabra precedente de la oración, \textquote{\ldots de manera que sea para nosotros Cuerpo y Sangre de tu Hijo amado, Jesucristo, nuestro Señor}. De este modo, el relato está unido a la oración anterior, a todo el Canon, y se hace él mismo oración. En efecto, en modo alguno se trata de un relato sencillamente insertado aquí; tampoco se trata de palabras aisladas de autoridad, que quizás interrumpirían la oración. Es oración. Y solamente en la oración se cumple el acto sacerdotal de la consagración que se convierte en transformación, transustanciación de nuestros dones de pan y vino en el Cuerpo y la Sangre de Cristo. Rezando en este momento central, la Iglesia concuerda totalmente con el acontecimiento del Cenáculo, ya que el actuar de Jesús se describe con las palabras:\textquote{\textit{gratias agens benedixit}} – \textquote{te dio gracias con la plegaria de bendición}. Con esta expresión, la Liturgia romana ha dividido en dos palabras, lo que en hebreo es una sola, \textit{berakha}, que en griego, en cambio, aparece en los dos términos de \textit{eucharistía} y \textit{eulogía}. El Señor agradece. Al agradecer, reconocemos que una cosa determinada es un don de otro. El Señor agradece, y de este modo restituye a Dios el pan, \textquote{fruto de la tierra y del trabajo del hombre}, para poder recibirlo nuevamente de Él. Agradecer se transforma en bendecir. Lo que ha sido puesto en las manos de Dios, vuelve de Él bendecido y transformado. Por tanto, la Liturgia romana tiene razón al interpretar nuestro orar en este momento sagrado con las palabras: \textquote{ofrecemos}, \textquote{pedimos}, \textquote{acepta}, \textquote{bendice esta ofrenda}. Todo esto se oculta en la palabra \textit{eucharistia.}

Hay otra particularidad en el relato de la institución del Canon Romano que queremos meditar en esta hora. La Iglesia orante se fija en las manos y los ojos del Señor. Quiere casi observarlo, desea percibir el gesto de su orar y actuar en aquella hora singular, encontrar la figura de Jesús, por decirlo así, también a través de los sentidos. \textquote{Tomó pan en sus santas y venerables manos}. Nos fijamos en las manos con las que Él ha curado a los hombres; en las manos con las que ha bendecido a los niños; en las manos que ha impuesto sobre los hombres; en las manos clavadas en la Cruz y que llevarán siempre los estigmas como signos de su amor dispuesto a morir. Ahora tenemos el encargo de hacer lo que Él ha hecho: tomar en las manos el pan para que sea convertido mediante la plegaria eucarística. En la Ordenación sacerdotal, nuestras manos fueron ungidas, para que fuesen manos de bendición. Pidamos al Señor ahora que nuestras manos sirvan cada vez más para llevar la salvación, para llevar la bendición, para hacer presente su bondad.

De la introducción a la Oración sacerdotal de Jesús (cf. \textit{Jn} 17, 1), el Canon usa luego las palabras: \textquote{elevando los ojos al cielo, hacia ti, Dios, Padre suyo todopoderoso}. El Señor nos enseña a levantar los ojos y sobre todo el corazón. A levantar la mirada, apartándola de las cosas del mundo, a orientarnos hacia Dios en la oración y así elevar nuestro ánimo. En un himno de la Liturgia de las Horas pedimos al Señor que custodie nuestros ojos, para que no acojan ni dejen que en nosotros entren las \textquote{\textit{vanitates}}, las vanidades, la banalidad, lo que sólo es apariencia. Pidamos que a través de los ojos no entre el mal en nosotros, falsificando y ensuciando así nuestro ser. Pero queremos pedir sobre todo que tengamos ojos que vean todo lo que es verdadero, luminoso y bueno, para que seamos capaces de ver la presencia de Dios en el mundo. Pidamos, para que miremos el mundo con ojos de amor, con los ojos de Jesús, reconociendo así a los hermanos y las hermanas que nos necesitan, que están esperando nuestra palabra y nuestra acción.

Después de bendecir, el Señor parte el pan y lo da a los discípulos. Partir el pan es el gesto del padre de familia que se preocupa de los suyos y les da lo que necesitan para la vida. Pero es también el gesto de la hospitalidad con que se acoge al extranjero, al huésped, y se le permite participar en la propia vida. Dividir, com-partir, es unir. A través del compartir se crea comunión. En el pan partido, el Señor se reparte a sí mismo. El gesto del partir alude misteriosamente también a su muerte, al amor hasta la muerte. Él se da a sí mismo, que es el verdadero \textquote{pan para la vida del mundo} (cf. \textit{Jn} 6, 51). El alimento que el hombre necesita en lo más hondo es la comunión con Dios mismo. Al agradecer y bendecir, Jesús transforma el pan, y ya no es pan terrenal lo que da, sino la comunión consigo mismo. Esta transformación, sin embargo, quiere ser el comienzo de la transformación del mundo. Para que llegue a ser un mundo de resurrección, un mundo de Dios. Sí, se trata de transformación. Del hombre nuevo y del mundo nuevo que comienzan en el pan consagrado, transformado, transustanciado.

Hemos dicho que partir el pan es un gesto de comunión, de unir mediante el compartir. Así, en el gesto mismo se alude ya a la naturaleza íntima de la Eucaristía: ésta es \textit{agape}, es amor hecho corpóreo. En la palabra \textquote{\textit{agape}}, se compenetran los significados de Eucaristía y amor. En el gesto de Jesús que parte el pan, el amor que se comparte ha alcanzado su extrema radicalidad: Jesús se deja partir como pan vivo. En el pan distribuido reconocemos el misterio del grano de trigo que muere y así da fruto. Reconocemos la nueva multiplicación de los panes, que deriva del morir del grano de trigo y continuará hasta el fin del mundo. Al mismo tiempo vemos que la Eucaristía nunca puede ser sólo una acción litúrgica. Sólo es completa, si el \textit{agape} litúrgico se convierte en amor cotidiano. En el culto cristiano, las dos cosas se transforman en una, el ser agraciados por el Señor en el acto cultual y el cultivo del amor respecto al prójimo. Pidamos en esta hora al Señor la gracia de aprender a vivir cada vez mejor el misterio de la Eucaristía, de manera que comience así la transformación del mundo.

Después del pan, Jesús toma el cáliz de vino. El Canon Romano designa el cáliz que el Señor da a los discípulos, como \textquote{\textit{praeclarus calix}}, cáliz glorioso, aludiendo con ello al Salmo 23 [22], el Salmo que habla de Dios como del Pastor poderoso y bueno. En él se lee: \textquote{preparas una mesa ante mí, enfrente de mis enemigos; \ldots y mi copa rebosa} (v. 5), \textit{calix praeclarus}. El Canon Romano interpreta esta palabra del Salmo como una profecía que se cumple en la Eucaristía. Sí, el Señor nos prepara la mesa en medio de las amenazas de este mundo, y nos da el cáliz glorioso, el cáliz de la gran alegría, de la fiesta verdadera que todos anhelamos, el cáliz rebosante del vino de su amor. El cáliz significa la boda: ahora ha llegado \textquote{la hora} a la que en las bodas de Caná se aludía de forma misteriosa. Sí, la Eucaristía es más que un banquete, es una fiesta de boda. Y esta boda se funda en la autodonación de Dios hasta la muerte. En las palabras de la última Cena de Jesús y en el Canon de la Iglesia, el misterio solemne de la boda se esconde bajo la expresión \textquote{\textit{novum Testamentum}}. Este cáliz es el nuevo Testamento, \textquote{la nueva Alianza sellada con mi sangre}, según la palabra de Jesús sobre el cáliz, que Pablo transmite en la \textbf{segunda lectura} de hoy (cf. \textit{1 Co} 11, 25). El Canon Romano añade: \textquote{de la alianza nueva y eterna}, para expresar la indisolubilidad del vínculo nupcial de Dios con la humanidad. El motivo por el cual las traducciones antiguas de la Biblia no hablan de Alianza, sino de Testamento, es que no se trata de dos contrayentes iguales quienes la establecen, sino que entra en juego la infinita distancia entre Dios y el hombre. Lo que nosotros llamamos nueva y antigua Alianza no es un acuerdo entre dos partes iguales, sino un mero don de Dios, que nos deja como herencia su amor, a sí mismo. Y ciertamente, a través de este don de su amor Él, superando cualquier distancia, nos convierte verdaderamente en \textit{partner} y se realiza el misterio nupcial del amor.

Para poder comprender lo que allí ocurre en profundidad, hemos de escuchar más cuidadosamente aún las palabras de la Biblia y su sentido originario. Los estudiosos nos dicen que, en los tiempos remotos de que hablan las historias de los Patriarcas de Israel, \textquote{ratificar una alianza} significaba \textquote{entrar con otros en una unión fundada en la sangre, o bien acoger a alguien en la propia federación y entrar así en una comunión de derechos recíprocos}. De este modo se crea una consanguinidad real, aunque no material. Los aliados se convierten en cierto modo en \textquote{hermanos de la misma carne y la misma sangre}. La alianza realiza un conjunto que significa paz (cf. ThWNT II 105-137). ¿Podemos ahora hacernos al menos una idea de lo que ocurrió en la hora de la última Cena y que, desde entonces, se renueva cada vez que celebramos la Eucaristía? Dios, el Dios vivo establece con nosotros una comunión de paz, más aún, Él crea una \textquote{consanguinidad} entre Él y nosotros. Por la encarnación de Jesús, por su sangre derramada, hemos sido injertados en una consanguinidad muy real con Jesús y, por tanto, con Dios mismo. La sangre de Jesús es su amor, en el que la vida divina y la humana se han hecho una cosa sola. Pidamos al Señor que comprendamos cada vez más la grandeza de este misterio. Que Él despliegue su fuerza trasformadora en nuestro interior, de modo que lleguemos a ser realmente consanguíneos de Jesús, llenos de su paz y, así, también en comunión unos con otros.

Sin embargo, ahora surge aún otra pregunta. En el Cenáculo, Cristo entrega a los discípulos su Cuerpo y su Sangre, es decir, Él mismo en la totalidad de su persona. Pero, ¿puede hacerlo? Todavía está físicamente presente entre ellos, está ante ellos. La respuesta es que, en aquella hora, Jesús cumple lo que previamente había anunciado en el discurso sobre el Buen Pastor: \textquote{Nadie me quita la vida, sino que yo la entrego libremente. Tengo poder para entregarla y tengo poder para recuperarla} (cf. \textit{Jn} 10, 18). Nadie puede quitarle la vida: la da por libre decisión. En aquella hora anticipa la crucifixión y la resurrección. Lo que, por decirlo así, se cumplirá físicamente en Él, Él ya lo lleva a cabo anticipadamente en la libertad de su amor. Él entrega su vida y la recupera en la resurrección para poderla compartir para siempre.

Señor, Tú nos entregas hoy tu vida, Tú mismo te nos das. Llénanos de tu amor. Haznos vivir en tu \textquote{hoy}. Haznos instrumentos de tu paz. Amén.
\end{body}

\label{b-05-01-2009H}
\newpage
\subsubsection{Homilía (2012): Obediencia y libertad}

\src{Basílica de San Juan de Letrán. 5 de abril de 2012.}

\begin{body}
\ltr{E}{l} Jueves Santo no es sólo el día de la Institución de la Santa Eucaristía, cuyo esplendor ciertamente se irradia sobre todo lo demás y, por así decir, lo atrae dentro de sí. También forma parte del Jueves Santo la noche oscura del Monte de los Olivos, hacia la cual Jesús se dirige con sus discípulos; forma parte también la soledad y el abandono de Jesús que, orando, va al encuentro de la oscuridad de la muerte; forma parte de este Jueves Santo la traición de Judas y el arresto de Jesús, así como también la negación de Pedro, la acusación ante el Sanedrín y la entrega a los paganos, a Pilato. En esta hora, tratemos de comprender con más profundidad estos eventos, porque en ellos se lleva a cabo el misterio de nuestra Redención.

Jesús sale en la noche. La noche significa falta de comunicación, una situación en la que uno no ve al otro. Es un símbolo de la incomprensión, del ofuscamiento de la verdad. Es el espacio en el que el mal, que debe esconderse ante la luz, puede prosperar. Jesús mismo es la luz y la verdad, la comunicación, la pureza y la bondad. Él entra en la noche. La noche, en definitiva, es símbolo de la muerte, de la pérdida definitiva de comunión y de vida. Jesús entra en la noche para superarla e inaugurar el nuevo día de Dios en la historia de la humanidad.

Durante este camino, él ha cantado con sus Apóstoles los Salmos de la liberación y de la redención de Israel, que recuerdan la primera Pascua en Egipto, la noche de la liberación. Como él hacía con frecuencia, ahora se va a orar solo y hablar como Hijo con el Padre. Pero, a diferencia de lo acostumbrado, quiere cerciorarse de que estén cerca tres discípulos: Pedro, Santiago y Juan. Son los tres que habían tenido la experiencia de su Transfiguración –la manifestación luminosa de la gloria de Dios a través de su figura humana– y que lo habían visto en el centro, entre la Ley y los Profetas, entre Moisés y Elías. Habían escuchado cómo hablaba con ellos de su \textquote{éxodo} en Jerusalén. El éxodo de Jesús en Jerusalén, ¡qué palabra misteriosa!; el éxodo de Israel de Egipto había sido el episodio de la fuga y la liberación del pueblo de Dios. ¿Qué aspecto tendría el éxodo de Jesús, en el cual debía cumplirse definitivamente el sentido de aquel drama histórico?; ahora, los discípulos son testigos del primer tramo de este éxodo, de la extrema humillación que, sin embargo, era el paso esencial para salir hacia la libertad y la vida nueva, hacia la que tiende el éxodo. Los discípulos, cuya cercanía quiso Jesús en esta hora de extrema tribulación, como elemento de apoyo humano, pronto se durmieron. No obstante, escucharon algunos fragmentos de las palabras de la oración de Jesús y observaron su actitud. Ambas cosas se grabaron profundamente en sus almas, y ellos las transmitieron a los cristianos para siempre. Jesús llama a Dios \textquote{Abbá}. Y esto significa –como ellos añaden– \textquote{Padre}. Pero no de la manera en que se usa habitualmente la palabra \textquote{padre}, sino como expresión del lenguaje de los niños, una palabra afectuosa con la cual no se osaba dirigirse a Dios. Es el lenguaje de quien es verdaderamente \textquote{niño}, Hijo del Padre, de aquel que se encuentra en comunión con Dios, en la más profunda unidad con él.

Si nos preguntamos cuál es el elemento más característico de la imagen de Jesús en los evangelios, debemos decir: su relación con Dios. Él está siempre en comunión con Dios. El ser con el Padre es el núcleo de su personalidad. A través de Cristo, conocemos verdaderamente a Dios. \textquote{A Dios nadie lo ha visto jamás}, dice san Juan. Aquel \textquote{que está en el seno del Padre\ldots lo ha dado a conocer} (1, 18). Ahora conocemos a Dios tal como es verdaderamente. Él es Padre, bondad absoluta a la que podemos encomendarnos. El evangelista Marcos, que ha conservado los recuerdos de Pedro, nos dice que Jesús, al apelativo \textquote{Abbá}, añadió aún: Todo es posible para ti, tú lo puedes todo (cf. 14, 36). Él, que es la bondad, es al mismo tiempo poder, es omnipotente. El poder es bondad y la bondad es poder. Esta confianza la podemos aprender de la oración de Jesús en el Monte de los Olivos.

Antes de reflexionar sobre el contenido de la petición de Jesús, debemos prestar atención a lo que los evangelistas nos relatan sobre la actitud de Jesús durante su oración. Mateo y Marcos dicen que \textquote{cayó rostro en tierra} (\textit{Mt} 26, 39; cf. \textit{Mc} 14, 35); asume por consiguiente la actitud de total sumisión, que ha sido conservada en la liturgia romana del Viernes Santo. Lucas, en cambio, afirma que Jesús oraba arrodillado. En los Hechos de los Apóstoles, habla de los santos, que oraban de rodillas: Esteban durante su lapidación, Pedro en el contexto de la resurrección de un muerto, Pablo en el camino hacia el martirio. Así, Lucas ha trazado una pequeña historia del orar arrodillados de la Iglesia naciente. Los cristianos, al arrodillarse, se ponen en comunión con la oración de Jesús en el Monte de los Olivos. En la amenaza del poder del mal, ellos, en cuanto arrodillados, están de pie ante el mundo, pero, en cuanto hijos, están de rodillas ante el Padre. Ante la gloria de Dios, los cristianos nos arrodillamos y reconocemos su divinidad, pero expresando también en este gesto nuestra confianza en que él triunfe.

Jesús forcejea con el Padre. Combate consigo mismo. Y combate por nosotros. Experimenta la angustia ante el poder de la muerte. Esto es ante todo la turbación propia del hombre, más aún, de toda creatura viviente ante la presencia de la muerte. En Jesús, sin embargo, se trata de algo más. En las noches del mal, él ensancha su mirada. Ve la marea sucia de toda la mentira y de toda la infamia que le sobreviene en aquel cáliz que debe beber. Es el estremecimiento del totalmente puro y santo frente a todo el caudal del mal de este mundo, que recae sobre él. Él también me ve, y ora también por mí. Así, este momento de angustia mortal de Jesús es un elemento esencial en el proceso de la Redención. Por eso, la Carta a los Hebreos ha definido el combate de Jesús en el Monte de los Olivos como un acto sacerdotal. En esta oración de Jesús, impregnada de una angustia mortal, el Señor ejerce el oficio del sacerdote: toma sobre sí el pecado de la humanidad, a todos nosotros, y nos conduce al Padre.

Finalmente, debemos prestar atención aún al contenido de la oración de Jesús en el Monte de los Olivos. Jesús dice: \textquote{Padre: tú lo puedes todo, aparta de mí ese cáliz. Pero no sea como yo quiero, sino como tú quieres} (\textit{Mc} 14, 36). La voluntad natural del hombre Jesús retrocede asustada ante algo tan ingente. Pide que se le evite eso. Sin embargo, en cuanto Hijo, abandona esta voluntad humana en la voluntad del Padre: no yo, sino tú. Con esto ha transformado la actitud de Adán, el pecado primordial del hombre, salvando de este modo al hombre. La actitud de Adán había sido: No lo que tú has querido, Dios; quiero ser dios yo mismo. Esta soberbia es la verdadera esencia del pecado. Pensamos ser libres y verdaderamente nosotros mismos sólo si seguimos exclusivamente nuestra voluntad. Dios aparece como el antagonista de nuestra libertad. Debemos liberarnos de él, pensamos nosotros; sólo así seremos libres. Esta es la rebelión fundamental que atraviesa la historia, y la mentira de fondo que desnaturaliza la vida. Cuando el hombre se pone contra Dios, se pone contra la propia verdad y, por tanto, no llega a ser libre, sino alienado de sí mismo. Únicamente somos libres si estamos en nuestra verdad, si estamos unidos a Dios. Entonces nos hacemos verdaderamente \textquote{como Dios}, no oponiéndonos a Dios, no desentendiéndonos de él o negándolo. En el forcejeo de la oración en el Monte de los Olivos, Jesús ha deshecho la falsa contradicción entre obediencia y libertad, y abierto el camino hacia la libertad. Oremos al Señor para que nos adentre en este \textquote{sí} a la voluntad de Dios, haciéndonos verdaderamente libres. Amén.
\end{body}

\newsection
\subsection{Francisco, papa}

\subsubsection{Homilía (2015): Se hizo esclavo por nosotros}

\src{Iglesia \textquote{Padre Nuestro}, \\Nuevo Complejo Penitenciario de Rebibbia, Roma. \\2 de abril de 2015.}

\begin{body}
\ltr{E}{ste} jueves, Jesús estaba en la mesa con los discípulos, celebrando la fiesta de la Pascua. Y el pasaje del \textbf{Evangelio} que hemos escuchado contiene una frase que es precisamente el centro de lo que hizo Jesús por todos nosotros: \textquote{Habiendo amado a los suyos que estaban en el mundo, los amó hasta el extremo} (\textit{Jn} 13, 1). Jesús nos amó. Jesús nos ama. Sin límites, siempre, hasta el extremo. El amor de Jesús por nosotros no tiene límites: cada vez más, cada vez más. No se cansa de amar. A ninguno. Nos ama a todos nosotros, hasta el punto de dar la vida por nosotros. Sí, dar la vida por nosotros; sí, dar la vida por todos nosotros, dar la vida por cada uno de nosotros. Y cada uno puede decir: \textquote{Dio la vida por mí}. Por cada uno. Ha dado la vida por ti, por ti, por ti, por mí, por él\ldots por cada uno, con nombre y apellido. Su amor es así: personal. El amor de Jesús nunca defrauda, porque Él no se cansa de amar, como no se cansa de perdonar, no se cansa de abrazarnos. Esta es la primera cosa que quería deciros: Jesús nos amó, a cada uno de nosotros, hasta el extremo.

Y luego, hizo lo que los discípulos no comprendieron: lavar los pies. En ese tiempo era habitual, era una costumbre, porque cuando la gente llegaba a una casa tenía los pies sucios por el polvo del camino; no existían los adoquines en ese tiempo\ldots Había polvo por el camino. Y en el ingreso de la casa se lavaban los pies. Pero esto no lo hacía el dueño de casa, lo hacían los esclavos. Era un trabajo de esclavos. Y Jesús lava como esclavo nuestros pies, los pies de los discípulos, y por eso dice: \textquote{Lo que yo hago, tú no lo entiendes ahora –dice a Pedro–, pero lo comprenderás más tarde} (\textit{Jn} 13, 7). Es tan grande el amor de Jesús que se hizo esclavo para servirnos, para curarnos, para limpiarnos.

Y hoy, en esta misa, la Iglesia quiere que el sacerdote lave los pies de doce personas, en memoria de los doce apóstoles. Pero en nuestro corazón debemos tener la certeza, debemos estar seguros de que el Señor, cuando nos lava los pies, nos lava todo, nos purifica, nos hace sentir de nuevo su amor. En la Biblia hay una frase, del profeta Isaías, muy bella, que dice: \textquote{¿Puede una madre olvidar a su hijo? Aunque ella se olvidara de su hijo, yo nunca me olvidaré de ti} (cf. 49, 15). Así es el amor de Dios por nosotros.

Y yo lavaré hoy los pies de doce de vosotros, pero en estos hermanos y hermanas estáis todos vosotros, todos, todos. Todos los que viven aquí. Vosotros los representáis a ellos. Y también yo necesito ser lavado por el Señor, y por eso rezad durante esta misa para que el Señor lave también mis suciedades, para que yo llegue a ser un mejor siervo vuestro, un mejor siervo al servicio de la gente, como lo fue Jesús.

Ahora comenzaremos esta parte de la celebración.
\end{body}

\label{b-05-01-2015H}

\begin{patercite}
(\ldots) Participando en la Eucaristía, vivimos de modo extraordinario la oración que Jesús hizo y hace continuamente por cada uno a fin de que el mal, que todos encontramos en la vida, no llegue a vencer, y obre en nosotros la fuerza transformadora de la muerte y resurrección de Cristo. En la Eucaristía la Iglesia responde al mandamiento de Jesús: \textquote{Haced esto en memoria mía} (\emph{Lc} 22, 19; cf. \emph{1 Co} 11, 24-26); repite la oración de acción de gracias y de bendición y, con ella, las palabras de la transustanciación del pan y del vino en el Cuerpo y la Sangre del Señor. En nuestras Eucaristías somos atraídos a aquel momento de oración, nos unimos siempre de nuevo a la oración de Jesús. Desde el principio, la Iglesia comprendió las palabras de la consagración como parte de la \emph{oración rezada junto con Jesús}; como parte central de la alabanza impregnada de gratitud, a través de la cual Dios nos dona nuevamente el fruto de la tierra y del trabajo del hombre como cuerpo y sangre de Jesús, como auto-donación de Dios mismo en el amor del Hijo que nos acoge (cf. \emph{Jesús de Nazaret}, II, p. 154). Participando en la Eucaristía, nutriéndonos de la carne y de la Sangre del Hijo de Dios, unimos nuestra oración a la del Cordero pascual en su noche suprema, para que nuestra vida no se pierda, no obstante nuestra debilidad y nuestras infidelidades, sino que sea transformada.

(\ldots) Pidamos al Señor que nuestra participación en su Eucaristía, indispensable para la vida cristiana, después de prepararnos debidamente, también con el sacramento de la Penitencia, sea siempre el punto más alto de toda nuestra oración. Pidamos que, unidos profundamente en su mismo ofrecimiento al Padre, también nosotros transformemos nuestras cruces en sacrificio, libre y responsable, de amor a Dios y a los hermanos.

\textbf{Benedicto XVI}, papa, \textit{Catequesis}, Audiencia general, 11 de enero de 2012, parr. 10-11.
\end{patercite}

\newpage
\subsubsection{Homilía (2018): Jesús arriesga por nosotros}

\src{Cárcel de \textquote{Regina Coeli}, Roma. \\29 de marzo de 2018.}

\begin{body}
\ltr{J}{esús} termina su discurso diciendo: \textquote{Porque os he dado ejemplo, para que también vosotros hagáis como yo he hecho con vosotros} (\textit{Jn} 13, 15). Lavar los pies. Los pies, en esa época, eran lavados por los esclavos. La gente recorría el camino, no había asfalto, no había \textquote{sanpietrini}; en aquel tiempo había polvo en el camino y la gente se manchaba los pies. Y en la entrada de la casa estaban los esclavos que lavaban los pies. Era un trabajo para esclavos. Pero era un servicio: un servicio hecho por esclavos. Y Jesús quiere hacer este servicio, para darnos un ejemplo de cómo nosotros debemos servirnos los unos a los otros. 

Una vez, cuando estaban en camino, dos de los discípulos que querían hacer carrera, habían pedido a Jesús ocupar puestos importantes, uno a la derecha y otro a la izquierda (cf. \textit{Mc} 10, 35-45). Y Jesús los miró con amor –Jesús miraba siempre con amor– y dijo: \textquote{No sabéis lo que pedís} (v. 38). Los jefes de las naciones –dice Jesús– mandan, se hacen servir, y ellos están bien (cf. v. 42). Pensemos en esa época de los reyes, de los emperadores tan crueles, que se hacían servir por los esclavos\ldots Pero entre vosotros –dice Jesús– no debe ser lo mismo: quien manda debe servir. Vuestro jefe debe ser vuestro servidor (cf. v. 43). Jesús da la vuelta a la costumbre histórica, cultural de esa época –también esta de hoy– aquel que manda, para ser un buen jefe, sea donde sea, debe servir. Yo pienso muchas veces –no en este tiempo porque cada uno todavía está vivo y tiene la oportunidad de cambiar de vida y no podemos juzgar, pero pensemos en la historia– si muchos reyes, emperadores, jefes de Estado hubieran entendido esta enseñanza de Jesús y en vez de mandar, ser crueles, matar gente, hubieran hecho esto, ¡cuántas guerras no se hubieran hecho! El servicio: realmente hay gente que no facilita esta actitud, gente soberbia, gente odiosa, gente que quizá nos desea el mal; pero nosotros estamos llamados a servirles más. Y también hay gente que sufre, que está descartada por la sociedad, al menos por un periodo, y Jesús va ahí a decirles: Tú eres importante para mí. Jesús viene a servirnos, y la señal que Jesús nos sirve hoy aquí, [en la cárcel de Regina Coeli,] es que ha querido elegir a doce de vosotros, como los doce apóstoles, para lavar los pies. Jesús arriesga sobre cada uno de nosotros. Sabed esto: Jesús se llama Jesús, no se llama Poncio Pilato. Jesús no sabe lavarse las manos: ¡solamente sabe arriesgar! Mirad esta imagen tan bonita: Jesús arrodillado entre las espinas, arriesgando herirse para tomar la oveja perdida. 

Hoy yo, que soy pecador como vosotros, pero represento a Jesús, soy embajador de Jesús. Hoy, cuando yo me arrodillo delante de cada uno de vosotros, pensad: \textquote{Jesús ha arriesgado en este hombre, un pecador, para venir a mí y decirme que me ama}. Este es el servicio, este es Jesús: no nos abandona nunca, no se cansa nunca de perdonarnos. Nos ama mucho. ¡Mirad cómo arriesga Jesús! Y así, con estos sentimientos, vamos adelante con esta ceremonia que es simbólica. Antes de darnos su cuerpo y su sangre, Jesús arriesga por cada uno de nosotros, y arriesga en el servicio porque nos ama mucho. 
\end{body}


\begin{patercite}
(\ldots) Has sido invitado a un gran banquete: considera atentamente qué manjares te ofrecen, pues también tú	debes preparar lo que a ti te han ofrecido. Es realmente sublime el	banquete donde se sirve, como alimento, el mismo Señor que invita al	banquete. Nadie, en efecto, alimenta de sí mismo a los que invita, pero el Señor Jesucristo ha hecho precisamente esto: él, que es quien invita, se da a sí mismo como comida y bebida. Y los mártires, entendiendo bien lo que habían comido y bebido, devolvieron al Señor lo mismo que de él	habían recibido. 
	
Pero, ¿cómo podrían devolver tales dones si no fuera por concesión de aquel que fue el primero en concedérselos? Esto es lo que nos enseña el	salmo que hemos cantado: \textit{Mucho le place al Señor la muerte de sus fieles}. En este salmo el autor consideró cuán grandes cosas había recibido del Señor; contempló la grandeza de los dones del Todopoderoso, que lo había creado, que cuando se había perdido lo buscó, que una vez encontrado le dio su perdón, que lo ayudó, cuando luchaba, en su debilidad, que no se	apartó en el momento de las pruebas, que lo coronó en la victoria y se le dio a sí mismo como premio; consideró todas estas cosas y exclamó: \textit{¿Cómo pagaré al Señor todo el bien que me ha hecho? Alzaré la copa de la salvación}.
	
¿De qué copa se trata? Sin duda de la copa de la pasión, copa amarga y	saludable, copa que debe beber primero el médico para quitar las aprensiones del enfermo. Es ésta la copa: la reconocemos por las	palabras de Cristo, cuando dice: \textit{Padre, si es posible, que se aleje de mí ese cáliz}.
	
De este mismo cáliz, afirmaron, pues, los mártires: \textit{Alzaré la copa de la salvación, invocando su nombre}. \textquote{¿Tienes miedo de no poder resistir?}, \textquote{No}, dice el mártir. \textquote{¿Por qué?} \textquote{Porque he invocado el nombre del Señor}. ¿Cómo podrían haber triunfado los mártires si en ellos no hubiera vencido aquel que afirmó: \textit{Tened valor: yo he vencido al mundo}? El que reina en el cielo regía la mente y la lengua de sus mártires, y	por medio de ellos, en la tierra, vencía al diablo y, en el cielo, coronaba a sus mártires. ¡Dichosos los que así bebieron este cáliz! Se	acabaron los dolores y han recibido el honor. Por tanto, queridos hermanos, concebid en vuestra mente y en vuestro	espíritu lo que no podéis ver con vuestros ojos, y sabed que \textit{mucho le place al Señor la muerte de sus fieles}.
	
\textbf{San Agustín}, obispo, \textit{Sermón} 329, en el natalicio de los mártires, cf. 1-2: PL 38,1454-1456.
	
\end{patercite}

\newsection 
\section{Temas}

\cceth{Institución de la Eucaristía} 
\cceref{CEC 1337-1344}

\begin{ccebody}

\ccesec{La institución de la Eucaristía}

\n{1337} El Señor, habiendo amado a los suyos, los amó hasta el fin. Sabiendo que había llegado la hora de partir de este mundo para retornar a su Padre, en el transcurso de una cena, les lavó los pies y les dio el mandamiento del amor (\textit{Jn} 13,1-17). Para dejarles una prenda de este amor, para no alejarse nunca de los suyos y hacerles partícipes de su Pascua, instituyó la Eucaristía como memorial de su muerte y de su resurrección y ordenó a sus apóstoles celebrarlo hasta su retorno, \textquote{constituyéndoles entonces sacerdotes del Nuevo Testamento} (Concilio de Trento: DS 1740).

\n{1338} Los tres evangelios sinópticos y san Pablo nos han transmitido el relato de la institución de la Eucaristía; por su parte, san Juan relata las palabras de Jesús en la sinagoga de Cafarnaúm, palabras que preparan la institución de la Eucaristía: Cristo se designa a sí mismo como el pan de vida, bajado del cielo (cf. \textit{Jn} 6).

\n{1339} Jesús escogió el tiempo de la Pascua para realizar lo que había anunciado en Cafarnaúm: dar a sus discípulos su Cuerpo y su Sangre:

\ccecite{\textquote{Llegó el día de los Ázimos, en el que se había de inmolar el cordero de Pascua; [Jesús] envió a Pedro y a Juan, diciendo: \textquote{Id y preparadnos la Pascua para que la comamos} [\ldots] fueron [\ldots] y prepararon la Pascua. Llegada la hora, se puso a la mesa con los Apóstoles; y les dijo: \textquote{Con ansia he deseado comer esta Pascua con vosotros antes de padecer; porque os digo que ya no la comeré más hasta que halle su cumplimiento en el Reino de Dios} [\ldots] Y tomó pan, dio gracias, lo partió y se lo dio diciendo: \textquote{Esto es mi cuerpo que va a ser entregado por vosotros; haced esto en recuerdo mío}. De igual modo, después de cenar, tomó el cáliz, diciendo: \textquote{Este cáliz es la Nueva Alianza en mi sangre, que va a ser derramada por vosotros}} (\textit{Lc} 22,7-20; cf. \textit{Mt} 26,17-29; \textit{Mc} 14,12-25; \textit{1 Co} 11,23-26).}

\n{1340} Al celebrar la última Cena con sus Apóstoles en el transcurso del banquete pascual, Jesús dio su sentido definitivo a la pascua judía. En efecto, el paso de Jesús a su Padre por su muerte y su resurrección, la Pascua nueva, es anticipada en la Cena y celebrada en la Eucaristía que da cumplimiento a la pascua judía y anticipa la pascua final de la Iglesia en la gloria del Reino.

\newpage
\ccesec{\textquote{Haced esto en memoria mía}}

\n{1341} El mandamiento de Jesús de repetir sus gestos y sus palabras \textquote{hasta que venga} (\textit{1 Co} 11,26), no exige solamente acordarse de Jesús y de lo que hizo. Requiere la celebración litúrgica por los Apóstoles y sus sucesores del \textit{memorial} de Cristo, de su vida, de su muerte, de su resurrección y de su intercesión junto al Padre.

\n{1342} Desde el comienzo la Iglesia fue fiel a la orden del Señor. De la Iglesia de Jerusalén se dice:

\ccecite{\textquote{Acudían asiduamente a la enseñanza de los apóstoles, fieles a la comunión fraterna, a la fracción del pan y a las oraciones [\ldots] Acudían al Templo todos los días con perseverancia y con un mismo espíritu, partían el pan por las casas y tomaban el alimento con alegría y con sencillez de corazón} (\textit{Hch} 2,42.46).}

\n{1343} Era sobre todo \textquote{el primer día de la semana}, es decir, el domingo, el día de la resurrección de Jesús, cuando los cristianos se reunían para \textquote{partir el pan} (\textit{Hch} 20,7). Desde entonces hasta nuestros días, la celebración de la Eucaristía se ha perpetuado, de suerte que hoy la encontramos por todas partes en la Iglesia, con la misma estructura fundamental. Sigue siendo el centro de la vida de la Iglesia.

\n{1344} Así, de celebración en celebración, anunciando el misterio pascual de Jesús \textquote{hasta que venga} (\textit{1 Co} 11,26), el pueblo de Dios peregrinante \textquote{camina por la senda estrecha de la cruz} (AG 1) hacia el banquete celestial, donde todos los elegidos se sentarán a la mesa del Reino.
\end{ccebody}

\cceth{La Eucaristía como acción de gracias} 
\cceref{CEC 1359-1361}

\begin{ccebody}
	
\ccesec{La acción de gracias y la alabanza al Padre}

\n{1359} La Eucaristía, sacramento de nuestra salvación realizada por Cristo en la cruz, es también un sacrificio de alabanza en acción de gracias por la obra de la creación. En el Sacrificio Eucarístico, toda la creación amada por Dios es presentada al Padre a través de la muerte y resurrección de Cristo. Por Cristo, la Iglesia puede ofrecer el sacrificio de alabanza en acción de gracias por todo lo que Dios ha hecho de bueno, de bello y de justo en la creación y en la humanidad.

\n{1360} La Eucaristía es un sacrificio de acción de gracias al Padre, una bendición por la cual la Iglesia expresa su reconocimiento a Dios por todos sus beneficios, por todo lo que ha realizado mediante la creación, la redención y la santificación. \textquote{Eucaristía} significa, ante todo, acción de gracias.

\n{1361} La Eucaristía es también el sacrificio de alabanza por medio del cual la Iglesia canta la gloria de Dios en nombre de toda la creación. Este sacrificio de alabanza sólo es posible a través de Cristo: Él une los fieles a su persona, a su alabanza y a su intercesión, de manera que el sacrificio de alabanza al Padre es ofrecido \textit{por} Cristo y \textit{con} Cristo para ser aceptado \textit{en} él.
\end{ccebody}

\newpage
\cceth{La Eucaristía como sacrificio} 
\cceref{CEC 610, 1362-1372, 1382, 1436}

\begin{ccebody}
\ccesec{Jesús anticipó en la cena la ofrenda libre de su vida}

\n{610} Jesús expresó de forma suprema la ofrenda libre de sí mismo en la cena tomada con los doce Apóstoles (cf. \textit{Mt} 26, 20), en \textquote{la noche en que fue entregado} (\textit{1 Co} 11, 23). En la víspera de su Pasión, estando todavía libre, Jesús hizo de esta última Cena con sus Apóstoles el memorial de su ofrenda voluntaria al Padre (cf. \textit{1 Co} 5, 7), por la salvación de los hombres: \textquote{Este es mi Cuerpo que va a \textit{ser entregado} por vosotros} (\textit{Lc} 22, 19). \textquote{Esta es mi sangre de la Alianza que va a \textit{ser derramada} por muchos para remisión de los pecados} (\textit{Mt} 26, 28).

\ccesec{El memorial sacrificial de Cristo y de su Cuerpo, que es la Iglesia}

\n{1362} La Eucaristía es el memorial de la Pascua de Cristo, la actualización y la ofrenda sacramental de su único sacrificio, en la liturgia de la Iglesia que es su Cuerpo. En todas las plegarias eucarísticas encontramos, tras las palabras de la institución, una oración llamada \textit{anámnesis} o memorial.

\n{1363} En el sentido empleado por la Sagrada Escritura, el \textit{memorial} no es solamente el recuerdo de los acontecimientos del pasado, sino la proclamación de las maravillas que Dios ha realizado en favor de los hombres (cf. \textit{Ex} 13,3). En la celebración litúrgica, estos acontecimientos se hacen, en cierta forma, presentes y actuales. De esta manera Israel entiende su liberación de Egipto: cada vez que es celebrada la pascua, los acontecimientos del Éxodo se hacen presentes a la memoria de los creyentes a fin de que conformen su vida a estos acontecimientos.

\n{1364} El memorial recibe un sentido nuevo en el Nuevo Testamento. Cuando la Iglesia celebra la Eucaristía, hace memoria de la Pascua de Cristo y ésta se hace presente: el sacrificio que Cristo ofreció de una vez para siempre en la cruz, permanece siempre actual (cf. \textit{Hb} 7,25-27): \textquote{Cuantas veces se renueva en el altar el sacrificio de la cruz, en el que \textquote{Cristo, nuestra Pascua, fue inmolado} (\textit{1Co} 5, 7), se realiza la obra de nuestra redención} (LG 3).

\n{1365} Por ser memorial de la Pascua de Cristo, \textit{la Eucaristía es también un sacrificio}. El carácter sacrificial de la Eucaristía se manifiesta en las palabras mismas de la institución: \textquote{Esto es mi Cuerpo que será entregado por vosotros} y \textquote{Esta copa es la nueva Alianza en mi sangre, que será derramada por vosotros} (\textit{Lc} 22,19-20). En la Eucaristía, Cristo da el mismo cuerpo que por nosotros entregó en la cruz, y la sangre misma que \textquote{derramó por muchos [\ldots] para remisión de los pecados} (\textit{Mt} 26,28).


\newpage 
\n{1366} La Eucaristía es, pues, un sacrificio porque \textit{representa} (= hace presente) el sacrificio de la cruz, porque es su \textit{memorial} y \textit{aplica} su fruto:

\ccecite{\textquote{(Cristo), nuestro Dios y Señor [\ldots] se ofreció a Dios Padre [\ldots] una vez por todas, muriendo como intercesor sobre el altar de la cruz, a fin de realizar para ellos (los hombres) la redención eterna. Sin embargo, como su muerte no debía poner fin a su sacerdocio (\textit{Hb} 7,24.27), en la última Cena, \textquote{la noche en que fue entregado} (\textit{1 Co}11,23), quiso dejar a la Iglesia, su esposa amada, un sacrificio visible (como lo reclama la naturaleza humana) [\ldots] donde se representara el sacrificio sangriento que iba a realizarse una única vez en la cruz, cuya memoria se perpetuara hasta el fin de los siglos (\textit{1 Co} 11,23) y cuya virtud saludable se aplicara a la remisión de los pecados que cometemos cada día} (Concilio de Trento: DS 1740).}

\n{1367} El sacrificio de Cristo y el sacrificio de la Eucaristía son, pues, \textit{un único sacrificio}: \textquote{La víctima es una y la misma. El mismo el que se ofrece ahora por el ministerio de los sacerdotes, el que se ofreció a sí mismo en la cruz, y solo es diferente el modo de ofrecer} (Concilio de Trento: DS 1743). \textquote{Y puesto que en este divino sacrificio que se realiza en la misa, se contiene e inmola incruentamente el mismo Cristo que en el altar de la cruz \textquote{se ofreció a sí mismo una vez de modo cruento}; [\ldots] este sacrificio [es] verdaderamente propiciatorio} (\textit{Ibíd}).

\n{1368} \textit{La Eucaristía es igualmente el sacrificio de la Iglesia}. La Iglesia, que es el Cuerpo de Cristo, participa en la ofrenda de su Cabeza. Con Él, ella se ofrece totalmente. Se une a su intercesión ante el Padre por todos los hombres. En la Eucaristía, el sacrificio de Cristo se hace también el sacrificio de los miembros de su Cuerpo. La vida de los fieles, su alabanza, su sufrimiento, su oración y su trabajo se unen a los de Cristo y a su total ofrenda, y adquieren así un valor nuevo. El sacrificio de Cristo presente sobre el altar da a todas las generaciones de cristianos la posibilidad de unirse a su ofrenda.

En las catacumbas, la Iglesia es con frecuencia representada como una mujer en oración, los brazos extendidos en actitud de orante. Como Cristo que extendió los brazos sobre la cruz, por él, con él y en él, la Iglesia se ofrece e intercede por todos los hombres.

\n{1369} \textit{Toda la Iglesia se une a la ofrenda y a la intercesión de Cristo}. Encargado del ministerio de Pedro en la Iglesia, \textit{el Papa} es asociado a toda celebración de la Eucaristía en la que es nombrado como signo y servidor de la unidad de la Iglesia universal. \textit{El obispo} del lugar es siempre responsable de la Eucaristía, incluso cuando es presidida por un \textit{presbítero}; el nombre del obispo se pronuncia en ella para significar su presidencia de la Iglesia particular en medio del presbiterio y con la asistencia de los \textit{diáconos}. La comunidad intercede también por todos los ministros que, por ella y con ella, ofrecen el Sacrificio Eucarístico:

\ccecite{\textquote{Que sólo sea considerada como legítima la Eucaristía que se hace bajo la presidencia del obispo o de quien él ha señalado para ello} (San Ignacio de Antioquía, \textit{Epistula ad Smyrnaeos} 8,1).}

\ccecite{\textquote{Por medio del ministerio de los presbíteros, se realiza a la perfección el sacrificio espiritual de los fieles en unión con el sacrificio de Cristo, único Mediador. Este, en nombre de toda la Iglesia, por manos de los presbíteros, se ofrece incruenta y sacramentalmente en la Eucaristía, hasta que el Señor venga} (PO 2).}

\newpage
\n{1370} A la ofrenda de Cristo se unen no sólo los miembros que están todavía aquí abajo, sino también los que están ya \textit{en la gloria del cielo}: La Iglesia ofrece el Sacrificio Eucarístico en comunión con la santísima Virgen María y haciendo memoria de ella, así como de todos los santos y santas. En la Eucaristía, la Iglesia, con María, está como al pie de la cruz, unida a la ofrenda y a la intercesión de Cristo.

\n{1371} El Sacrificio Eucarístico es también ofrecido \textit{por los fieles difuntos} \textquote{que han muerto en Cristo y todavía no están plenamente purificados} (Concilio de Trento: DS 1743), para que puedan entrar en la luz y la paz de Cristo:

\ccecite{\textquote{Enterrad [\ldots] este cuerpo en cualquier parte; no os preocupe más su cuidado; solamente os ruego que, dondequiera que os hallareis, os acordéis de mí ante el altar del Señor} (San Agustín, \textit{Confessiones}, 9, 11, 27; palabras de santa Mónica, antes de su muerte, dirigidas a san Agustín y a su hermano).}

\ccecite{\textquote{A continuación oramos (en la anáfora) por los santos padres y obispos difuntos, y en general por todos los que han muerto antes que nosotros, creyendo que será de gran provecho para las almas, en favor de las cuales es ofrecida la súplica, mientras se halla presente la santa y adorable víctima [\ldots] Presentando a Dios nuestras súplicas por los que han muerto, aunque fuesen pecadores [\ldots], presentamos a Cristo inmolado por nuestros pecados, haciendo propicio para ellos y para nosotros al Dios amigo de los hombres} (San Cirilo de Jerusalén, \textit{Catecheses mistagogicae} 5, 9.10).}

\n{1372} San Agustín ha resumido admirablemente esta doctrina que nos impulsa a una participación cada vez más completa en el sacrificio de nuestro Redentor que celebramos en la Eucaristía:

\ccecite{\textquote{Esta ciudad plenamente rescatada, es decir, la asamblea y la sociedad de los santos, es ofrecida a Dios como un sacrificio universal [\ldots] por el Sumo Sacerdote que, bajo la forma de esclavo, llegó a ofrecerse por nosotros en su pasión, para hacer de nosotros el cuerpo de una tan gran Cabeza [\ldots] Tal es el sacrificio de los cristianos: \textquote{siendo muchos, no formamos más que un sólo cuerpo en Cristo} (\textit{Rm} 12,5). Y este sacrificio, la Iglesia no cesa de reproducirlo en el Sacramento del altar bien conocido de los fieles, donde se muestra que en lo que ella ofrece se ofrece a sí misma} (San Agustín, \textit{De civitate Dei} 10, 6).}

\ccesec{El banquete pascual}

\n{1382} La misa es, a la vez e inseparablemente, el memorial sacrificial en que se perpetúa el sacrificio de la cruz, y el banquete sagrado de la comunión en el Cuerpo y la Sangre del Señor. Pero la celebración del sacrificio eucarístico está totalmente orientada hacia la unión íntima de los fieles con Cristo por medio de la comunión. Comulgar es recibir a Cristo mismo que se ofrece por nosotros.

\n{1436} \textit{Eucaristía y Penitencia}. La conversión y la penitencia diarias encuentran su fuente y su alimento en la Eucaristía, pues en ella se hace presente el sacrificio de Cristo que nos reconcilió con Dios; por ella son alimentados y fortificados los que viven de la vida de Cristo; \textquote{es el antídoto que nos libera de nuestras faltas cotidianas y nos preserva de pecados mortales} (Concilio de Trento: DS 1638).
\end{ccebody}


\newpage
\cceth{La presencia real de Cristo en la Eucaristía} 
\cceref{CEC 1373-1381}

\begin{ccebody}

\ccesec{La presencia de Cristo por el poder de su Palabra y del Espíritu Santo}

\n{1373} \textquote{Cristo Jesús que murió, resucitó, que está a la derecha de Dios e intercede por nosotros} (\textit{Rm} 8,34), está presente de múltiples maneras en su Iglesia (cf. LG 48): en su Palabra, en la oración de su Iglesia, \textquote{allí donde dos o tres estén reunidos en mi nombre} (\textit{Mt} 18,20), en los pobres, los enfermos, los presos (\textit{Mt} 25,31-46), en los sacramentos de los que Él es autor, en el sacrificio de la misa y en la persona del ministro. Pero, \textquote{\textit{sobre todo}, (está presente) \textit{bajo las especies eucarísticas}} (SC 7).

\n{1374} El modo de presencia de Cristo bajo las especies eucarísticas es singular. Eleva la Eucaristía por encima de todos los sacramentos y hace de ella \textquote{como la perfección de la vida espiritual y el fin al que tienden todos los sacramentos} (Santo Tomás de Aquino, \textit{Summa theologiae} 3, q. 73, a. 3). En el Santísimo Sacramento de la Eucaristía están \textquote{contenidos \textit{verdadera, real y substancialmente} el Cuerpo y la Sangre junto con el alma y la divinidad de nuestro Señor Jesucristo, y, por consiguiente, \textit{Cristo entero}} (Concilio de Trento: DS 1651). \textquote{Esta presencia se denomina \textquote{real}, no a título exclusivo, como si las otras presencias no fuesen \textquote{reales}, sino por excelencia, porque es \textit{substancial}, y por ella Cristo, Dios y hombre, se hace totalmente presente} (MF 39).

\n{1375} Mediante la \textit{conversión} del pan y del vino en su Cuerpo y Sangre, Cristo se hace presente en este sacramento. Los Padres de la Iglesia afirmaron con fuerza la fe de la Iglesia en la eficacia de la Palabra de Cristo y de la acción del Espíritu Santo para obrar esta conversión. Así, san Juan Crisóstomo declara que:

\ccecite{\textquote{No es el hombre quien hace que las cosas ofrecidas se conviertan en Cuerpo y Sangre de Cristo, sino Cristo mismo que fue crucificado por nosotros. El sacerdote, figura de Cristo, pronuncia estas palabras, pero su eficacia y su gracia provienen de Dios. \textit{Esto es mi Cuerpo}, dice. Esta palabra transforma las cosas ofrecidas} (\textit{De proditione Iudae homilia} 1, 6).}

Y san Ambrosio dice respecto a esta conversión:

\ccecite{\textquote{Estemos bien persuadidos de que esto no es lo que la naturaleza ha producido, sino lo que la bendición ha consagrado, y de que la fuerza de la bendición supera a la de la naturaleza, porque por la bendición la naturaleza misma resulta cambiada} (\textit{De mysteriis} 9, 50). \textquote{La palabra de Cristo, que pudo hacer de la nada lo que no existía, ¿no podría cambiar las cosas existentes en lo que no eran todavía? Porque no es menos dar a las cosas su naturaleza primera que cambiársela} (\textit{Ibíd.}, 9,50.52).}

\n{1376} El Concilio de Trento resume la fe católica cuando afirma: \textquote{Porque Cristo, nuestro Redentor, dijo que lo que ofrecía bajo la especie de pan era verdaderamente su Cuerpo, se ha mantenido siempre en la Iglesia esta convicción, que declara de nuevo el Santo Concilio: por la consagración del pan y del vino se opera la conversión de toda la substancia del pan en la substancia del Cuerpo de Cristo nuestro Señor y de toda la substancia del vino en la substancia de su Sangre; la Iglesia católica ha llamado justa y apropiadamente a este cambio \textit{transubstanciación}} (DS 1642).


\newpage
\n{1377} La presencia eucarística de Cristo comienza en el momento de la consagración y dura todo el tiempo que subsistan las especies eucarísticas. Cristo está todo entero presente en cada una de las especies y todo entero en cada una de sus partes, de modo que la fracción del pan no divide a Cristo (cf. Concilio de Trento: DS 1641).

\n{1378} \textit{El culto de la Eucaristía}. En la liturgia de la misa expresamos nuestra fe en la presencia real de Cristo bajo las especies de pan y de vino, entre otras maneras, arrodillándonos o inclinándonos profundamente en señal de adoración al Señor. \textquote{La Iglesia católica ha dado y continua dando este culto de adoración que se debe al sacramento de la Eucaristía no solamente durante la misa, sino también fuera de su celebración: conservando con el mayor cuidado las hostias consagradas, presentándolas a los fieles para que las veneren con solemnidad, llevándolas en procesión en medio de la alegría del pueblo} (MF 56).

\n{1379} El sagrario (tabernáculo) estaba primeramente destinado a guardar dignamente la Eucaristía para que pudiera ser llevada a los enfermos y ausentes fuera de la misa. Por la profundización de la fe en la presencia real de Cristo en su Eucaristía, la Iglesia tomó conciencia del sentido de la adoración silenciosa del Señor presente bajo las especies eucarísticas. Por eso, el sagrario debe estar colocado en un lugar particularmente digno de la iglesia; debe estar construido de tal forma que subraye y manifieste la verdad de la presencia real de Cristo en el santísimo sacramento.

\n{1380} Es grandemente admirable que Cristo haya querido hacerse presente en su Iglesia de esta singular manera. Puesto que Cristo iba a dejar a los suyos bajo su forma visible, quiso darnos su presencia sacramental; puesto que iba a ofrecerse en la cruz por muestra salvación, quiso que tuviéramos el memorial del amor con que nos había amado \textquote{hasta el fin} (\textit{Jn} 13,1), hasta el don de su vida. En efecto, en su presencia eucarística permanece misteriosamente en medio de nosotros como quien nos amó y se entregó por nosotros (cf. \textit{Ga} 2,20), y se queda bajo los signos que expresan y comunican este amor:

\ccecite{\textquote{La Iglesia y el mundo tienen una gran necesidad del culto eucarístico. Jesús nos espera en este sacramento del amor. No escatimemos tiempo para ir a encontrarlo en la adoración, en la contemplación llena de fe y abierta a reparar las faltas graves y delitos del mundo. No cese nunca nuestra adoración} (Juan Pablo II, Carta \textit{Dominicae Cenae}, 3).}

\n{1381} \textquote{La presencia del verdadero Cuerpo de Cristo y de la verdadera Sangre de Cristo en este sacramento, \textquote{no se conoce por los sentidos, dice santo Tomás, sino sólo \textit{por la fe}, la cual se apoya en la autoridad de Dios}. Por ello, comentando el texto de san Lucas 22, 19: \textquote{\textit{Esto es mi Cuerpo que será entregado por vosotros}}, san Cirilo declara: \textquote{No te preguntes si esto es verdad, sino acoge más bien con fe las palabras del Salvador, porque Él, que es la Verdad, no miente}} (MF 18; cf. Santo Tomás de Aquino, Summa theologiae 3, q. 75, a. 1; San Cirilo de Alejandría, \textit{Commentarius in Lucam} 22, 19):


\newpage
\begin{cceprose}
Adoro Te devote, latens Deitas,
   Quae sub his figuris vere latitas:
   Tibi se cor meum totum subjicit,
   Quia Te contemplans totum deficit.

Visus, gustus, tactus in te fallitur,
   Sed auditu solo tuto creditur:
   Credo quidquid dixit Dei Filius:
   Nil hoc Veritatis verbo verius.

Adórote devotamente, oculta Deidad,
   que bajo estas sagradas especies te ocultas verdaderamente:
   A ti mi corazón totalmente se somete,
   pues al contemplarte, se siente desfallecer por completo.

La vista, el tacto, el gusto, son aquí falaces;
   sólo con el oído se llega a tener fe segura.
   Creo todo lo que ha dicho el Hijo de Dios,
   nada más verdadero que esta palabra de Verdad. 

[AHMA 50, 589]
\end{cceprose}
\end{ccebody}

\cceth{La Comunión} 
\cceref{CEC 1384-1401, 2837}

\begin{ccebody}
\n{1384} El Señor nos dirige una invitación urgente a recibirle en el sacramento de la Eucaristía: \textquote{En verdad, en verdad os digo: si no coméis la carne del Hijo del hombre, y no bebéis su sangre, no tendréis vida en vosotros} (\textit{Jn} 6,53).

\n{1385} Para responder a esta invitación, debemos \textit{prepararnos} para este momento tan grande y santo. San Pablo exhorta a un examen de conciencia: \textquote{Quien coma el pan o beba el cáliz del Señor indignamente, será reo del Cuerpo y de la Sangre del Señor. Examínese, pues, cada cual, y coma entonces del pan y beba del cáliz. Pues quien come y bebe sin discernir el Cuerpo, come y bebe su propio castigo} (\textit{1 Co} 11,27-29). Quien tiene conciencia de estar en pecado grave debe recibir el sacramento de la Reconciliación antes de acercarse a comulgar.

\n{1386} Ante la grandeza de este sacramento, el fiel sólo puede repetir humildemente y con fe ardiente las palabras del Centurión (cf. \textit{Mt} 8,8): \textquote{\textit{Señor, no soy digno de que entres en mi casa, pero una palabra tuya bastará para sanarme}}. En la Liturgia de san Juan Crisóstomo, los fieles oran con el mismo espíritu:

\ccecite{\textquote{A tomar parte en tu cena sacramental invítame hoy, Hijo de Dios: no revelaré a tus enemigos el misterio, no te te daré el beso de Judas; antes como el ladrón te reconozco y te suplico: ¡Acuérdate de mí, Señor, en tu reino!} (Liturgia Bizantina. \textit{Anaphora Iohannis Chrysostomi}, Oración antes de la Comunión).}

\n{1387} Para prepararse convenientemente a recibir este sacramento, los fieles deben observar el ayuno prescrito por la Iglesia (cf. CIC can. 919). Por la actitud corporal (gestos, vestido) se manifiesta el respeto, la solemnidad, el gozo de ese momento en que Cristo se hace nuestro huésped.

\n{1388} Es conforme al sentido mismo de la Eucaristía que los fieles, con las debidas disposiciones (cf. CIC, cans. 916-917), comulguen cuando participan en la misa [Los fieles pueden recibir la Sagrada Eucaristía solamente dos veces el mismo día. Pontificia Comisión para la auténtica interpretación del Código de Derecho Canónico, \textit{Responsa ad proposita dubia} 1]. \textquote{Se recomienda especialmente la participación más perfecta en la misa, recibiendo los fieles, después de la comunión del sacerdote, del mismo sacrificio, el cuerpo del Señor} (SC 55).

\n{1389} La Iglesia obliga a los fieles \textquote{a participar los domingos y días de fiesta en la divina liturgia} (cf. OE 15) y a recibir al menos una vez al año la Eucaristía, si es posible en tiempo pascual (cf. CIC can. 920), preparados por el sacramento de la Reconciliación. Pero la Iglesia recomienda vivamente a los fieles recibir la santa Eucaristía los domingos y los días de fiesta, o con más frecuencia aún, incluso todos los días.

\n{1390} Gracias a la presencia sacramental de Cristo bajo cada una de las especies, la comunión bajo la sola especie de pan ya hace que se reciba todo el fruto de gracia propio de la Eucaristía. Por razones pastorales, esta manera de comulgar se ha establecido legítimamente como la más habitual en el rito latino. \textquote{La comunión tiene una expresión más plena por razón del signo cuando se hace bajo las dos especies. Ya que en esa forma es donde más perfectamente se manifiesta el signo del banquete eucarístico} (\textit{Institución general del Misal Romano}, 240). Es la forma habitual de comulgar en los ritos orientales.

\n{1391} \textit{La comunión acrecienta nuestra unión con Cristo}. Recibir la Eucaristía en la comunión da como fruto principal la unión íntima con Cristo Jesús. En efecto, el Señor dice: \textquote{Quien come mi Carne y bebe mi Sangre habita en mí y yo en él} (\textit{Jn} 6,56). La vida en Cristo encuentra su fundamento en el banquete eucarístico: \textquote{Lo mismo que me ha enviado el Padre, que vive, y yo vivo por el Padre, también el que me coma vivirá por mí} (\textit{Jn} 6,57):

\ccecite{\textquote{Cuando en las fiestas [del Señor] los fieles reciben el Cuerpo del Hijo, proclaman unos a otros la Buena Nueva, se nos han dado las arras de la vida, como cuando el ángel dijo a María [de Magdala]: \textquote{¡Cristo ha resucitado!} He aquí que ahora también la vida y la resurrección son comunicadas a quien recibe a Cristo} (\textit{Fanqîth, Breviarium iuxta ritum Ecclesiae Antiochenae Syrorum}, v. 1).}

\n{1392} Lo que el alimento material produce en nuestra vida corporal, la comunión lo realiza de manera admirable en nuestra vida espiritual. La comunión con la Carne de Cristo resucitado, \textquote{vivificada por el Espíritu Santo y vivificante} (PO 5), conserva, acrecienta y renueva la vida de gracia recibida en el Bautismo. Este crecimiento de la vida cristiana necesita ser alimentado por la comunión eucarística, pan de nuestra peregrinación, hasta el momento de la muerte, cuando nos sea dada como viático.

\n{1393} \textit{La comunión nos separa del pecado}. El Cuerpo de Cristo que recibimos en la comunión es \textquote{entregado por nosotros}, y la Sangre que bebemos es \textquote{derramada por muchos para el perdón de los pecados}. Por eso la Eucaristía no puede unirnos a Cristo sin purificarnos al mismo tiempo de los pecados cometidos y preservarnos de futuros pecados:

\ccecite{\textquote{Cada vez que lo recibimos, anunciamos la muerte del Señor (cf. \textit{1 Co} 11,26). Si anunciamos la muerte del Señor, anunciamos también el perdón de los pecados. Si cada vez que su Sangre es derramada, lo es para el perdón de los pecados, debo recibirle siempre, para que siempre me perdone los pecados. Yo que peco siempre, debo tener siempre un remedio} (San Ambrosio, \textit{De sacramentis} 4, 28).}

\n{1394} Como el alimento corporal sirve para restaurar la pérdida de fuerzas, la Eucaristía fortalece la caridad que, en la vida cotidiana, tiende a debilitarse; y esta caridad vivificada \textit{borra los pecados veniales} (cf. Concilio de Trento: DS 1638). Dándose a nosotros, Cristo reaviva nuestro amor y nos hace capaces de romper los lazos desordenados con las criaturas y de arraigarnos en Él:

\ccecite{\textquote{Porque Cristo murió por nuestro amor, cuando hacemos conmemoración de su muerte en nuestro sacrificio, pedimos que venga el Espíritu Santo y nos comunique el amor; suplicamos fervorosamente que aquel mismo amor que impulsó a Cristo a dejarse crucificar por nosotros sea infundido por el Espíritu Santo en nuestro propios corazones, con objeto de que consideremos al mundo como crucificado para nosotros, y sepamos vivir crucificados para el mundo [\ldots] y, llenos de caridad, muertos para el pecado vivamos para Dios} (San Fulgencio de Ruspe, \textit{Contra gesta Fabiani} 28, 17-19).}

\n{1395} Por la misma caridad que enciende en nosotros, la Eucaristía nos \textit{preserva de futuros pecados mortales}. Cuanto más participamos en la vida de Cristo y más progresamos en su amistad, tanto más difícil se nos hará romper con Él por el pecado mortal. La Eucaristía no está ordenada al perdón de los pecados mortales. Esto es propio del sacramento de la Reconciliación. Lo propio de la Eucaristía es ser el sacramento de los que están en plena comunión con la Iglesia.

\n{1396} \textit{La unidad del Cuerpo místico: La Eucaristía hace la Iglesia}. Los que reciben la Eucaristía se unen más estrechamente a Cristo. Por ello mismo, Cristo los une a todos los fieles en un solo cuerpo: la Iglesia. La comunión renueva, fortifica, profundiza esta incorporación a la Iglesia realizada ya por el Bautismo. En el Bautismo fuimos llamados a no formar más que un solo cuerpo (cf. \textit{1 Co} 12,13). La Eucaristía realiza esta llamada: \textquote{El cáliz de bendición que bendecimos ¿no es acaso comunión con la sangre de Cristo? y el pan que partimos ¿no es comunión con el Cuerpo de Cristo? Porque aun siendo muchos, un solo pan y un solo cuerpo somos, pues todos participamos de un solo pan} (\textit{1 Co} 10,16-17):

\ccecite{\textquote{Si vosotros mismos sois Cuerpo y miembros de Cristo, sois el sacramento que es puesto sobre la mesa del Señor, y recibís este sacramento vuestro. Respondéis \textquote{Amén} [es decir, \textquote{sí}, \textquote{es verdad}] a lo que recibís, con lo que, respondiendo, lo reafirmáis. Oyes decir \textquote{el Cuerpo de Cristo}, y respondes \textquote{amén}. Por lo tanto, sé tú verdadero miembro de Cristo para que tu \textquote{amén} sea también verdadero} (San Agustín, \textit{Sermo} 272).}

\n{1397} \textit{La Eucaristía entraña un compromiso en favor de los pobres:} Para recibir en la verdad el Cuerpo y la Sangre de Cristo entregados por nosotros debemos reconocer a Cristo en los más pobres, sus hermanos (cf. \textit{Mt} 25,40):

\ccecite{\textquote{Has gustado la sangre del Señor y no reconoces a tu hermano. [\ldots] Deshonras esta mesa, no juzgando digno de compartir tu alimento al que ha sido juzgado digno [\ldots] de participar en esta mesa. Dios te ha liberado de todos los pecados y te ha invitado a ella. Y tú, aún así, no te has hecho más misericordioso} (San Juan Crisóstomo, hom. in 1 Co 27,4).}

\n{1398} \textit{La Eucaristía y la unidad de los cristianos}. Ante la grandeza de esta misterio, san Agustín exclama: \textit{O sacramentum pietatis! O signum unitatis! O vinculum caritatis!} – \textquote{¡Oh sacramento de piedad, oh signo de unidad, oh vínculo de caridad!} (\textit{In Iohannis evangelium tractatus} 26,13; cf. SC 47). Cuanto más dolorosamente se hacen sentir las divisiones de la Iglesia que rompen la participación común en la mesa del Señor, tanto más apremiantes son las oraciones al Señor para que lleguen los días de la unidad completa de todos los que creen en Él.


\newpage
\n{1399} Las Iglesias orientales que no están en plena comunión con la Iglesia católica celebran la Eucaristía con gran amor. \textquote{Estas Iglesias, aunque separadas, [tienen] verdaderos sacramentos [\ldots] y sobre todo, en virtud de la sucesión apostólica, el sacerdocio y la Eucaristía, con los que se unen aún más con nosotros con vínculo estrechísimo} (UR 15). Una cierta comunión \textit{in sacris}, por tanto, en la Eucaristía, \textquote{no solamente es posible, sino que se aconseja\ldots en circunstancias oportunas y aprobándolo la autoridad eclesiástica} (UR 15, cf. CIC can. 844, § 3).

\n{1400} Las comunidades eclesiales nacidas de la Reforma, separadas de la Iglesia católica, \textquote{sobre todo por defecto del sacramento del orden, no han conservado la sustancia genuina e íntegra del misterio eucarístico} (UR 22). Por esto, para la Iglesia católica, la intercomunión eucarística con estas comunidades no es posible. Sin embargo, estas comunidades eclesiales \textquote{al conmemorar en la Santa Cena la muerte y la resurrección del Señor, profesan que en la comunión de Cristo se significa la vida, y esperan su venida gloriosa} (UR 22).

\n{1401} Si, a juicio del Ordinario, se presenta una necesidad grave, los ministros católicos pueden administrar los sacramentos (Eucaristía, Penitencia, Unción de los enfermos) a cristianos que no están en plena comunión con la Iglesia católica, pero que piden estos sacramentos con deseo y rectitud: en tal caso se precisa que profesen la fe católica respecto a estos sacramentos y estén bien dispuestos (cf. CIC, can. 844, § 4).

\n{2837} [Nuestro Pan] \textquote{\textit{De cada día}}. La palabra griega, \textit{epiousion}, no tiene otro sentido en el Nuevo Testamento. Tomada en un sentido temporal, es una repetición pedagógica de \textquote{hoy} (cf. \textit{Ex} 16, 19-21) para confirmarnos en una confianza \textquote{sin reserva}. Tomada en un sentido cualitativo, significa lo necesario a la vida, y más ampliamente cualquier bien suficiente para la subsistencia (cf. \textit{1 Tm} 6, 8). Tomada al pie de la letra (\textit{epiousion}: \textquote{lo más esencial}), designa directamente el Pan de Vida, el Cuerpo de Cristo, \textquote{remedio de inmortalidad} (San Ignacio de Antioquía, \textit{Epistula ad Ephesios,} 20, 2) sin el cual no tenemos la Vida en nosotros (cf. \textit{Jn} 6, 53-56) Finalmente, ligado a lo que precede, el sentido celestial es claro: este \textquote{día} es el del Señor, el del Festín del Reino, anticipado en la Eucaristía, en que pregustamos el Reino venidero. Por eso conviene que la liturgia eucarística se celebre \textquote{cada día}.

\ccecite{\textquote{La Eucaristía es nuestro pan cotidiano [\ldots] La virtud propia de este divino alimento es una fuerza de unión: nos une al Cuerpo del Salvador y hace de nosotros sus miembros para que vengamos a ser lo que recibimos [\ldots] Este pan cotidiano se encuentra, además, en las lecturas que oís cada día en la Iglesia, en los himnos que se cantan y que vosotros cantáis. Todo eso es necesario en nuestra peregrinación} (San Agustín, \textit{Sermo} 57, 7, 7).}

El Padre del cielo nos exhorta a pedir como hijos del cielo el Pan del cielo (cf. \textit{Jn} 6, 51). Cristo \textquote{mismo es el pan que, sembrado en la Virgen, florecido en la Carne, amasado en la Pasión, cocido en el Horno del sepulcro, reservado en la iglesia, llevado a los altares, suministra cada día a los fieles un alimento celestial} (San Pedro Crisólogo, \textit{Sermo} 67, 7).
\end{ccebody}


\newpage
\cceth{La Eucaristía \textquote{prenda de la gloria futura}}
\cceref{CEC 1402-1405}

\begin{ccebody}
\ccesec{La Eucaristía, \textquote{Pignus futurae gloriae}}

\n{1402} En una antigua oración, la Iglesia aclama el misterio de la Eucaristía: \textit{O sacrum convivium in quo Christus sumitur. Recolitur memoria passionis Eius; mens impletur gratia et futurae gloriae nobis pignus datur} – \textquote{¡Oh sagrado banquete, en que Cristo es nuestra comida; se celebra el memorial de su pasión; el alma se llena de gracia, y se nos da la prenda de la gloria futura!} (\textit{Solemnidad del Santísimo Cuerpo y Sangre de Cristo}, Antífona del \textquote{Magnificat} para las II Vísperas: \textit{Liturgia de las Horas}). Si la Eucaristía es el memorial de la Pascua del Señor y si por nuestra comunión en el altar somos colmados \textquote{de gracia y bendición} (\textit{Plegaria Eucarística I o Canon Romano} 96: \textit{Misal Romano}), la Eucaristía es también la anticipación de la gloria celestial.

\n{1403} En la última Cena, el Señor mismo atrajo la atención de sus discípulos hacia el cumplimiento de la Pascua en el Reino de Dios: \textquote{Y os digo que desde ahora no beberé de este fruto de la vid hasta el día en que lo beba con vosotros, de nuevo, en el Reino de mi Padre} (\textit{Mt} 26,29; cf. \textit{Lc} 22,18; \textit{Mc} 14,25). Cada vez que la Iglesia celebra la Eucaristía recuerda esta promesa y su mirada se dirige hacia \textquote{el que viene} (\textit{Ap} 1,4). En su oración, implora su venida: \textit{Marana tha} (\textit{1 Co} 16,22), \textquote{Ven, Señor Jesús} (\textit{Ap} 22,20), \textquote{que tu gracia venga y que este mundo pase} (\textit{Didaché} 10,6).

\n{1404} La Iglesia sabe que, ya ahora, el Señor viene en su Eucaristía y que está ahí en medio de nosotros. Sin embargo, esta presencia está velada. Por eso celebramos la Eucaristía \textit{expectantes beatam spem et adventum Salvatoris nostri Jesu Christi} – \textquote{Mientras esperamos la gloriosa venida de Nuestro Salvador Jesucristo} (\textit{Ritual de la Comunión}, 126 [Embolismo después del \textquote{Padrenuestro}]: \textit{Misal Romano}; cf. \textit{Tit} 2,13), pidiendo entrar \textquote{[en tu Reino], donde esperamos gozar todos juntos de la plenitud eterna de tu gloria; allí enjugarás las lágrimas de nuestros ojos, porque, al contemplarte como Tú eres, Dios nuestro, seremos para siempre semejantes a ti y cantaremos eternamente tus alabanzas, por Cristo, Señor Nuestro} (\textit{Plegaria Eucarística III}, 116: \textit{Misal Romano}).

\n{1405} De esta gran esperanza, la de los cielos nuevos y la tierra nueva en los que habitará la justicia (cf. \textit{2 P} 3,13), no tenemos prenda más segura, signo más manifiesto que la Eucaristía. En efecto, cada vez que se celebra este misterio, \textquote{se realiza la obra de nuestra redención} (LG 3) y \textquote{partimos un mismo pan [\ldots] que es remedio de inmortalidad, antídoto para no morir, sino para vivir en Jesucristo para siempre} (San Ignacio de Antioquía, \textit{Epistula ad Ephesios}, 20, 2).


\newpage
\cceth{La institución del sacerdocio en la Última Cena} 
\cceref{CEC 611, 1366}

\n{611} La Eucaristía que instituyó en este momento será el \textquote{memorial} (\textit{1 Co} 11, 25) de su sacrificio. Jesús incluye a los Apóstoles en su propia ofrenda y les manda perpetuarla (cf. \textit{Lc} 22, 19). Así Jesús instituye a sus apóstoles sacerdotes de la Nueva Alianza: \textquote{Por ellos me consagro a mí mismo para que ellos sean también consagrados en la verdad} (\textit{Jn} 17, 19; cf. Concilio de Trento: DS, 1752; 1764).

\n{1366} La Eucaristía es, pues, un sacrificio porque \textit{representa} (= hace presente) el sacrificio de la cruz, porque es su \textit{memorial} y \textit{aplica} su fruto:

\ccecite{\textquote{(Cristo), nuestro Dios y Señor [\ldots] se ofreció a Dios Padre [\ldots] una vez por todas, muriendo como intercesor sobre el altar de la cruz, a fin de realizar para ellos (los hombres) la redención eterna. Sin embargo, como su muerte no debía poner fin a su sacerdocio (\textit{Hb} 7,24.27), en la última Cena, \textquote{la noche en que fue entregado} (\textit{1 Co}11,23), quiso dejar a la Iglesia, su esposa amada, un sacrificio visible (como lo reclama la naturaleza humana) [\ldots] donde se representara el sacrificio sangriento que iba a realizarse una única vez en la cruz, cuya memoria se perpetuara hasta el fin de los siglos (\textit{1 Co} 11,23) y cuya virtud saludable se aplicara a la remisión de los pecados que cometemos cada día} (Concilio de Trento: DS 1740).}
\end{ccebody}

\begin{patercite}
(\ldots) En el gesto de lavar los pies a sus discípulos, Jesús revela la profundidad del amor de Dios por el hombre: ¡en Él, Dios mismo se pone al servicio de los hombres! Él revela al mismo tiempo el sentido de la vida cristiana y, con mayor motivo, de la vida consagrada, que es \textit{vida de amor oblativo}, de concreto y generoso servicio. Siguiendo los pasos del Hijo del hombre, que \textquote{no ha venido a ser servido, sino a servir} (\textit{Mt} 20, 28), la vida consagrada se ha caracterizado por este \textquote{lavar los pies}, es decir, por el servicio, especialmente a los más pobres y necesitados. Ella, por una parte, contempla el misterio sublime del Verbo en el seno del Padre (cf. \textit{Jn} 1, 1), mientras que, por otra, sigue al mismo Verbo que se hace carne (cf. \textit{Jn} 1, 14), se abaja, se humilla para servir a los hombres. (\ldots) Él llama continuamente a nuevos discípulos para comunicarles, mediante la efusión del Espíritu (cf. \textit{Rm} 5, 5), el \textit{ágape} divino, su modo de amar, apremiándolos a servir a los demás en la entrega humilde de sí mismos, lejos de cualquier cálculo interesado. A Pedro que, extasiado ante la luz de la Transfiguración, exclama: \textquote{Señor, bueno es estarnos aquí} (\textit{Mt} 17, 4), le invita a volver a los caminos del mundo para continuar sirviendo el Reino de Dios: \textquote{Desciende, Pedro; tú, que deseabas descansar en el monte, desciende y predica la Palabra, insiste a tiempo y a destiempo, arguye y exhorta, increpa con toda longanimidad y doctrina. Trabaja, suda, padece algunos tormentos a fin de llegar, por el brillo y hermosura de las obras hechas en caridad, a poseer eso que simbolizan los blancos vestidos del Señor}. La mirada fija en el rostro del Señor no atenúa en el apóstol el compromiso por el hombre; más bien lo potencia, capacitándole para incidir mejor en la historia y liberarla de todo lo que la desfigura.  (\ldots)

\textbf{San Juan Pablo II,} papa, \textit{Vita Consecrata}, n. 75.
\end{patercite}