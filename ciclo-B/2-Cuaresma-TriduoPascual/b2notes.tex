%% LaTeX2e file `b2notes.tex'
%% generated by the `filecontents' environment
%% from source `_homilias-b2-main' on 2021/02/04.
%%
% \anotecontent{id1}{\label{id1}This is the footnote with id1.}

\anotecontent{id1}{\label{id1}A partir de esa nueva traducción de la Biblia se publicaron los nuevos Leccionarios de la Misa, que son los leccionarios oficiales desde el mes de septiembre del año 2016.}

\anotecontent{id2}{\label{id2}Cf. NUALC nn. 27-31.}

\anotecontent{id3}{Cf. Concilio Vaticano II, Const. \textit{Sacrosanctum Concilium} sobre la sagrada Liturgia, n. 109.}

\anotecontent{id4}{Cf. Pablo VI, Const. Apost. \textit{P}œ\textit{nitemini}, del 17 de febrero de 1966, II, párr. 3: A.A.S. 58 (1966) p. 184.}

\anotecontent{id5}{Por motivos pastorales, esta celebración podría tener lugar otro día durante la Semana Santa.}

\anotecontent{id6}{\textit{Prenotandos} del Leccionario de la Misa, n. 97.}

\anotecontent{id7}{Cf. Congregación para el Culto Divino y la Disciplina de los Sacramentos, \textit{Directorio sobre la piedad popular y la liturgia}, Ciduad del Vaticano (2002), nn. 124-139.}

\anotecontent{id8}{El Papa hace referencia al pasaje de la Transfiguración según san Mateo, porque aún no se había reformado el Leccionario de la Misa, ocurrido dos años después, en 1969.}

\anotecontent{id9}{Cf. NUALC nn. 18-21.}

\anotecontent{id10}{Cf. Congregación para el Culto Divino y la Disciplina de los Sacramentos, \textit{Directorio sobre la piedad popular y la liturgia}, Ciduad del Vaticano (2002), nn. 140-151.}

\anotecontent{id11}{Cf. Ceremonial de los Obispos, n. 297.}

\anotecontent{id12}{Cf. Ceremonial de los Obispos, n. 312.}

\anotecontent{id13}{Concilio Vaticano II, Constitución sobre la sagrada liturgia, \textit{Sacrosanctum Concilium}, n. 5.}

\anotecontent{id14}{Cf. Ceremonial de los Obispos, nn. 332-335.}

\anotecontent{id15}{San Agustín, \textit{Sermo} 219: PL 38, 108.}

\anotecontent{id16}{Les aseguro que si ellos callan, gritarán las piedras.}

\anotecontent{id17}{El Papa evoca los principales lugares del mundo donde actualmente hay conflictos.}

\anotecontent{id18}{El Papa recuerda los principales conflictos en el mundo.}

\anotecontent{id19}{El Papa recuerda los principales conflictos en el mundo, pidiendo por la paz.}

\anotecontent{id20}{El Papa evoca los principales conflictos en el mundo.}