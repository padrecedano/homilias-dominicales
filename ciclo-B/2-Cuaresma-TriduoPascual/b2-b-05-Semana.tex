\chapter{Domingo V de Cuaresma (B)}

\section{Lecturas}

\rtitle{PRIMERA LECTURA}

\rbook{Del libro del profeta Jeremías} \rred{31, 31-34}

\rtheme{Haré una alianza nueva y no recordaré los pecados}

\begin{scripture}
Ya llegan días –oráculo del Señor– en que haré con la casa de Israel y la casa de Judá una alianza nueva. No será una alianza como la que hice con sus padres, cuando los tomé de la mano para sacarlos de Egipto, pues quebrantaron mi alianza, aunque yo era su Señor –oráculo del Señor–.

Esta será la alianza que haré con ellos después de aquellos días –oráculo del Señor–: Pondré mi ley en su interior y la escribiré en sus corazones; yo seré su Dios y ellos serán mi pueblo. Ya no tendrán que enseñarse unos a otros diciendo: \textquote{Conoced al Señor}, pues todos me conocerán, desde el más pequeño al mayor –oráculo del Señor–, cuando perdone su culpa y no recuerde ya sus pecados.
\end{scripture}


%\begin{centering}
%	\vfill
	\img{px300}
%	\vfill
%\end{centering}


	
\newpage 
\rtitle{SALMO RESPONSORIAL}

\rbook{Salmo} \rred{50, 3-4. 12-13. 14-15}

\rtheme{Oh, Dios, crea en mí un corazón puro}

\begin{psbody}
Misericordia, Dios mío, por tu bondad,
por tu inmensa compasión borra mi culpa;
lava del todo mi delito,
limpia mi pecado. 

Oh, Dios, crea en mí un corazón puro,
renuévame por dentro con espíritu firme.
No me arrojes lejos de tu rostro,
no me quites tu santo espíritu. 

Devuélveme la alegría de tu salvación,
afiánzame con espíritu generoso.
Enseñaré a los malvados tus caminos,
los pecadores volverán a ti. 
\end{psbody}

\rtitle{SEGUNDA LECTURA}

\rbook{De la carta a los Hebreos} \rred{5, 7-9}

\rtheme{Aprendió a obedecer; y se convirtió en autor de salvación eterna}

\begin{scripture}
Cristo, en los días de su vida mortal, a gritos y con lágrimas, presentó oraciones y súplicas al que podía salvarlo de la muerte, siendo escuchado por su piedad filial.

Y, aun siendo Hijo, aprendió, sufriendo, a obedecer. Y, llevado a la consumación, se convirtió, para todos los que lo obedecen, en autor de salvación eterna.
\end{scripture}

\newpage 
\rtitle{EVANGELIO}

\rbook{Del Evangelio según san Juan} \rred{12, 20-33}

\rtheme{Si el grano de trigo cae en tierra y muere, da mucho fruto}

\begin{scripture}
En aquel tiempo, entre los que habían venido a celebrar la fiesta había algunos griegos; estos, acercándose a Felipe, el de Betsaida de Galilea, le rogaban:

\>{Señor, queremos ver a Jesús}.

Felipe fue a decírselo a Andrés; y Andrés y Felipe fueron a decírselo a Jesús.

Jesús les contestó:

\>{Ha llegado la hora de que sea glorificado el Hijo del hombre. \\En verdad, en verdad os digo: si el grano de trigo no cae en tierra y muere, queda infecundo; pero si muere, da mucho fruto. El que se ama a sí mismo, se pierde, y el que se aborrece a sí mismo en este mundo, se guardará para la vida eterna. El que quiera servirme, que me siga, y donde esté yo, allí también estará mi servidor; a quien me sirva, el Padre lo honrará. \\Ahora mi alma está agitada, y ¿qué diré? ¿Padre, líbrame de esta hora? Pero si por esto he venido, para esta hora: Padre, glorifica tu nombre}.

Entonces vino una voz del cielo:

\>{Lo he glorificado y volveré a glorificarlo}.

La gente que estaba allí y lo oyó, decía que había sido un trueno; otros decían que le había hablado un ángel.

Jesús tomó la palabra y dijo:

\>{Esta voz no ha venido por mí, sino por vosotros. Ahora va a ser juzgado el mundo; ahora el príncipe de este mundo va a ser echado fuera. Y cuando yo sea elevado sobre la tierra, atraeré a todos hacia mí}.

Esto lo decía dando a entender la muerte de que iba a morir.
\end{scripture}


\newsection
\section{Comentario Patrístico}

\subsection{San Cirilo de Alejandría, obispo}

\ptheme{Cristo brotó en medio de nosotros como una espiga de trigo; murió~y~produce~mucho~fruto}

\src{Comentario sobre el libro de los Números, 2: PG 69, 619-623.}

\begin{body}
\ltr{C}{risto} fue la primicia de este trigo, Él el único que escapó de la maldición, precisamente cuando quiso hacerse maldición por nosotros. Es más, venció incluso a los agentes de la corrupción, volviendo por sí mismo a la existencia \textit{libre entre los muertos}. De hecho resucitó derrotando la muerte, y subió al Padre como don ofrecido, cual primicia de la naturaleza humana, renovada en la incorruptibilidad. Efectivamente, \textit{Cristo ha entrado no en un santuario construido por hombres –imagen del auténtico–, sino en el mismo cielo, para ponerse ante Dios, intercediendo por nosotros}.

Que Cristo sea aquel pan de vida bajado del cielo; que además perdone los pecados y libere a los hombres de sus transgresiones ofreciéndose a sí mismo a Dios Padre como víctima de suave olor, lo podrás comprender perfectamente si, con los ojos de la mente, lo contemplas como aquel novillo sacrificado y como aquel macho cabrío inmolado por los pecados del pueblo. Cristo, en efecto, ofreció su vida por nosotros, para cancelar los pecados del mundo.

Por lo tanto, así como en el pan vemos a Cristo como vida y dador de vida, en el novillo lo vemos inmolado, ofreciéndose nuevamente a Dios Padre en olor de suavidad; y en la figura del macho cabrío lo contemplamos convertido por nosotros en pecado y en víctima por los pecados, así también podemos considerarlo como una gavilla de trigo. Qué puede representar esta gavilla, os lo explicaré en pocas palabras.

El género humano puede ser comparado a las espigas de un campo: nace en cierto modo de la tierra, se desarrolla buscando su normal crecimiento, y es segado en el momento en que la muerte lo cosecha. El mismo Cristo habló de esto a sus discípulos, diciendo: ¿No decís vosotros que faltan todavía cuatro meses para la cosecha? Yo os digo esto: Levantad los ojos y contemplad los campos, que están ya dorados para la siega; el segador ya está recibiendo el salario y almacenando fruto para la vida eterna.

Los habitantes de la tierra pueden, pues, compararse y con razón, a la mies de los campos. Y Cristo, modelado según nuestra naturaleza, nació de la Santísima Virgen cual espiga de trigo. En realidad, es el mismo Cristo quien se da el nombre de grano de trigo: \textit{Os aseguro, que si el grano de trigo no cae en tierra y muere, queda infecundo; pero si muere, da mucho fruto}. Por esta razón, Cristo se convirtió por nosotros en anatema, es decir, en algo consagrado y ofrecido al Padre, a la manera de una gavilla o como las primicias de la tierra. Una única espiga, pero considerada no aisladamente, sino unida a todos nosotros que, cual gavilla formada de muchas espigas, formamos un solo haz.

Pues bien, esta realidad es necesaria para nuestra utilidad y provecho y suple el símbolo del misterio. Pues Cristo Jesús es uno, pero puede ser considerado –y lo es realmente– como apretada gavilla, por cuanto contiene en sí a todos los creyentes, con una unión preferentemente espiritual. De lo contrario, ¿cómo por ejemplo hubiera podido escribir san Pablo: \textit{Nos ha resucitado con Cristo Jesús y nos ha sentado en el cielo con él}? Siendo él uno de nosotros, comulgamos con él en un mismo cuerpo y, mediante la carne, hemos conseguido la unión con él. Y ésta es la razón por la que, en otro lugar, él mismo dirige a Dios, Padre celestial, estas palabras: \textit{Padre, éste es mi deseo: que todos sean uno, como tú, Padre, en mí y yo en ti, que ellos también lo sean en nosotros}.
\end{body}

\begin{patercite}
	[\ldots]  Ellos tramaban contra mí, diciendo: \textit{Metamos un leño en su pan}. El pan de Jesús, del que nosotros nos alimentamos, es su palabra. Y como, cuando enseñaba, algunos intentaron poner obstáculos a su enseñanza, crucificándolo dijeron: \textit{Venid, metamos un leño en su pan}. A la palabra y a la enseñanza de Jesús le hicieron seguir la crucifixión del Maestro: éste es el leño metido en su pan. Ellos, es verdad, dijeron insidiosamente: \textit{Venid, metamos un leño en su pan}, pero yo voy a decir algo realmente maravilloso: el leño metido en su pan mejoró el pan. (\ldots)  \textit{Arranquémoslo de la tierra vital, que su nombre no se pronuncie más}. Lo mataron con la intención de erradicar totalmente su nombre. Pero Jesús sabe por qué y cómo morir. Por eso dice: \textit{Si el grano de trigo no cae en tierra y muere, queda infecundo}. Por tanto, la muerte de Jesucristo, cual espiga de trigo, produjo siete veces y mucho más de lo que se había sembrado. Pensemos por un momento en la eventualidad de que no hubiera sido crucificado ni, después de la muerte, descendido a los infiernos: el grano de trigo hubiera quedado solo y de él no habrían nacido otros. Presta mucha atención a las palabras divinas, para ver qué es lo que quieren darnos a entender: \textit{Si el grano de trigo no cae en tierra y muere, queda infecundo}. La muerte de Jesús dio como fruto todos éstos. Por tanto, si la muerte ha producido una cosecha tan abundante, ¿de qué abundancia no será portadora la resurrección?
	
	\textbf{Orígenes, presbítero}, \textit{Homilía} 10 sobre el libro del profeta Jeremías, cf. 1-3: PG 13, 358-362.
\end{patercite}





\newsection
\section{Homilías}

\subsection{San Juan Pablo II, papa}

\subsubsection{Homilía (1979): Queremos ver a Jesús}

\src{Visita Pastoral a la Parroquia Romana de San Buenaventura. \\1 de abril de 1979.}

\begin{body}
\textquote{Señor, queremos ver a Jesús} (\textit{Jn} 12, 21).

\ltr[1. ]{A}{sí} dijo a Felipe, que era de Betsaida, la gente que había llegado a Jerusalén de diversas partes. [Cuando aquí, en este lugar, en los límites de la gran Roma, donde hasta hace algún tiempo todo era solamente campo, llegó la gente de varias partes de Italia, parecía que dijesen lo mismo:] ¡Queremos ver a Cristo en medio de nosotros! Queremos que Él habite con nosotros; que aquí se levante su casa. Nos conocemos poco entre nosotros. Queremos que Él nos haga conocernos mutuamente, que nos haga acercarnos recíprocamente, para que ya no seamos extraños, sino que lleguemos a ser una comunidad\ldots

\txtsmall{[Así habló la gente que había llegado aquí de diversas partes de Italia. Así habéis hablado vosotros, queridos feligreses de esta parroquia joven de San Buenaventura de Bagnoregio.]} 

Y éstas, o parecidas, palabras son todavía actuales: se escuchan incluso ahora. 

\txtsmall{[Vuestra parroquia es muy joven. Nació aquí por vuestra fe, sobre este terreno hace poco todavía baldío. Y nació por vuestra firme voluntad de hacer habitar a Jesús en medio de vosotros. Y nació por la iniciativa que manifestasteis ante las autoridades eclesiásticas, e incluso ante las civiles. Gracias a ello surgió esta iglesia que sirve ya a vuestra comunidad cristiana. Y funcionan otros medios útiles para la vida parroquial. Sé bien que ya se ha realizado mucho trabajo con método y abnegación, a pesar de las muchas dificultades encontradas, y que deseáis continuar la hermosa obra desarrollándola según las líneas de un aumento progresivo que se amplíe cada día más para llegar a todas las necesidades de esta familia parroquial. El Papa os acompaña con su benevolencia y con su deseo paterno: ¡Queremos ver a Jesús!]

[\ldots]}

\newpage


3. Y ahora permitid que me refiera de nuevo a las lecturas litúrgicas de este domingo. El \textbf{profeta Jeremías} habla en la primera lectura de la alianza cada vez más estrecha que Dios quiere hacer con la casa de Israel. Dado que el pueblo de Israel no mantuvo la alianza precedente, Dios quiere constituir con él otra más sólida e interior: \textquote{Pondré mi ley en su interior y la escribiré en su corazón, y seré su Dios y ellos serán mi Pueblo} (\textit{Jer} 31, 33).

Queridos hermanos y hermanas: Dios ha realizado con nosotros la nueva y a la vez definitiva alianza en Jesucristo, que, como dice hoy \textbf{San Pablo}, \textquote{vino a ser para todos los que le obedecen causa de salud eterna} (\textit{Heb} 5, 9). Esta alianza se basa en la perfecta obediencia del Hijo al Padre. En virtud de esta obediencia, Cristo \textquote{fue escuchado} (\textit{Heb} 5, 7), y es escuchado siempre; Él mantiene ininterrumpidamente esta unión del hombre con Dios que se estableció en su cruz. \textquote{La Iglesia –como afirma el Concilio– es sacramento o signo e instrumento de la íntima unión con Dios y de la unidad de todo el género humano} (\textit{Lumen gentium}, 1). Vosotros que habéis formado aquí una célula viva de la Iglesia, esto es, vuestra parroquia, habéis expresado de modo particular esta alianza con Dios en la que queréis perseverar con la gracia de Jesucristo.

Si alguno os preguntase por qué lo habéis hecho, le podríais responder así, como dice hoy el \textbf{Profeta}: nosotros queremos que Él sea nuestro Dios y nosotros su Pueblo; queremos que sus leyes estén escritas en nuestro corazón.

Vosotros buscáis un apoyo para vuestros corazones y vuestras conciencias. Buscáis un apoyo para vuestras familias. Queréis que sean estables, que no se disuelvan; que constituyan esos hogares vivos del amor, en los cuales el hombre puede calentarse cada día. Perseverando en el vínculo sacramental del matrimonio, queréis transmitir la vida a vuestros hijos y, junto con la vida, la educación humana y cristiana. Cada uno de vosotros, queridos padres, advierte profundamente esta gran responsabilidad que está vinculada a la dignidad del padre y de la madre. Sabéis que de esto depende vuestra propia salvación y la de vuestros hijos. ¿Cómo soy padre? ¿Qué madre soy yo? He aquí las preguntas que os hacéis más de una vez. Vosotros os alegráis y yo con vosotros, de cada uno de los bienes que se manifiesta en vosotros, en vuestras familias, en vuestros hijos; me alegro con vosotros de sus progresos en la escuela, del desarrollo de sus conciencias jóvenes. Queréis que se hagan verdaderamente \textquote{hombres}. Y esto depende, en gran medida, de lo que adquieren en la casa paterna. Nadie puede sustituiros en esta obra. La sociedad, la nación, la Iglesia se construyen sobre la base de los fundamentos que echáis vosotros.


\newpage 
Miro a vuestros niños, a la juventud de vuestra parroquia. Están aquí presentes muy numerosos. [Es joven, verdaderamente joven esta parroquia.] Los niños, los jóvenes, ¡cuántas esperanzas ponen en la vida! ¡Y cuánta esperanza tenemos en ellos! Precisamente por esto es necesario que apoyemos fuertemente toda nuestra vida, y ante todo la vida familiar, sobre Jesucristo. Porque Él, que \textquote{vino a ser causa de salvación eterna para todos} (\textit{Heb} 5, 9), nos indica cada día los caminos de esta salvación. Con la palabra y el ejemplo nos enseña cómo debemos vivir. Nos muestra cuál es el sentido profundo y último de la vida humana.

Y si el hombre está seguro de este sentido de la vida, entonces todos los problemas, incluso los ordinarios y cotidianos, se resuelven en concordancia con él.

La vida se desarrolla entonces al mismo tiempo en el plano humano y divino.

Hoy oímos que el Señor Jesús preanuncia su muerte. Este es ya el V domingo de Cuaresma; estamos muy próximos a la Semana Santa, al triduo sacro que nos recordará nuevamente de modo particular su pasión, muerte y resurrección. Por esto las palabras con que el Señor anuncia su fin ya cercano hablan de la gloria: \textquote{Es llegada la hora en que el Hijo del hombre será glorificado\ldots Ahora mi alma se siente turbada. ¿Y qué diré?\ldots Padre, glorifica tu nombre} (\textit{Jn} 12, 23. 27-28). Y finalmente pronuncia las palabras que manifiestan tan profundamente el misterio de la muerte redentora: \textquote{Ahora es el juicio de este mundo\ldots Y yo, cuando sea levantado de la tierra, atraeré a todos hacia mí} (\textit{Jn} 12, 31-32). Esta elevación de Cristo sobre la tierra es anterior a la elevación en la gloria: elevación sobre el leño de la cruz, elevación de martirio, elevación de muerte.

Jesús preanuncia su muerte también en estas palabras misteriosas: \textquote{En verdad, en verdad os digo que, si el grano de trigo no cae en la tierra y muere, quedará solo; pero si muere, llevará mucho fruto} (\textit{Jn} 12, 24). Su muerte es prenda de la vida, es la fuente de la vida para todos nosotros. El Padre Eterno preordinó esta muerte en el orden de la gracia y de la salvación, igual que está establecida, en el orden de la naturaleza, la muerte del grano de trigo bajo la tierra, para que pueda despuntar la espiga dando fruto abundante. El hombre después se alimenta de este fruto que se hace pan cotidiano. También el sacrificio realizado en la muerte de Cristo se hace comida de nuestras almas bajo las apariencias de pan.

Preparémonos a vivir la Semana Santa, el triduo sacro, la muerte y la resurrección. Aceptemos esta vida cuya fuente es su sacrifico. Vivamos esta vida alimentándonos con la comida del Cuerpo y la Sangre del Redentor, crezcamos en ella para alcanzar la vida eterna.
\end{body}

\label{b2-03-05-1982A}
\newpage

\subsubsection{Homilía (1982): Sólo el amor atrae}

\src{28 de marzo de 1982.}

\begin{body}
1. \textquote{Si el grano de trigo no cae en tierra y muere, queda él solo; pero si muere, da mucho fruto} (\textit{Jn} 12, 24).

\ltr{Q}{ueridos} hermanos, con estas palabras de Jesús, narradas por el Evangelio según Juan y propuestas por la liturgia de este quinto domingo de Cuaresma, nos orientamos y acercamos más decididamente hacia la Semana Santa y la celebración de los misterios supremos de nuestra salvación. 

\txtsmall{[Precisamente hoy vosotros termináis la semana de Ejercicios espirituales anuales\ldots me alegra saludarlos y concluir así vuestra preparación para la próxima Pascua de Resurrección.]}

Cada año esta solemnidad única vuelve providencialmente para recordarnos y hacernos revivir el centro de la fe cristiana y, ciertamente, en los encuentros que habéis celebrado, vosotros también os habéis vuelto a confrontar con los misterios esenciales de nuestra fe, cuyo punto focal, pivote y fundamento se encuentra en la muerte y resurrección de Jesús. Y espero que hayáis tomado decisiones efectivas para vuestra vida individual, familiar y social.

2. La comparación del grano de trigo, formulada por Jesús, es sobre todo válida para él. De hecho, Él cayó al suelo. Él, sobre todo, murió, Él, por tanto, está cargado de abundantes y sabrosos frutos para la salvación de los hombres, para nuestra salvación. Ese grano se ha transformado verdaderamente en espiga, rico y fecundo, porque solo Jesús es el verdadero trigo que nos nutre y sustenta. Lo escuchamos de sus propios labios en el mismo Evangelio según Juan: \textquote{Yo soy el pan de vida; el que viene a mí no tendrá hambre jamás, y el que cree en mí, no tendrá sed jamás} (\textit{Jn} 6, 35). Es decir, Cristo es la respuesta a las preguntas y necesidades más profundas de nuestra alma y de nuestra vida. Responde a nuestras preguntas; ilumina nuestro camino; multiplica nuestras energías; en una palabra, satisface nuestra hambre y nuestra sed de vida eterna, colocándonos en una situación de comunión filial con Dios.

Pero todo esto lo hace a través de su muerte, que es una muerte en la cruz. También leemos estas palabras suyas en el \textbf{Evangelio}: \textquote{Cuando yo sea levantado de la tierra, atraeré a todos hacia mí. Esto lo dijo para indicar de qué tipo de muerte iba a morir} (\textit{Jn} 12, 33). Nuestra salvación pasa por su sacrificio. Y, en verdad, sólo una entrega total hecha con amor posee la fuerza de \textquote{atraer}, es decir, de subyugar nuestras mentes y corazones, casi para magnetizarnos, ya que verdaderamente \textquote{no hay amor más grande que este: dar la vida por los amigos} (\textit{Jn} 15, 13). Y eso es precisamente lo que Jesús hizo por nosotros.

\newpage 
3. Pero la comparación del grano de trigo también es válida para nosotros, como para todos los cristianos. De hecho, las palabras de Jesús continúan así: \textquote{El que quiera servirme, que me siga, y donde esté yo, allí también estará mi servidor} (\textit{Jn} 12, 26). Con nuestro Bautismo, todos hemos sido puestos en comunión de servicio con Jesús y para Jesús. Cada bautizado está llamado a vivir responsablemente en la Iglesia como sujeto activo, con plena conciencia de su dignidad de Hijo de Dios y de los deberes de su Testimonio cristiano, según el continuo progreso espiritual (cf. \textit{Ap} 2, 19).

La \textbf{primera lectura} bíblica, tomada del profeta Jeremías, nos recordó en términos muy claros: \textquote{Pondré mi ley en sus almas, la escribiré en sus corazones\ldots Todos me conocerán, desde el más pequeño hasta el más grande, dice el Señor} (\textit{Jer} 31, 33-34). Esto nos sucedió con el bautismo; pero todos los días estamos llamados a renovar nuestra identidad cristiana, mediante la reafirmación humilde y gozosa de nuestra fe y nuestra adhesión firme y viva al Señor Jesús. Frente a estas altísimas realidades, surge espontáneamente preguntarnos con absoluta sinceridad si realmente seguimos a Jesucristo dondequiera que vaya. \textquote{Donde yo esté, allí también estará mi siervo}. ¿En qué medida hacemos nuestra su total donación de amor? ¿Hasta qué punto mostramos nuestro interés por él, por los demás, por la Iglesia, como lo demostró él en la cruz? De hecho, sólo con nuestro servicio, que también es siempre renuncia, podemos producir, como él, \textquote{mucho fruto}.

4. Queridos hermanos, la Palabra de Dios en la liturgia de hoy nos inspira estos pensamientos. Tratemos de transformarlos en oración, para que penetren cada vez más en nosotros.

\txtsmall{[(\ldots) quiero aprovechar esta oportunidad para agradeceros a todos la obra, escondida pero muy preciosa, que prestáis (\ldots) y, mientras os aseguro mi benevolencia, os animo paternalmente a continuar siempre con entusiasmo y laboriosidad en vuestro exigente servicio. Que el Señor, cuya bendición invoco, os recompense generosamente, asistiendo siempre con su gracia a cada uno de vosotros, a vuestros familiares y a todos vuestros seres queridos.]}
\end{body}

\label{b2-03-05-1982H}
\newpage

\subsubsection{Homilía (1991): La hora de Cristo}

\src{Visita Pastoral a la Parroquia Romana de San Mauricio Mártir en Acilia. \\17 de marzo de 1991.}

\begin{body}
\textquote{Queremos ver a Jesús} (\textit{Jn} 12, 21).

\ltr[1. ]{Q}{ueridos} hermanos y hermanas, estas palabras dirigidas al apóstol Felipe por algunos griegos, que subieron a Jerusalén junto con la multitud de judíos reunidos para celebrar la Pascua, atraen nuestra especial atención hoy, último domingo de Cuaresma.

En la ciudad santa, a la que Jesús fue por última vez, había mucha gente. Estaban los pequeños y los sencillos que lo recibieron con gozo, reconociendo en él al enviado en el nombre del Señor y el Rey de Israel. Están los fariseos y las demás autoridades judías, que le temen, decididas a eliminarlo porque lo consideran subversivo y por tanto \textquote{inconveniente} para sus planes. También hay algunos paganos, los griegos, que sienten curiosidad por verlo y aprender más sobre él y las obras que realizó, la última de las cuales, la resurrección de su amigo Lázaro, despertó asombro y revuelo en todos. 

\textquote{Queremos ver a Jesús}: en estas palabras hay una petición que va más allá del episodio narrado por el evangelista. De hecho, manifiestan una aspiración y una exigencia que corren a través de los siglos y brota más o menos explícitamente del corazón de muchos hombres que han oído hablar de Cristo, pero aún no lo han encontrado.

2. Jesús capta esta petición y ofrece una respuesta que revela su identidad mesiánica y señala el camino para entrar en la experiencia de su \textquote{misterio} Redentor, que da vida a quienes creen en él y se abren a su palabra y a su acción salvadora.

En su respuesta, Jesús se refiere al momento que se prepara para vivir y más precisamente a la hora que está por llegar y que representa el punto de llegada de su existencia y de toda la historia de la salvación. Es la hora de la Cruz.

Es la hora del juicio, de la derrota de Satanás, príncipe del mal, y por tanto del triunfo definitivo del infinito amor de Dios. Cristo, levantado de la tierra, en efecto, está suspendido en el cadalso de la Cruz y voluntariamente se ofrece a la muerte; pero también es exaltado por el Padre, en la resurrección, para atraer a todos hacia sí y, por tanto, reconciliar a los hombres con Dios y entre sí.

Es la hora de su sacrificio pascual, tomado no sólo como momento de sufrimiento y muerte, sino sobre todo en su dimensión más profunda e interior de don y ofrenda; como acto supremo de obediencia y glorificación del Padre y como fuente de redención para los que creen en Él. Es la hora, finalmente, en la que se cumplirá la nueva y eterna alianza, anunciada por los profetas y en particular por \textbf{Jeremías}. Una Alianza ya no escrita en tablas de piedra, como la antigua alianza, sino sellada por la sangre de Cristo y por el don del Espíritu, que Él derramará desde su costado desgarrado en el corazón de los creyentes, para que sean uno en Él y se conviertan en heraldos y testigos de su amor.

3. Jesús, sin embargo, se propone a todos como Salvador, no sólo de palabra, sino también a través de una imagen sencilla y sugerente, rica también en implicaciones vitales para quienes se adherirán a él con fe y se convertirán en sus discípulos. Se trata de la imagen del \textbf{grano de trigo} que, enterrado en la tierra, muere para dar fruto.

Al respecto, un Padre de la Iglesia comenta: \textquote{En el grano de trigo nuestro Señor y Salvador indica su cuerpo. Cuando fue sepultado en la tierra, trajo abundantes frutos, porque la resurrección hizo brotar por toda la tierra frutos de virtudes y cosechas de pueblos fieles}. Por tanto, a imagen del grano de trigo es posible captar otro aspecto del misterio pascual de Cristo: el de la fecundidad del acontecimiento del que es protagonista. De su muerte y resurrección, en efecto, emana la fuerza del Espíritu, el agua viva de la gracia, que da lugar al pueblo de la nueva alianza, la Iglesia, una comunidad pascual abierta a todos los hombres.

Pero la muerte de Cristo, vista como el grano caido en tierra del que brota la vida, se convierte también en paradigma de la existencia del discípulo, llamado a seguirlo por el mismo camino, a hacer lo que él hizo, para convertirse así en colaborador y servidor del designio divino de la salvación universal. Y todos aquellos que han experimentado el poder de la Cruz pueden dar testimonio de ello. Muchos de vostros, al menos una vez, ya han experimentado el poder de la Cruz de Cristo.

4. Queridos hermanos y hermanas [de la parroquia de San Mauricio en Acilia,] estad atentos y disponibles a la \textquote{lección de vida} que os da el mensaje de la liturgia de hoy, especialmente en lo que respecta a la misión de evangelización a la que está llamada vuestra comunidad y toda la Iglesia de [Roma con el Sínodo pastoral diocesano].

Abrid los ojos a la realidad de vuestro barrio: recoged las preguntas que surgen del corazón de hombres y mujeres, especialmente de los jóvenes, que quieren \textquote{ver a Cristo} para que podáis darles las respuestas que os ofrece el Evangelio.

\txtsmall{[Con estas exhortaciones (\ldots) os expreso mi cordial saludo y agradezco vuestra participación en esta celebración eucarística, con motivo de mi visita a esta parroquia, que está en camino de convertirse en una comunidad cada vez más unida y solidaria, tanto en la expresión de la fe cristiana como en las iniciativas de carácter social.]

[\ldots]}

\newpage 
Os exhorto a profundizar cada vez más en el conocimiento y el amor de Jesús Nuestro Señor, a quien en estos días recordamos en su pasión, muerte y resurrección, es decir, en el misterio de la Pascua; procurad acudir a las celebraciones litúrgicas, sed asiduos participando en la Santa Misa; estad disponibles generosamente para ayudar concretamente a los que sufren y a los necesitados.

Anunciad a todos que Dios nos ama y está presente en el corazón del hombre y en la historia humana, aunque esté marcada por tantas contradicciones y laceraciones. Despertad en todos la cuestión religiosa, muchas veces latente, anunciando con valentía que en Cristo Jesús Dios ha revelado el rostro de su Padre y nos ha ofrecido una salvación plena y definitiva. Proclamad sin miedo y con libertad plena que Cristo es el camino, la verdad y la vida; raíz y fuente de libertad, justicia, solidaridad y paz.

5. Queridos amigos, la \textquote{buena noticia} que Jesús trajo al mundo es su cruz gloriosa, la expresión suprema de su vida dada por amor.

Haced vosotros lo mismo. Jesús lo pide con particular insistencia y fuerza a quienes quieren ser sus discípulos y testigos y, por tanto, ponerse al servicio de su Reino.

Esto es lo que nos recuerda el \textbf{Evangelio} de este domingo: \textquote{El que ama su vida, la pierde y el que odia su vida en este mundo, la guardará para una vida eterna. Si alguno quiere servirme, que me siga y donde yo esté, también estará mi servidor}.

El Señor Jesús os pide que compartáis sus elecciones, que déis testimonio de su caridad, que os convirtáis en granos de trigo, listos para morir al egoísmo, el orgullo y a la mentira para resucitar a una nueva vida en el Espíritu. En una palabra, os llama a la cita de la Cruz, donde el amor es capaz de entregarse totalmente y a todos.

Esta es la Pascua de Jesús, esta Pascua se acerca. Deseo que viváis profundamente, en las próximas semanas, este gran misterio de Jesús, misterio de la Iglesia, misterio nuestro. ¡Amén!
\end{body}

\label{b2-03-05-1991H}
\newpage

\subsubsection{Ángelus (1997): Quisiéramos ver a Jesús}

\src{16 de marzo de 1997.}

\begin{body}
\ltr[1. ]{E}{n} el \textbf{Evangelio} de este quinto domingo de Cuaresma, Jesús explica el sentido de su muerte sirviéndose de la imagen del grano de trigo que, muriendo, da fruto (cf. \textit{Jn} 12, 24). La ocasión para esta reflexión se la ofrece el hecho de que, entre la multitud que fue a recibirlo mientras se acercaba a Jerusalén, había también extranjeros, precisamente algunos griegos, que manifestaron a los Apóstoles su deseo de verlo: \textquote{Quisiéramos ver a Jesús} (\textit{Jn} 12, 21). Con estas palabras, se hacen en cierto modo portavoces de toda la humanidad, destacando el valor universal de la salvación ofrecida por Jesús.

2. ¡Quisiéramos ver a Jesús! Este es el grito que la humanidad dirige también hoy a los discípulos de Cristo, pidiéndoles que muestren, con su vida y sus obras, el rostro divino. Lo acogemos con emoción, sabiendo que, como dice el apóstol Pablo, llevamos un tesoro en \textquote{recipientes de barro} (\textit{2 Co} 4, 7). No ignoramos que la historia cristiana, aunque es tan rica en santidad, muestra también mucha fragilidad humana. El Concilio ha observado que, con frecuencia, precisamente la incoherencia de los creyentes constituye un obstáculo en el camino de cuantos buscan al Señor (cf. \textit{Gaudium et spes}, 19). Por esta razón, el camino de la Iglesia [hacia el tercer milenio] tiene que ser un serio itinerario de conversión, un esfuerzo de renovación personal y comunitaria a la luz del Evangelio. [Este, y sólo este, debe ser el gran jubileo del año 2000.] Cuanto más se refleje Cristo en nuestra vida, tanto más mostrará la atracción irresistible que él mismo anunció hablando de su muerte en la cruz: \textquote{Cuando yo sea elevado sobre la tierra, atraeré a todos hacia mí} (\textit{Jn} 12, 32).

3. Señor Jesús, ¡da al mundo la paz! Amadísimos hermanos y hermanas, os invito, una vez más, a implorar al Señor la paz \txtsmall{[para Albania]. [La crisis que está sacudiendo a esa nación, que acaba de salir de un largo período de dictadura inhumana, ya se ha extendido a todo su territorio, sumiendo a esas amadas poblaciones en la falta total de seguridad. Por el bien de Albania, a todos los que han empuñado un arma, les pido que la depongan: ciertamente, la violencia destructora no es el medio adecuado para resolver los problemas sociales. Al contrario, cada uno tiene que sentirse comprometido a colaborar, dentro del respeto a las personas y al derecho, en el restablecimiento de la confianza entre los ciudadanos y sus autoridades. Todo esto no puede realizarse sin el orden público.

Ciertamente, estos acontecimientos trágicos interpelan a toda Europa, que debe ayudar a los gobernantes y al pueblo albanés a construir su país sobre la base de la democracia y el diálogo político y social.]}

Que la Virgen María, nuestra Señora del Buen Consejo, interceda para que la fuerza de las armas no triunfe sobre la paz, y para que la indiferencia no prevalezca sobre la solidaridad.
\end{body}


\newsection
\subsection{Benedicto XVI, papa}

\subsubsection{Homilía (2009): Era necesario que muriera}

\src{Visita Pastoral a la Parroquia Romana del Santo Rostro de Jesús en la Magliana. \\29 de marzo del 2009.}

\begin{body}
\ltr{E}{n} el pasaje evangélico de hoy, \textbf{san Juan} refiere un episodio que aconteció en la última fase de la vida pública de Cristo, en la inminencia de la Pascua judía, que sería su Pascua de muerte y resurrección. Narra el evangelista que, mientras se encontraba en Jerusalén, algunos griegos, prosélitos del judaísmo, por curiosidad y atraídos por lo que Jesús estaba haciendo, se acercaron a Felipe, uno de los Doce, que tenía un nombre griego y procedía de Galilea. \textquote{Señor –le dijeron–, queremos ver a Jesús} (\textit{Jn} 12, 21). Felipe, a su vez, llamó a Andrés, uno de los primeros apóstoles, muy cercano al Señor, y que también tenía un nombre griego; y ambos \textquote{fueron a decírselo a Jesús} (\textit{Jn} 12, 22).

En la petición de estos griegos anónimos podemos descubrir la sed de ver y conocer a Cristo que experimenta el corazón de todo hombre. Y la respuesta de Jesús nos orienta al misterio de la Pascua, manifestación gloriosa de su misión salvífica. \textquote{Ha llegado la hora de que sea glorificado el Hijo del hombre} (\textit{Jn} 12, 23). Sí, está a punto de llegar la hora de la glorificación del Hijo del hombre, pero esto conllevará el paso doloroso por la pasión y la muerte en cruz. De hecho, sólo así se realizará el plan divino de la salvación, que es para todos, judíos y paganos, pues todos están invitados a formar parte del único pueblo de la alianza nueva y definitiva. A esta luz comprendemos también la solemne proclamación con la que se concluye el pasaje evangélico: \textquote{Yo, cuando sea levantado de la tierra, atraeré a todos hacia mí} (\textit{Jn} 12, 32), así como el comentario del Evangelista: \textquote{Decía esto para significar de qué muerte iba a morir} (\textit{Jn} 12, 33). La cruz: la altura del amor es la altura de Jesús, y a esta altura nos atrae a todos.

Muy oportunamente la liturgia nos hace meditar este texto del evangelio de san Juan en este quinto domingo de Cuaresma, mientras se acercan los días de la Pasión del Señor, en la que nos sumergiremos espiritualmente desde el próximo domingo, llamado precisamente domingo de Ramos y de la Pasión del Señor. Es como si la Iglesia nos estimulara a compartir el estado de ánimo de Jesús, queriéndonos preparar para revivir el misterio de su crucifixión, muerte y resurrección, no como espectadores extraños, sino como protagonistas juntamente con él, implicados en su misterio de cruz y resurrección. De hecho, donde está Cristo, allí deben encontrarse también sus discípulos, que están llamados a seguirlo, a solidarizarse con él en el momento del combate, para ser asimismo partícipes de su victoria.

El Señor mismo nos explica cómo podemos asociarnos a su misión. Hablando de su muerte gloriosa ya cercana, utiliza una imagen sencilla y a la vez sugestiva: \textquote{Si el grano de trigo no cae en tierra y muere, queda él solo; pero si muere, da mucho fruto} (\textit{Jn} 12, 24). Se compara a sí mismo con un \textquote{grano de trigo deshecho, para dar a todos mucho fruto}, como dice de forma eficaz san Atanasio. Y sólo mediante la muerte, mediante la cruz, Cristo da mucho fruto para todos los siglos. De hecho, no bastaba que el Hijo de Dios se hubiera encarnado. Para llevar a cabo el plan divino de la salvación universal era necesario que muriera y fuera sepultado: sólo así toda la realidad humana sería aceptada y, mediante su muerte y resurrección, se haría manifiesto el triunfo de la Vida, el triunfo del Amor; así se demostraría que el amor es más fuerte que la muerte.

Con todo, el hombre Jesús, que era un hombre verdadero, con nuestros mismos sentimientos, sentía el peso de la prueba y la amarga tristeza por el trágico fin que le esperaba. Precisamente por ser hombre-Dios, experimentaba con mayor fuerza el terror frente al abismo del pecado humano y a cuanto hay de sucio en la humanidad, que él debía llevar consigo y consumar en el fuego de su amor. Todo esto él lo debía llevar consigo y transformar en su amor. \textquote{Ahora –confiesa– mi alma está turbada. Y ¿que voy a decir? ¿Padre, líbrame de esta hora?} (\textit{Jn} 12, 27). Le asalta la tentación de pedir: \textquote{Sálvame, no permitas la cruz, dame la vida}. En esta apremiante invocación percibimos una anticipación de la conmovedora oración de Getsemaní, cuando, al experimentar el drama de la soledad y el miedo, implorará al Padre que aleje de él el cáliz de la pasión.

Sin embargo, al mismo tiempo, mantiene su adhesión filial al plan divino, porque sabe que precisamente para eso ha llegado a esta hora, y con confianza ora: \textquote{Padre, glorifica tu nombre} (\textit{Jn} 12, 28). Con esto quiere decir: \textquote{Acepto la cruz}, en la que se glorifica el nombre de Dios, es decir, la grandeza de su amor. También aquí Jesús anticipa las palabras del Monte de los Olivos: \textquote{No se haga mi voluntad, sino la tuya} (\textit{Lc} 22, 42). Transforma su voluntad humana y la identifica con la de Dios. Este es el gran acontecimiento del Monte de los Olivos, el itinerario que deberíamos seguir fundamentalmente en todas nuestras oraciones: transformar, dejar que la gracia transforme nuestra voluntad egoísta y la impulse a uniformarse a la voluntad divina.

Los mismos sentimientos afloran en el pasaje de la carta a los Hebreos que se ha proclamado en la \textbf{segunda lectura}. Postrado por una angustia extrema a causa de la muerte que se cierne sobre él, Jesús ofrece a Dios ruegos y súplicas \textquote{con poderoso clamor y lágrimas} (\textit{Hb} 5, 7). Invoca ayuda de Aquel que puede liberarlo, pero abandonándose siempre en las manos del Padre. Y precisamente por esta filial confianza en Dios –nota el autor– fue escuchado, en el sentido de que resucitó, recibió la vida nueva y definitiva. La carta a los Hebreos nos da a entender que estas insistentes oraciones de Jesús, con clamor y lágrimas, eran el verdadero acto del sumo sacerdote, con el que se ofrecía a sí mismo y a la humanidad al Padre, transformando así el mundo.

Queridos hermanos y hermanas, este es el camino exigente de la cruz que Jesús indica a todos sus discípulos. En diversas ocasiones dijo: \textquote{Si alguno me quiere servir, sígame}. No hay alternativa para el cristiano que quiera realizar su vocación. Es la \textquote{ley} de la cruz descrita con la imagen del grano de trigo que muere para germinar a una nueva vida; es la \textquote{lógica} de la cruz de la que nos habla también el \textbf{pasaje evangélico} de hoy: \textquote{El que ama su vida, la pierde; y el que odia su vida en este mundo, la guardará para la vida eterna} (\textit{Jn} 12, 25). \textquote{Odiar} la propia vida es una expresión semítica fuerte y encierra una paradoja; subraya muy bien la totalidad radical que debe caracterizar a quien sigue a Cristo y, por su amor, se pone al servicio de los hermanos: pierde la vida y así la encuentra. No existe otro camino para experimentar la alegría y la verdadera fecundidad del Amor: el camino de darse, entregarse, perderse para encontrarse.

[Queridos amigos, la invitación de Jesús resuena de forma muy elocuente en la celebración de hoy en vuestra parroquia, pues está dedicada al] Santo Rostro de Jesús: el Rostro que \textquote{algunos griegos}, de los que habla el evangelio, deseaban ver; el Rostro que en los próximos días de la Pasión contemplaremos desfigurado a causa de los pecados, la indiferencia y la ingratitud de los hombres; el Rostro radiante de luz y resplandeciente de gloria, que brillará en el alba del día de Pascua. Mantengamos fijos el corazón y la mente en el Rostro de Cristo\ldots

Queridos hermanos y hermanas, dejaos iluminar por el esplendor del Rostro de Cristo\ldots Es importante que la oración, tanto personal como litúrgica, ocupe siempre el primer lugar en nuestra vida. A vosotros, queridos jóvenes, quiero dirigiros en particular unas palabras de aliento: dejaos atraer por la fascinación de Cristo. Contemplando su Rostro con los ojos de la fe, pedidle: \textquote{Jesús, ¿qué quieres que haga yo contigo y por ti?}. Luego, permaneced a la escucha y, guiados por su Espíritu, cumplid el plan que él tiene para cada uno de vosotros. Preparaos seriamente para construir familias unidas y fieles al Evangelio, y para ser sus testigos en la sociedad. Y si él os llama, estad dispuestos a dedicar totalmente vuestra vida a su servicio en la Iglesia como sacerdotes o como religiosos y religiosas. Yo os aseguro mi oración\ldots

Queridos hermanos y hermanas de esta comunidad parroquial, el amor infinito de Cristo que brilla en su Rostro resplandezca en todas vuestras actitudes, y se convierta en vuestra \textquote{cotidianidad}. Como exhortaba san Agustín en una homilía pascual, \textquote{Cristo padeció; muramos al pecado. Cristo resucitó; vivamos para Dios. Cristo pasó de este mundo al Padre; que no se apegue aquí nuestro corazón, sino que lo siga en las cosas de arriba. Nuestro jefe fue colgado de un madero; crucifiquemos la concupiscencia de la carne. Yació en el sepulcro; sepultados con él, olvidemos las cosas pasadas. Está sentado en el cielo; traslademos nuestros deseos a las cosas supremas} (\textit{Discurso} 229, D, 1).

Animados por esta convicción, prosigamos la celebración eucarística, invocando la intercesión maternal de María para que nuestra vida sea un reflejo de la de Cristo. Oremos para que todos aquellos con quienes nos encontremos perciban siempre en nuestros gestos y en nuestras palabras la bondad pacificadora y consoladora de su Rostro. Amén.
\end{body}

\label{b2-03-05-2012H}

\begin{patercite}
	Yo (\ldots) moriré de buena gana por Dios, con tal que vosotros no me lo impidáis. Os lo pido por favor: no me demostréis una benevolencia inoportuna. Dejad que sea pasto de las fieras, ya que ello me hará posible alcanzar a Dios. Soy trigo de Dios, y he de ser molido por los dientes de las fieras, para llegar a ser pan limpio de Cristo. Rogad por mí a Cristo, para que, por medio de esos instrumentos, llegue a ser una víctima para Dios.
	
	De nada me servirían los placeres terrenales ni los reinos de este mundo. Prefiero morir en Cristo Jesús que reinar en los confines de la tierra. Todo mi deseo y mi voluntad están puestos en aquel que por nosotros murió y resucitó. Se acerca ya el momento de mi nacimiento a la vida nueva. Por favor, hermanos, no me privéis de esta vida, no queráis que muera; si lo que yo anhelo es pertenecer a Dios, no me entreguéis al mundo ni me seduzcáis con las cosas materiales; dejad que pueda contemplar la luz pura; entonces seré hombre en pleno sentido. Permitid que imite la pasión de mi Dios. El que tenga a Dios en sí entenderá lo que quiero decir y se compadecerá de mí, sabiendo cuál es el deseo que me apremia.
	
	El príncipe de este mundo me quiere arrebatar y pretende arruinar mi deseo que tiende hacia Dios. Que nadie de vosotros, los aquí presentes, lo ayude; poneos más bien de mi parte, esto es, de parte de Dios. No queráis a un mismo tiempo tener a Jesucristo en la boca y los deseos mundanos en el corazón. Que no habite la envidia entre vosotros. Ni me hagáis caso si, cuando esté aquí, os suplicare en sentido contrario; haced más bien caso de lo que ahora os escribo. Porque os escribo en vida, pero deseando morir. Mi amor está crucificado y ya no queda en mí el fuego de los deseos terrenos; únicamente siento en mi interior la voz de una agua viva que me habla y me dice: \textquote{Ven al Padre}. No encuentro ya deleite en el alimento material ni en los placeres de este mundo. Lo que deseo es el pan de Dios, que es la carne de Jesucristo, de la descendencia de David, y la bebida de su sangre, que es la caridad incorruptible.
	
	[\ldots]
	
	\textbf{San Ignacio de Antioquía}, \textit{Carta a los Romanos}, cf. Caps 4, 1-2; 6, 1-8, 3: Funk 1, 217-223, Oficio de Lectura del 17 de octubre.
\end{patercite}




\newpage

\subsubsection{Homilía (2012): La gloria de Cristo}

\src{Viaje Apostólico a México y a la República de Cuba. \\Santa Misa en el Parque Expo Bicentenario de León. \\25 de marzo del 2012.}

\begin{body}
\ltr[«]{C}{rea} en mí, Señor, un corazón puro» (\textit{Sal} 50, 12), hemos invocado en el \textbf{salmo responsorial}. Esta exclamación muestra la profundidad con la que hemos de prepararnos para celebrar la próxima semana el gran misterio de la pasión, muerte y resurrección del Señor. Nos ayuda asimismo a mirar muy dentro del corazón humano, especialmente en los momentos de dolor y de esperanza a la vez, [como los que atraviesa en la actualidad el pueblo mexicano y también otros de Latinoamérica].

El anhelo de un corazón puro, sincero, humilde, aceptable a Dios, era muy sentido ya por Israel, a medida que tomaba conciencia de la persistencia del mal y del pecado en su seno, como un poder prácticamente implacable e imposible de superar. Quedaba sólo confiar en la misericordia de Dios omnipotente y la esperanza de que él cambiara desde dentro, desde el corazón, una situación insoportable, oscura y sin futuro. Así fue abriéndose paso el recurso a la misericordia infinita del Señor, que no quiere la muerte del pecador, sino que se convierta y viva (cf. \textit{Ez} 33, 11). Un corazón puro, un corazón nuevo, es el que se reconoce impotente por sí mismo, y se pone en manos de Dios para seguir esperando en sus promesas. De este modo, el salmista puede decir convencido al Señor: \textquote{Volverán a ti los pecadores} (Sal \textit{50}, 15). Y, hacia el final del salmo, dará una explicación que es al mismo tiempo una firme confesión de fe: \textquote{Un corazón quebrantado y humillado, tú no lo desprecias} (Sal \textit{50}, 19).

La historia de Israel narra también grandes proezas y batallas, pero a la hora de afrontar su existencia más auténtica, su destino más decisivo, la salvación, más que en sus propias fuerzas, pone su esperanza en Dios, que puede recrear un corazón nuevo, no insensible y engreído. Esto nos puede recordar hoy a cada uno de nosotros y a nuestros pueblos que, cuando se trata de la vida personal y comunitaria, en su dimensión más profunda, no bastarán las estrategias humanas para salvarnos. Se ha de recurrir también al único que puede dar vida en plenitud, porque él mismo es la esencia de la vida y su autor, y nos ha hecho partícipes de ella por su Hijo Jesucristo.

El \textbf{Evangelio} de hoy prosigue haciéndonos ver cómo este antiguo anhelo de vida plena se ha cumplido realmente en Cristo. Lo explica \textbf{san Juan} en un pasaje en el que se cruza el deseo de unos griegos de ver a Jesús y el momento en que el Señor está por ser glorificado. A la pregunta de los griegos, representantes del mundo pagano, Jesús responde diciendo: \textquote{Ha llegado la hora de que el Hijo del hombre sea glorificado} (\textit{Jn} 12, 23). Respuesta extraña, que parece incoherente con la pregunta de los griegos. ¿Qué tiene que ver la glorificación de Jesús con la petición de encontrarse con él? Pero sí que hay una relación. Alguien podría pensar –observa san Agustín– que Jesús se sentía glorificado porque venían a él los gentiles. Algo parecido al aplauso de la multitud que da \textquote{gloria} a los grandes del mundo, diríamos hoy. Pero no es así. \textquote{Convenía que a la excelsitud de su glorificación precediese la humildad de su pasión} (\textit{In Joannis Ev.}, 51, 9: PL 35, 1766).

La respuesta de Jesús, anunciando su pasión inminente, viene a decir que un encuentro ocasional en aquellos momentos sería superfluo y tal vez engañoso. Al que los griegos quieren ver en realidad, lo verán levantado en la cruz, desde la cual atraerá a todos hacia sí (cf. \textit{Jn} 12, 32). Allí comenzará su \textquote{gloria}, a causa de su sacrificio de expiación por todos, como el grano de trigo caído en tierra que muriendo, germina y da fruto abundante. Encontrarán a quien seguramente sin saberlo andaban buscando en su corazón, al verdadero Dios que se hace reconocible para todos los pueblos. Este es también el modo en que Nuestra Señora de Guadalupe mostró su divino Hijo a san Juan Diego. No como a un héroe portentoso de leyenda, sino como al verdaderísimo Dios, por quien se vive, al Creador de las personas, de la cercanía y de la inmediación, del Cielo y de la Tierra (cf. \textit{Nican Mopohua}, v. 33). Ella hizo en aquel momento lo que ya había ensayado en las Bodas de Caná. Ante el apuro de la falta de vino, indicó claramente a los sirvientes que la vía a seguir era su Hijo: \textquote{Hagan lo que él les diga} (\textit{Jn} 2, 5).

\txtsmall{[Queridos hermanos, al venir aquí he podido acercarme al monumento a Cristo Rey, en lo alto del Cubilete. Mi venerado predecesor, el beato Papa Juan Pablo II, aunque lo deseó ardientemente, no pudo visitar este lugar emblemático de la fe del pueblo mexicano en sus viajes a esta querida tierra. Seguramente se alegrará hoy desde el cielo de que el Señor me haya concedido la gracia de poder estar ahora con ustedes, como también habrá bendecido a tantos millones de mexicanos que han querido venerar sus reliquias recientemente en todos los rincones del país. Pues bien,]} en este monumento se representa a Cristo Rey. Pero las coronas que le acompañan, una de soberano y otra de espinas, indican que su realeza no es como muchos la entendieron y la entienden. Su reinado no consiste en el poder de sus ejércitos para someter a los demás por la fuerza o la violencia. Se funda en un poder más grande que gana los corazones: el amor de Dios que él ha traído al mundo con su sacrificio y la verdad de la que ha dado testimonio. Éste es su señorío, que nadie le podrá quitar ni nadie debe olvidar. \txtsmall{[Por eso es justo que, por encima de todo, este santuario sea un lugar de peregrinación, de oración ferviente, de conversión, de reconciliación, de búsqueda de la verdad y acogida de la gracia.]} A él, a Cristo, le pedimos que reine en nuestros corazones haciéndolos puros, dóciles, esperanzados y valientes en la propia humildad.

También hoy, \txtsmall{[desde este parque con el que se quiere dejar constancia del bicentenario del nacimiento de la nación mexicana, aunando en ella muchas diferencias, pero con un destino y un afán común,]} pidamos a Cristo un corazón puro, donde él pueda habitar como príncipe de la paz, gracias al poder de Dios, que es el poder del bien, el poder del amor. Y, para que Dios habite en nosotros, hay que escucharlo, hay que dejarse interpelar por su Palabra cada día, meditándola en el propio corazón, a ejemplo de María (cf. \textit{Lc} 2, 51). Así crece nuestra amistad personal con él, se aprende lo que espera de nosotros y se recibe aliento para darlo a conocer a los demás.

\txtsmall{[En Aparecida, los Obispos de Latinoamérica y el Caribe han sentido con clarividencia la necesidad de confirmar, renovar y revitalizar la novedad del Evangelio arraigada en la historia de estas tierras \textquote{desde el encuentro personal y comunitario con Jesucristo, que suscite discípulos y misioneros} (\textit{Documento conclusivo}, 11). La Misión Continental, que ahora se está llevando a cabo diócesis por diócesis en este Continente, tiene precisamente el cometido de hacer llegar esta convicción a todos los cristianos y comunidades eclesiales, para que resistan a la tentación de una fe superficial y rutinaria, a veces fragmentaria e incoherente. También aquí se ha de superar el cansancio de la fe y recuperar \textquote{la alegría de ser cristianos, de estar sostenidos por la felicidad interior de conocer a Cristo y de pertenecer a su Iglesia. De esta alegría nacen también las energías para servir a Cristo en las situaciones agobiantes de sufrimiento humano, para ponerse a su disposición, sin replegarse en el propio bienestar} (\textit{Discurso} a la Curia Romana, 22 de diciembre de 2011). Lo vemos muy bien en los santos, que se entregaron de lleno a la causa del evangelio con entusiasmo y con gozo, sin reparar en sacrificios, incluso el de la propia vida. Su corazón era una apuesta incondicional por Cristo, de quien habían aprendido lo que significa verdaderamente amar hasta el final.]

[En este sentido, el Año de la fe, al que he convocado a toda la Iglesia, \textquote{es una invitación a una auténtica y renovada conversión al Señor, único Salvador del mundo (\ldots). La fe, en efecto, crece cuando se vive como experiencia de un amor que se recibe y se comunica como experiencia de gracia y gozo} (\textit{Porta fidei}, 11 octubre 2011, 6. 7).]}

Pidamos a la Virgen María que nos ayude a purificar nuestro corazón, especialmente ante la cercana celebración de las fiestas de Pascua, para que lleguemos a participar mejor en el misterio salvador de su Hijo, tal como ella lo dio a conocer en estas tierras. Y pidámosle también que siga acompañando y amparando a sus queridos hijos [mexicanos y latinoamericanos], para que Cristo reine en sus vidas y les ayude a promover audazmente la paz, la concordia, la justicia y la solidaridad.

Amén.
\end{body}



\newsection
\subsection{Francisco, papa}

\subsubsection{Ángelus (2015): Evangelio, cruz, testimonio}

\src{Plaza de San Pedro. \\22 de marzo de 2015.}

\begin{body}
\ltr{E}{n} este quinto domingo de Cuaresma, el \textbf{evangelista Juan} nos llama la atención con un particular curioso: algunos \textquote{griegos}, de religión judía, llegados a Jerusalén para la fiesta de la Pascua, se dirigen al apóstol Felipe y le dicen: \textquote{Queremos ver a Jesús} (\textit{Jn} 12, 21). En la ciudad santa, donde Jesús fue por última vez, hay mucha gente. Están los pequeños y los sencillos, que han acogido festivamente al profeta de Nazaret reconociendo en Él al Enviado del Señor. Están los sumos sacerdotes y los líderes del pueblo, que lo quieren eliminar porque lo consideran herético y peligroso. También hay personas, como esos \textquote{griegos}, que tienen curiosidad por verlo y por saber más acerca de su persona y de las obras realizadas por Él, la última de las cuales –la resurrección de Lázaro– causó mucha sensación.

\textquote{Queremos ver a Jesús}: estas palabras, al igual que muchas otras en los Evangelios, van más allá del episodio particular y expresan algo universal; revelan un deseo que atraviesa épocas y culturas, un deseo presente en el corazón de muchas personas que han oído hablar de Cristo, pero no lo han encontrado aún. \textquote{Yo deseo ver a Jesús}, así siente el corazón de esta gente.

Respondiendo indirectamente, de modo profético, a aquel pedido de poderlo ver, Jesús pronuncia una profecía que revela su identidad e indica el camino para conocerlo verdaderamente: \textquote{Ha llegado la hora de que sea glorificado el Hijo del hombre} (\textit{Jn} 12, 23). ¡Es la hora de la Cruz! Es la hora de la derrota de Satanás, príncipe del mal, y del triunfo definitivo del amor misericordioso de Dios. Cristo declara que será \textquote{levantado sobre la tierra} (\textit{Jn} 12, 32), una expresión con doble significado: \textquote{levantado} en cuanto crucificado, y \textquote{levantado} porque fue exaltado por el Padre en la Resurrección, para atraer a todos hacia sí y reconciliar a los hombres con Dios y entre ellos. La hora de la Cruz, la más oscura de la historia, es también la fuente de salvación para todos los que creen en Él.

Continuando con la profecía sobre su Pascua ya inminente, Jesús usa una imagen sencilla y sugestiva, la del \textquote{grano de trigo} que, al caer en la tierra, muere para dar fruto (cf. \textit{Jn} 12, 24). En esta imagen encontramos otro aspecto de la Cruz de Cristo: el de la fecundidad. La cruz de Cristo es fecunda. La muerte de Jesús, de hecho, es una fuente inagotable de vida nueva, porque lleva en sí la fuerza regeneradora del amor de Dios. Inmersos en este amor por el Bautismo, los cristianos pueden convertirse en \textquote{granos de trigo} y dar mucho fruto si, al igual que Jesús, \textquote{pierden la propia vida} por amor a Dios y a los hermanos (cf. \textit{Jn} 12, 25).

Por este motivo, a aquellos que también hoy \textquote{quieren ver a Jesús}, a los que están en búsqueda del rostro de Dios; a quien recibió una catequesis cuando era pequeño y luego no la profundizó más y quizá ha perdido la fe; a muchos que aún no han encontrado a Jesús personalmente\ldots; a todas estas personas podemos ofrecerles tres cosas: el Evangelio; el Crucifijo y el testimonio de nuestra fe, pobre pero sincera. El Evangelio: ahí podemos encontrar a Jesús, escucharlo, conocerlo. El Crucifijo: signo del amor de Jesús que se entregó por nosotros. Y luego, una fe que se traduce en gestos sencillos de caridad fraterna. Pero principalmente en la coherencia de vida: entre lo que decimos y lo que vivimos, coherencia entre nuestra fe y nuestra vida, entre nuestras palabras y nuestras acciones. Evangelio, Crucifijo y testimonio. Que la Virgen nos ayude a llevar estas tres cosas.
\end{body}

\label{b2-03-05-2015A}

\begin{patercite}
	La misión primera y fundamental que recibimos de los santos Misterios que celebramos es la de dar testimonio con nuestra vida. El asombro por el don que Dios nos ha hecho en Cristo infunde en nuestra vida un dinamismo nuevo, llamándonos a ser testigos de su amor. Nos convertimos en testigos cuando, por nuestras acciones, palabras y modo de ser, aparece Otro y se comunica. Se puede decir que el testimonio es el medio con el que la verdad del amor de Dios llega al hombre en la historia, invitándolo a acoger libremente esta novedad radical. En el testimonio Dios, por así decir, se expone al riesgo de la libertad del hombre. Jesús mismo es el testigo fiel y veraz (cf. \textit{Ap} 1,5; 3,14); vino para dar testimonio de la verdad (cf. \textit{Jn} 18,37). (\ldots) El testimonio hasta el don de sí mismos, hasta el martirio, ha sido considerado siempre en la historia de la Iglesia como la cumbre del nuevo culto espiritual: \textquote{Ofreced vuestros cuerpos} (\textit{Rm} 12,1). (\ldots) El cristiano que ofrece su vida en el martirio entra en plena comunión con la Pascua de Jesucristo y así se convierte con Él en Eucaristía. Tampoco faltan hoy en la Iglesia mártires en los que se manifiesta de modo supremo el amor de Dios. Sin embargo, aun cuando no se requiera la prueba del martirio, sabemos que el culto agradable a Dios implica también interiormente esta disponibilidad, y se manifiesta en el testimonio alegre y convencido ante el mundo de una vida cristiana coherente allí donde el Señor nos llama a anunciarlo.
	
	\textbf{Benedicto XVI, papa}, \textit{Sacramentum Caritatis}, n. 85.
\end{patercite}

\newpage

\subsubsection{Ángelus (2018): Entrar en las llagas de Cristo}

\src{Plaza de San Pedro.\\18 de marzo del 2018.}

\begin{body}
\ltr{E}{l} \textbf{Evangelio} de hoy (cf. \textit{Jn} 12, 20-33) cuenta un episodio sucedido en los últimos días de la vida de Jesús. La escena se desarrolla en Jerusalén, donde Él se encuentra por la fiesta de la Pascua hebrea. Para esta celebración, habían llegado también algunos griegos; se trata de hombres animados por sentimientos religiosos, atraídos por la fe del pueblo hebreo y que, habiendo escuchado hablar de este gran profeta, se acercaron a Felipe, uno de los doce apóstoles y le dijeron: \textquote{Señor, queremos ver a Jesús} (\textit{Jn} 12, 21). Juan resalta esta frase, centrada en el verbo \textit{ver}, que en el vocabulario del evangelista significa ir más allá de las apariencias para recoger el misterio de una persona. El verbo que utiliza Juan, \textquote{ver} es llegar hasta el corazón, llegar con la vista, con la comprensión hasta lo íntimo de la persona, dentro de la persona.

La reacción de Jesús es sorprendente. Él no responde con un \textquote{sí} o con un \textquote{no}, sino que dice: \textquote{Ha llegado la hora de que sea glorificado el Hijo del hombre} (\textit{Jn} 12, 23). Estas palabras, que parecen a primera vista ignorar la pregunta de aquellos griegos, en realidad dan la verdadera respuesta, porque quien quiere conocer a Jesús debe mirar dentro de la cruz, donde se revela su gloria. Mirar dentro de la cruz. El Evangelio de hoy nos invita a dirigir nuestra mirada hacia el crucifijo, que no es un objeto ornamental o un accesorio para vestir –¡a veces manido!– sino que es un símbolo religioso para contemplar y comprender. En la imagen de Jesús crucificado se desvela el misterio de la muerte del hijo como supremo acto de amor, fuente de vida y de salvación para la humanidad de todos los tiempos. En sus llagas fuimos curados.

Puedo pensar: \textquote{¿Cómo miro el crucifijo? ¿Como una obra de arte, para ver si es hermoso o no es hermoso? ¿O miro dentro, en las llagas de Jesús, hasta su corazón? ¿Miro el misterio del Dios aniquilado hasta la muerte, como un esclavo, como un criminal?}. No os olvidéis de esto: mirad el crucifijo, pero miradlo dentro. Está esta hermosa devoción de rezar un Padre Nuestro por cada una de las cinco llagas: cuando rezamos ese Padre Nuestro, intentamos entrar a través de las llagas de Jesús, dentro, precisamente a su corazón. Y allí aprenderemos la gran sabiduría del misterio de Cristo, la gran sabiduría de la cruz.

Y para explicar el significado de su muerte y resurrección, Jesús se sirve de una imagen y dice \textquote{si el grano de trigo no cae en tierra y muere, queda él solo; pero si muere, da mucho fruto} (\textit{Jn} 12, 24). Quiere hacer entender que su caso extremo –es decir, la cruz, muerte y resurrección– es un acto de fecundidad –sus llagas nos han curado–, una fecundidad que dará fruto para muchos. Así se compara a sí mismo con el grano de trigo que pudriéndose en la tierra genera nueva vida. Con la Encarnación, Jesús vino a la tierra; pero eso no basta: Él debe también morir, para rescatar a los hombres de la esclavitud del pecado y darles una nueva vida reconciliada en el amor. He dicho \textquote{para rescatar a los hombres}: pero, para rescatarme a mí, a ti, a todos nosotros, a cada uno de nosotros, Él pagó ese precio. Este es el misterio de Cristo. Ve hacia sus llagas. Entra, contempla; ve a Jesús, pero desde dentro.

Y este dinamismo del grano de trigo, cumplido en Jesús, debe realizarse también en nosotros sus discípulos: estamos llamados a hacer nuestra esa ley pascual del perder la vida para recibirla nueva y eterna. ¿Y qué significa perder la vida? Es decir, ¿qué significa ser el grano de trigo? Significa pensar menos en sí mismos, en los intereses personales y saber \textquote{ver} e ir al encuentro de las necesidades de nuestro prójimo, especialmente de los últimos. Cumplir con alegría obras de caridad hacia los que sufren en el cuerpo y en el espíritu es el modo más auténtico de vivir el Evangelio, es el fundamento necesario para que nuestras comunidades crezcan en la fraternidad y en la acogida recíproca. Quiero ver a Jesús, pero verlo desde dentro. Entra en sus llagas y contempla ese amor en su corazón por ti, por ti, por ti, por mí, por todos.

Que la Virgen María, que ha tenido siempre la mirada del corazón fija en su Hijo, desde el pesebre de Belén hasta la cruz en el Calvario, nos ayude a encontrarlo y conocerlo así como Él quiere, para que podamos vivir iluminados por Él y llevar al mundo frutos de justicia y de paz.
\end{body}


\begin{patercite}
	Con la comunión eucarística la Iglesia consolida también su unidad como cuerpo de Cristo. San Pablo se refiere a esta \textit{eficacia unificadora} de la participación en el banquete eucarístico cuando escribe a los Corintios: \textquote{Y el pan que partimos ¿no es comunión con el cuerpo de Cristo? Porque aun siendo muchos, un solo pan y un solo cuerpo somos, pues todos participamos de un solo pan} (\textit{1 Co} 10, 16-17). El comentario de san Juan Crisóstomo es detallado y profundo: \textquote{¿Qué es, en efecto, el pan? Es el cuerpo de Cristo. ¿En qué se transforman los que lo reciben? En cuerpo de Cristo; pero no muchos cuerpos sino un sólo cuerpo. En efecto, como el pan es sólo uno, por más que esté compuesto de muchos granos de trigo y éstos se encuentren en él, aunque no se vean, de tal modo que su diversidad desaparece en virtud de su perfecta fusión; de la misma manera, también nosotros estamos unidos recíprocamente unos a otros y, todos juntos, con Cristo}. La argumentación es terminante: nuestra unión con Cristo, que es don y gracia para cada uno, hace que en Él estemos asociados también a la unidad de su cuerpo que es la Iglesia. La Eucaristía consolida la incorporación a Cristo, establecida en el Bautismo mediante el don del Espíritu (cf. \textit{1 Co} 12, 13.27).
	
	\textbf{San Juan Pablo II, papa}, \textit{Ecclesia de Eucharistia}, n. 23.
\end{patercite}



\newsection
\section{Temas}

\cceth{La vida de Cristo se ofrece al Padre} 
\cceref{CEC 606-607}

\begin{ccebody}
\ccesec{Cristo se ofreció a su Padre por nuestros pecados \\Toda la vida de Cristo es oblación al Padre}

\n{606} El Hijo de Dios \textquote{bajado del cielo no para hacer su voluntad sino la del Padre que le ha enviado} (\textit{Jn} 6, 38), \textquote{al entrar en este mundo, dice: [\ldots] He aquí que vengo [\ldots] para hacer, oh Dios, tu voluntad [\ldots] En virtud de esta voluntad somos santificados, merced a la oblación de una vez para siempre del cuerpo de Jesucristo} (\textit{Hb} 10, 5-10). Desde el primer instante de su Encarnación el Hijo acepta el designio divino de salvación en su misión redentora: \textquote{Mi alimento es hacer la voluntad del que me ha enviado y llevar a cabo su obra} (\textit{Jn} 4, 34). El sacrificio de Jesús \textquote{por los pecados del mundo entero} (\textit{1 Jn} 2, 2), es la expresión de su comunión de amor con el Padre: \textquote{El Padre me ama porque doy mi vida} (\textit{Jn} 10, 17). \textquote{El mundo ha de saber que amo al Padre y que obro según el Padre me ha ordenado} (\textit{Jn} 14, 31).

\n{607} Este deseo de aceptar el designio de amor redentor de su Padre anima toda la vida de Jesús (cf. \textit{Lc} 12, 50; 22, 15; \textit{Mt} 16, 21-23) porque su Pasión redentora es la razón de ser de su Encarnación: \textquote{¡Padre líbrame de esta hora! Pero ¡si he llegado a esta hora para esto!} (\textit{Jn} 12, 27). \textquote{El cáliz que me ha dado el Padre ¿no lo voy a beber?} (\textit{Jn} 18, 11). Y todavía en la cruz antes de que \textquote{todo esté cumplido} (\textit{Jn} 19, 30), dice: \textquote{Tengo sed} (\textit{Jn} 19, 28).
\end{ccebody}

\cceth{El deseo de Cristo de dar su vida para nuestra salvación} 
\cceref{CEC 542, 607}

\begin{ccebody}
\n{542} Cristo es el corazón mismo de esta reunión de los hombres como \textquote{familia de Dios}. Los convoca en torno a él por su palabra, por sus señales que manifiestan el Reino de Dios, por el envío de sus discípulos. Sobre todo, él realizará la venida de su Reino por medio del gran Misterio de su Pascua: su muerte en la Cruz y su Resurrección. \textquote{Cuando yo sea levantado de la tierra, atraeré a todos hacia mí} (\textit{Jn} 12, 32). A esta unión con Cristo están llamados todos los hombres (cf. LG 3).

\n{607} Este deseo de aceptar el designio de amor redentor de su Padre anima toda la vida de Jesús (cf. \textit{Lc} 12, 50; 22, 15; \textit{Mt} 16, 21-23) porque su Pasión redentora es la razón de ser de su Encarnación: \textquote{¡Padre líbrame de esta hora! Pero ¡si he llegado a esta hora para esto!} (\textit{Jn} 12, 27). \textquote{El cáliz que me ha dado el Padre ¿no lo voy a beber?} (\textit{Jn} 18, 11). Y todavía en la cruz antes de que \textquote{todo esté cumplido} (\textit{Jn} 19, 30), dice: \textquote{Tengo sed} (\textit{Jn} 19, 28).
\end{ccebody}

\cceth{El Espíritu glorifica al Hijo, el Hijo glorifica al Padre} 
\cceref{CEC 690, 729}

\begin{ccebody}
\n{690} Jesús es Cristo, \textquote{ungido}, porque el Espíritu es su Unción y todo lo que sucede a partir de la Encarnación mana de esta plenitud (cf. \textit{Jn} 3, 34). Cuando por fin Cristo es glorificado (\textit{Jn} 7, 39), puede a su vez, de junto al Padre, enviar el Espíritu a los que creen en él: Él les comunica su Gloria (cf. \textit{Jn} 17, 22), es decir, el Espíritu Santo que lo glorifica (cf. \textit{Jn} 16, 14). La misión conjunta se desplegará desde entonces en los hijos adoptados por el Padre en el Cuerpo de su Hijo: la misión del Espíritu de adopción será unirlos a Cristo y hacerles vivir en Él:

\ccecite{\textquote{La noción de la unción sugiere [\ldots] que no hay ninguna distancia entre el Hijo y el Espíritu. En efecto, de la misma manera que entre la superficie del cuerpo y la unción del aceite ni la razón ni los sentidos conocen ningún intermediario, así es inmediato el contacto del Hijo con el Espíritu, de tal modo que quien va a tener contacto con el Hijo por la fe tiene que tener antes contacto necesariamente con el óleo. En efecto, no hay parte alguna que esté desnuda del Espíritu Santo. Por eso es por lo que la confesión del Señorío del Hijo se hace en el Espíritu Santo por aquellos que la aceptan, viniendo el Espíritu desde todas partes delante de los que se acercan por la fe} (San Gregorio de Nisa, \textit{Adversus Macedonianos de Spirirtu Sancto}, 16).}

\n{729} Solamente cuando ha llegado la hora en que va a ser glorificado Jesús \textit{promete} la venida del Espíritu Santo, ya que su Muerte y su Resurrección serán el cumplimiento de la Promesa hecha a los Padres (cf. \textit{Jn} 14, 16-17. 26; 15, 26; 16, 7-15; 17, 26): El Espíritu de Verdad, el otro Paráclito, será dado por el Padre en virtud de la oración de Jesús; será enviado por el Padre en nombre de Jesús; Jesús lo enviará de junto al Padre porque él ha salido del Padre. El Espíritu Santo vendrá, nosotros lo conoceremos, estará con nosotros para siempre, permanecerá con nosotros; nos lo enseñará todo y nos recordará todo lo que Cristo nos ha dicho y dará testimonio de Él; nos conducirá a la verdad completa y glorificará a Cristo. En cuanto al mundo, lo acusará en materia de pecado, de justicia y de juicio.
\end{ccebody}

\cceth{La Ascensión de Cristo a la gloria es nuestra victoria} 
\cceref{CEC 662, 2853}

\begin{ccebody}
\n{662} \textquote{Cuando yo sea levantado de la tierra, atraeré a todos hacia mí} (\textit{Jn} 12, 32). La elevación en la Cruz significa y anuncia la elevación en la Ascensión al cielo. Es su comienzo. Jesucristo, el único Sacerdote de la Alianza nueva y eterna, \textquote{no [\ldots] penetró en un Santuario hecho por mano de hombre [\ldots], sino en el mismo cielo, para presentarse ahora ante el acatamiento de Dios en favor nuestro} (\textit{Hb} 9, 24). En el cielo, Cristo ejerce permanentemente su sacerdocio. \textquote{De ahí que pueda salvar perfectamente a los que por él se llegan a Dios, ya que está siempre vivo para interceder en su favor} (\textit{Hb} 7, 25). Como \textquote{Sumo Sacerdote de los bienes futuros} (\textit{Hb} 9, 11), es el centro y el oficiante principal de la liturgia que honra al Padre en los cielos (cf. \textit{Ap} 4, 6-11).

\n{2853} La victoria sobre el \textquote{príncipe de este mundo} (\textit{Jn} 14, 30) se adquirió de una vez por todas en la Hora en que Jesús se entregó libremente a la muerte para darnos su Vida. Es el juicio de este mundo, y el príncipe de este mundo está \textquote{echado abajo} (\textit{Jn} 12, 31; \textit{Ap} 12, 11). \textquote{Él se lanza en persecución de la Mujer} (cf. \textit{Ap} 12, 13-16), pero no consigue alcanzarla: la nueva Eva, \textquote{llena de gracia} del Espíritu Santo es preservada del pecado y de la corrupción de la muerte (Concepción inmaculada y Asunción de la santísima Madre de Dios, María, siempre virgen). \textquote{Entonces despechado contra la Mujer, se fue a hacer la guerra al resto de sus hijos} (\textit{Ap} 12, 17). Por eso, el Espíritu y la Iglesia oran: \textquote{Ven, Señor Jesús} (\textit{Ap} 22, 17. 20) ya que su Venida nos librará del Maligno.
\end{ccebody}

\cceth{Historia de las alianzas} 
\cceref{CEC 56-64, 220, 715, 762, 1965}

\begin{ccebody}
\ccesec{La alianza con Noé}

\n{56} Una vez rota la unidad del género humano por el pecado, Dios decide desde el comienzo salvar a la humanidad a través de una serie de etapas. La alianza con Noé después del diluvio (cf. \textit{Gn} 9, 9) expresa el principio de la Economía divina con las \textquote{naciones}, es decir con los hombres agrupados \textquote{según sus países, cada uno según su lengua, y según sus clanes} (\textit{Gn} 10, 5; cf. \textit{Gn} 10, 20-31).

\n{57} Este orden a la vez cósmico, social y religioso de la pluralidad de las naciones (cf. \textit{Hch} 17, 26-27), está destinado a limitar el orgullo de una humanidad caída que, unánime en su perversidad (cf. \textit{Sb} 10, 5), quisiera hacer por sí misma su unidad a la manera de Babel (cf. \textit{Gn} 11, 4-6). Pero, a causa del pecado (cf. \textit{Rm} 1, 18-25), el politeísmo, así como la idolatría de la nación y de su jefe, son una amenaza constante de vuelta al paganismo para esta economía aún no definitiva.

\n{58} La alianza con Noé permanece en vigor mientras dura el tiempo de las naciones (cf. \textit{Lc} 21, 24), hasta la proclamación universal del Evangelio. La Biblia venera algunas grandes figuras de las \textquote{naciones}, como \textquote{Abel el justo}, el rey-sacerdote Melquisedec (cf. \textit{Gn} 14, 18), figura de Cristo (cf. \textit{Hb} 7, 3), o los justos \textquote{Noé, Daniel y Job} (\textit{Ez} 14, 14). De esta manera, la Escritura expresa qué altura de santidad pueden alcanzar los que viven según la alianza de Noé en la espera de que Cristo \textquote{reúna en uno a todos los hijos de Dios dispersos} (\textit{Jn} 11, 52).

\ccesec{Dios elige a Abraham}

\n{59} Para reunir a la humanidad dispersa, Dios elige a Abram llamándolo \textquote{fuera de su tierra, de su patria y de su casa} (\textit{Gn} 12, 1), para hacer de él \textquote{Abraham}, es decir, \textquote{el padre de una multitud de naciones} (\textit{Gn} 17, 5): \textquote{En ti serán benditas todas las naciones de la tierra} (\textit{Gn} 12, 3; cf. \textit{Ga} 3, 8).

\n{60} El pueblo nacido de Abraham será el depositario de la promesa hecha a los patriarcas, el pueblo de la elección (cf. \textit{Rm} 11, 28), llamado a preparar la reunión un día de todos los hijos de Dios en la unidad de la Iglesia (cf. \textit{Jn} 11, 52; 10, 16); ese pueblo será la raíz en la que serán injertados los paganos hechos creyentes (cf. \textit{Rm} 11, 17-18. 24).

\n{61} Los patriarcas, los profetas y otros personajes del Antiguo Testamento han sido y serán siempre venerados como santos en todas las tradiciones litúrgicas de la Iglesia.

\ccesec{Dios forma a su pueblo Israel}

\n{62} Después de la etapa de los patriarcas, Dios constituyó a Israel como su pueblo salvándolo de la esclavitud de Egipto. Estableció con él la alianza del Sinaí y le dio por medio de Moisés su Ley, para que lo reconociese y le sirviera como al único Dios vivo y verdadero, Padre providente y juez justo, y para que esperase al Salvador prometido (cf. DV 3).

\n{63} Israel es el pueblo sacerdotal de Dios (cf. \textit{Ex} 19, 6), \textquote{sobre el que es invocado el nombre del Señor} (\textit{Dt} 28, 10). Es el pueblo de aquellos \textquote{a quienes Dios habló primero} (\textit{Viernes Santo, Pasión y Muerte del Señor, Oración universal VI, Misal Romano}), el pueblo de los \textquote{hermanos mayores} en la fe de Abraham (cf. \textit{Discurso en la sinagoga ante la comunidad hebrea de Roma}, 13 abril 1986).

\n{64} Por los profetas, Dios forma a su pueblo en la esperanza de la salvación, en la espera de una Alianza nueva y eterna destinada a todos los hombres (cf. \textit{Is} 2,2-4), y que será grabada en los corazones (cf. \textit{Jr} 31, 31-34; \textit{Hb} 10, 16). Los profetas anuncian una redención radical del pueblo de Dios, la purificación de todas sus infidelidades (cf. \textit{Ez} 36), una salvación que incluirá a todas las naciones (cf. \textit{Is} 49, 5-6; 53, 11). Serán sobre todo los pobres y los humildes del Señor (cf. \textit{So} 2, 3) quienes mantendrán esta esperanza. Las mujeres santas como Sara, Rebeca, Raquel, Miriam, Débora, Ana, Judit y Ester conservaron viva la esperanza de la salvación de Israel. De ellas la figura más pura es María (cf. \textit{Lc} 1, 38).

\n{220} El amor de Dios es \textquote{eterno} (\textit{Is} 54, 8). \textquote{Porque los montes se correrán y las colinas se moverán, mas mi amor de tu lado no se apartará} (\textit{Is} 54, 10). \textquote{Con amor eterno te he amado: por eso he reservado gracia para ti} (\textit{Jr} 31, 3).

\n{715} Los textos proféticos que se refieren directamente al envío del Espíritu Santo son oráculos en los que Dios habla al corazón de su Pueblo en el lenguaje de la Promesa, con los acentos del \textquote{amor y de la fidelidad} (cf. \textit{Ez }11, 19; 36, 25-28; 37, 1-14; \textit{Jr} 31, 31-34; y \textit{Jl} 3, 1-5), cuyo cumplimiento proclamará San Pedro la mañana de Pentecostés (cf. \textit{Hch} 2, 17-21). Según estas promesas, en los \textquote{últimos tiempos}, el Espíritu del Señor renovará el corazón de los hombres grabando en ellos una Ley nueva; reunirá y reconciliará a los pueblos dispersos y divididos; transformará la primera creación y Dios habitará en ella con los hombres en la paz.

\n{762} La \textit{preparación} lejana de la reunión del pueblo de Dios comienza con la vocación de Abraham, a quien Dios promete que llegará a ser padre de un gran pueblo (cf. \textit{Gn} 12, 2; 15, 5-6). La preparación inmediata comienza con la elección de Israel como pueblo de Dios (cf. \textit{Ex} 19, 5-6; \textit{Dt} 7, 6). Por su elección, Israel debe ser el signo de la reunión futura de todas las naciones (cf. \textit{Is} 2, 2-5; \textit{Mi} 4, 1-4). Pero ya los profetas acusan a Israel de haber roto la alianza y haberse comportado como una prostituta (cf. \textit{Os} 1; \textit{Is} 1, 2-4; \textit{Jr} 2; etc.). Anuncian, pues, una Alianza nueva y eterna (cf. \textit{Jr} 31, 31-34; \textit{Is} 55, 3). \textquote{Jesús instituyó esta nueva alianza} (LG 9).

\ccesec{La Ley nueva o Ley evangélica}

\n{1965} La Ley nueva o Ley evangélica es la perfección aquí abajo de la ley divina, natural y revelada. Es obra de Cristo y se expresa particularmente en el Sermón de la Montaña. Es también obra del Espíritu Santo, y por él viene a ser la ley interior de la caridad: \textquote{Concertaré con la casa de Israel una alianza nueva [\ldots] pondré mis leyes en su mente, en sus corazones las grabaré; y yo seré su Dios y ellos serán mi pueblo} (\textit{Hb} 8, 8-10; cf. \textit{Jr} 31, 31-34).
\end{ccebody}
