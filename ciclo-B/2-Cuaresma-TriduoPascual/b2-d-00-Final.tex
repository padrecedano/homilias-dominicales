			\chapter{Nota final}

Esta obra nació en un momento de oscuridad: la epidemia del Coronavirus vino de improviso y con ella un confinamiento de muchos días. ¿Qué hacer con tantas horas que antes ocupaba en la misión de manera presencial? Me decidí por preparar y publicar esta colección, esperando que pueda servir también a otros hermanos sacerdotes y diáconos.

El precio de los libros se ha establecido con el menor margen de ganancia posible, porque los mismos no se publican con fines de lucro, sino sobre todo con una finalidad pastoral (quienes no puedan comprarlos podrán descargarlos en formato electrónico sin coste alguno). Las ganancias que me revengan serán destinadas a la ayuda pastoral de mi Parroquia San Pedro Claver en la República Dominicana. Es el lugar donde he recibido la fe y en el que la he vivido durante todos estos años unido a una comunidad de hermanos con rostros y vidas concretos. La tierra buena, donde Dios quiso plantarme.

Mi deseo sincero es que esta obra, nacida en un momento difícil para millones de personas, sea una luz a través del ministerio de la predicación.

Sevilla (España)

31 de mayo del 2020

Solemnidad de Pentecostés \\y Visitación de la Bienaventurada Virgen María

\begin{patercite}\textbf{El Espíritu Santo en la experiencia del desierto}\end{patercite}

\begin{patercite}Al \textquote{comienzo} de la misión mesiánica de Jesús vemos otro hecho interesante y sugestivo, narrado por los evangelistas, que lo hacen depender de la acción del Espíritu Santo: se trata de \textit{la experiencia del desierto}. Leemos en el evangelio según san Marcos: \textquote{A continuación (del bautismo), \textit{el Espíritu le empuja al desierto}} (\textit{Mc} 1, 12). Además, Mateo (4, 1) y Lucas (4, 1) afirman que Jesús \textquote{fue conducido por el Espíritu al desierto}. Estos textos ofrecen puntos de reflexión que nos llevan a una ulterior investigación sobre el misterio de la íntima unión de Jesús-Mesías con el Espíritu Santo, ya desde el inicio de la obra de la redención.\end{patercite}

\begin{patercite}En primer lugar, una observación de carácter lingüístico: los verbos usados por los evangelistas (\textquote{fue conducido} por Mateo y Lucas; \textquote{le empuja}, por Marcos) expresan \textit{una iniciativa especialmente enérgica} por parte del Espíritu Santo, iniciativa que se inserta en la lógica de la vida espiritual y en la misma psicología de Jesús: acaba de recibir de Juan un \textquote{bautismo de penitencia}, y por ello siente la necesidad de un período de reflexión y de austeridad, aunque personalmente no tenía necesidad de penitencia, dado que estaba \textquote{lleno de gracia} y era \textquote{santo} desde el momento de su concepción (cf. \textit{Jn} 1, 14; \textit{Lc} 1, 35): como preparación para su ministerio mesiánico. Su misión le exige también vivir en medio de los hombres-pecadores, a quienes ha sido enviado a evangelizar y salvar (cf. santo Tomás, \textit{Summa Theol.,} III, q. 40, a. 1), en lucha contra el poder del demonio. De aquí la conveniencia de esta pausa en el desierto \textquote{\textit{para ser tentado por el diablo}}. Por lo tanto, Jesús sigue el impulso interior y se dirige adonde le sugiere el Espíritu Santo.\end{patercite}

\begin{patercite}\textit{El desierto}, además de ser lugar de encuentro con Dios, es también lugar de tentación y de lucha espiritual. Durante la peregrinación a través del desierto, que se prolongó durante cuarenta años, el pueblo de Israel había sufrido muchas tentaciones y había cedido (cf. \textit{Ex} 32, 1-6; \textit{Nm} 14, 1-4; 21, 4-5; 25, 1-3; \textit{Sal} 78, 17; \textit{1 Co} 10, 7-10). Jesús va al desierto, casi remitiéndose a la experiencia histórica de su pueblo. Pero, a diferencia del comportamiento de Israel, en el momento de inaugurar su actividad mesiánica, es sobre todo \textit{dócil a la acción del Espíritu Santo, que le pide desde el interior aquella definitiva preparación para el cumplimiento de su misión}. Es un período de soledad y de prueba espiritual, que supera con la ayuda de la palabra de Dios y con la oración. \end{patercite}

\begin{patercite}En el espíritu de la tradición bíblica, y en la línea con la psicología israelita, aquel número de \textquote{cuarenta días} podía relacionarse fácilmente con otros acontecimientos históricos, llenos de significado para la historia de la salvación: los cuarenta días del diluvio (cf. \textit{Gn} 7, 4. 17); los cuarenta días de permanencia de Moisés en el monte (cf. \textit{Ex} 24, 18); los cuarenta días de camino de Elías, alimentado con el pan prodigioso que le había dado nueva fuerza (cf. \textit{1 R} 19, 8). Según los evangelistas, Jesús, bajo la moción del Espíritu Santo, se acomoda, en lo que se refiere a la permanencia en el desierto, a este número tradicional y casi sagrado (cf. \textit{Mt} 4, 1; Lc 4, 1). Lo mismo hará también en el período de las apariciones a los Apóstoles tras la resurrección y la ascensión al cielo (cf. \textit{Hch} 1, 3). [+]\end{patercite}

\begin{patercite}\textbf{San Juan Pablo II, papa}. \textit{Catequesis,} audiencia general, 21 de julio 1990, nn. 1-2.\end{patercite}

\begin{patercite}[+] Jesús, por tanto, es conducido al desierto con el fin de afrontar \textit{las tentaciones de Satanás} y para que pueda tener, a la vez, un contacto más libre e íntimo con el Padre. Aquí conviene tener presente que los evangelistas suelen presentarnos \textit{el desierto} como \textit{el lugar donde reside Satanás}: baste recordar el pasaje de Lucas sobre el \textquote{espíritu inmundo} que \textquote{cuando sale del hombre, anda vagando por lugares áridos, en busca de reposo\ldots} (\textit{Lc} 11, 24); y en el pasaje que nos narra el episodio del endemoniado de Gerasa que \textquote{era empujado por el demonio al desierto} (\textit{Lc} 8, 29).\end{patercite}

\begin{patercite}En el caso de las tentaciones de Jesús, el ir al desierto es obra del Espíritu Santo, y ante todo significa el inicio de una demostración –se podría decir, incluso, de una nueva toma de conciencia– de la lucha que deberá mantener hasta el final de su vida contra Satanás, artífice del pecado. Venciendo sus tentaciones, manifiesta su propio poder salvífico sobre el pecado y la llegada del reino de Dios, como dirá un día: \textquote{Si por el Espíritu de Dios expulso yo los demonios, es que ha llegado a vosotros el reino de Dios} (\textit{Mt} 12, 28).\end{patercite}

\begin{patercite}\textbf{San Juan Pablo II, papa}. \textit{Catequesis,} audiencia general, 21 de julio 1990, n. 3.\end{patercite}

\begin{patercite}[+] Si observamos bien, en las tentaciones sufridas y vencidas por Jesús durante la \textquote{experiencia del desierto} se nota la oposición de Satanás contra la llegada del reino de Dios al mundo humano, directa o indirectamente expresada en los textos de los evangelistas. Las respuestas que da Jesús al tentador desenmascaran las intenciones esenciales del \textquote{padre de la mentira} (\textit{Jn} 8, 44), que trata de servirse, de modo perverso, de las palabras de la Escritura para alcanzar sus objetivos. Pero Jesús lo refuta apoyándose en la misma palabra de Dios, aplicada correctamente. La narración de los evangelistas incluye, tal vez, alguna reminiscencia y establece un paralelismo tanto con las análogas tentaciones del pueblo de Israel en los cuarenta años de peregrinación por el desierto (la búsqueda de alimento: cf. \textit{Dt} 8, 3; \textit{Ex} 16; la pretensión de la protección divina para satisfacerse a sí mismos: cf. \textit{Dt} 6, 16; \textit{Ex} 17, 1-7; la idolatría: cf. \textit{Dt} 6, 13; \textit{Ex} 32, 1-6), como con diversos momentos de la vida de Moisés. Pero se podría decir que el episodio entra específicamente en la historia de Jesús por su lógica biográfica y teológica. Aún estando libre de pecado, Jesús pudo conocer las seducciones externas del mal (cf. \textit{Mt} 16, 23): y era conveniente que fuese tentado para llegar a ser el Nuevo Adán, nuestro guía, nuestro redentor clemente (cf. \textit{Mt} 26, 36-46; \textit{Hb} 2, 10. 17-18; 4, 15; 5, 2. 7-9).\end{patercite}

\begin{patercite}En el fondo de todas las tentaciones estaba la perspectiva de \textit{un mesianismo político y} \textit{glorioso}, como se había difundido y había penetrado en el alma del pueblo de Israel. El diablo trata de inducir a Jesús a acoger esta falsa perspectiva, \end{patercite}

\begin{patercite}porque es el enemigo del plan de Dios, de su ley, de su economía de salvación, y por tanto de Cristo, como aparece claro por el evangelio y los demás escritos del Nuevo Testamento (cf. \textit{Mt} 13, 39; \textit{Jn} 8, 44; 13, 2; \textit{Hch} 10, 38; \textit{Ef} 6, 11; \textit{1 Jn} 3, 8, etc.). Si también Cristo cayese, el imperio de Satanás, que se gloría de ser el amo del mundo (\textit{Lc} 4, 5-6), obtendría la victoria definitiva en la historia. Aquel momento de la lucha en el desierto es, por consiguiente, decisivo.\end{patercite}

\begin{patercite}Jesús es consciente de ser enviado por el Padre para hacer presente el reino de Dios entre los hombres. Con ese fin acepta la tentación, tomando su lugar entre los pecadores, como había hecho ya en el Jordán, para servirles a todos de ejemplo (cf. San Agustín, \textit{De Trinitate}, 4, 13). Pero, por otra parte, en virtud de la \textquote{unción} del Espíritu Santo, llega a las mismas raíces del pecado y derrota al \textquote{padre de la mentira} (\textit{Jn} 8, 44). Por eso, va voluntariamente al encuentro de la tentación desde el comienzo de su ministerio, siguiendo el impulso del Espíritu Santo (cf. San Agustín, \textit{De Trinitate}, 13, 13).\end{patercite}

\begin{patercite}Un día, dando cumplimiento a su obra, podrá proclamar: \textquote{Ahora es el juicio de este mundo; ahora el príncipe de este mundo será echado fuera} (\textit{Jn} 12, 31). Y la víspera de su pasión repetirá una vez más: \textquote{Llega el príncipe de este mundo. En mí no tiene ningún poder} (\textit{Jn} 14, 30); es más, \textquote{el príncipe de este mundo está (ya) juzgado} (\textit{Jn} 16, 11); \textquote{¡Ánimo!, yo he vencido al mundo} (\textit{Jn} 16, 33). La lucha contra el \textquote{padre de la mentira}, que es el \textquote{principe de este mundo}, iniciada en el desierto, alcanzará su culmen en el Gólgota: la victoria se alcanzará por medio de la cruz del Redentor.\end{patercite}

\begin{patercite}Estamos, por tanto, llamados a reconocer el valor integral del desierto como lugar de una particular experiencia de Dios, como sucedió con Moisés (cf. \textit{Ex} 24, 18), con Elías (\textit{1 R} 19, 8), y sobre todo con Jesús que, \textquote{conducido} por el Espíritu Santo, acepta realizar la misma experiencia: \textit{el contacto con Dios Padre} (cf. \textit{Os} 2, 16) \textit{en lucha contra las potencias} \textit{opuestas a Dios}. Su experiencia es ejemplar, y nos puede servir también como lección sobre la necesidad de la penitencia, no para Jesús que estaba libre de pecado, sino para todos nosotros. Jesús mismo un día alertará a sus discípulos sobre la necesidad \textit{de la oración y del ayuno} para echar a los \textquote{espíritus inmundos} (cf. \textit{Mc} 9, 29) y, en la tensión de la solitaria oración de Getsemaní, recomendará a los Apóstoles presentes: \textquote{\textit{Velad y orad,} \textit{para que no caigáis en tentación}; que el espíritu está pronto, pero la carne es débil} (\textit{Mc} 14, 38). Seamos conscientes de que, amoldándonos a Cristo victorioso en la experiencia del desierto, también nosotros tendremos un divino confortador: el Espíritu Santo Paráclito, pues el mismo Cristo ha prometido que \textquote{recibirá de lo suyo} y nos lo dará (cf. \textit{Jn} 16, 14): Él, que condujo al Mesías al desierto no sólo \textquote{para ser tentado} sino también para que diera la primera demostración de su poderosa victoria sobre el diablo y sobre su reino, tomará de la victoria de Cristo sobre el pecado y sobre Satanás, su primer artífice, para hacer partícipe de ella a todo el que sea tentado.\end{patercite}

\begin{patercite}\textbf{San Juan Pablo II, papa}. \textit{Catequesis,} audiencia general, 21 de julio 1990, nn. 4-6.\end{patercite}

