
\chapter{Viernes Santo en~la~Pasión~del~Señor}

\section{Lecturas}

\rtitle{PRIMERA LECTURA}

\rbook{Del libro del profeta Isaías} \rred{52, 13–53, 12}

\rtheme{Él fue traspasado por nuestras rebeliones}

\begin{readprose}
Mirad, mi siervo tendrá éxito, 
   subirá y crecerá mucho. 

Como muchos se espantaron de él 
   porque desfigurado no parecía hombre, 
   ni tenía aspecto humano, 
   así asombrará a muchos pueblos, 
   ante él los reyes cerrarán la boca, 
   al ver algo inenarrable 
   y comprender algo inaudito.
   
¿Quién creyó nuestro anuncio?; 
   ¿a quién se reveló el brazo del Señor? 

Creció en su presencia como brote, 
   como raíz en tierra árida, 
   sin figura, sin belleza. 
   
Lo vimos sin aspecto atrayente, 
   despreciado y evitado de los hombres, 
   como un hombre de dolores, 
   acostumbrado a sufrimientos, 
   ante el cual se ocultaban los rostros, 
   despreciado y desestimado. 

\newpage    
Él soportó nuestros sufrimientos 
   y aguantó nuestros dolores; 
   nosotros lo estimamos leproso, 
   herido de Dios y humillado; 
   pero él fue traspasado por nuestras rebeliones, 
   triturado por nuestros crímenes. 
   
Nuestro castigo saludable cayó sobre él, 
   sus cicatrices nos curaron. 
   
Todos errábamos como ovejas, 
   cada uno siguiendo su camino; 
   y el Señor cargó sobre él 
   todos nuestros crímenes. 
   
Maltratado, voluntariamente se humillaba 
   y no abría la boca: 
   como cordero llevado al matadero, 
   como oveja ante el esquilador, 
   enmudecía y no abría la boca. 

Sin defensa, sin justicia, se lo llevaron, 
   ¿quién se preocupará de su estirpe? 
   
Lo arrancaron de la tierra de los vivos,
   por los pecados de mi pueblo lo hirieron. 
   
Le dieron sepultura con los malvados 
   y una tumba con los malhechores, 
   aunque no había cometido crímenes 
   ni hubo engaño en su boca. 
   
El Señor quiso triturarlo con el sufrimiento, 
   y entregar su vida como expiación: 
   verá su descendencia, 
   prolongará sus años, 
   lo que el Señor quiere prosperará por su mano. 
   
Por los trabajos de su alma verá la luz, 
   el justo se saciará de conocimiento. 
   
Mi siervo justificará a muchos, 
   porque cargó con los crímenes de ellos. 
   
Le daré una multitud como parte,
   y tendrá como despojo una muchedumbre. 
   
Porque expuso su vida a la muerte
   y fue contado entre los pecadores, 
   él tomó el pecado de muchos 
   e intercedió por los pecadores.
\end{readprose}


\newpage 
\rtitle{SALMO RESPONSORIAL}

\rbook{Salmo} \rred{30, 2 y 6. 12-13. 15-16. 17 y 25}

\rtheme{Padre, a tus manos encomiendo mi espíritu}

\begin{psbody}
A ti , Señor, me acojo: 
no quede yo nunca defraudado; 
tú, que eres justo, ponme a salvo. 
A tus manos encomiendo mi espíritu: 
tú, el Dios leal, me librarás. 

Soy la burla de todos mis enemigos, 
la irrisión de mis vecinos, 
el espanto de mis conocidos: 
me ven por la calle, y escapan de mí. 
Me han olvidado como a un muerto, 
me han desechado como a un cacharro inútil. 

Pero yo confío en ti, Señor; 
te digo: \textquote{Tú eres mi Dios}. 
En tu mano están mis azares: 
líbrame de los enemigos que me persiguen. 

Haz brillar tu rostro sobre tu siervo, 
sálvame por tu misericordia. 
Sed fuertes y valientes de corazón, 
los que esperáis en el Señor. 
\end{psbody}


\newpage 
\rtitle{SEGUNDA LECTURA}

\rbook{De la carta a los Hebreos} \rred{4, 14-16; 5, 7-9}

\rtheme{Aprendió a obedecer; y se convirtió, para todos los que lo obedecen, en autor de salvación}

\begin{scripture}
Hermanos: 

Ya que tenemos un sumo sacerdote grande que ha atravesado el cielo, Jesús, Hijo de Dios, mantengamos firme la confesión de fe. 

No tenemos un sumo sacerdote incapaz de compadecerse de nuestras debilidades, sino que ha sido probado en todo, como nosotros, menos en el pecado. Por eso, comparezcamos confiados ante el trono de la gracia, para alcanzar misericordia y encontrar gracia para un auxilio oportuno. 

Cristo, en los días de su vida mortal, a gritos y con lágrimas, presentó oraciones y súplicas al que podía salvarlo de la muerte, siendo escuchado por su piedad filial. Y, aun siendo Hijo, aprendió, sufriendo, a obedecer. Y, llevado a la consumación, se convirtió, para todos los que lo obedecen, en autor de salvación eterna.
\end{scripture}


\rtitle{EVANGELIO}

\rbook{Del Evangelio según san Juan} \rred{18, 1–19, 42}

\rtheme{Pasión de nuestro Señor Jesucristo}

\rbr{Omitimos el texto del Evangelio debido a su gran extensión.}



\newsection
\section{Comentarios Patrísticos}

\subsection{San Juan Crisóstomo, obispo}

\ptheme{El valor de la sangre de Cristo}

\src{Catequesis 3, 13-19: SC 50, 174-177.}

\begin{body}
\ltr[¿]{D}{eseas} conocer el valor de la sangre de Cristo? Remontémonos a las figuras que la profetizaron y recordemos los antiguos relatos de Egipto. \textit{Inmolad} –dice Moisés– \textit{un cordero de un año; tomad su sangre y rociad las dos jambas y el dintel de la casa}. \textquote{¿Qué dices, Moisés? La sangre de un cordero irracional ¿puede salvar a los hombres dotados de razón?} \textquote{Sin duda –responde Moisés–: no porque se trate de sangre, sino porque en esta sangre se contiene una profecía de la sangre del Señor}.

Si hoy, pues, el enemigo, en lugar de ver las puertas rociadas con sangre simbólica, ve brillar en los labios de los fieles, puertas de los templos de Cristo, la sangre del verdadero Cordero, huirá todavía más lejos.

¿Deseas descubrir aún por otro medio el valor de esta sangre? Mira de dónde brotó y cuál sea su fuente. Empezó a brotar de la misma cruz y su fuente fue el costado del Señor. Pues muerto ya el Señor, dice el Evangelio, \textit{uno de los soldados se acercó con la lanza, le traspasó el costado, y al punto salió agua y sangre}: \textit{agua}, como símbolo del bautismo; \textit{sangre}, como figura de la eucaristía. El soldado le traspasó el costado, abrió una brecha en el muro del templo santo, y yo encuentro el tesoro escondido y me alegro con la riqueza hallada. Esto fue lo que ocurrió con el cordero: los judíos sacrificaron el cordero, y yo recibo el fruto del sacrificio.

Del costado salió \textit{sangre} y \textit{agua}. No quiero, amado oyente, que pases con indiferencia ante tan gran misterio, pues me falta explicarte aún otra interpretación mística. He dicho que esta \textit{agua} y esta \textit{sangre} eran símbolos del bautismo y de la eucaristía. Pues bien, con estos dos sacramentos se edifica la Iglesia: con el \textit{agua} de la regeneración y con la renovación del Espíritu Santo, es decir, con el bautismo y la eucaristía, que han brotado, ambos, del costado. Del costado de Jesús se formó, pues, la Iglesia, como del costado de Adán fue formada Eva.


\newpage 
Por esta misma razón, afirma san Pablo: \textit{Somos miembros de su cuerpo, formados de sus huesos}, aludiendo con ello al costado de Cristo. Pues del mismo modo que Dios formó a la mujer del costado de Adán, de igual manera Jesucristo nos dio el \textit{agua} y la \textit{sangre} salidas de su costado, para edificar la Iglesia. Y de la misma manera que entonces Dios tomó la costilla de Adán, mientras éste dormía, así también nos dio el \textit{agua} y la \textit{sangre} después que Cristo hubo muerto.

Mirad de qué manera Cristo se ha unido a su esposa, considerad con qué alimento la nutre. Con un mismo alimento hemos nacido y nos alimentamos. De la misma manera que la mujer se siente impulsada por su misma naturaleza a alimentar con su propia sangre y con su leche a aquel a quien ha dado a luz, así también Cristo alimenta siempre con su \textit{sangre} a aquellos a quienes él mismo ha hecho renacer.
\end{body}

\begin{patercite}
\textquote{Tu rostro buscaré, Señor, no me escondas tu rostro} (\textit{Sal} 26, 8-9). Verónica --Berenice, según la tradición griega-- encarna este anhelo que acomuna a todos los hombres píos del Antiguo Testamento, el anhelo de todos los creyentes de ver el rostro de Dios. Ella, en principio, en el Vía crucis de Jesús no hace más que prestar un servicio de bondad femenina: ofrece un paño a Jesús. No se deja contagiar ni por la brutalidad de los soldados, ni inmovilizar por el miedo de los discípulos. Es la imagen de la mujer buena que, en la turbación y en la oscuridad del corazón, mantiene el brío de la bondad, sin permitir que su corazón se oscurezca. \textquote{Bienaventurados los limpios de corazón –había dicho el Señor en el Sermón de la montaña–, porque verán a Dios} (\textit{Mt} 5, 8). Inicialmente, Verónica ve solamente un rostro maltratado y marcado por el dolor. Pero el acto de amor imprime en su corazón la verdadera imagen de Jesús: en el rostro humano, lleno de sangre y heridas, ella ve el rostro de Dios y de su bondad, que nos acompaña también en el dolor más profundo. Únicamente podemos ver a Jesús con el corazón. Solamente el amor nos deja ver y nos hace puros. Sólo el amor nos permite reconocer a Dios, que es el amor mismo.

\textbf{Joseph Card. Ratzinger}, \textit{Meditación} para la sexta estación del Vía Crucis del año 2005 en el Coliseo Romano. 
\end{patercite}

\newsection
\subsection{San Cirilo de Alejandría, obispo}

\ptheme{Cristo entregó su alma en manos del Padre, \\abriéndonos a nuevas y luminosas esperanzas}

\src{Comentario sobre el evangelio de san Juan, Lib. 12: PG 74, 667-670.}

\begin{body}
\ltr{J}{esús,} cuando tomó el vinagre, dijo: \textquote{\textit{Está cumplido}}. E, inclinando la cabeza, entregó el espíritu. Con razón dijo: \textquote{\textit{Está cumplido}}. Ha sonado ya la hora de llevar el mensaje de salvación a los espíritus que se encuentran en los abismos. Él vino efectivamente para establecer su señorío sobre vivos y muertos. Por nosotros soportó la misma muerte en la carne asunta, enteramente igual a la nuestra, él que por naturaleza, Dios como es, es la vida misma. Todo esto, lo ha querido él expresamente para destronar a los poderes abismales y preparar de este modo el retorno de la naturaleza humana a la vida verdadera, \textit{él primicia de todos los que han muerto y primogénito de toda criatura}.

\textit{Inclinando la cabeza:} es el gesto característico del que acaba de morir, cuando, al faltar el espíritu que mantiene unido a todo el cuerpo, los músculos y los nervios se relajan. Por eso, la expresión del evangelista no es del todo apropiada, aunque inmediatamente introduzca otra frase comúnmente utilizada, también ella, para indicar que uno ha muerto: \textit{entregó el espíritu.}

Parece como si impulsado por una particular inspiración, el evangelista no haya dicho simplemente \textit{murió}, sino \textit{entregó el espíritu}. Es decir, entregó su espíritu en manos de Dios Padre, de acuerdo con lo que él mismo había dicho, si bien a través de la profética voz del salmista: \textit{Padre, a tus manos encomiendo mi espíritu}. Y mientras tanto, la fuerza y el sentido de estas palabras constituían para nosotros el comienzo y el fundamento de una dichosa esperanza.

Debemos efectivamente creer que las almas de los santos, al salir del cuerpo, no sólo se confían a las manos del Padre amadísimo, Dios de bondad y de misericordia, sino que en la mayoría de los casos se apresuran al encuentro del Padre común y de nuestro Salvador Jesucristo, que nos despejó el camino. Ni es correcto pensar –como hacen los paganos–, que estas almas estén revoloteando en torno a sus tumbas, en espera de los sacrificios ofrecidos por los muertos, o bien que sean arrojadas, como las almas de los pecadores, en el lugar del inmenso suplicio, esto es, en el infierno.

Cristo entregó su alma en las manos del Padre, para que en ella y por ella logremos nosotros el comienzo de la luminosa esperanza, sintiendo y creyendo firmemente que, después de haber soportado la muerte de la carne, estaremos en las manos de Dios, en un estado de vida infinitamente mejor que el que teníamos mientras vivíamos en la carne. Por eso el Doctor de los gentiles escribe que es mucho mejor partir de este cuerpo para estar con Cristo.
\end{body}

\begin{patercite}
	Él vino desde los cielos a la tierra a causa de los sufrimientos humanos; se revistió de la naturaleza humana en el vientre virginal y apareció como hombre; hizo suyas las pasiones y sufrimientos humanos con su cuerpo sujeto a la pasión y destruyó las pasiones de la carne, de moda que quien por su espíritu no podía morir acabó con la muerte homicida.
	
	Se vio arrastrado como un cordero y degollado como una oveja, y así nos redimió de idolatrar al mundo, como en otro tiempo libró a los israelitas de Egipto, y nos salvó de la esclavitud diabólica, como en otro tiempo a Israel de la mano del Faraón; y marcó nuestras almas con su propio espíritu y los miembros de nuestro cuerpo con su sangre.
	
	Este es el que cubrió a la muerte de confusión y dejó sumido al demonio en el llanto, como Moisés al Faraón. Este fue el que derrotó a la iniquidad y a la injusticia, como Moisés castigó a Egipto con la esterilidad. Este es el que nos sacó de la servidumbre a la libertad, de las tinieblas a la luz, de la muerte a la vida, de la tiranía al recinto eterno, e hizo de nosotros un sacerdocio nuevo y un pueblo elegido y eterno. El es la Pascua de nuestra salvación.
	
	Este es el que tuvo que sufrir mucho y en muchas ocasiones: el mismo que fue asesinado en Abel y atado de pies y manos en Isaac, el mismo que peregrinó en Jacob y fue vendido en José, expuesto en Moisés y sacrificado en el cordero, perseguido en David y deshonrado en los profetas. Este es el que se encarnó en la Virgen, colgado del madero, sepultado en tierra, y el que, resucitado de entre los muertos, subió al cielo.
	
	Este es el cordero sin voz; el cordero inmolado; el mismo que nació de María, la hermosa cordera; el mismo que fue arrebatado del rebaño, empujado a la muerte, inmolado de vísperas y sepultado en la noche; que no fue quebrantado en el leño, ni se descompuso en la tierra; el mismo que resucitó de entre los muertos e hizo que el hombre surgiera desde lo más hondo del sepulcro."
	
	\textbf{Melitón de Sardes}, \textit{Homilía} sobre la Pascua,  65-71: SC 123, 97-101.	
\end{patercite}	

\newsection
\subsection{San Cirilo de Alejandría, obispo}

\ptheme{Con su muerte corporal, Cristo redimió la vida de todos}

\src{Comentario sobre el evangelio de san Juan, Lib. 12: PG 74, 679-682.}

\begin{body}
\ltr{T}{omaron} \textit{el cuerpo de Jesús y lo vendaron todo, con los aromas, según se acostumbra a enterrar entre los judíos. Había un huerto en el sitio donde lo crucificaron, y en el huerto un sepulcro nuevo donde nadie había sido enterrado todavía}. Fue contado entre los muertos el que por nosotros murió según la carne; huelga decir que él tiene la vida en sí mismo y en el Padre, pues ésta es la realidad. Mas para cumplir todo lo que Dios quiere, es decir, para compartir todas las exigencias inherentes a la condición humana, sometió el templo de su cuerpo no sólo a la muerte voluntariamente aceptada, sino asimismo a aquella serie de situaciones que son secuelas de la muerte: la sepultura y la colocación en una tumba. El evangelista precisa que en el huerto había un sepulcro y que este sepulcro era nuevo. Lo cual, a nivel de símbolo, significa que con la muerte de Cristo se nos preparaba y concedía el retorno al paraíso. Y allí, en efecto, entró Cristo como precursor nuestro.

La precisión de que el sepulcro era nuevo indica el nuevo e inaudito retorno de Jesús de la muerte a la vida, y la restauración por él operada como alternativa a la corrupción. Efectivamente, en lo sucesivo nuestra muerte se ha transformado, en virtud de la muerte de Cristo, en una especie de sueño o de descanso. Vivimos, en efecto, como aquellos que –según la Escritura–, \textit{viven para el Señor}. Por esta razón, el apóstol san Pablo, para designar a los que han muerto en Cristo, usa casi siempre la expresión \textquote{los que se durmieron}.

Es verdad que en el pasado prevaleció la fuerza de la muerte contra nuestra naturaleza. \textit{La muerte reinó desde Adán hasta Moisés, incluso sobre los que no habían pecado con un delito como el de Adán}, y, como él, llevamos la imagen del hombre terreno, soportando la muerte que nos amenazaba por la maldición de Dios. Pero cuando apareció entre nosotros el segundo Adán, divino y celestial que, combatiendo por la vida de todos, con su muerte corporal redimió la vida de todos y, resucitando, destruyó el reino de la muerte, entonces fuimos transformados a su imagen y nos enfrentamos a una muerte, en cierto sentido, nueva. De hecho esta muerte no nos disuelve en una corrupción sempiterna, sino que nos infunde un sueño lleno de consoladora esperanza, a semejanza del que para nosotros inauguró esta vía, es decir, de Cristo.
\end{body}


\newsection
\section{Homilías}

\rbr{Tradicionalmente el Santo Padre no preside la celebración litúrgica del Viernes Santo. Recogemos aquí las Alocuciones en el \textit{Vía Crucis}, que aplican también perfectamente para la celebración de la Pasión del Señor.}

\subsection{San Pablo VI, papa} 

\subsubsection{Alocución (1964): Sufrimiento que da fruto}

\src{27 de marzo de 1964.}

\begin{body}
\ltr{A}{cabamos} de contemplar la Pasión del Señor en el Señor. Queremos creer que todos vosotros habréis intuido su profundidad y riqueza. Ahora extenderemos una mirada a la irradiación de esta Pasión, única y típica, centro de los destinos humanos, sobre la humanidad misma. Es el faro que ilumina al mundo. \textit{Crux lux}.

Uno de estos aspectos es el sufrimiento humano. Está iluminado de un modo bien conocido, pero siempre singular, a la luz de la cruz el dolor (podríamos señalar todas las miserias, toda la pobreza, todas las enfermedades y hasta todas las debilidades, es decir, todas las condiciones que hacen una vida deficiente y necesitada de atenciones), el dolor aparece extrañamente asimilable a la Pasión de Cristo, como llamado a integrarse con ella, como constituyendo una condición de favor respecto a la redención obrada por la Cruz del Señor. El dolor se hace sagrado. Antes –y todavía, para quien se olvida que es cristiano– el sufrimiento parecía pura desgracia, pura inferioridad, más digna de desprecio y repugnancia que merecedora de comprensión, de compasión, de amor. Quien ha dado al dolor del hombre su carácter sobrehumano, objeto de respeto, de cuidados y de culto, es Cristo doliente, el gran hermano de todos los pobres, de todos los afligidos. Hay más, Cristo no demuestra solamente la dignidad del dolor; Cristo lanza un llamamiento al dolor. Esta voz, hijos y hermanos, es la más misteriosa y la más benéfica que ha atravesado la escena de la vida humana. Cristo invita al dolor a salir de su desesperada inutilidad, a ser, unido al suyo, fuente positiva de bien, fuente no sólo de las más sublimes virtudes –desde la paciencia hasta el heroísmo y la sabiduría–, sino también de capacidad expiadora, redentora, beatificante, propia de la Cruz de Cristo. El poder salvífico de la Pasión de Cristo puede hacerse universal e inmanente en nuestros sufrimientos, si –he ahí la condición– se acepta y soporta en comunión con sus sufrimientos. La \textquote{com-pasión}, de pasiva se hace activa; idealiza y santifica el dolor humano, lo complementa con el del Redentor (Cfr. \textit{Col} 1, 24). Todos, debemos recordar esta inefable posibilidad. Nuestros sufrimientos (siempre dignos de cuidados y remedios), se hacen buenos, preciosos. En el cristiano se inicia un arte extraño y estupendo, de saber sufrir, hacer que el propio dolor sirva para la redención propia y ajena.

Esta providencialidad del sufrimiento nos hace pensar en las condiciones, siempre tristes y ofensivas para los ideales humanos, en que la civilización moderna quisiera inspirarse, en las cuales todavía se encuentra en gran parte a la Iglesia católica. El cuerpo de Cristo está crucificado moralmente, pero con saña, todavía hoy, en muchas regiones del mundo; la Iglesia del silencio es todavía la Iglesia doliente, la Iglesia paciente, y en ciertos lugares, la Iglesia amordazada. Cristo podría preguntar, hoy también, a los modernos y hábiles perseguidores: \textquote{\ldots ¿por qué me persigues?} (\textit{Hch} 9, 4). Es triste para quien es objeto de tales tratos; es indigno para quienes los practican, aunque se enmascaren de hipocresías legales. Pero estamos seguros que estas prolongadas pasiones están fortificadas por la asistencia divina y consoladas por nuestra com-pasión y la de toda la fraternidad universal cristiana, y esperamos que sean precisamente, en virtud de la cruz de Cristo a la que se ofrecen y por la que sufren, fuente de gracia para cuantos las padecen, para toda la Iglesia y para todo el mundo.

Y otro aspecto, reflejo de la cruz de Cristo, sobre la faz de la tierra, es la paz. La paz, que es el bien supremo del orden humano, esa paz que es tanto más deseable, cuanto más se inclina el mundo a formas de vida interdependientes y comunitarias, de forma que una infracción de la paz en un punto determinado repercute sobre todo el sistema organizativo de las naciones; esa paz que se hace, por tanto, cada vez más necesaria y obligada; esa paz, que los esfuerzos humanos, aunque muy nobles y dignos de aplauso y de solidaridad, difícilmente consiguen tutelar en su integridad y sostener con otros medios que no sean el temor y el interés temporal. La paz de Cristo llueve de lo alto, es decir, proyecta sobre la tierra y entre los hombres motivos y sentimientos originales y prodigiosos; lo sabemos, y viene precisamente de Aquel, como escribe San Pablo, que \textquote{por divina complacencia debía recapitular en sí todas las cosas habiéndolas pacificado con su sangre desde su cruz} (Cfr. \textit{Col} 1, 20), de forma que los hombres, divididos y enemigos entre sí fueran \textquote{reconciliados en un cuerpo único por medio de la cruz} (Cfr. \textit{Ef} 2, 16). Cristo Redentor nos ha enseñado cómo y por qué los hombres debemos y podemos vivir en la verdadera paz, y nos la ha conseguido si de verdad queremos.

Terminaremos esta conmovida y pública oración del Viernes Santo pidiendo a Cristo \textquote{nuestra paz} (\textit{Ef} 2, 14.) la paz para el mundo. En este momento están presentes a nuestro espíritu, los puntos geográficos y políticos, donde está herida la paz, donde está amenazada. Enviamos nuestro pensamiento lleno de buenos augurios a los hombres que se esfuerzan rectamente por salvar la paz, y para que los hombres sepan mantenerse hermanos en Cristo enviamos al mundo –y a vosotros aquí presentes que oráis y esperáis–, nuestra bendición apostólica.
\end{body}

\subsubsection{Alocución (1967): Sello de autenticidad del discípulo}

\src{24 de marzo de 1967.}

\begin{body}
\ltr{E}{n} este lugar, que nos habla del testimonio de fe, fortaleza y sangre de tantos Mártires por el Nombre de Cristo, \txtsmall{[en este día, en el que el doloroso recuerdo de las víctimas de la Fosse Ardeatine revive en Roma] (\ldots)} En esta hora de la historia contemporánea (\ldots) meditamos sobre la Pasión del Señor. Esta forma de meditación, casi guionizada, y alternada con cánticos y oraciones, nos ayuda no solo a recordar los sufrimientos de Cristo, sino a descubrir, en cierta medida, la profundidad, el drama, el misterio sumamente complejo, donde el dolor humano en su grado más alto, el pecado humano en su más trágica repercusión, el amor en su expresión más generosa y heroica, la muerte en su más cruel victoria y su derrota definitiva\ldots adquieren la evidencia más impresionante. 

\homsec{Teniendo un concepto exacto de Cristo y del cristianismo}

Haremos bien en grabar esta dolorosa pero sabia meditación en nuestras almas; recordarlo, repetirlo. Por dos motivos. 

El primero, tener un concepto exacto de Cristo y del cristianismo. La Pasión de Cristo ocupa un lugar esencial en el Evangelio. Existe una tendencia generalizada a mantener cerradas las páginas del Evangelio, que documentan el trágico epílogo de la corta vida temporal de Jesús; son páginas inquietantes. Quisiéramos un Evangelio más sereno, más fácil, más cómodo, más acorde con nuestro muy fuerte instinto y nuestro muy hábil interés de quitar el dolor de la vida, y ante todo el dolor voluntario, es decir, el sacrificio. ¿Qué sería un Evangelio, es decir, un cristianismo, sin la cruz, sin dolor, sin el sacrificio de Jesús? Sería un evangelio, un cristianismo sin la redención, sin la salvación, de la que –aquí debemos reconocerlo con despiadada sinceridad– necesitamos absolutamente. El Señor nos salvó con la Cruz; nos devolvió la esperanza, el derecho a la vida con su muerte. No podemos honrar a Cristo si no lo reconocemos como nuestro Salvador; y no podemos reconocerlo como nuestro Salvador si no honramos el misterio de su Cruz. 


\newpage 
\homsec{Llevando nuestra cruz: Jesús estará con nosotros}

Y luego debemos repetir la invocación con la que varias veces, en cada estación del \textit{Vía Crucis}, solemos dirigirnos a Nuestra Señora, la más afligida Madre de Cristo: ¡eh! ¡dejad que las llagas del Señor se graben en mi corazón! 

¿Por qué esta impresión? ¿No es suficiente que hayamos contemplado las llagas en el mismo Cristo? ¿No ha satisfecho Él todo por nosotros? Sí, Él nos ha salvado y cargó su Cruz por nosotros, ¿por qué deberíamos llevarla nosotros también? Esta es la segunda enseñanza del \textit{Vía Crucis}: el Señor hizo del dolor un medio de redención; con su dolor, sí, nos ha redimido, siempre que no nos neguemos a unir nuestro dolor al suyo, y hacer de él un medio para nuestra redención. En otras palabras: también nosotros debemos llevar, de alguna manera y en cierta medida, nuestra cruz, validada para la salvación por la Cruz de Cristo. 

¡Llevar la cruz! ¡Una cosa grande, una cosa grande, queridos hijos! Significa afrontar la vida con valentía, sin pusilanimidad y sin cobardía; significa transformar las inevitables dificultades de nuestra existencia en energía moral; ¡significa saber comprender el dolor humano y finalmente saber cómo amar de verdad! Significa aceptar el sello de autenticidad de los discípulos y seguidores de Cristo y establecer con él una comunión incomparable.
\end{body}

\subsubsection{Alocución (1970): Soy culpable de su sangre}

\src{27 de marzo de 1970.}

\begin{body}
\ltr{E}{sta} oración peregrinante por el camino de la Cruz nos deja al final muy pensativos. Sentimos que nosotros mismos hemos entrado en el plan profético de este doloroso drama; Jesús mismo lo había predicho: \textquote{cuando yo sea levantado de la tierra, atraeré a todos hacia mí} (\textit{Jn} 12, 32). Nos sentimos descritos por el texto bíblico, con el que el evangelista Juan concluye su relato de la crucifixión del Señor: \textquote{mirarán a aquel a quien traspasaron} (\textit{Jn} 19, 37). 

Sí, estamos mirando. Por atroz que sea la imagen de Jesús crucificado, nos sentimos atraídos por este Hombre de dolor; y la espantosa repugnancia que suele despertar la visión del cadáver de un ejecutado todo lleno de llagas y ensangrentado, se ve superado por una singular fascinación, que fija no solo nuestra mirada, sino más aún el alma en esa figura \textquote{sin ninguna belleza ni esplendor} (cfr. \textit{Is} 53, 2). Inmediatamente nos convencemos de que nos encontramos ante una revelación que va más allá de la imagen sensible; la revelación intencional de un símbolo, de un prototipo, de una personificación extrema del sufrimiento humano. Jesús, el Cristo, quería ser presentado así. 

¿Por qué así? ¡Oh! ¡Qué exploración se ofrece a nuestra piedad, a nuestra ciencia del hombre, a nuestra teología! Ciertamente no podemos consumarlo aquí, pero solo, en algunos puntos, decirlo. ¡Aquí el dolor parece consciente! La terrible pasión era prevista. ¡La tortura y el deshonor de la Cruz eran conocidos! y fueron queridos en su cruel totalidad hasta el final, sin los habituales narcóticos, que mitigan nuestro sufrimiento: desconocimiento de sí, de cuándo, de cómo vendrá; o más bien el misericordioso y sabio alivio del arte médico. Jesús es el \textquote{que conoce la enfermedad} en toda su extensión, en toda su profundidad, en toda su intensidad, en todo su horror , tanto como para exprimir la sangre de sus venas en la agonía espiritual de Getsemaní. Y eso es suficiente para hacerlo hermano de todo hombre que llora y sufre; hermano mayor, nuestro hermano. Tiene una primacía que centra en él la simpatía, la solidaridad, la comunión de todo hombre sufriente. 

Y luego: vemos en este sublime protagonista del dolor humano otra nota, también brillando en él más que en cualquier otro afectado por nuestros dolores: la inocencia. Cuando nos encontramos con un niño que está sufriendo, cuando observamos a alguien que suma al sufrimiento físico o moral la agonía de una pregunta ciega, que parece quedar sin respuesta: ¿por qué? ¿Por qué este desorden, por qué este atropello inexplicable al derecho fundamental de la existencia, a vivir bien, cuando la experiencia del mal arrecia sin razón aparente? Misterio, sí, el dolor inocente es un misterio para nosotros; pero el encuentro que hacemos de este misterio en el divino Crucifijo, en Él, el Supremo, el verdaderamente inocente (cf. \textit{Lc} 23, 41) al menos detiene la blasfemia que vendría a nuestros labios. Jesús también era inocente, era un cordero, era el cordero de Dios, que humilde y débil se dejó llevar al matadero (\textit{Is} 53, 7). Si es así, la pregunta vuelve a surgir, pero ya no desesperada y rebelde, sino ansiosa por una respuesta anticipada y prodigiosa.

Y es esta: Jesús murió inocente, porque Él lo quiso (\textit{Ibid}.: \textit{Jn} 10, 17-18). Pero, ¿por qué lo quería? Aquí está la clave de toda esta tragedia: porque quería asumir toda la expiación de la humanidad (\textit{Is} 53, 6; \textit{Jn} 11, 51; \textit{2 Cor} 5, 21); Él se ofreció como víctima para reemplazarnos; sí, Él es \textquote{el Cordero de Dios que quita el pecado del mundo} (\textit{Jn} 1, 29); Él se sacrificó por nosotros; Él se entregó a sí mismo por nosotros; ¡así Él nos redimió! ¡Él es así nuestra salvación! 

Y por eso el Crucifijo encadena nuestra atención casi alucinada: si Cristo ha asumido sobre sí la deuda de la justicia de Dios por mis faltas, yo soy corresponsable, ¡soy culpable de su sangre! y entonces ese descubrimiento se convierte en alegría, que estalla en gratitud y amor: \textquote{Me amó y se sacrificó por mí} (\textit{Gá} 2, 20). 

Y todo desemboca en la verdadera ciencia del amor, que traeremos a nuestra vida desde este Viernes Santo: es el dolor consciente, inocente, sufrido por el amor que redime y salva; como Cristo, debemos entregarnos voluntaria, libremente y hasta dolorosamente, por el bien de los demás, por la redención de la humanidad, por la salvación y la paz del mundo. ¡Así volvemos afligidos, pensativos, valientes, después del Vía Crucis!
\end{body}


\subsubsection{Alocución (1973): Cruz que irradia la esperanza}

\src{20 de abril de 1973.}

\begin{body}
\ltr{E}{ste} doloroso camino de la Cruz nos ha llevado a la última estación, a la del sepulcro, a la de la piedad, donde María, la Madre del divino Hijo, tiene en su seno el fruto del dolor, el odio y la muerte: el cadáver de Cristo crucificado. 

Hermanos ¡viajeros de esta común peregrinación! ¿Sabéis por qué todos nos sentimos atraídos, casi a pesar de nosotros mismos, por este espectáculo tan triste? ¿Y por qué no podemos separar nuestro corazón de esta contemplación terminal y trágica? ¿Por qué nuestra mirada no se horroriza al ver un tormento humano tan cruel y tan horrendo, infligido al más bello y al mejor de los hijos de la humanidad, a nuestro hermano más cercano y solidario, al amigo, al maestro, al pastor de todos nosotros, al Verbo de Dios hecho hombre como nosotros? 

¿No sentimos asco y miedo? ¿No sentimos remordimiento? ¿No nos asalta un extraño e instintivo sentido de corresponsabilidad? ¿No nos sentimos todos nosotros cómplices de esta muerte, la más humillante, la más injusta entre las que han ensangrentado la tierra? ¿No se reflejan todos nuestros pecados en el Cristo inmolado, como en la víctima más sensible y central del mundo entero? Sí, cada uno de nosotros puede acusarse ante la pasión y muerte de Jesús: ¡también es culpa mía! 

Sin embargo, hacia Él, hacia el Cristo crucificado, nos sentimos atraídos, en esta hora de tinieblas, pero atravesados por los destellos de una nueva conciencia; atraídos, decimos: Él lo había predicho: \textquote{Cuando yo sea levantado de la tierra (quiso decir: cuando sea exaltado en la cruz), atraeré a todos hacia mí} (\textit{Jn} 3, 14; 12, 32).

Hermanos, dejemos que este encanto misterioso nos domine con su doble sentimiento, de reproche y de esperanza. 

De reproche: ¿acaso no reflejan crudamente las heridas todavía sangrantes de Cristo toda la violencia, la tortura, las masacres, la barbarie, el odio, la maldad, la soberbia, la insensibilidad de las que es capaz el hombre moderno? Sí, el hombre ha alcanzado grandes avances en la civilización, pero sigue siendo miope sobre cómo usarlos sabiamente. Y ahora nos decimos a nosotros mismos: ¡que cesen los atropellos contra la vida y la dignidad de los hombres! ¡Que acabe la impasible inhumanidad, que ataca la vida inocente e indefensa que nace en el útero! ¡Que cese el crimen, que hoy se profesionaliza y organiza! ¡Que termine la estrategia, que se basa en la competencia por el poder mortal de las armas científicas! ¡Que acabe la degradante licencia del vicioso placer, erigido como ideal de libertad y felicidad ciega y egoísta! Esta invectiva podría extenderse hasta donde llega la degradación humana, ¡muy lejos! 

Pero escuchemos más bien las efusiones de esperanza que irradia la Cruz de Cristo. La primera esperanza es la misericordia, el perdón, la reconciliación de Dios con nosotros. Así como el pecado, recordemos bien, es nuestra primera y más grave desgracia, porque corta nuestra relación con la verdadera Vida, que es Dios, así la liberación del pecado es nuestra primera e indispensable fortuna. Y qué suerte para nosotros saber que Cristo, con Su Sangre, ha pagado por nosotros, ha expiado por nosotros, ha reparado nuestra irremediable maldición; y nos hizo levantarnos a una nueva existencia, y tener la esperanza de la felicidad eterna.

Y nos devolvió, con su muerte por amor, el amor a nuestros hermanos, nos enseñó a perdonar, a sentir las necesidades de los demás, a servir a los más débiles, a sacrificarnos por los demás; es decir, llevar humanidad a los hombres, bondad y justicia al mundo. Y por tanto paz. Y luego también: nos enseñó el valor del sufrimiento y la fecundidad del dolor, la dignidad en la desgracia. Y nos ha concedido, porque así lo ha querido, que su cruz, profecía y garantía de resurrección, fuera plantada en cada tumba. 

¡Hermanos! No pongamos fin a este camino hacia la Cruz sin el secreto propósito personal de continuarlo. Cristianos somos y debemos ser, y por eso respondemos con el corazón y con nuestra forma de vivir a la invitación de Cristo: \textquote{Venid a mí todos los que estáis cansados y oprimidos, y yo os consolaré} (\textit{Mt} 11, 28).

Que nuestra Bendición Apostólica conforte ahora en cada uno de nosotros estos sentimientos y resoluciones.
\end{body}

\subsubsection{Alocución (1976): Una muerte que nos concierne}

\src{16 de abril de 1976.}

\begin{body}
\ltr{H}{emos} completado el \textquote{Vía Crucis}, el camino de la Cruz. Seguimos este triste y trágico itinerario, recordando paso a paso la ejecución bárbara y cruel del condenado Jesús, el Maestro, el predicador del reino de Dios, el buen Pastor \textquote{manso y humilde de corazón} (\textit{Mt} 11, 29), quien había pasado \textquote{haciendo el bien y sanando a todos} (\textit{Hch} 10, 38), y se había llamado a sí mismo el Hijo del hombre y luego el Hijo de Dios, el Mesías y por lo tanto, el \textquote{Rey de los judíos}, desatando contra sí mismo la furia de los líderes del pueblo y la tácita condena del Procurador Romano Poncio Pilato: un drama complicado de pretextos políticos (\textit{Jn} 11, 48; 19, 12), y aún más de razones religiosas (\textit{Mt} 26, 63-64; \textit{Jn} 11, 51; 19, 7); una muerte desgarradora e injusta; un episodio violento y doloroso, ciertamente, como el de quien hace de la muerte un testigo, un martirio; que terminó pronto, a la hora novena de ese día santo cercano a la Pascua Oficial, concluido súbitamente con un entierro apresurado. \textquote{\textit{Consummatum est} – Todo está cumplido} \textit{(Jn} 19, 30), exclamó Jesús agonizante.

¡Hijos y hermanos! en nuestros ojos, en nuestras almas, se ha reproducido la desgarradora historia de Jesús; nos cautiva y quizás incluso nos conmueve, como ocurre con las escenas sangrientas y los casos dramáticos y singulares. Pero queda una duda, una cuestión por resolver; que ahora nos concierne; nosotros personalmente: ¿estamos involucrados en esta historia? ¿cómo hemos asistido? ¿como meros y extraños espectadores? ¿como curiosos y eruditos de la muerte de un hombre sabio y justo, como lo fue, por ejemplo, la muerte de Sócrates? No, hermanos e hijos; ¡No asistimos como observadores curiosos e impasibles! No, todos prestamos atención a la conclusión de esta historia, que nos involucra a todos. Lo queramos o no, somos corresponsables de la muerte de Jesús, esta es la primera conclusión que este piadoso ejercicio del \textquote{Vía Crucis} debe suscitar en nuestras conciencias. Sabemos bien que la afirmación de nuestra culpa en la crucifixión de Cristo requeriría pruebas formidables, que nuestros tribunales no podrían reconocer como legales; pero la realidad de la historia humana, como nos recuerda la más sabia teología, hace de toda la humanidad la causa de la muerte de la víctima divina. Una solidaridad universal hace a todos los hijos de Adán culpables y a todos deudores ante Dios, con esta doble conclusión: la primera, que todo hombre pesa en la balanza de la redención, en la necesidad de expiación, de la que Cristo es víctima, \textquote{el Cordero de Dios que quita el pecado del mundo} (\textit{Jn} 1, 29. 36 ); los santos, estos conocedores de la conciencia humana profunda y real, sintieron esta experiencia moral, es decir, cómo cada uno de nosotros fue verdugo en la crucifixión del Señor (cf. \textit{Heb} 6, 6), porque todo pecado humano contribuye a la necesidad de una reparación, que sólo la Palabra de Dios Salvador, que vino al mundo por nuestra salud, podía ofrecer a la justicia y misericordia de Dios. Y segunda conclusión: de los crucificadores nos hemos convertido en beneficiarios, en salvados por la víctima misma sacrificada por nosotros, en nuestro nombre, por nuestra salvación. Cuando hablamos de redención, de sacrificio divino, nos referimos a este drama, donde el culpable puede ser recompensado por el arrepentimiento de su fechoría.

Tal es el misterio detrás del \textquote{Vía Crucis}. El misterio de la redención, el misterio de nuestra salvación, el misterio de la virtud redentora de nuestro dolor si se une al de la pasión de Cristo (cf. \textit{Col} 1, 24), el misterio del amor inmolado de Cristo, que él hizo de su muerte fuente de nuestra vida eterna (cf. \textit{Heb} 5, 9).

Así, disolviendo esta asamblea orante y ansiosa con nuestros deseos pascuales y con nuestra bendición apostólica, cada uno de nosotros puede hacer suyo el testimonio amargo, pero renovador y muy feliz del Centurión en el momento de la muerte de Cristo: \textquote{Verdaderamente este era el Hijo de Dios} (\textit{Mt} 27, 54).
\end{body}


\newsection
\subsection{San Juan Pablo II, papa}

\subsubsection{Alocución (1979): Quédate con nosotros}

\src{13 de abril de 1979.}

\begin{body}
\ltr[1. ]{E}{stamos} siempre presentes en espíritu allí donde \textit{este camino} [de la cruz] \textit{tuvo su lugar \textquote{históricamente}:} allí donde se desarrolló, a lo largo de las calles de Jerusalén, desde el Pretorio de Pilato hasta la cima del Gólgota, es decir, del Calvario, fuera de las murallas. Así, pues, también hoy hemos estado en espíritu allí, en la ciudad del \textquote{gran Rey}, que como signo de su realeza ha escogido la corona de espinas en vez de la corona real, y la cruz en lugar del trono.

¿No tenía razón Pilato cuando, presentándolo al pueblo que esperaba su condenación ante el Pretorio \textquote{por no contaminarse, para poder comer la Pascua} (\textit{Jn} 18, 28), en vez de decir \textquote{He aquí al rey}, dijo \textquote{Ahí tenéis al hombre} (\textit{Jn} 19, 5)? Y así reveló el programa de su reino que quiere verse libre de los atributos del poder terreno para descubrir los pensamientos de muchos corazones (cf. \textit{Lc} 2, 35) y para acercarlos a la verdad y al amor que proviene de Dios.

\textquote{Mi reino no es de este mundo\ldots Yo para esto he nacido y para esto he venido al mundo, para dar testimonio de la verdad} (\textit{Jn} 18, 36-37). Este testimonio ha permanecido en las esquinas de las calles de Jerusalén, en los recodos del \textit{Vía Crucis}, allí por donde caminaba, donde cayó por tres veces, donde aceptó la ayuda de Simón Cirineo y el velo de la Verónica, allí donde habló a algunas mujeres que se apiadaban de Él. Hoy día seguimos aún deseosos de este testimonio. Queremos conocer todos los detalles. Seguimos las huellas del \textit{Vía Crucis} en Jerusalén y a la vez en tantos otros lugares de nuestra tierra; y cada vez nos parece repetir a este Condenado, a este Hombre de dolores: \textquote{Señor, ¿a quién iríamos? Tú tienes palabras de vida eterna} (\textit{Jn} 6, 68).

\newpage 
2. [Hoy] (\ldots) estamos también sobre las huellas de Cristo, cuya cruz encontró sitio en los corazones de sus mártires y confesores. Ellos anunciaban a Cristo crucificado como \textquote{poder y sabiduría de Dios} (\textit{1 Cor} 1, 24). Tomaban cada día la cruz en unión con Cristo (cf. \textit{Lc} 9, 23), y cuando era necesario morían como Él en la cruz, o morían sobre la arena de la Roma antigua, devorados por las fieras, quemados vivos o torturados. El poder de Dios y la sabiduría de Dios revelados en la cruz, se manifestaban así más poderosamente en las debilidades humanas. Ellos no sólo aceptaban los sufrimientos y la muerte por Cristo, sino que se decidían como Él por el amor a los perseguidores y a los enemigos: \textquote{Padre, perdónalos, porque no saben lo que hacen} (\textit{Lc} 23, 34).

\txtsmall{[Por esto, \textit{sobre las ruinas del Coliseo se levanta la cruz.}]}

Mirando hacia esta cruz, la cruz de los comienzos de la Iglesia (\ldots) y la cruz de su historia, debemos sentir y expresar una \textit{solidaridad} particularmente profunda con todos nuestros hermanos en la fe, que también en nuestra época s\textit{on objeto de persecuciones y de discriminaciones} en diversos lugares de la tierra. Pensemos ante todo en aquellos que están condenados, en cierto sentido, a la \textquote{muerte civil} por la denegación del derecho a vivir según la propia fe, el propio rito, según las propias condiciones religiosas. Mirando hacia la cruz (\ldots), pedimos a Cristo que no les falte –al igual que a aquellos que en otro tiempo sufrieron (\ldots) el martirio– la fuerza del Espíritu, de que tienen necesidad los confesores y los mártires de nuestro tiempo.

Mirando a la cruz (\ldots) sentimos una unión aún más profunda con ellos, una solidaridad aún más fuerte. Al igual que Cristo tiene \textit{un lugar especial en nuestros corazones} por su pasión, así también ellos. Tenemos el deber de hablar de esta pasión de sus confesores contemporáneos, y darles testimonio ante la conciencia de la humanidad entera, que proclama la causa del hombre, como finalidad principal de todo progreso. ¿Cómo conciliar estas afirmaciones con la lesión causada a tantos hombres, que –mirando a la cruz de Cristo– confiesan a Dios y anuncian su amor?

3. ¡Cristo Jesús! Estamos para terminar este santo día del Viernes Santo a los pies de tu cruz. Así como en otro tiempo, en Jerusalén, a los pies de tu cruz se encontraban tu Madre, Juan, Magdalena y otras mujeres, así también estamos aquí nosotros. Estamos profundamente emocionados por la importancia del momento. Nos faltan las palabras para expresar todo lo que sienten nuestros corazones. 

Ahora, en esta noche –cuando después de haberte bajado de la cruz, te han colocado en un sepulcro en la ladera del Calvario–, queremos suplicarte \textit{que permanezcas con nosotros mediante tu cruz}:


\newpage 
\begin{bodyprose}
Tú que por la cruz te has separado de nosotros. 
   Te suplicamos que permanezcas con la Iglesia; 
   que permanezcas con la humanidad; 
   que no te asustes si muchos pasan 
   tal vez indiferentes al lado de tu cruz, 
   si algunos se alejan y otros no se llegan a ella.

No obstante, tal vez hoy más que nunca 
   el hombre tiene necesidad de esta fuerza 
   y de esta sabiduría que eres Tú mismo, 

¡Tú solo: mediante tu cruz!
   Quédate, pues, con nosotros 
   en este penetrante \textit{mysterium} de tu muerte, 
   con la que has revelado cuánto \textquote{ha amado Dios al mundo} (cf. \textit{Jn} 3, 16). 

Quédate con nosotros y atráenos hacia Ti (cf. \textit{Jn} 12, 32), 
   Tú que caíste bajo el peso de esta cruz. 

Quédate con nosotros a través de tu Madre, 
   a la que desde la cruz has encomendado 
   particularmente a cada hombre (cf. \textit{Jn} 19, 27).

¡Quédate con nosotros!
\end{bodyprose}

\textit{Stat crux, dum volvitur orbis!} 

Sí, \textquote{¡la cruz está alzada sobre el mundo que avanza!}

\end{body}

\img{cross_tau}

\newpage 
\subsubsection{Alocución (1982): La Cruz nos abre a Dios}

\src{9 de abril de 1982.}

\begin{body}
1. \textquote{Crucem tuam adoramus}. 

\ltr{E}{ste} es el día en que adoramos especialmente la Cruz. La Cruz de Cristo. Este signo, infame instrumento de muerte, ha surgido desde el alba, ante nosotros y penetra las horas del Viernes Santo, durante el cual nos apresuramos solícitos, con nuestros pensamientos y corazones, detrás de la pasión del Señor: el camino desde el Pretorio de Pilato al Calvario; la agonía del Calvario. La muerte. Estas horas, llenas de silencio religioso, se hicieron oír más tarde, en la elocuencia de la liturgia de la tarde: \textit{la adoración de la Cruz}. \txtsmall{[Y ahora, a última hora de la tarde, llegamos al Coliseo para abrazar una vez más el conjunto: \textit{El \textquote{Vía Crucis}}: crucifixión - muerte - entierro.]}

2. (\ldots) la cruz nos recuerda efectivamente a todos aquellos que [en las primeras generaciones de la Iglesia] fueron condenados a la cruz, arrojados a las bestias, torturados de otras formas, martirizados hasta la muerte. Cayeron al suelo \textit{como una semilla} que debe morir para dar fruto y, mirando la Cruz de Cristo, repiten quizás sin palabras: \textquote{Crucem tuam adoramus}. La Cruz se ha convertido para ellos en el \textit{signo de la Vida} que nace del sufrimiento y de la muerte: \textquote{et sanctam resurrectionem tuam laudamus et glorificamus}. 

3. ¿Por cuántos lugares de la tierra ha pasado esta Cruz? ¿Por cuántas \textit{generaciones}? ¿Para cuántos discípulos de Cristo se ha convertido en el principal referente en la peregrinación terrena? ¿A cuántos ha preparado para el sufrimiento y la muerte? ¿A cuántos al martirio por Cristo, al testimonio sangriento o incruento? ¿Y a cuántos prepara continuamente para todo esto? \textit{La historia de la Iglesia}, en los distintos continentes y en los distintos países, sólo puede registrar una parte de este \textquote{martirologio}. Los altares de las iglesias no han podido acoger en su gloria a los que han dado testimonio de Cristo a través de la cruz. Bastaría pensar en los que vivieron en nuestro siglo. 

4. \textquote{Crucem tuam adoramus, Domine}. Sí. En la Cruz Cristo \textit{demostró ser Señor}: aceptó la muerte y dio la vida. No está simplemente \textquote{muerto}, sino que \textquote{\textit{ha dado la vida}}. \textquote{Nadie tiene mayor amor que el que da la vida por sus amigos} (\textit{Jn} 15, 13). ¡Él ha dado su vida! Acogió la muerte y dio la vida. 

Sus últimas palabras en la Cruz: \textquote{¡Padre, en tus manos encomiendo\ldots encomiendo mi espíritu} (cf. \textit{Lc} 23, 46). Dio su vida por nosotros. Por \textit{todos} los hombres. \textquote{Nosotros} somos sólo una pequeña parte de todos aquellos por quienes Cristo dio su vida. No hay hombre, \textit{desde el principio hasta el fin del mundo}, por quien no haya dado su vida. Él dio su vida por todos. Ha redimido a todos. La Cruz es signo de \textit{redención universal}: \textquote{excepción enim propter lignum venit gaudium en universo mundo}. 

5. \textquote{Venit gaudium\ldots}. La Cruz es la puerta por la que Dios entró definitivamente en la historia de la humanidad. Y permanece en ella. La Cruz es la puerta por la que Dios entra incesantemente en nuestra vida. Precisamente por eso \textit{nos signamos con la señal de la Cruz}, y al mismo tiempo decimos \textquote{en el nombre del Padre y del Hijo y del Espíritu Santo}. Y mientras trazamos el signo de la Cruz en la frente, entre los hombros y en el corazón, también pronunciamos esas palabras. Estas palabras son \textit{una invitación a que venga Dios}. Y nos unimos al signo de la Cruz, para que Dios entre en el corazón del hombre a través de la Cruz. Y así entra en cada obra, pensamiento y palabra: \textit{en toda la vida del hombre y del mundo}. La Cruz nos abre a Dios. La Cruz abre el mundo a Dios. 

6. Y la bendición también se da con el signo de la Cruz. También los obispos y sacerdotes. Los padres también cuando bendicen a sus hijos. Por la Cruz de Cristo esperamos el bien definitivo de Dios mismo y todos los bienes que nos acercan a él. Todo esto se expresa \textit{en cada bendición}. También de la que impartiré en breve. 

\textquote{Stat crux, dum volvitur orbis}. Todo pasa; la Cruz permanece entre el mundo y Dios. Por la Cruz Dios permanece en el mundo. \textquote{Crucem tuam adoramus, Domine}. 

7. ¡Queridos hermanos y hermanas! Que este día de Viernes Santo, dedicado al misterio de la Cruz, en el que hoy hemos meditado, nos acerque cada vez más al Dios vivo: Padre, Hijo y Espíritu Santo. Que el signo de la muerte de Cristo \textit{vivifique} su presencia y su fuerza en nosotros. 

\textit{Amén.}
\end{body}

\newpage 
\subsubsection{Alocución (1985): Muerte y amor se encuentran}

\src{5 de abril de 1985.}

\begin{body}

1. \textquote{\textit{Padre, perdónalos, porque no saben lo que hacen}} (\textit{Lc} 23, 34). 

\ltr{A}{l} final de este día, de este Viernes Santo, volvemos una vez más bajo la cruz, al Calvario. Aunque Cristo ya haya sido bajado de la cruz y aunque haya sido colocado apresuradamente en un sepulcro, \textquote{con motivo de la fiesta de la Pascua}, \textit{escuchemos una vez más las palabras} que pronunció en el momento de su agonía en la cruz. ¿No es acaso una \textit{revelación del amor} \textit{que perdona}, Aquel que, ante todo, pide al Padre que perdone a los responsables de su tormento: \textquote{Perdónalos}? ¡Eso es lo que significa: \textquote{No saben lo que hacen}! 

2. \textquote{\textit{En verdad te digo que hoy estarás conmigo en el paraíso}} (\textit{Lc} 23, 43). ¿Quién es ese Crucificado que le habla así a otro condenado? ¿No es el mismo Jesús de Nazaret que al inicio de su misión mesiánica dijo: \textquote{El reino de Dios está cerca; convertíos y creed en el Evangelio} (\textit{Mc} 1, 14-15)? He aquí que, al final de su ministerio mesiánico, también recoge el fruto en el alma del criminal convertido: \textquote{Jesús, acuérdate de mí cuando entres en tu reino} (\textit{Lc} 23, 42). Las palabras de Jesús son una respuesta a esta misma pregunta. Hermano, el reino de Dios está cerca, el reino de Dios está en tu alma. 

3. Y he aquí, desde lo alto de la cruz, \textit{una nueva palabra del Hijo del Hombre}. ¡Cuán importante es esta palabra que en cierto sentido completa todo el Evangelio! ¡Cuán profundamente nace del corazón del Evangelio! 

\textquote{Mujer, he ahí a tu hijo\ldots Ahí tienes a tu madre} (\textit{Jn} 19, 27). La Madre pierde al hijo y, al mismo tiempo, \textit{recibe un hijo}; recibe muchos hijos e hijas. Todos y cada uno de aquellos a quienes el Hijo ha dado el poder \textquote{de ser hijos de Dios} (\textit{Jn} 1, 12);\textit{ hijos en el Hijo}. El discípulo recibe a la Madre. La Iglesia recibe a la Madre. \textit{La humanidad recibe a la Madre}. Es maravillosa la riqueza \textit{con la cual nos enriquece}, quien por nosotros se ha hecho pobre. 

4. \textquote{Tengo sed} (\textit{Jn} 19, 28). Sí. Los labios secos tienen sed, el paladar y la lengua quemados por la fiebre de la agonía. El alma de Cristo tiene aún más sed. Sed infinita que lo abraza todo. Sed que, desde el principio, va hasta el fin y más allá del fin: hasta \textquote{que el Hijo le someta todas las cosas, para que Dios sea todo en todos} (cf. \textit{1 Co} 15, 28). 

5. \textquote{Dios mío, Dios mío, ¿por qué me has abandonado}? (\textit{Sal} 22, 2). El Salmo 22 comienza con estas palabras, y el Hijo del hombre que muere en la cruz, el siervo de Yahvé de la profecía de Isaías, \textit{comienza a orar} con las palabras de este \textit{salmo}. Pero, ¿hasta dónde llegan estas palabras? ¿Cómo puede Dios ser abandonado por Dios? ¿El Hijo abandonado por el Padre? Desde el punto de vista humano, esto parece imposible e inconcebible. 

Sin embargo en Dios\ldots 

\textit{Cuando el Hijo es abandonado por el Padre en el Espíritu Santo} ese abandono contiene \textit{la plenitud definitiva de su amor salvador}: la plenitud de la unidad del Hijo con el Padre en el Espíritu Santo. 

6. \textquote{¡Todo está cumplido!} (\textit{Jn} 19, 30). Una vez salieron del salmo estas palabras: \textquote{He aquí que vengo, oh Dios, para hacer tu voluntad} (cf. \textit{Sal} 40, 8-9). Estas palabras pasaron por la agonía en Getsemaní: \textquote{Padre, si quieres, ¡que pase de mí esta copa! \textit{Pero no se haga mi voluntad, sino la tuya}}(\textit{Lc} 22, 42). Y ahora las mismas palabras vuelven al final, para asentarse como el sello sobre la víctima de la redención: \textquote{He aquí que vengo}\ldots \textquote{\textit{Todo está cumplido}}. 

7. \textquote{\textit{Padre, en tus manos encomiendo mi espíritu}} (\textit{Lc} 23, 46). Jesucristo, Hijo del hombre, acepta la muerte humana, herencia del primer Adán. Esta muerte del cuerpo es la \textquote{\textit{entrega del espíritu}} a Dios. Toda muerte humana encuentra su modelo en la muerte de Cristo: es la entrega del espíritu al que creó al hombre \textit{para la inmortalidad}. El Hijo de Dios, que como verdadero hombre entrega el espíritu, se une a través de esto \textit{con el Padre en el Espíritu Santo}, que es el amor mutuo del Padre y del Hijo: él es su aliento eterno. 

La muerte y el amor se encuentran al final del sacrificio de la cruz. La victoria de la cruz tiene su comienzo en este hecho. El amor gana con la muerte. Al final del Viernes Santo volvemos una vez más bajo la cruz de Cristo en el Calvario. Reflexionamos sobre las palabras que Cristo pronunció desde lo alto de la cruz. 

Al salir de este lugar, llevemoslas con nosotros como testimonio de nuestra redención. \textquote{Te adoramos a ti oh Cristo y te bendecimos, porque con tu santa cruz has redimido al mundo}. Amén.

\end{body}


\subsubsection{Alocución (1988): Espada que traspasa el alma}

\src{1 de abril de 1988.}

\begin{body}

\ltr[1. «]{E}{staban} junto a la cruz de Jesús su madre, la hermana de su madre, María la de Cleofás y María de Magdala. Entonces Jesús, viendo a su madre y junto a ella al discípulo a quien amaba, le dijo a su madre: ¡Mujer, aquí tienes a tu hijo! Luego dijo al discípulo: ¡Ahí tienes a tu madre! Y desde ese momento el discípulo la acogió en su casa» (\textit{Jn} 19, 25-27). 

2. \textquote{Stabat Mater\ldots}. [Acabamos de recorrer el \textquote{Vía Crucis}, a lo largo del cual meditamos sobre el encuentro de la Madre con el Hijo, en la cuarta estación.] El Concilio enseña: \textquote{La Santísima Virgen avanzó en la peregrinación de la fe, y mantuvo fielmente su unión con el Hijo hasta la cruz, junto a la cual, no sin designio divino, se mantuvo erguida} (\textit{Lumen Gentium}, 58). 

3. Este plan divino fue revelado a María ya cuarenta días después del nacimiento de Jesús cuando, durante la Presentación en el templo de Jerusalén, se escucharon las palabras proféticas del viejo Simeón: \textquote{Él está aquí para la ruina y resurrección de muchos en Israel, y será signo de contradicción} (\textit{Lc} 2, 34) esto, en cuanto al Hijo. Y luego a la Madre: \textquote{Y a ti, una espada te traspasará el alma, para que se revelen los pensamientos de muchos corazones} (cf. \textit{Lc} 2, 35). 

4. Entonces, \textquote{no sin un plan divino}, María estaba bajo la cruz en el Gólgota. La espada le atravesó el corazón y le provocó un dolor indecible: el mayor sufrimiento preparado para María en este camino de fe, en el que seguía a Cristo. 

\homsec{Sufrimiento -- consuelo.} 

El Concilio enseña que María correspondía al designio divino: \textquote{Sufrir profundamente con el unigénito y asociarse con su alma maternal} (\textit{Lumen gentium}, 58). La consolación une a la Madre con el Hijo, ya que solo la Madre Inmaculada pudo unirse con el Hijo de Dios en la cruz. \textquote{La espada del dolor} atravesó su alma hasta el punto de esta unión. 

5. El Concilio enseña además: María estuvo bajo la cruz \textquote{sufriendo profundamente\ldots consintiendo amorosamente la inmolación de la víctima que ella misma había engendrado} (\textit{Lumen Gentium}, 58). Aquí cada palabra tiene un peso particular. En la Anunciación, María había exclamado: \textquote{Que me suceda lo que has dicho} (\textit{Lc} 1, 38). Ahora renueva la misma disponibilidad en el momento del mayor dolor: \textquote{Consintiendo amorosamente} porque el que fue concebido por obra del Espíritu Santo, el \textquote{Santo de Dios}, su Hijo unigénito, sufrió como víctima el despojo en la cruz. 

6. Una mujer entre la multitud pronunció en voz alta una bendición ante Jesús por su Madre: \textquote{¡Bendito el vientre que te dio a luz y el pecho que te amamantó!} (\textit{Lc} 11, 27). Y Jesús respondió a estas palabras de manera maravillosa: \textquote{Bienaventurados los que escuchan la palabra de Dios y la guardan} (\textit{Lc} 11, 27). Ciertamente, en aquel momento, parecería que Cristo no acoge la bienaventuranza dirigida por aquella mujer a su Madre. Bajo la cruz se entiende que Cristo dirigió la bienaventuranza entonces manifestada hacia el futuro. ¿Quién es su Madre en este momento? He aquí, es ella la que está junto a la cruz, la que escucha con heroica obediencia de fe la palabra de Dios, que con todo el sufrimiento materno de su corazón \textquote{cumple}, junto con el Hijo, \textquote{la voluntad del Padre}. 

7. Y he aquí, así, en la agonía de la cruz de Cristo, ¡tu Madre, oh Juan, te es entregada! Así hemos recibido todos, queridos hermanos y hermanas, a María como Madre. Así recibiste tú, Iglesia del Pueblo de Dios, tu \textquote{figura} y tu modelo. 

\homsec{\textquote{Stabat mater\ldots}.}

Desde ese momento María \textquote{colaboró con su amor maternal en el nacimiento y la educación} de todos nosotros. De hecho, el Padre Eterno ha establecido que Cristo, Hijo de María, \textquote{sea el primogénito entre muchos hermanos} (\textit{Rom} 8, 29). 

\textquote{La maternidad de María en la economía de la gracia continúa sin cesar desde el momento del consentimiento dado en la Anunciación y mantenido sin vacilación bajo la cruz, hasta la coronación perpetua de todos los elegidos} (\textit{Lumen Gentium}, 62). 

8. Queridos fieles (\ldots). Que a la meditación sobre la pasión del Redentor se agregue esta palabra sobre la \textquote{espada del dolor}, que traspasó el corazón inmaculado de la Madre al pie de la cruz en el Gólgota. ¡Que a través de su sufrir-con el Hijo, \textquote{se revelen los pensamientos de muchos corazones} (cf. \textit{Lc} 2, 35)! ¡Que nuestros corazones estén unidos en el misterio de la redención del mundo, y que por la Madre de Dios permanezcan en unión con Cristo en el camino de la fe, la esperanza y la caridad! 

Amén.
\end{body}


\subsubsection{Alocución (1991): Amor redentor}

\src{29 de marzo de 1991.}

\begin{body}
\ltr{D}{e} la carta \textit{a los Hebreos}: \textquote{La sangre de Cristo, que con un Espíritu eterno se ofreció a sí mismo sin tacha a Dios, limpiará nuestra conciencia de las obras muertas, para que sirvamos al Dios vivo} (\textit{Heb} 9, 14). 

1. De la encíclica \textit{sobre el Espíritu Santo} \textit{Dominum et vivificantem}: \textquote{Las palabras de la Carta a los Hebreos ahora nos explican cómo Cristo \textquote{se ofreció sin tacha a Dios}, y cómo lo hizo \textquote{con un Espíritu eterno}. \textit{En el sacrificio del Hijo del Hombre está presente el Espíritu Santo y actúa} como actuó en su concepción, en su venida al mundo, en su vida oculta y en su ministerio público} (\textit{Dominum et vivificantem}, 40). Las mismas palabras también muestran \textquote{cómo la \textit{humanidad, sometida al pecado} en la descendencia del primer Adán, en Jesucristo se sometió perfectamente a Dios y se unió a él y, al mismo tiempo, se llenó de misericordia hacia los hombres} (\textit{Ibid.}). El mismo Cristo en su humanidad se abrió sin límite a la acción del Paráclito que hace brotar del sufrimiento el amor que salva. 

2. \textquote{Jesucristo, el Hijo de Dios, como hombre, en la oración ardiente de su pasión, permitió al Espíritu Santo, que ya había penetrado hasta el fondo su humanidad, transformarla en sacrificio perfecto mediante el acto de su muerte, como víctima del amor en la Cruz. \textit{Solo Él hizo esta oblación}\ldots En su humanidad fue digno de convertirse en tal sacrificio, ya que solo él estaba \textquote{sin defecto}. \textit{Pero lo ofreció \textquote{con un Espíritu eterno}}\ldots El Espíritu Santo actuó de manera especial en esta entrega absoluta del Hijo del Hombre, para transformar el sufrimiento en amor redentor} (\textit{Ibid}). 

\textquote{Por analogía se puede decir que el Espíritu Santo es \textquote{el fuego del cielo}, \textit{que obra en lo más profundo del misterio de la Cruz}\ldots desciende, en cierto sentido, al corazón mismo del sacrificio que se ofrece en la Cruz\ldots \textit{consume este sacrificio con el fuego del amor}, que une al Hijo con el Padre en comunión trinitaria} (\textit{Ibíd}. 41). 

3. Queridos hermanos y hermanas, peregrinos del Viernes Santo \txtsmall{[que participan del \textquote{Vía Crucis} en el Coliseo Romano, fieles de Roma y del mundo, presentes en este lugar o unidos a nosotros gracias a la radio y la televisión.]} Aquí termina la celebración del Día de la muerte de Cristo; día en el que la redención del mundo se hace presente de manera muy particular. En la Cruz el poder del mal es derrotado y la esperanza renace en todo hombre que sufre, perseguido, cansado y descorazonado. Silencioso y abandonado, el Crucificado consume en el amor su sacrificio de salvación por nosotros. La vida brota de su sangre; en el misterio de la Pasión triunfa la misericordia del Altísimo. 

¡Cruz de nuestra salvación, llevas colgado al Señor del mundo!

\textquote{\textit{Bendito sea Dios, Padre de nuestro Señor Jesucristo}\ldots en quien tenemos la redención por su sangre, el perdón de los pecados} (\textit{Ef} 1, 3-7). Después de la Resurrección, cuando se han cumplido los días de Pascua, el Resucitado entra al Cenáculo a puerta cerrada y muestra a los Apóstoles los signos de la crucifixión; sopla sobre ellos y dice: \textquote{\textit{Recibid el Espíritu Santo; a quienes perdonéis los pecados, les serán perdonados}} (\textit{Jn} 20, 22-23). 

\textquote{La sangre de Cristo, que con Espíritu eterno se ofreció a sí mismo sin tacha a Dios} (\textit{Hb} 9, 14), purificará nuestra conciencia en la fuerza de este Espíritu hasta el fin del mundo. 

¡Gloria a Ti, Palabra de Dios! ¡Gloria a Ti, Cristo, inmolado por nosotros! ¡Tu amor ha redimido al mundo y siempre lo salva! ¡Amén!
\end{body}


\subsubsection{Alocución (1994): Ne evacuetur Crux!}

\src{1 de abril de 1994.}

\begin{body}
\ltr[1. ]{H}{ermanos} y hermanas, hoy estamos aquí para contemplar el misterio de la Cruz, que adoramos en la liturgia del Viernes Santo: \textquote{Ecce lignum crucis, venite adoremus}. Adorémosle ahora, aquí, \txtsmall{[en el Coliseo. Aquí donde nuestros antepasados en la fe dieron testimonio, a través de su martirio hasta la muerte, del amor con que Cristo nos amó. Aquí, en este punto del globo, en la antigua Roma, pienso especialmente en la \textquote{Montaña de las Cruces} que se encuentra en Lituania, donde fui en una visita pastoral en septiembre pasado. Me conmovió ese otro Coliseo no de la época romana antigua, sino el Coliseo de nuestra época, del siglo pasado. Antes de ir a Lituania, a los países bálticos, recé por estos dos caminos de evangelización: uno que caminaba desde Roma hacia el norte, este, oeste; el otro andando desde Constantinopla, desde la Iglesia de Oriente. Estas dos carreteras se encuentran precisamente allí, en los países bálticos, entre Lituania y Rusia.]}

2. \txtsmall{[Hoy la sabiduría de la Tradición Oriental nos guió en nuestra meditación sobre el Vía Crucis a través de las palabras de nuestro amado Hermano Bartolomé de Constantinopla, Patriarca Ecuménico. Le agradecemos de todo corazón. Pensé en estos otros Coliseos, muy numerosos, en estas otras \textquote{Montañas de las Cruces} que están del otro lado, por la Rusia europea, por Siberia, muchas \textquote{Montañas de las Cruces}, muchos Coliseos de los nuevos tiempos. Hoy quisiera decir a este hermano mío de Constantinopla, a todos estos hermanos nuestros de Oriente: queridos amigos, estamos unidos en estos mártires entre Roma, la \textquote{Montaña de las Cruces} y las Islas Soloviesky y muchos otros campos de exterminio.]}

Estamos unidos en el contexto de los mártires, no podemos dejar de estar unidos. No podemos dejar de decir la misma verdad sobre la Cruz y ¿por qué no podemos no decirla? Porque el mundo de hoy trata de vaciar la Cruz. 

Esta es la tradición anticristiana que se ha extendido desde hace varios siglos y quiere vaciar la Cruz y quiere decirnos que el hombre no tiene sus raíces en la Cruz, ni siquiera tiene la perspectiva y la esperanza en la Cruz. El hombre es solo humano, debe existir como si Dios no existiera. 

3. Queridos amigos, tenemos esta tarea común, debemos decir juntos entre Oriente y Occidente: \textquote{Ne evacuetur Crux!} Que la Cruz de Cristo no se vacíe, porque si la Cruz de Cristo se vacía, el hombre ya no tiene raíces, ya no tiene perspectivas: ¡está destruido! 

Este es el grito de finales del siglo XX. Es el grito de Roma, el grito de Moscú, el grito de Constantinopla. Es el grito de todo el cristianismo: de América, de África, de Asia, de todos. Es el grito de la nueva evangelización. 

Jesús nos dice: ellos me persiguieron, ellos también te perseguirán a ti; me escucharon, recibieron mi Palabra, también recibirán la tuya. Recibirán, no tienen otra solución. Nadie tiene palabras de vida eterna, solo Él, solo Jesús, solo su Cruz. 

\txtsmall{[Y así, al final de este Vía Crucis en nuestro antiguo Coliseo de Roma, pensamos en todos los demás Coliseos y los saludamos con amor, con fe, con una esperanza común.]}

4. Nos encomendamos, toda la Iglesia y toda la humanidad, a esta Madre que está bajo la Cruz y que nos abraza a todos como niños. En su amor nosotros, como Juan, sentimos la fuerza de esta unidad, de esta comunión, de la Iglesia y del cristianismo y damos gracias al Padre, al Hijo y al Espíritu Santo por la Cruz de Cristo. 

¡Alabado sea Jesucristo! 

¡Feliz Pascua!
\end{body}


\newpage 
\subsubsection{Alocución (1997): Tomar parte en su Cruz}

\src{28 de marzo de 1997.}

\begin{body}
\textquote{\textit{Cristus factus est pro nobis oboediens usque ad mortem, mortem autem crucis}} (\textit{Flp} 2, 8).

\ltr[1. «]{C}{risto} por nosotros se sometió incluso a la muerte, y una muerte de cruz» (\textit{Flp} 2, 8-9). Estas palabras de San Pablo resumen el mensaje que el Viernes Santo nos quiere comunicar. La Iglesia no celebra este día la Eucaristía, casi como queriendo subrayar que no es posible, en el día en que se ha consumado el sacrificio \textit{cruento} de Cristo en la cruz, hacerlo presente de manera \textit{incruenta} en el Sacramento. La Liturgia eucarística se sustituye hoy por el sugestivo rito de la \textit{adoración de la Cruz}\ldots Quienes han tomado parte en ella conservan aún viva la emoción experimentada al escuchar los textos litúrgicos sobre la Pasión del Señor.

¿Cómo no sentirse conmovidos por la descripción detallada que hace Isaías del \textquote{varón de dolores}, despreciado y rechazado de los hombres, que ha tomado sobre sí el peso de nuestros sufrimientos, herido de Dios por nuestros pecados? (cf. \textit{Is} 53, 3ss). Y, ¿cómo permanecer insensibles ante \textquote{el poderoso clamor y lágrimas} de Cristo, evocadas por el autor de la Carta a los Hebreos? (cf. \textit{Hb} 5, 7).

2. (\ldots) siguiendo las estaciones del \textit{Vía crucis}, contemplamos las dramáticos etapas de la Pasión: Cristo que lleva la Cruz, que cae bajo su peso y agoniza en ella, que en el momento de la agonía ora con aquellas palabras: \textquote{Padre, en tus manos encomiendo mi espíritu} (\textit{Lc} 23, 46), manifestando su total y confiado abandono.

Hoy se concentra en la Cruz toda nuestra atención. Meditamos sobre el misterio de la Cruz, que se perpetúa a través de los siglos en el sacrificio de tantos creyentes, de tantos hombres y mujeres asociados a la muerte de Jesús con el martirio. Contemplamos el misterio de la agonía y de la muerte del Señor, que perdura también en nuestros día en el dolor y el sufrimiento de los pueblos e individuos afectados por la guerra y la violencia. Allí donde el hombre es golpeado y abatido, se ofende y crucifica a Cristo mismo. ¡Misterio de dolor, misterio de amor sin límites! 

Quedemos en un recogimiento silencioso ante este misterio insondable.

3. \textquote{\textit{Ecce Lignum crucis\ldots}} -- ̣\textquote{Mirad el árbol de la Cruz, donde estuvo clavada la salvación del mundo. ¡Venid a adorarlo!}

La Cruz brilla esta tarde con extraordinario fulgor (\ldots) Revivimos, año tras año, la pasión y la muerte de Cristo. ¡\textit{Ecce lignum Crucis}! ¡Cuántos hermanos y hermanas en la fe participaron de la Cruz de Cristo en [los períodos de persecución de la historia de la Iglesia]!

(\ldots) ¡Muchos hermanos y hermanas (\ldots) han tomado parte en la Cruz de Cristo con el sacrificio de sus vidas! Hoy, en unión con ellos y con todos cuantos, en cualquier rincón de la tierra, en cada continente y en los diversos países del orbe, participan en la Cruz de Cristo con sus sufrimientos y con la muerte, queremos repetir: \textquote{Ecce lignum Crucis\ldots}, \textquote{Mirad el árbol de la Cruz, donde estuvo clavada la salvación del mundo. ¡Venid a adorarlo!}

4. ¡Mientras se ciernen las sombras de la noche, imagen elocuente del misterio que envuelve nuestra existencia, nosotros gritamos a Ti, Cruz de nuestra salvación, nuestra fe! Señor, de tu Cruz se desprende un rayo de luz. En tu muerte ha sido vencida nuestra muerte y se nos ha ofrecido la esperanza de la resurrección. ¡Asidos a tu Cruz, quedamos en la espera confiada de tu vuelta, Señor Jesús, Redentor nuestro! 

\textquote{\textit{Anunciamos tu muerte, proclamamos tu resurrección. ¡Ven, Señor Jesús!}}.

¡Amén!
\end{body}

\subsubsection{Alocución (2000): Muerte necesaria}

\src{21 de abril del 2000.}

\begin{body}
\ltr[1. «¿]{N}{o} era necesario que el Cristo padeciera eso y entrara así en su gloria?» (\textit{Lc} 24, 26). Estas palabras de Jesús a dos discípulos camino de Emaús, resuenan en nuestro espíritu esta tarde (\ldots). También ellos, como nosotros, habían oído hablar de los acontecimientos concernientes a la pasión y la crucifixión de Jesús. De vuelta a su pueblo, Cristo se les acerca como un peregrino desconocido y ellos se apresuran a contarle \textquote{lo de Jesús el Nazareno, que fue un profeta poderoso en obras y palabras delante de Dios y de todo el pueblo} (\textit{Lc} 24, 19), y \textquote{cómo nuestros sumos sacerdotes y magistrados le condenaron a muerte y le crucificaron} (\textit{Lc} 24, 20). Con tristeza, terminan diciendo: \textquote{Nosotros esperábamos que sería él el que iba a librar a Israel; pero, con todas estas cosas, llevamos ya tres días desde que esto pasó} (\textit{Lc} 24, 21). \textquote{Nosotros esperábamos\ldots}. Los discípulos están desanimados y abatidos. También para nosotros es difícil entender por qué la vía de la salvación deba pasar por el sufrimiento y la muerte.

2. \textquote{¿No era necesario que el Cristo padeciera eso y entrara así en su gloria?} (\textit{Lc} 24, 26). Nos hacemos la misma pregunta [en este Viernes Santo] (\ldots). Dentro de poco, dejaremos este lugar (\ldots) y nos dispersaremos en diversas direcciones. Volveremos a nuestras casas, reflexionando sobre los mismos acontecimientos de los que hablaban los discípulos de Emaús. ¡Que Jesús se acerque a cada uno de nosotros y se haga también compañero nuestro de viaje! Mientras nos acompaña, Él nos explicará que ha subido al Calvario por nosotros, ha muerto por nosotros, cumpliendo las Escrituras. De este modo, la dolorosa escena de la crucifixión que acabamos de contemplar se convertirá para cada uno en una elocuente enseñanza.

Queridos Hermanos y Hermanas. El hombre contemporáneo tiene necesidad de encontrar a Jesús crucificado y resucitado. ¿Quién, si no es el divino Condenado, puede comprender plenamente la pena de quien sufre condenas injustas? ¿Quién, si no es el Rey ultrajado y humillado, puede satisfacer las expectativas de tantos hombres y mujeres sin esperanza y sin dignidad? ¿Quién, si no es el Hijo de Dios crucificado, puede entender el dolor de la soledad de tantas vidas truncadas y sin futuro?

El poeta francés Paul Claudel escribía que el Hijo de Dios \textquote{nos ha enseñando la vía de salida del dolor y la posibilidad de su transformación} (\textit{Positions et propositions}). Abramos el corazón a Cristo: será Él mismo quien responda a nuestra más profundas expectativas. Él mismo nos desvelará los misterios de su pasión y muerte en la cruz.

3. \textquote{Entonces se les abrieron los ojos y le reconocieron} (\textit{Lc} 24, 31). Con sus palabras, el corazón de los dos viandantes desconsolados adquirió serenidad y comenzó a henchirse de alegría. Reconocen a su Maestro al partir el pan. Que los hombres de hoy, como ellos, reconozcan en la Eucaristía la presencia de su Salvador. Que lo encuentren en el sacramento de su Pascua y lo acojan como compañero de su camino. Él sabrá escucharles y consolarles. Sabrá ser su guía para conducirles por los senderos de la vida hacia la casa del Padre.

\textquote{\textit{Adoramus te, Christe, et benedicimus tibi, quian per sanctam Crucem tuam redemisti mundum!}}
\end{body}


\subsubsection{Alocución (2003): Mirad el árbol de la Cruz}

\src{18 de abril de 2003.}

\begin{body}
\textquote{\textit{Ecce lignum crucis, in quo salus mundi pependit\ldots Venite adoremus}}.

\ltr{H}{emos} escuchado estas palabras en la liturgia de hoy: \textquote{Mirad el árbol de la cruz}. Son las palabras clave del Viernes santo. Ayer, en el primer día del Triduo sacro, el Jueves santo, escuchamos: \textquote{\textit{Hoc est corpus meum, quod pro vobis tradetur}. Esto es mi cuerpo, que será entregado por vosotros}. Hoy vemos cómo se han realizado esas palabras de ayer, Jueves santo: he aquí el Gólgota, he aquí el cuerpo de Cristo en la cruz. \textquote{\textit{Ecce lignum crucis, in quo salus mundi pependit}}.

¡Misterio de la fe! El hombre no podía imaginar este misterio, esta realidad. Sólo Dios la podía revelar. El hombre no tiene la posibilidad de dar la vida después de la muerte. La muerte de la muerte. En el orden humano, la muerte es la última palabra. La palabra que viene después, la palabra de la Resurrección, es una palabra exclusiva de Dios y por eso celebramos con gran fervor este Triduo sacro.

Hoy oramos a Cristo bajado de la cruz y sepultado. Se ha sellado su sepulcro. Y mañana, en todo el mundo, en todo el cosmos, en todos nosotros, reinará un profundo silencio. Silencio de espera. \textquote{\textit{Ecce lignum crucis, in quo salus mundi pependit}}. Este árbol de la muerte, el árbol en el que murió el Hijo de Dios, abre el camino al día siguiente: jueves, viernes, sábado, domingo. El domingo será Pascua. Y escucharemos las palabras de la liturgia. Hoy hemos escuchado: \textquote{\textit{Ecce lignum crucis, in quo salus mundi pependit}. Salus mundi!} ¡En la cruz! Y pasado mañana cantaremos: \textquote{\textit{Surrexit de sepulcro\ldots qui pro nobis pependit in ligno}}. He aquí la profundidad, la sencillez divina, de este Triduo pascual.

Ojalá que todos vivamos este Triduo lo más profundamente posible. Como cada año, nos encontramos aquí (\ldots).

\textquote{\textit{Ecce lignum crucis, in quo salus mundi pependit}. Salus mundi!}

A todos vosotros, amadísimos hermanos y hermanas, os deseo que viváis este Triduo sacro –Jueves, Viernes, Sábado santo, Vigilia pascual, y luego la Pascua– cada vez con más profundidad, y también que lo testimoniéis.

¡Alabado sea Jesucristo!
\end{body}

\begin{patercite}
Las palabras pronunciadas por Jesús después de la invocación \textquote{Padre} retoman una expresión del Salmo 31: \textquote{A tus manos encomiendo mi espíritu} (\textit{Sal} 31, 6). Estas palabras, sin embargo, no son una simple cita, sino que más bien manifiestan una decisión firme: Jesús se \textquote{entrega} al Padre en un acto de total abandono. Estas palabras son una oración de \textquote{abandono}, llena de confianza en el amor de Dios. La oración de Jesús ante la muerte es dramática como lo es para todo hombre, pero, al mismo tiempo, está impregnada de esa calma profunda que nace de la confianza en el Padre y de la voluntad de entregarse totalmente a él. En Getsemaní, cuando había entrado en el combate final y en la oración más intensa y estaba a punto de ser \textquote{entregado en manos de los hombres} (\textit{Lc} 9, 44), \textquote{le entró un sudor que caía hasta el suelo como si fueran gotas espesas de sangre} (\textit{Lc} 22, 44). Pero su corazón era plenamente obediente a la voluntad del Padre, y por ello \textquote{un ángel del cielo} vino a confortarlo (cf. \textit{Lc} 22, 42-43). Ahora, en los últimos momentos, Jesús se dirige al Padre diciendo cuáles son realmente las manos a las que él entrega toda su existencia. Antes de partir en viaje hacia Jerusalén, Jesús había insistido con sus discípulos: \textquote{Meteos bien en los oídos estas palabras: el Hijo del hombre va a ser entregado en manos de los hombres} (\textit{Lc} 9, 44). Ahora que su muerte es inminente, él sella en la oración su última decisión: Jesús se dejó entregar \textquote{en manos de los hombres}, pero su espíritu lo pone en las manos del Padre; así --como afirma el evangelista san Juan-- todo se cumplió, el supremo acto de amor se cumplió hasta el final, al límite y más allá del límite.

\textbf{Benedicto XVI}, papa, \textit{Catequesis} sobre el Salmo 22, 15 de febrero del 2012, parr. 7. 
\end{patercite}

\newsection
\subsection{Benedicto XVI, papa}

\subsubsection{Alocución (2006): Nuestro lugar ante la Cruz}

\src{14 de abril de 2006.}

\begin{body}
\txtsmall{[Hemos acompañado a Jesús en el vía crucis. Lo hemos acompañado aquí, por el camino de los mártires, en el Coliseo, donde tantos han sufrido por Cristo, han dado la vida por el Señor; donde el Señor mismo ha sufrido de nuevo en tantos.]}

\ltr{A}{sí} hemos comprendido que el vía crucis no es algo del pasado y de un lugar determinado de la tierra. La cruz del Señor abraza al mundo entero; su vía crucis atraviesa los continentes y los tiempos. En el vía crucis no podemos limitarnos a ser espectadores. Estamos implicados también nosotros; por eso, debemos buscar nuestro lugar. ¿Dónde estamos nosotros? 

En el vía crucis no se puede ser neutral. Pilatos, el intelectual escéptico, trató de ser neutral, de quedar al margen; pero, precisamente así, se puso contra la justicia, por el conformismo de su carrera. 

Debemos buscar nuestro lugar.

En el espejo de la cruz hemos visto todos los sufrimientos de la humanidad de hoy. En la cruz de Cristo hoy hemos visto el sufrimiento de los niños abandonados, de los niños víctimas de abusos; las amenazas contra la familia; la división del mundo en la soberbia de los ricos que no ven a Lázaro a su puerta y la miseria de tantos que sufren hambre y sed. 

Pero también hemos visto \textquote{estaciones} de consuelo. Hemos visto a la Madre, cuya bondad permanece fiel hasta la muerte y más allá de la muerte. Hemos visto a la mujer valiente que se acerca al Señor y no tiene miedo de manifestar solidaridad con este Varón de dolores. Hemos visto a Simón, el Cirineo, un africano, que lleva la cruz juntamente con Jesús. Y mediante estas \textquote{estaciones} de consuelo hemos visto, por último, que, del mismo modo que no acaban los sufrimientos, tampoco acaban los consuelos. 

Hemos visto cómo san Pablo encontró en el \textquote{camino de la cruz} el celo de su fe y encendió la luz del amor. Hemos visto cómo san Agustín halló su camino. Lo mismo san Francisco de Asís, san Vicente de Paúl, san Maximiliano Kolbe, la madre Teresa de Calcuta\ldots Del mismo modo también nosotros estamos invitados a encontrar nuestro lugar, a encontrar, como estos grandes y valientes santos, el camino con Jesús y por Jesús: el camino de la bondad, de la verdad; la valentía del amor.

Hemos comprendido que el vía crucis no es simplemente una colección de las cosas oscuras y tristes del mundo. Tampoco es un moralismo que, al final, resulta insuficiente. No es un grito de protesta que no cambia nada. El vía crucis es el camino de la misericordia, y de la misericordia que pone el límite al mal: eso lo hemos aprendido del Papa Juan Pablo II. Es el camino de la misericordia y, así, el camino de la salvación. De este modo estamos invitados a tomar el camino de la misericordia y a poner, juntamente con Jesús, el límite al mal. 

Pidamos al Señor que nos ayude, que nos ayude a ser \textquote{contagiados} por su misericordia. Pidamos a la santa Madre de Jesús, la Madre de la misericordia, que también nosotros seamos hombres y mujeres de la misericordia, para contribuir así a la salvación del mundo, a la salvación de las criaturas, para ser hombres y mujeres de Dios. Amén.
\end{body}


\subsubsection{Alocución (2009): Un hombre que cambió la historia}

\src{Colina del Palatino. 10 de abril de 2009.}

\begin{body}
\ltr{A}{l} terminar el relato dramático de la Pasión, anota el evangelista San Marcos: \textquote{El centurión que estaba enfrente, al ver cómo había expirado, dijo: \textquote{Realmente este hombre era Hijo de Dios}} (\textit{Mc} 15, 39). No deja de sorprendernos la profesión de fe de este soldado romano, que había asistido al desarrollo de las diferentes fases de la crucifixión. Cuando la oscuridad de la noche estaba por caer sobre aquel Viernes único de la historia, cuando el sacrificio de la cruz ya se había consumado y los que estaban allí se apresuraban para poder celebrar la Pascua judía a tenor de lo prescrito, las breves palabras oídas de labios de un comandante anónimo de la tropa romana resuenan en el silencio ante aquella muerte tan singular. Este oficial de la tropa romana, que había asistido a la ejecución de uno de tantos condenados a la pena capital, supo reconocer en aquel Hombre crucificado al Hijo de Dios, que expiraba en el más humillante abandono. Su fin ignominioso habría debido marcar el triunfo definitivo del odio y de la muerte sobre el amor y la vida. Pero no fue así. En el Gólgota se erguía la Cruz, de la que colgaba un hombre ya muerto, pero aquel Hombre era el \textquote{Hijo de Dios}, como confesó el centurión \textquote{al ver cómo había expirado}, en palabras del evangelista.

La profesión de fe de este soldado se repite cada vez que volvemos a escuchar el relato de la pasión según san Marcos. También nosotros esta noche, como él, nos detenemos a contemplar el rostro exánime del Crucificado (\ldots). Hemos revivido el episodio trágico de un Hombre único en la historia de todos los tiempos, que ha cambiado el mundo no abatiendo a otros, sino dejando que lo mataran clavado en una cruz. Este Hombre, uno de nosotros, que mientras lo están asesinando perdona a sus verdugos, es el \textquote{Hijo de Dios} que, como nos recuerda el apóstol Pablo, \textquote{no hizo alarde de su categoría de Dios; al contrario, se despojó de su rango, y tomó la condición de esclavo\ldots se rebajó hasta someterse incluso a la muerte, y una muerte de cruz} (\textit{Flp} 2, 6-8).

La pasión dolorosa del Señor Jesús suscita necesariamente piedad hasta en los corazones más duros, ya que es el culmen de la revelación del amor de Dios por cada uno de nosotros. Observa san Juan: \textquote{Tanto amó Dios al mundo, que entregó a su Hijo único, para que no perezca ninguno de los que creen en Él, sino que tengan vida eterna} (\textit{Jn} 3, 16). Cristo murió en la cruz por amor. A lo largo de los milenios, muchedumbres de hombres y mujeres han quedado seducidos por este misterio y le han seguido, haciendo al mismo tiempo de su vida un don a los hermanos, como Él y gracias a su ayuda. Son los santos y los mártires, muchos de los cuales nos son desconocidos. También en nuestro tiempo, cuántas personas, en el silencio de su existencia cotidiana, unen sus padecimientos a los del Crucificado y se convierten en apóstoles de una auténtica renovación espiritual y social. ¿Qué sería del hombre sin Cristo? San Agustín señala: \textquote{Una inacabable miseria se hubiera apoderado de ti, si no se hubiera llevado a cabo esta misericordia. Nunca hubieras vuelto a la vida, si Él no hubiera venido al encuentro de tu muerte. Te hubieras derrumbado, si Él no te hubiera ayudado. Hubieras perecido, si Él no hubiera venido} (\textit{Sermón} 185, 1). Entonces, ¿por qué no acogerlo en nuestra vida?

Detengámonos esta noche contemplando su rostro desfigurado: es el rostro del Varón de dolores, que ha cargado sobre sí todas nuestras angustias mortales. Su rostro se refleja en el de cada persona humillada y ofendida, enferma o que sufre, sola, abandonada y despreciada. Al derramar su sangre, Él nos ha rescatado de la esclavitud de la muerte, roto la soledad de nuestras lágrimas, y entrado en todas nuestras penas y en todas nuestras inquietudes.

Hermanos y hermanas, mientras se yergue la Cruz sobre el Gólgota, la mirada de nuestra fe se proyecta hacia el amanecer del Día nuevo y gustamos ya el gozo y el fulgor de la Pascua. \textquote{Si hemos muerto con Cristo –escribe san Pablo–, creemos que también viviremos con Él} (\textit{Rm} 6, 8). Con esta certeza, continuamos nuestro camino. Mañana, Sábado Santo, velaremos en oración. Pero ya ahora oremos con María, la Virgen Dolorosa, oremos con todos los adolorados, [oremos sobre todo con los afectados por el terremoto de L’Aquila:] oremos para que también brille para ellos en esta noche oscura la estrella de la esperanza, la luz del Señor resucitado.

Desde ahora, deseo a todos una feliz Pascua en la luz del Señor Resucitado.
\end{body}

\newpage 
\subsubsection{Alocución (2012): Triunfo definitivo}

\src{Palatino. 6 de abril de 2012.}

\begin{body}
\ltr{H}{emos} recordado en la meditación, la oración y el canto, el camino de Jesús en la vía de la cruz: una vía que parecía sin salida y que, sin embargo, ha cambiado la vida y la historia del hombre, ha abierto el paso hacia los \textquote{cielos nuevos y la tierra nueva} (cf.\textit{ Ap} 21, 1). Especialmente en este día del Viernes Santo, la Iglesia celebra con íntima devoción espiritual la memoria de la muerte en cruz del Hijo de Dios y, en su cruz, ve el árbol de la vida, fecundo de una nueva esperanza. La experiencia del sufrimiento y de la cruz marca la humanidad, marca incluso la familia; cuántas veces el camino se hace fatigoso y difícil. Incomprensiones, divisiones, preocupaciones por el futuro de los hijos, enfermedades, dificultades de diverso tipo. En nuestro tiempo, además, la situación de muchas familias se ve agravada [por la precariedad del trabajo y por otros efectos negativos de la crisis económica. El camino del \textit{Via Crucis}, que hemos recorrido esta noche espiritualmente,] es una invitación para todos nosotros, y especialmente para las familias, a contemplar a Cristo crucificado para tener la fuerza de ir más allá de las dificultades. La cruz de Jesús es el signo supremo del amor de Dios para cada hombre, la respuesta sobreabundante a la necesidad que tiene toda persona de ser amada. Cuando nos encontramos en la prueba, cuando nuestras familias deben afrontar el dolor, la tribulación, miremos a la cruz de Cristo: allí encontramos el valor y la fuerza para seguir caminando; allí podemos repetir con firme esperanza las palabras de san Pablo: \textquote{¿Quién nos separará del amor de Cristo?: ¿la tribulación?, ¿la angustia?, ¿la persecución?, ¿el hambre?, ¿la desnudez?, ¿el peligro?, ¿la espada?\ldots Pero en todo esto vencemos de sobra gracias a aquel que nos ha amado} (\textit{Rm} 8, 35. 37).

En la aflicción y la dificultad, no estamos solos; la familia no está sola: Jesús está presente con su amor, la sostiene con su gracia y le da la fuerza para seguir adelante, para afrontar los sacrificios y superar todo obstáculo. Y es a este amor de Cristo al que debemos acudir cuando las vicisitudes humanas y las dificultades amenazan con herir la unidad de nuestra vida y de la familia. El misterio de la pasión, muerte y resurrección de Cristo alienta a seguir adelante con esperanza: la estación del dolor y de la prueba, si la vivimos con Cristo, con fe en él, encierra ya la luz de la resurrección, la vida nueva del mundo resucitado, la pascua de cada hombre que cree en su Palabra. En aquel hombre crucificado, que es el Hijo de Dios, incluso la muerte misma adquiere un nuevo significado y orientación, es rescatada y vencida, es el paso hacia la nueva vida: \textquote{si el grano de trigo no cae en tierra y muere, queda infecundo; pero si muere, da mucho fruto} (\textit{Jn} 12, 24). 

Encomendémonos a la Madre de Cristo. A ella, que ha acompañado a su Hijo por la vía dolorosa. Que ella, que estaba junto a la cruz en la hora de su muerte, que ha alentado a la Iglesia desde su nacimiento para que viva la presencia del Señor, dirija nuestros corazones, los corazones de todas las familias a través del inmenso \textit{mysterium passionis} hacia el \textit{mysterium paschale,} hacia aquella luz que prorrumpe de la Resurrección de Cristo y muestra el triunfo definitivo del amor, de la alegría, de la vida, sobre el mal, el sufrimiento, la muerte. Amén.
\end{body}

\begin{patercite}
Creado por Dios en la justicia, el hombre, sin embargo, por instigación del demonio, en el propio exordio de la historia, abusó de su libertad, levantándose contra Dios y pretendiendo alcanzar su propio fin al margen de Dios. Conocieron a Dios, pero no le glorificaron como a Dios. Obscurecieron su estúpido corazón y prefirieron servir a la criatura, no al Creador. Lo que la Revelación divina nos dice coincide con la experiencia. El hombre, en efecto, cuando examina su corazón, comprueba su inclinación al mal y se siente anegado por muchos males, que no pueden tener origen en su santo Creador. Al negarse con frecuencia a reconocer a Dios como su principio, rompe el hombre la debida subordinación a su fin último, y también toda su ordenación tanto por lo que toca a su propia persona como a las relaciones con los demás y con el resto de la creación.

Es esto lo que explica la división íntima del hombre. Toda la vida humana, la individual y la colectiva, se presenta como lucha, y por cierto dramática, entre el bien y el mal, entre la luz y las tinieblas. Más todavía, el hombre se nota incapaz de domeñar con eficacia por sí solo los ataques del mal, hasta el punto de sentirse como aherrojado entre cadenas. Pero el Señor vino en persona para liberar y vigorizar al hombre, renovándole interiormente y expulsando al príncipe de este mundo (cf. \textit{Jn} 12,31), que le retenía en la esclavitud del pecado. El pecado rebaja al hombre, impidiéndole lograr su propia plenitud.

A la luz de esta Revelación, la sublime vocación y la miseria profunda que el hombre experimenta hallan simultáneamente su última explicación.

\textbf{Concilio Vaticano II}, \textit{Gaudium et Spes}, n. 13.
\end{patercite}

\newsection
\subsection{Francisco, papa}

\subsubsection{Alocución (2015): Infinita misericordia}

\src{Palatino. 3 de abril de 2015.}

\begin{body}
\ltr{O}{h} Cristo crucificado y victorioso, tu \textit{Vía Crucis} es la síntesis de tu vida; es el icono de tu obediencia a la voluntad del Padre; es la realización de tu infinito amor por nosotros pecadores; es la prueba de tu misión; es la realización definitiva de la revelación y la historia de la salvación. El peso de tu cruz nos libera de todas nuestras cargas. 

En tu obediencia a la voluntad del Padre, caemos en la cuenta de nuestra rebelión y desobediencia. En ti vendido, traicionado y crucificado por tu gente y por tus seres queridos, vemos nuestras traiciones cotidianas y nuestras usuales infidelidades. En tu inocencia, Cordero inmaculado, vemos nuestra culpa. En tu rostro azotado, escupido y desfigurado, vemos toda la brutalidad de nuestros pecados. En la crueldad de tu Pasión, vemos la crueldad de nuestro corazón y de nuestras acciones. En tu sentirte \textquote{abandonado}, vemos a todos los abandonados por los familiares, la sociedad, la atención y la solidaridad. En tu cuerpo destrozado, desgarrado y lacerado, vemos los cuerpos de nuestros hermanos abandonados a lo largo de las calles, desfigurados por nuestra negligencia y nuestra indiferencia. En tu sed, Señor, vemos la sed de Tu Padre misericordioso que en Ti quiso abrazar, perdonar y salvar a toda la humanidad. En Ti, divino amor, vemos también hoy a nuestros hermanos perseguidos, decapitados y crucificados por su fe en Ti, ante nuestros ojos o a menudo con nuestro silencio cómplice. 

Imprime en nuestro corazón, Señor, sentimientos de fe, esperanza, caridad, de dolor por nuestros pecados y condúcenos a arrepentirnos de nuestros pecados que te han crucificado. Llévanos a transformar nuestra conversión hecha de palabras, en conversión de vida y de obras. Llévanos a custodiar en nosotros un recuerdo vivo de tu Rostro desfigurado, para no olvidar nunca el gran precio que has pagado para liberarnos. Jesús crucificado, refuerza en nosotros la fe para que no decaiga ante las tentaciones; reaviva en nosotros la esperanza, que no pierda el camino siguiendo las seducciones del mundo; custodia en nosotros la caridad para que no se deje engañar por la corrupción y la mundanidad. Enséñanos que la Cruz es el camino hacia la Resurrección. Enséñanos que el Viernes santo es camino hacia la Pascua de la luz; enséñanos que Dios nunca olvida a ninguno de sus hijos y nunca se cansa de perdonarnos y abrazarnos con su infinita misericordia. Pero enséñanos también a no cansarnos nunca de pedir perdón y creer en la misericordia sin límites del Padre. 

\begin{bodyprose}
\textit{Alma de Cristo, santifícanos.}
\textit{Cuerpo de Cristo, sálvanos.}
\textit{Sangre de Cristo, embriáganos.}

\textit{Agua del costado de Cristo, lávanos.}
\textit{Pasión de Cristo, confórtanos.}

\textit{Oh buen Jesús, óyenos.}
\textit{Dentro de tus llagas, escóndenos.}

\textit{No permitas que nos separemos de ti.}
\textit{Del maligno enemigo defiéndenos.}

\textit{En la hora de nuestra muerte llámanos.}
\textit{Y manda que vengamos a Ti} 
\textit{para que te alabemos con tus santos,}
\textit{por los siglos de los siglos. Amén.}
\end{bodyprose}
\end{body}



\newpage 
\subsubsection{Oración (2018): Vergüenza, arrepentimiento, esperanza}

\src{Palatino. 30 de marzo de 2018.}

\begin{body}
\ltr{S}{eñor} Jesús, nuestra mirada está dirigida a ti, llena de vergüenza, de arrepentimiento y de esperanza. Que ante tu amor supremo nos impregne la vergüenza para haberte dejado solo y sufrir por nuestros pecados: 

la vergüenza por haber escapado ante la prueba incluso habiendo dicho miles de veces: \textquote{aunque todos te dejen, yo no te dejaré}; 

la vergüenza de haber elegido a Barrabás y no a ti, el poder y no a ti, la apariencia y no a ti, el dios dinero y no a ti, la mundanidad y no la eternidad; 

la vergüenza por haberte tentando con la boca y con el corazón, cada vez que nos hemos encontrado ante una prueba, diciéndote: \textquote{¡si tú eres el mesías, sálvate y nosotros creeremos!};

la vergüenza porque muchas personas, e incluso algunos ministros tuyos, se han dejado engañar por la ambición y la vana gloria perdiendo su dignidad y su primer amor; 

la vergüenza porque nuestras generaciones están dejando a los jóvenes un mundo fracturado por las divisiones y las guerras; un mundo devorado por el egoísmo donde los jóvenes, los pequeños, los enfermos, los ancianos son marginados; 

la vergüenza de haber perdido la vergüenza;

Señor Jesús, ¡danos siempre la gracia de la santa vergüenza!

Nuestra mirada está llena también de un arrepentimiento que ante su silencio elocuente suplica tu misericordia:

el arrepentimiento que brota de la certeza de que solo tú puedes salvarnos del mal, solo tú puedes sanarnos de nuestra lepra de odio, de egoísmo, de soberbia, de codicia, de venganza, de avaricia, de idolatría, solo tú puedes abrazarnos y darnos de nuevo la dignidad filial y alegrarnos por nuestra vuelta a casa, a la vida; 

el arrepentimiento que florece del sentir nuestra pequeñez, nuestra nada, nuestra vanidad y que se deja acariciar por tu invitación suave y poderosa a la conversión; 

el arrepentimiento de David que del abismo de su miseria reencuentra en ti su única fuerza; 

el arrepentimiento que nace de nuestra vergüenza, que nace de la certeza de que nuestro corazón permanecerá siempre inquieto hasta que no te encuentre a ti y en ti su única fuente de plenitud y de quietud; 

el arrepentimiento de Pedro que encontrando tu mirada lloró amargamente por haberte negado delante de los hombres. 

Señor Jesús, ¡danos siempre la gracia del santo arrepentimiento! 

Delante de tu suprema majestad se enciende, en la tenebrosidad de nuestra desesperación, la chispa de la esperanza porque sabemos que tu única medida para amarnos es la de amarnos sin medida; 

la esperanza porque tu mensaje continúa inspirando, todavía hoy, a muchas personas y pueblos a los que solo el bien puede derrotar el mal y la maldad, solo el perdón puede derrumbar el rencor y la venganza, solo el abrazo fraterno puede dispersar la hostilidad y el miedo al otro; 

la esperanza porque tu sacrificio continúe, todavía hoy, emanando el perfume del amor divino que acaricia los corazones de tantos jóvenes que continúan consagrándote sus vidas convirtiéndose en ejemplos vivos de caridad y de gratuidad en este nuestro mundo devorado por la lógica del beneficio y de la fácil ganancia; 

la esperanza porque tantos misioneros y misioneras continúan, todavía hoy, desafiando la adormecida conciencia de la humanidad arriesgando la vida para servirte en los pobres, en los descartados, en los inmigrantes, en los invisibles, en los explotados, en los hambrientos y en los presos; 

la esperanza porque tu Iglesia, santa y hecha de pecadores, continúa, todavía hoy, no obstante todos los intentos de desacreditarla, siendo una luz que ilumina, alienta, levanta y testimonia tu amor ilimitado por la humanidad, un modelo de altruismo, un arca de salvación y una fuente de certeza y de verdad; 

la esperanza porque de tu cruz, fruto de la avaricia y la cobardía de tantos doctores de la Ley e hipócritas, ha surgido la Resurrección transformando las tinieblas de la tumba en el brillo del alba del Domingo sin puesta de sol, enseñándonos que tu amor es nuestra esperanza. 

¡Señor Jesús, danos siempre la gracia de la santa esperanza!

Ayúdanos, Hijo del hombre, a despojarnos de la arrogancia del ladrón puesto a tu izquierda y de los miopes y de los corruptos, que han visto en ti una oportunidad para explotar, un condenado para criticar, un derrotado del que burlarse, otra ocasión para cargar sobre otros, e incluso sobre Dios, las propias culpas. 

Sin embargo te pido, Hijo de Dios, que nos identifiquemos con el ladrón bueno que te ha mirado con ojos llenos de vergüenza, de arrepentimiento y de esperanza; que, con los ojos de la fe, ha visto en tu aparente derrota la divina victoria y así se ha arrodillado delante de tu misericordia y con honestidad ha robado el paraíso! ¡Amén!
\end{body}

\newsection
\section{Temas}

\cceth{La Pasión de Cristo} 

\cceref{CEC 602-618, 1992}

\begin{ccebody}

\ccesec{\textquote{Dios le hizo pecado por nosotros}}

\n{602} [\ldots] san Pedro pudo formular así la fe apostólica en el designio divino de salvación: \textquote{Habéis sido rescatados de la conducta necia heredada de vuestros padres, no con algo caduco, oro o plata, sino con una sangre preciosa, como de cordero sin tacha y sin mancilla, Cristo, predestinado antes de la creación del mundo y manifestado en los últimos tiempos a causa de vosotros} (\textit{1 P} 1, 18-20). Los pecados de los hombres, consecuencia del pecado original, están sancionados con la muerte (cf. \textit{Rm} 5, 12; \textit{1 Co} 15, 56). Al enviar a su propio Hijo en la condición de esclavo (cf. \textit{Flp} 2, 7), la de una humanidad caída y destinada a la muerte a causa del pecado (cf. \textit{Rm} 8, 3), \textquote{a quien no conoció pecado, Dios le hizo pecado por nosotros, para que viniésemos a ser justicia de Dios en él} (\textit{2 Co} 5, 21).

\n{603} Jesús no conoció la reprobación como si él mismo hubiese pecado (cf. \textit{Jn} 8, 46). Pero, en el amor redentor que le unía siempre al Padre (cf. \textit{Jn} 8, 29), nos asumió desde el alejamiento con relación a Dios por nuestro pecado hasta el punto de poder decir en nuestro nombre en la cruz: \textquote{Dios mío, Dios mío, ¿por qué me has abandonado?} (\textit{Mc} 15, 34; \textit{Sal} 22,2). Al haberle hecho así solidario con nosotros, pecadores, \textquote{Dios no perdonó ni a su propio Hijo, antes bien le entregó por todos nosotros} (\textit{Rm} 8, 32) para que fuéramos \textquote{reconciliados con Dios por la muerte de su Hijo} (\textit{Rm} 5, 10).

\ccesec{Dios tiene la iniciativa del amor redentor universal}

\n{604} Al entregar a su Hijo por nuestros pecados, Dios manifiesta que su designio sobre nosotros es un designio de amor benevolente que precede a todo mérito por nuestra parte: \textquote{En esto consiste el amor: no en que nosotros hayamos amado a Dios, sino en que él nos amó y nos envió a su Hijo como propiciación por nuestros pecados} (\textit{1 Jn} 4, 10; cf. \textit{Jn} 4, 19). \textquote{La prueba de que Dios nos ama es que Cristo, siendo nosotros todavía pecadores, murió por nosotros} (\textit{Rm} 5, 8).

\n{605} Jesús ha recordado al final de la parábola de la oveja perdida que este amor es sin excepción: \textquote{De la misma manera, no es voluntad de vuestro Padre celestial que se pierda uno de estos pequeños} (\textit{Mt} 18, 14). Afirma \textquote{dar su vida en rescate \textit{por muchos}} (\textit{Mt} 20, 28); este último término no es restrictivo: opone el conjunto de la humanidad a la única persona del Redentor que se entrega para salvarla (cf. \textit{Rm} 5, 18-19). La Iglesia, siguiendo a los Apóstoles (cf. \textit{2 Co} 5, 15; \textit{1 Jn} 2, 2), enseña que Cristo ha muerto por todos los hombres sin excepción: \textquote{no hay, ni hubo ni habrá hombre alguno por quien no haya padecido Cristo} (Concilio de Quiercy, año 853: DS, 624).

\ccesec{Cristo se ofreció a su Padre por nuestros pecados \\Toda la vida de Cristo es oblación al Padre}

\n{606} El Hijo de Dios \textquote{bajado del cielo no para hacer su voluntad sino la del Padre que le ha enviado} (\textit{Jn} 6, 38), \textquote{al entrar en este mundo, dice: [\ldots] He aquí que vengo [\ldots] para hacer, oh Dios, tu voluntad [\ldots] En virtud de esta voluntad somos santificados, merced a la oblación de una vez para siempre del cuerpo de Jesucristo} (\textit{Hb} 10, 5-10). Desde el primer instante de su Encarnación el Hijo acepta el designio divino de salvación en su misión redentora: \textquote{Mi alimento es hacer la voluntad del que me ha enviado y llevar a cabo su obra} (\textit{Jn} 4, 34). El sacrificio de Jesús \textquote{por los pecados del mundo entero} (\textit{1 Jn} 2, 2), es la expresión de su comunión de amor con el Padre: \textquote{El Padre me ama porque doy mi vida} (\textit{Jn} 10, 17). \textquote{El mundo ha de saber que amo al Padre y que obro según el Padre me ha ordenado} (\textit{Jn} 14, 31).

\n{607} Este deseo de aceptar el designio de amor redentor de su Padre anima toda la vida de Jesús (cf. \textit{Lc} 12, 50; 22, 15; \textit{Mt} 16, 21-23) porque su Pasión redentora es la razón de ser de su Encarnación: \textquote{¡Padre líbrame de esta hora! Pero ¡si he llegado a esta hora para esto!} (\textit{Jn} 12, 27). \textquote{El cáliz que me ha dado el Padre ¿no lo voy a beber?} (\textit{Jn} 18, 11). Y todavía en la cruz antes de que \textquote{todo esté cumplido} (\textit{Jn} 19, 30), dice: \textquote{Tengo sed} (\textit{Jn} 19, 28).

\ccesec{\textquote{El cordero que quita el pecado del mundo}}

\n{608} Juan Bautista, después de haber aceptado bautizarle en compañía de los pecadores (cf. \textit{Lc} 3, 21; \textit{Mt} 3, 14-15), vio y señaló a Jesús como el \textquote{Cordero de Dios que quita los pecados del mundo} (\textit{Jn} 1, 29; cf. \textit{Jn} 1, 36). Manifestó así que Jesús es a la vez el Siervo doliente que se deja llevar en silencio al matadero (\textit{Is} 53, 7; cf. \textit{Jr} 11, 19) y carga con el pecado de las multitudes (cf. \textit{Is} 53, 12) y el cordero pascual símbolo de la redención de Israel cuando celebró la primera Pascua (\textit{Ex} 12, 3-14; cf. \textit{Jn} 19, 36; \textit{1 Co} 5, 7). Toda la vida de Cristo expresa su misión: \textquote{Servir y dar su vida en rescate por muchos} (\textit{Mc} 10, 45).

\ccesec{Jesús acepta libremente el amor redentor del Padre}

\n{609} Jesús, al aceptar en su corazón humano el amor del Padre hacia los hombres, \textquote{los amó hasta el extremo} (\textit{Jn} 13, 1) porque \textquote{nadie tiene mayor amor que el que da su vida por sus amigos} (\textit{Jn} 15, 13). Tanto en el sufrimiento como en la muerte, su humanidad se hizo el instrumento libre y perfecto de su amor divino que quiere la salvación de los hombres (cf. \textit{Hb} 2, 10. 17-18; 4, 15; 5, 7-9). En efecto, aceptó libremente su pasión y su muerte por amor a su Padre y a los hombres que el Padre quiere salvar: \textquote{Nadie me quita [la vida]; yo la doy voluntariamente} (\textit{Jn} 10, 18). De aquí la soberana libertad del Hijo de Dios cuando Él mismo se encamina hacia la muerte (cf. \textit{Jn} 18, 4-6; \textit{Mt} 26, 53).


\ccesec{Jesús anticipó en la cena la ofrenda libre de su vida}

\n{610} Jesús expresó de forma suprema la ofrenda libre de sí mismo en la cena tomada con los doce Apóstoles (cf. \textit{Mt} 26, 20), en \textquote{la noche en que fue entregado} (\textit{1 Co} 11, 23). En la víspera de su Pasión, estando todavía libre, Jesús hizo de esta última Cena con sus Apóstoles el memorial de su ofrenda voluntaria al Padre (cf. \textit{1 Co} 5, 7), por la salvación de los hombres: \textquote{Este es mi Cuerpo que va a \textit{ser entregado} por vosotros} (\textit{Lc} 22, 19). \textquote{Esta es mi sangre de la Alianza que va a \textit{ser derramada} por muchos para remisión de los pecados} (\textit{Mt} 26, 28).

\n{611} La Eucaristía que instituyó en este momento será el \textquote{memorial} (\textit{1 Co} 11, 25) de su sacrificio. Jesús incluye a los Apóstoles en su propia ofrenda y les manda perpetuarla (cf. \textit{Lc} 22, 19). Así Jesús instituye a sus apóstoles sacerdotes de la Nueva Alianza: \textquote{Por ellos me consagro a mí mismo para que ellos sean también consagrados en la verdad} (\textit{Jn} 17, 19; cf. Concilio de Trento: DS, 1752; 1764).

\ccesec{La agonía de Getsemaní}

\n{612} El cáliz de la Nueva Alianza que Jesús anticipó en la Cena al ofrecerse a sí mismo (cf. \textit{Lc} 22, 20), lo acepta a continuación de manos del Padre en su agonía de Getsemaní (cf. \textit{Mt} 26, 42) haciéndose \textquote{obediente hasta la muerte} (\textit{Flp} 2, 8; cf. \textit{Hb} 5, 7-8). Jesús ora: \textquote{Padre mío, si es posible, que pase de mí este cáliz\ldots} (\textit{Mt} 26, 39). Expresa así el horror que representa la muerte para su naturaleza humana. Esta, en efecto, como la nuestra, está destinada a la vida eterna; además, a diferencia de la nuestra, está perfectamente exenta de pecado (cf. \textit{Hb} 4, 15) que es la causa de la muerte (cf. \textit{Rm} 5, 12); pero sobre todo está asumida por la persona divina del \textquote{Príncipe de la Vida} (\textit{Hch} 3, 15), de \textquote{el que vive}, \textit{Viventis assumpta} (\textit{Ap} 1, 18; cf. \textit{Jn} 1, 4; 5, 26). Al aceptar en su voluntad humana que se haga la voluntad del Padre (cf. \textit{Mt} 26, 42), acepta su muerte como redentora para \textquote{llevar nuestras faltas en su cuerpo sobre el madero} (\textit{1 P} 2, 24).

\ccesec{La muerte de Cristo es el sacrificio único y definitivo}

\n{613} La muerte de Cristo es a la vez el \textit{sacrificio} pascual que lleva a cabo la redención definitiva de los hombres (cf. \textit{1 Co} 5, 7; \textit{Jn} 8, 34-36) por medio del \textquote{Cordero que quita el pecado del mundo} (\textit{Jn} 1, 29; cf. \textit{1 P} 1, 19) y el \textit{sacrificio de la Nueva Alianza} (cf. \textit{1 Co} 11, 25) que devuelve al hombre a la comunión con Dios (cf. \textit{Ex} 24, 8) reconciliándole con Él por \textquote{la sangre derramada por muchos para remisión de los pecados} (\textit{Mt} 26, 28; cf. \textit{Lv} 16, 15-16).

\n{614} Este sacrificio de Cristo es único, da plenitud y sobrepasa a todos los sacrificios (cf. \textit{Hb} 10, 10). Ante todo es un don del mismo Dios Padre: es el Padre quien entrega al Hijo para reconciliarnos consigo (cf. \textit{1 Jn} 4, 10). Al mismo tiempo es ofrenda del Hijo de Dios hecho hombre que, libremente y por amor (cf. \textit{Jn} 15, 13), ofrece su vida (cf. \textit{Jn} 10, 17-18) a su Padre por medio del Espíritu Santo (cf. \textit{Hb} 9, 14), para reparar nuestra desobediencia.

\ccesec{Jesús reemplaza nuestra desobediencia por su obediencia}

\n{615} \textquote{Como [\ldots] por la desobediencia de un solo hombre, todos fueron constituidos pecadores, así también por la obediencia de uno solo todos serán constituidos justos} (\textit{Rm} 5, 19). Por su obediencia hasta la muerte, Jesús llevó a cabo la sustitución del Siervo doliente que \textquote{se dio a sí mismo en \textit{expiación}}, \textquote{cuando llevó el pecado de muchos}, a quienes \textquote{justificará y cuyas culpas soportará} (\textit{Is} 53, 10-12). Jesús repara por nuestras faltas y satisface al Padre por nuestros pecados (cf. Concilio de Trento: DS, 1529).

\ccesec{En la cruz, Jesús consuma su sacrificio}

\n{616} El \textquote{amor hasta el extremo} (\textit{Jn} 13, 1) es el que confiere su valor de redención y de reparación, de expiación y de satisfacción al sacrificio de Cristo. Nos ha conocido y amado a todos en la ofrenda de su vida (cf. \textit{Ga} 2, 20; \textit{Ef} 5, 2. 25). \textquote{El amor [\ldots] de Cristo nos apremia al pensar que, si uno murió por todos, todos por tanto murieron} (\textit{2 Co} 5, 14). Ningún hombre aunque fuese el más santo estaba en condiciones de tomar sobre sí los pecados de todos los hombres y ofrecerse en sacrificio por todos. La existencia en Cristo de la persona divina del Hijo, que al mismo tiempo sobrepasa y abraza a todas las personas humanas, y que le constituye Cabeza de toda la humanidad, hace posible su sacrificio redentor por todos.

\n{617} \textit{Sua sanctissima passione in ligno crucis nobis justificationem meruit} – \textquote{Por su sacratísima pasión en el madero de la cruz nos mereció la justificación}, enseña el Concilio de Trento (DS, 1529) subrayando el carácter único del sacrificio de Cristo como \textquote{causa de salvación eterna} (\textit{Hb} 5, 9). Y la Iglesia venera la Cruz cantando: \textit{O crux, ave, spes unica} – \textquote{Salve, oh cruz, única esperanza} (Añadidura litúrgica al himno \textquote{Vexilla Regis}: \textit{Liturgia de las Horas}).

\ccesec{Nuestra participación en el sacrificio de Cristo}

\n{618} La Cruz es el único sacrificio de Cristo \textquote{único mediador entre Dios y los hombres} (\textit{1 Tm} 2, 5). Pero, porque en su Persona divina encarnada, \textquote{se ha unido en cierto modo con todo hombre} (GS 22, 2) Él \textquote{ofrece a todos la posibilidad de que, en la forma de Dios sólo conocida [\ldots] se asocien a este misterio pascual} (GS 22, 5). Él llama a sus discípulos a \textquote{tomar su cruz y a seguirle} (\textit{Mt} 16, 24) porque Él \textquote{sufrió por nosotros dejándonos ejemplo para que sigamos sus huellas} (\textit{1 P} 2, 21). Él quiere, en efecto, asociar a su sacrificio redentor a aquellos mismos que son sus primeros beneficiarios (cf. \textit{Mc} 10, 39; \textit{Jn} 21, 18-19; \textit{Col} 1, 24). Eso lo realiza en forma excelsa en su Madre, asociada más íntimamente que nadie al misterio de su sufrimiento redentor (cf. \textit{Lc} 2, 35):

\ccecite{\textquote{Esta es la única verdadera escala del paraíso, fuera de la Cruz no hay otra por donde subir al cielo} (Santa Rosa de Lima, cf. P. Hansen, \textit{Vita mirabilis}, Lovaina, 1668)}

\n{1992} La justificación nos fue \textit{merecida por la pasión de Cristo}, que se ofreció en la cruz como hostia viva, santa y agradable a Dios y cuya sangre vino a ser instrumento de propiciación por los pecados de todos los hombres. La justificación es concedida por el Bautismo, sacramento de la fe. Nos asemeja a la justicia de Dios que nos hace interiormente justos por el poder de su misericordia. Tiene por fin la gloria de Dios y de Cristo, y el don de la vida eterna (cf. Concilio de Trento: DS 1529)

\ccecite{\textquote{Pero ahora, independientemente de la ley, la justicia de Dios se ha manifestado, atestiguada por la ley y los profetas, justicia de Dios por la fe en Jesucristo, para todos los que creen –pues no hay diferencia alguna; todos pecaron y están privados de la gloria de Dios– y son justificados por el don de su gracia, en virtud de la redención realizada en Cristo Jesús, a quien Dios exhibió como instrumento de propiciación por su propia sangre, mediante la fe, para mostrar su justicia, pasando por alto los pecados cometidos anteriormente, en el tiempo de la paciencia de Dios; en orden a mostrar su justicia en el tiempo presente, para ser él justo y justificador del que cree en Jesús} (\textit{Rm} 3, 21-26).}

\end{ccebody}

\newpage 
\cceth{La oración de Jesús} 

\cceref{CEC 612, 2606, 2741}

\begin{ccebody}

\ccesec{La agonía de Getsemaní}

\n{612} El cáliz de la Nueva Alianza que Jesús anticipó en la Cena al ofrecerse a sí mismo (cf. \textit{Lc} 22, 20), lo acepta a continuación de manos del Padre en su agonía de Getsemaní (cf. \textit{Mt} 26, 42) haciéndose \textquote{obediente hasta la muerte} (\textit{Flp} 2, 8; cf. \textit{Hb} 5, 7-8). Jesús ora: \textquote{Padre mío, si es posible, que pase de mí este cáliz\ldots} (\textit{Mt} 26, 39). Expresa así el horror que representa la muerte para su naturaleza humana. Esta, en efecto, como la nuestra, está destinada a la vida eterna; además, a diferencia de la nuestra, está perfectamente exenta de pecado (cf. \textit{Hb} 4, 15) que es la causa de la muerte (cf. \textit{Rm} 5, 12); pero sobre todo está asumida por la persona divina del \textquote{Príncipe de la Vida} (\textit{Hch} 3, 15), de \textquote{el que vive}, \textit{Viventis assumpta} (\textit{Ap} 1, 18; cf. \textit{Jn} 1, 4; 5, 26). Al aceptar en su voluntad humana que se haga la voluntad del Padre (cf. \textit{Mt} 26, 42), acepta su muerte como redentora para \textquote{llevar nuestras faltas en su cuerpo sobre el madero} (\textit{1 P} 2, 24).

\n{2606} Todos las angustias de la humanidad de todos los tiempos, esclava del pecado y de la muerte, todas las súplicas y las intercesiones de la historia de la salvación están recogidas en este grito del Verbo encarnado. He aquí que el Padre las acoge y, por encima de toda esperanza, las escucha al resucitar a su Hijo. Así se realiza y se consuma el drama de la oración en la Economía de la creación y de la salvación. El Salterio nos da la clave para la comprensión de este drama por medio de Cristo. Es en el \textquote{hoy} de la Resurrección cuando dice el Padre: \textquote{Tú eres mi Hijo; yo te he engendrado hoy. \textit{Pídeme}, y te \textit{daré} en herencia las naciones, en propiedad los confines de la tierra} (\textit{Sal} 2, 7-8; cf. \textit{Hch} 13, 33).

La carta a los Hebreos expresa en términos dramáticos cómo actúa la plegaria de Jesús en la victoria de la salvación: \textquote{El cual, habiendo ofrecido en los días de su vida mortal ruegos y súplicas con poderoso clamor y lágrimas al que podía salvarle de la muerte, fue escuchado por su actitud reverente, y aun siendo Hijo, con lo que padeció experimentó la obediencia; y llegado a la perfección, se convirtió en causa de salvación eterna para todos los que le obedecen} (Hb 5, 7-9).

\n{2741} Jesús ora también por nosotros, en nuestro lugar y en favor nuestro. Todas nuestras peticiones han sido recogidas una vez por todas en sus palabras en la Cruz; y escuchadas por su Padre en la Resurrección: por eso no deja de interceder por nosotros ante el Padre (cf. \textit{Hb} 5, 7; 7, 25; 9, 24). Si nuestra oración está resueltamente unida a la de Jesús, en la confianza y la audacia filial, obtenemos todo lo que pidamos en su Nombre, y aún más de lo que pedimos: recibimos al Espíritu Santo, que contiene todos los dones.

\end{ccebody}

\newpage 
\cceth{Cristo el sumo sacerdote} 

\cceref{CEC 467, 540, 1137}

\begin{ccebody}

\n{467} Los monofisitas afirmaban que la naturaleza humana había dejado de existir como tal en Cristo al ser asumida por su persona divina de Hijo de Dios. Enfrentado a esta herejía, el cuarto Concilio Ecuménico, en Calcedonia, confesó en el año 451:

\ccecite{\textquote{Siguiendo, pues, a los Santos Padres, enseñamos unánimemente que hay que confesar a un solo y mismo Hijo y Señor nuestro Jesucristo: perfecto en la divinidad, y perfecto en la humanidad; verdaderamente Dios y verdaderamente hombre compuesto de alma racional y cuerpo; consubstancial con el Padre según la divinidad, y consubstancial con nosotros según la humanidad, \textquote{en todo semejante a nosotros, excepto en el pecado} (\textit{Hb} 4, 15); nacido del Padre antes de todos los siglos según la divinidad; y por nosotros y por nuestra salvación, nacido en los últimos tiempos de la Virgen María, la Madre de Dios, según la humanidad}.}

\ccecite{\textquote{Se ha de reconocer a un solo y mismo Cristo Señor, Hijo único en dos naturalezas, sin confusión, sin cambio, sin división, sin separación. La diferencia de naturalezas de ningún modo queda suprimida por su unión, sino que quedan a salvo las propiedades de cada una de las naturalezas y confluyen en un solo sujeto y en una sola persona} (Concilio de Calcedonia; DS, 301-302).}

\n{540} La tentación de Jesús manifiesta la manera que tiene de ser Mesías el Hijo de Dios, en oposición a la que le propone Satanás y a la que los hombres (cf. \textit{Mt} 16, 21-23) le quieren atribuir. Por eso Cristo ha vencido al Tentador \textit{en beneficio nuestro}: \textquote{Pues no tenemos un Sumo Sacerdote que no pueda compadecerse de nuestras flaquezas, sino probado en todo igual que nosotros, excepto en el pecado} (\textit{Hb} 4, 15). La Iglesia se une todos los años, durante los cuarenta días de \textit{la Gran Cuaresma}, al Misterio de Jesús en el desierto.

\ccesec{Los celebrantes de la liturgia celestial}

\n{1137} El Apocalipsis de san Juan, leído en la liturgia de la Iglesia, nos revela primeramente que \textquote{un trono estaba erigido en el cielo y Uno sentado en el trono} (\textit{Ap} 4, 2): \textquote{el Señor Dios} (\textit{Is} 6, 1; cf. \textit{Ez} 1, 26-28). Luego revela al Cordero, \textquote{inmolado y de pie} (\textit{Ap} 5, 6; cf. \textit{Jn} 1, 29): Cristo crucificado y resucitado, el único Sumo Sacerdote del santuario verdadero (cf. \textit{Hb} 4, 14-15; 10, 19-21; etc), el mismo \textquote{que ofrece y que es ofrecido, que da y que es dado} (\textit{Liturgia Bizantina. Anaphora Iohannis Chrysostomi}). Y por último, revela \textquote{el río de agua de vida [\ldots] que brota del trono de Dios y del Cordero} (\textit{Ap} 22, 1), uno de los más bellos símbolos del Espíritu Santo (cf. \textit{Jn} 4, 10-14; \textit{Ap} 21, 6).

\end{ccebody}


\newpage 
\cceth{La obediencia de Cristo y la nuestra} 

\cceref{CEC 2825}

\begin{ccebody}
\n{2825} Jesús, \textquote{aun siendo Hijo, con lo que padeció, experimentó la obediencia} (\textit{Hb} 5, 8). ¡Con cuánta más razón la deberemos experimentar nosotros, criaturas y pecadores, que hemos llegado a ser hijos de adopción en Él! Pedimos a nuestro Padre que una nuestra voluntad a la de su Hijo para cumplir su voluntad, su designio de salvación para la vida del mundo. Nosotros somos radicalmente impotentes para ello, pero unidos a Jesús y con el poder de su Espíritu Santo, podemos poner en sus manos nuestra voluntad y decidir escoger lo que su Hijo siempre ha escogido: hacer lo que agrada al Padre (cf. \textit{Jn} 8, 29):

\ccecite{\textquote{Adheridos a Cristo, podemos llegar a ser un solo espíritu con Él, y así cumplir su voluntad: de esta forma ésta se hará tanto en la tierra como en el cielo} (Orígenes, \textit{De oratione}, 26, 3).}

\ccecite{\textquote{Considerad cómo [Jesucristo] nos enseña a ser humildes, haciéndonos ver que nuestra virtud no depende sólo de nuestro esfuerzo sino de la gracia de Dios. Él ordena a cada fiel que ora, que lo haga universalmente por toda la tierra. Porque no dice \textquote{Que tu voluntad se haga} en mí o en vosotros \textquote{sino en toda la tierra}: para que el error sea desterrado de ella, que la verdad reine en ella, que el vicio sea destruido en ella, que la virtud vuelva a florecer en ella y que la tierra ya no sea diferente del cielo} (San Juan Crisóstomo, \textit{In Matthaeum} homilia 19, 5).}
\end{ccebody}

\img{cross_of_jerusalem}