\chapter{Vigilia Pascual en la Noche Santa}

\section{Lecturas}

\rtitle{PRIMERA LECTURA}

\rbook{Del libro del Génesis} \rred{1, 1–2, 2}

\rtheme{Vio Dios todo lo que había hecho, y era muy bueno}

\begin{scripture}
Al principio creó Dios el cielo y la tierra. La tierra estaba informe y vacía; la tiniebla cubría la superficie del abismo, mientras el espíritu de Dios se cernía sobre la faz de las aguas. 

Dijo Dios: 

\>{Exista la luz}.

Y la luz existió. Vio Dios que la luz era buena. Y separó Dios la luz de la tiniebla. Llamó Dios a la luz \textquote{día} y a la tiniebla llamó \textquote{noche}. Pasó una tarde, pasó una mañana: el día primero. 

Y dijo Dios: 

\>{Exista un firmamento entre las aguas, que separe aguas de aguas}.

E hizo Dios el firmamento y separó las aguas de debajo del firmamento de las aguas de encima del firmamento. Y así fue. Llamó Dios al firmamento \textquote{cielo}. Pasó una tarde, pasó una mañana: el día segundo. 

Dijo Dios: 

\>{Júntense las aguas de debajo del cielo en un solo sitio, y que aparezca lo seco}.

Y así fue. Llamó Dios a lo seco \textquote{tierra}, y a la masa de las aguas llamó \textquote{mar}. Y vio Dios que era bueno. 

Dijo Dios: 

\>{Cúbrase la tierra de verdor, de hierba verde que engendre semilla, y de árboles frutales que den fruto según su especie y que lleven semilla sobre la tierra}.

Y así fue. La tierra brotó hierba verde que engendraba semilla según su especie, y árboles que daban fruto y llevaban semilla según su especie. Y vio Dios que era bueno. Pasó una tarde, pasó una mañana: el día tercero. 

Dijo Dios: 

\>{Existan lumbreras en el firmamento del cielo, para separar el día de la noche, para señalar las fiestas, los días y los años, y sirvan de lumbreras en el firmamento del cielo, para iluminar sobre la tierra}.

Y así fue. E hizo Dios dos lumbreras grandes: la lumbrera mayor para regir el día, la lumbrera menor para regir la noche; y las estrellas. Dios las puso en el firmamento del cielo para iluminar la tierra, para regir el día y la noche y para separar la luz de la tiniebla. Y vio Dios que era bueno. Pasó una tarde, pasó una mañana: el día cuarto. 

Dijo Dios: 

\>{Bullan las aguas de seres vivientes, y vuelen los pájaros sobre la tierra frente al firmamento del cielo}.

Y creó Dios los grandes cetáceos y los seres vivientes que se deslizan y que las aguas fueron produciendo según sus especies, y las aves aladas según sus especies. Y vio Dios que era bueno. 

Luego los bendijo Dios, diciendo: 

\>{Sed fecundos y multiplicaos, llenad las aguas del mar; y que las aves se multipliquen en la tierra}.

Pasó una tarde, pasó una mañana: el día quinto. 

Dijo Dios: 

\>{Produzca la tierra seres vivientes según sus especies: ganados, reptiles y fieras según sus especies}.

Y así fue. E hizo Dios las fieras según sus especies, los ganados según sus especies y los reptiles según sus especies. Y vio Dios que era bueno. 

Dijo Dios: 

\>{Hagamos al hombre a nuestra imagen y semejanza; que domine los peces del mar, las aves del cielo, los ganados y los reptiles de la tierra}.

Y creó Dios al hombre a su imagen, a imagen de Dios lo creó, varón y mujer los creó. 

Dios los bendijo; y les dijo Dios: 

\>{Sed fecundos y multiplicaos, llenad la tierra y sometedla; dominad los peces del mar, las aves del cielo y todos los animales que se mueven sobre la tierra}.

Y dijo Dios: 

\>{Mirad, os entrego todas las hierbas que engendran semilla sobre la superficie de la tierra y todos los árboles frutales que engendran semilla: os servirán de alimento. Y la hierba verde servirá de alimento a todas las fieras de la tierra, a todas las aves del cielo, a todos los reptiles de la tierra y a todo ser que respira}.

Y así fue. Vio Dios todo lo que había hecho, y era muy bueno. Pasó una tarde, pasó una mañana: el día sexto. 

Así quedaron concluidos el cielo, la tierra y todo el universo. Y habiendo concluido el día séptimo la obra que había hecho, descansó el día séptimo de toda la obra que había hecho.
\end{scripture}

\rtitle{Salmo responsorial a la primera lectura (opción 1)}

\rbook{Salmo} \rred{103, 1-2a. 5-6. 10 y 12. 13-14. 24 y 35c}

\rtheme{Envía tu espíritu, Señor, y repuebla la faz de la tierra.}

\begin{psbody}
Bendice, alma mía, al Señor: 
¡Dios mío, qué grande eres! 
Te vistes de belleza y majestad, 
a luz te envuelve como un manto. 

Asentaste la tierra sobre sus cimientos, 
y no vacilará jamás; 
la cubriste con el manto del océano, 
y las aguas se posaron sobre las montañas. 

De los manantiales sacas los ríos, 
para que fluyan entre los montes; 
junto a ellos habitan las aves del cielo, 
y entre las frondas se oye su canto. 

Desde tu morada riegas los montes, 
y la tierra se sacia de tu acción fecunda; 
haces brotar hierba para los ganados, 
y forraje para los que sirven al hombre. 
Él saca pan de los campos.
 
Cuántas son tus obras, Señor, 
y todas las hiciste con sabiduría; 
la tierra está llena de tus criaturas. 
¡Bendice, alma mía, al Señor! 
\end{psbody}

\newpage 

\rtitle{Salmo responsorial a la primera lectura (opción 2)}

\rbook{Salmo} \rred{32, 4-5. 6-7. 12-13. 20 y 22}

\rtheme{La misericordia del Señor llena la tierra.}

\begin{psbody}
La palabra del Señor es sincera, 
y todas sus acciones son leales; 
él ama la justicia y el derecho, 
y su misericordia llena la tierra. 

La palabra del Señor hizo el cielo; 
el aliento de su boca, sus ejércitos; 
encierra en un odre las aguas marinas, 
mete en un depósito el océano.

Dichosa la nación cuyo Dios es el Señor, 
el pueblo que él se escogió como heredad. 
El Señor mira desde el cielo, 
se fija en todos los hombres. 

Nosotros aguardamos al Señor: 
él es nuestro auxilio y escudo. 
Que tu misericordia, Señor, venga sobre nosotros, 
como lo esperamos de ti. 
\end{psbody}

\rtitle{SEGUNDA LECTURA}

\rbook{Del libro del Génesis} \rred{22, 1-18}

\rtheme{El sacrificio de Abrahán, nuestro padre en la fe}

\begin{scripture}
En aquellos días, Dios puso a prueba a Abrahán. Le dijo: 

\>{¡Abrahán!}.

Él respondió: 

\>{Aquí estoy}.

Dios dijo: 

\>{Toma a tu hijo único, al que amas, a Isaac, y vete a la tierra de Moria y ofrécemelo allí en holocausto en uno de los montes que yo te indicaré}.

Abrahán madrugó, aparejó el asno y se llevó consigo a dos criados y a su hijo Isaac; cortó leña para el holocausto y se encaminó al lugar que le había indicado Dios. 

Al tercer día levantó Abrahán los ojos y divisó el sitio desde lejos. Abrahán dijo a sus criados: 

\>{Quedaos aquí con el asno; yo con el muchacho iré hasta allá para adorar, y después volveremos con vosotros}.


Abrahán tomó la leña para el holocausto, se la cargó a su hijo Isaac, y él llevaba el fuego y el cuchillo. Los dos caminaban juntos. 

Isaac dijo a Abrahán, su padre: 

\>{Padre}.

Él respondió: 

\>{Aquí estoy, hijo mío}.

El muchacho dijo: 

\>{Tenemos fuego y leña, pero ¿dónde está el cordero para el holocausto?}.

Abrahán contestó: 

\>{Dios proveerá el cordero para el holocausto, hijo mío}.

Y siguieron caminando juntos. 

Cuando llegaron al sitio que le había dicho Dios, Abrahán levantó allí el altar y apiló la leña, luego ató a su hijo Isaac y lo puso sobre el altar, encima de la leña. Entonces Abrahán alargó la mano y tomó el cuchillo para degollar a su hijo. 

Pero el ángel del Señor le gritó desde el cielo: 

\>{¡Abrahán, Abrahán!}.

Él contestó: 

\>{Aquí estoy}.

El ángel le ordenó: 

\>{No alargues la mano contra el muchacho ni le hagas nada. Ahora he comprobado que temes a Dios, porque no te has reservado a tu hijo, a tu único hijo}.

Abrahán levantó los ojos y vio un carnero enredado por los cuernos en la maleza. Se acercó, tomó el carnero y lo ofreció en holocausto en lugar de su hijo. 

Abrahán llamó aquel sitio \textquote{El Señor ve}, por lo que se dice aún hoy \textquote{En el monte el Señor es visto}. 

El ángel del Señor llamó a Abrahán por segunda vez desde el cielo y le dijo: 

\>{Juro por mí mismo, oráculo del Señor: por haber hecho esto, por no haberte reservado tu hijo, tu hijo único, te colmaré de bendiciones y multiplicaré a tus descendientes como las estrellas del cielo y como la arena de la playa. Tus descendientes conquistarán las puertas de sus enemigos. Todas las naciones de la tierra se bendecirán con tu descendencia, porque has escuchado mi voz}.
\end{scripture}

\newpage 

\rtitle{Salmo responsorial a la segunda lectura}

\rbook{Salmo} \rred{15, 5 y 8. 9-10. 11}

\rtheme{Protégeme, Dios mío, que me refugio en ti}

\begin{psbody}
El Señor es el lote de mi heredad y mi copa; 
mi suerte está en tu mano. 
Tengo siempre presente al Señor, 
con él a mi derecha no vacilaré. 

Por eso se me alegra el corazón, 
se gozan mis entrañas, 
y mi carne descansa esperanzada. 
Porque no me abandonarás en la región de los muertos 
ni dejarás a tu fiel conocer la corrupción. 

Me enseñarás el sendero de la vida, 
me saciarás de gozo en tu presencia, 
de alegría perpetua a tu derecha. 
\end{psbody}

\rtitle{TERCERA LECTURA}

\rbook{Del libro del Éxodo} \rred{14, 15–15, 1}

\rtheme{Los hijos de Israel entraron en medio del mar, por lo seco}

\begin{scripture}
En aquellos días, el Señor dijo a Moisés: 

\>{¿Por qué sigues clamando a mí? Di a los hijos de Israel que se pongan en marcha. Y tú, alza tu cayado, extiende tu mano sobre el mar y divídelo, para que los hijos de Israel pasen por medio del mar, por lo seco. Yo haré que los egipcios se obstinen y entren detrás de vosotros, y me cubriré de gloria a costa del faraón y de todo su ejército, de sus carros y de sus jinetes. Así sabrán los egipcios que yo soy el Señor, cuando me haya cubierto de gloria a costa del faraón, de sus carros y de sus jinetes}.

Se puso en marcha el ángel del Señor, que iba al frente del ejército de Israel, y pasó a retaguardia. También la columna de nube, que iba delante de ellos, se desplazó y se colocó detrás, poniéndose entre el campamento de los egipcios y el campamento de Israel. La nube era tenebrosa y transcurrió toda la noche sin que los ejércitos pudieran aproximarse el uno al otro. Moisés extendió su mano sobre el mar y el Señor hizo retirarse el mar con un fuerte viento del Este que sopló toda la noche; el mar se secó y se dividieron las aguas. Los hijos de Israel entraron en medio del mar, en lo seco, y las aguas les hacían de muralla a derecha e izquierda. Los egipcios los persiguieron y entraron tras ellos, en medio del mar: todos los caballos del faraón, sus carros y sus jinetes. 

Era ya la vigilia matutina cuando el Señor miró desde la columna de fuego y humo hacia el ejército de los egipcios y sembró el pánico en el ejército egipcio. Trabó las ruedas de sus carros, haciéndolos avanzar pesadamente. 

Los egipcios dijeron: 

\>{Huyamos ante Israel, porque el Señor lucha por él contra Egipto}.

Luego dijo el Señor a Moisés: 

\>{Extiende tu mano sobre el mar, y vuelvan las aguas sobre los egipcios, sus carros y sus jinetes}.

Moisés extendió su mano sobre el mar; y al despuntar el día el mar recobró su estado natural, de modo que los egipcios, en su huida, toparon con las aguas. Así precipitó el Señor a los egipcios en medio del mar. 

Las aguas volvieron y cubrieron los carros, los jinetes y todo el ejército del faraón, que había entrado en el mar. Ni uno solo se salvó. 

Mas los hijos de Israel pasaron en seco por medio del mar, mientras las aguas hacían de muralla a derecha e izquierda. 

Aquel día salvó el Señor a Israel del poder de Egipto, e Israel vio a los egipcios muertos, en la orilla del mar. Vio, pues, Israel la mano potente que el Señor había desplegado contra los egipcios, y temió el pueblo al Señor, y creyó en el Señor y en Moisés, su siervo. 

Entonces Moisés y los hijos de Israel entonaron este canto al Señor:
\end{scripture}

\rtitle{Salmo responsorial a la tercera lectura}

\rbook{Del libro del Éxodo} \rred{15, 1-2. 3-4. 5-6. 17-18}

\rtheme{Cantaré al Señor, gloriosa es su victoria}

\begin{psbody}
Cantaré al Señor, gloriosa es su victoria, 
caballos y carros ha arrojado en el mar. 
Mi fuerza y mi poder es el Señor, 
El fue mi salvación. 
Él es mi Dios: yo lo alabaré; 
el Dios de mis padres: yo lo ensalzaré. 

El Señor es un guerrero, 
su nombre es \textquote{El Señor}. 
Los carros del faraón los lanzó al mar, 
ahogó en el mar Rojo a sus mejores capitanes. 

Las olas los cubrieron, 
bajaron hasta el fondo como piedras. 
Tu diestra, Señor, es magnífica en poder, 
tu diestra, Señor, tritura al enemigo. 

Lo introduces y lo plantas en el monte de tu heredad, 
lugar del que hiciste tu trono, Señor; 
santuario, Señor, que fundaron tus manos. 
El Señor reina por siempre jamás. 
\end{psbody}

\rtitle{CUARTA LECTURA}

\rbook{Del libro del profeta Isaías} \rred{54, 5-14}

\rtheme{Con amor eterno te quiere el Señor, tu libertador}

\begin{readprose}
Quien te desposa es tu Hacedor: 
   su nombre es Señor todopoderoso. 

Tu libertador es el Santo de Israel: 
   se llama \textquote{Dios de toda la tierra}. 

Como a mujer abandonada y abatida 
   te llama el Señor; 
   como a esposa de juventud, repudiada 
   –dice tu Dios–. 

Por un instante te abandoné, 
   pero con gran cariño te reuniré. 

En un arrebato de ira, 
   por un instante te escondí mi rostro, 
   pero con amor eterno te quiero 
   –dice el Señor, tu libertador–. 

Me sucede como en los días de Noé: 
   juré que las aguas de Noé 
   no volverían a cubrir la tierra; 
   así juro no irritarme contra ti 
   ni amenazarte. 

Aunque los montes cambiasen 
   y vacilaran las colinas, 
   no cambiaría mi amor, 
   ni vacilaría mi alianza de paz 
   –dice el Señor que te quiere–. 

¡Ciudad afligida, azotada por el viento, 
   a quien nadie consuela! 

Mira, yo mismo asiento tus piedras sobre azabaches, 
   tus cimientos sobre zafiros; 
   haré tus almenas de rubí, 
   tus puertas de esmeralda, 
   y de piedras preciosas tus bastiones. 

Tus hijos serán discípulos del Señor, 
   gozarán de gran prosperidad tus constructores. 

Tendrás tu fundamento en la justicia: 
   lejos de la opresión, no tendrás que temer; 
   lejos del terror, que no se acercará.
\end{readprose}


\rtitle{Salmo responsorial a la cuarta lectura}

\rbook{Salmo} \rred{29, 2 y 4. 5-6. 11-12a y 13b}

\rtheme{Te ensalzaré, Señor, porque me has librado}

\begin{psbody}
Te ensalzaré, Señor, porque me has librado 
y no has dejado que mis enemigos se rían de mí. 
Señor, sacaste mi vida del abismo, 
me hiciste revivir cuando bajaba a la fosa. 

Tañed para el Señor, fieles suyos, 
celebrad el recuerdo de su nombre santo; 
su cólera dura un instante; 
su bondad, de por vida; 
al atardecer nos visita el llanto; 
por la mañana, el júbilo. 

Escucha, Señor, y ten piedad de mí; 
Señor, socórreme. 
Cambiaste mi luto en danzas. 
Señor, Dios mío, te daré gracias por siempre. 
\end{psbody}

\rtitle{QUINTA LECTURA}

\rbook{Del libro del profeta Isaías} \rred{55, 1-11}

\rtheme{Venid a mí y viviréis. Sellaré con vosotros una alianza perpetua}

\begin{readprose}
Esto dice el Señor: 
	
«Oíd, sedientos todos, acudid por agua; 
   venid, también los que no tenéis dinero: 
   comprad trigo y comed, venid y comprad, 
   sin dinero y de balde, vino y leche. 

¿Por qué gastar dinero en lo que no alimenta 
   y el salario en lo que no da hartura? 

Escuchadme atentos y comeréis bien, 
   saborearéis platos sustanciosos. 

Inclinad vuestro oído, venid a mí: 
   escuchadme y viviréis. 

Sellaré con vosotros una alianza perpetua, 
   las misericordias firmes hechas a David: 
   lo hice mi testigo para los pueblos, 
   guía y soberano de naciones. 

Tú llamarás a un pueblo desconocido, 
   un pueblo que no te conocía correrá hacia ti; 
   porque el Señor tu Dios, 
   el Santo de Israel te glorifica. 

Buscad al Señor mientras se deja encontrar, 
   invocadlo mientras está cerca. 

Que el malvado abandone su camino, 
   y el malhechor sus planes; 
   que se convierta al Señor, y él tendrá piedad, 
   a nuestro Dios, que es rico en perdón. 

Porque mis planes no son vuestros planes, 
   vuestros caminos no son mis caminos 
   –oráculo del Señor–. 

Cuanto dista el cielo de la tierra, 
   así distan mis caminos de los vuestros, 
   y mis planes de vuestros planes. 

Como bajan la lluvia y la nieve desde el cielo, 
   y no vuelven allá, sino después de empapar la tierra, 
   de fecundarla y hacerla germinar, 
   para que dé semilla al sembrador 
   y pan al que come, 
   así será mi palabra que sale de mi boca: 
   no volverá a mí vacía, 
   sino que cumplirá mi deseo 
   y llevará a cabo mi encargo».
\end{readprose}


\rtitle{Salmo responsorial a la quinta lectura}

\rbook{Del libro del profeta Isaías} \rred{12, 2-3. 4bcd. 5-6}

\rtheme{Sacaréis aguas con gozo de las fuentes de la salvación}

\begin{psbody}
\textquote{El Señor es mi Dios y salvador:
confiaré y no temeré,
porque mi fuerza y mi poder es el Señor,
él fue mi salvación}.
Y sacaréis aguas con gozo
de las fuentes de la salvación. 

\textquote{Dad gracias al Señor,
invocad su nombre,
contad a los pueblos sus hazañas,
proclamad que su nombre es excelso}. 

Tañed para el Señor, que hizo proezas,
anunciadlas a toda la tierra;
gritad jubilosos, habitantes de Sión:
porque es grande medio de ti el Santo de Israel. 
\end{psbody}

\rtitle{SEXTA LECTURA}

\rbook{Del libro del profeta Baruc} \rred{3, 9-15. 32–4, 4}

\rtheme{Camina al resplandor del Señor}

\begin{readprose}
Escucha, Israel, mandatos de vida; 
   presta oído y aprende prudencia. 

¿Cuál es la razón, Israel, 
   de que sigas en país enemigo, 
   envejeciendo en tierra extranjera; 
   de que te crean un ser contaminado, 
   un muerto habitante del Abismo? 
   
¡Abandonaste la fuente de la sabiduría!

Si hubieras seguido el camino de Dios, 
   habitarías en paz para siempre.
   
Aprende dónde está la prudencia, 
   dónde el valor y la inteligencia, 
   dónde una larga vida, 
   la luz de los ojos y la paz.
   
¿Quién encontró su lugar 
   o tuvo acceso a sus tesoros?
   
El que todo lo sabe la conoce, 
   la ha examinado y la penetra; 
   el que creó la tierra para siempre 
   y la llenó de animales cuadrúpedos; 
   el que envía la luz y le obedece, 
   la llama y acude temblorosa; 
   a los astros que velan gozosos 
   arriba en sus puestos de guardia, 
   los llama, y responden: \textquote{Presentes}, 
   y brillan gozosos para su Creador. 
   
Este es nuestro Dios, 
   y no hay quien se le pueda comparar; 
   rastreó el camino de la inteligencia 
   y se lo enseñó a su hijo, Jacob, 
   se lo mostró a su amado, Israel. 
   
Después apareció en el mundo 
   y vivió en medio de los hombres. 
   
Es el libro de los mandatos de Dios, 
   la ley de validez eterna: 
   los que la guarden vivirán; 
   los que la abandonen morirán. 
   
Vuélvete, Jacob, a recibirla, 
   camina al resplandor de su luz; 
   no entregues a otros tu gloria, 
   ni tu dignidad a un pueblo extranjero. 
   
¡Dichosos nosotros, Israel, 
   que conocemos lo que agrada al Señor!
\end{readprose}


\rtitle{Salmo responsorial a la sexta lectura}

\rbook{Salmo} \rred{18, 8. 9. 10. 11}

\rtheme{Señor, tú tienes palabras de vida eterna}

\begin{psbody}
La ley del Señor es perfecta 
y es descanso del alma; 
el precepto del Señor es fiel 
e instruye a los ignorantes. 

Los mandatos del Señor son rectos 
y alegran el corazón; 
la norma del Señor es límpida 
y da luz a los ojos. 

El temor del Señor es puro 
y eternamente estable; 
los mandamientos del Señor son verdaderos 
y enteramente justos. 

Más preciosos que el oro, 
más que el oro fino; 
más dulces que la miel 
de un panal que destila. 
\end{psbody}

\newpage

\rtitle{SÉPTIMA LECTURA}

\rbook{Del libro del profeta Ezequiel} \rred{36, 16-28}

\rtheme{Derramaré sobre vosotros un agua pura, y os daré un corazón nuevo}

\begin{readprose}
Me vino esta palabra del Señor: 
	
«Hijo de hombre, la casa de Israel profanó con su conducta 
   y sus acciones la tierra en que habitaba. 

Me enfurecí contra ellos, 
   por la sangre que habían derramado en el país, 
   y por haberlo profanado con sus ídolos. 

Los dispersé por las naciones, 
   y anduvieron dispersos por diversos países. 

Los he juzgado según su conducta y sus acciones. 

Al llegar a las diversas naciones, 
   profanaron mi santo nombre, 
   ya que de ellos se decía: 
   “Estos son el pueblo del Señor 
   y han debido abandonar su tierra”. 

Así que tuve que defender mi santo nombre, 
   profanado por la casa de Israel 
   entre las naciones adonde había ido. 

Por eso, di a la casa de Israel: 
   “Esto dice el Señor Dios: 
   No hago esto por vosotros, casa de Israel, 
   sino por mi santo nombre, profanado por vosotros 
   en las naciones a las que fuisteis. 

Manifestaré la santidad de mi gran nombre, 
   profanado entre los gentiles, 
   porque vosotros lo habéis profanado en medio de ellos. 

Reconocerán las naciones que yo soy el Señor 
   –oráculo del Señor Dios–, 
   cuando por medio de vosotros les haga ver mi santidad. 

Os recogeré de entre las naciones, 
   os reuniré de todos los países 
   y os llevaré a vuestra tierra. 

Derramaré sobre vosotros un agua pura 
   que os purificará: 
   de todas vuestras inmundicias e idolatrías 
   os he de purificar; 
   y os daré un corazón nuevo, 
   y os infundiré un espíritu nuevo; 
   arrancaré de vuestra carne el corazón de piedra, 
   y os daré un corazón de carne. 

Os infundiré mi espíritu, 
   y haré que caminéis según mis preceptos, 
   y que guardéis y cumpláis mis mandatos. 

Y habitaréis en la tierra que di a vuestros padres. 

Vosotros seréis mi pueblo, 
   y yo seré vuestro Dios”».
\end{readprose}

\rtitle{Salmo responsorial a la séptima lectura}

\rbr{(cuando no hay Bautismo)}

\rbook{Salmo} \rred{41, 3. 5bcd; 42, 3. 4}

\rtheme{Como busca la cierva corrientes de agua, así mi alma te busca a ti, Dios mío}

\begin{psbody}
Mi alma tiene sed de Dios, del Dios vivo: 
cuándo entraré a ver el rostro de Dios? 

Cómo entraba en el recinto santo, 
cómo avanzaba hacia la casa de Dios, 
entre cantos de júbilo y alabanza, 
en el bullicio de la fiesta. 

Envía tu luz y tu verdad: 
que ellas me guíen 
y me conduzcan hasta tu monte santo, 
hasta tu morada. 

Me acercaré al altar de Dios, 
al Dios de mi alegría; 
y te daré gracias al son de la cítara, 
Dios, Dios mío. 
\end{psbody}

\newpage 

\rtitle{Salmo responsorial a la séptima lectura}

\rbr{(cuando hay Bautismo, opción 1)}

\rbook{Isaías \rred{12, 2-3. 4bcde. 5-6} }

\rtheme{Sacaréis aguas con gozo de las fuentes de la salvación.}

\begin{psbody}
\textquote{Él es mi Dios y Salvador:
confiaré y no temeré,
porque mi fuerza y mi poder es el Señor,
él fue mi salvación}.
Y sacaréis aguas con gozo
de las fuentes de la salvación. 

\textquote{Dad gracias al Señor,
invocad su nombre,
contad a los pueblos sus hazañas,
proclamad que su nombre es excelso}. 

Tañed para el Señor, que hizo proezas,
anunciadlas a toda la tierra;
gritad jubilosos, habitantes de Sion,
porque es grande en medio de ti el Santo de Israel. 
\end{psbody}

\title{Salmo responsorial a la séptima lectura}

\rbr{(cuando hay Bautismo, opción 2)}

\rbook{Salmo \rred{50, 12-13. 14-15. 18-19}}

\rtheme{Oh, Dios, crea en mí un corazón puro}

\begin{psbody}
Oh, Dios, crea en mí un corazón puro, 
renuévame por dentro con espíritu firme. 
No me arrojes lejos de tu rostro, 
no me quites tu santo espíritu. 

Devuélveme la alegría de tu salvación, 
afiánzame con espíritu generoso. 
Enseñaré a los malvados tus caminos, 
los pecadores volverán a ti. 

Los sacrificios no te satisfacen: 
si te ofreciera un holocausto, no lo querrías. 
El sacrificio agradable a Dios 
es un espíritu quebrantado; 
un corazón quebrantado y humillado, 
tú, oh, Dios, tú no lo desprecias. 
\end{psbody}

\rtitle{EPÍSTOLA}

\rbook{De la carta del apóstol san Pablo a los Romanos \rred{6, 3-11}}

\rtheme{Cristo, una vez resucitado de entre los muertos, ya no muere más}

\begin{scripture}
Hermanos: 

Cuantos fuimos bautizados en Cristo Jesús fuimos bautizados en su muerte. 

Por el bautismo fuimos sepultados con él en la muerte, para que, lo mismo que Cristo resucitó de entre los muertos por la gloria del Padre, así también nosotros andemos en una vida nueva. 

Pues si hemos sido incorporados a él en una muerte como la suya, lo seremos también en una resurrección como la suya; sabiendo que nuestro hombre viejo fue crucificado con Cristo, para que fuera destruido el cuerpo de pecado, y, de este modo, nosotros dejáramos de servir al pecado; porque quien muere ha quedado libre del pecado. 

Si hemos muerto con Cristo, creemos que también viviremos con él; pues sabemos que Cristo, una vez resucitado de entre los muertos, ya no muere más; la muerte ya no tiene dominio sobre él. Porque quien ha muerto, ha muerto al pecado de una vez para siempre; y quien vive, vive para Dios. 

Lo mismo vosotros, consideraos muertos al pecado y vivos para Dios en Cristo Jesús.
\end{scripture}

\rtitle{SALMO RESPONSORIAL}

\rbook{Salmo} \rred{117, 1-2. 16ab-17. 22-23}

\rtheme{Aleluya, aleluya, aleluya}

\begin{psbody}
Dad gracias al Señor porque es bueno, 
porque es eterna su misericordia. 
Diga la casa de Israel: 
eterna es su misericordia. 

\textquote{La diestra del Señor es poderosa, 
la diestra del Señor es excelsa}. 
No he de morir, viviré 
para contar las hazañas del Señor. 

La piedra que desecharon los arquitectos 
es ahora la piedra angular. 
Es el Señor quien lo ha hecho, 
ha sido un milagro patente. 
\end{psbody}

\rtitle{EVANGELIO}

\rbook{Del Santo Evangelio según san Marcos} \rred{16, 1-7}

\rtheme{Jesús el Nazareno, el crucificado, ha resucitado}

\begin{scripture}
Pasado el sábado, María Magdalena, María la de Santiago y Salomé compraron aromas para ir a embalsamar a Jesús. Y muy temprano, el primer día de la semana, al salir el sol, fueron al sepulcro. Y se decían unas a otras: 

\>{¿Quién nos correrá la piedra de la entrada del sepulcro?}. 

Al mirar, vieron que la piedra estaba corrida y eso que era muy grande. Entraron en el sepulcro y vieron a un joven sentado a la derecha, vestido de blanco. Y quedaron aterradas. Él les dijo: 

\>{No tengáis miedo. ¿Buscáis a Jesús el Nazareno, el crucificado? Ha resucitado. No está aquí. Mirad el sitio donde lo pusieron. \\Pero id a decir a sus discípulos y a Pedro: \textquote{Él va por delante de vosotros a Galilea. Allí lo veréis, como os dijo}}.
\end{scripture}


\newsection
\section{Comentario Patrístico}

\subsection{San Juan Pablo II, papa}

\ptheme{El primer signo de la Resurrección}

\src{\textit{Catequesis,} nn. 5-9: \\Audiencia General, 1 de febrero de 1989.}

\begin{body}
\textquote{Entraron en el sepulcro y vieron\ldots} (\textit{Mc} 16, 5)

\ltr{E}{n} el ámbito de los acontecimientos pascuales, el primer elemento ante el que nos encontramos es \textit{el \textquote{sepulcro vacío}}. Sin duda no es por sí mismo una prueba directa. La ausencia del cuerpo de Cristo en el sepulcro en el que había sido depositado podría \textit{explicarse de otra forma}, como de hecho pensó por un momento María Magdalena cuando, viendo el sepulcro vacío, supuso que alguno habría sustraído el cuerpo de Jesús (cf. \textit{Jn} 20, 13).

Más aún el Sanedrín trató de hacer correr la voz de que, mientras dormían los soldados, el cuerpo habría sido robado por los discípulos. \textquote{Y se corrió esa versión entre los judíos, ―anota Mateo― hasta el día de hoy} (\textit{Mt} 28, 12-15).

A pesar de esto el \textquote{\textit{sepulcro vacío}} ha constituido para todos, amigos y enemigos, un signo impresionante. Para las personas de buena voluntad su descubrimiento fue \textit{el primer paso hacia el reconocimiento del \textquote{hecho} de la resurrección como una verdad que no podía ser refutada}.

Así fue ante todo \textit{para las mujeres}, que muy de mañana se habían acercado al sepulcro para ungir el cuerpo de Cristo. Fueron las primeras en acoger el anuncio: \textquote{Ha resucitado, no está aquí\ldots Pero id a decir a sus discípulos y a Pedro\ldots} (\textit{Mc} 16, 6-7). \textquote{Recordad cómo os habló cuando estaba todavía en Galilea, diciendo: \textquote{Es necesario que el Hijo del hombre sea entregado en manos de los pecadores y sea crucificado, y al tercer día resucite}. Y ellas recordaron sus palabras} (\textit{Lc} 24, 6-8).

Ciertamente las mujeres estaban sorprendidas y asustadas (cf. \textit{Mc} 16, 8; \textit{Lc} 24, 5). Ni siquiera ellas estaban dispuestas a rendirse demasiado fácilmente a un hecho que, aún predicho por Jesús, estaba efectivamente por encima de toda posibilidad de imaginación y de invención. Pero en su sensibilidad y finura intuitiva ellas, y especialmente María Magdalena, se aferraron a la realidad y corrieron a donde estaban los Apóstoles para darles la alegre noticia.

El Evangelio de Mateo (28, 8-10) nos informa que a lo largo del camino Jesús mismo les salió al encuentro, las saludó y les renovó el mandato de llevar el anuncio a los hermanos (\textit{Mt} 28, 10). De esta forma las mujeres fueron las primeras mensajeras de la resurrección de Cristo, y lo fueron para los mismos Apóstoles (\textit{Lc} 24, 10). ¡Hecho elocuente sobre la importancia de la mujer ya en los días del acontecimiento pascual!

Entre los que recibieron el anuncio de María Magdalena estaban \textit{Pedro} y \textit{Juan} (cf. \textit{Jn} 20, 3-8). Ellos se acercaron al sepulcro no sin titubeos, tanto más cuanto que Marta les había hablado de una sustracción del cuerpo de Jesús del sepulcro (cf. \textit{Jn} 20, 2). Llegados al sepulcro, también ellos lo encontraron vacío. Terminaron creyendo, tras haber dudado no poco, porque, como dice Juan, \textquote{hasta entonces no habían comprendido que según la Escritura Jesús debía resucitar de entre los muertos} (\textit{Jn} 20, 9).

Digamos la verdad: el hecho era asombroso para aquellos hombres que se encontraban ante cosas demasiado superiores a ellos. La misma dificultad, que muestran las tradiciones del acontecimiento, al dar una relación de ello plenamente coherente, confirma su carácter extraordinario y el impacto desconcertante que tuvo en el ánimo de los afortunados testigos. La referencia\textit{ \textquote{a la Escritura}} es la prueba de la oscura percepción que tuvieron al encontrarse ante un misterio sobre el que sólo la Revelación podía dar luz.

Sin embargo, he aquí otro dato que se debe considerar bien: si el \textquote{\textit{sepulcro vacío}} dejaba estupefactos a primera vista y podía incluso generar una cierta sospecha, el gradual conocimiento de este hecho inicial, como lo anotan los Evangelios, terminó llevando al descubrimiento de la verdad de la resurrección.

En efecto, se nos dice que las mujeres, y sucesivamente los Apóstoles, se encontraron \textit{ante un \textquote{signo} particular: el signo de la victoria sobre la muerte}. Si el sepulcro mismo cerrado por una pesada losa, testimoniaba la muerte, el sepulcro vacío y la piedra removida daban el primer anuncio de que allí había sido derrotada la muerte.

No puede dejar de impresionar la consideración del estado de ánimo de las tres mujeres, que dirigiéndose al sepulcro al alba se decían entre sí: \textquote{\textit{¿Quién nos retirará la piedra de la puerta del sepulcro?}} (\textit{Mc} 16, 3), y que después, cuando llegaron al sepulcro, con gran maravilla constataron que \textquote{la piedra estaba corrida aunque era muy grande} (\textit{Mc} 16, 4). Según el Evangelio de Marcos encontraron en el sepulcro a alguno que les dio el anuncio de la resurrección (cf. \textit{Mc} 16, 5): pero ellas tuvieron miedo y, a pesar de las afirmaciones del joven vestido de blanco, \textquote{salieron huyendo del sepulcro, pues un gran temblor y espanto se había apoderado de ellas} (\textit{Mc} 16, 8). ¿Cómo no comprenderlas? Y sin embargo la comparación con los textos paralelos de los demás Evangelistas permite afirmar que, aunque temerosas, las mujeres llevaron el anuncio de la resurrección, de la que el \textquote{sepulcro vacío} con la piedra corrida fue el primer signo.

Para las mujeres y para los Apóstoles el camino abierto por \textquote{el signo} se concluye \textit{mediante el encuentro con el Resucitado}: entonces la percepción aún tímida e incierta se convierte en \textit{convicción} y, más aún, en fe en Aquel que \textquote{ha resucitado verdaderamente}. Así sucedió a las mujeres que al ver a Jesús en su camino y escuchar su saludo, se arrojaron a sus pies y lo adoraron (cf. \textit{Mt} 28, 9). Así le pasó especialmente a María Magdalena, que al escuchar que Jesús le llamaba por su nombre, le dirigió antes que nada el apelativo habitual: \textit{Rabbuní}, ¡Maestro! (\textit{Jn} 20, 16) y cuando Él la iluminó sobre el misterio pascual corrió radiante a llevar el anuncio a los discípulos: \textquote{¡He visto al Señor!} (\textit{Jn} 20, 18). Lo mismo ocurrió a los discípulos reunidos en el Cenáculo que la tarde de aquel \textquote{primer día después del sábado}, cuando vieron finalmente entre ellos a Jesús, se sintieron felices por la nueva certeza que había entrado en su corazón: \textquote{Se alegraron al ver al Señor} (cf. \textit{Jn} 20, 19-20).

¡El contacto directo con Cristo desencadena la chispa que hace saltar la fe!
\end{body}

\begin{patercite}
Los relatos de la creación en el \textit{Libro del Génesis} nos introducen también en este misterioso ámbito, ayudándonos a conocer el proyecto de Dios sobre el hombre. Antes que nada afirman que Dios formó al hombre con el polvo de la tierra (cf. \textit{Gn} 2, 7). Esto significa que no somos Dios, no nos hemos hecho solos, somos tierra; pero significa también que venimos de la tierra buena, por obra del Creador bueno. A esto se suma otra realidad fundamental: \textit{todos} los seres humanos son polvo, más allá de las distinciones obradas por la cultura y la historia, más allá de toda diferencia social; somos una única humanidad plasmada con la única tierra de Dios. Hay, luego, un segundo elemento: el ser humano se origina porque Dios sopla el aliento de vida en el cuerpo modelado de la tierra (cf. \textit{Gn} 2, 7). El ser humano está hecho a imagen y semejanza de Dios (cf. \textit{Gn} 1, 26-27). Todos, entonces, llevamos en nosotros el aliento vital de Dios, y toda vida humana --nos dice la Biblia-- está bajo la especial protección de Dios. Esta es la razón más profunda de la inviolabilidad de la dignidad humana contra toda tentación de valorar a la persona según criterios utilitaristas y de poder. El ser a imagen y semejanza de Dios indica luego que el hombre no está cerrado en sí mismo, sino que tiene una referencia esencial en Dios.

\textbf{Benedicto XVI}, papa, \textit{Catequesis}, audiencia general, 6 de febrero de 2013, parr. 5. 
\end{patercite}

\newsection
\section{Homilías}

\subsection{San Juan Pablo II, papa}

\subsubsection{Homilía (1979): Noche de la gran espera}

\src{Basílica de San Pedro. 14 de abril de 1979.}

\begin{body}
\ltr[1. ]{L}{a} palabra \textquote{muerte} se pronuncia con un nudo en la garganta. Aunque la humanidad, durante tantas generaciones, se haya acostumbrado de algún modo a la realidad inevitable de la muerte, sin embargo resulta siempre desconcertante. La muerte de Cristo había penetrado profundamente en los corazones de sus más allegados, en la conciencia de toda Jerusalén. El silencio que surgió después de ella llenó la tarde del viernes y todo el día siguiente del sábado. En este día, según las prescripciones de los judíos, nadie se había trasladado al lugar de la sepultura. Las tres mujeres, de las que habla el \textbf{Evangelio} de hoy, recuerdan muy bien la pesada piedra con que habían cerrado la entrada del sepulcro. \textit{Esta piedra}, en la que pensaban y de la que hablarían al día siguiente yendo al sepulcro, simboliza también el peso \textit{que había aplastado sus corazones}. La piedra que había separado al Muerto de los vivos, la piedra límite de la vida, el peso de la muerte. Las mujeres, que al amanecer del día después del sábado van al sepulcro, no hablarán de la muerte, sino de la piedra. Al llegar al sitio, comprobarán que la piedra no cierra ya la entrada del sepulcro. Ha sido derribada. No encontrarán a Jesús en el sepulcro. ¡Lo han buscado en vano! \textquote{No está aquí; ha resucitado, según lo había dicho} (\textit{Mt} 28, 6). Deben volver a la ciudad y anunciar a los discípulos que Él ha resucitado y que lo verán en Galilea. Las mujeres no son capaces de pronunciar una palabra. La noticia de la muerte se pronuncia en voz baja. Las palabras de la resurrección eran para ellas, desde luego, difíciles de comprender. \textit{Difíciles de repetir}, tanto ha influido la realidad de la muerte en el pensamiento y en el corazón del hombre.

2. Desde aquella noche y más aún desde la mañana siguiente, los discípulos de Cristo han aprendido a pronunciar la palabra \textquote{resurrección}. Y ha venido a ser la palabra más importante en su lenguaje, la palabra central, la palabra fundamental. Todo toma nuevamente origen de ella. Todo se confirma y se construye de nuevo: \textquote{La piedra que desecharon los arquitectos es \textit{ahora la piedra angular}. Es el Señor quien lo ha hecho, ha sido un milagro patente. Este es el día en que actuó el Señor. ¡Sea nuestra alegría y nuestro gozo!} (\textit{Sal} 117 [118], 22-24). Precisamente por esto la vigilia pascual –el día siguiente al Viernes Santo– no es ya sólo el día en que se pronuncia en voz baja la palabra \textquote{muerte}, en el que se recuerdan los últimos momentos de la vida del Muerto: \textit{es el día de una gran espera}. Es la Vigilia Pascual: el día y la noche de la espera del día que hizo el Señor.

El contenido litúrgico de la Vigilia se expresa mediante las distintas horas del breviario, para concentrarse después con toda su riqueza en esta liturgia de la noche, que alcanza su cumbre, después del período de Cuaresma, en el primer \textquote{Alleluia}. ¡\textit{Alleluia}: es el grito que expresa la alegría pascual! La exclamación que resuena todavía en la mitad de la noche de la espera y lleva ya consigo la alegría de la mañana. Lleva consigo la certeza de la resurrección. Lo que, en un primer momento, no han tenido la valentía de pronunciar ante el sepulcro los labios de las mujeres, o la boca de los Apóstoles, ahora la Iglesia, gracias a su testimonio, lo expresa con su Aleluya. Este canto de alegría, cantado casi a media noche, nos anuncia el Día Grande. (En algunas lenguas eslavas, la Pascua se llama la \textquote{Noche Grande}, después de la Noche Grande, llega el Día Grande: \textquote{Día hecho por el Señor}).

3. Y he aquí que estamos para ir al encuentro de este Día Grande con el fuego pascual encendido; en este fuego hemos encendido el cirio –luz de Cristo– y junto a él hemos proclamado la gloria de su resurrección en el canto del \textit{Exultet}. A continuación, hemos penetrado, mediante una serie de lecturas, en el gran proceso de la creación, del mundo, del hombre, del Pueblo de Dios; hemos penetrado en la preparación del conjunto de lo creado en este Día Grande, en el día de la victoria del bien sobre el mal, de la Vida sobre la muerte. ¡No se puede captar el misterio de la resurrección sino volviendo a los orígenes y siguiendo, después, todo el desarrollo de la historia de la economía salvífica hasta ese momento! El momento en que las tres mujeres de Jerusalén, que se detuvieron en el umbral del sepulcro vacío, oyeron el mensaje de un joven vestido de blanco: \textquote{No os asustéis. Buscáis a Jesús Nazareno, el crucificado; ha resucitado, no está aquí} (\textit{Mc} 16, 5-6).

4. Ese gran momento no nos consiente permanecer fuera de nosotros mismos; nos obliga a entrar en nuestra propia humanidad. Cristo no sólo nos ha revelado la victoria de la vida sobre la muerte, sino que nos ha traído con su resurrección la nueva vida. Nos ha dado esta nueva vida. He aquí cómo se expresa San Pablo: \textquote{¿O ignoráis que cuantos hemos sido bautizados en Cristo Jesús fuimos bautizados para participar en su muerte? Con Él hemos sido sepultados por el bautismo para participar en su muerte, para que como Él resucitó de entre los muertos por la gloria del Padre, así también nosotros vivamos una nueva vida} (\textit{Rom} 6, 3-4).

Las palabras \textquote{hemos sido bautizados en su muerte} dicen mucho. La muerte es el agua en la que se reconquista la vida: el agua \textquote{que salta hasta la vida eterna} (\textit{Jn} 4, 14). ¡Es necesario \textquote{sumergirse} en este agua; en esta muerte, \textit{para surgir después de ella como hombre nuevo}, como nueva criatura, como ser nuevo, esto es, \textit{vivificado por la potencia de la resurrección de Cristo!}

Este es el misterio del agua que esta noche bendecimos, que hacemos penetrar con la \textquote{luz de Cristo}, que hacemos penetrar con la nueva vida: ¡es el símbolo de la potencia de la resurrección! Este agua, en el sacramento del bautismo, se convierte en el signo \textit{de la victoria} sobre Satanás, sobre el pecado; el signo de la victoria que Cristo ha traído mediante la cruz, mediante la muerte y \textit{que nos trae} después \textit{a cada uno}: \textquote{Nuestro hombre viejo ha sido crucificado para que fuera destruido el cuerpo del pecado y ya no sirvamos al pecado} (\textit{Rom} 6, 6).

5. Es pues la noche de la gran espera. Esperemos en la fe, esperemos con todo nuestro ser humano a Aquel que al despuntar el alba ha roto la tiranía de la muerte, y ha revelado la potencia divina de la Vida: Él es nuestra esperanza.
\end{body}

\begin{patercite}
Lo que se pide a Abrahán es que se fíe de esta Palabra. La fe entiende que la palabra, aparentemente efímera y pasajera, cuando es pronunciada por el Dios fiel, se convierte en lo más seguro e inquebrantable que pueda haber, en lo que hace posible que nuestro camino tenga continuidad en el tiempo. La fe acoge esta Palabra como roca firme, para construir sobre ella con sólido fundamento. Por eso, la Biblia, para hablar de la fe, usa la palabra hebrea \textit{’emûnah}, derivada del verbo \textit{’amán}, cuya raíz significa \textquote{sostener}. El término \textit{’emûnah} puede significar tanto la fidelidad de Dios como la fe del hombre. El hombre fiel recibe su fuerza confiándose en las manos de Dios. Jugando con las dos acepciones de la palabra --presentes también en los correspondientes términos griego (\textit{pistós}) y latino (\textit{fidelis})--, san Cirilo de Jerusalén ensalza la dignidad del cristiano, que recibe el mismo calificativo que Dios: ambos son llamados \textquote{fieles}. San Agustín lo explica así: \textquote{El hombre es fiel creyendo a Dios, que promete; Dios es fiel dando lo que promete al hombre}.
	
	\textbf{Francisco}, papa, \textit{Lumen Fidei}, n. 10.
\end{patercite}

\newpage 

\subsubsection{Homilía (1982): Tumba vacía}

\src{Basílica de San Pedro, 10 de abril de 1982.}

\begin{body}
\ltr[1. ]{E}{n} el centro del día, que acaba de terminar, hay una tumba. El sepulcro de Cristo. Este fue el día del Sábado Santo. Víspera de Pascua. En el centro del Viernes Santo está la Cruz de Cristo. En el centro del Sábado Santo la tumba de Cristo. Esta tumba tiene a las tres mujeres ante sus ojos: María de Magdala, María madre de Santiago y Salomé, cuando al amanecer del día siguiente, \textquote{el día después del sábado}, van al lugar del entierro de Cristo incluso antes de levantarse el sol. Su principal preocupación se expresa en estas palabras: \textquote{¿Quién nos removerá la piedra de la entrada del sepulcro?} (\textit{Mc} 16, 3). 

El sepulcro: el lugar donde está enterrado Cristo, aquel cuyo cuerpo quieren embalsamar para protegerlo prontamente de la acción destructiva de la muerte. Y he aquí, el sepulcro está vacío. Las mujeres ven que la piedra ha sido quitada y entran en el sepulcro\ldots Al amanecer del día después del sábado, cambia radicalmente el horizonte de los pensamientos y sentimientos de todos los que vieron la cruz de Cristo, su muerte y su entierro. De los que vieron el sepulcro con la piedra removida al frente. 

La tumba vacía se coloca en medio de la noche siguiente y del amanecer del día siguiente al sábado. María de Magdala, María madre de Santiago y Salomé se asustaron al principio: \textquote{\ldots estaban llenas de temblor y espanto} (\textit{Mc} 16, 8). Estaban llenas de temblor y espanto, a pesar de lo que habían escuchado de labios del joven que habían encontrado en el sepulcro, vestido con una túnica blanca. A pesar, o quizás a causa de esto. El joven les había dicho: \textquote{Ha resucitado, no está aquí\ldots irá delante de vosotros a Galilea} (\textit{Mc} 16, 6s). Sin embargo, no pudieron repetir esta noticia. \textquote{Y no decían nada a nadie, porque tenían miedo} (\textit{Mc} 16, 8).

Esta es la primera imagen que la liturgia de la Vigilia Pascual, en su parte final, traza ante nosotros. 

2. La segunda imagen proviene de San Pablo. A partir del día siguiente, el día después del sábado, los discípulos de Cristo se familiarizaron con esta nueva realidad: el sepulcro vacío. Comenzaron a llamarla por su nombre. Poco a poco también comprendieron que en la resurrección del Señor se cumplía todo lo que Él había hecho y lo que había enseñado. Por lo tanto el apóstol Pablo en la carta a los Romanos, hacia el año 57, es decir, 25 años después del acontecimiento de la Pascua, escribe: \textquote{\ldots fuimos bautizados en Cristo Jesús, fuimos bautizados en su muerte\ldots por lo tanto fuimos sepultados junto con él en la muerte, para que como Cristo resucitó de entre los muertos por la gloria del Padre, así también nosotros podamos caminar en una vida nueva} (\textit{Rom} 6, 4). 

Por tanto, para ellos: para la primera generación apostólica de los confesores de Cristo, y también para nosotros en el centro de la vigilia pascual está primero el \textquote{hombre viejo}, el hombre de pecado, que debe morir junto con Cristo, debe ser junto con él sepultado –para que el pecado muera en la muerte redentora de Cristo– y para que al amanecer del Domingo de Resurrección nazca el \textquote{hombre nuevo}. El hombre que vuelve a la vida por Cristo. 

He aquí la analogía apostólica de \textquote{la tumba vacía}. \textquote{La tumba vacía} significa no solo la resurrección de Cristo. Significa una nueva vida, una vida en la Gracia. Significa \textquote{el hombre nuevo}. 

Entonces, primero tenemos la Cruz en el centro del Viernes Santo. Y Pablo escribe: \textquote{Nuestro hombre viejo fue crucificado en Cristo para que\ldots ya no fuéramos esclavos del pecado. De hecho, el que ha muerto ya está libre de pecado} (\textit{Rom} 6, 6s). Posteriormente, en el centro del Sábado Santo, se coloca el sepulcro. Y Pablo escribe: \textquote{Hemos sido completamente unidos a Él en una muerte como la suya} (\textit{Rom} 6, 5). El Sábado Santo es la víspera del Domingo de Resurrección. Al amanecer del domingo, las mujeres encuentran la tumba vacía. El Apóstol escribe (y estas palabras son como un clamor rotundo de fe y esperanza): \textquote{Cristo resucitado de entre los muertos ya no muere; la muerte ya no tiene poder sobre él} (\textit{Rom} 6, 9). Así, también vosotros, consideraos \textquote{muertos al pecado, pero vivos para Dios en Cristo Jesús} (\textit{Rom} 6, 11). Esta es la segunda imagen de la liturgia de la Vigilia. 

3. Acojamos el silencio de las mujeres asustadas al ver el sepulcro vacío, al amanecer del día siguiente al sábado. Acojamos este grito del Apóstol de la carta a los Romanos. [Acojedlo especialmente vosotros, queridos hermanos y hermanas, que durante esta noche de Vigilia recibís de Cristo la nueva vida en el sacramento del Bautismo.] Acojámoslo cada uno de nosotros, que hemos recibido esta nueva vida. Que lo acojan todos aquellos que se han renovado mediante el sacramento de la Penitencia. Cristo se ha convertido en la piedra angular del nuevo edificio en todos nosotros. 

4. Entonces, mientras todo está aún velado por la noche de Pascua, elevemos nuestro corazón hacia la Vida Nueva: \textquote{Es el Señor quien lo ha hecho, ha sido un milagro patente} (\textit{Sal} 117 [118], 23). Y junto al salmista damos gracias: \textquote{Dad gracias al Señor porque es bueno, porque es eterna su misericordia. Diga la casa de Israel: eterna es su misericordia\ldots La diestra del Señor es poderosa, la diestra del Señor es excelsa} (\textit{Sal} 117 [118], 1-2. 16). 

Esta noche de la Vigilia proclama el cumplimiento del misterio pascual: al centro del Viernes Santo está la Cruz, al centro del Sábado Santo el sepulcro de Cristo y al amanecer de la noche de la Vigilia se revela el poder de la diestra del Señor. El sepulcro vacío da testimonio de la resurrección de Cristo: estaremos injertados en Él \textquote{\ldots también en la resurrección} (\textit{Rm} 6, 5). 

Vosotros, queridos neófitos, todos nosotros, queridos hermanos y hermanas, participando en esta Eucaristía, renovamos esta certeza de fe, confesada por los labios del salmista: \textquote{No he de morir, viviré, para contar las hazañas del Señor} (\textit{Sal} 117 [118], 17).
\end{body}


\subsubsection{Homilía (1985): Poder creador y poder salvador}

\src{6 de abril de 1985.}

\begin{body}
\ltr[1. «]{\textit{O}} {} \textit{vere beata nox!}» -- ¡Oh noche verdaderamente dichosa! Así canta la Iglesia durante la vigilia pascual, velando junto al sepulcro de Cristo. En esta tumba fue puesto su cuerpo torturado, bajado de la cruz con rapidez con \textquote{motivo de la fiesta de la Pascua}. Todavía era la Pascua del antiguo pacto. 

2. \textquote{\textit{O vere beata nox!}} – ¡Oh noche verdaderamente dichosa! Así canta la Iglesia durante esta vigilia, que precede a la Pascua de la nueva alianza. En toda la superficie de la tierra la Iglesia está reunida en vigilia para adorar \textit{el poder del Altísimo}: \textquote{La diestra del Señor es poderosa, la diestra del Señor es excelsa} (\textit{Sal} 118, 16). Es el mismo \textit{poder} que se reveló al principio \textit{en la creación} del mundo. Dios dijo: \textquote{¡Hágase!}; y así \textquote{creó los cielos y la tierra} (\textit{Gen} 1, 1). Es el mismo poder que se manifestó en la liberación de Israel de Egipto, el poder que condujo al Pueblo Elegido a través del Mar Rojo, salvándolo de las manos del Faraón. La Iglesia, reunida en vigilia junto al sepulcro de Cristo, medita sobre los acontecimientos en los que se manifestó el poder de Dios: - el poder creador, - el poder salvador. 

3. \textquote{\textit{O vere beata nox!}}. Es verdaderamente dichosa esta noche en la que brilla de nuevo la luz de Cristo y en la que la vida vencerá a la muerte. Por eso, desde el fondo del mismo salmo nos habla el que \textquote{estaba muerto}: \textquote{No he de morir, viviré, para contar las hazañas del Señor\ldots La diestra del Señor es poderosa, la diestra del Señor es excelsa}. 

4. Estamos reunidos aquí, en vigilia, \textit{para experimentar este poder divino a través de la fe y el amor}. Estamos aquí para recibir su revelación, al igual que \textit{aquellas tres mujeres} que, temprano en la mañana, cuando aún estaba oscuro, fueron las primeras en ir al sepulcro y lo encontraron vacío. También como \textit{los apóstoles}, quienes luego fueron corriendo al sepulcro. En cuanto a las mujeres, el Evangelio dice que estaban \textquote{inseguras}, \textquote{asustadas} (cf. \textit{Lc} 24, 3-4), que \textquote{tenían miedo} (\textit{Mc} 16, 5). Encontraron un misterio que \textit{sobrepasa al hombre y lo asusta: \textquote{mysterium tremendum}}. 

5. Del miedo pasamos al \textit{asombro} (\textquote{\textit{mysterium fascinosum}}), a la admiración adoradora. Y la Iglesia reunida junto al sepulcro de Cristo, en el corazón de esta noche bendita, ve incluso el pecado con una nueva luz, porque se atreve a cantar: \textquote{O felix culpa, quae talem ac tantum meruit habere \textit{Redemptorem}} – \textquote{¡Oh feliz culpa, que mereció tan grande Redentor!} . Verdaderamente lo que sucedió esa noche es \textquote{obra del Señor: un milagro patente} (\textit{Sal} 118, 23). 

6. \txtsmall{[Invitamos de manera especial a acoger esta revelación del poder divino –del poder creador y salvador–, a \textit{vosotros}, queridos hermanos y hermanas, que durante esta liturgia de la Vigilia Pascual \textit{recibiréis el Bautismo}. La Iglesia os recibe con gran alegría, a vosotros que queréis de modo sacramental sumergiros con Cristo \textit{en su muerte} para resucitar junto \textit{con él a una nueva vida}. El obispo de Roma os saluda cordialmente invitándoos a las fuentes de la salvación. Sois un grupo de veinticuatro personas, todos jóvenes, de una docena de naciones y de varios continentes. Sois una señal de cómo Cristo llama a sus discípulos de todos los pueblos y naciones de la tierra. Mostráis la universalidad del mensaje del Evangelio. A través de vosotros, nuestra vigilia pascual se \textit{convierte en un signo particularmente} elocuente.]}

El misterio pascual de nuestro Señor Jesucristo \textit{está siempre presente} en el sacramento de la Iglesia. El poder de la muerte y la resurrección no deja de actuar en las almas de los hombres. 

7. Así, por obra del poder divino mismo: - poder creador, - poder salvador, \textit{la Iglesia nace a la vida} en la resurrección del Señor crucificado: \textquote{La piedra desechada por los constructores se \textit{ha convertido en la piedra angular}} (\textit{Sal} 118, 22). De ella nacemos todos: como piedras vivas, penetradas por el aliento vivificante de esta noche de Pascua; del aliento de la resurrección de Cristo. \textquote{Estamos muertos al pecado, pero vivimos para Dios en Cristo Jesús} (cf. \textit{Rm} 6, 11; \textit{Col} 2, 13).
\end{body}

\begin{patercite}
	Un último aspecto de la historia de Abrahán es importante para comprender su fe. La Palabra de Dios, aunque lleva consigo novedad y sorpresa, no es en absoluto ajena a la propia experiencia del patriarca. Abrahán reconoce en esa voz que se le dirige una llamada profunda, inscrita desde siempre en su corazón. Dios asocia su promesa a aquel \textquote{lugar} en el que la existencia del hombre se manifiesta desde siempre prometedora: la paternidad, la generación de una nueva vida: \textquote{Sara te va a dar un hijo; lo llamarás Isaac} (\textit{Gn} 17,19). El Dios que pide a Abrahán que se fíe totalmente de él, se revela como la fuente de la que proviene toda vida. De esta forma, la fe se pone en relación con la paternidad de Dios, de la que procede la creación: el Dios que llama a Abrahán es el Dios creador, que \textquote{llama a la existencia lo que no existe} (\textit{Rm} 4,17), que \textquote{nos eligió antes de la fundación del mundo\ldots y nos ha destinado a ser sus hijos} (\textit{Ef} 1,4-5). Para Abrahán, la fe en Dios ilumina las raíces más profundas de su ser, le permite reconocer la fuente de bondad que hay en el origen de todas las cosas, y confirmar que su vida no procede de la nada o la casualidad, sino de una llamada y un amor personal. El Dios misterioso que lo ha llamado no es un Dios extraño, sino aquel que es origen de todo y que todo lo sostiene. La gran prueba de la fe de Abrahán, el sacrificio de su hijo Isaac, nos permite ver hasta qué punto este amor originario es capaz de garantizar la vida incluso después de la muerte. La Palabra que ha sido capaz de suscitar un hijo con su cuerpo \textquote{medio muerto} y \textquote{en el seno estéril} de Sara (cf. \textit{Rm} 4,19), será también capaz de garantizar la promesa de un futuro más allá de toda amenaza o peligro (cf. \textit{Hb} 11,19; \textit{Rm} 4,21).
	
	\textbf{Francisco}, papa, \textit{Lumen Fidei}, n. 11.
\end{patercite}

\newpage  

\subsubsection{Homilía (1988): Luz que vence las tinieblas}

\src{3 de abril de 1988.}

\begin{body}
1. \textquote{Lumen Christi!}

\ltr{E}{n} la oscuridad que se extiende por todo el espacio de esta [Basílica de San Pedro], las palabras [del diácono] resuenan tres veces como un anuncio profético de la vigilia pascual: \textquote{Lumen Christi!} Poco a poco se iluminó el espacio exterior para expresar lo que ha traído esta noche después del sábado, de cara al amanecer del día. Todos entramos en esta noche, todavía trastornados por los acontecimientos de ayer, por la muerte de Jesús de Nazaret y por su sepultura no lejos de la cruz del Calvario. Caminamos, como aquellos dos discípulos en el camino de Jerusalén a Emaús (cf. \textit{Lc} 24, 13ss). 

2. Y he aquí que la Iglesia se nos acerca –como el desconocido que se acerca a los discípulos, caminando con ellos hacia Emaús– y despliega ante nosotros, en una serie de lecturas, su inspirada \textquote{pedagogía}. Muestra el designio eterno de Dios que se desarrolla a lo largo de la historia del hombre, a partir de la creación, a través de la vocación de Abraham y, más tarde, del pueblo que descendió de él. Los patriarcas y los profetas hablan, los acontecimientos hablan, todos juntos conducen finalmente al acontecimiento de esta noche de Pascua: \textquote{Lumen Christi!} 

3. Esta es la luz que ilumina todo el pasado, revela el sentido profundo de todos los libros del Antiguo Testamento y de todas las lecturas de esta liturgia. La luz de Cristo caminaba delante del hombre desde el comienzo de su historia terrena. ¡Desde la creación, desde el \textquote{árbol del conocimiento del bien y del mal}, desde la tentación y el pecado\ldots caminaba esta luz delante de él! 

Su poder es tan grande que la Iglesia no duda en la liturgia de esta noche de vigilia para exclamar: \textquote{O felix culpa, quae talem ac tantum meruit habere Redemptorem}. ¡Oh feliz culpa! Oímos estas palabras en el anuncio pascual del \textquote{Exsultet}, cantado por el diácono. De hecho, esta noche de vigilia nos invita a la mayor alegría, a la alegría de la Pascua de Cristo: \textquote{Lumen Christi!}. Tal es el poder de esta luz, que es capaz de transformar la oscuridad, tanto exterior como interior, en día: \textquote{haec est Dies} – ¡El Día hecho por el Señor! Con el poder de la Pascua de Cristo, donde \textquote{abundó el pecado}, es decir, la muerte; la gracia, es decir, la vida, pueden sobreabundar más (cf. \textit{Rm} 5, 20). 

4. Entonces, antes de que las tres mujeres, de las que habla el Evangelio de esta vigilia pascual, encuentren la piedra removida en el lugar del sepulcro de Cristo, la Iglesia desciende con nosotros a las profundidades de esta muerte, que ha provocado tal sobreabundancia de vida. 

\textquote{O Mors, ero mors tua!} – \textquote{¡Oh Muerte, yo seré tu muerte!}. Siguiendo las palabras del Apóstol en la carta a los Romanos, descendemos a la historia del pecado humano hasta su primer comienzo. 

Con el pecado entró la muerte en el mundo (cf. \textit{Rm} 5, 12). Y por eso durante esta noche de vigilia somos bautizados en la muerte de Cristo. Junto con él somos sepultados en la muerte, para que podamos caminar en una vida nueva, como Cristo (cf. \textit{Rm} 6, 4). De hecho, Cristo ha resucitado. \textquote{Muerto al pecado\ldots vive para Dios} (cf. \textit{Rm} 6, 10). Nuestro hombre viejo, el hombre de pecado, debe \textquote{ser crucificado con Él}, con Cristo, para que nosotros, participando en su muerte, en su muerte redentora, seamos liberados del pecado. 

5. \txtsmall{[Queridos hermanos y hermanas, que durante esta vigilia pascual recibiréis el Bautismo que os sumerge en la muerte de Cristo, toda la Iglesia y el Pueblo de Dios que llenan esta venerada Basílica de San Pedro os saludan, ya que estáis a punto de recibir nueva vida en Cristo. En ustedes deseo dirigir mi respetuoso saludo a sus respectivos países de donde proceden: Corea, Alemania, Japón, India, Indonesia, Islas de Cabo Verde, Italia, Perú, Estados Unidos de América, Hungría y Vietnam. Viniendo de diferentes partes del mundo, reflejáis la universalidad de la Iglesia, el ámbito universal de la redención. Vuestro nacimiento a través del Bautismo a una nueva vida en Cristo es para todos nosotros una fuente particular de alegría pascual.]}

\textquote{Dad gracias al Señor porque es bueno; porque es eterna su misericordia} (\textit{Sal} 118 [117], 1). 

6. Todos llevaremos en la mano, [junto a vosotros,] un cirio de Pascua encendido. Es un testimonio de nuestro bautismo, de nuestra fe, esperanza y caridad. Es testigo de esta noche de vigilia, en la que la Iglesia no duda en cantar \textquote{O felix culpa, quae talem ac tantum meruit habere Redemptorem}. 

He aquí que avanzan las horas de esta vigilia nocturna. Pronto llegará el amanecer. A las tres mujeres, habiendo encontrado la tumba de Cristo vacía, y la piedra removida, se les dirá: \textquote{Ha resucitado, no está aquí\ldots id, anunciadlo a sus discípulos y a Pedro} (\textit{Mc} 16, 6-7). 

\textquote{Haec est Dies, quam fecit Dominus} – \textquote{Este es el Día que hizo el Señor}. 

Todos entraremos en este día de la Pascua de Cristo, y las velas encendidas en la noche de vigilia darán testimonio del final de nuestros días terrenales: 

\textquote{Lumen Christi}! 

¡Sí, Cristo es la luz! Amén.
\end{body}

\newpage 

\subsubsection{Homilía (1991): Cielo nuevo y tierra nueva}

\src{30 de marzo de 1991.}

\begin{body}
\textquote{\textit{En el principio, Dios creó los cielos y la tierra}. La tierra estaba informe y vacía y las tinieblas cubrían el abismo y el espíritu de Dios se cernía sobre las aguas}. \textquote{\textit{El espíritu de Dios se cernía sobre las aguas}} (\textit{Gen} 1, 1-2). 

\ltr[1. ]{E}{n} esta noche, durante la Vigilia Pascual, el Sagrado Triduo nos hace volver al origen de la obra divina de la creación. La oscuridad se extendió hasta la basílica antes de que el grito del salmo resonara en las lecturas a medida que se desarrollaba la liturgia: \textquote{\textit{Envías tu espíritu, son creados y renuevas la faz de la tierra}} (\textit{Sal} 104, 30). 

\textit{Envía tu Espíritu, Señor, para renovar la faz de la tierra}. Retornamos al principio de la creación y al mismo tiempo experimentamos profundamente la \textit{noche} que cayó sobre Jerusalén, \textit{después de que Cristo}, bajado de la Cruz, fuera sepultado en la tumba. La muerte del Dios-Hombre inició la nueva creación. Cristo acogió la muerte para renovar el mundo. \textit{En su muerte, el grito por el advenimiento del Espíritu que da vida} adquirió una fuerza definitiva y eficaz. 

2. \txtsmall{[Ayer, después del \textquote{Vía Crucis}, se proclamaron las palabras de la \textit{Carta a los Hebreos}: \textquote{¿Cuánto más la sangre de Cristo, que con un Espíritu eterno se ofreció sin tacha a Dios, purificará nuestra conciencia de las obras muertas para servir al Dios vivo?} (\textit{Heb} 9, 14).]} 

\textit{El sacrificio de la Cruz} es obra mesiánica de Cristo, es el cumplimiento total de la Redención. Cristo lo realiza no sólo \textquote{con} un Espíritu eterno, sino también en este sacrificio en el que \textquote{\textit{recibe} el Espíritu Santo para \textquote{dárselo}} a los Apóstoles, a la Iglesia, a la humanidad. Una vez resucitado, Jesús se presentará a los Apóstoles reunidos en el Cenáculo y, soplando sobre ellos les dirá: \textquote{Recibid el Espíritu Santo; a quienes perdonéis los pecados, les serán perdonados} (\textit{Jn} 20, 22-23). 

\textit{El Espíritu Santo renueva la faz de la tierra, recurriendo al poder de la Cruz de Cristo}, recurriendo a los recursos infinitos de la Redención del mundo. Renueva todo en el hombre, en su corazón y en su conciencia. \textit{Todo se renueva a través del Amor}, que, precisamente en esta noche de Pascua, se revela más poderoso que la muerte y el pecado, que es la muerte del alma. 

3. Por eso, \textit{la Vigilia Pascual}, desde los primeros tiempos del cristianismo, ha sido para los catecúmenos \textit{la gran hora del Bautismo}. 

\txtsmall{[Ahora también somos testigos y participantes de ello en esta Basílica de San Pedro, donde el Obispo de Roma saluda con alegría a \textit{nuestros nuevos hermanos y hermanas en la fe} que están a punto de recibir el Santo Bautismo. Vosotros venís de Japón, Corea, Vietnam, China, Tailandia, Indonesia, Estados Unidos, Chile, Inglaterra e Italia.]}

Esta es la fe en Cristo que \textquote{resucitado de entre los muertos no muere más; la muerte ya no tiene poder sobre Él} (\textit{Rom} 6, 9). 

4. He aquí que el grito del salmista se realiza en vosotros, hermanos y hermanas: \textquote{\textit{Envía, Señor, tu Espíritu para renovar la faz de la tierra}}\ldots el Espíritu que desde el principio participó íntimamente en la obra de la creación. El Espíritu que flotaba sobre las aguas \textit{regenera}, en el sacramento del Bautismo, al \textit{hombre \textquote{por el agua}} que ha recibido la fuerza del Espíritu vivificante. 

Regeneración \textquote{en el agua y en el Espíritu} (cf. \textit{Jn} 3, 5), primer sacramento de la Pascua de Cristo. Junto a este Sacramento, cuando aparece el Día \textquote{hecho por el Señor}, la Iglesia espera para todos vosotros \textit{el cumplimiento de la promesa del profeta Ezequiel}: que el Señor os dé un corazón nuevo y os dé un espíritu nuevo\ldots que os haga vivir según los preceptos de Dios y os haga observar y poner en práctica sus leyes. Que seáis \textit{el pueblo de Dios y el Señor sea vuestro Dios} (cf. \textit{Ez} 36, 26-28). 

Finalmente, que podáis vivir eternamente en la tierra de los vivos (cf. \textit{Ap} 21, 1). La vigilia de Pascua es un anticipo de esta tierra y de esta morada. En efecto, es el comienzo del cielo nuevo y de la tierra nueva donde Dios será \textquote{todo en todos} (cf. \textit{1 Co 15, 28} ).
\end{body}

\begin{patercite}
En el libro del Éxodo, la historia del pueblo de Israel sigue la estela de la fe de Abrahán. La fe nace de nuevo de un don originario: Israel se abre a la intervención de Dios, que quiere librarlo de su miseria. La fe es la llamada a un largo camino para adorar al Señor en el Sinaí y heredar la tierra prometida. El amor divino se describe con los rasgos de un padre que lleva de la mano a su hijo por el camino (cf. \textit{Dt} 1,31). La confesión de fe de Israel se formula como narración de los beneficios de Dios, de su intervención para liberar y guiar al pueblo (cf. \textit{Dt} 26,5-11), narración que el pueblo transmite de generación en generación. Para Israel, la luz de Dios brilla a través de la memoria de las obras realizadas por el Señor, conmemoradas y confesadas en el culto, transmitidas de padres a hijos. Aprendemos así que la luz de la fe está vinculada al relato concreto de la vida, al recuerdo agradecido de los beneficios de Dios y al cumplimiento progresivo de sus promesas. La arquitectura gótica lo ha expresado muy bien: en las grandes catedrales, la luz llega del cielo a través de las vidrieras en las que está representada la historia sagrada. La luz de Dios nos llega a través de la narración de su revelación y, de este modo, puede iluminar nuestro camino en el tiempo, recordando los beneficios divinos, mostrando cómo se cumplen sus promesas.
	
	\textbf{Francisco}, papa, \textit{Lumen Fidei}, n. 12.
\end{patercite}

\newpage 

\subsubsection{Homilía (1994): Alegría de la Pascua}

\src{Basílica de San Pedro, 2 de abril de 1994.}

\begin{body}
1. \textquote{\textit{¡No tengáis miedo!}} (\textit{Mc} 16, 6). 

\ltr{M}{aría} Magdalena, María madre de Santiago y Salomé escuchan estas palabras a la entrada del sepulcro en el que fue depositado el cuerpo de Jesús. Se dan cuenta de que la piedra del sepulcro había sido removida y que el sepulcro estaba vacío. Son invadidas por el miedo y el asombro. Asombro que crece al escuchar las palabras desde el fondo del sepulcro: \textquote{¿Buscáis a Jesús Nazareno, el Crucificado? No está aquí, ha resucitado. Aquí está el lugar donde lo habían puesto. Ahora \textit{id, decid a sus discípulos y a Pedro} que Él va delante de vosotros a Galilea. Allí lo veréis, como os dijo} (\textit{Mc} 16, 6-7). Las mujeres se sorprenden y huyen de la tumba, temerosas de contarle a alguien lo que han visto. 

2. Este es precisamente el momento del misterio pascual, al que nos acercamos participando en la solemne vigilia de la noche de Pascua. El acontecimiento, descrito por el evangelista Marcos, es simple y, al mismo tiempo, impactante.  Por eso \textit{la liturgia de la vigilia pascual se refiere a las fuerzas de la naturaleza}. En esta noche debemos llamarlas de nuevo, porque reaccionaron precisamente en aquel momento. La tierra se movió y tembló cuando Cristo dejó la tumba. Un terremoto sacudió la roca que obstruía el sepulcro (cf. \textit{Mt} 28, 2). 

En esta noche la liturgia se vuelve \textit{fuego}, que posee un poder misterioso, un poder bendito, pero también un poder que destruye. El fuego consume y devora lo que encuentra a su paso, pero también puede ser una fuerza beneficiosa para los hombres. De hecho, las extremidades del cuerpo humano necesitan fuego para calentarse. También el fuego ilumina, ahuyentando las tinieblas y en esta noche la Iglesia lo enciende para extraer de él la luz que, más tarde, acompaña a la asamblea litúrgica en el templo con el canto: \textquote{Lumen Christi}. La luz de la llama se convierte en símbolo de la Resurrección. La liturgia de esta noche reserva el mayor espacio para el\textit{ poder del agua}. El agua también puede ser un signo de muerte. Según San Pablo es un símbolo de la muerte de Cristo (cf. \textit{Rm} 6, 3-4) y para pasar por esta muerte es necesario sumergirse en el agua. Inmersión en la muerte de Cristo que sirve no solo para ser lavado, sino, más aún, para ser vivificado. El agua que brota de la fuente es un alivio para el cuerpo cansado, restaura su fuerza; por eso el \textit{agua se ha convertido en el signo sacramental del renacimiento a través del bautismo}. Con este sacramento la Iglesia participa hoy de la Resurrección de Cristo. 

\txtsmall{[A través del bautismo vosotros, hermanos y hermanas, que recibiréis este sacramento esta noche, participáis de la Resurrección de Cristo. \textit{ El obispo de Roma os saluda cordialmente} mientras os preparáis para entrar en la nueva vida. Saluda a las naciones de donde venís: Corea, Filipinas, Japón, Guatemala, Hong Kong, Italia, Perú, Portugal, Eslovaquia, España y Suiza.]}

3. La nueva vida es siempre fuente de alegría. Sentimos la alegría de la Iglesia en las palabras cantadas por el diácono hace un momento. La primera palabra del anuncio de Pascua es \textquote{\textit{Exsultet}}: una llamada a la alegría. El gozo de esta noche es mayor que el temor de las mujeres de Jerusalén: es el gozo de la victoria sobre la muerte y el pecado. La Iglesia no duda en cantar: \textquote{Feliz culpa}; feliz porque ha encontrado al Redentor en esta noche; porque en su muerte la ha vencido. Cristo ha resucitado dando vida a todos los descendientes de Adán. 

4. Entonces la Iglesia, ya ahora, durante esta admirable vigilia pascual, nos invita a todos a la alegría. Alegrémonos porque en Cristo la vida es más fuerte que la muerte y la salvación es más fuerte que el pecado.

\textquote{Annuntio vobis gaudium magnum, quod est - ¡Aleluya!}. 

¡Sed testigos en el mundo de hoy de la alegría de la Pascua!
\end{body}

\begin{patercite}
	La Iglesia, como toda familia, transmite a sus hijos el contenido de su memoria. ¿Cómo hacerlo de manera que nada se pierda y, más bien, todo se profundice cada vez más en el patrimonio de la fe? Mediante la tradición apostólica, conservada en la Iglesia con la asistencia del Espíritu Santo, tenemos un contacto vivo con la memoria fundante. Como afirma el Concilio ecuménico Vaticano II, \textquote{lo que los Apóstoles transmitieron comprende todo lo necesario para una vida santa y para una fe creciente del Pueblo de Dios; así la Iglesia con su enseñanza, su vida, su culto, conserva y transmite a todas las edades lo que es y lo que cree} (DV 8). 
	
	En efecto, la fe necesita un ámbito en el que se pueda testimoniar y comunicar, un ámbito adecuado y proporcionado a lo que se comunica. Para transmitir un contenido meramente doctrinal, una idea, quizás sería suficiente un libro, o la reproducción de un mensaje oral. Pero lo que se comunica en la Iglesia, lo que se transmite en su Tradición viva, es la luz nueva que nace del encuentro con el Dios vivo, una luz que toca la persona en su centro, en el corazón, implicando su mente, su voluntad y su afectividad, abriéndola a relaciones vivas en la comunión con Dios y con los otros. Para transmitir esta riqueza hay un medio particular, que pone en juego a toda la persona, cuerpo, espíritu, interioridad y relaciones. Este medio son los sacramentos, celebrados en la liturgia de la Iglesia. En ellos se comunica una memoria encarnada, ligada a los tiempos y lugares de la vida, asociada a todos los sentidos; implican a la persona, como miembro de un sujeto vivo, de un tejido de relaciones comunitarias. Por eso, si bien, por una parte, los sacramentos son sacramentos de la fe (cf. SC 59), también se debe decir que la fe tiene una estructura sacramental. El despertar de la fe pasa por el despertar de un nuevo sentido sacramental de la vida del hombre y de la existencia cristiana, en el que lo visible y material está abierto al misterio de lo eterno.
	
	\textbf{Francisco}, papa, \textit{Lumen Fidei}, n. 40.
\end{patercite}

\newpage 

\subsubsection{Homilía (1997): Fuego y agua}

\src{29 de marzo de 1997.}

\begin{body}
1. ¡Que exista la luz! (\textit{Gn} 1,3)

\ltr{D}{urante} la Vigilia pascual, la liturgia proclama estas palabras del \textbf{Libro del Génesis}, las cuales son un elocuente motivo central de esta admirable celebración. Al empezar se bendice el \textquote{fuego nuevo}, y con él se enciende el cirio pascual, que es llevado en procesión hacia el altar. El cirio entra y avanza primero en la oscuridad, hasta el momento en que, después de cantar el tercer \textquote{\textit{Lumen Christi}}, se ilumina toda la Basílica.

De este modo están unidos entre sí \textit{los elementos de las tinieblas y de la luz, de la muerte y de la vida}. Con este fondo resuena la narración bíblica de la creación. Dios dice: \textquote{Que exista la luz}. \textit{Se trata, en cierto modo, del primer paso hacia la vid}a. En esta noche debe realizarse el singular paso de la muerte a la vida, y el rito de la luz, acompañado por las palabras del Génesis, ofrece el primer anuncio.

2. En el Prólogo de su Evangelio, san Juan dice que el Verbo se hizo carne: \textquote{En la Palabra había vida, y\textit{ la vida era la luz de los hombres}} (\textit{Jn} 1, 4). Esta noche santa se convierte pues en una extraordinaria manifestación de aquella vida que es la luz de los hombres. En esta manifestación participa toda la Iglesia y, de modo especial, los \textit{catecúmenos}, que durante esta Vigilia reciben el Bautismo.

\txtsmall{[La Basílica de san Pedro en esta solemne celebración os acoge a vosotros, \textit{amadísimos hermanos y hermanas, que dentro de poco seréis bautizados en Cristo nuestra Pascua}. Dos de vosotros provienen de Albania y dos del Zaire, Países que están viviendo horas dramáticas de su historia. ¡Que el Señor se digne escuchar el grito de los pobres y guiarlos en el camino hacia la paz y la libertad! Otros proceden de Benin, Cabo Verde, China y Taiwán. Ruego por cada uno de vosotros y de vosotras que, en esta asamblea representáis las primicias de la nueva humanidad redimida por Cristo, para que seáis siempre fieles testigos de su Evangelio.]}

Las lecturas litúrgicas de la Vigilia pascual unen entre sí \textit{los dos elementos del fuego y del agua}. El elemento fuego, que da la luz, y el elemento agua, que es la materia del sacramento del renacer, es decir, del santo Bautismo. \textquote{El que no nazca de agua y de Espíritu, no puede entrar en el Reino de Dios} (\textit{Jn} 3, 5). El paso de los Israelitas a través del Mar Rojo, es decir, la liberación de la esclavitud de Egipto, es figura y casi anticipación del Bautismo que libera de la esclavitud del pecado.

3. Los múltiples motivos que en esta liturgia de la Vigilia de Pascua encuentran su expresión en las Lecturas bíblicas, convergen y se interrelacionan así en una imagen unitaria. Del modo más completo es el apóstol Pablo quien presenta esta verdad en la \textbf{Carta a los Romanos}, proclamada hace poco: \textquote{Los que por el bautismo nos incorporamos a Cristo, fuimos incorporados a su muerte. Por el bautismo fuimos sepultados con él en la muerte, para que, así como Cristo fue despertado de entre los muertos por la gloria del Padre, así también nosotros andemos en una vida nueva} (\textit{Rm} 6, 3-4).

Estas palabras nos llevan \textit{al centro mismo de la verdad cristiana}. La muerte de Cristo, la muerte redentora, es el comienzo del paso a la vida, manifestado en la resurrección. \textquote{Si hemos muerto con Cristo –prosigue san Pablo–, creemos que también viviremos con él, pues sabemos que Cristo, una vez resucitado de entre los muertos, ya no muere más; la muerte ya no tiene dominio sobre él} (\textit{Rm} 6, 8-9).

4. Al llevar en las manos la antorcha de la Palabra de Dios, la Iglesia que celebra la Vigilia pascual se detiene como ante un último umbral. Se detiene en gran espera, durante toda esta noche. Junto al sepulcro esperamos el acontecimiento sucedido hace [dos mil años]. Primeros testigos de este suceso extraordinario fueron las mujeres de Jerusalén. Ellas llegaron al lugar donde Jesús había sido depositado el Viernes Santo y encontraron la tumba vacía. Una voz les sorprendió: \textquote{¿Buscáis a Jesús el Nazareno, el crucificado? No está aquí. Ha resucitado. Mirad el sitio donde lo pusieron. Ahora id a decir a sus discípulos y a Pedro: Él va por delante de vosotros a Galilea. Allí lo veréis, como os dijo} (\textit{Mc} 16, 6-7).

Nadie vio con sus propios ojos la resurrección de Cristo. Las mujeres, llegadas a la tumba, fueron las primeras en constatar el acontecimiento ya sucedido.

La Iglesia, congregada por la Vigilia pascual, escucha nuevamente, en silenciosa espera, este testimonio y manifiesta después su gran alegría. [La hemos escuchado anunciar hace poco por el diácono. \textquote{\textit{Annuntio vobis gaudium magnum\ldots}} -- \textquote{\textit{Os anuncio una gran alegría, ¡Aleluya!}}.]

Acojamos con corazón abierto este anuncio y participemos juntos en la gran alegría de la Iglesia.

¡Cristo ha resucitado verdaderamente! ¡Aleluya!
\end{body}

\newpage 

\subsubsection{Homilía (2000): La propia historia de salvación}

\src{22 de abril del 2000.}

\begin{body}
1. \textquote{\textit{Tenéis guardias. Id, aseguradlo como sabéis}} (\textit{Mt} 27, 65).

\ltr{L}{a} tumba de Jesús fue cerrada y sellada. Según la petición de los sumos sacerdotes y los fariseos, se pusieron soldados de guardia para que nadie pudiera robarlo (\textit{Mt} 27, 62-64). Este es el acontecimiento del que parte la liturgia de la Vigilia Pascual.

Vigilaban junto al sepulcro aquellos que habían querido la muerte de Cristo, considerándolo un \textquote{impostor} (\textit{Mt} 27, 63). Su deseo era que Él y su mensaje fueran enterrados para siempre.

Velan, no muy lejos de allí, María y, con ella, los Apóstoles y algunas mujeres. Tenían aún impresa en el corazón la imagen perturbadora de hechos que acaban de ocurrir.

2. Vela la Iglesia, esta noche, en todos los rincones de la tierra, y revive las etapas fundamentales de la historia de la salvación. La solemne liturgia que estamos celebrando es una expresión de este \textquote{vigilar} que, en cierto modo, recuerda el mismo de Dios, al que se refiere el \textbf{Libro del Éxodo}: \textquote{Noche de guardia fue ésta para Yahveh, para sacarlos de la tierra de Egipto. Esta misma noche será la noche de guardia en honor de Yahveh\ldots, por todas sus generaciones} (\textit{Ex} 12, 42).

En su amor providente y fiel, que supera el tiempo y el espacio, Dios vela sobre el mundo. Canta el salmista: \textquote{Yahveh es tu guardián, tu sombra, Yahveh, a tu diestra. De día el sol no te hará daño, ni la luna de noche. Te guarda Yahveh de todo mal, él guarda tu alma; \ldots desde ahora y por siempre} (\textit{Sal} 120, 4-5. 8).

También el pasaje que estamos viviendo entre el segundo y el tercer milenio está guardado en el misterio del Padre. Él \textquote{obra siempre} (\textit{Jn} 5, 7) por la salvación del mundo y, mediante el Hijo hecho hombre, guía a su pueblo de la esclavitud a la libertad. Toda la \textquote{obra} [del Gran Jubileo del año 2000] está, por decirlo así, inscrita en esta noche de Vigilia, que lleva a cumplimiento aquella del Nacimiento del Señor. Belén y el Calvario remiten al mismo misterio de amor de Dios, que tanto amó al mundo \textquote{que dio a su Hijo único, para que todo el que crea en él no perezca, sino que tenga vida eterna} (\textit{Jn} 3, 16).

3. En esta Noche, la Iglesia, en su velar, se centra sobre los textos de la Escritura, que trazan el designio divino de salvación desde el Génesis al Evangelio y que, gracias también a los ritos del agua y del fuego, confieren a esta singular celebración una dimensión cósmica. Todo el universo creado está llamado a velar en esta noche junto al sepulcro de Cristo. Pasa ante nuestros ojos la historia de la salvación, desde la creación a la redención, desde el éxodo a la Alianza en el Sinaí, de la antigua a la nueva y eterna Alianza. En esta noche santa se cumple el proyecto eterno de Dios que arrolla la historia del hombre y del cosmos.

4. En la vigilia pascual, madre de todas las vigilias, cada hombre puede reconocer también la propia historia de salvación, que tiene su punto fundamental en el renacer en Cristo mediante el Bautismo.

\txtsmall{[Esta es, de manera muy especial, vuestra experiencia, queridos Hermanos y Hermanas que dentro de poco recibiréis los sacramentos de la iniciación cristiana: el Bautismo, la Confirmación y la Eucaristía. Venís de diversos Países del mundo: Japón, China, Camerún, Albania e Italia. La variedad de vuestras naciones de origen pone de relieve la universalidad de la salvación traída por Cristo. Dentro de poco, queridos, seréis insertos íntimamente en el misterio de amor de Dios, Padre, Hijo y Espíritu Santo. Que vuestra existencia se haga un canto de alabanza a la Santísima Trinidad y un testimonio de amor que no conozca fronteras.]}

5. \textquote{\textit{Ecce lignum Crucis, in quo salus mundi pependit: venite adoremus}!} Esto ha cantado ayer la Iglesia, mostrando el árbol la Cruz, \textquote{donde estuvo clavada la salvación del mundo}. \textquote{Fue crucificado, muerto y sepultado}, recitamos en el Credo.

El sepulcro. El lugar donde lo habían puesto (cf. \textit{Mc} 16, 6). Allí está espiritualmente presente toda la Comunidad eclesial de cada rincón de la tierra. Estamos también nosotros con las tres mujeres que se acercan al sepulcro, antes del alba, para ungir el cuerpo sin vida de Jesús (cf. \textit{Mc} 16, 1). Su diligencia es nuestra diligencia. Con ellas descubrimos que la piedra sepulcral ha sido retirada y el cuerpo ya no está allí. \textquote{No está aquí}, anuncia el Ángel, mostrando el sepulcro vacío y las vendas por tierra. La muerte ya no tiene poder sobre Él (cf. \textit{Rm} 6, 9).

¡Cristo ha resucitado! Anuncia al final de esta noche de Pascua la Iglesia, que ayer había proclamado la muerte de Cristo en la Cruz. Es un anuncio de verdad y de vida.

\textquote{\textit{Surrexit Dominus de sepulcro, qui pro nobis pependit in ligno. Alleluia}!}

Ha resucitado del sepulcro el Señor, que por nosotros fue colgado a la cruz.

Sí, Cristo ha resucitado verdaderamente y nosotros somos testigos de ello.

Lo gritamos al mundo, para que la alegría que nos embarga llegue a tantos otros corazones, encendiendo en ellos la luz de la esperanza que no defrauda.

¡Cristo ha resucitado, aleluya!
\end{body}

\newpage 

\subsubsection{Homilía (2003): Todo vuelve a empezar}

\src{19 de abril de 2003.}

\begin{body}
1. \textquote{\textit{No os asustéis. ¿Buscáis a Jesús el Nazareno, el crucificado? No está aquí. Ha resucitado}} (\textit{Mc} 16, 6). 

\ltr{A}{l} alba del primer día después del sábado, como narra el Evangelio, algunas mujeres van al sepulcro para embalsamar el cuerpo de Jesús que, crucificado el viernes, rápidamente había sido envuelto en una sábana y depositado en el sepulcro. Lo buscan, pero no lo encuentran: \textit{ya no está donde había sido sepultado}. De Él \textit{sólo quedan las señales de la sepultura}: la tumba vacía, las vendas, la sábana. Las mujeres, sin embargo, quedan turbadas a la vista de un \textquote{\textit{joven vestido con una túnica blanca}}, que les anuncia: \textquote{\textit{No está aquí. Ha resucitado}}. Esta desconcertante noticia, destinada a cambiar el rumbo de la historia, desde entonces sigue resonando de generación en generación: anuncio antiguo y siempre nuevo. Ha resonado una vez más en esta Vigilia pascual, madre de todas las vigilias, y se está difundiendo en estas horas por toda la tierra.

2. ¡\textit{Oh sublime misterio de esta Noche Santa}! Noche en la cual revivimos ¡\textit{el extraordinario acontecimiento de la Resurrección}! Si Cristo hubiera quedado prisionero del sepulcro, la humanidad y toda la creación, en cierto modo, habrían perdido su sentido. Pero Tú, Cristo, ¡has resucitado verdaderamente! Entonces\textit{ se cumplen las Escrituras} que hace poco hemos escuchado de nuevo en la liturgia de la Palabra, recorriendo las etapas de todo el designio salvífico. Al comienzo de la creación \textquote{\textit{Vio Dios todo lo que había hecho: y era muy bueno}} (\textit{Gn} 1, 31). A Abrahán había prometido: \textquote{\textit{Todos los pueblos del mundo se bendecirán con tu descendencia}} (\textit{Gn} 22, 18). Se ha repetido uno de los cantos más antiguos de la tradición hebrea, que expresa el significado del antiguo éxodo, cuando \textquote{\textit{el Señor salvó a Israel de las manos de Egipto}} (\textit{Ex} 14, 30). Siguen cumpliéndose en nuestros días las promesas de los Profetas: \textquote{\textit{Os infundiré mi espíritu, y haré que caminéis\ldots}} (\textit{Ez} 36, 27).

3. En esta noche de \textit{Resurrección} todo vuelve a empezar desde el \textquote{principio}; \textit{la creación} recupera su auténtico significado en el plan de la salvación. Es como un \textit{nuevo comienzo} de la historia y del cosmos, porque \textquote{\textit{Cristo ha resucitado, primicia de todos los que han muerto}} (\textit{1 Co} 15, 20). Él, \textquote{\textit{el último Adán}}, se ha convertido en \textquote{\textit{un espíritu que da vida}} (\textit{1 Co} 15, 45). El mismo pecado de nuestros primeros padres es cantado en el \textit{Pregón pascual} como \textquote{\textit{felix culpa}}, \textquote{¡feliz culpa que mereció tal Redentor!}. Donde abundó el pecado, ahora sobreabundó la Gracia y \textquote{\textit{la piedra que desecharon los arquitectos es ahora la piedra angular}} (\textit{Salmo responsorial)} de un edificio espiritual indestructible. En esta Noche Santa ha nacido el nuevo pueblo con el cual \textit{Dios ha sellado una alianza eterna} con la sangre del Verbo encarnado, crucificado y resucitado.

4. Se entra a formar parte del pueblo de los redimidos mediante el Bautismo. \textquote{\textit{Por el bautismo} –nos ha recordado el apóstol Pablo en su Carta a los Romanos– \textit{fuimos sepultados con Él en la muerte, para que, así como Cristo fue despertado de entre los muertos por la gloria del Padre, así también nosotros andemos en una vida nueva}} (\textit{Rm} 6, 4).

\txtsmall{[Esta exhortación va dirigida especialmente a vosotros, queridos \textit{catecúmenos}, a quienes dentro de poco la Madre Iglesia comunicará el gran don de la vida divina. De diversas Naciones la divina Providencia os ha traído aquí, junto a la tumba de San Pedro, para recibir los Sacramentos de la \textit{iniciación cristiana}: el Bautismo, la Confirmación y la Eucaristía. Entráis así en la Casa del Señor, sois consagrados con el óleo de la alegría y podéis alimentaros con el Pan del cielo. Sostenidos por la fuerza del Espíritu Santo, \textit{perseverad en vuestra fidelidad a Cristo} y proclamad con valentía su Evangelio.]}

5. Queridos hermanos y hermanas aquí presentes. También nosotros, dentro de unos instantes, [nos uniremos a los catecúmenos para] renovar las promesas de nuestro Bautismo. Volveremos a renunciar a Satanás y a todas sus obras para seguir firmemente a Dios y sus planes de salvación. Expresaremos así \textit{un compromiso más fuerte de vida evangélica}.

Que María, testigo gozosa del acontecimiento de la Resurrección, ayude a todos a caminar \textquote{\textit{en una vida nueva}}; que haga a cada uno consciente de que, estando nuestro hombre viejo crucificado con Cristo, debemos considerarnos y comportarnos como hombres nuevos, personas que \textquote{viven para Dios, en Jesucristo} (cf. \textit{Rm} 6, 4. 11).

Amén. ¡Aleluya!
\end{body}


\newsection
\subsection{Benedicto XVI, papa}

\subsubsection{Homilía (2006): ¿La Resurrección te ha alcanzado?}

\src{Basílica Vaticana. 15 de abril de 2006.}

\begin{body}
\textquote{\textit{¿Buscáis a Jesús el Nazareno, el crucificado? No está aquí, ha resucitado}} (\textit{Mc} 16, 6). 

\ltr{A}{sí} dijo el mensajero de Dios, vestido de blanco, a las mujeres que buscaban el cuerpo de Jesús en el sepulcro. Y lo mismo nos dice también a nosotros el evangelista en esta noche santa: Jesús no es un personaje del pasado. Él vive y, como ser viviente, camina delante de nosotros; nos llama a seguirlo a Él, el viviente, y a encontrar así también nosotros el camino de la vida.

\textquote{\textit{Ha resucitado\ldots, no está aquí}}. Cuando Jesús habló por primera vez a los discípulos sobre la cruz y la resurrección, estos, mientras bajaban del monte de la Transfiguración, se preguntaban qué querría decir eso de \textquote{resucitar de entre los muertos} (\textit{Mc} 9, 10). En Pascua nos alegramos porque Cristo no ha quedado en el sepulcro, su cuerpo no ha conocido la corrupción; pertenece al mundo de los vivos, no al de los muertos; nos alegramos porque Él es –como proclamamos en el rito del cirio pascual– Alfa y al mismo tiempo Omega, y existe por tanto, no sólo ayer, sino también hoy y por la eternidad (cf. \textit{Hb} 13, 8). Pero, en cierto modo, vemos la resurrección tan fuera de nuestro horizonte, tan extraña a todas nuestras experiencias, que, entrando en nosotros mismos, continuamos con la discusión de los discípulos: ¿En qué consiste propiamente eso de \textquote{resucitar}? ¿Qué significa para nosotros? ¿Y para el mundo y la historia en su conjunto? Un teólogo alemán dijo una vez con ironía que el milagro de un cadáver reanimado –si es que eso hubiera ocurrido verdaderamente, algo en lo que no creía– sería a fin de cuentas irrelevante para nosotros porque, justamente, no nos concierne. En efecto, el que solamente una vez alguien haya sido reanimado, y nada más, ¿de qué modo debería afectarnos? Pero la resurrección de Cristo es precisamente algo más, una cosa distinta. Es –si podemos usar por una vez el lenguaje de la teoría de la evolución– la mayor \textquote{mutación}, el salto más decisivo en absoluto hacia una dimensión totalmente nueva, que se haya producido jamás en la larga historia de la vida y de sus desarrollos: un salto de un orden completamente nuevo, que nos afecta y que atañe a toda la historia.

Por tanto, la discusión comenzada con los discípulos comprendería las siguientes preguntas: ¿Qué es lo que sucedió allí? ¿Qué significa eso para nosotros, para el mundo en su conjunto y para mí personalmente? Ante todo: ¿Qué sucedió? Jesús ya no está en el sepulcro. Está en una vida nueva del todo. Pero, ¿cómo pudo ocurrir eso? ¿Qué fuerzas han intervenido? Es decisivo que este hombre Jesús no estuviera solo, no fuera un Yo cerrado en sí mismo. Él era uno con el Dios vivo, unido talmente a Él que formaba con Él una sola persona. Se encontraba, por así decir, en un mismo abrazo con Aquél que es la vida misma, un abrazo no solamente emotivo, sino que abarcaba y penetraba su ser. Su propia vida no era solamente suya, era una comunión existencial con Dios y un estar insertado en Dios, y por eso no se le podía quitar realmente. Él pudo dejarse matar por amor, pero justamente así destruyó el carácter definitivo de la muerte, porque en Él estaba presente el carácter definitivo de la vida. Él era una cosa sola con la vida indestructible, de manera que ésta brotó de nuevo a través de la muerte. Expresemos una vez más lo mismo desde otro punto de vista.

Su muerte fue un acto de amor. En la última Cena, Él anticipó la muerte y la transformó en el don de sí mismo. Su comunión existencial con Dios era concretamente una comunión existencial con el amor de Dios, y este amor es la verdadera potencia contra la muerte, es más fuerte que la muerte. La resurrección fue como un estallido de luz, una explosión del amor que desató el vínculo hasta entonces indisoluble del \textquote{morir y devenir}. Inauguró una nueva dimensión del ser, de la vida, en la que también ha sido integrada la materia, de manera transformada, y a través de la cual surge un mundo nuevo.

Está claro que este acontecimiento no es un milagro cualquiera del pasado, cuya realización podría ser en el fondo indiferente para nosotros. Es un salto cualitativo en la historia de la \textquote{evolución} y de la vida en general hacia una nueva vida futura, hacia un mundo nuevo que, partiendo de Cristo, entra ya continuamente en este mundo nuestro, lo transforma y lo atrae hacia sí. Pero, ¿cómo ocurre esto? ¿Cómo puede llegar efectivamente este acontecimiento hasta mí y atraer mi vida hacia Él y hacia lo alto? La respuesta, en un primer momento quizás sorprendente pero completamente real, es la siguiente: dicho acontecimiento me llega mediante la fe y el bautismo. Por eso el Bautismo es parte de la Vigilia pascual, como se subraya también en esta celebración [con la administración de los sacramentos de la iniciación cristiana a algunos adultos de diversos países.] El Bautismo significa precisamente que no es un asunto del pasado, sino un salto cualitativo de la historia universal que llega hasta mí, tomándome para atraerme. El Bautismo es algo muy diverso de un acto de socialización eclesial, de un ritual un poco fuera de moda y complicado para acoger a las personas en la Iglesia. También es más que una simple limpieza, una especie de purificación y embellecimiento del alma. Es realmente muerte y resurrección, renacimiento, transformación en una nueva vida.

¿Cómo lo podemos entender? Pienso que lo que ocurre en el Bautismo se puede aclarar más fácilmente para nosotros si nos fijamos en la parte final de la pequeña autobiografía espiritual que san Pablo nos ha dejado en su \textit{Carta a los Gálatas}. Concluye con las palabras que contienen también el núcleo de dicha biografía: \textquote{\textit{Vivo yo, pero no soy yo, es Cristo quien vive en mí}} (2, 20). Vivo, pero ya no soy yo. El yo mismo, la identidad esencial del hombre –de este hombre, Pablo– ha cambiado. Él todavía existe y ya no existe. Ha atravesado un \textquote{no} y sigue encontrándose en este \textquote{no}: \textit{Yo, pero \textquote{no} más yo}. Con estas palabras, Pablo no describe una experiencia mística cualquiera, que tal vez podía habérsele concedido y, si acaso, podría interesarnos desde el punto de vista histórico. No, esta frase es la expresión de lo que ha ocurrido en el Bautismo. Se me quita el propio yo y es insertado en un nuevo sujeto más grande. Así, pues, está de nuevo mi yo, pero precisamente transformado, bruñido, abierto por la inserción en el otro, en el que adquiere su nuevo espacio de existencia. Pablo nos explica lo mismo una vez más bajo otro aspecto cuando, en el tercer capítulo de la \textit{Carta a los Gálatas}, habla de la \textquote{promesa} diciendo que ésta se dio en singular, a uno solo: a Cristo. Sólo él lleva en sí toda la \textquote{promesa}.

Pero, ¿qué sucede entonces con nosotros? Vosotros habéis llegado a ser uno en Cristo, responde Pablo (cf. \textit{Ga} 3, 28). No sólo una cosa, sino uno, un único, un único sujeto nuevo. Esta liberación de nuestro yo de su aislamiento, este encontrarse en un nuevo sujeto es un encontrarse en la inmensidad de Dios y ser trasladados a una vida que ha salido ahora ya del contexto del \textquote{morir y devenir}. El gran estallido de la resurrección nos ha alcanzado en el Bautismo para atraernos.

Quedamos así asociados a una nueva dimensión de la vida en la que, en medio de las tribulaciones de nuestro tiempo, estamos ya de algún modo inmersos. Vivir la propia vida como un continuo entrar en este espacio abierto: éste es el sentido del ser bautizado, del ser cristiano. Ésta es la alegría de la Vigilia pascual. La resurrección no ha pasado, la resurrección nos ha alcanzado e impregnado. A ella, es decir al Señor resucitado, nos sujetamos, y sabemos que también Él nos sostiene firmemente cuando nuestras manos se debilitan. Nos agarramos a su mano, y así nos damos la mano unos a otros, nos convertimos en un sujeto único y no solamente en una sola cosa. \textit{Yo, pero no más yo:} ésta es la fórmula de la existencia cristiana fundada en el bautismo, la fórmula de la resurrección en el tiempo. \textit{Yo, pero no más yo:} si vivimos de este modo transformamos el mundo. Es la fórmula de contraste con todas las ideologías de la violencia y el programa que se opone a la corrupción y a las aspiraciones del poder y del poseer.

\textquote{\textit{Viviréis, porque yo sigo viviendo}}, dice Jesús en el \textit{Evangelio de San Juan} (14, 19) a sus discípulos, es decir, a nosotros. Viviremos mediante la comunión existencial con Él, por estar insertos en Él, que es la vida misma. La vida eterna, la inmortalidad beatífica, no la tenemos por nosotros mismos ni en nosotros mismos, sino por una relación, mediante la comunión existencial con Aquél que es la Verdad y el Amor y, por tanto, es eterno, es Dios mismo. La mera indestructibilidad del alma, por sí sola, no podría dar un sentido a una vida eterna, no podría hacerla una vida verdadera. La vida nos llega del ser amados por Aquél que es la Vida; nos viene del vivir con Él y del amar con Él. \textit{Yo, pero no más yo:} ésta es la vía de la Cruz, la vía que \textquote{cruza} una existencia encerrada solamente en el yo, abriendo precisamente así el camino a la alegría verdadera y duradera.

De este modo, llenos de gozo, podemos cantar con la Iglesia en el \textit{Exultet}: \textquote{Exulten por fin los coros de los ángeles\ldots Goce también la tierra}. La resurrección es un acontecimiento cósmico, que comprende cielo y tierra, y asocia el uno con la otra. Y podemos proclamar también con el \textit{Exultet}: \textquote{Cristo, tu hijo resucitado\ldots brilla sereno para el linaje humano, y vive y reina glorioso por los siglos de los siglos}. Amén.
\end{body}

\begin{patercite}
La transmisión de la fe se realiza en primer lugar mediante el bautismo. Pudiera parecer que el bautismo es sólo un modo de simbolizar la confesión de fe, un acto pedagógico para quien tiene necesidad de imágenes y gestos, pero del que, en último término, se podría prescindir. Unas palabras de san Pablo, a propósito del bautismo, nos recuerdan que no es así. Dice él que \textquote{por el bautismo fuimos sepultados en él en la muerte, para que, lo mismo que Cristo resucitó de entre los muertos por la gloria del Padre, así también nosotros andemos en una vida nueva} (\textit{Rm} 6,4). Mediante el bautismo nos convertimos en criaturas nuevas y en hijos adoptivos de Dios. El Apóstol afirma después que el cristiano ha sido entregado a un \textquote{modelo de doctrina} (\textit{typos didachés}), al que obedece de corazón (cf. \textit{Rm} 6,17). En el bautismo el hombre recibe también una doctrina que profesar y una forma concreta de vivir, que implica a toda la persona y la pone en el camino del bien. Es transferido a un ámbito nuevo, colocado en un nuevo ambiente, con una forma nueva de actuar en común, en la Iglesia. El bautismo nos recuerda así que la fe no es obra de un individuo aislado, no es un acto que el hombre pueda realizar contando sólo con sus fuerzas, sino que tiene que ser recibida, entrando en la comunión eclesial que transmite el don de Dios: nadie se bautiza a sí mismo, igual que nadie nace por su cuenta. Hemos sido bautizados. 
	
	\textbf{Francisco}, papa, \textit{Lumen Fidei}, n. 41.
\end{patercite}

\newpage 

\subsubsection{Homilía (2009): La luz, el agua y el aleluya}

\src{Basílica de San Pedro. 11 de abril de 2009.}

\begin{body}
\ltr{S}{an} Marcos nos relata en su Evangelio que los discípulos, bajando del monte de la Transfiguración, discutían entre ellos sobre lo que quería decir \textquote{resucitar de entre los muertos} (cf. \textit{Mc} 9, 10). Antes, el Señor les había anunciado su pasión y su resurrección a los tres días. Pedro había protestado ante el anuncio de la muerte. Pero ahora se preguntaban qué podía entenderse con el término \textquote{resurrección}. ¿Acaso no nos sucede lo mismo a nosotros? La Navidad, el nacimiento del Niño divino, nos resulta enseguida hasta cierto punto comprensible. Podemos amar al Niño, podemos imaginar la noche de Belén, la alegría de María, de san José y de los pastores, el júbilo de los ángeles. Pero resurrección, ¿qué es? No entra en el ámbito de nuestra experiencia y, así, el mensaje muchas veces nos parece en cierto modo incomprensible, como una cosa del pasado. La Iglesia trata de hacérnoslo comprender traduciendo este acontecimiento misterioso al lenguaje de los símbolos, en los que podemos contemplar de alguna manera este acontecimiento sobrecogedor. En la Vigilia Pascual nos indica el sentido de este día especialmente mediante tres símbolos: la luz, el agua y el canto nuevo, el Aleluya.

Primero la luz. La creación de Dios –lo acabamos de escuchar en el relato bíblico– comienza con la expresión: \textquote{Que exista la luz} (\textit{Gn} 1, 3). Donde hay luz, nace la vida, el caos puede transformarse en cosmos. En el mensaje bíblico, la luz es la imagen más inmediata de Dios: Él es todo Luminosidad, Vida, Verdad, Luz. En la Vigilia Pascual, la Iglesia lee la \textbf{narración de la creación} como profecía. En la resurrección se realiza del modo más sublime lo que este texto describe como el principio de todas las cosas. Dios dice de nuevo: \textquote{Que exista la luz}. La resurrección de Jesús es un estallido de luz. Se supera la muerte, el sepulcro se abre de par en par. El Resucitado mismo es Luz, la luz del mundo. Con la resurrección, el día de Dios entra en la noche de la historia. A partir de la resurrección, la luz de Dios se difunde en el mundo y en la historia. Se hace de día. Sólo esta Luz, Jesucristo, es la luz verdadera, más que el fenómeno físico de luz. Él es la pura Luz: Dios mismo, que hace surgir una nueva creación en aquella antigua, y transforma el caos en cosmos.

Tratemos de entender esto aún mejor. ¿Por qué Cristo es Luz? En el Antiguo Testamento, se consideraba a la Torah como la luz que procede de Dios para el mundo y la humanidad. Separa en la creación la luz de las tinieblas, es decir, el bien del mal. Indica al hombre la vía justa para vivir verdaderamente. Le indica el bien, le muestra la verdad y lo lleva hacia el amor, que es su contenido más profundo. Ella es \textquote{lámpara para mis pasos} y \textquote{luz en el sendero} (cf. \textit{Sal} 119, 105). Además, los cristianos sabían que en Cristo está presente la Torah, que la Palabra de Dios está presente en Él como Persona. La Palabra de Dios es la verdadera Luz que el hombre necesita. Esta Palabra está presente en Él, en el Hijo. El Salmo 19 compara la Torah con el sol que, al surgir, manifiesta visiblemente la gloria de Dios en todo el mundo. Los cristianos entienden: sí, en la resurrección, el Hijo de Dios ha surgido como Luz del mundo. Cristo es la gran Luz de la que proviene toda vida. Él nos hace reconocer la gloria de Dios de un confín al otro de la tierra. Él nos indica la senda. Él es el día de Dios que ahora, avanzando, se difunde por toda la tierra. Ahora, viviendo con Él y por Él, podemos vivir en la luz.

En la Vigilia Pascual, la Iglesia representa el misterio de luz de Cristo con el signo del cirio pascual, cuya llama es a la vez luz y calor. El simbolismo de la luz se relaciona con el del fuego: luminosidad y calor, luminosidad y energía transformadora del fuego: verdad y amor van unidos. El cirio pascual arde y, al arder, se consume: cruz y resurrección son inseparables. De la cruz, de la autoentrega del Hijo, nace la luz, viene la verdadera luminosidad al mundo. Todos nosotros encendemos nuestras velas del cirio pascual, sobre todo las de los recién bautizados, a los que, en este Sacramento, se les pone la luz de Cristo en lo más profundo de su corazón. La Iglesia antigua ha calificado el Bautismo como \textit{fotismos}, como Sacramento de la iluminación, como una comunicación de luz, y lo ha relacionado inseparablemente con la resurrección de Cristo. En el Bautismo, Dios dice al bautizando: \textquote{Recibe la luz}. El bautizando es introducido en la luz de Cristo. Ahora, Cristo separa la luz de las tinieblas. En Él reconocemos lo verdadero y lo falso, lo que es la luminosidad y lo que es la oscuridad. Con Él surge en nosotros la luz de la verdad y empezamos a entender. Una vez, cuando Cristo vio a la gente que había venido para escucharlo y esperaba de Él una orientación, sintió compasión de ellos, porque andaban como ovejas sin pastor (cf. \textit{Mc} 6, 34). Entre las corrientes contrastantes de su tiempo, no sabían dónde ir. Cuánta compasión debe sentir Cristo también en nuestro tiempo por tantas grandilocuencias, tras las cuales se esconde en realidad una gran desorientación. ¿Dónde hemos de ir? ¿Cuáles son los valores sobre los cuales regularnos? ¿Los valores en que podemos educar a los jóvenes, sin darles normas que tal vez no aguantan o exigirles algo que quizás no se les debe imponer? Él es la Luz. El cirio bautismal es el símbolo de la iluminación que recibimos en el Bautismo. Así, en esta hora, también san Pablo nos habla muy directamente. En la \textit{Carta a los Filipenses}, dice que, en medio de una generación tortuosa y convulsa, los cristianos han de brillar como lumbreras del mundo (cf. 2, 15). Pidamos al Señor que la llamita de la vela, que Él ha encendido en nosotros, la delicada luz de su palabra y su amor, no se apague entre las confusiones de estos tiempos, sino que sea cada vez más grande y luminosa, con el fin de que seamos con Él personas amanecidas, astros para nuestro tiempo.

El segundo símbolo de la Vigilia Pascual –la noche del Bautismo– es el agua. Aparece en la Sagrada Escritura y, por tanto, también en la estructura interna del Sacramento del Bautismo en dos sentidos opuestos. Por un lado está el mar, que se manifiesta como el poder antagonista de la vida sobre la tierra, como su amenaza constante, pero al que Dios ha puesto un límite. Por eso, el \textit{Apocalipsis} dice que en el mundo nuevo de Dios ya no habrá mar (cf. 21, 1). Es el elemento de la muerte. Y por eso se convierte en la representación simbólica de la muerte en cruz de Jesús: Cristo ha descendido en el mar, en las aguas de la muerte, como Israel en el Mar Rojo. Resucitado de la muerte, Él nos da la vida. Esto significa que el Bautismo no es sólo un lavacro, sino un nuevo nacimiento: con Cristo es como si descendiéramos en el mar de la muerte, para resurgir como criaturas nuevas.

El otro modo en que aparece el agua es como un manantial fresco, que da la vida, o también como el gran río del que proviene la vida. Según el primitivo ordenamiento de la Iglesia, se debía administrar el Bautismo con agua fresca de manantial. Sin agua no hay vida. Impresiona la importancia que tienen los pozos en la Sagrada Escritura. Son lugares de donde brota la vida. Junto al pozo de Jacob, Cristo anuncia a la Samaritana el pozo nuevo, el agua de la vida verdadera. Él se manifiesta como el nuevo Jacob, el definitivo, que abre a la humanidad el pozo que ella espera: ese agua que da la vida y que nunca se agota (cf. \textit{Jn} 4, 5. 15). San Juan nos dice que un soldado golpeó con una lanza el costado de Jesús, y que del costado abierto, del corazón traspasado, salió sangre y agua (cf. \textit{Jn} 19, 34). La Iglesia antigua ha visto aquí un símbolo del Bautismo y la Eucaristía, que provienen del corazón traspasado de Jesús. En la muerte, Jesús se ha convertido Él mismo en el manantial. El profeta Ezequiel percibió en una visión el Templo nuevo del que brota un manantial que se transforma en un gran río que da la vida (cf. 47, 1-12): en una Tierra que siempre sufría la sequía y la falta de agua, ésta era una gran visión de esperanza. El cristianismo de los comienzos entendió que esta visión se ha cumplido en Cristo. Él es el Templo auténtico y vivo de Dios. Y es la fuente de agua viva. De Él brota el gran río que fructifica y renueva el mundo en el Bautismo, el gran río de agua viva, su Evangelio que fecunda la tierra. Pero Jesús ha profetizado en un discurso durante la Fiesta de las Tiendas algo más grande aún. Dice: \textquote{El que cree en mí\ldots de sus entrañas manarán torrentes de agua viva} (\textit{Jn} 7, 38). En el Bautismo, el Señor no sólo nos convierte en personas de luz, sino también en fuentes de las que brota agua viva. Todos nosotros conocemos personas de este tipo, que nos dejan en cierto modo sosegados y renovados; personas que son como el agua fresca de un manantial. No hemos de pensar sólo en los grandes personajes, como Agustín, Francisco de Asís, Teresa de Ávila, Madre Teresa de Calcuta, y así sucesivamente; personas por las que han entrado en la historia realmente ríos de agua viva. Gracias a Dios, las encontramos continuamente también en nuestra vida cotidiana: personas que son una fuente. Ciertamente, conocemos también lo opuesto: gente de la que promana un vaho como el de un charco de agua putrefacta, o incluso envenenada. Pidamos al Señor, que nos ha dado la gracia del Bautismo, que seamos siempre fuentes de agua pura, fresca, saltarina del manantial de su verdad y de su amor.

El tercer gran símbolo de la Vigilia Pascual es de naturaleza singular, y concierne al hombre mismo. Es el cantar el canto nuevo, el aleluya. Cuando un hombre experimenta una gran alegría, no puede guardársela para sí mismo. Tiene que expresarla, transmitirla. Pero, ¿qué sucede cuando el hombre se ve alcanzado por la luz de la resurrección y, de este modo, entra en contacto con la Vida misma, con la Verdad y con el Amor? Simplemente, que no basta hablar de ello. Hablar no es suficiente. Tiene que cantar. En la Biblia, la primera mención de este cantar se encuentra después de la travesía del Mar Rojo. Israel se ha liberado de la esclavitud. Ha salido de las profundidades amenazadoras del mar. Es como si hubiera renacido. Está vivo y libre. La Biblia describe la reacción del pueblo a este gran acontecimiento de salvación con la expresión: \textquote{El pueblo creyó en el Señor y en Moisés, su siervo} (cf. \textit{Ex} 14, 31). Sigue a continuación la segunda reacción, que se desprende de la primera como una especie de necesidad interior: \textquote{Entonces Moisés y los hijos de Israel cantaron un cántico al Señor}. En la Vigilia Pascual, año tras año, los cristianos entonamos después de la tercera lectura este canto, lo entonamos como nuestro cántico, porque también nosotros, por el poder de Dios, hemos sido rescatados del agua y liberados para la vida verdadera.

La historia del canto de Moisés tras la liberación de Israel de Egipto y el paso del Mar Rojo, tiene un paralelismo sorprendente en el \textit{Apocalipsis} de san Juan. Antes del comienzo de las últimas siete plagas a las que fue sometida la tierra, al vidente se le aparece \textquote{una especie de mar de vidrio veteado de fuego; en la orilla estaban de pie los que habían vencido a la bestia, a su imagen y al número que es cifra de su nombre: tenían en sus manos las arpas que Dios les había dado. Cantaban el cántico de Moisés, el siervo de Dios, y el cántico del Cordero} (\textit{Ap} 15, 2s). Con esta imagen se describe la situación de los discípulos de Jesucristo en todos los tiempos, la situación de la Iglesia en la historia de este mundo. Humanamente hablando, es una situación contradictoria en sí misma. Por un lado, se encuentra en el éxodo, en medio del Mar Rojo. En un mar que, paradójicamente, es a la vez hielo y fuego. Y ¿no debe quizás la Iglesia, por decirlo así, caminar siempre sobre el mar, a través del fuego y del frío? Considerándolo humanamente, debería hundirse. Pero mientras aún camina por este Mar Rojo, canta, entona el canto de alabanza de los justos: el canto de Moisés y del Cordero, en el cual se armonizan la Antigua y la Nueva Alianza. Mientras que a fin de cuentas debería hundirse, la Iglesia entona el canto de acción de gracias de los salvados. Está sobre las aguas de muerte de la historia y, no obstante, ya ha resucitado. Cantando, se agarra a la mano del Señor, que la mantiene sobre las aguas. Y sabe que, con eso, está sujeta, fuera del alcance de la fuerza de gravedad de la muerte y del mal –una fuerza de la cual, de otro modo, no podría escapar–, sostenida y atraída por la nueva fuerza de gravedad de Dios, de la verdad y del amor. Por el momento, la Iglesia y todos nosotros nos encontramos entre los dos campos de gravitación. Pero desde que Cristo ha resucitado, la gravitación del amor es más fuerte que la del odio; la fuerza de gravedad de la vida es más fuerte que la de la muerte. ¿Acaso no es ésta realmente la situación de la Iglesia de todos los tiempos, nuestra propia situación? Siempre se tiene la impresión de que ha de hundirse, y siempre está ya salvada. San Pablo ha descrito así esta situación: \textquote{Somos\ldots los moribundos que están bien vivos} (\textit{2 Co} 6, 9). La mano salvadora del Señor nos sujeta, y así podemos cantar ya ahora el canto de los salvados, el canto nuevo de los resucitados: ¡aleluya! Amén.
\end{body}

\begin{patercite}
¿Cuáles son los elementos del bautismo que nos introducen en este nuevo \textquote{modelo de doctrina}? Sobre el catecúmeno se invoca, en primer lugar, el nombre de la Trinidad: Padre, Hijo y Espíritu Santo. Se le presenta así desde el principio un resumen del camino de la fe. El Dios que ha llamado a Abrahán y ha querido llamarse su Dios, el Dios que ha revelado su nombre a Moisés, el Dios que, al entregarnos a su Hijo, nos ha revelado plenamente el misterio de su Nombre, da al bautizado una nueva condición filial. Así se ve claro el sentido de la acción que se realiza en el bautismo, la inmersión en el agua: el agua es símbolo de muerte, que nos invita a pasar por la conversión del \textquote{yo}, para que pueda abrirse a un \textquote{Yo} más grande; y a la vez es símbolo de vida, del seno del que renacemos para seguir a Cristo en su nueva existencia. De este modo, mediante la inmersión en el agua, el bautismo nos habla de la estructura encarnada de la fe. La acción de Cristo nos toca en nuestra realidad personal, transformándonos radicalmente, haciéndonos hijos adoptivos de Dios, partícipes de su naturaleza divina; modifica así todas nuestras relaciones, nuestra forma de estar en el mundo y en el cosmos, abriéndolas a su misma vida de comunión. Este dinamismo de transformación propio del bautismo nos ayuda a comprender la importancia que tiene hoy el catecumenado para la nueva evangelización, también en las sociedades de antiguas raíces cristianas, en las cuales cada vez más adultos se acercan al sacramento del bautismo. El catecumenado es camino de preparación para el bautismo, para la transformación de toda la existencia en Cristo. 
	
Un texto del profeta Isaías, que ha sido relacionado con el bautismo en la literatura cristiana antigua, nos puede ayudar a comprender la conexión entre el bautismo y la fe: \textquote{Tendrá su alcázar en un picacho rocoso \ldots con provisión de agua} (\textit{Is} 33,16)(Cf. \textit{Epistula Barnabae}, 11, 5: SC 172, 162). El bautizado, rescatado del agua de la muerte, puede ponerse en pie sobre el \textquote{picacho rocoso}, porque ha encontrado algo consistente donde apoyarse. Así, el agua de muerte se transforma en agua de vida. El texto griego lo llama agua \textit{pistós}, agua \textquote{fiel}. El agua del bautismo es fiel porque se puede confiar en ella, porque su corriente introduce en la dinámica del amor de Jesús, fuente de seguridad para el camino de nuestra vida.
	
	\textbf{Francisco}, papa, \textit{Lumen Fidei}, n. 42.
\end{patercite}

\newpage 

\subsubsection{Homilía (2012): Nueva dimensión para el hombre}

\src{8 de abril de 2012. Basílica Vaticana.}

\begin{body}
\ltr{P}{ascua} es la fiesta de la nueva creación. Jesús ha resucitado y no morirá de nuevo. Ha descerrajado la puerta hacia una nueva vida que ya no conoce ni la enfermedad ni la muerte. Ha asumido al hombre en Dios mismo. \textquote{Ni la carne ni la sangre pueden heredar el reino de Dios}, dice Pablo en la \textit{Primera Carta a los Corintios} (15, 50). El escritor eclesiástico Tertuliano, en el siglo III, tuvo la audacia de escribir refriéndose a la resurrección de Cristo y a nuestra resurrección: \textquote{Carne y sangre, tened confianza, gracias a Cristo habéis adquirido un lugar en el cielo y en el reino de Dios} (\textit{CCL} II, 994). Se ha abierto una nueva dimensión para el hombre. La creación se ha hecho más grande y más espaciosa. La Pascua es el día de una nueva creación, pero precisamente por ello la Iglesia comienza la liturgia con la antigua creación, para que aprendamos a comprender la nueva. Así, en la Vigilia de Pascua, al principio de la Liturgia de la Palabra, se lee el relato de la \textbf{creación del mundo}. En el contexto de la liturgia de este día, hay dos aspectos particularmente importantes. En primer lugar, que se presenta a la creación como una totalidad, de la cual forma parte la dimensión del tiempo. Los siete días son una imagen de un conjunto que se desarrolla en el tiempo. Están ordenados con vistas al séptimo día, el día de la libertad de todas las criaturas para con Dios y de las unas para con las otras. Por tanto, la creación está orientada a la comunión entre Dios y la criatura; existe para que haya un espacio de respuesta a la gran gloria de Dios, un encuentro de amor y libertad. En segundo lugar, que en la Vigilia Pascual, la Iglesia comienza escuchando ante todo la primera frase de la historia de la creación: \textquote{Dijo Dios: \textquote{Que exista la luz}} (\textit{Gn} 1, 3). Como una señal, el relato de la creación inicia con la creación de la luz. El sol y la luna son creados sólo en el cuarto día. La narración de la creación los llama fuentes de luz, que Dios ha puesto en el firmamento del cielo. Con ello, los priva premeditadamente del carácter divino, que las grandes religiones les habían atribuido. No, ellos no son dioses en modo alguno. Son cuerpos luminosos, creados por el Dios único. Pero están precedidos por la luz, por la cual la gloria de Dios se refleja en la naturaleza de las criaturas.

¿Qué quiere decir con esto el relato de la creación? La luz hace posible la vida. Hace posible el encuentro. Hace posible la comunicación. Hace posible el conocimiento, el acceso a la realidad, a la verdad. Y, haciendo posible el conocimiento, hace posible la libertad y el progreso. El mal se esconde. Por tanto, la luz es también una expresión del bien, que es luminosidad y crea luminosidad. Es el día en el que podemos actuar. El que Dios haya creado la luz significa que Dios creó el mundo como un espacio de conocimiento y de verdad, espacio para el encuentro y la libertad, espacio del bien y del amor. La materia prima del mundo es buena, el ser es bueno en sí mismo. Y el mal no proviene del ser, que es creado por Dios, sino que existe sólo en virtud de la negación. Es el \textquote{no}.

En Pascua, en la mañana del primer día de la semana, Dios vuelve a decir: \textquote{Que exista la luz}. Antes había venido la noche del Monte de los Olivos, el eclipse solar de la pasión y muerte de Jesús, la noche del sepulcro. Pero ahora vuelve a ser el primer día, comienza la creación totalmente nueva. \textquote{Que exista la luz}, dice Dios, \textquote{y existió la luz}. Jesús resucita del sepulcro. La vida es más fuerte que la muerte. El bien es más fuerte que el mal. El amor es más fuerte que el odio. La verdad es más fuerte que la mentira. La oscuridad de los días pasados se disipa cuando Jesús resurge de la tumba y se hace él mismo luz pura de Dios. Pero esto no se refiere solamente a él, ni se refiere únicamente a la oscuridad de aquellos días. Con la resurrección de Jesús, la luz misma vuelve a ser creada. Él nos lleva a todos tras él a la vida nueva de la resurrección, y vence toda forma de oscuridad. Él es el nuevo día de Dios, que vale para todos nosotros.

Pero, ¿cómo puede suceder esto? ¿Cómo puede llegar todo esto a nosotros sin que se quede sólo en palabras sino que sea una realidad en la que estamos inmersos? Por el sacramento del bautismo y la profesión de la fe, el Señor ha construido un puente para nosotros, a través del cual el nuevo día viene a nosotros. En el bautismo, el Señor dice a aquel que lo recibe: \textit{Fiat lux}, que exista la luz. El nuevo día, el día de la vida indestructible llega también para nosotros. Cristo nos toma de la mano. A partir de ahora él te apoyará y así entrarás en la luz, en la vida verdadera. Por eso, la Iglesia antigua ha llamado al bautismo \textit{photismos}, iluminación.

¿Por qué? La oscuridad amenaza verdaderamente al hombre porque, sí, éste puede ver y examinar las cosas tangibles, materiales, pero no a dónde va el mundo y de dónde procede. A dónde va nuestra propia vida. Qué es el bien y qué es el mal. La oscuridad acerca de Dios y sus valores son la verdadera amenaza para nuestra existencia y para el mundo en general. Si Dios y los valores, la diferencia entre el bien y el mal, permanecen en la oscuridad, entonces todas las otras iluminaciones que nos dan un poder tan increíble, no son sólo progreso, sino que son al mismo tiempo también amenazas que nos ponen en peligro, a nosotros y al mundo. Hoy podemos iluminar nuestras ciudades de manera tan deslumbrante que ya no pueden verse las estrellas del cielo. ¿Acaso no es esta una imagen de la problemática de nuestro ser ilustrado? En las cosas materiales, sabemos y podemos tanto, pero lo que va más allá de esto, Dios y el bien, ya no lo conseguimos identificar. Por eso la fe, que nos muestra la luz de Dios, es la verdadera iluminación, es una irrupción de la luz de Dios en nuestro mundo, una apertura de nuestros ojos a la verdadera luz.

Queridos amigos, quisiera por último añadir todavía una anotación sobre la luz y la iluminación. En la Vigilia Pascual, la noche de la nueva creación, la Iglesia presenta el misterio de la luz con un símbolo del todo particular y muy humilde: el cirio pascual. Esta es una luz que vive en virtud del sacrificio. La luz de la vela ilumina consumiéndose a sí misma. Da luz dándose a sí misma. Así, representa de manera maravillosa el misterio pascual de Cristo que se entrega a sí mismo, y de este modo da mucha luz. Otro aspecto sobre el cual podemos reflexionar es que la luz de la vela es fuego. El fuego es una fuerza que forja el mundo, un poder que transforma. Y el fuego da calor. También en esto se hace nuevamente visible el misterio de Cristo. Cristo, la luz, es fuego, es llama que destruye el mal, transformando así al mundo y a nosotros mismos. Como reza una palabra de Jesús que nos ha llegado a través de Orígenes, \textquote{quien está cerca de mí, está cerca del fuego}. Y este fuego es al mismo tiempo calor, no una luz fría, sino una luz en la que salen a nuestro encuentro el calor y la bondad de Dios.

El gran himno del \textit{Exsultet}, que el diácono canta al comienzo de la liturgia de Pascua, nos hace notar, muy calladamente, otro detalle más. Nos recuerda que este objeto, el cirio, se debe principalmente a la labor de las abejas. Así, toda la creación entra en juego. En el cirio, la creación se convierte en portadora de luz. Pero, según los Padres, también hay una referencia implícita a la Iglesia. La cooperación de la comunidad viva de los fieles en la Iglesia es algo parecido al trabajo de las abejas. Construye la comunidad de la luz. Podemos ver así también en el cirio una referencia a nosotros y a nuestra comunión en la comunidad de la Iglesia, que existe para que la luz de Cristo pueda iluminar al mundo. Roguemos al Señor en esta hora que nos haga experimentar la alegría de su luz, y pidámosle que nosotros mismos seamos portadores de su luz, con el fin de que, a través de la Iglesia, el esplendor del rostro de Cristo entre en el mundo (cf. \textit{Lumen gentium}, 1). Amén.
\end{body}

\begin{patercite}
La naturaleza sacramental de la fe alcanza su máxima expresión en la eucaristía, que es el precioso alimento para la fe, el encuentro con Cristo presente realmente con el acto supremo de amor, el don de sí mismo, que genera vida. En la eucaristía confluyen los dos ejes por los que discurre el camino de la fe. Por una parte, el eje de la historia: la eucaristía es un acto de memoria, actualización del misterio, en el cual el pasado, como acontecimiento de muerte y resurrección, muestra su capacidad de abrir al futuro, de anticipar la plenitud final. La liturgia nos lo recuerda con su \textit{hodie}, el \textquote{hoy} de los misterios de la salvación. Por otra parte, confluye en ella también el eje que lleva del mundo visible al invisible. En la eucaristía aprendemos a ver la profundidad de la realidad. El pan y el vino se transforman en el Cuerpo y Sangre de Cristo, que se hace presente en su camino pascual hacia el Padre: este movimiento nos introduce, en cuerpo y alma, en el movimiento de toda la creación hacia su plenitud en Dios.

\textbf{Francisco}, papa, \textit{Lumen Fidei}, n. 44.
\end{patercite}

\newsection
\subsection{Francisco, papa}

\subsubsection{Homilía (2015): Entrar en el misterio}

\src{Basílica Vaticana. 4 de abril de 2015.}

\begin{body}
\ltr{E}{sta} noche es noche de vigilia. El Señor no duerme, vela el guardián de su pueblo (cf. \textit{Sal} 121, 4), para sacarlo de la esclavitud y para abrirle el camino de la libertad. El Señor vela y, con la fuerza de su amor, hace pasar al pueblo a través del Mar Rojo; y hace pasar a Jesús a través del abismo de la muerte y de los infiernos. Esta fue una noche de vela para los discípulos y las discípulas de Jesús. Noche de dolor y de temor. Los hombres permanecieron cerrados en el Cenáculo. Las mujeres, sin embargo, al alba del día siguiente al sábado, fueron al sepulcro para ungir el cuerpo de Jesús. Sus corazones estaban llenos de emoción y se preguntaban: \textquote{¿Cómo haremos para entrar?, ¿quién nos removerá la piedra de la tumba?\ldots}. Pero he aquí el primer signo del Acontecimiento: la gran piedra \textit{ya había sido removida}, y la tumba estaba abierta.

\textquote{Entraron en el sepulcro y vieron a un joven sentado a la derecha, vestido de blanco} (\textit{Mc} 16, 5). Las mujeres fueron las primeras que vieron este gran signo: el sepulcro vacío; y fueron las primeras en entrar. \textquote{Entraron en el sepulcro}. En esta noche de vigilia, nos viene bien detenernos a reflexionar sobre la experiencia de las discípulas de Jesús, que también nos interpela a nosotros. Efectivamente, para eso estamos aquí: para entrar, para \textit{entrar en el misterio} que Dios ha realizado con su vigilia de amor.

No se puede vivir la Pascua sin entrar en el misterio. No es un hecho intelectual, no es sólo conocer, leer\ldots Es más, es mucho más. \textquote{Entrar en el misterio} significa capacidad de asombro, de contemplación; capacidad de escuchar el silencio y sentir el susurro de ese hilo de silencio sonoro en el que Dios nos habla (cf. \textit{1 Re} 19, 12). Entrar en el misterio nos exige no tener miedo de la realidad: no cerrarse en sí mismos, no huir ante lo que no entendemos, no cerrar los ojos frente a los problemas, no negarlos, no eliminar los interrogantes\ldots Entrar en el misterio significa ir más allá de las cómodas certezas, más allá de la pereza y la indiferencia que nos frenan, y ponerse en busca de la verdad, la belleza y el amor, buscar un sentido no ya descontado, una respuesta no trivial a las cuestiones que ponen en crisis nuestra fe, nuestra fidelidad y nuestra razón.

Para entrar en el misterio se necesita humildad, la humildad de abajarse, de apearse del pedestal de nuestro yo, tan orgulloso, de nuestra presunción; la humildad para redimensionar la propia estima, reconociendo lo que realmente somos: criaturas con virtudes y defectos, pecadores necesitados de perdón. Para entrar en el misterio hace falta este abajamiento, que es impotencia, vaciamiento de las propias idolatrías\ldots adoración. Sin adorar no se puede entrar en el misterio.

Todo esto nos enseñan las mujeres discípulas de Jesús. Velaron aquella noche, junto a la Madre. Y ella, la Virgen Madre, les ayudó a no perder la fe y la esperanza. Así, no permanecieron prisioneras del miedo y del dolor, sino que salieron con las primeras luces del alba, llevando en las manos sus ungüentos y con el corazón ungido de amor. Salieron y encontraron la tumba abierta. Y entraron. Velaron, salieron y entraron en el misterio. Aprendamos de ellas a velar con Dios y con María, nuestra Madre, para entrar en el misterio que nos hace pasar de la muerte a la vida.
\end{body}

\begin{patercite}
En la celebración de los sacramentos, la Iglesia transmite su memoria, en particular mediante la profesión de fe. Ésta no consiste sólo en asentir a un conjunto de verdades abstractas. Antes bien, en la confesión de fe, toda la vida se pone en camino hacia la comunión plena con el Dios vivo. Podemos decir que en el \textit{Credo} el creyente es invitado a entrar en el misterio que profesa y a dejarse transformar por lo que profesa. Para entender el sentido de esta afirmación, pensemos antes que nada en el contenido del \textit{Credo}. Tiene una estructura trinitaria: el Padre y el Hijo se unen en el Espíritu de amor. El creyente afirma así que el centro del ser, el secreto más profundo de todas las cosas, es la comunión divina. Además, el \textit{Credo} contiene también una profesión cristológica: se recorren los misterios de la vida de Jesús hasta su muerte, resurrección y ascensión al cielo, en la espera de su venida gloriosa al final de los tiempos. Se dice, por tanto, que este Dios comunión, intercambio de amor entre el Padre y el Hijo en el Espíritu, es capaz de abrazar la historia del hombre, de introducirla en su dinamismo de comunión, que tiene su origen y su meta última en el Padre. Quien confiesa la fe, se ve implicado en la verdad que confiesa. No puede pronunciar con verdad las palabras del \textit{Credo} sin ser transformado, sin inserirse en la historia de amor que lo abraza, que dilata su ser haciéndolo parte de una comunión grande, del sujeto último que pronuncia el \textit{Credo}, que es la Iglesia. Todas las verdades que se creen proclaman el misterio de la vida nueva de la fe como camino de comunión con el Dios vivo.
	
\textbf{Francisco}, papa, \textit{Lumen Fidei}, n. 45.
\end{patercite}

\newpage 

\subsubsection{Homilía (2018): El triunfo de la Vida} 

\src{Basílica Vaticana. 31 de marzo de 2018.}

\begin{body}
\ltr{E}{sta} celebración la hemos comenzado fuera\ldots inmersos en la oscuridad de la noche y en el frío que la acompaña. Sentimos el peso del silencio ante la muerte del Señor, un silencio en el que cada uno de nosotros puede reconocerse y cala hondo en las hendiduras del corazón del discípulo que ante la cruz se queda sin palabras. Son las horas del discípulo enmudecido frente al dolor que genera la muerte de Jesús: ¿Qué decir ante tal situación? El discípulo que se queda sin palabras al tomar conciencia de sus reacciones durante las horas cruciales en la vida del Señor: frente a la injusticia que condenó al Maestro, los discípulos hicieron silencio; frente a las calumnias y al falso testimonio que sufrió el Maestro, los discípulos callaron. Durante las horas difíciles y dolorosas de la Pasión, los discípulos experimentaron de forma dramática su incapacidad de \textquote{jugársela} y de hablar en favor del Maestro. Es más, no lo conocían, se escondieron, se escaparon, callaron (cfr. \textit{Jn} 18, 25-27). 

Es la noche del silencio del discípulo que se encuentra entumecido y paralizado, sin saber hacia dónde ir frente a tantas situaciones dolorosas que lo agobian y rodean. Es el discípulo de hoy, enmudecido ante una realidad que se le impone haciéndole sentir, y lo que es peor, creer que nada puede hacerse para revertir tantas injusticias que viven en su carne nuestros hermanos. 

Es el discípulo atolondrado por estar inmerso en una rutina aplastante que le roba la memoria, silencia la esperanza y lo habitúa al \textquote{siempre se hizo así}. Es el discípulo enmudecido que, abrumado, termina \textquote{normalizando} y acostumbrándose a la expresión de Caifás: \textquote{¿No les parece preferible que un solo hombre muera por el pueblo y no perezca la nación entera?} (\textit{Jn} 11, 50). 

Y en medio de nuestros silencios, cuando callamos tan contundentemente, entonces las piedras empiezan a gritar (cf. \textit{Lc} 19,40)\footnote{16} y a dejar espacio para el mayor anuncio que jamás la historia haya podido contener en su seno: \textquote{No está aquí ha resucitado} (\textit{Mt} 28,6). La piedra del sepulcro gritó y en su grito anunció para todos un nuevo camino. Fue la creación la primera en hacerse eco del triunfo de la Vida sobre todas las formas que intentaron callar y enmudecer la alegría del evangelio. Fue la piedra del sepulcro la primera en saltar y a su manera entonar un canto de alabanza y admiración, de alegría y de esperanza al que todos somos invitados a tomar parte. 

Y si ayer, con las mujeres contemplábamos \textquote{al que traspasaron} (\textit{Jn} 19, 36; cf. \textit{Za} 12, 10); hoy con ellas somos invitados a contemplar la tumba vacía y a escuchar las palabras del ángel: \textquote{no tengan miedo\ldots ha resucitado} (\textit{Mt} 28, 5-6). Palabras que quieren tocar nuestras convicciones y certezas más hondas, nuestras formas de juzgar y enfrentar los acontecimientos que vivimos a diario; especialmente nuestra manera de relacionarnos con los demás. La tumba vacía quiere desafiar, movilizar, cuestionar, pero especialmente quiere animarnos a creer y a confiar que Dios \textquote{acontece} en cualquier situación, en cualquier persona, y que su luz puede llegar a los rincones menos esperados y más cerrados de la existencia. Resucitó de la muerte, resucitó del lugar del que nadie esperaba nada y nos espera –al igual que a las mujeres– para hacernos tomar parte de su obra salvadora. Este es el fundamento y la fuerza que tenemos los cristianos para poner nuestra vida y energía, nuestra inteligencia, afectos y voluntad en buscar, y especialmente en generar, caminos de dignidad. ¡No está aquí\ldots ha resucitado! Es el anuncio que sostiene nuestra esperanza y la transforma en gestos concretos de caridad. ¡Cuánto necesitamos dejar que nuestra fragilidad sea ungida por esta experiencia, cuánto necesitamos que nuestra fe sea renovada, cuánto necesitamos que nuestros miopes horizontes se vean cuestionados y renovados por este anuncio! Él resucitó y con él resucita nuestra esperanza y creatividad para enfrentar los problemas presentes, porque sabemos que no vamos solos. 

Celebrar la Pascua, es volver a creer que Dios irrumpe y no deja de irrumpir en nuestras historias desafiando nuestros \textquote{conformantes} y paralizadores determinismos. Celebrar la Pascua es dejar que Jesús venza esa pusilánime actitud que tantas veces nos rodea e intenta sepultar todo tipo de esperanza. La piedra del sepulcro tomó parte, las mujeres del evangelio tomaron parte, ahora la invitación va dirigida una vez más a ustedes y a mí: invitación a romper las rutinas, renovar nuestra vida, nuestras opciones y nuestra existencia. Una invitación que va dirigida allí donde estamos, en lo que hacemos y en lo que somos; con la \textquote{cuota de poder} que poseemos. ¿Queremos tomar parte de este anuncio de vida o seguiremos enmudecidos ante los acontecimientos?

¡No está aquí ha resucitado! Y te espera en Galilea, te invita a volver al tiempo y al lugar del primer amor y decirte: No tengas miedo, sígueme.
\end{body}

\newsection
\section{Temas}
\rbr{Ver los temas del Día de Pascua en la página \pageref{temasc04}.}

\begin{patercite}
Esta semana es la semana de la alegría: celebramos la Resurrección de Jesús. Es una alegría auténtica, profunda, basada en la certeza que Cristo resucitado ya no muere más, sino que está vivo y operante en la Iglesia y en el mundo. Tal certeza habita en el corazón de los creyentes desde esa mañana de Pascua, cuando las mujeres fueron al sepulcro de Jesús y los ángeles les dijeron: \textquote{¿Por qué buscáis entre los muertos al que vive?} (\emph{Lc} 24, 5). \textquote{¿Por qué buscáis entre los muertos al que vive?}. Estas palabras son como una piedra miliar en la historia; pero también una \textquote{piedra de tropiezo}, si no nos abrimos a la Buena Noticia, si pensamos que da menos fastidio un Jesús muerto que un Jesús vivo. En cambio, cuántas veces, en nuestro camino cotidiano, necesitamos que nos digan: \textquote{¿Por qué buscáis entre los muertos al que vive?}. Cuántas veces buscamos la vida entre las cosas muertas, entre las cosas que no pueden dar vida, entre las cosas que hoy están y mañana ya no estarán, las cosas que pasan\ldots \textquote{¿Por qué buscáis entre los muertos al que vive?}.

Lo necesitamos cuando nos encerramos en cualquier forma de egoísmo o de auto-complacencia; cuando nos dejamos seducir por los poderes terrenos y por las cosas de este mundo, olvidando a Dios y al prójimo; cuando ponemos nuestras esperanzas en vanidades mundanas, en el dinero, en el éxito. Entonces la Palabra de Dios nos dice: \textquote{¿Por qué buscáis entre los muertos al que vive?}. ¿Por qué lo estás buscando allí? Eso no te puede dar vida. Sí, tal vez te dará una alegría de un minuto, de un día, de una semana, de un mes\ldots ¿y luego? \textquote{¿Por qué buscáis entre los muertos al que vive?}. Esta frase debe entrar en el corazón y debemos repetirla. (\ldots) Hoy, cuando volvamos a casa, digámosla desde el corazón, en silencio, y hagámonos esta pregunta: ¿por qué yo en la vida busco entre los muertos a aquél que vive? Nos hará bien.

\textbf{Francisco}, papa, \textit{Catequesis} audiencia general, 23 de abril de 2014, parr. 1-2. 
\end{patercite}

