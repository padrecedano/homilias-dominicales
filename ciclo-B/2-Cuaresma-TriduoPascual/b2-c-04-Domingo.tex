\chapter{Día de Pascua}

\section{Lecturas}

\rtitle{PRIMERA LECTURA}

\rbook{De los Hechos de los Apóstoles} \rred{10, 34a. 37-43}

\rtheme{Hemos comido y bebido con él después de su resurrección de entre los muertos}

\begin{scripture}
En aquellos días, Pedro tomó la palabra y dijo: 

\>{Vosotros conocéis lo que sucedió en toda Judea, comenzando por Galilea, después del bautismo que predicó Juan. Me refiero a Jesús de Nazaret, ungido por Dios con la fuerza del Espíritu Santo, que pasó haciendo el bien y curando a todos los oprimidos por el diablo, porque Dios estaba con él}. 

\>{Nosotros somos testigos de todo lo que hizo en la tierra de los judíos y en Jerusalén. A este lo mataron, colgándolo de un madero. Pero Dios lo resucitó al tercer día y le concedió la gracia de manifestarse, no a todo el pueblo, sino a los testigos designados por Dios: a nosotros, que hemos comido y bebido con él después de su resurrección de entre los muertos}. 

\>{Nos encargó predicar al pueblo, dando solemne testimonio de que Dios lo ha constituido juez de vivos y muertos. De él dan testimonio todos los profetas: que todos los que creen en él reciben, por su nombre, el perdón de los pecados}.
\end{scripture}

\newpage 

\rtitle{SALMO RESPONSORIAL}

\rbook{Salmo} \rred{117, 1-2. 16ab-17. 22-23}

\rtheme{Este es el día que hizo el Señor: sea nuestra alegría y nuestro gozo}

\begin{psbody}
Dad gracias al Señor porque es bueno, 
porque es eterna su misericordia. 
Diga la casa de Israel: 
eterna es su misericordia. 

\textquote{La diestra del Señor es poderosa, 
la diestra del Señor es excelsa}. 
No he de morir, viviré 
para contar las hazañas del Señor. 

La piedra que desecharon los arquitectos 
es ahora la piedra angular. 
Es el Señor quien lo ha hecho, 
ha sido un milagro patente. 
\end{psbody}

\rtitle{SEGUNDA LECTURA (opción 1)}

\rbook{De la carta del apóstol san Pablo a los Colosenses} \rred{3, 1-4}

\rtheme{Buscad los bienes de allá arriba, donde está Cristo}

\begin{scripture}
Hermanos: 

Si habéis resucitado con Cristo, buscad los bienes de allá arriba, donde Cristo está sentado a la derecha de Dios; aspirad a los bienes de arriba, no a los de la tierra. 

Porque habéis muerto; y vuestra vida está con Cristo escondida en Dios. Cuando aparezca Cristo, vida vuestra, entonces también vosotros apareceréis gloriosos, juntamente con él.
\end{scripture}

\newpage

\rtitle{SEGUNDA LECTURA (opción 2)}

\rbook{De la primera carta del apóstol san Pablo a los Corintios} \rred{5, 6b-8}

\rtheme{Barred la levadura vieja para ser una masa nueva}

\begin{scripture}
Hermanos: 

¿No sabéis que un poco de levadura fermenta toda la masa? Barred la levadura vieja para ser una masa nueva, ya que sois panes ácimos. Porque ha sido inmolada nuestra víctima pascual: Cristo. 

Así, pues, celebremos la Pascua, no con levadura vieja (levadura de corrupción y de maldad), sino con los panes ácimos de la sinceridad y la verdad.
\end{scripture}

\rtitle{EVANGELIO}

\rbook{Del Evangelio según san Juan} \rred{20, 1-9}

\rtheme{Él había de resucitar de entre los muertos}

\begin{scripture}
El primer día de la semana, María la Magdalena fue al sepulcro al amanecer, cuando aún estaba oscuro, y vio la losa quitada del sepulcro. Echó a correr y fue donde estaban Simón Pedro y el otro discípulo, a quien Jesús amaba, y les dijo: 

\>{Se han llevado del sepulcro al Señor y no sabemos dónde lo han puesto}.

Salieron Pedro y el otro discípulo camino del sepulcro. Los dos corrían juntos, pero el otro discípulo corría más que Pedro; se adelantó y llegó primero al sepulcro; e, inclinándose, vio los lienzos tendidos; pero no entró. 

Llegó también Simón Pedro detrás de él y entró en el sepulcro: vio los lienzos tendidos y el sudario con que le habían cubierto la cabeza, no con los lienzos, sino enrollado en un sitio aparte. 

Entonces entró también el otro discípulo, el que había llegado primero al sepulcro; vio y creyó. 

Pues hasta entonces no habían entendido la Escritura: que él había de resucitar de entre los muertos.
\end{scripture}



\newsection 
\section{Comentario Patrístico}

\subsection{Gregorio de Palamás}

\ptheme{Juan es aquel a quien Cristo amó con amor de predilección}

\src{Homilía 20: PG 151, 266. 271.}

\begin{body}
\ltr{J}{uan} es aquel que permaneció virgen y recibió por gracia singular y como tesoro preciosísimo, a la Virgen Madre, única entre las madres; Juan es aquel a quien Cristo amó con amor de predilección y mereció ser llamado hijo, con preferencia a los otros evangelistas. Por eso hace resonar con fuerza la trompeta al anunciarnos los prodigios de la resurrección del Señor de entre los muertos, y al relatarnos con mayor claridad el modo cómo se manifestó a sus discípulos, según lo hallamos escrito en su evangelio, cuando nos dice: \textit{El primer día de la semana, María Magdalena fue al sepulcro al amanecer, cuando aún estaba oscuro y vio la losa quitada del sepulcro. Echó a correr y fue donde estaba Simón Pedro y el otro discípulo, a quien tanto quería Jesús}. Así es como se presenta a sí mismo.

Juan y Pedro, habiendo oído a María, van corriendo al sepulcro, donde vieron que había salido la Vida; y habiendo visto y creído, admirados por las pruebas se volvieron a casa.

Consideremos, hermanos, cuánta mayor dignidad que María Magdalena no tenía Pedro, el príncipe de los apóstoles, y el mismo Juan, a quien tanto quería Jesús, y sin embargo ella fue considerada digna de una gracia tan grande, con preferencia a ellos. Porque los apóstoles, corriendo al sepulcro, sólo vieron las vendas y el sudario; María, en cambio, por su firmeza y constancia, perseverando hasta el fin a la entrada del sepulcro, llegó a ver no sólo a los ángeles, sino al mismo Señor de los ángeles en la carne, antes que los apóstoles.

Este templo que veis, es un símbolo de aquel sepulcro; y no sólo un símbolo, sino una realidad mucho más sublime. Detrás de esa cortina, en el interior, está el lugar donde se coloca el cuerpo del Señor, y ahí está también la mesa o el altar santo. Así pues, lo mismo que María, todo el que se acerque con presteza a la recepción del misterio divino y persevere hasta el fin, teniendo recogida en Dios su propia alma, no sólo reconocerá las enseñanzas de la Escritura santa, redactada por el Espíritu de Dios, ni sólo a los ángeles que anunciaron el misterio de la divinidad y humanidad del Verbo de Dios, encarnado por nosotros, sino que verá también y sin ningún género de duda al mismo Señor con los ojos del alma, y también con los del cuerpo.

Pues aquel que con fe ve la mesa mística y el pan de vida depositado sobre ella ve al mismo Verbo de Dios oculto bajo las especies, hecho carne por nosotros y habitando en nosotros como en un sagrario. Más aún: si es considerado digno de recibirle, no sólo le ve, sino que participa de él, le recibe en sí mismo como huésped, y es enriquecido con el don de la misma gracia divina. Y así como María Magdalena vio lo que antes que nada los apóstoles deseaban ver, así el alma, poseída por la fe, será considerada digna de ver y de gozar de aquello que –según el apóstol– los \textit{ángeles desean penetrar}, divinizándose por completo, tanto por la contemplación como por la participación de estos misterios.
\end{body}

\begin{patercite}
(\ldots) \textquote{\emph{A este Jesús lo resucitó Dios, de lo cual todos nosotros somos testigos}} (\emph{Act} 2, 32). Esta vigorosa proclamación de Pedro en el alba de la predicación apostólica adquiere, efectivamente, un significado particular en el clima del Aleluya pascual, con el que la Iglesia mide durante cincuenta días los ritmos de las fiestas.
	
¡Cristo, realmente muerto, ha resucitado verdaderamente! En el curso de veinte siglos, la Iglesia ha continuado presentando ante el mundo este impresionante testimonio: lo ha hecho en todo contexto cultural y social, bajo todos los cielos, con la voz de sus Pastores, con el sacrificio de sus mártires, con la entrega de la falange innumerable de sus santos.
	
[\ldots]
	
El testimonio del Resucitado es un compromiso que vincula concretamente a todos los miembros del Pueblo de Dios. El Concilio lo ha hecho objeto de una explícita llamada a los fieles laicos, recapitulando la misión que les es propia en virtud de su incorporación a Cristo, mediante el bautismo, con estas comprometedoras palabras: \textquote{Cada laico debe ser ante el mundo un testigo de la resurrección y de la vida del Señor Jesús} (\emph{Lumen gentium}, 38).
	
Dar testimonio significa esencialmente atestiguar un hecho sobre la base de una certeza que, de algún modo, es fruto de experiencia personal. Las piadosas mujeres fueron los primeros testigos del retorno del Señor a la vida (cf. por ejemplo \emph{Mt} 28, 5-8). Ellas entonces no vieron a Jesús, pero adquirieron la certeza de su resurrección a base del descubrimiento del sepulcro vacío y de la explicación que les dio el Ángel, sobre el asombroso acontecimiento. Esta fue la experiencia inicial que tuvieron del misterio, fortalecida después por las apariciones del Resucitado.
	
\textbf{San Juan Pablo II}, papa, \textit{Catequesis} audiencia general, 25 de abril de 1984, n. 1-2b. 

\end{patercite}

\newsection
\subsection{Francisco, papa}

\ptheme{Mañana que cambió la historia}

\src{A los jóvenes italianos, \\11 de agosto de 2018, parr. 3-6. 9-13.}

\begin{body}
\ltr{E}{n} el pasaje del Evangelio que hemos escuchado (cf. \textit{Jn} 20, 1-8), Juan nos relata esa mañana inimaginable que cambió para siempre la historia de la humanidad. Imaginémosla esa mañana: a las primeras luces del alba del día después del sábado, alrededor de la tumba de Jesús, todos comienzan a correr. María de Magdala corre para advertir a los discípulos; Pedro y Juan corren hacia la tumba\ldots Todos corren, todos sienten la urgencia de moverse: no hay tiempo que perder, debemos apresurarnos\ldots Como había hecho María, ¿os acordáis? ―apenas concebido Jesús–, para ir a ayudar a Isabel.

Tenemos muchas razones para correr, a menudo solo porque hay muchas cosas que hacer y el tiempo nunca es suficiente. A veces nos apresuramos porque nos atrae algo nuevo, bello, interesante. A veces, por el contrario, corremos para escapar de una amenaza, de un peligro\ldots

Los discípulos de Jesús corren porque han recibido la noticia de que el cuerpo de Jesús ha desaparecido de la tumba. Los corazones de María Magdalena, de Simón Pedro, de Juan están llenos de amor y palpitan furiosamente después de la separación que parecía definitiva. ¡Quizás se ha reavivado en ellos la esperanza de volver a ver el rostro del Señor! Como en ese primer día cuando había prometido: \textquote{Venid y ved} (\textit{Jn} 1, 39). Juan es el que corre más deprisa, ciertamente porque es más joven, pero también porque no ha dejado de esperar después de ver a Jesús morir en la cruz con sus propios ojos; y también porque estaba cerca de María, y por eso estaba \textquote{contagiado} por su fe. Cuando sentimos que la fe nos falta o es tibia, vayamos a ella, a María, y ella nos enseñará, nos entenderá, nos hará sentir la fe.

Desde esa mañana (\ldots) la historia ya no es lo misma. Esa mañana cambió la historia. La hora en que la muerte parecía triunfar, en realidad se revela como el momento de su derrota. Incluso esa pesada roca, colocada ante el sepulcro, no pudo resistir. Y desde ese amanecer del primer día después del sábado, cada lugar donde la vida está oprimida, cada espacio en el que dominan la violencia, la guerra, la miseria, donde el hombre es humillado y pisoteado, en ese lugar todavía se puede reavivar la esperanza de la vida.

El Evangelio dice que Pedro entró el primero en el sepulcro y vio las sábanas por el suelo y el sudario envuelto en un lugar separado. Luego también entró el otro discípulo, quien –dice el Evangelio– \textquote{vio y creyó} (\textit{Jn} 20, 8). Este par de verbos es muy importante: ver y creer. A lo largo del Evangelio de Juan, se dice que los discípulos, viendo los signos que Jesús hacía, creyeron en Él. Ver y creer. ¿De qué signos se trata? El agua transformada en vino para la boda; de algunos enfermos curados; de un ciego que recobra la vista; de una gran multitud saciada con cinco panes y dos peces; de la resurrección de su amigo Lázaro, muerto desde hacía cuatro días. En todos estos signos, Jesús revela el rostro invisible de Dios.

No es la representación de la sublime perfección divina, la que se desprende de los signos de Jesús, sino la historia de la fragilidad humana que se encuentra con la Gracia que eleva. Hay una humanidad herida que es sanada tras el encuentro con Él; hay un hombre caído que encuentra una mano tendida para aferrarse; hay el desamparo de los derrotados que descubren una esperanza de redención. Y Juan, cuando entra en el sepulcro de Jesús, lleva en los ojos y en el corazón los signos hechos por Jesús que se sumerge en el drama humano para levantarlo. Jesucristo (\ldots) no es un héroe inmune a la muerte, sino Aquel que la transforma con el don de su vida. Y ese sudario cuidadosamente doblado dice que ya no lo necesitará: la muerte ya no tiene poder sobre él.

[\ldots] ¿Es posible encontrar vida en los lugares donde reina la muerte? Sí, es posible. Entrarían ganas de decir que no, que es mejor mantenerse alejado, largarse. Sin embargo, esta es la novedad revolucionaria del Evangelio: el sepulcro vacío de Cristo se convierte en el último signo en el que brilla la victoria definitiva de la Vida. ¡Y entonces no tengamos miedo! No nos alejemos de los lugares de sufrimiento, de derrota, de muerte. Dios nos ha dado un poder mayor que todas las injusticias y las debilidades de la historia, más grande que nuestro pecado: Jesús ha vencido la muerte dando su vida por nosotros. Él nos envía a anunciar a nuestros hermanos que Él es el Resucitado, es el Señor, y nos da su Espíritu para sembrar con Él el Reino de Dios. Aquella mañana del domingo de Pascua cambió la historia: ¡Tengamos valor!

¡Cuántos sepulcros ―por así decirlo― esperan hoy nuestra visita! Cuántos heridos, incluso jóvenes, han sellado su sufrimiento \textquote{poniendo –como se dice– una piedra encima}. Con el poder del Espíritu y la Palabra de Jesús podemos mover esos pedruscos y dejar que los rayos de luz entren en esos barrancos de tinieblas.

El camino para llegar a Roma ha sido hermoso y agotador; pensad, ¡cuánto esfuerzo, pero cuánta belleza! Pero el camino de regreso a vuestros hogares, a vuestros países, a vuestras comunidades será igual de hermoso y desafiante. Seguidlo con la confianza y la energía de Juan el \textquote{discípulo amado}. Sí, el secreto está todo allí, en ser y saber que eres \textquote{amado}, \textquote{amada} por Él, ¡Jesús, el Señor, nos ama! Y que cada uno de nosotros, volviendo a casa, se lo grabe en el corazón y en la mente: Jesús, el Señor, me ama. Soy amado. Soy amada. Sentir la ternura de Jesús que me ama. Recorre con valor y alegría el camino hacia casa, recorredlo con la certeza de ser amados por Jesús. Entonces, con este amor, la vida se convierte en algo bueno, sin ansiedad, sin miedo, esa palabra que nos destruye. Sin ansiedad y sin miedo. Una carrera hacia Jesús y hacia los hermanos, con un corazón lleno de amor, de fe y de alegría. ¡Adelante!
\end{body}

\begin{patercite}
(\ldots) Cada uno de los cristianos, bebiendo en la tradición histórica y, sobre todo, en las certezas de la fe, experimenta que Cristo es el Resucitado y, por lo mismo, el perennemente Viviente. Es una experiencia profunda y completa, que no puede quedar encerrada en el ámbito exclusivamente personal, sino que exige necesariamente difundirse: como la luz que se irradia; como la levadura que hace fermentar la masa del pan.

El auténtico cristiano es constitucionalmente un \textquote{Evangelio vivo}. No es, pues, el perezoso discípulo de una doctrina lejana en el tiempo y extraña a la realidad que vive; no es el mediocre repetidor de fórmulas carentes de garra sino el convencido y tenaz defensor de la contemporaneidad de Cristo y de la incesante novedad del Evangelio, siempre dispuesto, ante cualquiera y en todo momento, a dar razón de la esperanza que alimenta en el corazón (cf. \emph{1Pe} 3, 15).

El testimonio, como subrayaba mi predecesor Pablo VI, \textquote{es un elemento esencial, generalmente el primero, de la evangelización} (\emph{Evangelii nuntiandi}, 21). En nuestra época es particularmente urgente, dada la desorientación de los espíritus y el eclipse de los valores, que van configurando una crisis, que aparece cada vez más claramente como crisis total de civilización.

El hombre contemporáneo, embriagado por las conquistas materiales y, sin embargo, preocupado por las consecuencias destructoras que amenazan derivarse de esas conquistas, tiene necesidad de certezas absolutas, de horizontes capaces de resistir a la corrosión del tiempo. Insatisfecho o defraudado por el vagabundeo entre los meandros de sistemas ideológicos que lo alejan de sus aspiraciones más profundas, busca la verdad, busca la luz. Frecuentemente, quizá sin tener plena conciencia de ello, busca a Cristo.

Con la amargura de quien ha caminado en vano por los senderos de diferentes fórmulas culturales, el hombre de nuestro tiempo, según una aguda observación de Pablo VI, \textquote{escucha a los que dan testimonio más gustosamente que a los maestros, o si escucha a los maestros es porque dan testimonio} (\emph{AAS} 66, 1974, pág. 568; \emph{L'Osservatore Romano,} Edición en Lengua Española, 6 de octubre de 1974, pág. 3).

\textbf{San Juan Pablo II}, papa, \textit{Catequesis} audiencia general, 25 de abril de 1984, n. 2c-3. 
\end{patercite}

\newsection 
\section{Homilías}

\subsection{San Pablo VI, papa}

\subsubsection{Homilía (1970): El mayor acontecimiento}

\src{Parroquia de San Giorgio en Casa Palocco.\\29 de marzo de 1970.}

\begin{body}
[\ldots] Felices Pascuas: sea verdaderamente un gran y hermoso día. 

\ltr[¡]{H}{a} resucitado!: una frase que se imprimirá en la memoria como lo más grande del mundo, el acontecimiento más extraordinario de la historia. El hombre de hoy está acostumbrado a tener noticias de la conquista del espacio, de los maravillosos descubrimientos de la ciencia, de nuevos inventos. Pero saber que la vida, que nuestra existencia resurge, es algo mucho más asombroso y bello. Lo sabe bien quién ha estado enfermo y ha sido sanado, quién ha conocido las tinieblas de la guerra y ha encontrado la paz. La Pascua es la fiesta de la vida, la fiesta de la Resurrección, de la victoria sobre la muerte. Es el nuevo orden que el Señor quiere instaurar en la humanidad, y no es solo un hecho personal. El Señor ha resucitado por cada uno de nosotros, que estamos todos muriendo por la fragilidad de nuestra naturaleza. 

La vida de Jesús fue tal que el alma dominaba la materia, mientras que nosotros estamos fuertemente condicionados por la composición de nuestro cuerpo. El cuerpo está destinado a convertirse a su vez en un instrumento del alma, porque así lo ha establecido el Señor. Estamos hechos para vivir para siempre. Cuando una madre da a luz a un hijo, le da al mundo algo nuevo que nunca terminará. La vida es sagrada y debemos protegerla desde el útero. 

Cristo resucitó y todos los que creen en él resucitarán. Es necesario estar en convencida armonía con él, hacer como transfusión de la vida de Cristo a la nuestra. Si podemos estar en comunicación con esta fuente de vida, somos salvados. Si este hilo de conexión se rompe, estamos condenados. Estar con Cristo: eso es lo que significa ser cristiano. 

Nuestra Resurrección se realiza a través de tres fases. La primera es el Bautismo, a través del cual se infunde en la criatura como un alma nueva, el Espíritu Santo, la Gracia, una comunicación invisible pero real. Es un regalo que el cuerpo no ve, pero el alma sí. La segunda fase consiste en la coherencia, en la fidelidad. Debemos escuchar la Palabra del Señor, debemos convertirnos en discípulos, seguidores, creyentes. Después de todo, no hay personas felices en el mundo como los cristianos, si es que realmente lo son, porque siempre tienen la alegría de la Pascua en sus corazones. 

Jesús llamó a todos: al niño, al trabajador, al pobre. Derramó felicidad, gozo, alegría en el mundo. Tened siempre el alma llena de la alegría de Cristo. Después de esta fase, de la vida nueva, de la vida cristiana, estará la tercera: nuestra Resurrección. Es la Parusía, la última aparición después de nuestra muerte. Los cementerios se abrirán, los muertos resucitarán, la vida se reanudará animada por el alma inmortal. 

En una palabra: valentía, esperanza, alegría, la promesa de ser verdaderamente cristianos y la gratitud al Señor por hacernos vivir la Pascua, preludio de nuestra vida eterna.
\end{body}

\begin{patercite}
En estas jornadas pascuales (\ldots) adquiere un valor de gran actualidad la advertencia de
San Pablo: \textquote{Alejad la vieja levadura para ser nueva masa} (\emph{1Cor} 5, 7).

Cuanto más se ponen de relieve los caracteres contrastantes del tiempo presente, tanto más nos damos cuenta de que ésta es la hora de los cristianos auténticos, fuertes en la fe, audaces en la esperanza, generosos en la caridad, ardientes, por esto, en \textquote{dar testimonio de Cristo}, como dice también el nuevo \emph{Código de Derecho Canónico} (can. 225, par. 2), a propósito de los deberes de los laicos.

Ésta es la hora en que muchos de nuestros hermanos en la fe pagan a muy caro precio su testimonio. Son los mártires de los tiempos modernos, oprimidos por sistemas totalitarios en el ejercicio de la más elemental libertad de profesar abiertamente la fe religiosa. Con su cúmulo de sacrificios y de privaciones, con su intrepidez, constituyen una advertencia y un ejemplo. Quisiera que, lo mismo que ellos, cada uno de vosotros (\ldots) hiciese propia, con renovado fervor, la proclamación de Pedro: Cristo ha resucitado y yo soy testigo de ello.

Éste es el deseo que de corazón quiero presentar a todos con mi afectuosa bendición apostólica.

\textbf{San Juan Pablo II}, papa, \textit{Catequesis} audiencia general, 25 de abril de 1984, n. 4. 
\end{patercite}

\newsection

\subsection{San Juan Pablo II, papa}

\subsubsection{Urbi et Orbi (1980): Cristo ha resucitado}

\src{Mensaje Urbi et Orbi, \\Domingo de Resurrección, 6 de abril de 1980.}

\begin{body}
1. \textquote{\ldots y vio que la piedra había sido removida} (\textit{Jn} 20, 1).

\ltr{E}{n} la anotación de los acontecimientos del día que siguió a aquel sábado, estas palabras tienen un significado clave. Al lugar donde había sido puesto Jesús, la tarde del viernes, llega María Magdalena, llegan las otras mujeres. Jesús había sido colocado \textit{en una tumba} nueva, excavada en la roca, en la cual nadie había sido sepultado anteriormente. La tumba se hallaba a los pies del Gólgota, allí donde Jesús crucificado expiró, después de que el centurión le traspasara el costado con la lanza para constatar con certeza la realidad de su muerte. Jesús había sido envuelto en lienzos por las manos caritativas y afectuosas de las piadosas mujeres que, junto con su madre y con Juan, el discípulo predilecto, habían asistido a su extremo sacrificio. Pero, dado que caía rápidamente la tarde e iniciaba el sábado de pascua, las generosas y amorosas discípulas se vieron obligadas a dejar la unción del cuerpo santo y martirizado de Cristo para la próxima ocasión, apenas la ley religiosa de Israel lo permitiese.

Se dirigen pues al sepulcro, el día siguiente al sábado, temprano, es decir, \textit{al romper el día}, preocupadas de cómo remover la gran piedra que había sido puesta a la entrada del sepulcro, el cual además había sido sellado. Y he aquí que, llegadas al lugar, vieron que la piedra había sido removida del sepulcro.

2. Aquella piedra, colocada a la entrada de la tumba, se había convertido primeramente en un mudo \textit{testigo} de la muerte del Hijo del Hombre. Con piedras así se concluía el curso de la vida de tantos hombres de entonces en el cementerio de Jerusalén; más aún, el ciclo de la vida de todos los hombres en los cementerios de la tierra.

Bajo el peso de la losa sepulcral, tras su barrera imponente, se cumple en el silencio del sepulcro la obra de la muerte, es decir, el hombre salido del polvo se transforma lentamente en polvo (cf. \textit{Gén} 3, 19). La piedra puesta la tarde del Viernes Santo \textit{sobre la tumba de Jesús}, se ha convertido, como todas las losas sepulcrales, en el testigo mudo de la muerte del Hombre, del Hijo del Hombre.

¿Qué testimonia esta losa, el día después del sábado, en las primeras horas del día? ¿Qué nos dice? ¿Qué \textit{anuncia la piedra removida del sepulcro?}

En el Evangelio no hay una respuesta humana adecuada. No aparece en los labios de María de Magdala. Cuando asustada, por la ausencia del cuerpo de Jesús en la tumba, esta mujer corre a avisar a Simón Pedro y al otro discípulo al que Jesús amaba (cf. \textit{Jn} 20, 2), su \textit{lenguaje humano} encuentra solamente estas palabras para expresar lo sucedido: \textquote{Han tomado al Señor del monumento y no sabemos dónde lo han puesto} (\textit{Jn} 20, 2).

También Simón Pedro y el otro discípulo se dirigieron de prisa al sepulcro; y Pedro, entrando dentro, vio las vendas por tierra, y el sudario que había sido puesto sobre la cabeza de Jesús, al lado (cf. \textit{Jn} 20, 7).

Entonces entró también el otro discípulo, vio y creyó; \textquote{aún \textit{no se habían dado cuenta} de la Escritura, según la cual \textit{era preciso que Él resucitase de entre los muertos}} (\textit{Jn} 20, 9).

Vieron y comprendieron que los hombres no habían logrado derrotar a Jesús con la losa sepulcral, sellándola con la señal de la muerte.

3. La iglesia que hoy, como cada año, termina su triduo pascual con el Domingo de Resurrección, canta con alegría las palabras del antiguo: Salmo: 

textquote{Alabad a Yavé porque es bueno, porque es eterna su misericordia. Diga la Casa de Israel: Porque es eterna. su misericordia\ldots La diestra de Yavé ha sido ensalzada; la diestra de Yavé ha hecho proezas\ldots No moriré, sino que viviré para poder narrar las gestas de Yavé\ldots La piedra que rechazaron los constructores ha sido puesta por cabecera angular\ldots Obra de Yavé es ésta, y es admirable a nuestros ojos} (\textit{Sal} 117 [118], 1-2; 16-17; 22-23).

Los artífices de la muerte del Hijo del Hombre, para mayor seguridad, \textquote{pusieron guardia al sepulcro después de haber sellado la piedra} (\textit{Mt} 27, 66).

Muchas veces los constructores del Mundo, por el cual Cristo quiso morir han tratado de poner una piedra definitiva sobre su tumba.

Pero la piedra permanece siempre removida de su sepulcro; la piedra, testigo de la muerte, se ha convertido en testigo de la Resurrección: \textquote{la diestra de Yavé ha hecho proezas} (\textit{Sal} 117 [118], 16).

4. La Iglesia anuncia siempre y de nuevo la Resurrección de Cristo. La Iglesia repite con alegría a los hombres las palabras de los ángeles y de las mujeres pronunciadas en aquella radiante mañana en la que la muerte fue vencida.

La Iglesia anuncia que \textit{está vivo Aquel que se ha convertido en nuestra Pascua}. Aquel que ha muerto en la cruz, revela la plenitud de la \textit{Vida}.

Este mundo que por desgracia hoy, de diversas maneras, parece querer la \textquote{muerte de Dios}, escuche el mensaje de la Resurrección.

Todos vosotros que anunciáis \textquote{la muerte de Dios}, que tratáis de expulsar a Dios del mundo humano, deteneos y pensad que \textquote{\textit{la muerte de Dios}} puede comportar fatalmente \textquote{\textit{la muerte del hombre}}.

Cristo ha resucitado para que el hombre encuentre el auténtico significado de la existencia, para que el \textit{hombre viva en plenitud su propia vida}, para que el hombre, que viene \textit{de Dios}, viva \textit{en Dios}.

Cristo ha resucitado. El \textit{es la piedra angular}. Ya entonces se quiso rechazarlo y vencerlo con la piedra vigilada y sellada del sepulcro. Pero aquella piedra fue removida. Cristo ha resucitado.

No rechacéis a Cristo vosotros, los que construís el mundo humano.

No lo rechacéis vosotros, los que, de cualquier manera y en cualquier sector, construís el mundo de \textit{hoy} y de \textit{mañana}: el mundo de la cultura y de la civilización, el mundo de la economía y de la política, el mundo de la ciencia y de la información. \textit{Vosotros que construís el mundo de la paz\ldots, ¿o de la guerra?} Vosotros que construís el mundo del orden\ldots, ¿o del terror? No rechacéis a Cristo: ¡Él es la piedra angular!

Que no lo rechace ningún hombre, porque cada uno es responsable de su destino: constructor o destructor de la propia existencia.

Cristo resucitó ya antes de que el Ángel removiera la losa sepulcral. Él se reveló después como piedra angular, sobre la cual se construye la historia de la humanidad entera y la de cada uno de nosotros.

5. ¡Queridos hermanos y hermanas! Con sincera alegría acojamos este día tan esperado. Con viva alegría compartamos el mensaje pascual todos los que acogemos a Cristo como piedra angular.

En virtud de esta piedra angular que une, construyamos nuestra común esperanza con los hermanos en Cristo de Oriente y de Occidente, con quienes no nos une aún la plena comunión y la perfecta unidad.

Aceptad, queridos hermanos, nuestro beso pascual de paz y de amor. Cristo resucitado despierte en nosotros un deseo todavía más profundo de esta unidad por la cual oró la víspera de su pasión.

No cesemos de orar por ella en unión con Él. Pongamos nuestra confianza en la fuerza de la cruz y de la Resurrección; ¡tal fuerza es más poderosa que la debilidad de toda división humana!

Amadísimos hermanos: ¡Os anunció un gran gozo: Aleluya!

6. La Iglesia se acerca hoy a cada hombre con el \textit{deseo pascual}: el deseo de construir el mundo sobre Cristo: deseo que hace extensivo a toda la familia humana.

Ojalá acojan este deseo los que comparten con nosotros el mensaje de la resurrección y de la alegría pascual, y también los que, por desgracia, no lo comparten. Cristo, \textquote{nuestra Pascua}, no cesa de ser peregrino con nosotros en el camino de la historia, y cada uno puede encontrarlo, porque Él no cesa de ser el Hermano del hombre en cada época y en cada momento.

En \textit{su} nombre os hablo hoy a todos, y a todos os dirijo mi más ferviente y santa felicitación.
\end{body}


\newsection
\subsection{Benedicto XVI, papa}

\subsubsection{Urbi et Orbi (2006): Promesa cumplida}
\src{16 de abril del 2006.}

\begin{body}
\textit{Christus resurrexit!}- ¡Cristo ha resucitado! 

\ltr{L}{a} gran Vigilia de esta noche nos ha hecho revivir el acontecimiento decisivo y siempre actual de la Resurrección, misterio central de la fe cristiana. En las iglesias se han encendido innumerables cirios pascuales para simbolizar la luz de Cristo que ha iluminado e ilumina a la humanidad, venciendo para siempre las tinieblas del pecado y del mal. Y hoy resuenan con fuerza las palabras que asombraron a las mujeres que habían ido la madrugada del primer día de la semana al sepulcro donde habían puesto el cuerpo de Cristo, bajado apresuradamente de la cruz. Tristes y desconsoladas por la pérdida de su Maestro, encontraron apartada la gran piedra y, al entrar, no hallaron su cuerpo. Mientras estaban allí, perplejas y confusas, dos hombres con vestidos resplandecientes les sorprendieron, diciendo: \textquote{¿Por qué buscáis entre los muertos al que vive? No está aquí, ha resucitado} (\textit{Lc} 24, 5-6) \textquote{\textit{Non est hic, sed resurrexit}} (\textit{Lc} 24, 6). Desde aquella mañana, estas palabras siguen resonando en el universo como anuncio perenne, e impregnado a la vez de infinitos y siempre nuevos ecos, que atraviesa los siglos. 

\textquote{No está aquí\ldots ha resucitado}. Los mensajeros celestes comunican ante todo que Jesús \textquote{no está aquí}: el Hijo de Dios no ha quedado en el sepulcro, porque no podía permanecer bajo el dominio de la muerte (cf. \textit{Hch} 2, 24) y la tumba no podía retener \textquote{al que vive} (\textit{Ap} 1, 18), al que es la fuente misma de la vida. Porque, del mismo modo que Jonás estuvo en el vientre del cetáceo, también Cristo crucificado quedó sumido en el seno de la tierra (cf. \textit{Mt} 12, 40) hasta terminar un sábado. Aquel sábado fue ciertamente \textquote{un día solemne}, como escribe el evangelista Juan (19, 31), el más solemne de la historia, porque, en él, el \textquote{Señor del sábado} (\textit{Mt} 12, 8) llevó a término la obra de la creación (cf. \textit{Gn} 2, 1-4a), elevando al hombre y a todo el cosmos a la gloriosa libertad de los hijos de Dios (cf. \textit{Rm} 8, 21). Cumplida esta obra extraordinaria, el cuerpo exánime ha sido traspasado por el aliento vital de Dios y, rotas las barreras del sepulcro, ha resucitado glorioso. Por esto los ángeles proclaman \textquote{no está aquí}: ya no se le puede encontrase en la tumba. Ha peregrinado en la tierra de los hombres, ha terminado su camino en la tumba, como todos, pero ha vencido a la muerte y, de modo absolutamente nuevo, por un puro acto de amor, ha abierto la tierra de par en par hacia el Cielo. 

Su resurrección, gracias al Bautismo que nos \textquote{incorpora} a Él, es nuestra resurrección. Lo había preanunciado el profeta Ezequiel: \textquote{Yo mismo abriré vuestros sepulcros, y os haré salir de vuestros sepulcros, pueblo mío, y os traeré a la tierra de Israel} (\textit{Ez} 37, 12). Estas palabras proféticas adquieren un valor singular en el día de Pascua, porque hoy se cumple la promesa del Creador; hoy, también en esta época nuestra marcada por la inquietud y la incertidumbre, revivimos el acontecimiento de la resurrección, que ha cambiado el rostro de nuestra vida, ha cambiado la historia de la humanidad. Cuantos permanecen todavía bajo las cadenas del sufrimiento y la muerte, aguardan, a veces de modo inconsciente, la esperanza de Cristo resucitado. 

Que el espíritu del Resucitado traiga consuelo y seguridad [\ldots]\anote{id17} 

Que el Señor Resucitado haga sentir por todas partes su fuerza de vida, de paz y de libertad. Las palabras con las que el ángel confortó los corazones atemorizados de las mujeres en la mañana de Pascua, se dirigen a todos: \textquote{¡No tengáis miedo!\ldots No está aquí. Ha resucitado} (\textit{Mt} 28, 5-6). Jesús ha resucitado y nos da la paz; Él mismo es la paz. Por eso la Iglesia repite con firmeza: \textquote{Cristo ha resucitado – \textit{Christós anésti}}. Que la humanidad del tercer milenio no tenga miedo de abrirle el corazón. Su Evangelio sacia plenamente el anhelo de paz y de felicidad que habita en todo corazón humano. Cristo ahora está vivo y camina con nosotros. ¡Inmenso misterio de amor! \textit{Christus resurrexit, quia Deus caritas est! Alleluia.}
\end{body}


\newpage 


\subsubsection{Homilía (2009): No manda en ti la muerte}

\src{Domingo de Pascua, 12 de abril de 2009.}

\begin{body}
\textquote{\textit{Ha sido inmolado Cristo, nuestra Pascua}} (\textit{1 Co} 5, 7). 

\ltr{R}{esuena} en este día la exclamación de san Pablo que hemos escuchado en la \textbf{segunda lectura}, tomada de la primera \textit{Carta a los Corintios}. Un texto que se remonta a veinte años apenas después de la muerte y resurrección de Jesús y que, no obstante, contiene en una síntesis impresionante –como es típico de algunas expresiones paulinas– la plena conciencia de la novedad cristiana. El símbolo central de la historia de la salvación –el cordero pascual– se identifica aquí con Jesús, llamado precisamente \textquote{nuestra Pascua}. La Pascua judía, memorial de la liberación de la esclavitud de Egipto, prescribía el rito de la inmolación del cordero, un cordero por familia, según la ley mosaica. En su pasión y muerte, Jesús se revela como el Cordero de Dios \textquote{inmolado} en la cruz para quitar los pecados del mundo; fue muerto justamente en la hora en que se acostumbraba a inmolar los corderos en el Templo de Jerusalén. El sentido de este sacrificio suyo, lo había anticipado Él mismo durante la Última Cena, poniéndose en el lugar –bajo las especies del pan y el vino– de los elementos rituales de la cena de la Pascua. Así, podemos decir que Jesús, realmente, ha llevado a cumplimiento la tradición de la antigua Pascua y la ha transformado en \textit{su} Pascua. 

A partir de este nuevo sentido de la fiesta pascual, se comprende también la interpretación de san Pablo sobre los \textquote{ázimos}. El Apóstol se refiere a una antigua costumbre judía, según la cual en la Pascua había que limpiar la casa hasta de las migajas de pan fermentado. Eso formaba parte del recuerdo de lo que había pasado con los antepasados en el momento de su huída de Egipto: teniendo que salir a toda prisa del país, llevaron consigo solamente panes sin levadura. Pero, al mismo tiempo, \textquote{los ázimos} eran un símbolo de purificación: eliminar lo viejo para dejar espacio a lo nuevo. Ahora, como explica san Pablo, también esta antigua tradición adquiere un nuevo sentido, precisamente a partir del nuevo \textquote{éxodo} que es el paso de Jesús de la muerte a la vida eterna. Y puesto que Cristo, como el verdadero Cordero, se ha sacrificado a sí mismo por nosotros, también nosotros, sus discípulos –gracias a Él y por medio de Él– podemos y debemos ser \textquote{masa nueva}, \textquote{ázimos}, liberados de todo residuo del viejo fermento del pecado: ya no más malicia y perversidad en nuestro corazón.

\textquote{Así, pues, celebremos la Pascua\ldots con los panes ázimos de la sinceridad y la verdad}. Esta exhortación de san Pablo con que termina la breve lectura que se ha proclamado hace poco, [resuena aún más intensamente en el contexto del Año Paulino]. Queridos hermanos y hermanas, acojamos la invitación del Apóstol; abramos el corazón a Cristo muerto y resucitado para que nos renueve, para que nos limpie del veneno del pecado y de la muerte y nos infunda la savia vital del Espíritu Santo: la vida divina y eterna. En la secuencia pascual, como haciendo eco a las palabras del Apóstol, hemos cantado: \textquote{\textit{Scimus Christum surrexisse a mortuis vere}} – sabemos que estás resucitado, la muerte en ti no manda. Sí, éste es precisamente el núcleo fundamental de nuestra profesión de fe; éste es hoy el grito de victoria que nos une a todos. Y si Jesús ha resucitado, y por tanto está vivo, ¿quién podrá jamás separarnos de Él? ¿Quién podrá privarnos de su amor que ha vencido al odio y ha derrotado la muerte? Que el anuncio de la Pascua se propague por el mundo con el jubiloso canto del \textit{aleluya}. Cantémoslo con la boca, cantémoslo sobre todo con el corazón y con la vida, con un estilo de vida \textquote{ázimo}, simple, humilde, y fecundo de buenas obras. \textquote{\textit{Surrexit Christus spes mea: precedet vos in Galileam}} – ¡Resucitó de veras mi esperanza! Venid a Galilea, el Señor allí aguarda. El Resucitado nos precede y nos acompaña por las vías del mundo. Él es nuestra esperanza, Él es la verdadera paz del mundo. Amén.
\end{body}

\begin{patercite}
Cuando el cristiano, en sintonía con la voz orante de Israel, canta el salmo 117, experimenta en su interior una emoción particular. En efecto, encuentra en este himno, de intensa índole litúrgica, dos frases que resonarán dentro del Nuevo Testamento con una nueva tonalidad. La primera se halla en el versículo 22: \textquote{La piedra que desecharon los arquitectos es ahora la piedra angular}. Jesús cita esta frase, aplicándola a su misión de muerte y de gloria, después de narrar la parábola de los viñadores homicidas (cf.\emph{Mt} 21, 42). También la recoge san Pedro en los \emph{Hechos de los Apóstoles}: \textquote{Este Jesús es la piedra que vosotros, los constructores, habéis desechado y que se ha convertido en piedra angular. Porque no hay bajo el cielo otro nombre dado a los hombres por el que nosotros debamos salvarnos} (\emph{Hch} 4, 11-12). San Cirilo de Jerusalén comenta: \textquote{Afirmamos que el Señor Jesucristo es uno solo, para que la filiación sea única; afirmamos que es uno solo, para que no pienses que existe otro (\ldots). En efecto, le llamamos \emph{piedra}, no inanimada ni cortada por manos humanas, sino \emph{piedra angular}, porque quien crea en ella \emph{no quedará defraudado}} (\emph{Le Catechesi}, Roma 1993, pp. 312-313).

La segunda frase que el Nuevo Testamento toma del salmo 117 es la que cantaba la muchedumbre en la solemne entrada mesiánica de Cristo en Jerusalén: \textquote{¡Bendito el que viene en nombre del Señor!} (\emph{Mt} 21, 9; cf. \emph{Sal} 117, 26). La aclamación está enmarcada por un \textquote{Hosanna} que recoge la invocación hebrea \emph{hoshia' na':} \textquote{sálvanos}.
\end{patercite}

\newpage 


\subsubsection{Regina Caeli (2009): He resucitado y estoy contigo}

\src{Palacio pontificio de Castelgandolfo. \\Lunes de la octava de Pascua, 13 de abril de 2009.}

\begin{body}
\ltr{E}{n} estos días pascuales oiremos resonar a menudo las palabras de Jesús: \textquote{He resucitado y estoy siempre contigo}. La Iglesia, haciéndose eco de este anuncio, proclama con júbilo: \textquote{Era verdad, ha resucitado el Señor, aleluya. A él la gloria y el poder por toda la eternidad}. Toda la Iglesia en fiesta manifiesta sus sentimientos cantando: \textquote{Este es el día en que actuó el Señor}. En efecto, al resucitar de entre los muertos, Jesús inauguró su día eterno y también abrió la puerta de nuestra alegría. \textquote{No he de morir –dice–, viviré}. El Hijo del hombre crucificado, piedra desechada por los arquitectos, es ahora el sólido cimiento del nuevo edificio espiritual, que es la Iglesia, su Cuerpo místico. El pueblo de Dios, cuya Cabeza invisible es Cristo, está destinado a crecer a lo largo de los siglos, hasta el pleno cumplimiento del plan de la salvación. Entonces toda la humanidad se incorporará a él y toda realidad existente participará en su victoria definitiva. Entonces –escribe san Pablo–, él será \textquote{la plenitud de todas las cosas} (\textit{Ef} 1, 23) y \textquote{Dios será todo en todos} (\textit{1 Co} 15, 28). 

Por tanto, la comunidad cristiana se alegra porque la resurrección del Señor nos garantiza que el plan divino de la salvación se cumplirá con seguridad, no obstante toda la oscuridad de la historia. Precisamente por eso su Pascua es en verdad nuestra esperanza. Y nosotros, resucitados con Cristo mediante el Bautismo, debemos seguirlo ahora fielmente con una vida santa, caminando hacia la Pascua eterna, sostenidos por la certeza de que las dificultades, las luchas, las pruebas y los sufrimientos de nuestra existencia, incluida la muerte, ya no podrán separarnos de él y de su amor. Su resurrección ha creado un puente entre el mundo y la vida eterna, por el que todo hombre y toda mujer pueden pasar para llegar a la verdadera meta de nuestra peregrinación terrena. 

\textquote{He resucitado y estoy siempre contigo}. Esta afirmación de Jesús se realiza sobre todo en la Eucaristía; en toda celebración eucarística la Iglesia, y cada uno de sus miembros, experimentan su presencia viva y se benefician de toda la riqueza de su amor. En el sacramento de la Eucaristía está presente el Señor resucitado y, lleno de misericordia, nos purifica de nuestras culpas; nos alimenta espiritualmente y nos infunde vigor para afrontar las duras pruebas de la existencia y para luchar contra el pecado y el mal. Él es el apoyo seguro de nuestra peregrinación hacia la morada eterna del cielo. 

La Virgen María, que vivió junto a su divino Hijo cada fase de su misión en la tierra, nos ayude a acoger con fe el don de la Pascua y nos convierta en testigos felices, fieles y gozosos del Señor resucitado.
\end{body}

\subsubsection{Urbi et Orbi (2009): Nuestra esperanza}

\src{12 de abril del 2009.}

\begin{body}
\textit{Queridos hermanos y hermanas de Roma y del mundo entero:} 

\ltr{A}{ } todos vosotros dirijo de corazón la felicitación pascual con las palabras de san Agustín: \textquote{\textit{Resurrectio Domini, spes nostra}}, \textquote{la resurrección del Señor es nuestra esperanza} (\textit{Sermón} 261,1). Con esta afirmación, el gran Obispo explicaba a sus fieles que Jesús resucitó para que nosotros, aunque destinados a la muerte, no desesperáramos, pensando que con la muerte se acaba totalmente la vida; Cristo ha resucitado para darnos la esperanza (cf. \textit{ibíd}.). 

En efecto, una de las preguntas que más angustian la existencia del hombre es precisamente ésta: ¿qué hay después de la muerte? Esta solemnidad nos permite responder a este enigma afirmando que la muerte no tiene la última palabra, porque al final es la Vida la que triunfa. Nuestra certeza no se basa en simples razonamientos humanos, sino en un dato histórico de fe: Jesucristo, crucificado y sepultado, ha resucitado con su cuerpo glorioso. Jesús ha resucitado para que también nosotros, creyendo en Él, podamos tener la vida eterna. Este anuncio está en el corazón del mensaje evangélico. San Pablo lo afirma con fuerza: \textquote{Si Cristo no ha resucitado, nuestra predicación carece de sentido y vuestra fe lo mismo}. Y añade: \textquote{Si nuestra esperanza en Cristo acaba con esta vida, somos los hombres más desgraciados} (\textit{1 Co} 15, 14. 19). Desde la aurora de Pascua una nueva primavera de esperanza llena el mundo; desde aquel día nuestra resurrección ya ha comenzado, porque la Pascua no marca simplemente un momento de la historia, sino el inicio de una condición nueva: Jesús ha resucitado no porque su recuerdo permanezca vivo en el corazón de sus discípulos, sino porque Él mismo vive en nosotros y en Él ya podemos gustar la alegría de la vida eterna.

Por tanto, la resurrección no es una teoría, sino una realidad histórica revelada por el Hombre Jesucristo mediante su \textquote{pascua}, su \textquote{paso}, que ha abierto una \textquote{nueva vía} entre la tierra y el Cielo (cf. \textit{Hb} 10, 20). No es un mito ni un sueño, no es una visión ni una utopía, no es una fábula, sino un acontecimiento único e irrepetible: Jesús de Nazaret, hijo de María, que en el crepúsculo del Viernes fue bajado de la cruz y sepultado, ha salido vencedor de la tumba. En efecto, al amanecer del primer día después del sábado, Pedro y Juan hallaron la tumba vacía. Magdalena y las otras mujeres encontraron a Jesús resucitado; lo reconocieron también los dos discípulos de Emaús en la fracción del pan; el Resucitado se apareció a los Apóstoles aquella tarde en el Cenáculo y luego a otros muchos discípulos en Galilea. 

El anuncio de la resurrección del Señor ilumina las zonas oscuras del mundo en que vivimos. Me refiero particularmente al materialismo y al nihilismo, a esa visión del mundo que no logra transcender lo que es constatable experimentalmente, y se abate desconsolada en un sentimiento de la nada, que sería la meta definitiva de la existencia humana. En efecto, si Cristo no hubiera resucitado, el \textquote{vacío} acabaría ganando. Si quitamos a Cristo y su resurrección, no hay salida para el hombre, y toda su esperanza sería ilusoria. Pero, precisamente hoy, irrumpe con fuerza el anuncio de la resurrección del Señor, que responde a la pregunta recurrente de los escépticos, referida también por el libro del Eclesiastés:  \textquote{¿Acaso hay algo de lo que se pueda decir: \textquote{Mira, esto es nuevo?}} (\textit{Qo} 1,10). Sí, contestamos: todo se ha renovado en la mañana de Pascua. \textquote{Lucharon vida y muerte en singular batalla y, muerto el que es Vida, triunfante se levanta} (Secuencia Pascual). Ésta es la novedad. Una novedad que cambia la existencia de quien la acoge, como sucedió a lo santos. Así, por ejemplo, le ocurrió a san Pablo. 

\txtsmall{[En el contexto del Año Paulino, hemos tenido ocasión muchas veces de meditar sobre la experiencia del gran Apóstol.]} Saulo de Tarso, el perseguidor encarnizado de los cristianos, encontró a Cristo resucitado en el camino de Damasco y fue \textquote{conquistado} por Él. El resto lo sabemos. A Pablo le sucedió lo que más tarde él escribirá a los cristianos de Corinto: \textquote{El que vive con Cristo, es una criatura nueva; lo viejo ha pasado, ha llegado lo nuevo} (\textit{2 Co} 5, 17). Fijémonos en este gran evangelizador, que con el entusiasmo audaz de su acción apostólica, llevó el Evangelio a muchos pueblos del mundo de entonces. Su enseñanza y su ejemplo nos impulsan a buscar al Señor Jesús. Nos animan a confiar en Él, porque ahora el sentido de la nada, que tiende a intoxicar la humanidad, ha sido vencido por la luz y la esperanza que surgen de la resurrección. Ahora son verdaderas y reales las palabras del Salmo: \textquote{Ni la tiniebla es oscura para ti, la noche es clara como el día} (139[138], 12). Ya no es la nada la que envuelve todo, sino la presencia amorosa de Dios. Más aún, hasta el reino mismo de la muerte ha sido liberado, porque también al \textquote{abismo} ha llegado el Verbo de la vida, aventado por el soplo del Espíritu (v. 8). 

Aunque es verdad que la muerte ya no tiene poder sobre el hombre y el mundo, quedan todavía muchos, demasiados signos de su antiguo dominio. Aunque Cristo, por la Pascua, ha extirpado la raíz del mal, necesita hombres y mujeres que lo ayuden siempre y en todo lugar a afianzar su victoria con sus mismas armas: las armas de la justicia y de la verdad, de la misericordia, del perdón y del amor. [\ldots]\anote{id18} En un tiempo de carestía global de alimentos, de desbarajuste financiero, de pobrezas antiguas y nuevas, de cambios climáticos preocupantes, de violencias y miserias que obligan a muchos a abandonar su tierra buscando una supervivencia menos incierta, de terrorismo siempre amenazante, de miedos crecientes ante un porvenir problemático, es urgente descubrir nuevamente perspectivas capaces de devolver la esperanza. Que nadie se arredre en esta batalla pacífica comenzada con la Pascua de Cristo, el cual, lo repito, busca hombres y mujeres que lo ayuden a afianzar su victoria con sus mismas armas, las de la justicia y la verdad, la misericordia, el perdón y el amor.

\textquote{\textit{Resurrectio Domini, spes nostra}}. La resurrección de Cristo es nuestra esperanza. La Iglesia proclama hoy esto con alegría: anuncia la esperanza, que Dios ha hecho firme e invencible resucitando a Jesucristo de entre los muertos; comunica la esperanza, que lleva en el corazón y quiere compartir con todos, en cualquier lugar, especialmente allí donde los cristianos sufren persecución a causa de su fe y su compromiso por la justicia y la paz; invoca la esperanza capaz de avivar el deseo del bien, también y sobre todo cuando cuesta. Hoy la Iglesia canta \textquote{el día en que actuó el Señor} e invita al gozo. Hoy la Iglesia ora, invoca a María, Estrella de la Esperanza, para que conduzca a la humanidad hacia el puerto seguro de la salvación, que es el corazón de Cristo, la Víctima pascual, el Cordero que \textquote{ha redimido al mundo}, el Inocente que nos \textquote{ha reconciliado a nosotros, pecadores, con el Padre}. A Él, Rey victorioso, a Él, crucificado y resucitado, gritamos con alegría nuestro \textit{Alleluia}.
\end{body}


\subsubsection{Regina Caeli (2012): Misterio decisivo de la fe}

\src{Castelgandolfo. Lunes del Ángel, 9 de abril de 2012}

\begin{body}
\ltr[¡]{F}{eliz} día a todos vosotros! El lunes después de Pascua en muchos países es un día de vacación, en el que se puede dar un paseo en medio de la naturaleza o ir a visitar a parientes un poco lejanos para una reunión en familia. Pero quisiera que en la mente y en el corazón de los cristianos siempre estuviera presente el motivo de esta vacación, es decir, la resurrección de Jesús, el misterio decisivo de nuestra fe. De hecho, como escribe san Pablo a los Corintios, \textquote{si Cristo no ha resucitado, vana es nuestra predicación y vana también vuestra fe} (\textit{1 Co} 15, 14). Por eso, en estos días es importante releer los relatos de la resurrección de Cristo que encontramos en los cuatro Evangelios y leerlos con nuestro corazón. Se trata de relatos que, de modos diversos, presentan los encuentros de los discípulos con Jesús resucitado, y así nos permiten meditar en este acontecimiento estupendo que ha transformado la historia y da sentido a la existencia de todo hombre, de cada uno de nosotros.

Los evangelistas no describen el acontecimiento de la resurrección en cuanto tal. Ese acontecimiento permanece misterioso, no en el sentido de menos real, sino de oculto, más allá del alcance de nuestro conocimiento: como una luz tan deslumbrante que no se puede observar con los ojos, pues de lo contrario los cegaría. Los relatos comienzan, en cambio, desde que, al alba del día después del sábado, las mujeres se dirigieron al sepulcro y lo encontraron abierto y vacío. San Mateo habla también de un terremoto y de un ángel deslumbrante que corrió la gran piedra de la tumba y se sentó encima de ella (cf. \textit{Mt} 28, 2). Tras recibir del ángel el anuncio de la resurrección, las mujeres, llenas de miedo y de alegría, corrieron a dar la noticia a los discípulos, y precisamente en aquel momento se encontraron con Jesús, se postraron a sus pies y lo adoraron; y él les dijo: \textquote{No temáis; id a comunicar a mis hermanos que vayan a Galilea; allí me verán} (\textit{Mt} 28, 10). En todos los Evangelios las mujeres ocupan gran espacio en los relatos de las apariciones de Jesús resucitado, como también en los de la pasión y muerte de Jesús. En aquellos tiempos, en Israel, el testimonio de las mujeres no podía tener valor oficial, jurídico, pero las mujeres vivieron una experiencia de vínculo especial con el Señor, que es fundamental para la vida concreta de la comunidad cristiana, y esto siempre, en todas las épocas, no sólo al inicio del camino de la Iglesia.

Modelo sublime y ejemplar de esta relación con Jesús, de modo especial en su Misterio pascual, es naturalmente María, la Madre del Señor. Precisamente a través de la experiencia transformadora de la Pascua de su Hijo, la Virgen María se convierte también en Madre de la Iglesia, es decir, de cada uno de los creyentes y de toda la comunidad. A ella nos dirigimos ahora invocándola como \textit{Regina caeli}, con la oración que la tradición nos hace rezar en lugar del \textit{Ángelus} durante todo el tiempo pascual. Que María nos obtenga experimentar la presencia viva del Señor resucitado, fuente de esperanza y de paz.
\end{body}

\subsubsection{Urbi et Orbi (2012): Cristo, mi esperanza}

\src{8 de abril de 2012.}

\begin{body}

\ltr{S}{urrexit} Christus, spes mea -- \textquote{Resucitó Cristo, mi esperanza} (Secuencia pascual). Llegue a todos vosotros la voz exultante de la Iglesia, con las palabras que el antiguo himno pone en labios de María Magdalena, la primera en encontrar en la mañana de Pascua a Jesús resucitado. Ella corrió hacia los otros discípulos y, con el corazón sobrecogido, les anunció: \textquote{He visto al Señor} (\textit{Jn} 20, 18). También nosotros, que hemos atravesado el desierto de la Cuaresma y los días dolorosos de la Pasión, hoy abrimos las puertas al grito de victoria: \textquote{¡Ha resucitado! ¡Ha resucitado verdaderamente!}.

Todo cristiano revive la experiencia de María Magdalena. Es un encuentro que cambia la vida: el encuentro con un hombre único, que nos hace sentir toda la bondad y la verdad de Dios, que nos libra del mal, no de un modo superficial, momentáneo, sino que nos libra de él radicalmente, nos cura completamente y nos devuelve nuestra dignidad. He aquí por qué la Magdalena llama a Jesús \textquote{mi esperanza}: porque ha sido Él quien la ha hecho renacer, le ha dado un futuro nuevo, una existencia buena, libre del mal. \textquote{Cristo, mi esperanza}, significa que cada deseo mío de bien encuentra en Él una posibilidad real: con Él puedo esperar que mi vida sea buena y sea plena, eterna, porque es Dios mismo que se ha hecho cercano hasta entrar en nuestra humanidad. 

Pero María Magdalena, como los otros discípulos, han tenido que ver a Jesús rechazado por los jefes del pueblo, capturado, flagelado, condenado a muerte y crucificado. Debe haber sido insoportable ver la Bondad en persona sometida a la maldad humana, la Verdad escarnecida por la mentira, la Misericordia injuriada por la venganza. Con la muerte de Jesús, parecía fracasar la esperanza de cuantos confiaron en Él. Pero aquella fe nunca dejó de faltar completamente: sobre todo en el corazón de la Virgen María, la madre de Jesús, la llama quedó encendida con viveza también en la oscuridad de la noche. En este mundo, la esperanza no puede dejar de hacer cuentas con la dureza del mal. No es solamente el muro de la muerte lo que la obstaculiza, sino más aún las puntas aguzadas de la envidia y el orgullo, de la mentira y de la violencia. Jesús ha pasado por esta trama mortal, para abrirnos el paso hacia el reino de la vida. Hubo un momento en el que Jesús aparecía derrotado: las tinieblas habían invadido la tierra, el silencio de Dios era total, la esperanza una palabra que ya parecía vana. Y he aquí que, al alba del día después del sábado, se encuentra el sepulcro vacío. Después, Jesús se manifiesta a la Magdalena, a las otras mujeres, a los discípulos. La fe renace más viva y más fuerte que nunca, ya invencible, porque fundada en una experiencia decisiva: 

\begin{bodyprose}
«Lucharon vida y muerte  /  en singular batalla,  
  y, muerto el que es Vida, triunfante se levanta». 
\end{bodyprose}

Las señales de la resurrección testimonian la victoria de la vida sobre la muerte, del amor sobre el odio, de la misericordia sobre la venganza: 

\begin{bodyprose}
«Mi Señor glorioso,  /  la tumba abandonada, 
   los ángeles testigos, / sudarios y mortaja». 
\end{bodyprose}

Queridos hermanos y hermanas: si Jesús ha resucitado, entonces –y sólo entonces– ha ocurrido algo realmente nuevo, que cambia la condición del hombre y del mundo. Entonces Él, Jesús, es alguien del que podemos fiarnos de modo absoluto, y no solamente confiar en su mensaje, sino precisamente \textit{en} \textit{Él}, porque el resucitado no pertenece al \textit{pasado}, sino que \textit{está presente} hoy, vivo. Cristo es esperanza y consuelo de modo particular para las comunidades cristianas que más pruebas padecen a causa de la fe, por discriminaciones y persecuciones. Y está presente como fuerza de esperanza a través de su Iglesia, cercano a cada situación humana de sufrimiento e injusticia. 

[\ldots]\anote{id19}

Feliz Pascua a todos. 
\end{body}


\newsection
\subsection{Francisco, papa}

\subsubsection{Urbi et Orbi (2015): Inclinarse ante el misterio}

\src{Basílica Vaticana. 5 de abril de 2015.}

\begin{body}
\textit{Queridos hermanos y hermanas}

¡Feliz Pascua! ¡Jesucristo ha resucitado!

\ltr{E}{l} amor ha derrotado al odio, la vida ha vencido a la muerte, la luz ha disipado la oscuridad. Jesucristo, por amor a nosotros, se despojó de su gloria divina; se vació de sí mismo, asumió la forma de siervo y se humilló hasta la muerte, y muerte de cruz. Por esto Dios lo ha exaltado y le ha hecho Señor del universo. Jesús es el Señor.

Con su muerte y resurrección, Jesús muestra a todos la vía de la vida y la felicidad: esta vía es \textit{la humildad}, que comporta \textit{la humillación}. Este es el camino que conduce a la gloria. Sólo quien se humilla puede ir hacia los \textquote{bienes de allá arriba}, a Dios (cf. \textit{Col} 3, 1-4). El orgulloso mira \textquote{desde arriba hacia abajo}, el humilde, \textquote{desde abajo hacia arriba}.

La mañana de Pascua, Pedro y Juan, advertidos por las mujeres, corrieron al sepulcro y lo encontraron abierto y vacío. Entonces, se acercaron y se \textquote{inclinaron} para entrar en la tumba. Para entrar en el misterio hay que \textquote{inclinarse}, abajarse. Sólo quien se abaja comprende la glorificación de Jesús y puede seguirlo en su camino.

El mundo propone imponerse a toda costa, competir, hacerse valer\ldots Pero los cristianos, por la gracia de Cristo muerto y resucitado, \textit{son los brotes de otra humanidad}, en la cual tratamos de vivir al servicio de los demás, de no ser altivos, sino disponibles y respetuosos. Esto \textit{no es debilidad, sino auténtica fuerza}. Quien lleva en sí el poder de Dios, de su amor y su justicia, no necesita usar violencia, sino que habla y actúa con la fuerza de la verdad, de la belleza y del amor.

Imploremos hoy al Señor resucitado la gracia de no ceder al orgullo que fomenta la violencia y las guerras, sino de tener el valor humilde del perdón y de la paz. Pedimos a Jesús victorioso que alivie el sufrimiento de tantos hermanos nuestros perseguidos a causa de su nombre, así como de todos los que padecen injustamente las consecuencias de los conflictos y las violencias que se están produciendo, y que son tantas.

[\ldots] \anote{id20}

Pidamos paz y libertad para tantos hombres y mujeres sometidos a nuevas y antiguas formas de esclavitud por parte de personas y organizaciones criminales. Paz y libertad para las víctimas de los traficantes de droga, muchas veces aliados con los poderes que deberían defender la paz y la armonía en la familia humana. E imploremos la paz para este mundo sometido a los traficantes de armas, que se enriquecen con la sangre de hombres y mujeres.

Y que a los marginados, los presos, los pobres y los emigrantes, tan a menudo rechazados, maltratados y desechados; a los enfermos y los que sufren; a los niños, especialmente aquellos sometidos a la violencia; a cuantos hoy están de luto; y a todos los hombres y mujeres de buena voluntad, llegue la voz consoladora y curativa del Señor Jesús: \textquote{Paz a vosotros} (\textit{Lc} 24,36). \textquote{No temáis, he resucitado y siempre estaré con vosotros} (cf. \textit{Misal Romano}, Antífona de entrada del día de Pascua).
\end{body}

\subsubsection{Homilía (2018): Situarse ante la Pascua}

\src{Plaza de San Pedro. 1 de abril de 2018.}

\begin{body}
\ltr{D}{espués} de la escucha de la Palabra de Dios, de este paso del Evangelio, me nace decir tres cosas.

Primero: el \textit{anuncio}. Ahí hay un anuncio: el Señor ha resucitado. Este anuncio que desde los primeros tiempos de los cristianos iba de boca en boca; era el saludo: el Señor ha resucitado. Y las mujeres, que fueron a ungir el cuerpo del Señor, se encontraron frente a una sorpresa. La sorpresa\ldots Los anuncios de Dios son siempre sorpresas, porque nuestro Dios es el Dios de las sorpresas. Y así desde el inicio de la historia de la salvación, desde nuestro padre Abraham, Dios te sorprende: \textquote{Pero ve, ve, deja, vete de tu tierra}. Y siempre hay una sorpresa detrás de la otra. Dios no sabe hacer un anuncio sin sorprendernos. Y la sorpresa es lo que te conmueve el corazón, lo que te toca precisamente allí, donde tú no lo esperas. Para decirlo un poco con un lenguaje de los jóvenes: la sorpresa es un golpe bajo; tú no te lo esperas. Y Él va y te conmueve. Primero: el anuncio hecho sorpresa.

Segundo: la \textit{prisa}. Las mujeres corren, van deprisa a decir: \textquote{¡Pero hemos encontrado esto!}. Las sorpresas de Dios nos ponen en camino, inmediatamente, sin esperar. Y así corren para ver. Y Pedro y Juan corren. Los pastores la noche de Navidad corren: \textquote{Vamos a Belén a ver lo que nos han dicho los ángeles}. Y la Samaritana, corre para decir a su gente: \textquote{Esta es una novedad: he encontrado a un hombre que me ha dicho todo lo que he hecho}. Y la gente sabía las cosas que ella había hecho. Y aquella gente, corre, deja lo que está haciendo, también el ama de casa deja las patatas en la cazuela –las encontrará quemadas– pero lo importante es ir, correr, para ver esa sorpresa, ese anuncio. También hoy sucede.

En nuestros barrios, en los pueblos cuando sucede algo extraordinario, la gente corre a ver. Ir deprisa. Andrés no perdió tiempo y fue deprisa donde Pedro a decirle: \textquote{Hemos encontrado al Mesías}. Las sorpresas, las buenas noticias, se dan siempre así: deprisa. En el Evangelio hay uno que se toma un poco de tiempo; no quiere arriesgar. Pero el Señor es bueno, lo espera con amor, es Tomás. \textquote{Yo creeré cuando vea las llagas}, dice. También el Señor tiene paciencia para aquellos que no van tan deprisa.

El anuncio-sorpresa, la respuesta deprisa y lo tercero que yo quisiera decir hoy es una pregunta: \textquote{¿Y yo qué? ¿Tengo el corazón abierto a las sorpresas de Dios? ¿Soy capaz de ir deprisa, o siempre con esa cantilena, \textquote{veré mañana, mañana}? ¿Qué me dice a mí la sorpresa?}. Juan y Pedro fueron deprisa al sepulcro. De Juan el Evangelio nos dice: \textquote{Creed}. También Pedro: \textquote{Creed}, pero a su modo, con la fe un poco mezclada con el remordimiento de haber negado al Señor. El anuncio causó sorpresa, la carrera/ir deprisa y la pregunta: ¿Y yo hoy en esta Pascua de 2018 qué hago? ¿Tú, qué haces?
\end{body}

\subsubsection{Urbi et Orbi (2018): Más allá de la muerte}

\src{Basílica Vaticana. 1 de abril de 2018.}

\begin{body}
\textit{Queridos hermanos y hermanas, ¡Feliz Pascua!}

\ltr{J}{esús} ha resucitado de entre los muertos. Junto con el canto del aleluya, resuena en la Iglesia y en todo el mundo, este mensaje: Jesús es el Señor, el Padre lo ha resucitado y él vive para siempre en medio de nosotros.

Jesús mismo había preanunciado su muerte y resurrección con la imagen del \textit{grano de trigo}. Decía: \textquote{Si el grano de trigo no cae en tierra y muere, queda infecundo; pero si muere, da mucho fruto} (\textit{Jn} 12, 24). Y esto es lo que ha sucedido: Jesús, el grano de trigo sembrado por Dios en los surcos de la tierra, murió víctima del pecado del mundo, permaneció dos días en el sepulcro; pero en su muerte estaba presente toda la potencia del amor de Dios, que se liberó y se manifestó el tercer día, y que hoy celebramos: la Pascua de Cristo Señor.

Nosotros, cristianos, creemos y sabemos que la resurrección de Cristo es la verdadera esperanza del mundo, aquella que no defrauda. Es la fuerza del grano de trigo, del amor que se humilla y se da hasta el final, y que renueva realmente el mundo. También hoy esta fuerza produce fruto en los surcos de nuestra historia, marcada por tantas injusticias y violencias. Trae frutos de esperanza y dignidad donde hay miseria y exclusión, donde hay hambre y falta trabajo, a los prófugos y refugiados –tantas veces rechazados por la cultura actual del descarte–, a las víctimas del narcotráfico, de la trata de personas y de las distintas formas de esclavitud de nuestro tiempo.

\txtsmall{[Y, hoy, nosotros pedimos frutos de paz para el mundo entero, comenzando por la amada y martirizada Siria, cuya población está extenuada por una guerra que no tiene fin. Que la luz de Cristo resucitado ilumine en esta Pascua las conciencias de todos los responsables políticos y militares, para que se ponga fin inmediatamente al exterminio que se está llevando a cabo, se respete el derecho humanitario y se proceda a facilitar el acceso a las ayudas que estos hermanos y hermanas nuestros necesitan urgentemente, asegurando al mismo tiempo las condiciones adecuadas para el regreso de los desplazados.

Invocamos frutos de reconciliación para Tierra Santa, que en estos días también está siendo golpeada por conflictos abiertos que no respetan a los indefensos, para Yemen y para todo el Oriente Próximo, para que el diálogo y el respeto mutuo prevalezcan sobre las divisiones y la violencia. Que nuestros hermanos en Cristo, que sufren frecuentemente abusos y persecuciones, puedan ser testigos luminosos del Resucitado y de la victoria del bien sobre el mal.

Suplicamos en este día frutos de esperanza para cuantos anhelan una vida más digna, sobre todo en aquellas regiones del continente africano que sufren por el hambre, por conflictos endémicos y el terrorismo. Que la paz del Resucitado sane las heridas en Sudán del Sur: abra los corazones al diálogo y a la comprensión mutua. No olvidemos a las víctimas de ese conflicto, especialmente a los niños. Que nunca falte la solidaridad para las numerosas personas obligadas a abandonar sus tierras y privadas del mínimo necesario para vivir.

Imploramos frutos de diálogo para la península coreana, para que las conversaciones en curso promuevan la armonía y la pacificación de la región. Que los que tienen responsabilidades directas actúen con sabiduría y discernimiento para promover el bien del pueblo coreano y construir relaciones de confianza en el seno de la comunidad internacional.

Pedimos frutos de paz para Ucrania, para que se fortalezcan los pasos en favor de la concordia y se faciliten las iniciativas humanitarias que necesita la población.

Suplicamos frutos de consolación para el pueblo venezolano, el cual –como han escrito sus Pastores– vive en una especie de \textquote{tierra extranjera} en su propio país. Para que, por la fuerza de la resurrección del Señor Jesús, encuentre la vía justa, pacífica y humana para salir cuanto antes de la crisis política y humanitaria que lo oprime, y no falten la acogida y asistencia a cuantos entre sus hijos están obligados a abandonar su patria.]}

Traiga Cristo Resucitado frutos de vida nueva para los niños que, a causa de las guerras y el hambre, crecen sin esperanza, carentes de educación y de asistencia sanitaria; y también para los ancianos desechados por la cultura egoísta, que descarta a quien no es \textquote{productivo}.

Invocamos frutos de sabiduría para los que en todo el mundo tienen responsabilidades políticas, para que respeten siempre la dignidad humana, se esfuercen con dedicación al servicio del bien común y garanticen el desarrollo y la seguridad a los propios ciudadanos.

Queridos hermanos y hermanas:

También a nosotros, como a las mujeres que acudieron al sepulcro, van dirigidas estas palabras: \textquote{¿Por qué buscáis entre los muertos al que vive? No está aquí. Ha resucitado} (\textit{Lc} 24, 5-6). La muerte, la soledad y el miedo ya no son la última palabra. Hay una palabra que va más allá y que solo Dios puede pronunciar: es la palabra de la Resurrección (cf. Juan Pablo II, \textit{Palabras al término del Vía Crucis}). Ella, con la fuerza del amor de Dios, \textquote{ahuyenta los pecados, lava las culpas, devuelve la inocencia a los caídos, la alegría a los tristes, expulsa el odio, trae la concordia, doblega a los poderosos} (Pregón pascual).

¡Feliz Pascua a todos!
\end{body}


\begin{patercite}
(\ldots) El Triduo pascual (\ldots) renueva en los bautizados el sentimiento de su nueva condición, que el apóstol Pablo expresa siempre así: \textquote{Si habéis resucitado con Cristo, buscad las cosas de arriba [\ldots] Aspirad a las cosas de arriba, no a las de la tierra}. (\textit{Colosenses} 3, 1-3). Mirar arriba, mirar el horizonte, ampliar los horizontes: esta es nuestra fe, esta es nuestra justificación, ¡este es el estado de gracia! Por el bautismo, de hecho, resucitamos con Jesús y morimos para las cosas y la lógica del mundo; renacemos como criaturas nuevas: una realidad que pide convertirse en existencia concreta día a día. Un cristiano, si verdaderamente se deja lavar por Cristo, si verdaderamente se deja despojar por Él del hombre viejo para caminar en una vida nueva, incluso permaneciendo pecador --porque todos lo somos-- ya no puede ser corrupto, la justificación de Jesús nos salva de la corrupción, somos pecadores, pero no corruptos; ya no puede vivir con la muerte en el alma y tampoco ser causa de muerte. Y aquí debo decir una cosa triste y dolorosa\ldots{} Hay falsos cristianos: aquellos que dicen: \textquote{Jesús ha resucitado}, \textquote{yo he sido justificado por Jesús}, estoy en la vida nueva, pero vivo una vida corrupta. Y estos cristianos fingidos terminarán mal. El cristiano, repito, es pecador --todos lo somos, yo lo soy-- pero tenemos la seguridad de que cuando pedimos perdón, el Señor nos perdona. El corrupto hace como que es una persona honorable, pero, al final, en su corazón hay podredumbre. Una vida nueva nos da Jesús. El cristiano no puede vivir con la muerte en el alma, ni tampoco ser causa de muerte. Pensemos --para no ir lejos-- pensemos en casa, pensemos en los llamados \textquote{cristianos mafiosos}. Pero estos de cristianos no tienen nada: se dicen cristianos, pero llevan la muerte en el alma y a los demás. Recemos por ellos, para que el Señor toque su alma.
		
\textbf{Francisco}, papa, \textit{Catequesis}, audiencia general,  28 de marzo de 2018, cf. parr. 4.
	\end{patercite}

\newsection
\section{Temas}
\label{temasc04}
\cceth{La Resurrección de Cristo y nuestra resurrección} 

\cceref{CEC 638-655, 989, 1001-1002}

\ccesec{Al tercer día resucitó de entre los muertos}

\begin{ccebody}
\n{638} \textquote{Os anunciamos la Buena Nueva de que la Promesa hecha a los padres Dios la ha cumplido en nosotros, los hijos, al resucitar a Jesús} (\textit{Hch} 13, 32-33). La Resurrección de Jesús es la verdad culminante de nuestra fe en Cristo, creída y vivida por la primera comunidad cristiana como verdad central, transmitida como fundamental por la Tradición, establecida en los documentos del Nuevo Testamento, predicada como parte esencial del Misterio Pascual al mismo tiempo que la Cruz:

\begin{cceprose}
Cristo ha resucitado de los muertos,
   con su muerte ha vencido a la muerte.
   Y a los muertos ha dado la vida.
   (Liturgia bizantina: \textit{Tropario del día de Pascua})
\end{cceprose}

\n{639} El misterio de la resurrección de Cristo es un acontecimiento real que tuvo manifestaciones históricamente comprobadas como lo atestigua el Nuevo Testamento. Ya san Pablo, hacia el año 56, puede escribir a los Corintios: \textquote{Porque os transmití, en primer lugar, lo que a mi vez recibí: que Cristo murió por nuestros pecados, según las Escrituras; que fue sepultado y que resucitó al tercer día, según las Escrituras; que se apareció a Cefas y luego a los Doce} (\textit{1 Co} 15, 3-4). El apóstol habla aquí de \textit{la tradición viva de la Resurrección} que recibió después de su conversión a las puertas de Damasco (cf. \textit{Hch} 9, 3-18).

\ccesec{El sepulcro vacío}

\n{640} \textquote{¿Por qué buscar entre los muertos al que vive? No está aquí, ha resucitado} (\textit{Lc} 24, 5-6). En el marco de los acontecimientos de Pascua, el primer elemento que se encuentra es el sepulcro vacío. No es en sí una prueba directa. La ausencia del cuerpo de Cristo en el sepulcro podría explicarse de otro modo (cf. \textit{Jn} 20,13; \textit{Mt} 28, 11-15). A pesar de eso, el sepulcro vacío ha constituido para todos un signo esencial. Su descubrimiento por los discípulos fue el primer paso para el reconocimiento del hecho de la Resurrección. Es el caso, en primer lugar, de las santas mujeres (cf. \textit{Lc} 24, 3. 22- 23), después de Pedro (cf. \textit{Lc} 24, 12). \textquote{El discípulo que Jesús amaba} (\textit{Jn} 20, 2) afirma que, al entrar en el sepulcro vacío y al descubrir \textquote{las vendas en el suelo} (\textit{Jn} 20, 6) \textquote{vio y creyó} (\textit{Jn} 20, 8). Eso supone que constató en el estado del sepulcro vacío (cf. \textit{Jn} 20, 5-7) que la ausencia del cuerpo de Jesús no había podido ser obra humana y que Jesús no había vuelto simplemente a una vida terrenal como había sido el caso de Lázaro (cf. \textit{Jn} 11, 44).

\ccesec{Las apariciones del Resucitado}

\n{641} María Magdalena y las santas mujeres, que iban a embalsamar el cuerpo de Jesús (cf. \textit{Mc} 16, 1; \textit{Lc} 24, 1) enterrado a prisa en la tarde del Viernes Santo por la llegada del Sábado (cf. \textit{Jn} 19, 31. 42) fueron las primeras en encontrar al Resucitado (cf. \textit{Mt} 28, 9-10; \textit{Jn} 20, 11-18). Así las mujeres fueron las primeras mensajeras de la Resurrección de Cristo para los propios Apóstoles (cf. \textit{Lc} 24, 9-10). Jesús se apareció en seguida a ellos, primero a Pedro, después a los Doce (cf. \textit{1 Co} 15, 5). Pedro, llamado a confirmar en la fe a sus hermanos (cf. \textit{Lc} 22, 31-32), ve por tanto al Resucitado antes que los demás y sobre su testimonio es sobre el que la comunidad exclama: \textquote{¡Es verdad! ¡El Señor ha resucitado y se ha aparecido a Simón!} (\textit{Lc} 24, 34).

\n{642} Todo lo que sucedió en estas jornadas pascuales compromete a cada uno de los Apóstoles –y a Pedro en particular– en la construcción de la era nueva que comenzó en la mañana de Pascua. Como testigos del Resucitado, los Apóstoles son las piedras de fundación de su Iglesia. La fe de la primera comunidad de creyentes se funda en el testimonio de hombres concretos, conocidos de los cristianos y de los que la mayor parte aún vivían entre ellos. Estos \textquote{testigos de la Resurrección de Cristo} (cf. \textit{Hch} 1, 22) son ante todo Pedro y los Doce, pero no solamente ellos: Pablo habla claramente de más de quinientas personas a las que se apareció Jesús en una sola vez, además de Santiago y de todos los Apóstoles (cf. \textit{1 Co} 15, 4-8).

\n{643} Ante estos testimonios es imposible interpretar la Resurrección de Cristo fuera del orden físico, y no reconocerlo como un hecho histórico. Sabemos por los hechos que la fe de los discípulos fue sometida a la prueba radical de la pasión y de la muerte en cruz de su Maestro, anunciada por Él de antemano (cf. \textit{Lc} 22, 31-32). La sacudida provocada por la pasión fue tan grande que los discípulos (por lo menos, algunos de ellos) no creyeron tan pronto en la noticia de la resurrección. Los evangelios, lejos de mostrarnos una comunidad arrobada por una exaltación mística, nos presentan a los discípulos abatidos (\textquote{la cara sombría}: \textit{Lc} 24, 17) y asustados (cf. \textit{Jn} 20, 19). Por eso no creyeron a las santas mujeres que regresaban del sepulcro y \textquote{sus palabras les parecían como desatinos} (\textit{Lc} 24, 11; cf. \textit{Mc} 16, 11. 13). Cuando Jesús se manifiesta a los once en la tarde de Pascua \textquote{les echó en cara su incredulidad y su dureza de cabeza por no haber creído a quienes le habían visto resucitado} (\textit{Mc} 16, 14).

\n{644} Tan imposible les parece la cosa que, incluso puestos ante la realidad de Jesús resucitado, los discípulos dudan todavía (cf. \textit{Lc} 24, 38): creen ver un espíritu (cf. \textit{Lc} 24, 39). \textquote{No acaban de creerlo a causa de la alegría y estaban asombrados} (\textit{Lc} 24, 41). Tomás conocerá la misma prueba de la duda (cf. \textit{Jn} 20, 24-27) y, en su última aparición en Galilea referida por Mateo, \textquote{algunos sin embargo dudaron} (\textit{Mt} 28, 17). Por esto la hipótesis según la cual la resurrección habría sido un \textquote{producto} de la fe (o de la credulidad) de los apóstoles no tiene consistencia. Muy al contrario, su fe en la Resurrección nació –bajo la acción de la gracia divina– de la experiencia directa de la realidad de Jesús resucitado.

\newpage 

\ccesec{El estado de la humanidad resucitada de Cristo}

\n{645} Jesús resucitado establece con sus discípulos relaciones directas mediante el tacto (cf. \textit{Lc} 24, 39; \textit{Jn} 20, 27) y el compartir la comida (cf. \textit{Lc} 24, 30. 41-43; \textit{Jn} 21, 9. 13-15). Les invita así a reconocer que él no es un espíritu (cf. \textit{Lc} 24, 39), pero sobre todo a que comprueben que el cuerpo resucitado con el que se presenta ante ellos es el mismo que ha sido martirizado y crucificado, ya que sigue llevando las huellas de su pasión (cf. \textit{Lc} 24, 40; \textit{Jn} 20, 20. 27). Este cuerpo auténtico y real posee sin embargo al mismo tiempo, las propiedades nuevas de un cuerpo glorioso: no está situado en el espacio ni en el tiempo, pero puede hacerse presente a su voluntad donde quiere y cuando quiere (cf. \textit{Mt} 28, 9. 16-17; \textit{Lc} 24, 15. 36; \textit{Jn} 20, 14. 19. 26; 21, 4) porque su humanidad ya no puede ser retenida en la tierra y no pertenece ya más que al dominio divino del Padre (cf. \textit{Jn} 20, 17). Por esta razón también Jesús resucitado es soberanamente libre de aparecer como quiere: bajo la apariencia de un jardinero (cf. \textit{Jn} 20, 14-15) o \textquote{bajo otra figura} (\textit{Mc} 16, 12) distinta de la que les era familiar a los discípulos, y eso para suscitar su fe (cf. \textit{Jn} 20, 14. 16; 21, 4. 7).

\n{646} La Resurrección de Cristo no fue un retorno a la vida terrena como en el caso de las resurrecciones que él había realizado antes de Pascua: la hija de Jairo, el joven de Naím, Lázaro. Estos hechos eran acontecimientos milagrosos, pero las personas afectadas por el milagro volvían a tener, por el poder de Jesús, una vida terrena \textquote{ordinaria}. En cierto momento, volverán a morir. La Resurrección de Cristo es esencialmente diferente. En su cuerpo resucitado, pasa del estado de muerte a otra vida más allá del tiempo y del espacio. En la Resurrección, el cuerpo de Jesús se llena del poder del Espíritu Santo; participa de la vida divina en el estado de su gloria, tanto que san Pablo puede decir de Cristo que es \textquote{el hombre celestial} (cf. \textit{1 Co} 15, 35-50).

\ccesec{La Resurrección como acontecimiento transcendente}

\n{647} \textquote{¡Qué noche tan dichosa –canta el \textit{Exultet} de Pascua–, sólo ella conoció el momento en que Cristo resucitó de entre los muertos!}. En efecto, nadie fue testigo ocular del acontecimiento mismo de la Resurrección y ningún evangelista lo describe. Nadie puede decir cómo sucedió físicamente. Menos aún, su esencia más íntima, el paso a otra vida, fue perceptible a los sentidos. Acontecimiento histórico demostrable por la señal del sepulcro vacío y por la realidad de los encuentros de los Apóstoles con Cristo resucitado, no por ello la Resurrección pertenece menos al centro del Misterio de la fe en aquello que transciende y sobrepasa a la historia. Por eso, Cristo resucitado no se manifiesta al mundo (cf. \textit{Jn} 14, 22) sino a sus discípulos, \textquote{a los que habían subido con él desde Galilea a Jerusalén y que ahora son testigos suyos ante el pueblo} (\textit{Hch} 13, 31).

\ccesec{La Resurrección obra de la Santísima Trinidad}

\n{648} La Resurrección de Cristo es objeto de fe en cuanto es una intervención transcendente de Dios mismo en la creación y en la historia. En ella, las tres Personas divinas actúan juntas a la vez y manifiestan su propia originalidad. Se realiza por el poder del Padre que \textquote{ha resucitado} (\textit{Hch} 2, 24) a Cristo, su Hijo, y de este modo ha introducido de manera perfecta su humanidad –con su cuerpo– en la Trinidad. Jesús se revela definitivamente \textquote{Hijo de Dios con poder, según el Espíritu de santidad, por su resurrección de entre los muertos} (\textit{Rm} 1, 3-4). San Pablo insiste en la manifestación del poder de Dios (cf. \textit{Rm} 6, 4; 2 Co 13, 4; \textit{Flp} 3, 10; \textit{Ef} 1, 19-22; \textit{Hb} 7, 16) por la acción del Espíritu que ha vivificado la humanidad muerta de Jesús y la ha llamado al estado glorioso de Señor.

\n{649} En cuanto al Hijo, él realiza su propia Resurrección en virtud de su poder divino. Jesús anuncia que el Hijo del hombre deberá sufrir mucho, morir y luego resucitar (sentido activo del término) (cf. \textit{Mc} 8, 31; 9, 9-31; 10, 34). Por otra parte, él afirma explícitamente: \textquote{Doy mi vida, para recobrarla de nuevo \ldots Tengo poder para darla y poder para recobrarla de nuevo} (\textit{Jn} 10, 17-18). \textquote{Creemos que Jesús murió y resucitó} (\textit{1 Ts} 4, 14).

\n{650} Los Padres contemplan la Resurrección a partir de la persona divina de Cristo que permaneció unida a su alma y a su cuerpo separados entre sí por la muerte: \textquote{Por la unidad de la naturaleza divina que permanece presente en cada una de las dos partes del hombre, las que antes estaban separadas y segregadas, éstas se unen de nuevo. Así la muerte se produce por la separación del compuesto humano, y la Resurrección por la unión de las dos partes separadas} (San Gregorio de Nisa, \textit{De tridui inter mortem et resurrectionem Domini nostri Iesu Christi spatio}; cf. también DS 325; 359; 369; 539).

\ccesec{Sentido y alcance salvífico de la Resurrección}

\n{651} \textquote{Si no resucitó Cristo, vana es nuestra predicación, vana también vuestra fe} (\textit{1 Co} 15, 14). La Resurrección constituye ante todo la confirmación de todo lo que Cristo hizo y enseñó. Todas las verdades, incluso las más inaccesibles al espíritu humano, encuentran su justificación si Cristo, al resucitar, ha dado la prueba definitiva de su autoridad divina según lo había prometido.

\n{652} La Resurrección de Cristo \textit{es cumplimiento de las promesas} del Antiguo Testamento (cf. \textit{Lc} 24, 26-27. 44-48) y del mismo Jesús durante su vida terrenal (cf. \textit{Mt} 28, 6; \textit{Mc} 16, 7; \textit{Lc} 24, 6-7). La expresión \textquote{según las Escrituras} (cf. \textit{1 Co} 15, 3-4 y el Símbolo Niceno-Constantinopolitano: DS 150) indica que la Resurrección de Cristo cumplió estas predicciones.

\n{653} La verdad de \textit{la divinidad de Jesús} es confirmada por su Resurrección. Él había dicho: \textquote{Cuando hayáis levantado al Hijo del hombre, entonces sabréis que Yo Soy} (\textit{Jn} 8, 28). La Resurrección del Crucificado demostró que verdaderamente, él era \textquote{Yo Soy}, el Hijo de Dios y Dios mismo. San Pablo pudo decir a los judíos: \textquote{La Promesa hecha a los padres Dios la ha cumplido en nosotros [\ldots] al resucitar a Jesús, como está escrito en el salmo primero: \textquote{Hijo mío eres tú; yo te he engendrado hoy}} (\textit{Hch} 13, 32-33; cf. \textit{Sal} 2, 7). La Resurrección de Cristo está estrechamente unida al misterio de la Encarnación del Hijo de Dios: es su plenitud según el designio eterno de Dios.

\n{654} Hay un doble aspecto en el misterio pascual: por su muerte nos libera del pecado, por su Resurrección nos abre el acceso a una nueva vida. Esta es, en primer lugar, la \textit{justificación} que nos devuelve a la gracia de Dios (cf. \textit{Rm} 4, 25) \textquote{a fin de que, al igual que Cristo fue resucitado de entre los muertos [\ldots] así también nosotros vivamos una nueva vida} (\textit{Rm} 6, 4). Consiste en la victoria sobre la muerte y el pecado y en la nueva participación en la gracia (cf. \textit{Ef} 2, 4-5; \textit{1 P} 1, 3). Realiza la \textit{adopción filial} porque los hombres se convierten en hermanos de Cristo, como Jesús mismo llama a sus discípulos después de su Resurrección: \textquote{Id, avisad a mis hermanos} (\textit{Mt} 28, 10; \textit{Jn} 20, 17). Hermanos no por naturaleza, sino por don de la gracia, porque esta filiación adoptiva confiere una participación real en la vida del Hijo único, la que ha revelado plenamente en su Resurrección.

\n{655} Por último, la Resurrección de Cristo –y el propio Cristo resucitado– es principio y fuente de \textit{nuestra resurrección futura}: \textquote{Cristo resucitó de entre los muertos como primicias de los que durmieron [\ldots] del mismo modo que en Adán mueren todos, así también todos revivirán en Cristo} (\textit{1 Co} 15, 20-22). En la espera de que esto se realice, Cristo resucitado vive en el corazón de sus fieles. En Él los cristianos \textquote{saborean [\ldots] los prodigios del mundo futuro} (\textit{Hb} 6,5) y su vida es arrastrada por Cristo al seno de la vida divina (cf. \textit{Col} 3, 1-3) para que \textquote{ya no vivan para sí los que viven, sino para aquel que murió y resucitó por ellos} (\textit{2 Co} 5, 15).

\n{989} Creemos firmemente, y así lo esperamos, que del mismo modo que Cristo ha resucitado verdaderamente de entre los muertos, y que vive para siempre, igualmente los justos después de su muerte vivirán para siempre con Cristo resucitado y que Él los resucitará en el último día (cf. \textit{Jn} 6, 39-40). Como la suya, nuestra resurrección será obra de la Santísima Trinidad:

\ccecite{
\textquote{Si el Espíritu de Aquel que resucitó a Jesús de entre los muertos habita en vosotros, Aquel que resucitó a Jesús de entre los muertos dará también la vida a vuestros cuerpos mortales por su Espíritu que habita en vosotros} (\textit{Rm} 8, 11; cf. \textit{1 Ts} 4, 14; \textit{1 Co} 6, 14; \textit{2 Co} 4, 14; \textit{Flp} 3, 10-11).}

\n{1001} \textit{¿Cuándo?} Sin duda en el \textquote{último día} (\textit{Jn} 6, 39-40. 44. 54; 11, 24); \textquote{al fin del mundo} (LG 48). En efecto, la resurrección de los muertos está íntimamente asociada a la Parusía de Cristo:

\ccecite{\textquote{El Señor mismo, a la orden dada por la voz de un arcángel y por la trompeta de Dios, bajará del cielo, y los que murieron en Cristo resucitarán en primer lugar} (\textit{1 Ts} 4, 16).}

\ccesec{Resucitados con Cristo}

\n{1002} Si es verdad que Cristo nos resucitará en \textquote{el último día}, también lo es, en cierto modo, que nosotros ya hemos resucitado con Cristo. En efecto, gracias al Espíritu Santo, la vida cristiana en la tierra es, desde ahora, una participación en la muerte y en la Resurrección de Cristo:

\ccecite{\textquote{Sepultados con él en el Bautismo, con él también habéis resucitado por la fe en la acción de Dios, que le resucitó de entre los muertos [\ldots] Así pues, si habéis resucitado con Cristo, buscad las cosas de arriba, donde está Cristo sentado a la diestra de Dios} (\textit{Col} 2, 12; 3, 1).}
\end{ccebody}

\cceth{La Pascua, el Día del Señor} 

\cceref{CEC 647, 1167-1170, 1243, 1287}

\begin{ccebody}

\ccesec{La Resurrección como acontecimiento transcendente}

\n{647} \textquote{¡Qué noche tan dichosa –canta el \textit{Exultet} de Pascua–, sólo ella conoció el momento en que Cristo resucitó de entre los muertos!}. En efecto, nadie fue testigo ocular del acontecimiento mismo de la Resurrección y ningún evangelista lo describe. Nadie puede decir cómo sucedió físicamente. Menos aún, su esencia más íntima, el paso a otra vida, fue perceptible a los sentidos. Acontecimiento histórico demostrable por la señal del sepulcro vacío y por la realidad de los encuentros de los Apóstoles con Cristo resucitado, no por ello la Resurrección pertenece menos al centro del Misterio de la fe en aquello que transciende y sobrepasa a la historia. Por eso, Cristo resucitado no se manifiesta al mundo (cf. \textit{Jn} 14, 22) sino a sus discípulos, \textquote{a los que habían subido con él desde Galilea a Jerusalén y que ahora son testigos suyos ante el pueblo} (\textit{Hch} 13, 31).

\n{1167} El domingo es el día por excelencia de la asamblea litúrgica, en que los fieles \textquote{deben reunirse para, escuchando la Palabra de Dios y participando en la Eucaristía, recordar la pasión, la resurrección y la gloria del Señor Jesús y dar gracias a Dios, que los hizo renacer a la esperanza viva por la resurrección de Jesucristo de entre los muertos} (SC 106):

\ccecite{\textquote{Cuando meditamos, [oh Cristo], las maravillas que fueron realizadas en este día del domingo de tu santa y gloriosa Resurrección, decimos: Bendito es el día del domingo, porque en él tuvo comienzo la Creación [\ldots] la salvación del mundo [\ldots] la renovación del género humano [\ldots] en él el cielo y la tierra se regocijaron y el universo entero quedó lleno de luz. Bendito es el día del domingo, porque en él fueron abiertas las puertas del paraíso para que Adán y todos los desterrados entren en él sin temor} (\textit{Fanqîth, Breviarium iuxta ritum Ecclesiae Antiochenae Syrorum}, v. 6 [Mossul 1886] p. 193b).}

\ccesec{El año litúrgico}

\n{1168} A partir del \textquote{Triduo Pascual}, como de su fuente de luz, el tiempo nuevo de la Resurrección llena todo el año litúrgico con su resplandor. El año, gracias a esta fuente, queda progresivamente transfigurado por la liturgia. Es realmente \textquote{año de gracia del Señor} (cf. \textit{Lc} 4, 19). La economía de la salvación actúa en el marco del tiempo, pero desde su cumplimiento en la Pascua de Jesús y la efusión del Espíritu Santo, el fin de la historia es anticipado, como pregustado, y el Reino de Dios irrumpe en el tiempo de la humanidad.

\n{1169} Por ello, la \textit{Pascua} no es simplemente una fiesta entre otras: es la \textquote{Fiesta de las fiestas}, \textquote{Solemnidad de las solemnidades}, como la Eucaristía es el Sacramento de los sacramentos (el gran sacramento). San Atanasio la llama \textquote{el gran domingo} (\textit{Epistula festivalis} 1 [año 329], 10: PG 26, 1366), así como la Semana Santa es llamada en Oriente \textquote{la gran semana}. El Misterio de la Resurrección, en el cual Cristo ha aplastado a la muerte, penetra en nuestro viejo tiempo con su poderosa energía, hasta que todo le esté sometido.

\n{1170} En el Concilio de Nicea (año 325) todas las Iglesias se pusieron de acuerdo para que la Pascua cristiana fuese celebrada el domingo que sigue al plenilunio (14 del mes de Nisán) después del equinoccio de primavera. Por causa de los diversos métodos utilizados para calcular el 14 del mes de Nisán, en las Iglesias de Occidente y de Oriente no siempre coincide la fecha de la Pascua. Por eso, dichas Iglesias buscan hoy un acuerdo, para llegar de nuevo a celebrar en una fecha común el día de la Resurrección del Señor.

\n{1243} La \textit{vestidura blanca} simboliza que el bautizado se ha \textquote{revestido de Cristo} (\textit{Ga} 3,27): ha resucitado con Cristo. El \textit{cirio} que se enciende en el Cirio Pascual, significa que Cristo ha iluminado al neófito. En Cristo, los bautizados son \textquote{la luz del mundo} (\textit{Mt} 5,14; cf. \textit{Flp} 2,15).

El nuevo bautizado es ahora hijo de Dios en el Hijo Único. Puede ya decir la oración de los hijos de Dios: \textit{el Padre Nuestro}.

\n{1287} Ahora bien, esta plenitud del Espíritu no debía permanecer únicamente en el Mesías, sino que debía ser comunicada a \textit{todo el pueblo mesiánico} (cf. \textit{Ez} 36,25-27; \textit{Jl} 3,1-2). En repetidas ocasiones Cristo prometió esta efusión del Espíritu (cf. \textit{Lc} 12,12; \textit{Jn} 3,5-8; 7,37-39; 16,7-15; \textit{Hch} 1,8), promesa que realizó primero el día de Pascua (\textit{Jn} 20,22) y luego, de manera más manifiesta el día de Pentecostés (cf. \textit{Hch} 2,1-4). Llenos del Espíritu Santo, los Apóstoles comienzan a proclamar \textquote{las maravillas de Dios} (\textit{Hch} 2,11) y Pedro declara que esta efusión del Espíritu es el signo de los tiempos mesiánicos (cf. \textit{Hch} 2, 17-18). Los que creyeron en la predicación apostólica y se hicieron bautizar, recibieron a su vez el don del Espíritu Santo (cf. \textit{Hch} 2,38).
\end{ccebody}

\cceth{Los Sacramentos de la Iniciación cristiana} 

\cceref{CEC 1212}

\begin{ccebody}
\n{1212} Mediante los sacramentos de la iniciación cristiana, el Bautismo, la Confirmación y la Eucaristía, se ponen los \textit{fundamentos} de toda vida cristiana. \textquote{La participación en la naturaleza divina, que los hombres reciben como don mediante la gracia de Cristo, tiene cierta analogía con el origen, el crecimiento y el sustento de la vida natural. En efecto, los fieles renacidos en el Bautismo se fortalecen con el sacramento de la Confirmación y, finalmente, son alimentados en la Eucaristía con el manjar de la vida eterna, y, así por medio de estos sacramentos de la iniciación cristiana, reciben cada vez con más abundancia los tesoros de la vida divina y avanzan hacia la perfección de la caridad} (Pablo VI, Const. apost. \textit{Divinae consortium naturae}; cf. \textit{Ritual de Iniciación Cristiana de Adultos}, Prenotandos 1-2).
\end{ccebody}

\cceth{El Bautismo} 

\cceref{CEC 1214-1222, 1226-1228, 1234-1245, 1254} 

\begin{ccebody}
	
\ccesec{El nombre de este sacramento}

\n{1214} Este sacramento recibe el nombre de \textit{Bautismo} en razón del carácter del rito central mediante el que se celebra: bautizar (\textit{baptizein} en griego) significa \textquote{sumergir}, \textquote{introducir dentro del agua}; la \textquote{inmersión} en el agua simboliza el acto de sepultar al catecúmeno en la muerte de Cristo, de donde sale por la resurrección con Él (cf. \textit{Rm} 6,3-4; \textit{Col} 2,12) como \textquote{nueva criatura} (\textit{2 Co} 5,17; \textit{Ga} 6,15).

\n{1215} Este sacramento es llamado también \textquote{\textit{baño de regeneración y de renovación del Espíritu Santo}} (\textit{Tt} 3,5), porque significa y realiza ese nacimiento del agua y del Espíritu sin el cual \textquote{nadie puede entrar en el Reino de Dios} (\textit{Jn} 3,5).

\n{1216} \textquote{Este baño es llamado \textit{iluminación} porque quienes reciben esta enseñanza (catequética) su espíritu es iluminado} (San Justino, \textit{Apología} 1,61). Habiendo recibido en el Bautismo al Verbo, \textquote{la luz verdadera que ilumina a todo hombre} (\textit{Jn} 1,9), el bautizado, \textquote{tras haber sido iluminado} (\textit{Hb} 10,32), se convierte en \textquote{hijo de la luz} (\textit{1 Ts} 5,5), y en \textquote{luz} él mismo (\textit{Ef} 5,8):

\ccecite{El Bautismo \textquote{es el más bello y magnífico de los dones de Dios [\ldots] lo llamamos don, gracia, unción, iluminación, vestidura de incorruptibilidad, baño de regeneración, sello y todo lo más precioso que hay. \textit{Don}, porque es conferido a los que no aportan nada; \textit{gracia}, porque es dado incluso a culpables; \textit{bautismo}, porque el pecado es sepultado en el agua; \textit{unción}, porque es sagrado y real (tales son los que son ungidos); \textit{iluminación}, porque es luz resplandeciente; \textit{vestidura}, porque cubre nuestra vergüenza; \textit{baño}, porque lava; \textit{sello}, porque nos guarda y es el signo de la soberanía de Dios} (San Gregorio Nacianceno, \textit{Oratio} 40,3-4).}

\newpage 

\ccesec{Las prefiguraciones del Bautismo en la Antigua Alianza}

\n{1217} En la liturgia de la vigilia Pascual, cuando \textit{se bendice el agua bautismal}, la Iglesia hace solemnemente memoria de los grandes acontecimientos de la historia de la salvación que prefiguraban ya el misterio del Bautismo:

\ccecite{\textquote{¡Oh Dios! [\ldots] que realizas en tus sacramentos obras admirables con tu poder invisible, y de diversos modos te has servido de tu criatura el agua para significar la gracia del bautismo} (\textit{Vigilia Pascual, Bendición del agua: Misal Romano}).}

\n{1218} Desde el origen del mundo, el agua, criatura humilde y admirable, es la fuente de la vida y de la fecundidad. La Sagrada Escritura dice que el Espíritu de Dios \textquote{se cernía} sobre ella (cf. \textit{Gn} 1,2):

\ccecite{\textquote{¡Oh Dios!, cuyo Espíritu, en los orígenes del mundo, se cernía sobre las aguas, para que ya desde entonces concibieran el poder de santificar} (\textit{Vigilia Pascual, Bendición del agua: Misal Romano}).}

\n{1219} La Iglesia ha visto en el arca de Noé una prefiguración de la salvación por el bautismo. En efecto, por medio de ella \textquote{unos pocos, es decir, ocho personas, fueron salvados a través del agua} (\textit{1 P} 3,20):

\ccecite{\textquote{¡Oh Dios!, que incluso en las aguas torrenciales del diluvio prefiguraste el nacimiento de la nueva humanidad, de modo que una misma agua pusiera fin al pecado y diera origen a la santidad} (\textit{Vigilia Pascual, Bendición del agua: Misal Romano}).}

\n{1220} Si el agua de manantial simboliza la vida, el agua del mar es un símbolo de la muerte. Por lo cual, pudo ser símbolo del misterio de la Cruz. Por este simbolismo el bautismo significa la comunión con la muerte de Cristo.

\n{1221} Sobre todo el paso del mar Rojo, verdadera liberación de Israel de la esclavitud de Egipto, es el que anuncia la liberación obrada por el bautismo:

\ccecite{\textquote{Oh Dios!, que hiciste pasar a pie enjuto por el mar Rojo a los hijos de Abraham, para que el pueblo liberado de la esclavitud del faraón fuera imagen de la familia de los bautizados} (\textit{Vigilia Pascual, Bendición del agua: Misal Romano}).}

\n{1222} Finalmente, el Bautismo es prefigurado en el paso del Jordán, por el que el pueblo de Dios recibe el don de la tierra prometida a la descendencia de Abraham, imagen de la vida eterna. La promesa de esta herencia bienaventurada se cumple en la nueva Alianza.

\ccesec{El Bautismo en la Iglesia}

\n{1226} Desde el día de Pentecostés la Iglesia ha celebrado y administrado el santo Bautismo. En efecto, san Pedro declara a la multitud conmovida por su predicación: \textquote{Convertíos [\ldots] y que cada uno de vosotros se haga bautizar en el nombre de Jesucristo, para remisión de vuestros pecados; y recibiréis el don del Espíritu Santo} (\textit{Hch} 2,38). Los Apóstoles y sus colaboradores ofrecen el bautismo a quien crea en Jesús: judíos, hombres temerosos de Dios, paganos (\textit{Hch} 2,41; 8,12-13; 10,48; 16,15). El Bautismo aparece siempre ligado a la fe: \textquote{Ten fe en el Señor Jesús y te salvarás tú y tu casa}, declara san Pablo a su carcelero en Filipos. El relato continúa: \textquote{el carcelero inmediatamente recibió el bautismo, él y todos los suyos} (\textit{Hch} 16,31-33).

\newpage 

\n{1227} Según el apóstol san Pablo, por el Bautismo el creyente participa en la muerte de Cristo; es sepultado y resucita con Él:

\ccecite{\textquote{¿O es que ignoráis que cuantos fuimos bautizados en Cristo Jesús, fuimos bautizados en su muerte? Fuimos, pues, con él sepultados por el bautismo en la muerte, a fin de que, al igual que Cristo fue resucitado de entre los muertos por medio de la gloria del Padre, así también nosotros vivamos una vida nueva} (\textit{Rm} 6,3-4; cf. \textit{Col} 2,12).}

Los bautizados se han \textquote{revestido de Cristo} (\textit{Ga} 3,27). Por el Espíritu Santo, el Bautismo es un baño que purifica, santifica y justifica (cf. \textit{1 Co} 6,11; 12,13).

\n{1228} El Bautismo es, pues, un baño de agua en el que la \textquote{semilla incorruptible} de la Palabra de Dios produce su efecto vivificador (cf. \textit{1 P} 1,23; \textit{Ef} 5,26). San Agustín dirá del Bautismo: \textit{Accedit verbum ad elementum, et fit sacramentum} – \textquote{Se une la palabra a la materia, y se hace el sacramento} (\textit{In Iohannis evangelium tractatus} 80, 3).

\ccesec{La mistagogia de la celebración}

\n{1234} El sentido y la gracia del sacramento del Bautismo aparece claramente en los ritos de su celebración. Cuando se participa atentamente en los gestos y las palabras de esta celebración, los fieles se inician en las riquezas que este sacramento significa y realiza en cada nuevo bautizado.

\n{1235} \textit{La señal de la cruz}, al comienzo de la celebración, señala la impronta de Cristo sobre el que le va a pertenecer y significa la gracia de la redención que Cristo nos ha adquirido por su cruz.

\n{1236} \textit{El anuncio de la Palabra de Dios} ilumina con la verdad revelada a los candidatos y a la asamblea y suscita la respuesta de la fe, inseparable del Bautismo. En efecto, el Bautismo es de un modo particular \textquote{el sacramento de la fe} por ser la entrada sacramental en la vida de fe.

\n{1237} Puesto que el Bautismo significa la liberación del pecado y de su instigador, el diablo, se pronuncian uno o varios \textit{exorcismos} sobre el candidato. Este es ungido con el óleo de los catecúmenos o bien el celebrante le impone la mano y el candidato renuncia explícitamente a Satanás. Así preparado, puede \textit{confesar la fe de la Iglesia}, a la cual será \textquote{confiado} por el Bautismo (cf. \textit{Rm} 6,17).

\n{1238} El \textit{agua bautismal} es entonces consagrada mediante una oración de epíclesis (en el momento mismo o en la noche pascual). La Iglesia pide a Dios que, por medio de su Hijo, el poder del Espíritu Santo descienda sobre esta agua, a fin de que los que sean bautizados con ella \textquote{nazcan del agua y del Espíritu} (Jn 3,5).

\n{1239} Sigue entonces \textit{el rito esencial} del sacramento: \textit{el Bautismo} propiamente dicho, que significa y realiza la muerte al pecado y la entrada en la vida de la Santísima Trinidad a través de la configuración con el misterio pascual de Cristo. El Bautismo es realizado de la manera más significativa mediante la triple inmersión en el agua bautismal. Pero desde la antigüedad puede ser también conferido derramando tres veces agua sobre la cabeza del candidato.

\n{1240} En la Iglesia latina, esta triple infusión va acompañada de las palabras del ministro: \textquote{N., yo te bautizo en el nombre del Padre, y del Hijo y del Espíritu Santo}. En las liturgias orientales, estando el catecúmeno vuelto hacia el Oriente, el sacerdote dice: \textquote{El siervo de Dios, N., es bautizado en el nombre del Padre, y del Hijo y del Espíritu Santo}. Y mientras invoca a cada persona de la Santísima Trinidad, lo sumerge en el agua y lo saca de ella.

\n{1241} \textit{La unción con el santo crisma}, óleo perfumado y consagrado por el obispo, significa el don del Espíritu Santo al nuevo bautizado. Ha llegado a ser un cristiano, es decir, \textquote{ungido} por el Espíritu Santo, incorporado a Cristo, que es ungido sacerdote, profeta y rey (cf. \textit{Ritual del Bautismo de niños}, 62).

\n{1242} En la liturgia de las Iglesias de Oriente, la unción postbautismal es el sacramento de la Crismación (Confirmación). En la liturgia romana, dicha unción anuncia una segunda unción del santo crisma que dará el obispo: el sacramento de la Confirmación que, por así decirlo, \textquote{confirma} y da plenitud a la unción bautismal.

\n{1243} La \textit{vestidura blanca} simboliza que el bautizado se ha \textquote{revestido de Cristo} (\textit{Ga} 3,27): ha resucitado con Cristo. El \textit{cirio} que se enciende en el Cirio Pascual, significa que Cristo ha iluminado al neófito. En Cristo, los bautizados son \textquote{la luz del mundo} (\textit{Mt} 5,14; cf. \textit{Flp} 2,15).

El nuevo bautizado es ahora hijo de Dios en el Hijo Único. Puede ya decir la oración de los hijos de Dios: \textit{el Padre Nuestro}.

\n{1244} La \textit{primera comunión eucarística}. Hecho hijo de Dios, revestido de la túnica nupcial, el neófito es admitido \textquote{al festín de las bodas del Cordero} y recibe el alimento de la vida nueva, el Cuerpo y la Sangre de Cristo. Las Iglesias orientales conservan una conciencia viva de la unidad de la iniciación cristiana, por lo que dan la sagrada comunión a todos los nuevos bautizados y confirmados, incluso a los niños pequeños, recordando las palabras del Señor: \textquote{Dejad que los niños vengan a mí, no se lo impidáis} (\textit{Mc} 10,14). La Iglesia latina, que reserva el acceso a la Sagrada Comunión a los que han alcanzado el uso de razón, expresa cómo el Bautismo introduce a la Eucaristía acercando al altar al niño recién bautizado para la oración del Padre Nuestro.

\n{1245} La \textit{bendición solemne} cierra la celebración del Bautismo. En el Bautismo de recién nacidos, la bendición de la madre ocupa un lugar especial.

\n{1254} En todos los bautizados, niños o adultos, la fe debe crecer \textit{después} del Bautismo. Por eso, la Iglesia celebra cada año en la vigilia pascual la renovación de las promesas del Bautismo. La preparación al Bautismo sólo conduce al umbral de la vida nueva. El Bautismo es la fuente de la vida nueva en Cristo, de la cual brota toda la vida cristiana.
\end{ccebody}

\cceth{La Confirmación} 

\cceref{CEC 1286-1289}

\begin{ccebody}
\n{1286} En el Antiguo Testamento, los profetas anunciaron que el Espíritu del Señor reposaría sobre el Mesías esperado (cf. \textit{Is} 11,2) para realizar su misión salvífica (cf. \textit{Lc} 4,16-22; \textit{Is} 61,1). El descenso del Espíritu Santo sobre Jesús en su Bautismo por Juan fue el signo de que Él era el que debía venir, el Mesías, el Hijo de Dios (\textit{Mt} 3,13-17; \textit{Jn} 1,33- 34). Habiendo sido concedido por obra del Espíritu Santo, toda su vida y toda su misión se realizan en una comunión total con el Espíritu Santo que el Padre le da \textquote{sin medida} (\textit{Jn} 3,34).

\n{1287} Ahora bien, esta plenitud del Espíritu no debía permanecer únicamente en el Mesías, sino que debía ser comunicada a \textit{todo el pueblo mesiánico} (cf. \textit{Ez} 36,25-27; \textit{Jl} 3,1-2). En repetidas ocasiones Cristo prometió esta efusión del Espíritu (cf. \textit{Lc} 12,12; \textit{Jn} 3,5-8; 7,37-39; 16,7-15; \textit{Hch} 1,8), promesa que realizó primero el día de Pascua (\textit{Jn} 20,22) y luego, de manera más manifiesta el día de Pentecostés (cf. \textit{Hch} 2,1-4). Llenos del Espíritu Santo, los Apóstoles comienzan a proclamar \textquote{las maravillas de Dios} (\textit{Hch} 2,11) y Pedro declara que esta efusión del Espíritu es el signo de los tiempos mesiánicos (cf. \textit{Hch} 2, 17-18). Los que creyeron en la predicación apostólica y se hicieron bautizar, recibieron a su vez el don del Espíritu Santo (cf. \textit{Hch} 2,38).

\n{1288} \textquote{Desde [\ldots] aquel tiempo, los Apóstoles, en cumplimiento de la voluntad de Cristo, comunicaban a los neófitos, mediante la imposición de las manos, el don del Espíritu Santo, destinado a completar la gracia del Bautismo (cf. \textit{Hch} 8,15-17; 19,5-6). Esto explica por qué en la carta a los Hebreos se recuerda, entre los primeros elementos de la formación cristiana, la doctrina del Bautismo y de la imposición de las manos (cf. \textit{Hb} 6,2). Es esta imposición de las manos la que ha sido con toda razón considerada por la tradición católica como el primitivo origen del sacramento de la Confirmación, el cual perpetúa, en cierto modo, en la Iglesia, la gracia de Pentecostés} (Pablo VI, Const. apost. \textit{Divinae consortium naturae}).

\n{1289} Muy pronto, para mejor significar el don del Espíritu Santo, se añadió a la imposición de las manos una unción con óleo perfumado (crisma). Esta unción ilustra el nombre de \textquote{cristiano} que significa \textquote{ungido} y que tiene su origen en el nombre de Cristo, al que \textquote{Dios ungió con el Espíritu Santo} (\textit{Hch} 10,38). Y este rito de la unción existe hasta nuestros días tanto en Oriente como en Occidente. Por eso, en Oriente se llama a este sacramento crismación, unción con el crisma, o \textit{myron}, que significa \textquote{crisma}. En Occidente el nombre de \textit{Confirmación} sugiere que este sacramento al mismo tiempo confirma el Bautismo y robustece la gracia bautismal.
\end{ccebody}

\cceth{La Eucaristía}

\cceref{CEC 1322-1323}

\begin{ccebody}
\n{1322} La Sagrada Eucaristía culmina la iniciación cristiana. Los que han sido elevados a la dignidad del sacerdocio real por el Bautismo y configurados más profundamente con Cristo por la Confirmación, participan por medio de la Eucaristía con toda la comunidad en el sacrificio mismo del Señor.

\n{1323} \textquote{Nuestro Salvador, en la última Cena, la noche en que fue entregado, instituyó el Sacrificio Eucarístico de su cuerpo y su sangre para perpetuar por los siglos, hasta su vuelta, el sacrificio de la cruz y confiar así a su Esposa amada, la Iglesia, el memorial de su muerte y resurrección, sacramento de piedad, signo de unidad, vínculo de amor, banquete pascual en el que se recibe a Cristo, el alma se llena de gracia y se nos da una prenda de la gloria futura} (SC 47).
\end{ccebody}
